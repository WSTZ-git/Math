\documentclass[lang=cn,newtx,10pt,scheme=chinese]{elegantbook}

\title{高等代数白皮书(谢启鸿)解答}


\author{邹文杰}
\institute{无}
\date{2024/10/25}
\version{ElegantBook-4.5}
\bioinfo{自定义}{信息}

\extrainfo{宠辱不惊,闲看庭前花开花落;
\\
去留无意,漫随天外云卷云舒.}


\setcounter{tocdepth}{3}

\logo{logo-blue.png}
\cover{cover.png}

% 本文档额外使用的宏包和命令
\usepackage{mystyle-cn}

\begin{document}

\maketitle
\frontmatter

\tableofcontents

\mainmatter
\everymath{\displaystyle} % 让全文的行内公式都显示行间公式效果

\chapter{行列式}

\section{定义、定理和命题}

\begin{proposition}[\hypertarget{行列式计算常识}{行列式计算常识}]\label{pro:行列式计算常识}
(1)$\left| \begin{matrix}
&		&		&		a_n\\
&		&		\begin{turn}{80}$\ddots$\end{turn}&		\\
&		a_2&		&		\\
a_1&		&		&		\\
\end{matrix} \right|=\left( -1 \right) ^{\frac{n\left( n-1 \right)}{2}}a_1a_2\cdots a_n$
;\,\,$\left| \begin{matrix}
a_1&		&		&		&		\\
&		\ddots&		&		&		\\
b_1&		\cdots&		a_i&		\cdots&		b_n\\
&		&		&		\ddots&		\\
&		&		&		&		a_n\\
\end{matrix} \right|=a_1a_2\cdots a_n$.

(2)设$n$阶行列式$D=\det(a_{ij})$,把$D$上下翻转(\textbf{行倒排})、或左右翻转(\textbf{列倒排})分别得到$D_1$、$D_2$;把$D$\textbf{逆时针旋转$90^{\circ}$}、或\textbf{顺时针旋转$90^{\circ}$}分别得到$D_3$、$D_4$;把$D$\textbf{依副对角线翻转}、或\textbf{依主对角线翻转}分别得到$D_5$、$D_6$.易知
\begin{align*}
D_1=\left| \begin{matrix}
a_{n1}&		\cdots&		a_{nn}\\
\vdots&		&		\vdots\\
a_{11}&		\cdots&		a_{1n}\\
\end{matrix} \right|,D_2=\left| \begin{matrix}
a_{1n}&		\cdots&		a_{11}\\
\vdots&		&		\vdots\\
a_{nn}&		\cdots&		a_{n1}\\
\end{matrix} \right|,D_3=\left| \begin{matrix}
a_{1n}&		\cdots&		a_{nn}\\
\vdots&		&		\vdots\\
a_{11}&		\cdots&		a_{n1}\\
\end{matrix} \right|,
\\
D_4=\left| \begin{matrix}
a_{n1}&		\cdots&		a_{11}\\
\vdots&		&		\vdots\\
a_{nn}&		\cdots&		a_{1n}\\
\end{matrix} \right|,D_5=\left| \begin{matrix}
a_{nn}&		\cdots&		a_{1n}\\
\vdots&		&		\vdots\\
a_{n1}&		\cdots&		a_{11}\\
\end{matrix} \right|,D_6=\left| \begin{matrix}
a_{nn}&		\cdots&		a_{n1}\\
\vdots&		&		\vdots\\
a_{1n}&		\cdots&		a_{11}\\
\end{matrix} \right|.
\nonumber
\end{align*}
则一定有
\begin{gather*}
D_1=D_2=D_3=D_4=\left( -1 \right) ^{\frac{n\left( n-1 \right)}{2}}D,
\\
D_5=D_6=D.
\nonumber
\end{gather*}
(3)设\(A=(a_{i,j})\)为\(n\)阶复矩阵,则一定有\(\vert A\vert=\overline{\vert A\vert}\).

(4)若\(\vert A\vert\)是\(n\)阶行列式,\(\vert B\vert\)是\(m\)阶行列式,它们的值都不为零,则
\begin{align*}
\left| \left. \begin{matrix}
\boldsymbol{A}&		\boldsymbol{O}\\
\boldsymbol{O}&		\boldsymbol{B}\\
\end{matrix} \right. \right|=\left( -1 \right) ^{mn}\left. \left| \begin{matrix}
\boldsymbol{O}&		\boldsymbol{A}\\
\boldsymbol{B}&		\boldsymbol{O}\\
\end{matrix} \right| \right. .
\end{align*}
\end{proposition}
\begin{proof}
(1)运用行列式的定义即可得到结论.
\begin{align*}
(2)\,\,D_1&=\left| \begin{matrix}
a_{n1}&		\cdots&		a_{nn}\\
\vdots&		&		\vdots\\
a_{11}&		\cdots&		a_{1n}\\
\end{matrix} \right|\xlongequal[i=1,2,\cdots ,n-1]{r_i\longleftrightarrow r_{i+1}}\left( -1 \right) ^{n-1}\left| \begin{matrix}
a_{n-1,1}&		\cdots&		a_{n-1,n}\\
\vdots&		&		\vdots\\
a_{n1}&		\cdots&		a_{nn}\\
\end{matrix} \right|\xlongequal[i=1,2,\cdots ,n-2]{r_i\longleftrightarrow r_{i+1}}\left( -1 \right) ^{n-1+n-2}\left| \begin{matrix}
a_{n-2,1}&		\cdots&		a_{n-2,n}\\
\vdots&		&		\vdots\\
a_{n1}&		\cdots&		a_{nn}\\
\end{matrix} \right|
\\
&=\cdots =\left( -1 \right) ^{n-1+n-2+\cdots +1}\left| \begin{matrix}
a_{11}&		\cdots&		a_{1n}\\
\vdots&		&		\vdots\\
a_{n1}&		\cdots&		a_{nn}\\
\end{matrix} \right|=\left( -1 \right) ^{\frac{n\left( n-1 \right)}{2}}\left| \begin{matrix}
a_{11}&		\cdots&		a_{1n}\\
\vdots&		&		\vdots\\
a_{n1}&		\cdots&		a_{nn}\\
\end{matrix} \right|=\left( -1 \right) ^{\frac{n\left( n-1 \right)}{2}}D.
\end{align*}
\begin{align*}
D_2&=\left| \begin{matrix}
a_{1n}&		\cdots&		a_{11}\\
\vdots&		&		\vdots\\
a_{nn}&		\cdots&		a_{n1}\\
\end{matrix} \right|\xlongequal[i=1,2,\cdots ,n-1]{j_i\longleftrightarrow j_{i+1}}\left( -1 \right) ^{n-1}\left| \begin{matrix}
a_{1,n-1}&		\cdots&		a_{1n}\\
\vdots&		&		\vdots\\
a_{n,n-1}&		\cdots&		a_{nn}\\
\end{matrix} \right|\xlongequal[i=1,2,\cdots ,n-2]{j_i\longleftrightarrow j_{i+1}}\left( -1 \right) ^{n-1+n-2}\left| \begin{matrix}
a_{1,n-2}&		\cdots&		a_{1n}\\
\vdots&		&		\vdots\\
a_{n,n-2}&		\cdots&		a_{nn}\\
\end{matrix} \right|
\\
&=\cdots =\left( -1 \right) ^{n-1+n-2+\cdots +1}\left| \begin{matrix}
a_{11}&		\cdots&		a_{1n}\\
\vdots&		&		\vdots\\
a_{n1}&		\cdots&		a_{nn}\\
\end{matrix} \right|=\left( -1 \right) ^{\frac{n\left( n-1 \right)}{2}}\left| \begin{matrix}
a_{11}&		\cdots&		a_{1n}\\
\vdots&		&		\vdots\\
a_{n1}&		\cdots&		a_{nn}\\
\end{matrix} \right|=\left( -1 \right) ^{\frac{n\left( n-1 \right)}{2}}D.
\nonumber
\end{align*}
\begin{gather*}
D_3=\left| \begin{matrix}
a_{1n}&		\cdots&		a_{nn}\\
\vdots&		&		\vdots\\
a_{11}&		\cdots&		a_{n1}\\
\end{matrix} \right|\xlongequal{\text{行倒排}}\left( -1 \right) ^{\frac{n\left( n-1 \right)}{2}}\left| \begin{matrix}
a_{11}&		\cdots&		a_{n1}\\
\vdots&		&		\vdots\\
a_{1n}&		\cdots&		a_{nn}\\
\end{matrix} \right|=\left( -1 \right) ^{\frac{n\left( n-1 \right)}{2}}D^T=\left( -1 \right) ^{\frac{n\left( n-1 \right)}{2}}D.
\\
D_4=\left| \begin{matrix}
a_{n1}&		\cdots&		a_{11}\\
\vdots&		&		\vdots\\
a_{nn}&		\cdots&		a_{1n}\\
\end{matrix} \right|\xlongequal{\text{列倒排}}\left( -1 \right) ^{\frac{n\left( n-1 \right)}{2}}\left| \begin{matrix}
a_{11}&		\cdots&		a_{n1}\\
\vdots&		&		\vdots\\
a_{1n}&		\cdots&		a_{nn}\\
\end{matrix} \right|=\left( -1 \right) ^{\frac{n\left( n-1 \right)}{2}}D^T=\left( -1 \right) ^{\frac{n\left( n-1 \right)}{2}}D.
\\
D_5=\left| \begin{matrix}
a_{nn}&		\cdots&		a_{1n}\\
\vdots&		&		\vdots\\
a_{n1}&		\cdots&		a_{11}\\
\end{matrix} \right|\xlongequal[]{\text{逆时针旋转}90^{\circ}}\left( -1 \right) ^{\frac{n\left( n-1 \right)}{2}}\left| \begin{matrix}
a_{1n}&		\cdots&		a_{11}\\
\vdots&		&		\vdots\\
a_{nn}&		\cdots&		a_{n1}\\
\end{matrix} \right|\xlongequal[]{\text{列倒排}}\left( -1 \right) ^{\frac{n\left( n-1 \right)}{2}}\cdot \left( -1 \right) ^{\frac{n\left( n-1 \right)}{2}}\left| \begin{matrix}
a_{11}&		\cdots&		a_{1n}\\
\vdots&		&		\vdots\\
a_{n1}&		\cdots&		a_{nn}\\
\end{matrix} \right|=D.
\\
D_6=\left| \begin{matrix}
a_{nn}&		\cdots&		a_{n1}\\
\vdots&		&		\vdots\\
a_{1n}&		\cdots&		a_{11}\\
\end{matrix} \right|\xlongequal[]{\text{顺时针旋转}90^{\circ}}\left( -1 \right) ^{\frac{n\left( n-1 \right)}{2}}\left| \begin{matrix}
a_{1n}&		\cdots&		a_{nn}\\
\vdots&		&		\vdots\\
a_{11}&		\cdots&		a_{n1}\\
\end{matrix} \right|\xlongequal[]{\text{行倒排}}\left( -1 \right) ^{\frac{n\left( n-1 \right)}{2}}\cdot \left( -1 \right) ^{\frac{n\left( n-1 \right)}{2}}\left| \begin{matrix}
a_{11}&		\cdots&		a_{1n}\\
\vdots&		&		\vdots\\
a_{n1}&		\cdots&		a_{nn}\\
\end{matrix} \right|=D.
\nonumber
\end{gather*}
(3)复数的共轭保持加法和乘法:\(\overline{z_1 + z_2}=\overline{z_1}+\overline{z_2}\),\(\overline{z_1\cdot z_2}=\overline{z_1}\cdot\overline{z_2}\),故由行列式的组合定义可得
\begin{align*}
|A|&=\sum_{1\le k_1,k_2,\cdots ,k_n\le n}{\left( -1 \right) ^{\tau (k_1k_2\cdots k_n)}a_{k_{11}}a_{k_{22}}\cdots a_{k_{nn}}}
\\
&=\sum_{1\le k_1,k_2,\cdots ,k_n\le n}{\left( -1 \right) ^{\tau (k_1k_2\cdots k_n)}\overline{a_{k_{11}}}\cdot\overline{a_{k_{22}}}\cdots \overline{a_{k_{nn}}}}=|\overline{A}|.
\end{align*}

(4)将\(\vert A\vert\)的第一列依次和\(\vert B\vert\)的第\(m\)列,第\(m - 1\)列,…,第一列对换,共换了\(m\)次;再将\(\vert A\vert\)的第二列依次和\(\vert B\vert\)的第\(m\)列,第\(m - 1\)列,…,第一列对换,又换了\(m\)次;$\cdots$.依次类推,经过\(mn\)次对换可将第二个行列式变为第一个行列式.因此\(\vert D\vert=(-1)^{mn}\vert C\vert\),于是
由行列式的基本性质可得
\begin{align*}
\left| \left. \begin{matrix}
\boldsymbol{A}&		\boldsymbol{O}\\
\boldsymbol{O}&		\boldsymbol{B}\\
\end{matrix} \right. \right|=\left( -1 \right) ^{mn}\left. \left| \begin{matrix}
\boldsymbol{O}&		\boldsymbol{A}\\
\boldsymbol{B}&		\boldsymbol{O}\\
\end{matrix} \right| \right. .
\end{align*}
\end{proof}

\begin{proposition}[奇数阶反对称行列式的值等于零]\label{proposition:奇数阶反对称行列式的值等于零}
如果\(n\)阶行列式\(\vert A\vert\)的元素满足\(a_{ij}=-a_{ji}(1\leq i,j\leq n)\),则称为反对称行列式.求证:奇数阶反对称行列式的值等于零.
\end{proposition}
\begin{note}
{\color{blue}证法二}的想法是将行列式按组合的定义写成(n-1)!个单项的和.然后将其两两分组再求和(因为一共有(n-1)!个单项,即和式中共有偶数个单项,所以只要使用合适的分组方式就一定能够将其两两分组再求和),最后发现每组的和均为0.

构造的这个映射$\varphi$的目的是为了更加准确、严谨地说明分组的方式.证明这个映射$\varphi$是一个双射是为了保证原来的和式中的每一个单项都能与和式中另一个单项一一对应.\CJKunderline*{然后利用反证法证明了这两个一一对应的单项一定互不相同}\textbf{(注:我认为这步有些多余.这里应该只需要说明这两个一一对应的单项是原和式中不同的单项即可,即这两个单项的角标不完全相同就行,其实,这个在我们定义映射$\varphi$的时候就已经满足了.满足这个条件就足以说明原和式可以按照这种方式进行分组.并且利用反对称行列式的性质也能够证明这两个单项不仅互不相同,还能进一步得到这两个单项互为相反数)}.于是我们就可以将原和式中的每一个单项与其在双射$\varphi$作用下的像看成一组,按照这种方式就可以将原和式进行分组再求和.
\end{note}
\begin{proof}
{\color{blue}证法一(行列式的性质):}
由反对称行列式的定义可知,\(\vert A\vert\)的转置\(\vert A^{\prime}\vert\)与\(\vert A\vert\)的每个元素都相差一个符号,将\(\vert A^{\prime}\vert\)的每一行都提出公因子\(-1\)可得\(\vert A\vert=\vert A^{\prime}\vert=(-1)^{n}\vert A\vert=-\vert A\vert\),从而\(\vert A\vert = 0\).

{\color{blue}证法二(行列式的组合定义):}
由于\(\vert A\vert\)的主对角元全为0,故由组合定义,只需考虑下列单项:
\[
T = \{a_{k_11}a_{k_22}\cdots a_{k_{nn}} \mid k_i\neq i(1\leq i\leq n)\}
\]
定义映射\(\varphi:T\to T\),\(a_{k_11}a_{k_22}\cdots a_{k_{nn}}\mapsto a_{1k_1}a_{2k_2}\cdots a_{nk_n}\).显然\(\varphi^2 = \text{Id}_T\),于是\(\varphi\)是一个双射.我们断言:\(a_{k_11}a_{k_22}\cdots a_{k_{nn}}\)和\(a_{1k_1}a_{2k_2}\cdots a_{nk_n}\)作为\(\vert A\vert\)的单项不相同,否则\(\{1,2,\cdots,n\}\)必可分成若干对\((i_1,j_1),\cdots,(i_t,j_t)\),使得\(a_{k_11}a_{k_22}\cdots a_{k_{nn}}=a_{i_1j_1}a_{j_1i_1}\cdots a_{i_tj_t}a_{j_ti_t}\),这与\(n\)为奇数矛盾.将上述两个单项看成一组,则它们在\(\vert A\vert\)中符号均为\((-1)^{\tau(k_1k_2\cdots k_n)}\).由于\(\vert A\vert\)反对称,故
\[
a_{1k_1}a_{2k_2}\cdots a_{nk_n}=(-1)^n a_{k_11}a_{k_22}\cdots a_{k_{nn}}=-a_{k_11}a_{k_22}\cdots a_{k_{nn}}
\]
从而每组和为0,于是\(\vert A\vert = 0\).
\end{proof}

\begin{proposition}[\hypertarget{"爪"型行列式}{"爪"型行列式}]\label{"爪"型行列式}
证明$n$阶行列式:
\begin{gather}
|\boldsymbol{A}|=\left| \begin{matrix}
a_1&		b_2&		\cdots&		b_n\\
c_2&		a_2&		&		\\
\vdots&		&		\ddots&		\\
c_n&		&		&		a_n\\
\end{matrix} \right|
=a_1a_2\cdots a_n-\sum_{i=2}^n{a_2}\cdots \widehat{a_i}\cdots a_nb_ic_i.
\nonumber
\end{gather}
\end{proposition}
\begin{note}
记忆"爪"型行列式的计算方法和结论.
\end{note}
\begin{proof}
当$a_i\ne 0\left( \forall i\in \left[ 2,n \right] \cap \mathbb{N}  \right)$时,我们有
\begin{align*}
&|\boldsymbol{A}|=\left| \begin{matrix}
a_1&		b_2&		\cdots&		b_n\\
c_2&		a_2&		&		\\
\vdots&		&		\ddots&		\\
c_n&		&		&		a_n\\
\end{matrix} \right|\xlongequal[i=2,\cdots ,n]{\left( -\frac{c_i}{a_i} \right) j_i+j_1}\left| \begin{matrix}
a_1-\sum_{i=2}^n{\frac{b_ic_i}{a_i}}&		b_2&		\cdots&		b_n\\
0&		a_2&		&		\\
\vdots&		&		\ddots&		\\
0&		&		&		a_n\\
\end{matrix} \right|
\\
&=\left( a_1-\sum_{i=2}^n{\frac{b_ic_i}{a_i}} \right) \prod\limits_{i=2}^n{a_i}
\\
&=a_1a_2\cdots a_n-\sum_{i=2}^n{a_2}\cdots \widehat{a_i}\cdots a_nb_ic_i.
\end{align*}
当$\exists i\in \left[ 2,n \right] \cap \mathbb{N} \,\,s.t. \,\,a_i=0$时,则
$a_1a_2\cdots a_n-\sum_{i=2}^n{a_2}\cdots \widehat{a_i}\cdots a_nb_ic_i=-a_2\cdots \widehat{a_i}\cdots a_nb_ic_i$
.此时,我们有
\begin{align*}
|\boldsymbol{A}| &= \left| \begin{matrix}
a_1 & b_2 & \cdots & b_{i-1} & b_i & b_{i+1} & \cdots & b_n \\
c_2 & a_2 & & & & & & \\
\vdots & & \ddots & & & & & \\
c_{i-1} & & & a_{i-1} & & & & \\
c_i & & & & 0 & & & \\
c_{i+1} & & & & & a_{i+1} & & \\
\vdots & & & & & & \ddots & \\
c_n & & & & & & & a_n \\
\end{matrix} \right|
\xlongequal[(\text{按}c_i\text{所在行展开})]{\text{按第}i\text{行展开}} (-1)^{i+1}c_i \left| \begin{matrix}
b_2 & \cdots & b_{i-1} & b_i & b_{i+1} & \cdots & b_n \\
a_2 & & & & & & & \\
& \ddots & & & & & & \\
& & a_{i-1} & 0 & 0 & & & \\
& & 0 & 0 & a_{i+1} & & & \\
& & & & & \ddots & & \\
& & & & & & a_n \\
\end{matrix} \right| \\
&\xlongequal[(\text{按}b_i\text{所在列展开})]{\text{按第}i-1\text{列展开}} (-1)^{i+1}(-1)^{i}b_ic_i \left| \begin{matrix}
a_2 & & & & & \\
& \ddots & & & & \\
& & a_{i-1} & & & \\
& & & a_{i+1} & & \\
& & & & \ddots & \\
& & & & & a_n \\
\end{matrix} \right|      
= -a_2 \cdots \widehat{a_i} \cdots a_nb_ic_i.
\end{align*}
综上所述,原命题得证.
\end{proof}

\begin{proposition}[分块"爪"型行列式]\label{proposition:分块"爪"型行列式}
计算$n$阶行列式($a_{ii}\ne 0,i=k+1,k+2,\cdots,n$):
\begin{align*}
|\boldsymbol{A}|=\left| \begin{matrix}
a_{11}&		\cdots&		a_{1k}&		a_{1,k+1}&		\cdots&		a_{1n}\\
\vdots&		&		\vdots&		\vdots&		&		\vdots\\
a_{k1}&		\cdots&		a_{kk}&		a_{k,k+1}&		\cdots&		a_{kn}\\
a_{k+1,1}&		\cdots&		a_{k+1,k}&		a_{k+1,k+1}&		&		\\
\vdots&		&		\vdots&		&		\ddots&		\\
a_{n1}&		\cdots&		a_{nk}&		&		&		a_{nn}\\
\end{matrix} \right|.
\end{align*}
\end{proposition}
\begin{note}
记忆分块"爪"型行列式的计算方法即可,计算方法和"爪"型行列式的计算方法类似.
\end{note}
\begin{solution}
\begin{align*}
&|\boldsymbol{A}|=\left| \begin{matrix}
a_{11}&		\cdots&		a_{1k}&		a_{1,k+1}&		\cdots&		a_{1n}\\
\vdots&		&		\vdots&		\vdots&		&		\vdots\\
a_{k1}&		\cdots&		a_{kk}&		a_{k,k+1}&		\cdots&		a_{kn}\\
a_{k+1,1}&		\cdots&		a_{k+1,k}&		a_{k+1,k+1}&		&		\\
\vdots&		&		\vdots&		&		\ddots&		\\
a_{n1}&		\cdots&		a_{nk}&		&		&		a_{nn}\\
\end{matrix} \right|
\\
&\xlongequal[i=k+1,k+2,\cdots ,n]{-\frac{a_{i1}}{a_{ii}}j_i+j_1,-\frac{a_{i2}}{a_{ii}}j_i+j_2,\cdots ,-\frac{a_{in}}{a_{ii}}j_i+j_k}\left| \begin{matrix}
c_{11}&		\cdots&		c_{1k}&		a_{1,k+1}&		\cdots&		a_{1n}\\
\vdots&		&		\vdots&		\vdots&		&		\vdots\\
c_{k1}&		\cdots&		c_{kk}&		a_{k,k+1}&		\cdots&		a_{kn}\\
0&		\cdots&		0&		a_{k+1,k+1}&		&		\\
\vdots&		&		\vdots&		&		\ddots&		\\
0&		\cdots&		0&		&		&		a_{nn}\\
\end{matrix} \right|
\\
&=\left| \begin{matrix}
C&		B\\
O&		\Lambda\\
\end{matrix} \right|=|C|\cdot |\Lambda |=|C|\prod_{i=k+1}^n{a_{ii}}.
\end{align*}
其中$C=\left( \begin{matrix}
c_{11}&		\cdots&		c_{1k}\\
\vdots&		&		\vdots\\
c_{k1}&		\cdots&		c_{kk}\\
\end{matrix} \right) ,B=\left( \begin{matrix}
a_{1,k+1}&		\cdots&		a_{1n}\\
\vdots&		&		\vdots\\
a_{k,k+1}&		\cdots&		a_{kn}\\
\end{matrix} \right) ,\Lambda =\left( \begin{matrix}
a_{k+1}&		&		\\
&		\ddots&		\\
&		&		a_n\\
\end{matrix} \right).$
并且$c_{pq}=a_{pq}-\sum_{i=k+1}^n{\frac{a_{iq}a_{pi}}{a_{ii}}},p,q=1,2,\cdots ,n$.
\end{solution}

\begin{corollary}[\hypertarget{"爪"型行列式的推广}{"爪"型行列式的推广}]\label{"爪"型行列式的推广}
计算$n$阶行列式:
\begin{equation}
|\boldsymbol{A}|=\left| \begin{matrix}
x_1-a_1&		x_2&		x_3&		\cdots&		x_n\\
x_1&		x_2-a_2&		x_3&		\cdots&		x_n\\
x_1&		x_2&		x_3-a_3&		\cdots&		x_n\\
\vdots&		\vdots&		\vdots&		&		\vdots\\
x_1&		x_2&		x_3&		\cdots&		x_n-a_n\\
\end{matrix} \right|.
\nonumber
\end{equation}
\end{corollary}
\begin{note}
这是一个有用的模板(即\textbf{行列式除了主对角元素外,每行都一样}).

记忆该命题的计算方法即可.即先化为"爪"型行列式,再利用"爪"型行列式的计算结果.
\end{note}
\begin{solution}
当$a_i\ne 0\left( \forall i\in \left[ 2,n \right] \cap \mathbb{N}  \right)$时,我们有
\begin{equation}
\begin{split}
|\boldsymbol{A}|&=\left| \begin{matrix}
x_1-a_1&		x_2&		x_3&		\cdots&		x_n\\
x_1&		x_2-a_2&		x_3&		\cdots&		x_n\\
x_1&		x_2&		x_3-a_3&		\cdots&		x_n\\
\vdots&		\vdots&		\vdots&		&		\vdots\\
x_1&		x_2&		x_3&		\cdots&		x_n-a_n\\
\end{matrix} \right|\xlongequal[i=2,\cdots ,n]{\left( -1 \right) r_1+r_i}\left| \begin{matrix}
x_1-a_1&		x_2&		x_3&		\cdots&		x_n\\
a_1&		-a_2&		0&		\cdots&		0\\
a_1&		0&		-a_3&		\cdots&		0\\
\vdots&		\vdots&		\vdots&		&		\vdots\\
a_1&		0&		0&		\cdots&		-a_n\\
\end{matrix} \right|
\\
&\xlongequal{\text{命题}\ref{"爪"型行列式}}\left[ \left( x_1-a_1 \right) +\sum_{i=2}^n{\frac{a_1x_i}{a_i}} \right] \prod\limits_{i=2}^n{\left( -a_i \right)}=\left( -1 \right) ^{n-1}\left[ \left( x_1-a_1 \right) +\sum_{i=2}^n{\frac{a_1x_i}{a_i}} \right] \prod\limits_{i=2}^n{a_i}
\\
&=\left( -1 \right) ^{n-1}\left[ \left( x_1-a_1 \right) \prod\limits_{i=2}^n{a_i}+\sum_{i=2}^n{a_1a_2\cdots \widehat{a_i}\cdots a_n}x_i \right] .
\end{split}
\nonumber
\end{equation}

当$\exists i\in \left[ 2,n \right] \cap \mathbb{N}\,\,s.t.\,\, a_i=0$时,我们有
\begin{equation}
\begin{split}
|\boldsymbol{A}|&=\left| \begin{matrix}
x_1-a_1&		x_2&		x_3&		\cdots&		x_n\\
x_1&		x_2-a_2&		x_3&		\cdots&		x_n\\
x_1&		x_2&		x_3-a_3&		\cdots&		x_n\\
\vdots&		\vdots&		\vdots&		&		\vdots\\
x_1&		x_2&		x_3&		\cdots&		x_n-a_n\\
\end{matrix} \right|\xlongequal[i=2,\cdots ,n]{\left( -1 \right) r_1+r_i}\left| \begin{matrix}
x_1-a_1&		x_2&		x_3&		\cdots&		x_n\\
a_1&		-a_2&		0&		\cdots&		0\\
a_1&		0&		-a_3&		\cdots&		0\\
\vdots&		\vdots&		\vdots&		&		\vdots\\
a_1&		0&		0&		\cdots&		-a_n\\
\end{matrix} \right|
\\
&\xlongequal{\text{命题}\ref{"爪"型行列式}}\left( x_1-a_1 \right) \left( -a_2 \right) \left( -a_3 \right) \cdots \left( -a_n \right) -\sum_{i=2}^n{\left( -a_2 \right) \cdots \widehat{\left( -a_i \right) }\cdots \left( -a_n \right)}a_1x_i
\\
&=\left( -1 \right) ^{n-1}\left( x_1-a_1 \right) \prod\limits_{i=2}^n{a_i}+\left( -1 \right) ^{n-1}\sum_{i=2}^n{a_1a_2\cdots \widehat{a_i}\cdots a_n}x_i
\\
&=\left( -1 \right) ^{n-1}\left[ \left( x_1-a_1 \right) \prod\limits_{i=2}^n{a_i}+\sum_{i=2}^n{a_1a_2\cdots \widehat{a_i}\cdots a_n}x_i \right] .            
\end{split}
\nonumber
\end{equation}
综上所述,$|\boldsymbol{A}|=\left( -1 \right) ^{n-1}\left[ \left( x_1-a_1 \right) \prod\limits_{i=2}^n{a_i}+\sum_{i=2}^n{a_1a_2\cdots \widehat{a_i}\cdots a_nx_i} \right]$.
\end{solution}

\begin{proposition}\label{根据行列式代数余子式构造行列式}
设\(\vert A\vert=\vert a_{i}\vert\)是一个\(n\)阶行列式,\(A_{ij}\)是它的第\((i,j)\)元素的代数余子式,求证:
\begin{gather}
\left| \begin{matrix}
a_{11}&		a_{12}&		\cdots&		a_{1n}&		x_1\\
a_{21}&		a_{22}&		\cdots&		a_{2n}&		x_2\\
\vdots&		\vdots&		&		\vdots&		\vdots\\
a_{n1}&		a_{n2}&		\cdots&		a_{nn}&		x_n\\
y_1&		y_2&		\cdots&		y_n&		z\\
\end{matrix} \right|=z|\boldsymbol{A}|-\sum_{i=1}^n{\sum_{j=1}^n{A_{ij}x_iy_j.}}
\nonumber
\end{gather}
\end{proposition}
\begin{note}\label{关于行列式|A|所有代数余子式求和的构造}
根据这个命题可以得到一个\textbf{关于行列式$|\boldsymbol{A}|$的所有代数余子式求和的构造}:

\begin{align*}
-\sum_{i,j=1}^n{A_{ij}}=\left| \begin{matrix}
\boldsymbol{A}&		\mathbf{1}\\
\mathbf{1}'&		0\\
\end{matrix} \right|=\left| \begin{matrix}
\boldsymbol{\alpha }_{\mathbf{1}}&		\boldsymbol{\alpha }_{\mathbf{2}}&		\cdots&		\boldsymbol{\alpha }_{\boldsymbol{n}}&		\mathbf{1}\\
1&		1&		\cdots&		1&		0\\
\end{matrix} \right|=\left| \begin{matrix}
\boldsymbol{\beta }_{\mathbf{1}}&		1\\
\boldsymbol{\beta }_{\mathbf{2}}&		1\\
\vdots&		\vdots\\
\boldsymbol{\beta }_{\boldsymbol{n}}&		1\\
\mathbf{1}'&		0\\
\end{matrix} \right|.
\end{align*}
其中$|\boldsymbol{A}|$的列向量依次为$\boldsymbol{\alpha }_{\mathbf{1}},\boldsymbol{\alpha }_{\mathbf{2}},\cdots ,\boldsymbol{\alpha }_{\boldsymbol{n}}$,$|\boldsymbol{A}|$的行向量依次为$\boldsymbol{\beta }_{\mathbf{1}},\boldsymbol{\beta }_{\mathbf{2}},\cdots ,\boldsymbol{\beta }_{\boldsymbol{n}}$.并且$\mathbf{1}$表示元素均为1的列向量,$\mathbf{1}'$表示$\mathbf{1}$的转置.
(令上述命题中的$z=0,x_i=y_i=1,i=1,2,\cdots,n$即可得到.)
\end{note}
\begin{remark}
如果需要证明的是矩阵的代数余子式的相关命题,我们可以考虑一下这种构造,即令上述命题中的$z=0$并且待定/任取$x_i,y_i$.
\end{remark}
\begin{proof}
{\color{blue}证法一:}
将上述行列式先按最后一列展开,展开式的第一项为
\begin{equation}
\begin{split}
\left( -1 \right) ^{n+2}x_1\left| \begin{matrix}
a_{21}&		a_{22}&		\cdots&		a_{2n}\\
\vdots&		\vdots&		&		\vdots\\
a_{n1}&		a_{n2}&		\cdots&		a_{nn}\\
y_1&		y_2&		\cdots&		y_n\\
\end{matrix} \right|.
\end{split}
\nonumber
\end{equation}
再将上式按最后一行展开得到
\begin{equation}
\begin{split}
&\left( -1 \right) ^{n+2}x_1\left[ \left( -1 \right) ^{n+1}\left( -1 \right) ^{1+1}y_1A_{11}+\left( -1 \right) ^{n+2}\left( -1 \right) ^{1+2}y_2A_{12}+\cdots +\left( -1 \right) ^{n+n}\left( -1 \right) ^{1+n}y_nA_{1n} \right]
\\
&=\left( -1 \right) ^{n+2}x_1\left( -1 \right) ^{n+1}\left[ \left( -1 \right) ^2y_1A_{11}+\left( -1 \right) ^4y_2A_{12}+\cdots +\left( -1 \right) ^{2n}y_nA_{1n} \right] 
\\
&=-x_1\left( y_1A_{11}+y_2A_{12}+\cdots +y_nA_{1n} \right)
\\
&=-x_1\sum_{j=1}^n{y_jA_{1j}}.            
\end{split}
\nonumber
\end{equation}
同理可得原行列式展开式的第$i(i=1,2,\cdots,n-1)$项为
\begin{equation}
\begin{split}
\left( -1 \right) ^{n+1+i}x_i\left| \begin{matrix}
a_{11}&		a_{12}&		\cdots&		a_{1n}\\
\vdots&		\vdots&		&		\vdots\\
a_{i-1,1}&		a_{i-1,2}&		\cdots&		a_{i-1,n}\\
a_{i+1,1}&		a_{i+1,2}&		\cdots&		a_{i+1,n}\\
\vdots&		\vdots&		&		\vdots\\
a_{n1}&		a_{n2}&		\cdots&		a_{nn}\\
y_1&		y_2&		\cdots&		y_n\\
\end{matrix} \right|.
\end{split}
\nonumber
\end{equation}
将上式按最后一行展开得到$z\left|\boldsymbol{A}\right|$.
\begin{equation}
\begin{split}
&\left( -1 \right) ^{n+1+i}x_i\left[ \left( -1 \right) ^{n+1}\left( -1 \right) ^{i+1}y_1A_{i1}+\left( -1 \right) ^{n+2}\left( -1 \right) ^{i+2}y_2A_{i2}+\cdots +\left( -1 \right) ^{n+n}\left( -1 \right) ^{i+n}y_nA_{in} \right] 
\\
&=\left( -1 \right) ^{n+1+i}x_i\left( -1 \right) ^{n+1}\left[ \left( -1 \right) ^{i+1}y_1A_{i1}+\left( -1 \right) ^{i+2+1}y_2A_{i2}+\cdots +\left( -1 \right) ^{i+n+n-1}y_nA_{in} \right] 
\\
&=\left( -1 \right) ^{2i+1}y_1A_{i1}+\left( -1 \right) ^{2i+3}y_2A_{i2}+\cdots +\left( -1 \right) ^{2i+2n-1}y_nA_{in}
\\
&=-x_i\left( y_1A_{i1}+y_2A_{i2}+\cdots +y_nA_{in} \right) 
\\
&=-x_i\sum_{j=1}^n{y_jA_{ij}.}
\end{split}
\nonumber
\end{equation}
而展开式的最后一项为$z\left|\boldsymbol{A}\right|$.

因此,原行列式的值为
\begin{equation}
z|\boldsymbol{A}|-\sum_{i=1}^n{\sum_{j=1}^n{A_{ij}x_iy_j.}}
\nonumber
\end{equation}

{\color{blue}证法二:}设\(\boldsymbol{x}=(x_1,x_2,\cdots,x_n)',\boldsymbol{y}=(y_1,y_2,\cdots,y_n)'\). 若\(A\)是非异阵,则由降阶公式可得
\[
\begin{vmatrix}
A & \boldsymbol{x}\\
\boldsymbol{y}' & z
\end{vmatrix}=|A|(z - \boldsymbol{y}'A^{-1}\boldsymbol{x})=z|A| - \boldsymbol{y}'A^*\boldsymbol{x}.
\]

对于一般的方阵\(A\),可取到一列有理数\(t_k\rightarrow0\),使得\(t_kI_n + A\)为非异阵. 由非异阵情形的证明可得
\[
\begin{vmatrix}
t_kI_n + A & \boldsymbol{x}\\
\boldsymbol{y}' & z
\end{vmatrix}=z|t_kI_n + A| - \boldsymbol{y}'(t_kI_n + A)^*\boldsymbol{x}.
\]

注意到上式两边都是关于\(t_k\)的多项式,从而关于\(t_k\)连续. 上式两边同时取极限,令\(t_k\rightarrow0\),即有
\[
\begin{vmatrix}
A & \boldsymbol{x}\\
\boldsymbol{y}' & z
\end{vmatrix}=z|A| - \boldsymbol{y}'A^*\boldsymbol{x}=z|A|-\sum_{i = 1}^{n}\sum_{j = 1}^{n}A_{ij}x_iy_j.
\]
\end{proof}

\begin{example}\label{example:求矩阵代数余子式和的方法1}
设\(n\)阶行列式\(\vert \boldsymbol{A} \vert=\vert a_{ij}\vert\),\(A_{ij}\)是元素\(a_{ij}\)的代数余子式,求证:
\[
\vert B \vert = 
\begin{vmatrix}
a_{11}-a_{12} & a_{12}-a_{13} & \cdots & a_{1,n - 1}-a_{1n} & 1\\
a_{21}-a_{22} & a_{22}-a_{23} & \cdots & a_{2,n - 1}-a_{2n} & 1\\
a_{31}-a_{32} & a_{32}-a_{33} & \cdots & a_{3,n - 1}-a_{3n} & 1\\
\vdots & \vdots & \ddots & \vdots & \vdots\\
a_{n1}-a_{n2} & a_{n2}-a_{n3} & \cdots & a_{n,n - 1}-a_{nn} & 1
\end{vmatrix}
= \sum_{i,j = 1}^{n}A_{ij}.
\]
\end{example}
\begin{proof}
{\color{blue}证法一:}设\(|\boldsymbol{A}|\)的列向量依次为\(\boldsymbol{\alpha }_{\mathbf{1}},\boldsymbol{\alpha }_{\mathbf{2}},\cdots ,\boldsymbol{\alpha }_{\boldsymbol{n}}\),并且\(\mathbf{1}\)表示元素均为\(1\)的列向量.则
\begin{align*}
|\boldsymbol{B}|=|\boldsymbol{\alpha }_{\mathbf{1}}-\boldsymbol{\alpha }_{\mathbf{2}},\boldsymbol{\alpha }_{\mathbf{2}}-\boldsymbol{\alpha }_{\mathbf{3}},\cdots ,\boldsymbol{\alpha }_{\boldsymbol{n}-\mathbf{1}}-\boldsymbol{\alpha }_{\boldsymbol{n}},1|\xlongequal[i=n-1,n-2,\cdots ,2]{j_i+j_{i-1}}|\boldsymbol{\alpha }_{\mathbf{1}}-\boldsymbol{\alpha }_{\boldsymbol{n}},\boldsymbol{\alpha }_{\mathbf{2}}-\boldsymbol{\alpha }_{\boldsymbol{n}},\cdots ,\boldsymbol{\alpha }_{\boldsymbol{n}-\mathbf{1}}-\boldsymbol{\alpha }_{\boldsymbol{n}},1|.        
\end{align*}
将最后一列写成\((\boldsymbol{\alpha}_{\boldsymbol{n}} + \mathbf{1}) - \boldsymbol{\alpha}_{\boldsymbol{n}}\),进行拆分可得
\begin{align*}
&|\boldsymbol{B}| = |\boldsymbol{\alpha}_{\boldsymbol{1}} - \boldsymbol{\alpha}_{\boldsymbol{n}},\boldsymbol{\alpha}_{\boldsymbol{2}} - \boldsymbol{\alpha}_{\boldsymbol{n}},\cdots,\boldsymbol{\alpha}_{\boldsymbol{n - 1}} - \boldsymbol{\alpha}_n,(\boldsymbol{\alpha}_{\boldsymbol{n}} + \mathbf{1}) - \boldsymbol{\alpha}_{\boldsymbol{n}}|
\\
&= |\boldsymbol{\alpha}_{\boldsymbol{1}} - \boldsymbol{\alpha}_{\boldsymbol{n}},\boldsymbol{\alpha}_{\boldsymbol{2}} - \boldsymbol{\alpha}_{\boldsymbol{n}},\cdots,\boldsymbol{\alpha}_{\boldsymbol{n - 1}} - \boldsymbol{\alpha}_{\boldsymbol{n}},\boldsymbol{\alpha}_{\boldsymbol{n}} + \mathbf{1}| - |\boldsymbol{\alpha}_{\boldsymbol{1}} - \boldsymbol{\alpha}_{\boldsymbol{n}},\boldsymbol{\alpha}_{\boldsymbol{2}} - \boldsymbol{\alpha}_{\boldsymbol{n}},\cdots,\boldsymbol{\alpha}_{\boldsymbol{n - 1}} - \boldsymbol{\alpha}_{\boldsymbol{n}},\boldsymbol{\alpha}_{\boldsymbol{n}}|
\\
&= |\boldsymbol{\alpha}_{\boldsymbol{1}} + \mathbf{1},\boldsymbol{\alpha}_{\boldsymbol{2}} + \mathbf{1},\cdots,\boldsymbol{\alpha}_{\boldsymbol{n - 1}} + \mathbf{1},\boldsymbol{\alpha}_{\boldsymbol{n}} + \mathbf{1}| - |\boldsymbol{\alpha}_{\boldsymbol{1}},\boldsymbol{\alpha}_{\boldsymbol{2}},\cdots,\boldsymbol{\alpha}_{\boldsymbol{n-1}},\boldsymbol{\alpha}_{\boldsymbol{n}}|.
\end{align*}
根据行列式的性质将\(|\boldsymbol{\alpha}_{\boldsymbol{1}} + \mathbf{1},\boldsymbol{\alpha}_{\boldsymbol{2}} + \mathbf{1},\cdots,\boldsymbol{\alpha}_{\boldsymbol{n-1}} + \mathbf{1},\boldsymbol{\alpha}_{\boldsymbol{n}} + \mathbf{1}|\)每一列都拆分成两列,然后按\(1\)所在的列展开得到
\begin{align*}
&|\boldsymbol{B}| = |\boldsymbol{\alpha}_{\boldsymbol{1}} + \mathbf{1},\boldsymbol{\alpha}_{\boldsymbol{2}} + \mathbf{1},\cdots,\boldsymbol{\alpha}_{\boldsymbol{n-1}} + \mathbf{1},\boldsymbol{\alpha}_{\boldsymbol{n}} + \mathbf{1}| - |\boldsymbol{\alpha}_{\boldsymbol{1}},\boldsymbol{\alpha}_{\boldsymbol{2}},\cdots,\boldsymbol{\alpha}_{\boldsymbol{n-1}},\boldsymbol{\alpha}_{\boldsymbol{n}}|
\\
&= |\boldsymbol{\alpha}_{\boldsymbol{1}},\boldsymbol{\alpha}_{\boldsymbol{2}},\cdots,\boldsymbol{\alpha}_{\boldsymbol{n-1}},\boldsymbol{\alpha}_{\boldsymbol{n}}| + \sum_{i,j = 1}^{n}A_{ij} - |\boldsymbol{\alpha}_{\boldsymbol{1}},\boldsymbol{\alpha}_2,\cdots,\boldsymbol{\alpha}_{\boldsymbol{n-1}},\boldsymbol{\alpha}_{\boldsymbol{n}}| = \sum_{i,j = 1}^{n}A_{ij}.
\end{align*}

{\color{blue}证法二:}设\(|\boldsymbol{A}|\)的列向量依次为\(\boldsymbol{\alpha}_{\boldsymbol{1}},\boldsymbol{\alpha}_{\boldsymbol{2}},\cdots,\boldsymbol{\alpha}_{\boldsymbol{n}}\),并且\(\mathbf{1}\)表示元素均为\(1\)的列向量.\hyperref[关于行列式|A|所有代数余子式求和的构造]{注意到}
\begin{align*}
-\sum_{i,j=1}^n{A_{ij}}=\left| \begin{matrix}
\boldsymbol{\alpha }_{\mathbf{1}}&		\boldsymbol{\alpha }_{\mathbf{2}}&		\cdots&		\boldsymbol{\alpha }_{\boldsymbol{n}}&		\mathbf{1}\\
1&		1&		\cdots&		1&		0\\
\end{matrix} \right|.
\end{align*}
依次将第$i$列乘以$-1$加到第$i-1$列上去$(i=2,3,\cdots,n)$,再按第$n+1$行展开可得
\begin{align*}
-\sum_{i,j=1}^n{A_{ij}=\left| \begin{matrix}
\boldsymbol{\alpha }_{\mathbf{1}}-\boldsymbol{\alpha }_{\mathbf{2}}&		\boldsymbol{\alpha }_{\mathbf{2}}-\boldsymbol{\alpha }_{\mathbf{3}}&		\cdots&		\boldsymbol{\alpha }_{\boldsymbol{n}-\mathbf{1}}-\boldsymbol{\alpha }_{\boldsymbol{n}}&		\boldsymbol{\alpha }_{\boldsymbol{n}}&		1\\
0&		0&		\cdots&		0&		1&		0\\
\end{matrix} \right|}
\\
=-|\boldsymbol{\alpha }_{\mathbf{1}}-\boldsymbol{\alpha }_{\mathbf{2}},\boldsymbol{\alpha }_{\mathbf{2}}-\boldsymbol{\alpha }_{\mathbf{3}},\cdots ,\boldsymbol{\alpha }_{\boldsymbol{n}-\mathbf{1}}-\boldsymbol{\alpha }_{\boldsymbol{n}},1|=-|\boldsymbol{B}|.
\end{align*}
结论得证.
\end{proof}

\begin{example}
设\(n\)阶矩阵\(A\)的每一行、每一列的元素之和都为零,证明:\(A\)的每个元素的代数余子式都相等.
\end{example}
\begin{proof}
{\color{blue}证法一:}设\(A=(a_{ij})\),\(\boldsymbol{x}=(x_1,x_2,\cdots,x_n)'\),\(\boldsymbol{y}=(y_1,y_2,\cdots,y_n)'\),不妨设$x_iy_j$均不相同,$i,j=1,2,\cdots,n$.考虑如下\(n + 1\)阶矩阵的行列式求值:
\[
B=\begin{pmatrix}
A & \boldsymbol{x}\\
\boldsymbol{y}' & 0
\end{pmatrix}
\]
一方面,由\hyperref[根据行列式代数余子式构造行列式]{命题\ref{根据行列式代数余子式构造行列式}}可得\(|B|=-\sum_{i = 1}^{n}\sum_{j = 1}^{n}A_{ij}x_iy_j\). 另一方面,先把行列式\(|B|\)的第二行,\(\cdots\),第\(n\)行全部加到第一行上;再将第二列,\(\cdots\),第\(n\)列全部加到第一列上,可得
\[
\begin{vmatrix}
a_{11}&a_{12}&\cdots&a_{1n}&x_1\\
a_{21}&a_{22}&\cdots&a_{2n}&x_2\\
\vdots&\vdots&&\vdots&\vdots\\
a_{n1}&a_{n2}&\cdots&a_{nn}&x_n\\
y_1&y_2&\cdots&y_n&0
\end{vmatrix}=
\begin{vmatrix}
0&0&\cdots&0&\sum_{i = 1}^{n}x_i\\
a_{21}&a_{22}&\cdots&a_{2n}&x_2\\
\vdots&\vdots&&\vdots&\vdots\\
a_{n1}&a_{n2}&\cdots&a_{nn}&x_n\\
y_1&y_2&\cdots&y_n&0
\end{vmatrix}=
\begin{vmatrix}
0&0&\cdots&0&\sum_{i = 1}^{n}x_i\\
0&a_{22}&\cdots&a_{2n}&x_2\\
\vdots&\vdots&&\vdots&\vdots\\
0&a_{n2}&\cdots&a_{nn}&x_n\\
\sum_{j = 1}^{n}y_j&y_2&\cdots&y_n&0
\end{vmatrix}
\]
依次按照第一行和第一列进行展开,可得\(|B|=-A_{11}\sum_{i = 1}^{n}\sum_{j = 1}^{n}x_iy_j\). 比较上述两个结果,又由于$x_iy_j$均不
相同,因此可得\(A\)的所有代数余子式都相等.

{\color{blue}证法二:}由假设可知$\left| \boldsymbol{A} \right|=0$(每行元素全部加到第一行即得),从而\(\boldsymbol{A}\)是奇异矩阵. 若\(\boldsymbol{A}\)的秩小于\(n - 1\),则\(\boldsymbol{A}\)的任意一个代数余子式\(A_{ij}\)都等于零,结论显然成立. 若\(\boldsymbol{A}\)的秩等于\(n - 1\),则线性方程组\(\boldsymbol{A}\boldsymbol{x}=\boldsymbol{0}\)的基础解系只含一个向量. 又因为\(\boldsymbol{A}\)的每一行元素之和都等于零,所以由\hyperref[proposition:对矩阵行和和列和的一种刻画]{命题\ref{proposition:对矩阵行和和列和的一种刻画}}可知,我们可以选取\(\boldsymbol{\alpha}=(1,1,\cdots,1)'\)作为\(\boldsymbol{A}\boldsymbol{x}=\boldsymbol{0}\)的基础解系. 由\hyperref[proposition:奇异系数矩阵Ax=0的解空间]{命题\ref{proposition:奇异系数矩阵Ax=0的解空间}的证明}可知\(\boldsymbol{A}^*\)的每一列都是$\boldsymbol{A}\boldsymbol{x}=\boldsymbol{0}$的解,从而\(\boldsymbol{A}^*\)的每一列与\(\boldsymbol{\alpha}\)成比例,特别地,\(\boldsymbol{A}^*\)的每一行都相等. 对\(\boldsymbol{A}'\)重复上面的讨论,可得\((\boldsymbol{A}')^*\)的每一行都相等.注意到\((\boldsymbol{A}')^*=(\boldsymbol{A}^*)'\),从而\(\boldsymbol{A}^*\)的每一列都相等,于是\(\boldsymbol{A}\)的所有代数余子式\(A_{ij}\)都相等.
\end{proof}

\begin{proposition}[\hypertarget{三对角行列式}{三对角行列式}]\label{三对角行列式}
求下列行列式的递推关系式(空白处均为0):
\begin{equation}
\begin{split}
D_n=\left| \begin{matrix}
a_1&		b_1&		&		&		&		\\
c_1&		a_2&		b_2&		&		&		\\
&		c_2&		a_3&		\ddots&		&		\\
&		&		\ddots&		\ddots&		\ddots&		\\
&		&		&		\ddots&		a_{n-1}&		b_{n-1}\\
&		&		&		&		c_{n-1}&		a_n\\
\end{matrix} \right|.
\end{split}
\nonumber
\end{equation}
\end{proposition}
\begin{note}
记忆三对角行列式的计算方法和结果:
$\boldsymbol{D}_{\boldsymbol{n}}=\boldsymbol{a}_{\boldsymbol{n}}\boldsymbol{D}_{\boldsymbol{n}-\boldsymbol{1}}-\boldsymbol{b}_{\boldsymbol{n}-\boldsymbol{1}}\boldsymbol{c}_{\boldsymbol{n}-\boldsymbol{1}}\boldsymbol{D}_{\boldsymbol{n}-\boldsymbol{2}}\boldsymbol{(n}\ge \boldsymbol{2)}$,

即按最后一列(或行)展开得到递推公式.
\end{note}
\begin{solution}
显然$D_0=1,D_1=a_1$.当$n\ge2$时,我们有
\begin{align*}
D_n&=\left| \begin{matrix}
a_1&		b_1&		&		&		&		\\
c_1&		a_2&		b_2&		&		&		\\
&		c_2&		a_3&		\ddots&		&		\\
&		&		\ddots&		\ddots&		\ddots&		\\
&		&		&		\ddots&		a_{n-1}&		b_{n-1}\\
&		&		&		&		c_{n-1}&		a_n\\
\end{matrix} \right|=\left| \begin{matrix}
a_1&		b_1&		&		&		&		&		\\
c_1&		a_2&		b_2&		&		&		&		\\
&		c_2&		a_3&		\ddots&		&		&		\\
&		&		\ddots&		\ddots&		\ddots&		&		\\
&		&		&		\ddots&		a_{n-2}&		b_{n-2}&		\\
&		&		&		&		c_{n-2}&		a_{n-1}&		b_{n-1}\\
&		&		&		&		&		c_{n-1}&		a_n\\
\end{matrix} \right|
\\
&\xlongequal[]{\text{按最后一列展开}}a_n\left| \begin{matrix}
a_1&		b_1&		&		&		&		\\
c_1&		a_2&		b_2&		&		&		\\
&		c_2&		a_3&		\ddots&		&		\\
&		&		\ddots&		\ddots&		\ddots&		\\
&		&		&		\ddots&		a_{n-2}&		b_{n-2}\\
&		&		&		&		c_{n-2}&		a_{n-1}\\
\end{matrix} \right|-b_{n-1}\left| \begin{matrix}
a_1&		b_1&		&		&		&		&		\\
c_1&		a_2&		b_2&		&		&		&		\\
&		c_2&		a_3&		\ddots&		&		&		\\
&		&		\ddots&		\ddots&		\ddots&		&		\\
&		&		&		\ddots&		a_{n-3}&		b_{n-3}&		\\
&		&		&		&		c_{n-3}&		a_{n-2}&		b_{n-2}\\
&		&		&		&		&		0&		c_{n-1}\\
\end{matrix} \right|
\\
&\xlongequal[]{\text{第二项按最后}\mathbf{一行}\text{展开}}a_n\left| \begin{matrix}
a_1&		b_1&		&		&		&		\\
c_1&		a_2&		b_2&		&		&		\\
&		c_2&		a_3&		\ddots&		&		\\
&		&		\ddots&		\ddots&		\ddots&		\\
&		&		&		\ddots&		a_{n-2}&		b_{n-2}\\
&		&		&		&		c_{n-2}&		a_{n-1}\\
\end{matrix} \right|-b_{n-1}c_{n-1}\left| \begin{matrix}
a_1&		b_1&		&		&		&		\\
c_1&		a_2&		b_2&		&		&		\\
&		c_2&		a_3&		\ddots&		&		\\
&		&		\ddots&		\ddots&		\ddots&		\\
&		&		&		\ddots&		a_{n-3}&		b_{n-3}\\
&		&		&		&		c_{n-3}&		a_{n-2}\\
\end{matrix} \right|
\\
&=a_nD_{n-1}-b_{n-1}c_{n-1}D_{n-2}.
\nonumber
\end{align*}
\end{solution}

\begin{proposition}[\hypertarget{大拆分法}{大拆分法}]\label{大拆分法}
设\(t\)是一个参数,
\begin{align*}
|A(t)| = 
\begin{vmatrix}
a_{11}+t & a_{12}+t & \cdots & a_{1n}+t \\
a_{21}+t & a_{22}+t & \cdots & a_{2n}+t \\
\vdots & \vdots & \ddots & \vdots \\
a_{n1}+t & a_{n2}+t & \cdots & a_{nn}+t
\end{vmatrix}
\nonumber
\end{align*}

求证:
\begin{align*}
|A(t)| = |A(0)| + t \sum_{i,j = 1}^{n} A_{ij},
\nonumber
\end{align*}
其中\(A_{ij}\)是\(a_{ij}\)在\(|A(0)|\)中的代数余子式.
\end{proposition}
\begin{note}
大拆分法的想法:
\textbf{将行列式的每一行/列拆分成两行/列},得到
\begin{align*}
|\boldsymbol{A}(t)|=|\boldsymbol{A}(0)|+t\sum_{j=1}^n{|A_j|}.
\text{其中}A_j=\bordermatrix{%
&1 &	\cdots	&	i&	\cdots	&n		\cr
& a_{11}&		\cdots&		t&		\cdots&		a_{1n}\cr
&a_{21}&		\cdots&		t&		\cdots&		a_{2n}\cr
&\vdots&		&		\vdots&		&		\vdots\cr
&a_{n1}&		\cdots&		t&		\cdots&		a_{nn}
},j=1,2,\cdots ,n.
\end{align*}
大拆分法的关键是\textbf{拆分},根据行列式的性质将原行列式拆分成$2^n$个行列式.(不一定需要公共的$t$).不仅要熟悉大拆分法的想法还要记住大拆分法的这个命题.
\end{note}
\begin{remark}
大拆分法后续计算不一定要按行/列展开,拆分的方式一般比较多,只要拆分的方式方便后续计算即可.
\end{remark}
\begin{proof}
将行列式第一列拆成两列再展开得到
\begin{align*}
|\boldsymbol{A}(t)|=\left| \begin{matrix}
a_{11}&		a_{12}+t&		\cdots&		a_{1n}+t\\
a_{21}&		a_{22}+t&		\cdots&		a_{2n}+t\\
\vdots&		\vdots&		&		\vdots\\
a_{n1}&		a_{n2}+t&		\cdots&		a_{nn}+t\\
\end{matrix} \right|+\left| \begin{matrix}
t&		a_{12}+t&		\cdots&		a_{1n}+t\\
t&		a_{22}+t&		\cdots&		a_{2n}+t\\
\vdots&		\vdots&		&		\vdots\\
t&		a_{n2}+t&		\cdots&		a_{nn}+t\\
\end{matrix} \right|.
\nonumber
\end{align*}
将上式右边第二个行列式的第一列乘-1加到后面每一列上,得到
\begin{align*}
\left| \boldsymbol{A} \right|=\left| \begin{matrix}
a_{11}&		a_{12}+t&		\cdots&		a_{1n}+t\\
a_{21}&		a_{22}+t&		\cdots&		a_{2n}+t\\
\vdots&		\vdots&		&		\vdots\\
a_{n1}&		a_{n2}+t&		\cdots&		a_{nn}+t\\
\end{matrix} \right|+\left| \begin{matrix}
t&		a_{12}&		\cdots&		a_{1n}\\
t&		a_{22}&		\cdots&		a_{2n}\\
\vdots&		\vdots&		&		\vdots\\
t&		a_{n2}&		\cdots&		a_{nn}\\
\end{matrix} \right| .
\nonumber
\end{align*}
再对上式右边第一个行列式的第二列拆成两列展开,不断这样做下去就可得到
\begin{gather*}
|\boldsymbol{A}(t)|=\left| \begin{matrix}
a_{11}&		a_{12}&		\cdots&		a_{1n}\\
a_{21}&		a_{22}&		\cdots&		a_{2n}\\
\vdots&		\vdots&		&		\vdots\\
a_{n1}&		a_{n2}&		\cdots&		a_{nn}\\
\end{matrix} \right|+\left| \begin{matrix}
t&		a_{12}&		\cdots&		a_{1n}\\
t&		a_{22}&		\cdots&		a_{2n}\\
\vdots&		\vdots&		&		\vdots\\
t&		a_{n2}&		\cdots&		a_{nn}\\
\end{matrix} \right|+\cdots +\left| \begin{matrix}
a_{11}&		a_{1n}&		\cdots&		t\\
a_{21}&		a_{2n}&		\cdots&		t\\
\vdots&		\vdots&		&		\vdots\\
a_{n1}&		a_{nn}&		\cdots&		t\\
\end{matrix} \right|=|\boldsymbol{A}(0)|+\sum_{j=1}^n{|A_j|}.
\end{gather*}
其中$A_j=\bordermatrix{%
&1 &	\cdots	&	i&	\cdots	&n		\cr
& a_{11}&		\cdots&		t&		\cdots&		a_{1n}\cr
&a_{21}&		\cdots&		t&		\cdots&		a_{2n}\cr
&\vdots&		&		\vdots&		&		\vdots\cr
&a_{n1}&		\cdots&		t&		\cdots&		a_{nn}
}$,$j=1,2,\cdots ,n.$
将$A_j$按第$j$列展开可得
\begin{align*}
A_j=\left| \begin{matrix}
a_{11}&		\cdots&		t&		\cdots&		a_{1n}\\
a_{21}&		\cdots&		t&		\cdots&		a_{2n}\\
\vdots&		&		\vdots&		&		\vdots\\
a_{n1}&		\cdots&		t&		\cdots&		a_{nn}\\
\end{matrix} \right|=t\left( A_{1j}+A_{2j}+\cdots +A_{nj} \right) =t\sum_{i=1}^n{A_{ij}}.
\nonumber
\end{align*}
从而
\begin{align*}
|\boldsymbol{A}(t)|=|\boldsymbol{A}(0)|+\sum_{i=1}^n{A_i}=|\boldsymbol{A}(0)|+t\sum_{i=1}^n{\sum_{i=1}^n{A_{ij}}}=|\boldsymbol{A}(0)|+t\sum_{i,j=1}^n{A_{ij}}.
\end{align*}
\end{proof}

\begin{corollary}[\hypertarget{大拆分法的推广}{推广的大拆分法}]\label{大拆分法的推广}
设
\begin{align*}
|A| = 
\begin{vmatrix}
a_{11} & a_{12} & \cdots & a_{1n} \\
a_{21} & a_{22} & \cdots & a_{2n} \\
\vdots & \vdots & \ddots & \vdots \\
a_{n1} & a_{n2} & \cdots & a_{nn}
\end{vmatrix},
\nonumber
\end{align*}
则
\begin{align*}
|A(t_1,t_2,\cdots,t_n)| = 
\begin{vmatrix}
a_{11}+t_1 & a_{12}+t_2 & \cdots & a_{1n}+t_n \\
a_{21}+t_1 & a_{22}+t_2 & \cdots & a_{2n}+t_n \\
\vdots & \vdots & \ddots & \vdots \\
a_{n1}+t_1 & a_{n2}+t_2 & \cdots & a_{nn}+t_n
\end{vmatrix}
= |A| + \sum_{j = 1}^{n} \left( t_j \sum_{i = 1}^{n} A_{ij} \right).
\nonumber
\end{align*}
\end{corollary}
\begin{note}
记忆这种推广的大拆分法的想法(即\textbf{将行列式的每一行/列拆分成两行/列}).

这里推广的大拆分法的关键也是\textbf{要找到合适的$t_1,t_2,\cdots,t_n$}进行拆分将原行列式拆分成更好处理的形式.
\end{note}
\begin{remark}
大拆分法后续计算不一定要按行/列展开,拆分的方式一般比较多,只要拆分的方式方便后续计算即可.
\end{remark}
\begin{proof}
运用\hyperlink{大拆分法}{大拆分法}的证明方法不难得到.
\end{proof}

\begin{proposition}[\hypertarget{小拆分法}{小拆分法}]\label{小拆分法}
设
\begin{align*}
|A| = 
\begin{vmatrix}
a_{11} & a_{12} & \cdots & a_{1n} \\
a_{21} & a_{22} & \cdots & a_{2n} \\
\vdots & \vdots & \ddots & \vdots \\
a_{n1} & a_{n2} & \cdots & a_{nn}
\end{vmatrix},
\nonumber
\end{align*}
并且$a_{in}$可以拆分成$b_{in}+c_{in}$,$\,\,i=1,2,\cdots,n.$

则
\begin{align*}
\left| \boldsymbol{A} \right|=\left| \begin{matrix}
a_{11}&		a_{12}&		\cdots&		a_{1n}\\
a_{21}&		a_{22}&		\cdots&		a_{2n}\\
\vdots&		\vdots&		&		\vdots\\
a_{n1}&		a_{n2}&		\cdots&		a_{nn}\\
\end{matrix} \right|=\left| \begin{matrix}
a_{11}&		a_{12}&		\cdots&		b_{1n}+c_{1n}\\
a_{21}&		a_{22}&		\cdots&		b_{2n}+c_{2n}\\
\vdots&		\vdots&		&		\vdots\\
a_{n1}&		a_{n2}&		\cdots&		b_{nn}+c_{nn}\\
\end{matrix} \right|=\left| \begin{matrix}
a_{11}&		a_{12}&		\cdots&		b_{1n}\\
a_{21}&		a_{22}&		\cdots&		b_{2n}\\
\vdots&		\vdots&		&		\vdots\\
a_{n1}&		a_{n2}&		\cdots&		b_{nn}\\
\end{matrix} \right|+\left| \begin{matrix}
a_{11}&		a_{12}&		\cdots&		c_{1n}\\
a_{21}&		a_{22}&		\cdots&		c_{2n}\\
\vdots&		\vdots&		&		\vdots\\
a_{n1}&		a_{n2}&		\cdots&		c_{nn}\\
\end{matrix} \right|.
\end{align*}
\end{proposition}
\begin{note}
记忆小拆分法的想法(即\textbf{拆边列/行,再展开得到递推式}).
\end{note}
\begin{remark}
若已知的拆分不是最后一列而是其他的某一行或某一列,则可以通过\hyperref[pro:行列式计算常识]{倒排、旋转、翻转、两行或两列对换}的方法将这一行或一列变成最后一列,再按照上述方法进行拆分即可.

小拆分法后续计算也不一定要按行/列展开,拆分的方式一般比较多,只要拆分的方式方便后续计算即可. 
\end{remark}
\begin{proof}
由行列式的性质可直接得到结论.
\end{proof}

\begin{proposition}[行列式的求导运算]\label{proposition:行列式的求导运算}
设\(f_{ij}(t)\)是可微函数,
\begin{align*}
F(t) = 
\left| \begin{matrix}
f_{11}(t) & f_{12}(t) & \cdots & f_{1n}(t) \\
f_{21}(t) & f_{22}(t) & \cdots & f_{2n}(t) \\
\vdots & \vdots &  & \vdots \\
f_{n1}(t) & f_{n2}(t) & \cdots & f_{nn}(t)
\end{matrix} \right| 
\nonumber
\end{align*}
求证:$\frac{d}{dt}F\left( t \right) =\sum_{j=1}^n{F_j\left( t \right)}$,其中
\begin{align*}
F_{j}(t) = 
\left| \begin{matrix}
f_{11}(t)&		f_{12}(t)&		\cdots&		\frac{d}{dt}f_{1j}(t)&		\cdots&		f_{1n}(t)\\
f_{21}(t)&		f_{22}(t)&		\cdots&		\frac{d}{dt}f_{2j}(t)&		\cdots&		f_{2n}(t)\\
\vdots&		\vdots&		&		\vdots&		&		\vdots\\
f_{n1}(t)&		f_{n2}(t)&		\cdots&		\frac{d}{dt}f_{nj}(t)&		\cdots&		f_{nn}(t)\\
\end{matrix} \right| 
\nonumber
\end{align*}
\end{proposition}
\begin{proof}
{\color{blue}证法一(数学归纳法):}对阶数$n$进行归纳.当$n=1$时结论显然成立.假设$n-1$阶时结论成立,现证$n$阶的情形.

将$F(t)$按第一列展开得
\begin{align*}
F\left( t \right) =f_{11}\left( t \right) A_{11}\left( t \right) +f_{21}\left( t \right) A_{21}\left( t \right) +\cdots +f_{n1}\left( t \right) A_{n1}\left( t \right) .
\nonumber
\end{align*}
其中$A_{i1}(t)$是元素$f_{i1}(t)$的代数余子式.($i=1,2,\cdots,n$)

从而由归纳假设可得
\begin{gather*}
A_{i1}^{\prime}\left( t \right) =\frac{d}{dt}A_{i1}\left( t \right)=\sum_{k=2}^{n}{A_{i1}^{k}(t),i=1,2,\cdots ,n}. 
\\
\text{其中}A_{i1}^{k}(t)=\left| \begin{matrix}
f_{12}\left( t \right)&		\cdots&		\frac{d}{dt}f_{1k}\left( t \right)&		\cdots&		f_{1n}\left( t \right)\\
\vdots&		&		\vdots&		&		\vdots\\
f_{i-1,2}(t)&		\cdots&		\frac{d}{dt}f_{i-1,k}\left( t \right)&		\cdots&		f_{i-1,n}\left( t \right)\\
f_{i+1,2}\left( t \right)&		\cdots&		\frac{d}{dt}f_{i+1,k}(t)&		\cdots&		f_{i+1,n}\left( t \right)\\
\vdots&		&		\vdots&		&		\vdots\\
f_{n2}\left( t \right)&		\cdots&		\frac{d}{dt}f_{nk}\left( t \right)&		\cdots&		f_{nn}\left( t \right)\\
\end{matrix} \right|,k=2,3,\cdots ,n.
\nonumber
\end{gather*}
于是,我们就有
\begin{align*}
\frac{d}{dt}F\left( t \right) &=\frac{d}{dt}\left[ f_{11}\left( t \right) A_{11}\left( t \right) +f_{21}\left( t \right) A_{21}\left( t \right) +\cdots +f_{n1}\left( t \right) A_{n1}\left( t \right) \right] 
\\
&=f_{11}^{\prime}\left( t \right) A_{11}\left( t \right) +f_{21}^{\prime}\left( t \right) A_{21}\left( t \right) +\cdots +f_{n1}^{\prime}\left( t \right) A_{n1}\left( t \right) +f_{11}\left( t \right) A_{11}^{\prime}\left( t \right) +f_{21}\left( t \right) A_{21}^{\prime}\left( t \right) +\cdots +f_{n1}\left( t \right) A_{n1}^{\prime}\left( t \right) 
\\
&=\sum_{i=1}^n{f_{i1}^{\prime}\left( t \right) A_{i1}\left( t \right)}+f_{11}\left( t \right) \sum_{k=2}^{n}{A_{11}^{k}(t)}+f_{21}\left( t \right) \sum_{k=2}^{n}{A_{21}^{k}(t)}+\cdots +f_{n1}\left( t \right) \sum_{k=2}^{n}{A_{n1}^{k}(t)}
\\
&=\sum_{i=1}^n{f_{i1}^{\prime}\left( t \right) A_{i1}\left( t \right)}+\sum_{i=1}^n{\left( f_{i1}\left( t \right) \sum_{k=2}^n{A_{i1}^{k}\left( t \right)} \right)}
\\
&=\sum_{i=1}^n{f_{i1}^{\prime}\left( t \right) A_{i1}\left( t \right)}+\sum_{i=1}^n{f_{i1}\left( t \right) \left( A_{i1}^{2}+A_{i1}^{3}+\cdots +A_{i1}^{n} \right)}
\\
&=\sum_{i=1}^n{f_{i1}^{\prime}\left( t \right) A_{i1}\left( t \right)}+\sum_{i=1}^n{f_{i1}\left( t \right) A_{i1}^{2}}+\sum_{i=1}^n{f_{i1}\left( t \right) A_{i1}^{3}}+\cdots +\sum_{i=1}^n{f_{i1}\left( t \right) A_{i1}^{n}}
\\
&=F_1\left( t \right) +F_2\left( t \right) +F_3\left( t \right) +\cdots +F_n\left( t \right) 
\\
&=\sum_{j=1}^n{F_j\left( t \right)}.
\end{align*}
故由数学归纳法可知结论对任意正整数都成立.

{\color{blue}证法二(行列式的组合定义):}由行列式的组合定义可得
\begin{align*}
F(t)=\sum_{1\le k_1,k_2,\cdots ,k_n\le n}{(}-1)^{\tau (k_1k_2\cdots k_n)}f_{k_11}(t)f_{k_22}(t)\cdots f_{k_nn}(t).
\end{align*}
因此
\begin{align*}
\frac{d}{dt}F(t)&=\sum_{1\le k_1,k_2,\cdots ,k_n\le n}{(}-1)^{\tau (k_1k_2\cdots k_n)}f_{k_{11}}(t)f_{k_{22}}(t)\cdots f_{k_{nn}}(t)
\\
&\quad+\sum_{1\le k_1,k_2,\cdots ,k_n\le n}{(}-1)^{\tau (k_1k_2\cdots k_n)}f_{k_{11}}(t)f\prime_{k_{22}}(t)\cdots f_{k_{nn}}(t)
\\
&\quad+\cdots +\sum_{1\le k_1,k_2,\cdots ,k_n\le n}{(}-1)^{\tau (k_1k_2\cdots k_n)}f_{k_{11}}(t)f_{k_{22}}(t)\cdots f\prime_{k_{nn}}(t)
\\
&=F_1(t)+F_2(t)+\cdots +F_n(t).
\end{align*}
\end{proof}

\begin{proposition}[直接计算两个矩阵和的行列式]\label{proposition:直接计算两个矩阵和的行列式}
设\(A,B\)都是\(n\)阶矩阵,求证:
\begin{align*}
|\boldsymbol{A}+\boldsymbol{B}|=|\boldsymbol{A}|+|\boldsymbol{B}|+\sum_{1\le k\le n-1}{\left( \sum_{\substack{1\le i_1<i_2<\cdots <i_k\le n\\1\le j_1<j_2<\cdots <j_k\le n}}{\boldsymbol{A}\left( \begin{matrix}
i_1&		i_2&		\cdots&		i_k\\
j_1&		j_2&		\cdots&		j_k\\
\end{matrix} \right) \widehat{\boldsymbol{B}}\left( \begin{matrix}
i_1&		i_2&		\cdots&		i_k\\
j_1&		j_2&		\cdots&		j_k\\
\end{matrix} \right)} \right)}.
\end{align*}
其中$\widehat{\boldsymbol{B}}\left( \begin{matrix}
i_1&		i_2&		\cdots&		i_k\\
j_1&		j_2&		\cdots&		j_k\\
\end{matrix} \right)$是$|\boldsymbol{B}|$的$k$阶子式$\boldsymbol{B}\left( \begin{matrix}
i_1&		i_2&		\cdots&		i_k\\
j_1&		j_2&		\cdots&		j_k\\
\end{matrix} \right)$的代数余子式.
\end{proposition}
\begin{note}
当\(\boldsymbol{A}\),\(\boldsymbol{B}\)之一是比较简单的矩阵(例如对角矩阵或秩较小的矩阵)时,可利用这个命题计算$|\boldsymbol{A}+\boldsymbol{B}|$.
\end{note}
\begin{solution}
设\(|\boldsymbol{A}| = |\alpha_1,\alpha_2,\cdots,\alpha_n|\),\(|\boldsymbol{B}| = |\beta_1,\beta_2,\cdots,\beta_n|\),其中\(\alpha_j,\beta_j\)(\(j = 1,2,\cdots,n\))分别是\(\boldsymbol{A}\)和\(\boldsymbol{B}\)的列向量.注意到
\begin{align*}
|\boldsymbol{A} + \boldsymbol{B}| = |\alpha_1 + \beta_1,\alpha_2 + \beta_2,\cdots,\alpha_n + \beta_n|.
\end{align*}
对\(|\boldsymbol{A} + \boldsymbol{B}|\),按列用行列式的性质展开,使每个行列式的每一列或者只含有\(\alpha_j\),或者只含有\(\beta_j\)(即利用大拆分法按列向量将行列式完全拆分开),
则\(|\boldsymbol{A} + \boldsymbol{B}|\)可以表示为\(2^n\)个这样的行列式之和.即(并且单独把\(k = 0,n\)的项分离出来,即将\(|\boldsymbol{A}|\)、\(|\boldsymbol{B}|\)分离出来)
\begin{align*}
&|\boldsymbol{A} + \boldsymbol{B}| = |\alpha_1 + \beta_1,\alpha_2 + \beta_2,\cdots,\alpha_n + \beta_n| 
\\
&=|\boldsymbol{A}|+|\boldsymbol{B}|+\sum_{1\leqslant k\leqslant n-1}{\sum_{1\leq j_1\leq j_2\leq \cdots\leq j_k\leq n}{\begin{array}{c}
\begin{array}{c@{}c@{}c@{}c@{}c@{}c@{}c@{}c@{}c@{}c@{}c@{}}
& 1 & \cdots & j_1 &\cdots &j_2 &\cdots &j_k &\cdots &n \\
\left.\right|
&\beta _1,&\cdots ,&\alpha _{j_1},&\cdots ,&\alpha_{j_2},&\cdots ,&\alpha_{j_k},&\cdots ,&\beta_n& \left|\right.
\end{array}\\
\\
\end{array}}}.
\end{align*}
再对上式右边除\(|\boldsymbol{A}|\)、\(|\boldsymbol{B}|\)外的每个行列式用\(Laplace\)定理按含有\(\boldsymbol{A}\)的列向量的那些列展开得到
\begin{align*}
&|\boldsymbol{A} + \boldsymbol{B}| =|\boldsymbol{A}|+|\boldsymbol{B}|+\sum_{1\leqslant k\leqslant n-1}{\sum_{1\leq j_1\leq j_2\leq \cdots\leq j_k\leq n}{\begin{array}{c}
\begin{array}{c@{}c@{}c@{}c@{}c@{}c@{}c@{}c@{}c@{}c@{}c@{}}
& 1 & \cdots & j_1 &\cdots &j_2 &\cdots &j_k &\cdots &n \\
\left.\right|
&\beta _1,&\cdots ,&\alpha _{j_1},&\cdots ,&\alpha_{j_2},&\cdots ,&\alpha_{j_k},&\cdots ,&\beta_n& \left|\right.
\end{array}\\
\\
\end{array}}}
\\
&= |\boldsymbol{A}| + |\boldsymbol{B}| + \sum_{1\leqslant k\leqslant n - 1}\sum_{1\leqslant j_1,j_2,\cdots,j_k\leqslant n}\sum_{1\leqslant i_1,i_2,\cdots,i_k\leqslant n}\boldsymbol{A}\left(\begin{matrix}
i_1 & i_2 & \cdots & i_k\\
j_1 & j_2 & \cdots & j_k
\end{matrix}\right)\widehat{\boldsymbol{B}}\left(\begin{matrix}
i_1 & i_2 & \cdots & i_k\\
j_1 & j_2 & \cdots & j_k
\end{matrix}\right)
\\
&=|\boldsymbol{A}|+|\boldsymbol{B}|+\sum_{1\le k\le n-1}{\left( \sum_{\substack{1\le i_1<i_2<\cdots <i_k\le n\\1\le j_1<j_2<\cdots <j_k\le n}}{\boldsymbol{A}\left( \begin{matrix}
i_1&		i_2&		\cdots&		i_k\\
j_1&		j_2&		\cdots&		j_k\\
\end{matrix} \right) \widehat{\boldsymbol{B}}\left( \begin{matrix}
i_1&		i_2&		\cdots&		i_k\\
j_1&		j_2&		\cdots&		j_k\\
\end{matrix} \right)} \right)}.
\end{align*}
\end{solution}

\begin{example}
设
\[
f(x)=\left| \begin{matrix}
x-a_{11}&		-a_{12}&		\cdots&		-a_{1n}\\
-a_{21}&		x-a_{22}&		\cdots&		-a_{2n}\\
\vdots&		\vdots&		\ddots&		\vdots\\
-a_{n1}&		-a_{n2}&		\cdots&		x-a_{nn}\\
\end{matrix} \right|,
\]
其中\(x\)是未定元,\(a_{ij}\)是常数.证明:\(f(x)\)是一个最高次项系数为\(1\)的\(n\)次多项式,且其\(n - 1\)次项的系数等于\(-(a_{11}+a_{22}+\cdots + a_{nn})\).
\end{example}
\begin{note}
注意$f(x)$的每行每列除主对角元素外,其他元素均不相同.因此$f(x)$并不是\hyperref["爪"型行列式的推广]{推广的"爪"型行列式}.
\end{note}
\begin{solution}
由行列式的组合定义可知,\(f(x)\)的最高次项出现在组合定义展开式中的单项\((x - a_{11})(x - a_{22})\cdots(x - a_{nn})\)中,且展开式中的其他单项作为\(x\)的多项式其次数小于等于\(n - 2\).因此\(f(x)\)是一个最高次项系数为\(1\)的\(n\)次多项式,且其\(n - 1\)次项的系数等于\(-(a_{11}+a_{22}+\cdots + a_{nn})\).
\end{solution}
\begin{remark}
将这个例题进行推广再结合\hyperref[proposition:直接计算两个矩阵和的行列式]{直接计算两个矩阵和的行列式的结论}可以得到下述推论.
\end{remark}

\begin{corollary}
设\(A=(a_{ij})\)为\(n\)阶方阵,\(x\)为未定元,
\[
f(x)=\vert xI_n - A\vert = 
\begin{vmatrix}
x - a_{11} & -a_{12} & \cdots & -a_{1n} \\
-a_{21} & x - a_{22} & \cdots & -a_{2n} \\
\vdots & \vdots & \ddots & \vdots \\
-a_{n1} & -a_{n2} & \cdots & x - a_{nn}
\end{vmatrix}
\]

证明:\(f(x)=x^n + a_1x^{n - 1}+ \cdots + a_{n - 1}x + a_n\),其中
\[
a_k=(-1)^k \sum_{1\leq i_1 < i_2<\cdots <i_k\leq n} A
\begin{pmatrix}
i_1 & i_2 & \cdots & i_k \\
i_1 & i_2 & \cdots & i_k
\end{pmatrix}, 1\leq k\leq n.
\]
\end{corollary}
\begin{note}
需要注意上述推论中$a_1=-(a_{11}+a_{22}+\cdots+a_{nn}),a_n=\left( -1 \right) ^n\left| \boldsymbol{A} \right|.$
\end{note}
\begin{proof}
注意到 \(xI_{n}\) 非零的 \(n - k\) 阶子式只有 \(n - k\) 阶主子式,并且其值为 \(x^{n - k}\),其余$n-k$阶子式均为零.
记$\widehat{x\boldsymbol{I}_n}\left( \begin{matrix}
i_1&		i_2&		\cdots&		i_k\\
j_1&		j_2&		\cdots&		j_k\\
\end{matrix} \right)$是$x\boldsymbol{I}_n\left( \begin{matrix}
i_1&		i_2&		\cdots&		i_k\\
j_1&		j_2&		\cdots&		j_k\\
\end{matrix} \right)$的代数余子式,则$\widehat{x\boldsymbol{I}_n}\left( \begin{matrix}
i_1&		i_2&		\cdots&		i_k\\
j_1&		j_2&		\cdots&		j_k\\
\end{matrix} \right)$是\(xI_{n}\)非零的 \(n - k\) 阶子式.于是我们有
\begin{align*}
\widehat{x\boldsymbol{I}_n}\left( \begin{matrix}
i_1&		i_2&		\cdots&		i_k\\
j_1&		j_2&		\cdots&		j_k\\
\end{matrix} \right) =x^{n-k}.
\end{align*}
再利用\hyperref[proposition:直接计算两个矩阵和的行列式]{直接计算两个矩阵和的行列式的结论}就可以得到
\begin{align*}
&f(x)=|x\boldsymbol{I}_n-\boldsymbol{A}|=\left| \begin{matrix}
x-a_{11}&		-a_{12}&		\cdots&		-a_{1n}\\
-a_{21}&		x-a_{22}&		\cdots&		-a_{2n}\\
\vdots&		\vdots&		&		\vdots\\
-a_{n1}&		-a_{n2}&		\cdots&		x-a_{nn}\\
\end{matrix} \right|=\left| \left( \begin{matrix}
-a_{11}&		-a_{12}&		\cdots&		-a_{1n}\\
-a_{21}&		-a_{22}&		\cdots&		-a_{2n}\\
\vdots&		\vdots&		&		\vdots\\
-a_{n1}&		-a_{n2}&		\cdots&		-a_{nn}\\
\end{matrix} \right) +\left( \begin{matrix}
x&		0&		\cdots&		0\\
0&		x&		\cdots&		0\\
\vdots&		\vdots&		&		\vdots\\
0&		0&		\cdots&		x\\
\end{matrix} \right) \right|
\\
&=\left| \begin{matrix}
-a_{11}&		-a_{12}&		\cdots&		-a_{1n}\\
-a_{21}&		-a_{22}&		\cdots&		-a_{2n}\\
\vdots&		\vdots&		&		\vdots\\
-a_{n1}&		-a_{n2}&		\cdots&		-a_{nn}\\
\end{matrix} \right|+\left| \begin{matrix}
x&		0&		\cdots&		0\\
0&		x&		\cdots&		0\\
\vdots&		\vdots&		&		\vdots\\
0&		0&		\cdots&		x\\
\end{matrix} \right|+\sum_{1\le k\le n-1}{\sum_{\substack{1\le i_1,i_2,\cdots ,i_k\le n\\
1\le j_1,j_2,\cdots ,j_k\le n\\}
}{\left( -\boldsymbol{A} \right) \left( \begin{matrix}
i_1&		i_2&		\cdots&		i_k\\
j_1&		j_2&		\cdots&		j_k\\
\end{matrix} \right) \widehat{x\boldsymbol{I}_n}\left( \begin{matrix}
i_1&		i_2&		\cdots&		i_k\\
j_1&		j_2&		\cdots&		j_k\\
\end{matrix} \right)}}
\\
&=\left( -1 \right) ^n\left| \boldsymbol{A} \right|+x^n+\sum_{1\le k\le n-1}{\sum_{1\le i_1,i_2,\cdots ,i_k\le n}{\left( -1 \right) ^k\boldsymbol{A}\left( \begin{matrix}
i_1&		i_2&		\cdots&		i_k\\
i_1&		i_2&		\cdots&		i_k\\
\end{matrix} \right) \widehat{x\boldsymbol{I}_n}\left( \begin{matrix}
i_1&		i_2&		\cdots&		i_k\\
i_1&		i_2&		\cdots&		i_k\\
\end{matrix} \right)}}
\\
&=x^n+\sum_{1\le k\le n-1}{\left( -1 \right) ^k\sum_{1\le i_1,i_2,\cdots ,i_k\le n}{\boldsymbol{A}\left( \begin{matrix}
i_1&		i_2&		\cdots&		i_k\\
i_1&		i_2&		\cdots&		i_k\\
\end{matrix} \right) \cdot x^{n-k}}}+\left( -1 \right) ^n\left| \boldsymbol{A} \right|
\\
&=x^n+\sum_{1\le k\le n-1}{x^{n-k}\left( -1 \right) ^k\sum_{1\le i_1,i_2,\cdots ,i_k\le n}{\boldsymbol{A}\left( \begin{matrix}
i_1&		i_2&		\cdots&		i_k\\
i_1&		i_2&		\cdots&		i_k\\
\end{matrix} \right)}}+\left( -1 \right) ^n\left| \boldsymbol{A} \right|.
\end{align*}
因此
\(f(x)=x^n + a_1x^{n - 1}+ \cdots + a_{n - 1}x + a_n\),其中
\[
a_k=(-1)^k \sum_{1\leq i_1 < i_2<\cdots <i_k\leq n} A
\begin{pmatrix}
i_1 & i_2 & \cdots & i_k \\
i_1 & i_2 & \cdots & i_k
\end{pmatrix}, 1\leq k\leq n.
\]
\end{proof}

\begin{proposition}
设$f_k\left( x \right) =x^k+a_{k1}x^{k-1}+a_{k2}x^{k-2}+\cdots +a_{kk}
$,求下列行列式的值:
\begin{align*}
\left| \begin{matrix}
1&		f_1(x_1)&		f_2(x_1)&		\cdots&		f_{n-1}(x_1)\\
1&		f_1(x_2)&		f_2(x_2)&		\cdots&		f_{n-1}(x_2)\\
\vdots&		\vdots&		\vdots&		&		\vdots\\
1&		f_1(x_n)&		f_2(x_n)&		\cdots&		f_{n-1}(x_n)\\
\end{matrix} \right|.
\end{align*}
\end{proposition}
\begin{note}
知道这类行列式化简的操作即可.以后这种行列式化简操作不再作额外说明.
\end{note}
\begin{solution}
利用行列式的性质可得
\begin{align*}
&\left| \begin{matrix}
1&		f_1(x_1)&		f_2(x_1)&		\cdots&		f_{n-1}(x_1)\\
1&		f_1(x_2)&		f_2(x_2)&		\cdots&		f_{n-1}(x_2)\\
\vdots&		\vdots&		\vdots&		&		\vdots\\
1&		f_1(x_n)&		f_2(x_n)&		\cdots&		f_{n-1}(x_n)\\
\end{matrix} \right|
=\left| \begin{matrix}
1&		x_1+a_{11}&		x_{1}^{2}+a_{21}x_1+a_{22}&		\cdots&		x_{1}^{n-1}+a_{n-1,1}x_{1}^{n-2}+\cdots +a_{n-1,n-2}x_1+a_{n-1,n-1}\\
1&		x_2+a_{11}&		x_{2}^{2}+a_{21}x_2+a_{22}&		\cdots&		x_{2}^{n-1}+a_{n-1,1}x_{2}^{n-2}+\cdots +a_{n-1,n-2}x_2+a_{n-1,n-1}\\
\vdots&		\vdots&		\vdots&		&		\vdots\\
1&		x_n+a_{11}&		x_{n}^{2}+a_{21}x_n+a_{22}&		\cdots&		x_{n}^{n-1}+a_{n-1,1}x_{n}^{n-2}+\cdots +a_{n-1,n-2}x_n+a_{n-1,n-1}\\
\end{matrix} \right|
\\
&\xlongequal[\begin{array}{c}
\cdots\\
-a_{i,i-\left( n-3 \right)}j_{n-2}+j_{i+1},i=n-2,n-1\\
-a_{n-1,1}j_{n-1}+j_n\\
\end{array}]{\begin{array}{c}
-a_{ii}j_1+j_{i+1},i=1,2,\cdots n-1\\
-a_{i,i-1}j_2+j_{i+1},i=2,3,\cdots ,n-1\\
\end{array}}\left| \begin{matrix}
1&		x_1&		x_{1}^{2}&		\cdots&		x_{1}^{n-1}\\
1&		x_2&		x_{2}^{2}&		\cdots&		x_{2}^{n-1}\\
\vdots&		\vdots&		\vdots&		&		\vdots\\
1&		x_n&		x_{n}^{2}&		\cdots&		x_{n}^{n-1}\\
\end{matrix} \right|=\prod_{1\le i<j\le n}{\left( x_j-x_i \right) .}
\end{align*}
\end{solution}

\begin{proposition}[\hypertarget{多项式根的有限性}{多项式根的有限性}]\label{proposition:多项式根的有限性}
设多项式
\[
f(x)=a_nx^n + a_{n - 1}x^{n - 1}+\cdots+a_1x + a_0
\]
若\(f(x)\)有\(n + 1\)个不同的根\(b_1,b_2,\cdots,b_{n+1}\),即\(f(b_1)=f(b_2)=\cdots=f(b_{n+1})=0\),
求证:\(f(x)\)是零多项式,即\(a_n=a_{n - 1}=\cdots=a_1=a_0 = 0\).
\end{proposition}
\begin{proof}
由\(f(b_1)=f(b_2)=\cdots=f(b_{n+1})=0\),可知$x_0=a_0,x_1=a_1,\cdots ,x_{n-1}=a_{n-1},x_n=a_n$是下列线性方程组的解:
\begin{align*}
\left\{ \begin{aligned}
&x_0+b_1x_1+\cdots +b_{1}^{n-1}x_{n-1}+b_{1}^{n}x_n=0,\\
&x_0+b_2x_1+\cdots +b_{2}^{n-1}x_{n-1}+b_{2}^{n}x_n=0,\\
&\qquad \qquad \qquad \cdots \cdots \cdots \cdots\\
&x_0+b_{n+1}x_1+\cdots +b_{n+1}^{n-1}x_{n-1}+b_{n+1}^{n}x_n=0.\\
\end{aligned} \right. 
\end{align*}
上述线性方程组的系数行列式是一个Vandermode行列式,由于$b_1,b_2,\cdots,b_{n+1}$互不相同,所以系数行列式不等于零.由Crammer法则可知上述方程组只有零解.即有$a_n=a_{n - 1}=\cdots=a_1=a_0 = 0$.
\end{proof}

\begin{proposition}[\hypertarget{Cauchy行列式}{Cauchy行列式}]\label{Cauchy行列式}
计算$n$阶行列式:
\begin{gather}
|\boldsymbol{A}|=\left| \begin{matrix}
(a_1+b_1)^{-1}&		(a_1+b_2)^{-1}&		\cdots&		(a_1+b_n)^{-1}\\
(a_2+b_1)^{-1}&		(a_2+b_2)^{-1}&		\cdots&		(a_2+b_n)^{-1}\\
\vdots&		\vdots&		&		\vdots\\
(a_n+b_1)^{-1}&		(a_n+b_2)^{-1}&		\cdots&		(a_n+b_n)^{-1}\\
\end{matrix} \right|.
\nonumber
\end{gather}
\end{proposition}
\begin{note}
需要记忆$Cauchy$行列式的计算方法.

1.分式分母有公共部分可以作差,得到的分子会变得相对简便.

2.行列式内行列做加减一般都是加减同一行(或列).但是在\hyperlink{循环行列式}{循环行列式}中,我们一般采取相邻两行(或列)相加减的方法.
\end{note}
\begin{solution}
\begin{align*}
&|\boldsymbol{A}|=\left| \begin{matrix}
\frac{1}{a_1+b_1}&		\frac{1}{a_1+b_2}&		\cdots&		\frac{1}{a_1+b_n}\\
\frac{1}{a_2+b_1}&		\frac{1}{a_2+b_2}&		\cdots&		\frac{1}{a_2+b_n}\\
\vdots&		\vdots&		&		\vdots\\
\frac{1}{a_n+b_1}&		\frac{1}{a_n+b_2}&		\cdots&		\frac{1}{a_n+b_n}\\
\end{matrix} \right|
\\
&\xlongequal[i=n-1,\cdots ,1]{-j_n+j_i}\left| \begin{matrix}
\frac{b_n-b_1}{\left( a_1+b_1 \right) \left( a_1+b_n \right)}&		\frac{b_n-b_2}{\left( a_1+b_2 \right) \left( a_1+b_n \right)}&		\cdots&		\frac{b_n-b_{n-1}}{\left( a_1+b_{n-1} \right) \left( a_1+b_n \right)}&		\frac{1}{a_1+b_n}\\
\frac{b_n-b_1}{\left( a_2+b_1 \right) \left( a_2+b_n \right)}&		\frac{b_n-b_2}{\left( a_2+b_2 \right) \left( a_2+b_n \right)}&		\cdots&		\frac{b_n-b_{n-1}}{\left( a_1+b_{n-1} \right) \left( a_2+b_n \right)}&		\frac{1}{a_2+b_n}\\
\vdots&		\vdots&		&		\vdots&		\vdots\\
\frac{b_n-b_1}{\left( a_n+b_1 \right) \left( a_n+b_n \right)}&		\frac{b_n-b_2}{\left( a_n+b_2 \right) \left( a_n+b_n \right)}&		\cdots&		\frac{b_n-b_{n-1}}{\left( a_1+b_{n-1} \right) \left( a_n+b_n \right)}&		\frac{1}{a_n+b_n}\\
\end{matrix} \right|
\\
&=\frac{\prod\limits_{i=1}^{n-1}{\left( b_n-b_i \right)}}{\prod\limits_{j=1}^n{\left( a_j+b_n \right)}}\left| \begin{matrix}
\frac{1}{a_1+b_1}&		\frac{1}{a_1+b_2}&		\cdots&		\frac{1}{a_1+b_{n-1}}&		1\\
\frac{1}{a_2+b_1}&		\frac{1}{a_2+b_2}&		\cdots&		\frac{1}{a_2+b_{n-1}}&		1\\
\vdots&		\vdots&		&		\vdots&		\vdots\\
\frac{1}{a_n+b_1}&		\frac{1}{a_n+b_2}&		\cdots&		\frac{1}{a_n+b_{n-1}}&		1\\
\end{matrix} \right|
\\
&\xlongequal[i=n-1,\cdots ,1]{-r_n+r_i}\frac{\prod\limits_{i=1}^{n-1}{\left( b_n-b_i \right)}}{\prod\limits_{j=1}^n{\left( a_j+b_n \right)}}\left| \begin{matrix}
\frac{a_n-a_1}{\left( a_1+b_1 \right) \left( a_n+b_1 \right)}&		\frac{a_n-a_1}{\left( a_1+b_2 \right) \left( a_n+b_2 \right)}&		\cdots&		\frac{a_n-a_1}{\left( a_1+b_{n-1} \right) \left( a_n+b_{n-1} \right)}&		0\\
\frac{a_n-a_2}{\left( a_2+b_1 \right) \left( a_n+b_1 \right)}&		\frac{a_n-a_2}{\left( a_1+b_2 \right) \left( a_n+b_2 \right)}&		\cdots&		\frac{a_n-a_2}{\left( a_1+b_{n-1} \right) \left( a_n+b_{n-1} \right)}&		0\\
\vdots&		\vdots&		&		\vdots&		\vdots\\
\frac{a_n-a_{n-1}}{\left( a_{n-1}+b_1 \right) \left( a_n+b_1 \right)}&		\frac{a_n-a_{n-1}}{\left( a_{n-1}+b_2 \right) \left( a_n+b_2 \right)}&		\cdots&		\frac{a_n-a_{n-1}}{\left( a_{n-1}+b_{n-1} \right) \left( a_n+b_{n-1} \right)}&		0\\
\frac{1}{a_n+b_1}&		\frac{1}{a_n+b_2}&		\cdots&		\frac{1}{a_n+b_{n-1}}&		1\\
\end{matrix} \right|
\\
&=\frac{\prod\limits_{i=1}^{n-1}{\left( b_n-b_i \right)}}{\prod\limits_{j=1}^n{\left( a_j+b_n \right)}}\cdot \frac{\prod\limits_{i=1}^{n-1}{\left( a_n-a_i \right)}}{\prod\limits_{k=1}^{n-1}{\left( a_n+b_k \right)}}\left| \begin{matrix}
\frac{1}{a_1+b_1}&		\frac{1}{a_1+b_2}&		\cdots&		\frac{1}{a_1+b_{n-1}}&		0\\
\frac{1}{a_2+b_1}&		\frac{1}{a_2+b_2}&		\cdots&		\frac{1}{a_2+b_{n-1}}&		0\\
\vdots&		\vdots&		&		\vdots&		\vdots\\
\frac{1}{a_{n-1}+b_1}&		\frac{1}{a_{n-1}+b_2}&		\cdots&		\frac{1}{a_{n-1}+b_{n-1}}&		0\\
1&		1&		\cdots&		1&		1\\
\end{matrix} \right|
\\
&\xlongequal[]{\text{按最后一列展开}}\frac{\prod\limits_{i=1}^{n-1}{\left( b_n-b_i \right) \left( a_n-a_i \right)}}{\prod\limits_{j=1}^n{\left( a_j+b_n \right) \prod\limits_{k=1}^{n-1}{\left( a_n+b_k \right)}}}\left| \begin{matrix}
\frac{1}{a_1+b_1}&		\frac{1}{a_1+b_2}&		\cdots&		\frac{1}{a_1+b_{n-1}}\\
\frac{1}{a_2+b_1}&		\frac{1}{a_2+b_2}&		\cdots&		\frac{1}{a_2+b_{n-1}}\\
\vdots&		\vdots&		&		\vdots\\
\frac{1}{a_{n-1}+b_1}&		\frac{1}{a_{n-1}+b_2}&		\cdots&		\frac{1}{a_{n-1}+b_{n-1}}\\
\end{matrix} \right|
\\
&=\frac{\prod\limits_{i=1}^{n-1}{\left( b_n-b_i \right) \left( a_n-a_i \right)}}{\prod\limits_{j=1}^n{\left( a_j+b_n \right) \prod\limits_{k=1}^{n-1}{\left( a_n+b_k \right)}}}\cdot D_{n-1}.
\nonumber
\end{align*}
不断递推下去即得
\begin{align*}
&D_n=\frac{\prod\limits_{i=1}^{n-1}{\left( b_n-b_i \right) \left( a_n-a_i \right)}}{\prod\limits_{j=1}^n{\left( a_j+b_n \right) \prod\limits_{k=1}^{n-1}{\left( a_n+b_k \right)}}}\cdot D_{n-1}=\frac{\prod\limits_{i=1}^{n-1}{\left( b_n-b_i \right) \left( a_n-a_i \right)}}{\prod\limits_{j=1}^n{\left( a_j+b_n \right) \prod\limits_{k=1}^{n-1}{\left( a_n+b_k \right)}}}\cdot \frac{\prod\limits_{i=1}^{n-2}{\left( b_{n-1}-b_i \right) \left( a_{n-1}-a_i \right)}}{\prod\limits_{j=1}^{n-1}{\left( a_j+b_{n-1} \right) \prod\limits_{k=1}^{n-2}{\left( a_{n-1}+b_k \right)}}}\cdot D_{n-2}
\\
&=\cdots =\frac{\prod\limits_{i=1}^{n-1}{\left( b_n-b_i \right) \left( a_n-a_i \right)}}{\prod\limits_{j=1}^n{\left( a_j+b_n \right) \prod\limits_{k=1}^{n-1}{\left( a_n+b_k \right)}}}\cdot \frac{\prod\limits_{i=1}^{n-2}{\left( b_{n-1}-b_i \right) \left( a_{n-1}-a_i \right)}}{\prod\limits_{j=1}^{n-1}{\left( a_j+b_{n-1} \right) \prod\limits_{k=1}^{n-2}{\left( a_{n-1}+b_k \right)}}}\cdots \cdots \frac{\prod\limits_{i=1}^2{\left( b_3-b_i \right) \left( a_3-a_i \right)}}{\prod\limits_{j=1}^3{\left( a_j+b_3 \right) \prod\limits_{k=1}^2{\left( a_3+b_k \right)}}}\cdot D_2
\\
&=\frac{\prod\limits_{i=1}^{n-1}{\left( b_n-b_i \right) \left( a_n-a_i \right)}}{\prod\limits_{j=1}^n{\left( a_j+b_n \right) \prod\limits_{k=1}^{n-1}{\left( a_n+b_k \right)}}}\cdot \frac{\prod\limits_{i=1}^{n-2}{\left( b_{n-1}-b_i \right) \left( a_{n-1}-a_i \right)}}{\prod\limits_{j=1}^{n-1}{\left( a_j+b_{n-1} \right) \prod\limits_{k=1}^{n-2}{\left( a_{n-1}+b_k \right)}}}\cdots 
\frac{\prod\limits_{i=1}^2{\left( b_3-b_i \right) \left( a_3-a_i \right)}}{\prod\limits_{j=1}^3{\left( a_j+b_3 \right) \prod\limits_{k=1}^2{\left( a_3+b_k \right)}}}\cdot \frac{\left( b_2-b_1 \right) \left( a_2-a_1 \right)}{\prod\limits_{j=1}^2{\left( a_j+b_2 \right) \left( a_2+b_1 \right)}}\cdot D_1
\\
&=\frac{\prod\limits_{i=1}^{n-1}{\left( b_n-b_i \right) \left( a_n-a_i \right)}}{\prod\limits_{j=1}^n{\left( a_j+b_n \right) \prod\limits_{k=1}^{n-1}{\left( a_n+b_k \right)}}}\cdot \frac{\prod\limits_{i=1}^{n-2}{\left( b_{n-1}-b_i \right) \left( a_{n-1}-a_i \right)}}{\prod\limits_{j=1}^{n-1}{\left( a_j+b_{n-1} \right) \prod\limits_{k=1}^{n-2}{\left( a_{n-1}+b_k \right)}}}\cdots 
\frac{\prod\limits_{i=1}^2{\left( b_3-b_i \right) \left( a_3-a_i \right)}}{\prod\limits_{j=1}^3{\left( a_j+b_3 \right) \prod\limits_{k=1}^2{\left( a_3+b_k \right)}}}\cdot \frac{\left( b_2-b_1 \right) \left( a_2-a_1 \right)}{\prod\limits_{j=1}^2{\left( a_j+b_2 \right) \left( a_2+b_1 \right)}}\cdot \frac{1}{a_1+b_1}
\\
&=\frac{\prod\limits_{1\le i<j\le n}{(a_j}-a_i)(b_j-b_i)}{\prod\limits_{1\le i\le j\le n}{(a_i}+b_j)\prod\limits_{1\le j<i\le n}{(a_i}+b_j)}=\frac{\prod\limits_{1\le i<j\le n}{(a_j}-a_i)(b_j-b_i)}{\prod\limits_{i,j=1}^n{(a_i}+b_j)}.
\nonumber
\end{align*}
\end{solution}

\begin{proposition}\label{proposition:Vandermode行列式的"卷积"形式}
计算下列行列式的值:

\[
|\boldsymbol{A}|=\left| \begin{matrix}
a_{1}^{n-1}&		a_{1}^{n-2}b_1&		\cdots&		a_1b_{1}^{n-2}&		b_{1}^{n-1}\\
a_{2}^{n-1}&		a_{2}^{n-2}b_2&		\cdots&		a_2b_{2}^{n-2}&		b_{2}^{n-1}\\
\vdots&		\vdots&		&		\vdots&		\vdots\\
a_{n}^{n-1}&		a_{n}^{n-2}b_n&		\cdots&		a_nb_{n}^{n-2}&		b_{n}^{n-1}\\
\end{matrix} \right|.
\]
\end{proposition}
\begin{solution}
若所有的$a_i(i=1,2,\cdots,n)$都不为0,则有
\begin{align*}
|\boldsymbol{A}|&=\left| \begin{matrix}
a_{1}^{n-1}&		a_{1}^{n-2}b_1&		\cdots&		a_1b_{1}^{n-2}&		b_{1}^{n-1}\\
a_{2}^{n-1}&		a_{2}^{n-2}b_2&		\cdots&		a_2b_{2}^{n-2}&		b_{2}^{n-1}\\
\vdots&		\vdots&		&		\vdots&		\vdots\\
a_{n}^{n-1}&		a_{n}^{n-2}b_n&		\cdots&		a_nb_{n}^{n-2}&		b_{n}^{n-1}\\
\end{matrix} \right|=\prod_{i=1}^n{a_{i}^{n-1}}\left| \begin{matrix}
1&		\frac{b_1}{a_1}&		\cdots&		\frac{b_{1}^{n-2}}{a_{1}^{n-2}}&		\frac{b_{1}^{n-1}}{a_{1}^{n-1}}\\
1&		\frac{b_2}{a_2}&		\cdots&		\frac{b_{2}^{n-2}}{a_{2}^{n-2}}&		\frac{b_{2}^{n-1}}{a_{2}^{n-1}}\\
\vdots&		\vdots&		&		\vdots&		\vdots\\
1&		\frac{b_n}{a_n}&		\cdots&		\frac{b_{n}^{n-2}}{a_{n}^{n-2}}&		\frac{b_{n}^{n-2}}{a_{n}^{n-2}}\\
\end{matrix} \right|
\\
&=\prod_{i=1}^n{a_{i}^{n-1}}\prod_{1\le i<j\le n}{\left( \frac{b_j}{a_j}-\frac{b_i}{a_i} \right)}=\prod_{i=1}^n{a_{i}^{n-1}}\prod_{1\le i<j\le n}{\frac{a_ib_j-a_jb_i}{a_ja_i}}\hyperlink{连乘号计算小结论(1)}{=}\prod_{1\le i<j\le n}{(a_ib_j-a_jb_i)}.
\end{align*}
若只有一个$a_i$为0,则将原行列式按第$i$行展开得到具有相同类型的$n-1$阶行列式
\begin{align*}
|\boldsymbol{A}|&=\left| \begin{matrix}
a_{1}^{n-1}&		a_{1}^{n-2}b_1&		\cdots&		a_1b_{1}^{n-2}&		b_{1}^{n-1}\\
a_{2}^{n-1}&		a_{2}^{n-2}b_2&		\cdots&		a_2b_{2}^{n-2}&		b_{2}^{n-1}\\
\vdots&		\vdots&		&		\vdots&		\vdots\\
a_{i}^{n-1}&		a_{i}^{n-2}b_i&		\cdots&		a_ib_{i}^{n-2}&		b_{i}^{n-1}\\
\vdots&		\vdots&		&		\vdots&		\vdots\\
a_{n}^{n-1}&		a_{n}^{n-2}b_n&		\cdots&		a_nb_{n}^{n-2}&		b_{n}^{n-1}\\
\end{matrix} \right|=\left| \begin{matrix}
a_{1}^{n-1}&		a_{1}^{n-2}b_1&		\cdots&		a_1b_{1}^{n-2}&		b_{1}^{n-1}\\
a_{2}^{n-1}&		a_{2}^{n-2}b_2&		\cdots&		a_2b_{2}^{n-2}&		b_{2}^{n-1}\\
\vdots&		\vdots&		&		\vdots&		\vdots\\
0&		0&		\cdots&		0&		b_{i}^{n-1}\\
\vdots&		\vdots&		&		\vdots&		\vdots\\
a_{n}^{n-1}&		a_{n}^{n-2}b_n&		\cdots&		a_nb_{n}^{n-2}&		b_{n}^{n-1}\\
\end{matrix} \right|
\\
&\xlongequal{\text{按第}i\text{行展开}}\left( -1 \right) ^{n+i}b_{i}^{n-1}\left| \begin{matrix}
a_{1}^{n-1}&		a_{1}^{n-2}b_1&		\cdots&		a_1b_{1}^{n-2}\\
a_{2}^{n-1}&		a_{2}^{n-2}b_2&		\cdots&		a_2b_{2}^{n-2}\\
\vdots&		\vdots&		&		\vdots\\
a_{i-1}^{n-1}&		a_{i-1}^{n-2}b_{i-1}&		\cdots&		a_{i-1}b_{i-1}^{n-2}\\
a_{i+!}^{n-1}&		a_{i+1}^{n-2}b_{i+1}&		\cdots&		a_{i+1}b_{i+1}^{n-2}\\
\vdots&		\vdots&		&		\vdots\\
a_{n}^{n-1}&		a_{n}^{n-2}b_n&		\cdots&		a_nb_{n}^{n-2}\\
\end{matrix} \right|.
\end{align*}
此时同理可得
\begin{align*}
&|\boldsymbol{A}|=\left( -1 \right) ^{n+i}b_{i}^{n-1}\left| \begin{matrix}
a_{1}^{n-1}&		a_{1}^{n-2}b_1&		\cdots&		a_1b_{1}^{n-2}\\
a_{2}^{n-1}&		a_{2}^{n-2}b_2&		\cdots&		a_2b_{2}^{n-2}\\
\vdots&		\vdots&		&		\vdots\\
a_{i-1}^{n-1}&		a_{i-1}^{n-2}b_{i-1}&		\cdots&		a_{i-1}b_{i-1}^{n-2}\\
a_{i+1}^{n-1}&		a_{i+1}^{n-2}b_{i+1}&		\cdots&		a_{i+1}b_{i+1}^{n-2}\\
\vdots&		\vdots&		&		\vdots\\
a_{n}^{n-1}&		a_{n}^{n-2}b_n&		\cdots&		a_nb_{n}^{n-2}\\
\end{matrix} \right|=\left( -1 \right) ^{n+i}b_{i}^{n-1}\prod_{\substack{1\le k\le n\\
k\ne i\\}}{a_{k}^{n-1}}\left| \begin{matrix}
1&		\frac{b_1}{a_1}&		\cdots&		\frac{b_{1}^{n-2}}{a_{1}^{n-2}}\\
1&		\frac{b_2}{a_2}&		\cdots&		\frac{b_{2}^{n-2}}{a_{2}^{n-2}}\\
\vdots&		\vdots&		&		\vdots\\
1&		\frac{b_{i-1}}{a_{i-1}}&		\cdots&		\frac{b_{i-1}^{n-2}}{a_{i-1}^{n-2}}\\
1&		\frac{b_{i+1}}{a_{i+1}}&		\cdots&		\frac{b_{i+1}^{n-2}}{a_{i+1}^{n-2}}\\
\vdots&		\vdots&		&		\vdots\\
1&		\frac{b_n}{a_n}&		\cdots&		\frac{b_{n}^{n-2}}{a_{n}^{n-2}}\\
\end{matrix} \right|
\\
&=\left( -1 \right) ^{n+i}b_{i}^{n-1}\prod_{\substack{1\le k\le n\\
k\ne i\\}}{a_{k}^{n-1}}\prod_{\substack{1\le k<l\le n\\
k,l\ne i\\}}{\left( \frac{b_l}{a_l}-\frac{b_k}{a_k} \right)}=\left( -1 \right) ^{n+i}b_{i}^{n-1}\prod_{\substack{
1\le k\le n\\
k\ne i\\
}}{a_{k}^{n-1}}\prod_{\substack{
1\le k<l\le n\\
k,l\ne i\\
}}{\frac{a_kb_l-a_lb_k}{a_ka_l}}
\\
&\hyperlink{连乘号计算小结论(2)}{=}\left( -1 \right) ^{n+i}b_{i}^{n-1}\prod_{\substack{
1\le k\le n\\
k\ne i\\
}}{a_k}\cdot \prod_{\substack{
1\le k<l\le n\\
k,l\ne i\\
}}{\left( a_kb_l-a_lb_k \right)}=\left( -1 \right) ^{n-i}b_{i}^{n-1}\prod_{\substack{
1\le k\le n\\
k\ne i\\
}}{a_k}\cdot \prod_{\substack{
1\le k<l\le n\\
k,l\ne i\\
}}{\left( a_kb_l-a_lb_k \right)}
\\
&=\prod_{1\le k<i}{a_kb_i}\prod_{i<l\le n}{\left( -a_lb_i \right)}\cdot \prod_{\substack{
1\le k<l\le n\\
k,l\ne i\\
}}{\left( a_kb_l-a_lb_k \right)}
\\
&=\prod_{1\le k<l\le n}{\left( a_kb_l-a_lb_k \right)}.\left( a_i=0 \right).
\end{align*}
若至少有两个$a_i=a_j=0$,则第$i$行与第$j$行成比例,因此行列式的值等于0.经过计算发现,后面两种情形的答案都可以统一到第一种情形的答案.

综上所述,$|\boldsymbol{A}|=\prod_{1\le i<j\le n}{(a_ib_j-a_jb_i)}.$

\end{solution}
\begin{conclusion}\label{连乘号计算小技巧1}
\textbf{连乘号计算小结论:}

\hypertarget{连乘号计算小结论(1)}{(1)}$\prod_{1\le i<j\le n}{a_ia_j}=\prod_{i=1}^n{a_{i}^{n-1}}.$
\begin{align*}
\text{证明:}&\prod_{1\le i<j\le n}{a_ia_j}=\underset{n-1\text{组}}{\underbrace{a_2a_1\cdot a_3a_2a_3a_1\cdot a_4a_3a_4a_2a_4a_1\cdots \cdots \overset{k-1\text{对}}{\overbrace{a_ka_{k-1}a_ka_{k-2}\cdots a_ka_1}}\cdots \cdots \overset{n-1\text{对}}{\overbrace{a_na_{n-1}a_na_{n-2}\cdots a_na_1}}}}
\\
&\xlongequal{\text{从左往右按组计数}}a_{1}^{n-1}a_{2}^{1+n-2}a_{3}^{2+n-3}a_{4}^{3+n-4}\cdots a_{k}^{k-1+n-k}\cdots a_{n}^{n-1}=\prod_{i=1}^n{a_{i}^{n-1}}.
\end{align*}
\hypertarget{连乘号计算小结论(2)}{(2)}$\prod_{\substack{1\le i<j\le n\\i,j\ne k}}{a_ia_j}=\prod_{\substack{
1\le i\le n\\
i\ne k\\}}{a_{i}^{n-2}}$,其中$k\in [1,n]\cap \mathbb{N_+}$.

\begin{align*}
\text{证明:}&\prod_{\substack{
1\le i<j\le n\\
i,j\ne k\\
}}{a_ia_j}=\underset{n-2\text{组}}{\underbrace{a_2a_1\cdot a_3a_2a_3a_1\cdots \cdots \overset{k-2\text{对}}{\overbrace{a_{k-1}a_{k-2}\cdots a_{k-1}a_1}}\cdot \overset{k-1\text{对}}{\overbrace{a_{k+1}a_{k-1}\cdots a_{k+1}a_1}}\cdots \cdots \overset{n-2\text{对}}{\overbrace{a_na_{n-1}\cdots a_na_{k+1}a_na_{k-1}\cdots a_na_1}}}}
\\
&\xlongequal{\text{从左往右按组计数}}a_{1}^{n-2}a_{2}^{1+n-3}a_{3}^{2+n-4}a_{4}^{3+n-4}\cdots a_{k-1}^{k-2+n-k}a_{k+1}^{k-1+n-k-1}\cdots a_{n}^{n-2}=\prod_{\substack{
1\le i\le n\\
i\ne k\\}}{a_{i}^{n-2}}.
\end{align*}
注意:从第$k-1$组开始,后面每组都比原来少一对(后面每组均缺少原本含$a_k$的那一对).
\end{conclusion}

\begin{proposition}[友矩阵的特征多项式/行列式]\label{pro:友矩阵的特征多项式/行列式}
\begin{equation}
\boldsymbol{A}=\left( \begin{matrix}
0&		0&		0&		\cdots&		0&		-a_0\\
1&		0&		0&		\cdots&		0&		-a_1\\
0&		1&		0&		\cdots&		0&		-a_2\\
\vdots&		\vdots&		\vdots&		&		\vdots&		\vdots\\
0&		0&		0&		\cdots&		0&		-a_{n-2}\\
0&		0&		0&		\cdots&		1&		-a_{n-1}\\
\end{matrix} \right) 
\nonumber
\end{equation}
称为多项式$f\left( x \right) =x^n+a_{n-1}x^{n-1}+\cdots +a_1x+a_0$的
\textbf{友矩阵},则$\boldsymbol{A}$的特征多项式
\begin{equation}
|xE-\boldsymbol{A}|=\left| \begin{matrix}
x&		0&		0&		\cdots&		0&		a_0\\
-1&		x&		0&		\cdots&		0&		a_1\\
0&		-1&		x&		\cdots&		0&		a_2\\
\vdots&		\vdots&		\vdots&		&		\vdots&		\vdots\\
0&		0&		0&		\cdots&		x&		a_{n-2}\\
0&		0&		0&		\cdots&		-1&		x+a_{n-1}\\
\end{matrix} \right|=f(x).
\nonumber
\end{equation}
\end{proposition}
\begin{note}
只要从最后一行开始,每一行的$x$倍加到上一行就可以得到结果.
\end{note}
\begin{proof}
\begin{align*}
&|xE-\boldsymbol{A}|=\left| \begin{matrix}
x&		0&		0&		\cdots&		0&		a_0\\
-1&		x&		0&		\cdots&		0&		a_1\\
0&		-1&		x&		\cdots&		0&		a_2\\
\vdots&		\vdots&		\vdots&		&		\vdots&		\vdots\\
0&		0&		0&		\cdots&		x&		a_{n-2}\\
0&		0&		0&		\cdots&		-1&		x+a_{n-1}\\
\end{matrix} \right|\xlongequal[i=n,n-1,\cdots ,2]{xr_i+r_{i-1}}\left| \begin{matrix}
0&		0&		0&		\cdots&		0&		x^n+a_{n-1}x^{n-1}+\cdots +a_1x+a_0\\
-1&		0&		0&		\cdots&		0&		x^{n-1}+a_{n-1}x^{n-2}+\cdots +a_2x+a_1\\
0&		-1&		0&		\cdots&		0&		x^{n-2}+a_{n-1}x^{n-3}+\cdots +a_3x+a_2\\
\vdots&		\vdots&		\vdots&		&		\vdots&		\vdots\\
0&		0&		0&		\cdots&		0&		x^2+a_{n-1}x+a_{n-2}\\
0&		0&		0&		\cdots&		-1&		x+a_{n-1}\\
\end{matrix} \right|
\\
&\xlongequal{\text{按第一行展开}}\left( x^n+a_{n-1}x^{n-1}+\cdots +a_1x+a \right) \left( -1 \right) ^{n+1}\left| \begin{matrix}
-1&		0&		0&		\cdots&		0\\
0&		-1&		0&		\cdots&		0\\
\vdots&		\vdots&		\vdots&		&		\vdots\\
0&		0&		0&		\cdots&		0\\
0&		0&		0&		\cdots&		-1\\
\end{matrix} \right|
\\
&=\left( x^n+a_{n-1}x^{n-1}+\cdots +a_1x+a \right) \left( -1 \right) ^{n+1}\left( -1 \right) ^{n-1}
\\
&=x^n+a_{n-1}x^{n-1}+\cdots +a_1x+a=f\left( x \right) .
\end{align*}
\end{proof}

\begin{proposition}[行列式的刻画]\label{proposition:行列式的刻画}
设$f$为从$n$阶方阵全体构成的集合到数集上的映射,使得对任意的$n$阶方阵$\boldsymbol{A}$,任意的指标$1\leq i\leq n$,以及任意的常数$c$,满足下列条件:

(1) 设$\boldsymbol{A}$的第$i$列是方阵$\boldsymbol{B}$和$\boldsymbol{C}$的第$i$列之和,且$\boldsymbol{A}$的其余列与$\boldsymbol{B}$和$\boldsymbol{C}$的对应列完全相同,则$f(\boldsymbol{A})=f(\boldsymbol{B})+f(\boldsymbol{C})$;

(2) 将$\boldsymbol{A}$的第$i$列乘以常数$c$得到方阵$\boldsymbol{B}$,则$f(\boldsymbol{B})=cf(\boldsymbol{A})$;

(3) 对换$\boldsymbol{A}$的任意两列得到方阵$\boldsymbol{B}$,则$f(\boldsymbol{B})= - f(\boldsymbol{A})$;

(4) $f(\boldsymbol{I}_n)=1$,其中$\boldsymbol{I}_n$是$n$阶单位阵.

求证:$f(\boldsymbol{A})=\vert \boldsymbol{A}\vert$.
\end{proposition}
\begin{note}
这个命题给出了\textbf{行列式的刻画}:在方阵\(n\)个列向量上的多重线性和反对称性,以及正规性(即单位矩阵处的取值为\(1\)),唯一确定了行列式这个函数.
\end{note}
\begin{proof}
设\(\boldsymbol{A} = (\boldsymbol{\alpha}_1,\boldsymbol{\alpha}_2,\cdots,\boldsymbol{\alpha}_n)\),其中\(\boldsymbol{\alpha}_k\)为\(\boldsymbol{A}\)的第\(k\)列,\(\boldsymbol{e}_1,\boldsymbol{e}_2,\cdots,\boldsymbol{e}_n\)为标准单位列向量,则
\begin{align*}
\boldsymbol{\alpha}_j = a_{1j}\boldsymbol{e}_1 + a_{2j}\boldsymbol{e}_2 + \cdots + a_{nj}\boldsymbol{e}_n = \sum_{k = 1}^{n}a_{kj}\boldsymbol{e}_k,j = 1,2,\cdots,n.
\end{align*}
从而由条件\((1)\)和\((2)\)可得
\begin{align*}
&f\left( \boldsymbol{A} \right) =f\left( \boldsymbol{\alpha }_1,\boldsymbol{\alpha }_2,\cdots ,\boldsymbol{\alpha }_n \right) =f\left( \sum_{k_1=1}^n{a_{k_11}\boldsymbol{e}_k},\boldsymbol{\alpha }_2,\cdots ,\boldsymbol{\alpha }_n \right) 
\\
&=a_{11}f\left( \boldsymbol{e}_1,\boldsymbol{\alpha }_2,\cdots ,\boldsymbol{\alpha }_n \right) +a_{21}f\left( \boldsymbol{e}_2,\boldsymbol{\alpha }_2,\cdots ,\boldsymbol{\alpha }_n \right) +\cdots +a_{n1}f\left( \boldsymbol{e}_n,\boldsymbol{\alpha }_2,\cdots ,\boldsymbol{\alpha }_n \right) 
\\
&=\sum_{k_1=1}^n{a_{k_11}f\left( \boldsymbol{e}_{k_1},\boldsymbol{\alpha }_2,\cdots ,\boldsymbol{\alpha }_n \right)}=\sum_{k_1=1}^n{a_{k_11}f\left( \boldsymbol{e}_{k_1},\sum_{k_2=1}^n{a_{k_22}\boldsymbol{e}_{k_2}},\cdots ,\boldsymbol{\alpha }_n \right)}
\\
&=\sum_{k_1=1}^n{a_{k_11}\left[ a_{12}f\left( \boldsymbol{e}_{k_1},\boldsymbol{e}_1,\cdots ,\boldsymbol{\alpha }_n \right) +a_{22}f\left( \boldsymbol{e}_{k_1},\boldsymbol{e}_2,\cdots ,\boldsymbol{\alpha }_n \right) +\cdots +a_{n2}f\left( \boldsymbol{e}_{k_1},\boldsymbol{e}_n,\cdots ,\boldsymbol{\alpha }_n \right) \right]}
\\
&=\sum_{k_1=1}^n{a_{k_11}\sum_{k_2=1}^n{a_{k_22}}f\left( \boldsymbol{e}_{k_1},\boldsymbol{e}_{k_2},\cdots ,\boldsymbol{\alpha }_n \right)}=\cdots =\sum_{k_1=1}^n{a_{k1}\sum_{k_2=1}^n{a_{k_22}}\cdots \sum_{k_n=1}^n{a_{k_nn}f\left( \boldsymbol{e}_{k_1},\boldsymbol{e}_{k_2},\cdots ,\boldsymbol{e}_{k_n} \right)}}
\\
&=\sum_{k_1=1}^n{\sum_{k_2=1}^n{\cdots \sum_{k_n=1}^n{a_{k1}a_{k_22}\cdots a_{k_nn}f\left( \boldsymbol{e}_{k_1},\boldsymbol{e}_{k_2},\cdots ,\boldsymbol{e}_{k_n} \right)}}}=\sum_{\left( k_1,k_2,\cdots ,k_n \right)}{a_{k_11}a_{k_22}\cdots a_{k_nn}f\left( \boldsymbol{e}_{k_1},\boldsymbol{e}_{k_2},\cdots ,\boldsymbol{e}_{k_n} \right)}.
\end{align*}
若\(k_i = k_j\),则\((\boldsymbol{e}_{k_1},\boldsymbol{e}_{k_2},\cdots,\boldsymbol{e}_{k_n})\)的第\(i\)列和第\(j\)列对换后仍然是\((\boldsymbol{e}_{k_1},\boldsymbol{e}_{k_2},\cdots,\boldsymbol{e}_{k_n})\).由条件\((3)\)可知,\(f(\boldsymbol{e}_{k_1},\boldsymbol{e}_{k_2},\cdots,\boldsymbol{e}_{k_n}) = -f(\boldsymbol{e}_{k_1},\boldsymbol{e}_{k_2},\cdots,\boldsymbol{e}_{k_n})\),于是\(f(\boldsymbol{e}_{k_1},\boldsymbol{e}_{k_2},\cdots,\boldsymbol{e}_{k_n}) = 0\).
因此在\(f(\boldsymbol{A})\)的表示式中,只剩下\(k_i\)(\(i = 1,2,\cdots,n\))互不相同的项.
通过\(\tau(k_1k_2\cdots k_n)\)次相邻对换可将\((\boldsymbol{e}_{k_1},\boldsymbol{e}_{k_2},\cdots,\boldsymbol{e}_{k_n})\)变成\((\boldsymbol{e}_1,\boldsymbol{e}_2,\cdots,\boldsymbol{e}_n) = \boldsymbol{I}_n\),
故由条件\((3)\)和\((4)\)可得
\begin{align*}
f(\boldsymbol{e}_{k_1},\boldsymbol{e}_{k_2},\cdots,\boldsymbol{e}_{k_n}) = (-1)^{\tau(k_1k_2\cdots k_n)}f(\boldsymbol{I}_n) = (-1)^{\tau(k_1k_2\cdots k_n)}.
\end{align*}
于是由行列式的组合定义可知
\begin{align*}
f(\boldsymbol{A}) = \sum_{(k_1,k_2,\cdots,k_n)}a_{k_11}a_{k_22}\cdots a_{k_nn}f(\boldsymbol{e}_{k_1},\boldsymbol{e}_{k_2},\cdots,\boldsymbol{e}_{k_n}) = \sum_{(k_1,k_2,\cdots,k_n)}(-1)^{\tau(k_1k_2\cdots k_n)}a_{k_11}a_{k_22}\cdots a_{k_nn} = |\boldsymbol{A}|.
\end{align*}
\end{proof}

\section{练习}

\begin{exercise}
计算$n$阶行列式:
\begin{equation}
|\boldsymbol{A}|=\left| \begin{matrix}
1&		1&		\cdots&		1\\
1&		\mathrm{C}_{2}^{1}&		\cdots&		\mathrm{C}_{n}^{1}\\
1&		\mathrm{C}_{3}^{2}&		\cdots&		\mathrm{C}_{n+1}^{2}\\
\vdots&		\vdots&		&		\vdots\\
1&		\mathrm{C}_{n}^{n-1}&		\cdots&		\mathrm{C}_{2n-2}^{n-1}\\
\end{matrix} \right|.
\nonumber
\end{equation}
\begin{note}
组合数公式:
$\mathrm{C}_{m}^{k-1}+\mathrm{C}_{m}^{k}=\mathrm{C}_{m+1}^{k}$.

于是有
\begin{gather}
\mathrm{C}_{m}^{k}=\mathrm{C}_{m+1}^{k}-\mathrm{C}_{m}^{k-1}
\nonumber
\\
\mathrm{C}_{m}^{k-1}=\mathrm{C}_{m+1}^{k}-\mathrm{C}_{m}^{k}
\nonumber
\end{gather}
\end{note}
\begin{solution}
\begin{equation}
\begin{split}
|\boldsymbol{A}|&=\left| \begin{matrix}
1&		1&		\cdots&		1\\
1&		\mathrm{C}_{2}^{1}&		\cdots&		\mathrm{C}_{n}^{1}\\
1&		\mathrm{C}_{3}^{2}&		\cdots&		\mathrm{C}_{n+1}^{2}\\
\vdots&		\vdots&		&		\vdots\\
1&		\mathrm{C}_{n}^{n-1}&		\cdots&		\mathrm{C}_{2n-2}^{n-1}\\
\end{matrix} \right|\xlongequal[i=n,\cdots ,2]{\left( -1 \right) \cdot r_{i-1}+r_i}\left| \begin{matrix}
\mathrm{C}_{0}^{0}&		\mathrm{C}_{1}^{0}&		\cdots&		\mathrm{C}_{\mathrm{n}-1}^{0}\\
0&		\mathrm{C}_{2}^{1}-\mathrm{C}_{1}^{0}&		\cdots&		\mathrm{C}_{n}^{1}-\mathrm{C}_{\mathrm{n}-1}^{0}\\
0&		\mathrm{C}_{3}^{2}-\mathrm{C}_{2}^{1}&		\cdots&		\mathrm{C}_{n+1}^{2}-\mathrm{C}_{n}^{1}\\
\vdots&		\vdots&		&		\vdots\\
0&		\mathrm{C}_{n}^{n-1}-\mathrm{C}_{\mathrm{n}-1}^{\mathrm{n}-2}&		\cdots&		\mathrm{C}_{2n-2}^{n-1}-\mathrm{C}_{2\mathrm{n}-3}^{\mathrm{n}-2}\\
\end{matrix} \right|
\\
&=\left| \begin{matrix}
\mathrm{C}_{0}^{0}&		\mathrm{C}_{1}^{0}&		\cdots&		\mathrm{C}_{\mathrm{n}-1}^{0}\\
0&		\mathrm{C}_{1}^{1}&		\cdots&		\mathrm{C}_{\mathrm{n}-1}^{1}\\
0&		\mathrm{C}_{2}^{2}&		\cdots&		\mathrm{C}_{n}^{2}\\
\vdots&		\vdots&		&		\vdots\\
0&		\mathrm{C}_{\mathrm{n}-1}^{\mathrm{n}-1}&		\cdots&		\mathrm{C}_{2\mathrm{n}-3}^{\mathrm{n}-1}\\
\end{matrix} \right|\xlongequal{\text{按第一列展开}}\left| \begin{matrix}
\mathrm{C}_{1}^{1}&		\mathrm{C}_{2}^{1}&		\cdots&		\mathrm{C}_{\mathrm{n}-1}^{1}\\
\mathrm{C}_{2}^{2}&		\mathrm{C}_{3}^{2}&		\cdots&		\mathrm{C}_{n}^{2}\\
\vdots&		\vdots&		&		\vdots\\
\mathrm{C}_{\mathrm{n}-1}^{\mathrm{n}-1}&		\mathrm{C}_{\mathrm{n}}^{\mathrm{n}-1}&		\cdots&		\mathrm{C}_{2\mathrm{n}-3}^{\mathrm{n}-1}\\
\end{matrix} \right|
\\
&\xlongequal[i=n,\cdots ,2]{\left( -1 \right) \cdot j_{i-1}+j_i}\left| \begin{matrix}
\mathrm{C}_{1}^{1}&		\mathrm{C}_{2}^{1}-\mathrm{C}_{1}^{1}&		\cdots&		\mathrm{C}_{\mathrm{n}-1}^{1}-\mathrm{C}_{\mathrm{n}-2}^{1}\\
\mathrm{C}_{2}^{2}&		\mathrm{C}_{3}^{2}-\mathrm{C}_{2}^{2}&		\cdots&		\mathrm{C}_{n}^{2}-\mathrm{C}_{\mathrm{n}-1}^{2}\\
\vdots&		\vdots&		&		\vdots\\
\mathrm{C}_{\mathrm{n}-1}^{\mathrm{n}-1}&		\mathrm{C}_{\mathrm{n}}^{\mathrm{n}-1}-\mathrm{C}_{\mathrm{n}-1}^{\mathrm{n}-1}&		\cdots&		\mathrm{C}_{2\mathrm{n}-3}^{\mathrm{n}-1}-\mathrm{C}_{2\mathrm{n}-4}^{\mathrm{n}-1}\\
\end{matrix} \right|=\left| \begin{matrix}
\mathrm{C}_{1}^{1}&		\mathrm{C}_{1}^{0}&		\cdots&		\mathrm{C}_{\mathrm{n}-2}^{0}\\
\mathrm{C}_{2}^{2}&		\mathrm{C}_{2}^{1}&		\cdots&		\mathrm{C}_{\mathrm{n}-1}^{1}\\
\vdots&		\vdots&		&		\vdots\\
\mathrm{C}_{\mathrm{n}-1}^{\mathrm{n}-1}&		\mathrm{C}_{\mathrm{n}-1}^{\mathrm{n}-2}&		\cdots&		\mathrm{C}_{2\mathrm{n}-4}^{\mathrm{n}-2}\\
\end{matrix} \right|
\\
&=\left| \begin{matrix}
1&		1&		\cdots&		1\\
1&		\mathrm{C}_{2}^{1}&		\cdots&		\mathrm{C}_{\mathrm{n}-1}^{1}\\
\vdots&		\vdots&		&		\vdots\\
1&		\mathrm{C}_{\mathrm{n}-1}^{\mathrm{n}-2}&		\cdots&		\mathrm{C}_{2\mathrm{n}-4}^{\mathrm{n}-2}\\
\end{matrix} \right|
\end{split}.
\nonumber
\end{equation}
此时得到的行列式恰好是原行列式的左上角部分,并具有相同的规律.
不断这样做下去,最后可得$|\boldsymbol{A}|=1$
\end{solution}
\end{exercise}

\begin{exercise}
计算$n$阶行列式:
\begin{gather}
|\boldsymbol{A}|=\left| \begin{matrix}
1&		2&		3&		\cdots&		n\\
-1&		0&		3&		\cdots&		n\\
-1&		-2&		0&		\cdots&		n\\
\vdots&		\vdots&		\vdots&		&		\vdots\\
-1&		-2&		-3&		\cdots&		0\\
\end{matrix} \right|
\nonumber
\end{gather}
\begin{solution}
\begin{equation}
\begin{split}
|\boldsymbol{A}|=\left| \begin{matrix}
1&		2&		3&		\cdots&		n\\
-1&		0&		3&		\cdots&		n\\
-1&		-2&		0&		\cdots&		n\\
\vdots&		\vdots&		\vdots&		&		\vdots\\
-1&		-2&		-3&		\cdots&		0\\
\end{matrix} \right|
\xlongequal[i=2,\cdots ,n]{r_1+r_i}\left| \begin{matrix}
1&		2&		3&		\cdots&		n\\
0&		2&		*&		\cdots&		*\\
0&		0&		3&		\cdots&		*\\
\vdots&		\vdots&		\vdots&		&		\vdots\\
0&		0&		0&		\cdots&		n\\
\end{matrix} \right|=n!
\end{split}
\nonumber
\end{equation}
\end{solution}
\end{exercise}

\begin{exercise}
计算$n$阶行列式:
\begin{gather}
|\boldsymbol{A}|=\left| \begin{matrix}
a_1b_1&		a_1b_2&		a_1b_3&		\cdots&		a_1b_n\\
a_1b_2&		a_2b_2&		a_2b_3&		\cdots&		a_2b_n\\
a_1b_3&		a_2b_3&		a_3b_3&		\cdots&		a_3b_n\\
\vdots&		\vdots&		\vdots&		&		\vdots\\
a_1b_n&		a_2b_n&		a_3b_n&		\cdots&		a_nb_n\\
\end{matrix} \right|.
\nonumber
\end{gather}
\begin{solution}
\begin{equation}
\begin{split}
|\boldsymbol{A}|&=\left| \begin{matrix}
a_1b_1&		a_1b_2&		a_1b_3&		\cdots&		a_1b_n\\
a_1b_2&		a_2b_2&		a_2b_3&		\cdots&		a_2b_n\\
a_1b_3&		a_2b_3&		a_3b_3&		\cdots&		a_3b_n\\
\vdots&		\vdots&		\vdots&		&		\vdots\\
a_1b_n&		a_2b_n&		a_3b_n&		\cdots&		a_nb_n\\
\end{matrix} \right|=a_1\left| \begin{matrix}
b_1&		b_2&		b_3&		\cdots&		b_n\\
a_1b_2&		a_2b_2&		a_2b_3&		\cdots&		a_2b_n\\
a_1b_3&		a_2b_3&		a_3b_3&		\cdots&		a_3b_n\\
\vdots&		\vdots&		\vdots&		&		\vdots\\
a_1b_n&		a_2b_n&		a_3b_n&		\cdots&		a_nb_n\\
\end{matrix} \right|
\\
&\xlongequal[i=2,\cdots ,n]{\left( -a_i \right) r_1+r_i}a_1\left| \begin{matrix}
b_1&		b_2&		b_3&		\cdots&		b_n\\
a_1b_2-a_2b_1&		0&		0&		\cdots&		0\\
a_1b_3-a_3b_1&		a_2b_3-a_3b_2&		0&		\cdots&		0\\
\vdots&		\vdots&		\vdots&		&		\vdots\\
a_1b_n-a_nb_1&		a_2b_n-a_nb_2&		a_3b_n-a_nb_3&		\cdots&		0\\
\end{matrix} \right|
\\
&\xlongequal{\text{按第}n\text{列展开}}\left( -1 \right) ^{n+1}a_1b_n\left| \begin{matrix}
a_1b_2-a_2b_1&		0&		\cdots&		0\\
a_1b_3-a_3b_1&		a_2b_3-a_3b_2&		\cdots&		0\\
\vdots&		\vdots&		&		\vdots\\
a_1b_n-a_nb_1&		a_2b_n-a_nb_2&		\cdots&		a_{n-1}b_n-a_nb_{n-1}\\
\end{matrix} \right|
\\
&=\left( -1 \right) ^{n-1}a_1b_n\prod\limits_{i=1}^{n-1}{\left( a_ib_{i+1}-a_{i+1}b_i \right)}
\\
&=a_1b_n\prod\limits_{i=1}^{n-1}{\left( a_{i+1}b_i-a_ib_{i+1} \right)}.
\end{split}
\nonumber
\end{equation}
\end{solution}
\end{exercise}

\begin{exercise}
计算$n$阶行列式:
\begin{equation}
|\boldsymbol{A}|=\left| \begin{matrix}
a&		0&		\cdots&		0&		1\\
0&		a&		\cdots&		0&		0\\
\vdots&		\vdots&		&		\vdots&		\vdots\\
0&		0&		\cdots&		a&		0\\
1&		0&		\cdots&		0&		a\\
\end{matrix} \right|.
\nonumber
\end{equation}    
\begin{solution}
\begin{equation}
\begin{split}
|\boldsymbol{A}|=\left| \begin{matrix}
a&		0&		\cdots&		0&		1\\
0&		a&		\cdots&		0&		0\\
\vdots&		\vdots&		&		\vdots&		\vdots\\
0&		0&		\cdots&		a&		0\\
1&		0&		\cdots&		0&		a\\
\end{matrix} \right|\xlongequal{\text{按第一列展开}}a^n+\left( -1 \right) ^{n+1}\left| \begin{matrix}
0&		0&		\cdots&		0&		1\\
a&		0&		\cdots&		0&		0\\
\vdots&		\vdots&		&		\vdots&		\vdots\\
0&		0&		\cdots&		a&		0\\
\end{matrix} \right|=a^n+\left( -1 \right) ^{n+1+n}a^{n-2}=a^n-a^{n-2}.
\end{split}
\nonumber
\end{equation}
\end{solution}
\begin{remark}
本题也可由命题\ref{"爪"型行列式}直接得到,$|\boldsymbol{A}|=a^n-a^{n-2}$.
\end{remark}
\end{exercise}

\begin{exercise}
设$x_1,x_2,x_3$是方程$x^3+px+q=0$的3个根,求下列行列式的值:
\begin{gather}
\left| \boldsymbol{A} \right|=\left| \begin{matrix}
x_1&		x_2&		x_3\\
x_2&		x_3&		x_1\\
x_3&		x_1&		x_2\\
\end{matrix} \right|.
\nonumber
\end{gather}
\begin{solution}
由$Vieta$定理可知,$x_1+x_2+x_3=0$.因此,我们有
\begin{equation}
\begin{split}
\left| \boldsymbol{A} \right|=\left| \begin{matrix}
x_1&		x_2&		x_3\\
x_2&		x_3&		x_1\\
x_3&		x_1&		x_2\\
\end{matrix} \right|\xlongequal[i=2,3]{r_i+r_1}\left| \begin{matrix}
0&		0&		0\\
x_2&		x_3&		x_1\\
x_3&		x_1&		x_2\\
\end{matrix} \right|=0.
\end{split}
\nonumber
\end{equation}
\end{solution}
\end{exercise}

\begin{exercise}
设$b_{ij}=\left( a_{i1}+a_{i2}+\cdots +a_{in} \right) -a_{ij}$,求证:
\begin{equation}
\left| \begin{matrix}
b_{11}&		\cdots&		b_{1n}\\
\vdots&		&		\vdots\\
b_{n1}&		\cdots&		b_{nn}\\
\end{matrix} \right|=(-1)^{n-1}(n-1)\left| \begin{matrix}
a_{11}&		\cdots&		a_{1n}\\
\vdots&		&		\vdots\\
a_{n1}&		\cdots&		a_{nn}\\
\end{matrix} \right|.
\nonumber
\end{equation}
\begin{solution}
\begin{align*}
&\left| \begin{matrix}
b_{11}&		b_{12}&		\cdots&		b_{1n}\\
b_{21}&		b_{22}&		\cdots&		b_{2n}\\
\vdots&		\vdots&		&		\vdots\\
b_{n1}&		b_{n2}&		\cdots&		b_{nn}\\
\end{matrix} \right|=\left| \begin{matrix}
\left( a_{11}+a_{12}+\cdots +a_{1n} \right) -a_{11}&		\left( a_{11}+a_{12}+\cdots +a_{1n} \right) -a_{12}&		\cdots&		\left( a_{11}+a_{12}+\cdots +a_{1n} \right) -a_{1n}\\
\left( a_{21}+a_{22}+\cdots +a_{2n} \right) -a_{21}&		\left( a_{21}+a_{22}+\cdots +a_{2n} \right) -a_{22}&		\cdots&		\left( a_{21}+a_{22}+\cdots +a_{2n} \right) -a_{2n}\\
\vdots&		\vdots&		&		\vdots\\
\left( a_{n1}+a_{n2}+\cdots +a_{nn} \right) -a_{n1}&		\left( a_{n1}+a_{n2}+\cdots +a_{nn} \right) -a_{n2}&		\cdots&		\left( a_{n1}+a_{n2}+\cdots +a_{nn} \right) -a_{nn}\\
\end{matrix} \right|
\\
&\xlongequal[i=2,\cdots ,n]{j_i+j_1}\left| \begin{matrix}
\left( n-1 \right) \left( a_{11}+a_{12}+\cdots +a_{1n} \right)&		\left( a_{11}+a_{12}+\cdots +a_{1n} \right) -a_{12}&		\cdots&		\left( a_{11}+a_{12}+\cdots +a_{1n} \right) -a_{1n}\\
\left( n-1 \right) \left( a_{21}+a_{22}+\cdots +a_{2n} \right)&		\left( a_{21}+a_{22}+\cdots +a_{2n} \right) -a_{22}&		\cdots&		\left( a_{21}+a_{22}+\cdots +a_{2n} \right) -a_{2n}\\
\vdots&		\vdots&		&		\vdots\\
\left( n-1 \right) \left( a_{n1}+a_{n2}+\cdots +a_{nn} \right)&		\left( a_{n1}+a_{n2}+\cdots +a_{nn} \right) -a_{n2}&		\cdots&		\left( a_{n1}+a_{n2}+\cdots +a_{nn} \right) -a_{nn}\\
\end{matrix} \right|
\\
&=\left( n-1 \right) \left| \begin{matrix}
\left( a_{11}+a_{12}+\cdots +a_{1n} \right)&		\left( a_{11}+a_{12}+\cdots +a_{1n} \right) -a_{12}&		\cdots&		\left( a_{11}+a_{12}+\cdots +a_{1n} \right) -a_{1n}\\
\left( a_{21}+a_{22}+\cdots +a_{2n} \right)&		\left( a_{21}+a_{22}+\cdots +a_{2n} \right) -a_{22}&		\cdots&		\left( a_{21}+a_{22}+\cdots +a_{2n} \right) -a_{2n}\\
\vdots&		\vdots&		&		\vdots\\
\left( a_{n1}+a_{n2}+\cdots +a_{nn} \right)&		\left( a_{n1}+a_{n2}+\cdots +a_{nn} \right) -a_{n2}&		\cdots&		\left( a_{n1}+a_{n2}+\cdots +a_{nn} \right) -a_{nn}\\
\end{matrix} \right|
\\
&\xlongequal[i=2,\cdots ,n]{\left( -1 \right) j_1+j_i}\left( n-1 \right) \left| \begin{matrix}
\left( a_{11}+a_{12}+\cdots +a_{1n} \right)&		-a_{12}&		\cdots&		-a_{1n}\\
\left( a_{21}+a_{22}+\cdots +a_{2n} \right)&		-a_{22}&		\cdots&		-a_{2n}\\
\vdots&		\vdots&		&		\vdots\\
\left( a_{n1}+a_{n2}+\cdots +a_{nn} \right)&		-a_{n2}&		\cdots&		-a_{nn}\\
\end{matrix} \right|
\\
&\xlongequal[i=2,\cdots ,n]{j_i+j_1}\left( n-1 \right) \left| \begin{matrix}
a_{11}&		-a_{12}&		\cdots&		-a_{1n}\\
a_{21}&		-a_{22}&		\cdots&		-a_{2n}\\
\vdots&		\vdots&		&		\vdots\\
a_{n1}&		-a_{n2}&		\cdots&		-a_{nn}\\
\end{matrix} \right|
\\
&=(-1)^{n-1}(n-1)\left| \begin{matrix}
a_{11}&		-a_{12}&		\cdots&		-a_{1n}\\
a_{21}&		-a_{22}&		\cdots&		-a_{2n}\\
\vdots&		\vdots&		&		\vdots\\
a_{n1}&		-a_{n2}&		\cdots&		-a_{nn}\\
\end{matrix} \right|.
\end{align*}
\end{solution}
\end{exercise}
\begin{conclusion}\label{行列式计算:求和法}
第二个等号是行列式计算中的一个常用方法\hypertarget{行列式计算:求和法}{\textbf{求和法}}:

将除第一列外的其余列全部加到第一列上(或将除第一行外的其余行全部加到第一行上),
使第一列(或列)一样或者具有相同形式.
然后根据具体情况将第一列(或行)的倍数加到其余列(或行)上,
从而将行列式化为我们熟悉的形式.

应用该方法的一般情形:

1.行列式每行(或列)和相等时;

2.行列式每行(或列)和有一定规律时.

\end{conclusion}

\begin{exercise}
计算$n$阶行列式:
\begin{equation}
|\boldsymbol{A}|=\left| \begin{matrix}
0&		1&		\cdots&		1&		1\\
1&		0&		\cdots&		1&		1\\
\vdots&		\vdots&		&		\vdots&		\vdots\\
1&		1&		\cdots&		0&		1\\
1&		1&		\cdots&		1&		0\\
\end{matrix} \right|.
\nonumber
\end{equation}
\begin{solution}
\begin{equation}
\begin{split}
&|\boldsymbol{A}|=\left| \begin{matrix}
0&		1&		\cdots&		1&		1\\
1&		0&		\cdots&		1&		1\\
\vdots&		\vdots&		&		\vdots&		\vdots\\
1&		1&		\cdots&		0&		1\\
1&		1&		\cdots&		1&		0\\
\end{matrix} \right|\xlongequal[i=2,\cdots ,n]{j_i+j_1}\left| \begin{matrix}
n-1&		1&		\cdots&		1&		1\\
n-1&		0&		\cdots&		1&		1\\
\vdots&		\vdots&		&		\vdots&		\vdots\\
n-1&		1&		\cdots&		0&		1\\
n-1&		1&		\cdots&		1&		0\\
\end{matrix} \right|=\left( n-1 \right) \left| \begin{matrix}
1&		1&		\cdots&		1&		1\\
1&		0&		\cdots&		1&		1\\
\vdots&		\vdots&		&		\vdots&		\vdots\\
1&		1&		\cdots&		0&		1\\
1&		1&		\cdots&		1&		0\\
\end{matrix} \right|
\\
&\xlongequal[i=2,\cdots ,n]{\left( -1 \right) r_1+r_i}\left( n-1 \right) \left| \begin{matrix}
1&		1&		\cdots&		1&		1\\
0&		-1&		\cdots&		0&		0\\
\vdots&		\vdots&		&		\vdots&		\vdots\\
0&		0&		\cdots&		-1&		0\\
0&		0&		\cdots&		0&		-1\\
\end{matrix} \right|=\left( -1 \right) ^{n-1}\left( n-1 \right) .
\end{split}
\nonumber
\end{equation}
\end{solution}
\begin{remark}
因为$|\boldsymbol{A}|$除对角元素外,每行都一样,
所以本题也可以看成命题\ref{"爪"型行列式的推广}的应用,利用命题\ref{"爪"型行列式的推广}的计算方法直接得到结果.
\begin{equation}
|\boldsymbol{A}|=\left| \begin{matrix}
0&		1&		\cdots&		1&		1\\
1&		0&		\cdots&		1&		1\\
\vdots&		\vdots&		&		\vdots&		\vdots\\
1&		1&		\cdots&		0&		1\\
1&		1&		\cdots&		1&		0\\
\end{matrix} \right|\xlongequal[i=2,\cdots ,n]{\left( -1 \right) r_1+r_i}\left| \begin{matrix}
0&		1&		\cdots&		1&		1\\
1&		-1&		\cdots&		0&		0\\
\vdots&		\vdots&		&		\vdots&		\vdots\\
1&		0&		\cdots&		-1&		0\\
1&		0&		\cdots&		0&		-1\\
\end{matrix} \right|\xlongequal[]{\text{命题}1.2}-\sum_{i=2}^n{\left( -1 \right) ^{n-2}}=\left( -1 \right) ^{n-1}\left( n-1 \right) .            
\nonumber
\end{equation}
\end{remark}
\end{exercise}

\begin{exercise}
计算$n$阶行列式:
\begin{equation}
\begin{split}
|\boldsymbol{A}|=\left| \begin{matrix}
a_1+b&		a_2&		a_3&		\cdots&		a_n\\
a_1&		a_2+b&		a_3&		\cdots&		a_n\\
a_1&		a_2&		a_3+b&		\cdots&		a_n\\
\vdots&		\vdots&		\vdots&		&		\vdots\\
a_1&		a_2&		a_3&		\cdots&		a_n+b\\
\end{matrix} \right|.
\end{split}
\nonumber
\end{equation}
\begin{note}
既可以将$|\boldsymbol{A}|$看作命题\ref{"爪"型行列式的推广}的应用,
利用命题\ref{"爪"型行列式的推广}的计算方法直接得到结果.即下述解法一.

也可以利用\hyperlink{行列式计算:求和法}{求和法}将$|\boldsymbol{A}|$
化为上三角形行列式.即下述解法二.
\end{note}
\begin{solution}
{\color{blue} \text{解法一}}:
\begin{equation}
\begin{split}
&|\boldsymbol{A}|=\left| \begin{matrix}
a_1+b&		a_2&		a_3&		\cdots&		a_n\\
a_1&		a_2+b&		a_3&		\cdots&		a_n\\
a_1&		a_2&		a_3+b&		\cdots&		a_n\\
\vdots&		\vdots&		\vdots&		&		\vdots\\
a_1&		a_2&		a_3&		\cdots&		a_n+b\\
\end{matrix} \right|
\xlongequal[i=2,\cdots ,n]{-r_1+r_i}\left| \begin{matrix}
a_1+b&		a_2&		a_3&		\cdots&		a_n\\
-b&		b&		0&		\cdots&		0\\
-b&		0&		b&		\cdots&		0\\
\vdots&		\vdots&		\vdots&		&		\vdots\\
-b&		0&		0&		\cdots&		b\\
\end{matrix} \right|
\\
&\xlongequal[]{\text{命题}\ref{"爪"型行列式}}\left( a_1+b \right) b^{n-1}-\sum_{i=2}^n{b^{n-2}a_i\left( -b \right)}
=b^{n-1}\left[ \left( a_1+b \right) +\sum_{i=2}^n{a_i} \right] 
=\left(( b+\sum_{i=1}^n{a_i} \right) b^{n-1}.
\end{split}
\nonumber
\end{equation}
{\color{blue} \text{解法二}}:
\begin{equation}
\begin{split}
&|\boldsymbol{A}|=\left| \begin{matrix}
a_1+b&		a_2&		a_3&		\cdots&		a_n\\
a_1&		a_2+b&		a_3&		\cdots&		a_n\\
a_1&		a_2&		a_3+b&		\cdots&		a_n\\
\vdots&		\vdots&		\vdots&		&		\vdots\\
a_1&		a_2&		a_3&		\cdots&		a_n+b\\
\end{matrix} \right|
\xlongequal[i=2,\cdots ,n]{j_i+j_1}(b+\sum_{i=1}^n{a_i)\left| \begin{matrix}
1&		a_2&		a_3&		\cdots&		a_n\\
1&		a_2+b&		a_3&		\cdots&		a_n\\
1&		a_2&		a_3+b&		\cdots&		a_n\\
\vdots&		\vdots&		\vdots&		&		\vdots\\
1&		a_2&		a_3&		\cdots&		a_n+b\\
\end{matrix} \right|}
\\
&\xlongequal[i=2,\cdots ,n]{-a_i\cdot j_1+j_i}(b+\sum_{i=1}^n{a_i)\left| \begin{matrix}
1&		0&		0&		\cdots&		0\\
1&		b&		0&		\cdots&		0\\
1&		0&		b&		\cdots&		0\\
\vdots&		\vdots&		\vdots&		&		\vdots\\
1&		0&		0&		\cdots&		b\\
\end{matrix} \right|}
=(b+\sum_{i=1}^n{a_i)b^{n-1}}.
\end{split}
\nonumber
\end{equation}
\end{solution}
\end{exercise}

\begin{exercise}
计算$n$阶行列式:
\begin{equation}
\begin{split}
|\boldsymbol{A}|=\left| \begin{matrix}
1&		2&		3&		\cdots&		n-1&		n\\
n&		1&		2&		\cdots&		n-2&		n-1\\
n-1&		n&		1&		\cdots&		n-3&		n-2\\
\vdots&		\vdots&		\vdots&		&		\vdots&		\vdots\\
3&		4&		5&		\cdots&		1&		2\\
2&		3&		4&		\cdots&		n&		1\\
\end{matrix}\right|.
\end{split}
\nonumber
\end{equation}
\end{exercise}
\begin{note}
\hyperlink{行列式计算:求和法}{求和法}的经典应用.
\end{note}
\begin{solution}
\begin{equation}
\begin{split}
&|\boldsymbol{A}|=\left| \begin{matrix}
1&		2&		3&		\cdots&		n-1&		n\\
n&		1&		2&		\cdots&		n-2&		n-1\\
n-1&		n&		1&		\cdots&		n-3&		n-2\\
\vdots&		\vdots&		\vdots&		&		\vdots&		\vdots\\
3&		4&		5&		\cdots&		1&		2\\
2&		3&		4&		\cdots&		n&		1\\
\end{matrix} \right|\xlongequal[i=2,\cdots ,n]{j_i+j_1}\frac{n\left( n+1 \right)}{2}\left| \begin{matrix}
1&		2&		3&		\cdots&		n-1&		n\\
1&		1&		2&		\cdots&		n-2&		n-1\\
1&		n&		1&		\cdots&		n-3&		n-2\\
\vdots&		\vdots&		\vdots&		&		\vdots&		\vdots\\
1&		4&		5&		\cdots&		1&		2\\
1&		3&		4&		\cdots&		n&		1\\
\end{matrix} \right|
\\
&\xlongequal[i=2,\cdots ,n]{-r_1+r_i}\frac{n\left( n+1 \right)}{2}\left| \begin{matrix}
1&		2&		3&		\cdots&		n-1&		n\\
0&		-1&		-1&		\cdots&		-1&		-1\\
0&		n-2&		-2&		\cdots&		-2&		-2\\
\vdots&		\vdots&		\vdots&		&		\vdots&		\vdots\\
0&		2&		2&		\cdots&		2-n&		2-n\\
0&		1&		1&		\cdots&		1&		1-n\\
\end{matrix} \right|\xlongequal[]{\text{按第一列展开}}\frac{n\left( n+1 \right)}{2}\left| \begin{matrix}
-1&		-1&		\cdots&		-1&		-1\\
n-2&		-2&		\cdots&		-2&		-2\\
\vdots&		\vdots&		&		\vdots&		\vdots\\
2&		2&		\cdots&		2-n&		2-n\\
1&		1&		\cdots&		1&		1-n\\
\end{matrix} \right|
\\
&\xlongequal[i=2,\cdots ,n]{-j_1+j_i}\frac{n\left( n+1 \right)}{2}\left| \begin{matrix}
-1&		0&		\cdots&		0&		0\\
n-2&		-n&		\cdots&		-n&		-n\\
\vdots&		\vdots&		&		\vdots&		\vdots\\
2&		0&		\cdots&		-n&		-n\\
1&		0&		\cdots&		0&		-n\\
\end{matrix} \right|\xlongequal[]{\text{按第一行展开}}-\frac{n\left( n+1 \right)}{2}\left| \begin{matrix}
-n&		\cdots&		-n&		-n\\
\vdots&		&		\vdots&		\vdots\\
0&		\cdots&		-n&		-n\\
0&		\cdots&		0&		-n\\
\end{matrix} \right|
\\
&=-\frac{n\left( n+1 \right)}{2}\left( -n \right) ^{n-2}=\left( -1 \right) ^{n-1}\frac{n+1}{2}n^{n-1}.            
\end{split}
\nonumber
\end{equation}
\end{solution}

\begin{exercise}
计算$D_{n+1}=\left| \begin{matrix}
\left( a_0+b_0 \right) ^n&		\left( a_0+b_1 \right) ^n&		\cdots&		\left( a_0+b_n \right) ^n\\
\left( a_1+b_0 \right) ^n&		\left( a_1+b_1 \right) ^n&		\cdots&		\left( a_1+b_n \right) ^n\\
\vdots&		\vdots&		\vdots&		\\
\left( a_n+b_0 \right) ^n&		\left( a_n+b_1 \right) ^n&		\cdots&		\left( a_n+b_n \right) ^n\\
\end{matrix} \right|$.
\end{exercise}
\begin{solution}
由二项式定理可知
\begin{align*}
\left( a_i+b_j \right) ^n={a_i}^n+\mathrm{C}_{n}^{1}{a_i}^{n-1}b_j+\cdots +\mathrm{C}_{n}^{n-1}a_i{b_j}^{n-1}+{b_j}^n,\text{其中}i,j=0,1,\cdots ,n.
\nonumber
\end{align*}
从而
\begin{align*}
D_{n+1}&=\left| \begin{matrix}
{a_0}^n+\mathrm{C}_{n}^{1}{a_0}^{n-1}b_0+\cdots +\mathrm{C}_{n}^{n-1}a_0{b_0}^{n-1}+{b_0}^n&		\cdots&		{a_0}^n+\mathrm{C}_{n}^{1}{a_0}^{n-1}b_n+\cdots +\mathrm{C}_{n}^{n-1}a_0{b_n}^{n-1}+{b_n}^n\\
{a_1}^n+\mathrm{C}_{n}^{1}{a_1}^{n-1}b_0+\cdots +\mathrm{C}_{n}^{n-1}a_1{b_0}^{n-1}+{b_0}^n&		\cdots&		{a_1}^n+\mathrm{C}_{n}^{1}{a_1}^{n-1}b_n+\cdots +\mathrm{C}_{n}^{n-1}a_1{b_n}^{n-1}+{b_n}^n\\
\vdots&		&		\vdots\\
{a_{n-1}}^n+\mathrm{C}_{n}^{1}{a_{n\-1}}^{n-1}b_0+\cdots +\mathrm{C}_{n}^{n-1}a_{n-1}{b_0}^{n-1}+{b_0}^n&		\cdots&		{a_{n-1}}^n+\mathrm{C}_{n}^{1}{a_{n-1}}^{n-1}b_n+\cdots +\mathrm{C}_{n}^{n-1}a_{n-1}{b_n}^{n-1}+{b_n}^n\\
{a_n}^n+\mathrm{C}_{n}^{1}{a_n}^{n-1}b_0+\cdots +\mathrm{C}_{n}^{n-1}a_n{b_0}^{n-1}+{b_0}^n&		\cdots&		{a_n}^n+\mathrm{C}_{n}^{1}{a_n}^{n-1}b_n+\cdots +\mathrm{C}_{n}^{n-1}a_n{b_n}^{n-1}+{b_n}^n\\
\end{matrix} \right|
\\
&=\left| \begin{matrix}
{a_0}^n&		{a_0}^{n-1}&		\cdots&		a_0&		1\\
{a_1}^n&		{a_1}^{n-1}&		\cdots&		a_1&		1\\
\vdots&		\vdots&		&		\vdots&		\vdots\\
{a_{n-1}}^n&		{a_{n-1}}^{n-1}&		\cdots&		a_{n-1}&		1\\
{a_n}^n&		{a_n}^{n-1}&		\cdots&		a_n&		1\\
\end{matrix} \right|\cdot \left| \begin{matrix}
1&		1&		\cdots&		1&		1\\
\mathrm{C}_{n}^{1}b_0&		\mathrm{C}_{n}^{1}b_1&		\cdots&		\mathrm{C}_{n}^{1}b_{n-1}&		\mathrm{C}_{n}^{1}b_n\\
\vdots&		\vdots&		&		\vdots&		\vdots\\
\mathrm{C}_{n}^{n-1}{b_0}^{n-1}&		\mathrm{C}_{n}^{n-1}{b_1}^{n-1}&		\cdots&		\mathrm{C}_{n}^{n-1}{b_{n-1}}^{n-1}&		\mathrm{C}_{n}^{n-1}{b_n}^{n-1}\\
{b_0}^n&		{b_1}^n&		\cdots&		{b_{n-1}}^n&		{b_n}^n\\
\end{matrix} \right|
\\
&\xlongequal{\hyperlink{行列式计算常识}{\text{列倒排}}}\left( -1 \right) ^{\frac{n\left( n+1 \right)}{2}}\left| \begin{matrix}
1&		a_0&		\cdots&		{a_0}^{n-1}&		{a_0}^n\\
1&		a_1&		\cdots&		{a_1}^{n-1}&		{a_1}^n\\
\vdots&		\vdots&		&		\vdots&		\vdots\\
1&		a_n&		\cdots&		{a_n}^{n-1}&		{a_n}^n\\
\end{matrix} \right|\cdot \prod_{i=1}^{n-1}{\mathrm{C}_{n}^{i}\left| \begin{matrix}
1&		1&		\cdots&		1&		1\\
b_0&		b_1&		\cdots&		b_{n-1}&		b_n\\
\vdots&		\vdots&		&		\vdots&		\vdots\\
{b_0}^{n-1}&		{b_1}^{n-1}&		\cdots&		{b_{n-1}}^{n-1}&		{b_n}^{n-1}\\
{b_0}^n&		{b_1}^n&		\cdots&		{b_{n-1}}^n&		{b_n}^n\\
\end{matrix} \right|}
\\
&=\left( -1 \right) ^{\frac{n\left( n+1 \right)}{2}}\prod_{0\le j<i\le n}{\left( a_i-a_j \right)}\prod_{i=1}^{n-1}{\mathrm{C}_{n}^{i}\prod_{0\le j<i\le n}{\left( b_i-b_j \right)}}
= =\prod_{i=1}^{n-1}{\mathrm{C}_{n}^{i}\prod_{0\le j<i\le n}{\left( a_j-a_i \right) \left( b_i-b_j \right)}}.
\end{align*}
\end{solution}

\begin{exercise}
计算$n$阶行列式$(bc\ne0)$:
\begin{equation}
\begin{split}
D_n=\left| \begin{matrix}
a&		b&		&		&		&		\\
c&		a&		b&		&		&		\\
&		c&		a&		b&		&		\\
&		&		\ddots&		\ddots&		\ddots&		\\
&		&		&		c&		a&		b\\
&		&		&		&		c&		a\\
\end{matrix} \right|.
\end{split}
\nonumber
\end{equation}
\end{exercise}
\begin{solution}
由命题\ref{三对角行列式}可知,递推式为$D_n=aD_{n-1}-bcD_{n-2}(n\ge2)$.
又易知$D_0=1,D_1=a$.令$a=\alpha+\beta,bc=\alpha\beta$,其中$\alpha,\beta\in\mathbb{C}$是特征方程$\lambda^2=a\lambda-bc\lambda$的两个复根,
则$D_n=\left( \alpha +\beta \right) D_{n-1}-\alpha \beta D_{n-2}(n\ge2)$.
从而
\begin{gather}
D_n-\alpha D_{n-1}=\beta \left( D_{n-1}-\alpha D_{n-2} \right) ,D_n-\beta D_{n-1}=\alpha \left( D_{n-1}-\beta D_{n-2} \right).
\nonumber
\end{gather}
于是
\begin{gather}
D_n-\alpha D_{n-1}=\beta ^{n-1}\left( D_1-\alpha D_0 \right) =\beta ^{n-1}\left( a-\alpha \right) =\beta ^n,
\nonumber\\
D_n-\beta D_{n-1}=\alpha ^{n-1}\left( D_1-\beta D_0 \right) =\alpha ^{n-1}\left( a-\beta \right) =\alpha ^n.
\nonumber
\end{gather}
因此,若$a^2\ne4bc(\text{即}\alpha\ne\beta)$,则联立上面两式,解得
\begin{equation}
D_n=\frac{\alpha ^{n+1}-\beta ^{n+1}}{\alpha -\beta}; 
\nonumber
\end{equation}
若$a^2=4bc(\text{即}\alpha=\beta)$,则由$a=\alpha+\beta$可知,$\alpha=\beta=\frac{a}{2}$.
又由$D_n-\alpha D_{n-1}=\beta ^n$可得
\begin{gather}
D_n=\left( \frac{a}{2} \right) ^n+\frac{a}{2}D_{n-1}=\left( \frac{a}{2} \right) ^n+\frac{a}{2}\left( \left( \frac{a}{2} \right) ^{n-1}+\frac{a}{2}D_{n-2} \right) =2\left( \frac{a}{2} \right) ^n+\left( \frac{a}{2} \right) ^2D_{n-2}=\cdots =n\left( \frac{a}{2} \right) ^n+\left( \frac{a}{2} \right) ^nD_0=\left( n+1 \right) \left( \frac{a}{2} \right) ^n.
\nonumber
\end{gather}
\end{solution}

\begin{exercise}\label{三对角行列式例题1}
求证:$n$阶行列式
\begin{equation}
|\boldsymbol{A}|=\left| \begin{matrix}
\cos x&		1&		0&		0&		\cdots&		0&		0&		0\\
1&		2\cos x&		1&		0&		\cdots&		0&		0&		0\\
0&		1&		2\cos x&		1&		\cdots&		0&		0&		0\\
\vdots&		\vdots&		\vdots&		\vdots&		&		\vdots&		\vdots&		\vdots\\
0&		0&		0&		0&		\cdots&		1&		2\cos x&		1\\
0&		0&		0&		0&		\cdots&		0&		1&		2\cos x\\
\end{matrix} \right|=\cos nx.
\nonumber
\end{equation}
\end{exercise}
\begin{solution}
{\color{blue} \text{解法一:}}

设$|\boldsymbol{A}|=D_n$,其中$n$表示$|\boldsymbol{A}|$的阶数$(n\ge0)$.易知$D_0=1,D_1=\cos x$.

从而$|\boldsymbol{A}|=D_n\xlongequal[\text{命题}\ref{三对角行列式}]{\text{按最后一列展开}}2\cos xD_{n-1}-D_{n-2}\left( n\ge 2 \right)$.

其对应的特征方程为$\lambda ^2=2\cos x\lambda -1$,解得$\lambda _1=\cos x+i\sin x,\lambda _2=\cos x-i\sin x$.

于是当$n\ge2$时,我们有$D_n=\left( \lambda _1+\lambda _2 \right) D_{n-1}+\lambda _1\lambda _2D_{n-2}$.

进而
\begin{equation}
\label{eq:递推式1.1}
\begin{split}
&D_n-\lambda _1D_{n-1}=\lambda _2\left( D_n-\lambda _1D_{n-1} \right),
\\
&D_n-\lambda _2D_{n-1}=\lambda _1\left( D_n-\lambda _2D_{n-1} \right).
\end{split}
\end{equation}
由此可得
\begin{gather}
D_n-\lambda _1D_{n-1}={\lambda _2}^{n-1}\left( D_1-\lambda _1D_0 \right) =-i\sin x\cdot {\lambda _2}^{n-1},
\nonumber\\
D_n-\lambda _2D_{n-1}={\lambda _1}^{n-1}\left( D_1-\lambda _2D_0 \right) =i\sin x\cdot {\lambda _1}^{n-1}.
\nonumber
\end{gather}
若$x\ne k\pi(k\in\mathbb{Z})$,则联立上面两式,解得
\begin{equation}
\begin{split}
D_n&=\frac{i\sin x\cdot {\lambda _1}^n+i\sin x\cdot {\lambda _2}^n}{\lambda _1-\lambda _2}=\frac{i\sin x\cdot \left( \cos x+i\sin x \right) ^n+i\sin x\cdot \left( \cos x-i\sin x \right) ^n}{2i\sin x}
\\
&\xlongequal[e^{ix}=\cos x+i\sin x,e^{-ix}=\cos x-i\sin x.]{Euler\text{公式}}\frac{i\sin x\cdot e^{nxi}+i\sin x\cdot e^{-nxi}}{2i\sin x}=\frac{i\sin x\cdot \left( \cos nx+i\sin nx \right) +i\sin x\cdot \left( \cos nx-i\sin nx \right)}{2i\sin x}
\\
&=\frac{2i\sin x\cdot \cos nx}{2i\sin x}=\cos nx.
\end{split}
\nonumber
\end{equation}
若$x=k\pi(k\in\mathbb{Z})$,则$\lambda _1=\lambda _2=\cos k\pi$.
从而由\eqref{eq:递推式1.1}式可得,$D_n-\cos k\pi D_{n-1}=-i\sin x\cdot \left( \cos k\pi \right) =0.$

于是
\begin{align*}
D_n=\cos k\pi D_{n-1}=\left( \cos k\pi \right) ^2D_{n-2}=\cdots =\left( \cos k\pi \right) ^nD_0=\left( \cos k\pi \right) ^n=\left( -1 \right) ^{kn}=\cos \left( nk\pi \right) =\cos nx.
\nonumber
\end{align*}
{\color{blue} \text{解法二:}}仿照练习\ref{使用数学归纳法计算行列式例题1}中的数学归纳法证明.
\end{solution}

\begin{exercise}\label{三对角行列式例题2}
求下列$n$阶行列式的值:
\begin{gather}
D_{n}=\begin{vmatrix}1-a_{1}&a_{2}&0&0&\cdots&0&0\\ -1&1-a_{2}&a_{3}&0&\cdots&0&0\\ 0&-1&1-a_{3}&a_{4}&\cdots&0&0\\ \vdots&\vdots&\vdots&\vdots&&\vdots&\vdots\\ 0&0&0&0&\cdots&-1&1-a_{n}\end{vmatrix}.
\nonumber
\end{gather}
\end{exercise}
\begin{note}
观察原行列式我们可以得到,$D_n$的每列和有一定的规律,即除了第一列和最后一列,中间每列和均为0.并且$D_n$是三对角行列式.
因此,我们既可以直接应用三对角行列式的结论(即命题\ref{三对角行列式}),又可以使用求和法进行求解.
如果我们直接应用三对角行列式的结论(即命题\ref{三对角行列式}),按照对一般的三对角行列式展开的方法能得到相应递推式,但是这样得到的递推式并不是相邻两项之间的递推,后续求解通项并不简便.
又因为使用求和法计算行列式后续计算一般比较简便所以我们先采用求和法进行尝试.
\end{note}
\begin{solution}
{\color{blue} \text{解法一:}}
当$n\ge1$时,我们有
\begin{align}
D_n&=\left| \begin{matrix}
1-a_1&		a_2&		0&		0&		\cdots&		0&		0\\
-1&		1-a_2&		a_3&		0&		\cdots&		0&		0\\
0&		-1&		1-a_3&		a_4&		\cdots&		0&		0\\
\vdots&		\vdots&		\vdots&		\vdots&		&		\vdots&		\vdots\\
0&		0&		0&		0&		\cdots&		-1&		1-a_n\\
\end{matrix} \right|\xlongequal[i=2,\cdots ,n]{r_i+r_1}\left| \begin{matrix}
-a_1&		0&		0&		0&		\cdots&		0&		1\\
-1&		1-a_2&		a_3&		0&		\cdots&		0&		0\\
0&		-1&		1-a_3&		a_4&		\cdots&		0&		0\\
\vdots&		\vdots&		\vdots&		\vdots&		&		\vdots&		\vdots\\
0&		0&		0&		0&		\cdots&		-1&		1-a_n\\
\end{matrix} \right|
\nonumber\\\nonumber
&\xlongequal[]{\text{按第一行展开}}-a_1D_{n-1}+\left( -1 \right) ^{n+1}\left| \begin{matrix}
-1&		1-a_2&		a_3&		0&		\cdots&		0\\
0&		-1&		1-a_3&		a_4&		\cdots&		0\\
\vdots&		\vdots&		\vdots&		\vdots&		&		\vdots\\
0&		0&		0&		0&		\cdots&		-1\\
\end{matrix} \right|
\\\nonumber
&=-a_1D_{n-1}+\left( -1 \right) ^{n+1}\left( -1 \right) ^{n-1}
\\\nonumber
&=1-a_1D_{n-1}.
\end{align}
其中$D_{n-i}$表示$D_{n-i+1}$去掉第一行和第一列得到的$n-i$阶行列式,$i=1,2,\cdots,n-1$.
(或者称$D_{n-i}$表示以$a_{i+1},\cdots,a_n$为未定元的$n-i$阶行列式,$i=1,2,\cdots,n-1$)

由递推不难得到
\begin{align*}
D_n=1-a_1\left( 1-a_2D_{n-2} \right) =1-a_1+a_1a_2D_{n-2}=\cdots =1-a_1+a_1a_2-a_1a_2a_3+\cdots +\left( -1 \right) ^na_1a_2\cdots a_n.
\nonumber
\end{align*}
{\color{blue} \text{解法二:}}仿照练习\ref{使用数学归纳法计算行列式例题1}中的数学归纳法证明.
\end{solution}

\begin{exercise}\label{使用数学归纳法计算行列式例题1}
设$n$阶行列式
\begin{align}
A_n=\left| \begin{matrix}
a_0+a_1&		a_1&		0&		0&		\cdots&		0&		0\\
a_1&		a_1+a_2&		a_2&		0&		\cdots&		0&		0\\
0&		a_2&		a_2+a_3&		a_3&		\cdots&		0&		0\\
\vdots&		\vdots&		\vdots&		\vdots&		&		\vdots&		\vdots\\
0&		0&		0&		0&		\cdots&		a_{n-1}&		a_{n-1}+a_n\\
\end{matrix} \right|,
\nonumber
\end{align}
求证:
\begin{align}
A_n=a_0a_1\cdots a_n\left( \frac{1}{a_0}+\frac{1}{a_1}+\cdots +\frac{1}{a_n} \right) .
\nonumber
\end{align}
\end{exercise}
\begin{note}
用\hypertarget{用数学归纳法与行列式有关的结论}{\textbf{数学归纳法}}证明与行列式有关的结论.

练习\ref{三对角行列式例题1}和练习\ref{三对角行列式例题2}都可同理使用用数学归纳法证明(对阶数$n$进行归纳即可).
\end{note}
\begin{proof}
(数学归纳法)对阶数$n$进行归纳.当$n=1,2$时,结论显然成立.假设阶数小于$n$结论成立.

现证明$n$阶的情形.注意到
\begin{align*}
A_n=\left| \begin{matrix}
a_0+a_1&		a_1&		0&		0&		\cdots&		0	&0\\
a_1&		a_1+a_2&		a_2&		0&		\cdots&		0&		0\\
0&		a_2&		a_2+a_3&		a_3&		\cdots&		0&		0\\
\vdots&		\vdots&		\vdots&		\vdots&		&		\vdots&		\vdots\\
0&		0&		0&		0&		\cdots&		a_{n-1}&		a_{n-1}+a_n\\
\end{matrix} \right|=\left( a_{n-1}+a_n \right) A_{n-1}-a_{n-1}^{2}A_{n-2}.
\nonumber
\end{align*}
将归纳假设代入上面的式子中得
\begin{align*}
A_n&=\left( a_{n-1}+a_n \right) A_{n-1}-a_{n-1}^{2}A_{n-2}
\\
&=\left( a_{n-1}+a_n \right) a_0a_1\cdots a_{n-1}\left( \frac{1}{a_0}+\frac{1}{a_1}+\cdots +\frac{1}{a_{n-1}} \right) -a_{n-1}^{2}a_0a_1\cdots a_{n-2}\left( \frac{1}{a_0}+\frac{1}{a_1}+\cdots +\frac{1}{a_{n-2}} \right) 
\\
&=a_0a_1\cdots a_n\left( \frac{1}{a_0}+\frac{1}{a_1}+\cdots +\frac{1}{a_{n-1}} \right) +a_0a_1\cdots a_{n-2}a_{n-1}^{2}\frac{1}{a_{n-1}}
\\
&=a_0a_1\cdots a_{n-1}\left[ a_n\left( \frac{1}{a_0}+\frac{1}{a_1}+\cdots +\frac{1}{a_{n-1}} \right) +1 \right] 
\\
&=a_0a_1\cdots a_{n-1}a_n\left( \frac{1}{a_0}+\frac{1}{a_1}+\cdots +\frac{1}{a_{n-1}}+\frac{1}{a_n} \right) .
\nonumber
\end{align*}
故由数学归纳法可知,结论对任意正整数$n$都成立.
\end{proof}

\begin{exercise}
设\(n(n > 2)\)阶行列式\(\vert A \vert\)的所有元素或为\(1\)或为\(-1\),求证:\(\vert A \vert\)的绝对值小于等于\(\frac{2}{3}n!\).
\end{exercise}
\begin{solution}
对阶数$n$进行归纳.当$n=3$时,将$\left| A \right|$的第一列元素为-1的行都乘以-1,再将$\left| A \right|$的第一行元素为1的列都乘以-1,$\left| A \right|$的绝对值不改变.

因此不妨设$\left| A \right|=\left| \begin{matrix}
1&		-1&		-1\\
1&		a_0&		b_0\\
1&		c_0&		d_0\\
\end{matrix} \right|,\text{其中}a_0,b_0,c_0,d_0=1\text{或}-1.$

从而
\begin{align*}
\left| A \right|=\left| \begin{matrix}
1&		-1&		-1\\
1&		a_0&		b_0\\
1&		c_0&		d_0\\
\end{matrix} \right|\xlongequal[i=2,3]{j_1+j_i}\left| \begin{matrix}
1&		0&		0\\
1&		a&		b\\
1&		c&		d\\
\end{matrix} \right|,\text{其中}a,b,c,d=0\text{或}2.
\nonumber
\end{align*}
于是
\begin{align*}
abs \left( \left| A \right| \right) =abs \left( \left| \begin{matrix}
1&		0&		0\\
1&		a&		b\\
1&		c&		d\\
\end{matrix} \right| \right) =abs \left( ad-bc \right) \leqslant 4=\frac{2}{3}\cdot 3!
\nonumber
\end{align*}
假设n-1阶时结论成立,现证$n$阶的情形.将$\left| A \right|$按第一行展开得
\begin{align*}
\left| A \right|=a_{11}A_{11}+a_{12}A_{12}+\cdots +a_{1n}A_{1n},\text{其中}a_{1i}=1\text{或}-1\left( i=1,2\cdots ,n \right) .
\nonumber
\end{align*}
从而由归纳假设可得
\begin{align*}
abs \left( \left| A \right| \right) &=abs \left( a_{11}A_{11}+a_{12}A_{12}+\cdots +a_{1n}A_{1n} \right) \leqslant abs \left( A_{11} \right) +abs \left( A_{12} \right) +\cdots +abs \left( A_{1n} \right) 
\\
&\leqslant \frac{2}{3}\left( n-1 \right) !+\frac{2}{3}\left( n-1 \right) !+\cdots +\frac{2}{3}\left( n-1 \right) !
\\
&=n\cdot \frac{2}{3}\left( n-1 \right) !=\frac{2}{3}n!.
\nonumber
\end{align*}
故由数学归纳法可知结论对任意正整数都成立.
\end{solution}

\begin{exercise}\label{大/小拆分法例题1}
计算$n$阶行列式:
\begin{align*}
|\boldsymbol{A}|=\left| \begin{matrix}
a&		b&		\cdots&		b\\
b&		a&		\cdots&		b\\
\vdots&		\vdots&		&		\vdots\\
b&		b&		\cdots&		a\\
\end{matrix} \right|.
\nonumber
\end{align*}
\end{exercise}
\begin{note}
{\color{blue}\text{解法一(\hyperlink{大拆分法}{大拆分法}):}}
注意到
\begin{align*}
|\boldsymbol{A}|&=\left| \begin{matrix}
a&		b&		\cdots&		b\\
b&		a&		\cdots&		b\\
\vdots&		\vdots&		&		\vdots\\
b&		b&		\cdots&		a\\
\end{matrix} \right|=\left| \begin{matrix}
b+\left( a-b \right)&		b+0&		\cdots&		b+0\\
b+0&		b+\left( a-b \right)&		\cdots&		b+0\\
\vdots&		\vdots&		&		\vdots\\
b+0&		b+0&		\cdots&		b+\left( a-b \right)\\
\end{matrix} \right|
\\
&=\left| \begin{matrix}
a-b&		0&		\cdots&		0\\
0&		a-b&		\cdots&		0\\
\vdots&		\vdots&		&		\vdots\\
0&		0&		\cdots&		a-b\\
\end{matrix} \right|+\sum_{i=1}^n{A_i}=\left( a-b \right) ^n+\sum_{i=1}^n{A_i}.
\end{align*}
其中$A_i$是第$i$行元素全为$b$,主对角元素除了$( i,i )$元外都为$a-b$,其他元素都为0的$n$阶行列式.

又因为
\begin{align*}
A_i=\begin{array}{l}
1\\
\vdots\\
i\\
\vdots\\
n\\
\end{array}\left| \begin{matrix}
a-b&		&		&		&		\\
&		\ddots&		&		&		\\
b&		\cdots&		b&		\cdots&		b\\
&		&		&		\ddots&		\\
&		&		&		&		a-b\\
\end{matrix} \right|=b\left( a-b \right) ^{n-1},i=1,2,\cdots ,n. 
\end{align*}
所以
\begin{align*}
|\boldsymbol{A}|=\left( a-b \right) ^n+\sum_{i=1}^n{A_i}=\left( a-b \right) ^n+nb\left( a-b \right) ^{n-1}=\left[ a+\left( n-1 \right) b \right] \left( a-b \right) ^{n-1}.
\end{align*}
{\color{blue}\text{解法二(\hyperlink{小拆分法}{小拆分法}):}}
记原行列式为$D_n$,其中$n$为原行列式的阶数.则将原行列式按第一列拆开为两个行列式得
\begin{align*}
D_n&=\left| \begin{matrix}
a&		b&		\cdots&		b\\
b&		a&		\cdots&		b\\
\vdots&		\vdots&		&		\vdots\\
b&		b&		\cdots&		a\\
\end{matrix} \right|=\left| \begin{matrix}
b+\left( a-b \right)&		b&		\cdots&		b\\
b+0&		a&		\cdots&		b\\
\vdots&		\vdots&		&		\vdots\\
b+0&		b&		\cdots&		a\\
\end{matrix} \right|=\left| \begin{matrix}
b&		b&		\cdots&		b\\
b&		a&		\cdots&		b\\
\vdots&		\vdots&		&		\vdots\\
b&		b&		\cdots&		a\\
\end{matrix} \right|+\left| \begin{matrix}
a-b&		b&		\cdots&		b\\
0&		a&		\cdots&		b\\
\vdots&		\vdots&		&		\vdots\\
0&		b&		\cdots&		a\\
\end{matrix} \right|
\\
&=\left| \begin{matrix}
b&		b&		\cdots&		b\\
0&		a-b&		\cdots&		0\\
\vdots&		\vdots&		&		\vdots\\
0&		0&		\cdots&		a-b\\
\end{matrix} \right|+\left( a-b \right) D_{n-1}=b\left( a-b \right) ^{n-1}+\left( a-b \right) D_{n-1}.
(n\ge2)
\end{align*}
从而由上式递推可得
\begin{align*}
D_n&=b\left( a-b \right) ^{n-1}+\left( a-b \right) D_{n-1}
\\
&=b\left( a-b \right) ^{n-1}+\left( a-b \right) \left[ b\left( a-b \right) ^{n-2}+\left( a-b \right) D_{n-2} \right] 
=2b\left( a-b \right) ^{n-1}+\left( a-b \right) ^2D_{n-2}
\\
&=\cdots =\left( n-1 \right) b\left( a-b \right) ^{n-1}+\left( a-b \right) ^{n-1}D_1
\\
&=\left( n-1 \right) b\left( a-b \right) ^{n-1}+\left( a-b \right) ^{n-1}a
\\
&=\left[ a+\left( n-1 \right) b \right] \left( a-b \right) ^{n-1}.
\end{align*}
{\color{blue}\text{解法三(\hyperlink{行列式计算:求和法}{求和法}):}}
\begin{align*}
|\boldsymbol{A}|&=\left| \begin{matrix}
a&		b&		\cdots&		b\\
b&		a&		\cdots&		b\\
\vdots&		\vdots&		&		\vdots\\
b&		b&		\cdots&		a\\
\end{matrix} \right|\xlongequal[i=2,3,\cdots ,n]{j_i+j_1}\left| \begin{matrix}
a+\left( n-1 \right) b&		b&		\cdots&		b\\
a+\left( n-1 \right) b&		a&		\cdots&		b\\
\vdots&		\vdots&		&		\vdots\\
a+\left( n-1 \right) b&		b&		\cdots&		a\\
\end{matrix} \right|=\left[ a+\left( n-1 \right) b \right] \left| \begin{matrix}
1&		b&		\cdots&		b\\
1&		a&		\cdots&		b\\
\vdots&		\vdots&		&		\vdots\\
1&		b&		\cdots&		a\\
\end{matrix} \right|
\\
&\xlongequal[i=2,3,\cdots ,n]{-r_1+r_i}\left[ a+\left( n-1 \right) b \right] \left| \begin{matrix}
1&		b&		\cdots&		b\\
0&		a-b&		\cdots&		0\\
\vdots&		\vdots&		&		\vdots\\
0&		0&		\cdots&		a-b\\
\end{matrix} \right|=\left[ a+\left( n-1 \right) b \right] \left( a-b \right) ^{n-1}.
\end{align*}
{\color{blue}\text{解法四(\hyperlink{"爪"型行列式的推广}{"爪"型行列式的推广}):}}
\begin{align*}
|\boldsymbol{A}|&=\left| \begin{matrix}
a&		b&		\cdots&		b\\
b&		a&		\cdots&		b\\
\vdots&		\vdots&		&		\vdots\\
b&		b&		\cdots&		a\\
\end{matrix} \right|\xlongequal[i=2,3,\cdots ,n]{-r_1+r_i}\left| \begin{matrix}
a&		b&		\cdots&		b\\
b-a&		a-b&		\cdots&		0\\
\vdots&		\vdots&		&		\vdots\\
b-a&		0&		\cdots&		a-b\\
\end{matrix} \right|
\\
&\xlongequal[i=2,3,\cdots ,n]{-j_i+j_1}\left| \begin{matrix}
a-\left( n-1 \right) b&		b&		\cdots&		b\\
0&		a-b&		\cdots&		0\\
\vdots&		\vdots&		&		\vdots\\
0&		0&		\cdots&		a-b\\
\end{matrix} \right|=\left[ a-\left( n-1 \right) b \right] \left( a-b \right) ^{n-1}.
\end{align*}
\end{note}

\begin{exercise}
计算$n$阶行列式:
\begin{align*}
|\boldsymbol{A}|=\left| \begin{matrix}
a&		b&		\cdots&		b\\
c&		a&		\cdots&		b\\
\vdots&		\vdots&		&		\vdots\\
c&		c&		\cdots&		a\\
\end{matrix} \right|.
\nonumber
\end{align*}
\end{exercise}
\begin{solution}
{\color{blue}\text{解法一(\hyperlink{大拆分法}{大拆分法}):}}
令
\begin{align*}
|\boldsymbol{A}(t)|=\left| \begin{matrix}
a+t&		b+t&		\cdots&		b+t\\
c+t&		a+t&		\cdots&		b+t\\
\vdots&		\vdots&		&		\vdots\\
c+t&		c+t&		\cdots&		a+t\\
\end{matrix} \right|=|\boldsymbol{A}|+tu,  u=\sum_{i,j=1}^n{A_{ij}.}
\nonumber
\end{align*}
当$t=-b$时,可得
\begin{align*}
|\boldsymbol{A}(-b)|=\left| \begin{matrix}
a-b&		0&		\cdots&		0\\
c-b&		a-b&		\cdots&		0\\
\vdots&		\vdots&		&		\vdots\\
c-b&		c-b&		\cdots&		a-b\\
\end{matrix} \right|=|\boldsymbol{A}|-bu=(a-b)^n.
\end{align*}
当$t=-c$时,可得
\begin{align*}
|\boldsymbol{A}(-c)|=\left| \begin{matrix}
a-c&		b-c&		\cdots&		b-c\\
0&		a-c&		\cdots&		b-c\\
\vdots&		\vdots&		&		\vdots\\
0&		0&		\cdots&		a-c\\
\end{matrix} \right|=|\boldsymbol{A}|-cu=(a-c)^n.
\end{align*}
若$b\ne c$,则联立上面两式可得
\begin{align*}
\left| \boldsymbol{A} \right|=\frac{b\left( a-c \right) ^n-c\left( a-b \right) ^n}{b-c}.
\nonumber
\end{align*}
若$b=c$,则由练习\ref{大/小拆分法例题1}可知
\begin{align*}
|\boldsymbol{A}|=\left[ a+\left( n-1 \right) b \right] \left( a-b \right) ^{n-1}.
\nonumber
\end{align*}
{\color{blue}\text{解法二(\hyperlink{小拆分法}{小拆分法}):}}
记原行列式为$D_n$,其中$n$为原行列式的阶数.则将原行列式分别按第一行、第一列拆开为两个行列式得
\begin{align*}
D_n&=\left| \begin{matrix}
a&		b&		\cdots&		b\\
c&		a&		\cdots&		b\\
\vdots&		\vdots&		&		\vdots\\
c&		c&		\cdots&		a\\
\end{matrix} \right|=\left| \begin{matrix}
b+\left( a-b \right)&		b+0&		\cdots&		b+0\\
c&		a&		\cdots&		b\\
\vdots&		\vdots&		&		\vdots\\
c&		c&		\cdots&		a\\
\end{matrix} \right|=\left| \begin{matrix}
b&		b&		\cdots&		b\\
c&		a&		\cdots&		b\\
\vdots&		\vdots&		&		\vdots\\
c&		c&		\cdots&		a\\
\end{matrix} \right|+\left| \begin{matrix}
a-b&		0&		\cdots&		0\\
c&		a&		\cdots&		b\\
\vdots&		\vdots&		&		\vdots\\
c&		c&		\cdots&		a\\
\end{matrix} \right|
\\
&=b\left| \begin{matrix}
1&		1&		\cdots&		1\\
c&		a&		\cdots&		b\\
\vdots&		\vdots&		&		\vdots\\
c&		c&		\cdots&		a\\
\end{matrix} \right|+\left( a-b \right) D_{n-1}=b\left| \begin{matrix}
1&		1&		\cdots&		1\\
0&		a-c&		\cdots&		b-c\\
\vdots&		\vdots&		&		\vdots\\
0&		0&		\cdots&		a-c\\
\end{matrix} \right|+\left( a-b \right) D_{n-1}
\\
&=b\left( a-c \right) ^{n-1}++\left( a-b \right) D_{n-1}.\left( n\ge 2 \right) 
\end{align*}
\begin{align*}
D_n&=\left| \begin{matrix}
a&		b&		\cdots&		b\\
c&		a&		\cdots&		b\\
\vdots&		\vdots&		&		\vdots\\
c&		c&		\cdots&		a\\
\end{matrix} \right|=\left| \begin{matrix}
c+\left( a-c \right)&		b&		\cdots&		b\\
c+0&		a&		\cdots&		b\\
\vdots&		\vdots&		&		\vdots\\
c+0&		c&		\cdots&		a\\
\end{matrix} \right|=\left| \begin{matrix}
c&		b&		\cdots&		b\\
c&		a&		\cdots&		b\\
\vdots&		\vdots&		&		\vdots\\
c&		c&		\cdots&		a\\
\end{matrix} \right|+\left| \begin{matrix}
a-c&		b&		\cdots&		b\\
0&		a&		\cdots&		b\\
\vdots&		\vdots&		&		\vdots\\
0&		c&		\cdots&		a\\
\end{matrix} \right|
\\
&=c\left| \begin{matrix}
1&		b&		\cdots&		b\\
1&		a&		\cdots&		b\\
\vdots&		\vdots&		&		\vdots\\
1&		c&		\cdots&		a\\
\end{matrix} \right|+\left( a-c \right) D_{n-1}=c\left| \begin{matrix}
1&		0&		\cdots&		0\\
1&		a-b&		\cdots&		0\\
\vdots&		\vdots&		&		\vdots\\
1&		c-b&		\cdots&		a-b\\
\end{matrix} \right|+\left( a-c \right) D_{n-1}
\\
&=c\left( a-b \right) ^{n-1}++\left( a-c \right) D_{n-1}.\left( n\ge 2 \right) 
\end{align*}
若$b\ne c$,则联立上面两式可得
\begin{align*}
\left| \boldsymbol{A} \right|=D_n=\frac{b\left( a-c \right) ^n-c\left( a-b \right) ^n}{b-c}.
\nonumber
\end{align*}
若$b=c$,则由上面式子递推可得
\begin{align*}
\left| \boldsymbol{A} \right|=D_n&=b\left( a-b \right) ^{n-1}+\left( a-b \right) D_{n-1}
\\
&=b\left( a-b \right) ^{n-1}+\left( a-b \right) \left[ b\left( a-b \right) ^{n-2}+\left( a-b \right) D_{n-2} \right] 
=2b\left( a-b \right) ^{n-1}+\left( a-b \right) ^2D_{n-2}
\\
&=\cdots =\left( n-1 \right) b\left( a-b \right) ^{n-1}+\left( a-b \right) ^{n-1}D_1
\\
&=\left( n-1 \right) b\left( a-b \right) ^{n-1}+\left( a-b \right) ^{n-1}a
\\
&=\left[ a+\left( n-1 \right) b \right] \left( a-b \right) ^{n-1}.
\end{align*}

当$b=c$时,也可以由练习\ref{大/小拆分法例题1}可知
\begin{align*}
|\boldsymbol{A}|=\left[ a+\left( n-1 \right) b \right] \left( a-b \right) ^{n-1}.
\nonumber
\end{align*}
\end{solution}

\begin{exercise}
设\(f_1(x), f_2(x), \cdots, f_n(x)\)是次数不超过\(n - 2\)的多项式,求证:对任意\(n\)个数\(a_1, a_2, \cdots, a_n\),均有
\begin{align*}
\begin{vmatrix}
f_1(a_1) & f_2(a_1) & \cdots & f_n(a_1) \\
f_1(a_2) & f_2(a_2) & \cdots & f_n(a_2) \\
\vdots & \vdots & \ddots & \vdots \\
f_1(a_n) & f_2(a_n) & \cdots & f_n(a_n)
\end{vmatrix} = 0.
\end{align*}
\end{exercise}
\begin{proof}
{\color{blue}\text{证法一(\hyperlink{大拆分法}{大拆分法}):}}
因为\(f_k(x)(1 \leq k \leq n)\)的次数不超过\(n - 2\),所以它们都是单项式\(1,x,\cdots,x^{n - 2}\)的线性组合.将原行列式中每一列的多项式都按这\(n - 1\)个单项式进行拆分,最后得到至多$(n-1)!$个简单行列式之和,这些行列式中每一列的多项式只是单项式.由于每个简单行列式都有\(n\)列,根据抽屉原理,每个简单行列式中至少有两列是共用同一个单项式(可能相差一个常系数),于是这两列成比例,从而所有这样的简单行列式都等于零,因此原行列式也等于零.

{\color{blue}\text{证法二(\hyperlink{多项式根的有限性}{多项式根的有限性}):}}
令$f\left( x \right) =\left| \begin{matrix}
f_1(x)&		f_2(a_1)&		\cdots&		f_n(a_1)\\
f_1(x)&		f_2(a_2)&		\cdots&		f_n(a_2)\\
\vdots&		\vdots&		\ddots&		\vdots\\
f_1(x)&		f_2(a_n)&		\cdots&		f_n(a_n)\\
\end{matrix} \right|$,则将$f(x)$按第一列展开得到
\begin{align*}
f\left( x \right) =k_1f_1\left( x \right) +k_2f_2\left( x \right) +\cdots +k_nf_n\left( x \right) .
\end{align*}
其中$k_i$为行列式$f\left( x \right)$的第$\left( i,1 \right)$元素的代数余子式,$i=1,2,\cdots ,n$.

注意$k_i$与$x$无关,均为常数.若$f(x)$不恒为0,则又因为\(f_k(x)(1 \leq k \leq n)\)的次数不超过\(n - 2\),所以$degf(x)\le n-2$.
但是,注意到$f(a_2)=f(a_3)=\cdots=f(a_n)=0$,即$f(x)$有$n-1$个根.于是由\hyperlink{余数定理}{余数定理}可知,$\left( x-a_2 \right) \cdots \left( x-a_n \right) |f\left( x \right)$.从而$n-1=deg\left( x-a_2 \right) \cdots \left( x-a_n \right) \ge degf\left( x \right)$.这与$degf(x)\le n-2$矛盾.故$f(x)\equiv 0$,当然也有$f(a_1)=0$.

{\color{blue}证法三:}

设多项式
\[
f_k(x)=c_{k,n - 2}x^{n - 2}+\cdots+c_{k1}x + c_{k0},1\leq k\leq n.
\]
则有如下的矩阵分解:
\[
\begin{pmatrix}
f_1(a_1) & f_2(a_1) & \cdots & f_n(a_1)\\
f_1(a_2) & f_2(a_2) & \cdots & f_n(a_2)\\
\vdots & \vdots & & \vdots\\
f_1(a_n) & f_2(a_n) & \cdots & f_n(a_n)
\end{pmatrix}
=
\begin{pmatrix}
1 & a_1 & \cdots & a_1^{n - 2}\\
1 & a_2 & \cdots & a_2^{n - 2}\\
\vdots & \vdots & & \vdots\\
1 & a_n & \cdots & a_n^{n - 2}
\end{pmatrix}
\begin{pmatrix}
c_{10} & c_{20} & \cdots & c_{n0}\\
c_{11} & c_{21} & \cdots & c_{n1}\\
\vdots & \vdots & & \vdots\\
c_{1,n - 2} & c_{2,n - 2} & \cdots & c_{n,n - 2}
\end{pmatrix}.
\]
注意到上式右边的两个矩阵分别是\(n\times(n - 1)\)和\((n - 1)\times n\)矩阵,故由\hyperref[theorem:Cauchy-Binet公式]{Cauchy - Binet公式}马上得到左边矩阵的行列式值等于零.
\end{proof}

\begin{exercise}
计算$n$阶行列式:
\begin{align}
D_n=\left| \begin{matrix}
x_1&		y&		y&		\cdots&		y&		y\\
z&		x_2&		y&		\cdots&		y&		y\\
z&		z&		x_3&		\cdots&		y&		y\\
\vdots&		\vdots&		\vdots&		&		\vdots&		\vdots\\
z&		z&		z&		\cdots&		x_{n-1}&		y\\
z&		z&		z&		\cdots&		z&		x_n\\
\end{matrix} \right|.
\nonumber
\end{align}
\end{exercise}
\begin{solution}(\hyperref[小拆分法]{小拆分法})
对第$n$列进行拆分即可得到递推式:
(对第1或n行(或列)拆分都可以得到相同结果)
\begin{align}
&D_n=\left| \begin{matrix}
x_1&		y&		y&		\cdots&		y&		y+0\\
z&		x_2&		y&		\cdots&		y&		y+0\\
z&		z&		x_3&		\cdots&		y&		y+0\\
\vdots&		\vdots&		\vdots&		&		\vdots&		\vdots\\
z&		z&		z&		\cdots&		x_{n-1}&		y+0\\
z&		z&		z&		\cdots&		z&		y+x_n-y\\
\end{matrix} \right|=\left| \begin{matrix}
x_1&		y&		y&		\cdots&		y&		y\\
z&		x_2&		y&		\cdots&		y&		y\\
z&		z&		x_3&		\cdots&		y&		y\\
\vdots&		\vdots&		\vdots&		&		\vdots&		\vdots\\
z&		z&		z&		\cdots&		x_{n-1}&		y\\
z&		z&		z&		\cdots&		z&		y\\
\end{matrix} \right|+\left| \begin{matrix}
x_1&		y&		y&		\cdots&		y&		0\\
z&		x_2&		y&		\cdots&		y&		0\\
z&		z&		x_3&		\cdots&		y&		0\\
\vdots&		\vdots&		\vdots&		&		\vdots&		\vdots\\
z&		z&		z&		\cdots&		x_{n-1}&		0\\
z&		z&		z&		\cdots&		z&		x_n-y\\
\end{matrix} \right|
\nonumber\\
&=\left| \begin{matrix}
x_1-z&		0&		0&		\cdots&		0&		0\\
0&		x_2-z&		0&		\cdots&		0&		0\\
0&		0&		x_3-z&		\cdots&		0&		0\\
\vdots&		\vdots&		\vdots&		&		\vdots&		\vdots\\
0&		0&		0&		\cdots&		x_{n-1}-z&		0\\
z&		z&		z&		\cdots&		z&		y\\
\end{matrix} \right|+\left( x_n-y \right) D_{n-1}=y\prod\limits_{i=1}^{n-1}{\left( x_i-z \right)}+\left( x_n-y \right) D_{n-1}.
\label{eq:递推式1.2}
\end{align}
将原行列式转置后,同理可得
\begin{align}
&D_n=D_{n}^{T}=\left| \begin{matrix}
x_1&		z&		z&		\cdots&		z&		z+0\\
y&		x_2&		z&		\cdots&		z&		z+0\\
y&		y&		x_3&		\cdots&		z&		z+0\\
\vdots&		\vdots&		\vdots&		&		\vdots&		\vdots\\
y&		y&		y&		\cdots&		x_{n-1}&		z+0\\
y&		y&		y&		\cdots&		y&		z+x_n-z\\
\end{matrix} \right|=\left| \begin{matrix}
x_1&		z&		z&		\cdots&		z&		z\\
y&		x_2&		z&		\cdots&		z&		z\\
y&		y&		x_3&		\cdots&		z&		z\\
\vdots&		\vdots&		\vdots&		&		\vdots&		\vdots\\
y&		y&		y&		\cdots&		x_{n-1}&		z\\
y&		y&		y&		\cdots&		y&		z\\
\end{matrix} \right|+\left| \begin{matrix}
x_1&		z&		z&		\cdots&		z&		0\\
y&		x_2&		z&		\cdots&		z&		0\\
y&		y&		x_3&		\cdots&		z&		0\\
\vdots&		\vdots&		\vdots&		&		\vdots&		\vdots\\
y&		y&		y&		\cdots&		x_{n-1}&		0\\
y&		y&		y&		\cdots&		y&		x_n-z\\
\end{matrix} \right|
\nonumber\\
&=\left| \begin{matrix}
x_1-y&		0&		0&		\cdots&		0&		0\\
0&		x_2-y&		0&		\cdots&		0&		0\\
0&		0&		x_3-y&		\cdots&		0&		0\\
\vdots&		\vdots&		\vdots&		&		\vdots&		\vdots\\
0&		0&		0&		\cdots&		x_{n-1}-y&		0\\
y&		y&		y&		\cdots&		y&		z\\
\end{matrix} \right|+\left( x_n-z \right) D_{n-1}^{T}=z\prod\limits_{i=1}^{n-1}{\left( x_i-y \right)}+\left( x_n-z \right) D_{n-1}.
\label{eq:递推式1.3}
\end{align}
若$z\ne y$,则联立\eqref{eq:递推式1.2}\eqref{eq:递推式1.3}式,解得
\begin{equation}
D_n=\frac{1}{z-y}\biggl[ z\prod\limits_{i=1}^n{(x_i}-y)-y\prod\limits_{i=1}^n{(x_i}-z) \biggr];
\nonumber
\end{equation}
若$z= y$,则由\eqref{eq:递推式1.2}式递推可得
\begin{equation}
\begin{split}
D_n&=y\prod\limits_{i=1}^{n-1}{\left( x_i-y \right)}+\left( x_n-y \right) D_{n-1}
\\
&=y\prod\limits_{i=1}^{n-1}{\left( x_i-y \right)}+\left( x_n-y \right) \left( y\prod\limits_{i=1}^{n-2}{\left( x_i-y \right)}+\left( x_{n-1}-y \right) D_{n-2} \right) 
\\
&=y\prod\limits_{j\ne n}^{}{\left( x_i-y \right)}+y\prod\limits_{j\ne n-1}^{}{\left( x_i-y \right)}+\left( x_n-y \right) \left( x_{n-1}-y \right) D_{n-2}
\\
&=\cdots =y\sum_{i=1}^n{\prod\limits_{j\ne i}{(x_j}}-y)+\prod\limits_{i=1}^n{(x_i}-y)D_0
\\
&=y\sum_{i=1}^n{\prod\limits_{j\ne i}{(x_j}}-y)+\prod\limits_{i=1}^n{(x_i}-y).
\end{split}
\nonumber
\end{equation}
\end{solution}

\begin{exercise}
求下列$n$阶行列式的值:
\begin{align*}
D_n = 
\begin{vmatrix}
1 + a_1^2 & a_1 a_2 & \cdots & a_1 a_n \\
a_2 a_1 & 1 + a_2^2 & \cdots & a_2 a_n \\
\vdots & \vdots & \ddots & \vdots \\
a_n a_1 & a_n a_2 & \cdots & 1 + a_n^2
\end{vmatrix}
\end{align*}
\end{exercise}
\begin{note}
本题行列式每行或每列求和后得到的结果不具备明显的规律性,故不适合使用\hyperref[行列式计算:求和法]{求和法}.

本题行列式难以找到合适的$t$对其进行\hyperref[大拆分法]{大拆分},故也不适合使用大拆分法.(并且因为难以找到合适的$t_i$,所以\hyperref[大拆分法的推广]{推广的大拆分}也不行)
\end{note}
\begin{solution}
(\hyperlink{小拆分法}{小拆分法})
将$D_n$最后一列拆成两列得
\begin{align*}
D_n&=\left| \begin{matrix}
1+a_{1}^{2}&		a_1a_2&		\cdots&		a_1a_n\\
a_2a_1&		1+a_{2}^{2}&		\cdots&		a_2a_n\\
\vdots&		\vdots&		\ddots&		\vdots\\
a_na_1&		a_na_2&		\cdots&		1+a_{n}^{2}\\
\end{matrix} \right|=\left| \begin{matrix}
1+a_{1}^{2}&		a_1a_2&		\cdots&		a_1a_n\\
a_2a_1&		1+a_{2}^{2}&		\cdots&		a_2a_n\\
\vdots&		\vdots&		\ddots&		\vdots\\
a_na_1&		a_na_2&		\cdots&		a_{n}^{2}\\
\end{matrix} \right|+\left| \begin{matrix}
1+a_{1}^{2}&		a_1a_2&		\cdots&		0\\
a_2a_1&		1+a_{2}^{2}&		\cdots&		0\\
\vdots&		\vdots&		\ddots&		\vdots\\
a_na_1&		a_na_2&		\cdots&		1\\
\end{matrix} \right|
\\
&=\left| \begin{matrix}
1+a_{1}^{2}&		a_1a_2&		\cdots&		a_1a_n\\
a_2a_1&		1+a_{2}^{2}&		\cdots&		a_2a_n\\
\vdots&		\vdots&		\ddots&		\vdots\\
a_na_1&		a_na_2&		\cdots&		a_{n}^{2}\\
\end{matrix} \right|+D_{n-1}.
\end{align*}
若$a_n\ne0$,则由上式可得
\begin{align*}
D_n=a_n\left| \begin{matrix}
1+a_{1}^{2}&		a_1a_2&		\cdots&		a_1\\
a_2a_1&		1+a_{2}^{2}&		\cdots&		a_2\\
\vdots&		\vdots&		\ddots&		\vdots\\
a_na_1&		a_na_2&		\cdots&		a_n\\
\end{matrix} \right|+D_{n-1}\xlongequal[i=1,2,\cdots ,n]{\text{对第一个行列式}:-a_ij_n+j_i}a_n\left| \begin{matrix}
1&		0&		\cdots&		a_1\\
0&		1&		\cdots&		a_2\\
\vdots&		\vdots&		\ddots&		\vdots\\
0&		0&		\cdots&		a_n\\
\end{matrix} \right|+D_{n-1}=a_{n}^{2}+D_{n-1}.\left( n\ge 2 \right) 
\end{align*}
若$a_n=0$,则上面第一个行列式等于0,进而$D_n=D_{n-1}(n\ge0)$.仍然满足上述递推式.

从而由上式递推可得
\begin{align*}
D_n=a_{n}^{2}+D_{n-1}=a_{n}^{2}+\left( a_{n-1}^{2}+D_{n-2} \right) =\cdots =\sum_{i=2}^n{a_{i}^{2}}+D_1=1+\sum_{i=1}^n{a_{i}^{2}}.
\end{align*}
\end{solution}

\begin{exercise}\label{Vandermode行列式三角函数例题}
求下列行列式的值:
\begin{align*}
|\boldsymbol{A}|=\left| \begin{matrix}
1&		\cos \theta _1&		\cos 2\theta _1&		\cdots&		\cos\mathrm{(}n-1)\theta _1\\
1&		\cos \theta _2&		\cos 2\theta _2&		\cdots&		\cos\mathrm{(}n-1)\theta _2\\
\vdots&		\vdots&		\vdots&		&		\vdots\\
1&		\cos \theta _n&		\cos 2\theta _n&		\cdots&		\cos\mathrm{(}n-1)\theta _n\\
\end{matrix} \right|.
\end{align*}
\end{exercise}
\begin{solution}
由\(De\,\,Moivre\)公式及二项式定理,可得
\begin{align*}
&\cos k\theta +\mathrm{i}\sin k\theta =(\cos \theta +\mathrm{i}\sin \theta )^k
\\
&=\cos ^k\theta +\mathrm{iC}_{k}^{1}\cos ^{k-1}\theta \sin \theta -\mathrm{C}_{k}^{2}\cos ^{k-2}\theta \sin ^2\theta +\mathrm{iC}_{k}^{3}\cos ^{k-3}\theta \sin ^3\theta -\cdots 
\\
&=\cos ^k\theta +\mathrm{iC}_{k}^{1}\cos ^{k-1}\theta \sin \theta -\mathrm{C}_{k}^{2}\cos ^{k-2}\theta \left( 1-\cos ^2\theta \right) +\mathrm{iC}_{k}^{3}\cos ^{k-3}\theta \sin ^3\theta -\cdots 
\end{align*}
比较实部可得
\begin{align*}
\cos k\theta& =\cos ^k\theta \left( 1+\mathrm{C}_{k}^{2}+\mathrm{C}_{k}^{4}+\cdots \right) -\mathrm{C}_{k}^{2}\cos ^{k-2}+\mathrm{C}_{k}^{4}\cos ^{k-4}-\cdots 
\\
&\hyperlink{组合式计算常用公式}{=}2^{k-1}\cos ^k\theta -\mathrm{C}_{k}^{2}\cos ^{k-2}+\mathrm{C}_{k}^{4}\cos ^{k-4}-\cdots 
\end{align*}

利用这个事实,依次将原行列式各列表示成\(\cos \theta _j\)(\(j = 2,3,\cdots,n\))的多项式.

再利用行列式的性质,可依次将第\(3,4,\cdots,n\)列消去除最高次项外的其他项,从而得到
\begin{align*}
|\boldsymbol{A}|&=\left| \begin{matrix}
1&		\cos \theta _1&		2\cos ^2\theta _1&		\cdots&		2^{n-2}\cos ^{n-1}\theta _1\\
1&		\cos \theta _2&		2\cos ^2\theta _2&		\cdots&		2^{n-2}\cos ^{n-1}\theta _2\\
\vdots&		\vdots&		\vdots&		&		\vdots\\
1&		\cos \theta _n&		2\cos ^2\theta _n&		\cdots&		2^{n-2}\cos ^{n-1}\theta _n\\
\end{matrix} \right|=2^{\frac{1}{2}(n-1)(n-2)}\left| \begin{matrix}
1&		\cos \theta _1&		\cos ^2\theta _1&		\cdots&		\cos ^{n-1}\theta _1\\
1&		\cos \theta _2&		\cos ^2\theta _2&		\cdots&		\cos ^{n-1}\theta _2\\
\vdots&		\vdots&		\vdots&		&		\vdots\\
1&		\cos \theta _n&		\cos ^2\theta _n&		\cdots&		\cos ^{n-1}\theta _n\\
\end{matrix} \right|
\\
&=2^{\frac{1}{2}(n-1)(n-2)}\prod_{1\le i<j\le n}{\left( \cos \theta _j-\cos \theta _i \right)}.
\end{align*}
\end{solution}
\begin{conclusion}
\hypertarget{组合式计算常用公式}{组合式计算常用公式:}

(1)$\mathrm{C}_{n}^{m}=\mathrm{C}_{n-1}^{m}+\mathrm{C}_{n-1}^{m-1}$

(2)$\mathrm{C}_{n}^{0}+\mathrm{C}_{n}^{2}+\cdots =\mathrm{C}_{n}^{1}+\mathrm{C}_{n}^{3}+\cdots =2^{n-1}$

证明:(1)\begin{align*}
\mathrm{C}_{n}^{m}&=\frac{n!}{m!\left( n-m \right) !}=\frac{\left( n-1 \right) !\left( n-m+m \right)}{m!\left( n-m \right) !}=\frac{\left( n-1 \right) !\left( n-m \right)}{m!\left( n-m \right) !}+\frac{\left( n-1 \right) !m}{m!\left( n-m \right) !}
\\
&=\frac{\left( n-1 \right) !}{m!\left( n-m-1 \right) !}+\frac{\left( n-1 \right) !}{\left( m-1 \right) !\left( n-m \right) !}=\mathrm{C}_{n-1}^{m}+\mathrm{C}_{n-1}^{m-1}
\end{align*}
(2)(i)当\(n\)为奇数时,由\(C_{n}^{m}=C_{n - 1}^{m - 1} + C_{n - 1}^{m}\),可得
\begin{align*}
&C_{n}^{0} + C_{n}^{2} + C_{n}^{4} \cdots + C_{n}^{n - 1} = C_{n - 1}^{0} + C_{n - 1}^{1} + C_{n - 1}^{2} + C_{n - 1}^{3} + C_{n - 1}^{4} \cdots + C_{n - 1}^{n - 2} + C_{n - 1}^{n - 1}
\\
&C_{n}^{1} + C_{n}^{3} + C_{n}^{5} \cdots + C_{n}^{n} = C_{n - 1}^{0} + C_{n - 1}^{1} + C_{n - 1}^{2} + C_{n - 1}^{3} + C_{n - 1}^{4} + C_{n - 1}^{5} + \cdots + C_{n - 1}^{n - 1} + C_{n - 1}^{n}
\end{align*}
由于\(C_{n - 1}^{n} = 0\),再对比上面两式每一项可知,上面两式相等.

而上面两式相加,得$
C_{n}^{0} + C_{n}^{1} + C_{n}^{2} \cdots + C_{n}^{n - 1} + C_{n}^{n} = (1 + 1)^n = 2^n.$

故\(C_{n}^{0} + C_{n}^{2} + C_{n}^{4} \cdots + C_{n}^{n - 1} = C_{n}^{1} + C_{n}^{3} + C_{n}^{5} \cdots + C_{n}^{n} = 2^{n - 1}\).

(ii)当\(n\)为偶数时,由\(C_{n}^{m} = C_{n - 1}^{m - 1} + C_{n - 1}^{m}\),可得
\begin{align*}
&C_{n}^{0} + C_{n}^{2} + C_{n}^{4} \cdots + C_{n}^{n} = C_{n - 1}^{0} + C_{n - 1}^{1} + C_{n - 1}^{2} + C_{n - 1}^{3} + C_{n - 1}^{4} \cdots + C_{n - 1}^{n - 1} + C_{n - 1}^{n} 
\\
&C_{n}^{1} + C_{n}^{3} + C_{n}^{5} \cdots + C_{n}^{n - 1} = C_{n - 1}^{0} + C_{n - 1}^{1} + C_{n - 1}^{2} + C_{n - 1}^{3} + C_{n - 1}^{4} + C_{n - 1}^{5} + \cdots + C_{n - 1}^{n - 2} + C_{n - 1}^{n - 1}
\end{align*}
由于\(C_{n - 1}^{n} = 0\),再对比上面两式每一项可知,上面两式相等.

而上面两式相加,得
$C_{n}^{0} + C_{n}^{1} + C_{n}^{2} \cdots + C_{n}^{n - 1} + C_{n}^{n} = (1 + 1)^n = 2^n.$

故\(C_{n}^{0} + C_{n}^{2} + C_{n}^{4} \cdots + C_{n}^{n - 1} = C_{n}^{1} + C_{n}^{3} + C_{n}^{5} \cdots + C_{n}^{n} = 2^{n - 1}\).

综上所述,\(C_{n}^{0} + C_{n}^{2} + \cdots = C_{n}^{1} + C_{n}^{3} + \cdots = 2^{n - 1}\). 
\end{conclusion}

\begin{exercise}
求下列行列式式的值:
\begin{align*}
|\boldsymbol{A}|=\left| \begin{matrix}
\sin \theta _1&		\sin 2\theta _1&		\cdots&		\sin n\theta _1\\
\sin \theta _2&		\sin 2\theta _2&		\cdots&		\sin n\theta _2\\
\vdots&		\vdots&		&		\vdots\\
\sin \theta _n&		\sin 2\theta _n&		\cdots&		\sin n\theta _n\\
\end{matrix} \right|.
\end{align*}
\end{exercise}
\begin{note}
可以利用\hyperref[Vandermode行列式三角函数例题]{上一题}类似的方法求解.但我们给出另外一种解法,目的是直接利用\hyperref[Vandermode行列式三角函数例题]{上一题}的结论.
\end{note}
\begin{solution}
根据和差化积公式,可得
\begin{align*}
\sin k\theta -\sin \left( k-2 \right) \theta =2\sin \theta \cos \left( k-1 \right) \theta ,k=2,3,\cdots ,n.
\end{align*}
再结合上一题结论,可得
\begin{align*}
|\boldsymbol{A}|&=\left| \begin{matrix}
\sin \theta _1&		\sin 2\theta _1&		\cdots&		\sin n\theta _1\\
\sin \theta _2&		\sin 2\theta _2&		\cdots&		\sin n\theta _2\\
\vdots&		\vdots&		&		\vdots\\
\sin \theta _n&		\sin 2\theta _n&		\cdots&		\sin n\theta _n\\
\end{matrix} \right|=\left| \begin{matrix}
\sin \theta _1&		2\sin \theta _1\cos \theta _1&		\cdots&		2\sin \theta _1\cos \left( n-1 \right) \theta _1\\
\sin \theta _2&		2\sin \theta _2\cos \theta _2&		\cdots&		2\sin \theta _2\cos \left( n-1 \right) \theta _2\\
\vdots&		\vdots&		&		\vdots\\
\sin \theta _n&		2\sin \theta _n\cos \theta _n&		\cdots&		2\sin \theta _n\cos \left( n-1 \right) \theta _n\\
\end{matrix} \right|
\\
&=2^{n-1}\prod_{i=1}^n{\sin \theta _i}\left| \begin{matrix}
\cos \theta _1&		\cos 2\theta _1&		\cdots&		\cos (n-1)\theta _1\\
\cos \theta _2&		\cos 2\theta _2&		\cdots&		\cos (n-1)\theta _2\\
\vdots&		\vdots&		&		\vdots\\
\cos \theta _n&		\cos 2\theta _n&		\cdots&		\cos (n-1)\theta _n\\
\end{matrix} \right|=2^{\frac{1}{2}\left( n-2 \right) \left( n-1 \right) +n-1}\prod_{i=1}^n{\sin \theta _i}\prod_{1\le i<j\le n}{\left( \cos \theta _j-\cos \theta _i \right)}
\\
&=2^{\frac{1}{2}n\left( n-1 \right)}\prod_{i=1}^n{\sin \theta _i}\prod_{1\le i<j\le n}{\left( \cos \theta _j-\cos \theta _i \right)}.
\end{align*}
\end{solution}

\begin{exercise}\label{升阶法的应用(1)例题}
计算$n$阶行列式:
\begin{align*}
|\boldsymbol{A}|=\left| \begin{matrix}
1+x_1&		1+x_2&		\cdots&		1+x_{1}^{n}\\1+x_{1}^{2}
&		1+x_{2}^{2}&		\cdots&		1+x_{2}^{n}\\
\vdots&		\vdots&		&		\vdots\\
1+x_n&		1+x_{n}^{2}&		\cdots&		1+x_{n}^{n}\\
\end{matrix} \right|.
\end{align*}
\end{exercise}
\begin{note}
本题也可以使用\hyperref[大拆分法]{大拆分法}进行求解.但我们以本题为例介绍利用\textbf{升阶法}计算行列式.
\end{note}
\begin{solution}
{\color{blue}解法一\hyperref[行列式计算:升阶法]{升阶法}:}
\begin{align*}
|\boldsymbol{A}|&=\left| \begin{matrix}
1&		0&		0&		\cdots&		0\\
1&		1+x_1&		1+x_{1}^{2}&		\cdots&		1+x_{1}^{n}\\
1&		1+x_2&		1+x_{2}^{2}&		\cdots&		1+x_{2}^{n}\\
\vdots&		\vdots&		\vdots&		&		\vdots\\
1&		1+x_n&		1+x_{n}^{2}&		\cdots&		1+x_{n}^{n}\\
\end{matrix} \right|=\left| \begin{matrix}
1&		-1&		-1&		\cdots&		-1\\
1&		x_1&		x_{1}^{2}&		\cdots&		x_{1}^{n}\\
1&		x_2&		x_{2}^{2}&		\cdots&		x_{2}^{n}\\
\vdots&		\vdots&		\vdots&		&		\vdots\\
1&		x_n&		x_{n}^{2}&		\cdots&		x_{n}^{n}\\
\end{matrix} \right|
\\
&\xlongequal{\hyperlink{小拆分法}{\text{小拆分法}}}\left| \begin{matrix}
2&		0&		0&		\cdots&		0\\
1&		x_1&		x_{1}^{2}&		\cdots&		x_{1}^{n}\\
1&		x_2&		x_{2}^{2}&		\cdots&		x_{2}^{n}\\
\vdots&		\vdots&		\vdots&		&		\vdots\\
1&		x_n&		x_{n}^{2}&		\cdots&		x_{n}^{n}\\
\end{matrix} \right|+\left| \begin{matrix}
-1&		-1&		-1&		\cdots&		-1\\
1&		x_1&		x_{1}^{2}&		\cdots&		x_{1}^{n}\\
1&		x_2&		x_{2}^{2}&		\cdots&		x_{2}^{n}\\
\vdots&		\vdots&		\vdots&		&		\vdots\\
1&		x_n&		x_{n}^{2}&		\cdots&		x_{n}^{n}\\
\end{matrix} \right|
\\
&=2\left| \begin{matrix}
x_1&		x_{1}^{2}&		\cdots&		x_{1}^{n}\\
x_2&		x_{2}^{2}&		\cdots&		x_{2}^{n}\\
\vdots&		\vdots&		&		\vdots\\
x_n&		x_{n}^{2}&		\cdots&		x_{n}^{n}\\
\end{matrix} \right|-\left| \begin{matrix}
1&		1&		1&		\cdots&		1\\
1&		x_1&		x_{1}^{2}&		\cdots&		x_{1}^{n}\\
1&		x_2&		x_{2}^{2}&		\cdots&		x_{2}^{n}\\
\vdots&		\vdots&		\vdots&		&		\vdots\\
1&		x_n&		x_{n}^{2}&		\cdots&		x_{n}^{n}\\
\end{matrix} \right|
\\
&=2x_1x_2\cdots x_n\left| \begin{matrix}
1&		x_1&		\cdots&		x_{1}^{n-1}\\
1&		x_2&		\cdots&		x_{2}^{n-1}\\
\vdots&		\vdots&		&		\vdots\\
1&		x_n&		\cdots&		x_{n}^{n-1}\\
\end{matrix} \right|-\left( x_1-1 \right) \left( x_2-1 \right) \cdots \left( x_n-1 \right) \prod_{1\le i<j\le n}{\left( x_j-x_i \right)}
\\
&=2x_1x_2\cdots x_n\prod_{1\le i<j\le n}{\left( x_j-x_i \right)}-\left( x_1-1 \right) \left( x_2-1 \right) \cdots \left( x_n-1 \right) \prod_{1\le i<j\le n}{\left( x_j-x_i \right)}
\\
&=\left[ 2x_1x_2\cdots x_n-\left( x_1-1 \right) \left( x_2-1 \right) \cdots \left( x_n-1 \right) \right] \prod_{1\le i<j\le n}{\left( x_j-x_i \right)}.
\end{align*}

{\color{blue}解法二(\hyperref[大拆分法]{大拆分法}):}
设\(\vert\boldsymbol{B}(t)\vert=\left|\begin{matrix}
x_1 + t & x_{1}^{2} + t & \cdots & x_{1}^{n} + t\\
x_2 + t & x_{2}^{2} + t & \cdots & x_{2}^{n} + t\\
\vdots & \vdots &  & \vdots\\
x_n + t & x_{n}^{2} + t & \cdots & x_{n}^{n} + t
\end{matrix}\right|\),且\(B_{ij}\)是\(\vert\boldsymbol{B}(0)\vert\)的第\((i,j)\)元素的代数余子式.

根据行列式的性质将\(\vert\boldsymbol{A}\vert\)每一列都拆分成两列,然后按\(t\)所在的列展开得到
\begin{gather*}
\vert\boldsymbol{A}\vert=\vert\boldsymbol{B}(1)\vert=\vert\boldsymbol{B}(0)\vert+\sum_{i,j = 1}^{n}B_{ij},
\\
\vert\boldsymbol{B}(-1)\vert=\vert\boldsymbol{B}(0)\vert-\sum_{i,j = 1}^{n}B_{ij}.
\end{gather*}
于是\(\vert\boldsymbol{A}\vert = 2\vert\boldsymbol{B}(0)\vert - \vert\boldsymbol{B}(-1)\vert\).注意到
\begin{align*}
\vert\boldsymbol{B}(0)\vert=\left|\begin{matrix}
x_1 & x_{1}^{2} & \cdots & x_{1}^{n}\\
x_2 & x_{2}^{2} & \cdots & x_{2}^{n}\\
\vdots & \vdots &  & \vdots\\
x_n & x_{n}^{2} & \cdots & x_{n}^{n}
\end{matrix}\right|=x_1x_2\cdots x_n\left|\begin{matrix}
1 & x_1 & \cdots & x_{1}^{n}\\
1 & x_2 & \cdots & x_{2}^{n}\\
\vdots & \vdots &  & \vdots\\
1 & x_n & \cdots & x_{n}^{n}
\end{matrix}\right|=x_1x_2\cdots x_n\prod_{1\leqslant i < j\leqslant n}(x_j - x_i). 
\end{align*}
又由行列式性质可得
\begin{align*}
\vert\boldsymbol{B}(-1)\vert&=\left|\begin{matrix}
x_1 - 1 & x_{1}^{2} - 1 & \cdots & x_{1}^{n} - 1\\
x_2 - 1 & x_{2}^{2} - 1 & \cdots & x_{2}^{n} - 1\\
\vdots & \vdots &  & \vdots\\
x_n - 1 & x_{n}^{2} - 1 & \cdots & x_{n}^{n} - 1
\end{matrix}\right|
=(x_1 - 1)(x_2 - 1)\cdots (x_n - 1)\left|\begin{matrix}
1 & x_1 + 1 & \cdots & x_{1}^{n - 1} + x_{1}^{n - 2}\cdots + x_1 + 1\\
1 & x_2 + 1 & \cdots & x_{2}^{n - 1} + x_{2}^{n - 2}\cdots + x_2 + 1\\
\vdots & \vdots &  & \vdots\\
1 & x_n + 1 & \cdots & x_{n}^{n - 1} + x_{n}^{n - 2}\cdots + x_n + 1
\end{matrix}\right|
\\
&=(x_1 - 1)(x_2 - 1)\cdots (x_n - 1)\left|\begin{matrix}
1 & x_1 & \cdots & x_{1}^{n - 1}\\
1 & x_2 & \cdots & x_{2}^{n - 1}\\
\vdots & \vdots &  & \vdots\\
1 & x_n & \cdots & x_{n}^{n - 1}
\end{matrix}\right|
=(x_1 - 1)(x_2 - 1)\cdots (x_n - 1)\prod_{1\leqslant i < j\leqslant n}(x_j - x_i).
\end{align*}
故可得
\begin{align*}
\vert\boldsymbol{A}\vert &= 2\vert\boldsymbol{B}(0)\vert - \vert\boldsymbol{B}(-1)\vert
=2x_1x_2\cdots x_n\prod_{1\leqslant i < j\leqslant n}(x_j - x_i)-(x_1 - 1)(x_2 - 1)\cdots (x_n - 1)\prod_{1\leqslant i < j\leqslant n}(x_j - x_i)
\\
&=\left[2x_1x_2\cdots x_n-(x_1 - 1)(x_2 - 1)\cdots (x_n - 1)\right]\prod_{1\leqslant i < j\leqslant n}(x_j - x_i).
\end{align*}
\end{solution}
\begin{conclusion}\label{行列式计算:升阶法}
\hypertarget{行列式计算:升阶法}{\textbf{升阶法:}}
将原行列式加上一行和一列使得到到新行列式的阶数比原行列式要高一阶.

\textbf{升阶法的应用:}

(1)当原行列式每一行具有相同的结构时,我们可以在原行列式的基础上加上一行和一列,新加上的一列和一行需要满足:新的一列除了与新的一行交叉位置的元素为1外其余全为0(这样才能保证新的行列式按新的一行或一列展开后与原行列式相同),并且新加上的一行除1以外其他位置的元素就取原行列式中每一行所具有的相同结构(这样可以利用行列式的性质将每一行中的相同的结构减去,进而达到简化原行列式的目的).具体例子见练习\ref{升阶法的应用(1)例题}.

(2)当原行列式是我们由熟悉的行列式去掉某一行、或某一列、或某一行和一列得到的,我们可以在原行列式的基础上补充上缺少的那一行和一列,再进行计算得到新行列式的式子.再将新行列式按照新添加的一行或一列展开得到的对应元素乘与其对应的代数余子式,而新添加的一行和一列交叉位置的元素对应的余子式就是原行列式,最后两边式子比较系数一般就能得到原行列式的值.
具体例子见练习\ref{升阶法的应用(2)例题}.
\end{conclusion}

\begin{exercise}\label{升阶法的应用(2)例题}
求下列$n$阶行列式的值($1\le i\le n-1$):
\begin{align*}
|\boldsymbol{A}|=\left| \begin{matrix}
1&		x_1&		\cdots&		x_{1}^{i-1}&		x_{1}^{i+1}&		\cdots&		x_{1}^{n}\\
1&		x_2&		\cdots&		x_{2}^{i-1}&		x_{2}^{i+1}&		\cdots&		x_{2}^{n}\\
\vdots&		\vdots&		&		\vdots&		\vdots&		&		\vdots\\
1&		x_n&		\cdots&		x_{n}^{i-1}&		x_{n}^{i+1}&		\cdots&		x_{n}^{n}\\
\end{matrix} \right|.
\end{align*}
\end{exercise}
\begin{solution}
令
\begin{align}\label{eq:1.4(Vandermode行列式升阶法)}
|\boldsymbol{B}|=\left|\begin{matrix}
1 & x_1 & \cdots & x_{1}^{i - 1} & x_{1}^{i} & x_{1}^{i + 1} & \cdots & x_{1}^{n}\\
1 & x_2 & \cdots & x_{2}^{i - 1} & x_{2}^{i} & x_{2}^{i + 1} & \cdots & x_{2}^{n}\\
\vdots & \vdots &  & \vdots & \vdots & \vdots &  & \vdots\\
1 & x_n & \cdots & x_{n}^{i - 1} & x_{n}^{i} & x_{n}^{i + 1} & \cdots & x_{n}^{n}\\
1 & y & \cdots & y^{i - 1} & y^i & y^{i + 1} & \cdots & y^n
\end{matrix}\right|=(y - x_1)(y - x_2)\cdots (y - x_n)\prod_{1\leqslant i < j\leqslant n}(x_j - x_i).
\end{align}
而上式右边是关于\(y\)的\(n\)次多项式,并且其\(y^i\)前的系数是
\begin{align*}
\sum_{1\leqslant k_1 < k_2 < \cdots < k_{n - i}\leqslant n}(-1)^{n - i}x_{k_1}x_{k_2}\cdots x_{k_{n - i}}\prod_{1\leqslant i < j\leqslant n}(x_j - x_i).
\end{align*}
将\(|\boldsymbol{B}|\)按最后一行展开,得
\begin{align*}
|\boldsymbol{B}|=A_{n1} + A_{n2}y + \cdots + A_{ni}y^i + \cdots + A_{nn}y^n, 
\end{align*}
其中\(A_{nk}\)为\(|\boldsymbol{B}|\)的\((n,k)\)位置元素的代数余子式,\(k = 1,2,\cdots,n\).

注意到\(A_{nk}\)均与\(y\)无关.因此\(|\boldsymbol{B}|\)作为关于\(y\)的\(n\)次多项式,其\(y^i\)前的系数是
\begin{align*}
A_{ni}=(-1)^{n + 1 + i + 1}|\boldsymbol{A}|=(-1)^{n + i}|\boldsymbol{A}|.
\end{align*}
再结合\eqref{eq:1.4(Vandermode行列式升阶法)}式,可知
\begin{align*}
(-1)^{n + i}|\boldsymbol{A}|=\sum_{1\leqslant k_1 < k_2 < \cdots < k_{n - i}\leqslant n}(-1)^{n - i}x_{k_1}x_{k_2}\cdots x_{k_{n - i}}\prod_{1\leqslant i < j\leqslant n}(x_j - x_i). 
\end{align*}
故\(|\boldsymbol{A}|=x_{k_1}x_{k_2}\cdots x_{k_{n - i}}\prod_{1\leqslant i < j\leqslant n}(x_j - x_i)\).
\end{solution}

\begin{exercise}
求下列$n$阶行列式的值,其中$a_i\ne 0(1\le i\le n)$:
\begin{align*}
|\boldsymbol{A}|=\left| \begin{matrix}
0&		a_1+a_2&		\cdots&		a_1+a_{n-1}&		a_1+a_n\\
a_2+a_1&		0&		\cdots&		a_2+a_{n-1}&		a_2+a_n\\
\vdots&		\vdots&		&		\vdots&		\vdots\\
a_{n-1}+a_1&		a_{n-1}+a_2&		\cdots&		0&		a_{n-1}+a_n\\
a_n+a_1&		a_n+a_2&		\cdots&		a_n+a_{n-1}&		0\\
\end{matrix} \right|.
\end{align*}
\end{exercise}
\begin{note}
{\color{blue}解法一}中不仅使用了\hyperlink{行列式计算:升阶法}{升阶法}还使用了\hyperref[proposition:分块"爪"型行列式]{分块"爪"型行列式的计算方法}.观察到各行各列有不同的公共项,因此可以利用升阶法将各行各列的公共项消去.
\end{note}
\begin{solution}
{\color{blue}解法一(\hyperlink{行列式计算:升阶法}{升阶法}):}
\begin{align*}
&|\boldsymbol{A}|\xlongequal[]{\text{升阶}}\left| \begin{matrix}
1&		-a_1&		-a_2&		\cdots&		-a_{n-1}&		-a_n\\
0&		0&		a_1+a_2&		\cdots&		a_1+a_{n-1}&		a_1+a_n\\
0&		a_2+a_1&		0&		\cdots&		a_2+a_{n-1}&		a_2+a_n\\
\vdots&		\vdots&		\vdots&		&		\vdots&		\vdots\\
0&		a_{n-1}+a_1&		a_{n-1}+a_2&		\cdots&		0&		a_{n-1}+a_n\\
0&		a_n+a_1&		a_n+a_2&		\cdots&		a_n+a_{n-1}&		0\\
\end{matrix} \right|
\\
&\xlongequal[i=1,2,\cdots ,n+1]{r_1+r_i}\left| \begin{matrix}
1&		-a_1&		-a_2&		\cdots&		-a_{n-1}&		-a_n\\
1&		-a_1&		a_1&		\cdots&		a_1&		a_1\\
1&		a_2&		-a_2&		\cdots&		a_2&		a_2\\
\vdots&		\vdots&		\vdots&		&		\vdots&		\vdots\\
1&		a_{n-1}&		a_{n-1}&		\cdots&		-a_{n-1}&		a_{n-1}\\
1&		a_n&		a_n&		\cdots&		a_n&		-a_n\\
\end{matrix} \right|\xlongequal[]{\text{升阶}}\left| \begin{matrix}
1&		0&		0&		0&		\cdots&		0&		0\\
0&		1&		-a_1&		-a_2&		\cdots&		-a_{n-1}&		-a_n\\
-a_1&		1&		-a_1&		a_1&		\cdots&		a_1&		a_1\\
-a_2&		1&		a_2&		-a_2&		\cdots&		a_2&		a_2\\
\vdots&		\vdots&		\vdots&		\vdots&		&		\vdots&		\vdots\\
-a_{n-1}&		1&		a_{n-1}&		a_{n-1}&		\cdots&		-a_{n-1}&		a_{n-1}\\
-a_n&		1&		a_n&		a_n&		\cdots&		a_n&		-a_n\\
\end{matrix} \right|
\\
&\xlongequal[i=1,3,4\cdots ,n+2]{j_1+j_i}\left| \begin{matrix}
1&		0&		1&		1&		\cdots&		1&		1\\
0&		1&		-a_1&		-a_2&		\cdots&		-a_{n-1}&		-a_n\\
-a_1&		1&		-2a_1&		0&		\cdots&		0&		0\\
-a_2&		1&		0&		-2a_2&		\cdots&		0&		0_2\\
\vdots&		\vdots&		\vdots&		\vdots&		&		\vdots&		\vdots\\
-a_{n-1}&		1&		0&		0&		\cdots&		-2a_{n-1}&		0\\
-a_n&		1&		0&		0&		\cdots&		0&		-2a_n\\
\end{matrix} \right|
\\
&\xlongequal[i=3,4\cdots ,n+2]{\begin{array}{c}
-\frac{1}{2}j_i+j_1\\
\frac{1}{2a_{i-2}}j_i+j_2\\
\end{array}}\left| \begin{matrix}
1-\frac{n}{2}&		\frac{S}{2}&		1&		1&		\cdots&		1&		1\\
\frac{T}{2}&		1-\frac{n}{2}&		-a_1&		-a_2&		\cdots&		-a_{n-1}&		-a_n\\
0&		0&		-2a_1&		0&		\cdots&		0&		0\\
0&		0&		0&		-2a_2&		\cdots&		0&		0_2\\
\vdots&		\vdots&		\vdots&		\vdots&		&		\vdots&		\vdots\\
0&		0&		0&		0&		\cdots&		-2a_{n-1}&		0\\
0&		0&		0&		0&		\cdots&		0&		-2a_n\\
\end{matrix} \right|. 
\end{align*}
其中\(S = a_1 + a_2 + \cdots + a_n\),\(T = \frac{1}{a_1} + \frac{1}{a_2} + \cdots + \frac{1}{a_n}\).注意到上述行列式是分块上三角行列式,从而可得
\begin{align*}
\vert\boldsymbol{A}\vert = (-2)^n\prod_{i = 1}^n{a_i} \cdot \frac{(n - 2)^2 - ST}{4} = (-2)^{n - 2}\prod_{i = 1}^n{a_i}[(n - 2)^2 - (\sum_{i = 1}^n{a_i})(\sum_{i = 1}^n{\frac{1}{a_i}})].
\end{align*}

{\color{blue}解法二(\hyperref[proposition:直接计算两个矩阵和的行列式]{直接计算两个矩阵和的行列式}):}

设\(\boldsymbol{B}=\left(\begin{matrix}
2a_1 & a_1 + a_2 & \cdots & a_1 + a_n\\
a_2 + a_1 & 2a_2 & \cdots & a_2 + a_n\\
\vdots & \vdots &  & \vdots\\
a_n + a_1 & a_n + a_2 & \cdots & 2a_n
\end{matrix}\right)\),\(\boldsymbol{C}=\left(\begin{matrix}
-2a_1 &  &  & \\
& -2a_2 &  & \\
&  & \ddots & \\
&  &  & -2a_n
\end{matrix}\right)\),则\(|\boldsymbol{A}| = |\boldsymbol{B} + \boldsymbol{C}|\).

从而利用\hyperref[proposition:直接计算两个矩阵和的行列式]{直接计算两个矩阵和的行列式}的结论得到
\begin{align}\label{eq(行列式):1.5式}
&|\boldsymbol{A}| = |\boldsymbol{B}| + |\boldsymbol{C}| + \sum_{1\leqslant k\leqslant n - 1}\left(\sum_{\begin{array}{c}
1\leqslant i_1 < i_2 < \cdots < i_k\leqslant n\\
1\leqslant j_1 < j_2 < \cdots < j_k\leqslant n
\end{array}}\boldsymbol{B}\left(\begin{matrix}
i_1 & i_2 & \cdots & i_k\\
j_1 & j_2 & \cdots & j_k
\end{matrix}\right)\widehat{\boldsymbol{C}}\left(\begin{matrix}
i_1 & i_2 & \cdots & i_k\\
j_1 & j_2 & \cdots & j_k
\end{matrix}\right)\right)
\end{align}
其中\(\widehat{\boldsymbol{C}}\left(\begin{matrix}
i_1 & i_2 & \cdots & i_k\\
j_1 & j_2 & \cdots & j_k
\end{matrix}\right)\)是\(\boldsymbol{C}\left(\begin{matrix}
i_1 & i_2 & \cdots & i_k\\
j_1 & j_2 & \cdots & j_k
\end{matrix}\right)\)的代数余子式.

我们先来计算\(\boldsymbol{B}\left(\begin{matrix}
i_1 & i_2 & \cdots & i_k\\
j_1 & j_2 & \cdots & j_k
\end{matrix}\right)\),\(k = 1,2,\cdots,n\).拆分\(\boldsymbol{B}\left(\begin{matrix}
i_1 & i_2 & \cdots & i_k\\
j_1 & j_2 & \cdots & j_k
\end{matrix}\right)\)的第一列得到
\begin{align*}
&\boldsymbol{B}\left(\begin{matrix}
i_1 & i_2 & \cdots & i_k\\
j_1 & j_2 & \cdots & j_k
\end{matrix}\right) = \left|\begin{matrix}
a_{i_1} + a_{j_1} & a_{i_1} + a_{j_2} & \cdots & a_{i_1} + a_{j_k}\\
a_{i_2} + a_{j_1} & a_{i_2} + a_{j_2} & \cdots & a_{i_2} + a_{j_k}\\
\vdots & \vdots &  & \vdots\\
a_{i_k} + a_{j_1} & a_{i_k} + a_{j_2} & \cdots & a_{i_k} + a_{j_k}
\end{matrix}\right|
\\
&=\left|\begin{matrix}
a_{i_1} & a_{i_1} + a_{j_2} & \cdots & a_{i_1} + a_{j_k}\\
a_{i_2} & a_{i_2} + a_{j_2} & \cdots & a_{i_2} + a_{j_k}\\
\vdots & \vdots &  & \vdots\\
a_{i_k} & a_{i_k} + a_{j_2} & \cdots & a_{i_k} + a_{j_k}
\end{matrix}\right| + \left|\begin{matrix}
a_{j_1} & a_{i_1} + a_{j_2} & \cdots & a_{i_1} + a_{j_k}\\
a_{j_1} & a_{i_2} + a_{j_2} & \cdots & a_{i_2} + a_{j_k}\\
\vdots & \vdots &  & \vdots\\
a_{j_1} & a_{i_k} + a_{j_2} & \cdots & a_{i_k} + a_{j_k}
\end{matrix}\right|
\\
&=\left|\begin{matrix}
a_{i_1} & a_{j_2} & \cdots & a_{j_k}\\
a_{i_2} & a_{j_2} & \cdots & a_{j_k}\\
\vdots & \vdots &  & \vdots\\
a_{i_k} & a_{j_2} & \cdots & a_{j_k}
\end{matrix}\right| + \left|\begin{matrix}
a_{j_1} & a_{i_1} & \cdots & a_{i_1}\\
a_{j_1} & a_{i_2} & \cdots & a_{i_2}\\
\vdots & \vdots &  & \vdots\\
a_{j_1} & a_{i_k} & \cdots & a_{i_k}
\end{matrix}\right|
\end{align*}
因此当\(k\geqslant 3\)时,\(\boldsymbol{B}\left(\begin{matrix}
i_1 & i_2 & \cdots & i_k\\
j_1 & j_2 & \cdots & j_k
\end{matrix}\right) = 0\);
当\(k = 2\)时,\(\boldsymbol{B}\left(\begin{matrix}
i_1 & i_2 & \cdots & i_k\\
j_1 & j_2 & \cdots & j_k
\end{matrix}\right) = \boldsymbol{B}\left(\begin{matrix}
i_1 & i_2\\
j_1 & j_2
\end{matrix}\right) = \left|\begin{matrix}
a_{i_1} & a_{j_2}\\
a_{i_2} & a_{j_2}
\end{matrix}\right| + \left|\begin{matrix}
a_{j_1} & a_{i_1}\\
a_{j_1} & a_{i_2}
\end{matrix}\right| = (a_{i_1}a_{j_2} - a_{i_2}a_{j_2})(a_{i_2}a_{j_1} - a_{i_1}a_{j_1})\);
当\(k = 1\)时,\(\boldsymbol{B}\left(\begin{matrix}
i_1 & i_2 & \cdots & i_k\\
j_1 & j_2 & \cdots & j_k
\end{matrix}\right) = \boldsymbol{B}\left(\begin{array}{c}
i_1\\
j_1
\end{array}\right) = a_{i_1} + a_{j_1}\).

又注意到\(|\boldsymbol{C}|\)只有主子式非零,而其主子式\(\boldsymbol{C}\left(\begin{matrix}
i_1 & i_2 & \cdots & i_k\\
i_1 & i_2 & \cdots & i_k
\end{matrix}\right) = (-2)^ka_{i_1}a_{i_2}\cdots a_{i_k}\).
于是当\(\exists m\in \{1,2,\cdots,k\}\),使得\(i_m\neq j_m\)时,\(\widehat{\boldsymbol{C}}\left(\begin{matrix}
i_1 & i_2 & \cdots & i_k\\
j_1 & j_2 & \cdots & j_k
\end{matrix}\right) = 0\);
当\(i_m\neq j_m\),\(m = 1,2,\cdots,k\)时,\(\widehat{\boldsymbol{C}}\left(\begin{matrix}
i_1 & i_2 & \cdots & i_k\\
j_1 & j_2 & \cdots & j_k
\end{matrix}\right) = \widehat{\boldsymbol{C}}\left(\begin{matrix}
i_1 & i_2 & \cdots & i_k\\
i_1 & i_2 & \cdots & i_k
\end{matrix}\right) = (-2)^{n - k}a_1\cdots \hat{a}_{i_1}\cdots \hat{a}_{i_2}\cdots \hat{a}_{i_k}\cdots a_n\).

故当\(n\geqslant 3\)时,\eqref{eq(行列式):1.5式}式可化为
\begin{align*}
&|\boldsymbol{A}| = |\boldsymbol{B}| + |\boldsymbol{C}| + \sum_{1\leqslant k\leqslant n - 1}\left(\sum_{\substack{
1\leqslant i_1 < i_2 < \cdots < i_k\leqslant n\\
1\leqslant j_1 < j_2 < \cdots < j_k\leqslant n
}}\boldsymbol{B}\left(\begin{matrix}
i_1 & i_2 & \cdots & i_k\\
j_1 & j_2 & \cdots & j_k
\end{matrix}\right)\widehat{\boldsymbol{C}}\left(\begin{matrix}
i_1 & i_2 & \cdots & i_k\\
j_1 & j_2 & \cdots & j_k
\end{matrix}\right)\right) 
\\
&= |\boldsymbol{C}| + \sum_{\substack{
1\leqslant i_1\leqslant n\\
1\leqslant j_1\leqslant n
}}\boldsymbol{B}\left(\substack{
i_1\\
j_1
}\right)\widehat{\boldsymbol{C}}\left(\begin{array}{c}
i_1\\
j_1
\end{array}\right) + \sum_{\substack{
1\leqslant i_1 < i_2\leqslant n\\
1\leqslant j_1 < j_2\leqslant n
}}\boldsymbol{B}\left(\begin{matrix}
i_1 & i_2\\
j_1 & j_2
\end{matrix}\right)\widehat{\boldsymbol{C}}\left(\begin{matrix}
i_1 & i_2\\
j_1 & j_2
\end{matrix}\right)
\\
&= |\boldsymbol{C}| + \sum_{1\leqslant i_1\leqslant n}\boldsymbol{B}\left(\begin{array}{c}
i_1\\
i_1
\end{array}\right)\widehat{\boldsymbol{C}}\left(\begin{array}{c}
i_1\\
i_1
\end{array}\right) + \sum_{1\leqslant i_1 < i_2\leqslant n}\boldsymbol{B}\left(\begin{matrix}
i_1 & i_2\\
i_1 & i_2
\end{matrix}\right)\widehat{\boldsymbol{C}}\left(\begin{matrix}
i_1 & i_2\\
i_1 & i_2
\end{matrix}\right)
= |\boldsymbol{C}| + \sum_{1\leqslant i\leqslant n}\boldsymbol{B}\left(\begin{matrix}
i\\
i
\end{matrix}
\right)\widehat{\boldsymbol{C}}\left(\begin{matrix}
i\\
i
\end{matrix}
\right) + \sum_{1\leqslant i < j\leqslant n}\boldsymbol{B}\left(\begin{matrix}
i & j\\
i & j
\end{matrix}\right)\widehat{\boldsymbol{C}}\left(\begin{matrix}
i & j\\
i & j
\end{matrix}\right)
\\
&= (-2)^na_1a_2\cdots a_n + \sum_{1\leqslant i\leqslant n}2a_i(-2)^{n - 1}a_1\cdots \hat{a}_i\cdots a_n
+\sum_{1\leqslant i < j\leqslant n}[(a_ia_j - a_{j}^{2})(a_ia_j - a_{i}^{2})(-2)^{n - 2}a_1\cdots \hat{a}_i\cdots \hat{a}_j\cdots a_n]
\\
&= (-2)^na_1a_2\cdots a_n - (-2)^n\sum_{1\leqslant i\leqslant n}a_1a_2\cdots \cdots a_n
+ (-2)^{n - 2}\sum_{1\leqslant i < j\leqslant n}[-(a_i - a_j)^2a_1\cdots \hat{a}_i\cdots \hat{a}_j\cdots a_n]
\\
&= (-2)^na_1a_2\cdots a_n - (-2)^nna_1a_2\cdots \cdots a_n
- (-2)^{n - 2}\sum_{1\leqslant i < j\leqslant n}[(a_i - a_j)^2a_1\cdots \hat{a}_i\cdots \hat{a}_j\cdots a_n]
\\
&= (-2)^n\prod_{i = 1}^n{a_i}(1 - n) - (-2)^{n - 2}\prod_{i = 1}^n{a_i}\sum_{1\leqslant i < j\leqslant n}\frac{(a_i - a_j)^2}{a_{i}a_{j}}
\\
& = (-2)^{n - 2}\prod_{i = 1}^n{a_i}[(n - 2)^2 - (\sum_{i = 1}^n{a_i})(\sum_{i = 1}^n{\frac{1}{a_i}})]
\\
&=(-2)^n\prod_{i=1}^n{a_i(1}-n)-(-2)^{n-2}\prod_{i=1}^n{a_i\sum_{1\leqslant i<j\leqslant n}{\frac{(a_i-a_j)^2}{a_ia_j}}}
\\
&=\left( -2 \right) ^{n-2}\prod_{i=1}^n{a_i}\left[ 4-4n-\sum_{1\leqslant i<j\leqslant n}{\frac{(a_i-a_j)^2}{a_ia_j}} \right] 
\\
&=\left( -2 \right) ^{n-2}\prod_{i=1}^n{a_i}\left[ 4-4n-\sum_{1\leqslant i<j\leqslant n}{\left( \frac{a_j}{a_i}+\frac{a_i}{a_j}-2 \right)} \right] 
\\
&=\left( -2 \right) ^{n-2}\prod_{i=1}^n{a_i}\left[ 4-4n-\sum_{\substack{
1\leqslant i,j\leqslant n\\
i\ne j\\
}}{\frac{a_i}{a_j}}+\sum_{1\leqslant i<j\leqslant n}{2} \right] 
\\
&=\left( -2 \right) ^{n-2}\prod_{i=1}^n{a_i}\left[ 4-4n-\left( \sum_{1\leqslant i,j\leqslant n}{\frac{a_i}{a_j}}-\sum_{i=1}^n{\frac{a_i}{a_i}} \right) +\sum_{i=1}^{n-1}{\sum_{j=i+1}^n{2}} \right] 
\\
&=\left( -2 \right) ^{n-2}\prod_{i=1}^n{a_i}\left[ 4-4n-\left( \sum_{1\leqslant i,j\leqslant n}{\frac{a_i}{a_j}}-n \right) +2\sum_{i=1}^{n-1}{\left( n-i \right)} \right] 
\\
&=\left( -2 \right) ^{n-2}\prod_{i=1}^n{a_i}\left[ 4-4n+n+n\left( n-1 \right) -\sum_{i=1}^n{\sum_{j=1}^n{\frac{a_i}{a_j}}} \right] 
\\
&=\left( -2 \right) ^{n-2}\prod_{i=1}^n{a_i}\left[ n^2-4n+4-\sum_{i=1}^n{a_i\sum_{j=1}^n{\frac{1}{a_j}}} \right] 
\\
&=\left( -2 \right) ^{n-2}\prod_{i=1}^n{a_i[(n}-2)^2-(\sum_{i=1}^n{a_i)(\sum_{i=1}^n{\frac{1}{a_i})]}}.
\end{align*}
{\color{blue}解法三:}
令$\varLambda=\left( \begin{matrix}
a_1&		1\\
a_2&		1\\
\vdots&		\vdots\\
a_n&		1\\
\end{matrix} \right) ,B=\left( \begin{matrix}
-2a_1&		&		&		\\
&		-2a_2&		&		\\
&		&		\ddots&		\\
&		&		&		-2a_n\\
\end{matrix} \right) $,则
\begin{align*}
A=\left( \begin{matrix}
-2a_1&		&		&		\\
&		-2a_2&		&		\\
&		&		\ddots&		\\
&		&		&		-2a_n\\
\end{matrix} \right) +\left( \begin{matrix}
a_1&		1\\
a_2&		1\\
\vdots&		\vdots\\
a_n&		1\\
\end{matrix} \right) I_{2}^{-1}\left( \begin{matrix}
1&		1&		\cdots&		1\\
a_1&		a_2&		\cdots&		a_n\\
\end{matrix} \right) =B+\varLambda I_{2}^{-1}\varLambda '.
\end{align*}
于是由降价公式(打洞原理)我们有
\begin{align*}
|A|&=|I|\left|B + \Lambda I_{2}^{-1}\Lambda '\right|=\left|\begin{matrix}
I_2 & \Lambda '\\
\Lambda & B
\end{matrix}\right|=|B|\left|I_2 - \Lambda 'B^{-1}\Lambda\right|\\
&=\left|\begin{matrix}
-2a_1 & & & \\
& -2a_2 & & \\
& & \ddots & \\
& & & -2a_n
\end{matrix}\right|\cdot\left|I_2 - \left(\begin{matrix}
1 & 1 & \cdots & 1\\
a_1 & a_2 & \cdots & a_n
\end{matrix}\right)\left(\begin{matrix}
-\frac{1}{2a_1} & & & \\
& -\frac{1}{2a_2} & & \\
& & \ddots & \\
& & & -\frac{1}{2a_n}
\end{matrix}\right)\left(\begin{matrix}
a_1 & 1\\
a_2 & 1\\
\vdots & \vdots\\
a_n & 1
\end{matrix}\right)\right|\\
&=(-2)^n\prod_{i = 1}^n a_i\left|I_2 - \left(\begin{matrix}
-\frac{1}{2a_1} & -\frac{1}{2a_2} & \cdots & -\frac{1}{2a_n}\\
-\frac{1}{2} & -\frac{1}{2} & \cdots & -\frac{1}{2}
\end{matrix}\right)\left(\begin{matrix}
a_1 & 1\\
a_2 & 1\\
\vdots & \vdots\\
a_n & 1
\end{matrix}\right)\right|\\
&=(-2)^n\prod_{i = 1}^n a_i\left|I_2 - \left(\begin{matrix}
-\frac{n}{2} & -\frac{1}{2}\sum_{i = 1}^n\frac{1}{a_i}\\
-\frac{1}{2}\sum_{i = 1}^n a_i & -\frac{n}{2}
\end{matrix}\right)\right|=(-2)^n\prod_{i = 1}^n a_i\left|\begin{matrix}
\frac{n + 2}{2} & \frac{1}{2}\sum_{i = 1}^n\frac{1}{a_i}\\
\frac{1}{2}\sum_{i = 1}^n a_i & \frac{n + 2}{2}
\end{matrix}\right|\\
&=(-2)^{n - 2}\prod_{i = 1}^n a_i\left[(n + 2)^2 - \left(\sum_{i = 1}^n a_i\right)\left(\sum_{i = 1}^n\frac{1}{a_i}\right)\right].
\end{align*}
\end{solution}
\begin{conclusion}\label{对角矩阵行列式的子式和余子式}
\hypertarget{对角矩阵行列式的子式和余子式}{\textbf{对角矩阵行列式的子式和余子式:}}

设\(|\boldsymbol{A}|=\left|\begin{matrix}
a_1 & 0 & \cdots & 0\\
0 & a_2 & \cdots & 0\\
\vdots & \vdots & \ddots & \vdots\\
0 & 0 & \cdots & a_n
\end{matrix}\right|\),则其\(k\)阶子式\(\boldsymbol{A}\left(\begin{matrix}
i_1 & i_2 & \cdots & i_k\\
j_1 & j_2 & \cdots & j_k
\end{matrix}\right)\)除\(k\)阶主子式\(\boldsymbol{A}\left(\begin{matrix}
i_1 & i_2 & \cdots & i_k\\
i_1 & i_2 & \cdots & i_k
\end{matrix}\right)\)外都为零,其中\(k = 1,2,\cdots,n\).

记\(\widehat{\boldsymbol{A}}\left(\begin{matrix}
i_1 & i_2 & \cdots & i_k\\
j_1 & j_2 & \cdots & j_k
\end{matrix}\right)\)为\(\boldsymbol{A}\left(\begin{matrix}
i_1 & i_2 & \cdots & i_k\\
j_1 & j_2 & \cdots & j_k
\end{matrix}\right)\)的代数余子式(\(n - k\)阶).于是\(\widehat{\boldsymbol{A}}\left(\begin{matrix}
i_1 & i_2 & \cdots & i_k\\
j_1 & j_2 & \cdots & j_k
\end{matrix}\right)\)除\(\widehat{\boldsymbol{A}}\left(\begin{matrix}
i_1 & i_2 & \cdots & i_k\\
i_1 & i_2 & \cdots & i_k
\end{matrix}\right)\)外也都为零,其中\(k = 1,2,\cdots,n\).

并且
\begin{align*}
&\boldsymbol{A}\left(\begin{matrix}
i_1 & i_2 & \cdots & i_k\\
i_1 & i_2 & \cdots & i_k
\end{matrix}\right) = a_{i_1}a_{i_2}\cdots a_{i_k},
\\
&\widehat{\boldsymbol{A}}\left(\begin{matrix}
i_1 & i_2 & \cdots & i_k\\
i_1 & i_2 & \cdots & i_k
\end{matrix}\right) = a_1\cdots \hat{a}_{i_1}\cdots \hat{a}_{i_2}\cdots \hat{a}_{i_k}\cdots a_n\,
\end{align*}
其中\(k = 1,2,\cdots,n\).
\end{conclusion}

\begin{exercise}
若\(n\)阶行列式\(\vert A\vert\)中零元素的个数超过\(n^2 - n\)个,证明:\(\vert A\vert = 0\).
\end{exercise}
\begin{solution}
证明 由行列式的组合定义可得
\[
|A|=\sum_{1\leq k_1k_2\cdots k_n\leq n}(-1)^{\tau (k_1,k_2,\cdots,k_n)}a_{k_{11}}a_{k_{22}}\cdots a_{k_{nn}}
\]
由于\(|A|\)中零元素的个数超过\(n^2 - n\)个,故\(a_{k_{11}},a_{k_{22}},\cdots,a_{k_{nn}}\)中至少有一个为零,从而\(a_{k_{11}}a_{k_{22}}\cdots a_{k_{nn}} = 0\),因此\(|A| = 0\).如直接利用行列式的性质,也可以这样来证明:因为\(|A|\)中零元素的个数超过\(n^2 - n\)个,由抽屉原理可知,\(|A|\)至少有一列其零元素的个数大于等于\(\left\lfloor\frac{n^2 - n}{n}\right\rfloor+ 1=n\),即\(|A|\)至少有一列其元素全为零,因此\(|A| = 0\).
\end{solution}

\begin{exercise}
设\(A=(a_{ij})\)是\(n(n\geq2)\)阶非异整数方阵,满足对任意的\(i,j\),\(\vert A\vert\)均可整除\(a_{ij}\),证明:\(\vert A\vert=\pm1\).
\end{exercise}
\begin{solution}
\(\vert A\vert\)可整除每个元素\(a_{i,j}\),故由行列式的组合定义
\[
\sum_{1\le k_1,k_2,\cdots ,k_n\le n}{\left( -1 \right) ^{\tau \left( k_1k_2\cdots k_n \right)}a_{k_{11}}a_{k_{22}}\cdots a_{k_{nn}}}
\]
可知\(\vert A\vert^n\)可整除\(\vert A\vert\)中每个单项\(a_{k_{11}}a_{k_{22}}\cdots a_{k_{nn}}\),从而\(\vert A\vert^n\)可整除\(\vert A\vert\),即有\(\vert A\vert^{n - 1}\)可整除\(1\),于是\(\vert A\vert^{n - 1}=\pm1\).又由行列式的组合定义可知\(\vert A\vert\)是整数,从而只能是\(\vert A\vert=\pm1\).
\end{solution}

\begin{exercise}
利用行列式的$Laplace$定理证明恒等式:
\[
(ab' - a'b)(cd' - c'd)-(ac' - a'c)(bd' - b'd)+(ad' - a'd)(bc' - b'c)=0.
\]
\end{exercise}
\begin{solution}
显然下列行列式的值为零:
\begin{align*}
\left| \begin{matrix}
a&		a^{\prime}&		a&		a^{\prime}\\
b&		b^{\prime}&		b&		b^{\prime}\\
c&		c^{\prime}&		c&		c^{\prime}\\
d&		d^{\prime}&		d&		d^{\prime}\\
\end{matrix} \right|.
\end{align*}
利用$Laplace$定理按第一、二列展开得
\begin{align*}
\left| \begin{matrix}
a&		a^{\prime}&		a&		a^{\prime}\\
b&		b^{\prime}&		b&		b^{\prime}\\
c&		c^{\prime}&		c&		c^{\prime}\\
d&		d^{\prime}&		d&		d^{\prime}\\
\end{matrix} \right|&=\left( -1 \right) ^{1+2+1+2}\left| \begin{matrix}
a&		a^{\prime}\\
b&		b^{\prime}\\
\end{matrix} \right|\left| \begin{matrix}
c&		c^{\prime}\\
d&		d^{\prime}\\
\end{matrix} \right|+\left( -1 \right) ^{1+2+1+3}\left| \begin{matrix}
a&		a^{\prime}\\
c&		c^{\prime}\\
\end{matrix} \right|\left| \begin{matrix}
b&		b^{\prime}\\
d&		d^{\prime}\\
\end{matrix} \right|+\left( -1 \right) ^{1+2+1+4}\left| \begin{matrix}
a&		a^{\prime}\\
d&		d^{\prime}\\
\end{matrix} \right|\left| \begin{matrix}
b&		b^{\prime}\\
c&		c^{\prime}\\
\end{matrix} \right|
\\
&\quad+\left( -1 \right) ^{1+2+2+3}\left| \begin{matrix}
b&		b^{\prime}\\
c&		c^{\prime}\\
\end{matrix} \right|\left| \begin{matrix}
a&		a^{\prime}\\
d&		d^{\prime}\\
\end{matrix} \right|+\left( -1 \right) ^{1+2+2+4}\left| \begin{matrix}
b&		b^{\prime}\\
d&		d^{\prime}\\
\end{matrix} \right|\left| \begin{matrix}
a&		a^{\prime}\\
c&		c^{\prime}\\
\end{matrix} \right|
\\
&\quad+\left( -1 \right) ^{1+2+3+4}\left| \begin{matrix}
c&		c^{\prime}\\
d&		d^{\prime}\\
\end{matrix} \right|\left| \begin{matrix}
a&		a^{\prime}\\
b&		b^{\prime}\\
\end{matrix} \right|
\\
&=2\left| \begin{matrix}
a&		a^{\prime}\\
b&		b^{\prime}\\
\end{matrix} \right|\left| \begin{matrix}
c&		c^{\prime}\\
d&		d^{\prime}\\
\end{matrix} \right|-2\left| \begin{matrix}
a&		a^{\prime}\\
c&		c^{\prime}\\
\end{matrix} \right|\left| \begin{matrix}
b&		b^{\prime}\\
d&		d^{\prime}\\
\end{matrix} \right|+2\left| \begin{matrix}
a&		a^{\prime}\\
d&		d^{\prime}\\
\end{matrix} \right|\left| \begin{matrix}
b&		b^{\prime}\\
c&		c^{\prime}\\
\end{matrix} \right|=0.
\end{align*}
上式等价于
\begin{align*}
\left| \begin{matrix}
a&		a^{\prime}\\
b&		b^{\prime}\\
\end{matrix} \right|\left| \begin{matrix}
c&		c^{\prime}\\
d&		d^{\prime}\\
\end{matrix} \right|-\left| \begin{matrix}
a&		a^{\prime}\\
c&		c^{\prime}\\
\end{matrix} \right|\left| \begin{matrix}
b&		b^{\prime}\\
d&		d^{\prime}\\
\end{matrix} \right|+\left| \begin{matrix}
a&		a^{\prime}\\
d&		d^{\prime}\\
\end{matrix} \right|\left| \begin{matrix}
b&		b^{\prime}\\
c&		c^{\prime}\\
\end{matrix} \right|=0.
\end{align*}
整理可得
\begin{align*}
(ab' - a'b)(cd' - c'd)-(ac' - a'c)(bd' - b'd)+(ad' - a'd)(bc' - b'c)=0.
\end{align*}
\end{solution}

\begin{exercise}
求\(2n\)阶行列式的值(空缺处都是零):
\begin{align*}
\left| \begin{matrix}
a&		&		&		&		&		b\\
&		\ddots&		&		&		\begin{turn}{80}$\ddots$\end{turn}&		\\
&		&		a&		b&		&		\\
&		&		b&		a&		&		\\
&		\begin{turn}{80}$\ddots$\end{turn}&		&		&		\ddots&		\\
b&		&		&		&		&		a\\
\end{matrix} \right|.
\end{align*}
\end{exercise}
\begin{solution}
设原行列式为$D_{2n}$,其中$2n$为行列式的阶数.
不断用$Laplace$定理按第一行及最后一行展开,可得
\begin{align*}
D_{2n}=\left| \begin{matrix}
a&		&		&		&		&		b\\
&		\ddots&		&		&		\begin{turn}{80}$\ddots$\end{turn}&		\\
&		&		a&		b&		&		\\
&		&		b&		a&		&		\\
&		\begin{turn}{80}$\ddots$\end{turn}&		&		&		\ddots&		\\
b&		&		&		&		&		a\\
\end{matrix} \right|\xlongequal[]{\text{按第一行及最后一行展开}}\left| \begin{matrix}
a&		b\\
b&		a\\
\end{matrix} \right|D_{2n-2}=\left( a^2-b^2 \right) D_{2\left( n-1 \right)}.
\end{align*}
进而,由上述递推式可得
\begin{align*}
D_{2n}&=\left( a^2-b^2 \right) D_{2\left( n-1 \right)}=\left( a^2-b^2 \right) ^2D_{2\left( n-2 \right)}=\cdots =\left( a^2-b^2 \right) ^{n-1}D_2
\\
&=\left( a^2-b^2 \right) ^{n-1}\left| \begin{matrix}
a&		b\\
b&		a\\
\end{matrix} \right|=\left( a^2-b^2 \right) ^n.
\end{align*}
\end{solution}

\begin{exercise}
求下列$n$阶行列式的值:
\[
\left| A \right|=\begin{vmatrix}
(x - a_1)^2 & a_2^2 & \cdots & a_n^2 \\
a_1^2 & (x - a_2)^2 & \cdots & a_n^2 \\
\vdots & \vdots & \ddots & \vdots \\
a_1^2 & a_2^2 & \cdots & (x - a_n)^2
\end{vmatrix}.
\]
\end{exercise}
\begin{note}
注意到这个行列式每行元素除了主对角元素外,其余位置元素都相同.因此这个行列式是\hyperref["爪"型行列式的推广]{推广的"爪"型行列式}.
\end{note}
\begin{solution}
\begin{align*}
&\left| A \right|=\left| \begin{matrix}
(x-a_1)^2&		a_{2}^{2}&		\cdots&		a_{n}^{2}\\
a_{1}^{2}&		(x-a_2)^2&		\cdots&		a_{n}^{2}\\
\vdots&		\vdots&		\ddots&		\vdots\\
a_{1}^{2}&		a_{2}^{2}&		\cdots&		(x-a_n)^2\\
\end{matrix} \right|=\left| \begin{matrix}
(x-a_1)^2&		a_{2}^{2}&		\cdots&		a_{n}^{2}\\
2a_1x-x^2&		x^2-2a_2x&		\cdots&		0\\
\vdots&		\vdots&		\ddots&		\vdots\\
2a_1x-x^2&		0&		\cdots&		x^2-2a_nx\\
\end{matrix} \right|
\\
&\xlongequal{\hyperref["爪"型行列式]{\text{"爪"型行列式}}}(x-a_1)^2\prod_{i=2}^n{\left( x^2-2a_ix \right)}-\sum_{i=2}^n{a_{i}^{2}\left( 2a_1x-x^2 \right) \left( x^2-2a_2x \right) \cdots \widehat{\left( x^2-2a_ix \right) }\cdots}\left( x^2-2a_nx \right) 
\\
&=(x-a_1)^2\prod_{i=2}^n{\left( x^2-2a_ix \right)}+\sum_{i=2}^n{a_{i}^{2}\left( x^2-2a_1x \right) \left( x^2-2a_2x \right) \cdots \widehat{\left( x^2-2a_ix \right) }\cdots}\left( x^2-2a_nx \right) 
\\
&=(x-a_1)^2\prod_{i=2}^n{\left( x^2-2a_ix \right)}+\sum_{i=2}^n{\left( x^2-2a_1x \right) \cdots \left( x^2-2a_{i-1}x \right) a_{i}^{2}\left( x^2-2a_{i+1}x \right) \cdots}\left( x^2-2a_nx \right) 
\\
&=\left[ \left( x^2-2a_1x \right) +a_{1}^{2} \right] \prod_{i=2}^n{\left( x^2-2a_ix \right)}+\sum_{i=2}^n{\left( x^2-2a_1x \right) \cdots \left( x^2-2a_{i-1}x \right) a_{i}^{2}\left( x^2-2a_{i+1}x \right) \cdots}\left( x^2-2a_nx \right) 
\\
&=\prod_{i=1}^n{\left( x^2-2a_ix \right)}+\sum_{i=1}^n{\left( x^2-2a_1x \right) \cdots \left( x^2-2a_{i-1}x \right) a_{i}^{2}\left( x^2-2a_{i+1}x \right) \cdots}\left( x^2-2a_nx \right).
\end{align*}
\end{solution}

\begin{exercise}
求下列行列式式的值:
\[
\left| \boldsymbol{A} \right|=\begin{vmatrix}
(a + b)^2 & c^2 & c^2 \\
a^2 & (b + c)^2 & a^2 \\
b^2 & b^2 & (c + a)^2
\end{vmatrix}.
\]
\end{exercise}
\begin{solution}
{\color{blue}解法一:}
\begin{align*}
&\left| \boldsymbol{A} \right|=\left| \begin{matrix}
(a+b)^2&		c^2&		c^2\\
a^2&		(b+c)^2&		a^2\\
b^2&		b^2&		(c+a)^2\\
\end{matrix} \right|\xlongequal[i=1,2]{-j_1+j_i}\left| \begin{matrix}
(a+b)^2-c^2&		c^2&		0\\
a^2-(b+c)^2&		(b+c)^2&		a^2-(b+c)^2\\
0&		b^2&		(c+a)^2-b^2\\
\end{matrix} \right|
\\
&=(a+b+c)^2\left| \begin{matrix}
a+b-c&		c^2&		0\\
a-b-c&		(b+c)^2&		a-b-c\\
0&		b^2&		a+c-b\\
\end{matrix} \right|\xlongequal[i=1,2]{-r_i+r_2}(a+b+c)^2\left| \begin{matrix}
a+b-c&		c^2&		0\\
-2b&		2bc&		-2c\\
0&		b^2&		a+c-b\\
\end{matrix} \right|
\\
&\xlongequal[\frac{b}{2}j_3+j_2]{\frac{c}{2}j_1+j_2}(a+b+c)^2\left| \begin{matrix}
a+b-c&		\frac{c}{2}\left( a+b+c \right)&		0\\
-2b&		0&		-2c\\
0&		\frac{b}{2}\left( a+b+c \right)&		a+c-b\\
\end{matrix} \right|=(a+b+c)^3\left| \begin{matrix}
a+b-c&		\frac{c}{2}&		0\\
-2b&		0&		-2c\\
0&		\frac{b}{2}&		a+c-b\\
\end{matrix} \right|
\\
&=2abc(a+b+c)^3.
\end{align*}

{\color{blue}解法二(求根法):}
\end{solution}

\begin{exercise}
证明:若一个\(n(n>1)\)阶行列式中元素或为\(1\)或为\(-1\),则其值必为偶数.
\end{exercise}
\begin{proof}
将该行列式的任意一行加到另一行上去得到的行列式有一行元素全是偶数(注意:零也是偶数),由行列式的基本性质知道,可将因子2提出,剩下的行列式的元素都是整数,其值也是整数,乘以2后必是偶数.
\end{proof}

\begin{exercise}
\(n\) 阶行列式\(\vert \boldsymbol{A}\vert\)的值为\(c\),若从第二列开始每一列加上它前面的一列,同时对第一列加上\(\vert \boldsymbol{A}\vert\)的第\(n\)列,求得到的新行列式\(\vert \boldsymbol{B}\vert\)的值.
\end{exercise}
\begin{solution}
\begin{align*}
\left| \boldsymbol{B} \right|&=\left| \boldsymbol{\alpha }_1+\boldsymbol{\alpha }_n,\boldsymbol{\alpha }_2+\boldsymbol{\alpha }_1,\cdots ,\boldsymbol{\alpha }_n+\boldsymbol{\alpha }_{n-1} \right|
\\
&=\left| \boldsymbol{\alpha }_1,\boldsymbol{\alpha }_2,\cdots ,\boldsymbol{\alpha }_n \right|+\left| \boldsymbol{\alpha }_n,\boldsymbol{\alpha }_1,\cdots ,\boldsymbol{\alpha }_{n-1} \right|
+\sum_{1\leqslant k\leqslant n-2}{\sum_{2\leq j_1\leq j_2\leq \cdots\leq j_k\leq n}{\begin{array}{c}
\begin{array}{c@{}c@{}c@{}c@{}c@{}c@{}c@{}c@{}c@{}c@{}c@{}}
& 1 & \cdots & j_1 &\cdots &j_2 &\cdots &j_k &\cdots &n \\
\left.\right|
&\boldsymbol{\alpha }_n,&\cdots ,&\boldsymbol{\alpha }_{j_1+1},&\cdots ,&\boldsymbol{\alpha }_{j_2+1},&\cdots ,&\boldsymbol{\alpha }_{j_k+1},&\cdots ,&\boldsymbol{\alpha }_{n-1}& \left|\right.
\end{array}\\
\\
\end{array}}}
\\
&\quad +\sum_{1\leqslant k\leqslant n-2}{\sum_{2\leq j_1\leq j_2\leq \cdots\leq j_k\leq n}{\begin{array}{c}
\begin{array}{c@{}c@{}c@{}c@{}c@{}c@{}c@{}c@{}c@{}c@{}c@{}}
& 1 & \cdots & j_1 &\cdots &j_2 &\cdots &j_k &\cdots &n \\
\left.\right|
&\boldsymbol{\alpha }_1,&\cdots ,&\boldsymbol{\alpha }_{j_1+1},&\cdots ,&\boldsymbol{\alpha }_{j_2+1},&\cdots ,&\boldsymbol{\alpha }_{j_k+1},&\cdots ,&\boldsymbol{\alpha }_{n-1}& \left|\right.
\end{array}\\
\\
\end{array}}}.
\\
&=\left| \boldsymbol{\alpha }_1,\boldsymbol{\alpha }_2,\cdots ,\boldsymbol{\alpha }_n \right|+\left| \boldsymbol{\alpha }_n,\boldsymbol{\alpha }_1,\cdots ,\boldsymbol{\alpha }_{n-1} \right|
\\
&=c+\left( -1 \right) ^{n-1}\left| \boldsymbol{\alpha }_1,\boldsymbol{\alpha }_2,\cdots ,\boldsymbol{\alpha }_n \right|
\\
&=c+\left( -1 \right) ^{n-1}c
\\
&=\begin{cases}
0 \,\,,n\text{为偶数}\\
2c,n\text{为奇数}\\
\end{cases}
\end{align*}
\end{solution}

\begin{exercise}
令
\[
\left( a_{1} a_{2} \cdots a_{n} \right) = 
\begin{vmatrix}
a_{1} & 1 &   &   &   \\
-1 & a_{2} & 1 &   &   \\
& -1 & a_{3} & \ddots &   \\
&   & \ddots & \ddots & 1 \\
&   &   & -1 & a_{n}
\end{vmatrix},
\]
证明关于连分数的如下等式成立:
\[
a_{1} + \frac{1}{a_{2} + \frac{1}{a_{3} + \cdots + \frac{1}{a_{n - 1} + \frac{1}{a_{n}}}}} = \frac{\left( a_{1} a_{2} \cdots a_{n} \right)}{\left( a_{2} a_{3} \cdots a_{n} \right)}.
\]
\end{exercise}
\begin{solution}
假设等式对$\forall n\leq k-1,k\in \mathbb{N}_+$都成立.则当$n=k$时,将行列式$(a_1a_2,\cdots,a_k)$按第一列展开得
\begin{align*}
\left( a_1a_2\cdots a_k \right) &=\left| \begin{matrix}
a_1&		1&		&		&		\\
-1&		a_2&		1&		&		\\
&		-1&		a_3&		\ddots&		\\
&		&		\ddots&		\ddots&		1\\
&		&		&		-1&		a_k\\
\end{matrix} \right|=a_1\left| \begin{matrix}
a_2&		1&		&		\\
-1&		a_3&		\ddots&		\\
&		\ddots&		\ddots&		1\\
&		&		-1&		a_k\\
\end{matrix} \right|+\left| \begin{matrix}
a_3&		1&		&		\\
-1&		a_4&		\ddots&		\\
&		\ddots&		\ddots&		1\\
&		&		-1&		a_k\\
\end{matrix} \right|
\\
&=a_1\left( a_2a_3\cdots a_k \right) +\left( a_3a_4\cdots a_k \right).
\end{align*}
从而
\begin{align*}
\frac{\left( a_1a_2\cdots a_k \right)}{\left( a_2a_3\cdots a_k \right)}=a_1+\frac{\left( a_3a_4\cdots a_k \right)}{\left( a_2a_3\cdots a_k \right)}=a_1+\frac{1}{\frac{\left( a_2a_3\cdots a_k \right)}{\left( a_3a_4\cdots a_k \right)}}.
\end{align*}
于是由归纳假设可知
\begin{align*}
\frac{\left( a_1a_2\cdots a_k \right)}{\left( a_2a_3\cdots a_k \right)}=a_1+\frac{1}{\frac{\left( a_2a_3\cdots a_k \right)}{\left( a_3a_4\cdots a_k \right)}}=a_1+\frac{1}{a_2+\frac{1}{a_3+\cdots +\frac{1}{a_{n-1}+\frac{1}{a_n}}}}.
\end{align*}
故由数学归纳法可知结论成立.
\end{solution}

\begin{exercise}
设\(\vert A\vert\)是\(n\)阶行列式,\(\vert A\vert\)的第\((i,j)\)元素\(a_{ij}=\max\{i,j\}\),试求\(\vert A\vert\)的值.
\end{exercise}
\begin{solution}
\begin{align*}
\left| \boldsymbol{A} \right|=\left| \begin{matrix}
1&		2&		3&		\cdots&		n\\
2&		2&		3&		\cdots&		n\\
3&		3&		3&		\cdots&		n\\
\vdots&		\vdots&		\vdots&		&		\vdots\\
n&		n&		n&		\cdots&		n\\
\end{matrix} \right|\xlongequal[i=n,n-1,\cdots ,2]{-r_i+r_{i-1}}\left| \begin{matrix}
-1&		0&		0&		\cdots&		0\\
2&		-1&		0&		\cdots&		0\\
3&		3&		-1&		\cdots&		0\\
\vdots&		\vdots&		\vdots&		&		\vdots\\
n&		n&		n&		\cdots&		n\\
\end{matrix} \right|=\left( -1 \right) ^{n-1}n.
\end{align*}
\end{solution}

\begin{exercise}
设\(\vert A\vert\)是\(n\)阶行列式,\(\vert A\vert\)的第\((i,j)\)元素\(a_{ij}=\vert i - j\vert\),试求\(\vert A\vert\)的值.
\end{exercise}
\begin{note}
注意:这只是一个\textbf{对称行列式},不是循环行列式.
类似这种每行、每列元素有一定的等差递进关系的行列式,都可以先尝试用每一列减去前面一列.
\end{note}
\begin{solution}
\begin{align*}
\left| \boldsymbol{A} \right|&=\left| \begin{matrix}
0&		1&		2&		\cdots&		n-2&		n-1\\
1&		0&		1&		\cdots&		n-3&		n-2\\
2&		1&		0&		\cdots&		n-4&		n-3\\
\vdots&		\vdots&		\vdots&		&		\vdots&		\vdots\\
n-1&		n-2&		n-3&		\cdots&		1&		0\\
\end{matrix} \right|\xlongequal[i=n,n-1,\cdots ,2]{-j_{i-1}+j_i}\left| \begin{matrix}
0&		1&		1&		\cdots&		1&		1\\
1&		-1&		1&		\cdots&		1&		1\\
2&		-1&		-1&		\cdots&		1&		1\\
\vdots&		\vdots&		\vdots&		&		\vdots&		\vdots\\
n-1&		-1&		-1&		\cdots&		-1&		-1\\
\end{matrix} \right|
\\
&\xlongequal[i=n-1,n-2,\cdots ,1]{r_n+r_i}\left| \begin{matrix}
n-1&		0&		0&		\cdots&		0&		0\\
n&		-2&		0&		\cdots&		0&		0\\
n+1&		-2&		-2&		\cdots&		0&		0\\
\vdots&		\vdots&		\vdots&		&		\vdots&		\vdots\\
n-1&		-1&		-1&		\cdots&		-1&		-1\\
\end{matrix} \right|=\left( -2 \right) ^{n-2}\left( n-1 \right) .
\end{align*}
\end{solution}

\begin{exercise}
求下列\(n\)阶行列式的值:
\[
\left| \boldsymbol{A} \right| = 
\begin{vmatrix}
1 & x_1(x_1 - a) & x_1^2(x_1 - a) & \cdots & x_1^{n - 1}(x_1 - a)\\
1 & x_2(x_2 - a) & x_2^2(x_2 - a) & \cdots & x_2^{n - 1}(x_2 - a)\\
\vdots & \vdots & \vdots & \ddots & \vdots\\
1 & x_n(x_n - a) & x_n^2(x_n - a) & \cdots & x_n^{n - 1}(x_n - a)
\end{vmatrix}.
\]
\end{exercise}
\begin{note}
当行列式的行或列有一定的规律性时,但是由于缺少一行或一列导致这个行列式行或列的规律性并不完整.此时我们可以尝试\hyperlink{行列式计算:升阶法}{升阶法}补全这个行列式行或列的规律,再对行列式进行化简.

本题若直接使用\hyperref[大拆分法]{大拆分法}会得到比较多的行列式,而且每个行列式并不是完整的$Vandermode$行列式.后续求解很繁琐,因此不采取\hyperref[大拆分法]{大拆分法}.
\end{note}
\begin{solution}
(\hyperlink{行列式计算:升阶法}{升阶法})考虑$n+1$阶行列式\(|\boldsymbol{B}|=\left|\begin{matrix}
1 & x_1 - a & x_1(x_1 - a) & x_{1}^{2}(x_1 - a) & \cdots & x_{1}^{n - 1}(x_1 - a)\\
1 & x_2 - a & x_2(x_2 - a) & x_{2}^{2}(x_2 - a) & \cdots & x_{2}^{n - 1}(x_2 - a)\\
\vdots & \vdots & \vdots & \vdots &  & \vdots\\
1 & x_n - a & x_n(x_n - a) & x_{n}^{2}(x_n - a) & \cdots & x_{n}^{n - 1}(x_n - a)\\
1 & y - a & y(y - a) & y^2(y - a) & \cdots & y^{n - 1}(y - a)
\end{matrix}\right|\),则
\begin{align*}
|\boldsymbol{B}|=\left|\begin{matrix}
1 & x_1 & x_{1}^{2} & x_{1}^{3} & \cdots & x_{1}^{n}\\
1 & x_2 & x_{2}^{2} & x_{2}^{3} & \cdots & x_{2}^{n}\\
\vdots & \vdots & \vdots & \vdots &  & \vdots\\
1 & x_n & x_{n}^{2} & x_{n}^{3} & \cdots & x_{n}^{n}\\
1 & y & y^2 & y^3 & \cdots & y^n
\end{matrix}\right|=\prod_{k = 1}^{n}(y - x_k)\prod_{1\leqslant i < j\leqslant n}(x_j - x_i).
\end{align*}
由上式可知,\(|\boldsymbol{B}|\)可以看作一个关于\(y\)的\(n\)次多项式.
将\(|\boldsymbol{B}|\)按最后一行展开得到
\begin{align*}
|\boldsymbol{B}|=\sum_{i = 1}^{n + 1}(-1)^{n + i}B_{n + 1,i}y^{i - 1},\text{其中}B_{ni}\text{是}|\boldsymbol{B}|\text{的第}(n + 1,i)\text{元的余子式},i = 1,2,\cdots,n + 1.
\end{align*}
从而
\begin{align}\label{eq:两多项式相等1.1}
|\boldsymbol{B}|=(-1)^{n + 2}B_{n + 1,1}+\sum_{i = 2}^{n + 1}(-1)^{n + i + 1}B_{n + 1,i}y^{i - 2}(y - a)=\prod_{k = 1}^{n}(y - x_k)\prod_{1\leqslant i < j\leqslant n}(x_j - x_i).
\end{align}
又易知\(B_{n + 1,2}=|\boldsymbol{A}|\),而当\(a = 0\)时,由等式\eqref{eq:两多项式相等1.1}可知,\(|\boldsymbol{B}|\)中\(y\)前面的系数只有\(B_{n + 1,2}\).比较等式\eqref{eq:两多项式相等1.1}两边\(y\)的系数可得
\begin{align*}
(-1)^{n + 3}|\boldsymbol{A}|=(-1)^{n + 3}B_{n + 1,2}=\prod_{1\leqslant i < j\leqslant n}(x_j - x_i)\left(\sum_{i = 1}^{n}(-x_1)\cdots (-x_{i - 1})(-x_{i + 1})\cdots (-x_n)\right).
\end{align*}
于是\(|\boldsymbol{A}|=(-1)^{n + 3}(-1)^{n - 1}\prod_{1\leqslant i < j\leqslant n}(x_j - x_i)\left(\sum_{i = 1}^{n}x_1\cdots x_{i - 1}x_{i + 1}\cdots x_n\right)=\prod_{1\leqslant i < j\leqslant n}(x_j - x_i)\left(\sum_{i = 1}^{n}x_1\cdots x_{i - 1}x_{i + 1}\cdots x_n\right)\).

当\(a\neq 0\)时,由等式\((1.1)\)可知,\(|\boldsymbol{B}|\)中\(y\)前面的系数不只有\(B_{n + 1,2}\),但是,我们比较等式\eqref{eq:两多项式相等1.1}两边的常数项可得
\begin{align}\label{eq:等式1.2}
(-1)^{n + 2}B_{n + 1,1}-a(-1)^{n + 3}B_{n + 1,2}=\prod_{1\leqslant i < j\leqslant n}(x_j - x_i)\prod_{k = 1}^{n}(-x_k).
\end{align}
又因为
\begin{align*}
B_{n + 1,1}&=\left|\begin{matrix}
x_1 - a & x_1(x_1 - a) & x_{1}^{2}(x_1 - a) & \cdots & x_{1}^{n - 1}(x_1 - a)\\
x_2 - a & x_2(x_2 - a) & x_{2}^{2}(x_2 - a) & \cdots & x_{2}^{n - 1}(x_2 - a)\\
\vdots & \vdots & \vdots &  & \vdots\\
x_n - a & x_n(x_n - a) & x_{n}^{2}(x_n - a) & \cdots & x_{n}^{n - 1}(x_n - a)
\end{matrix}\right|
\\
&=\prod_{i = 1}^{n}(x_i - a)\left|\begin{matrix}
1 & x_1 & x_{1}^{2} & x_{1}^{3} & \cdots & x_{1}^{n - 1}\\
1 & x_2 & x_{2}^{2} & x_{2}^{3} & \cdots & x_{2}^{n - 1}\\
\vdots & \vdots & \vdots & \vdots &  & \vdots\\
1 & x_n & x_{n}^{2} & x_{n}^{3} & \cdots & x_{n}^{n - 1}
\end{matrix}\right|=\prod_{i = 1}^{n}(x_i - a)\prod_{1\leqslant i < j\leqslant n}(x_j - x_i).
\end{align*}
所以再结合等式\eqref{eq:等式1.2}可得
\begin{align*}
-a(-1)^{n + 3}|\boldsymbol{A}|&=-a(-1)^{n + 3}B_{n + 1,2}=\prod_{1\leqslant i < j\leqslant n}(x_j - x_i)\prod_{k = 1}^{n}(-x_k)-(-1)^{n + 2}B_{n + 1,1}
\\
&=(-1)^n\prod_{k = 1}^{n}x_k\prod_{1\leqslant i < j\leqslant n}(x_j - x_i)+(-1)^{n + 1}\prod_{i = 1}^{n}(x_i - a)\prod_{1\leqslant i < j\leqslant n}(x_j - x_i)
\\
&=(-1)^n\prod_{1\leqslant i < j\leqslant n}(x_j - x_i)\left[\prod_{k = 1}^{n}x_k-\prod_{i = 1}^{n}(x_i - a)\right].
\end{align*}
故此时\(|\boldsymbol{A}|=\prod_{1\leqslant i < j\leqslant n}(x_j - x_i)\left(\prod_{k = 1}^{n}x_k-\prod_{i = 1}^{n}(x_i - a)\right)\).
\end{solution}

\begin{exercise}
求下列行列式式的值($n$为偶数)
\begin{align*}
I=\left| \begin{matrix}
1&		1&		\cdots&		1&		1\\
2&		2^2&		\cdots&		2^n&		2^{n+1}\\
\vdots&		\vdots&		\ddots&		\vdots&		\vdots\\
n&		n^2&		\cdots&		n^n&		n^{n+1}\\
\frac{n}{2}&		\frac{n^2}{3}&		\cdots&		\frac{n^n}{n+1}&		\frac{n^{n+1}}{n+2}\\
\end{matrix} \right|.
\end{align*}
\end{exercise}
\begin{note}
应用\hyperref[proposition:行列式的求导运算]{行列式函数求导求行列式}的值.
\end{note}
\begin{solution}
令\(G(x)=\left|\begin{matrix}
1 & 1 & \cdots & 1 & 1\\
2 & 2^2 & \cdots & 2^n & 2^{n + 1}\\
\vdots & \vdots & \ddots & \vdots & \vdots\\
n & n^2 & \cdots & n^n & n^{n + 1}\\
\frac{x^2}{2} & \frac{x^3}{3} & \cdots & \frac{x^{n + 1}}{n + 1} & \frac{x^{n + 2}}{n + 2}
\end{matrix}\right|\),则\(I = \frac{G(n)}{n}\)且\(G(0) = 0\).      
利用行列式求导公式,可得
\begin{align*}
G'(x)&=\left|\begin{matrix}
1 & 1 & \cdots & 1 & 1\\
2 & 2^2 & \cdots & 2^n & 2^{n + 1}\\
\vdots & \vdots & \ddots & \vdots & \vdots\\
n & n^2 & \cdots & n^n & n^{n + 1}\\
x & x^2 & \cdots & x^n & x^{n + 1}
\end{matrix}\right|
= n!x\left|\begin{matrix}
1 & 1 & \cdots & 1 & 1\\
1 & 2 & \cdots & 2^{n - 1} & 2^n\\
\vdots & \vdots & \ddots & \vdots & \vdots\\
1 & n & \cdots & n^{n - 1} & n^n\\
1 & x & \cdots & x^{n - 1} & x^n
\end{matrix}\right|
= n!\prod_{1\leqslant i < j\leqslant n}(j - i)\prod_{k = 0}^{n}(x - k).
\end{align*}
因此
\begin{align*}
I &= \frac{G(n)}{n}=\frac{\int_{0}^{n}G'(x)dx}{n}=(n - 1)!\prod_{1\leqslant i < j\leqslant n}(j - i)\int_{0}^{n}\prod_{k = 0}^{n}(x - k)dx
\\
&\stackrel{\text{区间再现}}{=}(n - 1)!\prod_{1\leqslant i < j\leqslant n}(j - i)\int_{0}^{n}\prod_{k = 0}^{n}(n - k - x)dx
\\
&= (-1)^{n + 1}(n - 1)!\prod_{1\leqslant i < j\leqslant n}(j - i)\int_{0}^{n}\prod_{k = 0}^{n}(x - k)dx
\\
&= (-1)^{n + 1}I.
\end{align*}      
由于\(n\)为偶数,所以\((-1)^{n + 1} = -1\).于是\(I = -I\).故\(I = 0\). 
\end{solution}

\chapter{矩阵}

\section{矩阵的运算}

\begin{proposition}[标准单位向量和基础矩阵]\label{proposition:标准单位向量和基础矩阵}
\large{\textbf{1.标准单位向量}}

\(n\)维标准单位列向量是指下列\(n\)个\(n\)维列向量:
\[
\boldsymbol{e}_{1}=\left(\begin{array}{c}
1 \\
0 \\
\vdots \\
0
\end{array}\right), \quad \boldsymbol{e}_{2}=\left(\begin{array}{c}
0 \\
1 \\
\vdots \\
0
\end{array}\right), \quad \cdots, \quad \boldsymbol{e}_{n}=\left(\begin{array}{c}
0 \\
0 \\
\vdots \\
1
\end{array}\right)
\]

向量组\(\boldsymbol{e}_{1}', \boldsymbol{e}_{2}', \cdots, \boldsymbol{e}_{n}'\)则被称为\(n\)维标准单位行向量,容易验证标准单位向量有下列基本性质:
\begin{enumerate}
\item 若\(i \neq j\),则\(\boldsymbol{e}_{i}' \boldsymbol{e}_{j}=0\),而\(\boldsymbol{e}_{i}' \boldsymbol{e}_{i}=1\);

\item 若\(A=(a_{ij})\)是\(m\times n\)矩阵,则\(A\boldsymbol{e}_{i}\)是\(A\)的第\(i\)个列向量;\(\boldsymbol{e}_{i}'A\)是\(A\)的第\(i\)个行向量;

\item 若\(A=(a_{ij})\)是\(m\times n\)矩阵,则\(\boldsymbol{e}_{i}'A\boldsymbol{e}_{j}=a_{ij}\);

\item \label{矩阵相等的判定准则}\hypertarget{proposition:矩阵相等的判定法则}{\textbf{判定准则:}}设\(A,B\)都是\(m\times n\)矩阵,则\(A = B\)当且仅当\(A\boldsymbol{e}_{i}=B\boldsymbol{e}_{i}(1\leq i\leq n)\)成立,也当且仅当\(\boldsymbol{e}_{i}'A=\boldsymbol{e}_{i}'B(1\leq i\leq m)\)成立.
\end{enumerate}

\large{\textbf{2.基础矩阵}}

\(n\)阶基础矩阵(又称初级矩阵)是指\(n^{2}\)个\(n\)阶矩阵\(\{E_{ij}, 1\leq i,j\leq n\}\).这里\(E_{ij}\)是一个\(n\)阶矩阵,它的第\((i,j)\)元素等于\(1\),其他元素全为\(0\).基础矩阵也可以看成是标准单位向量的积:\(E_{ij}=\boldsymbol{e}_{i}\boldsymbol{e}_{j}^{T}\).由此不难证明基础矩阵的下列性质:
\begin{enumerate}
\item 若\(j\neq k\),则\(E_{ij}E_{kl} = 0\);

\item  若\(j = k\),则\(E_{ij}E_{kl}=E_{il}\);

\item  若\(A\)是\(n\)阶矩阵且\(A=(a_{ij})\),则\(A=\sum_{i = 1}^{n}\sum_{j = 1}^{n}a_{ij}E_{ij}\);

\item  若\(A\)是\(n\)阶矩阵且\(A=(a_{ij})\),则\(E_{ij}A\)的第\(i\)行是\(A\)的第\(j\)行,\(E_{ij}A\)的其他行全为零;

\item  若\(A\)是\(n\)阶矩阵且\(A=(a_{ij})\),则\(AE_{ij}\)的第\(j\)列是\(A\)的第\(i\)列,\(AE_{ij}\)的其他列全为零;

\item  若\(A\)是\(n\)阶矩阵且\(A=(a_{ij})\),则\(E_{ij}AE_{kl}=a_{jk}E_{il}\).
\end{enumerate}
\end{proposition}
\begin{note}
标准单位向量和基础矩阵虽然很简单,但如能灵活应用就可以得到意外的结果.我们在今后将经常应用它们,因此请读者熟记这些结论.

一些常见的想法:

\textbf{1.可以将一般的矩阵写成标准单位列向量或基础矩阵的形式(这个形式可以是和式的形式,也可以是分块的形式)}.

\textbf{2.如果要证明两个矩阵相等,那么我们就可以考虑\hyperlink{proposition:矩阵相等的判定法则}{判定法则}}.

\textbf{3.如果某种等价关系蕴含了一种递减的规律(项数减少,阶数降低等),那么我们就可以考虑数学归纳法,去尝试根据这个规律得到一些结论}.
\end{note}

\begin{definition}[循环矩阵]\label{definition:循环矩阵}
1.    下列形状的$n$阶矩阵称为\(n\)阶基础循环矩阵:
\[ 
\boldsymbol{J}=
\left( \begin{matrix}
0&		1&		0&		\cdots&		0\\
0&		0&		1&		\cdots&		0\\
\vdots&		\vdots&		\vdots&	\ddots&		\vdots\\
0&		0&		0&		\cdots&		1\\
1&		0&		0&		\cdots&		0\\
\end{matrix} \right).
\]

2.下列形状的矩阵称为循环矩阵:
\[
\begin{pmatrix}
a_1 & a_2 & a_3 & \cdots & a_n \\
a_n & a_1 & a_2 & \cdots & a_{n - 1} \\
a_{n - 1} & a_n & a_1 & \cdots & a_{n - 2} \\
\vdots & \vdots & \vdots & & \vdots \\
a_2 & a_3 & a_4 & \cdots & a_1
\end{pmatrix}.
\]
\end{definition}
\begin{note}
记\(C_n(\mathbb{K})\)为\(\mathbb{K}\)上所有\(n\)阶循环矩阵构成的集合.
\end{note}

\begin{proposition}[循环矩阵的性质]\label{proposition:循环矩阵的性质}
\begin{enumerate}
\item 若$\boldsymbol{J}$为$n$阶基础循环矩阵,
则
\[
\boldsymbol{J}^{k} = 
\begin{pmatrix}
O & I_{n - k} \\
I_{k} & O
\end{pmatrix},  1 \leq k \leq n.
\]
\item 若循环矩阵\begin{align*}
A=\begin{pmatrix}
a_1 & a_2 & a_3 & \cdots & a_n \\
a_n & a_1 & a_2 & \cdots & a_{n - 1} \\
a_{n - 1} & a_n & a_1 & \cdots & a_{n - 2} \\
\vdots & \vdots & \vdots & & \vdots \\
a_2 & a_3 & a_4 & \cdots & a_1
\end{pmatrix}.
\end{align*}
,则循环矩阵$A$可以表示为基础循环矩阵$J$的多项式:
\begin{align*}
A=a_1I_n+a_2J+a_3J^2+\cdots+a_nJ^{n-1}.
\end{align*}
反之,若一个矩阵能表示为基础循环矩阵$J$的多项式,则它必是循环矩阵.
\item \label{example:iten546641856}同阶循环矩阵之积仍是循环矩阵.
\item 基础循环矩阵$\boldsymbol{J}=
\begin{pmatrix}
O & I_{n - k} \\
I_{k} & O
\end{pmatrix} (1 \leq k \leq n)$的逆仍是循环矩阵,并且
\begin{align*}
\boldsymbol{J}^{-1} =\begin{pmatrix}
O & I_{k} \\
I_{n-k} & O
\end{pmatrix},1 \leq k \leq n.
\end{align*}
\end{enumerate}
\end{proposition}
\begin{note}
循环矩阵的性质及应用详见\href{https://www.cnblogs.com/torsor/p/8848641.html}{谢启鸿博客}.
\end{note}
\begin{proof}
\begin{enumerate}
\item 将$\boldsymbol{J}$写作$(e_n,e_1,\cdots,e_{n-1})$,其中$e_i$是标准单位列向量($i=1,2,\cdots,n$).由分块矩阵乘法并注意到$\boldsymbol{J}e_i$就是$\boldsymbol{J}$的第$i$列,可得
\begin{align*}
\boldsymbol{J}^2=\boldsymbol{J}\left( e_{n,}e_1,\cdots ,e_{n-1} \right) =\left( \boldsymbol{J}e_{n,}\boldsymbol{J}e_1,\cdots ,\boldsymbol{J}e_{n-1} \right) =\left( e_{n-1},e_n,\cdots ,e_{n-2} \right) .
\end{align*}
不断这样做下去就可以得到结论.
\item 由\hyperref[definition:循环矩阵]{循环矩阵和基础循环矩阵的定义}和\hyperref[proposition:循环矩阵的性质]{循环矩阵的性质2}容易得到证明.
\item 由\hyperref[proposition:循环矩阵的性质]{循环矩阵的性质2},可知两个循环矩阵之积可写为基础循环矩阵$J$的两个多项式之积.又由\hyperref[proposition:循环矩阵的性质]{循环矩阵的性质1},可知$J^n=I_n$.因此两个循环矩阵之积可以表示为基础循环矩阵$J$的多项式,故由\hyperref[proposition:循环矩阵的性质]{循环矩阵的性质1}即得结论.
\item 利用矩阵初等行变换可得
\begin{align*}
\left( \begin{matrix}
O&		\boldsymbol{I}_{n-k}&		\boldsymbol{I}_{n-k}&		O\\
\boldsymbol{I}_k&		O&		O&		\boldsymbol{I}_k\\
\end{matrix} \right) \rightarrow \left( \begin{matrix}
\boldsymbol{I}_k&		O&		O&		\boldsymbol{I}_k\\
O&		\boldsymbol{I}_{n-k}&		\boldsymbol{I}_{n-k}&		O\\
\end{matrix} \right) ,1\le k\le n.
\end{align*}
从而$\boldsymbol{J}^{-1}=\left( \begin{matrix}
O&		\boldsymbol{I}_k\\
\boldsymbol{I}_{n-k}&		O\\
\end{matrix} \right) ,1\le k\le n.$
\end{enumerate}
\end{proof}

\begin{proposition}[循环行列式计算公式]\label{proposition:循环行列式计算公式}
已知下列循环矩阵\(A\):
\[
A = 
\begin{pmatrix}
a_1 & a_2 & a_3 & \cdots & a_n\\
a_n & a_1 & a_2 & \cdots & a_{n - 1}\\
a_{n - 1} & a_n & a_1 & \cdots & a_{n - 2}\\
\vdots & \vdots & \vdots & & \vdots\\
a_2 & a_3 & a_4 & \cdots & a_1
\end{pmatrix}.
\]
则\(A\)的行列式的值为:
\[
|A|| = f(\varepsilon_1)f(\varepsilon_2)\cdots f(\varepsilon_n).
\]
其中\(f(x)=a_1 + a_2x + a_3x^2+\cdots+a_nx^{n - 1}\),\(\varepsilon_1,\varepsilon_2,\cdots,\varepsilon_n\)是\(1\)的所有\(n\)次方根.
\end{proposition}
\begin{note}
关键是要注意到
\begin{align*}
AV = 
\begin{pmatrix}
f(\varepsilon_1) & f(\varepsilon_2) & f(\varepsilon_3) & \cdots & f(\varepsilon_n)\\
\varepsilon_1f(\varepsilon_1) & \varepsilon_2f(\varepsilon_2) & \varepsilon_3f(\varepsilon_3) & \cdots & \varepsilon_nf(\varepsilon_n)\\
\varepsilon_1^2f(\varepsilon_1) & \varepsilon_2^2f(\varepsilon_2) & \varepsilon_3^2f(\varepsilon_3) & \cdots & \varepsilon_n^2f(\varepsilon_n)\\
\vdots & \vdots & \vdots & & \vdots\\
\varepsilon_1^{n - 1}f(\varepsilon_1) & \varepsilon_2^{n - 1}f(\varepsilon_2) & \varepsilon_3^{n - 1}f(\varepsilon_3) & \cdots & \varepsilon_n^{n - 1}f(\varepsilon_n)
\end{pmatrix}.
\end{align*}
然后再利用\hyperref[proposition:一些能写成两个向量乘积的矩阵]{命题\ref{proposition:一些能写成两个向量乘积的矩阵}}就能得到分解$AV = V\Lambda$.
\end{note}
\begin{proof}
作多项式\(f(x)=a_1 + a_2x + a_3x^2+\cdots+a_nx^{n - 1}\),令\(\varepsilon_1,\varepsilon_2,\cdots,\varepsilon_n\)是\(1\)的所有\(n\)次方根.又令
\[
V = 
\begin{pmatrix}
1 & 1 & 1 & \cdots & 1\\
\varepsilon_1 & \varepsilon_2 & \varepsilon_3 & \cdots & \varepsilon_n\\
\varepsilon_1^2 & \varepsilon_2^2 & \varepsilon_3^2 & \cdots & \varepsilon_n^2\\
\vdots & \vdots & \vdots & & \vdots\\
\varepsilon_1^{n - 1} & \varepsilon_2^{n - 1} & \varepsilon_3^{n - 1} & \cdots & \varepsilon_n^{n - 1}
\end{pmatrix},
\Lambda = 
\begin{pmatrix}
f(\varepsilon_1) & 0 & 0 & \cdots & 0\\
0 & f(\varepsilon_2) & 0 & \cdots & 0\\
0 & 0 & f(\varepsilon_3) & \cdots & 0\\
\vdots & \vdots & \vdots & & \vdots\\
0 & 0 & 0 & \cdots & f(\varepsilon_n)
\end{pmatrix}
\]
则
\[
AV = 
\begin{pmatrix}
f(\varepsilon_1) & f(\varepsilon_2) & f(\varepsilon_3) & \cdots & f(\varepsilon_n)\\
\varepsilon_1f(\varepsilon_1) & \varepsilon_2f(\varepsilon_2) & \varepsilon_3f(\varepsilon_3) & \cdots & \varepsilon_nf(\varepsilon_n)\\
\varepsilon_1^2f(\varepsilon_1) & \varepsilon_2^2f(\varepsilon_2) & \varepsilon_3^2f(\varepsilon_3) & \cdots & \varepsilon_n^2f(\varepsilon_n)\\
\vdots & \vdots & \vdots & & \vdots\\
\varepsilon_1^{n - 1}f(\varepsilon_1) & \varepsilon_2^{n - 1}f(\varepsilon_2) & \varepsilon_3^{n - 1}f(\varepsilon_3) & \cdots & \varepsilon_n^{n - 1}f(\varepsilon_n)
\end{pmatrix}
= V\Lambda
\]
因此
\[
|A||V| = |AV| = |V\Lambda| = |V||\Lambda|.
\]
因为\(\varepsilon_i\)互不相同,所以\(|V|\neq0\),从而
\[
|A| = |\Lambda| = f(\varepsilon_1)f(\varepsilon_2)\cdots f(\varepsilon_n).
\]
\end{proof}

\begin{proposition}[b-循环矩阵]\label{proposition:b-循环矩阵}
设\(b\)为非零常数,下列形状的矩阵称为b -循环矩阵:
\[
A = 
\begin{pmatrix}
a_1 & a_2 & a_3 & \cdots & a_n\\
ba_n & a_1 & a_2 & \cdots & a_{n - 1}\\
ba_{n - 1} & ba_n & a_1 & \cdots & a_{n - 2}\\
\vdots & \vdots & \vdots & & \vdots\\
ba_2 & ba_3 & ba_4 & \cdots & a_1
\end{pmatrix}
\]
\begin{enumerate}[(1)]
\item 证明:同阶b -循环矩阵的乘积仍然是b-循环矩阵;
\item 求上述b-循环矩阵\(A\)的行列式的值.
\end{enumerate}
\end{proposition}
\begin{proof}
\begin{enumerate}[(1)]
\item (证明类似于\hyperref[example:iten546641856]{循环矩阵的性质\ref{example:iten546641856}})设\(J_b=\begin{pmatrix}
O & I_{n - 1}\\
b & O
\end{pmatrix}\),则$J_{b}^{k}=\left( \begin{matrix}
O&		I_{n-k}\\
bI_k&		O\\
\end{matrix} \right) ,0\le k\le n-1$.从而\(J_b^n = bI_n\)且\(A = a_1I_n + a_2J_b + a_3J_b^2+\cdots+ a_nJ_b^{n - 1}\). 因此同阶\(b -\)循环阵的乘积仍然可以写成$J_b$的$n-1$次多项式,故同阶\(b -\)循环阵的乘积仍然是\(b -\)循环矩阵.

\item (证明完全类似\hyperref[proposition:循环行列式计算公式]{循环行列式计算公式的证明})作多项式\(f(x)=a_1 + a_2x + a_3x^2+\cdots + a_nx^{n - 1}\),令\(\varepsilon_1,\varepsilon_2,\cdots,\varepsilon_n\)是\(b\)的所有\(n\)次方根.又令
\[
V = 
\begin{pmatrix}
1 & 1 & 1 & \cdots & 1\\
\varepsilon_1 & \varepsilon_2 & \varepsilon_3 & \cdots & \varepsilon_n\\
\varepsilon_1^2 & \varepsilon_2^2 & \varepsilon_3^2 & \cdots & \varepsilon_n^2\\
\vdots & \vdots & \vdots & & \vdots\\
\varepsilon_1^{n - 1} & \varepsilon_2^{n - 1} & \varepsilon_3^{n - 1} & \cdots & \varepsilon_n^{n - 1}
\end{pmatrix},
\Lambda = 
\begin{pmatrix}
f(\varepsilon_1) & 0 & 0 & \cdots & 0\\
0 & f(\varepsilon_2) & 0 & \cdots & 0\\
0 & 0 & f(\varepsilon_3) & \cdots & 0\\
\vdots & \vdots & \vdots & & \vdots\\
0 & 0 & 0 & \cdots & f(\varepsilon_n)
\end{pmatrix}
\]
则
\[
AV = 
\begin{pmatrix}
f(\varepsilon_1) & f(\varepsilon_2) & f(\varepsilon_3) & \cdots & f(\varepsilon_n)\\
\varepsilon_1f(\varepsilon_1) & \varepsilon_2f(\varepsilon_2) & \varepsilon_3f(\varepsilon_3) & \cdots & \varepsilon_nf(\varepsilon_n)\\
\varepsilon_1^2f(\varepsilon_1) & \varepsilon_2^2f(\varepsilon_2) & \varepsilon_3^2f(\varepsilon_3) & \cdots & \varepsilon_n^2f(\varepsilon_n)\\
\vdots & \vdots & \vdots & & \vdots\\
\varepsilon_1^{n - 1}f(\varepsilon_1) & \varepsilon_2^{n - 1}f(\varepsilon_2) & \varepsilon_3^{n - 1}f(\varepsilon_3) & \cdots & \varepsilon_n^{n - 1}f(\varepsilon_n)
\end{pmatrix}
= V\Lambda
\]
因此
\[
|A||V| = |AV| = |V\Lambda| = |V||\Lambda|.
\]
因为\(\varepsilon_i\)互不相同,所以\(|V|\neq0\),从而
\[
|A| = |\Lambda| = f(\varepsilon_1)f(\varepsilon_2)\cdots f(\varepsilon_n).
\]
\end{enumerate}
\end{proof}


\begin{proposition}[幂零Jordan块]\label{proposition:幂零Jordan块}
设\(n\)阶幂零Jordan块
\[
\boldsymbol{A}=\left(\begin{array}{ccccc}
0 & 1 & 0 & \cdots & 0 \\
0 & 0 & 1 & \cdots & 0 \\
\vdots & \vdots & \vdots & \ddots & \vdots \\
0 & 0 & 0 & \cdots & 1 \\
0 & 0 & 0 & \cdots & 0
\end{array}\right)
\]
则

\[
\boldsymbol{A}^{k}=\left(\begin{array}{cc}
O & I_{n - k} \\
O & O
\end{array}\right),1 \leq k \leq n.
\]
\end{proposition}
\begin{proof}
将\(\boldsymbol{A}\)写为\(\boldsymbol{A}=(0,\mathbf{e}_{1},\mathbf{e}_{2},\cdots,\mathbf{e}_{n - 1})\),其中\(\mathbf{e}_{i}\)是标准单位列向量.由分块矩阵乘法并注意\(\boldsymbol{A}\mathbf{e}_{i}\)就是\(\boldsymbol{A}\)的第\(i\)列,因此
\[
\boldsymbol{A}^{2}=(0,\boldsymbol{A}\mathbf{e}_{1},\boldsymbol{A}\mathbf{e}_{2},\cdots,\boldsymbol{A}\mathbf{e}_{n - 1})=(0,0,\mathbf{e}_{1},\cdots,\mathbf{e}_{n - 2})
\]
不断这样做下去就可得到结论.
\end{proof}

\begin{example}\label{example:56471456}
设\(A\)是\(n\)阶矩阵,\(A\)适合\(A^n = O\)时,\(I_n - A\)必是可逆矩阵.
\end{example}
\begin{proof}
注意到
\begin{align*}
I_n=I_n-A^n=\left( I_n-A \right) \left( I_n+A+A^2+\cdots +A^{n-1} \right) .
\end{align*}
故此时\(I_n - A\)必是可逆矩阵.
\end{proof}

\begin{example}
设\(A\)是\(n\)阶矩阵,\(A\)适合\(AB = B(I_n - A)\)对任意\(n\)阶矩阵\(B\)成立,那么\(B = O\).
\end{example}
\begin{note}
若已知矩阵乘法的相关等式,可以尝试得到一些递推等式.
\end{note}
\begin{proof}
假设\(A^k = O\),其中\(k\)为某个正整数. 由条件可得\(AB = B(I_n - A)\),于是\(O = A^kB = B(I_n - A)^k\). 由\hyperref[example:56471456]{上一题}知\(I_n - A\)是可逆矩阵,从而\(B = O\).
\end{proof}


\begin{proposition}[多项式的友矩和Frobenius块]\label{proposition:多项式的友矩和Frobenius块}
设首一多项式\(f(x)=x^{n}+a_{1}x^{n - 1}+\cdots+a_{n - 1}x + a_{n}\),\(f(x)\)的友阵
\[
\boldsymbol{C}(f(x))=\left(\begin{array}{cccccc}
0 & 0 & \cdots & 0 & 0 & -a_{n} \\
1 & 0 & \cdots & 0 & 0 & -a_{n - 1} \\
0 & 1 & \cdots & 0 & 0 & -a_{n - 2} \\
\vdots & \vdots & \ddots & \vdots & \vdots & \vdots \\
0 & 0 & \cdots & 1 & 0 & -a_{2} \\
0 & 0 & \cdots & 0 & 1 & -a_{1}
\end{array}\right),
\]

则
\(\vert x\boldsymbol{I}_{n}-\boldsymbol{C}(f(x))\vert=f(x)\).

\(\boldsymbol{C}(f(x))\)的转置\(F(f(x))\)称为\(f(x)\)的Frobenius块.即
\begin{align*}
\boldsymbol{C}^T(f(x))=\left( \begin{matrix}
0&		1&		0&		\cdots&		0&		0\\
0&		0&		1&		\cdots&		0&		0\\
\vdots&		\vdots&		\vdots&		\ddots&		\vdots&		\vdots\\
0&		0&		0&		\cdots&		1&		0\\
0&		0&		0&		\cdots&		0&		1\\
-a_n&		-a_{n-1}&		-a_{n-2}&		\cdots&		-a_2&		-a_1\\
\end{matrix} \right).
\end{align*}
并且容易验证$\boldsymbol{C}(f(x))$具有以下性质,其中$\boldsymbol{e}_i$是标准单位列向量($i=1,2,\cdots,n$):
\begin{align*}
C(f(x))\boldsymbol{e}_i = \boldsymbol{e}_{i + 1} \ (1 \leq i \leq n - 1), \ C(f(x))\boldsymbol{e}_n = - \sum_{i = 1}^{n} a_{n - i + 1} \boldsymbol{e}_i .
\end{align*}
\end{proposition}
\begin{proof}
\(\vert x\boldsymbol{I}_{n}-\boldsymbol{C}(f(x))\vert=f(x)\)的证明见\hyperref[pro:友矩阵的特征多项式/行列式]{友矩阵的特征多项式/行列式}.
\end{proof}

\begin{example}
求下列矩阵的逆矩阵\((a_n\neq0)\):
\[
F = 
\begin{pmatrix}
0 & 0 & \cdots & 0 & -a_n\\
1 & 0 & \cdots & 0 & -a_{n - 1}\\
0 & 1 & \cdots & 0 & -a_{n - 2}\\
\vdots & \vdots & & \vdots & \vdots\\
0 & 0 & \cdots & 1 & -a_1
\end{pmatrix}.
\]
\end{example}
\begin{solution}
用初等变换法不难求得
\[
F^{-1} = 
\begin{pmatrix}
-\frac{a_{n - 1}}{a_n} & 1 & 0 & \cdots & 0\\
-\frac{a_{n - 2}}{a_n} & 0 & 1 & \cdots & 0\\
-\frac{a_{n - 3}}{a_n} & 0 & 0 & \cdots & 0\\
\vdots & \vdots & \vdots & & \vdots\\
-\frac{1}{a_n} & 0 & 0 & \cdots & 0
\end{pmatrix}.
\]
\end{solution}


\begin{proposition}\label{proposition:与对角矩阵可交换的矩阵必是对角阵}
和所有$n$阶对角矩阵乘法可交换的矩阵必是对角矩阵.
\end{proposition}
\begin{proof}
由矩阵乘法易得.
\end{proof}

\begin{proposition}[纯量矩阵的刻画]\label{proposition:纯量阵的刻画}
\begin{enumerate}[(1)]
\item 和所有\(n\)阶奇异阵乘法可交换的矩阵必是纯量阵\(kI_{n}\).

\item 和所有\(n\)阶非奇异阵乘法可交换的矩阵必是纯量阵\(kI_{n}\).

\item 和所有\(n\)阶正交阵乘法可交换的矩阵必是纯量阵\(kI_{n}\).

\item 和所有\(n\)阶矩阵乘法可交换的矩阵必是纯量阵\(kI_{n}\).
\end{enumerate}

\end{proposition}
\begin{proof}
首先设$A=(a_{ij})_{n\times n}$.
\begin{enumerate}
\item 设\(E_{ij}(1\leq i\neq j\leq n)\)为基础矩阵,因为基础矩阵都是奇异阵,所以由条件可知\(E_{ij}A = AE_{ij}\).注意到\(E_{ij}A\)是将\(A\)的第\(j\)行变为第\(i\)行而其他行都是零的\(n\)阶矩阵,\(AE_{ij}\)是将\(A\)的第\(i\)列变为第\(j\)列而其他列都是零的\(n\)阶矩阵,于是我们有
\begin{align*}
\bordermatrix{%
&    &		&		&		j&		&
\cr
&    &		&		&		&		&		\cr
&   &		&		&		&		&		\cr
i&    a_{j1}&		a_{j2}&		\cdots&		a_{jj}&		\cdots&		a_{jn}
\cr
&    &		&		&		&		&		\cr
&    &		&		&		&		&		\cr
} \quad= \quad \bordermatrix{%
&    &       &             j&     &
\cr
&    &		&		a_{1i}&		&		\cr
&    &		&		a_{2i}&		&		\cr
&    &		&		\vdots&		&		\cr
i&    &		&		a_{ii}&		&		\cr
&    &		&		\vdots&		&		\cr
&    &		&		a_{ni}&		&		\cr
}.
\end{align*}
从而比较上述等式两边矩阵的每个元素可得\(a_{ij}=0(i\neq j)\),\(a_{ii}=a_{jj}(1\leq i\neq j\leq n)\),因此\(A\)是纯量阵.

\item 设\(D=\text{diag}\{1,2,\cdots,n\}\)为对角阵,因为$D$为非奇异阵,所以由条件可知\(AD = DA\).进而
\begin{gather*}
AD=DA
\\
\Leftrightarrow A\left( \boldsymbol{e}_1,2\boldsymbol{e}_2,\cdots ,n\boldsymbol{e}_n \right) =\left( \boldsymbol{e}_1,2\boldsymbol{e}_2,\cdots ,n\boldsymbol{e}_n \right) A
\\
\Leftrightarrow \left( A\boldsymbol{e}_1,2A\boldsymbol{e}_2,\cdots ,nA\boldsymbol{e}_n \right) =\left( \boldsymbol{e}_1A,2\boldsymbol{e}_2A,\cdots ,n\boldsymbol{e}_nA \right) 
\\
\Leftrightarrow \left( \begin{matrix}
a_{11}&		2a_{12}&		\cdots&		na_{1n}\\
a_{21}&		2a_{22}&		\cdots&		na_{2n}\\
\vdots&		\vdots&		\ddots&		\vdots\\
a_{n1}&		2a_{n2}&		\cdots&		na_{nn}\\
\end{matrix} \right) =\left( \begin{matrix}
a_{11}&		a_{12}&		\cdots&		a_{1n}\\
2a_{21}&		2a_{22}&		\cdots&		2a_{2n}\\
\vdots&		\vdots&		\ddots&		\vdots\\
na_{n1}&		na_{n2}&		\cdots&		na_{nn}\\
\end{matrix} \right).
\end{gather*}
比较上述等式两边矩阵的每个元素可得$ja_{ij}=ia_{ij}\left( i\ne j \right)$,从而$\left( i-j \right) a_{ij}=0\left( i\ne j \right)$,于是$a_{ij}=0\left( i\ne j \right) $.故\(A=\text{diag}\{a_{11},a_{22},\cdots,a_{nn}\}\)也为对角阵.

设\(P_{ij}(1\leq i\neq j\leq n)\)为第一类初等阵,因为第一类初等阵均为非奇异阵,所以由条件可知\(AP_{ij}=P_{ij}A\).进而可得
\begin{align*}
\bordermatrix{%
&    &		&		i&		&		j&		&		\cr
&    a_{11}&		&		&		&		&		&		\cr
&    &		\ddots&		&		&		&		&		\cr
i&    &		&		0&		\cdots&		a_{jj}&		&		\cr
&   &		&		\vdots&		\ddots&		\vdots&		&		\cr
j&    &		&		a_{ii}&		\cdots&		0&		&		\cr
&    &		&		&		&		&		\ddots&		\cr
&    &		&		&		&		&		&		a_{nn}\cr
} \quad =\quad \bordermatrix{%
&    &		&		i&		&		j&		&		\cr
&    a_{11}&		&		&		&		&		&		\cr
&    &		\ddots&		&		&		&		&		\cr
i&    &		&		0&		\cdots&		a_{ii}&		&		\cr
&    &		&		\vdots&		\ddots&		\vdots&		&		\cr
j&   &		&		a_{jj}&		\cdots&		0&		&		\cr
&    &		&		&		&		&		\ddots&		\cr
&    &		&		&		&		&		&		a_{nn}\cr
} .
\end{align*}
从而比较上述等式两边矩阵的每个元素可得\(a_{ii}=a_{jj}(1\leq i\neq j\leq n)\),于是\(A\)为纯量阵.

\item 设第二类初等阵\(P_{i}(-1)(1\leq i\leq n)\),因为$P_{i}(-1)(1\leq i\leq n)$都是正交阵,所以由条件可知$P_{i}(-1)A=AP_{i}(-1)$.进而可得
\begin{align*}
\left( \begin{matrix}
a_{11}&		a_{12}&		\cdots&		a_{1i}&		\cdots&		a_{1n}\\
\vdots&		\vdots&		&		\vdots&		&		\vdots\\
-a_{i1}&		-a_{i2}&		\cdots&		-a_{ii}&		\cdots&		-a_{in}\\
\vdots&		\vdots&		&		\vdots&		&		\vdots\\
a_{n1}&		a_{n2}&		\cdots&		a_{ni}&		\cdots&		a_{nn}\\
\end{matrix} \right) =\left( \begin{matrix}
a_{11}&		\cdots&		-a_{1i}&		\cdots&		a_{1n}\\
a_{21}&		\cdots&		-a_{2i}&		\cdots&		a_{2n}\\
\vdots&		&		\vdots&		&		\vdots\\
a_{i1}&		\cdots&		-a_{ii}&		\cdots&		a_{in}\\
\vdots&		&		\vdots&		&		\vdots\\
a_{n1}&		\cdots&		-a_{ni}&		\cdots&		a_{nn}\\
\end{matrix} \right) .
\end{align*}
比较上述等式两边矩阵的每个元素可得$a_{ij}=-a_{ij}\left( i\ne j \right) $,从而$a_{ij}=0\left( i\ne j \right) $.于是\(A=\text{diag}\{a_{11},a_{22},\cdots,a_{nn}\}\)为对角阵.

设\(P_{ij}(1\leq i\neq j\leq n)\)为第一类初等阵,因为第一类初等阵均为正交阵,所以由条件可知\(AP_{ij}=P_{ij}A\).进而可得
\begin{align*}
\bordermatrix{%
&    &		&		i&		&		j&		&		\cr
&    a_{11}&		&		&		&		&		&		\cr
&    &		\ddots&		&		&		&		&		\cr
i&    &		&		0&		\cdots&		a_{jj}&		&		\cr
&   &		&		\vdots&		\ddots&		\vdots&		&		\cr
j&    &		&		a_{ii}&		\cdots&		0&		&		\cr
&    &		&		&		&		&		\ddots&		\cr
&    &		&		&		&		&		&		a_{nn}\cr
} \quad =\quad \bordermatrix{%
&    &		&		i&		&		j&		&		\cr
&    a_{11}&		&		&		&		&		&		\cr
&    &		\ddots&		&		&		&		&		\cr
i&    &		&		0&		\cdots&		a_{ii}&		&		\cr
&    &		&		\vdots&		\ddots&		\vdots&		&		\cr
j&   &		&		a_{jj}&		\cdots&		0&		&		\cr
&    &		&		&		&		&		\ddots&		\cr
&    &		&		&		&		&		&		a_{nn}\cr
} .
\end{align*}
从而比较上述等式两边矩阵的每个元素可得\(a_{ii}=a_{jj}(1\leq i\neq j\leq n)\),于是\(A\)为纯量阵.

\item 可以由上面(1)(2)(3)中任意一个证明得到.注意如果此时用(3)的证明方法,那么我们可以先考虑$A$与第一类初等矩阵\(P_{i}(c)(c\ne 1,1\leq i\leq n)\)的乘法交换性.而不是像(3)中只能考虑\(P_{i}(-1)(1\leq i\leq n)\).
\end{enumerate}
\end{proof}

\begin{proposition}[零矩阵的充要条件]\label{proposition:零矩阵的充要条件}
\begin{enumerate}
\item \(m\times n\)实矩阵\(A = O\)的充要条件是适合条件\(AA' = O\)或\(\mathrm{tr}(AA')\geq0\),等号成立;

\item \(m\times n\)复矩阵\(A = O\)的充要条件是适合条件\(A\overline{A}'= O\)或\(\mathrm{tr}(A\overline{A}')\geq0\),等号成立.
\end{enumerate}
\end{proposition}
\begin{proof}
\begin{enumerate}
\item (1) 设\(A=(a_{ij})_{m\times n}\),则\(AA'\)的第\((i,i)\)元素等于零,即
\begin{align*}
a_{i1}^{2}+a_{i2}^{2}+\cdots +a_{in}^{2}=0,i=1,2,\cdots,m.
\end{align*}
又因为\(a_{ij}\)都是实数,所以必有\(a_{ij}=0,i=1,2,\cdots,m,j=1,2,\cdots,n\).故$A=O$.

(2)设\(A=(a_{ij})\)为\(m\times n\)实矩阵,则通过计算可得
\[
\mathrm{tr}(AA')=\sum_{i = 1}^{m}\sum_{j = 1}^{n}a_{ij}^2\geq0,
\]
等号成立当且仅当\(a_{ij}=0(1\leq i\leq m,1\leq j\leq n)\),即\(A = O\).

\item (1)设\(A=(a_{ij})_{m\times n}\),则\(A\overline{A'}\)的第\((i,i)\)元素等于零,即
\begin{align*}
\left| a_{i1} \right|^2+\left| a_{i2} \right|^2+\cdots +\left| a_{in} \right|^2=0,i=1,2,\cdots ,m.
\end{align*}
又因为\(a_{ij}\)都是复数,所以可设$a_{ij}=b_{ij}+\mathrm{i}c_{ij}$,其中$b_{ij},c_{ij}\in \mathbb{R},i=1,2,\cdots,m,j=1,2,\cdots,n$.于是
\begin{align*}
b_{i1}^{2}+c_{i1}^{2}+b_{i2}^{2}+c_{i2}^{2}+\cdots +b_{in}^{2}+c_{in}^{2}=0,i=1,2,\cdots ,m.
\end{align*}
再结合$b_{ij},c_{ij}\in \mathbb{R}$,可知\(b_{ij}=c_{ij}=0\).即\(a_{ij}=0,i=1,2,\cdots,m,j=1,2,\cdots,n\).故$A=O$.

(2)设\(A=(a_{ij})\)为\(m\times n\)复矩阵,则通过计算可得
\[
\mathrm{tr}(A\overline{A}')=\sum_{i = 1}^{m}\sum_{j = 1}^{n}|a_{ij}|^2\geq0,
\]
等号成立当且仅当\(a_{ij}=0(1\leq i\leq m,1\leq j\leq n)\),即\(A = O\).
\end{enumerate}



\end{proof}

\begin{proposition}[对称阵是零矩阵的充要条件]\label{proposition:对称阵是零矩阵的充要条件}
设\(\boldsymbol{A}\)为\(n\)阶对称阵,则\(\boldsymbol{A}\)是零矩阵的充要条件是对任意的\(n\)维列向量\(\boldsymbol{\alpha}\),有
\[
\boldsymbol{\alpha'} \boldsymbol{A} \boldsymbol{\alpha} = 0.
\]
\end{proposition}
\begin{proof}
只要证明充分性.设\(\boldsymbol{A}=(a_{ij})\),令\(\alpha = \boldsymbol{e}_{i}\),是第\(i\)个标准单位列向量.因为\(\boldsymbol{e}_{i}'\boldsymbol{A}\boldsymbol{e}_{i}\)是\(\boldsymbol{A}\)的第\((i, i)\)元素,故\(a_{ii}=0\).又令\(\boldsymbol{\alpha}=\boldsymbol{e}_{i}+\boldsymbol{e}_{j}(i \neq j)\),则
\[
0 = (\boldsymbol{e}_{i}+\boldsymbol{e}_{j})'\boldsymbol{A}(\boldsymbol{e}_{i}+\boldsymbol{e}_{j})=a_{ii}+a_{jj}+a_{ij}+a_{ji}.
\]
由于\(\boldsymbol{A}\)是对称阵,故\(a_{ij}=a_{ji}\),又上面已经证明\(a_{ii}=a_{jj}=0\),从而\(a_{ij}=0\),这就证明了\(\boldsymbol{A} = \boldsymbol{O}\).
\end{proof}

\begin{proposition}[反对称阵的刻画]\label{proposition:反对称阵的刻画}
设\(\boldsymbol{A}\)为\(n\)阶方阵,则\(\boldsymbol{A}\)是反称阵的充要条件是对任意的\(n\)维列向量\(\boldsymbol{\alpha}\),有
\[
\boldsymbol{\alpha'} \boldsymbol{A} \boldsymbol{\alpha} = 0.
\]
\end{proposition}
\begin{proof}
必要性($\Rightarrow $):若\(\boldsymbol{A}\)是反称阵,则对任意的\(n\)维列向量\(\boldsymbol{\alpha}\),有\((\boldsymbol{\alpha' A \alpha})' = - \boldsymbol{\alpha' A \alpha}\).而\(\boldsymbol{\alpha' A \alpha}\)是数,因此\((\boldsymbol{\alpha' A \alpha})' = \boldsymbol{\alpha' A \alpha}\).比较上面两个式子便有\(\boldsymbol{\alpha' A \alpha} = 0\).

充分性($\Leftarrow $):若上式对任意的\(n\)维列向量\(\boldsymbol{\alpha}\)成立,则由\(\boldsymbol{\alpha' A \alpha}\)是数,可知$\boldsymbol{\alpha }'\boldsymbol{A\alpha }=\left( \boldsymbol{\alpha }'\boldsymbol{A\alpha } \right) '=\boldsymbol{\alpha }'\boldsymbol{A}'\boldsymbol{\alpha }=0$,故\(\boldsymbol{\alpha'}(\boldsymbol{A} + \boldsymbol{A'})\boldsymbol{\alpha} = 0\).因为矩阵\(\boldsymbol{A} + \boldsymbol{A'}\)是对称阵,故由\hyperref[proposition:对称阵是零矩阵的充要条件]{对称阵是零矩阵的充要条件}可得\(\boldsymbol{A} + \boldsymbol{A'} = \boldsymbol{O}\),即\(\boldsymbol{A'} = - \boldsymbol{A}\),\(\boldsymbol{A}\)是反称阵.
\end{proof}

\begin{proposition}\label{proposition:任一阶方阵可表示为对称阵与反对称阵之和}
任一\(n\)阶方阵均可表示为一个对称阵与一个反对称阵之和.
\end{proposition}
\begin{note}
构造思路:设$A=B+C$,且$B$为对称矩阵,$C$为反称矩阵.则两边取转置可得
\begin{align*}
\begin{cases}
A=B+C\\
A'=\left( B+C \right)'=B-C\\
\end{cases}
\end{align*}
解得:$B=\frac{1}{2}(\boldsymbol{A}+\boldsymbol{A}'),C=\frac{1}{2}(\boldsymbol{A}-\boldsymbol{A}')$.
\end{note}
\begin{proof}
设\(\boldsymbol{A}\)是\(n\)阶方阵,则\(\boldsymbol{A}+\boldsymbol{A}'\)是对称阵,\(\boldsymbol{A}-\boldsymbol{A}'\)是反对称阵,并且
\[
\boldsymbol{A}=\frac{1}{2}(\boldsymbol{A}+\boldsymbol{A}')+\frac{1}{2}(\boldsymbol{A}-\boldsymbol{A}').
\]
\end{proof}
\begin{remark}
上例中的\(\frac{1}{2}(\boldsymbol{A}+\boldsymbol{A}')\)称为\(\boldsymbol{A}\)的对称化,\(\frac{1}{2}(\boldsymbol{A}-\boldsymbol{A}')\)称为\(\boldsymbol{A}\)的反对称化.
\end{remark}

\begin{proposition}[上三角阵性质]\label{proposition:上三角阵性质}
(1) 设\(\boldsymbol{A}\)是\(n\)阶上三角阵且主对角线上元素全为零,则\(\boldsymbol{A}^{n}=O\).

\hypertarget{proposition:上三角阵的性质第2条性质}{(2)}
设\(\boldsymbol{A}\)是\(n(n\geq2)\)阶上三角阵,若\(i < j\),则\(A_{ij}=M_{ij}=0\).

(3)上(下)三角阵的加减、数乘、乘积(幂)、多项式、伴随和求逆仍然是上(下)三角阵,并且所得上(下)三角阵的主对角元是原上(下)三角阵对应主对角元的加减、数乘、乘积(幂)、多项式、伴随和求逆.
\end{proposition}
\begin{note}

\end{note}
\begin{proof}
(1) {\color{blue}证法一(抽屉原理):}
设\(A=(a_{ij})\),当\(i\geq j\)时,\(a_{ij} = 0\).将\(A\)表示为基础矩阵\(E_{ij}\)之和:
\[
A=\sum_{i>j}a_{ij}E_{ij}
\]
因为当\(j\neq k\)时,\(E_{ij}E_{kl}=\boldsymbol{O}\),故在\(A^{n}\)的乘法展开式中,可能非零的项只能是具有形式\(E_{i_{1}j_{1}}E_{i_{2}j_{2}}\cdots E_{i_{n - 1}j_{n - 1}}\),但足标必须满足条件\(1\leq i_{1}<j_{1}<i_{2}<j_{2}<\cdots <j_{n - 1}\leq n\).根据\hyperlink{抽屉原理}可知,这样的项也不存在,因此\(A^{n}=\boldsymbol{O}\).

{\color{blue}证法二(数学归纳法):} 
由假设\(Ae_{i}=a_{i1}e_{1}+\cdots +a_{i, i - 1}e_{i - 1}(1\leq i\leq n)\),我们只要用归纳法证明:\(A^{k}e_{k}=0\)对任意的\(1\leq k\leq n\)都成立,则\(A^{n}e_{i}=A^{n-i}\cdot A^ie_i=A^{n-i}\cdot 0=0\)对任意的\(1\leq i\leq n\)都成立,从而由\hyperlink{proposition:矩阵相等的判定法则}{判定法则}可知\(A^{n}=O\)成立.显然,\(Ae_{1}=0\)成立.假设\(A^{k}e_{k}=0\)对任意的\(1 < k < n\)都成立,则
\[
A^{k}e_{k}=A^{k - 1}(Ae_{k})=A^{k - 1}(a_{k1}e_{1}+\cdots +a_{k, k - 1}e_{k - 1})
\]
\[
=a_{k1}A^{k - 1}e_{1}+\cdots +a_{k, k - 1}A^{k - 1}e_{k - 1}=0.
\]

(2) 根据条件可设$\left| \boldsymbol{A} \right|=\left| \begin{matrix}
a_{11}&		a_{12}&		\cdots&		a_{1n}\\
0&		a_{22}&		\cdots&		a_{2n}\\
\vdots&		\vdots&		&		\vdots\\
0&		0&		\cdots&		a_{nn}\\
\end{matrix} \right|$,则当$i<j$时,有
\begin{align*}
& M_{ij}=\left| \begin{matrix}
a_{11}&		a_{12}&		\cdots&		a_{1i}&		a_{1,i+1}&		\cdots&		\MyTikzmark{topA}{a_{1j}}&		\cdots&		a_{1n}\\
0&		a_{22}&		\cdots&		a_{2i}&		a_{2,i+1}&		\cdots&		a_{2j}&		\cdots&		a_{2n}\\
\vdots&		\vdots&		&		\vdots&		\vdots&		&		\vdots&		&		\vdots\\
\MyTikzmark{leftA}{0}&		0&		\cdots&		a_{ii}&		a_{i,i+1}&		\cdots&		a_{ij}&		\cdots&		\MyTikzmark{rightA}{a_{in}}\\
0&		0&		\cdots&		0&		a_{i+1,i+1}&		\cdots&		a_{i+1,j}&		\cdots&		a_{i+1,n}\\
\vdots&		\vdots&		&		\vdots&		\vdots&		&		\vdots&		&		\vdots\\
\vdots&		\vdots&		&		\vdots&		\vdots&		&		\vdots&		&		\vdots\\
0&		0&		\cdots&		0&		0&		\cdots&		\MyTikzmark{bottomA}{0}&		\cdots&		a_{nn}\\
\end{matrix} \right|
\text{\DrawVLine[red, thick, opacity=0.5]{topA}{bottomA}
\DrawHLine[red, thick, opacity=0.5]{leftA}{rightA}}
\\
&=\left| \begin{matrix}
a_{11}&		a_{12}&		\cdots&		a_{1i}&		a_{1,i+1}&		\cdots&		a_{1n}\\
0&		a_{22}&		\cdots&		a_{2i}&		a_{2,i+1}&		\cdots&		a_{2n}\\
\vdots&		\vdots&		&		\vdots&		\vdots&		&		\vdots\\
0&		0&		\cdots&		0&		a_{i+1,i+1}&		\cdots&		a_{i+1,n}\\
\vdots&		\vdots&		&		\vdots&		\vdots&		&		\vdots\\
\vdots&		\vdots&		&		\vdots&		\vdots&		&		\vdots\\
0&		0&		\cdots&		0&		0&		\cdots&		a_{nn}\\
\end{matrix} \right|=\left| \begin{matrix}
a_{11}&		a_{12}&		\cdots&		a_{1i}&		\cdots&		a_{1n}\\
&		a_{22}&		\cdots&		a_{2i}&		\cdots&		a_{2n}\\
&		&		\ddots&		\vdots&		&		\vdots\\
&		&		&		0&		\cdots&		a_{i+1,n}\\
&		&		&		&		\ddots&		\vdots\\
&		&		&		&		&		a_{nn}\\
\end{matrix} \right|=0.
\end{align*}
故\(A_{ij}=M_{ij}=0\).

(3)只证上三角阵的情形,下三角阵的情形完全类似.上三角阵的加减、数乘、乘积(幂)以及多项式结论的证明是显然的.下面我们来证明伴随和求逆的结论.设$\boldsymbol{A}=(a_{ij})$为$n$阶上三角阵,即满足$a_{ij}=0,(\forall i>j)$.由\hyperlink{proposition:上三角阵的性质第2条性质}{(2)}可知$\boldsymbol{A}$的代数余子式$A_{ij}=0,\forall i<j$.于是
\begin{align*}
\boldsymbol{A}^*=\left( \begin{matrix}
A_{11}&		A_{21}&		\cdots&		A_{n1}\\
0&		A_{22}&		\cdots&		A_{n2}\\
\vdots&		\vdots&		\ddots&		\vdots\\
0&		0&		\cdots&		A_{nn}\\
\end{matrix} \right) .
\end{align*}
故$\boldsymbol{A}^*$也是上三角阵.而对$\forall i\in [1,n]\cap N$,有

我们又\textbf{将$A_ii=a_{11}\cdots \widehat{a_{ii}}\cdots a_{nn}$这个数称为$a_ii$的伴随}.这就完成了$\boldsymbol{A}^*$结论的证明.
\begin{align*}
&A_{ii}=\left( -1 \right) ^{2i}M_{ii}=M_{ii}=\left| \begin{matrix}
a_{11}&		a_{12}&		\cdots&		a_{1,i-1}&		\MyTikzmark{topB}{a_{1i}}&		a_{1,i+1}&		\cdots&		a_{1n}\\
0&		a_{22}&		\cdots&		a_{2,i-1}&		a_{2i}&		a_{2,i+1}&		\cdots&		a_{2n}\\
\vdots&		\vdots&		\ddots&		\vdots&		\vdots&		\vdots&		&		\vdots\\
0&		0&		\cdots&		a_{i-1,i-1}&		a_{i-1,i}&		a_{i-1,i+1}&		\cdots&		a_{i-1,n}\\
\MyTikzmark{leftB}{0}&		0&		\cdots&		0&		a_{ii}&		a_{i,i+1}&		\cdots&		\MyTikzmark{rightB}{a_{in}}\\
0&		0&		\cdots&		0&		0&		a_{i+1,i+1}&		\cdots&		a_{i+1,n}\\
\vdots&		\vdots&		&		\vdots&		\vdots&		\vdots&		\ddots&		\vdots\\
0&		0&		\cdots&		0&		\MyTikzmark{bottomB}{0}&		0&		\cdots&		a_{nn}\\
\end{matrix} \right|\
\intertext{\DrawVLine[red, thick, opacity=0.5]{topB}{bottomB}
\DrawHLine[red, thick, opacity=0.5]{leftB}{rightB}}
\\
&=\left| \begin{matrix}
a_{11}&		a_{12}&		\cdots&		a_{1,i-1}&		a_{1,i+1}&		\cdots&		a_{1n}\\
0&		a_{22}&		\cdots&		a_{2,i-1}&		a_{2,i+1}&		\cdots&		a_{2n}\\
\vdots&		\vdots&		\ddots&		\vdots&		\vdots&		&		\vdots\\
0&		0&		\cdots&		a_{i-1,i-1}&		a_{i-1,i+1}&		\cdots&		a_{i-1,n}\\
0&		0&		\cdots&		0&		a_{i+1,i+1}&		\cdots&		a_{i+1,n}\\
\vdots&		\vdots&		&		\vdots&		\vdots&		\ddots&		\vdots\\
0&		0&		\cdots&		0&		0&		\cdots&		a_{nn}\\
\end{matrix} \right|=a_{11}\cdots \widehat{a_{ii}}\cdots a_{nn}.
\end{align*}

由于当$\left| \boldsymbol{A} \right|\ne 0$时,我们有$\boldsymbol{A}^{-1}=\frac{1}{\left| \boldsymbol{A} \right|}\boldsymbol{A}^*$,故由上三角阵的数乘结论可知,$\boldsymbol{A}^{-1}$也是上三角阵,其主对角元为$\frac{1}{\left| \boldsymbol{A} \right|}A_{ii}=a_{ii}^{-1}$.结论得证.
\end{proof}

\begin{proposition}
若\(\boldsymbol{A}\),\(\boldsymbol{B}\)都是由非负实数组成的矩阵且\(AB\)有一行等于零,则或者\(\boldsymbol{A}\)有一行为零,或者\(\boldsymbol{B}\)有一行为零.
\end{proposition}
\begin{proof}
设\(\boldsymbol{A} = (a_{ij})_{n\times m}\),\(\boldsymbol{B} = (b_{ij})_{m\times s}\).假设\(\boldsymbol{C} = \boldsymbol{AB}\),\(\boldsymbol{C} = (c_{ij})_{n\times s}\)的第\(i\)行全为零.
则对\(\forall j\in [1,s]\cap N\),都有
\begin{align*}
c_{ij} = a_{i1}b_{1j} + a_{i2}b_{2j} + \cdots + a_{im}b_{nj} = 0 .
\end{align*}
已知对\(\forall i\in [1,n]\cap N\),\(j\in [1,m]\cap N\),有\(a_{ij} \geq 0\);对\(\forall i\in [1,m]\cap N\),\(j\in [1,s]\cap N\),有\(b_{ij} \geq 0\).从而
\begin{align*}
a_{i1}b_{1j} = a_{i2}b_{2j} = \cdots = a_{im}b_{nj} = 0,\forall j\in [1,s]\cap N.
\end{align*}
若\(\boldsymbol{A}\)的第\(i\)行不全为零,不妨设\(a_{ik} \neq 0\),\(k\in [1,m]\cap N\),则由\(a_{ik}b_{kj} = 0\),\(\forall j\in [1,s]\cap N\)可得\(b_{kj} = 0\),对\(\forall j\in [1,s]\cap N\)都成立,即\(\boldsymbol{B}\)的第\(k\)行全为零. 
\end{proof}

\begin{proposition}[对矩阵行和和列和的一种刻画]\label{proposition:对矩阵行和和列和的一种刻画}
\begin{enumerate}[(1)]
\item $n$阶矩阵$A$第$i$行元素之和为$a_i(i=1,2,\cdots,n)$当且仅当
\begin{align*}
A\left( \begin{array}{c}
1\\
1\\
\vdots\\
1\\
\end{array} \right) =\left( \begin{array}{c}
a_1\\
a_2\\
\vdots\\
a_n\\
\end{array} \right)
.
\end{align*}
特别地,$n$阶矩阵$A$的每一行元素之和等于$c$当且仅当$A\alpha=c\cdot\alpha$,其中$\alpha=(1,1,\cdots,1)'$.
\item $n$阶矩阵$A$第$i$列元素之和为$a_i(i=1,2,\cdots,n)$当且仅当
\begin{align*}
\left( \begin{matrix}
1&		1&		\cdots&		1\\
\end{matrix} \right) A=\left( \begin{matrix}
a_1&		a_2&		\cdots&		a_n\\
\end{matrix} \right). 
\end{align*}
特别地,$n$阶矩阵$A$的每一列元素之和等于$c$当且仅当$\alpha A=c\cdot\alpha$,其中$\alpha=(1,1,\cdots,1)$.
\end{enumerate}
\end{proposition}
\begin{proof}
由矩阵乘法容易得到证明.
\end{proof}

\begin{example}
设\(n\)阶方阵\(\boldsymbol{A}\)的每一行元素之和等于常数\(c\),求证:

(1) 对任意的正整数\(k\),\(\boldsymbol{A}^{k}\)的每一行元素之和等于\(c^{k}\);

(2) 若\(\boldsymbol{A}\)为可逆阵,则\(c \neq 0\)并且\(\boldsymbol{A}^{-1}\)的每一行元素之和等于\(c^{-1}\).
\end{example}
\begin{note}
核心想法是利用\hyperref[proposition:对矩阵行和和列和的一种刻画]{命题\ref{proposition:对矩阵行和和列和的一种刻画}}.
\end{note}
\begin{proof}
设\(\boldsymbol{\alpha} = (1,1,\cdots,1)'\),则由矩阵乘法可知,\(\boldsymbol{A}\)的每一行元素之和等于\(c\)当且仅当\(\boldsymbol{A}\boldsymbol{\alpha} = c \cdot \boldsymbol{\alpha}\)成立.

(1) 由\(\boldsymbol{A}\boldsymbol{\alpha} = c \cdot \boldsymbol{\alpha}\)不断递推可得\(\boldsymbol{A}^{k}\boldsymbol{\alpha} = c^{k} \cdot \boldsymbol{\alpha}\),故结论成立.

(2) 若\(c = 0\),则由\(\boldsymbol{A}\)可逆以及\(\boldsymbol{A}\boldsymbol{\alpha} = \boldsymbol{0}\)可得\(\boldsymbol{\alpha} = \boldsymbol{0}\),矛盾.在\(\boldsymbol{A}\boldsymbol{\alpha} = c \cdot \boldsymbol{\alpha}\)的两边同时左乘\(c^{-1}\boldsymbol{A}^{-1}\),可得\(\boldsymbol{A}^{-1}\boldsymbol{\alpha} = c^{-1} \cdot \boldsymbol{\alpha}\),由此即得结论.
\end{proof}

\begin{proposition}[矩阵可逆的等价命题]\label{proposition:矩阵可逆的等价命题}
(1)\(n\)阶方阵\(\boldsymbol{A}\)可逆.

(2)存在矩阵$\boldsymbol{B}$,使得$\boldsymbol{AB}=\boldsymbol{BA}=\boldsymbol{I}_n$(这个等式同时也说明$\boldsymbol{B}$可逆).

(3)\(\boldsymbol{A}\)的行列式\(\vert \boldsymbol{A}\vert\neq0\).

(4)\(\boldsymbol{A}\)等价(相抵)于\(n\)阶单位矩阵.

(5)\(\boldsymbol{A}\)可以表示为有限个初等矩阵的积.

(6)\(\boldsymbol{A}\)的\(n\)个行向量(列向量)线性无关.
\end{proposition}

\begin{proposition}\label{proposition:由矩阵的零化多项式构造其逆矩阵}
(1)若已知\(\lambda_1\boldsymbol{A}^2+\lambda_2\boldsymbol{A}+\lambda_3\boldsymbol{I}_n=\boldsymbol{O}\),其中\(\lambda_1,\lambda_2,\lambda_3\in\mathbb{R}\),$\lambda_1\ne 0$,并且$\lambda_1x^2+\lambda_2x+\lambda_3=0$无实根(即原等式左边不可因式分解成$\left( a_1\boldsymbol{I}_n+a_2\boldsymbol{A} \right) \left( b_1\boldsymbol{I}_n+b_2\boldsymbol{A} \right)$),
则对任何$c,d\in\mathbb{R}$,都有\(c\boldsymbol{A}+d\boldsymbol{I}_n\)可逆.

(2)若已知\(\lambda_1\boldsymbol{A}^2+\lambda_2\boldsymbol{A}+\lambda_3\boldsymbol{I}_n=(a_1\boldsymbol{A}+b_1\boldsymbol{I}_n)(a_2\boldsymbol{A}+b_2\boldsymbol{I}_n)=\boldsymbol{O}\),其中\(a_1,a_2,b_1,b_2\in\mathbb{R}\),且\(\lambda_1=a_1a_2,\lambda_2=a_1b_2+a_2b_1,\lambda_3=b_1b_2,a_1\ne 0,a_2\ne 0\).
则对任何实数对\(\left( c,d \right) \ne \left( a_1,b_1 \right) ,\left( a_2,b_2 \right)\),都有\(c\boldsymbol{A}+d\boldsymbol{I}_n\)可逆.
\end{proposition}
\begin{note}
\textbf{构造逆矩阵的方法:}不妨设\(k(c\boldsymbol{A}+d\boldsymbol{I}_n)^{-1}=(p\boldsymbol{A}+q\boldsymbol{I}_n)\),其中\(k,p,q\)为待定系数.则
\begin{align*}
(c\boldsymbol{A}+d\boldsymbol{I}_n)\cdot k(c\boldsymbol{A}+d\boldsymbol{I}_n)^{-1}=(c\boldsymbol{A}+d\boldsymbol{I}_n)(p\boldsymbol{A}+q\boldsymbol{I}_n)=pc\boldsymbol{A}^2+(cq + dp)\boldsymbol{A}+dq\boldsymbol{I}_n=k\boldsymbol{I}_n.
\end{align*}
令\(pc=\lambda_1,cq + dp=\lambda_2\),则\(p=\frac{\lambda_1}{c},q=\frac{\lambda_2}{c}-\frac{\lambda_1d}{c^2}\).于是由已知条件可得
\begin{align*}
(c\boldsymbol{A}+d\boldsymbol{I}_n)(p\boldsymbol{A}+q\boldsymbol{I}_n)=(c\boldsymbol{A}+d\boldsymbol{I}_n)\left(\frac{\lambda_1}{c}\boldsymbol{A}+\left(\frac{\lambda_2}{c}-\frac{\lambda_1d}{c^2}\right)\boldsymbol{I}_n\right)=\lambda_1\boldsymbol{A}^2+\lambda_2\boldsymbol{A}+d\left(\frac{\lambda_2}{c}-\frac{\lambda_1d}{c^2}\right)\boldsymbol{I}_n=\left(\frac{\lambda_2d}{c}-\frac{\lambda_1d^2}{c^2}-\lambda_3\right)\boldsymbol{I}_n.
\end{align*}
从而\(k=\frac{\lambda_2d}{c}-\frac{\lambda_1d^2}{c^2}-\lambda_3\).因此\((c\boldsymbol{A}+d\boldsymbol{I}_n)^{-1}=\frac{1}{k}(p\boldsymbol{A}+q\boldsymbol{I}_n)=\frac{1}{\frac{\lambda_2d}{c}-\frac{\lambda_1d^2}{c^2}-\lambda_3}\left(\frac{\lambda_1}{c}\boldsymbol{A}+\left(\frac{\lambda_2}{c}-\frac{\lambda_1d}{c^2}\right)\boldsymbol{I}_n\right)\).

{\color{blue}实际做题中只需要先设\(k(c\boldsymbol{A}+d\boldsymbol{I}_n)^{-1}=(p\boldsymbol{A}+q\boldsymbol{I}_n)\),其中\(k,p,q\)为待定系数.则有\((c\boldsymbol{A}+d\boldsymbol{I}_n)(p\boldsymbol{A}+q\boldsymbol{I}_n)=k\boldsymbol{I}_n\).然后通过比较二次项和一次项的系数得到方程组\(\begin{cases}
pc=\lambda_1\\
cq + dp=\lambda_2\\
\end{cases}\)(即要凑出合适的p,q,使得$(c\boldsymbol{A}+d\boldsymbol{I}_n)(p\boldsymbol{A}+q\boldsymbol{I}_n)$与$\lambda_1\boldsymbol{A}^2+\lambda_2\boldsymbol{A}+\lambda_3\boldsymbol{I}_n$的二次项和一次项的系数相等),解出\(p,q\)的值.最后将已知条件\(\lambda_1\boldsymbol{A}^2+\lambda_2\boldsymbol{A}+\lambda_3\boldsymbol{I}_n=\boldsymbol{O}\)代入\((c\boldsymbol{A}+d\boldsymbol{I}_n)(p\boldsymbol{A}+q\boldsymbol{I}_n)=k\boldsymbol{I}_n\)即可得到\(k\)的值.}

熟悉这种方式之后就能快速构造出我们需要的逆矩阵.
\end{note}
\begin{proof}
(1)和(2)的证明相同.如下(这里我们是利用了上述构造逆矩阵的方法直接构造出逆矩阵,再根据逆矩阵的定义直接得到证明):

当\(c = 0\)时,\(c\boldsymbol{A}+d\boldsymbol{I}_n=d\boldsymbol{I}_n\)显然可逆.

当\(c\neq 0\)时,注意到\((c\boldsymbol{A}+d\boldsymbol{I}_n)\left(\frac{\lambda_1}{c}\boldsymbol{A}+\left(\frac{\lambda_2}{c}-\frac{\lambda_1d}{c^2}\right)\boldsymbol{I}_n\right)=\left(\frac{\lambda_2d}{c}-\frac{\lambda_1d^2}{c^2}-\lambda_3\right)\boldsymbol{I}_n\),故\(c\boldsymbol{A}+d\boldsymbol{I}_n\)可逆.
\end{proof}

\begin{example}
设\(n\)阶方阵\(\boldsymbol{A}\)适合等式\(\boldsymbol{A}^{2}-3\boldsymbol{A}+2\boldsymbol{I}_{n}=\boldsymbol{O}\),求证:\(\boldsymbol{A}\)和\(\boldsymbol{A}+\boldsymbol{I}_{n}\)都是可逆阵,而若\(\boldsymbol{A}\neq\boldsymbol{I}_{n}\),则\(\boldsymbol{A}-2\boldsymbol{I}_{n}\)必不是可逆阵.
\end{example}
\begin{note}
这里构造逆矩阵利用了\hyperref[proposition:由矩阵的零化多项式构造其逆矩阵]{命题\ref{proposition:由矩阵的零化多项式构造其逆矩阵}}.
\end{note}
\begin{proof}
由已知得\(\boldsymbol{A}(\boldsymbol{A}-3\boldsymbol{I}_{n})=-2\boldsymbol{I}_{n}\),因此\(\boldsymbol{A}\)是可逆阵.又\(\boldsymbol{A}^{2}-3\boldsymbol{A}-4\boldsymbol{I}_{n}=-6\boldsymbol{I}_{n}\),于是\((\boldsymbol{A}+\boldsymbol{I}_{n})(\boldsymbol{A}-4\boldsymbol{I}_{n})=-6\boldsymbol{I}_{n}\),故\(\boldsymbol{A}+\boldsymbol{I}_{n}\)也是可逆阵.

另一方面,由已知等式可得\((\boldsymbol{A}-\boldsymbol{I}_{n})(\boldsymbol{A}-2\boldsymbol{I}_{n})=\boldsymbol{O}\),如果\(\boldsymbol{A}-2\boldsymbol{I}_{n}\)可逆,则\(\boldsymbol{A}-\boldsymbol{I}_{n}=\boldsymbol{O}\),\(\boldsymbol{A}=\boldsymbol{I}_{n}\)和假设不合,因此\(\boldsymbol{A}-2\boldsymbol{I}_{n}\)不是可逆阵.
\end{proof}

\begin{proposition}\label{proposition:由矩阵等式构造逆矩阵}
(1)若已知\(\lambda _1\boldsymbol{AB}+\lambda _2\boldsymbol{A}+\lambda _3\boldsymbol{B}+\lambda _4\boldsymbol{I}_n=\boldsymbol{O}\),其中\(\lambda_1,\lambda_2,\lambda_3,\lambda_4\in\mathbb{R}\),$\lambda_1\ne 0$,并且$\lambda_1x^2+(\lambda_2+\lambda_3)x+\lambda_4=0$无实根(即原等式左边不可因式分解成$\left( a_1\boldsymbol{I}_n+a_2\boldsymbol{A} \right) \left( b_1\boldsymbol{I}_n+b_2\boldsymbol{B} \right)$),
则对任何$c,d\in\mathbb{R}$,都有\(a\boldsymbol{I}_n+b\boldsymbol{A},c\boldsymbol{I}_n+d\boldsymbol{B}\)可逆.

(2)若已知\(\lambda _1\boldsymbol{AB}+\lambda _2\boldsymbol{A}+\lambda _3\boldsymbol{B}+\lambda _4\boldsymbol{I}_n=\left( a_1\boldsymbol{I}_n+b_1\boldsymbol{A} \right) \left( a_2\boldsymbol{I}_n+b_2\boldsymbol{B} \right) =\boldsymbol{O}\),其中\(a_1,a_2,b_1,b_2\in\mathbb{R}\),且\(\lambda _1=b_1b_2,\lambda _2=a_2b_1,\lambda _3=a_1b_2,\lambda _4=a_1a_2,b_1\ne 0,b_2\ne 0\).
则对任何实数对\(\left( a,b \right),\left( c,d \right) \ne \left( a_1,b_1 \right) ,\left( a_2,b_2 \right)\),都有\(a\boldsymbol{I}_n+b\boldsymbol{A},c\boldsymbol{I}_n+d\boldsymbol{B}\)可逆.
\end{proposition}
\begin{proof}
证明方法与\hyperref[proposition:由矩阵的零化多项式构造其逆矩阵]{命题\ref{proposition:由矩阵的零化多项式构造其逆矩阵}}类似,构造逆矩阵的方法也与其类似.这里不再赘述.
\end{proof}

\begin{example}
\begin{enumerate}
\item 求证:不存在\(n\)阶奇异矩阵\(A\),适合条件\(A^2 + A+I_n = O\).
\item 设\(A\)是\(n\)阶矩阵,且\(A^2 = A\),求证:\(I_n - 2A\)是可逆矩阵.
\item \label{example:item3}若\(A\)是\(n\)阶矩阵,且\(2A(A - I_n)=A^3\),求证:\(I_n - A\)可逆.
\end{enumerate}
\end{example}
\begin{note}
这类问题构造逆矩阵的方法(以\ref{example:item3}为例):已知条件等价于$A^3-2A^2+2A=O$,设$(I_n-A)^{-1}=aA^2+bA+cI_n$,其中$a,b,c$为待定系数,使得\begin{align*}
\left( I_n-A \right) \left( aA^2+bA+cI_n \right) =A^3-2A^2+2A+kI_n=kI_n,k\text{为待定常数}.
\end{align*}
比较等式两边系数可得
\begin{align*}
\begin{cases}
-a=1\\
a-b=-2\\
b-c=2\\
c=k\\
\end{cases}\Rightarrow \begin{cases}
a=-1\\
b=1\\
k=c=-1\\
\end{cases}
\end{align*}
于是$\left( I_n-A \right) \left( -A^2+A-I_n \right) =-I_n.$从而$\left( I_n-A \right) ^{-1}=A^2-A+I_n.$
\end{note}
\begin{proof}
\begin{enumerate}
\item 由已知\(A^2 + A+I_n = O\),则\((A - I_n)(A^2 + A+I_n)=A^3 - I_n = O\),即\(A^3 = I_n\),于是\(A\)是可逆矩阵.
\item 因为\((I_n - 2A)^2 = I_n - 4A + 4A^2 = I_n\),故\(I_n - 2A\)是可逆矩阵.
\item 由已知\(A^3 - 2A^2 + 2A - I_n=-I_n\),即\((A - I_n)(A^2 - A + I_n)=-I_n\),于是\((I_n - A)^{-1}=A^2 - A + I_n\).
\end{enumerate}
\end{proof}

\begin{example}
设\(n\)阶方阵\(\boldsymbol{A}\)和\(\boldsymbol{B}\)满足\(\boldsymbol{A}+\boldsymbol{B}=\boldsymbol{A}\boldsymbol{B}\),求证:\(\boldsymbol{I}_{n}-\boldsymbol{A}\)是可逆阵且\(\boldsymbol{A}\boldsymbol{B}=\boldsymbol{B}\boldsymbol{A}\).
\end{example}
\begin{proof}
因为
\[
(\boldsymbol{I}_{n}-\boldsymbol{A})(\boldsymbol{I}_{n}-\boldsymbol{B})=\boldsymbol{I}_{n}-\boldsymbol{A}-\boldsymbol{B}+\boldsymbol{A}\boldsymbol{B}=\boldsymbol{I}_{n},
\]
所以\(\boldsymbol{I}_{n}-\boldsymbol{A}\)是可逆阵.另一方面,由上式可得\((\boldsymbol{I}_{n}-\boldsymbol{A})^{-1}=(\boldsymbol{I}_{n}-\boldsymbol{B})\),故
\[
\boldsymbol{I}_{n}=(\boldsymbol{I}_{n}-\boldsymbol{B})(\boldsymbol{I}_{n}-\boldsymbol{A})=\boldsymbol{I}_{n}-\boldsymbol{B}-\boldsymbol{A}+\boldsymbol{B}\boldsymbol{A},
\]
从而\(\boldsymbol{B}\boldsymbol{A}=\boldsymbol{A}+\boldsymbol{B}=\boldsymbol{A}\boldsymbol{B}\).
\end{proof}

\begin{proposition}[矩阵转置的性质]\label{proposition:矩阵转置的性质}
设矩阵$A,B$,则有
\begin{enumerate}
\item $(A')'=A$;
\item $(A+B)'=A'+B'$;
\item $(kA)'=kA'$;
\item $(AB)'=B'A'$.
\end{enumerate}
\end{proposition}
\begin{proof}
由矩阵的性质易证.
\end{proof}

\begin{proposition}[矩阵的逆运算]\label{proposition:矩阵的逆运算}
设矩阵$\boldsymbol{A},\boldsymbol{B},\boldsymbol{C}$可逆,则有

常规逆运算:

\begin{enumerate}
\item $\left( \boldsymbol{AB} \right) ^{-1}=\boldsymbol{B}^{-1}\boldsymbol{A}^{-1}$.

\item $\left( \boldsymbol{AC}+\boldsymbol{BC} \right) ^{-1}=\boldsymbol{C}^{-1}\left( \boldsymbol{A}+\boldsymbol{B} \right) ^{-1}$.

\item $\left( \boldsymbol{A}+\boldsymbol{B} \right) ^{-1}\boldsymbol{C}=\left( \boldsymbol{C}^{-1}\boldsymbol{A}+\boldsymbol{C}^{-1}\boldsymbol{B} \right) ^{-1}$.

\item $\boldsymbol{C}\left( \boldsymbol{A}+\boldsymbol{B} \right) ^{-1}=\left( \boldsymbol{AC}^{-1}+\boldsymbol{BC}^{-1} \right) ^{-1}$.
\end{enumerate}
\noindent{\textbf{凑因子:}}
\begin{enumerate}
\item $\boldsymbol{A}=\left( \boldsymbol{AB}^{-1} \right) \boldsymbol{B}=\left( \boldsymbol{AB} \right) \boldsymbol{B}^{-1}=\boldsymbol{B}\left( \boldsymbol{B}^{-1}\boldsymbol{A} \right) =\boldsymbol{B}^{-1}\left( \boldsymbol{BA} \right)$ .

\item $\boldsymbol{A}+\boldsymbol{B}=\left( \boldsymbol{AC}^{-1}+\boldsymbol{BC}^{-1} \right) \boldsymbol{C}=\left( \boldsymbol{AC}+\boldsymbol{BC} \right) \boldsymbol{C}^{-1}=\boldsymbol{C}\left( \boldsymbol{C}^{-1}\boldsymbol{A}+\boldsymbol{C}^{-1}\boldsymbol{B} \right) =\boldsymbol{C}^{-1}\left( \boldsymbol{CA}+\boldsymbol{CB} \right)$ .
\end{enumerate}
\end{proposition}
\begin{note}
无需额外记忆这些公式,只需要知道\textbf{凑因子}的想法,即\textbf{在矩阵可逆的条件下,我们可以利用矩阵$\boldsymbol{I}_n=\boldsymbol{AA}^{-1}=\boldsymbol{A}^{-1}\boldsymbol{A}$的性质,将原本矩阵没有的因子凑出来,然后提取我们需要的矩阵因子到矩阵逆的外面或将其乘入矩阵逆的内部,从而达到化简原矩阵的目的.}
\end{note}
\begin{proof}
由矩阵的运算性质不难证明.
\end{proof}
\begin{remark}
\textbf{凑因子想法}的应用:\hyperref[example:凑因子例1]{例题\ref{example:凑因子例1}},\hyperref[example:凑因子例2]{例题\ref{example:凑因子例2}},\hyperref[Sherman-Morrison-Woodbury公式]{例题\ref{Sherman-Morrison-Woodbury公式}}.
\end{remark}

\begin{example}
设\(A,B,A - B\)都是\(n\)阶可逆阵,证明:
\[
B^{-1} - A^{-1} = (B + B(A - B)^{-1}B)^{-1}.
\]    
\end{example}
\begin{note}
直接运用逆矩阵的定义验证即可.
\end{note}
\begin{proof}
\begin{align*}
&\left( B^{-1}-A^{-1} \right) \left( B+B\left( A-B \right) ^{-1}B \right) 
\\
&=I_n+\left( A-B \right) ^{-1}B-A^{-1}B-A^{-1}B\left( A-B \right) -1B
\\
&=I_n+\left( A-B \right) ^{-1}B-A^{-1}B\left( I_n+\left( A-B \right) -1B \right) 
\\
&=\left( I_n-A^{-1}B \right) \left( I_n+\left( A-B \right) ^{-1}B \right) 
\\
&=A^{-1}\left( A-B \right) \left[ \left( A-B \right) ^{-1}\left( A-+B \right) \right] 
\\
&=A^{-1}\left( A-B \right) \left( A-B \right) ^{-1}A=I_n.
\end{align*}
\end{proof}

\begin{example}[$\,\,$Sherman-Morrison公式]\label{Sherman-Morrison公式}
设\(A\)是\(n\)阶可逆阵,\(\alpha,\beta\)是\(n\)维列向量,且\(1 + \beta'A^{-1}\alpha \neq 0\).求证:
\[
(A + \alpha\beta')^{-1} = A^{-1} - \frac{1}{1 + \beta'A^{-1}\alpha}A^{-1}\alpha\beta'A^{-1}.
\]
\end{example}
\begin{note}
直接运用逆矩阵的定义验证即可,注意$\beta'A^{-1}\alpha $是一个数可以提出来.
\end{note}
\begin{proof}
\begin{align*}
&\left( A+\alpha \beta ' \right) \left( A^{-1}-\frac{1}{1+\beta 'A^{-1}\alpha}A^{-1}\alpha \beta 'A^{-1} \right) 
\\
&=I_n-\frac{1}{1+\beta 'A^{-1}\alpha}\alpha \beta 'A^{-1}+\alpha \beta 'A^{-1}-\frac{1}{1+\beta 'A^{-1}\alpha}\alpha \left( \beta 'A^{-1}\alpha \right) \beta 'A^{-1}
\\
&=I_n+\alpha \beta 'A^{-1}-\frac{1}{1+\beta 'A^{-1}\alpha}\alpha \beta 'A^{-1}-\frac{\beta 'A^{-1}\alpha}{1+\beta 'A^{-1}\alpha}\alpha \beta 'A^{-1}
\\
&=I_n+\alpha \beta 'A^{-1}-\frac{1+\beta 'A^{-1}\alpha}{1+\beta 'A^{-1}\alpha}\alpha \beta 'A^{-1}=I_n.
\end{align*}
\end{proof}


\begin{proposition}[一些矩阵等式]\label{proposition:一些矩阵等式}
\begin{enumerate}
\item 设$\boldsymbol{A}$为$m\times n$矩阵,$\boldsymbol{B}$为$n\times m$矩阵.则有$\boldsymbol{A}\left( \boldsymbol{I}_{\boldsymbol{n}}+\boldsymbol{BA} \right) =\left( \boldsymbol{I}_{\boldsymbol{m}}+\boldsymbol{AB} \right) \boldsymbol{A}$.

\item 设$\boldsymbol{A},\boldsymbol{B}$均为$n$阶可逆矩阵,则有$\boldsymbol{A}+\boldsymbol{B}=\boldsymbol{A}(\boldsymbol{A}^{-1}+\boldsymbol{B}^{-1})\boldsymbol{B}$.

\item 若$n$阶矩阵$A,B$满足$A^2=B^2$,则$A(A+B)=A^2+AB=B^2+AB=(A+B)B$.
\end{enumerate}
\end{proposition}
\begin{note}
这是一些常见的矩阵等式.可以通过反复\hyperref[proposition:矩阵的逆运算]{凑因子}得到.
\end{note}
\begin{proof}
由矩阵的运算性质不难证明.
\end{proof}

\begin{example}\label{example:凑因子例1}
设\(\boldsymbol{A},\boldsymbol{B},\boldsymbol{AB}-\boldsymbol{I}_{n}\)都是\(n\)阶可逆阵,证明:\(\boldsymbol{A}-\boldsymbol{B}^{-1}\)与\((\boldsymbol{A}-\boldsymbol{B}^{-1})^{-1}-\boldsymbol{A}^{-1}\)均可逆,并求它们的逆矩阵.
\end{example}
\begin{note}
核心想法是利用\hyperref[proposition:矩阵的逆运算]{命题\ref{proposition:矩阵的逆运算}}和\hyperref[proposition:一些矩阵等式]{命题\ref{proposition:一些矩阵等式}}.
\end{note}
\begin{proof}
注意到\(\boldsymbol{A}-\boldsymbol{B}^{-1}=(\boldsymbol{AB}-\boldsymbol{I}_{n})\boldsymbol{B}^{-1}\),故\(\boldsymbol{A}-\boldsymbol{B}^{-1}\)是可逆矩阵,并且\((\boldsymbol{A}-\boldsymbol{B}^{-1})^{-1}=\boldsymbol{B}(\boldsymbol{AB}-\boldsymbol{I}_{n})^{-1}\).注意到如下变形:
\begin{align*}
&(\boldsymbol{A}-\boldsymbol{B}^{-1})^{-1}-\boldsymbol{A}^{-1}\\
=&\boldsymbol{B}(\boldsymbol{AB}-\boldsymbol{I}_{n})^{-1}-\boldsymbol{A}^{-1}=\boldsymbol{A}^{-1}(\boldsymbol{AB}(\boldsymbol{AB}-\boldsymbol{I}_{n})^{-1}-\boldsymbol{I}_{n})\\
=&\boldsymbol{A}^{-1}(\boldsymbol{AB}-(\boldsymbol{AB}-\boldsymbol{I}_{n}))(\boldsymbol{AB}-\boldsymbol{I}_{n})^{-1}=\boldsymbol{A}^{-1}(\boldsymbol{AB}-\boldsymbol{I}_{n})^{-1}.
\end{align*}
故\((\boldsymbol{A}-\boldsymbol{B}^{-1})^{-1}-\boldsymbol{A}^{-1}\)可逆,并且\((( \boldsymbol{A}-\boldsymbol{B}^{-1})^{-1}-\boldsymbol{A}^{-1})^{-1}=(\boldsymbol{AB}-\boldsymbol{I}_{n})\boldsymbol{A}\).
\end{proof}

\begin{proposition}\label{proposition:矩阵可逆的重要结论1}
设\(\boldsymbol{A}\)为\(m\times n\)矩阵,\(\boldsymbol{B}\)为\(n\times m\)矩阵,使得\(\boldsymbol{I}_{m}+\boldsymbol{AB}\)可逆,则\(\boldsymbol{I}_{n}+\boldsymbol{BA}\)也可逆,并且\((\boldsymbol{I}_{n}+\boldsymbol{BA})^{-1}=\boldsymbol{I}_{n}-\boldsymbol{B}(\boldsymbol{I}_{m}+\boldsymbol{AB})^{-1}\boldsymbol{A}\).
\end{proposition}
\begin{note}
\hyperref[proposition:矩阵可逆的重要结论1]{命题\ref{proposition:矩阵可逆的重要结论1}}的应用:一般对于求只含有两项的矩阵和式的逆矩阵,我们可以利用\hyperref[proposition:矩阵的逆运算]{矩阵的逆运算(凑因子)}的方法将原矩阵和式转化为$\boldsymbol{C}\left( \boldsymbol{I}_n+\boldsymbol{AB} \right)$或$\left( \boldsymbol{I}_n+\boldsymbol{AB} \right) \boldsymbol{C}$的形式,再利用\hyperref[proposition:矩阵可逆的重要结论1]{这个命题}求得原矩阵的逆.
\end{note}
\begin{remark}
证法一只能得到\(\boldsymbol{I}_{n}+\boldsymbol{BA}\)可逆,并不能得到具体的逆矩阵.而证法二可以求出\((\boldsymbol{I}_{n}+\boldsymbol{BA})^{-1}=\boldsymbol{I}_{n}-\boldsymbol{B}(\boldsymbol{I}_{m}+\boldsymbol{AB})^{-1}\boldsymbol{A}\).
\end{remark}
\begin{proof}
{\color{blue}证法一(\hyperref[proposition:打洞原理]{打洞原理}):}根据分块矩阵的初等变换可得\begin{align*}
\left| \begin{matrix}
\boldsymbol{I}_m&		-\boldsymbol{A}\\
\boldsymbol{B}&		\boldsymbol{I}_n\\
\end{matrix} \right|=\left| \begin{matrix}
\boldsymbol{I}_m&		\boldsymbol{O}\\
-\boldsymbol{B}&		\boldsymbol{I}_n\\
\end{matrix} \right|\left| \begin{matrix}
\boldsymbol{I}_m&		-\boldsymbol{A}\\
\boldsymbol{B}&		\boldsymbol{I}_n\\
\end{matrix} \right|=\left| \left( \begin{matrix}
\boldsymbol{I}_m&		\boldsymbol{O}\\
-\boldsymbol{B}&		\boldsymbol{I}_n\\
\end{matrix} \right) \left( \begin{matrix}
\boldsymbol{I}_m&		-\boldsymbol{A}\\
\boldsymbol{B}&		\boldsymbol{I}_n\\
\end{matrix} \right) \right|=\left| \begin{matrix}
\boldsymbol{I}_m&		-\boldsymbol{A}\\
\boldsymbol{O}&		\boldsymbol{I}_n+\boldsymbol{BA}\\
\end{matrix} \right|=\left| \boldsymbol{I}_n+\boldsymbol{BA} \right|.
\\
\left| \begin{matrix}
\boldsymbol{I}_m&		-\boldsymbol{A}\\
\boldsymbol{B}&		\boldsymbol{I}_n\\
\end{matrix} \right|=\left| \begin{matrix}
\boldsymbol{I}_m&		\boldsymbol{A}\\
\boldsymbol{O}&		\boldsymbol{I}_n\\
\end{matrix} \right|\left| \begin{matrix}
\boldsymbol{I}_m&		-\boldsymbol{A}\\
\boldsymbol{B}&		\boldsymbol{I}_n\\
\end{matrix} \right|=\left| \left( \begin{matrix}
\boldsymbol{I}_m&		\boldsymbol{A}\\
\boldsymbol{O}&		\boldsymbol{I}_n\\
\end{matrix} \right) \left( \begin{matrix}
\boldsymbol{I}_m&		-\boldsymbol{A}\\
\boldsymbol{B}&		\boldsymbol{I}_n\\
\end{matrix} \right) \right|=\left| \begin{matrix}
\boldsymbol{I}_m+\boldsymbol{AB}&		\boldsymbol{O}\\
\boldsymbol{B}&		\boldsymbol{I}_n\\
\end{matrix} \right|=\left| \boldsymbol{I}_m+\boldsymbol{AB} \right|.
\end{align*}
故$\left| \begin{matrix}
\boldsymbol{I}_m&		-\boldsymbol{A}\\
\boldsymbol{B}&		\boldsymbol{I}_n\\
\end{matrix} \right|=\left| \boldsymbol{I}_m+\boldsymbol{AB} \right|=\left| \boldsymbol{I}_n+\boldsymbol{BA} \right|$.又因为$\boldsymbol{I}_m+\boldsymbol{AB}$可逆,所以$\left| \boldsymbol{I}_n+\boldsymbol{BA} \right|=\left| \boldsymbol{I}_m+\boldsymbol{AB} \right|\ne0$.因此$\boldsymbol{I}_n+\boldsymbol{BA}$也可逆.

{\color{blue}证法二(\hyperref[proposition:矩阵的逆运算]{矩阵的逆运算}):}
注意到\(\boldsymbol{A}(\boldsymbol{I}_{n}+\boldsymbol{BA})=(\boldsymbol{I}_{m}+\boldsymbol{AB})\boldsymbol{A}\),故\((\boldsymbol{I}_{m}+\boldsymbol{AB})^{-1}\boldsymbol{A}(\boldsymbol{I}_{n}+\boldsymbol{BA})=\boldsymbol{A}\),
于是\(\boldsymbol{B}(\boldsymbol{I}_{m}+\boldsymbol{AB})^{-1}\boldsymbol{A}(\boldsymbol{I}_{n}+\boldsymbol{BA})=\boldsymbol{BA}\),从而
\begin{align*}
\boldsymbol{I}_{n}&=\boldsymbol{I}_{n}+\boldsymbol{BA}-\boldsymbol{BA}=(\boldsymbol{I}_{n}+\boldsymbol{BA})-\boldsymbol{B}(\boldsymbol{I}_{m}+\boldsymbol{AB})^{-1}\boldsymbol{A}(\boldsymbol{I}_{n}+\boldsymbol{BA})\\
&=(\boldsymbol{I}_{n}-\boldsymbol{B}(\boldsymbol{I}_{m}+\boldsymbol{AB})^{-1}\boldsymbol{A})(\boldsymbol{I}_{n}+\boldsymbol{BA}).
\end{align*}
于是\((\boldsymbol{I}_{n}+\boldsymbol{BA})^{-1}=\boldsymbol{I}_{n}-\boldsymbol{B}(\boldsymbol{I}_{m}+\boldsymbol{AB})^{-1}\boldsymbol{A}\).
\end{proof}

\begin{example}\label{example:凑因子例2}
设\(\boldsymbol{A},\boldsymbol{B}\)均为\(n\)阶可逆阵,使得\(\boldsymbol{A}^{-1}+\boldsymbol{B}^{-1}\)可逆,证明:\(\boldsymbol{A}+\boldsymbol{B}\)也可逆,并且
\[
(\boldsymbol{A}+\boldsymbol{B})^{-1}=\boldsymbol{A}^{-1}-\boldsymbol{A}^{-1}(\boldsymbol{A}^{-1}+\boldsymbol{B}^{-1})^{-1}\boldsymbol{A}^{-1}.
\]
\end{example}
\begin{proof}
注意到\(\boldsymbol{A}+\boldsymbol{B}=\boldsymbol{A}(\boldsymbol{A}^{-1}+\boldsymbol{B}^{-1})\boldsymbol{B}\),故\(\boldsymbol{A}+\boldsymbol{B}\)可逆.由\hyperref[proposition:矩阵可逆的重要结论1]{命题\ref{proposition:矩阵可逆的重要结论1}}可得
\[
(\boldsymbol{I}_{n}+\boldsymbol{A}^{-1}\boldsymbol{B})^{-1}=\boldsymbol{I}_{n}-\boldsymbol{A}^{-1}(\boldsymbol{I}_{n}+\boldsymbol{B}\boldsymbol{A}^{-1})^{-1}\boldsymbol{B}=\boldsymbol{I}_{n}-\boldsymbol{A}^{-1}(\boldsymbol{A}^{-1}+\boldsymbol{B}^{-1})^{-1},
\]
于是
\begin{align*}
(\boldsymbol{A}+\boldsymbol{B})^{-1}&=(\boldsymbol{A}(\boldsymbol{I}_{n}+\boldsymbol{A}^{-1}\boldsymbol{B}))^{-1}=(\boldsymbol{I}_{n}+\boldsymbol{A}^{-1}\boldsymbol{B})^{-1}\boldsymbol{A}^{-1}\\
&=\boldsymbol{A}^{-1}-\boldsymbol{A}^{-1}(\boldsymbol{A}^{-1}+\boldsymbol{B}^{-1})^{-1}\boldsymbol{A}^{-1}.\square
\end{align*}
\end{proof}

\begin{example}[$\,\,$Sherman-Morrison-Woodbury公式]\label{Sherman-Morrison-Woodbury公式}

设\(\boldsymbol{A}\)为\(n\)阶可逆阵,\(\boldsymbol{C}\)为\(m\)阶可逆阵,\(\boldsymbol{B}\)为\(n\times m\)矩阵,\(\boldsymbol{D}\)为\(m\times n\)矩阵,使得\(\boldsymbol{C}^{-1}+\boldsymbol{D}\boldsymbol{A}^{-1}\boldsymbol{B}\)可逆.求证:\(\boldsymbol{A}+\boldsymbol{B}\boldsymbol{C}\boldsymbol{D}\)也可逆,并且
\[
(\boldsymbol{A}+\boldsymbol{B}\boldsymbol{C}\boldsymbol{D})^{-1}=\boldsymbol{A}^{-1}-\boldsymbol{A}^{-1}\boldsymbol{B}(\boldsymbol{C}^{-1}+\boldsymbol{D}\boldsymbol{A}^{-1}\boldsymbol{B})^{-1}\boldsymbol{D}\boldsymbol{A}^{-1}.
\]
\end{example}
\begin{remark}
若已知矩阵逆的表达式,也可以采取利用矩阵逆的定义直接验证的方法进行证明.
\end{remark}
\begin{proof}
注意到\(\boldsymbol{A}+\boldsymbol{B}\boldsymbol{C}\boldsymbol{D}=\boldsymbol{A}(\boldsymbol{I}_{n}+\boldsymbol{A}^{-1}\boldsymbol{B}\boldsymbol{C}\boldsymbol{D})\),将\(\boldsymbol{A}^{-1}\boldsymbol{B}\)和\(\boldsymbol{C}\boldsymbol{D}\)分别看成整体,此时
\(\boldsymbol{I}_{m}+(\boldsymbol{C}\boldsymbol{D})(\boldsymbol{A}^{-1}\boldsymbol{B})=\boldsymbol{C}(\boldsymbol{C}^{-1}+\boldsymbol{D}\boldsymbol{A}^{-1}\boldsymbol{B})\)可逆,故由\hyperref[proposition:矩阵可逆的重要结论1]{命题\ref{proposition:矩阵可逆的重要结论1}}的结论可知\(\boldsymbol{I}_{n}+(\boldsymbol{A}^{-1}\boldsymbol{B})(\boldsymbol{C}\boldsymbol{D})\)也可逆,并且
\begin{align*}
(\boldsymbol{I}_{n}+\boldsymbol{A}^{-1}\boldsymbol{B}\boldsymbol{C}\boldsymbol{D})^{-1}&=\boldsymbol{I}_{n}-\boldsymbol{A}^{-1}\boldsymbol{B}(\boldsymbol{I}_{m}+\boldsymbol{C}\boldsymbol{D}\boldsymbol{A}^{-1}\boldsymbol{B})^{-1}\boldsymbol{C}\boldsymbol{D}\\
&=\boldsymbol{I}_{n}-\boldsymbol{A}^{-1}\boldsymbol{B}(\boldsymbol{C}^{-1}+\boldsymbol{D}\boldsymbol{A}^{-1}\boldsymbol{B})^{-1}\boldsymbol{D}.
\end{align*}
于是\(\boldsymbol{A}+\boldsymbol{B}\boldsymbol{C}\boldsymbol{D}=\boldsymbol{A}(\boldsymbol{I}_{n}+\boldsymbol{A}^{-1}\boldsymbol{B}\boldsymbol{C}\boldsymbol{D})\)也可逆,并且
\[
(\boldsymbol{A}+\boldsymbol{B}\boldsymbol{C}\boldsymbol{D})^{-1}=\boldsymbol{A}^{-1}-\boldsymbol{A}^{-1}\boldsymbol{B}(\boldsymbol{C}^{-1}+\boldsymbol{D}\boldsymbol{A}^{-1}\boldsymbol{B})^{-1}\boldsymbol{D}\boldsymbol{A}^{-1}.
\]
\end{proof}


\subsection{练习}


\begin{exercise}
计算下列矩阵的\(k\)次幂,其中\(k\)为正整数:

(1) \(A=\begin{pmatrix}
a & 1 & 0 \\
0 & a & 1 \\
0 & 0 & a
\end{pmatrix}\);\quad
(2) \(A=\begin{pmatrix}
1 & 2 & 4 \\
2 & 4 & 8 \\
3 & 6 & 12
\end{pmatrix}\).
\end{exercise}
\begin{note}
第(2)问核心想法是利用\hyperref[proposition:可以写成两个矩阵(向量)乘积的矩阵]{命题\ref{proposition:可以写成两个矩阵(向量)乘积的矩阵}}.
\end{note}
\begin{solution}
(1)设\(J=\begin{pmatrix}
0 & 1 & 0 \\
0 & 0 & 1 \\
0 & 0 & 0
\end{pmatrix}\),则\(A = aI_{3}+J\).注意到\(aI_{3}\)和\(J\)乘法可交换,$J$是幂零阵并且\(J^{3}=O\),因此我们可用二项式定理来求\(A\)的\(k\)次幂:
\begin{align*}
A^{k}&=(aI_{3}+J)^{k}=(aI_{3})^{k}+C_{k}^{1}(aI_{3})^{k - 1}J+C_{k}^{2}(aI_{3})^{k - 2}J^{2}\\
&=a^{k}I_{3}+C_{k}^{1}a^{k - 1}J+C_{k}^{2}a^{k - 2}J^{2}=\begin{pmatrix}
a^{k}&C_{k}^{1}a^{k - 1}&C_{k}^{2}a^{k - 2}\\
0&a^{k}&C_{k}^{1}a^{k - 1}\\
0&0&a^{k}
\end{pmatrix}
\end{align*}

(2)注意到\(A\)的列向量成比例,故可设\(\alpha=(1,2,3)\),\(\beta=(1,2,4)\),则\(A = \alpha\beta'\).
由矩阵乘法的结合律并注意到\(\beta\alpha' = 17\),可得
\begin{align*}
A^{k}&=(\alpha\beta')(\alpha\beta')\cdots(\alpha\beta')=\alpha(\beta'\alpha)(\beta'\alpha)\cdots(\beta'\alpha)\beta'\\
&=(\beta'\alpha)^{k - 1}\alpha\beta'=17^{k - 1}A=
\begin{pmatrix}
17^{k - 1}&2\cdot17^{k - 1}&4\cdot17^{k - 1}\\
2\cdot17^{k - 1}&4\cdot17^{k - 1}&8\cdot17^{k - 1}\\
3\cdot17^{k - 1}&6\cdot17^{k - 1}&12\cdot17^{k - 1}
\end{pmatrix}
\end{align*}
\end{solution}

\begin{exercise}
设\(k\)是正整数,计算\(\begin{pmatrix}
\cos\theta & \sin\theta\\
-\sin\theta & \cos\theta
\end{pmatrix}^k\).
\end{exercise}
\begin{solution}
已知$k=1$时,有\(\begin{pmatrix}
\cos\theta & \sin\theta\\
-\sin\theta & \cos\theta
\end{pmatrix}\).假设$k=n$时,有$\left( \begin{matrix}
\cos \theta&		\sin \theta\\
-\sin \theta&		\cos \theta\\
\end{matrix} \right) ^n=\left( \begin{matrix}
\cos n\theta&		\sin n\theta\\
-\sin n\theta&		\cos n\theta\\
\end{matrix} \right) $.则当$k=n+1$时,有
\begin{align*}
&\left( \begin{matrix}
\cos \theta&		\sin \theta\\
-\sin \theta&		\cos \theta\\
\end{matrix} \right) ^{n+1}=\left( \begin{matrix}
\cos \theta&		\sin \theta\\
-\sin \theta&		\cos \theta\\
\end{matrix} \right) ^n\left( \begin{matrix}
\cos \theta&		\sin \theta\\
-\sin \theta&		\cos \theta\\
\end{matrix} \right) =\left( \begin{matrix}
\cos n\theta&		\sin n\theta\\
-\sin n\theta&		\cos n\theta\\
\end{matrix} \right) \left( \begin{matrix}
\cos \theta&		\sin \theta\\
-\sin \theta&		\cos \theta\\
\end{matrix} \right) 
\\
&=\left( \begin{matrix}
\cos n\theta \cos \theta -\sin n\theta \sin \theta&		\cos n\theta \sin \theta +\sin n\theta \cos \theta\\
-\left( \cos n\theta \sin \theta +\sin n\theta \cos \theta \right)&		\cos n\theta \cos \theta -\sin n\theta \sin \theta\\
\end{matrix} \right) 
=\left( \begin{matrix}
\cos \left( n+1 \right) \theta&		\sin \left( n+1 \right) \theta\\
-\sin \left( n+1 \right) \theta&		\cos \left( n+1 \right) \theta\\
\end{matrix} \right) .
\end{align*}
从而由数学归纳法可知,对任意正整数\(k\),有$\left( \begin{matrix}
\cos \theta&		\sin \theta\\
-\sin \theta&		\cos \theta\\
\end{matrix} \right) ^k=\left( \begin{matrix}
\cos k\theta&		\sin k\theta\\
-\sin k\theta&		\cos k\theta\\
\end{matrix} \right)$.
\end{solution}

\begin{exercise}
求矩阵\(A\)的逆阵:
\[
A = 
\begin{pmatrix}
1 & 2 & 3 & \cdots & n - 1 & n \\
n & 1 & 2 & \cdots & n - 2 & n - 1 \\
n - 1 & n & 1 & \cdots & n - 3 & n - 2 \\
\vdots & \vdots & \vdots & & \vdots & \vdots \\
2 & 3 & 4 & \cdots & n & 1
\end{pmatrix}.
\]
\end{exercise}
\begin{solution}
对\(\left( \begin{matrix}
A&		I_n\\
\end{matrix} \right) \)用初等变换法,将所有行加到第一行上,再将第一行乘以\(s^{-1}\),其中\(s = \frac{1}{2}n(n + 1)\),得到
\setcounter{MaxMatrixCols}{20} % 将矩阵最大列数设置为20
\begin{gather*}
\left( \begin{matrix}
1&		2&		3&		\cdots&		n-1&		n&		1&		0&		0&		\cdots&		0&		0\\
n&		1&		2&		\cdots&		n-2&		n-1&		0&		1&		0&		\cdots&		0&		0\\
n-1&		n&		1&		\cdots&		n-3&		n-2&		0&		0&		1&		\cdots&		0&		0\\
\vdots&		\vdots&		\vdots&		&		\vdots&		\vdots&		\vdots&		\vdots&		\vdots&		&		\vdots&		\vdots\\
2&		3&		4&		\cdots&		n&		1&		0&		0&		0&		\cdots&		0&		1\\
\end{matrix} \right) \rightarrow
\\
\left( \begin{matrix}
1&		1&		1&		\cdots&		1&		1&		\frac{1}{s}&		\frac{1}{s}&		\frac{1}{s}&		\cdots&		\frac{1}{s}&		\frac{1}{s}\\
n&		1&		2&		\cdots&		n-2&		n-1&		0&		1&		0&		\cdots&		0&		0\\
n-1&		n&		1&		\cdots&		n-3&		n-2&		0&		0&		1&		\cdots&		0&		0\\
\vdots&		\vdots&		\vdots&		&		\vdots&		\vdots&		\vdots&		\vdots&		\vdots&		&		\vdots&		\vdots\\
2&		3&		4&		\cdots&		n&		1&		0&		0&		0&		\cdots&		0&		1\\
\end{matrix} \right) .
\end{gather*}
从第二行起依次减去下一行,得到
\begin{gather*}
\left( \begin{matrix}
1&		1&		1&		\cdots&		1&		1&		\frac{1}{s}&		\frac{1}{s}&		\frac{1}{s}&		\cdots&		\frac{1}{s}&		\frac{1}{s}\\
1&		1-n&		1&		\cdots&		1&		1&		0&		1&		-1&		\cdots&		0&		0\\
1&		1&		1-n&		\cdots&		1&		1&		0&		0&		1&		\cdots&		0&		0\\
\vdots&		\vdots&		\vdots&		&		\vdots&		\vdots&		\vdots&		\vdots&		\vdots&		&		\vdots&		\vdots\\
2&		3&		4&		\cdots&		n&		1&		0&		0&		0&		\cdots&		0&		1\\
\end{matrix} \right) .
\end{gather*}
消去第一列除第一行外的所有元素后,得到
\begin{gather*}
\left( \begin{matrix}
1&		1&		1&		\cdots&		1&		1&		\frac{1}{s}&		\frac{1}{s}&		\frac{1}{s}&		\cdots&		\frac{1}{s}&		\frac{1}{s}\\
0&		-n&		0&		\cdots&		0&		0&		-\frac{1}{s}&		\frac{s-1}{s}&		-\frac{s+1}{s}&		\cdots&		-\frac{1}{s}&		-\frac{1}{s}\\
0&		0&		-n&		\cdots&		0&		0&		-\frac{1}{s}&		-\frac{1}{s}&		\frac{s-1}{s}&		\cdots&		-\frac{1}{s}&		-\frac{1}{s}\\
\vdots&		\vdots&		\vdots&		&		\vdots&		\vdots&		\vdots&		\vdots&		\vdots&		&		\vdots&		\vdots\\
0&		1&		2&		\cdots&		n-2&		-1&		-\frac{2}{s}&		-\frac{2}{s}&		-\frac{2}{s}&		\cdots&		-\frac{2}{s}&		\frac{s-2}{s}\\
\end{matrix} \right) .
\end{gather*}
从第二行到第\(n - 1\)行分别乘以\(-\frac{1}{n}\),得到
\begin{gather*}
\left( \begin{matrix}
1&		1&		1&		\cdots&		1&		1&		\frac{1}{s}&		\frac{1}{s}&		\frac{1}{s}&		\cdots&		\frac{1}{s}&		\frac{1}{s}\\
0&		1&		0&		\cdots&		0&		0&		\frac{1}{ns}&		\frac{1-s}{ns}&		\frac{s+1}{ns}&		\cdots&		\frac{1}{ns}&		\frac{1}{ns}\\
0&		0&		1&		\cdots&		0&		0&		\frac{1}{ns}&		\frac{1}{ns}&		\frac{1-s}{ns}&		\cdots&		\frac{1}{ns}&		\frac{1}{ns}\\
\vdots&		\vdots&		\vdots&		&		\vdots&		\vdots&		\vdots&		\vdots&		\vdots&		&		\vdots&		\vdots\\
0&		1&		2&		\cdots&		n-2&		-1&		-\frac{2}{s}&		-\frac{2}{s}&		-\frac{2}{s}&		\cdots&		-\frac{2}{s}&		\frac{s-2}{s}\\
\end{matrix} \right) .
\end{gather*}
将第一行依次减去第二行,第三行,\(\cdots\),第\(n - 1\)行,得到
\begin{gather*}
\left( \begin{matrix}
1&		0&		0&		\cdots&		0&		1&		\frac{2}{ns}&		\frac{s+2}{ns}&		\frac{2}{ns}&		\cdots&		\frac{2}{ns}&		\frac{2-s}{ns}\\
0&		1&		0&		\cdots&		0&		0&		\frac{1}{ns}&		\frac{1-s}{ns}&		\frac{s+1}{ns}&		\cdots&		\frac{1}{ns}&		\frac{1}{ns}\\
0&		0&		1&		\cdots&		0&		0&		\frac{1}{ns}&		\frac{1}{ns}&		\frac{1-s}{ns}&		\cdots&		\frac{1}{ns}&		\frac{1}{ns}\\
\vdots&		\vdots&		\vdots&		&		\vdots&		\vdots&		\vdots&		\vdots&		\vdots&		&		\vdots&		\vdots\\
0&		1&		2&		\cdots&		n-2&		-1&		-\frac{2}{s}&		-\frac{2}{s}&		-\frac{2}{s}&		\cdots&		-\frac{2}{s}&		\frac{s-2}{s}\\
\end{matrix} \right) .
\end{gather*}
将最后一行加到第一行,再将最后一行乘以\(-1\),得到
\begin{gather*}
\left( \begin{matrix}
1&		0&		0&		\cdots&		0&		0&		\frac{1-s}{ns}&		\frac{1+s}{ns}&		\frac{1}{ns}&		\cdots&		\frac{1}{ns}&		\frac{1}{ns}\\
0&		1&		0&		\cdots&		0&		0&		\frac{1}{ns}&		\frac{1-s}{ns}&		\frac{s+1}{ns}&		\cdots&		\frac{1}{ns}&		\frac{1}{ns}\\
0&		0&		1&		\cdots&		0&		0&		\frac{1}{ns}&		\frac{1}{ns}&		\frac{1-s}{ns}&		\cdots&		\frac{1}{ns}&		\frac{1}{ns}\\
\vdots&		\vdots&		\vdots&		&		\vdots&		\vdots&		\vdots&		\vdots&		\vdots&		&		\vdots&		\vdots\\
0&		0&		0&		\cdots&		0&		1&		\frac{s+1}{ns}&		\frac{1}{ns}&		\frac{1}{ns}&		\cdots&		\frac{1}{ns}&		\frac{1-s}{ns}\\
\end{matrix} \right) .
\end{gather*}
因此
\begin{align*}
A^{-1}=\frac{1}{ns}\left( \begin{matrix}
1-s&		1+s&		1&		\cdots&		1&		1\\
1&		1-s&		1+s&		\cdots&		1&		1\\
1&		1&		1-s&		\cdots&		1&		1\\
\vdots&		\vdots&		\vdots&		&		\vdots&		\vdots\\
1+s&		1&		1&		\cdots&		1&		1-s\\
\end{matrix} \right) .
\end{align*}
\end{solution}

\begin{exercise}\label{exercise2.3}
求下列\(n\)阶矩阵的逆阵,其中\(a_i \neq 0(1\leq i\leq n)\):
\[
A = 
\begin{pmatrix}
1 + a_1 & 1 & 1 & \cdots & 1 \\
1 & 1 + a_2 & 1 & \cdots & 1 \\
1 & 1 & 1 + a_3 & \cdots & 1 \\
\vdots & \vdots & \vdots & & \vdots \\
1 & 1 & 1 & \cdots & 1 + a_n
\end{pmatrix}.
\]
\end{exercise}
\begin{solution}
对\(\left( \begin{matrix}
A&		I_n\\
\end{matrix} \right) \)用初等变换法,将第\(i\)行乘以\(a_i^{-1}(1\leq i\leq n)\),有
\begin{gather*}
\left( \begin{matrix}
1+a_1&		1&		1&		\cdots&		1&		1&		0&		0&		\cdots&		0\\
1&		1+a_2&		1&		\cdots&		1&		0&		1&		0&		\cdots&		0\\
1&		1&		1+a_3&		\cdots&		1&		0&		0&		1&		\cdots&		0\\
\vdots&		\vdots&		\vdots&		&		\vdots&		\vdots&		\vdots&		\vdots&		&		\vdots\\
1&		1&		1&		\cdots&		1+a_n&		0&		0&		0&		\cdots&		1\\
\end{matrix} \right) \rightarrow 
\\
\left( \begin{matrix}
1+\frac{1}{a_1}&		\frac{1}{a_1}&		\frac{1}{a_1}&		\cdots&		\frac{1}{a_1}&		\frac{1}{a_1}&		0&		0&		\cdots&		0\\
\frac{1}{a_2}&		1+\frac{1}{a_2}&		\frac{1}{a_2}&		\cdots&		\frac{1}{a_2}&		0&		\frac{1}{a_2}&		0&		\cdots&		0\\
\frac{1}{a_3}&		\frac{1}{a_3}&		1+\frac{1}{a_3}&		\cdots&		\frac{1}{a_3}&		0&		0&		\frac{1}{a_3}&		\cdots&		0\\
\vdots&		\vdots&		\vdots&		&		\vdots&		\vdots&		\vdots&		\vdots&		&		\vdots\\
\frac{1}{a_n}&		\frac{1}{a_n}&		\frac{1}{a_n}&		\cdots&		1+\frac{1}{a_n}&		0&		0&		0&		\cdots&		\frac{1}{a_n}\\
\end{matrix} \right) .
\end{gather*}
将下面的行都加到第一行上,并令\(s = 1 + \frac{1}{a_1} + \frac{1}{a_2} + \cdots + \frac{1}{a_n}\),则上面的矩阵变为
\begin{gather*}
\left( \begin{matrix}
s&		s&		s&		\cdots&		s&		\frac{1}{a_1}&		\frac{1}{a_2}&		\frac{1}{a_3}&		\cdots&		\frac{1}{a_n}\\
\frac{1}{a_2}&		1+\frac{1}{a_2}&		\frac{1}{a_2}&		\cdots&		\frac{1}{a_2}&		0&		\frac{1}{a_2}&		0&		\cdots&		0\\
\frac{1}{a_3}&		\frac{1}{a_3}&		1+\frac{1}{a_3}&		\cdots&		\frac{1}{a_3}&		0&		0&		\frac{1}{a_3}&		\cdots&		0\\
\vdots&		\vdots&		\vdots&		&		\vdots&		\vdots&		\vdots&		\vdots&		&		\vdots\\
\frac{1}{a_n}&		\frac{1}{a_n}&		\frac{1}{a_n}&		\cdots&		1+\frac{1}{a_n}&		0&		0&		0&		\cdots&		\frac{1}{a_n}\\
\end{matrix} \right) \rightarrow 
\\
\left( \begin{matrix}
1&		1&		1&		\cdots&		1&		\frac{1}{sa_1}&		\frac{1}{sa_2}&		\frac{1}{sa_3}&		\cdots&		\frac{1}{sa_n}\\
\frac{1}{a_2}&		1+\frac{1}{a_2}&		\frac{1}{a_2}&		\cdots&		\frac{1}{a_2}&		0&		\frac{1}{a_2}&		0&		\cdots&		0\\
\frac{1}{a_3}&		\frac{1}{a_3}&		1+\frac{1}{a_3}&		\cdots&		\frac{1}{a_3}&		0&		0&		\frac{1}{a_3}&		\cdots&		0\\
\vdots&		\vdots&		\vdots&		&		\vdots&		\vdots&		\vdots&		\vdots&		&		\vdots\\
\frac{1}{a_n}&		\frac{1}{a_n}&		\frac{1}{a_n}&		\cdots&		1+\frac{1}{a_n}&		0&		0&		0&		\cdots&		\frac{1}{a_n}\\
\end{matrix} \right) \rightarrow 
\\
\left( \begin{matrix}
1&		1&		1&		\cdots&		1&		\frac{1}{sa_1}&		\frac{1}{sa_2}&		\frac{1}{sa_3}&		\cdots&		\frac{1}{sa_n}\\
0&		1&		0&		\cdots&		0&		-\frac{1}{sa_1a_2}&		\frac{sa_2-1}{sa_{2}^{2}}&		-\frac{1}{sa_3a_2}&		\cdots&		-\frac{1}{sa_na_2}\\
0&		0&		1&		\cdots&		0&		-\frac{1}{sa_1a_3}&		-\frac{1}{sa_2a_3}&		\frac{sa_3-1}{sa_{3}^{2}}&		\cdots&		-\frac{1}{sa_na_3}\\
\vdots&		\vdots&		\vdots&		&		\vdots&		\vdots&		\vdots&		\vdots&		&		\vdots\\
0&		0&		0&		\cdots&		1&		-\frac{1}{sa_1a_n}&		-\frac{1}{sa_2a_n}&		-\frac{1}{sa_3a_n}&		\cdots&		\frac{sa_n-1}{sa_{n}^{2}}\\
\end{matrix} \right) .
\end{gather*}
再消去第一行的后$n-1$个1就得到
\begin{align*}
A^{-1}=-\frac{1}{s}\left( \begin{matrix}
\frac{1}{a_1}&		\frac{1}{a_2}&		\frac{1}{a_3}&		\cdots&		\frac{1}{a_n}\\
-\frac{1}{a_1a_2}&		\frac{sa_2-1}{a_{2}^{2}}&		-\frac{1}{a_3a_2}&		\cdots&		-\frac{1}{a_na_2}\\
-\frac{1}{a_1a_3}&		-\frac{1}{a_2a_3}&		\frac{sa_3-1}{a_{3}^{2}}&		\cdots&		-\frac{1}{a_na_3}\\
\vdots&		\vdots&		\vdots&		&		\vdots\\
-\frac{1}{a_1a_n}&		-\frac{1}{a_2a_n}&		-\frac{1}{a_3a_n}&		\cdots&		\frac{sa_n-1}{a_{n}^{2}}\\
\end{matrix} \right) .
\end{align*}

\end{solution}

\begin{exercise}
求下列\(n\)阶矩阵的逆矩阵:
\[
A = 
\begin{pmatrix}
0 & 1 & 1 & \cdots & 1 \\
1 & 0 & 1 & \cdots & 1 \\
1 & 1 & 0 & \cdots & 1 \\
\vdots & \vdots & \vdots & & \vdots \\
1 & 1 & 1 & \cdots & 0
\end{pmatrix}
\]    
\end{exercise}
\begin{note}
解法一和解法二的核心想法是:先假设(猜测)矩阵$A$的逆矩阵与其具有相似的结构,再结合逆矩阵的定义,使用待定系数法求出矩阵$A$的逆矩阵.
\end{note}
\begin{solution}
{\color{blue}解法一:}
设\(\alpha = (1,1,\cdots,1)'\),则\(A = -I_n + \alpha\alpha'\).设\(B = cI_n + d\alpha\alpha'\),其中\(c,d\)为待定系数.则\(AB = -cI_n + (c + (n - 1)d)\alpha\alpha'\).令\(c = -1\),\(c + (n - 1)d = 0\),则\(d = \frac{1}{n - 1}\).于是\(AB = I_n\),从而\(A^{-1} = B = -I_n + \frac{1}{n - 1}\alpha\alpha'\).

{\color{blue}解法二\hyperref[Sherman-Morrison公式]{(Sherman-Morrison公式)}:}设\(\alpha = (1,1,\cdots,1)'\),则\(A = -I_n + \alpha\alpha'\).由\hyperref[Sherman-Morrison公式]{Sherman-Morrison公式}可得
\begin{align*}
\boldsymbol{A}^{-1}=\left( -\boldsymbol{I}_n+\boldsymbol{\alpha \alpha }^{\prime} \right) ^{-1}=\left( -\boldsymbol{I}_n \right) ^{-1}-\frac{1}{1+\boldsymbol{\alpha }^{\prime}\left( -\boldsymbol{I}_n \right) ^{-1}\boldsymbol{\alpha }}\left( -\boldsymbol{I}_n \right) ^{-1}\boldsymbol{\alpha \alpha }^{\prime}\left( -\boldsymbol{I}_n \right) ^{-1}=-\boldsymbol{I}_n+\frac{1}{n-1}\boldsymbol{\alpha \alpha }^{\prime}.
\end{align*}

{\color{blue}解法三(循环矩阵):}
设\(J\)为基础循环矩阵,则\(A = J + J^2 + \cdots + J^{n - 1}\).设\(B = cI_n + J + J^2 + \cdots + J^{n - 1}\)(因为循环矩阵的逆仍是循环矩阵),其中\(c\)为待定系数.则
\begin{align*}
AB = (n - 1)I_n + (c + n - 2)(J + J^2 + \cdots + J^{n - 1}) 
\end{align*}
只要令\(c = 2 - n\),则\(AB = (n - 1)I_n\).于是\(A^{-1} = \frac{1}{n - 1}B = \frac{2 - n}{n - 1}I_n + J + J^2 + \cdots + J^{n - 1}\).

{\color{blue}解法四(初等变换):}本题是\hyperref[exercise2.3]{练习\ref{exercise2.3}}的特例,都利用相同的初等变换方法求逆矩阵.
\end{solution}

\begin{exercise}
设\(\boldsymbol{A}\)是非零实矩阵且\(\boldsymbol{A}^* = \boldsymbol{A}'\).求证:\(\boldsymbol{A}\)是可逆阵.
\end{exercise}
\begin{proof}
设\(\boldsymbol{A} = (a_{ij})\),\(a_{ij}\)的代数余子式记为\(A_{ij}\),由已知,\(a_{ij} = A_{ij}\).由于\(\boldsymbol{A}\)是非零实矩阵,故必有某个\(a_{rs} \neq 0\),将\(|\boldsymbol{A}|\)按第\(r\)行展开,可得
\begin{align*}
|\boldsymbol{A}| = a_{r1}A_{r1} + \cdots + a_{rs}A_{rs} + \cdots + a_{rn}A_{rn} = a_{r1}^2 + \cdots + a_{rs}^2 + \cdots + a_{rn}^2 > 0.
\end{align*}
特别地,\(|\boldsymbol{A}| \neq 0\),即\(\boldsymbol{A}\)是可逆阵.
\end{proof}

\begin{exercise}
设\(\boldsymbol{A}\)是奇数阶矩阵,满足\(\boldsymbol{A}\boldsymbol{A}' = \boldsymbol{I}_{n}\)且\(|\boldsymbol{A}| > 0\),证明:\(\boldsymbol{I}_{n} - \boldsymbol{A}\)是奇异阵.
\end{exercise}
\begin{proof}
由\(1 = |\boldsymbol{I}_{n}| = |\boldsymbol{A}\boldsymbol{A}'| = |\boldsymbol{A}||\boldsymbol{A}'| = |\boldsymbol{A}|^{2}\)以及\(|\boldsymbol{A}| > 0\)可得\(|\boldsymbol{A}| = 1\).因为
\begin{align*}
|\boldsymbol{I}_{n} - \boldsymbol{A}| = |\boldsymbol{A}\boldsymbol{A}' - \boldsymbol{A}| = |\boldsymbol{A}||\boldsymbol{A}' - \boldsymbol{I}_{n}| = |(\boldsymbol{A} - \boldsymbol{I}_{n})'| = |\boldsymbol{A} - \boldsymbol{I}_{n}| = (-1)^{n}|\boldsymbol{I}_{n} - \boldsymbol{A}|.
\end{align*}
又\(n\)是奇数,故\(|\boldsymbol{I}_{n} - \boldsymbol{A}| = -|\boldsymbol{I}_{n} - \boldsymbol{A}|\),从而\(|\boldsymbol{I}_{n} - \boldsymbol{A}| = 0\),即\(\boldsymbol{I}_{n} - \boldsymbol{A}\)是奇异阵.
\end{proof}

\begin{exercise}
设\(\boldsymbol{A},\boldsymbol{B}\)为\(n\)阶可逆阵,满足\(\boldsymbol{A}^{2} = \boldsymbol{B}^{2}\)且\(|\boldsymbol{A}| + |\boldsymbol{B}| = 0\),求证:\(\boldsymbol{A} + \boldsymbol{B}\)是奇异阵.
\end{exercise}
\begin{proof}
由已知\(\boldsymbol{A},\boldsymbol{B}\)都是可逆阵且\(|\boldsymbol{B}| = -|\boldsymbol{A}|\),因此
\begin{align*}
|\boldsymbol{A}||\boldsymbol{A} + \boldsymbol{B}| = |\boldsymbol{A}^{2} + \boldsymbol{A}\boldsymbol{B}| = |\boldsymbol{B}^{2} + \boldsymbol{A}\boldsymbol{B}| = |\boldsymbol{B} + \boldsymbol{A}||\boldsymbol{B}| = -|\boldsymbol{A}||\boldsymbol{A} + \boldsymbol{B}|.
\end{align*}
于是\(|\boldsymbol{A}||\boldsymbol{A} + \boldsymbol{B}| = 0\).因为\(|\boldsymbol{A}| \neq 0\),故\(|\boldsymbol{A} + \boldsymbol{B}| = 0\),即\(\boldsymbol{A} + \boldsymbol{B}\)是奇异阵.
\end{proof}

\section{矩阵的初等变换}

\subsection{相抵标准型}


\begin{definition}[矩阵相抵的定义]\label{definition:矩阵相抵的定义}
设矩阵$A,B$,若$A$经有限次初等变换后变成$B$,则称$A$与$B$\textbf{相抵},记作$A\sim B$.
\end{definition}
\begin{note}
容易验证相抵是\(M_{s\times n}(K)\)上的一个等价关系.在相抵关系下,矩阵\(A\)的等价类称为\(A\)的相抵类.
\end{note}

\begin{proposition}[矩阵相抵的等价命题]\label{proposition:矩阵相抵的等价命题}
数域\(K\)上\(s\times n\)矩阵\(A\)和\(B\)相抵等价于:
\begin{enumerate}
\item $A$\text{可以经过初等行变换和初等列变换变成}\(B\).

\item \text{存在}$K$\text{上}$s$\text{级初等矩阵}$P_1,P_2,\cdots,P_t$\text{与}$n$\text{级初等矩阵}$Q_1,Q_2,\cdots,Q_m$,\text{使得}
$P_t\cdots P_2P_1AQ_1Q_2\cdots Q_m = B$.

\item \text{存在}\(K\)\text{上}\(s\)\text{级可逆矩阵}\(P\)\text{与}\(n\)\text{级可逆矩阵}\(Q\),\text{使得}:
\(PAQ = B\).
\end{enumerate}
\end{proposition}

\begin{theorem}[相抵标准型]\label{theorem:相抵标准型}
设数域\(K\)上\(s\times n\)矩阵\(A\)的秩为\(r\).如果\(r > 0\),那么\(A\)相抵于下述形式的矩阵:
\begin{align}\label{equation:相抵标准型1}
\begin{pmatrix}
I_r & 0 \\
0 & 0
\end{pmatrix}   
\end{align}
称矩阵\eqref{equation:相抵标准型1}为\(A\)的相抵标准形;如果\(r = 0\),那么\(A\)相抵于零矩阵,此时称\(A\)的相抵标准形是零矩阵.
\end{theorem}

\begin{corollary}
\begin{enumerate}
\item 数域\(K\)上\(s\times n\)矩阵\(A\)和\(B\)相抵当且仅当它们的秩相等.

\item 设数域\(K\)上\(s\times n\)矩阵\(A\)的秩为\(r(r > 0)\),则存在\(K\)上的\(s\)级、\(n\)级可逆矩阵\(P\)、\(Q\),使得
\(A = P\begin{pmatrix}
I_r & 0 \\
0 & 0
\end{pmatrix}Q\).
\end{enumerate}
\end{corollary}

\begin{proposition}[奇异阵的充要条件]\label{proposition:奇异阵的充要条件}
数域$K$上的$n$阶矩阵$A$是奇异阵的充要条件有:
\begin{enumerate}
\item\label{proposition:奇异阵的充要条件1} 存在数域$K$上不为零的同阶方阵$B$,使得$AB=O$.
\item\label{proposition:奇异阵的充要条件2} 存在数域$K$上的$n$维非零列向量$x$,使得$Ax=0$.
\end{enumerate}
\end{proposition}
\begin{proof}
\begin{enumerate}
\item 充分性($\Leftarrow$):显然若\(A\)可逆,则从\(AB = O\)可得到\(B = O\),因此充分性成立.

必要性($\Rightarrow$):反之,若\(A\)是奇异阵,则存在数域$K$上的可逆阵\(P,Q\),使得\(PAQ = \begin{pmatrix}I_r & O \\ O & O\end{pmatrix}\),其中\(r < n\).令\(C = \begin{pmatrix}O & O \\ O & I_{n - r}\end{pmatrix}\),则\(PAQC = O\).又因为\(P\)可逆,故\(AQC = O\).只要令\(B = QC\in K\)就得到了结论.

\item 充分性($\Leftarrow$):显然若\(A\)可逆,则从\(Ax = 0\)可得到\(x = 0\),因此充分性成立.

必要性($\Rightarrow$):反之,若\(A\)是奇异阵,则存在数域$K$上的可逆阵\(P,Q\),使\(PAQ = \begin{pmatrix}I_r & O \\ O & O\end{pmatrix}\),其中\(r < n\).
令\(y = (0,\cdots,0,1)'\)为\(n\)维列向量,则\(PAQy = 0\).又因为\(P\)可逆,故\(AQy = 0\).
只要令\(x = Qy\in K\)就得到了结论.
\end{enumerate}
\end{proof}

\subsection{练习}

\begin{exercise}
设\(A\)为\(n\)阶实反对称阵,证明:\(I_n - A\)是非异阵.
\end{exercise}
\begin{proof}
(反证法)假设是$I_n-A$是奇异阵,则由\hyperref[proposition:奇异阵的充要条件2]{命题\ref{proposition:奇异阵的充要条件}的2},可知存在\(n\)维非零实列向量\(x\),使得\((I_n - A)x = 0\),即\(Ax = x\).设\(x = (a_1,a_2,\cdots,a_n)'\),其中\(a_i\)都是实数,则由\(A\)的反对称性以及\hyperref[proposition:反对称阵的刻画]{命题\ref{proposition:反对称阵的刻画}},可知
\begin{align*}
0 = x'Ax = x'x = a_1^2 + a_2^2 + \cdots + a_n^2.
\end{align*}
从而\(a_1 = a_2 = \cdots = a_n = 0\),即\(x = 0\),这与已知矛盾.
\end{proof}


\begin{exercise}\label{可逆阵都能只用第三类初等变换化为对角阵}
设\(A\)为\(n\)阶可逆阵,求证:只用第三类初等变换就可以将\(A\)化为如下形状:
\[
\mathrm{diag}\{1,\cdots,1,|A|\}.
\]
\end{exercise}
\begin{proof}
假设\(A\)的第\((1,1)\)元素等于零,因为\(A\)可逆,故第一行必有元素不为零.用第三初等变换将非零元素所在的列加到第一列,则到的矩阵中第\((1,1)\)元素不为零.因此不设\(A\)的第\((1,1)\)元素非零,于是可用三类初等变换将\(A\)的第一行及第一列其余素都消为零.这就是说,\(A\)经过第三类初变换可化为如下形状:
\[
\begin{pmatrix}
a & O \\
O & A_1
\end{pmatrix}.
\]
再对\(A_1\)同样处理,不断做下去,可将\(A\)化为对角阵,并且对角元素均非零.因此我们只要对对角阵证明结论即可.为简化讨论,我们先考虑二阶对角阵:
\[
\begin{pmatrix}
a & 0 \\
0 & b
\end{pmatrix}.
\]
将其第一行乘以\((1 - a)a^{-1}\)加到第行上,再将第二行加到第一行上得到:
\[
\begin{pmatrix}
a & 0 \\
0 & b
\end{pmatrix} \to
\begin{pmatrix}
a & 0 \\
1 - a & b
\end{pmatrix} \to
\begin{pmatrix}
1 & b \\
1 - a & b
\end{pmatrix}.
\]
将其第一列乘以\(-b\)加到第二列上,再将第行乘以\(a - 1\)加到第二行上得到:
\[
\begin{pmatrix}
1 & b \\
1 - a & b
\end{pmatrix} \to
\begin{pmatrix}
1 & 0 \\
1 - a & ab
\end{pmatrix} \to
\begin{pmatrix}
1 & 0 \\
0 & ab
\end{pmatrix}.
\]
从而原结论对二阶对角阵成立.对于$n$阶对角阵$B=diag\{a_1,a_2,\cdots,a_n\}$而言,按照上述方法对$B\left( \begin{matrix}
1&		2\\
1&		2\\
\end{matrix} \right) $所对应的子矩阵进行第三类初等变换得到
\begin{align*}
\left( \begin{matrix}
a_1&		&		&		\\
&		a_2&		&		\\
&		&		\ddots&		\\
&		&		&		a_n\\
\end{matrix} \right) \longrightarrow \left( \begin{matrix}
1&		&		&		\\
&		a_1a_2&		&		\\
&		&		\ddots&		\\
&		&		&		a_n\\
\end{matrix} \right) .
\end{align*}
按照上述方法对再对$B\left( \begin{matrix}
2&		3\\
2&		3\\
\end{matrix} \right) $所对应的子矩阵进行第三类初等变换得到
\begin{align*}
\left( \begin{matrix}
1&		&		&		&		\\
&		a_1a_2&		&		&		\\
&		&		a_3&		&		\\
&		&		&		\ddots&		\\
&		&		&		&		a_n\\
\end{matrix} \right) \longrightarrow \left( \begin{matrix}
1&		&		&		&		\\
&		1&		&		&		\\
&		&		a_1a_2a_3&		&		\\
&		&		&		\ddots&		\\
&		&		&		&		a_n\\
\end{matrix} \right) .
\end{align*}
同理依次对$B\left( \begin{matrix}
k&		k+1\\
k&		k+1\\
\end{matrix} \right),k=1,2\cdots,n-1$所对应的子矩阵按照上述方法进行第三类初等变换,最后得到
\begin{align*}
B=\left( \begin{matrix}
a_1&		&		&		\\
&		a_2&		&		\\
&		&		\ddots&		\\
&		&		&		a_n\\
\end{matrix} \right) \longrightarrow \left( \begin{matrix}
1&		&		&		\\
&		1&		&		\\
&		&		\ddots&		\\
&		&		&		a_1a_2\cdots a_n\\
\end{matrix} \right) .
\end{align*}
于是原结论对对角阵也成立.而我们所用的初等变换始终是第三类初等变换.这就得到了结论.
\end{proof}

\begin{exercise}
求证:任一\(n\)阶矩阵均可表示为形如\(I_n + a_{ij}E_{ij}\)这样的矩阵之积,其中\(E_{ij}\)是\(n\)阶基础矩阵.
\end{exercise}
\begin{proof}
由\hyperref[theorem:相抵标准型]{命题\ref{theorem:相抵标准型}}可知任意一个\(n\)阶矩阵都可表示为有限个初等阵和具有下列形状的对角阵\(D\)之积:
\[
D = \mathrm{diag}\{1,\cdots,1,0,\cdots,0\},
\]
故只要对初等阵和\(D\)证明结论即可.对\(D\),假设\(D\)有\(r\)个\(1\),则
\[
D = (I_n - E_{r + 1,r + 1})\cdots(I_n - E_{nn}).
\]
第三类初等阵已经是这种形状了,即$P_{ij}\left( c \right) =I_n+cE_{ij}$.对第二类初等阵\(P_i(c)\),显然我们有\(P_i(c) = I_n + (c - 1)E_{ii}\).对第一类初等阵\(P_{ij}\),由\hyperref[可逆阵都能只用第三类初等变换化为对角阵]{练习\ref{可逆阵都能只用第三类初等变换化为对角阵}}可知,只用第三类初等变换就可以将\(P_{ij}\)化为\(P_n(-1) = \mathrm{diag}\{1,\cdots,1,-1\}\),因此对第一类初等阵结论也成立.具体地,我们有
\begin{align*}
P_{ij}\cdot P_{ij}\left( -1 \right) P_j\left( -1 \right) P_{ji}\left( -1 \right) P_{ij}\left( 1 \right) =I_n. 
\end{align*}
由此可得
\begin{align*}
&P_{ij}=\left[ P_{ij}\left( -1 \right) P_j\left( -1 \right) P_{ji}\left( -1 \right) P_{ij}\left( 1 \right) \right] ^{-1}=P_{ij}^{-1}\left( 1 \right) P_{ji}^{-1}\left( -1 \right) P_{j}^{-1}\left( -1 \right) P_{ij}^{-1}\left( -1 \right) 
\\
&=P_{ij}\left( -1 \right) P_{ji}\left( 1 \right) P_j\left( -1 \right) P_{ij}\left( 1 \right) =\left( I_n-E_{ij} \right) \left( I_n+E_{ji} \right) \left( I_n-2E_{jj} \right) \left( I_n+E_{ij} \right) .
\end{align*}
\end{proof}


\section{伴随矩阵}

\begin{definition}[伴随矩阵定义]\label{definition:伴随矩阵定义}
设$A=(a_{ij})_{n\times n}$,若
\[
A^* = 
\begin{pmatrix}
A_{11} & A_{21} & \cdots & A_{n - 1,1} & A_{n1} \\
A_{12} & A_{22} & \cdots & A_{n - 1,2} & A_{n2} \\
\vdots & \vdots & & \vdots & \vdots \\
A_{1,n - 1} & A_{2,n - 1} & \cdots & A_{n - 1,n - 1} & A_{n - 1,n} \\
A_{1n} & A_{2n} & \cdots & A_{n,n - 1} & A_{nn}
\end{pmatrix}
\]
其中\(A_{ij}\)是\(a_{ij}\)的代数余子式.则称\(A^*\)为\(A\)的\textbf{伴随矩阵}.
\end{definition}

\begin{theorem}\label{theorem:伴随矩阵的基本性质}
设$A$为$n$阶矩阵,$n\geq 2$,则
\begin{enumerate}[(i)]
\item $AA^*=A^*A=\left| A \right|I_n$.
\item\label{伴随矩阵基本性质2} 当$A$可逆时,有$A^{-1}=\frac{1}{\left| A \right|}A^*$.
\end{enumerate}
\end{theorem}
\begin{proof}
由伴随矩阵的定义不难证明.
\end{proof}

\begin{proposition}
设\(A\)为\(n\)阶矩阵,满足\(A^m = I_n\),则\((A^*)^m = I_n\).
\end{proposition}
\begin{proof}
由\(A^m = I_n\)得\(|A|^m = 1\ne 0\),于是矩阵$A$可逆.又\(A^* = |A|A^{-1}\),故$(A^*)^m = |A|^m(A^{-1})^m = (A^m)^{-1} = I_n$.
\end{proof}

\begin{theorem}[矩阵乘积的伴随]\label{theorem:矩阵乘积的伴随}
设\(A,B\)为\(n\)阶矩阵,\(n\geq 2\),则\((AB)^* = B^*A^*\).
\end{theorem}
\begin{proof}
{\color{blue}证法一(\hyperref[corollary:Cauchy-Binet公式推论]{Cauchy-Binet公式推论}):}设\(C = AB\).记\(M_{ij}, N_{ij}, P_{ij}\)分别是\(A, B, C\)中第\((i, j)\)元素的余子式,\(A_{ij}, B_{ij}, C_{ij}\)分别是\(A, B, C\)中第\((i, j)\)元素的代数余子式.注意到
\[
A^* = 
\begin{pmatrix}
A_{11} & A_{21} & \cdots & A_{n1} \\
A_{12} & A_{22} & \cdots & A_{n2} \\
\vdots & \vdots & & \vdots \\
A_{1n} & A_{2n} & \cdots & A_{nn}
\end{pmatrix},
\quad
B^* = 
\begin{pmatrix}
B_{11} & B_{21} & \cdots & B_{n1} \\
B_{12} & B_{22} & \cdots & B_{n2} \\
\vdots & \vdots & & \vdots \\
B_{1n} & B_{2n} & \cdots & B_{nn}
\end{pmatrix},
\]
\(B^*A^*\)的第\((i, j)\)元素为\(\sum_{k = 1}^{n} B_{ki}A_{jk}\).而\(C^*\)的第\((i, j)\)元素就是\(C_{ji} = (-1)^{j + i}P_{ji}\).

由\hyperref[corollary:Cauchy-Binet公式推论]{Cauchy-Binet公式推论}可得
\begin{align*}
C_{ji} &= (-1)^{j + i}P_{ji} = (-1)^{j + i} \sum_{k = 1}^{n} M_{jk}N_{ki}\\
&= \sum_{k = 1}^{n} (-1)^{j + k}M_{jk}(-1)^{i + k}N_{ki} = \sum_{k =1}^{n} A_{jk}B_{ki}
\end{align*}
故结论成立.

{\color{blue}证法二(\hyperref[example:摄动法]{摄动法}):}若\(A,B\)均为非异阵,则\(A^* = |A|A^{-1},B^* = |B|B^{-1}\),从而
\[
(AB)^* = |AB|(AB)^{-1} = |A||B|(B^{-1}A^{-1})=(|B|B^{-1})(|A|A^{-1}) = B^*A^*.
\]

由\hyperref[proposition:摄动法基本命题]{命题\ref{proposition:摄动法基本命题}},可知对于一般的方阵\(A,B\),可取到一列有理数\(t_k\rightarrow0\),使得\(t_kI_n + A\)与\(t_kI_n + B\)均为非异阵. 由非异阵情形的证明可得
\[
((t_kI_n + A)(t_kI_n + B))^*=(t_kI_n + B)^*(t_kI_n + A)^*.
\]
注意到上式两边均为\(n\)阶方阵,其元素都是\(t_k\)的多项式,从而关于\(t_k\)连续. 上式两边同时取极限,令\(t_k\rightarrow0\),即有\((AB)^* = B^*A^*\)成立.
\end{proof}

\begin{theorem}[伴随矩阵的秩]\label{theorem:伴随矩阵的秩}
设\(A\)为\(n\)阶矩阵,\(n\geq 2\),则
\[
\mathrm{rank}A^* = 
\begin{cases}
n, & \mathrm{rank}A = n, \\
1, & \mathrm{rank}A = n - 1, \\
0, & \mathrm{rank}A < n - 1.
\end{cases}
\]
\end{theorem}
\begin{proof}
当\(\mathrm{rank}A = n\)时,则\(\vert A\vert\neq 0\),\(A\)可逆,又$AA^*=A^*A=\left| A \right|I_n$,两边同时取行列式,可得\(\left| A^* \right|\cdot \left| A \right|=\left| A^*A \right|=\left| \left| A \right|I_n \right|=\left| A \right|^n
\),于是$\left| A^* \right|=\left| A \right|^{n-1}\ne 0$.所以\(\mathrm{rank}A^* = n\).

当\(\mathrm{rank}A = n - 1\)时,\(A\)至少存在一个\(n - 1\)阶子式不等于\(0\),故\(A^*\neq 0\),即\(\mathrm{rank}A^*\geq 1\);由\(\mathrm{rank}A < n\)知\(\vert A\vert = 0\),从而\(AA^* = \vert A\vert E = 0\),故由\hyperref[corollary:矩阵的秩不等式1]{定理\ref{corollary:矩阵的秩不等式1}}可知\(\mathrm{rank}A^* \leq n - \mathrm{rank}A = 1\),于是\(\mathrm{rank}A^* = 1\).
(另证:若\(\boldsymbol{A}\)的秩等于\(n - 1\),则由\hyperref[proposition:奇异系数矩阵Ax=0的解空间]{命题\ref{proposition:奇异系数矩阵Ax=0的解空间}}可知\(\boldsymbol{A}^*\)的\(n\)个列向量都成比例且至少有一列不为零,故\(\boldsymbol{A}^*\)的秩等于\(1\).)

当\(\mathrm{rank}A < n - 1\)时,\(A\)的所有\(n - 1\)阶子式均等于\(0\),即\(A^* = 0\),故\(\mathrm{rank}A^* = 0\).
\end{proof}


\begin{proposition}[伴随矩阵的性质]\label{proposition:伴随矩阵的性质}
设\(A\)为\(n\)阶矩阵,\(n\geq 2\),则
\begin{enumerate}
\item\label{伴随矩阵的性质1}  \((A^{\mathrm{T}})^* = (A^*)^{\mathrm{T}}\).

\item  \((kA)^* = k^{n - 1}A^*\),\(k\)为常数.

\item\label{伴随矩阵的性质3}  若$A$为可逆阵,则$A^*$也可逆,并且\((A^{-1})^* = (A^*)^{-1}\).

\item  \((A^{m})^* = (A^*)^{m}\),\(m\)为正整数.

\item\label{伴随矩阵的性质5}  \(|A^*| = |A|^{n - 1}\).

\item  \((A^*)^* = |A|^{n - 2}A\).
\end{enumerate}
\end{proposition}
\begin{proof}
\begin{enumerate}
\item 由伴随矩阵的定义及行列式的性质即得.

\item 由伴随矩阵的定义及行列式的性质即得.

\item 由\hyperref[theorem:矩阵乘积的伴随]{定理\ref{theorem:矩阵乘积的伴随}}可知
$A^*\left( A^{-1} \right) ^*=\left( A^{-1}A \right) ^*=I_{n}^{*}=I_n$.从而$(A^{-1})^* = (A^*)^{-1}$.

\item 多次利用\hyperref[theorem:矩阵乘积的伴随]{定理\ref{theorem:矩阵乘积的伴随}}即得.

\item 
{\color{blue}证法一:}当\(A\)可逆时,有\(A^* = |A|A^{-1}\),从而\(|A^*| = |A|^{n - 1}\);当\(A\)不可逆时,有$\mathrm{rank}A<n$,由\hyperref[theorem:伴随矩阵的秩]{定理\ref{theorem:伴随矩阵的秩}}知$\mathrm{rank}A^*<n$.于是\(|A^*| = |A| = 0\),故\(|A^*| = |A|^{n - 1}\).

若\(A\)是非异阵,有\(A^* = |A|A^{-1}\),从而\(|A^*| = |A|^{n - 1}\). 对于一般的方阵\(A\),由\hyperref[proposition:摄动法基本命题]{命题\ref{proposition:摄动法基本命题}}可知,可取到一列有理数\(t_k\rightarrow0\),使得\(t_kI_n + A\)为非异阵. 由非异阵情形的证明可得
\[
|(t_kI_n + A)^*| = |t_kI_n + A|^{n - 1}.
\]
注意到上式两边均为行列式的幂次,其值都是\(t_k\)的多项式,从而关于\(t_k\)连续. 上式两边同时取极限(上式两边都是关于$t_k$的多项式函数),令\(t_k\rightarrow0\),即有\(|A^*| = |A|^{n - 1}\)成立.

{\color{blue}证法二:}见白皮书.

\item {\color{blue}证法一:}当\(A\)可逆时,\(A^*\)也可逆,且\((A^*)^{-1} = \frac{1}{|A|}A\),从而由 \hyperref[伴随矩阵的性质5]{伴随矩阵的性质\ref{伴随矩阵的性质5}}得
\[
(A^*)^* = |A^*|(A^*)^{-1} = |A|^{n - 1}\frac{1}{|A|}A = |A|^{n - 2}A.
\]

当\(A\)不可逆时,则\(|A| = 0\),且由\hyperref[theorem:伴随矩阵的秩]{定理\ref{theorem:伴随矩阵的秩}}及$n\geq 2$知\(\mathrm{rank}A^* \leq 1<n-1\),从而\(\mathrm{rank}(A^*)^* = 0\),即\((A^*)^* = 0\),因此\((A^*)^* = |A|^{n - 2}A\).

若\(A\)是非异阵,\(A^*\)也可逆,且\((A^*)^{-1} = \frac{1}{|A|}A\),从而由 \hyperref[伴随矩阵的性质5]{伴随矩阵的性质\ref{伴随矩阵的性质5}}得
\[
(A^*)^* = |A^*|(A^*)^{-1} = |A|^{n - 1}\frac{1}{|A|}A = |A|^{n - 2}A.
\]
对于一般的方阵\(A\),由\hyperref[proposition:摄动法基本命题]{命题\ref{proposition:摄动法基本命题}}可知,可取到一列有理数\(t_k\rightarrow0\),使得\(t_kI_n + A\)为非异阵. 由非异阵情形的证明可得
\[
((t_kI_n + A)^*)^* = |t_kI_n + A|^{n - 2}(t_kI_n + A).
\]
注意到上式两边均为\(n\)阶方阵,其元素都是\(t_k\)的多项式(上式两边的矩阵每个元素都是关于$t_k$的多项式函数),从而关于\(t_k\)连续. 上式两边同时取极限,令\(t_k\rightarrow0\),即有\((A^*)^* = |A|^{n - 2}A\)成立. 

{\color{blue}证法二:}见白皮书.
\end{enumerate}
\end{proof}

\begin{proposition}[伴随矩阵的继承性]\label{proposition:伴随矩阵的继承性}
\begin{enumerate}
\item\label{伴随矩阵的继承性1} 对角矩阵的伴随矩阵是对角矩阵;

\item 对称矩阵的伴随矩阵是对称矩阵;

\item 上(下)三角矩阵的伴随矩阵是上(下)三角矩阵;

\item 可逆矩阵的伴随矩阵是可逆;

\item 正交矩阵的伴随矩阵是正交矩阵;

\item 半正定(正定)矩阵的伴随矩阵是半正定(正定)矩阵;

\item 可对角化矩阵的伴随矩阵是可对角化矩阵.
\end{enumerate}
\end{proposition}
\begin{proof}
\begin{enumerate}
\item 设\(n\)阶矩阵\(A = (a_{ij}),\ n \geq 2\).

\item 若\(A\)为对角矩阵,则\(a_{ij} = 0 (i \neq j)\),从而\(i \neq j\)时,\(M_{ij}\)是对角行列式,且主对角元必有零,即\(M_{ij} = 0\),故\(A_{ij} = 0\),于是\(A^*\)为对角矩阵.

\item 若\(A\)为对称矩阵,则\(a_{ij} = a_{ji} (i, j = 1, 2, \cdots, n)\),因此\(i, j = 1, 2, \cdots, n\)时,\(M_{ij}\)是对称行列式,从而\(A_{ij} = A_{ji}\),即\(A^*\)为对称矩阵.

\item 若\(A\)为上三角矩阵,则\(1 \leq j < i \leq n\)时,\(a_{ij} = 0\),所以\(1 \leq i < j \leq n\)时,\(M_{ij}\)是上三角行列式,且主对角元必有零,即\(M_{ij} = 0\),从而\(A_{ij} = 0\),所以\(A^*\)为上三角矩阵.同理可证:下三角矩阵的伴随矩阵是下三角矩阵.

\item 由\(\vert A \vert \neq 0\)和\(A^* = \vert A \vert A^{-1}\)即知.

\item 因为\(A\)为正交矩阵等价于\(A^{-1} = A^{\mathrm{T}}\),所以\(\vert A \vert^{-1} = \vert A \vert\).从而由\hyperref[伴随矩阵基本性质2]{定理\ref{theorem:伴随矩阵的基本性质}\ref{伴随矩阵基本性质2}},有
\[
(A^*)^{-1} = \vert A \vert^{-1} A = (\vert A \vert A^{\mathrm{T}})^{\mathrm{T}} = (A^*)^{\mathrm{T}},
\]
故\(A^*\)为正交矩阵.

\item 由于\(A\)为半正定矩阵等价于存在实矩阵\(C\),使得\(A = C^{\mathrm{T}}C\),因此由\hyperref[theorem:矩阵乘积的伴随]{定理\ref{theorem:矩阵乘积的伴随}}和\hyperref[伴随矩阵的性质1]{伴随矩阵的性质\ref{伴随矩阵的性质1}},有
\[
A^* = (C^{\mathrm{T}}C)^* = C^*(C^{\mathrm{T}})^* = C^*(C^*)^{\mathrm{T}},
\]
于是\(A^*\)为半正定矩阵.当\(A\)为正定矩阵时,同理可证\(A^*\)为正定矩阵.

\item 若\(A\)可对角化,则存在可逆矩阵\(P\),使得\(A = P\Lambda P^{\mathrm{T}}\),其中\(\Lambda\)为对角矩阵,从而由\hyperref[theorem:矩阵乘积的伴随]{定理\ref{theorem:矩阵乘积的伴随}}和\hyperref[伴随矩阵的性质1]{伴随矩阵的性质\ref{伴随矩阵的性质1}},有
\[
A^* = (P^{\mathrm{T}})^*\Lambda^*P^* = (P^*)^{\mathrm{T}}\Lambda^*P^*,
\]
再根据\hyperref[伴随矩阵的继承性1]{伴随矩阵的继承性\ref{伴随矩阵的继承性1}}和\hyperref[伴随矩阵的性质3]{伴随矩阵的性质\ref{伴随矩阵的性质3}},知\(\Lambda^*\)为对角矩阵,\(P^*\)为可逆矩阵,故\(A^*\)可对角化.
\end{enumerate}
\end{proof}

\begin{proposition}[分块矩阵的伴随矩阵]\label{proposition:分块矩阵的伴随矩阵}
设\(A\)为\(m\)阶矩阵,\(B\)为\(n\)阶矩阵,分块对角阵\(C\)为
\[
C = 
\begin{pmatrix}
A & O \\
O & B
\end{pmatrix}.
\]
则分块对角阵$C$的伴随矩阵为:
\[
C^* = 
\begin{pmatrix}
|B|A^* & O \\
O & |A|B^*
\end{pmatrix}.
\]
\end{proposition}
\begin{proof}
{\color{blue}证法一:}
设\(A = (a_{ij})_{m\times m}\),元素\(a_{ij}\)的余子式和代数余子式分别记为\(M_{ij}\)和\(A_{ij}\);\(B = (b_{ij})_{n\times n}\),元素\(b_{ij}\)的余子式和代数余子式分别记为\(N_{ij}\)和\(B_{ij}\).利用Laplace定理可以容易地计算出:当\(1 \leq i, j \leq m\)时,\(C\)的第\((i, j)\)元素的代数余子式为\((-1)^{i + j}M_{ij}|B| = |B|A_{ij}\);当\(m + 1 \leq i, j \leq m + n\)时,由Laplace定理,可知\(C\)的第\((i, j)\)元素的代数余子式为\((-1)^{i + j}N_{i - m, j - m}|A| = |A|B_{i - m, j - m}\);当\(i, j\)属于其他范围时,由Laplace定理,当$1\leq i\leq m,m\leq j\leq m+n$时,将其按前$m$列展开,当$m\leq i\leq m+n,1\leq j\leq m$时,将其按前$m$行展开,可得\(C\)的第\((i, j)\)元素的代数余子式等于零.因此我们有
\[
C^* = 
\begin{pmatrix}
|B|A^* & O \\
O & |A|B^*
\end{pmatrix}.
\]
{\color{blue}证法二:}若\(A,B\)均为非异阵,则
\begin{align*}
C\begin{pmatrix}
|B|A^* & O\\
O & |A|B^*
\end{pmatrix}&=\begin{pmatrix}
A & O\\
O & B
\end{pmatrix}\begin{pmatrix}
|B|A^* & O\\
O & |A|B^*
\end{pmatrix}
=\begin{pmatrix}
|B|AA^* & O\\
O & |A|BB^*
\end{pmatrix}=\begin{pmatrix}
|A||B|I_m & O\\
O & |A||B|I_n
\end{pmatrix}
=|C|I_{m + n}=CC^*,
\end{align*}
注意到\(C\)非异,故由上式可得
\[
C^*=\begin{pmatrix}
A & O\\
O & B
\end{pmatrix}^*=\begin{pmatrix}
|B|A^* & O\\
O & |A|B^*
\end{pmatrix}.
\]

对于一般的方阵\(A,B\),由\hyperref[proposition:摄动法基本命题]{命题\ref{proposition:摄动法基本命题}}可知,可取到一列有理数\(t_k\rightarrow0\),使得\(t_kI_m + A\)与\(t_kI_n + B\)均为非异阵. 由非异阵情形的证明可得
\[
\begin{pmatrix}
t_kI_m + A & O\\
O & t_kI_n + B
\end{pmatrix}^*=\begin{pmatrix}
|t_kI_n + B|(t_kI_m + A)^* & O\\
O & |t_kI_m + A|(t_kI_n + B)^*
\end{pmatrix}.
\]
注意到上式两边均为\(m + n\)阶方阵,其元素都是\(t_k\)的多项式,从而关于\(t_k\)连续. 上式两边同时取极限,令\(t_k\rightarrow0\),即有\(\begin{pmatrix}
A & O\\
O & B
\end{pmatrix}^*=\begin{pmatrix}
|B|A^* & O\\
O & |A|B^*
\end{pmatrix}\)成立.
\end{proof}


\subsection{练习}

\begin{exercise}
设\(A,B\)为\(n\)阶方阵,满足\(AB = BA\),证明:\(AB^* = B^*A\).
\end{exercise}
\begin{proof}
若\(B\)为非异阵,则由\(AB = BA\)可得\(AB^{-1}=B^{-1}A\). 又\(B^* = |B|B^{-1}\),于是\(AB^* = B^*A\)成立. 对于一般的方阵\(B\),可取到一列有理数\(t_k\rightarrow0\),使得\(t_kI_n + B\)为非异阵,此时\(A(t_kI_n + B)=(t_kI_n + B)A\)仍然成立. 由非异阵情形的证明可得
\[
A(t_kI_n + B)^*=(t_kI_n + B)^*A.
\]
注意到上式两边均为\(n\)阶方阵,其元素都是\(t_k\)的多项式,从而关于\(t_k\)连续. 上式两边同时取极限,令\(t_k\rightarrow0\),即有\(AB^* = B^*A\)成立.
\end{proof}

\begin{exercise}
设\(n\)阶矩阵
\[
A = 
\begin{pmatrix}
2 & 2 & 2 & \cdots & 2 \\
0 & 1 & 1 & \cdots & 1 \\
0 & 0 & 1 & \cdots & 1 \\
\vdots & \vdots & \vdots & & \vdots \\
0 & 0 & 0 & \cdots & 1
\end{pmatrix},
\]
求\(\sum_{i,j = 1}^{n} A_{ij}\).
\end{exercise}
\begin{solution}
{\color{blue}解法一:}显然\(\vert A\vert = 2\),用初等变换不难求出
\[
A^{-1} = 
\begin{pmatrix}
\frac{1}{2} & -1 & 0 & \cdots & 0 & 0 \\
0 & 1 & -1 & \cdots & 0 & 0 \\
\vdots & \vdots & \vdots & & \vdots & \vdots \\
0 & 0 & 0 & \cdots & 1 & -1 \\
0 & 0 & 0 & \cdots & 0 & 1
\end{pmatrix},
\]
故
\[
A^* = 2A^{-1} = 
\begin{pmatrix}
1 & -2 & 0 & \cdots & 0 & 0 \\
0 & 2 & -2 & \cdots & 0 & 0 \\
\vdots & \vdots & \vdots & & \vdots & \vdots \\
0 & 0 & 0 & \cdots & 2 & -2 \\
0 & 0 & 0 & \cdots & 0 & 2
\end{pmatrix}.
\]
将\(A^*\)的所有元素加起来,可得\(\sum_{i,j = 1}^{n} A_{ij} = 1\).

{\color{blue}解法二:}由\hyperref[根据行列式代数余子式构造行列式]{命题\ref{根据行列式代数余子式构造行列式}}可得
\begin{align*}
-\sum_{i,j=1}^n{A_{ij}}=\left| \begin{matrix}
2&		2&		2&		\cdots&		2&		1\\
0&		1&		1&		\cdots&		1&		1\\
0&		0&		1&		\cdots&		1&		1\\
\vdots&		\vdots&		\vdots&		&		\vdots&		\vdots\\
0&		0&		0&		\cdots&		1&		1\\
1&		1&		1&		\cdots&		1&		0\\
\end{matrix} \right|=\left| \begin{matrix}
2&		2&		2&		\cdots&		2&		1\\
0&		1&		1&		\cdots&		1&		1\\
0&		0&		1&		\cdots&		1&		1\\
\vdots&		\vdots&		\vdots&		&		\vdots&		\vdots\\
0&		0&		0&		\cdots&		1&		1\\
0&		0&		0&		\cdots&		0&		-\frac{1}{2}\\
\end{matrix} \right|=-1.
\end{align*}
于是$\sum_{i,j=1}^n{A_{ij}}=1$.

{\color{blue}解法三:}由\hyperref[大拆分法]{大拆分法}可得$\left| A\left( -1 \right) \right|=\left| A \right|-\sum_{i,j=1}^n{A_{ij}}$,且
\begin{align*}
\left| A\left( -1 \right) \right|=\left| \begin{matrix}
1&		1&		\cdots&		1&		1\\
-1&		0&		\cdots&		0&		0\\
-1&		-1&		\cdots&		0&		0\\
\vdots&		\vdots&		&		\vdots&		\vdots\\
-1&		-1&		\cdots&		-1&		0\\
\end{matrix} \right|=\left( -1 \right) ^{n+1}\left| \begin{matrix}
-1&		0&		\cdots&		0\\
-1&		-1&		\cdots&		0\\
\vdots&		\vdots&		&		\vdots\\
-1&		-1&		\cdots&		-1\\
\end{matrix} \right|=1.
\end{align*}
故$\sum_{i,j=1}^n{A_{ij}}=\left| A\left( -1 \right) \right|-\left| A \right|$.

{\color{blue}解法四:}由\hyperref[example:求矩阵代数余子式和的方法1]{例题\ref{example:求矩阵代数余子式和的方法1}}可得
\begin{align*}
\sum_{i,j=1}^n{A_{ij}}=\left| \begin{matrix}
0&		0&		\cdots&		0&		1\\
-1&		0&		\cdots&		0&		1\\
0&		-1&		\cdots&		0&		1\\
\vdots&		\vdots&		&		\vdots&		\vdots\\
0&		0&		\cdots&		-1&		1\\
\end{matrix} \right|=(-1)^{n+1}\left| \begin{matrix}
-1&		0&		\cdots&		0\\
0&		-1&		\cdots&		0\\
\vdots&		\vdots&		&		\vdots\\
0&		0&		\cdots&		-1\\
\end{matrix} \right|=1.
\end{align*}
\end{solution}

\section{矩阵的迹}

\begin{proposition}[矩阵迹的性质]\label{proposition:矩阵矩阵迹的性质}
设\(A,B\)是\(n\)阶矩阵,则有:
\begin{enumerate}
\item (线性)\(\mathrm{tr}(A + B)=\mathrm{tr}(A)+\mathrm{tr}(B)\),\(\mathrm{tr}(kA)=k\mathrm{tr}(A)\);
\item (对称性)\(\mathrm{tr}(A')=\mathrm{tr}(A)\);
\item (交换性)\(\mathrm{tr}(AB)=\mathrm{tr}(BA)\).
\end{enumerate}
\end{proposition}
\begin{proof}
根据矩阵迹的定义及矩阵乘法的定义容易验证.
\end{proof}

\begin{proposition}[矩阵迹的刻画]\label{proposition:矩阵迹的刻画}
设\(f\)是数域\(\mathbb{F}\)上\(n\)阶矩阵集合到\(\mathbb{F}\)的一个映射,它满足下列条件:

(1) 对任意的\(n\)阶矩阵\(A,B\),\(f(A + B)=f(A)+f(B)\);

(2) 对任意的\(n\)阶矩阵\(A\)和\(\mathbb{F}\)中的数\(k\),\(f(kA)=kf(A)\);

(3) 对任意的\(n\)阶矩阵\(A,B\),\(f(AB)=f(BA)\);

(4) \(f(I_n)=n\).

求证:\(f\)就是迹,即\(f(A)=\mathrm{tr}(A)\)对一切\(\mathbb{F}\)上\(n\)阶矩阵\(A\)成立.
\end{proposition}
\begin{note}
这个命题给出了迹的刻画,它告诉我们迹函数由线性、交换性和正规性(即单位矩阵处的取值为其阶数)唯一决定.
\end{note}
\begin{proof}
设\(E_{ij}\)是\(n\)阶基础矩阵.由(1)和(4),有
\[
n = f(I_n)=f(E_{11}+E_{22}+\cdots+E_{nn})=f(E_{11})+f(E_{22})+\cdots+f(E_{nn}).
\]
又由(3),有
\[
f(E_{ii})=f(E_{ij}E_{ji})=f(E_{ji}E_{ij})=f(E_{jj}),
\]
所以\(f(E_{ii}) = 1(1\leq i\leq n)\).另一方面,若\(i\neq j\),则
\[
f(E_{ij})=f(E_{i1}E_{1j})=f(E_{1j}E_{i1})=f(O)=f(0\cdot I_n)=0\cdot f(I_n)=0.
\]
设\(n\)阶矩阵\(A=(a_{ij})\),则
\[
f(A)=f\left(\sum_{i,j = 1}^{n}a_{ij}E_{ij}\right)=\sum_{i,j = 1}^{n}a_{ij}f(E_{ij})=\sum_{i = 1}^{n}a_{ii}=\mathrm{tr}(A).
\]
\end{proof}

\begin{example}
求证:不存在\(n\)阶矩阵\(A,B\),使得\(AB - BA = kI_n(k\neq0)\).
\end{example}
\begin{proof}
用反证法证明. 若存在\(n\)阶矩阵\(A,B\)满足条件\(AB - BA = kI_n(k\neq0)\),则
\[
kn=\mathrm{tr}(kI_n)=\mathrm{tr}(AB - BA)=\mathrm{tr}(AB)-\mathrm{tr}(BA)=0
\]
矛盾.
\end{proof}

\begin{example}
设\(A\)是\(n\)阶矩阵,\(P\)是同阶可逆阵,求证:\(\mathrm{tr}(P^{-1}AP)=\mathrm{tr}(A)\),即相似矩阵具有相同的迹.
\end{example}
\begin{proof}
因为\(\mathrm{tr}(AB)=\mathrm{tr}(BA)\),故\(\mathrm{tr}(P^{-1}AP)=\mathrm{tr}(APP^{-1})=\mathrm{tr}(A)\).
\end{proof}

\begin{example}
设\(A_1,A_2,\cdots,A_k\)是实对称阵且\(A_1^2 + A_2^2+\cdots+A_k^2 = O\),证明:每个\(A_i = O\).
\end{example}
\begin{proof}
对题设中的等式两边同时取迹,可得
\[
0=\mathrm{tr}(O)=\mathrm{tr}(A_1^2 + A_2^2+\cdots+A_k^2)=\mathrm{tr}(A_1A_1')+\mathrm{tr}(A_2A_2')+\cdots+\mathrm{tr}(A_kA_k').
\]
又由于\(\mathrm{tr}(A_iA_i')\geq0\),从而只可能是\(\mathrm{tr}(A_iA_i') = 0(1\leq i\leq k)\),再次由\hyperref[proposition:零矩阵的充要条件]{零矩阵的充要条件}可得\(A_i = O(1\leq i\leq k)\).
\end{proof}

\begin{proposition}\label{proposition:反称/反酉矩阵为零矩阵的充要条件}
\begin{enumerate}
\item 设\(n\)阶实矩阵\(A\)适合\(A'=-A\),如果存在同阶实矩阵\(B\),使得\(AB = B\),则\(B = O\);
\item 设\(n\)阶复矩阵\(A\)适合\(\overline{A}'=-A\),如果存在同阶复矩阵\(B\),使得\(AB = B\),则\(B = O\).
\end{enumerate}
\end{proposition}
\begin{proof}
\begin{enumerate}
\item 在等式\(AB = B\)两边同时左乘\(B'\)可得
\[
B'AB = B'B.
\]
上式两边同时转置并注意到\(A'=-A\),可得
\[
B'B=(B'B)'=(B'AB)'=B'A'B=-B'AB=-B'B,
\]
从而有\(B'B = O\).两边同时取迹,由\hyperref[proposition:零矩阵的充要条件]{零矩阵的充要条件}可得\(B = O\).
\item 证明与1类似.
\end{enumerate}
\end{proof}

\begin{proposition}\label{proposition:矩阵迹的不等式}
设\(A\)为\(n\)阶实矩阵,求证:\(\mathrm{tr}(A^2)\leq\mathrm{tr}(AA')\),等号成立当且仅当\(A\)是对称阵.
\end{proposition}
\begin{proof}
若已知$\mathrm{tr}(AA') \geq \mathrm{tr}(A^2)$,则由迹的线性、对称性、交换性和正定性可得
\begin{align*}
&\mathrm{tr}((A - A')(A - A')')\\
=&\mathrm{tr}((A - A')(A' - A))=\mathrm{tr}(AA' - A^2 - (A')^2 + A'A)\\
=&2\mathrm{tr}(AA') - 2\mathrm{tr}(A^2)\geq0,
\end{align*}
故要证的不等式成立.若上述不等式的等号成立,则由迹的正定性可知\(A - A' = O\),即\(A\)为对称阵.
若已知$A$为对称阵,则$\mathrm{tr}(AA') = \mathrm{tr}(A^2)$显然成立.
\end{proof}

\begin{proposition}\label{proposition:矩阵可交换关于迹的充分条件}
设\(A,B\)是两个\(n\)阶矩阵,使得\(\mathrm{tr}(ABC)=\mathrm{tr}(CBA)\)对任意\(n\)阶矩阵\(C\)成立,求证:\(AB = BA\).
\end{proposition}
\begin{proof}
设\(AB=(d_{ij}),BA=(e_{ij})\),令\(C = E_{kl}(1\leq k,l\leq n)\),则
\[
\mathrm{tr}(ABC)=d_{lk},\mathrm{tr}(CBA)=e_{lk},
\]
因此\(d_{lk}=e_{lk}(1\leq k,l\leq n)\),即有\(AB = BA\).
\end{proof}
\begin{remark}
注 若\(A,B\)是实(复)矩阵,我们还可以通过迹的正定性来证明结论.事实上,由迹的交换性和线性可得\(\mathrm{tr}((AB - BA)C)=0\),令\(C\)为\(AB - BA\)的转置(共轭转置),再由\hyperref[proposition:零矩阵的充要条件]{零矩阵的充要条件}即得结论.
\end{remark}

\begin{example}
若\(n\)阶实方阵\(A\)满足\(AA' = I_n\),则称为正交矩阵. 证明:不存在\(n\)阶正交矩阵\(A,B\)满足\(A^2 = cAB + B^2\),其中\(c\)是非零常数.
\end{example}
\begin{proof}
用反证法,设存在\(n\)阶正交阵\(A,B\),使得\(A^2 = cAB + B^2(c\neq0)\). 在等式两边同时左乘\(A'\),右乘\(B'\),可得\(AB' = cI_n + A'B\),从而\(cI_n = A'B - AB'\). 两边同时取迹,可得\(0\ne nc=\text{tr}(cI_n)=\text{tr}(A'B)-\text{tr}(AB')=\text{tr}((A'B)') - \text{tr}(AB')=\text{tr}(B'A)-\text{tr}(AB') = 0\),矛盾.
\end{proof}

\begin{example}
设\(A,B\)为\(n\)阶实对称阵,证明:\(\text{tr}((AB)^2)\leq\text{tr}(A^2B^2)\),并求等号成立的充要条件.
\end{example}
\begin{proof}
由\hyperref[proposition:矩阵迹的不等式]{命题\ref{proposition:矩阵迹的不等式}},再结合$A,B$的对称性可得
\begin{align*}
\mathrm{tr}\left( \left( AB \right) ^2 \right) \leqslant \mathrm{tr}\left( \left( AB \right) \left( AB \right) \prime \right) =\mathrm{tr}\left( ABBA \right) =\mathrm{tr}\left( A^2B^2 \right) .
\end{align*}
等号成立当且仅当$AB$也为实对称矩阵,即\(AB = B'A'=BA\).
\end{proof}


\section{矩阵乘法与行列式计算}

\begin{proposition}[可以写成两个矩阵(向量)乘积的矩阵]\label{proposition:可以写成两个矩阵(向量)乘积的矩阵}
若已知矩阵\begin{align*}
A=\left( \begin{matrix}
a_{11}b_{11}+a_{12}b_{12}+\cdots +a_{1n}b_{1n}&		a_{21}b_{11}+a_{22}b_{12}+\cdots +a_{2n}b_{1n}&		\cdots&		a_{n1}b_{11}+a_{n2}b_{12}+\cdots +a_{nn}b_{1n}\\
a_{11}b_{21}+a_{12}b_{22}+\cdots +a_{1n}b_{2n}&		a_{21}b_{21}+a_{22}b_{22}+\cdots +a_{2n}b_{2n}&		\cdots&		a_{n1}b_{21}+a_{n2}b_{22}+\cdots +a_{nn}b_{2n}\\
\vdots&		\vdots&		&		\vdots\\
a_{11}b_{n1}+a_{12}b_{n2}+\cdots +a_{1n}b_{nn}&		a_{21}b_{n1}+a_{22}b_{n2}+\cdots +a_{2n}b_{nn}&		\cdots&		a_{n1}b_{n1}+a_{n2}b_{n2}+\cdots +a_{nn}b_{nn}\\
\end{matrix} \right) .
\end{align*}
则矩阵$A$可以写成$BC$,其中
\begin{align*}
B=\left( \begin{matrix}
b_{11}&		b_{12}&		\cdots&		b_{1n}\\
b_{21}&		b_{22}&		\cdots&		b_{2n}\\
\vdots&		\vdots&		&		\vdots\\
b_{n1}&		b_{n2}&		\cdots&		b_{nn}\\
\end{matrix} \right),
C=\left( \begin{matrix}
a_{11}&		a_{21}&		\cdots&		a_{n1}\\
a_{12}&		a_{22}&		\cdots&		a_{n2}\\
\vdots&		\vdots&		&		\vdots\\
a_{1n}&		a_{2n}&		\cdots&		a_{nn}\\
\end{matrix} \right) .
\end{align*}
即\begin{align*}
A=\left( \begin{matrix}
b_{11}&		b_{12}&		\cdots&		b_{1n}\\
b_{21}&		b_{22}&		\cdots&		b_{2n}\\
\vdots&		\vdots&		&		\vdots\\
b_{n1}&		b_{n2}&		\cdots&		b_{nn}\\
\end{matrix} \right) \left( \begin{matrix}
a_{11}&		a_{21}&		\cdots&		a_{n1}\\
a_{12}&		a_{22}&		\cdots&		a_{n2}\\
\vdots&		\vdots&		&		\vdots\\
a_{1n}&		a_{2n}&		\cdots&		a_{nn}\\
\end{matrix} \right) =BC.
\end{align*}
特别地,若矩阵$A$的行/列向量成比例,不妨设
\begin{align*}
A=\left( \begin{matrix}
a_1b_1&		a_2b_1&		\cdots&		a_nb_1\\
a_1b_2&		a_2b_2&		\cdots&		a_nb_2\\
\vdots&		\vdots&		&		\vdots\\
a_1b_n&		a_2b_n&		\cdots&		a_nb_n\\
\end{matrix} \right) .
\end{align*}
则令$\alpha =\left( a_1,a_2,\cdots ,a_n \right) ,\beta =\left( b_1,b_2,\cdots ,b_n \right)$,就有$A=\beta'\alpha=\left( \begin{array}{c}
b_1\\
b_2\\
\vdots\\
b_n\\
\end{array} \right) \left( \begin{matrix}
a_1&		a_2&		\cdots&		a_n\\
\end{matrix} \right)$.
\end{proposition}
\begin{remark}
若矩阵的列向量成比例,则行向量也一定成比例.反之也成立.
\end{remark}
\begin{note}
观察原矩阵$A$不难发现:\textbf{矩阵$A$的每一行沿行方向只有$a_{ij}$(的角标)改变,而$b_{kl}$(的角标)并不改变;而矩阵$A$的每一列沿列方向只有$b_{kl}$(的角标)改变,$a_{ij}$(的角标)并不改变.}因此具有这种性质的矩阵,都可以按照\hyperref[proposition:可以写成两个矩阵乘积的矩阵]{这个命题}将其写成两个矩阵的乘积.特别地,\textbf{若矩阵的行/列向量成比例,则一定可以将其写成两个向量的乘积.}

记忆小技巧:只需要记住矩阵$B$的形式(沿行方向不变的项写在前面作为矩阵$B$的元素),然后结合原矩阵,利用矩阵乘法就能写出矩阵$C$.即\textbf{按行变化的项写左边(作为矩阵$B$的元素),按列变化的项写右边(作为矩阵$C$的元素).}
\end{note}
\begin{remark}
拆分后的矩阵$B$的行数与原矩阵$A$相同,矩阵$C$的列数与原矩阵$A$相同.但是矩阵$B$的列数与矩阵$C$的行数可以任意选取,只要满足$BC=A$即可.
\end{remark}
\begin{proof}
利用矩阵乘法容易得到证明.
\end{proof}

\begin{proposition}[一些能写成两个向量乘积的矩阵]\label{proposition:一些能写成两个向量乘积的矩阵}
1.$\quad$$\left( \begin{matrix}
1&		1&		\cdots&		1\\
1&		1&		\cdots&		1\\
\vdots&		\vdots&		&		\vdots\\
1&		1&		\cdots&		1\\
\end{matrix} \right) =\alpha\alpha'$,其中$\alpha =\left( 1,1,\cdots ,1 \right)' $.

2.若矩阵$A$的行/列向量成比例,不妨设
\begin{align*}
A=\left( \begin{matrix}
a_1b_1&		a_1b_2&		\cdots&		a_1b_n\\
a_2b_1&		a_2b_2&		\cdots&		a_2b_n\\
\vdots&		\vdots&		&		\vdots\\
a_nb_1&		a_nb_2&		\cdots&		a_nb_n\\
\end{matrix} \right) .
\end{align*}
则有$A =\left( \begin{array}{c}
a_1\\
a_2\\
\vdots\\
a_n\\
\end{array} \right) \left( \begin{matrix}
b_1&		b_2&		\cdots&		b_n\\
\end{matrix} \right) $.
\end{proposition}
\begin{note}
这里的$a_i$可以是行向量,$b_i$可以是列向量.此时矩阵$A$的元素就是$a_ib_i$仍然是一个数.并且此时矩阵$A$能够分解的条件应该改为\textbf{矩阵$A$的每一行都有公共的行向量$a_i$,每一列都有公共的列向量$b_i$.}

\begin{remark}
若$a_i,b_i$是上述向量,则根据矩阵乘法,可知$a_i$的列数可以任意选取,$b_i$的行数可以任意选取.此时只要确定每个向量$a_i,b_i$就可以确定矩阵$A$的分解式.
\end{remark}
\end{note}

\begin{example}
设\(s_k = x_1^k + x_2^k+\cdots+x_n^k(k\geq1)\),\(s_0 = n\),
\[
S = 
\begin{pmatrix}
s_0 & s_1 & s_2 & \cdots & s_{n - 1}\\
s_1 & s_2 & s_3 & \cdots & s_{n}\\
s_2 & s_3 & s_4 & \cdots & s_{n + 1}\\
\vdots & \vdots & \vdots & & \vdots\\
s_{n - 1} & s_{n} & s_{n + 1} & \cdots & s_{2n - 2}
\end{pmatrix}.
\]
求\(|S|\)的值并证明若\(x_i\)是实数,则\(|S|\geq0\).
\end{example}
\begin{solution}
设
\[
V = 
\begin{pmatrix}
1 & 1 & 1 & \cdots & 1\\
x_1 & x_2 & x_3 & \cdots & x_n\\
x_1^2 & x_2^2 & x_3^2 & \cdots & x_n^2\\
\vdots & \vdots & \vdots & & \vdots\\
x_1^{n - 1} & x_2^{n - 1} & x_3^{n - 1} & \cdots & x_n^{n - 1}
\end{pmatrix}
\]
则\(S = VV'\),因此
\[
|S| = |V|^2=\prod_{1\leq i<j\leq n}(x_j - x_i)^2\geq0.
\]
\end{solution}

\begin{example}
设\(s_k = x_1^k + x_2^k+\cdots + x_n^k(k\geq1)\),\(s_0 = n\),计算矩阵\(A\)的行列式的值:
\[
A = 
\begin{pmatrix}
s_0 & s_1 & \cdots & s_{n - 1} & 1\\
s_1 & s_2 & \cdots & s_n & x\\
\vdots & \vdots & & \vdots & \vdots\\
s_n & s_{n + 1} & \cdots & s_{2n - 1} & x^n
\end{pmatrix}.
\]
\end{example}
\begin{solution}
将矩阵\(A\)分解为两个矩阵的乘积:
\[
A = 
\begin{pmatrix}
1 & 1 & \cdots & 1 & 1\\
x_1 & x_2 & \cdots & x_n & x\\
\vdots & \vdots & & \vdots & \vdots\\
x_1^{n - 1} & x_2^{n - 1} & \cdots & x_n^{n - 1} & x^{n - 1}\\
x_1^n & x_2^n & \cdots & x_n^n & x^n
\end{pmatrix}
\begin{pmatrix}
1 & x_1 & \cdots & x_1^{n - 1} & 0\\
1 & x_2 & \cdots & x_2^{n - 1} & 0\\
\vdots & \vdots & & \vdots & \vdots\\
1 & x_n & \cdots & x_n^{n - 1} & 0\\
0 & 0 & \cdots & 0 & 1
\end{pmatrix}
\]
因此\(|A|=(x - x_1)(x - x_2)\cdots(x - x_n)\prod_{1\leq i < j\leq n}(x_j - x_i)^2\).
\end{solution}


\begin{example}
计算下列矩阵\(A\)的行列式的值:
\[
A = 
\begin{pmatrix}
x & -y & -z & -w\\
y & x & -w & z\\
z & w & x & -y\\
w & -z & y & x
\end{pmatrix}.
\]
\end{example}
\begin{solution}
{\color{blue}解法一:}
注意到
\[
AA' = 
\begin{pmatrix}
x & -y & -z & -w\\
y & x & -w & z\\
z & w & x & -y\\
w & -z & y & x
\end{pmatrix}
\begin{pmatrix}
x & y & z & w\\
-y & x & w & -z\\
-z & -w & x & y\\
-w & z & -y & x
\end{pmatrix}
=
\begin{pmatrix}
u & 0 & 0 & 0\\
0 & u & 0 & 0\\
0 & 0 & u & 0\\
0 & 0 & 0 & u
\end{pmatrix}
\]
其中\(u = x^2 + y^2 + z^2 + w^2\),因此
\[
|A|^2=(x^2 + y^2 + z^2 + w^2)^4.
\]
故
\[
|A|=(x^2 + y^2 + z^2 + w^2)^2.
\]
在矩阵\(A\)中令\(x = 1,y = z = w = 0\),显然\(|A| = 1\).

{\color{blue}解法二:}
令
\[
B = 
\begin{pmatrix}
x & -y\\
y & x
\end{pmatrix},
C = 
\begin{pmatrix}
z & w\\
w & -z
\end{pmatrix},
\]
则\(|A| = 
\begin{vmatrix}
B & -C\\
C & B
\end{vmatrix}\).由\hyperref[proposition:对角相同的复分块矩阵行列式计算]{命题\ref{proposition:对角相同的复分块矩阵行列式计算}}可得
\begin{align*}
|A|&=|B + \mathrm{i}C||B - \mathrm{i}C|
=\begin{vmatrix}
x + \mathrm{i}z & -y + \mathrm{i}w\\
y + \mathrm{i}w & x - \mathrm{i}z
\end{vmatrix}\begin{vmatrix}
x - \mathrm{i}z & -y - \mathrm{i}w\\
y - \mathrm{i}w & x + \mathrm{i}z
\end{vmatrix}
=(x^2 + y^2 + z^2 + w^2)^2.
\end{align*}
\end{solution}

\begin{example}
计算下列矩阵\(A\)的行列式的值:
\[
A = 
\begin{pmatrix}
\cos\theta & \cos2\theta & \cos3\theta & \cdots & \cos n\theta\\
\cos n\theta & \cos\theta & \cos2\theta & \cdots & \cos(n - 1)\theta\\
\cos(n - 1)\theta & \cos n\theta & \cos\theta & \cdots & \cos(n - 2)\theta\\
\vdots & \vdots & \vdots & & \vdots\\
\cos2\theta & \cos3\theta & \cos4\theta & \cdots & \cos\theta
\end{pmatrix}.
\]
\end{example}
\begin{solution}
解 由\hyperref[proposition:循环行列式计算公式]{上面的结论}可知
\[
|A| = f(\varepsilon_1)f(\varepsilon_2)\cdots f(\varepsilon_n),
\]
其中\(\varepsilon_1,\varepsilon_2,\cdots,\varepsilon_n\)是\(1\)的所有\(n\)次方根,\(f(x)=\cos\theta + x\cos2\theta+\cdots+x^{n - 1}\cos n\theta\).令
\[
g(x)=\sin\theta + x\sin2\theta+\cdots+x^{n - 1}\sin n\theta,
\]
则由De Moivre公式可得
\begin{align*}
&f(x)+\mathrm{i}g(x)=\left( \cos \theta +\mathrm{i}\sin \theta \right) +x\left( \cos \theta +\mathrm{i}\sin \theta \right) ^2+\cdots +x^{n-1}\left( \cos \theta +\mathrm{i}\sin \theta \right) ^n
\\
&=\frac{\left( \cos \theta +\mathrm{i}\sin \theta \right) \left[ 1-x^n\left( \cos \theta +\mathrm{i}\sin \theta \right) ^n \right]}{1-x\left( \cos \theta +\mathrm{i}\sin \theta \right)}=\frac{1-x^n\left( \cos \theta +\mathrm{i}\sin \theta \right) ^n}{\cos \theta -x-\mathrm{i}\sin \theta}=\frac{\left( 1-x^n\cos n\theta +\mathrm{i}x^n\sin n\theta \right) \left( \cos \theta -x+\mathrm{i}\sin \theta \right)}{\left[ \left( \cos \theta -\mathrm{i}\sin \theta \right) -x \right] \left[ \left( \cos \theta +\mathrm{i}\sin \theta \right) -x \right]}
\\
&=\frac{\left( 1-x^n\cos n\theta +\mathrm{i}x^n\sin n\theta \right) \left( \cos \theta -x+\mathrm{i}\sin \theta \right)}{\left( \cos \theta -x \right) ^2+\sin ^2\theta}=\frac{\left( 1-x^n\cos n\theta +\mathrm{i}x^n\sin n\theta \right) \left( \cos \theta -x+\mathrm{i}\sin \theta \right)}{x^2-2x\cos \theta +1}
\\
&=\frac{\left[ x^{n+1}\cos n\theta -x^n\cos \left( n+1 \right) \theta -x+\cos \theta \right] +\mathrm{i}\left[ x^{n+1}\sin n\theta +x^n\sin \left( n-1 \right) \theta +\sin \theta \right]}{x^2-2x\cos \theta +1}.
\end{align*}
再比较实部,可得
\[
f(x)=\frac{\cos n\theta\cdot x^{n + 1}-\cos(n + 1)\theta\cdot x^n - x+\cos\theta}{x^2 - 2\cos\theta\cdot x + 1}.
\]
对任意的\(\varepsilon_i\),经计算并化简,可得
\[
f(\varepsilon _i)=\frac{\left( \cos \theta -\cos \left( n+1 \right) \theta \right) -\varepsilon _i\left( 1-\cos n\theta \right)}{\left[ \left( \cos \theta +\mathrm{i}\sin \theta \right) -\varepsilon _i \right] \left[ \left( \cos \theta -\mathrm{i}\sin \theta \right) -\varepsilon _i \right]}.
\]
注意到对任意的\(a,b\),有\(a^n - b^n=(a - \varepsilon_1b)(a - \varepsilon_2b)\cdots(a - \varepsilon_nb)\),因此
\begin{align*}
&\left| A \right|=\prod_{i=1}^n{f(\varepsilon _i)}=\frac{(\cos \theta -\cos\mathrm{(}n+1)\theta )^n-(1-\cos n\theta )^n}{(\cos n\theta +\mathrm{i}\sin n\theta -1)(\cos n\theta -\mathrm{i}\sin n\theta -1)}
\\
&=\frac{(\cos \theta -\cos\mathrm{(}n+1)\theta )^n-(1-\cos n\theta )^n}{2(1-\cos n\theta )}
\\
&=\frac{2^n\sin ^n\frac{n\theta}{2}\sin ^n\frac{\left( n-2 \right) \theta}{2}-2^n\sin ^{2n}\frac{n\theta}{2}}{4\sin ^2\frac{n\theta}{2}}
\\
&=2^{n-2}\sin ^{n-2}\frac{n\theta}{2}\left( \sin ^n\frac{(n+2)\theta}{2}-\sin ^n\frac{n\theta}{2} \right) .
\end{align*}
\end{solution}

\section{Cauchy-Binet公式}

\begin{theorem}[Cauchy-Binet公式]\label{theorem:Cauchy-Binet公式}
设\(A=(a_{ij})\)是\(m\times n\)矩阵,\(B=(b_{ij})\)是\(n\times m\)矩阵.\(A\left(\begin{matrix}
i_1 & \cdots & i_s\\
j_1 & \cdots & j_s
\end{matrix}\right)\)表示\(A\)的一个\(s\)阶子式,它是由\(A\)的第\(i_1,\cdots,i_s\)行与第\(j_1,\cdots,j_s\)列交点上的元素按原次序排列组成的行列式.同理定义\(B\)的\(s\)阶子式.

(1) 若\(m > n\),则有\(\vert AB\vert=0\);

(2) 若\(m\leq n\),则有
\[\vert AB\vert=\sum_{1\leq j_1<j_2<\cdots<j_m\leq n}A\left(\begin{matrix}
1 & 2 & \cdots & m\\
j_1 & j_2 & \cdots & j_m
\end{matrix}\right)B\left(\begin{matrix}
j_1 & j_2 & \cdots & j_m\\
1 & 2 & \cdots & m
\end{matrix}\right).\]
\end{theorem}

\begin{corollary}[Cauchy-Binet公式推论]\label{corollary:Cauchy-Binet公式推论}
设\(A=(a_{ij})\)是\(m\times n\)矩阵,\(B=(b_{ij})\)是\(n\times m\)矩阵,\(r\)是一个正整数且\(r\leq m\).

(1) 若\(r > n\),则\(AB\)的任意\(r\)阶子式都等于零;

(2) 若\(r\leq n\),则\(AB\)的\(r\)阶子式
\[AB\left(\begin{matrix}
i_1 & i_2 & \cdots & i_r\\
j_1 & j_2 & \cdots & j_r
\end{matrix}\right)=\sum_{1\leq k_1<k_2<\cdots<k_r\leq n}A\left(\begin{matrix}
i_1 & i_2 & \cdots & i_r\\
k_1 & k_2 & \cdots & k_r
\end{matrix}\right)B\left(\begin{matrix}
k_1 & k_2 & \cdots & k_r\\
j_1 & j_2 & \cdots & j_r
\end{matrix}\right).\]
\end{corollary}

\begin{example}
设\(n\geq3\),证明下列矩阵是奇异阵:
\[
A = 
\begin{pmatrix}
\cos(\alpha_1 - \beta_1) & \cos(\alpha_1 - \beta_2) & \cdots & \cos(\alpha_1 - \beta_n)\\
\cos(\alpha_2 - \beta_1) & \cos(\alpha_2 - \beta_2) & \cdots & \cos(\alpha_2 - \beta_n)\\
\vdots & \vdots & & \vdots\\
\cos(\alpha_n - \beta_1) & \cos(\alpha_n - \beta_2) & \cdots & \cos(\alpha_n - \beta_n)
\end{pmatrix}.
\]    
\end{example}
\begin{solution}
注意到
\begin{align*}
A=\left( \begin{matrix}
\cos\mathrm{(}\alpha _1-\beta _1)&		\cos\mathrm{(}\alpha _1-\beta _2)&		\cdots&		\cos\mathrm{(}\alpha _1-\beta _n)\\
\cos\mathrm{(}\alpha _2-\beta_1)&		\cos\mathrm{(}\alpha_2-\beta _2)&		\cdots&	\cos\mathrm{(}\alpha _2-\beta _n)\\
\vdots&		\vdots&		&	\vdots\\
\cos\mathrm{(}\alpha _n-\beta_1)&		\cos\mathrm{(}\alpha_n-\beta _2)&		\cdots&	\cos\mathrm{(}\alpha _n-\beta _n)\\
\end{matrix} \right) 
=\left( \begin{matrix}
\cos \alpha _1&		\sin \alpha_1\\
\cos \alpha _2&		\sin \alpha_2\\
\vdots&		\vdots\\
\cos \alpha _n&		\sin \alpha_n\\
\end{matrix} \right) \left( \begin{matrix}
\cos \beta _1&		\cos \beta _2&		\cdots&		\cos \beta _n\\
\sin \beta _1&		\sin \beta _2&		\cdots&		\sin \beta _n\\
\end{matrix} \right) .
\end{align*}
由\hyperref[theorem:Cauchy-Binet公式]{Cauchy-Binet公式},可知$\left| A \right|=0$.
\end{solution}

\begin{example}
设\(A\)是\(m\times n\)实矩阵,求证:矩阵\(AA'\)的任一主子式都非负.
\end{example}
\begin{proof}
若\(r\leq n\),则由\hyperref[corollary:Cauchy-Binet公式推论]{Cauchy-Binet公式推论}可得
\[
AA'\begin{pmatrix}
i_1 & i_2 & \cdots & i_r\\
i_1 & i_2 & \cdots & i_r
\end{pmatrix}=\sum_{1\leq j_1<j_2<\cdots<j_r\leq n}\left(A\begin{pmatrix}
i_1 & i_2 & \cdots & i_r\\
j_1 & j_2 & \cdots & j_r
\end{pmatrix}\right)^2\geq0;
\]
若\(r > n\),则\(AA'\)的任一\(r\)阶主子式都等于零,结论也成立.
\end{proof}

\begin{example}
设\(A\)是\(n\)阶实方阵且\(AA' = I_n\).求证:若\(1\leq i_1 < i_2 < \cdots < i_r\leq n\),则
\[
\sum_{1\leq j_1<j_2<\cdots<j_r\leq n}\left(A\begin{pmatrix}
i_1 & i_2 & \cdots & i_r\\
j_1 & j_2 & \cdots & j_r
\end{pmatrix}\right)^2 = 1.
\]
\end{example}
\begin{proof}
对等式$AA' = I_n$两边同时求r阶子式,因为$r\leq n$,所以由\hyperref[theorem:Cauchy-Binet公式]{Cauchy-Binet公式}即得结论成立.
\end{proof}

\begin{example}
设\(A,B\)分别是\(m\times n\),\(n\times m\)矩阵,求证:\(AB\)和\(BA\)的\(r\)阶主子式之和相等,其中\(1\leq r\leq\min\{m,n\}\).
\end{example}
\begin{proof}
由\hyperref[theorem:Cauchy-Binet公式]{Cauchy-Binet公式}可得
\begin{align*}
&\sum_{1\leq i_1 < i_2 < \cdots < i_r\leq m}AB\begin{pmatrix}
i_1 & i_2 & \cdots & i_r\\
i_1 & i_2 & \cdots & i_r
\end{pmatrix}\\
=&\sum_{1\leq i_1 < i_2 < \cdots < i_r\leq m}\sum_{1\leq j_1 < j_2 < \cdots < j_r\leq n}A\begin{pmatrix}
i_1 & i_2 & \cdots & i_r\\
j_1 & j_2 & \cdots & j_r
\end{pmatrix}B\begin{pmatrix}
j_1 & j_2 & \cdots & j_r\\
i_1 & i_2 & \cdots & i_r
\end{pmatrix}\\
=&\sum_{1\leq j_1 < j_2 < \cdots < j_r\leq n}\sum_{1\leq i_1 < i_2 < \cdots < i_r\leq m}B\begin{pmatrix}
j_1 & j_2 & \cdots & j_r\\
i_1 & i_2 & \cdots & i_r
\end{pmatrix}A\begin{pmatrix}
i_1 & i_2 & \cdots & i_r\\
j_1 & j_2 & \cdots & j_r
\end{pmatrix}\\
=&\sum_{1\leq j_1 < j_2 < \cdots < j_r\leq n}BA\begin{pmatrix}
j_1 & j_2 & \cdots & j_r\\
j_1 & j_2 & \cdots & j_r
\end{pmatrix}.
\end{align*}
\end{proof}

\begin{lemma}[Lagrange恒等式]\label{lemma:Lagrange恒等式}
证明Lagrange恒等式\((n\geq2)\):
\[
\left(\sum_{i = 1}^{n}a_i^2\right)\left(\sum_{i = 1}^{n}b_i^2\right)-\left(\sum_{i = 1}^{n}a_ib_i\right)^2=\sum_{1\leq i<j\leq n}(a_ib_j - a_jb_i)^2.
\]
\end{lemma}
\begin{proof}
左边的式子等于
\[
\begin{vmatrix}
\sum_{i = 1}^{n}a_i^2 & \sum_{i = 1}^{n}a_ib_i\\
\sum_{i = 1}^{n}a_ib_i & \sum_{i = 1}^{n}b_i^2
\end{vmatrix},
\]
这个行列式对应的矩阵可化为
\[
\begin{pmatrix}
a_1 & a_2 & \cdots & a_n\\
b_1 & b_2 & \cdots & b_n
\end{pmatrix}
\begin{pmatrix}
a_1 & b_1\\
a_2 & b_2\\
\vdots & \vdots\\
a_n & b_n
\end{pmatrix}.
\]
由Cauchy - Binet公式可得
\[
\begin{vmatrix}
\sum_{i = 1}^{n}a_i^2 & \sum_{i = 1}^{n}a_ib_i\\
\sum_{i = 1}^{n}a_ib_i & \sum_{i = 1}^{n}b_i^2
\end{vmatrix}
=\sum_{1\leq i<j\leq n}
\begin{vmatrix}
a_i & a_j\\
b_i & b_j
\end{vmatrix}
\begin{vmatrix}
a_i & b_i\\
a_j & b_j
\end{vmatrix}
=\sum_{1\leq i<j\leq n}(a_ib_j - a_jb_i)^2.
\]  
\end{proof}

\begin{theorem}[Cauchy - Schwarz不等式]\label{theorem:Cauchy - Schwarz不等式}
设\(a_i,b_i\)都是实数,证明Cauchy - Schwarz不等式:
\[
\left(\sum_{i = 1}^{n}a_i^2\right)\left(\sum_{i = 1}^{n}b_i^2\right)\geq\left(\sum_{i = 1}^{n}a_ib_i\right)^2.
\]
\end{theorem}
\begin{proof}
由\hyperref[lemma:Lagrange恒等式]{Lagrange恒等式},恒等式右边总非负,即得结论.
\end{proof}

\begin{example}
设\(A,B\)都是\(m\times n\)实矩阵,求证:
\[
|AA'||BB'|\geq|AB'|^2.
\]
\end{example}
\begin{proof}
若\(m > n\),则\(|AA'| = |BB'| = |AB'| = 0\),结论显然成立.
若\(m\leq n\),则由Cauchy - Binet公式可得
\[
|AA'|=\sum_{1\leq j_1<j_2<\cdots<j_m\leq n}\left(A\begin{pmatrix}
1 & 2 & \cdots & m\\
j_1 & j_2 & \cdots & j_m
\end{pmatrix}\right)^2;
\]
\[
|BB'|=\sum_{1\leq j_1<j_2<\cdots<j_m\leq n}\left(B\begin{pmatrix}
1 & 2 & \cdots & m\\
j_1 & j_2 & \cdots & j_m
\end{pmatrix}\right)^2;
\]
\[
|AB'|=\sum_{1\leq j_1<j_2<\cdots<j_m\leq n}A\begin{pmatrix}
1 & 2 & \cdots & m\\
j_1 & j_2 & \cdots & j_m
\end{pmatrix}B\begin{pmatrix}
1 & 2 & \cdots & m\\
j_1 & j_2 & \cdots & j_m
\end{pmatrix},
\]
再由\hyperref[theorem:Cauchy - Schwarz不等式]{Cauchy - Schwarz不等式}即得结论.
\end{proof}

\section{分块矩阵的初等变换与降价公式(打洞原理)}

\begin{proposition}[打洞原理]\label{proposition:打洞原理}
(1)设
\[
\boldsymbol{M} = 
\begin{pmatrix}
\boldsymbol{A} & \boldsymbol{B} \\
\boldsymbol{C} & \boldsymbol{D}
\end{pmatrix}_{(n + m) \times (n + m)}
\]
是一个方阵,并且\(\boldsymbol{A}\)为\(n\)阶可逆子方阵,那么
\[
|\boldsymbol{M}| = |\boldsymbol{A}| \cdot |\boldsymbol{D} - \boldsymbol{C}\boldsymbol{A}^{-1}\boldsymbol{B}|.
\]

(2)设
\[
\boldsymbol{M} = 
\begin{pmatrix}
\boldsymbol{A} & \boldsymbol{B} \\
\boldsymbol{C} & \boldsymbol{D}
\end{pmatrix}_{(n + m) \times (n + m)}
\]
是一个方阵,并且\(\boldsymbol{D}\)为\(n\)阶可逆子方阵,那么
\[
|\boldsymbol{M}| = |\boldsymbol{D}| \cdot |\boldsymbol{A} - \boldsymbol{B}\boldsymbol{D}^{-1}\boldsymbol{C}|.
\]
\end{proposition}
\begin{note}
\hyperref[proposition:打洞原理]{打洞原理}是一个重要结论,必须要熟练掌握.但是在实际解题中我们一般不会直接套用\hyperref[proposition:打洞原理]{打洞原理}的结论,而是利用分块矩阵的初等变换书写过程.

记忆打洞原理公式的小技巧:先记住一个模板$\left| \Box \right|=\left| \Box \right|\left| \Box -\Box \Box ^{-1}\Box \right|$,然后从左往右填入子矩阵(每个子矩阵只能填一次),第一个$\Box$填相应的可逆子矩阵,再从主对角线上另外一个子矩阵开始,按顺时针顺序将子矩阵填入$\Box$即可.
\end{note}
\begin{proof}
(核心想法:利用分块矩阵的初等变换消去$\boldsymbol{B}$或$\boldsymbol{C}$)

(1)根据分块矩阵的初等变换,对$\boldsymbol{M}$的第一行左乘$(-\boldsymbol{CA}^{-1})$再加到第二行得到
\begin{align*}
\left( \begin{matrix}
\boldsymbol{I}_n&		\boldsymbol{O}\\
-\boldsymbol{CA}^{-1}&		\boldsymbol{I}_m\\
\end{matrix} \right) \left( \begin{matrix}
\boldsymbol{A}&		\boldsymbol{B}\\
\boldsymbol{C}&		\boldsymbol{D}\\
\end{matrix} \right) =\left( \begin{matrix}
\boldsymbol{A}&		\boldsymbol{B}\\
\boldsymbol{O}&		\boldsymbol{D}-\boldsymbol{CA}^{-1}\boldsymbol{B}\\
\end{matrix} \right) .
\end{align*}
然后两边同时取行列式就得到
\begin{align*}
\left| \boldsymbol{M} \right|=\left| \begin{matrix}
\boldsymbol{I}_n&		\boldsymbol{O}\\
-\boldsymbol{CA}^{-1}&		\boldsymbol{I}_m\\
\end{matrix} \right|\left| \begin{matrix}
\boldsymbol{A}&		\boldsymbol{B}\\
\boldsymbol{C}&		\boldsymbol{D}\\
\end{matrix} \right|=\left| \left( \begin{matrix}
\boldsymbol{I}_n&		\boldsymbol{O}\\
-\boldsymbol{CA}^{-1}&		\boldsymbol{I}_m\\
\end{matrix} \right) \left( \begin{matrix}
\boldsymbol{A}&		\boldsymbol{B}\\
\boldsymbol{C}&		\boldsymbol{D}\\
\end{matrix} \right) \right|=\left| \begin{matrix}
\boldsymbol{A}&		\boldsymbol{B}\\
\boldsymbol{O}&		\boldsymbol{D}-\boldsymbol{CA}^{-1}\boldsymbol{B}\\
\end{matrix} \right|=|\boldsymbol{A}|\cdot |\boldsymbol{D}-\boldsymbol{CA}^{-1}\boldsymbol{B}|.
\end{align*}

(2)根据分块矩阵的初等变换,对$\boldsymbol{M}$的第二行左乘$(-\boldsymbol{BD}^{-1})$再加到第一行得到
\begin{align*}
\left( \begin{matrix}
\boldsymbol{I}_n&		-\boldsymbol{BD}^{-1}\\
\boldsymbol{O}&		\boldsymbol{I}_m\\
\end{matrix} \right) \left( \begin{matrix}
\boldsymbol{A}&		\boldsymbol{B}\\
\boldsymbol{C}&		\boldsymbol{D}\\
\end{matrix} \right) =\left( \begin{matrix}
\boldsymbol{A}-\boldsymbol{BD}^{-1}\boldsymbol{C}&		\boldsymbol{O}\\
\boldsymbol{C}&		\boldsymbol{D}\\
\end{matrix} \right) .
\end{align*}
然后两边同时取行列式就得到
\begin{align*}
\left| \boldsymbol{M} \right|=\left| \begin{matrix}
\boldsymbol{I}_n&		-\boldsymbol{BD}^{-1}\\
\boldsymbol{O}&		\boldsymbol{I}_m\\
\end{matrix} \right|\left| \begin{matrix}
\boldsymbol{A}&		\boldsymbol{B}\\
\boldsymbol{C}&		\boldsymbol{D}\\
\end{matrix} \right|=\left| \left( \begin{matrix}
\boldsymbol{I}_n&		-\boldsymbol{BD}^{-1}\\
\boldsymbol{O}&		\boldsymbol{I}_m\\
\end{matrix} \right) \left( \begin{matrix}
\boldsymbol{A}&		\boldsymbol{B}\\
\boldsymbol{C}&		\boldsymbol{D}\\
\end{matrix} \right) \right|=\left| \begin{matrix}
\boldsymbol{A}-\boldsymbol{BD}^{-1}\boldsymbol{C}&		\boldsymbol{O}\\
\boldsymbol{C}&		\boldsymbol{D}\\
\end{matrix} \right|=|\boldsymbol{D}|\cdot |\boldsymbol{A}-\boldsymbol{BD}^{-1}\boldsymbol{C}|.
\end{align*}

\end{proof}

\begin{corollary}[打洞原理推论]\label{corollary:打洞原理推论}
设\(\boldsymbol{A}\)是\(n\times m\)矩阵,\(\boldsymbol{B}\)是\(m\times n\)矩阵,则
\[
\lambda^{m}|\lambda\boldsymbol{I}_{n}-\boldsymbol{AB}|=\lambda^{n}|\lambda\boldsymbol{I}_{m}-\boldsymbol{BA}|.
\]
\end{corollary}
\begin{note}
这个推论能将原本复杂的矩阵$AB$通过交换顺序变成相对简单的矩阵$BA$.例如:\hyperref[example:1895]{例题\ref{example:1895}}.
\end{note}
\begin{remark}
这是由\hyperref[proposition:打洞原理]{打洞原理}得到的一个重要结论,也需要熟练掌握.同样地,在实际解题中如果不能直接套用\hyperref[corollary:打洞原理推论]{打洞原理推论}的结论,就需要利用分块矩阵的初等变换书写过程.
\end{remark}
\begin{proof}
当$\lambda=0$时,结论显然成立.

当$\lambda\ne 0$时,根据分块矩阵的初等变换可知
\begin{gather*}
\left( \begin{matrix}
\boldsymbol{I}_n&		-\boldsymbol{A}\\
\boldsymbol{O}&		\boldsymbol{I}_m\\
\end{matrix} \right) \left( \begin{matrix}
\lambda \boldsymbol{I}_n&		\boldsymbol{A}\\
\boldsymbol{B}&		\boldsymbol{I}_m\\
\end{matrix} \right) =\left( \begin{matrix}
\lambda \boldsymbol{I}_n-\boldsymbol{AB}&		\boldsymbol{O}\\
\boldsymbol{B}&		\boldsymbol{I}_m\\
\end{matrix} \right) ,
\\
\left( \begin{matrix}
\boldsymbol{I}_n&		\boldsymbol{O}\\
-\frac{1}{\lambda}\boldsymbol{B}&		\boldsymbol{I}_m\\
\end{matrix} \right) \left( \begin{matrix}
\lambda \boldsymbol{I}_n&		\boldsymbol{A}\\
\boldsymbol{B}&		\boldsymbol{I}_m\\
\end{matrix} \right) =\left( \begin{matrix}
\lambda \boldsymbol{I}_n&		\boldsymbol{A}\\
\boldsymbol{O}&		\boldsymbol{I}_m-\frac{1}{\lambda}\boldsymbol{BA}\\
\end{matrix} \right) .
\end{gather*}
再对上式两边分别取行列式得到
\begin{gather*}
\left| \begin{matrix}
\lambda \boldsymbol{I}_n&		\boldsymbol{A}\\
\boldsymbol{B}&		\boldsymbol{I}_m\\
\end{matrix} \right|=\left| \begin{matrix}
\boldsymbol{I}_n&		-\boldsymbol{A}\\
\boldsymbol{O}&		\boldsymbol{I}_m\\
\end{matrix} \right|\left| \begin{matrix}
\lambda \boldsymbol{I}_n&		\boldsymbol{A}\\
\boldsymbol{B}&		\boldsymbol{I}_m\\
\end{matrix} \right|=\left| \left( \begin{matrix}
\boldsymbol{I}_n&		-\boldsymbol{A}\\
\boldsymbol{O}&		\boldsymbol{I}_m\\
\end{matrix} \right) \left( \begin{matrix}
\lambda \boldsymbol{I}_n&		\boldsymbol{A}\\
\boldsymbol{B}&		\boldsymbol{I}_m\\
\end{matrix} \right) \right|=\left| \begin{matrix}
\lambda \boldsymbol{I}_n-\boldsymbol{AB}&		\boldsymbol{O}\\
\boldsymbol{B}&		\boldsymbol{I}_m\\
\end{matrix} \right|=\left| \lambda \boldsymbol{I}_n-\boldsymbol{AB} \right|.
\\
\left| \begin{matrix}
\lambda \boldsymbol{I}_n&		\boldsymbol{A}\\
\boldsymbol{B}&		\boldsymbol{I}_m\\
\end{matrix} \right|=\left| \begin{matrix}
\boldsymbol{I}_n&		\boldsymbol{O}\\
-\frac{1}{\lambda}\boldsymbol{B}&		\boldsymbol{I}_m\\
\end{matrix} \right|\left| \begin{matrix}
\lambda \boldsymbol{I}_n&		\boldsymbol{A}\\
\boldsymbol{B}&		\boldsymbol{I}_m\\
\end{matrix} \right|=\left| \left( \begin{matrix}
\boldsymbol{I}_n&		\boldsymbol{O}\\
-\frac{1}{\lambda}\boldsymbol{B}&		\boldsymbol{I}_m\\
\end{matrix} \right) \left( \begin{matrix}
\lambda \boldsymbol{I}_n&		\boldsymbol{A}\\
\boldsymbol{B}&		\boldsymbol{I}_m\\
\end{matrix} \right) \right|=\left| \begin{matrix}
\lambda \boldsymbol{I}_n&		\boldsymbol{A}\\
\boldsymbol{O}&		\boldsymbol{I}_m-\frac{1}{\lambda}\boldsymbol{BA}\\
\end{matrix} \right|=\lambda ^n\left| \boldsymbol{I}_m-\frac{1}{\lambda}\boldsymbol{BA} \right|=\lambda ^{n-m}\left| \lambda \boldsymbol{I}_m-\boldsymbol{BA} \right|.    
\end{gather*}
于是$\left| \begin{matrix}
\lambda \boldsymbol{I}_n&		\boldsymbol{A}\\
\boldsymbol{B}&		\boldsymbol{I}_m\\
\end{matrix} \right|=\left| \lambda \boldsymbol{I}_n-\boldsymbol{AB} \right|=\lambda ^{n-m}\left| \lambda \boldsymbol{I}_m-\boldsymbol{BA} \right|.
$.故$\lambda ^m\left| \lambda \boldsymbol{I}_n-\boldsymbol{AB} \right|=\lambda ^n\left| \lambda \boldsymbol{I}_m-\boldsymbol{BA} \right|$.
\end{proof}

\begin{example}\label{example:1895}
求下列矩阵的行列式的值:
\[
A = 
\begin{pmatrix}
a_1^2 & a_1a_2 + 1 & \cdots & a_1a_n + 1\\
a_2a_1 + 1 & a_2^2 & \cdots & a_2a_n + 1\\
\vdots & \vdots & & \vdots\\
a_na_1 + 1 & a_na_2 + 1 & \cdots & a_n^2
\end{pmatrix}.
\]
\end{example}
\begin{solution}
令$\boldsymbol{\varLambda }=\left( \begin{matrix}
a_1&		1\\
a_2&		1\\
\vdots&		\vdots\\
a_n&		1\\
\end{matrix} \right)$,则由降价公式(打洞原理)我们有
\begin{align*}
&\boldsymbol{A}=-\boldsymbol{I}_n+\left( \begin{matrix}
a_1&		1\\
a_2&		1\\
\vdots&		\vdots\\
a_n&		1\\
\end{matrix} \right) \boldsymbol{I}_{2}^{-1}\left( \begin{matrix}
a_1&		a_2&		\cdots&		a_n\\
1&		1&		\cdots&		1\\
\end{matrix} \right) =\left( -1 \right) ^n\left| \boldsymbol{I}_2 \right|\left| \boldsymbol{I}_n-\left( \begin{matrix}
a_1&		1\\
a_2&		1\\
\vdots&		\vdots\\
a_n&		1\\
\end{matrix} \right) \boldsymbol{I}_{2}^{-1}\left( \begin{matrix}
a_1&		a_2&		\cdots&		a_n\\
1&		1&		\cdots&		1\\
\end{matrix} \right) \right|
\\
&=\left( -1 \right) ^n\left| \begin{matrix}
\boldsymbol{I}_2&		\boldsymbol{\varLambda }'\\
\boldsymbol{\varLambda }&		\boldsymbol{I}_n\\
\end{matrix} \right|=\left( -1 \right) ^n\left| \boldsymbol{I}_n \right|\left| \boldsymbol{I}_2-\left( \begin{matrix}
a_1&		a_2&		\cdots&		a_n\\
1&		1&		\cdots&		1\\
\end{matrix} \right) \boldsymbol{I}_{n}^{-1}\left( \begin{matrix}
a_1&		1\\
a_2&		1\\
\vdots&		\vdots\\
a_n&		1\\
\end{matrix} \right) \right|
\\
&=\left( -1 \right) ^n\left| \boldsymbol{I}_2-\left( \begin{matrix}
\sum_{i=1}^n{a_{i}^{2}}&		\sum_{i=1}^n{a_i}\\
\sum_{i=1}^n{a_i}&		n\\
\end{matrix} \right) \right|=\left( -1 \right) ^n\left[ \left( 1-n \right) \left( 1-\sum_{i=1}^n{a_{i}^{2}} \right) -\left( \sum_{i=1}^n{a_i} \right) ^2 \right] .
\end{align*}
\end{solution}

\begin{example}
计算矩阵\(A\)的行列式的值:
\[
A = 
\begin{pmatrix}
1 + a_1^2 & a_1a_2 & \cdots & a_1a_n\\
a_2a_1 & 1 + a_2^2 & \cdots & a_2a_n\\
\vdots & \vdots & & \vdots\\
a_na_1 & a_na_2 & \cdots & 1 + a_n^2
\end{pmatrix}.
\]
\end{example}
\begin{solution}
注意到\begin{align*}
A-I_n=\left( \begin{array}{c}
a_1\\
a_2\\
\vdots\\
a_n\\
\end{array} \right) \left( \begin{matrix}
a_1&		a_2&		\cdots&		a_n\\
\end{matrix} \right) .
\end{align*}
从而由降价公式可得\begin{align*}
\left| A \right|=\left| I_n+\left( \begin{array}{c}
a_1\\
a_2\\
\vdots\\
a_n\\
\end{array} \right) \left( \begin{matrix}
a_1&		a_2&		\cdots&		a_n\\
\end{matrix} \right) \right|=\left| I_n \right|\left| 1+\left( \begin{matrix}
a_1&		a_2&		\cdots&		a_n\\
\end{matrix} \right) I_{n}^{-1}\left( \begin{array}{c}
a_1\\
a_2\\
\vdots\\
a_n\\
\end{array} \right) \right|=1+\sum_{i=1}^n{a_{i}^{2}}.
\end{align*}
\end{solution}

\begin{example}
计算矩阵\(A\)的行列式的值:
\[
A = 
\begin{pmatrix}
a_1 - b_1 & a_1 - b_2 & \cdots & a_1 - b_n\\
a_2 - b_1 & a_2 - b_2 & \cdots & a_2 - b_n\\
\vdots & \vdots & & \vdots\\
a_n - b_1 & a_n - b_2 & \cdots & a_n - b_n
\end{pmatrix}.
\]
\end{example}
\begin{solution}
注意到\begin{align*}
A=\left( \begin{matrix}
a_1-b_1&		a_1-b_2&		\cdots&		a_1-b_n\\
a_2-b_1&		a_2-b_2&		\cdots&		a_2-b_n\\
\vdots&		\vdots&		&		\vdots\\
a_n-b_1&		a_n-b_2&		\cdots&		a_n-b_n\\
\end{matrix} \right) =\left( \begin{matrix}
a_1&		-1\\
a_2&		-1\\
\vdots&		\vdots\\
a_n&		-1\\
\end{matrix} \right) \left( \begin{matrix}
1&		1&		\cdots&		1\\
b_1&		b_2&		\cdots&		b_n\\
\end{matrix} \right) .
\end{align*}
当$n>2$时,由Cauchy-Binet公式可知$\left| A \right|=0$.当$n=2$时,$\left| A \right|=a_1b_1+a_2b_2-a_1b_2-b_1a_2$.当$n=1$时,$\left| A \right|=a_1-b_1$.
\end{solution}

\begin{example}
求下列矩阵的行列式的值:
\[
A = 
\begin{pmatrix}
0 & 2 & 3 & \cdots & n\\
1 & 0 & 3 & \cdots & n\\
1 & 2 & 0 & \cdots & n\\
\vdots & \vdots & \vdots & & \vdots\\
1 & 2 & 3 & \cdots & 0
\end{pmatrix}.
\]
\end{example}
\begin{solution}
将\(A\)化为
\[
A = 
\begin{pmatrix}
-1 & 0 & \cdots & 0\\
0 & -2 & \cdots & 0\\
\vdots & \vdots & & \vdots\\
0 & 0 & \cdots & -n
\end{pmatrix}
+
\begin{pmatrix}
1\\
1\\
\vdots\\
1
\end{pmatrix}
(1,2,\cdots,n),
\]
利用降阶公式容易求得\(|A| = (-1)^nn!(1 - n)\).
\end{solution}

\begin{proposition}\label{proposition:对角相同分块矩阵行列式计算}
设\(A,B\)是\(n\)阶矩阵,求证:
\[
\begin{vmatrix}
A & B\\
B & A
\end{vmatrix}=|A + B||A - B|.
\]
\end{proposition}
\begin{proof}
将分块矩阵的第二行加到第一行上,再将第二列减去第一列,可得
\[
\begin{pmatrix}
A & B\\
B & A
\end{pmatrix}\to\begin{pmatrix}
A + B & A + B\\
B & A
\end{pmatrix}\to\begin{pmatrix}
A + B & O\\
B & A - B
\end{pmatrix}.
\]
第三类分块初等变换不改变行列式的值,因此可得
\[
\begin{vmatrix}
A & B\\
B & A
\end{vmatrix}=\begin{vmatrix}
A + B & O\\
B & A - B
\end{vmatrix}=|A + B||A - B|.
\]
\end{proof}

\begin{example}
计算:
\[
|A| = 
\begin{vmatrix}
x & y & z & w\\
y & x & w & z\\
z & w & x & y\\
w & z & y & x
\end{vmatrix}.
\]
\end{example}
\begin{solution}
{\color{blue}解法一:}
令
\[
B = 
\begin{pmatrix}
x & y\\
y & x
\end{pmatrix},
C = 
\begin{pmatrix}
z & w\\
w & z
\end{pmatrix},
\]
则\(|A| = 
\begin{vmatrix}
B & C\\
C & B
\end{vmatrix}\).由\hyperref[proposition:对角相同分块矩阵行列式计算]{命题\ref{proposition:对角相同分块矩阵行列式计算}}可得
\begin{align*}
|A|&=|B + C||B - C|
=\begin{vmatrix}
x + z & y + w\\
y + w & x + z
\end{vmatrix}\begin{vmatrix}
x - z & y - w\\
y - w & x - z
\end{vmatrix}\\
&=(x + y + z + w)(x + z - y - w)(x + y - z - w)(x + w - y - z).
\end{align*}
{\color{blue}解法二(求根法):}
\end{solution}

\begin{proposition}\label{proposition:对角相同的复分块矩阵行列式计算}
设\(A,B\)是\(n\)阶复矩阵,求证:
\[
\begin{vmatrix}
A & -B\\
B & A
\end{vmatrix}=|A + \mathrm{i}B||A - \mathrm{i}B|.
\]
\end{proposition}
\begin{proof}
将分块矩阵的第二行乘以\(\mathrm{i}\)加到第一行上,再将第一列乘以\(-\mathrm{i}\)加到第二列上,可得
\[
\begin{pmatrix}
A & -B\\
B & A
\end{pmatrix}\to\begin{pmatrix}
A + \mathrm{i}B & \mathrm{i}A - B\\
B & A
\end{pmatrix}\to\begin{pmatrix}
A + \mathrm{i}B & O\\
B & A - \mathrm{i}B
\end{pmatrix}.
\]
第三类分块初等变换不改变行列式的值,因此可得
\[
\begin{vmatrix}
A & -B\\
B & A
\end{vmatrix}=\begin{vmatrix}
A + \mathrm{i}B & O\\
B & A - \mathrm{i}B
\end{vmatrix}=|A + \mathrm{i}B||A - \mathrm{i}B|.
\]
\end{proof}

\begin{example}
设\(A,B,C,D\)都是\(n\)阶矩阵,求证:
\[
|M| = 
\begin{vmatrix}
A & B & C & D\\
B & A & D & C\\
C & D & A & B\\
D & C & B & A
\end{vmatrix}
= |A + B + C + D||A + B - C - D||A - B + C - D||A - B - C + D|.
\]
\end{example}
\begin{solution}
反复利用\hyperref[proposition:对角相同分块矩阵行列式计算]{命题\ref{proposition:对角相同分块矩阵行列式计算}}的结论可得
\begin{align*}
|M|&=\left|\begin{pmatrix}
A & B\\
B & A
\end{pmatrix}+\begin{pmatrix}
C & D\\
D & C
\end{pmatrix}\right|\cdot\left|\begin{pmatrix}
A & B\\
B & A
\end{pmatrix}-\begin{pmatrix}
C & D\\
D & C
\end{pmatrix}\right|
=\begin{vmatrix}
A + C & B + D\\
B + D & A + C
\end{vmatrix}\cdot\begin{vmatrix}
A - C & B - D\\
B - D & A - C
\end{vmatrix}\\
&=|A + B + C + D||A - B + C - D||A + B - C - D||A - B - C + D|.
\end{align*}
\end{solution}


\begin{example}\label{example:2.28}
设\(A,B\)是\(n\)阶矩阵且\(AB = BA\),求证:
\[
\begin{vmatrix}
A & -B\\
B & A
\end{vmatrix}=|A^2 + B^2|.
\]
\end{example}
\begin{proof}
由\hyperref[proposition:对角相同的复分块矩阵行列式计算]{命题\ref{proposition:对角相同的复分块矩阵行列式计算}}的结论可得
\begin{align*}
\begin{vmatrix}
A & -B\\
B & A
\end{vmatrix}&=|A + \mathrm{i}B|\cdot|A - \mathrm{i}B|
=|(A + \mathrm{i}B)(A - \mathrm{i}B)|
=|A^2 + B^2 - \mathrm{i}(AB - BA)|
=|A^2 + B^2|.
\end{align*}
\end{proof}

\begin{example}
设\(A,B\)是\(n\)阶实矩阵,求证:
\[
\begin{vmatrix}
A & -B\\
B & A
\end{vmatrix}\geq0.
\]
\end{example}
\begin{proof}
注意到\(A,B\)都是实矩阵,故\(\overline{|A + \mathrm{i}B|}=|\overline{A + \mathrm{i}B}|=|A - \mathrm{i}B|\),再由\hyperref[proposition:对角相同的复分块矩阵行列式计算]{命题\ref{proposition:对角相同的复分块矩阵行列式计算}}的结论可得
\begin{align*}
\begin{vmatrix}
A & -B\\
B & A
\end{vmatrix}&=|A + \mathrm{i}B|\cdot|A - \mathrm{i}B|=|A + \mathrm{i}B|\cdot\overline{|A + \mathrm{i}B|}\geq0.
\end{align*}
\end{proof}

\section{摄动法}
\textbf{摄动法的原理}

(1) 证明矩阵问题对非异阵成立.

(2) 对任意的 \(n\) 阶矩阵 \(A\), 由上例可知, 存在一列有理数 \(t_k\rightarrow0\), 使得 \(t_kI_n + A\) 都是非异阵. 验证 \(t_kI_n + A\) 仍满足矩阵问题的条件, 从而该问题对 \(t_kI_n + A\) 成立.

(3) 若矩阵问题关于 \(t_k\) 连续, 则可取极限令 \(t_k\rightarrow0\), 从而得到该问题对一般的矩阵 \(A\) 也成立.

\begin{remark}
\begin{enumerate}
\item 矩阵问题对非异阵成立以及矩阵问题关于 \(t_k\) 连续, 这两个要求缺一不可, 否则将不能使用摄动法进行证明.
\item 验证摄动矩阵仍然满足矩阵问题的条件是必要的. 例如, 若矩阵问题中有 \(AB=-BA\) 这一条, 但 \((t_kI_n + A)B\neq -B(t_kI_n + A)\), 因此便不能使用摄动法.
\item 根据实际问题的需要, 也可以使用其他非异阵来替代 \(I_n\) 对 \(A\) 进行摄动.
\end{enumerate}
\end{remark}
\begin{note}
关于伴随矩阵的问题中经常会使用摄动法.
\end{note}


\begin{proposition}\label{proposition:摄动法基本命题}
设\(A\)是一个\(n\)阶方阵,求证:存在一个正数\(a\),使得对任意的\(0 < t < a\),矩阵\(tI_n+A\)都是非异阵. 
\end{proposition}
\begin{note}
这个命题告诉我们对任意的 \(n\) 阶矩阵 \(A\),经过微小的一维摄动之后,\(tI_n + A\) 总能成为一个非异阵.
\end{note}
\begin{proof}
通过简单的计算可得
\[
|tI_n + A| = t^n + a_1t^{n - 1}+\cdots+a_{n - 1}t + a_n,
\]
这是一个关于未定元\(t\)的\(n\)次多项式. 由\hyperref[proposition:多项式根的有限性]{多项式根的有限性}可知上述多项式至多只有\(n\)个不同的根. 若上述多项式的根都是零,则不妨取\(a = 1\);若上述多项式有非零根, 则令\(a\)为\(|tI_n + A|\)所有非零根的模长的最小值. 因此对任意的\(0 < t_0 < a\), \(t_0\)都不是\(|tI_n + A|\)的根, 即\(|t_0I_n + A|\neq0\), 从而\(t_0I_n + A\)是非异阵.
\end{proof}

\begin{example}
设\(A,B,C,D\)是\(n\)阶矩阵且\(AC = CA\),求证:
\[
\begin{vmatrix}
A & B\\
C & D
\end{vmatrix}=|AD - CB|.
\]
\end{example}
\begin{note}
本题也给出了\hyperref[example:2.28]{例题\ref{example:2.28}}的摄动法证明.
\end{note}
\begin{proof}
若\(A\)为非异阵,则由降阶公式,再结合条件\(AC = CA\)可得
\[
\left|\begin{matrix}
A & B\\
C & D
\end{matrix}\right| = |A|\left|D - CA^{-1}B\right| = \left|AD - ACA^{-1}B\right| = |AD - CB|.
\]
对于一般的方阵\(A\),由命题\(2.11\)可知,存在一列有理数\(t_k\rightarrow 0\),使得\(t_kI_n + A\)是非异阵,并且条件\((t_kI_n + A)C = C(t_kI_n + A)\)仍然成立.于是
\[
\left|\begin{matrix}
t_kI_n + A & B\\
C & D
\end{matrix}\right| = \left|(t_kI_n + A)D - CB\right|.
\]
上式两边都是行列式,其值都是\(t_k\)的多项式,从而都关于\(t_k\)连续.上式两边同时令\(t_k\rightarrow 0\),即有\(\left|\begin{matrix}
A & B\\
C & D
\end{matrix}\right| = |AD - CB|\)成立.
\end{proof}

\section{练习}

\begin{exercise}
设\(A=(a_{ij})\)为\(n\)阶方阵,定义函数\(f(A)=\sum_{i,j = 1}^{n}a_{ij}^2\). 设\(P\)为\(n\)阶可逆矩阵,使得对任意的\(n\)阶方阵\(A\)成立:\(f(PAP^{-1}) = f(A)\). 证明:存在非零常数\(c\),使得\(P'P = cI_n\).
\end{exercise}
\begin{proof}
由假设知\(f(A)=\text{tr}(AA')\),因此
\[
f(PAP^{-1})=\text{tr}(PAP^{-1}(P')^{-1}A'P')=\text{tr}((P'P)A(P'P)^{-1}A')=\text{tr}(AA').
\]
以下设\(P'P=(c_{ij})\),\((P'P)^{-1}=(d_{ij})\). 注意\(P'P\)是对称矩阵,后面要用到. 令\(A = E_{ij}\),其中$1\leq i,j\leq n$.并将其代入$(P'P)A(P'P)^{-1}A'$可得
\begin{align*}
&(P'P)A(P'P)^{-1}A'=(P'P)E_{ij}(P'P)^{-1}E_{ji}
\\
&=\bordermatrix{%
&    &       &             j&     &
\cr
&    &		&		c_{1i}&		&		\cr
&    &		&		c_{2i}&		&		\cr
&    &		&		\vdots&		&		\cr
&    &		&		c_{ii}&		&		\cr
&    &		&		\vdots&		&		\cr
&    &		&		c_{ni}&		&		\cr
}\bordermatrix{%
&    &       &             i&     &
\cr
&    &		&		d_{1j}&		&		\cr
&    &		&		d_{2j}&		&		\cr
&    &		&		\vdots&		&		\cr
&    &		&		d_{jj}&		&		\cr
&    &		&		\vdots&		&		\cr
&    &		&		d_{nj}&		&		\cr
}=\bordermatrix{%
&    &       &             i&     &
\cr
&    &		&		c_{1i}d_{jj}&		&		\cr
&    &		&		c_{2i}d_{jj}&		&		\cr
&    &		&		\vdots&		&		\cr
&    &		&		c_{ii}d_{jj}&		&		\cr
&    &		&		\vdots&		&		\cr
&    &		&		c_{ni}d_{jj}&		&		\cr
}
\end{align*}
于是$\mathrm{tr}\left( \left( P'P \right) A\left( P'P \right) ^{-1}A' \right) =c_{ii}d_{jj}$.
而$\mathrm{tr}\left( AA\prime \right) =\mathrm{tr}\left( E_{ij}E_{ji} \right) =\mathrm{tr}\left( E_{ii} \right) =1$.
则由$\text{tr}((P'P)A(P'P)^{-1}A')=\text{tr}(AA')$可知
\begin{align}\label{equation:eq542}
c_{ii}d_{jj}=1. 
\end{align}
再令\(A = E_{ij}+E_{kl}\),其中$1\leq i,j,k,l\leq n$.不妨设$k\geq i,l\geq j$,将其代入$(P'P)A(P'P)^{-1}A'$可得
\begin{align*}
&(P'P)A(P'P)^{-1}A'=(P'P)(E_{ij}+E_{kl})(P'P)^{-1}(E_{ji}+E_{lk})
\\
&=\left[\bordermatrix{%
&    &       &             j&     &
\cr
&    &		&		c_{1i}&		&		\cr
&    &		&		c_{2i}&		&		\cr
&    &		&		\vdots&		&		\cr
&    &		&		c_{ii}&		&		\cr
&    &		&		\vdots&		&		\cr
&    &		&		c_{ki}&		&		\cr
&    &		&		\vdots&		&		\cr
&    &		&		c_{ni}&		&		\cr
}+\bordermatrix{%
&    &       &             l&     &
\cr
&    &		&		c_{1k}&		&		\cr
&    &		&		c_{2k}&		&		\cr
&    &		&		\vdots&		&		\cr
&    &		&		c_{ik}&		&		\cr
&    &		&		\vdots&		&		\cr
&    &		&		c_{kk}&		&		\cr
&    &		&		\vdots&		&		\cr
&    &		&		c_{nk}&		&		\cr
}\right]\left[\bordermatrix{%
&    &       &             i&     &
\cr
&    &		&		d_{1j}&		&		\cr
&    &		&		d_{2j}&		&		\cr
&    &		&		\vdots&		&		\cr
&    &		&		d_{jj}&		&		\cr
&    &		&		\vdots&		&		\cr
&    &		&		d_{lj}&		&		\cr
&    &		&		\vdots&		&		\cr
&    &		&		d_{nj}&		&		\cr
}+\bordermatrix{%
&    &       &             k&     &
\cr
&    &		&		d_{1l}&		&		\cr
&    &		&		d_{2l}&		&		\cr
&    &		&		\vdots&		&		\cr
&    &		&		d_{jl}&		&		\cr
&    &		&		\vdots&		&		\cr
&    &		&		d_{ll}&		&		\cr
&    &		&		\vdots&		&		\cr
&    &		&		d_{nl}&		&		\cr
}\right]\\
&=\bordermatrix{%
&    &       &   j&  &          l&     &
\cr
&    &		&	c_{1i}& \cdots&c_{1k}&		&		\cr
&    &		&	c_{2i}&	\cdots&c_{2k}&		&		\cr
&    &		&	\vdots&	&\vdots&		&		\cr
&    &		&	c_{ii}&	\cdots&c_{ik}&		&		\cr
&    &		&	\vdots&	&\vdots&		&		\cr
&    &		&	c_{ki}&	\cdots&c_{kk}&		&		\cr
&    &		&	\vdots&	&\vdots&		&		\cr
&    &		&	c_{ni}&	\cdots&c_{nk}&		&		\cr
}\bordermatrix{%
&    &       &   i&  &          k&     &
\cr
&    &		&	d_{1j}& \cdots&	d_{1l}&		&		\cr
&    &		&	d_{2j}&	\cdots&d_{2l}&		&		\cr
&    &		&	\vdots&	&\vdots&		&		\cr
&    &		&	d_{jj}&	\cdots&d_{jl}&		&		\cr
&    &		&	\vdots&	&\vdots&		&		\cr
&    &		&	d_{lj}&	\cdots&d_{ll}&		&		\cr
&    &		&	\vdots&	&\vdots&		&		\cr
&    &		&	d_{nj}&	\cdots&d_{nl}&		&		\cr
}=\bordermatrix{%
&    &       &   i&  &          k&     &
\cr
&    &		&	c_{1i}d_{jj}+c_{1k}d_{lj}& \cdots&	c_{1i}d_{jl}+c_{1k}d_{ll}&		&		\cr
&    &		&	c_{2i}d_{jj}+c_{2k}d_{lj}& \cdots&	c_{2i}d_{jl}+c_{2k}d_{ll}&		&		\cr
&    &		&	\vdots&	&\vdots&		&		\cr
&    &		&	c_{ii}d_{jj}+c_{ik}d_{lj}& \cdots&	c_{ii}d_{jl}+c_{ik}d_{ll}&		&		\cr
&    &		&	\vdots&	&\vdots&		&		\cr
&    &		&	c_{ki}d_{jj}+c_{kk}d_{lj}& \cdots&	c_{ki}d_{jl}+c_{kk}d_{ll}&		&		\cr
&    &		&	\vdots&	&\vdots&		&		\cr
&    &		&	c_{ni}d_{jj}+c_{nk}d_{lj}& \cdots&	c_{ni}d_{jl}+c_{nk}d_{ll}&		&		\cr
}
\end{align*}
从而$\mathrm{tr}\left( \left( P'P \right) A\left( P'P \right) ^{-1}A' \right) =c_{ii}d_{jj}+c_{kk}d_{ll}+c_{ki}d_{jl}+c_{ik}d_{lj}$.而
\begin{align*}
\mathrm{tr}\left( AA' \right) =\mathrm{tr}\left( \left( E_{ij}+E_{kl} \right) \left( E_{ji}+E_{lk} \right) \right) =\mathrm{tr}\left( E_{ij}E_{ji}+E_{ij}E_{lk}+E_{kl}E_{ji}+E_{kl}E_{lk} \right)=2 + 2\delta_{ik}\delta_{jl}.    
\end{align*}
于是由$\text{tr}((P'P)A(P'P)^{-1}A')=\text{tr}(AA')$可知
\begin{align}\label{equation:eq745}
c_{ii}d_{jj}+c_{kk}d_{ll}+c_{ki}d_{jl}+c_{ik}d_{lj}=2 + 2\delta_{ik}\delta_{jl},  
\end{align}
其中\(\delta_{ik}\)是Kronecker符号. 由上述\eqref{equation:eq542}\eqref{equation:eq745}两式可得
\[
c_{ki}d_{jl}+c_{ik}d_{lj}=2\delta_{ik}\delta_{jl}.
\]
在上式中令\(j = l\),\(i\neq k\),注意到\(d_{jj}\neq0\),故有\(c_{ik}+c_{ki}=0\),又因为\(P'P\)是对称矩阵,所以\(c_{ik}=c_{ki}\).故\(c_{ik}=0\),\(\forall i\neq k\). 于是\(P'P\)是一个对角矩阵,从而由\eqref{equation:eq542}式可得\(d_{jj}=c_{jj}^{-1}\),由此可得\(c_{ii}=c_{jj}\),\(\forall i,j\). 因此\(P'P = cI_n\),其中\(c = c_{11}\neq0\).
\end{proof}




\chapter{线性空间与线性方程组}

\section{向量的线性关系}

\begin{theorem}\label{theorem:向量的线性关系定理1}
设\(\boldsymbol{\alpha}_1,\boldsymbol{\alpha}_2,\cdots,\boldsymbol{\alpha}_m,\boldsymbol{\beta}\)是线性空间\(V\)中的向量.

(1) 若\(\boldsymbol{\alpha}_1,\boldsymbol{\alpha}_2,\cdots,\boldsymbol{\alpha}_m\)线性相关,则任意一组包含这组向量的向量组必线性相关;若\(\boldsymbol{\alpha}_1,\boldsymbol{\alpha}_2,\cdots,\boldsymbol{\alpha}_m\)线性无关,则从这组向量中任意取出一组向量必线性无关.

(2) 向量组\(\boldsymbol{\alpha}_1,\boldsymbol{\alpha}_2,\cdots,\boldsymbol{\alpha}_m\)线性相关的充要条件是其中至少有一个向量可以表示为其余向量的线性组合.

(3) 若\(\boldsymbol{\beta}\)可表示为\(\boldsymbol{\alpha}_1,\boldsymbol{\alpha}_2,\cdots,\boldsymbol{\alpha}_m\)的线性组合,即
\[
\boldsymbol{\beta}=k_1\boldsymbol{\alpha}_1 + k_2\boldsymbol{\alpha}_2+\cdots + k_m\boldsymbol{\alpha}_m,
\]
则表示唯一的充要条件是向量\(\boldsymbol{\alpha}_1,\boldsymbol{\alpha}_2,\cdots,\boldsymbol{\alpha}_m\)线性无关.
\end{theorem}
\begin{proof}

\end{proof}

\begin{theorem}\label{theorem:向量的线性关系定理2}
(1)设\(A,B\)是两组向量,\(A\)含有\(r\)个向量,\(B\)含有\(s\)个向量,且\(A\)中每个向量均可用\(B\)中向量线性表示. 若\(A\)中向量线性无关,则\(r\leq s\).

(2)设\(A,B\)是两组向量,\(A\)含有\(r\)个向量,\(B\)含有\(s\)个向量,且\(A\)中每个向量均可用\(B\)中向量线性表示. 若\(r > s\),则\(A\)中向量线性相关.
\end{theorem}
\begin{proof}
(2)是(1)的逆否命题,因此我们只证明(1).
\end{proof}



\begin{proposition}\label{proposition:阶梯形矩阵的非零行对应其列向量的极大无关组}
设\(A\)是\(m\times n\)阶梯形矩阵,证明:\(A\)的秩等于其非零行的个数,且阶梯点所在的列向量是\(A\)的列向量的极大无关组.
\end{proposition}
\begin{proof}
设
\[
A=\left( \begin{matrix}
0&		\cdots&		a_{1k_1}&		\cdots&		\cdots&		\cdots&		\cdots&		\cdots\\
0&		\cdots&		0&		\cdots&		a_{2k_2}&		\cdots&		\cdots&		\cdots\\
\vdots&		&		\vdots&		&		\vdots&		&		\vdots&		\vdots\\
0&		\cdots&		0&		\cdots&		0&		\cdots&		a_{rk_r}&		\cdots\\
0&		\cdots&		0&		\cdots&		0&		\cdots&		0&		\cdots\\
\vdots&		&		\vdots&		&		\vdots&		&		\vdots&		\\
0&		\cdots&		0&		\cdots&		0&		\cdots&		0&		\cdots\\
\end{matrix} \right) .
\]
其中\(a_{1k_1},a_{2k_2},\cdots,a_{rk_r}\)是\(A\)的阶梯点. 设\(\boldsymbol{\alpha}_1,\boldsymbol{\alpha}_2,\cdots,\boldsymbol{\alpha}_r\)是\(A\)的前\(r\)行,我们先证明它们线性无关. 设
\[
c_1\boldsymbol{\alpha}_1 + c_2\boldsymbol{\alpha}_2+\cdots + c_r\boldsymbol{\alpha}_r=\boldsymbol{0}
\]
其中\(c_1,c_2,\cdots,c_r\)是常数. 上式是关于\(n\)维行向量的等式,先考察行向量的第\(k_1\)分量,可得\(c_1a_{1k_1}=0\). 因为\(a_{1k_1}\neq0\),故\(c_1 = 0\);再依次考察行向量的第\(k_2,\cdots,k_r\)分量,最后可得\(c_1 = c_2=\cdots = c_r = 0\). 因此\(\boldsymbol{\alpha}_1,\boldsymbol{\alpha}_2,\cdots,\boldsymbol{\alpha}_r\)线性无关,从而\(A\)的秩等于\(r\),即其非零行的个数.

再将\(r\)个阶梯点所在的列向量取出,拼成一个新的矩阵:
\[
B=\left( \begin{matrix}
a_{1k_1}&		\cdots&		\cdots&		\cdots\\
0&		a_{2k_2}&		\cdots&		\cdots\\
\vdots&		\vdots&		&		\vdots\\
0&		0&		\cdots&		a_{rk_r}\\
0&		0&		\cdots&		0\\
\vdots&		\vdots&		&		\vdots\\
0&		0&		\cdots&		0\\
\end{matrix} \right) .
\]
采用相同的方法可证明矩阵\(B\)的前\(r\)行线性无关,因此\(\text{r}(B)=r\),从而阶梯点所在的列向量组的秩也等于\(r\). 又因为\(\text{r}(A)=r\),故它们是\(A\)的列向量的极大无关组.
\end{proof}

\begin{proposition}\label{proposition:乘可逆矩阵不改变向量组的极大无关组的位置}
设\(A = (\boldsymbol{\alpha}_1,\boldsymbol{\alpha}_2,\cdots,\boldsymbol{\alpha}_n)\)是一个\(m\times n\)矩阵,\(\boldsymbol{\alpha}_1,\boldsymbol{\alpha}_2,\cdots,\boldsymbol{\alpha}_n\)是列向量. \(P\)是一个\(m\)阶可逆矩阵,\(B = PA=(\boldsymbol{\beta}_1,\boldsymbol{\beta}_2,\cdots,\boldsymbol{\beta}_n)\),其中\(\boldsymbol{\beta}_j = P\boldsymbol{\alpha}_j(1\leq j\leq n)\). 证明:若\(\boldsymbol{\alpha}_{i_1},\boldsymbol{\alpha}_{i_2},\cdots,\boldsymbol{\alpha}_{i_r}\)是\(A\)的列向量的极大无关组,则\(\boldsymbol{\beta}_{i_1},\boldsymbol{\beta}_{i_2},\cdots,\boldsymbol{\beta}_{i_r}\)是\(B\)的列向量的极大无关组.
\end{proposition}
\begin{proof}
先证明向量组\(\boldsymbol{\beta}_{i_1},\boldsymbol{\beta}_{i_2},\cdots,\boldsymbol{\beta}_{i_r}\)线性无关. 设
\[
c_1\boldsymbol{\beta}_{i_1}+c_2\boldsymbol{\beta}_{i_2}+\cdots + c_r\boldsymbol{\beta}_{i_r}=\boldsymbol{0},
\]
即
\[
c_1P\boldsymbol{\alpha}_{i_1}+c_2P\boldsymbol{\alpha}_{i_2}+\cdots + c_rP\boldsymbol{\alpha}_{i_r}=\boldsymbol{0}.
\]
已知\(P\)是可逆矩阵,因此
\[
c_1\boldsymbol{\alpha}_{i_1}+c_2\boldsymbol{\alpha}_{i_2}+\cdots + c_r\boldsymbol{\alpha}_{i_r}=\boldsymbol{0}.
\]
而向量组\(\boldsymbol{\alpha}_{i_1},\boldsymbol{\alpha}_{i_2},\cdots,\boldsymbol{\alpha}_{i_r}\)线性无关,故\(c_1 = c_2=\cdots = c_r = 0\),这证明了向量组\(\boldsymbol{\beta}_{i_1},\boldsymbol{\beta}_{i_2},\cdots,\boldsymbol{\beta}_{i_r}\)线性无关. 要证这是\(B\)的列向量的极大无关组,只需证明\(B\)的任意一个列向量都是这些向量的线性组合即可. 设\(\boldsymbol{\beta}_j\)是\(B\)的任意一个列向量,则\(\boldsymbol{\beta}_j = P\boldsymbol{\alpha}_j\). 因为\(\boldsymbol{\alpha}_{i_1},\boldsymbol{\alpha}_{i_2},\cdots,\boldsymbol{\alpha}_{i_r}\)是\(A\)的列向量的极大无关组,故\(\boldsymbol{\alpha}_j\)可用\(\boldsymbol{\alpha}_{i_1},\boldsymbol{\alpha}_{i_2},\cdots,\boldsymbol{\alpha}_{i_r}\)线性表示. 不妨设
\[
\boldsymbol{\alpha}_j = b_1\boldsymbol{\alpha}_{i_1}+b_2\boldsymbol{\alpha}_{i_2}+\cdots + b_r\boldsymbol{\alpha}_{i_r},
\]
则
\[
P\boldsymbol{\alpha}_j = b_1P\boldsymbol{\alpha}_{i_1}+b_2P\boldsymbol{\alpha}_{i_2}+\cdots + b_rP\boldsymbol{\alpha}_{i_r},
\]
即
\[
\boldsymbol{\beta}_j = b_1\boldsymbol{\beta}_{i_1}+b_2\boldsymbol{\beta}_{i_2}+\cdots + b_r\boldsymbol{\beta}_{i_r}.
\]
\end{proof}

\begin{corollary}\label{corollary:线性无关向量组乘可逆矩阵仍然线性无关}
设\(n\)维向量\(\boldsymbol{\alpha}_1,\boldsymbol{\alpha}_2,\cdots,\boldsymbol{\alpha}_m\)线性无关,\(A\)为\(n\)阶可逆矩阵,求证:\(A\boldsymbol{\alpha}_1,A\boldsymbol{\alpha}_2,\cdots,A\boldsymbol{\alpha}_m\)线性无关.
\end{corollary}
\begin{proof}
由\hyperref[proposition:乘可逆矩阵不改变向量组的极大无关组的位置]{命题\ref{proposition:乘可逆矩阵不改变向量组的极大无关组的位置}}即得.
\end{proof}

\begin{proposition}\label{proposition:线性无关向量组的命题1}
设\(\boldsymbol{\alpha}_1,\boldsymbol{\alpha}_2,\cdots,\boldsymbol{\alpha}_m\)是线性空间\(V\)中一组线性无关的向量,\(\boldsymbol{\beta}\)是\(V\)中的向量. 求证:或者\(\boldsymbol{\alpha}_1,\boldsymbol{\alpha}_2,\cdots,\boldsymbol{\alpha}_m,\boldsymbol{\beta}\)线性无关,或者\(\boldsymbol{\beta}\)是\(\boldsymbol{\alpha}_1,\boldsymbol{\alpha}_2,\cdots,\boldsymbol{\alpha}_m\)的线性组合.
\end{proposition}
\begin{proof}
若\(\boldsymbol{\alpha}_1,\boldsymbol{\alpha}_2,\cdots,\boldsymbol{\alpha}_m,\boldsymbol{\beta}\)线性无关,则结论得证. 若\(\boldsymbol{\alpha}_1,\boldsymbol{\alpha}_2,\cdots,\boldsymbol{\alpha}_m,\boldsymbol{\beta}\)线性相关,则存在不全为零的数\(c_1,c_2,\cdots,c_m,d\),使得
\[
c_1\boldsymbol{\alpha}_1 + c_2\boldsymbol{\alpha}_2+\cdots + c_m\boldsymbol{\alpha}_m + d\boldsymbol{\beta}=\boldsymbol{0}.
\]
若\(d = 0\),则\(c_1,c_2,\cdots,c_m\)不全为零且\(c_1\boldsymbol{\alpha}_1 + c_2\boldsymbol{\alpha}_2+\cdots + c_m\boldsymbol{\alpha}_m=\boldsymbol{0}\),这与\(\boldsymbol{\alpha}_1,\boldsymbol{\alpha}_2,\cdots,\boldsymbol{\alpha}_m\)线性无关矛盾. 因此\(d\neq0\),从而
\[
\boldsymbol{\beta}=-\frac{c_1}{d}\boldsymbol{\alpha}_1-\frac{c_2}{d}\boldsymbol{\alpha}_2-\cdots-\frac{c_m}{d}\boldsymbol{\alpha}_m,
\]
即\(\boldsymbol{\beta}\)是\(\boldsymbol{\alpha}_1,\boldsymbol{\alpha}_2,\cdots,\boldsymbol{\alpha}_m\)的线性组合.
\end{proof}

\begin{corollary}\label{corollary:线性无关向量组的命题1}
若\(\boldsymbol{\alpha}_1,\boldsymbol{\alpha}_2,\cdots,\boldsymbol{\alpha}_m\)线性无关且\(\boldsymbol{\beta}\notin L(\boldsymbol{\alpha}_1,\boldsymbol{\alpha}_2,\cdots,\boldsymbol{\alpha}_m)\),则\(\boldsymbol{\alpha}_1,\boldsymbol{\alpha}_2,\cdots,\boldsymbol{\alpha}_m,\boldsymbol{\beta}\)线性无关. 
\end{corollary}
\begin{note}
这个推论与\hyperref[proposition:线性无关向量组的命题1]{上一个命题\ref{proposition:线性无关向量组的命题1}}等价.虽然这个等价命题很简单,但后面经常会用到.
\end{note}

\begin{proposition}\label{proposition:线性无关向量组的命题2}
设向量\(\boldsymbol{\beta}\)可由向量\(\boldsymbol{\alpha}_1,\boldsymbol{\alpha}_2,\cdots,\boldsymbol{\alpha}_m\)线性表示,但不能由其中任何个数少于\(m\)的部分向量线性表示,则这\(m\)个向量线性无关.
\end{proposition}
\begin{proof}
用反证法,设\(\boldsymbol{\alpha}_1,\boldsymbol{\alpha}_2,\cdots,\boldsymbol{\alpha}_m\)线性相关,则至少有一个向量是其余向量的线性组合. 不妨设\(\boldsymbol{\alpha}_m\)是\(\boldsymbol{\alpha}_1,\boldsymbol{\alpha}_2,\cdots,\boldsymbol{\alpha}_{m - 1}\)的线性组合,则由线性组合的传递性可知,\(\boldsymbol{\beta}\)也是\(\boldsymbol{\alpha}_1,\boldsymbol{\alpha}_2,\cdots,\boldsymbol{\alpha}_{m - 1}\)的线性组合,这与假设矛盾. 
\end{proof}

\begin{proposition}\label{proposition:线性无关向量组的命题3}
设线性空间\(V\)中向量\(\boldsymbol{\alpha}_1,\boldsymbol{\alpha}_2,\cdots,\boldsymbol{\alpha}_r\)线性无关,已知有序向量组\(\{\boldsymbol{\beta},\boldsymbol{\alpha}_1,\boldsymbol{\alpha}_2,\cdots,\boldsymbol{\alpha}_r\}\)线性相关,求证:最多只有一个\(\boldsymbol{\alpha}_i\)可以表示为前面向量的线性组合.
\end{proposition}
\begin{proof}
用反证法,设存在\(1\leq i<j\leq r\),使得
\begin{align*}
\boldsymbol{\alpha}_i&=b\boldsymbol{\beta}+a_1\boldsymbol{\alpha}_1 + a_2\boldsymbol{\alpha}_2+\cdots + a_{i - 1}\boldsymbol{\alpha}_{i - 1},\\
\boldsymbol{\alpha}_j&=d\boldsymbol{\beta}+c_1\boldsymbol{\alpha}_1 + c_2\boldsymbol{\alpha}_2+\cdots + c_{j - 1}\boldsymbol{\alpha}_{j - 1}.
\end{align*}
由于\(\boldsymbol{\alpha}_1,\boldsymbol{\alpha}_2,\cdots,\boldsymbol{\alpha}_r\)线性无关,故\(b\neq0\).若$b=0$,则\(\boldsymbol{\alpha}_i\)是\(\boldsymbol{\alpha}_1,\boldsymbol{\alpha}_2,\cdots,\boldsymbol{\alpha}_{i - 1}\)的线性组合,这与\(\boldsymbol{\alpha}_1,\boldsymbol{\alpha}_2,\cdots,\boldsymbol{\alpha}_r\)线性无关矛盾.将第一个等式乘以\(-\frac{d}{b}\)加到第二个等式上,可得\(\boldsymbol{\alpha}_j\)是\(\boldsymbol{\alpha}_1,\boldsymbol{\alpha}_2,\cdots,\boldsymbol{\alpha}_{j - 1}\)的线性组合,这与\(\boldsymbol{\alpha}_1,\boldsymbol{\alpha}_2,\cdots,\boldsymbol{\alpha}_r\)线性无关矛盾.
\end{proof}

\begin{proposition}\label{proposition:线性无关向量组的命题4}
设\(A\)是\(n\times m\)矩阵,\(B\)是\(m\times n\)矩阵,满足\(AB = I_n\),求证:\(B\)的\(n\)个列向量线性无关.
\end{proposition}
\begin{note}
实际上,由\(AB = I_n\)可知$A,B$互为逆矩阵,从而$A,B$满秩,结论得证.
\end{note}
\begin{proof}
设\(B = (\boldsymbol{\beta}_1,\boldsymbol{\beta}_2,\cdots,\boldsymbol{\beta}_n)\)为列分块,则\(AB=(A\boldsymbol{\beta}_1,A\boldsymbol{\beta}_2,\cdots,A\boldsymbol{\beta}_n)\). 由\(AB = I_n\)可得\(A\boldsymbol{\beta}_i=\boldsymbol{e}_i(1\leq i\leq n)\),其中\(\boldsymbol{e}_i\)是\(n\)维标准单位列向量. 设
\[
c_1\boldsymbol{\beta}_1 + c_2\boldsymbol{\beta}_2+\cdots + c_n\boldsymbol{\beta}_n=\boldsymbol{0},
\]
上式两边同时左乘\(A\),可得
\[
\boldsymbol{0}=c_1A\boldsymbol{\beta}_1 + c_2A\boldsymbol{\beta}_2+\cdots + c_nA\boldsymbol{\beta}_n=c_1\boldsymbol{e}_1 + c_2\boldsymbol{e}_2+\cdots + c_n\boldsymbol{e}_n=(c_1,c_2,\cdots,c_n)',
\]
因此\(c_1 = c_2=\cdots = c_n = 0\),即\(B\)的\(n\)个列向量\(\boldsymbol{\beta}_1,\boldsymbol{\beta}_2,\cdots,\boldsymbol{\beta}_n\)线性无关.
\end{proof}

\begin{proposition}[缩短向量与延伸向量]\label{proposition:线性相关向量组的缩短组也线性相关}
1.设\(\{\boldsymbol{\alpha}_i=(a_{i1},a_{i2},\cdots,a_{in}),1\leq i\leq m\}\)是一组\(n\)维行向量,\(1\leq j_1<j_2<\cdots<j_t\leq n\)是给定的\(t(t < n)\)个指标. 定义\(\widetilde{\boldsymbol{\alpha}}_i=(a_{ij_1},a_{ij_2},\cdots,a_{ij_t})\),称\(\widetilde{\boldsymbol{\alpha}}_i\)为\(\boldsymbol{\alpha}_i\)的\(t\)维缩短向量. 则

(1) 若\(\boldsymbol{\alpha}_1,\boldsymbol{\alpha}_2,\cdots,\boldsymbol{\alpha}_m\)线性相关,则\(\widetilde{\boldsymbol{\alpha}}_1,\widetilde{\boldsymbol{\alpha}}_2,\cdots,\widetilde{\boldsymbol{\alpha}}_m\)也线性相关;

(2) 设\(n\)维行向量\(\boldsymbol{\alpha}=(a_1,a_2,\cdots,a_n)\)是\(\boldsymbol{\alpha}_1,\boldsymbol{\alpha}_2,\cdots,\boldsymbol{\alpha}_m\)的线性组合,则\(\widetilde{\boldsymbol{\alpha}}\)也是\(\widetilde{\boldsymbol{\alpha}}_1,\widetilde{\boldsymbol{\alpha}}_2,\cdots,\widetilde{\boldsymbol{\alpha}}_m\)的线性组合.

2.设\(\{\boldsymbol{\alpha}_i=(a_{i1},a_{i2},\cdots,a_{in}),1\leq i\leq m\}\)是一组\(n\)维行向量,\(j_1,j_2,\cdots,j_t\geq 1\)是给定的\(t(t > n)\)个指标. 定义\(\overline{\boldsymbol{\alpha}}_i=(a_{ij_1},a_{ij_2},\cdots,a_{ij_t})\),称\(\overline{\boldsymbol{\alpha}}_i\)为\(\boldsymbol{\alpha}_i\)的\(t\)维延伸向量.则

若\(\boldsymbol{\alpha}_1,\boldsymbol{\alpha}_2,\cdots,\boldsymbol{\alpha}_m\)线性无关,则\(\overline{\boldsymbol{\alpha}}_1,\overline{\boldsymbol{\alpha}}_2,\cdots,\overline{\boldsymbol{\alpha}}_m\)也线性无关.
\end{proposition}
\begin{note}
这个命题告诉我们:\textbf{线性相关向量组的任意缩短组也是线性相关的,线性无关向量组的任意延伸组也是线性无关的.}
\end{note}
\begin{proof}
1. (1) 由假设存在不全为零的数\(c_1,c_2,\cdots,c_m\),使得
\[
\boldsymbol{0}=c_1\boldsymbol{\alpha}_1 + c_2\boldsymbol{\alpha}_2+\cdots + c_m\boldsymbol{\alpha}_m=\left(\sum_{i = 1}^{m}c_ia_{i1},\sum_{i = 1}^{m}c_ia_{i2},\cdots,\sum_{i = 1}^{m}c_ia_{in}\right).
\]
在等式两边同时取\(t\)维缩短向量,可得
\[
\boldsymbol{0}=\left(\sum_{i = 1}^{m}c_ia_{ij_1},\sum_{i = 1}^{m}c_ia_{ij_2},\cdots,\sum_{i = 1}^{m}c_ia_{ij_t}\right)=c_1\widetilde{\boldsymbol{\alpha}}_1 + c_2\widetilde{\boldsymbol{\alpha}}_2+\cdots + c_m\widetilde{\boldsymbol{\alpha}}_m,
\]
从而结论成立.

(2) 设\(\boldsymbol{\alpha}=c_1\boldsymbol{\alpha}_1 + c_2\boldsymbol{\alpha}_2+\cdots + c_m\boldsymbol{\alpha}_m\),则
\[
\boldsymbol{\alpha}=c_1\boldsymbol{\alpha}_1 + c_2\boldsymbol{\alpha}_2+\cdots + c_m\boldsymbol{\alpha}_m=\left(\sum_{i = 1}^{m}c_ia_{i1},\sum_{i = 1}^{m}c_ia_{i2},\cdots,\sum_{i = 1}^{m}c_ia_{in}\right).
\]
在等式两边同时取\(t\)维缩短向量,可得
\[
\widetilde{\boldsymbol{\alpha}}=\left(\sum_{i = 1}^{m}c_ia_{ij_1},\sum_{i = 1}^{m}c_ia_{ij_2},\cdots,\sum_{i = 1}^{m}c_ia_{ij_t}\right)=c_1\widetilde{\boldsymbol{\alpha}}_1 + c_2\widetilde{\boldsymbol{\alpha}}_2+\cdots + c_m\widetilde{\boldsymbol{\alpha}}_m,
\]
从而结论成立. 

2.这个命题就是1(1)的逆否命题,从而结论成立.
\end{proof}

\begin{example}\label{example:3.1.1.1}
设\(V\)是实数域上连续函数全体构成的实线性空间,求证下列函数线性无关:

(1) \(\sin x,\sin 2x,\cdots,\sin nx\);

(2) \(1,\cos x,\cos 2x,\cdots,\cos nx\);

(3) \(1,\sin x,\cos x,\sin 2x,\cos 2x,\cdots,\sin nx,\cos nx\).
\end{example}
\begin{proof}
{\color{blue}证法一:}根据向量线性无关的基本性质,我们只要证明(3)即可. 对\(n\)进行归纳,当\(n = 0\)时,显然\(1\)作为一个函数线性无关. 假设命题对小于\(n\)的自然数成立,现证明等于\(n\)的情形. 设
\[
a + b_1\sin x + c_1\cos x + b_2\sin 2x + c_2\cos 2x+\cdots + b_n\sin nx + c_n\cos nx = 0,
\]
其中\(a,b_i,c_i\)都是实数. 对上式两次求导,可得
\[
-b_1\sin x - c_1\cos x - 4b_2\sin 2x - 4c_2\cos 2x-\cdots - n^2b_n\sin nx - n^2c_n\cos nx = 0,
\]
再将第一个式子乘以\(n^2\)加到第二个式子上,可得
\[
an^2+\sum_{i = 1}^{n - 1}b_i(n^2 - i^2)\sin ix+\sum_{i = 1}^{n - 1}c_i(n^2 - i^2)\cos ix = 0,
\]
由归纳假设即得\(a = b_1 = c_1=\cdots = b_{n - 1} = c_{n - 1} = 0\). 将此结论代入第一个式子可得\(b_n\sin nx + c_n\cos nx = 0\). 若\(b_n\neq0\) (\(c_n\neq0\)),则\(\tan nx=-c_n/b_n\) (\(\cot nx=-b_n/c_n\))为常数($\tan nx,\cot nx$都不是常函数),矛盾. 因此,\(b_n = c_n = 0\).

{\color{blue}证法二:} 设
\[
f(x)=a + b_1\sin x + c_1\cos x + b_2\sin 2x + c_2\cos 2x+\cdots + b_n\sin nx + c_n\cos nx = 0,
\]
其中\(a,b_i,c_i\)都是实数. 依次设\(g(x)=1,\sin x,\cos x,\sin 2x,\cos 2x,\cdots,\sin nx,\cos nx\),并分别计算定积分\(\int_{0}^{2\pi}f(x)g(x)\mathrm{d}x\),可得\(a = b_1 = c_1=\cdots = b_n = c_n = 0\). 
\end{proof}



\begin{proposition}\label{proposition:线性无关向量组的命题5}
设向量组\(\boldsymbol{\alpha}_1,\boldsymbol{\alpha}_2,\cdots,\boldsymbol{\alpha}_r\)线性无关,又
\[
\begin{cases}
\boldsymbol{\beta}_1 = a_{11}\boldsymbol{\alpha}_1 + a_{12}\boldsymbol{\alpha}_2+\cdots + a_{1r}\boldsymbol{\alpha}_r,\\
\boldsymbol{\beta}_2 = a_{21}\boldsymbol{\alpha}_1 + a_{22}\boldsymbol{\alpha}_2+\cdots + a_{2r}\boldsymbol{\alpha}_r,\\
\cdots\cdots\cdots\cdots\\
\boldsymbol{\beta}_r = a_{r1}\boldsymbol{\alpha}_1 + a_{r2}\boldsymbol{\alpha}_2+\cdots + a_{rr}\boldsymbol{\alpha}_r.
\end{cases}
\]
则:\(\boldsymbol{\beta}_1,\boldsymbol{\beta}_2,\cdots,\boldsymbol{\beta}_r\)线性相关的充要条件是系数矩阵\(A=(a_{ij})_{r\times r}\)的行列式为零.
\end{proposition}
\begin{proof}
记\(A\)的行向量为\(\boldsymbol{\gamma}_1,\boldsymbol{\gamma}_2,\cdots,\boldsymbol{\gamma}_r\). 若\(|A| = 0\),则\(A\)的行向量线性相关,即存在不全为零的\(r\)个数\(c_1,c_2,\cdots,c_r\),使得
\[
c_1\boldsymbol{\gamma}_1 + c_2\boldsymbol{\gamma}_2+\cdots + c_r\boldsymbol{\gamma}_r=\boldsymbol{0}.
\]
设$\boldsymbol{\alpha }=\left( \boldsymbol{\alpha }_1,\boldsymbol{\alpha }_2,\cdots ,\boldsymbol{\alpha }_m \right)'$,则经简单计算可得
\[
c_1\boldsymbol{\beta }_1+c_2\boldsymbol{\beta }_2+\cdots +c_r\boldsymbol{\beta }_r=c_1\boldsymbol{\gamma }_1\boldsymbol{\alpha }+c_1\boldsymbol{\gamma }_2\boldsymbol{\alpha }+\cdots +c_1\boldsymbol{\gamma }_r\boldsymbol{\alpha }=\left( c_1\boldsymbol{\gamma }_1+c_2\boldsymbol{\gamma }_2+\cdots +c_r\boldsymbol{\gamma }_r \right) \boldsymbol{\alpha }=\mathbf{0}\boldsymbol{\alpha }=\mathbf{0},
\]
从而\(\boldsymbol{\beta}_1,\boldsymbol{\beta}_2,\cdots,\boldsymbol{\beta}_r\)线性相关.

反之,若\(A\)可逆,如有\(k_1,k_2,\cdots,k_r\),使得
\[
k_1\boldsymbol{\beta}_1 + k_2\boldsymbol{\beta}_2+\cdots + k_r\boldsymbol{\beta}_r=\boldsymbol{0},
\]
将\(\boldsymbol{\beta}_i\)代入,并利用\(\boldsymbol{\alpha}_1,\boldsymbol{\alpha}_2,\cdots,\boldsymbol{\alpha}_r\)的线性无关性,可得以\(k_i\)为未知数的线性方程组:
\[
\begin{cases}
a_{11}k_1 + a_{21}k_2+\cdots + a_{r1}k_r = 0,\\
a_{12}k_1 + a_{22}k_2+\cdots + a_{r2}k_r = 0,\\
\cdots\cdots\cdots\cdots\\
a_{1r}k_1 + a_{2r}k_2+\cdots + a_{rr}k_r = 0.
\end{cases}
\]
因为\(A\)可逆,所以该方程组只有零解,从而\(\boldsymbol{\beta}_1,\boldsymbol{\beta}_2,\cdots,\boldsymbol{\beta}_r\)线性无关. 
\end{proof}

\begin{proposition}\label{proposition:线性无关向量组的命题6}
设\(\boldsymbol{\alpha}_1,\boldsymbol{\alpha}_2,\cdots,\boldsymbol{\alpha}_m\)是一组线性无关的向量,向量组\(\boldsymbol{\beta}_1,\boldsymbol{\beta}_2,\cdots,\boldsymbol{\beta}_k\)可用\(\boldsymbol{\alpha}_1,\boldsymbol{\alpha}_2,\cdots,\boldsymbol{\alpha}_m\)线性表示如下:
\[
\begin{cases}
\boldsymbol{\beta}_1 = a_{11}\boldsymbol{\alpha}_1 + a_{12}\boldsymbol{\alpha}_2+\cdots + a_{1m}\boldsymbol{\alpha}_m,\\
\boldsymbol{\beta}_2 = a_{21}\boldsymbol{\alpha}_1 + a_{22}\boldsymbol{\alpha}_2+\cdots + a_{2m}\boldsymbol{\alpha}_m,\\
\cdots\cdots\cdots\cdots\\
\boldsymbol{\beta}_k = a_{k1}\boldsymbol{\alpha}_1 + a_{k2}\boldsymbol{\alpha}_2+\cdots + a_{km}\boldsymbol{\alpha}_m.
\end{cases}
\]
记表示矩阵\(A=(a_{ij})_{k\times m}\),求证:向量组\(\boldsymbol{\beta}_1,\boldsymbol{\beta}_2,\cdots,\boldsymbol{\beta}_k\)的秩等于\(\text{r}(A)\).
\end{proposition}
\begin{proof}
{\color{blue}证法一:}
设\(\text{r}(A)=r\),记\(A\)的\(k\)个行向量为\(\boldsymbol{\gamma}_1,\boldsymbol{\gamma}_2,\cdots,\boldsymbol{\gamma}_k\). 不失一般性,可假设\(A\)的前\(r\)个行向量线性无关,其余向量均可用前\(r\)个行向量线性表示. 若
\[
\boldsymbol{\gamma}_i = c_1\boldsymbol{\gamma}_1 + c_2\boldsymbol{\gamma}_2+\cdots + c_r\boldsymbol{\gamma}_r,
\]
设$\boldsymbol{\alpha }=\left( \boldsymbol{\alpha }_1,\boldsymbol{\alpha }_2,\cdots ,\boldsymbol{\alpha }_m \right)'$,则经过简单计算可得
\begin{align*}
\boldsymbol{\beta }_i&=\boldsymbol{\gamma }_i\boldsymbol{\alpha }=\left( c_1\boldsymbol{\gamma }_1+c_2\boldsymbol{\gamma }_2+\cdots +c_r\boldsymbol{\gamma }_r \right) \boldsymbol{\alpha }
\\
&=c_1\boldsymbol{\gamma }_1\boldsymbol{\alpha }+c_1\boldsymbol{\gamma }_2\boldsymbol{\alpha }+\cdots +c_1\boldsymbol{\gamma }_r\boldsymbol{\alpha }
\\
&=c_1\boldsymbol{\beta }_1+c_2\boldsymbol{\beta }_2+\cdots +c_r\boldsymbol{\beta }_r. 
\end{align*}
另一方面,若
\[
c_1\boldsymbol{\beta}_1 + c_2\boldsymbol{\beta}_2+\cdots + c_r\boldsymbol{\beta}_r=\boldsymbol{0},
\]
则
\[
c_1(a_{11}\boldsymbol{\alpha}_1+\cdots + a_{1m}\boldsymbol{\alpha}_m)+\cdots + c_r(a_{r1}\boldsymbol{\alpha}_1+\cdots + a_{rm}\boldsymbol{\alpha}_m)=\boldsymbol{0},
\]
即
\[
(c_1a_{11}+\cdots + c_ra_{r1})\boldsymbol{\alpha}_1+\cdots + (c_1a_{1m}+\cdots + c_ra_{rm})\boldsymbol{\alpha}_m=\boldsymbol{0}.
\]
因为\(\boldsymbol{\alpha}_1,\cdots,\boldsymbol{\alpha}_m\)线性无关,故可得
\[
\begin{cases}
a_{11}c_1 + a_{21}c_2+\cdots + a_{r1}c_r = 0,\\
a_{12}c_1 + a_{22}c_2+\cdots + a_{r2}c_r = 0,\\
\cdots\cdots\cdots\cdots\\
a_{1m}c_1 + a_{2m}c_2+\cdots + a_{rm}c_r = 0.
\end{cases}
\]
将上述方程组看成是未知数\(c_i\)的齐次线性方程组,其系数矩阵的秩为\(r\),未知数个数也是\(r\),因此只有唯一一组解,即零解. 这表明\(\boldsymbol{\beta}_1,\boldsymbol{\beta}_2,\cdots,\boldsymbol{\beta}_r\)是向量组\(\boldsymbol{\beta}_1,\boldsymbol{\beta}_2,\cdots,\boldsymbol{\beta}_k\)的极大无关组,因此向量组\(\boldsymbol{\beta}_1,\boldsymbol{\beta}_2,\cdots,\boldsymbol{\beta}_k\)的秩等于\(r\).

{\color{blue}证法二:}令\(V\)是由\(\boldsymbol{\alpha}_1,\boldsymbol{\alpha}_2,\cdots,\boldsymbol{\alpha}_m\)生成的向量空间. 因为\(\boldsymbol{\alpha}_1,\boldsymbol{\alpha}_2,\cdots,\boldsymbol{\alpha}_m\)线性无关,故它们组成\(V\)的一组基,\(V\)的维数等于\(m\). 注意到\(\boldsymbol{\beta}_i\)在这组基下的坐标向量为\((a_{i1},a_{i2},\cdots,a_{im})'\),故由这些列向量组成的矩阵就是\(\boldsymbol{A}'\),从而
\begin{align*}
\mathrm{r}\left( \boldsymbol{\beta }_1,\boldsymbol{\beta }_2,\cdots ,\boldsymbol{\beta }_k \right) =\mathrm{r}\left( \boldsymbol{A}' \right) =\mathrm{r}\left( \boldsymbol{A} \right) .
\end{align*}
\end{proof}

\begin{proposition}\label{proposition:表出向量组的秩不超过原向量组的秩}
设\(\boldsymbol{\alpha}_1,\boldsymbol{\alpha}_2,\cdots,\boldsymbol{\alpha}_m\)是向量空间\(V\)中一组向量,向量组\(\boldsymbol{\beta}_1,\boldsymbol{\beta}_2,\cdots,\boldsymbol{\beta}_k\)可用\(\boldsymbol{\alpha}_1,\boldsymbol{\alpha}_2,\cdots,\boldsymbol{\alpha}_m\)线性表出,求证:向量组\(\boldsymbol{\beta}_1,\boldsymbol{\beta}_2,\cdots,\boldsymbol{\beta}_k\)的秩小于等于向量组\(\boldsymbol{\alpha}_1,\boldsymbol{\alpha}_2,\cdots,\boldsymbol{\alpha}_m\)的秩.
\end{proposition}
\begin{remark}
如果将向量组\(\boldsymbol{\alpha}_1,\boldsymbol{\alpha}_2,\cdots,\boldsymbol{\alpha}_m\)称为原向量组,将向量组\(\boldsymbol{\beta}_1,\boldsymbol{\beta}_2,\cdots,\boldsymbol{\beta}_k\)称为表出向量组,则例3.19可简述为:“\textbf{表出向量组的秩不超过原向量组的秩}.”从几何上看,这是一个自然的结果. 因为每个\(\boldsymbol{\beta}_i\)都属于由\(\boldsymbol{\alpha}_1,\boldsymbol{\alpha}_2,\cdots,\boldsymbol{\alpha}_m\)生成的子空间,故它们的秩不会超过该子空间的维数.
\end{remark}
\begin{proof}
不失一般性,可设\(\boldsymbol{\alpha}_1,\boldsymbol{\alpha}_2,\cdots,\boldsymbol{\alpha}_r\)是向量组\(\boldsymbol{\alpha}_1,\boldsymbol{\alpha}_2,\cdots,\boldsymbol{\alpha}_m\)的极大无关组,\(\boldsymbol{\beta}_1,\boldsymbol{\beta}_2,\cdots,\boldsymbol{\beta}_s\)是向量组\(\boldsymbol{\beta}_1,\boldsymbol{\beta}_2,\cdots,\boldsymbol{\beta}_k\)的极大无关组. 因为\(\boldsymbol{\alpha}_1,\boldsymbol{\alpha}_2,\cdots,\boldsymbol{\alpha}_m\)可用\(\boldsymbol{\alpha}_1,\boldsymbol{\alpha}_2,\cdots,\boldsymbol{\alpha}_r\)线性表出,所以\(\boldsymbol{\beta}_1,\boldsymbol{\beta}_2,\cdots,\boldsymbol{\beta}_s\)也可用\(\boldsymbol{\alpha}_1,\boldsymbol{\alpha}_2,\cdots,\boldsymbol{\alpha}_r\)线性表出,从而由\hyperref[theorem:向量的线性关系定理2]{定理\ref{theorem:向量的线性关系定理2}(1)}可知\(s\leq r\),结论成立.
\end{proof}

\begin{proposition}\label{proposition:极大无关组的判定条件}
设\(\boldsymbol{\alpha}_1,\boldsymbol{\alpha}_2,\cdots,\boldsymbol{\alpha}_m\)是向量空间\(V\)中一组向量且其秩等于\(r\),\(\boldsymbol{\alpha}_{i_1},\boldsymbol{\alpha}_{i_2},\cdots,\boldsymbol{\alpha}_{i_r}\)是其中\(r\)个向量. 若下列条件之一成立:

(1) \(\boldsymbol{\alpha}_{i_1},\boldsymbol{\alpha}_{i_2},\cdots,\boldsymbol{\alpha}_{i_r}\)线性无关;

(2) 任一\(\boldsymbol{\alpha}_i\)均可由\(\boldsymbol{\alpha}_{i_1},\boldsymbol{\alpha}_{i_2},\cdots,\boldsymbol{\alpha}_{i_r}\)线性表示.

则\(\boldsymbol{\alpha}_{i_1},\boldsymbol{\alpha}_{i_2},\cdots,\boldsymbol{\alpha}_{i_r}\)是向量组的极大无关组.
\end{proposition}
\begin{proof}
(1) 设\(\boldsymbol{\alpha}_{i_1},\boldsymbol{\alpha}_{i_2},\cdots,\boldsymbol{\alpha}_{i_r}\)线性无关,又设\(\boldsymbol{\alpha}_{j_1},\boldsymbol{\alpha}_{j_2},\cdots,\boldsymbol{\alpha}_{j_r}\)是向量组的极大无关组. 对任意的\(1\leq i\leq m\),\(\boldsymbol{\alpha}_{i_1},\boldsymbol{\alpha}_{i_2},\cdots,\boldsymbol{\alpha}_{i_r},\boldsymbol{\alpha}_i\)均可由\(\boldsymbol{\alpha}_{j_1},\boldsymbol{\alpha}_{j_2},\cdots,\boldsymbol{\alpha}_{j_r}\)线性表示,由\hyperref[theorem:向量的线性关系定理2]{定理\ref{theorem:向量的线性关系定理2}(2)}可知\(\boldsymbol{\alpha}_{i_1},\boldsymbol{\alpha}_{i_2},\cdots,\boldsymbol{\alpha}_{i_r},\boldsymbol{\alpha}_i\)必线性相关. 再由\hyperref[proposition:线性无关向量组的命题1]{命题\ref{proposition:线性无关向量组的命题1}}可知\(\boldsymbol{\alpha}_i\)可由\(\boldsymbol{\alpha}_{i_1},\boldsymbol{\alpha}_{i_2},\cdots,\boldsymbol{\alpha}_{i_r}\)线性表示,从而\(\boldsymbol{\alpha}_{i_1},\boldsymbol{\alpha}_{i_2},\cdots,\boldsymbol{\alpha}_{i_r}\)也是向量组的极大无关组.

(2) 设任一\(\boldsymbol{\alpha}_i\)均可由\(\boldsymbol{\alpha}_{i_1},\boldsymbol{\alpha}_{i_2},\cdots,\boldsymbol{\alpha}_{i_r}\)线性表示. 不失一般性,可设\(\boldsymbol{\alpha}_{i_1},\boldsymbol{\alpha}_{i_2},\cdots,\boldsymbol{\alpha}_{i_s}\)是向量组\(\boldsymbol{\alpha}_{i_1},\boldsymbol{\alpha}_{i_2},\cdots,\boldsymbol{\alpha}_{i_r}\)的极大无关组. 因此,\(\boldsymbol{\alpha}_{i_1},\boldsymbol{\alpha}_{i_2},\cdots,\boldsymbol{\alpha}_{i_s}\)线性无关. 再由线性组合的传递性可知,任一\(\boldsymbol{\alpha}_i\)均可由\(\boldsymbol{\alpha}_{i_1},\boldsymbol{\alpha}_{i_2},\cdots,\boldsymbol{\alpha}_{i_s}\)线性表示,故\(\boldsymbol{\alpha}_{i_1},\boldsymbol{\alpha}_{i_2},\cdots,\boldsymbol{\alpha}_{i_s}\)是原向量组的极大无关组,从而\(s = r\),即\(\boldsymbol{\alpha}_{i_1},\boldsymbol{\alpha}_{i_2},\cdots,\boldsymbol{\alpha}_{i_r}\)是原向量组的极大无关组.
\end{proof}

\begin{proposition}\label{proposition:对称矩阵或反称矩阵的极大无关组}
若\(\boldsymbol{A}\)是对称矩阵或反称矩阵,并且\(\boldsymbol{A}\)的第\(i_1,\cdots,i_r\)行是\(\boldsymbol{A}\)的行向量的极大无关组,则它的第\(i_1,\cdots,i_r\)列也是\(\boldsymbol{A}\)的列向量的极大无关组.
\end{proposition}
\begin{proof}
设\(\boldsymbol{A}=\begin{pmatrix}
\boldsymbol{\alpha}_1 \\
\boldsymbol{\alpha}_2 \\
\vdots \\
\boldsymbol{\alpha}_m
\end{pmatrix}=(\boldsymbol{\beta}_1,\boldsymbol{\beta}_2,\cdots,\boldsymbol{\beta}_n)\)为矩阵\(\boldsymbol{A}\)的行分块和列分块,则由条件可知,\(\boldsymbol{\alpha}_{i_1},\boldsymbol{\alpha}_{i_2},\cdots,\boldsymbol{\alpha}_{i_r}\)线性无关.从而存在一组不全为零的数\(k_1,k_2,\cdots,k_r\),使得
\[k_1\boldsymbol{\alpha}_{i_1}+k_2\boldsymbol{\alpha}_{i_2}+\cdots +k_r\boldsymbol{\alpha}_{i_r}=0.\]
又因为\(\boldsymbol{A}\)为对称或反称矩阵,所以\(\boldsymbol{\alpha}_i = \pm\boldsymbol{\beta}_{i}^{\prime}, i = 1,2,\cdots,n\).代入上式可得
\[k_1\boldsymbol{\beta}_{i_1}^{\prime}+k_2\boldsymbol{\beta}_{i_2}^{\prime}+\cdots +k_r\boldsymbol{\beta}_{i_r}^{\prime}=0.\]
再对上式两边同时取转置可得
\[k_1\boldsymbol{\beta}_{i_1}+k_2\boldsymbol{\beta}_{i_2}+\cdots +k_r\boldsymbol{\beta}_{i_r}=0.\]
故\(\boldsymbol{\beta}_{i_1},\boldsymbol{\beta}_{i_2},\cdots,\boldsymbol{\beta}_{i_r}\)线性无关.
\end{proof}

\begin{proposition}[向量组等价的充要条件]\label{proposition:向量组等价的充要条件}
设有两个向量组\(A = \{\boldsymbol{\alpha}_1,\boldsymbol{\alpha}_2,\cdots,\boldsymbol{\alpha}_m\}\)和\(B = \{\boldsymbol{\beta}_1,\boldsymbol{\beta}_2,\cdots,\boldsymbol{\beta}_n\}\). 求证:它们等价的充要条件是它们的秩相等且其中一组向量可以用另外一组向量线性表示.
\end{proposition}
\begin{note}
遇到向量组相关的问题,一般都会先设出各个向量组的极大无关组.
\end{note}
\begin{proof}
必要性由向量组等价的定义和\hyperref[proposition:表出向量组的秩不超过原向量组的秩]{命题\ref{proposition:表出向量组的秩不超过原向量组的秩}}即得,下证充分性. 假设向量组\(A\)可用向量组\(B\)线性表示,且它们的秩都等于\(r\). 不失一般性,设\(\boldsymbol{\alpha}_1,\boldsymbol{\alpha}_2,\cdots,\boldsymbol{\alpha}_r\)是向量组\(A\)的极大无关组,\(\boldsymbol{\beta}_1,\boldsymbol{\beta}_2,\cdots,\boldsymbol{\beta}_r\)是向量组\(B\)的极大无关组. 考虑向量组\(C = \{\boldsymbol{\alpha}_1,\boldsymbol{\alpha}_2,\cdots,\boldsymbol{\alpha}_r,\boldsymbol{\beta}_1,\boldsymbol{\beta}_2,\cdots,\boldsymbol{\beta}_r\}\). 因为\(\boldsymbol{\alpha}_1,\boldsymbol{\alpha}_2,\cdots,\boldsymbol{\alpha}_r\)可用\(\boldsymbol{\beta}_1,\boldsymbol{\beta}_2,\cdots,\boldsymbol{\beta}_r\)线性表示,故\(\boldsymbol{\beta}_1,\boldsymbol{\beta}_2,\cdots,\boldsymbol{\beta}_r\)是向量组\(C\)的极大无关组,从而向量组\(C\)的秩等于\(r\). 又因为\(\boldsymbol{\alpha}_1,\boldsymbol{\alpha}_2,\cdots,\boldsymbol{\alpha}_r\)线性无关,故由\hyperref[proposition:表出向量组的秩不超过原向量组的秩]{命题\ref{proposition:表出向量组的秩不超过原向量组的秩}(1)}可知,\(\boldsymbol{\alpha}_1,\boldsymbol{\alpha}_2,\cdots,\boldsymbol{\alpha}_r\)也是向量组\(C\)的极大无关组,从而\(\boldsymbol{\beta}_1,\boldsymbol{\beta}_2,\cdots,\boldsymbol{\beta}_r\)可用\(\boldsymbol{\alpha}_1,\boldsymbol{\alpha}_2,\cdots,\boldsymbol{\alpha}_r\)线性表示,于是向量组\(B\)也可用向量组\(A\)线性表示. 因此,向量组\(A\)与向量组\(B\)等价. 
\end{proof}

\section{线性空间}

\begin{example}[$\,\,$常见的线性空间]\label{example:常见的线性空间}
\begin{enumerate}[(1)]
\item 数域\(\mathbb{K}\)上\(n\)维行(列)向量集合\(\mathbb{K}_n(\mathbb{K}^n)\),在行(列)向量的加法和数乘下成为\(\mathbb{K}\)上的线性空间,称为数域\(\mathbb{K}\)上的\(n\)维行(列)向量空间.
\item 数域\(\mathbb{K}\)上的一元多项式全体\(\mathbb{K}[x]\),在多项式的加法和数乘下成为\(\mathbb{K}\)上的线性空间. 在\(\mathbb{K}[x]\)中,取次数小于等于\(n\)的多项式全体,记这个集合为\(\mathbb{K}_n[x]\),则\(\mathbb{K}_n[x]\)也是\(\mathbb{K}\)上的线性空间.
\item 数域\(\mathbb{K}\)上\(m\times n\)矩阵全体\(M_{m\times n}(\mathbb{K})\),在矩阵的加法和数乘下成为\(\mathbb{K}\)上的线性空间.
\item \label{example:常见的线性空间(4)}实数域\(\mathbb{R}\)上的连续函数全体记为\(C(\mathbb{R})\),函数的加法及数乘分别定义为\((f + g)(x)=f(x)+g(x)\),\((kf)(x)=kf(x)\),则\(C(\mathbb{R})\)是\(\mathbb{R}\)上的线性空间.
\end{enumerate}
\end{example}

\begin{proposition}\label{proposition:数域上的线性空间}
若两个数域\(\mathbb{K}_1\subseteq\mathbb{K}_2\),则\(\mathbb{K}_2\)可以看成是\(\mathbb{K}_1\)上的线性空间. 向量就是\(\mathbb{K}_2\)中的数,向量的加法就是数的加法,数乘就是\(\mathbb{K}_1\)中的数乘以\(\mathbb{K}_2\)中的数. 特别地,数域\(\mathbb{K}\)也可以看成是\(\mathbb{K}\)自身上的线性空间.
\end{proposition}

\begin{example}\label{example:3.3}
判断下列集合是否构成实数域\(\mathbb{R}\)上的线性空间:
\begin{enumerate}[(1)]
\item \(V\)为次数等于\(n(n\geq1)\)的实系数多项式全体,加法和数乘就是多项式的加法和数乘.
\item \(V = M_n(\mathbb{R})\),数乘就是矩阵的数乘,加法\(\oplus\)定义为\(\boldsymbol{A}\oplus\boldsymbol{B}=\boldsymbol{A}\boldsymbol{B}-\boldsymbol{B}\boldsymbol{A}\),其中等式右边是矩阵的乘法和减法.
\item \(V = M_n(\mathbb{R})\),数乘就是矩阵的数乘,加法\(\oplus\)定义为\(\boldsymbol{A}\oplus\boldsymbol{B}=\boldsymbol{A}\boldsymbol{B}+\boldsymbol{B}\boldsymbol{A}\),其中等式右边是矩阵的乘法和加法.
\item \label{example:3.3(4)}\(V\)是以\(0\)为极限的实数数列全体,即\(V = \left\{\{a_n\}\mid\lim_{n\rightarrow\infty}a_n = 0\right\}\),定义两个数列的加法\(\oplus\)及数乘\(\circ\)为:\(\{a_n\}\oplus\{b_n\}=\{a_n + b_n\}\),\(k\circ\{a_n\}=\{ka_n\}\),其中等式右边分别是数的加法和乘法.
\item \label{example:3.3(5)}\(V\)是正实数全体\(\mathbb{R}^+\),加法\(\oplus\)定义为\(a\oplus b = ab\),数乘\(\circ\)定义为\(k\circ a = a^k\),其中等式右边分别是数的乘法和乘方.
\item \label{example:3.3(6)}\(V\)为实数对全体\(\{(a,b)\mid a,b\in\mathbb{R}\}\),加法\(\oplus\)定义为\((a_1,b_1)\oplus(a_2,b_2)=(a_1 + a_2,b_1 + b_2 + a_1a_2)\),数乘\(\circ\)定义为\(k\circ(a,b)=(ka,kb+\frac{k(k - 1)}{2}a^2)\),其中等式右边分别是数的加法和乘法.
\end{enumerate}
\end{example}
\begin{solution}
(1) \(V\)不是线性空间, 因为加法不封闭.

(2) \(V\)不是线性空间, 因为加法不满足交换律, 即 \(\boldsymbol{A}\oplus\boldsymbol{B}\neq\boldsymbol{B}\oplus\boldsymbol{A}\).

(3) \(V\)不是线性空间, 因为加法不满足结合律, 即 \((\boldsymbol{A}\oplus\boldsymbol{B})\oplus\boldsymbol{C}\neq\boldsymbol{A}\oplus(\boldsymbol{B}\oplus\boldsymbol{C})\).

(4)、(5)、(6) \(V\)都是线性空间, 特别是 (5) 和 (6), 其加法和数乘的定义都不是线性的, 但它们竟然都是线性空间! 请读者自己验证线性空间的 8 条公理的确成立, 在下一节我们会从线性同构的角度来说明它们成为线性空间的深层次理由.
\end{solution}

\begin{proposition}
设\(V\)是\(n\)维线性空间,\(\boldsymbol{e}_1,\boldsymbol{e}_2,\cdots,\boldsymbol{e}_n\)是\(V\)中\(n\)个向量. 若它们满足下列条件之一:

(1) \(\boldsymbol{e}_1,\boldsymbol{e}_2,\cdots,\boldsymbol{e}_n\)线性无关;

(2) \(V\)中任一向量均可由\(\boldsymbol{e}_1,\boldsymbol{e}_2,\cdots,\boldsymbol{e}_n\)线性表示,
求证:\(\boldsymbol{e}_1,\boldsymbol{e}_2,\cdots,\boldsymbol{e}_n\)是\(V\)的一组基.
\end{proposition}
\begin{proof}
证明完全类似\hyperref[proposition:极大无关组的判定条件]{命题\ref{proposition:极大无关组的判定条件}}.
\end{proof}

\begin{theorem}[基扩充定理]\label{theorem:基扩充定理}
设\(V\)是\(n\)维线性空间,\(\boldsymbol{v}_1,\boldsymbol{v}_2,\cdots,\boldsymbol{v}_m\)是一组线性无关的向量(\(V\)的子空间\(U\)的一组基),\(\boldsymbol{e}_1,\boldsymbol{e}_2,\cdots,\boldsymbol{e}_n\)是\(V\)的一组基. 则必可在\(\boldsymbol{e}_1,\boldsymbol{e}_2,\cdots,\boldsymbol{e}_n\)中选出\(n - m\)个向量,使之和\(\boldsymbol{v}_1,\boldsymbol{v}_2,\cdots,\boldsymbol{v}_m\)一起组成\(V\)的一组基.
\end{theorem}
\begin{proof}
若\(m< n\),将\(\boldsymbol{e}_i(1\leq i\leq n)\)依次加入向量组\(\boldsymbol{v}_1,\boldsymbol{v}_2,\cdots,\boldsymbol{v}_m\),则必有一个\(\boldsymbol{e}_i\),使得\(\boldsymbol{v}_1,\boldsymbol{v}_2,\cdots,\boldsymbol{v}_m,\boldsymbol{e}_i\)线性无关. 这是因为若任意一个\(\boldsymbol{e}_i\)加入\(\boldsymbol{v}_1,\boldsymbol{v}_2,\cdots,\boldsymbol{v}_m\)后均线性相关,则由\hyperref[proposition:线性无关向量组的命题1]{命题\ref{proposition:线性无关向量组的命题1}}可知,每个\(\boldsymbol{e}_i\)都可用\(\boldsymbol{v}_1,\boldsymbol{v}_2,\cdots,\boldsymbol{v}_m\)线性表示,由\hyperref[theorem:向量的线性关系定理2]{定理\ref{theorem:向量的线性关系定理2}}可得\(n\leq m\),矛盾. 将新加入的向量$e_i$记作$v_{m+1}$,则原线性无关向量组扩张为$v_1,v_2,\cdots,v_{m+1}$,并且仍线性无关.若\(m + 1< n\),则同理又可从\(\boldsymbol{e}_1,\boldsymbol{e}_2,\cdots,\boldsymbol{e}_n\)中找到一个向量,加入$v_1,v_2,\cdots,v_{m+1}$之后仍线性无关.将新加入的向量记作$v_{m+2}$,则原线性无关向量组扩张为$v_1,v_2,\cdots,v_{m+2}$,并且仍线性无关.不断这样做下去,直到$m+n-m-1=n-1<n$时,同理可从\(\boldsymbol{e}_1,\boldsymbol{e}_2,\cdots,\boldsymbol{e}_n\)中找到一个向量,加入$v_1,v_2,\cdots,v_{n-1}$之后仍线性无关.将新加入的向量记作$v_{n}$,则可将\(\boldsymbol{v}_1,\boldsymbol{v}_2,\cdots,\boldsymbol{v}_m\)扩张成为$v_1,v_2,\cdots,v_{n}$,并且仍线性无关.此时$v_1,v_2,\cdots,v_{n}$就是\(V\)的一组基.
\end{proof}

\begin{proposition}\label{proposition:包含所有向量的空间也包含这些向量张成的空间}
若$\boldsymbol{\alpha }_1,\boldsymbol{\alpha }_2,\cdots ,\boldsymbol{\alpha }_n\subset V$,则$L\left( \boldsymbol{\alpha }_1,\boldsymbol{\alpha }_2,\cdots ,\boldsymbol{\alpha }_n \right) \subset V$.
\end{proposition}
\begin{proof}
证明是显然的.
\end{proof}

\begin{proposition}\label{proposition:与全空间维数相同的子空间等于全空间}
设\(V_1,V_2\)均为线性空间,若\(V_1\subset V_2\),并且\(\dim V_1 = \dim V_2\),则\(V_1 = V_2\).
\end{proposition}
\begin{proof}
取$V_1$的一组基即可得到证明.
\end{proof}

\begin{proposition}\label{proposition:和空间包含于空间的和}
证明:$L\left( \boldsymbol{\alpha }_1+\boldsymbol{\beta }_1,\boldsymbol{\alpha }_2+\boldsymbol{\beta }_2,\cdots ,\boldsymbol{\alpha }_n+\boldsymbol{\beta }_n \right) \subset L\left( \boldsymbol{\alpha }_1,\boldsymbol{\alpha }_2,\cdots ,\boldsymbol{\alpha }_n \right) +L\left( \boldsymbol{\beta }_1,\boldsymbol{\beta }_2,\cdots ,\boldsymbol{\beta }_n \right)$.
\end{proposition}
\begin{proof}
$\forall \boldsymbol{\alpha }\in L(\boldsymbol{\alpha }_1+\boldsymbol{\beta }_1,\boldsymbol{\alpha }_2+\boldsymbol{\beta }_2,\cdots,\boldsymbol{\alpha }_n+\boldsymbol{\beta }_n)$,则
\[
\boldsymbol{\alpha } = k_1(\boldsymbol{\alpha }_1+\boldsymbol{\beta }_1)+k_2(\boldsymbol{\alpha }_2+\boldsymbol{\beta }_2)+\cdots +k_n(\boldsymbol{\alpha }_n+\boldsymbol{\beta }_n) = k_1\boldsymbol{\alpha }_1 + k_2\boldsymbol{\alpha }_2+\cdots +k_n\boldsymbol{\alpha }_n + k_1\boldsymbol{\beta }_1 + k_2\boldsymbol{\beta }_2+\cdots +k_n\boldsymbol{\beta }_n.
\]
其中$k_1,k_2,\cdots,k_n\in \mathbb{R}$.

令$\boldsymbol{\beta } = k_1\boldsymbol{\alpha }_1 + k_2\boldsymbol{\alpha }_2+\cdots +k_n\boldsymbol{\alpha }_n$,$\boldsymbol{\gamma } = k_1\boldsymbol{\beta }_1 + k_2\boldsymbol{\beta }_2+\cdots +k_n\boldsymbol{\beta }_n$,则$\boldsymbol{\beta }\in L(\boldsymbol{\alpha }_1,\boldsymbol{\alpha }_2,\cdots,\boldsymbol{\alpha }_n)$,$\boldsymbol{\gamma }\in L(\boldsymbol{\beta }_1,\boldsymbol{\beta }_2,\cdots,\boldsymbol{\beta }_n)$.从而
\[
\boldsymbol{\alpha }=\boldsymbol{\beta }+\boldsymbol{\gamma }\in L(\boldsymbol{\alpha }_1,\boldsymbol{\alpha }_2,\cdots,\boldsymbol{\alpha }_n)+L(\boldsymbol{\beta }_1,\boldsymbol{\beta }_2,\cdots,\boldsymbol{\beta }_n).
\]
故$L\left( \boldsymbol{\alpha }_1+\boldsymbol{\beta }_1,\boldsymbol{\alpha }_2+\boldsymbol{\beta }_2,\cdots ,\boldsymbol{\alpha }_n+\boldsymbol{\beta }_n \right) \subset L\left( \boldsymbol{\alpha }_1,\boldsymbol{\alpha }_2,\cdots ,\boldsymbol{\alpha }_n \right) +L\left( \boldsymbol{\beta }_1,\boldsymbol{\beta }_2,\cdots ,\boldsymbol{\beta }_n \right)$.
\end{proof}


\begin{example}[$\,\,$一些常见线性空间的基]\label{example:一些常见线性空间的基}
\begin{enumerate}[(1)]
\item \label{example:一些常见线性空间的基(1)}设\(V\)是数域\(\mathbb{K}\)上次数不超过\(n\)的多项式全体构成的线性空间,求证:\(\{1,x,x^2,\cdots,x^n\}\)是\(V\)的一组基,并且\(\{1,x + 1,(x + 1)^2,\cdots,(x + 1)^n\}\)也是\(V\)的一组基.
\item  \label{example:一些常见线性空间的基(2)}设\(V\)是数域\(\mathbb{K}\)上次数小于\(n\)的多项式全体构成的线性空间,\(a_1,a_2,\cdots,a_n\)是\(\mathbb{K}\)中互不相同的\(n\)个数,\(f(x)=(x - a_1)(x - a_2)\cdots(x - a_n)\),\(f_i(x)=f(x)/(x - a_i)\),求证:\(\{f_1(x),f_2(x),\cdots,f_n(x)\}\)组成\(V\)的一组基.
\item \label{example:一些常见线性空间的基(3)}设\(V\)是数域\(\mathbb{K}\)上\(m\times n\)矩阵全体组成的线性空间,令\(\boldsymbol{E}_{ij}(1\leq i\leq m,1\leq j\leq n)\)是第\((i,j)\)元素为\(1\)、其余元素为\(0\)的\(m\times n\)矩阵,求证:全体\(\boldsymbol{E}_{ij}\)组成了\(V\)的一组基,从而\(V\)是\(mn\)维线性空间.
\item  \label{example:一些常见线性空间的基(4)}\(V\)是数域\(\mathbb{K}\)上\(n\)阶上三角矩阵全体组成的线性空间.容易验证\(\{\boldsymbol{E}_{ij}(1\leq i\leq j\leq n)\}\)是\(V\)的一组基,因此\(\dim V=\frac{n(n + 1)}{2}\).
\item  \label{example:一些常见线性空间的基(5)}\(V\)是数域\(\mathbb{K}\)上\(n\)阶对称矩阵全体组成的线性空间.容易验证\(\{\boldsymbol{E}_{ii}(1\leq i\leq n);\boldsymbol{E}_{ij}+\boldsymbol{E}_{ji}(1\leq i< j\leq n)\}\)是\(V\)的一组基,因此\(\dim V=\frac{n(n + 1)}{2}\).
\item  \label{example:一些常见线性空间的基(6)}\(V\)是数域\(\mathbb{K}\)上\(n\)阶反对称矩阵全体组成的线性空间.容易验证\(\{\boldsymbol{E}_{ij}-\boldsymbol{E}_{ji}(1\leq i< j\leq n)\}\)是\(V\)的一组基,因此\(\dim V=\frac{n(n - 1)}{2}\).
\end{enumerate}
\end{example}
\begin{proof}
\begin{enumerate}[(1)]
\item 根据多项式的定义容易验证\(\{1,x,x^2,\cdots,x^n\}\)是\(V\)的一组基,特别地,\(\dim V=n + 1\). 对任意的\(f(x)\in V\),设\(y=x + 1\),则
\[
f(x)=f(y - 1)=b_ny^n+\cdots+b_1y + b_0=b_n(x + 1)^n+\cdots+b_1(x + 1)+b_0,
\]
其中\(b_n,\cdots,b_1,b_0\)是\(\mathbb{K}\)中的数. 因此,\(V\)中任一多项式\(f(x)\)均可由\(1,x + 1,(x + 1)^2,\cdots,(x + 1)^n\)线性表示. 由例3.24可知,\(\{1,x + 1,(x + 1)^2,\cdots,(x + 1)^n\}\)是\(V\)的一组基.
\item 因为\(V\)是\(n\)维线性空间,故由例3.24只需证明\(n\)个向量\(f_1(x),f_2(x),\cdots,f_n(x)\)线性无关即可. 设
\[
k_1f_1(x)+k_2f_2(x)+\cdots+k_nf_n(x)=0,
\]
依次令\(x = a_1,a_2,\cdots,a_n\),即可求出\(k_1 = k_2=\cdots=k_n = 0\). 
\item 一方面,对任意的\(\boldsymbol{A}=(a_{ij})\in V\),容易验证\(\boldsymbol{A}=\sum_{i = 1}^{m}\sum_{j = 1}^{n}a_{ij}\boldsymbol{E}_{ij}\). 另一方面,设\(mn\)个数\(c_{ij}(1\leq i\leq m,1\leq j\leq n)\)满足\(\sum_{i = 1}^{m}\sum_{j = 1}^{n}c_{ij}\boldsymbol{E}_{ij}=\boldsymbol{O}\),则由矩阵相等的定义可得所有的\(c_{ij}=0\). 因此,全体\(\boldsymbol{E}_{ij}\)组成了\(V\)的一组基,从而\(\dim V = mn\).
\item 
\item 
\item 
\end{enumerate}
\end{proof}

\begin{example}[$\,\,$\(n\)阶(斜)Hermite矩阵全体构成的线性空间]\label{example:n阶(斜)Hermite矩阵全体构成的线性空间}

设\(V_1 = \{ \boldsymbol{A} \in M_n(\mathbb{C})|\overline{\boldsymbol{A}}'=\boldsymbol{A}\}\)为\(n\)阶Hermite矩阵全体,\(V_2 = \{ \boldsymbol{A} \in M_n(\mathbb{C})|\overline{\boldsymbol{A}}'=-\boldsymbol{A}\}\)为\(n\)阶斜Hermite矩阵全体,求证:在矩阵加法和实数关于矩阵的数乘下,\(V_1,V_2\)成为实数域\(\mathbb{R}\)上的线性空间,并且具有相同的维数.
\end{example}
\begin{proof}
首先,容易证明对任意的\(\boldsymbol{A},\boldsymbol{B} \in V_i,c \in \mathbb{R}\),我们有\(\boldsymbol{A}+\boldsymbol{B} \in V_i,c\boldsymbol{A} \in V_i\),这就验证了上述加法和数乘是定义好的运算. 其次,容易验证线性空间的8条公理成立,因此\(V_1,V_2\)是实线性空间(注意虽然向量都是复矩阵,但它们绝不是复线性空间). 最后,容易验证\(\{\boldsymbol{E}_{ii}(1\leq i\leq n);\boldsymbol{E}_{ij}+\boldsymbol{E}_{ji}(1\leq i< j\leq n);\mathrm{i}\boldsymbol{E}_{ij}-\mathrm{i}\boldsymbol{E}_{ji}(1\leq i< j\leq n)\}\)是\(V_1\)的一组基,\(\{\mathrm{i}\boldsymbol{E}_{ii}(1\leq i\leq n);\boldsymbol{E}_{ij}-\boldsymbol{E}_{ji}(1\leq i< j\leq n);\mathrm{i}\boldsymbol{E}_{ij}+\mathrm{i}\boldsymbol{E}_{ji}(1\leq i< j\leq n)\}\)是\(V_2\)的一组基,因此\(\dim_{\mathbb{R}}V_1=\dim_{\mathbb{R}}V_2=n^2\).
\end{proof}

\begin{example}
设\(\mathbb{Q}(\sqrt[3]{2})=\{a + b\sqrt[3]{2}+c\sqrt[3]{4}\}\),其中\(a,b,c\)均是有理数,证明:\(\mathbb{Q}(\sqrt[3]{2})\)是有理数域上的线性空间并求其维数.
\end{example}
\begin{proof}
事实上,我们可以证明\(\mathbb{Q}(\sqrt[3]{2})\)是一个数域. 加法、减法和乘法的封闭性都是显然的,我们只要证明除法封闭,或等价地证明非零数的倒数封闭即可. 为此首先需要找出一个数非零的充要条件. 我们断言以下3个结论等价:
\begin{gather*}
(1) a + b\sqrt[3]{2}+c\sqrt[3]{4}=0;\quad(2) a^3 + 2b^3+4c^3-6abc = 0;\quad (3) a = b = c = 0.
\end{gather*}
由公式\((x + y + z)(x^2 + y^2+z^2-xy - yz - zx)=x^3 + y^3+z^3-3xyz\)很容易从(1)推出(2). 假设(2)对不全为零的有理数\(a,b,c\)成立,将(2)式两边同时乘以\(a,b,c\)公分母的立方,可将\(a,b,c\)化为整数; 又可将整数\(a,b,c\)的最大公因数从(2)式提出,因此不妨假设满足(2)式的\(a,b,c\)是互素的整数. 由(2)式可得\(a\)是偶数,可设\(a = 2a_1\),代入(2)式可得\((2')4a_1^3 + b^3+2c^3-6a_1bc = 0\); 由\((2')\)式可得\(b\)是偶数,可设\(b = 2b_1\),代入\((2')\)式可得\((2'')2a_1^3 + 4b_1^3+c^3-6a_1b_1c = 0\); 由\((2'')\)式可得\(c\)是偶数,可设\(c = 2c_1\),这样\(a,b,c\)就有了公因子\(2\),这与它们互素矛盾. 因此,从(2)可以推出(3). 从(3)推出(1)是显然的.

任取\(\mathbb{Q}(\sqrt[3]{2})\)中的非零数\(a + b\sqrt[3]{2}+c\sqrt[3]{4}\),由上述充要条件以及公式可得
\[
(a + b\sqrt[3]{2}+c\sqrt[3]{4})((a^2 - 2bc)+(c^2 - ab)\sqrt[3]{2}+(b^2 - ac)\sqrt[3]{4})=a^3 + 2b^3+4c^3-6abc\neq0,
\]
从而\((a^2 - 2bc)+(c^2 - ab)\sqrt[3]{2}+(b^2 - ac)\sqrt[3]{4}\neq0\). 将倒数\(\frac{1}{a + b\sqrt[3]{2}+c\sqrt[3]{4}}\)的分子分母同时乘以非零数\((a^2 - 2bc)+(c^2 - ab)\sqrt[3]{2}+(b^2 - ac)\sqrt[3]{4}\)进行化简,可得
\[
\frac{1}{a + b\sqrt[3]{2}+c\sqrt[3]{4}}=\frac{(a^2 - 2bc)+(c^2 - ab)\sqrt[3]{2}+(b^2 - ac)\sqrt[3]{4}}{a^3 + 2b^3+4c^3-6abc}\in\mathbb{Q}(\sqrt[3]{2}).
\]
这就证明了\(\mathbb{Q}(\sqrt[3]{2})\)是一个数域. 因为\(\mathbb{Q}\subseteq\mathbb{Q}(\sqrt[3]{2})\),故由\hyperref[proposition:数域上的线性空间]{命题\ref{proposition:数域上的线性空间}}可知,\(\mathbb{Q}(\sqrt[3]{2})\)是有理数域上的线性空间.

由\(\mathbb{Q}(\sqrt[3]{2})\)的定义可知,\(\mathbb{Q}(\sqrt[3]{2})\)中每个数都是\(1,\sqrt[3]{2},\sqrt[3]{4}\)的\(\mathbb{Q}\)-线性组合; 又由上述充要条件可知,\(1,\sqrt[3]{2},\sqrt[3]{4}\)是\(\mathbb{Q}\)-线性无关的. 因此,\(\{1,\sqrt[3]{2},\sqrt[3]{4}\}\)是\(\mathbb{Q}(\sqrt[3]{2})\)的一组基. 特别地,\(\dim_{\mathbb{Q}}\mathbb{Q}(\sqrt[3]{2}) = 3\).
\end{proof}

\begin{proposition}\label{proposition:数域上的线性空间的维数的传递性}
设\(\mathbb{K}_1,\mathbb{K}_2,\mathbb{K}_3\)是数域且\(\mathbb{K}_1\subseteq\mathbb{K}_2\subseteq\mathbb{K}_3\).若将\(\mathbb{K}_2\)看成是\(\mathbb{K}_1\)上的线性空间,其维数为\(m\),又将\(\mathbb{K}_3\)看成是\(\mathbb{K}_2\)上的线性空间,其维数为\(n\).则如将\(\mathbb{K}_3\)看成是\(\mathbb{K}_1\)上的线性空间,则其维数为\(mn\).
\end{proposition}
\begin{proof}
\(\mathbb{K}_2\)作为\(\mathbb{K}_1\)上的线性空间,取其一组基为\(\{\alpha_1,\alpha_2,\cdots,\alpha_m\}\);\(\mathbb{K}_3\)作为\(\mathbb{K}_2\)上的线性空间,取其一组基为\(\{\beta_1,\beta_2,\cdots,\beta_n\}\).注意到\(\alpha_i,\beta_j\)都是数,现在我们断言:\(\mathbb{K}_3\)作为\(\mathbb{K}_1\)上的线性空间,\(\{\alpha_i\beta_j(1\leq i\leq m,1\leq j\leq n)\}\)恰为其一组基.

一方面,对\(\mathbb{K}_3\)中任一数\(a\),存在\(\mathbb{K}_2\)中的数\(b_1,b_2,\cdots,b_n\),使得
\[
a = b_1\beta_1 + b_2\beta_2+\cdots + b_n\beta_n.
\]
又对\(b_j\in\mathbb{K}_2\),存在\(\mathbb{K}_1\)中的数\(c_{1j},c_{2j},\cdots,c_{mj}\),使得
\[
b_j = c_{1j}\alpha_1 + c_{2j}\alpha_2+\cdots + c_{mj}\alpha_m,1\leq j\leq n.
\]
将上述两式进行整理,可得
\[
a=\sum_{j = 1}^{n}b_j\beta_j=\sum_{j = 1}^{n}\left(\sum_{i = 1}^{m}c_{ij}\alpha_i\right)\beta_j=\sum_{j = 1}^{n}\sum_{i = 1}^{m}c_{ij}\alpha_i\beta_j,
\]
即\(\mathbb{K}_3\)中任一数均可由\(\{\alpha_i\beta_j(1\leq i\leq m,1\leq j\leq n)\}\)线性表示.

另一方面,设有\(\mathbb{K}_1\)中的数\(k_{ij}(1\leq i\leq m,1\leq j\leq n)\),使得
\[
\sum_{j = 1}^{n}\sum_{i = 1}^{m}k_{ij}\alpha_i\beta_j = 0,
\]
则经过变形可得
\[
\sum_{j = 1}^{n}\left(\sum_{i = 1}^{m}k_{ij}\alpha_i\right)\beta_j = 0.
\]
注意到\(\sum_{i = 1}^{m}k_{ij}\alpha_i\in\mathbb{K}_2\)且\(\beta_1,\beta_2,\cdots,\beta_n\)是\(\mathbb{K}_3/\mathbb{K}_2\)的一组基,故有\(\sum_{i = 1}^{m}k_{ij}\alpha_i = 0(1\leq j\leq n)\).又因为\(\{\alpha_1,\alpha_2,\cdots,\alpha_m\}\)是\(\mathbb{K}_2/\mathbb{K}_1\)的一组基,故有\(k_{ij}=0(1\leq i\leq m,1\leq j\leq n)\),即\(\{\alpha_i\beta_j(1\leq i\leq m,1\leq j\leq n)\}\)是\(\mathbb{K}_1\) - 线性无关的.

综上所述,\(\{\alpha_i\beta_j(1\leq i\leq m,1\leq j\leq n)\}\)是\(\mathbb{K}_3/\mathbb{K}_1\)的一组基,特别地,\(\dim_{\mathbb{K}_1}\mathbb{K}_3=mn=\dim_{\mathbb{K}_1}\mathbb{K}_2\cdot\dim_{\mathbb{K}_2}\mathbb{K}_3\).
\end{proof}

\begin{proposition}\label{proposition:数域上的线性空间的基的传递性}
设\(\mathbb{F}\subseteq\mathbb{K}\)为数域,\(\mathbb{K}\)作为\(\mathbb{F}\)上的线性空间,一组基为\(\{\alpha_1,\alpha_2,\cdots,\alpha_m\}\);设\(V\)为\(\mathbb{K}\)上的\(n\)维线性空间,一组基为\(\{\boldsymbol{e}_1,\boldsymbol{e}_2,\cdots,\boldsymbol{e}_n\}\).则\(V\)是\(\mathbb{F}\)上的\(mn\)维线性空间,一组基可选择为\(\{\alpha_i\boldsymbol{e}_j(1\leq i\leq m,1\leq j\leq n)\}\).
\end{proposition}
\begin{proof}
证明与\hyperref[proposition:数域上的线性空间的维数的传递性]{命题\ref{proposition:数域上的线性空间的维数的传递性}}完全类似.
\end{proof}

\begin{example}
证明下列线性空间是实数域上的无限维线性空间:

(1) 实数域\(\mathbb{R}\)上的连续函数全体构成的线性空间\(C(\mathbb{R})\)(见\hyperref[example:常见的线性空间(4)]{例题\ref{example:常见的线性空间}\ref{example:常见的线性空间(4)}});

(2) 以\(0\)为极限的实数数列全体构成的线性空间\(V = \left\{\{a_n\}|\lim_{n\rightarrow\infty}a_n = 0\right\}\)(见\hyperref[example:3.3(4)]{例题\ref{example:3.3}\ref{example:3.3(4)}}).
\end{example}
\begin{proof}
我们用反证法来证明.

(1) 若\(C(\mathbb{R})\)是有限维线性空间,则可取到正整数\(k>\dim C(\mathbb{R})\). 然而由\hyperref[example:3.1.1.1]{例题\ref{example:3.1.1.1}}可知\(\sin x,\sin 2x,\cdots,\sin kx\)是\(\mathbb{R}\) - 线性无关的,矛盾.

(2) 若\(V\)是有限维线性空间,则可取到正整数\(k>\dim V\). 构造\(V\)中\(k\)个数列:
\[
\left\{a_n^{(1)}=\frac{1}{n}\right\},\left\{a_n^{(2)}=\frac{1}{n^2}\right\},\cdots,\left\{a_n^{(k)}=\frac{1}{n^k}\right\}.
\]
设有实数\(c_1,c_2,\cdots,c_k\),使得
\[
c_1\left\{a_n^{(1)}\right\}+c_2\left\{a_n^{(2)}\right\}+\cdots + c_k\left\{a_n^{(k)}\right\}=\{0\},
\]
则对于任意的正整数\(n\),成立
\[
\frac{c_1}{n}+\frac{c_2}{n^2}+\cdots+\frac{c_k}{n^k}=0.
\]
任取\(k\)个不同的正整数代入上式,并利用Vandermonde行列式即得\(c_1 = c_2=\cdots = c_k = 0\),从而上述\(k\)个数列线性无关,矛盾. 
\end{proof}

\begin{definition}[无限维空间基的定义]\label{definition:无限维空间基的定义}
设\(B = \{ \boldsymbol{e}_i\}_{i\in I}\)为线性空间\(V\)中的向量族,若\(B\)中任意有限个向量都线性无关,则称向量族\(B\)线性无关;若向量\(\boldsymbol{\alpha}\)可表示为\(B\)中有限个向量的线性组合,则称\(\boldsymbol{\alpha}\)可被向量族\(B\)线性表示. 若线性空间\(V\)中存在线性无关的向量族\(B\),使得\(V = L(B)\),即\(V\)中任一向量都可被\(B\)线性表示,则称向量族\(B\)是\(V\)的一组基.
\end{definition}
\begin{note}
再利用选择公理或Zorn引理就可以证明任意线性空间中基的存在性了.因此这个定义是良定义.
\end{note}

\section{线性同构和几何问题代数化}

\hypertarget{线性同构1}{我们有一类特别重要的线性同构}. 设\(V\)是数域\(\mathbb{K}\)上的\(n\)维线性空间,\(\{\boldsymbol{e}_1,\boldsymbol{e}_2,\cdots,\boldsymbol{e}_n\}\)是\(V\)的一组基并固定次序. 对任一向量\(\boldsymbol{\alpha}\in V\),设\(\boldsymbol{\alpha}=\lambda_1\boldsymbol{e}_1+\lambda_2\boldsymbol{e}_2+\cdots+\lambda_n\boldsymbol{e}_n\),则映射\(\eta:V\to\mathbb{K}^n\)定义为:\(\eta(\boldsymbol{\alpha})=(\lambda_1,\lambda_2,\cdots,\lambda_n)'\),即\(\eta\)将\(V\)中的向量映射到它在给定基下的坐标向量. 容易验证\(\eta:V\to\mathbb{K}^n\)是一个线性同构. 因此,通过这个线性同构,我们可将抽象的线性空间\(V\)和具体的列向量空间\(\mathbb{K}^n\)等同起来.

\begin{theorem}\label{theorem:线性空间的同构}
\begin{enumerate}[(1)]
\item 同构关系是一种等价关系;
\item 线性同构不仅将线性相关的向量组映射为线性相关的向量组,而且将线性无关的向量组映射为线性无关的向量组;
\item 同一个数域\(\mathbb{F}\)上的线性空间同构的充要条件是它们具有相同的维数.
\end{enumerate}
\end{theorem}
\begin{proof}

\end{proof}

\begin{theorem}
定理 假设和记号同上,设\(\boldsymbol{\alpha}_1,\boldsymbol{\alpha}_2,\cdots,\boldsymbol{\alpha}_m,\boldsymbol{\beta}\)是\(V\)中向量,它们在给定基下的坐标向量记为\(\widetilde{\boldsymbol{\alpha}}_1,\widetilde{\boldsymbol{\alpha}}_2,\cdots,\widetilde{\boldsymbol{\alpha}}_m,\widetilde{\boldsymbol{\beta}}\),则
\begin{enumerate}[(1)]
\item \(\boldsymbol{\alpha}_1,\boldsymbol{\alpha}_2,\cdots,\boldsymbol{\alpha}_m\)线性无关的充要条件是\(\widetilde{\boldsymbol{\alpha}}_1,\widetilde{\boldsymbol{\alpha}}_2,\cdots,\widetilde{\boldsymbol{\alpha}}_m\)线性无关.

\item \(\boldsymbol{\beta}\)可以用\(\boldsymbol{\alpha}_1,\boldsymbol{\alpha}_2,\cdots,\boldsymbol{\alpha}_m\)线性表示的充要条件是\(\widetilde{\boldsymbol{\beta}}\)可以用\(\widetilde{\boldsymbol{\alpha}}_1,\widetilde{\boldsymbol{\alpha}}_2,\cdots,\widetilde{\boldsymbol{\alpha}}_m\)线性表示,
并且线性表示的系数不变.即\begin{align*}
\boldsymbol{\beta }=c_1\boldsymbol{\alpha }_1+c_2\boldsymbol{\alpha }_2+\cdots +c_m\boldsymbol{\alpha }_m\Leftrightarrow \widetilde{\boldsymbol{\beta }}=c_1\widetilde{\boldsymbol{\alpha }}_1+c_2\widetilde{\boldsymbol{\alpha }}_2+\cdots +c_m\widetilde{\boldsymbol{\alpha }}_m.
\end{align*}

\item \(\boldsymbol{\alpha}_{i_1},\boldsymbol{\alpha}_{i_2},\cdots,\boldsymbol{\alpha}_{i_r}\)是向量组\(\boldsymbol{\alpha}_1,\boldsymbol{\alpha}_2,\cdots,\boldsymbol{\alpha}_m\)的极大无关组的充要条件是\(\widetilde{\boldsymbol{\alpha}}_{i_1},\widetilde{\boldsymbol{\alpha}}_{i_2},\cdots,\widetilde{\boldsymbol{\alpha}}_{i_r}\)是向量组\(\widetilde{\boldsymbol{\alpha}}_1,\widetilde{\boldsymbol{\alpha}}_2,\cdots,\widetilde{\boldsymbol{\alpha}}_m\)的极大无关组. 特别地,我们有
\[
\mathrm{r}(\boldsymbol{\alpha}_1,\boldsymbol{\alpha}_2,\cdots,\boldsymbol{\alpha}_m)=\mathrm{r}(\widetilde{\boldsymbol{\alpha}}_1,\widetilde{\boldsymbol{\alpha}}_2,\cdots,\widetilde{\boldsymbol{\alpha}}_m).
\]
\end{enumerate}
\end{theorem}
\begin{note}
由上述定理,我们可以将抽象线性空间\(V\)中向量组线性关系的判定和秩的计算,转化为具体列向量空间\(\mathbb{K}^n\)中由它们的坐标向量构成的列向量组线性关系的判定和秩的计算. 由于后者通常可以通过矩阵的方法来处理,故上述过程被称为“几何问题代数化”.
\end{note}
\begin{proof}
将\hyperref[theorem:线性空间的同构]{定理\ref{theorem:线性空间的同构}}运用到\hyperlink{线性同构1}{线性同构$\eta$}上就能得到证明.
\end{proof}

\begin{proposition}\label{proposition:系数矩阵与向量组的秩}
设\(\boldsymbol{\alpha}_1,\boldsymbol{\alpha}_2,\cdots,\boldsymbol{\alpha}_k;\boldsymbol{\beta}_1,\boldsymbol{\beta}_2,\cdots,\boldsymbol{\beta}_m\)是向量空间\(V\)中的向量,且满足:
\[
\begin{cases}
\boldsymbol{\beta}_1 = c_{11}\boldsymbol{\alpha}_1 + c_{12}\boldsymbol{\alpha}_2+\cdots + c_{1k}\boldsymbol{\alpha}_k,\\
\boldsymbol{\beta}_2 = c_{21}\boldsymbol{\alpha}_1 + c_{22}\boldsymbol{\alpha}_2+\cdots + c_{2k}\boldsymbol{\alpha}_k,\\
\cdots\cdots\cdots\cdots\\
\boldsymbol{\beta}_m = c_{m1}\boldsymbol{\alpha}_1 + c_{m2}\boldsymbol{\alpha}_2+\cdots + c_{mk}\boldsymbol{\alpha}_k.
\end{cases}
\]
记上述表示式中的系数矩阵为\(\boldsymbol{C}=(c_{ij})_{m\times k}\),则
\begin{enumerate}[(1)]
\item 若\(\mathrm{r}(\boldsymbol{C}) = k\),则这两组向量等价.
\item 若\(\mathrm{r}(\boldsymbol{C}) = r\),则向量组\(\boldsymbol{\beta}_1,\boldsymbol{\beta}_2,\cdots,\boldsymbol{\beta}_m\)的秩不超过\(r\).
\end{enumerate}
\end{proposition}
\begin{proof}
\begin{enumerate}[(1)]
\item 在\(V\)中取定一组基\(\boldsymbol{e}_1,\boldsymbol{e}_2,\cdots,\boldsymbol{e}_n\),假设在这组基下\(\boldsymbol{\alpha}_i\)的坐标向量是\(\widetilde{\boldsymbol{\alpha}}_i(1\leq i\leq k)\),\(\boldsymbol{\beta}_j\)的坐标向量是\(\widetilde{\boldsymbol{\beta}}_j(1\leq j\leq m)\),则
\[
\begin{cases}
\widetilde{\boldsymbol{\beta}}_1 = c_{11}\widetilde{\boldsymbol{\alpha}}_1 + c_{12}\widetilde{\boldsymbol{\alpha}}_2+\cdots + c_{1k}\widetilde{\boldsymbol{\alpha}}_k,\\
\widetilde{\boldsymbol{\beta}}_2 = c_{21}\widetilde{\boldsymbol{\alpha}}_1 + c_{22}\widetilde{\boldsymbol{\alpha}}_2+\cdots + c_{2k}\widetilde{\boldsymbol{\alpha}}_k,\\
\cdots\cdots\cdots\cdots\\
\widetilde{\boldsymbol{\beta}}_m = c_{m1}\widetilde{\boldsymbol{\alpha}}_1 + c_{m2}\widetilde{\boldsymbol{\alpha}}_2+\cdots + c_{mk}\widetilde{\boldsymbol{\alpha}}_k,
\end{cases}
\]
写成矩阵形式为
\[
(\widetilde{\boldsymbol{\beta}}_1,\widetilde{\boldsymbol{\beta}}_2,\cdots,\widetilde{\boldsymbol{\beta}}_m)=(\widetilde{\boldsymbol{\alpha}}_1,\widetilde{\boldsymbol{\alpha}}_2,\cdots,\widetilde{\boldsymbol{\alpha}}_k)\boldsymbol{C}'.
\]
因为\(\boldsymbol{C}'\)是一个行满秩\(k\times m\)矩阵,故由\hyperref[proposition:行/列满秩矩阵性质]{行满秩矩阵性质}可知,存在\(m\times k\)矩阵\(\boldsymbol{T}\),使得\(\boldsymbol{C}'\boldsymbol{T}=\boldsymbol{I}_k\),于是
\[
(\widetilde{\boldsymbol{\beta}}_1,\widetilde{\boldsymbol{\beta}}_2,\cdots,\widetilde{\boldsymbol{\beta}}_m)\boldsymbol{T}=(\widetilde{\boldsymbol{\alpha}}_1,\widetilde{\boldsymbol{\alpha}}_2,\cdots,\widetilde{\boldsymbol{\alpha}}_k).
\]
这表明\(\boldsymbol{\alpha}_1,\boldsymbol{\alpha}_2,\cdots,\boldsymbol{\alpha}_k\)可用\(\boldsymbol{\beta}_1,\boldsymbol{\beta}_2,\cdots,\boldsymbol{\beta}_m\)来线性表示,于是这两组向量等价.

\item 在\(V\)中取定一组基\(\boldsymbol{e}_1,\boldsymbol{e}_2,\cdots,\boldsymbol{e}_n\),假设在这组基下\(\boldsymbol{\alpha}_i\)的坐标向量是\(\widetilde{\boldsymbol{\alpha}}_i(1\leq i\leq k)\),\(\boldsymbol{\beta}_j\)的坐标向量是\(\widetilde{\boldsymbol{\beta}}_j(1\leq j\leq m)\),则
\[
\begin{cases}
\widetilde{\boldsymbol{\beta}}_1 = c_{11}\widetilde{\boldsymbol{\alpha}}_1 + c_{12}\widetilde{\boldsymbol{\alpha}}_2+\cdots + c_{1k}\widetilde{\boldsymbol{\alpha}}_k,\\
\widetilde{\boldsymbol{\beta}}_2 = c_{21}\widetilde{\boldsymbol{\alpha}}_1 + c_{22}\widetilde{\boldsymbol{\alpha}}_2+\cdots + c_{2k}\widetilde{\boldsymbol{\alpha}}_k,\\
\cdots\cdots\cdots\cdots\\
\widetilde{\boldsymbol{\beta}}_m = c_{m1}\widetilde{\boldsymbol{\alpha}}_1 + c_{m2}\widetilde{\boldsymbol{\alpha}}_2+\cdots + c_{mk}\widetilde{\boldsymbol{\alpha}}_k,
\end{cases}
\]
写成矩阵形式为
\[
(\widetilde{\boldsymbol{\beta}}_1,\widetilde{\boldsymbol{\beta}}_2,\cdots,\widetilde{\boldsymbol{\beta}}_m)=(\widetilde{\boldsymbol{\alpha}}_1,\widetilde{\boldsymbol{\alpha}}_2,\cdots,\widetilde{\boldsymbol{\alpha}}_k)\boldsymbol{C}'.
\]
由于两个矩阵乘积的秩不超过每个矩阵的秩,因此
\begin{align*}
\mathrm{r}\left( \boldsymbol{\beta }_1,\boldsymbol{\beta }_2,\cdots ,\boldsymbol{\beta }_m \right) =\mathrm{r}\left( \widetilde{\boldsymbol{\beta }}_1,\widetilde{\boldsymbol{\beta }}_2,\cdots ,\widetilde{\boldsymbol{\beta }}_m \right) =\mathrm{r}\left( \left( \widetilde{\boldsymbol{\alpha }}_1,\widetilde{\boldsymbol{\alpha }}_2,\cdots ,\widetilde{\boldsymbol{\alpha }}_k \right) \boldsymbol{C}' \right) \leqslant \mathrm{r}\left( \boldsymbol{C}'\right) =\mathrm{r}\left( \boldsymbol{C} \right) =r.
\end{align*}
\end{enumerate}
\end{proof}

\begin{proposition}\label{proposition:向量方程的解空间}
设\(\boldsymbol{\alpha}_1,\boldsymbol{\alpha}_2,\cdots,\boldsymbol{\alpha}_m\)是数域\(\mathbb{F}\)上\(n\)维线性空间\(V\)中的\(m\)个向量,且已知它们的秩等于\(r\). 求证: 全体满足\(x_1\boldsymbol{\alpha}_1 + x_2\boldsymbol{\alpha}_2+\cdots + x_m\boldsymbol{\alpha}_m = \boldsymbol{0}\)的列向量\((x_1,x_2,\cdots,x_m)'(x_i\in\mathbb{F})\)构成数域\(\mathbb{F}\)上\(m\)维列向量空间\(\mathbb{F}^m\)的\(m - r\)维子空间.
\end{proposition}
\begin{proof}
在\(V\)中引进基以后,记\(\widetilde{\boldsymbol{\alpha}}_i\)是\(\boldsymbol{\alpha}_i\)的坐标向量,则\(x_1\boldsymbol{\alpha}_1 + x_2\boldsymbol{\alpha}_2+\cdots + x_m\boldsymbol{\alpha}_m = \boldsymbol{0}\)等价于\(x_1\widetilde{\boldsymbol{\alpha}}_1 + x_2\widetilde{\boldsymbol{\alpha}}_2+\cdots + x_m\widetilde{\boldsymbol{\alpha}}_m = \boldsymbol{0}\). 而后者是一个齐次线性方程组,其系数矩阵的秩等于\(r\)(将\(x_i\)视为未知数),故其解构成\(\mathbb{F}^m\)的\(m - r\)维子空间.
\end{proof}


\begin{example}
设\(\{\boldsymbol{e}_1,\boldsymbol{e}_2,\cdots,\boldsymbol{e}_n\}\)是线性空间\(V\)的一组基,问:\(\{\boldsymbol{e}_1,\boldsymbol{e}_1+\boldsymbol{e}_2,\cdots,\boldsymbol{e}_1+\boldsymbol{e}_2+\cdots+\boldsymbol{e}_n\}\)是否也是\(V\)的基?
\end{example}
\begin{note}
利用\hyperref[theorem:线性空间的同构]{定理\ref{theorem:线性空间的同构}}即可.
\end{note}
\begin{solution}
将\(\{\boldsymbol{e}_1,\boldsymbol{e}_1+\boldsymbol{e}_2,\cdots,\boldsymbol{e}_1+\boldsymbol{e}_2+\cdots+\boldsymbol{e}_n\}\)对应的坐标向量拼成如下矩阵:
\[
\boldsymbol{A}=\begin{pmatrix}
1&1&\cdots&1\\
0&1&\cdots&1\\
\vdots&\vdots&&\vdots\\
0&0&\cdots&1
\end{pmatrix}.
\]
显然\(|\boldsymbol{A}| = 1\),从而\(\boldsymbol{A}\)是满秩阵,于是\(\{\boldsymbol{e}_1,\boldsymbol{e}_1+\boldsymbol{e}_2,\cdots,\boldsymbol{e}_1+\boldsymbol{e}_2+\cdots+\boldsymbol{e}_n\}\)也是\(V\)的一组基. 
\end{solution}

\begin{example}
已知向量组\(\{\boldsymbol{\alpha}_1,\boldsymbol{\alpha}_2,\cdots,\boldsymbol{\alpha}_s\}(s > 1)\)是线性空间\(V\)的一组基,设\(\boldsymbol{\beta}_1=\boldsymbol{\alpha}_1+\boldsymbol{\alpha}_2,\boldsymbol{\beta}_2=\boldsymbol{\alpha}_2+\boldsymbol{\alpha}_3,\cdots,\boldsymbol{\beta}_s=\boldsymbol{\alpha}_s+\boldsymbol{\alpha}_1\). 讨论向量\(\boldsymbol{\beta}_1,\boldsymbol{\beta}_2,\cdots,\boldsymbol{\beta}_s\)的线性相关性.
\end{example}
\begin{note}
利用\hyperref[theorem:线性空间的同构]{定理\ref{theorem:线性空间的同构}}即可.
\end{note}
\begin{solution}
将\(\boldsymbol{\beta}_1,\boldsymbol{\beta}_2,\cdots,\boldsymbol{\beta}_s\)对应的坐标向量拼成如下矩阵:
\[
\boldsymbol{A}=\begin{pmatrix}
1&0&\cdots&1\\
1&1&\cdots&0\\
0&1&\cdots&0\\
\vdots&\vdots&&\vdots\\
0&0&\cdots&1
\end{pmatrix}.
\]
经计算可得\(|\boldsymbol{A}| = 1+(-1)^{s + 1}\). 因此当\(s\)为偶数时,\(|\boldsymbol{A}| = 0\),从而向量\(\boldsymbol{\beta}_1,\boldsymbol{\beta}_2,\cdots,\boldsymbol{\beta}_s\)线性相关;当\(s\)为奇数时,\(|\boldsymbol{A}| = 2\),从而向量\(\boldsymbol{\beta}_1,\boldsymbol{\beta}_2,\cdots,\boldsymbol{\beta}_s\)线性无关. 
\end{solution}

\begin{example}
设\(\{\boldsymbol{e}_1,\boldsymbol{e}_2,\boldsymbol{e}_3,\boldsymbol{e}_4\}\)是线性空间\(V\)的一组基,已知
\[
\begin{cases}
\boldsymbol{\alpha}_1=\boldsymbol{e}_1+\boldsymbol{e}_2+\boldsymbol{e}_3 + 3\boldsymbol{e}_4,\\
\boldsymbol{\alpha}_2=-\boldsymbol{e}_1-3\boldsymbol{e}_2 + 5\boldsymbol{e}_3+\boldsymbol{e}_4,\\
\boldsymbol{\alpha}_3=3\boldsymbol{e}_1+2\boldsymbol{e}_2-\boldsymbol{e}_3 + 4\boldsymbol{e}_4,\\
\boldsymbol{\alpha}_4=-2\boldsymbol{e}_1-6\boldsymbol{e}_2 + 10\boldsymbol{e}_3+2\boldsymbol{e}_4,
\end{cases}
\]
求\(\boldsymbol{\alpha}_1,\boldsymbol{\alpha}_2,\boldsymbol{\alpha}_3,\boldsymbol{\alpha}_4\)的一个极大无关组.
\end{example}
\begin{note}
利用\hyperref[theorem:线性空间的同构]{定理\ref{theorem:线性空间的同构}}即可.
\end{note}
\begin{solution}
将\(\boldsymbol{\alpha}_1,\boldsymbol{\alpha}_2,\boldsymbol{\alpha}_3,\boldsymbol{\alpha}_4\)对应的坐标向量拼成如下矩阵,并用初等行变换将其化为阶梯形矩阵:
\[
\boldsymbol{A}=\begin{pmatrix}
1&-1&3&-2\\
1&-3&2&-6\\
1&5&-1&10\\
3&1&4&2
\end{pmatrix}\to\begin{pmatrix}
1&-1&3&-2\\
0&-2&-1&-4\\
0&0&-7&0\\
0&0&0&0
\end{pmatrix}.
\]
因此,矩阵\(\boldsymbol{A}\)的第一列、第二列和第三列是坐标向量组的极大无关组,从而\(\boldsymbol{\alpha}_1,\boldsymbol{\alpha}_2,\boldsymbol{\alpha}_3\)是\(\boldsymbol{\alpha}_1,\boldsymbol{\alpha}_2,\boldsymbol{\alpha}_3,\boldsymbol{\alpha}_4\)的一个极大无关组. 
\end{solution}

\begin{example}\label{example:3.110.1}
设\(a_1,a_2,\cdots,a_n\)是\(n\)个不同的数,\(\{\boldsymbol{e}_1,\boldsymbol{e}_2,\cdots,\boldsymbol{e}_n\}\)是线性空间\(V\)的一组基,已知
\[
\begin{cases}
\boldsymbol{\alpha}_1=\boldsymbol{e}_1 + a_1\boldsymbol{e}_2+\cdots + a_1^{n - 1}\boldsymbol{e}_n,\\
\boldsymbol{\alpha}_2=\boldsymbol{e}_1 + a_2\boldsymbol{e}_2+\cdots + a_2^{n - 1}\boldsymbol{e}_n,\\
\cdots\cdots\cdots\cdots\\
\boldsymbol{\alpha}_n=\boldsymbol{e}_1 + a_n\boldsymbol{e}_2+\cdots + a_n^{n - 1}\boldsymbol{e}_n,
\end{cases}
\]
求证:\(\{\boldsymbol{\alpha}_1,\boldsymbol{\alpha}_2,\cdots,\boldsymbol{\alpha}_n\}\)也是\(V\)的一组基.
\end{example}
\begin{note}
利用\hyperref[theorem:线性空间的同构]{定理\ref{theorem:线性空间的同构}}即可.
\end{note}
\begin{proof}
将\(\boldsymbol{\alpha}_1,\boldsymbol{\alpha}_2,\cdots,\boldsymbol{\alpha}_n\)对应的坐标向量拼成如下矩阵:
\[
\boldsymbol{A}=\begin{pmatrix}
1&1&\cdots&1\\
a_1&a_2&\cdots&a_n\\
\vdots&\vdots&&\vdots\\
a_1^{n - 1}&a_2^{n - 1}&\cdots&a_n^{n - 1}
\end{pmatrix}.
\]
显然,\(|\boldsymbol{A}|=\prod_{1\leq i<j\leq n}(a_j - a_i)\neq0\),故\(\boldsymbol{A}\)是满秩阵,从而\(\{\boldsymbol{\alpha}_1,\boldsymbol{\alpha}_2,\cdots,\boldsymbol{\alpha}_n\}\)也是\(V\)的一组基. 
\end{proof}

\section{基变换与过渡矩阵}

\begin{definition}[过渡矩阵]\label{definition:过渡矩阵}
设\(\{\boldsymbol{e}_1,\boldsymbol{e}_2,\cdots,\boldsymbol{e}_n\}\)和\(\{\boldsymbol{f}_1,\boldsymbol{f}_2,\cdots,\boldsymbol{f}_n\}\)是\(n\)维线性空间\(V\)的两组基,若
\[
\begin{cases}
\boldsymbol{f}_1 = a_{11}\boldsymbol{e}_1 + a_{21}\boldsymbol{e}_2+\cdots + a_{n1}\boldsymbol{e}_n,\\
\boldsymbol{f}_2 = a_{12}\boldsymbol{e}_1 + a_{22}\boldsymbol{e}_2+\cdots + a_{n2}\boldsymbol{e}_n,\\
\cdots\cdots\cdots\cdots\\
\boldsymbol{f}_n = a_{1n}\boldsymbol{e}_1 + a_{2n}\boldsymbol{e}_2+\cdots + a_{nn}\boldsymbol{e}_n,
\end{cases}
\]
则矩阵
\[
\boldsymbol{A}=\begin{pmatrix}
a_{11}&a_{12}&\cdots&a_{1n}\\
a_{21}&a_{22}&\cdots&a_{2n}\\
\vdots&\vdots&&\vdots\\
a_{n1}&a_{n2}&\cdots&a_{nn}
\end{pmatrix}
\]
称为从基\(\{\boldsymbol{e}_1,\boldsymbol{e}_2,\cdots,\boldsymbol{e}_n\}\)到基\(\{\boldsymbol{f}_1,\boldsymbol{f}_2,\cdots,\boldsymbol{f}_n\}\)的过渡矩阵.并且$\left( \boldsymbol{f}_1,\boldsymbol{f}_2,\cdots ,\boldsymbol{f}_n \right) =\left( \boldsymbol{e}_1,\boldsymbol{e}_2,\cdots ,\boldsymbol{e}_n \right) \boldsymbol{A}$.
\end{definition}

\begin{theorem}[同一向量在不同基下坐标向量的关系]\label{theorem:同一向量在不同基下坐标向量的关系}
设\(V\)是数域\(\mathbb{F}\)上\(n\)维线性空间,从基\(\{\boldsymbol{e}_1,\boldsymbol{e}_2,\cdots,\boldsymbol{e}_n\}\)到基\(\{\boldsymbol{f}_1,\boldsymbol{f}_2,\cdots,\boldsymbol{f}_n\}\)的过渡矩阵为\(\boldsymbol{A}=(a_{ij})\). 若\(V\)中向量\(\boldsymbol{\alpha}\)在基\(\{\boldsymbol{e}_1,\boldsymbol{e}_2,\cdots,\boldsymbol{e}_n\}\)下的坐标向量是\((x_1,x_2,\cdots,x_n)'\),在基\(\{\boldsymbol{f}_1,\boldsymbol{f}_2,\cdots,\boldsymbol{f}_n\}\)下的坐标向量是\((y_1,y_2,\cdots,y_n)'\),则
\begin{gather*}
\begin{pmatrix}
x_1\\
x_2\\
\vdots\\
x_n
\end{pmatrix}=\begin{pmatrix}
a_{11}&a_{12}&\cdots&a_{1n}\\
a_{21}&a_{22}&\cdots&a_{2n}\\
\vdots&\vdots&&\vdots\\
a_{n1}&a_{n2}&\cdots&a_{nn}
\end{pmatrix}\begin{pmatrix}
y_1\\
y_2\\
\vdots\\
y_n
\end{pmatrix}.
\end{gather*}
\end{theorem}
\begin{proof}
由过渡矩阵定义可得
\begin{align*}
\boldsymbol{\alpha }=\left( \boldsymbol{e}_1,\boldsymbol{e}_2,\cdots ,\boldsymbol{e}_n \right) \left( \begin{array}{c}
x_1\\
x_2\\
\vdots\\
x_n\\
\end{array} \right) =\left( \boldsymbol{f}_1,\boldsymbol{f}_2,\cdots ,\boldsymbol{f}_n \right) \left( \begin{array}{c}
y_1\\
y_2\\
\vdots\\
y_n\\
\end{array} \right) =\left( \boldsymbol{e}_1,\boldsymbol{e}_2,\cdots ,\boldsymbol{e}_n \right) \boldsymbol{A}\left( \begin{array}{c}
y_1\\
y_2\\
\vdots\\
y_n\\
\end{array} \right) .
\end{align*}
又因为$\left( \boldsymbol{e}_1,\boldsymbol{e}_2,\cdots ,\boldsymbol{e}_n \right) $可逆,所以
\begin{gather*}
\begin{pmatrix}
x_1\\
x_2\\
\vdots\\
x_n
\end{pmatrix}=\begin{pmatrix}
a_{11}&a_{12}&\cdots&a_{1n}\\
a_{21}&a_{22}&\cdots&a_{2n}\\
\vdots&\vdots&&\vdots\\
a_{n1}&a_{n2}&\cdots&a_{nn}
\end{pmatrix}\begin{pmatrix}
y_1\\
y_2\\
\vdots\\
y_n
\end{pmatrix}.
\end{gather*}
\end{proof}

\begin{theorem}\label{theorem:两次基变换后的过渡矩阵}
矩阵\(\boldsymbol{A}\)是\(n\)维线性空间\(V\)的基\(\{\boldsymbol{e}_1,\boldsymbol{e}_2,\cdots,\boldsymbol{e}_n\}\)到基\(\{\boldsymbol{f}_1,\boldsymbol{f}_2,\cdots,\boldsymbol{f}_n\}\)的过渡矩阵,则\(\boldsymbol{A}\)是可逆矩阵且从基\(\{\boldsymbol{f}_1,\boldsymbol{f}_2,\cdots,\boldsymbol{f}_n\}\)到基\(\{\boldsymbol{e}_1,\boldsymbol{e}_2,\cdots,\boldsymbol{e}_n\}\)的过渡矩阵为\(\boldsymbol{A}^{-1}\). 又若\(\boldsymbol{B}\)是从基\(\{\boldsymbol{f}_1,\boldsymbol{f}_2,\cdots,\boldsymbol{f}_n\}\)到基\(\{\boldsymbol{g}_1,\boldsymbol{g}_2,\cdots,\boldsymbol{g}_n\}\)的过渡矩阵,则从基\(\{\boldsymbol{e}_1,\boldsymbol{e}_2,\cdots,\boldsymbol{e}_n\}\)到基\(\{\boldsymbol{g}_1,\boldsymbol{g}_2,\cdots,\boldsymbol{g}_n\}\)的过渡矩阵为\(\boldsymbol{A}\boldsymbol{B}\).
\end{theorem}

\begin{example}
设\(\{\boldsymbol{u}_1,\boldsymbol{u}_2,\cdots,\boldsymbol{u}_n\},\{\boldsymbol{e}_1,\boldsymbol{e}_2,\cdots,\boldsymbol{e}_n\},\{\boldsymbol{f}_1,\boldsymbol{f}_2,\cdots,\boldsymbol{f}_n\}\)是向量空间\(V\)的3组基. 若从\(\boldsymbol{u}_1,\boldsymbol{u}_2,\cdots,\boldsymbol{u}_n\)到\(\boldsymbol{e}_1,\boldsymbol{e}_2,\cdots,\boldsymbol{e}_n\)的过渡矩阵是\(\boldsymbol{A}\),从\(\boldsymbol{u}_1,\boldsymbol{u}_2,\cdots,\boldsymbol{u}_n\)到\(\boldsymbol{f}_1,\boldsymbol{f}_2,\cdots,\boldsymbol{f}_n\)的过渡矩阵是\(\boldsymbol{B}\),求从\(\boldsymbol{e}_1,\boldsymbol{e}_2,\cdots,\boldsymbol{e}_n\)到\(\boldsymbol{f}_1,\boldsymbol{f}_2,\cdots,\boldsymbol{f}_n\)的过渡矩阵.
\end{example}
\begin{solution}
从\(\boldsymbol{e}_1,\boldsymbol{e}_2,\cdots,\boldsymbol{e}_n\)到\(\boldsymbol{u}_1,\boldsymbol{u}_2,\cdots,\boldsymbol{u}_n\)的过渡矩阵为\(\boldsymbol{A}^{-1}\),故从\(\boldsymbol{e}_1,\boldsymbol{e}_2,\cdots,\boldsymbol{e}_n\)到\(\boldsymbol{f}_1,\boldsymbol{f}_2,\cdots,\boldsymbol{f}_n\)的过渡矩阵为\(\boldsymbol{A}^{-1}\boldsymbol{B}\).
\end{solution}

\begin{example}
在四维行向量空间中求从基\(\boldsymbol{e}_1,\boldsymbol{e}_2,\cdots,\boldsymbol{e}_n\)到\(\boldsymbol{f}_1,\boldsymbol{f}_2,\cdots,\boldsymbol{f}_n\)的过渡矩阵,其中
\begin{align*}
\boldsymbol{e}_1&=(1,1,0,1),\boldsymbol{e}_2=(2,1,2,0),\boldsymbol{e}_3=(1,1,0,0),\boldsymbol{e}_4=(0,1,-1,-1),\\
\boldsymbol{f}_1&=(1,0,0,1),\boldsymbol{f}_2=(0,0,1,-1),\boldsymbol{f}_3=(2,1,0,3),\boldsymbol{f}_4=(-1,0,1,2).
\end{align*}
\end{example}
\begin{note}
这类题如用求解线性方程组的方法比较繁,可采用下列方法.
\end{note}
\begin{solution}
设该向量空间的标准基为
\[
\boldsymbol{u}_1=(1,0,0,0),\boldsymbol{u}_2=(0,1,0,0),\boldsymbol{u}_3=(0,0,1,0),\boldsymbol{u}_4=(0,0,0,1),
\]
则由条件可知从\(\boldsymbol{u}_1,\boldsymbol{u}_2,\boldsymbol{u}_3,\boldsymbol{u}_4\)到\(\boldsymbol{e}_1,\boldsymbol{e}_2,\boldsymbol{e}_3,\boldsymbol{e}_4\)的过渡矩阵为
\[
\boldsymbol{A}=\begin{pmatrix}
1&2&1&0\\
1&1&1&1\\
0&2&0&-1\\
1&0&0&-1
\end{pmatrix},
\]
于是就有
\begin{align}\label{equation:3.131.1}
\left( \boldsymbol{e}_1,\boldsymbol{e}_2,\boldsymbol{e}_3,\boldsymbol{e}_4 \right) =\left( \boldsymbol{u}_1,\boldsymbol{u}_2,\boldsymbol{u}_3,\boldsymbol{u}_4 \right) \boldsymbol{A}\Rightarrow \left( \boldsymbol{u}_1,\boldsymbol{u}_2,\boldsymbol{u}_3,\boldsymbol{u}_4 \right) =\left( \boldsymbol{e}_1,\boldsymbol{e}_2,\boldsymbol{e}_3,\boldsymbol{e}_4 \right) \boldsymbol{A}^{-1}.
\end{align}
又由条件可知从\(\boldsymbol{u}_1,\boldsymbol{u}_2,\boldsymbol{u}_3,\boldsymbol{u}_4\)到\(\boldsymbol{f}_1,\boldsymbol{f}_2,\boldsymbol{f}_3,\boldsymbol{f}_4\)的过渡矩阵为
\[
\boldsymbol{B}=\begin{pmatrix}
1&0&2&-1\\
0&0&1&0\\
0&1&0&1\\
1&-1&3&2
\end{pmatrix}.
\]
于是就有
\begin{align}\label{equation:3.131.2}
\left( \boldsymbol{f}_1,\boldsymbol{f}_2,\boldsymbol{f}_3,\boldsymbol{f}_4 \right) =\left( \boldsymbol{u}_1,\boldsymbol{u}_2,\boldsymbol{u}_3,\boldsymbol{u}_4 \right) \boldsymbol{B}.
\end{align}
从而由\eqref{equation:3.131.1}\eqref{equation:3.131.2}式可得
\begin{align*}
\left( \boldsymbol{f}_1,\boldsymbol{f}_2,\boldsymbol{f}_3,\boldsymbol{f}_4 \right) =\left( \boldsymbol{u}_1,\boldsymbol{u}_2,\boldsymbol{u}_3,\boldsymbol{u}_4 \right) \boldsymbol{B}=\left( \boldsymbol{e}_1,\boldsymbol{e}_2,\boldsymbol{e}_3,\boldsymbol{e}_4 \right) \boldsymbol{A}^{-1}\boldsymbol{B}.
\end{align*}
故从基\(\boldsymbol{e}_1,\boldsymbol{e}_2,\boldsymbol{e}_3,\boldsymbol{e}_4\)到\(\boldsymbol{f}_1,\boldsymbol{f}_2,\boldsymbol{f}_3,\boldsymbol{f}_4\)的过渡矩阵为\(\boldsymbol{A}^{-1}\boldsymbol{B}\). 它可以用初等变换和求逆矩阵类似的方法直接求得(对矩阵\((\boldsymbol{A}|\boldsymbol{B})\)进行初等行变换,将\(\boldsymbol{A}\)化为单位矩阵,则右边一块就化为了\(\boldsymbol{A}^{-1}\boldsymbol{B}\))
因此,所求之过渡矩阵为
\[
\begin{pmatrix}
-1&0&1&5\\
-1&1&-1&2\\
4&-2&3&-10\\
-2&1&-2&3
\end{pmatrix}.
\] 
\end{solution}

\begin{example}
设\(a\)为常数,求向量\(\boldsymbol{\alpha}=(a_1,a_2,\cdots,a_n)\)在基\(\{\boldsymbol{f}_1=(a^{n - 1},a^{n - 2},\cdots,a,1),\boldsymbol{f}_2=(a^{n - 2},a^{n - 3},\cdots,1,0),\cdots,\boldsymbol{f}_n=(1,0,\cdots,0,0)\}\)下的坐标.
\end{example}
\begin{proof}
设\(\boldsymbol{e}_1,\boldsymbol{e}_2,\cdots,\boldsymbol{e}_n\)是标准单位行向量,则\(\boldsymbol{\alpha}\)在\(\{\boldsymbol{e}_1,\boldsymbol{e}_2,\cdots,\boldsymbol{e}_n\}\)下的坐标向量就是$\boldsymbol{\alpha}'$,并且从\(\{\boldsymbol{e}_1,\boldsymbol{e}_2,\cdots,\boldsymbol{e}_n\}\)到\(\{\boldsymbol{f}_1,\boldsymbol{f}_2,\cdots,\boldsymbol{f}_n\}\)的过渡矩阵是
\[
\boldsymbol{A}=\begin{pmatrix}
a^{n - 1}&a^{n - 2}&\cdots&1\\
a^{n - 2}&a^{n - 3}&\cdots&0\\
\vdots&\vdots&&\vdots\\
a&1&\cdots&0\\
1&0&\cdots&0
\end{pmatrix}.
\]
设\(\boldsymbol{\alpha}\)在\(\{\boldsymbol{f}_1,\boldsymbol{f}_2,\cdots,\boldsymbol{f}_n\}\)下的坐标向量为\(\boldsymbol{x}=(x_1,x_2,\cdots,x_n)\).则由\hyperref[theorem:同一向量在不同基下坐标向量的关系]{同一向量在不同基下坐标向量的关系}有\(\boldsymbol{A}\boldsymbol{x}'=\boldsymbol{\alpha}'\). 这是一个非齐次线性方程组,可由初等行变换求出方程组的解:
\begin{gather*}
\left( \begin{matrix}
a^{n-1}&		a^{n-2}&		\cdots&		1&		a_1\\
a^{n-2}&		a^{n-3}&		\cdots&		0&		a_2\\
\vdots&		\vdots&		&		\vdots&		\vdots\\
a&		1&		\cdots&		0&		a_{n-1}\\
1&		0&		\cdots&		0&		a_n\\
\end{matrix} \right) \rightarrow \left( \begin{matrix}
0&		a^{n-2}&		\cdots&		1&		a_1-a^{n-1}a_n\\
0&		a^{n-3}&		\cdots&		0&		a_2-a^{n-2}a_n\\
\vdots&		\vdots&		&		\vdots&		\vdots\\
0&		1&		\cdots&		0&		a_{n-1}-aa_n\\
1&		0&		\cdots&		0&		a_n\\
\end{matrix} \right) 
\\
\rightarrow \left( \begin{matrix}
0&		0&		\cdots&		1&		a_1-aa_2\\
0&		0&		\cdots&		0&		a_2-aa_3\\
\vdots&		\vdots&		&		\vdots&		\vdots\\
0&		1&		\cdots&		0&		a_{n-1}-aa_n\\
1&		0&		\cdots&		0&		a_n\\
\end{matrix} \right) \rightarrow \left( \begin{matrix}
1&		0&		\cdots&		0&		a_n\\
0&		1&		\cdots&		0&		a_{n-1}-aa_n\\
\vdots&		\vdots&		&		\vdots&		\vdots\\
0&		0&		\cdots&		0&		a_2-aa_3\\
0&		0&		\cdots&		1&		a_1-aa_2\\
\end{matrix} \right),
\end{gather*}
因此\(\boldsymbol{x}=(a_n,a_{n - 1} - a a_n,\cdots,a_2 - a a_3,a_1 - a a_2)\).
\end{proof}

\begin{example}
设\(V\)是次数不超过\(n\)的实系数多项式全体组成的线性空间,求从基\(\{1,x,x^2,\cdots,x^n\}\)到基\(\{1,x - a,(x - a)^2,\cdots,(x - a)^n\}\)的过渡矩阵,并以此证明多项式的Taylor公式:
\[
f(x)=f(a)+\frac{f'(a)}{1!}(x - a)+\frac{f''(a)}{2!}(x - a)^2+\cdots+\frac{f^{(n)}(a)}{n!}(x - a)^n,
\]
其中\(f^{(n)}(x)\)表示\(f(x)\)的\(n\)次导数.
\end{example}
\begin{solution}
从基\(\{1,x,x^2,\cdots,x^n\}\)到基\(\{1,x - a,(x - a)^2,\cdots,(x - a)^n\}\)的过渡矩阵\((n + 1\)阶\()\)利用二项式定理容易求出为
\[
\boldsymbol{P}=\begin{pmatrix}
1&-a&a^2&\cdots&(-1)^na^n\\
0&1&-2a&\cdots&(-1)^{n - 1}na^{n - 1}\\
0&0&1&\cdots&(-1)^{n - 2}\frac{n(n - 1)}{2!}a^{n - 2}\\
\vdots&\vdots&\vdots&&\vdots\\
0&0&0&\cdots&1
\end{pmatrix}.
\]
注意\(\boldsymbol{P}\)的逆矩阵实际上就是从基\(\{1,x - a,(x - a)^2,\cdots,(x - a)^n\}\)到基\(\{1,x,x^2,\cdots,x^n\}\)的过渡矩阵,结合$x^n=[(x-a)+a]^n$,再利用二项式定理可以马上得到(不必用初等变换法求逆矩阵):
\[
\boldsymbol{P}^{-1}=\begin{pmatrix}
1&a&a^2&\cdots&a^n\\
0&1&2a&\cdots&na^{n - 1}\\
0&0&1&\cdots&\frac{n(n - 1)}{2!}a^{n - 2}\\
\vdots&\vdots&\vdots&&\vdots\\
0&0&0&\cdots&1
\end{pmatrix}.
\]
设\(f(x)=a_0 + a_1x + a_2x^2+\cdots + a_nx^n\),则\(f(x)\)在基\(\{1,x ,x^2,\cdots,x^n\}\)下的坐标向量为$(a_0,a_1,\cdots,a_n)'$.设\(f(x)\)在基\(\{1,x - a,(x - a)^2,\cdots,(x - a)^n\}\)下的坐标向量为$(y_0,y_1,\cdots,y_n)'$.则由\hyperref[theorem:同一向量在不同基下坐标向量的关系]{同一向量在不同基下坐标向量的关系}可知
\begin{align*}
\left( \begin{array}{c}
a_0\\
a_1\\
a_2\\
\vdots\\
a_n\\
\end{array} \right) =\boldsymbol{P}\left( \begin{array}{c}
y_0\\
y_1\\
y_2\\
\vdots\\
y_n\\
\end{array} \right) .
\end{align*}
于是\begin{align*}
\left( \begin{array}{c}
y_0\\
y_1\\
y_2\\
\vdots\\
y_n\\
\end{array} \right) =\boldsymbol{P}^{-1}\left( \begin{array}{c}
a_0\\
a_1\\
a_2\\
\vdots\\
a_n\\
\end{array} \right) =\left( \begin{matrix}
1&		a&		a^2&		\cdots&		a^n\\
0&		1&		2a&		\cdots&		na^{n-1}\\
0&		0&		1&		\cdots&		\frac{n(n-1)}{2!}a^{n-2}\\
\vdots&		\vdots&		\vdots&		&		\vdots\\
0&		0&		0&		\cdots&		1\\
\end{matrix} \right) \left( \begin{array}{c}
a_0\\
a_1\\
a_2\\
\vdots\\
a_n\\
\end{array} \right) =\left( \begin{array}{c}
f(a)\\
\frac{f'(a)}{1!}\\
\vdots\\
\frac{f^{(n)}(a)}{n!}\\
\end{array} \right) .
\end{align*}
由此可得\begin{align*}
f(x)=\left( 1,x-a,(x-a)^2,\cdots ,(x-a)^n \right) \left( \begin{array}{c}
f(a)\\
\frac{f'(a)}{1!}\\
\vdots\\
\frac{f^{(n)}(a)}{n!}\\
\end{array} \right) =f(a)+\frac{f'(a)}{1!}(x-a)+\frac{f''(a)}{2!}(x-a)^2+\cdots +\frac{f^{(n)}(a)}{n!}(x-a)^n.        
\end{align*}
\end{solution}

\subsection{练习}

\begin{exercise}
验证下列映射是线性同构:
\begin{enumerate}[(1)]
\item 一维实行向量空间\(\mathbb{R}\),\hyperref[example:3.3(5)]{例题}\ref{example:3.3}\ref{example:3.3(5)}中的实线性空间\(\mathbb{R}^+\),映射\(\varphi:\mathbb{R}\to\mathbb{R}^+\)定义为\(\varphi(x)=\mathrm{e}^x\);
\item 二维实行向量空间\(\mathbb{R}_2\),\hyperref[example:3.3(6)]{例题}\ref{example:3.3}\ref{example:3.3(6)}中的实线性空间\(V\),映射\(\varphi:\mathbb{R}_2\to V\)定义为\(\varphi(a,b)=(a,b+\frac{1}{2}a^2)\).
\end{enumerate}
\end{exercise}
\begin{solution}
\begin{enumerate}[(1)]
\item  \(\varphi\)的逆映射是\(\psi:\mathbb{R}^+\to\mathbb{R},\psi(y)=\ln y\),故\(\varphi\)是一一对应的. 根据加法和数乘的定义可得
\[
\varphi(x + y)=\mathrm{e}^{x + y}=\mathrm{e}^x\mathrm{e}^y=\varphi(x)\oplus\varphi(y),\varphi(kx)=\mathrm{e}^{kx}=(\mathrm{e}^x)^k=k\circ\varphi(x),
\]
因此\(\varphi:\mathbb{R}\to\mathbb{R}^+\)是线性同构.
\item \(\varphi\)的逆映射是\(\psi:V\to\mathbb{R}_2,\psi(x,y)=(x,y-\frac{x^2}{2})\),故\(\varphi\)是一一对应的. 根据具体的计算可得
\[
\varphi(a_1 + a_2,b_1 + b_2)=\varphi(a_1,b_1)\oplus\varphi(a_2,b_2),\varphi(ka,kb)=k\circ\varphi(a,b),
\]
因此\(\varphi:\mathbb{R}_2\to V\)是线性同构.
\end{enumerate}
\end{solution}

\begin{exercise}
构造下列线性空间之间的线性同构:
\begin{enumerate}[(1)]
\item \(V\)是数域\(\mathbb{K}\)上的\(n\)阶上三角矩阵构成的线性空间,\(U\)是数域\(\mathbb{K}\)上的\(n\)阶对称矩阵构成的线性空间(\hyperref[example:一些常见线性空间的基(4)]{例题\ref{example:一些常见线性空间的基}\ref{example:一些常见线性空间的基(4)}});
\item \(V\)是数域\(\mathbb{K}\)上主对角元全为零的\(n\)阶上三角矩阵构成的线性空间,\(U\)是数域\(\mathbb{K}\)上的\(n\)阶反对称矩阵构成的线性空间(\hyperref[example:一些常见线性空间的基(6)]{例题\ref{example:一些常见线性空间的基}\ref{example:一些常见线性空间的基(6)}});
\item \(V\)是\(n\)阶Hermite矩阵构成的实线性空间,\(U\)是\(n\)阶斜Hermite矩阵构成的实线性空间(\hyperref[example:n阶(斜)Hermite矩阵全体构成的线性空间]{例题\ref{example:n阶(斜)Hermite矩阵全体构成的线性空间}}).
\end{enumerate}
\end{exercise}
\begin{solution}
\begin{enumerate}[(1)]
\item  \(\varphi:V\to U\)定义为:对任意的\(\boldsymbol{A}=(a_{ij})\in V\),当\(i\leq j\)时,矩阵\(\varphi(\boldsymbol{A})\)的第\((i,j)\)元素为\(a_{ij}\);当\(i > j\)时,矩阵\(\varphi(\boldsymbol{A})\)的第\((i,j)\)元素为\(a_{ji}\). 容易验证\(\varphi:V\to U\)是定义好的映射,并且是数域\(\mathbb{K}\)上的线性同构.
\item \(\varphi:V\to U\)定义为:对任意的\(\boldsymbol{A}=(a_{ij})\in V\),\(\varphi(\boldsymbol{A})=\boldsymbol{A}-\boldsymbol{A}'\). 容易验证\(\varphi:V\to U\)是定义好的映射,并且是数域\(\mathbb{K}\)上的线性同构.
\item \(\varphi:V\to U\)定义为:对任意的\(\boldsymbol{A}=(a_{ij})\in V\),\(\varphi(\boldsymbol{A})=\mathrm{i}\boldsymbol{A}\). 容易验证\(\varphi:V\to U\)是定义好的映射,并且是实数域上的线性同构. 注意到\(\varphi\)的逆映射\(\psi:U\to V\)为:\(\psi(\boldsymbol{B})=-\mathrm{i}\boldsymbol{B}\). 
\end{enumerate}
\end{solution}


\section{子空间、直和与商空间}

\begin{definition}[直和]\label{definition:直和}
设\(V_1,V_2,\cdots,V_k\)是线性空间\(V\)的子空间,若对任意的\(i(1\leq i\leq k)\),均有
\[
V_i\cap(V_1+\cdots+V_{i - 1}+V_{i + 1}+\cdots+V_k)=0,
\]
则称和\(V_1 + V_2+\cdots+V_k\)是直接和,简称直和,记为\(V_1\oplus V_2\oplus\cdots\oplus V_k\).
\end{definition}

\begin{theorem}[直和的等价条件]\label{theorem:直和的等价条件}
设\(V_1,V_2,\cdots,V_k\)是线性空间\(V_0\)的子空间,\(V_0 = V_1 + V_2+\cdots+V_k\),则下列命题等价:
\begin{enumerate}[(1)]
\item \label{theorem:直和的等价条件1}\(V_0 = V_1\oplus V_2\oplus\cdots\oplus V_k\);
\item \label{theorem:直和的等价条件2}对任意的\(2\leq i\leq k\),有\(V_i\cap(V_1 + V_2+\cdots+V_{i - 1}) = 0\);
\item \label{theorem:直和的等价条件3}\(\dim V_0=\dim V_1+\dim V_2+\cdots+\dim V_k\);
\item \label{theorem:直和的等价条件4}\(V_1,V_2,\cdots,V_k\)的一组基可以拼成\(V_0\)的一组基;
\item \label{theorem:直和的等价条件5}\(V_0\)中的向量表示为\(V_1,V_2,\cdots,V_k\)中的向量之和时其表示唯一.
\item 零向量表示唯一.
\end{enumerate}
\end{theorem}
\begin{proof}

\end{proof} 

\begin{theorem}[维数公式]\label{theorem:维数公式}
设\(V_1,V_2\)是线性空间\(V\)的两个子空间,则
\[
\dim(V_1 + V_2)=\dim V_1+\dim V_2-\dim(V_1\cap V_2).
\]
\end{theorem}
\begin{proof}

\end{proof}

\subsection{证明直和的方法}
证明直和的方法大致有两种:

第一种:先证和,再证直和. 

第二种:对于给定的\(V,V_1,V_2\),求证\(V = V_1\oplus V_2\)的题目,如果“和”不好证明的话,可以记\(W = V_1 + V_2\),先证\(W = V_1\oplus V_2\),再证\(V = W\)(证明\(V = W\)通常会利用\hyperref[proposition:与全空间维数相同的子空间等于全空间]{命题\ref{proposition:与全空间维数相同的子空间等于全空间}}).具体例子见\hyperref[example:561.16]{例题\ref{example:561.16}}

\begin{proposition}\label{proposition:矩阵空间可以分解为对称和反称矩阵空间的直和}
设\(V\)是数域\(\mathbb{F}\)上\(n\)阶矩阵组成的向量空间,\(V_1\)和\(V_2\)分别是\(\mathbb{F}\)上对称矩阵和反对称矩阵组成的子集. 求证:\(V_1\)和\(V_2\)都是\(V\)的子空间且\(V = V_1\oplus V_2\).
\end{proposition}
\begin{note}
要证明向量空间\(V\)是其子空间\(V_1,V_2\)的直和,只需证明两件事:一是证明\(V\)中任一向量均可表示为\(V_1\)与\(V_2\)中向量之和,即\(V = V_1 + V_2\);二是证明\(V_1\)与\(V_2\)的交等于零.
\end{note}
\begin{proof}
由于对称矩阵之和仍是对称矩阵,一个数乘以对称矩阵仍是对称矩阵,因此\(V_1\)是\(V\)的子空间. 同理\(V_2\)也是\(V\)的子空间. 又由\hyperref[proposition:任一阶方阵可表示为对称阵与反对称阵之和]{命题\ref{proposition:任一阶方阵可表示为对称阵与反对称阵之和}}可知,任一\(n\)阶矩阵都可以表示为一个对称矩阵和一个反对称矩阵之和,故\(V = V_1 + V_2\). 若一个矩阵既是对称矩阵又是反对称矩阵,则它一定是零矩阵. 这就是说\(V_1\cap V_2 =\mathbf{0}\). 于是\(V = V_1\oplus V_2\). 
\end{proof}

\begin{example}
设\(V_1,V_2\)分别是数域\(\mathbb{F}\)上的齐次线性方程组\(x_1 = x_2=\cdots = x_n\)与\(x_1 + x_2+\cdots + x_n = 0\)的解空间,求证:\(\mathbb{F}^n = V_1\oplus V_2\).
\end{example}
\begin{note}
要证明向量空间\(V\)是其子空间\(V_1,V_2\)的直和,只需证明两件事:一是证明\(V\)中任一向量均可表示为\(V_1\)与\(V_2\)中向量之和,即\(V = V_1 + V_2\);二是证明\(V_1\)与\(V_2\)的交等于零.
\end{note}
\begin{proof}
由线性方程组解的定理知,\(V_1\)的维数是\(1\),\(V_2\)的维数是\(n - 1\). 若列向量\(\boldsymbol{\alpha}\in V_1\cap V_2\),则\(\boldsymbol{\alpha}\)既是第一个线性方程组的解,也是第二个线性方程组的解,不难看出\(\boldsymbol{\alpha}\)只能等于零向量,因此\(V_1\cap V_2 = 0\). 又因为
\[
\dim(V_1\oplus V_2)=\dim V_1+\dim V_2=1+(n - 1)=n=\dim\mathbb{F}^n,
\]
故\(\mathbb{F}^n = V_1\oplus V_2\).
\end{proof}

\begin{example}
设\(U,V\)是数域\(\mathbb{K}\)上的两个线性空间,\(W = U\times V\)是\(U\)和\(V\)的积集合,即\(W=\{(\boldsymbol{u},\boldsymbol{v})|\boldsymbol{u}\in U,\boldsymbol{v}\in V\}\). 现在\(W\)上定义加法和数乘:
\[
(\boldsymbol{u}_1,\boldsymbol{v}_1)+(\boldsymbol{u}_2,\boldsymbol{v}_2)=(\boldsymbol{u}_1+\boldsymbol{u}_2,\boldsymbol{v}_1+\boldsymbol{v}_2),k(\boldsymbol{u},\boldsymbol{v})=(k\boldsymbol{u},k\boldsymbol{v}).
\]
验证:\(W\)是\(\mathbb{K}\)上的线性空间(这个线性空间称为\(U\)和\(V\)的外直和).

又若设\(U'=\{(\boldsymbol{u},\boldsymbol{0})|\boldsymbol{u}\in U\},V'=\{(\boldsymbol{0},\boldsymbol{v})|\boldsymbol{v}\in V\}\),求证:\(U',V'\)是\(W\)的子空间,\(U'\)和\(U\)同构,\(V'\)和\(V\)同构,并且\(W = U'\oplus V'\).
\end{example}
\begin{proof}
易验证\(W\)在上述加法和数乘下满足线性空间的8条公理,从而是\(\mathbb{K}\)上的线性空间. 任取\((\boldsymbol{u}_1,\boldsymbol{0}),(\boldsymbol{u}_2,\boldsymbol{0})\in U',k\in\mathbb{K}\),则\((\boldsymbol{u}_1,\boldsymbol{0})+(\boldsymbol{u}_2,\boldsymbol{0})=(\boldsymbol{u}_1+\boldsymbol{u}_2,\boldsymbol{0})\in U',k(\boldsymbol{u}_1,\boldsymbol{0})=(k\boldsymbol{u}_1,\boldsymbol{0})\in U'\),因此\(U'\)是\(W\)的子空间. 同理可证\(V'\)是\(W\)的子空间. 构造映射\(\varphi:U\to U',\varphi(\boldsymbol{u})=(\boldsymbol{u},\boldsymbol{0})\),容易验证\(\varphi\)是一一对应并且保持加法和数乘运算,所以\(\varphi:U\to U'\)是一个线性同构. 构造映射\(\psi:V\to V',\psi(\boldsymbol{v})=(\boldsymbol{0},\boldsymbol{v})\),同理可证\(\psi:V\to V'\)是一个线性同构. 显然\(U'\cap V' = 0\),又对\(W\)中任一向量\((\boldsymbol{u},\boldsymbol{v})\),有\((\boldsymbol{u},\boldsymbol{v})=(\boldsymbol{u},\boldsymbol{0})+(\boldsymbol{0},\boldsymbol{v})\in U'+V'\),因此\(W = U'\oplus V'\). 
\end{proof}

\begin{example}\label{example:561.16}
给定数域\(P\),设\(\boldsymbol{A}\)是数域\(P\)上的一个\(n\)级可逆方阵,\(\boldsymbol{A}\)的前\(r\)个行向量组成的矩阵为\(\boldsymbol{B}\),后\(n - r\)个行向量组成的矩阵为\(\boldsymbol{C}\),\(n\)元线性方程组\(\boldsymbol{B}\boldsymbol{X}=0\)与\(\boldsymbol{C}\boldsymbol{X}=0\)的解空间分别为\(V_1,V_2\),证\(P^n = V_1\oplus V_2\).
\end{example}
\begin{proof}
先记\(W = V_1 + V_2\). 若\(\boldsymbol{\alpha}\in V_1\cap V_2\),则\(\boldsymbol{B}\boldsymbol{\alpha}=\boldsymbol{C}\boldsymbol{\alpha}=0\),所以
\[
\boldsymbol{A}\boldsymbol{\alpha}=\begin{pmatrix}
\boldsymbol{B}\\
\boldsymbol{C}
\end{pmatrix}\boldsymbol{\alpha}=0.
\]
由于\(\boldsymbol{A}\)可逆,知\(\boldsymbol{\alpha}=0\),所以\(V_1\cap V_1 = \{0\}\),即\(W = V_1\oplus V_2\).

最后说\(W = P^n\):显然\(r(\boldsymbol{B}) = r\),\(r(\boldsymbol{C}) = n - r\),则\(\dim V_1 = n - r\),\(\dim V_2 = n-(n - r)=r\). 所以
\[
\dim W=\dim V_1+\dim V_2=n=\dim P^n.
\]
又\(W = V_1\oplus V_2\subseteq P^n\),从而\(W = P^n\),即
\[
P^n = V_1\oplus V_2.
\]  
\end{proof}


\begin{proposition}[任意子空间一定存在相应的补空间]\label{proposition:补空间}
设\(U\)是\(V\)的子空间,则一定存在\(V\)的子空间\(W\),使得\(V = U\oplus W\). 这样的子空间\(W\)称为子空间\(U\)在\(V\)中的\textbf{补空间}.
\end{proposition}
\begin{remark}
在这个命题中\(U\cap W = \{ \boldsymbol{0}\}\),而不是\(U\cap W=\varnothing\);同时\(V = U + W\)是子空间的和,而不是\(V = U\cup W\). 因此,补空间绝不是补集,请读者务必注意!一般来说,补空间并不唯一. 例如下面证明中,取$U$中不同的基,再将基扩张得到的补空间也不相同.
还例如,若\(\dim V-\dim U\geq1\)且\(\dim U\geq1\),则\(U\)有无限个补空间.
\end{remark}
\begin{proof}
取子空间\(U\)的一组基\(\{\boldsymbol{e}_1,\cdots,\boldsymbol{e}_m\}\),由\hyperref[theorem:基扩充定理]{基扩张定理}可将其扩张为\(V\)的一组基\(\{\boldsymbol{e}_1,\cdots,\boldsymbol{e}_m,\boldsymbol{e}_{m + 1},\cdots,\boldsymbol{e}_n\}\). 令\(W = L(\boldsymbol{e}_{m + 1},\cdots,\boldsymbol{e}_n)\),则\(V = U+W\). 由于\(\{\boldsymbol{e}_{m + 1},\cdots,\boldsymbol{e}_n\}\)是\(W\)的一组基,故\(\dim V=\dim U+\dim W\),从而\(V = U\oplus W\). 
\end{proof}

\begin{proposition}\label{proposition:直和的传递性}
若\(V = U\oplus W\)且\(U = U_1\oplus U_2\),求证:\(V = U_1\oplus U_2\oplus W\). 
\end{proposition}
\begin{proof}
由\(U = U_1\oplus U_2\)可得\(U_1\cap U_2 = 0\);由\(V = U\oplus W\)可得\((U_1 + U_2)\cap W=U\cap W = 0\),因此由\hyperref[theorem:直和的等价条件2]{定理\ref{theorem:直和的等价条件}\ref{theorem:直和的等价条件2}}可得\(U_1 + U_2+W\)是直和,从而\(V = U_1 + U_2+W = U_1\oplus U_2\oplus W\).
\end{proof}

\begin{proposition}\label{proposition:n维线性空间的一维直和分解}
每一个\(n\)维线性空间均可表示为\(n\)个一维子空间的直和.
\end{proposition}
\begin{proof}
设\(V\)是\(n\)维线性空间,取其一组基为\(\{\boldsymbol{e}_1,\boldsymbol{e}_2,\cdots,\boldsymbol{e}_n\}\). 设\(V_i = L(\boldsymbol{e}_i)(1\leq i\leq n)\),则\(V_i\)是\(V\)的一维子空间.任取$\alpha\in V$,存在唯一一组常数$k_1,k_2,\cdots,k_n$,使得$\alpha =k_1\boldsymbol{e}_1+k_2\boldsymbol{e}_2+\cdots +k_n\boldsymbol{e}_n$,而$k_i\boldsymbol{e}_i\in V_i,i=1,2,\cdots ,n.$因此\(V = V_1 + V_2+\cdots+V_n\). 注意到\(\dim V = n=\dim V_1+\dim V_2+\cdots+\dim V_n\),故由\hyperref[theorem:直和的等价条件3]{定理\ref{theorem:直和的等价条件}\ref{theorem:直和的等价条件3}}可知,\(V = V_1\oplus V_2\oplus\cdots\oplus V_n\). 

(注意到\(V_i\)的基是\(\{\boldsymbol{e}_i\}\),因此\(V_i(1\leq i\leq n)\)的基能拼成\(V\)的基,故由\hyperref[theorem:直和的等价条件4]{定理\ref{theorem:直和的等价条件}\ref{theorem:直和的等价条件4}}也可得到结论. 再注意到\(V\)中任一向量写成基向量\(\{\boldsymbol{e}_1,\boldsymbol{e}_2,\cdots,\boldsymbol{e}_n\}\)的线性组合时,其表示是唯一的. 这就是说,\(V\)中任一向量写成\(V_i\)中的向量之和时,其表示是唯一的,故由\hyperref[theorem:直和的等价条件5]{定理\ref{theorem:直和的等价条件}\ref{theorem:直和的等价条件5}}同样可得结论. )
\end{proof}

\begin{proposition}\label{proposition:真子空间至多包含n-1个基向量}
设\(V_0\)是数域\(\mathbb{F}\)上$n$维向量空间\(V\)的真子空间,则\(V_0\)至多包含$n-1$个$V$中的基向量.
\end{proposition}
\begin{proof}
反证法,若$V_0$包含\(n\)个\(V\)中的基向量,则$V_i$就包含了$V$的一组基.不妨设$V_0$中的这组基向量为$\{e_1,e_2,\cdots e_n\}$,则$\forall \alpha\in V$,有$\alpha =k_1e_1+k_2e_2+\cdots +k_ne_n \in V_0$,其中$k_i\in \mathbb{F}$,$i=1,2,\cdots,n$.故$V_0\supset V$,又$V_0\subset V$,因此$V_0=V$.这与\(V_0\)是\(V\)的真子空间矛盾.
\end{proof}

\begin{proposition}\label{proposition:真子空间外仍有向量存在}
设\(V_1,V_2,\cdots,V_m\)是数域\(\mathbb{F}\)上向量空间\(V\)的\(m\)个真子空间,证明:在\(V\)中必存在一个向量\(\boldsymbol{\alpha}\),它不属于任何一个\(V_i\).
\end{proposition}
\begin{note}
这个命题表明:\textbf{有限个真子空间不能覆盖全空间}.
\end{note}
\begin{proof}
{\color{blue}证法一:}
对个数\(m\)进行归纳,当\(m = 1\)时结论显然成立. 设\(m = k\)时结论成立,现要证明\(m = k + 1\)时结论也成立. 由归纳假设,存在向量\(\boldsymbol{\alpha}\),它不属于任何一个\(V_i(1\leq i\leq k)\). 若\(\boldsymbol{\alpha}\)也不属于\(V_{k + 1}\),则结论已成立,因此可设\(\boldsymbol{\alpha}\in V_{k + 1}\). 在\(V_{k + 1}\)外选一个向量\(\boldsymbol{\beta}\),作集合
\begin{align*}
M = \{t\boldsymbol{\alpha}+\boldsymbol{\beta}|t\in\mathbb{F}\}.
\end{align*}
事实上,我们可将\(M\)看成是通过\(\boldsymbol{\beta}\)的终点且平行于\(\boldsymbol{\alpha}\)的一根“直线”,现要证明它和每个\(V_i\)最多只有一个交点. 首先,\(M\)和\(V_{k + 1}\)无交点,因为若\(t\boldsymbol{\alpha}+\boldsymbol{\beta}\in V_{k + 1}\),则从\(t\boldsymbol{\alpha}\in V_{k + 1}\)可推出\(\boldsymbol{\beta}\in V_{k + 1}\),与假设矛盾. 又若对某个\(V_i(i<k + 1)\),存在\(t_1\neq t_2\),使得\(t_1\boldsymbol{\alpha}+\boldsymbol{\beta}\in V_i\),\(t_2\boldsymbol{\alpha}+\boldsymbol{\beta}\in V_i\),则\((t_1 - t_2)\boldsymbol{\alpha}\in V_i\),从而导致\(\boldsymbol{\alpha}\in V_i\),与假设矛盾. 因此,\(M\)和每个\(V_i\)最多只有一个交点,从而\(M\)中只有有限个向量属于\(V_i\)的并集,而\(t\)有无穷多个选择,由此即得结论. 

{\color{blue}证法二:}任取\(V\)的一组基\(\{\boldsymbol{e}_1,\boldsymbol{e}_2,\cdots,\boldsymbol{e}_n\}\). 对任意的正整数\(k\),构造\(V\)中向量\(\boldsymbol{\alpha}_k=\boldsymbol{e}_1 + k\boldsymbol{e}_2+\cdots + k^{n - 1}\boldsymbol{e}_n\),设向量族\(S = \{\boldsymbol{\alpha}_k|k = 1,2,\cdots\}\). 由\hyperref[example:3.110.1]{例题\ref{example:3.110.1}}可知,\(S\)中任意\(n\)个不同的向量都构成\(V\)的一组基. 因为\(V_i\)都是\(V\)的真子空间,所以每个\(V_i\)至多包含\(S\)中\(n - 1\)个向量.因此$\bigcup_{i=1}^m{V_i}$至多包含$S$中$m(n-1)$个向量.
又由于\(S\)是无限集合,故存在某个向量\(\boldsymbol{\alpha}_k\),使得\(\boldsymbol{\alpha}_k\)不属于任何一个\(V_i\).
\end{proof}
\begin{remark}
上述证明要用到任意一个数域都有无穷个元素这一事实. 因此,对于有限域(读者以后可能会学到)上的向量空间,上例结论不一定成立.
\end{remark}

\begin{proposition}\label{proposition:真子空间外仍有一组基存在}
设\(V_1,V_2,\cdots,V_m\)是数域\(\mathbb{F}\)上向量空间\(V\)的\(m\)个真子空间,证明:\(V\)中必有一组基,使得每个基向量都不在诸\(V_i\)的并中.
\end{proposition}
\begin{proof}
{\color{blue}证法一:}
由\hyperref[proposition:真子空间外仍有向量存在]{命题\ref{proposition:真子空间外仍有向量存在}}可知,存在非零向量\(\boldsymbol{e}_1\in V\),使得\(\boldsymbol{e}_1\notin\bigcup_{i = 1}^{m}V_i\). 定义\(V_{m + 1}=L(\boldsymbol{e}_1)\),再由\hyperref[proposition:真子空间外仍有向量存在]{命题\ref{proposition:真子空间外仍有向量存在}}可知,存在向量\(\boldsymbol{e}_2\in V\),使得\(\boldsymbol{e}_2\notin\bigcup_{i = 1}^{m + 1}V_i\). 由\hyperref[corollary:线性无关向量组的命题1]{推论\ref{corollary:线性无关向量组的命题1}}可知,\(\boldsymbol{e}_2\notin L(\boldsymbol{e}_1)\)意味着\(\boldsymbol{e}_1,\boldsymbol{e}_2\)线性无关. 重新定义\(V_{m + 1}=L(\boldsymbol{e}_1,\boldsymbol{e}_2)\),再由\hyperref[proposition:真子空间外仍有向量存在]{命题\ref{proposition:真子空间外仍有向量存在}}可知,存在向量\(\boldsymbol{e}_3\in V\),使得\(\boldsymbol{e}_3\notin\bigcup_{i = 1}^{m + 1}V_i\). 再由\hyperref[corollary:线性无关向量组的命题1]{推论\ref{corollary:线性无关向量组的命题1}}可知,\(\boldsymbol{e}_3\notin L(\boldsymbol{e}_1,\boldsymbol{e}_2)\)意味着\(\boldsymbol{e}_1,\boldsymbol{e}_2,\boldsymbol{e}_3\)线性无关. 不断重复上述讨论,即添加线性无关的向量重新定义\(V_{m + 1}\),并反复利用\hyperref[proposition:真子空间外仍有向量存在]{命题\ref{proposition:真子空间外仍有向量存在}}和\hyperref[corollary:线性无关向量组的命题1]{推论\ref{corollary:线性无关向量组的命题1}}的结论,最后可以得到\(n\)个线性无关的向量\(\boldsymbol{e}_1,\boldsymbol{e}_2,\cdots,\boldsymbol{e}_n\),它们构成\(V\)的一组基,且满足\(\boldsymbol{e}_j\notin\bigcup_{i = 1}^{m}V_i(1\leq j\leq n)\). 

{\color{blue}证法二:}任取\(V\)的一组基\(\{\boldsymbol{e}_1,\boldsymbol{e}_2,\cdots,\boldsymbol{e}_n\}\). 对任意的正整数\(k\),构造\(V\)中向量\(\boldsymbol{\alpha}_k=\boldsymbol{e}_1 + k\boldsymbol{e}_2+\cdots + k^{n - 1}\boldsymbol{e}_n\),设向量族\(S = \{\boldsymbol{\alpha}_k|k = 1,2,\cdots\}\). 由\hyperref[example:3.110.1]{例题\ref{example:3.110.1}}可知,\(S\)中任意\(n\)个不同的向量都构成\(V\)的一组基. 因为\(V_i\)都是\(V\)的真子空间,所以每个\(V_i\)至多包含\(S\)中\(n - 1\)个向量.因此$\bigcup_{i=1}^m{V_i}$至多包含$S$中$m(n-1)$个向量.
又由于\(S\)是无限集合,故存在某个向量\(\boldsymbol{\alpha}_k\),使得\(\boldsymbol{\alpha}_k\)不属于任何一个\(V_i\).
进一步,在\(S\)中一定还存在\(n\)个不同的向量\(\boldsymbol{\alpha}_{k_1},\boldsymbol{\alpha}_{k_2},\cdots,\boldsymbol{\alpha}_{k_n}\),使得每个\(\boldsymbol{\alpha}_{k_j}\)都不属于任何一个\(V_i\),此时\(\{\boldsymbol{\alpha}_{k_1},\boldsymbol{\alpha}_{k_2},\cdots,\boldsymbol{\alpha}_{k_n}\}\)就构成了\(V\)的一组基.
\end{proof}

\begin{definition}[$U-$陪集与商空间]\label{definition:U-陪集与商空间}
设\(V\)是数域\(\mathbb{K}\)上的线性空间,\(U\)是\(V\)的子空间. 对任意的\(\boldsymbol{v}\in V\),集合\(\boldsymbol{v}+U:=\{\boldsymbol{v}+\boldsymbol{u}|\boldsymbol{u}\in U\}\)称为\(\boldsymbol{v}\)的\textbf{\(U -\)陪集}. 在所有\(U -\)陪集构成的集合\(S = \{\boldsymbol{v}+U|\boldsymbol{v}\in V\}\)中,定义加法和数乘如下,其中\(\boldsymbol{v}_1,\boldsymbol{v}_2\in V\),\(k\in\mathbb{K}\):
\[
(\boldsymbol{v}_1 + U)+(\boldsymbol{v}_2 + U):=(\boldsymbol{v}_1+\boldsymbol{v}_2)+U,\ k\cdot(\boldsymbol{v}_1 + U):=k\cdot\boldsymbol{v}_1+U.
\]

\(S\)在上述加法和数乘下成为数域\(\mathbb{K}\)上的线性空间,称为\(V\)关于子空间\(U\)的\textbf{商空间},记为\(V/U\).
\end{definition}
\begin{note}
容易验证\(S\)在上述加法和数乘下满足线性空间的8条公理,因此商空间是良定义的.故任意$V$的子空间$U$都存在相应的商空间.
\end{note}
\begin{remark}
商空间的向量是$U-$陪集.商空间的零向量就是$\boldsymbol{0} + U=U$.
\end{remark}

\begin{proposition}[$U-$陪集的性质]\label{proposition:U-陪集的性质}
\begin{enumerate}[(1)]
\item \(U -\)陪集之间的关系是:作为集合或者相等,或者不相交;
\item  \(\boldsymbol{v}_1 + U=\boldsymbol{v}_2 + U\)(作为集合相等)当且仅当\(\boldsymbol{v}_1-\boldsymbol{v}_2\in U\). 特别地,\(\boldsymbol{v}+U\)是\(V\)的子空间当且仅当\(\boldsymbol{v}\in U\);
\item  \(S\)中的加法以及\(\mathbb{K}\)关于\(S\)的数乘不依赖于代表元的选取,即若\(\boldsymbol{v}_1 + U=\boldsymbol{v}_1'+U\)以及\(\boldsymbol{v}_2 + U=\boldsymbol{v}_2'+U\),则\((\boldsymbol{v}_1 + U)+(\boldsymbol{v}_2 + U)=(\boldsymbol{v}_1'+U)+(\boldsymbol{v}_2'+U)\),以及\(k\cdot(\boldsymbol{v}_1 + U)=k\cdot(\boldsymbol{v}_1'+U)\);
\end{enumerate}
\end{proposition}
\begin{proof}
\begin{enumerate}[(1)]
\item 设\((\boldsymbol{v}_1 + U)\cap(\boldsymbol{v}_2 + U)\neq\varnothing\),即存在\(\boldsymbol{u}_1,\boldsymbol{u}_2\in U\),使得\(\boldsymbol{v}_1+\boldsymbol{u}_1=\boldsymbol{v}_2+\boldsymbol{u}_2\),从而\(\boldsymbol{v}_1-\boldsymbol{v}_2=\boldsymbol{u}_2-\boldsymbol{u}_1\in U\),于是
\[
\boldsymbol{v}_1 + U=\boldsymbol{v}_2+(\boldsymbol{v}_1 - \boldsymbol{v}_2)+U\subseteq\boldsymbol{v}_2 + U,\ \boldsymbol{v}_2 + U=\boldsymbol{v}_1+(\boldsymbol{v}_2 - \boldsymbol{v}_1)+U\subseteq\boldsymbol{v}_1 + U,
\]
因此\(\boldsymbol{v}_1 + U=\boldsymbol{v}_2 + U\).
\item 由(1)的证明过程即得. 特别地,\(\boldsymbol{v}+U\)是\(V\)的子空间\(\Rightarrow\mathbf{0}\in \boldsymbol{v}+U\Rightarrow \text{存在}\boldsymbol{u}\in U,\text{使得}\mathbf{0}=\boldsymbol{v}+\boldsymbol{u}\Rightarrow \boldsymbol{v}=-\boldsymbol{u}\in U \).

若$\boldsymbol{v}\in U$,则一方面,$\forall \boldsymbol{\alpha }\in \boldsymbol{v}+U$,存在$\boldsymbol{u}'\in U$,使得$\boldsymbol{\alpha }=\boldsymbol{v}+\boldsymbol{u}'$.又$\boldsymbol{v}\in U$,因此$\boldsymbol{\alpha }=\boldsymbol{v}+\boldsymbol{u}'\in U$.故$\boldsymbol{v}+U\subset U$.
另一方面,$\forall \boldsymbol{\beta }\in U$,有$\boldsymbol{\beta }=\boldsymbol{v}+\boldsymbol{\beta }-\boldsymbol{v}$.又由$\boldsymbol{v}\in U$可知$\boldsymbol{\beta }-\boldsymbol{v}\in U$,于是$\boldsymbol{\beta }=\boldsymbol{v}+\boldsymbol{\beta }-\boldsymbol{v}\in \boldsymbol{v}+U$.故$\boldsymbol{v}+U\supset U$.因此$\boldsymbol{v}+U = U$是\(V\)的子空间.

(实际上,若$\boldsymbol{v}\in U$,则因为$\boldsymbol{v}\in U\text{并且}\boldsymbol{v}\in \boldsymbol{v}+U$,所以$\boldsymbol{v}+U\cap U\ne \varnothing$.故由(1)可知$\boldsymbol{v}+U=U$是\(V\)的子空间.这样也能得到证明.)

\item 若\(\boldsymbol{v}_1 + U=\boldsymbol{v}_1'+U\)以及\(\boldsymbol{v}_2 + U=\boldsymbol{v}_2'+U\),则\hyperlink{remark1陪集}{存在\(\boldsymbol{u}_1,\boldsymbol{u}_2\in U\),使得\(\boldsymbol{v}_1-\boldsymbol{v}_1'=\boldsymbol{u}_1\),\(\boldsymbol{v}_2-\boldsymbol{v}_2'=\boldsymbol{u}_2\)},从而\((\boldsymbol{v}_1+\boldsymbol{v}_2)-(\boldsymbol{v}_1'+\boldsymbol{v}_2')=\boldsymbol{u}_1+\boldsymbol{u}_2\in U\),\(k\cdot\boldsymbol{v}_1 - k\cdot\boldsymbol{v}_1'=k\cdot\boldsymbol{u}_1\in U\),于是由(2)可得
\[
(\boldsymbol{v}_1 + U)+(\boldsymbol{v}_2 + U)=(\boldsymbol{v}_1+\boldsymbol{v}_2)+U=(\boldsymbol{v}_1'+\boldsymbol{v}_2')+U=(\boldsymbol{v}_1'+U)+(\boldsymbol{v}_2'+U),
\]
\[
k\cdot(\boldsymbol{v}_1 + U)=k\cdot\boldsymbol{v}_1+U=k\cdot\boldsymbol{v}_1'+U=k\cdot(\boldsymbol{v}_1'+U).
\]
\end{enumerate}
\end{proof}
\begin{remark}
\hypertarget{remark1陪集}{若}\(\boldsymbol{v}_1 + U=\boldsymbol{v}_1'+U\)以及\(\boldsymbol{v}_2 + U=\boldsymbol{v}_2'+U\),则$\forall \boldsymbol{u}_{1}^{\prime}\in U$,有$\boldsymbol{v}_1+\boldsymbol{u}_{1}^{\prime}\in \boldsymbol{v}_1 + U = \boldsymbol{v}_1' + U$.从而存在$\boldsymbol{u}_{1}''\in U$,使得$\boldsymbol{v}_1+\boldsymbol{u}_{1}^{\prime}=\boldsymbol{v}_1'+\boldsymbol{u}_{1}''$.于是$\boldsymbol{v}_1 - \boldsymbol{v}_1'=\boldsymbol{u}_{1}''-\boldsymbol{u}_{1}^{\prime}$.再令$\boldsymbol{u}_1=\boldsymbol{u}_{1}''-\boldsymbol{u}_{1}^{\prime}$,则$\boldsymbol{v}_1 - \boldsymbol{v}_1'=\boldsymbol{u}_1\in U$.同理可得,存在$u_2\in U$,使得$\boldsymbol{v}_2-\boldsymbol{v}_{2}^{\prime}=\boldsymbol{u}_2\in U$.
\end{remark}


\begin{proposition}[商空间的维数公式和商空间与补空间同构]\label{proposition:商空间的维数公式和商空间与补空间同构}
设\(V\)是数域\(\mathbb{K}\)上的\(n\)维线性空间,\(U\)是\(V\)的子空间,\(W\)是\(U\)的补空间,证明:\(\dim V/U=\dim V-\dim U\),并且存在线性同构\(\varphi:W\to V/U\).
\end{proposition}
\begin{proof}
取子空间\(U\)的一组基\(\{\boldsymbol{e}_1,\cdots,\boldsymbol{e}_m\}\),补空间\(W\)的一组基\(\{\boldsymbol{e}_{m + 1},\cdots,\boldsymbol{e}_n\}\),则\(\{\boldsymbol{e}_1,\cdots,\boldsymbol{e}_m,\boldsymbol{e}_{m + 1},\cdots,\boldsymbol{e}_n\}\)是\(V\)的一组基. 我们断言\(\{\boldsymbol{e}_{m + 1}+U,\cdots,\boldsymbol{e}_n+U\}\)是商空间\(V/U\)的一组基. 一方面,对任意的\(\boldsymbol{v}\in V\),设\(\boldsymbol{v}=\sum_{i = 1}^{n}a_i\boldsymbol{e}_i\),则
\[
\boldsymbol{v}+U=\left(\sum_{i = 1}^{n}a_i\boldsymbol{e}_i\right)+U=\left(\sum_{i = m + 1}^{n}a_i\boldsymbol{e}_i\right)+U=\sum_{i = m + 1}^{n}a_i(\boldsymbol{e}_i+U).
\]
另一方面,设\(a_{m + 1},\cdots,a_n\in\mathbb{K}\),使得\(\sum_{i = m + 1}^{n}a_i(\boldsymbol{e}_i+U)=\boldsymbol{0}+U\),即\(\left(\sum_{i = m + 1}^{n}a_i\boldsymbol{e}_i\right)+U = U\),从而\(\sum_{i = m + 1}^{n}a_i\boldsymbol{e}_i\in U\). 于是存在\(a_1,\cdots,a_m\in\mathbb{K}\),使得\(\sum_{i = m + 1}^{n}a_i\boldsymbol{e}_i=-\sum_{i = 1}^{m}a_i\boldsymbol{e}_i\),即\(\sum_{i = 1}^{n}a_i\boldsymbol{e}_i=\boldsymbol{0}\),从而\(a_i = 0(1\leq i\leq n)\).于是\(\{\boldsymbol{e}_{m + 1}+U,\cdots,\boldsymbol{e}_n+U\}\)线性无关. 因此,\(\dim V/U=n - m=\dim V-\dim U\).

对任意的\(\boldsymbol{w}\in W\),设\(\boldsymbol{w}=\sum_{i = m + 1}^{n}a_i\boldsymbol{e}_i\),定义映射\(\varphi:W\to V/U\)为
\[
\varphi(\boldsymbol{w})=\boldsymbol{w}+U=\sum_{i = m + 1}^{n}a_i(\boldsymbol{e}_i+U).
\]
容易验证\(\varphi\)保持加法和数乘,并且是一一对应($W$的基$e_i$映射过去得到$\varphi(e_i)$仍是$V/U$的基,$i=m+1,\cdots,n$.),从而是线性同构. 
\end{proof}



\subsection{练习}

\begin{exercise}\label{exercise:矩阵乘法可交换的子空间C(A)}
设\(V = M_n(\mathbb{K})\)是数域\(\mathbb{K}\)上的\(n\)阶矩阵全体组成的线性空间,\(\boldsymbol{A}\in V\),求证:与\(\boldsymbol{A}\)乘法可交换的矩阵全体\(C(\boldsymbol{A})\)组成\(V\)的子空间且其维数不为零. 又若\(T\)是\(V\)的非空子集,求证:与\(T\)中任一矩阵乘法可交换的矩阵全体\(C(T)\)也构成\(V\)的子空间且其维数不为零.
\end{exercise}
\begin{proof}
由于纯量阵\(c\boldsymbol{I}_n\)与任一\(n\)阶矩阵\(\boldsymbol{A}\)乘法可交换,故\(L(\boldsymbol{I}_n)\subseteq C(\boldsymbol{A})\). 任取\(\boldsymbol{B},\boldsymbol{C}\in C(\boldsymbol{A})\),\(k\in\mathbb{K}\),容易验证\(\boldsymbol{B}+\boldsymbol{C}\in C(\boldsymbol{A})\),\(k\boldsymbol{B}\in C(\boldsymbol{A})\),故\(C(\boldsymbol{A})\)是\(M_n(\mathbb{K})\)的子空间且其维数不为零. \(C(T)\)的结论同理可证.
\end{proof}

\begin{exercise}
设\(\boldsymbol{\alpha}_1=(1,0, - 1,0),\boldsymbol{\alpha}_2=(0,1,2,1),\boldsymbol{\alpha}_3=(2,1,0,1)\)是四维实行向量空间\(V\)中的向量,它们生成的子空间为\(V_1\),又向量\(\boldsymbol{\beta}_1=(-1,1,1,1),\boldsymbol{\beta}_2=(1,-1,-3,-1),\boldsymbol{\beta}_3=(-1,1,-1,1)\)生成的子空间为\(V_2\),求子空间\(V_1 + V_2\)和\(V_1\cap V_2\)的基.
\end{exercise}
\begin{solution}
{\color{blue}解法一:}\(V_1 + V_2\)是由\(\boldsymbol{\alpha}_i\)和\(\boldsymbol{\beta}_i\)生成的,因此只要求出这6个向量的极大无关组即可. 将这6个向量按列分块方式拼成矩阵,并用初等行变换将其化为阶梯形矩阵:
\[
\begin{pmatrix}
1&0&2&-1&1&-1\\
0&1&1&1&-1&1\\
-1&2&0&1&-3&-1\\
0&1&1&1&-1&1
\end{pmatrix}\to
\begin{pmatrix}
1&0&2&-1&1&-1\\
0&1&1&1&-1&1\\
0&2&2&0&-2&-2\\
0&0&0&0&0&0
\end{pmatrix}\to
\begin{pmatrix}
1&0&2&-1&1&-1\\
0&1&1&1&-1&1\\
0&0&0&-2&0&-4\\
0&0&0&0&0&0
\end{pmatrix},
\]
故可取\(\boldsymbol{\alpha}_1,\boldsymbol{\alpha}_2,\boldsymbol{\beta}_1\)为\(V_1 + V_2\)的基(不唯一).

再来求\(V_1\cap V_2\)的基. 首先注意到\(\boldsymbol{\alpha}_1,\boldsymbol{\alpha}_2\)是\(V_1\)的基(从上面的矩阵即可看出),又不难验证\(\boldsymbol{\beta}_1,\boldsymbol{\beta}_2\)是\(V_2\)的基,\(V_2\)中的向量可以表示为\(\boldsymbol{\beta}_1,\boldsymbol{\beta}_2\)的线性组合. 假设\(t_1\boldsymbol{\beta}_1 + t_2\boldsymbol{\beta}_2\)属于\(V_1\),则向量组\(\boldsymbol{\alpha}_1,\boldsymbol{\alpha}_2,t_1\boldsymbol{\beta}_1 + t_2\boldsymbol{\beta}_2\)和向量组\(\boldsymbol{\alpha}_1,\boldsymbol{\alpha}_2\)的秩相等(因为\(\boldsymbol{\alpha}_1,\boldsymbol{\alpha}_2\)是\(V_1\)的基). 因此,我们可以用矩阵方法来求出参数\(t_1,t_2\). 注意到
\[
\begin{pmatrix}
1&0&-t_1 + t_2\\
0&1&t_1 - t_2\\
-1&2&t_1 - 3t_2\\
0&1&t_1 - t_2
\end{pmatrix}\to
\begin{pmatrix}
1&0&-t_1 + t_2\\
0&1&t_1 - t_2\\
0&2&-2t_2\\
0&0&0
\end{pmatrix}\to
\begin{pmatrix}
1&0&-t_1 + t_2\\
0&1&t_1 - t_2\\
0&0&-2t_1\\
0&0&0
\end{pmatrix},
\]
故可得\(t_1 = 0\),所以\(V_1\cap V_2\)的基可取为\(\boldsymbol{\beta}_2\).

{\color{blue}解法二:} 求\(V_1 + V_2\)的基同解法1,现用解线性方程组的方法来求\(V_1\cap V_2\)的基. 因为\(\boldsymbol{\alpha}_1,\boldsymbol{\alpha}_2\)是\(V_1\)的基,\(\boldsymbol{\beta}_1,\boldsymbol{\beta}_2\)是\(V_2\)的基,故对任一向量\(\boldsymbol{\gamma}\in V_1\cap V_2\),\(\boldsymbol{\gamma}=x_1\boldsymbol{\alpha}_1 + x_2\boldsymbol{\alpha}_2=(-x_3)\boldsymbol{\beta}_1 + (-x_4)\boldsymbol{\beta}_2\). 因此,求向量\(\boldsymbol{\gamma}\)等价于求解线性方程组
\[
x_1\boldsymbol{\alpha}_1 + x_2\boldsymbol{\alpha}_2 + x_3\boldsymbol{\beta}_1 + x_4\boldsymbol{\beta}_2=\boldsymbol{0}.
\]
通过初等行变换将其系数矩阵\((\boldsymbol{\alpha}_1,\boldsymbol{\alpha}_2,\boldsymbol{\beta}_1,\boldsymbol{\beta}_2)\)进行化简:
\[
\begin{pmatrix}
1&0&-1&1\\
0&1&1&-1\\
0&0&-2&0\\
0&0&0&0
\end{pmatrix}\to
\begin{pmatrix}
1&0&0&1\\
0&1&0&-1\\
0&0&1&0\\
0&0&0&0
\end{pmatrix},
\]
故上述线性方程组的通解为\((x_1,x_2,x_3,x_4)=k(-1,1,0,1)\),从而\(\boldsymbol{\gamma}=-k(\boldsymbol{\alpha}_1 - \boldsymbol{\alpha}_2)=-k\boldsymbol{\beta}_2(k\in\mathbb{R})\),于是\(\boldsymbol{\beta}_2\)是\(V_1\cap V_2\)的基. 
\end{solution}


\section{矩阵的秩}

\subsection{初等变换法}

矩阵的秩在初等变换或分块初等变换下不变.

\textbf{想法:遇到关于秩不等式的问题,可以考虑构造分块矩阵,对其做适当的初等变换,再利用秩的基本公式.}

\begin{theorem}\label{theorem:矩阵的秩与子式}
矩阵\(\boldsymbol{A}\)的秩等于\(r\)的充要条件是\(\boldsymbol{A}\)有一个\(r\)阶子式不等于零,而\(\boldsymbol{A}\)的所有\(r + 1\)阶子式都等于零.
\end{theorem}

\begin{proposition}[矩阵秩的基本公式]\label{proposition:矩阵秩的基本公式}
\begin{enumerate}[(1)]
\item \label{矩阵秩的基本公式1}若\(k\neq0\),\(\mathrm{r}(k\boldsymbol{A})=\mathrm{r}(\boldsymbol{A})\);

\item \label{矩阵秩的基本公式2}\(\mathrm{r}(\boldsymbol{A}\boldsymbol{B})\leq\min\{\mathrm{r}(\boldsymbol{A}),\mathrm{r}(\boldsymbol{B})\}\);

\item \label{矩阵秩的基本公式3}\(\mathrm{r}\begin{pmatrix}\boldsymbol{A}&\boldsymbol{O}\\\boldsymbol{O}&\boldsymbol{B}\end{pmatrix}=\mathrm{r}(\boldsymbol{A})+\mathrm{r}(\boldsymbol{B})\);

\item \label{矩阵秩的基本公式4}\(\mathrm{r}\begin{pmatrix}\boldsymbol{A}&\boldsymbol{C}\\\boldsymbol{O}&\boldsymbol{B}\end{pmatrix}\geq\mathrm{r}(\boldsymbol{A})+\mathrm{r}(\boldsymbol{B})\),\(\mathrm{r}\begin{pmatrix}\boldsymbol{A}&\boldsymbol{O}\\\boldsymbol{D}&\boldsymbol{B}\end{pmatrix}\geq\mathrm{r}(\boldsymbol{A})+\mathrm{r}(\boldsymbol{B})\);

\item \label{矩阵秩的基本公式5}\(\mathrm{r}\left( \begin{matrix}
\boldsymbol{A}&		\boldsymbol{B}\\
\end{matrix} \right) \leq\mathrm{r}(\boldsymbol{A})+\mathrm{r}(\boldsymbol{B})\),\(\mathrm{r}\begin{pmatrix}\boldsymbol{A}\\\boldsymbol{B}\end{pmatrix}\leq\mathrm{r}(\boldsymbol{A})+\mathrm{r}(\boldsymbol{B})\);

\item \label{矩阵秩的基本公式6}\(\mathrm{r}(\boldsymbol{A}+\boldsymbol{B})\leq\mathrm{r}(\boldsymbol{A})+\mathrm{r}(\boldsymbol{B})\),\(\mathrm{r}(\boldsymbol{A}-\boldsymbol{B})\leq\mathrm{r}(\boldsymbol{A})+\mathrm{r}(\boldsymbol{B})\);

\item \label{矩阵秩的基本公式7}\(\mathrm{r}(\boldsymbol{A}-\boldsymbol{B})\geq|\mathrm{r}(\boldsymbol{A})-\mathrm{r}(\boldsymbol{B})|\).
\end{enumerate}
\end{proposition}
\begin{proof}
\begin{enumerate}[(1)]
\item 由于\(k\boldsymbol{A}=\boldsymbol{P}_1(k)\boldsymbol{P}_2(k)\cdots\boldsymbol{P}_m(k)\boldsymbol{A}\),故\(\mathrm{r}(k\boldsymbol{A})=\mathrm{r}(\boldsymbol{A})\).

\item  {\color{blue}证法一:}设\(\boldsymbol{A}\)是\(m\times n\)矩阵,\(\boldsymbol{B}\)是\(n\times s\)矩阵. 将矩阵\(\boldsymbol{B}\)按列分块,\(\boldsymbol{B}=(\boldsymbol{\beta}_1,\boldsymbol{\beta}_2,\cdots,\boldsymbol{\beta}_s)\),则\(\boldsymbol{A}\boldsymbol{B}=(\boldsymbol{A}\boldsymbol{\beta}_1,\boldsymbol{A}\boldsymbol{\beta}_2,\cdots,\boldsymbol{A}\boldsymbol{\beta}_s)\). 若\(\boldsymbol{B}\)列向量的极大无关组为\(\{\boldsymbol{\beta}_{j_1},\boldsymbol{\beta}_{j_2},\cdots,\boldsymbol{\beta}_{j_r}\}\),则\(\boldsymbol{B}\)的任一列向量\(\boldsymbol{\beta}_j\)均可用\(\{\boldsymbol{\beta}_{j_1},\boldsymbol{\beta}_{j_2},\cdots,\boldsymbol{\beta}_{j_r}\}\)线性表示. 于是任一\(\boldsymbol{A}\boldsymbol{\beta}_j\)也可用\(\{\boldsymbol{A}\boldsymbol{\beta}_{j_1},\boldsymbol{A}\boldsymbol{\beta}_{j_2},\cdots,\boldsymbol{A}\boldsymbol{\beta}_{j_r}\}\)来线性表示. 因此,向量组\(\{\boldsymbol{A}\boldsymbol{\beta}_1,\boldsymbol{A}\boldsymbol{\beta}_2,\cdots,\boldsymbol{A}\boldsymbol{\beta}_s\}\)的秩不超过\(r\),即\(\mathrm{r}(\boldsymbol{A}\boldsymbol{B})\leq\mathrm{r}(\boldsymbol{B})\). 同理,对矩阵\(\boldsymbol{A}\)用行分块的方法可以证明\(\mathrm{r}(\boldsymbol{A}\boldsymbol{B})\leq\mathrm{r}(\boldsymbol{A})\).

{\color{blue}证法二:}见\hyperref[example:4.243456]{例题\ref{example:4.243456}}.

\item  设\(\boldsymbol{A},\boldsymbol{B}\)的秩分别为\(r_1,r_2\),则存在非异阵\(\boldsymbol{P}_1,\boldsymbol{Q}_1\)和非异阵\(\boldsymbol{P}_2,\boldsymbol{Q}_2\),使得
\[
\boldsymbol{P}_1\boldsymbol{A}\boldsymbol{Q}_1=\begin{pmatrix}
\boldsymbol{I}_{r_1}&\boldsymbol{O}\\
\boldsymbol{O}&\boldsymbol{O}
\end{pmatrix}, \boldsymbol{P}_2\boldsymbol{B}\boldsymbol{Q}_2=\begin{pmatrix}
\boldsymbol{I}_{r_2}&\boldsymbol{O}\\
\boldsymbol{O}&\boldsymbol{O}
\end{pmatrix}.
\]
于是
\[
\begin{pmatrix}
\boldsymbol{P}_1&\boldsymbol{O}\\
\boldsymbol{O}&\boldsymbol{P}_2
\end{pmatrix}
\begin{pmatrix}
\boldsymbol{A}&\boldsymbol{O}\\
\boldsymbol{O}&\boldsymbol{B}
\end{pmatrix}
\begin{pmatrix}
\boldsymbol{Q}_1&\boldsymbol{O}\\
\boldsymbol{O}&\boldsymbol{Q}_2
\end{pmatrix}=
\begin{pmatrix}
\boldsymbol{P}_1\boldsymbol{A}\boldsymbol{Q}_1&\boldsymbol{O}\\
\boldsymbol{O}&\boldsymbol{P}_2\boldsymbol{B}\boldsymbol{Q}_2
\end{pmatrix}=
\begin{pmatrix}
\boldsymbol{I}_{r_1}&\boldsymbol{O}&\boldsymbol{O}&\boldsymbol{O}\\
\boldsymbol{O}&\boldsymbol{O}&\boldsymbol{O}&\boldsymbol{O}\\
\boldsymbol{O}&\boldsymbol{O}&\boldsymbol{I}_{r_2}&\boldsymbol{O}\\
\boldsymbol{O}&\boldsymbol{O}&\boldsymbol{O}&\boldsymbol{O}
\end{pmatrix}.
\]
因此,\(\mathrm{r}\begin{pmatrix}
\boldsymbol{A}&\boldsymbol{O}\\
\boldsymbol{O}&\boldsymbol{B}
\end{pmatrix}=r_1 + r_2=\mathrm{r}(\boldsymbol{A})+\mathrm{r}(\boldsymbol{B})\).

\item  {\color{blue}证法一:} 我们只证明第一个不等式,第二个不等式同理可证. 设\(\boldsymbol{A},\boldsymbol{B}\)的秩分别为\(r_1,r_2\),则存在非异阵\(\boldsymbol{P}_1,\boldsymbol{Q}_1\)和非异阵\(\boldsymbol{P}_2,\boldsymbol{Q}_2\),使得
\[
\boldsymbol{P}_1\boldsymbol{A}\boldsymbol{Q}_1=\begin{pmatrix}
\boldsymbol{I}_{r_1}&\boldsymbol{O}\\
\boldsymbol{O}&\boldsymbol{O}
\end{pmatrix}, \boldsymbol{P}_2\boldsymbol{B}\boldsymbol{Q}_2=\begin{pmatrix}
\boldsymbol{I}_{r_2}&\boldsymbol{O}\\
\boldsymbol{O}&\boldsymbol{O}
\end{pmatrix}.
\]
于是
\[
\begin{pmatrix}
\boldsymbol{P}_1&\boldsymbol{O}\\
\boldsymbol{O}&\boldsymbol{P}_2
\end{pmatrix}
\begin{pmatrix}
\boldsymbol{A}&\boldsymbol{C}\\
\boldsymbol{O}&\boldsymbol{B}
\end{pmatrix}
\begin{pmatrix}
\boldsymbol{Q}_1&\boldsymbol{O}\\
\boldsymbol{O}&\boldsymbol{Q}_2
\end{pmatrix}=
\begin{pmatrix}
\boldsymbol{P}_1\boldsymbol{A}\boldsymbol{Q}_1&\boldsymbol{P}_1\boldsymbol{C}\boldsymbol{Q}_2\\
\boldsymbol{O}&\boldsymbol{P}_2\boldsymbol{B}\boldsymbol{Q}_2
\end{pmatrix}=
\begin{pmatrix}
\boldsymbol{I}_{r_1}&\boldsymbol{O}&\boldsymbol{C}_{11}&\boldsymbol{C}_{12}\\
\boldsymbol{O}&\boldsymbol{O}&\boldsymbol{C}_{21}&\boldsymbol{C}_{22}\\
\boldsymbol{O}&\boldsymbol{O}&\boldsymbol{I}_{r_2}&\boldsymbol{O}\\
\boldsymbol{O}&\boldsymbol{O}&\boldsymbol{O}&\boldsymbol{O}
\end{pmatrix}.
\]
在上面的分块矩阵中实施第三类分块初等变换,用\(\boldsymbol{I}_{r_1}\)消去同行的矩阵;用\(\boldsymbol{I}_{r_2}\)消去
同列的矩阵,再将\(\boldsymbol{C}_{22}\)对换到第\((2,2)\)位置:
\[
\begin{pmatrix}
\boldsymbol{I}_{r_1}&\boldsymbol{O}&\boldsymbol{C}_{11}&\boldsymbol{C}_{12}\\
\boldsymbol{O}&\boldsymbol{O}&\boldsymbol{C}_{21}&\boldsymbol{C}_{22}\\
\boldsymbol{O}&\boldsymbol{O}&\boldsymbol{I}_{r_2}&\boldsymbol{O}\\
\boldsymbol{O}&\boldsymbol{O}&\boldsymbol{O}&\boldsymbol{O}
\end{pmatrix}
\rightarrow
\begin{pmatrix}
\boldsymbol{I}_{r_1}&\boldsymbol{O}&\boldsymbol{O}&\boldsymbol{O}\\
\boldsymbol{O}&\boldsymbol{O}&\boldsymbol{O}&\boldsymbol{C}_{22}\\
\boldsymbol{O}&\boldsymbol{O}&\boldsymbol{I}_{r_2}&\boldsymbol{O}\\
\boldsymbol{O}&\boldsymbol{O}&\boldsymbol{O}&\boldsymbol{O}
\end{pmatrix}
\rightarrow
\begin{pmatrix}
\boldsymbol{I}_{r_1}&\boldsymbol{O}&\boldsymbol{O}&\boldsymbol{O}\\
\boldsymbol{O}&\boldsymbol{C}_{22}&\boldsymbol{O}&\boldsymbol{O}\\
\boldsymbol{O}&\boldsymbol{O}&\boldsymbol{I}_{r_2}&\boldsymbol{O}\\
\boldsymbol{O}&\boldsymbol{O}&\boldsymbol{O}&\boldsymbol{O}
\end{pmatrix},
\]
最后由(3)的结论可得
\[
\mathrm{r}\begin{pmatrix}
\boldsymbol{A}&\boldsymbol{C}\\
\boldsymbol{O}&\boldsymbol{B}
\end{pmatrix}=\mathrm{r}(\boldsymbol{I}_{r_1})+\mathrm{r}(\boldsymbol{C}_{22})+\mathrm{r}(\boldsymbol{I}_{r_2})\geq r_1 + r_2=\mathrm{r}(\boldsymbol{A})+\mathrm{r}(\boldsymbol{B}).
\]
{\color{blue}证法二:} 我们也可用子式法来证明. 设\(\mathrm{r}\begin{pmatrix}
\boldsymbol{A}&\boldsymbol{O}\\
\boldsymbol{O}&\boldsymbol{B}
\end{pmatrix}=r\),则由\hyperref[theorem:矩阵的秩与子式]{定理\ref{theorem:矩阵的秩与子式}}可知,\(\begin{pmatrix}
\boldsymbol{A}&\boldsymbol{O}\\
\boldsymbol{O}&\boldsymbol{B}
\end{pmatrix}\)有一个\(r\)阶子式不为零,不妨设为\(\begin{vmatrix}
\boldsymbol{A}_1&\boldsymbol{O}\\
\boldsymbol{O}&\boldsymbol{B}_1
\end{vmatrix}\),其中\(\boldsymbol{A}_1,\boldsymbol{B}_1\)分别是\(\boldsymbol{A},\boldsymbol{B}\)的子阵. 注意\(\boldsymbol{A}_1\)或\(\boldsymbol{B}_1\)允许是零阶矩阵,这对应于该子式完全包含在\(\boldsymbol{B}\)或\(\boldsymbol{A}\)中,但若\(\boldsymbol{A}_1,\boldsymbol{B}_1\)的阶数都大于零,则通过该子式非零,再结合由Laplace定理容易验证\(\boldsymbol{A}_1,\boldsymbol{B}_1\)都是方阵. 设在矩阵\(\begin{pmatrix}
\boldsymbol{A}&\boldsymbol{C}\\
\boldsymbol{O}&\boldsymbol{B}
\end{pmatrix}\)中对应的\(r\)阶子式是\(\begin{vmatrix}
\boldsymbol{A}_1&\boldsymbol{C}_1\\
\boldsymbol{O}&\boldsymbol{B}_1
\end{vmatrix}\),则由Laplace定理可得\(\begin{vmatrix}
\boldsymbol{A}_1&\boldsymbol{C}_1\\
\boldsymbol{O}&\boldsymbol{B}_1
\end{vmatrix}=|\boldsymbol{A}_1||\boldsymbol{B}_1|=\begin{vmatrix}
\boldsymbol{A}_1&\boldsymbol{O}\\
\boldsymbol{O}&\boldsymbol{B}_1
\end{vmatrix}\neq0\),再次由\hyperref[theorem:矩阵的秩与子式]{定理\ref{theorem:矩阵的秩与子式}}可得
\[
\mathrm{r}\begin{pmatrix}
\boldsymbol{A}&\boldsymbol{C}\\
\boldsymbol{O}&\boldsymbol{B}
\end{pmatrix}\geq r=\mathrm{r}\begin{pmatrix}
\boldsymbol{A}&\boldsymbol{O}\\
\boldsymbol{O}&\boldsymbol{B}
\end{pmatrix}=\mathrm{r}(\boldsymbol{A})+\mathrm{r}(\boldsymbol{B}). 
\]
{\color{blue}证法三:}设\(\boldsymbol{A}=(\boldsymbol{\alpha}_1,\boldsymbol{\alpha}_2,\cdots,\boldsymbol{\alpha}_n)\)是\(\boldsymbol{A}\)的列分块,\(\boldsymbol{\alpha}_{i_1},\boldsymbol{\alpha}_{i_2},\cdots,\boldsymbol{\alpha}_{i_r}\)是\(\boldsymbol{A}\)的列向量的极大无关组;设\(\boldsymbol{B}=(\boldsymbol{\beta}_1,\boldsymbol{\beta}_2,\cdots,\boldsymbol{\beta}_l)\),\(\boldsymbol{C}=(\boldsymbol{\gamma}_1,\boldsymbol{\gamma}_2,\cdots,\boldsymbol{\gamma}_l)\)是\(\boldsymbol{B},\boldsymbol{C}\)的列分块,\(\boldsymbol{\beta}_{j_1},\boldsymbol{\beta}_{j_2},\cdots,\boldsymbol{\beta}_{j_s}\)是\(\boldsymbol{B}\)的列向量的极大无关组,则\(\mathrm{r}(\boldsymbol{A}) = r\)且\(\mathrm{r}(\boldsymbol{B}) = s\). 我们接下来证明:作为\(\begin{pmatrix}
\boldsymbol{A}&\boldsymbol{C}\\
\boldsymbol{O}&\boldsymbol{B}
\end{pmatrix}\)的列向量,\(\begin{pmatrix}
\boldsymbol{\alpha}_{i_1}\\
\boldsymbol{0}
\end{pmatrix},\cdots,\begin{pmatrix}
\boldsymbol{\alpha}_{i_r}\\
\boldsymbol{0}
\end{pmatrix},\begin{pmatrix}
\boldsymbol{\gamma}_{j_1}\\
\boldsymbol{\beta}_{j_1}
\end{pmatrix},\cdots,\begin{pmatrix}
\boldsymbol{\gamma}_{j_s}\\
\boldsymbol{\beta}_{j_s}
\end{pmatrix}\)线性无关. 设
\[
c_1\begin{pmatrix}
\boldsymbol{\alpha}_{i_1}\\
\boldsymbol{0}
\end{pmatrix}+\cdots + c_r\begin{pmatrix}
\boldsymbol{\alpha}_{i_r}\\
\boldsymbol{0}
\end{pmatrix}+d_1\begin{pmatrix}
\boldsymbol{\gamma}_{j_1}\\
\boldsymbol{\beta}_{j_1}
\end{pmatrix}+\cdots + d_s\begin{pmatrix}
\boldsymbol{\gamma}_{j_s}\\
\boldsymbol{\beta}_{j_s}
\end{pmatrix}=\boldsymbol{0},
\]
即
\[
c_1\boldsymbol{\alpha}_{i_1}+\cdots + c_r\boldsymbol{\alpha}_{i_r}+d_1\boldsymbol{\gamma}_{j_1}+\cdots + d_s\boldsymbol{\gamma}_{j_s}=\boldsymbol{0},d_1\boldsymbol{\beta}_{j_1}+\cdots + d_s\boldsymbol{\beta}_{j_s}=\boldsymbol{0}.
\]
由上面的假设即得\(c_1=\cdots = c_r = d_1=\cdots = d_s = 0\),于是上述结论得证. 因为\(\begin{pmatrix}
\boldsymbol{A}&\boldsymbol{C}\\
\boldsymbol{O}&\boldsymbol{B}
\end{pmatrix}\)的列向量中有\(r + s\)个线性无关,故\(\mathrm{r}\begin{pmatrix}
\boldsymbol{A}&\boldsymbol{C}\\
\boldsymbol{O}&\boldsymbol{B}
\end{pmatrix}\geq r + s=\mathrm{r}(\boldsymbol{A})+\mathrm{r}(\boldsymbol{B})\).

\item  注意到
\[
\left( \begin{matrix}
\boldsymbol{I}&		\boldsymbol{I}\\
\end{matrix} \right) \left( \begin{matrix}
\boldsymbol{A}&		\boldsymbol{O}\\
\boldsymbol{O}&		\boldsymbol{B}\\
\end{matrix} \right) =\left( \begin{matrix}
\boldsymbol{A}&		\boldsymbol{B}\\
\end{matrix} \right) ,\left( \begin{matrix}
\boldsymbol{A}&		\boldsymbol{O}\\
\boldsymbol{O}&		\boldsymbol{B}\\
\end{matrix} \right) \left( \begin{array}{c}
\boldsymbol{I}\\
\boldsymbol{I}\\
\end{array} \right) =\left( \begin{array}{c}
\boldsymbol{A}\\
\boldsymbol{B}\\
\end{array} \right) .
\]
故由(2)和(3)可得
\begin{align*}
\mathrm{r}\left( \begin{matrix}
\boldsymbol{A}&		\boldsymbol{B}\\
\end{matrix} \right) =\mathrm{r}\left( \left( \begin{matrix}
\boldsymbol{I}&		\boldsymbol{I}\\
\end{matrix} \right) \left( \begin{matrix}
\boldsymbol{A}&		\boldsymbol{O}\\
\boldsymbol{O}&		\boldsymbol{B}\\
\end{matrix} \right) \right) \leqslant \mathrm{r}\left( \begin{matrix}
\boldsymbol{A}&		\boldsymbol{O}\\
\boldsymbol{O}&		\boldsymbol{B}\\
\end{matrix} \right) =\mathrm{r}\left( \boldsymbol{A} \right) +\mathrm{r}\left( \boldsymbol{B} \right) ,
\\
\mathrm{r}\left( \begin{array}{c}
\boldsymbol{A}\\
\boldsymbol{B}\\
\end{array} \right) =\mathrm{r}\left( \left( \begin{matrix}
\boldsymbol{A}&		\boldsymbol{O}\\
\boldsymbol{O}&		\boldsymbol{B}\\
\end{matrix} \right) \left( \begin{array}{c}
\boldsymbol{I}\\
\boldsymbol{I}\\
\end{array} \right) \right) \leqslant \mathrm{r}\left( \begin{matrix}
\boldsymbol{A}&		\boldsymbol{O}\\
\boldsymbol{O}&		\boldsymbol{B}\\
\end{matrix} \right) =\mathrm{r}\left( \boldsymbol{A} \right) +\mathrm{r}\left( \boldsymbol{B} \right) .
\end{align*}

\item  注意到
\[
\left( \begin{matrix}
\boldsymbol{A}&		\boldsymbol{B}\\
\end{matrix} \right) \left( \begin{array}{c}
\boldsymbol{I}\\
\boldsymbol{I}\\
\end{array} \right) =\boldsymbol{A}+\boldsymbol{B},\left( \begin{matrix}
\boldsymbol{A}&		\boldsymbol{B}\\
\end{matrix} \right) \left( \begin{array}{c}
\boldsymbol{I}\\
-\boldsymbol{I}\\
\end{array} \right) =\boldsymbol{A}-\boldsymbol{B}.
\]
故由(2)和(5)可得
\begin{align*}
\mathrm{r}\left( \boldsymbol{A}+\boldsymbol{B} \right) =\mathrm{r}\left( \left( \begin{matrix}
\boldsymbol{A}&		\boldsymbol{B}\\
\end{matrix} \right) \left( \begin{array}{c}
\boldsymbol{I}\\
\boldsymbol{I}\\
\end{array} \right) \right) \leqslant \mathrm{r}\left( \begin{matrix}
\boldsymbol{A}&		\boldsymbol{B}\\
\end{matrix} \right) \leqslant \mathrm{r}\left( \boldsymbol{A} \right) +\mathrm{r}\left( \boldsymbol{B} \right) ,
\\
\mathrm{r}\left( \boldsymbol{A}-\boldsymbol{B} \right) =\mathrm{r}\left( \left( \begin{matrix}
\boldsymbol{A}&		\boldsymbol{B}\\
\end{matrix} \right) \left( \begin{array}{c}
\boldsymbol{I}\\
-\boldsymbol{I}\\
\end{array} \right) \right) \leqslant \mathrm{r}\left( \begin{matrix}
\boldsymbol{A}&		\boldsymbol{B}\\
\end{matrix} \right) \leqslant \mathrm{r}\left( \boldsymbol{A} \right) +\mathrm{r}\left( \boldsymbol{B} \right) .
\end{align*}

\item 由于\(\mathrm{r}(\boldsymbol{A}-\boldsymbol{B})=\mathrm{r}(\boldsymbol{B}-\boldsymbol{A})\),故不妨设\(\mathrm{r}(\boldsymbol{A})\geq\mathrm{r}(\boldsymbol{B})\),则由(6)可得\(\mathrm{r}(\boldsymbol{A}-\boldsymbol{B})+\mathrm{r}(\boldsymbol{B})\geq\mathrm{r}(\boldsymbol{A}-\boldsymbol{B}+\boldsymbol{B})=\mathrm{r}(\boldsymbol{A})\),即\(\mathrm{r}(\boldsymbol{A}-\boldsymbol{B})\geq\mathrm{r}(\boldsymbol{A})-\mathrm{r}(\boldsymbol{B})\).
\end{enumerate}
\end{proof}

\begin{example}
设\(\boldsymbol{A}=(a_{ij}),\boldsymbol{B}=(b_{ij})\)是\(m\times n\)矩阵,且\(b_{ij}=(-1)^{i + j}a_{ij}\). 求证:\(\mathrm{r}(\boldsymbol{A})=\mathrm{r}(\boldsymbol{B})\).
\end{example}
\begin{proof}
将\(\boldsymbol{A}\)的第\(i\)行乘以\((-1)^i\),又将第\(j\)列乘以\((-1)^j\),即得矩阵\(\boldsymbol{B}\),因此\(\boldsymbol{A}\)和\(\boldsymbol{B}\)相抵,故结论成立. 
\end{proof}

\begin{proposition}[Sylvester不等式]\label{proposition:Sylvester不等式}
设$\boldsymbol{A}$是$m\times n$矩阵,$\boldsymbol{B}$是$n\times t$矩阵,求证:
\[
\mathrm{r}(\boldsymbol{A}\boldsymbol{B})\geq\mathrm{r}(\boldsymbol{A})+\mathrm{r}(\boldsymbol{B}) - n.
\]
\end{proposition}
\begin{proof}
{\color{blue}证法一:}
考虑下列矩阵的分块初等变换:
\[
\begin{pmatrix}
\boldsymbol{I}_n&\boldsymbol{O}\\
\boldsymbol{O}&\boldsymbol{A}\boldsymbol{B}
\end{pmatrix}\to
\begin{pmatrix}
\boldsymbol{I}_n&\boldsymbol{O}\\
\boldsymbol{A}&\boldsymbol{A}\boldsymbol{B}
\end{pmatrix}\to
\begin{pmatrix}
\boldsymbol{I}_n&-\boldsymbol{B}\\
\boldsymbol{A}&\boldsymbol{O}
\end{pmatrix}\to
\begin{pmatrix}
\boldsymbol{B}&\boldsymbol{I}_n\\
\boldsymbol{O}&\boldsymbol{A}
\end{pmatrix},
\]
由\hyperref[矩阵秩的基本公式3]{矩阵秩的基本公式(3)}和\hyperref[矩阵秩的基本公式4]{矩阵秩的基本公式(4)}可得
\[
\mathrm{r}(\boldsymbol{A}\boldsymbol{B}) + n=\mathrm{r}\begin{pmatrix}
\boldsymbol{I}_n&\boldsymbol{O}\\
\boldsymbol{O}&\boldsymbol{A}\boldsymbol{B}
\end{pmatrix}=\mathrm{r}\begin{pmatrix}
\boldsymbol{B}&\boldsymbol{I}_n\\
\boldsymbol{O}&\boldsymbol{A}
\end{pmatrix}\geq\mathrm{r}(\boldsymbol{A})+\mathrm{r}(\boldsymbol{B}),
\]
即\(\mathrm{r}(\boldsymbol{A}\boldsymbol{B})\geq\mathrm{r}(\boldsymbol{A})+\mathrm{r}(\boldsymbol{B}) - n\).

{\color{blue}证法二:}见\hyperref[example:4.243456]{例题\ref{example:4.243456}}.
\end{proof}

\begin{corollary}\label{corollary:矩阵的秩不等式1}
若\(\boldsymbol{A}\)是\(m\times n\)矩阵,\(\boldsymbol{B}\)是\(n\times t\)矩阵且\(\boldsymbol{A}\boldsymbol{B}=\boldsymbol{O}\),则\(\mathrm{r}(\boldsymbol{A})+\mathrm{r}(\boldsymbol{B})\leq n\).
\end{corollary}

\begin{proposition}[Sylvester不等式的推广]\label{proposition:Sylvester不等式的推广}
设\(\boldsymbol{A}_1,\boldsymbol{A}_2,\cdots,\boldsymbol{A}_m\)为\(n\)阶方阵,求证:
\[
\mathrm{r}(\boldsymbol{A}_1)+\mathrm{r}(\boldsymbol{A}_2)+\cdots+\mathrm{r}(\boldsymbol{A}_m)\leq(m - 1)n+\mathrm{r}(\boldsymbol{A}_1\boldsymbol{A}_2\cdots\boldsymbol{A}_m).
\]
特别地,若\(\boldsymbol{A}_1\boldsymbol{A}_2\cdots\boldsymbol{A}_m=\boldsymbol{O}\),则\(\mathrm{r}(\boldsymbol{A}_1)+\mathrm{r}(\boldsymbol{A}_2)+\cdots+\mathrm{r}(\boldsymbol{A}_m)\leq(m - 1)n\).
\end{proposition}
\begin{proof}
反复利用\hyperref[proposition:Sylvester不等式]{Sylvester不等式}可得
\begin{align*}
&\mathrm{r}(\boldsymbol{A}_1)+\mathrm{r}(\boldsymbol{A}_2)+\mathrm{r}(\boldsymbol{A}_3)+\cdots+\mathrm{r}(\boldsymbol{A}_m)\\
\leq&n+\mathrm{r}(\boldsymbol{A}_1\boldsymbol{A}_2)+\mathrm{r}(\boldsymbol{A}_3)+\cdots+\mathrm{r}(\boldsymbol{A}_m)\\
\leq&2n+\mathrm{r}(\boldsymbol{A}_1\boldsymbol{A}_2\boldsymbol{A}_3)+\cdots+\mathrm{r}(\boldsymbol{A}_m)\\
\leq&\cdots\leq(m - 1)n+\mathrm{r}(\boldsymbol{A}_1\boldsymbol{A}_2\cdots\boldsymbol{A}_m).
\end{align*}
\end{proof}

\begin{example}
设\(\boldsymbol{A},\boldsymbol{B}\)为\(n\)阶方阵,满足\(\boldsymbol{A}\boldsymbol{B}=\boldsymbol{O}\). 证明:若\(n\)是奇数,则\(\boldsymbol{A}\boldsymbol{B}'+\boldsymbol{A}'\boldsymbol{B}\)必为奇异阵;若\(n\)为偶数,举例说明上述结论一般不成立.
\end{example}
\begin{proof}
由\hyperref[corollary:矩阵的秩不等式1]{推论\ref{corollary:矩阵的秩不等式1}}可知,\(\mathrm{r}(\boldsymbol{A})+\mathrm{r}(\boldsymbol{B})\leq n\). 若\(n\)为奇数,则\(\mathrm{r}(\boldsymbol{A}),\mathrm{r}(\boldsymbol{B})\)中至少有一个小于等于\(\frac{n}{2}\),从而小于等于\(\frac{n - 1}{2}\). 不妨设\(\mathrm{r}(\boldsymbol{A})\leq\frac{n - 1}{2}\),于是
\[
\mathrm{r}(\boldsymbol{A}\boldsymbol{B}'+\boldsymbol{A}'\boldsymbol{B})\leq\mathrm{r}(\boldsymbol{A}\boldsymbol{B}')+\mathrm{r}(\boldsymbol{A}'\boldsymbol{B})\leq\mathrm{r}(\boldsymbol{A})+\mathrm{r}(\boldsymbol{A}') = 2\mathrm{r}(\boldsymbol{A})\leq n - 1,
\]
从而\(\boldsymbol{A}\boldsymbol{B}'+\boldsymbol{A}'\boldsymbol{B}\)为奇异阵. 例如,当\(n = 2\)时,令\(\boldsymbol{A}=\boldsymbol{B}=\begin{pmatrix}
0&1\\
0&0
\end{pmatrix}\),则\(\boldsymbol{A}\boldsymbol{B}=\boldsymbol{O}\),但\(\boldsymbol{A}\boldsymbol{B}'+\boldsymbol{A}'\boldsymbol{B}=\boldsymbol{I}_2\)为非异阵.
\end{proof}

\begin{proposition}[Frobenius不等式]\label{proposition:Frobenius不等式}
证明:\(\mathrm{r}(\boldsymbol{A}\boldsymbol{B}\boldsymbol{C})\geq\mathrm{r}(\boldsymbol{A}\boldsymbol{B})+\mathrm{r}(\boldsymbol{B}\boldsymbol{C})-\mathrm{r}(\boldsymbol{B})\).
\end{proposition}
\begin{proof}
{\color{blue}证法一:}
考虑下列分块初等变换:
\[
\begin{pmatrix}
\boldsymbol{A}\boldsymbol{B}\boldsymbol{C}&\boldsymbol{O}\\
\boldsymbol{O}&\boldsymbol{B}
\end{pmatrix}\to
\begin{pmatrix}
\boldsymbol{A}\boldsymbol{B}\boldsymbol{C}&\boldsymbol{A}\boldsymbol{B}\\
\boldsymbol{O}&\boldsymbol{B}
\end{pmatrix}\to
\begin{pmatrix}
\boldsymbol{O}&\boldsymbol{A}\boldsymbol{B}\\
-\boldsymbol{B}\boldsymbol{C}&\boldsymbol{B}
\end{pmatrix}\to
\begin{pmatrix}
\boldsymbol{A}\boldsymbol{B}&\boldsymbol{O}\\
\boldsymbol{B}&\boldsymbol{B}\boldsymbol{C}
\end{pmatrix}.
\]
由\hyperref[矩阵秩的基本公式3]{矩阵秩的基本公式(3)}和\hyperref[矩阵秩的基本公式4]{矩阵秩的基本公式(4)}可得
\[
\mathrm{r}(\boldsymbol{A}\boldsymbol{B}\boldsymbol{C})+\mathrm{r}(\boldsymbol{B})=\mathrm{r}\begin{pmatrix}
\boldsymbol{A}\boldsymbol{B}\boldsymbol{C}&\boldsymbol{O}\\
\boldsymbol{O}&\boldsymbol{B}
\end{pmatrix}=\mathrm{r}\begin{pmatrix}
\boldsymbol{A}\boldsymbol{B}&\boldsymbol{O}\\
\boldsymbol{B}&\boldsymbol{B}\boldsymbol{C}
\end{pmatrix}\geq\mathrm{r}(\boldsymbol{A}\boldsymbol{B})+\mathrm{r}(\boldsymbol{B}\boldsymbol{C}),
\]
由此即得结论. 

{\color{blue}证法二(几何方法):}将问题转化成几何的语言即为:设\(\varphi:V_1\to V_2,\psi:V_2\to V_3,\theta:V_3\to V_4\)是线性映射,证明:\(\text{r}(\theta\psi\varphi)\geq\text{r}(\theta\psi)+\text{r}(\psi\varphi)-\text{r}(\psi)\).

下面考虑通过定义域的限制得到的线性映射. 将\(\theta\)的定义域限制在\(\text{Im}\psi\varphi\)上可得线性映射\(\theta_1:\text{Im}\psi\varphi\to V_4\),它的像空间是\(\text{Im}\theta\psi\varphi\),核空间是\(\text{Ker}\theta\cap\text{Im}\psi\varphi\);将\(\theta\)的定义域限制在\(\text{Im}\psi\)上可得线性映射\(\theta_2:\text{Im}\psi\to V_4\),它的像空间是\(\text{Im}\theta\psi\),核空间是\(\text{Ker}\theta\cap\text{Im}\psi\),故由\hyperref[proposition:值域和核空间维数之和等于原像空间维数]{线性映射的维数公式}可得
\begin{align}
\dim(\text{Im}\psi\varphi)&=\dim(\text{Ker}\theta\cap\text{Im}\psi\varphi)+\dim(\text{Im}\theta\psi\varphi),\label{equation:3.334.1}\\
\dim(\text{Im}\psi)&=\dim(\text{Ker}\theta\cap\text{Im}\psi)+\dim(\text{Im}\theta\psi).\label{equation:3.334.2}
\end{align}
注意到\(\text{Im}\psi\varphi\subseteq\text{Im}\psi\),故\(\dim(\text{Ker}\theta\cap\text{Im}\psi\varphi)\leq\dim(\text{Ker}\theta\cap\text{Im}\psi)\),从而由\eqref{equation:3.334.1}式和\eqref{equation:3.334.2}式可得
\begin{align*}
&\mathrm{dim}\left( \mathrm{Im}\psi \varphi \right) -\mathrm{dim}\left( \mathrm{Im}\theta \psi \varphi \right) =\mathrm{dim}\left( \mathrm{Ker}\theta \cap \mathrm{Im}\psi \varphi \right) 
\\
&\leqslant \mathrm{dim}\left( \mathrm{Ker}\theta \cap \mathrm{Im}\psi \right) =\mathrm{dim}\left( \mathrm{Im}\psi \right) -\mathrm{dim}\left( \mathrm{Im}\theta \psi \right) .
\end{align*}
于是
\begin{gather*}
\mathrm{dim}\left( \mathrm{Im}\psi \varphi \right) -\mathrm{dim}\left( \mathrm{Im}\theta \psi \varphi \right) \leqslant \mathrm{dim}\left( \mathrm{Im}\psi \right) -\mathrm{dim}\left( \mathrm{Im}\theta \psi \right) 
\\
\Leftrightarrow \text{r}(\psi\varphi)-\text{r}(\theta\psi\varphi)\leq\text{r}(\psi)-\text{r}(\theta\psi),
\end{gather*}
结论得证. 
\end{proof}

\begin{proposition}[幂等矩阵关于秩的判定准则]\label{proposition:幂等矩阵关于秩的判定准则}
求证:\(n\)阶矩阵\(\boldsymbol{A}\)是幂等矩阵(即\(\boldsymbol{A}^2 = \boldsymbol{A}\))的充要条件是:
\[
\mathrm{r}(\boldsymbol{A})+\mathrm{r}(\boldsymbol{I}_n - \boldsymbol{A}) = n.
\]
\end{proposition}
\begin{proof}
在下列矩阵的分块初等变换中矩阵的秩保持不变:
\[
\begin{pmatrix}
\boldsymbol{A}&\boldsymbol{O}\\
\boldsymbol{O}&\boldsymbol{I}-\boldsymbol{A}
\end{pmatrix}\to
\begin{pmatrix}
\boldsymbol{A}&\boldsymbol{A}\\
\boldsymbol{O}&\boldsymbol{I}-\boldsymbol{A}
\end{pmatrix}\to
\begin{pmatrix}
\boldsymbol{A}&\boldsymbol{A}\\
\boldsymbol{A}&\boldsymbol{I}
\end{pmatrix}\to
\begin{pmatrix}
\boldsymbol{A}-\boldsymbol{A}^2&\boldsymbol{A}\\
\boldsymbol{O}&\boldsymbol{I}
\end{pmatrix}\to
\begin{pmatrix}
\boldsymbol{A}-\boldsymbol{A}^2&\boldsymbol{O}\\
\boldsymbol{O}&\boldsymbol{I}
\end{pmatrix}.
\]
因此
\[
\mathrm{r}\begin{pmatrix}
\boldsymbol{A}&\boldsymbol{O}\\
\boldsymbol{O}&\boldsymbol{I}-\boldsymbol{A}
\end{pmatrix}=\mathrm{r}\begin{pmatrix}
\boldsymbol{A}-\boldsymbol{A}^2&\boldsymbol{O}\\
\boldsymbol{O}&\boldsymbol{I}
\end{pmatrix},
\]
即\(\mathrm{r}(\boldsymbol{A})+\mathrm{r}(\boldsymbol{I}-\boldsymbol{A})=\mathrm{r}(\boldsymbol{A}-\boldsymbol{A}^2)+n\),由此即得结论.  
\end{proof}

\begin{proposition}[对合矩阵关于秩的判定准则]\label{proposition:对合矩阵关于秩的判定准则}
求证:\(n\)阶矩阵\(\boldsymbol{A}\)是对合矩阵(即\(\boldsymbol{A}^2 = \boldsymbol{I}_n\))的充要条件是:
\[
\mathrm{r}(\boldsymbol{I}_n+\boldsymbol{A})+\mathrm{r}(\boldsymbol{I}_n - \boldsymbol{A}) = n.
\]
\end{proposition}
\begin{proof}
在下列矩阵的分块初等变换中,矩阵的秩保持不变:
\[
\begin{pmatrix}
\boldsymbol{I}_n+\boldsymbol{A}&\boldsymbol{O}\\
\boldsymbol{O}&\boldsymbol{I}_n - \boldsymbol{A}
\end{pmatrix}\to
\begin{pmatrix}
\boldsymbol{I}_n+\boldsymbol{A}&\boldsymbol{I}_n+\boldsymbol{A}\\
\boldsymbol{O}&\boldsymbol{I}_n - \boldsymbol{A}
\end{pmatrix}\to
\begin{pmatrix}
\boldsymbol{I}_n+\boldsymbol{A}&\boldsymbol{I}_n+\boldsymbol{A}\\
\boldsymbol{I}_n+\boldsymbol{A}&2\boldsymbol{I}_n
\end{pmatrix}\to
\begin{pmatrix}
\frac{1}{2}(\boldsymbol{I}_n - \boldsymbol{A}^2)&\boldsymbol{I}_n+\boldsymbol{A}\\
\boldsymbol{O}&2\boldsymbol{I}_n
\end{pmatrix}\to
\begin{pmatrix}
\frac{1}{2}(\boldsymbol{I}_n - \boldsymbol{A}^2)&\boldsymbol{O}\\
\boldsymbol{O}&2\boldsymbol{I}_n
\end{pmatrix}.
\]
因此
\[
\mathrm{r}\begin{pmatrix}
\boldsymbol{I}_n+\boldsymbol{A}&\boldsymbol{O}\\
\boldsymbol{O}&\boldsymbol{I}_n - \boldsymbol{A}
\end{pmatrix}=\mathrm{r}\begin{pmatrix}
\frac{1}{2}(\boldsymbol{I}_n - \boldsymbol{A}^2)&\boldsymbol{O}\\
\boldsymbol{O}&2\boldsymbol{I}_n
\end{pmatrix},
\]
即\(\mathrm{r}(\boldsymbol{I}_n+\boldsymbol{A})+\mathrm{r}(\boldsymbol{I}_n - \boldsymbol{A})=\mathrm{r}(\boldsymbol{I}_n - \boldsymbol{A}^2)+n\),由此即得结论. 
\end{proof}

\begin{example}
设\(\boldsymbol{A}\)是\(n\)阶矩阵,求证:\(\mathrm{r}(\boldsymbol{A})+\mathrm{r}(\boldsymbol{I}_n+\boldsymbol{A})\geq n\).
\end{example}
\begin{proof}
{\color{blue}证法一:}
由下列分块初等变换即得结论
\[
\begin{pmatrix}
\boldsymbol{A}&\boldsymbol{O}\\
\boldsymbol{O}&\boldsymbol{I}+\boldsymbol{A}
\end{pmatrix}\to
\begin{pmatrix}
\boldsymbol{A}&\boldsymbol{A}\\
\boldsymbol{O}&\boldsymbol{I}+\boldsymbol{A}
\end{pmatrix}\to
\begin{pmatrix}
\boldsymbol{A}&\boldsymbol{A}\\
-\boldsymbol{A}&\boldsymbol{I}
\end{pmatrix}\to
\begin{pmatrix}
\boldsymbol{A}+\boldsymbol{A}^2&\boldsymbol{A}\\
\boldsymbol{O}&\boldsymbol{I}
\end{pmatrix}\to
\begin{pmatrix}
\boldsymbol{A}+\boldsymbol{A}^2&\boldsymbol{O}\\
\boldsymbol{O}&\boldsymbol{I}
\end{pmatrix}.
\]

{\color{blue}证法二:}
\(\mathrm{r}(\boldsymbol{A})+\mathrm{r}(\boldsymbol{I}+\boldsymbol{A})=\mathrm{r}(-\boldsymbol{A})+\mathrm{r}(\boldsymbol{I}+\boldsymbol{A})\geq\mathrm{r}(-\boldsymbol{A}+\boldsymbol{I}+\boldsymbol{A})=\mathrm{r}(\boldsymbol{I}) = n\).
\end{proof}

\begin{proposition}[秩的降阶公式]\label{proposition:秩的降阶公式}
设有分块矩阵\(M = \begin{pmatrix}
\boldsymbol{A}&\boldsymbol{B}\\
\boldsymbol{C}&\boldsymbol{D}
\end{pmatrix}\),证明:
\begin{enumerate}[(1)]
\item 若\(\boldsymbol{A}\)可逆,则\(\mathrm{r}(M)=\mathrm{r}(\boldsymbol{A})+\mathrm{r}(\boldsymbol{D}-\boldsymbol{C}\boldsymbol{A}^{-1}\boldsymbol{B})\);

\item 若\(\boldsymbol{D}\)可逆,则\(\mathrm{r}(M)=\mathrm{r}(\boldsymbol{D})+\mathrm{r}(\boldsymbol{A}-\boldsymbol{B}\boldsymbol{D}^{-1}\boldsymbol{C})\);

\item 若\(\boldsymbol{A},\boldsymbol{D}\)都可逆,则\(\mathrm{r}(\boldsymbol{A})+\mathrm{r}(\boldsymbol{D}-\boldsymbol{C}\boldsymbol{A}^{-1}\boldsymbol{B})=\mathrm{r}(\boldsymbol{D})+\mathrm{r}(\boldsymbol{A}-\boldsymbol{B}\boldsymbol{D}^{-1}\boldsymbol{C})\).
\end{enumerate}
\end{proposition}
\begin{proof}
\begin{enumerate}[(1)]
\item 由分块初等变换可得
\[
\begin{pmatrix}
\boldsymbol{A}&\boldsymbol{B}\\
\boldsymbol{C}&\boldsymbol{D}
\end{pmatrix}\to
\begin{pmatrix}
\boldsymbol{A}&\boldsymbol{B}\\
\boldsymbol{O}&\boldsymbol{D}-\boldsymbol{C}\boldsymbol{A}^{-1}\boldsymbol{B}
\end{pmatrix}\to
\begin{pmatrix}
\boldsymbol{A}&\boldsymbol{O}\\
\boldsymbol{O}&\boldsymbol{D}-\boldsymbol{C}\boldsymbol{A}^{-1}\boldsymbol{B}
\end{pmatrix},
\]
由此即得结论.

\item 同理可证明.

\item 由(1)和(2)即得. 
\end{enumerate}
\end{proof}

\begin{example}
设
\[
M = \begin{pmatrix}
a_1^2&a_1a_2 + 1&\cdots&a_1a_n + 1\\
a_2a_1 + 1&a_2^2&\cdots&a_2a_n + 1\\
\vdots&\vdots&&\vdots\\
a_na_1 + 1&a_na_2 + 1&\cdots&a_n^2
\end{pmatrix},
\]
证明:\(\mathrm{r}(M)\geq n - 1\),等号成立当且仅当\(|M| = 0\).
\end{example}
\begin{proof}
若\(n = 1\),结论显然成立. 下设\(n\geq2\). 取\(\boldsymbol{A}=\begin{pmatrix}
a_1&a_2&\cdots&a_n\\
1&1&\cdots&1
\end{pmatrix}\),则
\begin{align*}
\boldsymbol{M}=-\boldsymbol{I}_n+\left( \begin{matrix}
a_1&		1\\
a_2&		1\\
\vdots&		\vdots\\
a_n&		1\\
\end{matrix} \right) \boldsymbol{I}_{2}^{-1}\left( \begin{matrix}
a_1&		a_2&		\cdots&		a_n\\
1&		1&		\cdots&		1\\
\end{matrix} \right) =-\boldsymbol{I}_n+\boldsymbol{A}'\boldsymbol{I}_{2}^{-1}\boldsymbol{A}=-\left( \boldsymbol{I}_n-\boldsymbol{A}'\boldsymbol{I}_{2}^{-1}\boldsymbol{A} \right) .
\end{align*}
由\hyperref[proposition:秩的降阶公式]{秩的降阶公式}可得
\begin{align*}
2+\mathrm{r}\left( \boldsymbol{M} \right) =2+\mathrm{r}\left( -\boldsymbol{M} \right) =\mathrm{r}\left( \boldsymbol{I}_2 \right) +\mathrm{r}\left( \boldsymbol{I}_n-\boldsymbol{AI}_{2}^{-1}\boldsymbol{A}' \right) =\mathrm{r}\left( \boldsymbol{I}_n \right) +\mathrm{r}\left( \boldsymbol{I}_2-\boldsymbol{A}'\boldsymbol{I}_{n}^{-1}\boldsymbol{A} \right) =n+\mathrm{r}\left( \begin{matrix}
\sum_{i=1}^n{a_{i}^{2}}-1&		\sum_{i=1}^n{a_i}\\
\sum_{i=1}^n{a_i}&		n-1\\
\end{matrix} \right) .
\end{align*}
而$\mathrm{r}\left( \begin{matrix}
\sum_{i=1}^n{a_{i}^{2}}-1&		\sum_{i=1}^n{a_i}\\
\sum_{i=1}^n{a_i}&		n-1\\
\end{matrix} \right) \geqslant 1$,
于是\(\mathrm{r}\left( \boldsymbol{M} \right) =n-2+\mathrm{r}\left( \begin{matrix}
\sum_{i=1}^n{a_{i}^{2}}-1&		\sum_{i=1}^n{a_i}\\
\sum_{i=1}^n{a_i}&		n-1\\
\end{matrix} \right) \geqslant n-1.\),等号成立当且仅当\(M\)不满秩,即\(|M| = 0\).
\end{proof}

\begin{proposition}\label{proposition:矩阵乘法可交换的秩不等式}
设\(\boldsymbol{A},\boldsymbol{B}\)都是数域\(\mathbb{K}\)上的\(n\)阶矩阵且\(\boldsymbol{A}\boldsymbol{B}=\boldsymbol{B}\boldsymbol{A}\),证明:
\[
\mathrm{r}(\boldsymbol{A}+\boldsymbol{B})\leq\mathrm{r}(\boldsymbol{A})+\mathrm{r}(\boldsymbol{B})-\mathrm{r}(\boldsymbol{A}\boldsymbol{B}).
\]
\end{proposition}
\begin{remark}
{\color{blue}证法一:}这里乘的不只是初等变换矩阵.记住这个分块矩阵乘法和构造.

{\color{blue}证法二:}和{\color{blue}证法三:}思路分析:将秩不等式转化为维数公式就能自然得到证明的想法.
\end{remark}
\begin{proof}
{\color{blue}证法一:}
考虑如下分块矩阵的乘法:
\[
\begin{pmatrix}
\boldsymbol{I}&\boldsymbol{I}\\
\boldsymbol{O}&\boldsymbol{I}
\end{pmatrix}
\begin{pmatrix}
\boldsymbol{A}&\boldsymbol{O}\\
\boldsymbol{O}&\boldsymbol{B}
\end{pmatrix}
\begin{pmatrix}
\boldsymbol{I}&-\boldsymbol{B}\\
\boldsymbol{I}&\boldsymbol{A}
\end{pmatrix}=
\begin{pmatrix}
\boldsymbol{A}+\boldsymbol{B}&-\boldsymbol{A}\boldsymbol{B}+\boldsymbol{B}\boldsymbol{A}\\
\boldsymbol{B}&\boldsymbol{B}\boldsymbol{A}
\end{pmatrix}=
\begin{pmatrix}
\boldsymbol{A}+\boldsymbol{B}&\boldsymbol{O}\\
\boldsymbol{B}&\boldsymbol{A}\boldsymbol{B}
\end{pmatrix}.
\]
由\hyperref[矩阵秩的基本公式2]{矩阵秩的基本公式(2)}和\hyperref[矩阵秩的基本公式4]{矩阵秩的基本公式(4)}可得
\[
\mathrm{r}(\boldsymbol{A})+\mathrm{r}(\boldsymbol{B})=\mathrm{r}\begin{pmatrix}
\boldsymbol{A}&\boldsymbol{O}\\
\boldsymbol{O}&\boldsymbol{B}
\end{pmatrix}\geq\mathrm{r}\begin{pmatrix}
\boldsymbol{A}+\boldsymbol{B}&\boldsymbol{O}\\
\boldsymbol{B}&\boldsymbol{B}\boldsymbol{A}
\end{pmatrix}\geq\mathrm{r}(\boldsymbol{A}+\boldsymbol{B})+\mathrm{r}(\boldsymbol{A}\boldsymbol{B}),
\]
由此即得结论.

{\color{blue}证法二:}设\(V_{\boldsymbol{A}}\)是方程组\(\boldsymbol{A}\boldsymbol{x}=\boldsymbol{0}\)的解空间,\(V_{\boldsymbol{B}},V_{\boldsymbol{A}\boldsymbol{B}},V_{\boldsymbol{A}+\boldsymbol{B}}\)的意义同理. 若列向量\(\boldsymbol{\alpha}\in V_{\boldsymbol{A}}\cap V_{\boldsymbol{B}}\),即\(\boldsymbol{\alpha}\)满足\(\boldsymbol{A}\boldsymbol{\alpha}=\boldsymbol{0}\)且\(\boldsymbol{B}\boldsymbol{\alpha}=\boldsymbol{0}\),于是\((\boldsymbol{A}+\boldsymbol{B})\boldsymbol{\alpha}=\boldsymbol{0}\),即\(\boldsymbol{\alpha}\in V_{\boldsymbol{A}+\boldsymbol{B}}\),从而\(V_{\boldsymbol{A}}\cap V_{\boldsymbol{B}}\subseteq V_{\boldsymbol{A}+\boldsymbol{B}}\). 同理可证\(V_{\boldsymbol{A}}\subseteq V_{\boldsymbol{B}\boldsymbol{A}}\),\(V_{\boldsymbol{B}}\subseteq V_{\boldsymbol{A}\boldsymbol{B}}\). 因为\(\boldsymbol{A}\boldsymbol{B}=\boldsymbol{B}\boldsymbol{A}\),所以\(V_{\boldsymbol{B}\boldsymbol{A}} = V_{\boldsymbol{A}\boldsymbol{B}}\),从而\(V_{\boldsymbol{A}}+V_{\boldsymbol{B}}\subseteq V_{\boldsymbol{A}\boldsymbol{B}}\). 因此,我们有
\[
\dim(V_{\boldsymbol{A}}\cap V_{\boldsymbol{B}})\leq\dim V_{\boldsymbol{A}+\boldsymbol{B}}=n - \mathrm{r}(\boldsymbol{A}+\boldsymbol{B}),\quad\dim(V_{\boldsymbol{A}}+V_{\boldsymbol{B}})\leq\dim V_{\boldsymbol{A}\boldsymbol{B}}=n - \mathrm{r}(\boldsymbol{A}\boldsymbol{B}).
\]
将上面两个不等式相加,再由交和空间维数公式可得
\begin{align*}
n - \mathrm{r}(\boldsymbol{A}+\boldsymbol{B})+n - \mathrm{r}(\boldsymbol{A}\boldsymbol{B})&\geq\dim(V_{\boldsymbol{A}}\cap V_{\boldsymbol{B}})+\dim(V_{\boldsymbol{A}}+V_{\boldsymbol{B}})\\
&=\dim V_{\boldsymbol{A}}+\dim V_{\boldsymbol{B}}=n - \mathrm{r}(\boldsymbol{A})+n - \mathrm{r}(\boldsymbol{B}),
\end{align*}
因此\(\mathrm{r}(\boldsymbol{A}+\boldsymbol{B})+\mathrm{r}(\boldsymbol{A}\boldsymbol{B})\leq\mathrm{r}(\boldsymbol{A})+\mathrm{r}(\boldsymbol{B})\),结论得证.

{\color{blue}证法三:}设\(\boldsymbol{A}=(\boldsymbol{\alpha}_1,\boldsymbol{\alpha}_2,\cdots,\boldsymbol{\alpha}_n)\)为\(\boldsymbol{A}\)的列分块,\(\boldsymbol{B}=(\boldsymbol{\beta}_1,\boldsymbol{\beta}_2,\cdots,\boldsymbol{\beta}_n)\)为\(\boldsymbol{B}\)的列分块. 记\(U_{\boldsymbol{A}} = L(\boldsymbol{\alpha}_1,\boldsymbol{\alpha}_2,\cdots,\boldsymbol{\alpha}_n)\)为\(\boldsymbol{A}\)的列向量生成的\(\mathbb{K}^n\)的子空间,\(U_{\boldsymbol{B}},U_{\boldsymbol{A}\boldsymbol{B}},U_{\boldsymbol{A}+\boldsymbol{B}}\)的意义同理. 因为向量组\(\boldsymbol{\alpha}_1,\boldsymbol{\alpha}_2,\cdots,\boldsymbol{\alpha}_n\)的极大无关组就是\(L(\boldsymbol{\alpha}_1,\boldsymbol{\alpha}_2,\cdots,\boldsymbol{\alpha}_n)\)的一组基,故\(\mathrm{r}(\boldsymbol{A})=\dim U_{\boldsymbol{A}}\),关于\(\boldsymbol{B},\boldsymbol{A}\boldsymbol{B},\boldsymbol{A}+\boldsymbol{B}\)的等式同理可得. 显然,我们有\(U_{\boldsymbol{A}+\boldsymbol{B}}\subseteq U_{\boldsymbol{A}}+U_{\boldsymbol{B}}\). 注意到\(\boldsymbol{A}\boldsymbol{B}=(\boldsymbol{A}\boldsymbol{\beta}_1,\boldsymbol{A}\boldsymbol{\beta}_2,\cdots,\boldsymbol{A}\boldsymbol{\beta}_n)\),若设\(\boldsymbol{\beta}_j=(b_{1j},b_{2j},\cdots,b_{nj})'\),则\(\boldsymbol{A}\boldsymbol{B}\)的列向量\(\boldsymbol{A}\boldsymbol{\beta}_j=b_{1j}\boldsymbol{\alpha}_1 + b_{2j}\boldsymbol{\alpha}_2+\cdots + b_{nj}\boldsymbol{\alpha}_n\in U_{\boldsymbol{A}}\),从而\(U_{\boldsymbol{A}\boldsymbol{B}}\subseteq U_{\boldsymbol{A}}\). 同理可得\(U_{\boldsymbol{B}\boldsymbol{A}}\subseteq U_{\boldsymbol{B}}\).又因为\(\boldsymbol{A}\boldsymbol{B}=\boldsymbol{B}\boldsymbol{A}\),故\(U_{\boldsymbol{A}\boldsymbol{B}}\subseteq U_{\boldsymbol{A}}\cap U_{\boldsymbol{B}}\). 最后,由上述包含关系以及交和空间维数公式可得
\begin{align*}
\mathrm{r}(\boldsymbol{A}+\boldsymbol{B})+\mathrm{r}(\boldsymbol{A}\boldsymbol{B})&=\dim U_{\boldsymbol{A}+\boldsymbol{B}}+\dim U_{\boldsymbol{A}\boldsymbol{B}}\leq\dim(U_{\boldsymbol{A}}+U_{\boldsymbol{B}})+\dim(U_{\boldsymbol{A}}\cap U_{\boldsymbol{B}})\\
&=\dim U_{\boldsymbol{A}}+\dim U_{\boldsymbol{B}}=\mathrm{r}(\boldsymbol{A})+\mathrm{r}(\boldsymbol{B}).
\end{align*}
\end{proof}

\subsection{利用线性方程组的求解理论讨论矩阵的秩}

\begin{theorem}\label{theorem:系数矩阵的秩与解空间的维数}
设\(\boldsymbol{A}\)是数域\(\mathbb{K}\)上的\(m\times n\)矩阵,则齐次线性方程组\(\boldsymbol{A}\boldsymbol{x}=\boldsymbol{0}\)的解集\(V_{\boldsymbol{A}}\)是\(n\)维列向量空间\(\mathbb{K}^n\)的子空间. 根据线性方程组的求解理论,我们有
\[
\dim V_{\boldsymbol{A}}+\mathrm{r}(\boldsymbol{A}) = n,
\]
\end{theorem}
\begin{note}
即齐次线性方程组解空间的维数与系数矩阵的秩之和等于未知数的个数. 根据上述公式,由矩阵的秩可以讨论线性方程组解的性质;反过来,也可以由线性方程组解的性质讨论矩阵的秩. 
\end{note}

\begin{corollary}\label{corollary:线性方程组只有零解的充要条件}
线性方程组\(\boldsymbol{A}\boldsymbol{x}=\boldsymbol{0}\)只有零解的充要条件是\(\boldsymbol{A}\)为列满秩阵. 特别地,若\(\boldsymbol{A}\)是方阵,则线性方程组\(\boldsymbol{A}\boldsymbol{x}=\boldsymbol{0}\)只有零解的充要条件是\(\boldsymbol{A}\)为非异阵. 
\end{corollary}
\begin{proof}
由\hyperref[theorem:系数矩阵的秩与解空间的维数]{定理\ref{theorem:系数矩阵的秩与解空间的维数}中的公式}\(\dim V_{\boldsymbol{A}}+\mathrm{r}(\boldsymbol{A}) = n\)即可得到证明.
\end{proof}


\begin{proposition}\label{proposition:r(AA')=r(A)}
\begin{enumerate}[(1)]
\item 设\(\boldsymbol{A}\)是\(m\times n\)实矩阵,求证:\(\mathrm{r}(\boldsymbol{A}'\boldsymbol{A})=\mathrm{r}(\boldsymbol{A}\boldsymbol{A}')=\mathrm{r}(\boldsymbol{A})\).

\item 若\(\boldsymbol{A}\)是\(m\times n\)复矩阵,则\(\mathrm{r}\left(\overline{\boldsymbol{A}}'\boldsymbol{A}\right)=\mathrm{r}\left(\boldsymbol{A}\overline{\boldsymbol{A}}'\right)=\mathrm{r}(\boldsymbol{A})\).
\end{enumerate}
\end{proposition}
\begin{proof}
\begin{enumerate}[(1)]
\item 首先证明\(\mathrm{r}(\boldsymbol{A}'\boldsymbol{A})=\mathrm{r}(\boldsymbol{A})\),为此我们将证明齐次线性方程组\(\boldsymbol{A}\boldsymbol{x}=\boldsymbol{0}\)和\(\boldsymbol{A}'\boldsymbol{A}\boldsymbol{x}=\boldsymbol{0}\)同解. 显然\(\boldsymbol{A}\boldsymbol{x}=\boldsymbol{0}\)的解都是\(\boldsymbol{A}'\boldsymbol{A}\boldsymbol{x}=\boldsymbol{0}\)的解. 反之,任取方程组\(\boldsymbol{A}'\boldsymbol{A}\boldsymbol{x}=\boldsymbol{0}\)的解\(\boldsymbol{\alpha}\in\mathbb{R}^n\),则\(\boldsymbol{\alpha}'\boldsymbol{A}'\boldsymbol{A}\boldsymbol{\alpha}=0\),即\((\boldsymbol{A}\boldsymbol{\alpha})'(\boldsymbol{A}\boldsymbol{\alpha}) = 0\). 记\(\boldsymbol{A}\boldsymbol{\alpha}=(b_1,b_2,\cdots,b_m)'\in\mathbb{R}^m\),则
\[
b_1^2 + b_2^2+\cdots + b_m^2 = 0.
\]
因为\(b_i\)是实数,故每个\(b_i = 0\),即\(\boldsymbol{A}\boldsymbol{\alpha}=\boldsymbol{0}\),也即\(\boldsymbol{\alpha}\)是\(\boldsymbol{A}\boldsymbol{x}=0\)的解. 这就证明了方程组\(\boldsymbol{A}\boldsymbol{x}=0\)和\(\boldsymbol{A}'\boldsymbol{A}\boldsymbol{x}=0\)同解,即\(V_{\boldsymbol{A}}=V_{\boldsymbol{A}'\boldsymbol{A}}\),于是由\hyperref[theorem:系数矩阵的秩与解空间的维数]{定理\ref{theorem:系数矩阵的秩与解空间的维数}}可得\(\mathrm{r}(\boldsymbol{A}'\boldsymbol{A})=\mathrm{r}(\boldsymbol{A})\). 在上述等式中用\(\boldsymbol{A}'\)替代\(\boldsymbol{A}\)可得\(\mathrm{r}(\boldsymbol{A}\boldsymbol{A}')=\mathrm{r}(\boldsymbol{A}')\),又因为\(\mathrm{r}(\boldsymbol{A})=\mathrm{r}(\boldsymbol{A}')\),故结论得证.

\item 由(1)类似的方法可以证明.
\end{enumerate}
\end{proof}

\begin{example}
设\(\boldsymbol{A}\)和\(\boldsymbol{B}\)是数域\(\mathbb{K}\)上的\(n\)阶矩阵,若线性方程组\(\boldsymbol{A}\boldsymbol{x}=\boldsymbol{0}\)和\(\boldsymbol{B}\boldsymbol{x}=\boldsymbol{0}\)同解,且每个方程组的基础解系含\(m\)个线性无关的向量,求证:\(\mathrm{r}(\boldsymbol{A}-\boldsymbol{B})\leq n - m\).
\end{example}
\begin{proof}
由方程组\(\boldsymbol{A}\boldsymbol{x}=\boldsymbol{0}\)和\(\boldsymbol{B}\boldsymbol{x}=\boldsymbol{0}\)同解可知,\(\boldsymbol{A}\boldsymbol{x}=\boldsymbol{0}\)的解都是\((\boldsymbol{A}-\boldsymbol{B})\boldsymbol{x}=\boldsymbol{0}\)的解,即\(V_{\boldsymbol{A}}\subseteq V_{\boldsymbol{A}-\boldsymbol{B}}\),从而\(\dim V_{\boldsymbol{A}-\boldsymbol{B}}\geq\dim V_{\boldsymbol{A}} = m\),于是由\hyperref[theorem:系数矩阵的秩与解空间的维数]{定理\ref{theorem:系数矩阵的秩与解空间的维数}}可得$\mathrm{r(}\boldsymbol{A}-\boldsymbol{B})=n-\mathrm{dim}V_{\boldsymbol{A}-\boldsymbol{B}}\le n-m$.
\end{proof}

\begin{proposition}\label{proposition:ABx=O与Bx=O同解的充要条件}
设\(\boldsymbol{A}\)是\(m\times n\)矩阵,\(\boldsymbol{B}\)是\(n\times k\)矩阵,证明:方程组\(\boldsymbol{A}\boldsymbol{B}\boldsymbol{x}=\boldsymbol{0}\)和方程组\(\boldsymbol{B}\boldsymbol{x}=\boldsymbol{0}\)同解的充要条件是\(\mathrm{r}(\boldsymbol{A}\boldsymbol{B})=\mathrm{r}(\boldsymbol{B})\).
\end{proposition}
\begin{proof}
显然方程组\(\boldsymbol{B}\boldsymbol{x}=\boldsymbol{0}\)的解都是方程组\(\boldsymbol{A}\boldsymbol{B}\boldsymbol{x}=\boldsymbol{0}\)的解,即\(V_{\boldsymbol{B}}\subseteq V_{\boldsymbol{A}\boldsymbol{B}}\),于是两个线性方程组同解,即\(V_{\boldsymbol{B}} = V_{\boldsymbol{A}\boldsymbol{B}}\)的充要条件是\(\dim V_{\boldsymbol{B}}=\dim V_{\boldsymbol{A}\boldsymbol{B}}\). 又由\hyperref[theorem:系数矩阵的秩与解空间的维数]{定理\ref{theorem:系数矩阵的秩与解空间的维数}}可知\(\dim V_{\boldsymbol{B}}=k - \mathrm{r}(\boldsymbol{B})\),\(\dim V_{\boldsymbol{A}\boldsymbol{B}}=k - \mathrm{r}(\boldsymbol{A}\boldsymbol{B})\),因此上述两个方程组同解的充要条件是\(\mathrm{r}(\boldsymbol{A}\boldsymbol{B})=\mathrm{r}(\boldsymbol{B})\). 
\end{proof}

\begin{proposition}\label{proposition:r(AB)=r(B)可推出r(ABC)=r(BC)}
设\(\boldsymbol{A}\)是\(m\times n\)矩阵,\(\boldsymbol{B}\)是\(n\times k\)矩阵. 若\(\boldsymbol{A}\boldsymbol{B}\)和\(\boldsymbol{B}\)有相同的秩,求证:对任意的\(k\times l\)矩阵\(\boldsymbol{C}\),矩阵\(\boldsymbol{A}\boldsymbol{B}\boldsymbol{C}\)和矩阵\(\boldsymbol{B}\boldsymbol{C}\)也有相同的秩.
\end{proposition}
\begin{proof}
{\color{blue}证法一:}
由假设和\hyperref[proposition:ABx=O与Bx=O同解的充要条件]{命题\ref{proposition:ABx=O与Bx=O同解的充要条件}}可知,方程组\(\boldsymbol{A}\boldsymbol{B}\boldsymbol{x}=\boldsymbol{0}\)和方程组\(\boldsymbol{B}\boldsymbol{x}=\boldsymbol{0}\)同解. 要证明\(\mathrm{r}(\boldsymbol{A}\boldsymbol{B}\boldsymbol{C})=\mathrm{r}(\boldsymbol{B}\boldsymbol{C})\),我们只要证明方程组\(\boldsymbol{A}\boldsymbol{B}\boldsymbol{C}\boldsymbol{x}=\boldsymbol{0}\)和方程组\(\boldsymbol{B}\boldsymbol{C}\boldsymbol{x}=\boldsymbol{0}\)同解即可. 显然方程组\(\boldsymbol{B}\boldsymbol{C}\boldsymbol{x}=\boldsymbol{0}\)的解都是方程组\(\boldsymbol{A}\boldsymbol{B}\boldsymbol{C}\boldsymbol{x}=\boldsymbol{0}\)的解. 反之,若列向量\(\boldsymbol{\alpha}\)是方程组\(\boldsymbol{A}\boldsymbol{B}\boldsymbol{C}\boldsymbol{x}=\boldsymbol{0}\)的解,则\(\boldsymbol{C}\boldsymbol{\alpha}\)是方程组\(\boldsymbol{A}\boldsymbol{B}\boldsymbol{x}=\boldsymbol{0}\)的解,因此\(\boldsymbol{C}\boldsymbol{\alpha}\)也是方程组\(\boldsymbol{B}\boldsymbol{x}=\boldsymbol{0}\)的解,即\(\boldsymbol{B}\boldsymbol{C}\boldsymbol{\alpha}=\boldsymbol{0}\),于是\(\boldsymbol{\alpha}\)也是方程组\(\boldsymbol{B}\boldsymbol{C}\boldsymbol{x}=\boldsymbol{0}\)的解. 这就证明了方程组\(\boldsymbol{A}\boldsymbol{B}\boldsymbol{C}\boldsymbol{x}=\boldsymbol{0}\)和方程组\(\boldsymbol{B}\boldsymbol{C}\boldsymbol{x}=\boldsymbol{0}\)同解,从而结论得证.

{\color{blue}证法二:}
由\hyperref[proposition:Frobenius不等式]{Frobenius不等式}可得
\[
\mathrm{r}(\boldsymbol{A}\boldsymbol{B}\boldsymbol{C})\geq\mathrm{r}(\boldsymbol{A}\boldsymbol{B})+\mathrm{r}(\boldsymbol{B}\boldsymbol{C})-\mathrm{r}(\boldsymbol{B})=\mathrm{r}(\boldsymbol{B}\boldsymbol{C}),
\]
又由\hyperref[矩阵秩的基本公式2]{矩阵秩的基本公式(2)}可知\(\mathrm{r}(\boldsymbol{A}\boldsymbol{B}\boldsymbol{C})\leq\mathrm{r}(\boldsymbol{B}\boldsymbol{C})\),故结论得证.
\end{proof}

\begin{proposition}\label{proposition:奇异系数矩阵Ax=0的解空间}
设数域\(\mathbb{K}\)上的\(n\)阶矩阵\(\boldsymbol{A}=(a_{ij})\)满足:\(|\boldsymbol{A}| = 0\)且某个元素\(a_{ij}\)的代数余子式\(A_{ij}\neq0\). 求证:齐次线性方程组\(\boldsymbol{A}\boldsymbol{x}=\boldsymbol{0}\)的所有解都可写为下列形式:
\[
k\begin{pmatrix}
A_{i1}\\
A_{i2}\\
\vdots\\
A_{in}
\end{pmatrix},k\in\mathbb{K}.
\]
\end{proposition}
\begin{proof}
由条件和\hyperref[theorem:矩阵的秩与子式]{定理\ref{theorem:矩阵的秩与子式}}可知\(\boldsymbol{A}\)的秩等于\(n - 1\),因此线性方程组\(\boldsymbol{A}\boldsymbol{x}=\boldsymbol{0}\)的基础解系只含一个向量. 注意到\(|\boldsymbol{A}| = 0\),故\(\boldsymbol{A}\boldsymbol{A}^*=|\boldsymbol{A}|\boldsymbol{I}_n=\boldsymbol{O}\),于是伴随矩阵\(\boldsymbol{A}^*\)的任一列向量都是\(\boldsymbol{A}\boldsymbol{x}=\boldsymbol{0}\)的解. 又已知\(A_{ij}\neq0\),因此\(\boldsymbol{A}^*\)的第\(i\)个列向量\((A_{i1},A_{i2},\cdots,A_{in})'\)(不是零向量)是\(\boldsymbol{A}\boldsymbol{x}=\boldsymbol{0}\)的基础解系.   
\end{proof}

\begin{example}
设\(\boldsymbol{A}\)是\(n\)阶实反对称阵,\(\boldsymbol{D}=\mathrm{diag}\{d_1,d_2,\cdots,d_n\}\)是同阶对角阵且主对角元素全大于零,求证:\(|\boldsymbol{A}+\boldsymbol{D}|>0\). 特别地,\(|\boldsymbol{I}_n\pm\boldsymbol{A}|>0\),从而\(\boldsymbol{I}_n\pm\boldsymbol{A}\)都是非异阵.
\end{example}
\begin{note}
利用行列式构造连续的多项式函数,再利用函数连续的性质证明.
\end{note}
\begin{proof}
先证明\(|\boldsymbol{A}+\boldsymbol{D}|\neq0\),只需证明\((\boldsymbol{A}+\boldsymbol{D})\boldsymbol{x}=\boldsymbol{0}\)只有零解. 因为\(\boldsymbol{x}'(\boldsymbol{A}+\boldsymbol{D})\boldsymbol{x}=0\),转置可得\(\boldsymbol{x}'(-\boldsymbol{A}+\boldsymbol{D})\boldsymbol{x}=0\),上述两式相加即得\(\boldsymbol{x}'\boldsymbol{D}\boldsymbol{x}=0\). 若设\(\boldsymbol{x}=(x_1,x_2,\cdots,x_n)'\),则有\(d_1x_1^2 + d_2x_2^2+\cdots + d_nx_n^2 = 0\). 由于\(d_i\)都大于零并且\(x_i\)都是实数,故只能是\(x_1 = x_2=\cdots = x_n = 0\),即有\(\boldsymbol{x}=\boldsymbol{0}\).

再证明本题的结论. 设\(f(t)=|t\boldsymbol{A}+\boldsymbol{D}|\),则\(f(t)\)是关于\(t\)的多项式,从而是关于\(t\)的连续函数. 注意到对任意的实数\(t\),\(t\boldsymbol{A}\)仍是实反对称阵,故由上面的讨论可得\(f(t)=|t\boldsymbol{A}+\boldsymbol{D}|\neq0\),即\(f(t)\)是\(\mathbb{R}\)上处处不为零的连续函数. 注意到当\(t = 0\)时,\(f(0)=|\boldsymbol{D}|>0\),因此\(f(t)\)只能是\(\mathbb{R}\)上取值恒为正数的连续函数(原因见:\hyperref[Accumulation of mathematical techniques-proposition:连续函数无零点则一定恒大于零或恒小于零]{命题\ref{Accumulation of mathematical techniques-proposition:连续函数无零点则一定恒大于零或恒小于零}}).特别地,\(f(1)=|\boldsymbol{A}+\boldsymbol{D}|>0\). 
\end{proof}

\begin{definition}[严格对角占优阵]\label{definition:严格对角占优阵}
如果\(n\)阶实方阵\(\boldsymbol{A}=(a_{ij})\)适合条件:
\[
|a_{ii}|>\sum_{j = 1,j\neq i}^{n}|a_{ij}|,1\leq i\leq n,
\]
则称\(\boldsymbol{A}\)是\textbf{严格对角占优阵}.
\end{definition}

\begin{proposition}[严格对角占优阵必是非异阵]\label{proposition:严格对角占优阵必是非异阵}
如果\(n\)阶实方阵\(\boldsymbol{A}=(a_{ij})\)是严格对角占优阵,则\(\boldsymbol{A}\)必是非异阵.
\end{proposition}
\begin{proof}
只需证明线性方程组\(\boldsymbol{A}\boldsymbol{x}=\boldsymbol{0}\)只有零解. 若有非零解,设为\((c_1,c_2,\cdots,c_n)\),假设\(c_k\)是其中绝对值最大者. 将解代入该方程组的第\(k\)个方程式,得
\[
a_{k1}c_1+\cdots + a_{kk}c_k+\cdots + a_{kn}c_n = 0,
\]
即有
\[
-a_{kk}c_k=a_{k1}c_1+\cdots + a_{k,k - 1}c_{k - 1}+a_{k,k + 1}c_{k + 1}+\cdots + a_{kn}c_n.
\]
上式两边取绝对值,由三角不等式以及\(c_k\)是绝对值最大的假设可得
\begin{align*}
|a_{kk}||c_k|&\leq|a_{k1}||c_1|+\cdots + |a_{k,k - 1}||c_{k - 1}|+|a_{k,k + 1}||c_{k + 1}|+\cdots + |a_{kn}||c_n|\leq\left(\sum_{j = 1,j\neq k}^{n}|a_{kj}|\right)|c_k|,
\end{align*}
从而有
\[
|a_{kk}|\leq\sum_{j = 1,j\neq k}^{n}|a_{kj}|,
\]
得到矛盾. 因此,方程组\(\boldsymbol{A}\boldsymbol{x}=\boldsymbol{0}\)只有零解.
\end{proof}

\begin{proposition}\label{proposition:更严格对角占优阵行列式必大于零}
若\(n\)阶实方阵\(\boldsymbol{A}=(a_{ij})\)满足
\[
a_{ii}>\sum_{j = 1,j\neq i}^{n}|a_{ij}|,1\leq i\leq n,
\]
求证: \(|\boldsymbol{A}|>0\).
\end{proposition}
\begin{proof}
考虑矩阵\(t\boldsymbol{I}_n+\boldsymbol{A}\),当\(t\geq0\)时,这是一个严格对角占优阵,因此由\hyperref[proposition:严格对角占优阵必是非异阵]{上一个命题}可知其行列式\(f(t)=|t\boldsymbol{I}_n+\boldsymbol{A}|\)不为零. 又\(f(t)\)是关于\(t\)的多项式且首项系数为\(1\),所以当\(t\)充分大时,\(f(t)>0\). 注意到\(f(t)\)是\([0,+\infty)\)上处处不为零的连续函数,并且当\(t\)充分大时取值为正,因此\(f(t)\)在\([0,+\infty)\)上取值恒为正(原因见:\hyperref[Accumulation of mathematical techniques-proposition:连续函数无零点则一定恒大于零或恒小于零]{命题\ref{Accumulation of mathematical techniques-proposition:连续函数无零点则一定恒大于零或恒小于零}}). 特别地,\(f(0)=|\boldsymbol{A}|>0\).
\end{proof}

\begin{proposition}\label{proposition:实对称矩阵性质11}
设\(\boldsymbol{A}\)是\(n\)阶实对称阵,求证:\(\boldsymbol{I}_n + \mathrm{i}\boldsymbol{A}\)和\(\boldsymbol{I}_n - \mathrm{i}\boldsymbol{A}\)都是非异阵.
\end{proposition}
\begin{proof}
只需证明\((\boldsymbol{I}_n + \mathrm{i}\boldsymbol{A})\boldsymbol{x}=\boldsymbol{0}\)只有零解. 由\(\overline{\boldsymbol{x}}'(\boldsymbol{I}_n + \mathrm{i}\boldsymbol{A})\boldsymbol{x}=0\)共轭转置可得\(\overline{\boldsymbol{x}}'(\boldsymbol{I}_n - \mathrm{i}\boldsymbol{A})\boldsymbol{x}=0\). 上述两式相加,可得\(\overline{\boldsymbol{x}}'\boldsymbol{I}_n\boldsymbol{x}=0\),因此\(\boldsymbol{x}=\boldsymbol{0}\).
\end{proof}


\subsection{利用线性空间理论讨论矩阵的秩}
按照最初的定义,矩阵的秩就是矩阵的行(列)向量组的秩,因此通过线性空间理论去讨论矩阵的秩是十分自然的事情.

\begin{proposition}\label{proposition:矩阵的秩与子式及其加边子式的关系}
求证: 矩阵\(\boldsymbol{A}\)的秩等于\(r\)的充要条件是\(\boldsymbol{A}\)存在一个\(r\)阶子式\(|\boldsymbol{D}|\)不等于零,而\(|\boldsymbol{D}|\)的所有\(r + 1\)阶加边子式全等于零.
\end{proposition}
\begin{proof}
必要性由\hyperref[theorem:矩阵的秩与子式]{定理\ref{theorem:矩阵的秩与子式}}可直接得到,只需证明充分性. 不失一般性,我们可设\(|\boldsymbol{D}|\)是由\(\boldsymbol{A}\)的前\(r\)行和前\(r\)列构成的\(r\)阶子式. 设
\[
\boldsymbol{A}=\begin{pmatrix}
\boldsymbol{\alpha}_1\\
\boldsymbol{\alpha}_2\\
\vdots\\
\boldsymbol{\alpha}_m
\end{pmatrix}=(\boldsymbol{\beta}_1,\boldsymbol{\beta}_2,\cdots,\boldsymbol{\beta}_n)
\]
为矩阵\(\boldsymbol{A}\)的行分块和列分块,记\(\tau_{\leq r}\boldsymbol{\alpha}_i\)为行向量\(\boldsymbol{\alpha}_i\)关于前\(r\)列的缩短向量,\(\tau_{\leq r}\boldsymbol{\beta}_j\)为列向量\(\boldsymbol{\beta}_j\)关于前\(r\)行的缩短向量. 由\(|\boldsymbol{D}|\neq0\)可得\(\tau_{\leq r}\boldsymbol{\alpha}_1,\cdots,\tau_{\leq r}\boldsymbol{\alpha}_r\)线性无关,由\hyperref[proposition:线性相关向量组的缩短组也线性相关]{命题\ref{proposition:线性相关向量组的缩短组也线性相关}}可知\(\boldsymbol{\alpha}_1,\cdots,\boldsymbol{\alpha}_r\)线性无关. 

我们只要证明\(\boldsymbol{\alpha}_1,\cdots,\boldsymbol{\alpha}_r\)是\(\boldsymbol{A}\)的行向量的极大无关组即可得到\(\mathrm{r}(\boldsymbol{A}) = r\). 用反证法证明,若它们不是极大无关组,则可以添加一个行向量,不妨设为\(\boldsymbol{\alpha}_{r + 1}\),使得\(\boldsymbol{\alpha}_1,\cdots,\boldsymbol{\alpha}_r,\boldsymbol{\alpha}_{r + 1}\)线性无关. 设\(\boldsymbol{A}_1\)是\(\boldsymbol{A}\)的前\(r + 1\)行构成的矩阵,则\(\boldsymbol{A}_1 = (\tau_{\leq r + 1}\boldsymbol{\beta}_1,\tau_{\leq r + 1}\boldsymbol{\beta}_2,\cdots,\tau_{\leq r + 1}\boldsymbol{\beta}_n)\)且\(\mathrm{r}(\boldsymbol{A}_1)=r + 1\). 由\(|\boldsymbol{D}|\neq0\)可得\(\tau_{\leq r}\boldsymbol{\beta}_1,\cdots,\tau_{\leq r}\boldsymbol{\beta}_r\)线性无关,由\hyperref[proposition:线性相关向量组的缩短组也线性相关]{命题\ref{proposition:线性相关向量组的缩短组也线性相关}}可知\(\tau_{\leq r + 1}\boldsymbol{\beta}_1,\cdots,\tau_{\leq r + 1}\boldsymbol{\beta}_r\)线性无关. 因为\(\mathrm{r}(\boldsymbol{A}_1)=r + 1\),故存在\(\boldsymbol{A}_1\)的一个列向量,不妨设为\(\tau_{\leq r + 1}\boldsymbol{\beta}_{r + 1}\),使得\(\tau_{\leq r + 1}\boldsymbol{\beta}_1,\cdots,\tau_{\leq r + 1}\boldsymbol{\beta}_r,\tau_{\leq r + 1}\boldsymbol{\beta}_{r + 1}\)
线性无关. 设\(\boldsymbol{A}_2 = (\tau_{\leq r + 1}\boldsymbol{\beta}_1,\cdots,\tau_{\leq r + 1}\boldsymbol{\beta}_r,\tau_{\leq r + 1}\boldsymbol{\beta}_{r + 1})\),即\(\boldsymbol{A}_2\)是\(\boldsymbol{A}\)的前\(r + 1\)行和前\(r + 1\)列构成的方阵,则\(\mathrm{r}(\boldsymbol{A}_2)=r + 1\). 因此,\(|\boldsymbol{A}_2|\neq0\)是包含\(|\boldsymbol{D}|\)的\(r + 1\)阶加边子式,这与假设矛盾.
\end{proof}

\begin{proposition}\label{proposition:极大无关组对应的子式不为零}
设\(m\times n\)矩阵\(\boldsymbol{A}\)的\(m\)个行向量为\(\boldsymbol{\alpha}_1,\boldsymbol{\alpha}_2,\cdots,\boldsymbol{\alpha}_m\),且\(\boldsymbol{\alpha}_{i_1},\boldsymbol{\alpha}_{i_2},\cdots,\boldsymbol{\alpha}_{i_r}\)是其极大无关组,又设\(\boldsymbol{A}\)的\(n\)个列向量为\(\boldsymbol{\beta}_1,\boldsymbol{\beta}_2,\cdots,\boldsymbol{\beta}_n\),且\(\boldsymbol{\beta}_{j_1},\boldsymbol{\beta}_{j_2},\cdots,\boldsymbol{\beta}_{j_r}\)是其极大无关组. 证明:\(\boldsymbol{\alpha}_{i_1},\boldsymbol{\alpha}_{i_2},\cdots,\boldsymbol{\alpha}_{i_r}\)和\(\boldsymbol{\beta}_{j_1},\boldsymbol{\beta}_{j_2},\cdots,\boldsymbol{\beta}_{j_r}\)交叉点上的元素组成的子矩阵\(\boldsymbol{D}\)的行列式\(|\boldsymbol{D}|\neq0\).
\end{proposition}
\begin{proof}
因为\(\boldsymbol{\alpha}_{i_1},\boldsymbol{\alpha}_{i_2},\cdots,\boldsymbol{\alpha}_{i_r}\)是极大无关组,故\(\boldsymbol{A}\)的任一行向量\(\boldsymbol{\alpha}_s\)均可表示为\(\boldsymbol{\alpha}_{i_1},\boldsymbol{\alpha}_{i_2},\cdots,\boldsymbol{\alpha}_{i_r}\)的线性组合. 记\(\widetilde{\boldsymbol{\alpha}}_{i_1},\widetilde{\boldsymbol{\alpha}}_{i_2},\cdots,\widetilde{\boldsymbol{\alpha}}_{i_r},\widetilde{\boldsymbol{\alpha}}_s\)分别是\(\boldsymbol{\alpha}_{i_1},\boldsymbol{\alpha}_{i_2},\cdots,\boldsymbol{\alpha}_{i_r},\boldsymbol{\alpha}_s\)在\(j_1,j_2,\cdots,j_r\)列处的缩短向量,由\hyperref[proposition:线性相关向量组的缩短组也线性相关]{命题\ref{proposition:线性相关向量组的缩短组也线性相关}}可知,\(\widetilde{\boldsymbol{\alpha}}_s\)均可表示为\(\widetilde{\boldsymbol{\alpha}}_{i_1},\widetilde{\boldsymbol{\alpha}}_{i_2},\cdots,\widetilde{\boldsymbol{\alpha}}_{i_r}\)的线性组合.

考虑由列向量\(\boldsymbol{\beta}_{j_1},\boldsymbol{\beta}_{j_2},\cdots,\boldsymbol{\beta}_{j_r}\)组成的矩阵\(\boldsymbol{B}=(\boldsymbol{\beta}_{j_1},\boldsymbol{\beta}_{j_2},\cdots,\boldsymbol{\beta}_{j_r})\),这是一个\(m\times r\)矩阵且秩等于\(r\). 由于矩阵\(\boldsymbol{B}\)的任一行向量\(\widetilde{\boldsymbol{\alpha}}_s\)均可用\(\widetilde{\boldsymbol{\alpha}}_{i_1},\widetilde{\boldsymbol{\alpha}}_{i_2},\cdots,\widetilde{\boldsymbol{\alpha}}_{i_r}\)线性表示,并且\(\boldsymbol{B}\)的行秩等于\(r\),故由\hyperref[proposition:极大无关组的判定条件]{命题\ref{proposition:极大无关组的判定条件}}可知,\(\widetilde{\boldsymbol{\alpha}}_{i_1},\widetilde{\boldsymbol{\alpha}}_{i_2},\cdots,\widetilde{\boldsymbol{\alpha}}_{i_r}\)是\(\boldsymbol{B}\)的行向量的极大无关组,从而它们线性无关. 注意到\(r\)阶方阵\(\boldsymbol{D}\)的行向量恰好是\(\widetilde{\boldsymbol{\alpha}}_{i_1},\widetilde{\boldsymbol{\alpha}}_{i_2},\cdots,\widetilde{\boldsymbol{\alpha}}_{i_r}\),因此\(\boldsymbol{D}\)是满秩阵,从而\(|\boldsymbol{D}|\neq0\).
\end{proof}


\begin{proposition}\label{proposition:对称阵或反对称阵必有非零主子式}
设\(\boldsymbol{A}\)是一个\(n\)阶方阵,\(\boldsymbol{A}\)的第\(i_1,\cdots,i_r\)行和第\(i_1,\cdots,i_r\)列交叉点上的元素组成的子式称为\(\boldsymbol{A}\)的主子式. 若\(\boldsymbol{A}\)是对称阵或反对称阵且秩等于\(r\),求证:\(\boldsymbol{A}\)必有一个\(r\)阶主子式不等于零.
\end{proposition}
\begin{proof}
由对称性或反对称性,设\(\boldsymbol{A}\)的第\(i_1,\cdots,i_r\)行是\(\boldsymbol{A}\)的行向量的极大无关组,则由\hyperref[proposition:对称矩阵或反称矩阵的极大无关组]{命题\ref{proposition:对称矩阵或反称矩阵的极大无关组}}它的第\(i_1,\cdots,i_r\)列也是\(\boldsymbol{A}\)的列向量的极大无关组,因此由\hyperref[proposition:极大无关组对应的子式不为零]{命题\ref{proposition:极大无关组对应的子式不为零}}可知,它们交叉点上的元素组成的\(r\)阶主子式不等于零. 
\end{proof}

\begin{proposition}[反对称阵的秩必为偶数]\label{proposition:反对称阵的秩必为偶数}
证明:反对称阵的秩必为偶数.
\end{proposition}
\begin{proof}
用反证法,设反对称阵\(\boldsymbol{A}\)的秩等于\(2r + 1\),则由\hyperref[proposition:对称阵或反对称阵必有非零主子式]{命题\ref{proposition:对称阵或反对称阵必有非零主子式}}可知,\(\boldsymbol{A}\)有一个\(2r + 1\)阶主子式\(|\boldsymbol{D}|\)不等于零. 注意到反对称阵的主子式是反对称行列式,而由\hyperref[proposition:奇数阶反对称行列式的值等于零]{命题\ref{proposition:奇数阶反对称行列式的值等于零}}奇数阶反对称行列式的值等于零,从而\(|\boldsymbol{D}| = 0\),矛盾.
\end{proof}









\section{相抵标准型及其应用}

\begin{theorem}[矩阵的相抵标准型]\label{theorem:矩阵的相抵标准型}
对任意一个秩为\(r\)的\(m\times n\)矩阵\(\boldsymbol{A}\),总存在\(m\)阶非异阵\(\boldsymbol{P}\)和\(n\)阶非异阵\(\boldsymbol{Q}\),使得
\[
\boldsymbol{P}\boldsymbol{A}\boldsymbol{Q}=\begin{pmatrix}
\boldsymbol{I}_r&\boldsymbol{O}\\
\boldsymbol{O}&\boldsymbol{O}
\end{pmatrix}.
\]
\end{theorem}
\begin{proof}

\end{proof}

\begin{proposition}[矩阵的秩1分解]\label{proposition:矩阵的秩1分解}
求证: 秩等于\(r\)的矩阵可以表示为\(r\)个秩等于\(1\)的矩阵之和,但不能表示为少于\(r\)个秩为\(1\)的矩阵之和.
\end{proposition}
\begin{proof}
将\(\boldsymbol{A}\)化为相抵标准型,即存在非异矩阵\(\boldsymbol{P}\)及\(\boldsymbol{Q}\),使得
\begin{align*}
\boldsymbol{A}&=\boldsymbol{P}\left( \begin{matrix}
\boldsymbol{I}_r&		\boldsymbol{O}\\
\boldsymbol{O}&		\boldsymbol{O}\\
\end{matrix} \right) \boldsymbol{Q}=\boldsymbol{P}\left( \boldsymbol{E}_{11}+\boldsymbol{E}_{22}+\cdots +\boldsymbol{E}_{rr} \right) \boldsymbol{Q}
\\
&=\boldsymbol{PE}_{11}\boldsymbol{Q}+\boldsymbol{PE}_{22}\boldsymbol{Q}+\cdots +\boldsymbol{PE}_{rr}\boldsymbol{Q}.
\end{align*}
于是记$\boldsymbol{A}_1=\boldsymbol{PE}_{11}\boldsymbol{Q},\boldsymbol{A}_2=\boldsymbol{PE}_{22}\boldsymbol{Q},\cdots ,\boldsymbol{A}_r=\boldsymbol{PE}_{rr}\boldsymbol{Q}$,则每个$\boldsymbol{A}_i$的秩都等于1.故$\boldsymbol{A}$可以化为$r$个秩等于1的矩阵之和.

若\(\boldsymbol{A}=\boldsymbol{B}_1+\boldsymbol{B}_2+\cdots+\boldsymbol{B}_k,k < r\),且每个\(\boldsymbol{B}_i\)的秩都等于\(1\),则由\hyperref[矩阵秩的基本公式6]{命题\ref{proposition:矩阵秩的基本公式}\ref{矩阵秩的基本公式6}}可知\(\mathrm{r}(\boldsymbol{A})\leq\mathrm{r}(\boldsymbol{B}_1)+\mathrm{r}(\boldsymbol{B}_2)+\cdots+\mathrm{r}(\boldsymbol{B}_k)=k\),这与\(\mathrm{r}(\boldsymbol{A}) = r\)矛盾,故不可能.
\end{proof}

\begin{proposition}[对称矩阵的秩1分解]\label{proposition:对称矩阵的秩1分解}
秩等于$r$的对称矩阵可以表成$r$个秩等于1的对称矩阵之和.
\end{proposition}
\begin{proof}
设\(A\)是一个秩为\(r\)的对称矩阵,则存在一个可逆矩阵\(C\),使得
\[
C^TAC = \left( \begin{matrix}
I_r & \\
& O
\end{matrix} \right) = E_{11} + E_{22} + \cdots + E_{rr}.
\]
从而
\begin{align*}
A &= (C^T)^{-1}(E_{11} + E_{22} + \cdots + E_{rr})C^{-1}
\\
&=(C^T)^{-1}E_{11}C^{-1} + (C^T)^{-1}E_{22}C^{-1} + \cdots + (C^T)^{-1}E_{nn}C^{-1}.
\end{align*}
因为\(E_{ii}\)的秩为\(1\),且\((C^T)^{-1}\),\(C^{-1}\)均可逆,所以\((C^T)^{-1}E_{ii}C^{-1}\)的秩也为\(1\).又由于
\begin{align*}
((C^T)^{-1}E_{ii}C^{-1})^T = (C^{-1})^TE_{ii}^TC^{-1} = (C^T)^{-1}E_{ii}C^{-1}.
\end{align*}
因此\((C^T)^{-1}E_{ii}C^{-1}\)也是对称矩阵.故$A$可以表成$r$个秩等于1的对称矩阵之和.
\end{proof}

\begin{example}\label{example:3.251111}
设\(\boldsymbol{A},\boldsymbol{B},\boldsymbol{C}\)分别为\(m\times n,p\times q\)和\(m\times q\)矩阵,\(\boldsymbol{M}=\begin{pmatrix}
\boldsymbol{A}&\boldsymbol{C}\\
\boldsymbol{O}&\boldsymbol{B}
\end{pmatrix}\). 证明:\(\mathrm{r}(\boldsymbol{M})=\mathrm{r}(\boldsymbol{A})+\mathrm{r}(\boldsymbol{B})\)成立的充要条件是矩阵方程\(\boldsymbol{A}\boldsymbol{X}+\boldsymbol{Y}\boldsymbol{B}=\boldsymbol{C}\)有解,其中\(\boldsymbol{X},\boldsymbol{Y}\)分别是\(n\times q\)和\(m\times p\)未知矩阵.
\end{example}
\begin{note}
证明必要性时不妨设的原因:假设当\(\boldsymbol{A}=\begin{pmatrix}
\boldsymbol{I}_r & \boldsymbol{O} \\
\boldsymbol{O} & \boldsymbol{O}
\end{pmatrix}\),\(\boldsymbol{B}=\begin{pmatrix}
\boldsymbol{I}_s & \boldsymbol{O} \\
\boldsymbol{O} & \boldsymbol{O}
\end{pmatrix}\)时,结论成立.则当\(\boldsymbol{A}\neq\begin{pmatrix}
\boldsymbol{I}_r & \boldsymbol{O} \\
\boldsymbol{O} & \boldsymbol{O}
\end{pmatrix}\),\(\boldsymbol{B}\neq\begin{pmatrix}
\boldsymbol{I}_s & \boldsymbol{O} \\
\boldsymbol{O} & \boldsymbol{O}
\end{pmatrix}\)时,记\(\boldsymbol{A}_1 = \boldsymbol{P}_1\boldsymbol{AQ}_1=\begin{pmatrix}
\boldsymbol{I}_r & \boldsymbol{O} \\
\boldsymbol{O} & \boldsymbol{O}
\end{pmatrix}\),\(\boldsymbol{B}_1 = \boldsymbol{P}_2\boldsymbol{BQ}_2=\begin{pmatrix}
\boldsymbol{I}_{s} & \boldsymbol{O} \\
\boldsymbol{O} & \boldsymbol{O}
\end{pmatrix}\),\(\boldsymbol{C}_1 = \boldsymbol{P}_1\boldsymbol{CQ}_2\),\(\boldsymbol{M}_1=\begin{pmatrix}
\boldsymbol{A}_1 & \boldsymbol{C}_1 \\
\boldsymbol{O} & \boldsymbol{B}_1
\end{pmatrix}\).

由于矩阵乘可逆矩阵不改变其秩,因此
\begin{align*}
\mathrm{r}(\boldsymbol{A})&=\mathrm{r}(\boldsymbol{P}_1\boldsymbol{AQ}_1)=\mathrm{r}\begin{pmatrix}
\boldsymbol{I}_r & \boldsymbol{O} \\
\boldsymbol{O} & \boldsymbol{O}
\end{pmatrix}=\mathrm{r}(\boldsymbol{A}_1),\\
\mathrm{r}(\boldsymbol{B})&=\mathrm{r}(\boldsymbol{P}_2\boldsymbol{BQ}_2)=\mathrm{r}\begin{pmatrix}
\boldsymbol{I}_s & \boldsymbol{O} \\
\boldsymbol{O} & \boldsymbol{O}
\end{pmatrix}=\mathrm{r}(\boldsymbol{B}_1),\\
\mathrm{r}(\boldsymbol{M})&=\mathrm{r}\begin{pmatrix}
\boldsymbol{A} & \boldsymbol{C} \\
\boldsymbol{O} & \boldsymbol{B}
\end{pmatrix}=\mathrm{r}\left(\begin{pmatrix}
\boldsymbol{P}_1 & \boldsymbol{O} \\
\boldsymbol{O} & \boldsymbol{P}_2
\end{pmatrix}\begin{pmatrix}
\boldsymbol{A} & \boldsymbol{C} \\
\boldsymbol{O} & \boldsymbol{B}
\end{pmatrix}\begin{pmatrix}
\boldsymbol{Q}_1 & \boldsymbol{O} \\
\boldsymbol{O} & \boldsymbol{Q}_2
\end{pmatrix}\right)
=\mathrm{r}\begin{pmatrix}
\boldsymbol{P}_1\boldsymbol{AQ}_1 & \boldsymbol{P}_1\boldsymbol{CQ}_2 \\
\boldsymbol{O} & \boldsymbol{P}_2\boldsymbol{BQ}_2
\end{pmatrix}=\mathrm{r}(\boldsymbol{M}_1).
\end{align*}
从而
\[
\mathrm{r}(\boldsymbol{M})=\mathrm{r}(\boldsymbol{A})+\mathrm{r}(\boldsymbol{B})\Leftrightarrow\mathrm{r}(\boldsymbol{M}_1)=\mathrm{r}\begin{pmatrix}
\boldsymbol{A}_1 & \boldsymbol{C}_1 \\
\boldsymbol{O} & \boldsymbol{B}_1
\end{pmatrix}=\mathrm{r}(\boldsymbol{A}_1)+\mathrm{r}(\boldsymbol{B}_1).
\]
于是由假设可知\(\boldsymbol{A}_1\boldsymbol{X}_1+\boldsymbol{Y}_1\boldsymbol{B}_1=\boldsymbol{C}_1\)有解\(\boldsymbol{X}_1,\boldsymbol{Y}_1\).记\(\boldsymbol{X}=\boldsymbol{Q}_1\boldsymbol{X}_1\boldsymbol{Q}_{2}^{-1}\),\(\boldsymbol{Y}=\boldsymbol{P}_{1}^{-1}\boldsymbol{Y}_1\boldsymbol{P}_2\),则
\begin{align*}
&\boldsymbol{A}_1\boldsymbol{X}_1+\boldsymbol{Y}_1\boldsymbol{B}_1=\boldsymbol{C}_1\text{有解}\boldsymbol{X}_1,\boldsymbol{Y}_1\\
\Leftrightarrow&\boldsymbol{P}_1\boldsymbol{AQ}_1\boldsymbol{X}_1+\boldsymbol{Y}_1\boldsymbol{P}_2\boldsymbol{BQ}_2=\boldsymbol{P}_1\boldsymbol{CQ}_2\text{有解}\boldsymbol{X}_1,\boldsymbol{Y}_1\\
\Leftrightarrow&\boldsymbol{AQ}_1\boldsymbol{X}_1\boldsymbol{Q}_{2}^{-1}+\boldsymbol{P}_{1}^{-1}\boldsymbol{Y}_1\boldsymbol{P}_2\boldsymbol{B}=\boldsymbol{C}\text{有解}\boldsymbol{X}_1,\boldsymbol{Y}_1\\
\Leftrightarrow&\boldsymbol{AX}+\boldsymbol{YB}=\boldsymbol{C}\text{有解}\boldsymbol{X},\boldsymbol{Y}
\end{align*}
故可以不妨设\(\boldsymbol{A}=\begin{pmatrix}
\boldsymbol{I}_r & \boldsymbol{O} \\
\boldsymbol{O} & \boldsymbol{O}
\end{pmatrix}\),\(\boldsymbol{B}=\begin{pmatrix}
\boldsymbol{I}_s & \boldsymbol{O} \\
\boldsymbol{O} & \boldsymbol{O}
\end{pmatrix}\).
\end{note}
\begin{proof}
先证充分性. 设\(\boldsymbol{X}=\boldsymbol{X}_0,\boldsymbol{Y}=\boldsymbol{Y}_0\)是矩阵方程\(\boldsymbol{A}\boldsymbol{X}+\boldsymbol{Y}\boldsymbol{B}=\boldsymbol{C}\)的解,则将\(\boldsymbol{M}\)的第一分块列右乘\(-\boldsymbol{X}_0\)加到第二分块列上,再将第二分块行左乘\(-\boldsymbol{Y}_0\)加到第一分块行上,可得分块对角阵\(\begin{pmatrix}
\boldsymbol{A}&\boldsymbol{O}\\
\boldsymbol{O}&\boldsymbol{B}
\end{pmatrix}\),于是\(\mathrm{r}(\boldsymbol{M})=\mathrm{r}\begin{pmatrix}
\boldsymbol{A}&\boldsymbol{O}\\
\boldsymbol{O}&\boldsymbol{B}
\end{pmatrix}=\mathrm{r}(\boldsymbol{A})+\mathrm{r}(\boldsymbol{B})\).

再证必要性. 设\(\boldsymbol{P}_1\boldsymbol{A}\boldsymbol{Q}_1=\begin{pmatrix}
\boldsymbol{I}_r&\boldsymbol{O}\\
\boldsymbol{O}&\boldsymbol{O}
\end{pmatrix},\boldsymbol{P}_2\boldsymbol{B}\boldsymbol{Q}_2=\begin{pmatrix}
\boldsymbol{I}_s&\boldsymbol{O}\\
\boldsymbol{O}&\boldsymbol{O}
\end{pmatrix}\),其中\(\boldsymbol{P}_1,\boldsymbol{Q}_1,\boldsymbol{P}_2,\boldsymbol{Q}_2\)为非异阵,\(r = \mathrm{r}(\boldsymbol{A}),s=\mathrm{r}(\boldsymbol{B})\). 注意到问题的条件和结论在相抵变换:
\(\boldsymbol{A}\mapsto\boldsymbol{P}_1\boldsymbol{A}\boldsymbol{Q}_1,\boldsymbol{B}\mapsto\boldsymbol{P}_2\boldsymbol{B}\boldsymbol{Q}_2,\boldsymbol{C}\mapsto\boldsymbol{P}_1\boldsymbol{C}\boldsymbol{Q}_2,\boldsymbol{X}\mapsto\boldsymbol{Q}_1^{-1}\boldsymbol{X}\boldsymbol{Q}_2,\boldsymbol{Y}\mapsto\boldsymbol{P}_1\boldsymbol{Y}\boldsymbol{P}_2^{-1}\)
下保持不变,故不妨从一开始就假设\(\boldsymbol{A}=\begin{pmatrix}
\boldsymbol{I}_r&\boldsymbol{O}\\
\boldsymbol{O}&\boldsymbol{O}
\end{pmatrix},\boldsymbol{B}=\begin{pmatrix}
\boldsymbol{I}_s&\boldsymbol{O}\\
\boldsymbol{O}&\boldsymbol{O}
\end{pmatrix}\)都是相抵标准型. 设\(\boldsymbol{C}=\begin{pmatrix}
\boldsymbol{C}_1&\boldsymbol{C}_2\\
\boldsymbol{C}_3&\boldsymbol{C}_4
\end{pmatrix},\boldsymbol{X}=\begin{pmatrix}
\boldsymbol{X}_1&\boldsymbol{X}_2\\
\boldsymbol{X}_3&\boldsymbol{X}_4
\end{pmatrix},\boldsymbol{Y}=\begin{pmatrix}
\boldsymbol{Y}_1&\boldsymbol{Y}_2\\
\boldsymbol{Y}_3&\boldsymbol{Y}_4
\end{pmatrix}\)为对应的分块. 考虑\(\boldsymbol{M}\)的如下分块初等变换:
\[
\boldsymbol{M}=\begin{pmatrix}
\boldsymbol{I}_r&\boldsymbol{O}&\boldsymbol{C}_1&\boldsymbol{C}_2\\
\boldsymbol{O}&\boldsymbol{O}&\boldsymbol{C}_3&\boldsymbol{C}_4\\
\boldsymbol{O}&\boldsymbol{O}&\boldsymbol{I}_s&\boldsymbol{O}\\
\boldsymbol{O}&\boldsymbol{O}&\boldsymbol{O}&\boldsymbol{O}
\end{pmatrix}\to\begin{pmatrix}
\boldsymbol{I}_r&\boldsymbol{O}&\boldsymbol{O}&\boldsymbol{O}\\
\boldsymbol{O}&\boldsymbol{O}&\boldsymbol{O}&\boldsymbol{C}_4\\
\boldsymbol{O}&\boldsymbol{O}&\boldsymbol{I}_s&\boldsymbol{O}\\
\boldsymbol{O}&\boldsymbol{O}&\boldsymbol{O}&\boldsymbol{O}
\end{pmatrix},
\]
由于\(\mathrm{r}(\boldsymbol{M})=\mathrm{r}(\boldsymbol{A})+\mathrm{r}(\boldsymbol{B})=r + s\),故\(\boldsymbol{C}_4=\boldsymbol{O}\). 于是矩阵方程\(\boldsymbol{A}\boldsymbol{X}+\boldsymbol{Y}\boldsymbol{B}=\boldsymbol{C}\),即
\[
\begin{pmatrix}
\boldsymbol{X}_1&\boldsymbol{X}_2\\
\boldsymbol{O}&\boldsymbol{O}
\end{pmatrix}+\begin{pmatrix}
\boldsymbol{Y}_1&\boldsymbol{O}\\
\boldsymbol{Y}_3&\boldsymbol{O}
\end{pmatrix}=\begin{pmatrix}
\boldsymbol{X}_1+\boldsymbol{Y}_1&\boldsymbol{X}_2\\
\boldsymbol{Y}_3&\boldsymbol{O}
\end{pmatrix}=\begin{pmatrix}
\boldsymbol{C}_1&\boldsymbol{C}_2\\
\boldsymbol{C}_3&\boldsymbol{O}
\end{pmatrix}
\]
有解,例如\(\boldsymbol{X}_1=\boldsymbol{C}_1,\boldsymbol{X}_2=\boldsymbol{C}_2,\boldsymbol{Y}_1=\boldsymbol{O},\boldsymbol{Y}_3=\boldsymbol{C}_3\),其余分块取法任意.
\end{proof}

\begin{proposition}[行/列满秩矩阵性质]\label{proposition:行/列满秩矩阵性质}
由\hyperref[theorem:相抵标准型]{矩阵的相抵标准型}可设\(\boldsymbol{A}\)是\(m\times n\)矩阵,则
\begin{enumerate}[(1)]
\item 若\(\mathrm{r}(\boldsymbol{A}) = n\),即\(\boldsymbol{A}\)是列满秩阵,则必存在秩等于\(n\)的\(n\times m\)矩阵\(\boldsymbol{B}\)(行满秩),使得\(\boldsymbol{B}\boldsymbol{A}=\boldsymbol{I}_n\)(这样的矩阵\(\boldsymbol{B}\)称为\(\boldsymbol{A}\)的左逆);
\item 若\(\mathrm{r}(\boldsymbol{A}) = m\),即\(\boldsymbol{A}\)是行满秩阵,则必存在秩等于\(m\)的\(n\times m\)矩阵\(\boldsymbol{C}\)(列满秩),使得\(\boldsymbol{A}\boldsymbol{C}=\boldsymbol{I}_m\)(这样的矩阵\(\boldsymbol{C}\)称为\(\boldsymbol{A}\)的右逆).
\end{enumerate}
\end{proposition}
\begin{proof}
\begin{enumerate}[(1)]
\item 设\(\boldsymbol{P}\)为\(m\)阶非异阵,\(\boldsymbol{Q}\)为\(n\)阶非异阵,使得
\[
\boldsymbol{P}\boldsymbol{A}\boldsymbol{Q}=\begin{pmatrix}
\boldsymbol{I}_n\\
\boldsymbol{O}
\end{pmatrix},
\]
因此\((\boldsymbol{I}_n,\boldsymbol{O})\boldsymbol{P}\boldsymbol{A}\boldsymbol{Q}=\boldsymbol{I}_n\),即\((\boldsymbol{I}_n,\boldsymbol{O})\boldsymbol{P}\boldsymbol{A}=\boldsymbol{Q}^{-1}\),于是\(\boldsymbol{Q}(\boldsymbol{I}_n,\boldsymbol{O})\boldsymbol{P}\boldsymbol{A}=\boldsymbol{I}_n\). 令\(\boldsymbol{B}=\boldsymbol{Q}(\boldsymbol{I}_n,\boldsymbol{O})\boldsymbol{P}\)即可.

\item 同理可证,或者考虑\(\boldsymbol{A}'\)并利用(1)的结论.
\end{enumerate}
\end{proof}

\begin{corollary}\label{corollary:列满秩矩阵适合左消去律/行满秩矩阵适合右消去律}
列满秩矩阵适合左消去律,即若\(\boldsymbol{A}\)列满秩且\(\boldsymbol{A}\boldsymbol{D}=\boldsymbol{A}\boldsymbol{E}\),则\(\boldsymbol{D}=\boldsymbol{E}\). 同理,行满秩矩阵适合右消去律,即若\(\boldsymbol{A}\)行满秩且\(\boldsymbol{D}\boldsymbol{A}=\boldsymbol{E}\boldsymbol{A}\),则\(\boldsymbol{D}=\boldsymbol{E}\).
\end{corollary}

\begin{proposition}[满秩分解]\label{proposition:矩阵的满秩分解}
设\(m\times n\)矩阵\(\boldsymbol{A}\)的秩为\(r\),证明:
\begin{enumerate}[(1)]
\item \(\boldsymbol{A}=\boldsymbol{B}\boldsymbol{C}\),其中\(\boldsymbol{B}\)是\(m\times r\)(列满秩)矩阵且\(\mathrm{r}(\boldsymbol{B}) = r\),\(\boldsymbol{C}\)是\(r\times n\)(行满秩)矩阵且\(\mathrm{r}(\boldsymbol{C}) = r\),这种分解称为\(\boldsymbol{A}\)的满秩分解;
\item 若\(\boldsymbol{A}\)有两个满秩分解\(\boldsymbol{A}=\boldsymbol{B}_1\boldsymbol{C}_1=\boldsymbol{B}_2\boldsymbol{C}_2\),则存在\(r\)阶非异阵\(\boldsymbol{P}\),使得\(\boldsymbol{B}_2=\boldsymbol{B}_1\boldsymbol{P}\),\(\boldsymbol{C}_2=\boldsymbol{P}^{-1}\boldsymbol{C}_1\).
\end{enumerate}
\end{proposition}
\begin{proof}
\begin{enumerate}[(1)]
\item 设\(\boldsymbol{P}\)为\(m\)阶非异阵,\(\boldsymbol{Q}\)为\(n\)阶非异阵,使得
\[
\boldsymbol{A}=\boldsymbol{P}\begin{pmatrix}
\boldsymbol{I}_r&\boldsymbol{O}\\
\boldsymbol{O}&\boldsymbol{O}
\end{pmatrix}\boldsymbol{Q}=\boldsymbol{P}\begin{pmatrix}
\boldsymbol{I}_r\\
\boldsymbol{O}
\end{pmatrix}(\boldsymbol{I}_r,\boldsymbol{O})\boldsymbol{Q}.
\]
令\(\boldsymbol{B}=\boldsymbol{P}\begin{pmatrix}
\boldsymbol{I}_r\\
\boldsymbol{O}
\end{pmatrix}\),\(\boldsymbol{C}=(\boldsymbol{I}_r,\boldsymbol{O})\boldsymbol{Q}\),即得结论.

\item  由\hyperref[proposition:行/列满秩矩阵性质]{行/列满秩矩阵性质}可知,存在\(r\times m\)行满秩阵\(\boldsymbol{S}_2\),\(n\times r\)列满秩阵\(\boldsymbol{T}_2\),使得\(\boldsymbol{S}_2\boldsymbol{B}_2=\boldsymbol{I}_r\),\(\boldsymbol{C}_2\boldsymbol{T}_2=\boldsymbol{I}_r\),于是
\begin{align*}
\boldsymbol{B}_2&=\boldsymbol{B}_2(\boldsymbol{C}_2\boldsymbol{T}_2)=(\boldsymbol{B}_2\boldsymbol{C}_2)\boldsymbol{T}_2=(\boldsymbol{B}_1\boldsymbol{C}_1)\boldsymbol{T}_2=\boldsymbol{B}_1(\boldsymbol{C}_1\boldsymbol{T}_2),\\
\boldsymbol{C}_2&=(\boldsymbol{S}_2\boldsymbol{B}_2)\boldsymbol{C}_2=\boldsymbol{S}_2(\boldsymbol{B}_2\boldsymbol{C}_2)=\boldsymbol{S}_2(\boldsymbol{B}_1\boldsymbol{C}_1)=(\boldsymbol{S}_2\boldsymbol{B}_1)\boldsymbol{C}_1,\\
(\boldsymbol{S}_2\boldsymbol{B}_1)(\boldsymbol{C}_1\boldsymbol{T}_2)&=\boldsymbol{S}_2(\boldsymbol{B}_1\boldsymbol{C}_1)\boldsymbol{T}_2=\boldsymbol{S}_2(\boldsymbol{B}_2\boldsymbol{C}_2)\boldsymbol{T}_2=(\boldsymbol{S}_2\boldsymbol{B}_2)(\boldsymbol{C}_2\boldsymbol{T}_2)=\boldsymbol{I}_r.
\end{align*}
令\(\boldsymbol{P}=\boldsymbol{C}_1\boldsymbol{T}_2\),即得结论.
\end{enumerate}
\end{proof}

\begin{proposition}\label{proposition:几何观点下的矩阵满秩分解}
\(\boldsymbol{A}=\boldsymbol{B}\boldsymbol{C}\)是满秩分解当且仅当\(\boldsymbol{B}\)的\(r\)个列向量是\(\boldsymbol{A}\)的\(n\)个列向量张成线性空间的一组基,也当且仅当\(\boldsymbol{C}\)的\(r\)个行向量是\(\boldsymbol{A}\)的\(m\)个行向量张成线性空间的一组基.
\end{proposition}

\begin{proof}

\end{proof}

\begin{example}\label{example:3.26111}
设\(\boldsymbol{A}\)为\(m\times n\)矩阵,证明:存在\(n\times m\)矩阵\(\boldsymbol{B}\),使得\(\boldsymbol{A}\boldsymbol{B}\boldsymbol{A}=\boldsymbol{A}\).
\end{example}
\begin{note}
{\color{blue}证法一}的不妨设原因与\hyperref[example:3.251111]{例题\ref{example:3.251111}}类似.
\end{note}
\begin{proof}
{\color{blue}证法一:}
设\(\boldsymbol{P}\boldsymbol{A}\boldsymbol{Q}=\begin{pmatrix}
\boldsymbol{I}_r&\boldsymbol{O}\\
\boldsymbol{O}&\boldsymbol{O}
\end{pmatrix}\),其中\(\boldsymbol{P}\)是\(m\)阶非异阵,\(\boldsymbol{Q}\)是\(n\)阶非异阵. 注意到问题的条件和结论在相抵变换:\(\boldsymbol{A}\mapsto\boldsymbol{P}\boldsymbol{A}\boldsymbol{Q},\boldsymbol{B}\mapsto\boldsymbol{Q}^{-1}\boldsymbol{B}\boldsymbol{P}^{-1}\)下保持不变,故不妨从一开始就假设\(\boldsymbol{A}=\begin{pmatrix}
\boldsymbol{I}_r&\boldsymbol{O}\\
\boldsymbol{O}&\boldsymbol{O}
\end{pmatrix}\)是相抵标准型. 设\(\boldsymbol{B}=\begin{pmatrix}
\boldsymbol{B}_1&\boldsymbol{B}_2\\
\boldsymbol{B}_3&\boldsymbol{B}_4
\end{pmatrix}\)为对应的分块,由\(\boldsymbol{A}\boldsymbol{B}\boldsymbol{A}=\boldsymbol{A}\)可得\(\boldsymbol{B}_1=\boldsymbol{I}_r\),其余分块取法任意.

{\color{blue}证法二:}设\(\boldsymbol{A}=\boldsymbol{C}\boldsymbol{D}\)为\(\boldsymbol{A}\)的满秩分解,\(\boldsymbol{E}\)为列满秩阵\(\boldsymbol{C}\)的左逆,\(\boldsymbol{F}\)是行满秩阵\(\boldsymbol{D}\)的右逆. 令\(\boldsymbol{B}=\boldsymbol{F}\boldsymbol{E}\),则
\[
\boldsymbol{A}\boldsymbol{B}\boldsymbol{A}=(\boldsymbol{C}\boldsymbol{D})(\boldsymbol{F}\boldsymbol{E})(\boldsymbol{C}\boldsymbol{D})=\boldsymbol{C}(\boldsymbol{D}\boldsymbol{F})(\boldsymbol{E}\boldsymbol{C})\boldsymbol{D}=\boldsymbol{C}\boldsymbol{D}=\boldsymbol{A}.
\]
\end{proof}

\begin{example}
设\(\boldsymbol{A},\boldsymbol{B}\)分别是\(3\times2,2\times3\)矩阵且满足
\[
\boldsymbol{A}\boldsymbol{B}=\begin{pmatrix}
8&2&-2\\
2&5&4\\
-2&4&5
\end{pmatrix},
\]
试求\(\boldsymbol{B}\boldsymbol{A}\).
\end{example}
\begin{proof}
{\color{blue}解法一:}
通过简单的计算可得\(\mathrm{r}(\boldsymbol{A}\boldsymbol{B}) = 2\),从而\(\mathrm{r}(\boldsymbol{A})\geq2,\mathrm{r}(\boldsymbol{B})\geq2\). 又因为矩阵的秩不超过行数和列数的最小值,故\(\mathrm{r}(\boldsymbol{A})=\mathrm{r}(\boldsymbol{B}) = 2\),即\(\boldsymbol{A}\)是列满秩阵,\(\boldsymbol{B}\)是行满秩阵. 又注意到\((\boldsymbol{A}\boldsymbol{B})^2 = 9\boldsymbol{A}\boldsymbol{B}\),经整理可得\(\boldsymbol{A}(\boldsymbol{B}\boldsymbol{A}-9\boldsymbol{I}_2)\boldsymbol{B}=\boldsymbol{O}\). 根据\hyperref[corollary:列满秩矩阵适合左消去律/行满秩矩阵适合右消去律]{推论\ref{corollary:列满秩矩阵适合左消去律/行满秩矩阵适合右消去律}},可以在上式的左边消去\(\boldsymbol{A}\),右边消去\(\boldsymbol{B}\),从而可得\(\boldsymbol{B}\boldsymbol{A}=9\boldsymbol{I}_2\).

{\color{blue}解法二:}由{\color{blue}解法一}中矩阵秩的计算可知,\(\boldsymbol{A}\boldsymbol{B}\)是题中\(3\)阶矩阵\(\boldsymbol{C}\)的满秩分解. 注意到\(\boldsymbol{C}\)的后两列线性无关,因此可取另一种满秩分解为
\[
\boldsymbol{C}=\begin{pmatrix}
2&-2\\
5&4\\
4&5
\end{pmatrix}\begin{pmatrix}
2&1&0\\
-2&0&1
\end{pmatrix}=\boldsymbol{A}_1\boldsymbol{B}_1.
\]
由\hyperref[proposition:矩阵的满秩分解]{矩阵的满秩分解(2)}可知,存在可逆矩阵$\boldsymbol{P}$,使得$\boldsymbol{A}_1=\boldsymbol{AP},\boldsymbol{B}_1=\boldsymbol{P}^{-1}\boldsymbol{B}$.于是$\boldsymbol{B}_1\boldsymbol{A}_1=\boldsymbol{P}^{-1}\boldsymbol{BAP}$,故
\(\boldsymbol{B}\boldsymbol{A}\)相似于\(\boldsymbol{B}_1\boldsymbol{A}_1 = 9\boldsymbol{I}_2\),从而\(\boldsymbol{B}\boldsymbol{A}=\boldsymbol{P}^{-1}(9\boldsymbol{I}_2)\boldsymbol{P}=9\boldsymbol{I}_2\). 
\end{proof}

\begin{proposition}[幂等矩阵关于满秩分解的刻画]\label{proposition:幂等矩阵关于满秩分解的刻画}
设\(\boldsymbol{A}\)是\(n\)阶方阵且\(\mathrm{r}(\boldsymbol{A}) = r\),求证:\(\boldsymbol{A}^2=\boldsymbol{A}\)的充要条件是存在秩等于\(r\)的\(n\times r\)矩阵\(\boldsymbol{S}\)和秩等于\(r\)的\(r\times n\)矩阵\(\boldsymbol{T}\),使得\(\boldsymbol{A}=\boldsymbol{S}\boldsymbol{T},\boldsymbol{T}\boldsymbol{S}=\boldsymbol{I}_r\).
\end{proposition}
\begin{proof}
充分性显然,现证必要性. 设\(\boldsymbol{P},\boldsymbol{Q}\)为\(n\)阶非异阵,使得
\[
\boldsymbol{A}=\boldsymbol{P}\begin{pmatrix}
\boldsymbol{I}_r&\boldsymbol{O}\\
\boldsymbol{O}&\boldsymbol{O}
\end{pmatrix}\boldsymbol{Q}.
\]
代入\(\boldsymbol{A}^2=\boldsymbol{A}\)消去两侧的非异阵\(\boldsymbol{P}\)和\(\boldsymbol{Q}\),可得
\[
\begin{pmatrix}
\boldsymbol{I}_r&\boldsymbol{O}\\
\boldsymbol{O}&\boldsymbol{O}
\end{pmatrix}=\begin{pmatrix}
\boldsymbol{I}_r&\boldsymbol{O}\\
\boldsymbol{O}&\boldsymbol{O}
\end{pmatrix}\boldsymbol{Q}\boldsymbol{P}\begin{pmatrix}
\boldsymbol{I}_r&\boldsymbol{O}\\
\boldsymbol{O}&\boldsymbol{O}
\end{pmatrix}.
\]
只需令
\[
\boldsymbol{S}=\boldsymbol{P}\begin{pmatrix}
\boldsymbol{I}_r&\boldsymbol{O}\\
\boldsymbol{O}&\boldsymbol{O}
\end{pmatrix}\begin{pmatrix}
\boldsymbol{I}_r\\
\boldsymbol{O}
\end{pmatrix},\boldsymbol{T}=(\boldsymbol{I}_r,\boldsymbol{O})\begin{pmatrix}
\boldsymbol{I}_r&\boldsymbol{O}\\
\boldsymbol{O}&\boldsymbol{O}
\end{pmatrix}\boldsymbol{Q},
\]
则$\boldsymbol{S}$列满秩,$\boldsymbol{I}$行满秩,经简单计算即得结论.
\end{proof}

\begin{corollary}[幂等矩阵的迹和秩相等]\label{corollary:幂等矩阵的迹和秩相等}
设\(\boldsymbol{A}\)为\(n\)阶幂等矩阵,则\(\mathrm{tr}(\boldsymbol{A})=\mathrm{r}(\boldsymbol{A})\).
\end{corollary}
\begin{proof}
{\color{blue}证法一:}由\hyperref[proposition:幂等矩阵关于满秩分解的刻画]{命题\ref{proposition:幂等矩阵关于满秩分解的刻画}}可知,\(\mathrm{tr}(\boldsymbol{A})=\mathrm{tr}(\boldsymbol{S}\boldsymbol{T})=\mathrm{tr}(\boldsymbol{T}\boldsymbol{S})=\mathrm{tr}(\boldsymbol{I}_r)=r=\mathrm{r}(\boldsymbol{A})\).

{\color{blue}证法二(相似标准型):}事实上,由\(\boldsymbol{A}^2=\boldsymbol{A}\)可知,存在可逆矩阵\(\boldsymbol{P}\),使得
\[
\boldsymbol{A}=\boldsymbol{P}\begin{pmatrix}
\boldsymbol{I}_r&\boldsymbol{O}\\
\boldsymbol{O}&\boldsymbol{O}
\end{pmatrix}\boldsymbol{P}^{-1}=\boldsymbol{P}\begin{pmatrix}
\boldsymbol{I}_r\\
\boldsymbol{O}
\end{pmatrix}(\boldsymbol{I}_r,\boldsymbol{O})\boldsymbol{P}^{-1},
\]
令\(\boldsymbol{S}=\boldsymbol{P}\begin{pmatrix}
\boldsymbol{I}_r\\
\boldsymbol{O}
\end{pmatrix},\boldsymbol{T}=(\boldsymbol{I}_r,\boldsymbol{O})\boldsymbol{P}^{-1}\),可得\(\mathrm{tr}(\boldsymbol{A})=\mathrm{tr}(\boldsymbol{S}\boldsymbol{T})=\mathrm{tr}(\boldsymbol{T}\boldsymbol{S})=\mathrm{tr}(\boldsymbol{I}_r)=r=\mathrm{r}(\boldsymbol{A})\).
\end{proof}

\section{线性方程组的解及其应用}

\subsection{线性方程组的解的讨论}

\begin{proposition}\label{proposition:线性方程组同解系数矩阵秩相同}
线性方程组$\boldsymbol{Ax}=\mathbf{0}$与$\boldsymbol{Bx}=\mathbf{0}$同解当且仅当$r\left( \boldsymbol{A} \right) =\mathrm{r}\left( \boldsymbol{B} \right)$ .
\end{proposition}
\begin{proof}

\end{proof}

\begin{example}
设\(\boldsymbol{A}\)是一个\(m\times n\)矩阵,记\(\boldsymbol{\alpha}_i\)是\(\boldsymbol{A}\)的第\(i\)个行向量,\(\boldsymbol{\beta}=(b_1,b_2,\cdots,b_n)\). 求证:若齐次线性方程组\(\boldsymbol{A}\boldsymbol{x}=\boldsymbol{0}\)的解全是方程\(b_1x_1 + b_2x_2+\cdots + b_nx_n = 0\)的解,则\(\boldsymbol{\beta}\)是\(\boldsymbol{\alpha}_1,\boldsymbol{\alpha}_2,\cdots,\boldsymbol{\alpha}_m\)的线性组合.
\end{example}
\begin{proof}
令\(\boldsymbol{B}=\begin{pmatrix}
\boldsymbol{A}\\
\boldsymbol{\beta}
\end{pmatrix}\),由已知,方程组\(\boldsymbol{A}\boldsymbol{x}=\boldsymbol{0}\)和方程组\(\boldsymbol{B}\boldsymbol{x}=\boldsymbol{0}\)同解,故\(\mathrm{r}(\boldsymbol{A})=\mathrm{r}(\boldsymbol{B})\),从而\(\boldsymbol{A}\)的行向量的极大无关组也是\(\boldsymbol{B}\)的行向量的极大无关组. 因此,\(\boldsymbol{\beta}\)可表示为\(\boldsymbol{\alpha}_1,\boldsymbol{\alpha}_2,\cdots,\boldsymbol{\alpha}_m\)的线性组合. 
\end{proof}

\begin{example}
设\(\boldsymbol{A}\boldsymbol{x}=\boldsymbol{\beta}\)是\(m\)个方程式\(n\)个未知数的线性方程组,求证:它有解的充要条件是方程组\(\boldsymbol{A}'\boldsymbol{y}=\boldsymbol{0}\)的任一解\(\boldsymbol{\alpha}\)均适合等式\(\boldsymbol{\alpha}'\boldsymbol{\beta}=0\).
\end{example}
\begin{proof}
方程组\(\boldsymbol{A}\boldsymbol{x}=\boldsymbol{\beta}\)有解当且仅当\(\mathrm{r}\left( \boldsymbol{A}\,|\,\boldsymbol{\beta } \right) =\mathrm{r}(\boldsymbol{A})\),当且仅当\(\mathrm{r}\begin{pmatrix}
\boldsymbol{A}'\\
\boldsymbol{\beta}'
\end{pmatrix}=\mathrm{r}(\boldsymbol{A}')\),当且仅当方程组\(\begin{pmatrix}
\boldsymbol{A}'\\
\boldsymbol{\beta}'
\end{pmatrix}\boldsymbol{y}=\boldsymbol{0}\)与\(\boldsymbol{A}'\boldsymbol{y}=\boldsymbol{0}\)同解,而这当且仅当\(\boldsymbol{A}'\boldsymbol{y}=\boldsymbol{0}\)的任一解\(\boldsymbol{\alpha}\)均适合等式\(\boldsymbol{\beta}'\boldsymbol{\alpha}=0\),即\(\boldsymbol{\alpha}'\boldsymbol{\beta}=0\).
\end{proof}

\begin{example}
设有两个线性方程组:
\begin{align}\label{equation:3.301.1}
\begin{cases}
a_{11}x_1 + a_{12}x_2+\cdots + a_{1n}x_n = b_1,\\
a_{21}x_1 + a_{22}x_2+\cdots + a_{2n}x_n = b_2,\\
\cdots\cdots\cdots\cdots\\
a_{m1}x_1 + a_{m2}x_2+\cdots + a_{mn}x_n = b_m;
\end{cases} 
\end{align}
\begin{align}\label{equation:3.301.2}
\begin{cases}
a_{11}x_1 + a_{21}x_2+\cdots + a_{m1}x_m = 0,\\
a_{12}x_1 + a_{22}x_2+\cdots + a_{m2}x_m = 0,\\
\cdots\cdots\cdots\cdots\\
a_{1n}x_1 + a_{2n}x_2+\cdots + a_{mn}x_m = 0,\\
b_1x_1 + b_2x_2+\cdots + b_mx_m = 1.
\end{cases}
\end{align}
求证: 方程组\eqref{equation:3.301.1}有解的充要条件是方程组\eqref{equation:3.301.2}无解.
\end{example}
\begin{proof}
设第一个线性方程组的系数矩阵为\(\boldsymbol{A}\), 常数向量为\(\boldsymbol{\beta}\), 则第二个线性方程组的系数矩阵和增广矩阵分别为
\[
\boldsymbol{B}=\begin{pmatrix}
\boldsymbol{A}'\\
\boldsymbol{\beta}'
\end{pmatrix},\widetilde{\boldsymbol{B}}=\begin{pmatrix}
\boldsymbol{A}'&\boldsymbol{O}\\
\boldsymbol{\beta}'&1
\end{pmatrix}.
\]
显然,由矩阵初等变换可知,我们有\(\mathrm{r}(\widetilde{\boldsymbol{B}})=\mathrm{r}(\boldsymbol{A}') + 1=\mathrm{r}(\boldsymbol{A})+1\).

若方程组\eqref{equation:3.301.1}有解, 则\(\mathrm{r}\left( \boldsymbol{A}\,|\,\boldsymbol{\beta } \right) =\mathrm{r}(\boldsymbol{A})\), 故\(\mathrm{r}(\boldsymbol{B})=\mathrm{r}(\boldsymbol{B}')=\mathrm{r}\left( \boldsymbol{A}\,|\,\boldsymbol{\beta } \right) =\mathrm{r}(\boldsymbol{A})\neq\mathrm{r}(\widetilde{\boldsymbol{B}})\). 因此, 方程组\eqref{equation:3.301.2}无解.

反之, 若方程组\eqref{equation:3.301.1}无解, 则\(\mathrm{r}\left( \boldsymbol{A}\,|\,\boldsymbol{\beta } \right) =\mathrm{r}(\boldsymbol{A}) + 1\), 故\(\mathrm{r}(\boldsymbol{B})=\mathrm{r}(\boldsymbol{B}')=\mathrm{r}\left( \boldsymbol{A}\,|\,\boldsymbol{\beta } \right) =\mathrm{r}(\boldsymbol{A})+1=\mathrm{r}(\widetilde{\boldsymbol{B}})\). 因此, 方程组\eqref{equation:3.301.2}有解.
\end{proof}

\begin{proposition}\label{proposition:线性方程组的解结论1}
设\(\boldsymbol{A}\)是秩为\(r\)的\(m\times n\)矩阵,求证:必存在秩为\(n - r\)的\(n\times(n - r)\)矩阵\(\boldsymbol{B}\),使得\(\boldsymbol{A}\boldsymbol{B}=\boldsymbol{O}\).
\end{proposition}
\begin{proof}
考虑线性方程组\(\boldsymbol{A}\boldsymbol{x}=\boldsymbol{0}\),它有\(n - r\)个基础解系,不妨设为\(\boldsymbol{\beta}_1,\cdots,\boldsymbol{\beta}_{n - r}\). 令\(\boldsymbol{B}=(\boldsymbol{\beta}_1,\cdots,\boldsymbol{\beta}_{n - r})\),则\(\boldsymbol{A}\boldsymbol{B}=(\boldsymbol{A}\boldsymbol{\beta}_1,\cdots,\boldsymbol{A}\boldsymbol{\beta}_{n - r})=\boldsymbol{O}\),结论得证. 
\end{proof}

\begin{example}
设
\[
\boldsymbol{A}=\begin{pmatrix}
a_{11}&a_{12}&\cdots&a_{1n}\\
a_{21}&a_{22}&\cdots&a_{2n}\\
\vdots&\vdots&&\vdots\\
a_{m1}&a_{m2}&\cdots&a_{mn}
\end{pmatrix}(m < n),
\]
已知\(\boldsymbol{A}\boldsymbol{x}=\boldsymbol{0}\)的基础解系为\(\boldsymbol{\beta}_i=(b_{i1},b_{i2},\cdots,b_{in})'(1\leq i\leq n - m)\),试求齐次线性方程组
\[
\sum_{j = 1}^{n}b_{ij}y_j = 0(i = 1,2,\cdots,n - m)
\]
的基础解系.
\end{example}
\begin{solution}
令\(\boldsymbol{B}=(\boldsymbol{\beta}_1,\boldsymbol{\beta}_2,\cdots,\boldsymbol{\beta}_{n - m})\),则\(\boldsymbol{A}\boldsymbol{B}=\boldsymbol{O},\boldsymbol{B}'\boldsymbol{A}'=\boldsymbol{O}\). 因为\(\boldsymbol{A}\boldsymbol{x}=\boldsymbol{0}\)有$n-m$个基础解系,所以\(\boldsymbol{A}\)的秩为\(m\).又由于$\mathrm{r}\left( \boldsymbol{B} \right) =\mathrm{r}\left( \boldsymbol{B}‘ \right) =n-m$,因此\(\boldsymbol{B}'\boldsymbol{y}=\boldsymbol{0}\)的基础解系有$m$个.故\(\boldsymbol{B}'\boldsymbol{y}=\boldsymbol{0}\)的基础解系为\(\boldsymbol{A}'\)的全部列向量,即\(\boldsymbol{A}\)的所有行向量.
\end{solution}

\begin{proposition}\label{proposition:一个线性子空间对应一个线性方程组的系数矩阵}
设\(V_0\)是数域\(\mathbb{K}\)上\(n\)维列向量空间的真子空间,求证:必存在矩阵\(\boldsymbol{A}\),使得\(V_0\)是\(n\)元齐次线性方程组\(\boldsymbol{A}\boldsymbol{x}=\boldsymbol{0}\)的解空间.
\end{proposition}
\begin{proof}
设\(\boldsymbol{\beta}_1,\cdots,\boldsymbol{\beta}_r\)是子空间\(V_0\)的一组基. 令\(\boldsymbol{B}=(\boldsymbol{\beta}_1,\cdots,\boldsymbol{\beta}_r)\),这是一个\(n\times r\)矩阵. 考虑齐次线性方程组\(\boldsymbol{B}'\boldsymbol{x}=\boldsymbol{0}\),因为\(\boldsymbol{B}\)的秩等于\(r\),故其基础解系含\(n - r\)个向量,记为\(\boldsymbol{\alpha}_1,\cdots,\boldsymbol{\alpha}_{n - r}\). 令\(\boldsymbol{A}=(\boldsymbol{\alpha}_1,\cdots,\boldsymbol{\alpha}_{n - r})'\),这是个\((n - r)\times n\)矩阵且秩为\(n - r\). 由\(\boldsymbol{B}'\boldsymbol{A}'=\boldsymbol{O}\)可得\(\boldsymbol{A}\boldsymbol{B}=\boldsymbol{O}\),因此齐次线性方程组\(\boldsymbol{A}\boldsymbol{x}=\boldsymbol{0}\)的基础解系是\(\boldsymbol{\beta}_1,\cdots,\boldsymbol{\beta}_r\),其解空间就是\(V_0\).
\end{proof}
\begin{remark}
设\(\boldsymbol{\beta}_1,\cdots,\boldsymbol{\beta}_r\)是子空间\(V_0\)的一组基. 令\(\boldsymbol{B}=(\boldsymbol{\beta}_1,\cdots,\boldsymbol{\beta}_r)\),这是一个\(n\times r\)矩阵.也可以由\hyperref[proposition:线性方程组的解结论1]{命题\ref{proposition:线性方程组的解结论1}}直接得到存在矩阵$\boldsymbol{A}$,使得\(\boldsymbol{A}\boldsymbol{B}=\boldsymbol{O}\),因此齐次线性方程组\(\boldsymbol{A}\boldsymbol{x}=\boldsymbol{0}\)的基础解系是\(\boldsymbol{\beta}_1,\cdots,\boldsymbol{\beta}_r\),其解空间就是\(V_0\)..
\end{remark}

\begin{example}
设\(\boldsymbol{A}\)是秩为\(r\)的\(m\times n\)矩阵,\(\boldsymbol{\alpha}_1,\cdots,\boldsymbol{\alpha}_{n - r}\)与\(\boldsymbol{\beta}_1,\cdots,\boldsymbol{\beta}_{n - r}\)是齐次线性方程组\(\boldsymbol{A}\boldsymbol{x}=\boldsymbol{0}\)的两个基础解系. 求证:必存在\(n - r\)阶可逆矩阵\(\boldsymbol{P}\),使得
\[
(\boldsymbol{\beta}_1,\cdots,\boldsymbol{\beta}_{n - r})=(\boldsymbol{\alpha}_1,\cdots,\boldsymbol{\alpha}_{n - r})\boldsymbol{P}.
\]
\end{example}
\begin{proof}
设\(U\)是齐次线性方程组\(\boldsymbol{A}\boldsymbol{x}=\boldsymbol{0}\)的解空间,则向量组\(\boldsymbol{\alpha}_1,\cdots,\boldsymbol{\alpha}_{n - r}\)与\(\boldsymbol{\beta}_1,\cdots,\boldsymbol{\beta}_{n - r}\)是\(U\)的两组基. 令\(\boldsymbol{P}\)是这两组基之间的过渡矩阵,则
\[
(\boldsymbol{\beta}_1,\cdots,\boldsymbol{\beta}_{n - r})=(\boldsymbol{\alpha}_1,\cdots,\boldsymbol{\alpha}_{n - r})\boldsymbol{P}. 
\]  
\end{proof}

\begin{theorem}\label{theorem:矩阵方程有解的充要条件}
设\(\boldsymbol{A},\boldsymbol{B}\)为\(m\times n\)和\(m\times p\)矩阵,\(\boldsymbol{X}\)为\(n\times p\)未知矩阵,证明:矩阵方程\(\boldsymbol{A}\boldsymbol{X}=\boldsymbol{B}\)有解的充要条件是\(\mathrm{r}\left( \boldsymbol{A}\,|\,\boldsymbol{B} \right)=\mathrm{r}(\boldsymbol{A})\).
\end{theorem}
\begin{proof}
设\(\boldsymbol{A}=(\boldsymbol{\alpha}_1,\cdots,\boldsymbol{\alpha}_n),\boldsymbol{B}=(\boldsymbol{\beta}_1,\cdots,\boldsymbol{\beta}_p),\boldsymbol{X}=(\boldsymbol{x}_1,\cdots,\boldsymbol{x}_p)\)为对应的列分块. 设\(\mathrm{r}(\boldsymbol{A}) = r\)且\(\boldsymbol{\alpha}_{i_1},\cdots,\boldsymbol{\alpha}_{i_r}\)是\(\boldsymbol{A}\)的列向量的极大无关组. 注意到矩阵方程\(\boldsymbol{A}\boldsymbol{X}=\boldsymbol{B}\)有解当且仅当\(p\)个线性方程组\(\boldsymbol{A}\boldsymbol{x}_i=\boldsymbol{\beta}_i(1\leq i\leq p)\)都有解. 因此,若\(\boldsymbol{A}\boldsymbol{X}=\boldsymbol{B}\)有解,则每个\(\boldsymbol{\beta}_i\)都是\(\boldsymbol{A}\)的列向量的线性组合,从而是\(\boldsymbol{\alpha}_{i_1},\cdots,\boldsymbol{\alpha}_{i_r}\)的线性组合,于是\(\boldsymbol{\alpha}_{i_1},\cdots,\boldsymbol{\alpha}_{i_r}\)是\(\left( \boldsymbol{A}\,|\,\boldsymbol{B} \right)\)的列向量的极大无关组,故\(\mathrm{r}\left( \boldsymbol{A}\,|\,\boldsymbol{B} \right) = r\). 反之,若\(\mathrm{r}\left( \boldsymbol{A}\,|\,\boldsymbol{B} \right) = r\),则由\hyperref[proposition:表出向量组的秩不超过原向量组的秩]{命题\ref{proposition:表出向量组的秩不超过原向量组的秩}}可知,\(\boldsymbol{\alpha}_{i_1},\cdots,\boldsymbol{\alpha}_{i_r}\)是\(\left( \boldsymbol{A}\,|\,\boldsymbol{B} \right)\)的列向量的极大无关组,于是每个\(\boldsymbol{\beta}_i\)都是\(\boldsymbol{A}\)的列向量的线性组合,从而\(\boldsymbol{A}\boldsymbol{X}=\boldsymbol{B}\)有解.
\end{proof}

\begin{proposition}
矩阵方程\(\boldsymbol{A}\boldsymbol{X}=\boldsymbol{B}\)有解当且仅当\(p\)个线性方程组\(\boldsymbol{A}\boldsymbol{x}_i=\boldsymbol{\beta}_i(1\leq i\leq p)\)都有解.从而每个\(\boldsymbol{\beta}_i\)都是\(\boldsymbol{A}\)的列向量的线性组合.
\end{proposition}
\begin{proof}
证明是显然的.
\end{proof}

\begin{proposition}
设\(\boldsymbol{A},\boldsymbol{B}\)为\(m\times n\)和\(n\times p\)矩阵,证明:存在\(p\times n\)矩阵\(\boldsymbol{C}\),使得\(\boldsymbol{A}\boldsymbol{B}\boldsymbol{C}=\boldsymbol{A}\)的充要条件是\(\mathrm{r}(\boldsymbol{A})=\mathrm{r}(\boldsymbol{A}\boldsymbol{B})\).
\end{proposition}
\begin{proof}
必要性由秩的不等式\(\mathrm{r}(\boldsymbol{A})\geq\mathrm{r}(\boldsymbol{A}\boldsymbol{B})\geq\mathrm{r}(\boldsymbol{A}\boldsymbol{B}\boldsymbol{C})=\mathrm{r}(\boldsymbol{A})\)即得. 

充分性由秩的不等式可知\(\mathrm{r}\left( \boldsymbol{A} \right) =\mathrm{r}\left( \boldsymbol{AB} \right) \le \mathrm{r}\left( \boldsymbol{AB}\,|\,\boldsymbol{B} \right) =\mathrm{r}\left( \boldsymbol{A}\left( \boldsymbol{B}\,|\,\boldsymbol{I}_n \right) \right) \le \mathrm{r}\left( \boldsymbol{A} \right)\).故$\mathrm{r}\left( \boldsymbol{A} \right) =\mathrm{r}\left( \boldsymbol{AB}\,|\,\boldsymbol{B} \right)$.于是由
\hyperref[theorem:矩阵方程有解的充要条件]{定理\ref{theorem:矩阵方程有解的充要条件}}可知,矩阵方程$\boldsymbol{ABX}=\boldsymbol{A}$有解.即存在\(p\times n\)矩阵\(\boldsymbol{C}\),使得\(\boldsymbol{A}\boldsymbol{B}\boldsymbol{C}=\boldsymbol{A}\)的充要条件是\(\mathrm{r}(\boldsymbol{A})=\mathrm{r}(\boldsymbol{A}\boldsymbol{B})\).
\end{proof}

\subsection{线性方程组的公共解}

对两个非齐次线性方程组,若只已知它们的通解,而不知道方程组本身,要求它们的公共解,我们可以这样来做:
设\(\boldsymbol{A}\boldsymbol{x}=\boldsymbol{\beta}_1,\boldsymbol{B}\boldsymbol{x}=\boldsymbol{\beta}_2\)是两个含\(n\)个未知数的非齐次线性方程组. 方程组\(\boldsymbol{A}\boldsymbol{x}=\boldsymbol{\beta}_1\)有特解\(\boldsymbol{\gamma}\)且\(\boldsymbol{A}\boldsymbol{x}=\boldsymbol{0}\)的基础解系为\(\boldsymbol{\eta}_1,\cdots,\boldsymbol{\eta}_{n - r}\). 方程组\(\boldsymbol{B}\boldsymbol{x}=\boldsymbol{\beta}_2\)有特解\(\boldsymbol{\delta}\)且\(\boldsymbol{B}\boldsymbol{x}=\boldsymbol{0}\)的基础解系为\(\boldsymbol{\xi}_1,\cdots,\boldsymbol{\xi}_{n - s}\).

{\color{blue}方法一:} 假设它们的公共解为\(\boldsymbol{\gamma}+t_1\boldsymbol{\eta}_1+\cdots+t_{n - r}\boldsymbol{\eta}_{n - r}\),则\(\boldsymbol{\gamma}+t_1\boldsymbol{\eta}_1+\cdots+t_{n - r}\boldsymbol{\eta}_{n - r}-\boldsymbol{\delta}\)是\(\boldsymbol{B}\boldsymbol{x}=\boldsymbol{0}\)的解,因此可以表示为\(\boldsymbol{\xi}_1,\cdots,\boldsymbol{\xi}_{n - s}\)的线性组合. 于是矩阵\((\boldsymbol{\xi}_1,\cdots,\boldsymbol{\xi}_{n - s},\boldsymbol{\gamma}+t_1\boldsymbol{\eta}_1+\cdots+t_{n - r}\boldsymbol{\eta}_{n - r}-\boldsymbol{\delta})\)的秩等于\(n - s\). 由此可以求出\(t_1,\cdots,t_{n - r}\),从而求出公共解.

{\color{blue}方法二:}假设它们的公共解为\(\boldsymbol{\zeta}\),则
\[
\boldsymbol{\zeta}=\boldsymbol{\gamma}+t_1\boldsymbol{\eta}_1+\cdots+t_{n - r}\boldsymbol{\eta}_{n - r}=\boldsymbol{\delta}+(-u_1)\boldsymbol{\xi}_1+\cdots+(-u_{n - s})\boldsymbol{\xi}_{n - s}.
\]
要求公共解\(\boldsymbol{\zeta}\)等价于求解下列关于未定元\(t_1,\cdots,t_{n - r};u_1,\cdots,u_{n - s}\)的线性方程组:
\[
t_1\boldsymbol{\eta}_1+\cdots+t_{n - r}\boldsymbol{\eta}_{n - r}+u_1\boldsymbol{\xi}_1+\cdots+u_{n - s}\boldsymbol{\xi}_{n - s}=\boldsymbol{\delta}-\boldsymbol{\gamma}.
\]

\begin{example}
设有两个非齐次线性方程组(I), (II),它们的通解分别为
\[
\boldsymbol{\gamma}+t_1\boldsymbol{\eta}_1 + t_2\boldsymbol{\eta}_2;\boldsymbol{\delta}+k_1\boldsymbol{\xi}_1 + k_2\boldsymbol{\xi}_2,
\]
其中\(\boldsymbol{\gamma}=(5,-3,0,0)',\boldsymbol{\eta}_1=(-6,5,1,0)',\boldsymbol{\eta}_2=(-5,4,0,1)';\boldsymbol{\delta}=(-11,3,0,0)',\boldsymbol{\xi}_1=(8,-1,1,0)',\boldsymbol{\xi}_2=(10,-2,0,1)'\). 求这两个方程组的公共解.
\end{example}
\begin{proof}
{\color{blue}解法一:} 设公共解为
\[
\boldsymbol{\gamma}+t_1\boldsymbol{\eta}_1 + t_2\boldsymbol{\eta}_2=\begin{pmatrix}
5 - 6t_1 - 5t_2\\
-3 + 5t_1 + 4t_2\\
t_1\\
t_2
\end{pmatrix}.
\]
注意矩阵\((\boldsymbol{\xi}_1,\boldsymbol{\xi}_2,\boldsymbol{\gamma}-\boldsymbol{\delta}+t_1\boldsymbol{\eta}_1 + t_2\boldsymbol{\eta}_2)\)的秩等于2,对此矩阵作初等行变换:
\[
\begin{pmatrix}
8&10&16 - 6t_1 - 5t_2\\
-1&-2&-6 + 5t_1 + 4t_2\\
1&0&t_1\\
0&1&t_2
\end{pmatrix}\to\begin{pmatrix}
1&0&t_1\\
0&1&t_2\\
8&10&16 - 6t_1 - 5t_2\\
-1&-2&-6 + 5t_1 + 4t_2
\end{pmatrix}\to
\begin{pmatrix}
1&0&t_1\\
0&1&t_2\\
0&0&16 - 14t_1 - 15t_2\\
0&0&-6 + 6t_1 + 6t_2
\end{pmatrix}
\]
可得关于\(t_1,t_2\)的方程组
\[
\begin{cases}
14t_1 + 15t_2 = 16,\\
6t_1 + 6t_2 = 6.
\end{cases}
\]
解得\(t_1=-1,t_2 = 2\),所以公共解为 (只有一个向量)\(\boldsymbol{\gamma}-\boldsymbol{\eta}_1 + 2\boldsymbol{\eta}_2=(1,0,-1,2)'\).

{\color{blue}解法二:} 求公共解等价于求解下列线性方程组:
\[
t_1\boldsymbol{\eta}_1 + t_2\boldsymbol{\eta}_2 + u_1\boldsymbol{\xi}_1 + u_2\boldsymbol{\xi}_2=\boldsymbol{\delta}-\boldsymbol{\gamma}.
\]
对其增广矩阵实施初等行变换,可得
\[
\begin{pmatrix}
-6&-5&8&10&-16\\
5&4&-1&-2&6\\
1&0&1&0&0\\
0&1&0&1&0
\end{pmatrix}\to\begin{pmatrix}
1&0&0&0&-1\\
0&1&0&0&2\\
0&0&1&0&1\\
0&0&0&1&-2
\end{pmatrix},
\]
故\((t_1,t_2,u_1,u_2)\)只有唯一解\((-1,2,1,-2)\). 因此,公共解为\(\boldsymbol{\gamma}-\boldsymbol{\eta}_1 + 2\boldsymbol{\eta}_2=\boldsymbol{\delta}-\boldsymbol{\xi}_1 + 2\boldsymbol{\xi}_2=(1,0,-1,2)'\).
\end{proof}

\begin{example}
设有非齐次线性方程组 (I):
\[
\begin{cases}
7x_1 - 6x_2 + 3x_3 = b,\\
8x_1 - 9x_2 + ax_4 = 7.
\end{cases}
\]
又已知方程组 (II) 的通解为
\[
(1,1,1,0)' + t_1(1,0,-1,0)' + t_2(2,3,0,1)'.
\]
若这两个方程组有无穷多组公共解,求出 \(a,b\) 的值并求出公共解.
\end{example}
\begin{proof}
将 (II) 的通解写为 \((1 + t_1 + 2t_2,1 + 3t_2,-t_1,t_2)'\),代入方程组 (I) 化简得到
\[
\begin{cases}
4t_1 - 4t_2 = b - 1,\\
8t_1 + (a - 11)t_2 = 8.
\end{cases}
\]
要使这两个方程组有无穷多组公共解,\(t_1,t_2\) 必须有无穷多组解,于是上面方程组的系数矩阵和增广矩阵的秩都应该等于 1,从而 \(a = 3,b = 5\). 解出方程组得到 \(t_1 = t_2 + 1\),因此方程组 (I), (II) 的公共解为
\[
(1 + t_1 + 2t_2,1 + 3t_2,-t_1,t_2)' = (2,1,-1,0)' + t_2(3,3,-1,1)',
\]
其中 \(t_2\) 为任意数.
\end{proof}

\subsection{在解析几何上的应用}

\begin{proposition}\label{proposition:平面上点位于同一条直线上的充要条件}
求平面上 \(n\) 个点 \((x_1,y_1),(x_2,y_2),\cdots,(x_n,y_n)\) 位于同一条直线上的充要条件.
\end{proposition}
\begin{proof}
充要条件为第一个点和其余点代表的向量之差属于一个一维子空间,即 \((x_i - x_1,y_i - y_1)\) 都成比例. 写成矩阵形式为
\[
\mathrm{r}\begin{pmatrix}
x_2 - x_1&x_3 - x_1&\cdots&x_n - x_1\\
y_2 - y_1&y_3 - y_1&\cdots&y_n - y_1
\end{pmatrix}\leq1,
\]
或
\[
\mathrm{r}\begin{pmatrix}
x_1&x_2&x_3&\cdots&x_n\\
y_1&y_2&y_3&\cdots&y_n\\
1&1&1&\cdots&1
\end{pmatrix}\leq2. 
\]
\end{proof}

\begin{proposition}\label{proposition:三维实空间中4点共面}
求三维实空间中 4 点 \((x_i,y_i,z_i)(1\leq i\leq 4)\) 共面的充要条件.
\end{proposition}
\begin{proof}
设 4 点的向量为 \(\boldsymbol{\alpha}_1,\boldsymbol{\alpha}_2,\boldsymbol{\alpha}_3,\boldsymbol{\alpha}_4\),则 4 点共面的充要条件是:向量组 \(\boldsymbol{\alpha}_2 - \boldsymbol{\alpha}_1,\boldsymbol{\alpha}_3 - \boldsymbol{\alpha}_1,\boldsymbol{\alpha}_4 - \boldsymbol{\alpha}_1\) 的秩不超过 2. 不难将此写成矩阵形式:
\[
\mathrm{r}\begin{pmatrix}
x_2 - x_1&x_3 - x_1&x_4 - x_1\\
y_2 - y_1&y_3 - y_1&y_4 - y_1\\
z_2 - z_1&z_3 - z_1&z_4 - z_1
\end{pmatrix}\leq2,
\]
或
\[
\mathrm{r}\begin{pmatrix}
x_1&x_2&x_3&x_4\\
y_1&y_2&y_3&y_4\\
z_1&z_2&z_3&z_4\\
1&1&1&1
\end{pmatrix}\leq3. 
\]
\end{proof}

\begin{example}\label{example:3.35115}
证明: 通过平面内不在一条直线上的 3 点 \((x_1,y_1),(x_2,y_2),(x_3,y_3)\) 的圆方程为
\[
\begin{vmatrix}
x^2 + y^2&x&y&1\\
x_1^2 + y_1^2&x_1&y_1&1\\
x_2^2 + y_2^2&x_2&y_2&1\\
x_3^2 + y_3^2&x_3&y_3&1
\end{vmatrix}=0.
\]
\end{example}
\begin{proof}
圆方程可设为
\[
u_1(x^2 + y^2)+u_2x + u_3y + u_4 = 0,
\]
于是得到未知数 \(u_1,u_2,u_3,u_4\) 的方程组为
\[
\begin{cases}
(x_1^2 + y_1^2)u_1 + x_1u_2 + y_1u_3 + u_4 = 0,\\
(x_2^2 + y_2^2)u_1 + x_2u_2 + y_2u_3 + u_4 = 0,\\
(x_3^2 + y_3^2)u_1 + x_3u_2 + y_3u_3 + u_4 = 0.
\end{cases}
\]
上述方程组加上原方程组成一个含 4 个未知数、4 个方程式的齐次线性方程组,它有非零解的充要条件是系数行列式等于零,即
\[
\begin{vmatrix}
x^2 + y^2&x&y&1\\
x_1^2 + y_1^2&x_1&y_1&1\\
x_2^2 + y_2^2&x_2&y_2&1\\
x_3^2 + y_3^2&x_3&y_3&1
\end{vmatrix}=0.
\]
由\hyperref[proposition:平面上点位于同一条直线上的充要条件]{命题\ref{proposition:平面上点位于同一条直线上的充要条件}}可知 3 点不在一条直线上意味着
\[
\begin{vmatrix}
x_1&y_1&1\\
x_2&y_2&1\\
x_3&y_3&1
\end{vmatrix}\neq0,
\]
故圆方程不退化.  
\end{proof}

\begin{proposition}\label{proposition:平面上4点共圆的充要条件}
求平面上不在一条直线上的 4 个点 \((x_1,y_1),(x_2,y_2),(x_3,y_3),(x_4,y_4)\) 位于同一个圆上的充要条件.
\end{proposition}
\begin{proof}
由\hyperref[example:3.35115]{例题\ref{example:3.35115}}可得充要条件为
\[
\begin{vmatrix}
x_1^2 + y_1^2&x_1&y_1&1\\
x_2^2 + y_2^2&x_2&y_2&1\\
x_3^2 + y_3^2&x_3&y_3&1\\
x_4^2 + y_4^2&x_4&y_4&1
\end{vmatrix}=0.
\]
\end{proof}

\begin{example}
已知平面上两条不同的二次曲线 \(a_ix^2 + b_ixy + c_iy^2 + d_ix + e_iy + f_i = 0(i = 1,2)\) 交于 4 个不同的点 \((x_i,y_i)(1\leq i\leq 4)\). 求证: 过这 4 个点的二次曲线均可写为如下形状:
\[
\lambda_1(a_1x^2 + b_1xy + c_1y^2 + d_1x + e_1y + f_1)+\lambda_2(a_2x^2 + b_2xy + c_2y^2 + d_2x + e_2y + f_2)=0.
\]
\end{example}
\begin{proof}
显然上述曲线过这 4 个交点. 现设 \(ax^2 + bxy + cy^2 + dx + ey + f = 0\) 是过这 4 个交点的二次曲线,则有
\begin{align}\label{equation:3.363.3}
\begin{cases}
ax_1^2 + bx_1y_1 + cy_1^2 + dx_1 + ey_1 + f = 0,\\
ax_2^2 + bx_2y_2 + cy_2^2 + dx_2 + ey_2 + f = 0,\\
ax_3^2 + bx_3y_3 + cy_3^2 + dx_3 + ey_3 + f = 0,\\
ax_4^2 + bx_4y_4 + cy_4^2 + dx_4 + ey_4 + f = 0.
\end{cases}
\end{align}
视 \(a,b,c,d,e,f\) 为未知数,则线性方程组 \eqref{equation:3.363.3} 有线性无关的解 \((a_1,b_1,c_1,d_1,e_1,f_1)',(a_2,b_2,c_2,d_2,e_2,f_2)'\). 如果能证明方程组 \eqref{equation:3.363.3} 的系数矩阵的秩等于 4, 则这两个解就构成了基础解系,从而即得结论.

容易验证 4 个交点中的任意 3 个点都不共线,而且经过坐标轴适当的旋转,可以假设这 4 个交点的横坐标 \(x_1,x_2,x_3,x_4\) 互不相同. 用反证法证明结论,设方程组 \eqref{equation:3.363.3} 系数矩阵 \(\boldsymbol{A}\) 的秩小于 4. 由任意 3 个交点不共线以及\hyperref[proposition:平面上点位于同一条直线上的充要条件]{命题\ref{proposition:平面上点位于同一条直线上的充要条件}}知,\((x_1,x_2,x_3,x_4)',(y_1,y_2,y_3,y_4)',(1,1,1,1)'\) 线性无关,从而它们是 \(\boldsymbol{A}\) 的列向量的极大无关组,于是 \((x_1^2,x_2^2,x_3^2,x_4^2)'\) 是它们的线性组合,故可设 \(x_i^2 = rx_i + sy_i + t(1\leq i\leq 4)\),其中 \(r,s,t\) 是实数. 由于 \(x_1,x_2,x_3,x_4\) 互不相同,故 \(s\neq 0\),于是 \(y_i = \frac{1}{s}x_i^2 - \frac{r}{s}x_i - \frac{t}{s}(1\leq i\leq 4)\). 考虑 \(\boldsymbol{A}\) 的第一列、第二列、第四列和第六列构成的四阶行列式 \(|\boldsymbol{B}|\),利用 Vander - monde 行列式容易算出 \(|\boldsymbol{B}|=-\frac{1}{s}\prod_{1\leq i < j\leq 4}(x_i - x_j)\neq 0\),于是 \(\boldsymbol{A}\) 的秩等于 4, 这与假设矛盾. 因此方程组 \eqref{equation:3.363.3} 的系数矩阵的秩只能等于 4.
\end{proof}


\chapter{线性映射}

\begin{proposition}[满射的复合仍是满射]\label{proposition:满射的复合仍是满射}
若\(f,g\)均为满射,则\(f\circ g\)也是满射.
\end{proposition}
\begin{remark}
单射的复合不一定是单射.
\end{remark}
\begin{proof}
设\(f:V_1\rightarrow V_2\),\(g:V_2\rightarrow V_3\),则对\(\forall \alpha \in V_3\),由\(g\)为满射可知,存在\(\beta \in V_2\),使得\(\alpha = g(\beta)\).又由\(f\)为满射可知,存在\(\gamma \in V_1\),使得\(\beta = f(\gamma)\).从而\((g\circ f)(\gamma)=g(f(\gamma)) = g(\beta)=\alpha\).故\(g\circ f\)也是满射.
\end{proof}

\begin{definition}[线性映射的表示矩阵]\label{definition:线性映射的表示矩阵}
设\(\varphi\)是\(V\to U\)的线性映射,分别取\(V\)和\(U\)的基如下:
\[
V:\boldsymbol{e}_1,\boldsymbol{e}_2,\cdots,\boldsymbol{e}_n; \ U:\boldsymbol{f}_1,\boldsymbol{f}_2,\cdots,\boldsymbol{f}_m.
\]
假设有
\[
\begin{cases}
\varphi(\boldsymbol{e}_1)=a_{11}\boldsymbol{f}_1 + a_{21}\boldsymbol{f}_2+\cdots + a_{m1}\boldsymbol{f}_m,\\
\varphi(\boldsymbol{e}_2)=a_{12}\boldsymbol{f}_1 + a_{22}\boldsymbol{f}_2+\cdots + a_{m2}\boldsymbol{f}_m,\\
\cdots\cdots\cdots\\
\varphi(\boldsymbol{e}_n)=a_{1n}\boldsymbol{f}_1 + a_{2n}\boldsymbol{f}_2+\cdots + a_{mn}\boldsymbol{f}_m,
\end{cases}
\]
则矩阵
\[
\begin{pmatrix}
a_{11}&a_{12}&\cdots&a_{1n}\\
a_{21}&a_{22}&\cdots&a_{2n}\\
\vdots&\vdots&&\vdots\\
a_{m1}&a_{m2}&\cdots&a_{mn}
\end{pmatrix}
\]
称为线性映射\(\varphi\)在基$\left\{ \boldsymbol{e}_1,\boldsymbol{e}_2,\cdots ,\boldsymbol{e}_n \right\}$和$\left\{ \boldsymbol{f}_1,\boldsymbol{f}_2,\cdots ,\boldsymbol{f}_m \right\}$下的表示矩阵.
\end{definition}
\begin{remark}
若\(\varphi\)是向量空间\(V\)上的线性变换,则取\(V\)的一组基,而不取两组基.
\end{remark}

\begin{corollary}
设\(\varphi\)是\(V\rightarrow U\)的线性映射,分别取\(V\)和\(U\)的基如下:
\[
V: \boldsymbol{e}_1, \boldsymbol{e}_2, \cdots, \boldsymbol{e}_n; \quad U: \boldsymbol{f}_1, \boldsymbol{f}_2, \cdots, \boldsymbol{f}_m.
\]
并且设\(\varphi\)在基\(\{\boldsymbol{e}_1, \boldsymbol{e}_2, \cdots, \boldsymbol{e}_n\}\)和\(\{\boldsymbol{f}_1, \boldsymbol{f}_2, \cdots, \boldsymbol{f}_m\}\)下的表示矩阵为\(\boldsymbol{A}\).则有
\[
(\varphi(\boldsymbol{e}_1), \varphi(\boldsymbol{e}_2), \cdots, \varphi(\boldsymbol{e}_n)) = (\boldsymbol{f}_1, \boldsymbol{f}_2, \cdots, \boldsymbol{f}_m)\boldsymbol{A}.
\]
\(\forall \boldsymbol{\alpha} \in V\),设\(\boldsymbol{\alpha}\)在基\(\{\boldsymbol{e}_1, \boldsymbol{e}_2, \cdots, \boldsymbol{e}_n\}\)下的坐标向量为\((x_1, x_2, \cdots, x_n)'\),则
\[
\boldsymbol{\alpha} = x_1\boldsymbol{e}_1 + x_2\boldsymbol{e}_2 + \cdots + x_n\boldsymbol{e}_n = (\boldsymbol{e}_1, \boldsymbol{e}_2, \cdots, \boldsymbol{e}_n)\begin{pmatrix}
x_1 \\
x_2 \\
\vdots \\
x_n
\end{pmatrix},
\]
\begin{align*}
\varphi(\boldsymbol{\alpha}) &= \varphi(x_1\boldsymbol{e}_1 + x_2\boldsymbol{e}_2 + \cdots + x_n\boldsymbol{e}_n)
= (\varphi(\boldsymbol{e}_1), \varphi(\boldsymbol{e}_2), \cdots, \varphi(\boldsymbol{e}_n))\begin{pmatrix}
x_1 \\
x_2 \\
\vdots \\
x_n
\end{pmatrix} 
= (\boldsymbol{f}_1, \boldsymbol{f}_2, \cdots, \boldsymbol{f}_m)\boldsymbol{A}\begin{pmatrix}
x_1 \\
x_2 \\
\vdots \\
x_n
\end{pmatrix}.
\end{align*}
即\(\varphi(\boldsymbol{\alpha})\)在基\(\{\boldsymbol{f}_1, \boldsymbol{f}_2, \cdots, \boldsymbol{f}_m\}\)下的坐标向量为\(\boldsymbol{A}(x_1, x_2, \cdots, x_n)'\).
\end{corollary}

\begin{theorem}\label{theorem:向量空间上同一个线性变换在不同基下的表示矩阵必相似}
设\(V\)是数域\(\mathbb{F}\)上的\(n\)维向量空间,\(\varphi\)是\(V\)上的线性变换,\(\{\boldsymbol{e}_1,\boldsymbol{e}_2,\cdots,\boldsymbol{e}_n\}\)和\(\{\boldsymbol{f}_1,\boldsymbol{f}_2,\cdots,\boldsymbol{f}_n\}\)是\(V\)的两组基,从第一组基到第二组基的过渡矩阵为\(\boldsymbol{P}\). 假设\(\varphi\)在第一组基下的表示矩阵为\(\boldsymbol{A}\),在第二组基下的表示矩阵为\(\boldsymbol{B}\),则\(\boldsymbol{B}=\boldsymbol{P}^{-1}\boldsymbol{A}\boldsymbol{P}\),即向量空间上同一个线性变换在不同基下的表示矩阵必相似.
\end{theorem}

\section{线性映射及其运算}

在许多问题中,常常需要定义向量空间之间的线性映射(或某一向量空间上的线性变换). 一般来说,无须对向量空间中的每个元素进行定义,我们可采用下列两种方法来简化定义:第一,只要对向量空间的基向量进行定义即可;第二,若向量空间可分解为两个(或多个)子空间的直和,则只要对每个子空间进行定义即可.这两点可由下面两个定理(\hyperref[theorem:线性扩张定理]{定理\ref{theorem:线性扩张定理}}和定理\hyperref[theorem:利用直和构造线性映射]{定理\ref
{theorem:利用直和构造线性映射}})得到.

\begin{theorem}[线性扩张定理]\label{theorem:线性扩张定理}
设\(V\)和\(U\)是数域\(\mathbb{F}\)上的向量空间,\(\boldsymbol{e}_1,\boldsymbol{e}_2,\cdots,\boldsymbol{e}_n\)是\(V\)的一组基,\(\boldsymbol{u}_1,\boldsymbol{u}_2,\cdots,\boldsymbol{u}_n\)是\(U\)中\(n\)个向量,求证:存在唯一的\(V\)到\(U\)的线性映射\(\varphi\),使得\(\varphi(\boldsymbol{e}_i)=\boldsymbol{u}_i\).
\end{theorem}
\begin{remark}
这个线性扩张定理表明只要选定\(V\)的一组基和\(U\)中\(n\)个向量,则有且仅有一个线性映射将基向量映到对应的向量. 后面我们将经常采用线性扩张定理来构造线性映射以及判定两个线性映射是否相等.
\end{remark}
\begin{note}
这个线性扩张定理表明:

1.在向量空间上定义线性映射只要对向量空间的基向量进行定义即可.

2.判定两个线性映射是否相等只要判断这个两个线性映射原像空间的基向量的像是否相等即可.
\end{note}
\begin{proof}
先证存在性. 对任意的\(\boldsymbol{\alpha}\in V\),设\(\boldsymbol{\alpha}=a_1\boldsymbol{e}_1 + a_2\boldsymbol{e}_2+\cdots + a_n\boldsymbol{e}_n\),则\(a_1,a_2,\cdots,a_n\)被\(\boldsymbol{\alpha}\)唯一确定. 令
\[
\varphi(\boldsymbol{\alpha})=a_1\boldsymbol{u}_1 + a_2\boldsymbol{u}_2+\cdots + a_n\boldsymbol{u}_n,
\]
则\(\varphi\)是\(V\)到\(U\)的映射. 若另有\(\boldsymbol{\beta}=b_1\boldsymbol{e}_1 + b_2\boldsymbol{e}_2+\cdots + b_n\boldsymbol{e}_n\),则
\[
\varphi(\boldsymbol{\alpha}+\boldsymbol{\beta})=(a_1 + b_1)\boldsymbol{u}_1+(a_2 + b_2)\boldsymbol{u}_2+\cdots+(a_n + b_n)\boldsymbol{u}_n=\varphi(\boldsymbol{\alpha})+\varphi(\boldsymbol{\beta}).
\]
又对\(\mathbb{F}\)中的任意元素\(k\),有
\[
\varphi(k\boldsymbol{\alpha})=ka_1\boldsymbol{u}_1 + ka_2\boldsymbol{u}_2+\cdots + ka_n\boldsymbol{u}_n=k\varphi(\boldsymbol{\alpha}).
\]
因此\(\varphi\)是线性映射,显然它满足\(\varphi(\boldsymbol{e}_i)=\boldsymbol{u}_i\).

设另有\(V\)到\(U\)的线性映射\(\psi\)满足\(\psi(\boldsymbol{e}_i)=\boldsymbol{u}_i\),则对任意的\(\boldsymbol{\alpha}\in V\),有
\begin{align*}
\psi(\boldsymbol{\alpha})&=\psi(a_1\boldsymbol{e}_1 + a_2\boldsymbol{e}_2+\cdots + a_n\boldsymbol{e}_n)\\
&=a_1\psi(\boldsymbol{e}_1)+a_2\psi(\boldsymbol{e}_2)+\cdots + a_n\psi(\boldsymbol{e}_n)\\
&=a_1\boldsymbol{u}_1 + a_2\boldsymbol{u}_2+\cdots + a_n\boldsymbol{u}_n=\varphi(\boldsymbol{\alpha}).
\end{align*}
因此\(\psi = \varphi\),这就证明了唯一性. 
\end{proof}

\begin{theorem}\label{theorem:利用直和构造线性映射}
设线性空间\(V = V_1\oplus V_2\),并且\(\varphi_1\)及\(\varphi_2\)分别是\(V_1,V_2\)到\(U\)的线性映射,求证:存在唯一的从\(V\)到\(U\)的线性映射\(\varphi\),当\(\varphi\)限制在\(V_i\)上时等于\(\varphi_i\).
\end{theorem}
\begin{remark}
\hyperref[theorem:利用直和构造线性映射]{定理\ref{theorem:利用直和构造线性映射}}可以推广到多个子空间的情形:设\(V = V_1\oplus\cdots\oplus V_m\),给定线性映射\(\varphi_i:V_i\to U(1\leq i\leq m)\),则存在唯一的线性映射\(\varphi:V\to U\),使得\(\left.\varphi\right|_{V_i}=\varphi_i(1\leq i\leq m)\). 我们可以把这样的线性映射\(\varphi\)简记为\(\varphi_1\oplus\cdots\oplus\varphi_m\).
\end{remark}
\begin{note}
这个定理表明:在向量空间上定义线性映射,若这个向量空间可分解为两个(或多个)子空间的直和,则只要对每个子空间进行定义即可.
\end{note}
\begin{proof}
因为\(V = V_1\oplus V_2\),故对任意的\(\boldsymbol{\alpha}\in V\),\(\boldsymbol{\alpha}\)可唯一地写为\(\boldsymbol{\alpha}=\boldsymbol{\alpha}_1+\boldsymbol{\alpha}_2\),其中\(\boldsymbol{\alpha}_1\in V_1,\boldsymbol{\alpha}_2\in V_2\). 令\(\varphi(\boldsymbol{\alpha})=\varphi_1(\boldsymbol{\alpha}_1)+\varphi_2(\boldsymbol{\alpha}_2)\),则\(\varphi\)是\(V\)到\(U\)的映射. 不难验证\(\varphi\)保持加法和数乘,因此\(\varphi\)是线性映射. 若另有线性映射\(\psi\),它在\(V_i\)上的限制等于\(\varphi_i\),则
\[
\psi(\boldsymbol{\alpha})=\psi(\boldsymbol{\alpha}_1)+\psi(\boldsymbol{\alpha}_2)=\varphi_1(\boldsymbol{\alpha}_1)+\varphi_2(\boldsymbol{\alpha}_2)=\varphi(\boldsymbol{\alpha}).
\]
因此\(\psi = \varphi\),唯一性得证.
\end{proof}

\begin{example}
设\(\varphi\)是有限维线性空间\(V\)到\(U\)的线性映射,求证:必存在\(U\)到\(V\)的线性映射\(\psi\),使得\(\varphi\psi\varphi=\varphi\).
\end{example}
\begin{note}
\hypertarget{可以定义线性映射的原因}{可以直接定义\(\psi\)是\(U\)到\(V\)的线性映射的原因}:由\hyperref[theorem:线性扩张定理]{线性扩张定理}可知存在唯一的线性映射\(\psi\),使得它在基上的作用为
\[
\psi(\boldsymbol{f}_i)=\boldsymbol{e}_i,1\leq i\leq r;\psi(\boldsymbol{f}_j)=\boldsymbol{0},r + 1\leq j\leq m.
\]
以后这种利用\hyperref[theorem:线性扩张定理]{线性扩张定理}得到线性映射的存在性不再额外说明,而是直接定义.
\end{note}
\begin{proof}
{\color{blue}证法一:}
设\(V\)和\(U\)的维数分别是\(n\)和\(m\). 由\hyperref[proposition:线性映射在基下的矩阵为标准型]{命题\ref{proposition:线性映射在基下的矩阵为标准型}}可知,存在\(V\)和\(U\)的基\(\{\boldsymbol{e}_1,\boldsymbol{e}_2,\cdots,\boldsymbol{e}_n\},\{\boldsymbol{f}_1,\boldsymbol{f}_2,\cdots,\boldsymbol{f}_m\}\),使得\(\varphi\)在这两组基下的表示矩阵为
\[
\begin{pmatrix}
\boldsymbol{I}_r&\boldsymbol{O}\\
\boldsymbol{O}&\boldsymbol{O}
\end{pmatrix}.
\]
这就是\(\varphi(\boldsymbol{e}_i)=\boldsymbol{f}_i,1\leq i\leq r;\varphi(\boldsymbol{e}_j)=\boldsymbol{0},r + 1\leq j\leq n\).\hyperlink{可以定义线性映射的原因}{定义\(\psi\)是\(U\)到\(V\)的线性映射},它在基上的作用为
\[
\psi(\boldsymbol{f}_i)=\boldsymbol{e}_i,1\leq i\leq r;\psi(\boldsymbol{f}_j)=\boldsymbol{0},r + 1\leq j\leq m,
\]
则在\(V\)的基上,有
\begin{align*}
\varphi\psi\varphi(\boldsymbol{e}_i)&=\varphi\psi(\boldsymbol{f}_i)=\varphi(\boldsymbol{e}_i),1\leq i\leq r;\\
\varphi\psi\varphi(\boldsymbol{e}_j)&=\varphi\psi(\boldsymbol{0})=\boldsymbol{0}=\varphi(\boldsymbol{e}_j),r + 1\leq j\leq n.
\end{align*}
于是\(\varphi\psi\varphi=\varphi\).

{\color{blue}证法二(代数方法):}取定\(V\)和\(U\)的两组基,设\(\varphi\)在这两组基下的表示矩阵为\(m\times n\)矩阵\(\boldsymbol{A}\),则由\hyperref[example:3.26111]{例题\ref{example:3.26111}}可知,存在\(n\times m\)矩阵\(\boldsymbol{B}\),使得\(\boldsymbol{A}\boldsymbol{B}\boldsymbol{A}=\boldsymbol{A}\). 由矩阵\(\boldsymbol{B}\)可定义从\(U\)到\(V\)的线性映射\(\psi\),它适合\(\varphi\psi\varphi=\varphi\).
\end{proof}

\begin{proposition}
设有数域\(\mathbb{F}\)上的有限维线性空间\(V,V'\),又\(U\)是\(V\)的子空间,\(\varphi\)是\(U\)到\(V'\)的线性映射. 求证:必存在\(V\)到\(V'\)的线性映射\(\psi\),它在\(U\)上的限制就是\(\varphi\).
\end{proposition}
\begin{note}
\hypertarget{直接定义\(\psi\)为\(V\)到\(V'\)的线性映射的原因}{可以直接定义\(\psi\)为\(V\)到\(V'\)的线性映射的原因}:由\hyperref[theorem:利用直和构造线性映射]{定理\ref{theorem:利用直和构造线性映射}}可知,存在唯一的从\(V\)到\(V'\)的线性映射\(\psi\),使得它在\(U\)上的限制是\(\varphi\),它在\(W\)上的限制是零线性映射.

以后这种利用\hyperref[theorem:利用直和构造线性映射]{定理\ref{theorem:利用直和构造线性映射}}得到线性映射的存在性不再额外说明,而是直接定义.
\end{note}
\begin{proof}
令\(W\)是子空间\(U\)在\(V\)中的补空间,即\(V = U\oplus W\). \hypertarget{直接定义\(\psi\)为\(V\)到\(V'\)的线性映射的原因}{定义\(\psi\)为\(V\)到\(V'\)的线性映射},它在\(U\)上的限制是\(\varphi\),它在\(W\)上的限制是零线性映射,这样的\(\psi\)即为所求.     
\end{proof}

\begin{proposition}\label{proposition:线性映射是单射或满射的充要条件1}
设\(V,U\)是\(\mathbb{F}\)上的有限维线性空间,\(\varphi\)是\(V\)到\(U\)的线性映射,求证:
\begin{enumerate}[(1)]
\item  \(\varphi\)是单映射的充要条件是存在\(U\)到\(V\)的线性映射\(\psi\),使\(\psi\varphi = \text{Id}_V\),这里\(\text{Id}_V\)表示\(V\)上的恒等映射;

\item \(\varphi\)是满映射的充要条件是存在\(U\)到\(V\)的线性映射\(\eta\),使\(\varphi\eta = \text{Id}_U\),这里\(\text{Id}_U\)表示\(U\)上的恒等映射.
\end{enumerate}
\end{proposition}
\begin{proof}
{\color{blue}证法一:}
\begin{enumerate}[(1)]
\item 若\(\psi\varphi = \text{Id}_V\),则对任意的\(\boldsymbol{v}\in\text{Ker}\varphi\),\(\boldsymbol{v}=\psi(\varphi(\boldsymbol{v})) = \boldsymbol{0}\),即\(\text{Ker}\varphi = 0\),从而\(\varphi\)是单映射. 

反之,若\(\varphi\)是单映射,则定义映射\(\varphi_1:V\to\text{Im}\varphi\),它与\(\varphi\)有相同的映射法则,但值域变为\(\text{Im}\varphi\). 
因为\(\varphi_1\)与\(\varphi\)有相同的映射法则,所以对\(\forall \alpha \in \mathrm{Ker}\varphi_1\),有\(\varphi(\alpha)=\varphi_1(\alpha)=0\).于是\(\alpha \in \mathrm{Ker}\varphi\),故\(\mathrm{Ker}\varphi_1\subset \mathrm{Ker}\varphi\).又由\(\varphi\)是单射可知,\(\mathrm{Ker}\varphi = 0\).因此\(\mathrm{Ker}\varphi_1 = 0\),即\(\varphi_1\)也是单射.
因为\(\varphi_1\)与\(\varphi\)有相同的映射法则,所以对\(\forall \beta \in \mathrm{Im}\varphi\),存在\(b \in V\),使得\(\beta = \varphi(b)=\varphi_1(b) \in \mathrm{Im}\varphi_1\).故\(\mathrm{Im}\varphi\subset \mathrm{Im}\varphi_1\).又根据\(\varphi_1\)的定义可知,\(\mathrm{Im}\varphi_1\subset \mathrm{Im}\varphi\).因此\(\mathrm{Im}\varphi = \mathrm{Im}\varphi_1\),即\(\varphi_1\)是满射.
故\(\varphi_1\)是双射.由于\(\varphi_1\)与\(\varphi\)有相同的映射法则,因此容易验证\(\varphi_1\)是线性映射.综上,\(\varphi_1\)是线性同构.

设\(U_0\)是\(\text{Im}\varphi\)在\(U\)中的补空间,即\(U = \text{Im}\varphi\oplus U_0\). 定义\(\psi\)为\(U\)到\(V\)的线性映射,它在\(\text{Im}\varphi\)上的限制为\(\varphi_1^{-1}\),它在\(U_0\)上的限制是零线性映射,则容易验证\(\psi\varphi = \text{Id}_V\)成立.

\item 若\(\varphi\eta = \text{Id}_U\),则对任意的\(\boldsymbol{u}\in U\),\(\boldsymbol{u}=\varphi(\eta(\boldsymbol{u}))\),从而\(\varphi\)是满映射. 

反之,若\(\varphi\)是满映射,则可取\(U\)的一组基\(\boldsymbol{f}_1,\boldsymbol{f}_2,\cdots,\boldsymbol{f}_m\),一定存在\(V\)中的向量\(\boldsymbol{v}_1,\boldsymbol{v}_2,\cdots,\boldsymbol{v}_m\),使得\(\varphi(\boldsymbol{v}_i)=\boldsymbol{f}_i(1\leq i\leq m)\). 定义\(\eta\)为\(U\)到\(V\)的线性映射,它在基上的作用为\(\eta(\boldsymbol{f}_i)=\boldsymbol{v}_i(1\leq i\leq m)\),则容易验证\(\varphi\eta = \text{Id}_U\)成立.
\end{enumerate}
{\color{blue}证法二(代数方法):}

充分性同{\color{blue}证法一},现只证必要性. 取定\(V\)和\(U\)的两组基,设\(\varphi\)在这两组基下的表示矩阵为\(m\times n\)矩阵\(\boldsymbol{A}\).
\begin{enumerate}[(1)]
\item 若\(\varphi\)是单映射,则由\hyperref[proposition:行/列满秩矩阵对应满/单射]{命题\ref{proposition:行/列满秩矩阵对应满/单射}}可知\(\boldsymbol{A}\)是列满秩矩阵. 再由\hyperref[proposition:行/列满秩矩阵性质]{命题\ref{proposition:行/列满秩矩阵性质}(1)}可知,存在\(n\times m\)矩阵\(\boldsymbol{B}\),使得\(\boldsymbol{B}\boldsymbol{A}=\boldsymbol{I}_n\). 由矩阵\(\boldsymbol{B}\)可定义从\(U\)到\(V\)的线性映射\(\psi\),它适合\(\psi\varphi=\text{Id}_V\).

\item 若\(\varphi\)是满映射,则由\hyperref[proposition:行/列满秩矩阵对应满/单射]{命题\ref{proposition:行/列满秩矩阵对应满/单射}}可知\(\boldsymbol{A}\)是行满秩矩阵. 再由\hyperref[proposition:行/列满秩矩阵性质]{命题\ref{proposition:行/列满秩矩阵性质}(2)}可知,存在\(n\times m\)矩阵\(\boldsymbol{C}\),使得\(\boldsymbol{A}\boldsymbol{C}=\boldsymbol{I}_m\). 由矩阵\(\boldsymbol{C}\)可定义从\(U\)到\(V\)的线性映射\(\eta\),它适合\(\varphi\eta=\text{Id}_U\).
\end{enumerate}
\end{proof}

\begin{proposition}\label{proposition:核空间和值域与线性映射}
\begin{enumerate}[(1)]
\item 设 \(V,U\) 是数域 \(\mathbb{K}\) 上的有限维线性空间,\(\varphi,\psi:V\to U\) 是两个线性映射,证明:存在 \(U\) 上的线性变换 \(\xi\),使得 \(\psi = \xi\varphi\) 成立的充要条件是 \(\text{Ker}\varphi\subseteq\text{Ker}\psi\).

\item 设 \(V,U\) 是数域 \(\mathbb{K}\) 上的有限维线性空间,\(\varphi,\psi:V\to U\) 是两个线性映射,证明:存在 \(V\) 上的线性变换 \(\xi\),使得 \(\psi = \varphi\xi\) 成立的充要条件是 \(\text{Im}\psi\subseteq\text{Im}\varphi\).
\end{enumerate}
\end{proposition}
\begin{proof}
\begin{enumerate}[(1)]
\item 先证必要性:任取 \(\boldsymbol{v}\in\text{Ker}\varphi\),则 \(\psi(\boldsymbol{v}) = \xi\varphi(\boldsymbol{v}) = \boldsymbol{0}\),即有 \(\boldsymbol{v}\in\text{Ker}\psi\),从而 \(\text{Ker}\varphi\subseteq\text{Ker}\psi\). 

再证充分性:设 \(\dim V = n,\dim U = m,\dim\text{Ker}\varphi = n - r\). 取 \(\text{Ker}\varphi\) 的一组基 \(\boldsymbol{e}_{r + 1},\cdots,\boldsymbol{e}_n\),扩张为 \(V\) 的一组基 \(\boldsymbol{e}_1,\cdots,\boldsymbol{e}_r,\boldsymbol{e}_{r + 1},\cdots,\boldsymbol{e}_n\). 由\hyperref[corollary:由核的基导出值域的基]{推论\ref{corollary:由核的基导出值域的基}}可知,\(\varphi(\boldsymbol{e}_1),\cdots,\varphi(\boldsymbol{e}_r)\) 是 \(\text{Im}\varphi\) 的一组基,将其扩张为 \(U\) 的一组基 \(\varphi(\boldsymbol{e}_1),\cdots,\varphi(\boldsymbol{e}_r),\boldsymbol{g}_{r + 1},\cdots,\boldsymbol{g}_m\). 定义 \(\xi\) 为 \(U\) 上的线性变换,它在基上的作用为:\(\xi(\varphi(\boldsymbol{e}_i)) = \psi(\boldsymbol{e}_i)(1\leq i\leq r),\xi(\boldsymbol{g}_j) = \boldsymbol{0}(r + 1\leq j\leq m)\). 由于 \(\text{Ker}\varphi\subseteq\text{Ker}\psi\),故容易验证 \(\psi(\boldsymbol{e}_i) = \xi\varphi(\boldsymbol{e}_i)(1\leq i\leq n)\) 成立,从而 \(\psi = \xi\varphi\).

\item 先证必要性:任取 \(\boldsymbol{v}\in V\),则 \(\psi(\boldsymbol{v}) = \varphi(\xi(\boldsymbol{v}))\in\text{Im}\varphi\),从而 \(\text{Im}\psi\subseteq\text{Im}\varphi\). 

再证充分性:取 \(V\) 的一组基 \(\boldsymbol{e}_1,\boldsymbol{e}_2,\cdots,\boldsymbol{e}_n\),则 \(\psi(\boldsymbol{e}_i)\in\text{Im}\psi\subseteq\text{Im}\varphi\),从而存在 \(\boldsymbol{f}_i\in V\),使得 \(\varphi(\boldsymbol{f}_i) = \psi(\boldsymbol{e}_i)(1\leq i\leq n)\). 定义 \(\xi\) 为 \(V\) 上的线性变换,它在基上的作用为:\(\xi(\boldsymbol{e}_i) = \boldsymbol{f}_i(1\leq i\leq n)\). 容易验证 \(\psi(\boldsymbol{e}_i) = \varphi\xi(\boldsymbol{e}_i)(1\leq i\leq n)\) 成立,从而 \(\psi = \varphi\xi\).
\end{enumerate}
\end{proof}

\begin{proposition}\label{proposition:幂零线性变换的一组基}
设\(\varphi\)是\(n\)维线性空间\(V\)上的线性变换,\(\boldsymbol{\alpha}\in V\). 若\(\varphi^{m - 1}(\boldsymbol{\alpha})\neq\boldsymbol{0}\),而\(\varphi^{m}(\boldsymbol{\alpha})=\boldsymbol{0}\),求证:\(\boldsymbol{\alpha},\varphi(\boldsymbol{\alpha}),\varphi^{2}(\boldsymbol{\alpha}),\cdots,\varphi^{m - 1}(\boldsymbol{\alpha})\)线性无关.
\end{proposition}
\begin{proof}
设有\(m\)个数\(a_0,a_1,\cdots,a_{m - 1}\),使得
\[
a_0\boldsymbol{\alpha}+a_1\varphi(\boldsymbol{\alpha})+\cdots+a_{m - 1}\varphi^{m - 1}(\boldsymbol{\alpha})=\boldsymbol{0}.
\]
上式两边同时作用\(\varphi^{m - 1}\),则有\(a_0\varphi^{m - 1}(\boldsymbol{\alpha})=\boldsymbol{0}\),由于\(\varphi^{m - 1}(\boldsymbol{\alpha})\neq\boldsymbol{0}\),故\(a_0 = 0\). 上式两边同时作用\(\varphi^{m - 2}\),则有\(a_1\varphi^{m - 1}(\boldsymbol{\alpha})=\boldsymbol{0}\),由于\(\varphi^{m - 1}(\boldsymbol{\alpha})\neq\boldsymbol{0}\),故\(a_1 = 0\). 不断这样做下去,最后可得\(a_0 = a_1=\cdots=a_{m - 1}=0\),于是\(\boldsymbol{\alpha},\varphi(\boldsymbol{\alpha}),\varphi^{2}(\boldsymbol{\alpha}),\cdots,\varphi^{m - 1}(\boldsymbol{\alpha})\)线性无关.
\end{proof}

\begin{corollary}\label{corollary:幂零线性变换基的表示矩阵}
设\(V\)是数域\(\mathbb{K}\)上的\(n\)维线性空间,\(\varphi\)是\(V\)上的幂零线性变换,满足\(\mathrm{r}(\varphi)=n - 1\). 求证:存在\(V\)的一组基,使得\(\varphi\)在这组基下的表示矩阵为
\[
\boldsymbol{A}=\begin{pmatrix}
0&0&\cdots&0&0\\
1&0&\cdots&0&0\\
0&1&\cdots&0&0\\
\vdots&\vdots&&\vdots&\vdots\\
0&0&\cdots&1&0
\end{pmatrix}.
\]
\end{corollary}
\begin{proof}
由假设存在正整数\(m\),使得\(\varphi^{m}=0,\varphi^{m - 1}\neq 0\),从而存在\(\boldsymbol{\alpha}\in V\),使得\(\varphi^{m}(\boldsymbol{\alpha}) = 0,\varphi^{m - 1}(\boldsymbol{\alpha})\neq 0\). 由\hyperref[proposition:幂零线性变换的一组基]{命题\ref{proposition:幂零线性变换的一组基}}可知,\(\boldsymbol{\alpha},\varphi(\boldsymbol{\alpha}),\cdots,\varphi^{m - 1}(\boldsymbol{\alpha})\)线性无关,于是\(m\leq\dim V=n\). 另一方面,由\hyperref[proposition:Sylvester不等式]{Sylvester不等式}以及\(\mathrm{r}(\varphi)=n - 1\)可知,\(\mathrm{r}(\varphi^{2})\geq 2\mathrm{r}(\varphi)-n=n - 2\). 不断这样讨论下去,最终可得\(0=\mathrm{r}(\varphi^{m})\geq n - m\),即有\(m\geq n\),从而\(m = n\). 于是\(\boldsymbol{\alpha},\varphi(\boldsymbol{\alpha}),\cdots,\varphi^{n - 1}(\boldsymbol{\alpha})\)是\(V\)的一组基,\(\varphi\)在这组基下的表示矩阵为\(\boldsymbol{A}\).
\end{proof}



\section{线性同构}
线性同构刻画了不同线性空间之间的相同本质,即同构的线性空间具有相同的线性结构(或从线性结构的观点来看没有任何区别).要证明线性映射\(\varphi:V\to U\)是线性同构,通常一方面需要验证\(\varphi\)是单映射(或等价地验证\(\text{Ker}\varphi = 0\)),另一方面需要验证\(\varphi\)是满映射(或等价地验证\(\text{Im}\varphi = U\)). 但若已知前后两个线性空间的维数相等,则由线性映射的维数公式容易证明,\(\varphi\)是线性同构当且仅当\(\varphi\)是单映射,也当且仅当\(\varphi\)是满映射,从而只需验证\(\varphi\)是单映射或满映射即可得到\(\varphi\)是线性同构.

\begin{proposition}\label{proposition:像和原像空间维数相同时线性同构的充要条件}
设$V,U$为两个线性空间,若\(\dim V = \dim U\),则线性映射\(\varphi:V\rightarrow U\)是线性同构当且仅当\(\varphi\)是单映射,也当且仅当\(\varphi\)是满映射.
\end{proposition}
\begin{proof}
由线性映射的维数公式容易证明.
\end{proof}

\begin{corollary}\label{corollary:线性变换自同构的充要条件}
设$V$为线性空间,则线性变换\(\varphi:V\rightarrow V\)是自同构当且仅当\(\varphi\)是单映射,也当且仅当\(\varphi\)是满映射.
\end{corollary}
\begin{proof}
由\hyperref[proposition:像和原像空间维数相同时线性同构的充要条件]{命题\ref{proposition:像和原像空间维数相同时线性同构的充要条件}}立得.
\end{proof}

\begin{lemma}\label{lemma:证明Lagrange插值公式}
设\(a_0,a_1,\cdots,a_n\)是数域\(\mathbb{F}\)中\(n + 1\)个不同的数,\(V\)是\(\mathbb{F}\)上次数不超过\(n\)的多项式全体组成的线性空间. 设\(\varphi\)是\(V\)到\(n + 1\)维行向量空间\(U\)的映射:
\[
\varphi(f)=(f(a_0),f(a_1),\cdots,f(a_n)),
\]
求证:\(\varphi\)是线性同构.
\end{lemma}
\begin{proof}
不难验证\(\varphi\)是一个线性映射. 若\(f(x)\in\text{Ker}\varphi\),则\(f(a_i)=0(0\leq i\leq n)\). 因为\(f(x)\)的次数不超过\(n\),故由\hyperref[proposition:多项式根的有限性]{多项式根的有限性}可知\(f(x)=0\),即\(\text{Ker}\varphi = 0\),这证明了映射\(\varphi\)是单映射. 注意到线性空间\(V\)和\(U\)的维数都等于\(n + 1\),因此由\hyperref[proposition:像和原像空间维数相同时线性同构的充要条件]{命题\ref{proposition:像和原像空间维数相同时线性同构的充要条件}}可知\(\varphi\)是线性同构.
\end{proof}

\begin{theorem}[Lagrange插值公式]\label{theorem:Lagrange插值公式}
设\(a_0,a_1,\cdots,a_n\)是数域\(\mathbb{F}\)中\(n + 1\)个不同的数,\(b_0,b_1,\cdots,b_n\)是\(\mathbb{F}\)中任意\(n + 1\)个数,求证:必存在\(\mathbb{F}\)上次数不超过\(n\)的多项式\(f(x)\),使得\(f(a_i)=b_i(0\leq i\leq n)\),并将\(f(x)\)构造出来.
\end{theorem}
\begin{proof}
由\hyperref[lemma:证明Lagrange插值公式]{引理\ref{lemma:证明Lagrange插值公式}}可知映射\(\varphi\)是映上的(满射),因此存在性已经证明. 现来构造\(f(x)\). 设\(\boldsymbol{e}_i=(0,\cdots,1,\cdots,0)(1\leq i\leq n + 1)\)是\(\mathbb{F}\)上的\(n + 1\)维标准单位行向量. 对任意的\(0\leq i\leq n\),令
\[
f_i(x)=\frac{(x - a_0)\cdots(x - a_{i - 1})(x - a_{i + 1})\cdots(x - a_n)}{(a_i - a_0)\cdots(a_i - a_{i - 1})(a_i - a_{i + 1})\cdots(a_i - a_n)},
\]
则\(f_i(a_i)=1,f_i(a_j)=0(j\neq i)\),于是\(\varphi(f_i)=\boldsymbol{e}_{i + 1}(0\leq i\leq n)\). 再令
\[
f(x)=b_0f_0(x)+b_1f_1(x)+\cdots + b_nf_n(x),
\]
则容易验证\(\varphi(f)=(b_0,b_1,\cdots,b_n)\),即\(f(a_i)=b_i(0\leq i\leq n)\)成立. 
\end{proof}

\subsection{证明线性变换可逆的方法}

要证明某个有限维线性空间\(V\)上的线性变换\(\varphi\)是自同构(可逆线性变换),通常有\(3\)种方法. 一是可尝试直接构造出\(\varphi\)的逆变换. 二是证明\(\varphi\)是单映射或者\(\varphi\)是满映射(两者只需其一)(\hyperref[proposition:像和原像空间维数相同时线性同构的充要条件]{命题\ref{proposition:像和原像空间维数相同时线性同构的充要条件}}). 三是用矩阵方法,即选取\(V\)的一组基,设\(\varphi\)在这组基下的表示矩阵为\(\boldsymbol{A}\),设法证明\(\boldsymbol{A}\)是可逆矩阵. 

对于无限维线性空间之间的线性映射,我们并没有定义表示矩阵这一概念,也没有维数公式等结论,因此研究线性映射或线性变换,无限维线性空间的情形远比有限维线性空间的情形难得多,也常出现对有限维线性空间成立的结论在无限维线性空间却不成立的情况. 例如,要证明无限维线性空间上的线性变换是自同构,只能按照定义证明它既是单映射又是满映射,而不能像有限维线性空间上的线性变换那样,只验证它是单映射或满映射即可.
\begin{example}
设\(\varphi\)是数域\(\mathbb{F}\)上线性空间\(V\)上的线性变换,若存在正整数\(n\)以及\(a_1,a_2,\cdots,a_n\in\mathbb{F}\),使得
\[
\varphi^{n}+a_1\varphi^{n - 1}+\cdots+a_{n - 1}\varphi+a_nI_V = 0,
\]
其中\(I_V\)表示恒等变换并且\(a_n\neq 0\),求证:\(\varphi\)是\(V\)上的自同构.
\end{example}
\begin{proof}
由条件可得
\[
\varphi^{n}+a_1\varphi^{n - 1}+\cdots+a_{n - 1}\varphi=-a_nI_V,
\]
从而
\[
\varphi\left(-\frac{1}{a_n}(\varphi^{n - 1}+\cdots+a_{n - 1}I_V)\right)=I_V,
\]
于是
\[
\varphi^{-1}=-\frac{1}{a_n}(\varphi^{n - 1}+\cdots+a_{n - 1}I_V). 
\]
\end{proof}

\begin{proposition}\label{proposition:线性变换是可逆变换的充要条件1}
设\(\varphi\)是\(n\)维线性空间\(V\)上的线性变换,证明:\(\varphi\)是可逆变换的充要条件是\(\varphi\)将\(V\)的基变为基.
\end{proposition}
\begin{proof}
若\(\varphi\)是可逆变换,则显然\(\varphi\)将\(V\)的基变为基.

反之,{\color{blue}证法一:}
若\(\boldsymbol{e}_1,\boldsymbol{e}_2,\cdots,\boldsymbol{e}_n\)和\(\boldsymbol{f}_1,\boldsymbol{f}_2,\cdots,\boldsymbol{f}_n\)是\(V\)的两组基,使得\(\varphi(\boldsymbol{e}_i)=\boldsymbol{f}_i(1\leq i\leq n)\),则对任意\(\boldsymbol{\alpha}\in V\),\(\boldsymbol{\alpha}=\lambda_1\boldsymbol{f}_1+\lambda_2\boldsymbol{f}_2+\cdots+\lambda_n\boldsymbol{f}_n\),有\(\varphi(\lambda_1\boldsymbol{e}_1+\lambda_2\boldsymbol{e}_2+\cdots+\lambda_n\boldsymbol{e}_n)=\boldsymbol{\alpha}\),即\(\varphi\)是满映射,从而是自同构.(我们也可以证明\(\varphi\)是单映射,从而是自同构.) 

{\color{blue}证法二:}若\(\boldsymbol{e}_1,\boldsymbol{e}_2,\cdots,\boldsymbol{e}_n\)和\(\boldsymbol{f}_1,\boldsymbol{f}_2,\cdots,\boldsymbol{f}_n\)是\(V\)的两组基,使得\(\varphi(\boldsymbol{e}_i)=\boldsymbol{f}_i(1\leq i\leq n)\).设从基\(\boldsymbol{e}_1,\boldsymbol{e}_2,\cdots,\boldsymbol{e}_n\)到基\(\boldsymbol{f}_1,\boldsymbol{f}_2,\cdots,\boldsymbol{f}_n\)的过渡矩阵为\(\boldsymbol{P}\),则\(\varphi\)在基\(\boldsymbol{e}_1,\boldsymbol{e}_2,\cdots,\boldsymbol{e}_n\)下的表示矩阵就是\(\boldsymbol{P}\),这是一个可逆矩阵,从而\(\varphi\)是可逆变换. 
\end{proof}

\begin{proposition}\label{proposition:两个维数相同的线性空间一定可以通过一个可逆线性变换联系起来}
设\(U_1,U_2\)是\(n\)维线性空间\(V\)的子空间,假设它们维数相同. 求证:存在\(V\)上的可逆线性变换\(\varphi\),使得\(U_2 = \varphi(U_1)\).
\end{proposition}
\begin{proof}
取\(U_1\)的一组基\(\boldsymbol{e}_1,\cdots,\boldsymbol{e}_m\),并扩张为\(V\)的一组基\(\boldsymbol{e}_1,\cdots,\boldsymbol{e}_m,\boldsymbol{e}_{m + 1},\cdots,\boldsymbol{e}_n\);取\(U_2\)的一组基\(\boldsymbol{f}_1,\cdots,\boldsymbol{f}_m\),并扩张为\(V\)的一组基\(\boldsymbol{f}_1,\cdots,\boldsymbol{f}_m,\boldsymbol{f}_{m + 1},\cdots,\boldsymbol{f}_n\). 定义\(\varphi\)为\(V\)上的线性变换,它在基上的作用为:\(\varphi(\boldsymbol{e}_i)=\boldsymbol{f}_i(1\leq i\leq n)\),则由\hyperref[proposition:线性变换是可逆变换的充要条件1]{命题\ref{proposition:线性变换是可逆变换的充要条件1}}可知,\(\varphi\)是可逆线性变换,再由定义容易验证\(\varphi(U_1)=U_2\)成立. 
\end{proof}

\begin{example}
设\(\varphi\)是\(n\)维线性空间\(V\)上的线性变换,若对\(V\)中任一向量\(\boldsymbol{\alpha}\),总存在正整数\(m\)(\(m\)可能和\(\boldsymbol{\alpha}\)有关),使得\(\varphi^{m}(\boldsymbol{\alpha}) = 0\). 求证:\(I_V-\varphi\)是自同构.
\end{example}
\begin{proof}
{\color{blue}证法一:}
首先证明线性变换\(\varphi\)是幂零的. 设\(\boldsymbol{e}_1,\boldsymbol{e}_2,\cdots,\boldsymbol{e}_n\)是线性空间\(V\)的一组基. 对每个\(\boldsymbol{e}_i\),都有\(m_i\),使得\(\varphi^{m_i}(\boldsymbol{e}_i)=0\),令\(m\)为诸\(m_i\)中最大者. 对\(V\)中任一向量\(\boldsymbol{v}\),设\(\boldsymbol{v}=a_1\boldsymbol{e}_1 + a_2\boldsymbol{e}_2+\cdots + a_n\boldsymbol{e}_n\),则有
\[
\varphi^{m}(\boldsymbol{v})=a_1\varphi^{m}(\boldsymbol{e}_1)+a_2\varphi^{m}(\boldsymbol{e}_2)+\cdots + a_n\varphi^{m}(\boldsymbol{e}_n)=\boldsymbol{0}.
\]
因此\(\varphi^{m}=0\).

注意到下列等式:
\[
(I_V - \varphi)(I_V+\varphi+\varphi^{2}+\cdots+\varphi^{m - 1})=I_V-\varphi^{m}=I_V.
\]
由此即知\(I_V - \varphi\)是自同构.

{\color{blue}证法二:} 只要证明\(I_V - \varphi\)是单映射即可. 任取\(\boldsymbol{\alpha}\in\text{Ker}(I_V - \varphi)\),即\((I_V - \varphi)(\boldsymbol{\alpha}) = 0\),则\(\varphi(\boldsymbol{\alpha})=\boldsymbol{\alpha}\). 设\(m\)为正整数,使得\(\varphi^{m}(\boldsymbol{\alpha}) = 0\),则\(0=\varphi^{m}(\boldsymbol{\alpha})=\varphi^{m - 1}(\boldsymbol{\alpha})=\cdots=\varphi(\boldsymbol{\alpha})=\boldsymbol{\alpha}\),故\(\text{Ker}(I_V - \varphi)=0\),即\(I_V - \varphi\)是单映射. 
\end{proof}

\begin{example}
设\(V = M_n(\mathbb{F})\)是\(\mathbb{F}\)上\(n\)阶矩阵全体组成的线性空间,\(\boldsymbol{A},\boldsymbol{B}\)是两个\(n\)阶矩阵,定义\(V\)上的变换:\(\varphi(\boldsymbol{X})=\boldsymbol{A}\boldsymbol{X}\boldsymbol{B}\). 求证:\(\varphi\)是\(V\)上的线性变换,\(\varphi\)是可逆变换的充要条件是\(\boldsymbol{A}\)和\(\boldsymbol{B}\)都是可逆矩阵.
\end{example}
\begin{remark}
用\hyperref[proposition:无限维线性空间的可逆线性变换充要条件1]{命题\ref{proposition:无限维线性空间的可逆线性变换充要条件1}}的结论来看这个例题,就能发现\(D\)之所以不是可逆变换,是因为它的右逆变换除了\(S\)之外,还有无穷多个.
\end{remark}
\begin{proof}
容易验证\(\varphi\)是线性变换. 若\(\boldsymbol{A},\boldsymbol{B}\)都是可逆矩阵,则\(\psi(\boldsymbol{X})=\boldsymbol{A}^{-1}\boldsymbol{X}\boldsymbol{B}^{-1}\)是\(\varphi\)的逆线性变换. 下面用两种方法来证明必要性.

{\color{blue}证法一:} 若\(\boldsymbol{A}\)是不可逆矩阵,则我们可证明\(\varphi\)不是单映射,即存在\(\boldsymbol{X}\neq\boldsymbol{O}\),使得\(\varphi(\boldsymbol{X})=\boldsymbol{A}\boldsymbol{X}\boldsymbol{B}=\boldsymbol{O}\),从而\(\varphi\)不是可逆变换. 事实上,若\(\boldsymbol{A}\)的秩等于\(r < n\),则存在可逆矩阵\(\boldsymbol{P}\)和\(\boldsymbol{Q}\),使得\(\boldsymbol{P}\boldsymbol{A}\boldsymbol{Q}=\begin{pmatrix}\boldsymbol{I}_r&\boldsymbol{O}\\\boldsymbol{O}&\boldsymbol{O}\end{pmatrix}\). 令\(\boldsymbol{C}=\begin{pmatrix}\boldsymbol{O}&\boldsymbol{O}\\\boldsymbol{O}&\boldsymbol{I}_{n - r}\end{pmatrix}\),则\(\boldsymbol{P}\boldsymbol{A}\boldsymbol{Q}\boldsymbol{C}=\boldsymbol{O}\),而\(\boldsymbol{P}\)是可逆矩阵,故\(\boldsymbol{A}\boldsymbol{Q}\boldsymbol{C}=\boldsymbol{O}\),再令\(\boldsymbol{X}=\boldsymbol{Q}\boldsymbol{C}\)即可. 同理,若\(\boldsymbol{B}\)的秩小于\(n\),也可以证明\(\varphi\)不是可逆变换.

{\color{blue}证法二:} 若\(\boldsymbol{A}\)是不可逆矩阵,则对任意的\(n\)阶矩阵\(\boldsymbol{X}\),\(\varphi(\boldsymbol{X})=\boldsymbol{A}\boldsymbol{X}\boldsymbol{B}\)总是不可逆矩阵(行列式一定为零). 因此\(\varphi\)不可能是映上的(可逆矩阵不在值域里但是在像空间中). 同理,若\(\boldsymbol{B}\)是不可逆矩阵,\(\varphi\)也不是映上的.
\end{proof}

\begin{example}
设\(V\)是实系数多项式全体构成的实线性空间,定义\(V\)上的变换\(D,S\)如下:
\[
D(f(x))=\frac{\mathrm{d}}{\mathrm{d}x}f(x),\ S(f(x))=\int_{0}^{x}f(t)\mathrm{d}t.
\]
证明:\(D,S\)均为\(V\)上的线性变换且\(DS = I_V\),但\(SD\neq I_V\).
\end{example}
\begin{proof}
简单验证即得结论. 由\(DS = I_V\)可知,\(S\)是单线性映射,\(D\)是满线性映射. 又容易看出\(S\)不是满映射(值域不包含常数),\(D\)不是单映射,从而它们都不是自同构.
\end{proof}

\begin{proposition}\label{proposition:无限维线性空间的可逆线性变换充要条件1}
设\(V\)是\(\mathbb{K}\)上的无限维线性空间,\(\varphi,\psi\)是\(V\)上的线性变换.
\begin{enumerate}[(1)]
\item 证明:\(\varphi\)和\(\psi\)都是可逆变换的充要条件是\(\varphi\psi\)和\(\psi\varphi\)都是可逆变换;

\item 若\(\psi\varphi = I_V\),则称\(\psi\)是\(\varphi\)的左逆变换,\(\varphi\)是\(\psi\)的右逆变换. 证明:\(\varphi\)是可逆变换的充要条件是\(\varphi\)有且仅有一个左逆变换(右逆变换).
\end{enumerate}
\end{proposition}
\begin{note}
这个命题对有限维空间仍成立.
\end{note}
\begin{proof}
\begin{enumerate}[(1)]
\item 若\(\varphi\)和\(\psi\)都是可逆变换,则\((\psi^{-1}\varphi^{-1})(\varphi\psi)=(\varphi\psi)(\psi^{-1}\varphi^{-1}) = I_V\),\((\varphi^{-1}\psi^{-1})(\psi\varphi)=(\psi\varphi)(\varphi^{-1}\psi^{-1}) = I_V\),因此\(\varphi\psi\)和\(\psi\varphi\)都是可逆变换. 反之,若\(\varphi\psi\)和\(\psi\varphi\)都是可逆变换,则存在\(V\)上的线性变换\(\xi,\eta\),使得\(\varphi\psi\xi=\xi\varphi\psi = I_V\),\(\psi\varphi\eta=\eta\psi\varphi = I_V\). 由\(\varphi\psi\xi = I_V\)及\hyperref[proposition:值域和核空间维数之和等于原像空间维数]{命题\ref{proposition:值域和核空间维数之和等于原像空间维数}(2)}可得\(\varphi\)是满映射,由\(\eta\psi\varphi = I_V\)及\hyperref[proposition:值域和核空间维数之和等于原像空间维数]{命题\ref{proposition:值域和核空间维数之和等于原像空间维数}(1)}可得\(\varphi\)是单映射,从而\(\varphi\)是可逆变换. 同理可证\(\psi\)也是可逆变换.

\item 若\(\varphi\)是可逆变换,任取\(\varphi\)的一个左逆变换\(\psi\),则
\[
\psi=\psi I_V=\psi\varphi\varphi^{-1}=I_V\varphi^{-1}=\varphi^{-1},
\]
即\(\varphi\)的任一左逆变换都是逆变换\(\varphi^{-1}\). 由逆变换的唯一性可知,\(\varphi\)有且仅有一个左逆变换. 反之,若\(\varphi\)有且仅有一个左逆变换\(\psi\),则\(\psi\varphi = I_V\),且有
\[
(\psi+\varphi\psi - I_V)\varphi=\psi\varphi+\varphi\psi\varphi-\varphi=I_V+\varphi-\varphi=I_V,
\]
即\(\psi+\varphi\psi - I_V\)也是\(\varphi\)的左逆变换,从而\(\psi+\varphi\psi - I_V=\psi\),即\(\varphi\psi = I_V\). 因此\(\psi\)也是\(\varphi\)的右逆变换,从而\(\varphi\)是可逆变换. 同理可证关于右逆变换的结论.
\end{enumerate}
\end{proof}

\begin{example}
试构造无限维线性空间\(V\)以及\(V\)上的线性变换\(\varphi,\psi\),使得\(\varphi\psi-\psi\varphi = I_V\).
\end{example}
\begin{solution}
设\(V\)是实系数多项式全体构成的实线性空间,线性变换\(\varphi,\psi\)定义为:对任一\(f(x)\in V\),\(\varphi(f(x)) = f^\prime(x)\),\(\psi(f(x)) = xf(x)\). 容易验证\(\varphi\psi-\psi\varphi = I_V\)成立.
\end{solution}
\begin{remark}
事实上,满足上述性质的线性变换\(\varphi,\psi\)绝不可能存在于有限维线性空间\(V\)上. 若存在,取\(V\)的一组基并设\(\varphi,\psi\)的表示矩阵为\(\boldsymbol{A},\boldsymbol{B}\),则有\(\boldsymbol{A}\boldsymbol{B}-\boldsymbol{B}\boldsymbol{A}=\boldsymbol{I}\)成立. 上式两边同时取迹,可得
\[
0=\text{tr}(\boldsymbol{A}\boldsymbol{B}-\boldsymbol{B}\boldsymbol{A})=\text{tr}(\boldsymbol{I})=\dim V,
\]
导出矛盾.
\end{remark}

\section{线性映射与矩阵}

线性映射与矩阵的关系是这一章的核心. 线性映射是一个几何概念,矩阵是一个代数概念,它们之间的关系需要掌握以下几点:
\begin{enumerate}[(1)]
\item\label{线性映射与矩阵基本结论1} 记数域\(\mathbb{F}\)上\(n\)维向量空间\(V\)到\(m\)维向量空间\(U\)的线性映射全体为\(\mathcal{L}(V,U)\),\(\mathbb{F}\)上\(m\times n\)矩阵全体为\(M_{m\times n}(\mathbb{F})\). 各自取定\(V\)和\(U\)的一组基,设\(\varphi\in\mathcal{L}(V,U)\)在给定基下的表示矩阵为\(\boldsymbol{A}\),则\(\varphi\mapsto\boldsymbol{A}\)定义了从\(\mathcal{L}(V,U)\)到\(M_{m\times n}(\mathbb{F})\)的一一对应,这个对应还是一个线性同构. 若\(m = n\),则在这个对应下,线性同构(可逆线性映射)对应于可逆矩阵. 特别地,若\(V = U\),上述对应还定义了一个代数同构,即除了保持加法与数乘外,还保持乘法. 因此,两个向量空间之间线性映射的运算完全可以归结为矩阵的运算.

\item\label{线性映射与矩阵基本结论2} 设线性映射\(\varphi\)在给定基下的表示矩阵为\(\boldsymbol{A}\),则\(\text{Ker}\varphi\)和齐次线性方程组\(\boldsymbol{A}\boldsymbol{x}=0\)的解空间同构,\(\text{Im}\varphi\)和\(\boldsymbol{A}\)的全体列向量张成的向量空间同构.因此$\mathrm{dim}\,\mathrm{Im}\varphi =\mathrm{r}\left( \varphi \right) =\mathrm{r}\left( \boldsymbol{A} \right) ,\mathrm{dim}\,\mathrm{Ker}\varphi =n-\mathrm{r}\left( \varphi \right) =n-\mathrm{r}\left( \boldsymbol{A} \right)$.
\end{enumerate}
这两点由\hyperref[theorem:线性映射与矩阵基本定理]{定理\ref{theorem:线性映射与矩阵基本定理}}的结论即得.

\begin{theorem}\label{theorem:线性映射与矩阵基本定理}
设\(\varphi\)是数域\(\mathbb{F}\)上\(n\)维线性空间\(V\)到\(m\)维线性空间\(U\)的线性映射. 令\(\mathbb{F}^n\)和\(\mathbb{F}^m\)分别是\(\mathbb{F}\)上\(n\)维和\(m\)维列向量空间. 又设\(\boldsymbol{e}_1,\boldsymbol{e}_2,\cdots,\boldsymbol{e}_n\)和
\(\boldsymbol{f}_1,\boldsymbol{f}_2,\cdots,\boldsymbol{f}_m\)分别是\(V\)和\(U\)的基,\(\varphi\)在给定基下的表示矩阵为\(\boldsymbol{A}\). 记\(\eta_1:V\to\mathbb{F}^n\)为\(V\)中向量映射到它在基\(\boldsymbol{e}_1,\boldsymbol{e}_2,\cdots,\boldsymbol{e}_n\)下的坐标向量的线性同构,\(\eta_2:U\to\mathbb{F}^m\)为\(U\)中向量映射到它在基\(\boldsymbol{f}_1,\boldsymbol{f}_2,\cdots,\boldsymbol{f}_m\)下的坐标向量的线性同构,\(\boldsymbol{A}:\mathbb{F}^n\to\mathbb{F}^m\)为矩阵乘法诱导的线性映射,即\(\boldsymbol{A}(\boldsymbol{\alpha})=\boldsymbol{A}\boldsymbol{\alpha}\). 求证:\(\eta_2\varphi=\boldsymbol{A}\eta_1\),即下列图交换,并且\(\eta_1:\text{Ker}\varphi\to\text{Ker}\boldsymbol{A},\eta_2:\text{Im}\varphi\to\text{Im}\boldsymbol{A}\)都是线性同构.
\[\begin{tikzcd}
V \arrow[r, "\varphi"] \arrow[d, "\eta_1"'] & U \arrow[d, "\eta_2"] \\
\mathbb{F}^n \arrow[r, "\boldsymbol{A}"]    & \mathbb{F}^m         
\end{tikzcd}
\]
\end{theorem}
\begin{proof}

\end{proof}


\begin{proposition}\label{proposition:线性映射与过渡矩阵}
设\(\varphi\)是线性空间\(V\)到\(U\)的线性映射,\(\{\boldsymbol{e}_1,\boldsymbol{e}_2,\cdots,\boldsymbol{e}_n\}\)和\(\{\boldsymbol{f}_1,\boldsymbol{f}_2,\cdots,\boldsymbol{f}_n\}\)是\(V\)的两组基,\(\{\boldsymbol{e}_1,\boldsymbol{e}_2,\cdots,\boldsymbol{e}_n\}\)到\(\{\boldsymbol{f}_1,\boldsymbol{f}_2,\cdots,\boldsymbol{f}_n\}\)的过渡矩阵为\(\boldsymbol{P}\). \(\{\boldsymbol{g}_1,\boldsymbol{g}_2,\cdots,\boldsymbol{g}_m\}\)和\(\{\boldsymbol{h}_1,\boldsymbol{h}_2,\cdots,\boldsymbol{h}_m\}\)是\(U\)的两组基,\(\{\boldsymbol{g}_1,\boldsymbol{g}_2,\cdots,\boldsymbol{g}_m\}\)到\(\{\boldsymbol{h}_1,\boldsymbol{h}_2,\cdots,\boldsymbol{h}_m\}\)的过渡矩阵为\(\boldsymbol{Q}\). 又设\(\varphi\)在基\(\{\boldsymbol{e}_1,\boldsymbol{e}_2,\cdots,\boldsymbol{e}_n\}\)和基\(\{\boldsymbol{g}_1,\boldsymbol{g}_2,\cdots,\boldsymbol{g}_m\}\)下的表示矩阵为\(\boldsymbol{A}\),在基\(\{\boldsymbol{f}_1,\boldsymbol{f}_2,\cdots,\boldsymbol{f}_n\}\)和基\(\{\boldsymbol{h}_1,\boldsymbol{h}_2,\cdots,\boldsymbol{h}_m\}\)下的表示矩阵为\(\boldsymbol{B}\). 求证:\(\boldsymbol{B}=\boldsymbol{Q}^{-1}\boldsymbol{A}\boldsymbol{P}\).
\end{proposition}
\begin{proof}
任取\(\boldsymbol{v}\in V\),设它在基\(\{\boldsymbol{e}_1,\boldsymbol{e}_2,\cdots,\boldsymbol{e}_n\}\)下的坐标向量为\((x_1,x_2,\cdots,x_n)'\),则它在基\(\{\boldsymbol{f}_1,\boldsymbol{f}_2,\cdots,\boldsymbol{f}_n\}\)下的坐标向量为\(\boldsymbol{P}^{-1}(x_1,x_2,\cdots,x_n)'\). \(\varphi(\boldsymbol{v})\)在基\(\{\boldsymbol{g}_1,\boldsymbol{g}_2,\cdots,\boldsymbol{g}_m\}\)下的坐标向量为\(\boldsymbol{A}(x_1,x_2,\cdots,x_n)'\),在基\(\{\boldsymbol{h}_1,\boldsymbol{h}_2,\cdots,\boldsymbol{h}_m\}\)下的坐标向量为\(\boldsymbol{B}\boldsymbol{P}^{-1}(x_1,x_2,\cdots,x_n)'\). 由于从\(\{\boldsymbol{g}_1,\boldsymbol{g}_2,\cdots,\boldsymbol{g}_m\}\)到\(\{\boldsymbol{h}_1,\boldsymbol{h}_2,\cdots,\boldsymbol{h}_m\}\)的过渡矩阵
为\(\boldsymbol{Q}\),故
\begin{align*}
\boldsymbol{A}(x_1,x_2,\cdots,x_n)'=\boldsymbol{Q}\boldsymbol{B}\boldsymbol{P}^{-1}(x_1,x_2,\cdots,x_n)'.  
\end{align*}
因为\((x_1,x_2,\cdots,x_n)'\)是任意的,
故\(\boldsymbol{A}=\boldsymbol{Q}\boldsymbol{B}\boldsymbol{P}^{-1}\),即\(\boldsymbol{B}=\boldsymbol{Q}^{-1}\boldsymbol{A}\boldsymbol{P}\).
\end{proof}


\begin{proposition}\label{proposition:线性映射在基下的矩阵为标准型}
设\(\varphi\)是有限维线性空间\(V\)到\(U\)的线性映射,求证:必存在\(V\)和\(U\)的两组基,使线性映射\(\varphi\)在两组基下的表示矩阵为\(\begin{pmatrix}\boldsymbol{I}_r&\boldsymbol{O}\\\boldsymbol{O}&\boldsymbol{O}\end{pmatrix}\).
\end{proposition}
\begin{proof}
设\(\{\boldsymbol{e}_1,\boldsymbol{e}_2,\cdots,\boldsymbol{e}_n\}\)是\(V\)的一组基,\(\{\boldsymbol{g}_1,\boldsymbol{g}_2,\cdots,\boldsymbol{g}_m\}\)是\(U\)的一组基,\(\varphi\)在这两组基下的表示矩阵为\(\boldsymbol{A}\). 由相抵标准型理论可知,存在\(m\)阶非异阵\(\boldsymbol{Q}\),\(n\)阶非异阵\(\boldsymbol{P}\),使得\(\boldsymbol{Q}^{-1}\boldsymbol{A}\boldsymbol{P}=\begin{pmatrix}\boldsymbol{I}_r&\boldsymbol{O}\\\boldsymbol{O}&\boldsymbol{O}\end{pmatrix}\). 设\(\{\boldsymbol{f}_1,\boldsymbol{f}_2,\cdots,\boldsymbol{f}_n\}\)是\(V\)的一组新基,使得从\(\{\boldsymbol{e}_1,\boldsymbol{e}_2,\cdots,\boldsymbol{e}_n\}\)到\(\{\boldsymbol{f}_1,\boldsymbol{f}_2,\cdots,\boldsymbol{f}_n\}\)的过渡矩阵为\(\boldsymbol{P}\);设\(\{\boldsymbol{h}_1,\boldsymbol{h}_2,\cdots,\boldsymbol{h}_m\}\)是\(U\)的一组新基,使得从\(\{\boldsymbol{g}_1,\boldsymbol{g}_2,\cdots,\boldsymbol{g}_m\}\)到\(\{\boldsymbol{h}_1,\boldsymbol{h}_2,\cdots,\boldsymbol{h}_m\}\)的过渡矩阵为\(\boldsymbol{Q}\),则由\hyperref[proposition:线性映射与过渡矩阵]{命题\ref{proposition:线性映射与过渡矩阵}}可知,\(\varphi\)在两组新基下的表示矩阵为\(\boldsymbol{Q}^{-1}\boldsymbol{A}\boldsymbol{P}=\begin{pmatrix}\boldsymbol{I}_r&\boldsymbol{O}\\\boldsymbol{O}&\boldsymbol{O}\end{pmatrix}\). 
\end{proof}
\begin{remark}
利用这个命题可以得到\(\text{Ker}\varphi = L(\boldsymbol{f}_{r + 1},\cdots,\boldsymbol{f}_n),\text{Im}\varphi = L(\boldsymbol{h}_1,\cdots,\boldsymbol{h}_r)\),由此即得线性映射的维数公式.
\end{remark}

\begin{proposition}[线性映射维数公式]\label{proposition:值域和核空间维数之和等于原像空间维数}
设\(V,U\)是数域\(\mathbb{K}\)上的有限维线性空间,\(\varphi:V\to U\)是线性映射,\(\varphi:V\to U\)为线性映射,求证:\[
\dim\text{Ker}\varphi+\dim\text{Im}\varphi=\dim V.
\]
\end{proposition}
\begin{proof}
{\color{blue}证法一:}
设\(\dim V = n,\dim\text{Ker}\varphi = k\),我们只要证明\(\dim\text{Im}\varphi=n - k\)即可. 取\(\text{Ker}\varphi\)的一组基\(\boldsymbol{e}_1,\cdots,\boldsymbol{e}_k\),并将其扩张为\(V\)的一组基\(\boldsymbol{e}_1,\cdots,\boldsymbol{e}_k,\boldsymbol{e}_{k + 1},\cdots,\boldsymbol{e}_n\). 任取\(\boldsymbol{\alpha}\in V\),设\(\boldsymbol{\alpha}=c_1\boldsymbol{e}_1+\cdots + c_k\boldsymbol{e}_k + c_{k + 1}\boldsymbol{e}_{k + 1}+\cdots + c_n\boldsymbol{e}_n\),则\(\varphi(\boldsymbol{\alpha})=c_{k + 1}\varphi(\boldsymbol{e}_{k + 1})+\cdots + c_n\varphi(\boldsymbol{e}_n)\),即\(\text{Im}\varphi\)中任一向量都是\(\varphi(\boldsymbol{e}_{k + 1}),\cdots,\varphi(\boldsymbol{e}_n)\)的线性组合. 下证\(\varphi(\boldsymbol{e}_{k + 1}),\cdots,\varphi(\boldsymbol{e}_n)\)线性无关. 设\(\lambda_{k + 1}\varphi(\boldsymbol{e}_{k + 1})+\cdots + \lambda_n\varphi(\boldsymbol{e}_n)=\boldsymbol{0}\),则\(\varphi(\lambda_{k + 1}\boldsymbol{e}_{k + 1}+\cdots + \lambda_n\boldsymbol{e}_n)=\boldsymbol{0}\),即\(\lambda_{k + 1}\boldsymbol{e}_{k + 1}+\cdots + \lambda_n\boldsymbol{e}_n\in\text{Ker}\varphi\),故可设\(\lambda_{k + 1}\boldsymbol{e}_{k + 1}+\cdots + \lambda_n\boldsymbol{e}_n=\lambda_1\boldsymbol{e}_1+\cdots + \lambda_k\boldsymbol{e}_k\),再由\(\boldsymbol{e}_1,\cdots,\boldsymbol{e}_k,\boldsymbol{e}_{k + 1},\cdots,\boldsymbol{e}_n\)线性无关可知\(\lambda_1=\cdots=\lambda_k=\lambda_{k + 1}=\cdots=\lambda_n = 0\). 因此\(\varphi(\boldsymbol{e}_{k + 1}),\cdots,\varphi(\boldsymbol{e}_n)\)是\(\text{Im}\varphi\)的一组基,从而\(\dim\text{Im}\varphi=n - k\),结论得证. 

{\color{blue}证法二(从商空间的角度):}
设由\(\varphi\)诱导的线性映射\(\overline{\varphi}:V/\text{Ker}\varphi\to\text{Im}\varphi\),\(\overline{\varphi}(\boldsymbol{v}+\text{Ker}\varphi)=\varphi(\boldsymbol{v})\). 先证是$\overline{\varphi}$线性同构的.

首先,\(\overline{\varphi}\)的定义不依赖于\(\text{Ker}\varphi -\)陪集代表元的选取. 事实上,若\(\boldsymbol{v}_1+\text{Ker}\varphi=\boldsymbol{v}_2+\text{Ker}\varphi\),即\(\boldsymbol{v}_1 - \boldsymbol{v}_2\in\text{Ker}\varphi\),则\(0=\varphi(\boldsymbol{v}_1 - \boldsymbol{v}_2)=\varphi(\boldsymbol{v}_1)-\varphi(\boldsymbol{v}_2)\),即\(\varphi(\boldsymbol{v}_1)=\varphi(\boldsymbol{v}_2)\). 其次,容易验证\(\overline{\varphi}\)是一个线性映射. 再次,由\(\overline{\varphi}\)的定义不难看出它是满射. 最后,由\(\overline{\varphi}\)的定义可知\(\text{Ker}\overline{\varphi}=\{\boldsymbol{0}+\text{Ker}\varphi\}\)是商空间\(V/\text{Ker}\varphi\)的零子空间,故为单射,从而\(\overline{\varphi}:V/\text{Ker}\varphi\to\text{Im}\varphi\)是线性同构. 由\hyperref[proposition:商空间的维数公式和商空间与补空间同构]{商空间的维数公式}可得
\[
\dim\text{Im}\varphi=\dim(V/\text{Ker}\varphi)=\dim V-\dim\text{Ker}\varphi,
\]
由此即得线性映射的维数公式.
\end{proof}

\begin{corollary}\label{corollary:由核的基导出值域的基}
设\(\varphi:V\to U\)为线性映射,$\mathrm{Ker}\varphi $的一组基为$\boldsymbol{e}_{r+1}\cdots ,\boldsymbol{e}_n,$并将其扩张为$V$的一组基$\boldsymbol{e}_1,\boldsymbol{e}_2,\cdots ,\boldsymbol{e}_n.$
则$\varphi \left( \boldsymbol{e}_1 \right) ,\cdots ,\varphi \left( \boldsymbol{e}_r \right)$ 一定是Im$\varphi$ 的一组基.
\end{corollary}
\begin{proof}
由\hyperref[proposition:值域和核空间维数之和等于原像空间维数]{命题\ref{proposition:值域和核空间维数之和等于原像空间维数}}的证法一立得.

\end{proof}
\begin{proposition}\label{proposition:行/列满秩矩阵对应满/单射}
设\(\varphi\)是\(n\)维线性空间\(V\)到\(m\)维线性空间\(U\)的线性映射,\(\varphi\)在给定基下的表示矩阵为\(\boldsymbol{A}_{m\times n}\). 求证:\(\varphi\)是满映射的充要条件是\(\text{r}(\boldsymbol{A}) = m\)($\boldsymbol{A}$行满秩),\(\varphi\)是单映射的充要条件是\(\text{r}(\boldsymbol{A}) = n\)($\boldsymbol{A}$列满秩).
\end{proposition}
\begin{note}
\(\dim\text{Im}\varphi=\text{r}(\boldsymbol{A})\)和\(\dim\text{Ker}\varphi=n - \text{r}(\boldsymbol{A})\)的原因见\hyperref[线性映射与矩阵基本结论2]{线性映射与矩阵基本结论\ref{线性映射与矩阵基本结论2}}.
\end{note}
\begin{proof}
注意到\(\dim\text{Im}\varphi=\text{r}(\boldsymbol{A})\),并且\(\varphi\)是满映射的充要条件是\(\text{Im}\varphi = U\),这也等价于\(\dim\text{Im}\varphi=\dim U = m\),故第一个结论成立.

注意到\(\dim\text{Ker}\varphi=n - \text{r}(\boldsymbol{A})\),并且\(\varphi\)是单映射的充要条件是\(\text{Ker}\varphi = 0\),这也等价于\(\dim\text{Ker}\varphi = 0\),故第二个结论成立.
\end{proof}

\begin{proposition}\label{proposition:线性映射的秩1分解}
设\(\varphi:V\to U\)为线性映射且\(\varphi\)的秩为\(r\),证明:存在\(r\)个秩为\(1\)的线性映射\(\varphi_i:V\to U(1\leq i\leq r)\),使得\(\varphi=\varphi_1+\cdots+\varphi_r\).
\end{proposition}
\begin{proof}
取定\(V\)和\(U\)的两组基,设\(\varphi\)在这两组基下的表示矩阵为\(\boldsymbol{A}\),则\(\text{r}(\boldsymbol{A})=\text{r}(\varphi)=r\). 由\hyperref[proposition:矩阵的秩1分解]{矩阵的秩1分解}可知,存在\(r\)个秩为\(1\)的矩阵\(\boldsymbol{A}_i(1\leq i\leq r)\),使得\(\boldsymbol{A}=\boldsymbol{A}_1+\cdots+\boldsymbol{A}_r\). 由于线性映射和表示矩阵之间一一对应,故存在线性映射\(\varphi_i:V\to U(1\leq i\leq r)\),使得\(\varphi=\varphi_1+\cdots+\varphi_r\),且\(\text{r}(\varphi_i)=\text{r}(\boldsymbol{A}_i)=1\).
\end{proof}

\begin{proposition}\label{proposition:任一组基下的表示矩阵都相同的线性变换是纯量变换}
设\(\varphi\)是线性空间\(V\)上的线性变换,若它在\(V\)的任一组基下的表示矩阵都相同,求证:\(\varphi\)是纯量变换,即存在常数\(k\),使得\(\varphi(\boldsymbol{\alpha}) = k\boldsymbol{\alpha}\)对一切\(\boldsymbol{\alpha}\in V\)都成立.
\end{proposition}
\begin{proof}
取定\(V\)的一组基,设\(\varphi\)在这组基下的表示矩阵是\(\boldsymbol{A}\). 由已知条件可知,对任意一个同阶可逆矩阵\(\boldsymbol{P}\),\(\boldsymbol{A}=\boldsymbol{P}^{-1}\boldsymbol{A}\boldsymbol{P}\),即\(\boldsymbol{P}\boldsymbol{A}=\boldsymbol{A}\boldsymbol{P}\). 因此矩阵\(\boldsymbol{A}\)和任意一个可逆矩阵乘法可交换,于是由\hyperref[proposition:纯量阵的刻画]{命题\ref{proposition:纯量阵的刻画}}可知\(\boldsymbol{A}=k\boldsymbol{I}_n\),由此即知\(\varphi\)是纯量变换.
\end{proof}

\subsection{将矩阵问题转化为线性映射问题}
我们将线性映射的问题转化为矩阵问题来处理. 反之,我们也可将矩阵问题转化为线性映射(线性变换)问题来处理. 一般的处理方式如下:

设\(\boldsymbol{A}\)是数域\(\mathbb{F}\)上的\(m\times n\)矩阵,定义列向量空间\(\mathbb{F}^n\)到\(\mathbb{F}^m\)的线性映射:\(\varphi(\boldsymbol{\alpha})=\boldsymbol{A}\boldsymbol{\alpha}\),容易验证在\(\mathbb{F}^n\)和\(\mathbb{F}^m\)的标准单位列向量构成的基下,\(\varphi\)的表示矩阵就是\(\boldsymbol{A}\). 同理,若\(\boldsymbol{A}\)是\(\mathbb{F}\)上的\(n\)阶矩阵,定义\(\mathbb{F}^n\)上的线性变换:\(\varphi(\boldsymbol{\alpha})=\boldsymbol{A}\boldsymbol{\alpha}\),容易验证在\(\mathbb{F}^n\)的标准单位列向量构成的基下,\(\varphi\)的表示矩阵就是\(\boldsymbol{A}\). 

因此,我们有时就把这个线性映射(线性变换)写为\(\boldsymbol{A}\). 上述把代数问题转化成几何问题的语言表述,在后面的章节中一直会用到. 某些矩阵问题采用这种方式转化为线性映射(线性变换)问题后,往往变得比较容易解决或者可以充分利用几何直观去得到解题思路.
\begin{example}
设\(\boldsymbol{A},\boldsymbol{B}\)都是数域\(\mathbb{F}\)上的\(m\times n\)矩阵,求证:方程组\(\boldsymbol{A}\boldsymbol{x}=0,\boldsymbol{B}\boldsymbol{x}=0\)同解的充要条件是存在可逆矩阵\(\boldsymbol{P}\),使得\(\boldsymbol{B}=\boldsymbol{P}\boldsymbol{A}\).
\end{example}
\begin{proof}
因为\(\boldsymbol{P}\)是可逆矩阵,充分性是显然的. 现通过两种方法来证明必要性.

{\color{blue}证法一(代数方法):} 由条件可得方程组\(\boldsymbol{A}\boldsymbol{x}=0,\boldsymbol{B}\boldsymbol{x}=0,\begin{pmatrix}\boldsymbol{A}\\\boldsymbol{B}\end{pmatrix}\boldsymbol{x}=0\)都同解,从而有
\[
\text{r}(\boldsymbol{A})=\text{r}(\boldsymbol{B})=\text{r}\begin{pmatrix}\boldsymbol{A}\\\boldsymbol{B}\end{pmatrix}.
\]
注意到结论\(\boldsymbol{B}=\boldsymbol{P}\boldsymbol{A}\)就是说\(\boldsymbol{A},\boldsymbol{B}\)可以通过初等行变换相互转化,因此在证明的过程中,对\(\boldsymbol{A}\)或\(\boldsymbol{B}\)实施初等行变换不影响结论的证明. 设
\[
\boldsymbol{A}=\begin{pmatrix}\boldsymbol{\alpha}_1\\\boldsymbol{\alpha}_2\\\vdots\\\boldsymbol{\alpha}_m\end{pmatrix},\boldsymbol{B}=\begin{pmatrix}\boldsymbol{\beta}_1\\\boldsymbol{\beta}_2\\\vdots\\\boldsymbol{\beta}_m\end{pmatrix}
\]
分别为\(\boldsymbol{A},\boldsymbol{B}\)的行分块. 不妨对\(\boldsymbol{A},\boldsymbol{B}\)都进行行对换,故可设\(\boldsymbol{\alpha}_1,\cdots,\boldsymbol{\alpha}_r\)是\(\boldsymbol{A}\)的行向量的极大无关组,\(\boldsymbol{\beta}_1,\cdots,\boldsymbol{\beta}_r\)是\(\boldsymbol{B}\)的行向量的极大无关组. 由于\(\text{r}\begin{pmatrix}\boldsymbol{A}\\\boldsymbol{B}\end{pmatrix}=r\),故由\hyperref[proposition:表出向量组的秩不超过原向量组的秩]{命题\ref{proposition:表出向量组的秩不超过原向量组的秩}}可知,\(\boldsymbol{\alpha}_1,\cdots,\boldsymbol{\alpha}_r\)和\(\boldsymbol{\beta}_1,\cdots,\boldsymbol{\beta}_r\)是向量组\(\boldsymbol{\alpha}_1,\boldsymbol{\alpha}_2,\cdots,\boldsymbol{\alpha}_m,\)\(\boldsymbol{\beta}_1,\boldsymbol{\beta}_2,\cdots,\boldsymbol{\beta}_m\)的两组极大无关组. 设\(\boldsymbol{\beta}_i=\sum_{j = 1}^{r}c_{ij}\boldsymbol{\alpha}_j(1\leq i\leq r)\),则容易验证\(r\)阶方阵\(\boldsymbol{C}=(c_{ij})\)是非异阵. 设\(\boldsymbol{\beta}_i-\boldsymbol{\alpha}_i=\sum_{j = 1}^{r}d_{ij}\boldsymbol{\alpha}_j(r + 1\leq i\leq m)\),\(\boldsymbol{D}=(d_{ij})\)是\((m - r)\times r\)矩阵,则容易验证\(\boldsymbol{P}=\begin{pmatrix}\boldsymbol{C}&\boldsymbol{O}\\\boldsymbol{D}&\boldsymbol{I}_{m - r}\end{pmatrix}\)是\(m\)阶非异阵,并且满足\(\boldsymbol{B}=\boldsymbol{P}\boldsymbol{A}\).

{\color{blue}证法二(几何方法):}  将问题转化成几何的语言即为:设\(V\)是\(\mathbb{F}\)上的\(n\)维线性空间,\(U\)是\(\mathbb{F}\)上的\(m\)维线性空间,\(\varphi,\psi:V\to U\)是两个线性映射. 求证:若\(\text{Ker}\varphi=\text{Ker}\psi\),则存在\(U\)上的自同构\(\sigma\),使得\(\psi=\sigma\varphi\).

设\(\text{r}(\varphi)=r\),则\(\dim\text{Ker}\varphi=\dim\text{Ker}\psi=n - r\). 取\(\text{Ker}\varphi=\text{Ker}\psi\)的一组基\(\boldsymbol{e}_{r + 1},\cdots,\boldsymbol{e}_n\),并将其扩张为\(V\)的一组基\(\boldsymbol{e}_1,\cdots,\boldsymbol{e}_r,\boldsymbol{e}_{r + 1},\cdots,\boldsymbol{e}_n\). 根据\hyperref[corollary:由核的基导出值域的基]{推论\ref{corollary:由核的基导出值域的基}}可知,\(\varphi(\boldsymbol{e}_1),\cdots,\varphi(\boldsymbol{e}_r)\)是\(\text{Im}\varphi\)的一组基,故可将其扩张为\(U\)的一组基\(\varphi(\boldsymbol{e}_1),\cdots,\varphi(\boldsymbol{e}_r),\boldsymbol{f}_{r + 1},\cdots,\boldsymbol{f}_m\). 同理可知,\(\psi(\boldsymbol{e}_1),\cdots,\psi(\boldsymbol{e}_r)\)是\(\text{Im}\psi\)的一组基,故可将其扩张为\(U\)的一组基\(\psi(\boldsymbol{e}_1),\cdots,\psi(\boldsymbol{e}_r),\boldsymbol{g}_{r + 1},\cdots,\boldsymbol{g}_m\). 定义\(U\)上的线性变换\(\sigma\)如下:
\[
\sigma(\varphi(\boldsymbol{e}_i))=\psi(\boldsymbol{e}_i),1\leq i\leq r;\ \sigma(\boldsymbol{f}_j)=\boldsymbol{g}_j,r + 1\leq j\leq m.
\]
因为\(\sigma\)把\(U\)的一组基映射为\(U\)的另一组基,故\(\sigma\)是\(U\)的自同构. 又对\(r + 1\leq j\leq n\),\(\sigma(\varphi(\boldsymbol{e}_j))=0=\psi(\boldsymbol{e}_j)\),故\(\sigma\varphi=\psi\)成立. 
\end{proof}

\begin{proposition}\label{proposition:相似矩阵可看作一个线性变换在不同基下的表示矩阵}
若数域\(\mathbb{F}\)上的\(n\)阶方阵\(\boldsymbol{A}\)和\(\boldsymbol{B}\)相似,求证:它们可以看成是某个线性空间上同一个线性变换在不同基下的表示矩阵.
\end{proposition}
\begin{note}
由下面的证明可知这个线性变换\(\varphi\)就是由矩阵\(\boldsymbol{A}\)的乘法诱导的线性变换.两组不同的基就是标准基与可逆矩阵\(\boldsymbol{P}\)的列向量.
\end{note}
\begin{proof}
令\(V = \mathbb{F}^n\)是\(n\)维列向量空间,\(\{\boldsymbol{e}_1,\boldsymbol{e}_2,\cdots,\boldsymbol{e}_n\}\)是由\(n\)维标准单位列向量构成的基,\(\varphi\)是由矩阵\(\boldsymbol{A}\)的乘法诱导的线性变换,容易验证\(\varphi\)在基\(\{\boldsymbol{e}_1,\boldsymbol{e}_2,\cdots,\boldsymbol{e}_n\}\)下的表示矩阵就是\(\boldsymbol{A}\). 已知\(\boldsymbol{A}\)和\(\boldsymbol{B}\)相似,即存在可逆矩阵\(\boldsymbol{P}\),使得\(\boldsymbol{B}=\boldsymbol{P}^{-1}\boldsymbol{A}\boldsymbol{P}\).
令\(\boldsymbol{P}=(\boldsymbol{f}_1,\boldsymbol{f}_2,\cdots,\boldsymbol{f}_n)\)为其列分块,由于\(\boldsymbol{P}\)可逆,故\(\boldsymbol{f}_1,\boldsymbol{f}_2,\cdots,\boldsymbol{f}_n\)线性无关,从而是\(V\)的一组基. 注意到从基\(\{\boldsymbol{e}_1,\boldsymbol{e}_2,\cdots,\boldsymbol{e}_n\}\)到基\(\{\boldsymbol{f}_1,\boldsymbol{f}_2,\cdots,\boldsymbol{f}_n\}\)的过渡矩阵就是\(\boldsymbol{P}\),因此线性变换\(\varphi\)在基\(\{\boldsymbol{f}_1,\boldsymbol{f}_2,\cdots,\boldsymbol{f}_n\}\)下的表示矩阵为\(\boldsymbol{P}^{-1}\boldsymbol{A}\boldsymbol{P}=\boldsymbol{B}\). 
\end{proof}

\begin{example}
设\(V\)是数域\(\mathbb{F}\)上\(n\)阶矩阵全体构成的线性空间,\(\varphi\)是\(V\)上的线性变换:\(\varphi(\boldsymbol{A})=\boldsymbol{A}'\). 证明:存在\(V\)的一组基,使得\(\varphi\)在这组基下的表示矩阵是一个对角矩阵且主对角元素全是\(1\)或\(-1\),并求出\(1\)和\(-1\)的个数.
\end{example}
\begin{proof}
设\(V_1\)是由\(n\)阶对称矩阵组成的子空间,\(V_2\)是由反对称矩阵组成的子空间,则由\hyperref[proposition:矩阵空间可以分解为对称和反称矩阵空间的直和]{命题\ref{proposition:矩阵空间可以分解为对称和反称矩阵空间的直和}}可得
\[
V = V_1\oplus V_2.
\]
取\(V_1\)的一组基和\(V_2\)的一组基拼成\(V\)的一组基,则\(\varphi\)在这组基下的表示矩阵是对角矩阵且主对角元素或为\(1\)或为\(-1\). 因为\(\dim V_1=\frac{1}{2}n(n + 1)\),\(\dim V_2=\frac{1}{2}n(n - 1)\),故\(1\)的个数为\(\frac{1}{2}n(n + 1)\),\(-1\)的个数为\(\frac{1}{2}n(n - 1)\).
\end{proof}

\begin{example}
设\(V\)是数域\(\mathbb{K}\)上的\(n\)维线性空间,\(\varphi,\psi\)是\(V\)上的线性变换且\(\varphi^2 = 0\),\(\psi^2 = 0\),\(\varphi\psi+\psi\varphi=\boldsymbol{I}\),\(\boldsymbol{I}\)是\(V\)上的恒等变换. 求证:
\begin{enumerate}[(1)]
\item \(V=\text{Ker}\varphi\oplus\text{Ker}\psi\);

\item 若\(V\)是二维空间,则存在\(V\)的基\(\boldsymbol{e}_1,\boldsymbol{e}_2\),使得\(\varphi,\psi\)在这组基下的表示矩阵分别为
\[
\boldsymbol{A}=\begin{pmatrix}0&0\\1&0\end{pmatrix},\boldsymbol{B}=\begin{pmatrix}0&1\\0&0\end{pmatrix};
\]

\item \(V\)必是偶数维空间且若\(V\)是\(2k\)维空间,则存在\(V\)的一组基,使得\(\varphi,\psi\)在这组基下的表示矩阵分别为下列分块对角矩阵:
\[
\begin{pmatrix}
\boldsymbol{A}&\boldsymbol{O}&\cdots&\boldsymbol{O}\\
\boldsymbol{O}&\boldsymbol{A}&\cdots&\boldsymbol{O}\\
\vdots&\vdots&&\vdots\\
\boldsymbol{O}&\boldsymbol{O}&\cdots&\boldsymbol{A}
\end{pmatrix},\begin{pmatrix}
\boldsymbol{B}&\boldsymbol{O}&\cdots&\boldsymbol{O}\\
\boldsymbol{O}&\boldsymbol{B}&\cdots&\boldsymbol{O}\\
\vdots&\vdots&&\vdots\\
\boldsymbol{O}&\boldsymbol{O}&\cdots&\boldsymbol{B}
\end{pmatrix},
\]
其中主对角线上分别有\(k\)个\(\boldsymbol{A}\)和\(k\)个\(\boldsymbol{B}\).
\end{enumerate}
\end{example}
\begin{proof}
\begin{enumerate}[(1)]
\item 任取\(\boldsymbol{\alpha}\in V\),则由\(\boldsymbol{I}=\varphi\psi+\psi\varphi\)得到\(\boldsymbol{\alpha}=\varphi\psi(\boldsymbol{\alpha})+\psi\varphi(\boldsymbol{\alpha})\). 注意到\(\varphi\psi(\boldsymbol{\alpha})\in\text{Ker}\varphi\),\(\psi\varphi(\boldsymbol{\alpha})\in\text{Ker}\psi\),因此\(V=\text{Ker}\varphi+\text{Ker}\psi\). 又若\(\boldsymbol{\beta}\in\text{Ker}\varphi\cap\text{Ker}\psi\),则\(\boldsymbol{\beta}=\varphi\psi(\boldsymbol{\beta})+\psi\varphi(\boldsymbol{\beta}) = 0\),即\(\text{Ker}\varphi\cap\text{Ker}\psi = 0\). 于是\(V=\text{Ker}\varphi\oplus\text{Ker}\psi\).

\item 取\(\boldsymbol{0}\neq\boldsymbol{e}_1\in\text{Ker}\psi\),\(\boldsymbol{e}_2 = \varphi(\boldsymbol{e}_1)\),则\(\varphi(\boldsymbol{e}_2)=\varphi^2(\boldsymbol{e}_1)=0\),即\(\boldsymbol{e}_2\in\text{Ker}\varphi\). 又若\(\boldsymbol{e}_2 = \boldsymbol{0}\),则\(\boldsymbol{e}_1\in\text{Ker}\varphi\cap\text{Ker}\psi = 0\),和假设矛盾,于是\(\boldsymbol{e}_2\neq\boldsymbol{0}\). 因此\(\boldsymbol{e}_1,\boldsymbol{e}_2\)组成\(V\)的一组基,不难验证在这组基下,\(\varphi,\psi\)的表示矩阵符合要求($\psi \left( \boldsymbol{e}_2 \right) =\psi \left( \varphi \left( \boldsymbol{e}_1 \right) \right) =\psi \varphi \left( \boldsymbol{e}_1 \right) =\boldsymbol{I}\left( \boldsymbol{e}_1 \right) -\varphi \psi \left( \boldsymbol{e}_1 \right) =\boldsymbol{e}_1$).

\item 设\(\dim\text{Ker}\psi=k\),并取\(\text{Ker}\psi\)的一组基\(\boldsymbol{e}_1,\boldsymbol{e}_2,\cdots,\boldsymbol{e}_k\). 令\(\boldsymbol{e}_{k + 1}=\varphi(\boldsymbol{e}_1)\),\(\boldsymbol{e}_{k + 2}=\varphi(\boldsymbol{e}_2),\cdots,\boldsymbol{e}_{2k}=\varphi(\boldsymbol{e}_k)\),则由\(\varphi^2 = 0\)可得\(\boldsymbol{e}_{k + 1},\boldsymbol{e}_{k + 2},\cdots,\boldsymbol{e}_{2k}\)都属于\(\text{Ker}\varphi\). 我们先证明向量组\(\boldsymbol{e}_{k + 1},\boldsymbol{e}_{k + 2},\cdots,\boldsymbol{e}_{2k}\)是线性无关的. 设有
\[
c_1\boldsymbol{e}_{k + 1}+c_2\boldsymbol{e}_{k + 2}+\cdots + c_k\boldsymbol{e}_{2k}=\boldsymbol{0},
\]
两边作用\(\psi\),可得
\[
c_1\psi(\boldsymbol{e}_{k + 1})+c_2\psi(\boldsymbol{e}_{k + 2})+\cdots + c_k\psi(\boldsymbol{e}_{2k})=\boldsymbol{0}.
\]
注意到\(\boldsymbol{e}_1=\varphi\psi(\boldsymbol{e}_1)+\psi\varphi(\boldsymbol{e}_1)=\psi(\boldsymbol{e}_{k + 1})\),同理\(\boldsymbol{e}_2=\psi(\boldsymbol{e}_{k + 2}),\cdots,\boldsymbol{e}_k=\psi(\boldsymbol{e}_{2k})\). 因此上式就是
\[
c_1\boldsymbol{e}_1+c_2\boldsymbol{e}_2+\cdots + c_k\boldsymbol{e}_k=\boldsymbol{0}.
\]
而\(\boldsymbol{e}_1,\boldsymbol{e}_2,\cdots,\boldsymbol{e}_k\)线性无关,故\(c_1 = c_2=\cdots = c_k = 0\),即向量组\(\boldsymbol{e}_{k + 1},\boldsymbol{e}_{k + 2},\cdots,\boldsymbol{e}_{2k}\)线性无关. 特别地,我们有\(\dim\text{Ker}\varphi\geq k=\dim\text{Ker}\psi\). 由于\(\varphi,\psi\)的地位是对称的,故同理可证\(\dim\text{Ker}\psi\geq\dim\text{Ker}\varphi\),从而\(\dim\text{Ker}\varphi=\dim\text{Ker}\psi = k\),并且\(\boldsymbol{e}_{k + 1},\boldsymbol{e}_{k + 2},\cdots,\boldsymbol{e}_{2k}\)是\(\text{Ker}\varphi\)的一组基. 因为\(V=\text{Ker}\varphi\oplus\text{Ker}\psi\),故\(\boldsymbol{e}_1,\cdots,\boldsymbol{e}_k,\boldsymbol{e}_{k + 1},\cdots,\boldsymbol{e}_{2k}\)组成\(V\)的一组基. 现将基向量排列如下:
\[
\boldsymbol{e}_1,\boldsymbol{e}_{k + 1},\boldsymbol{e}_2,\boldsymbol{e}_{k + 2},\cdots,\boldsymbol{e}_k,\boldsymbol{e}_{2k}.
\]
不难验证,在这组基下\(\varphi,\psi\)的表示矩阵即为所求.
\end{enumerate}
\end{proof}

\section{像空间和核空间}

\begin{proposition}\label{proposition:像空间和核空间的子空间链}
设\(\varphi\)是向量空间\(V\)上的线性变换,则
\[
V\supseteq \mathrm{Im}\varphi \supseteq \mathrm{Im}\varphi^2 \supseteq \cdots \supseteq \mathrm{Im}\varphi^n \supseteq \mathrm{Im}\varphi^{n + 1} \supseteq \cdots,
\]
\[
\mathrm{Ker}\varphi \subseteq \mathrm{Ker}\varphi ^2\subseteq \cdots \subseteq \mathrm{Ker}\varphi ^n\subseteq \mathrm{Ker}\varphi ^{n+1}\subseteq \cdots \subseteq V.
\]
\end{proposition}
\begin{proof}
由像空间和核空间的定义易证.
\end{proof}

\begin{example}
设线性空间\(V\)上的线性变换\(\varphi\)在基\(\{\boldsymbol{e}_1,\boldsymbol{e}_2,\boldsymbol{e}_3,\boldsymbol{e}_4\}\)下的表示矩阵为
\[
\boldsymbol{A}=\begin{pmatrix}
1&0&2&1\\
-1&2&1&3\\
1&2&5&5\\
2&-2&1&-2
\end{pmatrix},
\]
求\(\varphi\)的核空间与像空间(用基的线性组合来表示).
\end{example}
\begin{proof}
像空间通过坐标向量同构于\(\boldsymbol{A}\)的列向量生成的子空间,通过计算可得\(\boldsymbol{A}\)的秩等于\(2\),且\(\boldsymbol{A}\)的第一、第二列向量线性无关,于是\(\text{Im}\varphi\)的基的坐标向量为\((1,-1,1,2)^\prime,(0,2,2,-2)^\prime\),从而\(\text{Im}\varphi = k_1(\boldsymbol{e}_1 - \boldsymbol{e}_2 + \boldsymbol{e}_3 + 2\boldsymbol{e}_4)+k_2(2\boldsymbol{e}_2 + 2\boldsymbol{e}_3 - 2\boldsymbol{e}_4)\). 核空间通过坐标向量同构于齐次线性方程组\(\boldsymbol{A}\boldsymbol{x}=0\)的解空间,通过计算可得该方程组的基础解系为\((-4,-3,2,0)^\prime,(-1,-2,0,1)^\prime\),此即\(\text{Ker}\varphi\)的基的坐标向量,于是\(\text{Ker}\varphi = k_1(-4\boldsymbol{e}_1 - 3\boldsymbol{e}_2 + 2\boldsymbol{e}_3)+k_2(-\boldsymbol{e}_1 - 2\boldsymbol{e}_2 + \boldsymbol{e}_4)\).
\end{proof}

\begin{proposition}\label{proposition:一组有限个非零线性变换一定存在非零像}
设\(V\)是数域\(\mathbb{F}\)上的线性空间,\(\varphi_1,\varphi_2,\cdots,\varphi_k\)是\(V\)上的非零线性变换. 求证:存在\(\boldsymbol{\alpha}\in V\),使得\(\varphi_i(\boldsymbol{\alpha})\neq 0(1\leq i\leq k)\).
\end{proposition}
\begin{proof}
因为\(\varphi_i\neq 0\),所以\(\text{Ker}\varphi_i\)是\(V\)的真子空间. 由\hyperref[proposition:真子空间外仍有向量存在]{命题\ref{proposition:真子空间外仍有向量存在}}可知,有限个真子空间\(\text{Ker}\varphi_i\)不能覆盖全空间\(V\),故必存在\(\boldsymbol{\alpha}\in V\),使得\(\boldsymbol{\alpha}\)不属于任意一个\(\text{Ker}\varphi_i\),从而结论得证.
\end{proof}

\begin{proposition}\label{proposition:一组有限个非零线性变换一定存在互不相同的像}
设\(V\)是数域\(\mathbb{F}\)上的线性空间,\(\varphi_1,\varphi_2,\cdots,\varphi_k\)是\(V\)上互不相同的线性变换. 求证:存在\(\boldsymbol{\alpha}\in V\),使得\(\varphi_1(\boldsymbol{\alpha}),\varphi_2(\boldsymbol{\alpha}),\cdots,\varphi_k(\boldsymbol{\alpha})\)互不相同.
\end{proposition}
\begin{proof}
令\(\varphi_{ij}=\varphi_i - \varphi_j(1\leq i<j\leq k)\),则\(\varphi_{ij}\)是\(V\)上的非零线性变换. 由\hyperref[proposition:一组有限个非零线性变换一定存在非零像]{命题\ref{proposition:一组有限个非零线性变换一定存在非零像}}可知,存在\(\boldsymbol{\alpha}\in V\),使得\(\varphi_{ij}(\boldsymbol{\alpha})\neq 0\),即\(\varphi_i(\boldsymbol{\alpha})\neq\varphi_j(\boldsymbol{\alpha})(1\leq i<j\leq k)\),从而结论得证. 
\end{proof}

\begin{proposition}\label{proposition:矩阵n次方幂秩等式}
设\(\boldsymbol{A}\)是\(n\)阶方阵,求证:\(\text{r}(\boldsymbol{A}^n)=\text{r}(\boldsymbol{A}^{n + 1})=\text{r}(\boldsymbol{A}^{n + 2})=\cdots\).
\end{proposition}
\begin{proof}
{\color{blue}证法一(代数方法):} 由秩的不等式可得
\[
n=\text{r}(\boldsymbol{I}_n)\geq\text{r}(\boldsymbol{A})\geq\text{r}(\boldsymbol{A}^2)\geq\cdots\geq\text{r}(\boldsymbol{A}^n)\geq\text{r}(\boldsymbol{A}^{n + 1})\geq 0.
\]
上述\(n + 2\)个整数都在\([0,n]\)之间,故由抽屉原理可知,存在某个整数\(m\in[0,n]\),使得\(\text{r}(\boldsymbol{A}^m)=\text{r}(\boldsymbol{A}^{m + 1})\). 对任意的\(k\geq m\),由\hyperref[proposition:Frobenius不等式]{矩阵秩的Frobenius不等式}可得
\[
\text{r}(\boldsymbol{A}^{k + 1})=\text{r}(\boldsymbol{A}^{k - m}\boldsymbol{A}^m\boldsymbol{A})\geq\text{r}(\boldsymbol{A}^{k - m}\boldsymbol{A}^m)+\text{r}(\boldsymbol{A}^m\boldsymbol{A})-\text{r}(\boldsymbol{A}^m)=\text{r}(\boldsymbol{A}^k),
\]
又\(\text{r}(\boldsymbol{A}^{k + 1})\leq\text{r}(\boldsymbol{A}^k)\),故\(\text{r}(\boldsymbol{A}^{k + 1})=\text{r}(\boldsymbol{A}^k)\)对任意的\(k\geq m\)成立,结论得证.

{\color{blue}证法二(几何方法):} 令\(\varphi\)为在\(n\)维列向量空间上由矩阵\(\boldsymbol{A}\)乘法诱导的线性变换,则\(\varphi\)在标准基下的表示矩阵就是\(\boldsymbol{A}\),并且不难发现对\(\forall k\in \mathbb{N}\),\(\varphi^k\)在标准基下的表示矩阵是\(\boldsymbol{A}^k\).因此\(\mathrm{r}(\boldsymbol{A}^k) = \mathrm{dim} \,\mathrm{Im}\varphi^k\),故原命题等价于$\mathrm{dim}\,\mathrm{Im}\varphi ^n=\mathrm{dim}\,\mathrm{Im}\varphi ^{n+1}=\mathrm{dim}\,\mathrm{Im}\varphi ^{n+2}=\cdots 
$.注意下列子空间链:
\[
V\supseteq\text{Im}\varphi\supseteq\text{Im}\varphi^2\supseteq\cdots\supseteq\text{Im}\varphi^n\supseteq\text{Im}\varphi^{n + 1}.
\]
上述\(n + 2\)个子空间的维数都在\([0,n]\)之间,故由抽屉原理可知,存在某个整数\(m\in[0,n]\),使得\(\text{Im}\varphi^m=\text{Im}\varphi^{m + 1}\). 现要证明对任意的\(k\geq m\),\(\text{Im}\varphi^k=\text{Im}\varphi^{k + 1}\). 一方面,\(\text{Im}\varphi^{k + 1}\subseteq\text{Im}\varphi^k\)是显然的. 另一方面,任取\(\boldsymbol{\alpha}\in\text{Im}\varphi^k\),则存在\(\boldsymbol{\beta}\in V\),使得\(\boldsymbol{\alpha}=\varphi^k(\boldsymbol{\beta})\). 由于\(\varphi^m(\boldsymbol{\beta})\in\text{Im}\varphi^m=\text{Im}\varphi^{m + 1}\),故存在\(\boldsymbol{\gamma}\in V\),使得\(\varphi^m(\boldsymbol{\beta})=\varphi^{m + 1}(\boldsymbol{\gamma})\),从而
\[
\boldsymbol{\alpha}=\varphi^k(\boldsymbol{\beta})=\varphi^{k - m}(\varphi^m(\boldsymbol{\beta}))=\varphi^{k - m}(\varphi^{m + 1}(\boldsymbol{\gamma}))=\varphi^{k + 1}(\boldsymbol{\gamma})\in\text{Im}\varphi^{k + 1},
\]
故\(\text{Im}\varphi^k=\text{Im}\varphi^{k + 1}\)对任意的\(k\geq m\)成立,取维数后即得结论. 
\end{proof}

\begin{corollary}\label{corollary:n次方幂秩/值域维数等式}
\begin{enumerate}[(1)]
\item 设\(\boldsymbol{A}\)是\(n\)阶方阵,则一定存在整数$m\in[0,n]$,使得\(\text{r}(\boldsymbol{A}^m)=\text{r}(\boldsymbol{A}^{m + 1})=\text{r}(\boldsymbol{A}^{m + 2})=\cdots\).

\item 设\(\varphi\)是\(n\)维线性空间\(V\)上的线性变换,则必存在整数\(m\in[0,n]\),使得
\[
\mathrm{dim}\,\mathrm{Im}\varphi ^m=\mathrm{dim}\,\mathrm{Im}\varphi ^{m+1}=\mathrm{dim}\,\mathrm{Im}\varphi ^{m+2}=\cdots.
\]
\end{enumerate}
\end{corollary}
\begin{proof}
\begin{enumerate}[(1)]
\item {\color{blue}证法一(代数方法):} 由秩的不等式可得
\[
n=\text{r}(\boldsymbol{I}_n)\geq\text{r}(\boldsymbol{A})\geq\text{r}(\boldsymbol{A}^2)\geq\cdots\geq\text{r}(\boldsymbol{A}^n)\geq\text{r}(\boldsymbol{A}^{n + 1})\geq 0.
\]
上述\(n + 2\)个整数都在\([0,n]\)之间,故由抽屉原理可知,存在某个整数\(m\in[0,n]\),使得\(\text{r}(\boldsymbol{A}^m)=\text{r}(\boldsymbol{A}^{m + 1})\). 对任意的\(k\geq m\),由\hyperref[proposition:Frobenius不等式]{矩阵秩的Frobenius不等式}可得
\[
\text{r}(\boldsymbol{A}^{k + 1})=\text{r}(\boldsymbol{A}^{k - m}\boldsymbol{A}^m\boldsymbol{A})\geq\text{r}(\boldsymbol{A}^{k - m}\boldsymbol{A}^m)+\text{r}(\boldsymbol{A}^m\boldsymbol{A})-\text{r}(\boldsymbol{A}^m)=\text{r}(\boldsymbol{A}^k),
\]
又\(\text{r}(\boldsymbol{A}^{k + 1})\leq\text{r}(\boldsymbol{A}^k)\),故\(\text{r}(\boldsymbol{A}^{k + 1})=\text{r}(\boldsymbol{A}^k)\)对任意的\(k\geq m\)成立,结论得证.

{\color{blue}证法二(几何方法):} 令\(\varphi\)为在\(n\)维列向量空间上由矩阵\(\boldsymbol{A}\)乘法诱导的线性变换,则\(\varphi\)在标准基下的表示矩阵就是\(\boldsymbol{A}\),并且不难发现对\(\forall k\in \mathbb{N}\),\(\varphi^k\)在标准基下的表示矩阵是\(\boldsymbol{A}^k\).因此\(\mathrm{r}(\boldsymbol{A}^k) = \mathrm{dim} \,\mathrm{Im}\varphi^k\),故原命题等价于$\mathrm{dim}\,\mathrm{Im}\varphi ^n=\mathrm{dim}\,\mathrm{Im}\varphi ^{n+1}=\mathrm{dim}\,\mathrm{Im}\varphi ^{n+2}=\cdots 
$.注意下列子空间链:
\[
V\supseteq\text{Im}\varphi\supseteq\text{Im}\varphi^2\supseteq\cdots\supseteq\text{Im}\varphi^n\supseteq\text{Im}\varphi^{n + 1}.
\]
上述\(n + 2\)个子空间的维数都在\([0,n]\)之间,故由抽屉原理可知,存在某个整数\(m\in[0,n]\),使得\(\text{Im}\varphi^m=\text{Im}\varphi^{m + 1}\). 现要证明对任意的\(k\geq m\),\(\text{Im}\varphi^k=\text{Im}\varphi^{k + 1}\). 一方面,\(\text{Im}\varphi^{k + 1}\subseteq\text{Im}\varphi^k\)是显然的. 另一方面,任取\(\boldsymbol{\alpha}\in\text{Im}\varphi^k\),则存在\(\boldsymbol{\beta}\in V\),使得\(\boldsymbol{\alpha}=\varphi^k(\boldsymbol{\beta})\). 由于\(\varphi^m(\boldsymbol{\beta})\in\text{Im}\varphi^m=\text{Im}\varphi^{m + 1}\),故存在\(\boldsymbol{\gamma}\in V\),使得\(\varphi^m(\boldsymbol{\beta})=\varphi^{m + 1}(\boldsymbol{\gamma})\),从而
\[
\boldsymbol{\alpha}=\varphi^k(\boldsymbol{\beta})=\varphi^{k - m}(\varphi^m(\boldsymbol{\beta}))=\varphi^{k - m}(\varphi^{m + 1}(\boldsymbol{\gamma}))=\varphi^{k + 1}(\boldsymbol{\gamma})\in\text{Im}\varphi^{k + 1},
\]
故\(\text{Im}\varphi^k=\text{Im}\varphi^{k + 1}\)对任意的\(k\geq m\)成立,取维数后即得结论. 

\item 注意下列子空间链:
\[
V\supseteq\text{Im}\varphi\supseteq\text{Im}\varphi^2\supseteq\cdots\supseteq\text{Im}\varphi^n\supseteq\text{Im}\varphi^{n + 1}.
\]
上述\(n + 2\)个子空间的维数都在\([0,n]\)之间,故由抽屉原理可知,存在某个整数\(m\in[0,n]\),使得\(\text{Im}\varphi^m=\text{Im}\varphi^{m + 1}\). 现要证明对任意的\(k\geq m\),\(\text{Im}\varphi^k=\text{Im}\varphi^{k + 1}\). 一方面,\(\text{Im}\varphi^{k + 1}\subseteq\text{Im}\varphi^k\)是显然的. 另一方面,任取\(\boldsymbol{\alpha}\in\text{Im}\varphi^k\),则存在\(\boldsymbol{\beta}\in V\),使得\(\boldsymbol{\alpha}=\varphi^k(\boldsymbol{\beta})\). 由于\(\varphi^m(\boldsymbol{\beta})\in\text{Im}\varphi^m=\text{Im}\varphi^{m + 1}\),故存在\(\boldsymbol{\gamma}\in V\),使得\(\varphi^m(\boldsymbol{\beta})=\varphi^{m + 1}(\boldsymbol{\gamma})\),从而
\[
\boldsymbol{\alpha}=\varphi^k(\boldsymbol{\beta})=\varphi^{k - m}(\varphi^m(\boldsymbol{\beta}))=\varphi^{k - m}(\varphi^{m + 1}(\boldsymbol{\gamma}))=\varphi^{k + 1}(\boldsymbol{\gamma})\in\text{Im}\varphi^{k + 1},
\]
故\(\text{Im}\varphi^k=\text{Im}\varphi^{k + 1}\)对任意的\(k\geq m\)成立,取维数后即得结论. 
\end{enumerate}
\end{proof}

\begin{proposition}\label{proposition:线性映射像空间与和空间等式链}
设\(\varphi\)是\(n\)维线性空间\(V\)上的线性变换,求证:必存在整数\(m\in[0,n]\),使得
\[
\text{Im}\varphi^m = \text{Im}\varphi^{m + 1} = \text{Im}\varphi^{m + 2} = \cdots, \text{Ker}\varphi^m = \text{Ker}\varphi^{m + 1} = \text{Ker}\varphi^{m + 2} = \cdots, V = \text{Im}\varphi^m\oplus\text{Ker}\varphi^m.
\]
\end{proposition}
\begin{proof}
根据\hyperref[corollary:n次方幂秩/值域维数等式]{推论\ref{corollary:n次方幂秩/值域维数等式}(2)}可知,存在整数\(m\in[0,n]\),使得
\[
\text{Im}\varphi^m = \text{Im}\varphi^{m + 1} = \text{Im}\varphi^{m + 2} = \cdots.
\]
注意到对任意的正整数\(i\),\(\text{Ker}\varphi^i\subseteq\text{Ker}\varphi^{i + 1}\). 再由维数公式可知,对任意的\(i\geq m\),\(\dim\text{Ker}\varphi^i=\dim V - \dim\text{Im}\varphi^i=n - \dim\text{Im}\varphi^m\)是一个不依赖于\(i\)的常数,因此由\hyperref[proposition:与全空间维数相同的子空间等于全空间]{命题\ref{proposition:与全空间维数相同的子空间等于全空间}}可得
\[
\text{Ker}\varphi^m = \text{Ker}\varphi^{m + 1} = \text{Ker}\varphi^{m + 2} = \cdots.
\]
若\(\boldsymbol{\alpha}\in\text{Im}\varphi^m\cap\text{Ker}\varphi^m\),则\(\boldsymbol{\alpha}=\varphi^m(\boldsymbol{\beta})\),\(\varphi^m(\boldsymbol{\alpha}) = 0\). 于是\(0=\varphi^m(\boldsymbol{\alpha})=\varphi^{2m}(\boldsymbol{\beta})\),即\(\boldsymbol{\beta}\in\text{Ker}\varphi^{2m}=\text{Ker}\varphi^m\),从而\(\boldsymbol{\alpha}=\varphi^m(\boldsymbol{\beta}) = 0\),这证明了\(\text{Im}\varphi^m\cap\text{Ker}\varphi^m = 0\). 又对\(V\)中任一向量\(\boldsymbol{\alpha}\),因为\(\varphi^m(\boldsymbol{\alpha})\in\text{Im}\varphi^m=\text{Im}\varphi^{2m}\),所以\(\varphi^m(\boldsymbol{\alpha})=\varphi^{2m}(\boldsymbol{\beta})\),其中\(\boldsymbol{\beta}\in V\). 我们有分解式
\[
\boldsymbol{\alpha}=\varphi^m(\boldsymbol{\beta})+(\boldsymbol{\alpha}-\varphi^m(\boldsymbol{\beta})).
\]
注意到\(\varphi^m(\boldsymbol{\alpha}-\varphi^m(\boldsymbol{\beta})) = 0\),即\(\boldsymbol{\alpha}-\varphi^m(\boldsymbol{\beta})\in\text{Ker}\varphi^m\),这就证明了\(V=\text{Im}\varphi^m+\text{Ker}\varphi^m\). 因此
\[
V = \text{Im}\varphi^m\oplus\text{Ker}\varphi^m.
\]
\end{proof}
\begin{remark}
也可不证明\(\text{Im}\varphi^m\cap\text{Ker}\varphi^m = 0\),改由\hyperref[proposition:值域和核空间维数之和等于原像空间维数]{线性映射维数公式}\(\dim\text{Im}\varphi^m+\dim\text{Ker}\varphi^m=n\)直接得到\(V=\text{Im}\varphi^m\oplus\text{Ker}\varphi^m\).
\end{remark}

\begin{proposition}\label{proposition:像空间和核空间的直和分解}
设\(V\)是数域\(\mathbb{K}\)上的\(n\)维线性空间,\(\varphi\)是\(V\)上的线性变换,证明以下9个结论等价:

(1) \(V=\text{Ker}\varphi\oplus\text{Im}\varphi\);

(2) \(V=\text{Ker}\varphi+\text{Im}\varphi\);

(3) \(\text{Ker}\varphi\cap\text{Im}\varphi = 0\);

(4) \(\text{Ker}\varphi=\text{Ker}\varphi^2\),或等价地,\(\dim\text{Ker}\varphi=\dim\text{Ker}\varphi^2\);

(5) \(\text{Ker}\varphi=\text{Ker}\varphi^2=\text{Ker}\varphi^3=\cdots\),或等价地,\(\dim\text{Ker}\varphi=\dim\text{Ker}\varphi^2=\dim\text{Ker}\varphi^3=\cdots\);

(6) \(\text{Im}\varphi=\text{Im}\varphi^2\),或等价地,\(\text{r}(\varphi)=\text{r}(\varphi^2)\);

(7) \(\text{Im}\varphi=\text{Im}\varphi^2=\text{Im}\varphi^3=\cdots\),或等价地,\(\text{r}(\varphi)=\text{r}(\varphi^2)=\text{r}(\varphi^3)=\cdots\);

(8) \(\text{Ker}\varphi\)存在\(\varphi\)-不变补空间,即存在\(\varphi\)-不变子空间\(U\),使得\(V=\text{Ker}\varphi\oplus U\)(实际上,$U = \text{Im}\varphi$);

(9) \(\text{Im}\varphi\)存在\(\varphi\)-不变补空间,即存在\(\varphi\)-不变子空间\(W\),使得\(V=\text{Im}\varphi\oplus W\)(实际上,$W = \text{Ker}\varphi$).
\end{proposition}
\begin{proof}
由直和的定义可知\((1)\Leftrightarrow(2)+(3)\),于是\((1)\Rightarrow(2)\)和\((1)\Rightarrow(3)\)都是显然的. 根据交空间维数公式和线性映射维数公式可知
\begin{align*}
\dim(\text{Ker}\varphi+\text{Im}\varphi)&=\dim\text{Ker}\varphi+\dim\text{Im}\varphi-\dim(\text{Ker}\varphi\cap\text{Im}\varphi)\\
&=\dim V-\dim(\text{Ker}\varphi\cap\text{Im}\varphi),
\end{align*}
于是\((2)\Leftrightarrow(3)\)成立,从而前3个结论两两等价.

\((3)\Rightarrow(4)\):显然\(\text{Ker}\varphi\subseteq\text{Ker}\varphi^2\)成立. 任取\(\boldsymbol{\alpha}\in\text{Ker}\varphi^2\),则\(\varphi(\boldsymbol{\alpha})\in\text{Ker}\varphi\cap\text{Im}\varphi = 0\),于是\(\varphi(\boldsymbol{\alpha}) = 0\),即\(\boldsymbol{\alpha}\in\text{Ker}\varphi\),从而\(\text{Ker}\varphi^2\subseteq\text{Ker}\varphi\)也成立,故\((4)\)成立.

\((4)\Rightarrow(3)\):任取\(\boldsymbol{\alpha}\in\text{Ker}\varphi\cap\text{Im}\varphi\),则存在\(\boldsymbol{\beta}\in V\),使得\(\boldsymbol{\alpha}=\varphi(\boldsymbol{\beta})\),于是\(0=\varphi(\boldsymbol{\alpha})=\varphi^2(\boldsymbol{\beta})\),即\(\boldsymbol{\beta}\in\text{Ker}\varphi^2=\text{Ker}\varphi\),从而\(\boldsymbol{\alpha}=\varphi(\boldsymbol{\beta}) = 0\),\((3)\)成立.

\((5)\Rightarrow(4)\)是显然的,下证\((4)\Rightarrow(5)\):设\(\text{Ker}\varphi^k=\text{Ker}\varphi^{k + 1}\)已对正整数\(k\)成立,先证\(\text{Ker}\varphi^{k + 1}=\text{Ker}\varphi^{k + 2}\)也成立,然后用归纳法即得结论. \(\text{Ker}\varphi^{k + 1}\subseteq\text{Ker}\varphi^{k + 2}\)是显然的. 任取\(\boldsymbol{\alpha}\in\text{Ker}\varphi^{k + 2}\),即\(0=\varphi^{k + 2}(\boldsymbol{\alpha})=\varphi^{k + 1}(\varphi(\boldsymbol{\alpha}))\),于是\(\varphi(\boldsymbol{\alpha})\in\text{Ker}\varphi^{k + 1}=\text{Ker}\varphi^k\),从而\(\varphi^{k + 1}(\boldsymbol{\alpha})=\varphi^k(\varphi(\boldsymbol{\alpha})) = 0\),即\(\boldsymbol{\alpha}\in\text{Ker}\varphi^{k + 1}\),于是\(\text{Ker}\varphi^{k + 2}\subseteq\text{Ker}\varphi^{k + 1}\)也成立.

\((3)\Leftrightarrow(6)\):考虑\(\varphi\)在不变子空间\(\text{Im}\varphi\)上的限制变换\(\varphi|_{\text{Im}\varphi}:\text{Im}\varphi\to\text{Im}\varphi\),由限制的定义可知它的核等于\(\text{Ker}\varphi\cap\text{Im}\varphi\),它的像等于\(\text{Im}\varphi^2\). 由于有限维线性空间上的线性变换是单射当且仅当它是满射,当且仅当它是同构,故\((3)\Leftrightarrow(6)\)成立.

\((7)\Rightarrow(6)\)是显然的,下证\((6)\Rightarrow(7)\):设\(\text{Im}\varphi^k=\text{Im}\varphi^{k + 1}\)已对正整数\(k\)成立,先证\(\text{Im}\varphi^{k + 1}=\text{Im}\varphi^{k + 2}\)也成立,然后用归纳法即得结论. \(\text{Im}\varphi^{k + 2}\subseteq\text{Im}\varphi^{k + 1}\)是显然的. 任取\(\boldsymbol{\alpha}\in\text{Im}\varphi^{k + 1}\),即存在\(\boldsymbol{\beta}\in V\),使得\(\boldsymbol{\alpha}=\varphi^{k + 1}(\boldsymbol{\beta})\). 由于\(\varphi^k(\boldsymbol{\beta})\in\text{Im}\varphi^k=\text{Im}\varphi^{k + 1}\),故存在\(\boldsymbol{\gamma}\in V\),使得\(\varphi^k(\boldsymbol{\beta})=\varphi^{k + 1}(\boldsymbol{\gamma})\),于是\(\boldsymbol{\alpha}=\varphi^{k + 1}(\boldsymbol{\beta})=\varphi(\varphi^k(\boldsymbol{\beta}))=\varphi(\varphi^{k + 1}(\boldsymbol{\gamma}))=\varphi^{k + 2}(\boldsymbol{\gamma})\in\text{Im}\varphi^{k + 2}\),从而\(\text{Im}\varphi^{k + 1}\subseteq\text{Im}\varphi^{k + 2}\)也成立.

(1) \(\Rightarrow\) (8) 是显然的,下证 (8) \(\Rightarrow\) (1).我们先证 \(\text{Im}\varphi\subseteq U\):任取 \(\varphi(\boldsymbol{v})\in\text{Im}\varphi\),由直和分解可设 \(\boldsymbol{v}=\boldsymbol{v}_1+\boldsymbol{u}\),其中 \(\boldsymbol{v}_1\in\text{Ker}\varphi\),\(\boldsymbol{u}\in U\),则由 \(U\) 的 \(\varphi -\)不变性可得 \(\varphi(\boldsymbol{v})=\varphi(\boldsymbol{v}_1)+\varphi(\boldsymbol{u})=\varphi(\boldsymbol{u})\in U\).考虑不等式
\[
\dim V=\dim(\text{Ker}\varphi\oplus U)=\dim\text{Ker}\varphi+\dim U\geq\dim\text{Ker}\varphi+\dim\text{Im}\varphi=\dim V,
\]
从而只能是 \(U = \text{Im}\varphi\),于是 (1) 成立.

(1) \(\Rightarrow\) (9) 是显然的,下证 (9) \(\Rightarrow\) (1).我们先证 \(W\subseteq\text{Ker}\varphi\):任取 \(\boldsymbol{w}\in W\),则由 \(W\) 的 \(\varphi -\)不变性可得 \(\varphi(\boldsymbol{w})\in\text{Im}\varphi\cap W = 0\),即有 \(\boldsymbol{w}\in\text{Ker}\varphi\).考虑不等式
\[
\dim V=\dim(\text{Im}\varphi\oplus W)=\dim\text{Im}\varphi+\dim W\leq\dim\text{Im}\varphi+\dim\text{Ker}\varphi=\dim V,
\]
从而只能是 \(W = \text{Ker}\varphi\),于是 (1) 成立.
\end{proof}

\begin{proposition}\label{proposition:由维数公式得到线性映射的存在性}
设\(U,W\)是\(n\)维线性空间\(V\)的子空间且\(\dim U+\dim W=\dim V\). 求证:存在\(V\)上的线性变换\(\varphi\),使得\(\text{Ker}\varphi = U\),\(\text{Im}\varphi = W\).
\end{proposition}
\begin{proof}
取\(U\)的一组基\(\boldsymbol{e}_1,\cdots,\boldsymbol{e}_m\),并将其扩张为\(V\)的一组基\(\boldsymbol{e}_1,\cdots,\boldsymbol{e}_m,\boldsymbol{e}_{m + 1},\cdots,\boldsymbol{e}_n\),再取\(W\)的一组基\(\boldsymbol{f}_{m + 1},\cdots,\boldsymbol{f}_n\). 定义\(\varphi\)为\(V\)上的线性变换,它在基上的作用为:\(\varphi(\boldsymbol{e}_i)=\boldsymbol{0}(1\leq i\leq m)\),\(\varphi(\boldsymbol{e}_j)=\boldsymbol{f}_j(m + 1\leq j\leq n)\). 注意到\(\boldsymbol{f}_{m + 1},\cdots,\boldsymbol{f}_n\)是\(W\)的一组基,故通过简单的验证可得\(\text{Ker}\varphi = U\),\(\text{Im}\varphi = W\). 
\end{proof}

\begin{example}
设\(V = M_n(\mathbb{F})\)是\(\mathbb{F}\)上\(n\)阶矩阵全体构成的线性空间,\(\varphi:V\to\mathbb{F}\)是迹函数,即对任意的\(\boldsymbol{A}=(a_{ij})\in V\),
\[
\varphi(\boldsymbol{A})=a_{11}+a_{22}+\cdots + a_{nn}.
\]
求证:\(\varphi\)是\(V\)到一维空间\(\mathbb{F}\)上的线性映射,并求\(\text{Ker}\varphi\)的维数及其一组基.
\end{example}
\begin{proof}
容易验证\(\varphi\)是线性映射且是映上的. 注意到\(V\)是
$n^2$维线性空间,由线性映射的维数公式可知
\[
\mathrm{dimKer}\varphi =\mathrm{dim}V-\mathrm{dim}\,\mathrm{Im}\varphi =\mathrm{dim}V-\mathrm{dim}\,\mathbb{F} =n^2-1.
\]
记\(\boldsymbol{E}_{ij}\)为\(n\)阶基础矩阵,即第\((i,j)\)元素为\(1\),其余元素为\(0\)的矩阵. 容易验证下列\(n^2 - 1\)个矩阵迹为零且线性无关,因此它们组成了\(\text{Ker}\varphi\)的一组基:
\[
\boldsymbol{E}_{ij}(i\neq j),\boldsymbol{E}_{11}-\boldsymbol{E}_{22},\boldsymbol{E}_{22}-\boldsymbol{E}_{33},\cdots,\boldsymbol{E}_{n - 1,n - 1}-\boldsymbol{E}_{nn}. 
\]
\end{proof}

\begin{example}
设\(\varphi\)是有限维线性空间\(V\)到\(U\)的线性映射,且\(V\)的维数大于\(U\)的维数,求证:\(\text{Ker}\varphi\neq 0\).
\end{example}
\begin{proof}
由线性映射的维数公式
\[
\dim V=\dim\text{Im}\varphi+\dim\text{Ker}\varphi,
\]
以及\(\dim\text{Im}\varphi\leq\dim U<\dim V\)可得\(\dim\text{Ker}\varphi>0\),即\(\text{Ker}\varphi\neq 0\).
\end{proof}

\begin{example}
设\(\varphi\)是有限维线性空间\(V\)到\(U\)的满线性映射,求证:必存在\(V\)的子空间\(W\),使得\(V = W\oplus\text{Ker}\varphi\),且\(\varphi\)在\(W\)上的限制是\(W\)到\(U\)上的线性同构.
\end{example}
\begin{proof}
{\color{blue}证法一:}取\(\text{Ker}\varphi\)的一组基\(\boldsymbol{e}_1,\cdots,\boldsymbol{e}_k\),并将其扩张为\(V\)的一组基\(\boldsymbol{e}_1,\cdots,\boldsymbol{e}_k,\boldsymbol{e}_{k + 1},\cdots,\boldsymbol{e}_n\). 令\(W = L(\boldsymbol{e}_{k + 1},\cdots,\boldsymbol{e}_n)\),则显然\(V = W\oplus\text{Ker}\varphi\). 由\hyperref[corollary:由核的基导出值域的基]{推论\ref{corollary:由核的基导出值域的基}}可知,\(\varphi(\boldsymbol{e}_{k + 1}),\cdots,\varphi(\boldsymbol{e}_n)\)是\(\text{Im}\varphi = U\)的一组基,故\(\varphi\)在\(W\)上的限制将\(W\)的一组基\(\boldsymbol{e}_{k + 1},\cdots,\boldsymbol{e}_n\)映射为\(U\)的一组基\(\varphi(\boldsymbol{e}_{k + 1}),\cdots,\varphi(\boldsymbol{e}_n)\),从而由\hyperref[proposition:线性变换是可逆变换的充要条件1]{命题\ref{proposition:线性变换是可逆变换的充要条件1}}可知$\varphi \mid_{W}$必为线性同构.

{\color{blue}证法二:} 取\(W\)为\(\text{Ker}\varphi\)在\(V\)中的补空间. 对任意的\(\boldsymbol{u}\in U\),由于\(\varphi\)是映上的,故存在\(\boldsymbol{v}=\boldsymbol{w}+\boldsymbol{v}_1\),其中\(\boldsymbol{w}\in W\),\(\boldsymbol{v}_1\in\text{Ker}\varphi\),使得\(\boldsymbol{u}=\varphi(\boldsymbol{v})=\varphi(\boldsymbol{w})\),于是\(\varphi\)在\(W\)上的限制也是映上的,故$\dim U=\dim \text{Im}\varphi|_W$.另一方面,由维数公式可知,\(\dim W=\dim V-\dim\text{Ker}\varphi=\mathrm{dim}\,\mathrm{Im}\varphi =\dim U\). 再对\(\varphi\)在\(W\)上的限制用\hyperref[proposition:值域和核空间维数之和等于原像空间维数]{线性映射维数公式}可知,$\mathrm{dim}\,\mathrm{Ker}\varphi |_W=\mathrm{dim}W-\mathrm{dim}\,\mathrm{Im}\varphi |_W=\mathrm{dim}U-\mathrm{dim}\,\mathrm{Im}\varphi |_W=0$.从而它必是单映射,于是\(\varphi\)在\(W\)上的限制是\(W\)到\(U\)上的线性同构. 
\end{proof}

\begin{example}
设\(\varphi\)是有限维线性空间\(V\)到\(V'\)的线性映射,\(U\)是\(V'\)的子空间且\(U\subseteq\text{Im}\varphi\),求证:\(\varphi^{-1}(U)=\{\boldsymbol{v}\in V|\varphi(\boldsymbol{v})\in U\}\)是\(V\)的子空间,且
\[
\dim U+\dim\text{Ker}\varphi=\dim\varphi^{-1}(U).
\]
\end{example}
\begin{note}
注意对线性映射做限制这个操作.
\end{note}
\begin{proof}
容易验证\(\varphi^{-1}(U)\)是\(V\)的子空间. 将\(\varphi\)限制在\(\varphi^{-1}(U)\)上,它是到\(U\)上的线性映射. 因为\(\boldsymbol{0}\in U\),故\(\text{Ker}\varphi\subseteq\varphi^{-1}(U)\). 从而$\text{Ker}\varphi |_{\varphi ^{-1}\left( U \right)}=\mathrm{Ker}\varphi \cap \varphi ^{-1}\left( U \right) =\mathrm{Ker}\varphi $,又显然$\text{Im}\varphi |_{\varphi ^{-1}\left( U \right)}=U$.再对\(\varphi\)在\(\varphi^{-1}(U)\)上的限制用\hyperref[proposition:值域和核空间维数之和等于原像空间维数]{线性映射维数公式}即得
\[
\mathrm{dim}\varphi ^{-1}\left( U \right) =\mathrm{dim}\,\mathrm{Im}\varphi |_{\varphi ^{-1}\left( U \right)}+\mathrm{dim}\,\mathrm{Ker}\varphi |_{\varphi ^{-1}\left( U \right)}=\mathrm{dim}U+\mathrm{dimKer}\varphi .
\]
\end{proof}
\begin{remark}
\(\varphi^{-1}(U)=\{\boldsymbol{v}\in V|\varphi(\boldsymbol{v})\in U\}\)称为$U$在线性映射$\varphi$下的原像集.
\end{remark}

\begin{proposition}\label{proposition:线性变换关于子空间的维数不等式}
设\(U\)是有限维线性空间\(V\)的子空间,\(\varphi\)是\(V\)上的线性变换,求证:
\begin{enumerate}[(1)]
\item \(\dim U-\dim\text{Ker}\varphi\leq\dim\varphi(U)\leq\dim U\);

\item \(\dim\varphi^{-1}(U)\leq\dim U+\dim\text{Ker}\varphi\).
\end{enumerate}
\end{proposition}
\begin{proof}
\begin{enumerate}[(1)]
\item 注意到当\(\varphi\)限制在\(U\)上时,\(\text{Ker}(\varphi|_U)=U\cap\text{Ker}\varphi\),故由线性映射的维数公式可得
\[
\mathrm{dim}U=\mathrm{dim(}U\cap \mathrm{Ker}\varphi )+\mathrm{dim}\varphi (U)\leqslant \mathrm{dim}\,\mathrm{Ker}\varphi +\mathrm{dim}\varphi (U).
\]
于是
\[
\dim U - \dim\text{Ker}\varphi\leq\dim\varphi(U),
\]
而由\hyperref[proposition:值域和核空间维数之和等于原像空间维数]{线性映射维数公式},可知$\mathrm{dim}U=\mathrm{dim}\,\mathrm{Ker}\varphi |_U+\mathrm{dim}\varphi (U)$.进而\(\dim\varphi(U)\leq\dim U\).

\item 设\(\overline{\varphi}\)是线性变换\(\varphi\)在子空间\(\varphi^{-1}(U)\)上的限制,则\(\text{Im}\overline{\varphi}=U\cap\text{Im}\varphi\).因为\(\boldsymbol{0}\in U\),故\(\text{Ker}\varphi\subseteq\varphi^{-1}(U)\).从而\(\text{Ker}\overline{\varphi}=\text{Ker}\varphi\cap\varphi^{-1}(U)=\text{Ker}\varphi\). 由线性映射的维数公式可得
\[
\dim\varphi^{-1}(U)=\dim(U\cap\text{Im}\varphi)+\dim\text{Ker}\varphi.
\]
显然,由\(\dim(U\cap\text{Im}\varphi)\leq\dim U\)可推出
\[
\dim\varphi^{-1}(U)\leq\dim U+\dim\text{Ker}\varphi. 
\]
\end{enumerate}
\end{proof}

\begin{example}\label{example:4.243456}
证明:若\(\boldsymbol{A},\boldsymbol{B}\)是数域\(\mathbb{F}\)上两个\(n\)阶方阵,则
\[
\text{r}(\boldsymbol{A})+\text{r}(\boldsymbol{B})-n\leq\text{r}(\boldsymbol{A}\boldsymbol{B})\leq\min\{\text{r}(\boldsymbol{A}),\text{r}(\boldsymbol{B})\}.
\]
\end{example}
\begin{proof}
令\(V\)是\(\mathbb{F}\)上\(n\)维列向量空间,则将\(\boldsymbol{A}\)和\(\boldsymbol{B}\)看成是\(V\)上由矩阵\(\boldsymbol{A}\)和\(\boldsymbol{B}\)乘法诱导的线性变换. 又令\(U = \boldsymbol{B}(V)\),注意到\(\boldsymbol{A}(U)=\boldsymbol{A}\boldsymbol{B}(V)\),故\(\mathrm{dim}\boldsymbol{A}(U)=\mathrm{dim}\boldsymbol{AB}(V)=\mathrm{r(}\boldsymbol{AB})\),\(\dim\text{Ker}\boldsymbol{A}=n - \text{r}(\boldsymbol{A})\),即线性方程组\(\boldsymbol{A}\boldsymbol{x}=\boldsymbol{0}\)的解空间维数. 而\(\dim U=\dim\boldsymbol{B}(V)=\text{r}(\boldsymbol{B})\),由\hyperref[proposition:线性变换关于子空间的维数不等式]{命题\ref{proposition:线性变换关于子空间的维数不等式}(1)}的结论,可得
\[
\mathrm{r(}\boldsymbol{A})+\mathrm{r(}\boldsymbol{B})-n=\mathrm{dim}U-\mathrm{dim}\,\mathrm{Ker}\boldsymbol{A}\leqslant \mathrm{dim}\boldsymbol{A}\left( U \right) =\mathrm{r(}\boldsymbol{AB})\leqslant \mathrm{dim}U=\mathrm{r}\left( \boldsymbol{B} \right).
\]
又显然有\(\dim\boldsymbol{A}(U)\leq\dim\boldsymbol{A}(V)\),故得\(\text{r}(\boldsymbol{A}\boldsymbol{B})\leq\text{r}(\boldsymbol{A})\).
\end{proof}



\section{不变子空间}

\begin{example}
设线性空间\(V\)上的线性变换\(\varphi\)在基\(\{\boldsymbol{e}_1,\boldsymbol{e}_2,\boldsymbol{e}_3,\boldsymbol{e}_4\}\)下的表示矩阵为
\[
\boldsymbol{A}=\begin{pmatrix}
1&0&2&-1\\
0&1&4&-2\\
2&-1&0&1\\
2&-1&-1&2
\end{pmatrix},
\]
求证:\(U = L(\boldsymbol{e}_1 + 2\boldsymbol{e}_2,\boldsymbol{e}_3 + \boldsymbol{e}_4,\boldsymbol{e}_1 + \boldsymbol{e}_2)\)和\(W = L(\boldsymbol{e}_2 + \boldsymbol{e}_3 + 2\boldsymbol{e}_4)\)都是\(\varphi\)的不变子空间.
\end{example}
\begin{proof}
要证明由若干个向量生成的子空间是某个线性变换的不变子空间,通常只需证明这些向量在线性变换的作用下仍在这个子空间中即可.因此只需证明这些子空间的一组基在线性变换的作用下仍在这个子空间中即可.注意到\(\varphi(\boldsymbol{e}_1 + 2\boldsymbol{e}_2)\)的坐标向量为
\[
\begin{pmatrix}
1&0&2&-1\\
0&1&4&-2\\
2&-1&0&1\\
2&-1&-1&2
\end{pmatrix}
\begin{pmatrix}
1\\
2\\
0\\
0
\end{pmatrix}=
\begin{pmatrix}
1\\
2\\
0\\
0
\end{pmatrix},
\]
即\(\varphi(\boldsymbol{e}_1 + 2\boldsymbol{e}_2)=\boldsymbol{e}_1 + 2\boldsymbol{e}_2\in U\). 同理可计算出
\begin{align*}
\varphi(\boldsymbol{e}_3 + \boldsymbol{e}_4)&=(\boldsymbol{e}_1 + 2\boldsymbol{e}_2)+(\boldsymbol{e}_3 + \boldsymbol{e}_4)\in U,\\
\varphi(\boldsymbol{e}_1 + \boldsymbol{e}_2)&=(\boldsymbol{e}_1 + \boldsymbol{e}_2)+(\boldsymbol{e}_3 + \boldsymbol{e}_4)\in U,\\
\varphi(\boldsymbol{e}_2 + \boldsymbol{e}_3 + 2\boldsymbol{e}_4)&=\boldsymbol{e}_2 + \boldsymbol{e}_3 + 2\boldsymbol{e}_4\in W,
\end{align*}
因此结论成立. 
\end{proof}

\begin{proposition}\label{proposition:不变子空间的子空间、交与和仍是不变子空间}
设\(V_1,V_2\)是\(V\)上线性变换\(\varphi\)的不变子空间,任取$V_0\subset V$,求证:,$V_0$,\(V_1\cap V_2\),\(V_1 + V_2\)也是\(\varphi\)的不变子空间.
\end{proposition}
\begin{proof}
$V_0$是\(\varphi -\)不变子空间是显然的.

任取\(\boldsymbol{v}\in V_1\cap V_2\),则由\(\boldsymbol{v}\in V_i\)可得\(\varphi(\boldsymbol{v})\in V_i(i = 1,2)\),于是\(\varphi(\boldsymbol{v})\in V_1\cap V_2\),从而\(V_1\cap V_2\)是\(\varphi -\)不变子空间.

任取\(\boldsymbol{v}\in V_1 + V_2\),则\(\boldsymbol{v}=\boldsymbol{v}_1+\boldsymbol{v}_2\),其中\(\boldsymbol{v}_i\in V_i\),故\(\varphi(\boldsymbol{v}_i)\in V_i(i = 1,2)\),于是\(\varphi(\boldsymbol{v})=\varphi(\boldsymbol{v}_1)+\varphi(\boldsymbol{v}_2)\in V_1 + V_2\),从而\(V_1 + V_2\)是\(\varphi -\)不变子空间.
\end{proof}

\begin{proposition}\label{proposition:纯量变换关于不变子空间的等价条件}
设\(\varphi\)是\(n(n\geq 2)\)维线性空间\(V\)上的线性变换,证明以下\(n+1\)个结论等价:

(1) \(V\)的任一\(1\)维子空间都是\(\varphi -\)不变子空间;

\(\cdots\cdots\)

(r) \(V\)的任一\(r\)维子空间都是\(\varphi -\)不变子空间;

\(\cdots\cdots\)

(n - 1) \(V\)的任一\(n - 1\)维子空间都是\(\varphi -\)不变子空间;

(n)\(V\)本身就是\(\varphi -\)不变子空间;

(n+1) \(\varphi\)是纯量变换.
\end{proposition}
\begin{proof}
(n+1) \(\Rightarrow\) (n)是显然的.
注意到当\(1\leq i\leq n - 2\)时,任一\(i\)维子空间\(V_0\)都可表示为两个\(i + 1\)维子空间\(V_1,V_2\)的交;而$V$的任意$n-1$维子空间都是$V$的子空间.于是由\hyperref[proposition:不变子空间的子空间、交与和仍是不变子空间]{命题\ref{proposition:不变子空间的子空间、交与和仍是不变子空间}}可知:(n) \(\Rightarrow\) (n - 1) \(\Rightarrow\) (n - 2) \(\Rightarrow\cdots\Rightarrow\) (1) 显然成立,剩下只要证明 (1) \(\Rightarrow\) (n+1) 即可. 取\(V\)的一组基\(\{\boldsymbol{e}_1,\boldsymbol{e}_2,\cdots,\boldsymbol{e}_n\}\),由 (1) 可知$L\left( \boldsymbol{e}_1 \right) ,L\left( \boldsymbol{e}_2 \right) ,\cdots ,L\left( \boldsymbol{e}_n \right)$都是\(\varphi -\)不变子空间,设\(\varphi(\boldsymbol{e}_i)=\lambda_i\boldsymbol{e}_i(1\leq i\leq n)\). 只要证明\(\lambda_1=\lambda_2=\cdots=\lambda_n\)即可得到\(\varphi\)为纯量变换. 用反证法,不妨设\(\lambda_1\neq\lambda_2\),则由\(L(\boldsymbol{e}_1+\boldsymbol{e}_2)\)也是\(\varphi -\)不变子空间可设\(\varphi(\boldsymbol{e}_1+\boldsymbol{e}_2)=\lambda_0(\boldsymbol{e}_1+\boldsymbol{e}_2)\),于是\((\lambda_1 - \lambda_0)\boldsymbol{e}_1+(\lambda_2 - \lambda_0)\boldsymbol{e}_2=\boldsymbol{0}\),从而由$\boldsymbol{e}_1,\boldsymbol{e}_2$线性无关可知\(\lambda_1=\lambda_2=\lambda_0\),矛盾.
\end{proof}

\begin{proposition}\label{proposition:乘法可交换的线性变换值域和核互为不变子空间}
设\(\varphi,\psi\)是线性空间\(V\)上的线性变换且\(\varphi\psi = \psi\varphi\),求证:\(\text{Im}\varphi\)及\(\text{Ker}\varphi\)都是\(\psi\)的不变子空间.同理,\(\text{Im}\psi\)及\(\text{Ker}\psi\)也都是\(\varphi\)的不变子空间.
\end{proposition}
\begin{note}
显然\(\text{Im}\varphi\)及\(\text{Ker}\varphi\)都是\(\varphi\)自身的不变子空间,\(\text{Im}\psi\)及\(\text{Ker}\psi\)也都是\(\psi\)自身的不变子空间.
\end{note}
\begin{proof}
任取\(\boldsymbol{v}\in\text{Im}\varphi\),即\(\boldsymbol{v}=\varphi(\boldsymbol{u})\),则\(\psi(\boldsymbol{v})=\psi\varphi(\boldsymbol{u})=\varphi\psi(\boldsymbol{u})\in\text{Im}\varphi\),即\(\text{Im}\varphi\)是\(\psi\)的不变子空间.

任取\(\boldsymbol{v}\in\text{Ker}\varphi\),即\(\varphi(\boldsymbol{v}) = 0\),则\(\varphi\psi(\boldsymbol{v})=\psi\varphi(\boldsymbol{v}) = 0\). 因此,\(\psi(\boldsymbol{v})\in\text{Ker}\varphi\),即\(\text{Ker}\varphi\)是\(\psi\)的不变子空间.
\end{proof}

\begin{proposition}\label{proposition:线性变换在其不变子空间下的限制}
设\(\varphi\)是\(n\)维线性空间\(V\)上的线性变换,\(W\)为\(\varphi -\)不变子空间,\(\varphi\)在\(W\)上的限制为\(\varphi|_W\),
则\(\varphi|_W\)的像集与原像集相同且均为\(W\),对\(\forall k\in \mathbb{N}\),\((\varphi|_W)^k\)有意义并且\(\varphi^k|_W = (\varphi|_W)^k\).    
\end{proposition}
\begin{proof}
因为\(W\)为\(\varphi -\)不变子空间,所以\(\varphi|_W\)的像集与原像集相同且均为\(W\)是显然的.
并且对\(\forall \alpha \in W\),有\(\varphi|_W(\alpha) \in W\).因此对\(\forall k \in \mathbb{N}\),有\((\varphi|_W)^k(\alpha) \in W\).故\((\varphi|_W)^k\)有意义.下证\(\varphi^k|_W = (\varphi|_W)^k\).

对\(\forall k\in \mathbb{N}\),显然\(\varphi^k|_W\)和\((\varphi|_W)^k\)的定义域都是\(W\).从而对\(\forall a\in W\),有
\[
\varphi^k|_W(a)=\varphi^k(a),
\]
\[
(\varphi|_W)^k(a)=(\varphi|_W)^{k - 1}\varphi|_W(a)=(\varphi|_W)^{k - 1}\varphi(a)=\cdots=\varphi^k(a).
\]
因此\(\varphi^k|_W(a)=(\varphi|_W)^k(a)=\varphi^k(a)\).故\(\varphi^k|_W = (\varphi|_W)^k\).
\end{proof}

\begin{proposition}\label{proposition:自同构在其不变子空间下的限制}
设\(\varphi\)是\(n\)维线性空间\(V\)上的自同构,\(W\)为\(\varphi -\)不变子空间,\(\varphi\)在\(W\)上的限制为\(\varphi|_W\),
则\(\varphi|_W\)的像集与原像集相同且均为\(W\),\(\varphi|_W\)是\(W\)上的自同构并且\((\varphi|_W)^{-1} = \varphi^{-1}|_W\).
\end{proposition}
\begin{proof}

\end{proof}


\begin{example}
设\(\boldsymbol{A}\)为数域\(\mathbb{K}\)上的\(n\)阶幂零阵,\(\boldsymbol{B}\)为\(n\)阶方阵,满足\(\boldsymbol{A}\boldsymbol{B}=\boldsymbol{B}\boldsymbol{A}\)且\(\text{r}(\boldsymbol{A}\boldsymbol{B})=\text{r}(\boldsymbol{B})\). 求证:\(\boldsymbol{B}=\boldsymbol{O}\).
\end{example}
\begin{remark}
因为$\text{Im}\boldsymbol{B}$是$\boldsymbol{A}-$不变子空间,所以\(\boldsymbol{A}|_{\text{Im}\boldsymbol{B}}(\text{Im}\boldsymbol{B})\in \text{Im}\boldsymbol{B}\).因此$(\boldsymbol{A}|_{\text{Im}\boldsymbol{B}})^k $有意义.
对于一般的限制$W$,$(\boldsymbol{A}|_{W})^k $不一定有意义.见\hyperref[proposition:线性变换在其不变子空间下的限制]{命题\ref{proposition:线性变换在其不变子空间下的限制}}.
\end{remark}
\begin{proof}
将\(\boldsymbol{A},\boldsymbol{B}\)都看成是\(\mathbb{K}^n\)上(由矩阵\(\boldsymbol{A},\boldsymbol{B}\)乘法诱导)的线性变换,设\(\boldsymbol{A}^k = \boldsymbol{O}\),其中\(k\)为正整数. 由\(\boldsymbol{A}\boldsymbol{B}=\boldsymbol{B}\boldsymbol{A}\)以及\hyperref[proposition:乘法可交换的线性变换值域和核互为不变子空间]{命题\ref{proposition:乘法可交换的线性变换值域和核互为不变子空间}}可知\(\text{Im}\boldsymbol{B}\)是\(\boldsymbol{A}-\)不变子空间. 考虑\(\boldsymbol{A}\)在\(\text{Im}\boldsymbol{B}\)上的限制\(\boldsymbol{A}|_{\text{Im}\boldsymbol{B}}\),其像空间的维数\(\dim\boldsymbol{A}\boldsymbol{B}(\mathbb{K}^n)=\text{r}(\boldsymbol{A}\boldsymbol{B})=\text{r}(\boldsymbol{B})=\dim\text{Im}\boldsymbol{B}\),故\(\boldsymbol{A}|_{\text{Im}\boldsymbol{B}}\)是\(\text{Im}\boldsymbol{B}\)上的满线性变换. 于是由\hyperref[proposition:线性变换在其不变子空间下的限制]{命题\ref{proposition:线性变换在其不变子空间下的限制}}和\hyperref[proposition:满射的复合仍是满射]{满射的复合仍是满射}可知\((\boldsymbol{A}|_{\text{Im}\boldsymbol{B}})^k = \boldsymbol{A}^k|_{\text{Im}\boldsymbol{B}}=\boldsymbol{O}|_{\text{Im}\boldsymbol{B}}\)也是\(\text{Im}\boldsymbol{B}\)上的满线性变换,从而只能是\(\text{Im}\boldsymbol{B}=0\),即\(\boldsymbol{B}=\boldsymbol{O}\). 
\end{proof}

\begin{proposition}\label{proposition:可逆线性变换的不变子空间仍是其逆的不变子空间}
设\(\varphi\)是\(n\)维线性空间\(V\)上的自同构,若\(W\)是\(\varphi\)的不变子空间,求证:\(W\)也是\(\varphi^{-1}\)的不变子空间.
\end{proposition}
\begin{proof}
将\(\varphi\)限制在\(W\)上,得到$\varphi :W\rightarrow W$.它是\(W\)上的线性变换. 由于\(\varphi\)是单映射,故它在\(W\)上的限制也是单映射,从而由\hyperref[corollary:线性变换自同构的充要条件]{推论\ref{corollary:线性变换自同构的充要条件}}可知,\(\varphi\)在\(W\)上的限制也是满映射,即它是\(W\)上的自同构,于是结合\hyperref[proposition:自同构在其不变子空间下的限制]{命题\ref{proposition:自同构在其不变子空间下的限制}}可知\(W=\varphi |_W\left( W \right) =\varphi \left( W \right) \),对其两边同时取$\varphi ^{-1}$可得$\varphi ^{-1}\left( W \right) =W$.故结论得证.
\end{proof}
\begin{remark}
如果\(V\)是无限维线性空间,则这个命题的结论一般并不成立. 例如,\(V = \mathbb{K}[x^{-1},x]\)是由数域\(\mathbb{K}\)上的Laurent多项式\(f(x)=\sum_{i = -m}^{n}a_ix^i(m,n\in\mathbb{N})\)构成的线性空间,\(V\)上的线性变换\(\varphi,\psi\)定义为\(\varphi(f(x)) = xf(x)\),\(\psi(f(x)) = x^{-1}f(x)\). 显然,\(\varphi,\psi\)互为逆映射,从而都是自同构. 注意到\(W = \mathbb{K}[x]\)是\(V\)的\(\varphi -\)不变子空间,但\(W\)显然不是\(\varphi^{-1}-\)不变子空间.
\end{remark}

\begin{example}
设\(V\)是次数小于\(n\)的实系数多项式组成的线性空间,\(D\)是\(V\)上的求导变换. 求证:\(D\)的任一\(k(k\geq 1)\)维不变子空间必是由\(\{1,x,\cdots,x^{k - 1}\}\)生成的子空间. 特别地,向量\(1\)包含在\(D\)的任一非零不变子空间中.
\end{example}
\begin{proof}
任取\(D\)的一个\(k(k\geq 1)\)维不变子空间\(V_0\),再取出\(V_0\)中次数最高的一个多项式(不唯一)\(f(x)=a_lx^l + a_{l - 1}x^{l - 1}+\cdots + a_1x + a_0\),其中\(a_l\neq 0\). 注意到\(V_0\)是\(D -\)不变子空间,由\(D^lf(x)=a_ll!\in V_0\)可得\(1\in V_0\);由\(D^{l - 1}f(x)=a_ll!x + a_{l - 1}(l - 1)!\in V_0\)可得\(x\in V_0\);\(\cdots\);由\(Df(x)=a_llx^{l - 1}+a_{l - 1}(l - 1)x^{l - 2}+\cdots + a_1\in V_0\)可得\(x^{l - 1}\in V_0\);最后由\(f(x)\in V_0\)可得\(x^l\in V_0\). 因为\(V_0\)中所有多项式的次数都小于等于\(l\),所以\(\{1,x,\cdots,x^l\}\)构成了\(V_0\)的一组基,于是\(k = \dim V_0 = l + 1\),即\(l = k - 1\),从而结论得证.
\end{proof}

\begin{example}
设\(\varphi\)是\(n\)维线性空间\(V\)上的线性变换,\(\varphi\)在\(V\)的一组基下的表示矩阵为对角阵且主对角线上的元素互不相同,求\(\varphi\)的所有不变子空间.
\end{example}
\begin{proof}
设\(\varphi\)在基\(\boldsymbol{e}_1,\boldsymbol{e}_2,\cdots,\boldsymbol{e}_n\)下的表示矩阵为\(\text{diag}\{d_1,d_2,\cdots,d_n\}\),其中\(d_1,d_2,\cdots,d_n\)互不相同,则\(\varphi(\boldsymbol{e}_i)=d_i\boldsymbol{e}_i\). 对任意的指标集\(1\leq i_1 < i_2 < \cdots < i_r\leq n\),容易验证\(U = L(\boldsymbol{e}_{i_1},\boldsymbol{e}_{i_2},\cdots,\boldsymbol{e}_{i_r})\)是\(\varphi\)的不变子空间. 注意到\(1,2,\cdots,n\)的子集共有\(2^n\)个(空集对应于零子空间),故上述形式的\(\varphi -\)不变子空间共有\(2^n\)个. 下面我们证明\(\varphi\)的任一不变子空间都是上述不变子空间之一.

任取\(\varphi\)的非零不变子空间\(U\),设指标集
\[
I = \{i\in[1,n]|\text{存在某个}\boldsymbol{\alpha}\in U,\text{使得}\boldsymbol{\alpha}=c\boldsymbol{e}_i + \cdots,\text{其中}c\neq 0\}.
\]
因为\(U\neq 0\),故\(I\neq\varnothing\),不妨设\(I = \{i_1,i_2,\cdots,i_r\}\). 由指标集\(I\)的定义可知,\(U\subseteq L(\boldsymbol{e}_{i_1},\boldsymbol{e}_{i_2},\cdots,\boldsymbol{e}_{i_r})\). 下面我们证明\(\boldsymbol{e}_{i_j}\in U(j = 1,2,\cdots,r)\)成立. 不失一般性,我们只需证明\(\boldsymbol{e}_{i_1}\in U\)即可. 由指标集\(I\)的定义可知,存在\(\boldsymbol{\alpha}\in U\),使得
\[
\boldsymbol{\alpha}=c_1\boldsymbol{e}_{i_1}+c_2\boldsymbol{e}_{i_2}+\cdots + c_k\boldsymbol{e}_{i_k},
\]
其中\(c_1,c_2,\cdots,c_k\)都是非零常数. 将上式作用\(\varphi^l\),可得
\[
\varphi^l(\boldsymbol{\alpha})=c_1d_{i_1}^l\boldsymbol{e}_{i_1}+c_2d_{i_2}^l\boldsymbol{e}_{i_2}+\cdots + c_kd_{i_k}^l\boldsymbol{e}_{i_k},l = 1,2,\cdots,k - 1.
\]
因此,我们有
\[
(\boldsymbol{\alpha},\varphi(\boldsymbol{\alpha}),\cdots,\varphi^{k - 1}(\boldsymbol{\alpha}))=(\boldsymbol{e}_{i_1},\boldsymbol{e}_{i_2},\cdots,\boldsymbol{e}_{i_k})
\begin{pmatrix}
c_1&c_1d_{i_1}&\cdots&c_1d_{i_1}^{k - 1}\\
c_2&c_2d_{i_2}&\cdots&c_2d_{i_2}^{k - 1}\\
\vdots&\vdots&&\vdots\\
c_k&c_kd_{i_k}&\cdots&c_kd_{i_k}^{k - 1}
\end{pmatrix}.
\]
上式右边的矩阵记为\(\boldsymbol{A}\),由于\(|\boldsymbol{A}| = c_1c_2\cdots c_k\prod_{1\leq r < s\leq k}(d_{i_s}-d_{i_r})\neq 0\),故\(\boldsymbol{A}\)为可逆矩阵,从而
\[
(\boldsymbol{e}_{i_1},\boldsymbol{e}_{i_2},\cdots,\boldsymbol{e}_{i_k})=(\boldsymbol{\alpha},\varphi(\boldsymbol{\alpha}),\cdots,\varphi^{k - 1}(\boldsymbol{\alpha}))\boldsymbol{A}^{-1},
\]
特别地,\(\boldsymbol{e}_{i_1}\)可以表示为\(\boldsymbol{\alpha},\varphi(\boldsymbol{\alpha}),\cdots,\varphi^{k - 1}(\boldsymbol{\alpha})\)的线性组合. 因为\(U\)是\(\varphi\)的不变子空间,故\(\boldsymbol{\alpha},\varphi(\boldsymbol{\alpha}),\cdots,\varphi^{k - 1}(\boldsymbol{\alpha})\)都是\(U\)中的向量,从而\(\boldsymbol{e}_{i_1}\in U\),因此\(U = L(\boldsymbol{e}_{i_1},\boldsymbol{e}_{i_2},\cdots,\boldsymbol{e}_{i_r})\). 综上所述,\(\varphi\)的不变子空间共有\(2^n\)个. 
\end{proof}

\begin{theorem}\label{theorem:在不变子空间基下的矩阵}
设\(\varphi\)是数域\(\mathbb{F}\)上向量空间\(V\)上的线性变换,\(W\)是\(\varphi\)的不变子空间. 若取\(W\)的一组基\(\{e_1,\cdots,e_r\}\),再扩张为\(V\)的一组基\(\{e_1,\cdots,e_r,e_{r + 1},\cdots,e_n\}\),则\(\varphi\)在这组基下的表示矩阵具有下列分块上三角矩阵的形状:
\[
\begin{pmatrix}
\boldsymbol{A}_{11}&\boldsymbol{A}_{12}\\
\boldsymbol{O}&\boldsymbol{A}_{22}
\end{pmatrix},
\]
其中\(\boldsymbol{A}_{11}\)是一个\(r\)阶矩阵.
\end{theorem}

\begin{theorem}\label{theorem:在直和的基下的矩阵}
设\(\varphi\)是数域\(\mathbb{F}\)上向量空间\(V\)上的线性变换,\(V_1,V_2,\cdots,V_m\)是\(\varphi\)的不变子空间且\(V = V_1\oplus V_2\oplus\cdots\oplus V_m\). 若取\(V_i\)的基拼成\(V\)的一组基\(\{e_1,e_2,\cdots,e_n\}\),则\(\varphi\)在这组基下的表示矩阵具有下列分块对角矩阵的形状:
\[
\begin{pmatrix}
\boldsymbol{A}_{11}&&&\\
&\boldsymbol{A}_{22}&&\\
&&\ddots&\\
&&&\boldsymbol{A}_{mm}
\end{pmatrix}.
\]
\end{theorem}

\begin{proposition}\label{proposition:商空间下的线性变换对应表示矩阵}
设\(\varphi\)是\(n\)维线性空间\(V\)上的线性变换,\(U\)是\(r\)维\(\varphi -\)不变子空间. 取\(U\)的一组基\(\{\boldsymbol{e}_1,\cdots,\boldsymbol{e}_r\}\),并扩张为\(V\)的一组基\(\{\boldsymbol{e}_1,\cdots,\boldsymbol{e}_r,\boldsymbol{e}_{r + 1},\cdots,\boldsymbol{e}_n\}\). 设\(\varphi\)在这组基下的表示矩阵\(\boldsymbol{A}=(a_{ij})=\begin{pmatrix}\boldsymbol{A}_{11}&\boldsymbol{A}_{12}\\ \boldsymbol{O}&\boldsymbol{A}_{22}\end{pmatrix}\)为分块上三角阵,其中\(\boldsymbol{A}_{11}\)是\(\varphi\)在不变子空间\(U\)上的限制\(\varphi|_U\)在基\(\{\boldsymbol{e}_1,\cdots,\boldsymbol{e}_r\}\)下的表示矩阵. 证明:\(\varphi\)诱导的变换\(\overline{\varphi}(\boldsymbol{v}+U)=\varphi(\boldsymbol{v})+U\)是商空间\(V/U\)上的线性变换,并且在\(V/U\)的一组基\(\{\boldsymbol{e}_{r + 1}+U,\cdots,\boldsymbol{e}_n+U\}\)下的表示矩阵为\(\boldsymbol{A}_{22}\).
\end{proposition}
\begin{proof}
由\(U\)是\(\varphi -\)不变子空间容易验证\(\overline{\varphi}\)的定义不依赖于\(U -\)陪集代表元的选取,从而是定义好的变换. \(\overline{\varphi}\)的线性由\(\varphi\)的线性即得. 由\(\varphi\)的表示矩阵为\(\boldsymbol{A}\),再结合$\boldsymbol{e}_1,\boldsymbol{e}_2\cdots ,\boldsymbol{e}_r\in U$及\hyperref[proposition:U-陪集的性质]{U-陪集的性质(2)}和\hyperref[definition:U-陪集与商空间]{商空间的加法和数乘的定义}可得
\begin{align*}
&\begin{aligned}
\overline{\varphi }\left( \boldsymbol{e}_{r+1}+U \right) &=\varphi \left( \boldsymbol{e}_{r+1} \right) +U=a_{1,r+1}\boldsymbol{e}_1+\cdots +a_{r,r+1}\boldsymbol{e}_r+a_{r+1,r+1}\boldsymbol{e}_{r+1}+\cdots +a_{n,r+1}\boldsymbol{e}_n+U
\\
&=a_{r+1,r+1}\boldsymbol{e}_{r+1}+\cdots +a_{n,r+1}\boldsymbol{e}_n+U=a_{r+1,r+1}\left( \boldsymbol{e}_{r+1}+U \right) +\cdots +a_{n,r+1}\left( \boldsymbol{e}_n+U \right) ,
\end{aligned}
\\
&\cdots \cdots \cdots \cdots 
\\
&\overline{\varphi }(\boldsymbol{e}_n+U)=a_{r+1,n}(\boldsymbol{e}_{r+1}+U)+\cdots +a_{n,n}(\boldsymbol{e}_n+U),
\end{align*}
故\(\overline{\varphi}\)在基\(\{\boldsymbol{e}_{r + 1}+U,\cdots,\boldsymbol{e}_n+U\}\)下的表示矩阵为\(\boldsymbol{A}_{22}\). 
\end{proof}



\section{幂等变换}

\begin{definition}[幂等变换]\label{definition:幂等变换}
线性变换\(\varphi\)若满足\(\varphi^2 = \varphi\),则称为\textbf{幂等变换}. 
\end{definition}

\begin{definition}[投影变换]\label{definition:投影变换}
设\(V = V_1\oplus V_2\oplus\cdots\oplus V_m\)为线性空间\(V\)关于子空间\(V_i(1\leq i\leq m)\)的直和分解,则\(V\)中任一向量\(\boldsymbol{v}\)可唯一地分解为\(\boldsymbol{v}=\boldsymbol{v}_1+\boldsymbol{v}_2+\cdots+\boldsymbol{v}_m\),其中\(\boldsymbol{v}_i\in V_i\). 定义\(\varphi_i:V\to V\),\(\varphi_i(\boldsymbol{v})=\boldsymbol{v}_i(1\leq i\leq m)\),容易验证\(\varphi_i\)是\(V\)上的线性变换,称为\(V\)到\(V_i\)上的\textbf{投影变换}. 
\end{definition}

\begin{proposition}[投影变换的性质]\label{proposition:投影变换的性质}
设\(V = V_1\oplus V_2\oplus\cdots\oplus V_m\)为线性空间\(V\)关于子空间\(V_i(1\leq i\leq m)\)的直和分解,\(\varphi_i\)为\(V\)到\(V_i\)上的投影变换.投影变换\(\varphi_i\)满足如下性质:
\begin{enumerate}[(1)]
\item \(\varphi_i^2 = \varphi_i\),\(\varphi_i\varphi_j = \boldsymbol{0}(i\neq j)\),\(\boldsymbol{I}_V = \varphi_1+\varphi_2+\cdots+\varphi_m\);

\item \(\text{Im}\varphi_i = V_i\),\(\text{Ker}\varphi_i = \bigoplus_{j\neq i}V_j\),\(V = \text{Im}\varphi_i\oplus\text{Ker}\varphi_i\).

\item 投影变换\(\varphi_i\)都是幂等变换;

\item 若取\(V_i\)的一组基拼成\(V\)的一组基,则\(\varphi_i\)在这组基下的表示矩阵为\(\text{diag}\{0,\cdots,0,1,\cdots,1,0,\cdots,0\}\),其中有\(\dim V_i\)个\(1\);

\item \(V = \text{Im}\varphi_1\oplus\text{Im}\varphi_2\oplus\cdots\oplus\text{Im}\varphi_m\);

\item \(\text{Ker}\varphi_1\cap\text{Ker}\varphi_2\cap\cdots\cap\text{Ker}\varphi_m = 0\).
\end{enumerate}
\end{proposition}
\begin{note}
提示:两个集合相等等价于这两个集合相互包含.
\end{note}
\begin{proof}
\begin{enumerate}[(1)]
\item 证明是显然的.

\item 证明是显然的.

\item 由(1)易得.

\item 由(2)易得.

\item 由\((2)\)易知\(V = \text{Im}\varphi_1\oplus\text{Im}\varphi_2\oplus\cdots\oplus\text{Im}\varphi_m\).

\item 任取\(\alpha \in \mathrm{Ker}\varphi_1\cap \mathrm{Ker}\varphi_2\cap \cdots \cap \mathrm{Ker}\varphi_m\).由投影变换性质\((2)\),可知\(\mathrm{Ker}\varphi_i = \bigoplus_{j\neq i}V_j\).于是对任意整数\(i, j\in [1, m]\)且\(i\neq j\),有\(\alpha \in \mathrm{Ker}\varphi_i\cap \mathrm{Ker}\varphi_j = \bigoplus_{k\neq i}V_k\cap \bigoplus_{k\neq j}V_k\).从而
\[
\alpha = v_1 + \cdots + v_{i - 1} + v_{i + 1} + \cdots + v_m = u_1 + \cdots + u_{j - 1} + u_{j + 1} + \cdots + u_m,
\]
其中\(v_k, u_k\in V_k\).上式经整理可得\(v_j - u_i = \sum_{k\neq i, j}(u_k - v_k)\in \bigoplus_{k\neq i, j}V_k\).又\(v_j - u_i\in V_i\oplus V_j\).故\(v_j - u_i\in (V_i\oplus V_j)\cap \bigoplus_{k\neq i, j}V_k\).而由于\(V = \bigoplus_{k = 1}^m V_k\),因此\((V_i\oplus V_j)\cap \bigoplus_{k\neq i, j}V_k = \mathbf{0}\).故\(v_j - u_i = \mathbf{0}\),从而\(v_j = u_i\in V_i\cap V_j\).又因为\(V_i\oplus V_j\),所以\(V_i\cap V_j = \mathbf{0}\).故\(v_j = u_i = \mathbf{0}\).再由\(i, j\)的任意性可知,\(v_i = \mathbf{0}, i = 1, 2, \cdots, m\).因此\(\alpha = \sum_{k\neq i}v_k = \mathbf{0}\).故\(\mathrm{Ker}\varphi_1\cap \mathrm{Ker}\varphi_2\cap \cdots \cap \mathrm{Ker}\varphi_m = \mathbf{0}\).
\end{enumerate}
\end{proof}

\begin{proposition}\label{proposition:幂等变换就是投影变换}
设\(\varphi\)是\(n\)维线性空间\(V\)上的幂等变换,证明:\(V = U\oplus W\),其中\(U = \text{Im}\varphi=\text{Ker}(\boldsymbol{I}_V - \varphi)\),\(W = \text{Im}(\boldsymbol{I}_V - \varphi)=\text{Ker}\varphi\),且\(\varphi\)就是\(V\)到\(U\)上的投影变换.
\end{proposition}
\begin{note}
由上述命题可知\textbf{\(n\)维线性空间\(V\)上的幂等变换\(\varphi\)也是\(V\)到\(\text{Im}\varphi\)上的投影变换}.于是由\hyperref[proposition:幂等变换就是投影变换]{命题\ref{proposition:幂等变换就是投影变换}}和\hyperref[proposition:投影变换的性质]{命题\ref{proposition:投影变换的性质}}可知,投影变换可以看作幂等变换,幂等变换也可以看作投影变换.(即\hypertarget{幂等变换和投影变换等价}{幂等变换和投影变换等价})
\end{note}
\begin{proof}
因为\(\varphi^2 = \varphi\),故\(\text{Im}\varphi\subseteq\text{Ker}(\boldsymbol{I} - \varphi)\),\(\text{Im}(\boldsymbol{I} - \varphi)\subseteq\text{Ker}\varphi\). 对任意的\(\boldsymbol{\alpha}\in V\),\(\varphi(\boldsymbol{\alpha})\in\text{Ker}(\boldsymbol{I} - \varphi)\),\((\boldsymbol{I} - \varphi)(\boldsymbol{\alpha})\in\text{Ker}\varphi\),于是\(\boldsymbol{\alpha}=(\boldsymbol{I} - \varphi)(\boldsymbol{\alpha})+\varphi(\boldsymbol{\alpha})\in\text{Ker}\varphi+\text{Ker}(\boldsymbol{I} - \varphi)\),从而\(V = \text{Ker}\varphi+\text{Ker}(\boldsymbol{I} - \varphi)\). 任取\(\boldsymbol{\beta}\in\text{Ker}\varphi\cap\text{Ker}(\boldsymbol{I} - \varphi)\),则\(\boldsymbol{\beta}=(\boldsymbol{I}-\varphi)(\boldsymbol{\beta})+\varphi(\boldsymbol{\beta}) = \boldsymbol{0}\),即\(\text{Ker}\varphi\cap\text{Ker}(\boldsymbol{I}-\varphi)= 0\). 因此,\(V = \text{Ker}\varphi\oplus\text{Ker}(\boldsymbol{I}-\varphi)\). 特别地,由维数公式可得\(\dim\text{Im}\varphi=\dim\text{Ker}(\boldsymbol{I} - \varphi)\),\(\dim\text{Im}(\boldsymbol{I} - \varphi)=\dim\text{Ker}\varphi\),从而\(\text{Im}\varphi=\text{Ker}(\boldsymbol{I} - \varphi)\),\(\text{Im}(\boldsymbol{I} - \varphi)=\text{Ker}\varphi\).

令\(U = \text{Im}\varphi=\text{Ker}(\boldsymbol{I} - \varphi)\),\(W = \text{Im}(\boldsymbol{I} - \varphi)=\text{Ker}\varphi\),则\(V = U\oplus W\). 注意到对任意的\(\boldsymbol{\alpha}\in V\),\(\boldsymbol{\alpha}=\varphi(\boldsymbol{\alpha})+(\boldsymbol{I} - \varphi)(\boldsymbol{\alpha})\),其中\(\varphi(\boldsymbol{\alpha})\in U\),\((\boldsymbol{I} - \varphi)(\boldsymbol{\alpha})\in W\),故\(\varphi\)就是\(V\)到\(U\)上的投影变换.
\end{proof}

\begin{corollary}\label{corollary:幂等变换在某组基下的表示矩阵是标准型}
对线性空间\(V\)上的幂等变换\(\varphi\),总存在\(V\)的一组基(它由\(\mathrm{Im}\varphi\)的基和\(\mathrm{Ker}\varphi \)的基拼成),使得\(\varphi\)在这组基下的表示矩阵为下列对角矩阵:
\[
\begin{pmatrix}
\boldsymbol{I}_r&\boldsymbol{O}\\
\boldsymbol{O}&\boldsymbol{O}
\end{pmatrix},
\]
其中\(\boldsymbol{I}_r\)为\(r\)阶单位矩阵,\(r\)等于\(\dim \mathrm{Im}\varphi\),即\(\varphi\)的像空间的维数.
\end{corollary}
\begin{proof}
由\hyperref[proposition:幂等变换就是投影变换]{这个命题\ref{proposition:幂等变换就是投影变换}}和\hyperref[proposition:投影变换的性质]{投影变换的性质}容易证明.
\end{proof}

\begin{proposition}\label{proposition:幂等矩阵的性质1}
设\(\boldsymbol{A}\)是数域\(\mathbb{F}\)上的\(n\)阶幂等矩阵,求证:

(1) 存在\(n\)阶非异阵\(\boldsymbol{P}\),使得\(\boldsymbol{P}^{-1}\boldsymbol{A}\boldsymbol{P}=\begin{pmatrix}\boldsymbol{I}_r&\boldsymbol{O}\\ \boldsymbol{O}&\boldsymbol{O}\end{pmatrix}\),其中\(r = \text{r}(\boldsymbol{A})\);

(2) \(\text{r}(\boldsymbol{A})=\text{tr}(\boldsymbol{A})\).
\end{proposition}
\begin{proof}
将\(\boldsymbol{A}\)看成是\(n\)维列向量空间\(\mathbb{F}^n\)上(由矩阵$\boldsymbol{A}$乘法诱导)的线性变换,则它是幂等变换,因此由\hyperref[corollary:幂等变换在某组基下的表示矩阵是标准型]{推论\ref{corollary:幂等变换在某组基下的表示矩阵是标准型}}即得(1). 

注意到\(\text{tr}(\boldsymbol{A})=\text{tr}(\boldsymbol{P}^{-1}\boldsymbol{A}\boldsymbol{P})=\text{tr}\begin{pmatrix}\boldsymbol{I}_r&\boldsymbol{O}\\ \boldsymbol{O}&\boldsymbol{O}\end{pmatrix}=r = \text{r}(\boldsymbol{A})\),故(2)也成立.
\end{proof}

\begin{example}
设\(\boldsymbol{A},\boldsymbol{B}\)是数域\(\mathbb{F}\)上的\(n\)阶幂等矩阵,且\(\boldsymbol{A}\)和\(\boldsymbol{B}\)的秩相同,求证:必存在\(\mathbb{F}\)上的\(n\)阶可逆矩阵\(\boldsymbol{C}\),使得\(\boldsymbol{C}\boldsymbol{B}=\boldsymbol{A}\boldsymbol{C}\).
\end{example}
\begin{proof}
由\hyperref[proposition:幂等矩阵的性质1]{命题\ref{proposition:幂等矩阵的性质1}}可知,\(\boldsymbol{A}\)和\(\boldsymbol{B}\)均相似于矩阵\(\begin{pmatrix}\boldsymbol{I}_r&\boldsymbol{O}\\ \boldsymbol{O}&\boldsymbol{O}\end{pmatrix}\),于是\(\boldsymbol{A}\)和\(\boldsymbol{B}\)相似,即存在可逆矩阵\(\boldsymbol{C}\),使得\(\boldsymbol{B}=\boldsymbol{C}^{-1}\boldsymbol{A}\boldsymbol{C}\),即\(\boldsymbol{C}\boldsymbol{B}=\boldsymbol{A}\boldsymbol{C}\).
\end{proof}

\begin{proposition}\label{proposition:幂等变换值域与核空间相等的充要条件1}
设\(\varphi,\psi\)是\(n\)维线性空间\(V\)上的幂等线性变换,求证:

(1) \(\text{Im}\varphi=\text{Im}\psi\)的充要条件是\(\varphi\psi=\psi\),\(\psi\varphi=\varphi\);

(2) \(\text{Ker}\varphi=\text{Ker}\psi\)的充要条件是\(\varphi\psi=\varphi\),\(\psi\varphi=\psi\).
\end{proposition}
\begin{note}
也可由幂等变换等价于投影变换来给出直观的几何证明.
\end{note}
\begin{proof}
(1) 由\(\psi=\varphi\psi\)可得\(\text{Im}\psi\subseteq\text{Im}\varphi\).同理由\(\varphi=\psi\varphi\)可得\(\text{Im}\varphi\subseteq\text{Im}\psi\).因此\(\text{Im}\varphi=\text{Im}\psi\).

反之,若\(\text{Im}\varphi=\text{Im}\psi\),则对任意的\(\boldsymbol{\alpha}\in V\),\(\psi(\boldsymbol{\alpha})\in\text{Im}\psi=\text{Im}\varphi\),故存在\(\boldsymbol{\beta}\in V\),使得\(\psi(\boldsymbol{\alpha})=\varphi(\boldsymbol{\beta})\).注意到\(\varphi^2=\varphi\),故\(\varphi\psi(\boldsymbol{\alpha})=\varphi^2(\boldsymbol{\beta})=\varphi(\boldsymbol{\beta})=\psi(\boldsymbol{\alpha})\),于是\(\varphi\psi=\psi\).同理可证\(\psi\varphi=\varphi\).

(2) 设\(\varphi\psi=\varphi\),\(\psi\varphi=\psi\).对任意的\(\boldsymbol{\alpha}\in\text{Ker}\varphi\),即\(\varphi(\boldsymbol{\alpha}) = \boldsymbol{0}\),有\(\psi(\boldsymbol{\alpha})=\psi\varphi(\boldsymbol{\alpha}) = \boldsymbol{0}\),即\(\boldsymbol{\alpha}\in\text{Ker}\psi\),于是\(\text{Ker}\varphi\subseteq\text{Ker}\psi\).同理可证\(\text{Ker}\psi\subseteq\text{Ker}\varphi\),因此\(\text{Ker}\varphi=\text{Ker}\psi\).

反之,设\(\text{Ker}\varphi=\text{Ker}\psi\).对任意的\(\boldsymbol{\alpha}\in V\),有\(\psi(\boldsymbol{\alpha}-\psi(\boldsymbol{\alpha}))=\psi(\boldsymbol{\alpha})-\psi^2(\boldsymbol{\alpha}) = \boldsymbol{0}\),因此\(\boldsymbol{\alpha}-\psi(\boldsymbol{\alpha})\in\text{Ker}\psi=\text{Ker}\varphi\),从而\(\varphi(\boldsymbol{\alpha}-\psi(\boldsymbol{\alpha})) = \boldsymbol{0}\),即\(\varphi(\boldsymbol{\alpha})=\varphi\psi(\boldsymbol{\alpha})\),于是\(\varphi=\varphi\psi\).同理可证\(\psi\varphi=\psi\).
\end{proof}

\begin{proposition}\label{proposition:幂等变换的和与差仍是幂等变换的充要条件1}
设\(\varphi,\psi\)是\(n\)维线性空间\(V\)上的幂等线性变换,求证:

(1) \(\varphi+\psi\)是幂等变换的充要条件是\(\varphi\psi=\psi\varphi = 0\);

(2) \(\varphi-\psi\)是幂等变换的充要条件是\(\varphi\psi=\psi\varphi=\psi\).
\end{proposition}
\begin{note}
也可由幂等变换等价于投影变换来给出直观的几何证明.
\end{note}
\begin{proof}
充分性容易验证,下面证明必要性.

(1) 若\((\varphi+\psi)^2=\varphi+\psi\),则\(\varphi\psi+\psi\varphi = 0\),即\(\varphi\psi=-\psi\varphi\).将上式两边分别左乘及右乘\(\varphi\),可得\(\varphi\psi\varphi=-\varphi\psi=-\psi\varphi\).因此\(\varphi\psi=\psi\varphi = 0\).

(2) 若\((\varphi - \psi)^2=\varphi - \psi\),则\(\varphi\psi+\psi\varphi = 2\psi\).将上式两边分别左乘及右乘\(\varphi\),可得\(\varphi\psi\varphi=\varphi\psi=\psi\varphi\).因此\(\varphi\psi=\psi\varphi=\psi\).
\end{proof}

\begin{proposition}\label{proposition:由投影变换性质反推直和分解}
设\(\varphi_1,\cdots,\varphi_m\)是\(n\)维线性空间\(V\)上的线性变换,且适合条件:
\[
\varphi_i^2 = \varphi_i,\ \varphi_i\varphi_j = 0\ (i\neq j),\ \text{Ker}\varphi_1\cap\cdots\cap\text{Ker}\varphi_m = 0.
\]
求证:\(V = \text{Im}\varphi_1\oplus\text{Im}\varphi_2\oplus\cdots\oplus\text{Im}\varphi_m\).
\end{proposition}
\begin{proof}
任取\(\boldsymbol{\alpha}\in\text{Im}\varphi_i\cap(\sum_{j\neq i}\text{Im}\varphi_j)\),设\(\boldsymbol{\alpha}=\varphi_i(\boldsymbol{\beta})\),其中\(\boldsymbol{\beta}\in V\),则\(\varphi_i(\boldsymbol{\alpha})=\varphi_i^2(\boldsymbol{\beta})=\varphi_i(\boldsymbol{\beta})=\boldsymbol{\alpha}\).又可设
\[
\boldsymbol{\alpha}=\varphi_1(\boldsymbol{\alpha}_1)+\cdots+\varphi_{i - 1}(\boldsymbol{\alpha}_{i - 1})+\varphi_{i + 1}(\boldsymbol{\alpha}_{i + 1})+\cdots+\varphi_m(\boldsymbol{\alpha}_m),
\]
于是
\[
\boldsymbol{\alpha}=\varphi_i(\boldsymbol{\alpha})=\varphi_i(\varphi_1(\boldsymbol{\alpha}_1)+\cdots+\varphi_{i - 1}(\boldsymbol{\alpha}_{i - 1})+\varphi_{i + 1}(\boldsymbol{\alpha}_{i + 1})+\cdots+\varphi_m(\boldsymbol{\alpha}_m)) = 0.
\]
因此\(\text{Im}\varphi_i\cap(\sum_{j\neq i}\text{Im}\varphi_j)=0\).

对\(V\)中任一向量\(\boldsymbol{\alpha}\)以及任意的\(i\),有
\[
\varphi_i(\boldsymbol{\alpha}-(\varphi_1(\boldsymbol{\alpha})+\cdots+\varphi_m(\boldsymbol{\alpha})))=\varphi_i(\boldsymbol{\alpha})-\varphi_i^2(\boldsymbol{\alpha}) = 0,
\]
因此
\[
\boldsymbol{\alpha}-(\varphi_1(\boldsymbol{\alpha})+\cdots+\varphi_m(\boldsymbol{\alpha}))\in\text{Ker}\varphi_1\cap\cdots\cap\text{Ker}\varphi_m = 0,
\]
从而\(\boldsymbol{\alpha}-(\varphi_1(\boldsymbol{\alpha})+\cdots+\varphi_m(\boldsymbol{\alpha})) = 0\),即\(\boldsymbol{\alpha}=\varphi_1(\boldsymbol{\alpha})+\cdots+\varphi_m(\boldsymbol{\alpha})\),于是\(V=\text{Im}\varphi_1+\cdots+\text{Im}\varphi_m\).这就证明了\(V\)是\(\text{Im}\varphi_1,\cdots,\text{Im}\varphi_m\)的直和.
\end{proof}

\begin{proposition}\label{proposition:投影变换的性质x}
设\(\varphi,\varphi_1,\cdots,\varphi_m\)是\(n\)维线性空间\(V\)上的线性变换,满足:\(\varphi^2 = \varphi\)且\(\varphi=\varphi_1+\varphi_2+\cdots+\varphi_m\).求证:\(\text{r}(\varphi)=\text{r}(\varphi_1)+\text{r}(\varphi_2)+\cdots+\text{r}(\varphi_m)\)成立的充要条件是\(\varphi_i^2 = \varphi_i\),\(\varphi_i\varphi_j = 0\ (i\neq j)\).
\end{proposition}
\begin{note}
$\mathrm{r(}\varphi )=\mathrm{r(}\varphi _1)+\mathrm{r(}\varphi _2)+\cdots +\mathrm{r(}\varphi _m)$等价于$\mathrm{dim}\,\mathrm{Im}\varphi =\mathrm{dim}\,\mathrm{Im}\varphi _1+\mathrm{dim}\,\mathrm{Im}\varphi _2+\cdots +\mathrm{dim}\,\mathrm{Im}\varphi _m$.
\end{note}
\begin{proof}
{\color{blue}证法一(几何方法):} 令\(V_0 = \text{Im}\varphi\),\(V_i = \text{Im}\varphi_i\),则由\(\varphi=\varphi_1+\varphi_2+\cdots+\varphi_m\)可得\(V_0\subseteq V_1 + V_2+\cdots+V_m\).

先证充分性. 由\(\varphi_i^2 = \varphi_i\),\(\varphi_i\varphi_j = 0\ (i\neq j)\)可得\(\varphi_i=(\varphi_1+\varphi_2+\cdots+\varphi_m)\varphi_i=\varphi\varphi_i\),故\(V_i\subseteq V_0\),从而\(V_0 = V_1 + V_2+\cdots+V_m\).要证上述和为直和,只要证明零向量表示唯一即可.设
\[
\boldsymbol{0}=\boldsymbol{\alpha}_1+\boldsymbol{\alpha}_2+\cdots+\boldsymbol{\alpha}_m,\boldsymbol{\alpha}_i=\varphi_i(\boldsymbol{v}_i)\in V_i(1\leq i\leq m),
\]
则\(\boldsymbol{0}=\varphi_i(\varphi_1(\boldsymbol{v}_1))+\varphi_i(\varphi_2(\boldsymbol{v}_2))+\cdots+\varphi_i(\varphi_m(\boldsymbol{v}_m))=\varphi_i^2(\boldsymbol{v}_i)=\varphi_i(\boldsymbol{v}_i)=\boldsymbol{\alpha}_i\).因此\(V_0 = V_1\oplus V_2\oplus\cdots\oplus V_m\).两边同取维数即得\(\text{r}(\varphi)=\text{r}(\varphi_1)+\text{r}(\varphi_2)+\cdots+\text{r}(\varphi_m)\).

再证必要性.由于\(V_0\subseteq V_1 + V_2+\cdots+V_m\),于是
\[
\dim V_0\leq\dim(V_1 + V_2+\cdots+V_m)\leq\dim V_1+\dim V_2+\cdots+\dim V_m,
\]
故由\(\text{r}(\varphi)=\text{r}(\varphi_1)+\text{r}(\varphi_2)+\cdots+\text{r}(\varphi_m)\)可得\(\dim V_0=\dim V_1+\dim V_2+\cdots+\dim V_m\),从而上式中的不等号只能取等号.由\hyperref[proposition:与全空间维数相同的子空间等于全空间]{命题\ref{proposition:与全空间维数相同的子空间等于全空间}}及直和的充要条件可知,\(V_1 + V_2+\cdots+V_m\)是直和,并且
\[
V_0 = V_1\oplus V_2\oplus\cdots\oplus V_m.
\]
因为\(\text{Im}\varphi_i = V_i\subseteq V_0 = \text{Im}\varphi\),故对\(V\)中任一向量\(\boldsymbol{\alpha}\),存在\(\boldsymbol{\beta}\in V\),使得\(\varphi_i(\boldsymbol{\alpha})=\varphi(\boldsymbol{\beta})\),从而
\begin{align*}
\varphi_i(\boldsymbol{\alpha})&=\varphi(\boldsymbol{\beta})=\varphi^2(\boldsymbol{\beta})=(\varphi_1+\varphi_2+\cdots+\varphi_m)\varphi(\boldsymbol{\beta})\\
&=(\varphi_1+\varphi_2+\cdots+\varphi_m)\varphi_i(\boldsymbol{\alpha})\\
&=\varphi_1\varphi_i(\boldsymbol{\alpha})+\varphi_2\varphi_i(\boldsymbol{\alpha})+\cdots+\varphi_m\varphi_i(\boldsymbol{\alpha}).
\end{align*}
由直和表示的唯一性可知
\[
\varphi_i^2(\boldsymbol{\alpha})=\varphi_i(\boldsymbol{\alpha}),\varphi_j\varphi_i(\boldsymbol{\alpha}) = 0\ (j\neq i),
\]
于是\(\varphi_i^2 = \varphi_i\),\(\varphi_i\varphi_j = 0\ (i\neq j)\).

{\color{blue}证法二(代数方法):}把问题转换成代数的语言:设\(\boldsymbol{A},\boldsymbol{A}_1,\boldsymbol{A}_2,\cdots,\boldsymbol{A}_m\)是\(n\)阶矩阵,满足\(\boldsymbol{A}^2 = \boldsymbol{A}\)且\(\boldsymbol{A}=\boldsymbol{A}_1+\boldsymbol{A}_2+\cdots+\boldsymbol{A}_m\),求证:\(\text{r}(\boldsymbol{A})=\text{r}(\boldsymbol{A}_1)+\text{r}(\boldsymbol{A}_2)+\cdots+\text{r}(\boldsymbol{A}_m)\)成立的充要条件是\(\boldsymbol{A}_i^2 = \boldsymbol{A}_i\),\(\boldsymbol{A}_i\boldsymbol{A}_j = \boldsymbol{O}\ (i\neq j)\).

先证充分性. 若\(\boldsymbol{A}_i^2 = \boldsymbol{A}_i\),则由\hyperref[proposition:幂等矩阵的性质1]{命题\ref{proposition:幂等矩阵的性质1}}可知\(\text{r}(\boldsymbol{A}_i)=\text{tr}(\boldsymbol{A}_i)\),从而
\begin{align*}
\text{r}(\boldsymbol{A})&=\text{tr}(\boldsymbol{A})=\text{tr}(\boldsymbol{A}_1+\boldsymbol{A}_2+\cdots+\boldsymbol{A}_m)\\
&=\text{tr}(\boldsymbol{A}_1)+\text{tr}(\boldsymbol{A}_2)+\cdots+\text{tr}(\boldsymbol{A}_m)=\text{r}(\boldsymbol{A}_1)+\text{r}(\boldsymbol{A}_2)+\cdots+\text{r}(\boldsymbol{A}_m).
\end{align*}

再证必要性. 因为\(\boldsymbol{A}\)是幂等矩阵,故由\hyperref[proposition:幂等矩阵关于秩的判定准则]{命题\ref{proposition:幂等矩阵关于秩的判定准则}}可得\(n=\text{r}(\boldsymbol{I}_n - \boldsymbol{A})+\text{r}(\boldsymbol{A})\),从而\(n=\text{r}(\boldsymbol{I}_n - \boldsymbol{A})+\text{r}(\boldsymbol{A}_1)+\text{r}(\boldsymbol{A}_2)+\cdots+\text{r}(\boldsymbol{A}_m)\).构造如下分块对角矩阵并对其实施分块初等变换,可得
\[
\begin{pmatrix}
\boldsymbol{I}_n - \boldsymbol{A}&&&&\\
&\boldsymbol{A}_1&&&\\
&&\boldsymbol{A}_2&&\\
&&&\ddots&\\
&&&&\boldsymbol{A}_m
\end{pmatrix}\to
\left( \begin{matrix}
\boldsymbol{I}_n&		&		&		&		\\
\boldsymbol{A}_1&		\boldsymbol{A}_1&		&		&		\\
\boldsymbol{A}_2&		&		\boldsymbol{A}_2&		&		\\
\vdots&		&		&		\ddots&		\\
\boldsymbol{A}_m&		&		&		&		\boldsymbol{A}_m\\
\end{matrix} \right) \to
\]
\[
\begin{pmatrix}
\boldsymbol{I}_n&\boldsymbol{A}_1&\boldsymbol{A}_2&\cdots&\boldsymbol{A}_m\\
\boldsymbol{A}_1&\boldsymbol{A}_1&&&\\
\boldsymbol{A}_2&&\boldsymbol{A}_2&&\\
\vdots&&&\ddots&\\
\boldsymbol{A}_m&&&&\boldsymbol{A}_m
\end{pmatrix}\to
\begin{pmatrix}
\boldsymbol{I}_n&\boldsymbol{O}&\boldsymbol{O}&\cdots&\boldsymbol{O}\\
\boldsymbol{O}&\boldsymbol{A}_1 - \boldsymbol{A}_1^2&-\boldsymbol{A}_1\boldsymbol{A}_2&\cdots&-\boldsymbol{A}_1\boldsymbol{A}_m\\
\boldsymbol{O}&-\boldsymbol{A}_2\boldsymbol{A}_1&\boldsymbol{A}_2 - \boldsymbol{A}_2^2&\cdots&-\boldsymbol{A}_2\boldsymbol{A}_m\\
\vdots&\vdots&\vdots&&\vdots\\
\boldsymbol{O}&-\boldsymbol{A}_m\boldsymbol{A}_1&-\boldsymbol{A}_m\boldsymbol{A}_2&\cdots&\boldsymbol{A}_m - \boldsymbol{A}_m^2
\end{pmatrix}.
\]
由\(n=\text{r}(\boldsymbol{I}_n - \boldsymbol{A})+\text{r}(\boldsymbol{A}_1)+\text{r}(\boldsymbol{A}_2)+\cdots+\text{r}(\boldsymbol{A}_m)\)可得最后一个矩阵的右下角部分必为零矩阵,从而\(\boldsymbol{A}_i^2 = \boldsymbol{A}_i\),\(\boldsymbol{A}_i\boldsymbol{A}_j = \boldsymbol{O}\ (i\neq j)\).
\end{proof}

\begin{corollary}\label{corollary:恒等变换的幂等分解}
若取\(\boldsymbol{I}_V\)为$n$维线性空间\(V\)上的恒等变换,并且此时线性变换\(\varphi_i\)满足\(\varphi_1+\varphi_2+\cdots+\varphi_m = \boldsymbol{I}_n\).如果下列条件之一成立:

(1) \(\dim V=\dim\text{Im}\varphi_1+\dim\text{Im}\varphi_2+\cdots+\dim\text{Im}\varphi_m\);

(2) \(\varphi_i^2 = \varphi_i\),\(\varphi_i\varphi_j = 0\ (i\neq j)\),

则\(V = \text{Im}\varphi_1\oplus\text{Im}\varphi_2\oplus\cdots\oplus\text{Im}\varphi_m\),并且\(\varphi_i\)就是\(V\)到\(\text{Im}\varphi_i\)上的投影变换.
\end{corollary}
\begin{proof}
由\hyperref[proposition:投影变换的性质x]{命题\ref{proposition:投影变换的性质x}}可知条件(1)(2)等价,并且由\hyperref[proposition:投影变换的性质x]{命题\ref{proposition:投影变换的性质x}证法一的必要性证明过程}可直接由条件(1)推出\(V = \text{Im}\varphi_1\oplus\text{Im}\varphi_2\oplus\cdots\oplus\text{Im}\varphi_m\)(直和的证明也可由条件(2)及\hyperref[proposition:投影变换的性质]{投影变换的性质}直接得到).又因为\hyperlink{幂等变换和投影变换等价}{幂等变换和投影变换等价},故由条件(2)可直接得到\(\varphi_i\)就是\(V\)到\(\text{Im}\varphi_i\)上的投影变换.因此结论得证.
\end{proof}






\chapter{多项式}

\begin{proposition}[多项式次数的性质]\label{proposition:多项式次数的性质}
设$f(x),g(x)\in \mathbb{K}[x]$.则
\begin{enumerate}
\item $\mathrm{deg}\left( cf\left( x \right) \right) =\mathrm{deg}\,f\left( x \right) ,0\ne 0\in \mathbb{K} .$

\item $\mathrm{deg}\,\left( f\left( x \right) +g\left( x \right) \right) \leqslant\max \left\{ \mathrm{deg}\,f\left( x \right) ,\mathrm{deg}\,g\left( x \right) \right\} .$

\item $\mathrm{deg}\,\left( f\left( x \right) g\left( x \right) \right) =\mathrm{deg}\,f\left( x \right) +\mathrm{deg}\,g\left( x \right).$
\end{enumerate}
\end{proposition}

\section{整除与带余除法}

\begin{definition}[整除的定义]\label{definition:整除的定义}
设\(f(x),g(x)\)是\(\mathbb{F}\)上的多项式,若存在\(\mathbb{F}\)上的多项式\(h(x)\),使得
\[
f(x)=g(x)h(x),
\]
则称\(g(x)\)是\(f(x)\)的因式,或称\(g(x)\)可整除\(f(x)\)(也称\(f(x)\)可被\(g(x)\)整除),记为\(g(x)\mid f(x)\).
\end{definition}

\begin{proposition}[整除的基本性质]\label{proposition:整除的基本性质}
设\(f(x),g(x),h(x)\in\mathbb{K}[x],0\neq c\in\mathbb{K}\),则
\begin{enumerate}[(1)]
\item 若\(f(x)\mid g(x)\),则\(cf(x)\mid g(x)\),因此非零常数多项式\(c\)是任一非零多项式的因式;

\item  \(f(x)\mid f(x)\);

\item 若\(f(x)\mid g(x),g(x)\mid h(x)\),则\(f(x)\mid h(x)\);

\item 若\(f(x)\mid g(x),f(x)\mid h(x)\),则对任意的多项式\(u(x),v(x)\),有
\[
f(x)\mid g(x)u(x)+h(x)v(x);
\]

\item 设\(f(x)\mid g(x),g(x)\mid f(x)\)且\(f(x),g(x)\)都是非零多项式,则存在\(\mathbb{K}\)中非零元\(c\),使
\[
f(x)=cg(x).
\]
\end{enumerate}
\end{proposition}
\begin{proof}
\begin{enumerate}[(1)]
\item 若\(g(x)=f(x)p(x)\),则
\[
g(x)=(cf(x))(c^{-1}p(x)).
\]
此即\(cf(x)\mid g(x)\).

特别地,任取$a\in \mathbb{K}$,令$g(x)=a$,则$a\mid a$,从而$ca \mid a$,故$c$是$a$的因式.

\item 显然.

\item 若\(g(x)=f(x)p(x),h(x)=g(x)q(x)\),则
\[
h(x)=(f(x)p(x))q(x)=f(x)(p(x)q(x)).
\]

\item 若\(g(x)=f(x)p(x),h(x)=f(x)q(x)\),则
\[
g(x)u(x)+h(x)v(x)=f(x)(p(x)u(x)+q(x)v(x)).
\]

\item 设\(g(x)=f(x)p(x),f(x)=g(x)q(x)\),则
\[
f(x)=f(x)(p(x)q(x)).
\]
由此即得
\[
\mathrm{deg }f(x)=\mathrm{deg }f(x)+\mathrm{deg}(p(x)q(x)),
\]
从而
\[
\mathrm{deg}(p(x)q(x)) = 0,
\]
于是
\[
\mathrm{deg }p(x)=\mathrm{deg }q(x)=0.
\]
因此\(p(x)\)及\(q(x)\)均为非零常数多项式, 即\(f(x)\)和\(g(x)\)相差一个非零常数倍.
\end{enumerate}
\end{proof}

\begin{definition}[相伴多项式]\label{definition:}
若\(f(x)\mid g(x),g(x)\mid f(x)\)且\(f(x),g(x)\)都是非零多项式,则$f(x),g(x)$(即可以互相整除的两个多项式)称为\textbf{相伴多项式},记为\(f(x)\sim g(x)\).
\end{definition}
\begin{note}
由\hyperref[proposition:整除的基本性质]{整除的基本性质(5)}可知,相伴的多项式只相差一个非零常数倍.
\end{note}

\begin{proposition}[相伴多项式的基本性质]\label{proposition:相伴多项式的基本性质}
若$f(x)\sim g(x)$,则任意的多项式$u(x)$都有$f(x)u(x)\sim g(x)u(x)$.
\end{proposition}
\begin{proof}
由$f(x)\sim g(x)$及\hyperref[proposition:整除的基本性质]{整除的基本性质(4)}可知,任意的多项式$u(x)$都有$f(x)u(x)\mid g(x)u(x)$,$g(x)u(x)\mid f(x)u(x)$.故$f(x)u(x)\sim g(x)u(x)$.
\end{proof}

\begin{theorem}[多项式的带余除法]\label{theorem:多项式的带余除法}
设\(f(x),g(x)\in\mathbb{F}[x],g(x)\neq 0\),则必存在唯一的\(q(x),r(x)\in\mathbb{F}[x]\),使得
\[
f(x)=g(x)q(x)+r(x),
\]
且\(\text{deg }\,r(x)<\text{deg }\,g(x)\).
\end{theorem}
\begin{proof}
若\(\mathrm{deg }\,f(x)<\mathrm{deg }\,g(x)\),只需令\(q(x)=0,r(x)=f(x)\)即可.
现设\(\mathrm{deg }\,f(x)\geq\mathrm{deg }\,g(x)\),对\(f(x)\)的次数用数学归纳法. 若\(\mathrm{deg }\,f(x)=0\),则\(\mathrm{deg }\,g(x)=0\). 因此可设\(f(x)=a,g(x)=b(a\neq 0,b\neq 0)\). 这时令\(q(x)=ab^{-1},r(x)=0\)即可. 作为归纳假设,我们设结论对小于\(n\)次的多项式均成立. 设
\[
f(x)=a_nx^n + a_{n - 1}x^{n - 1}+\cdots+a_1x + a_0, a_n\neq 0,
\]
\[
g(x)=b_mx^m + b_{m - 1}x^{m - 1}+\cdots+b_1x + b_0, b_m\neq 0,
\]
由于\(n\geq m\),可令
\[
f_1(x)=f(x)-a_nb_m^{-1}x^{n - m}g(x),
\]
则\(\mathrm{deg }\,f_1(x)<n\). 由归纳假设,有
\[
f_1(x)=g(x)q_1(x)+r(x),
\]
且\(\mathrm{deg }\,r(x)<\mathrm{deg }\,g(x)\),于是
\[
f(x)-a_nb_m^{-1}x^{n - m}g(x)=g(x)q_1(x)+r(x).
\]
因此
\[
f(x)=g(x)(a_nb_m^{-1}x^{n - m}+q_1(x))+r(x).
\]
令
\[
q(x)=a_nb_m^{-1}x^{n - m}+q_1(x),
\]
即得$f(x)=g(x)q(x)+r(x)$.

再证明唯一性. 设另有\(p(x),t(x)\),使
\[
f(x)=g(x)p(x)+t(x),
\]
且\(\mathrm{deg }\,t(x)<\mathrm{deg }\,g(x)\),则
\[
g(x)(q(x)-p(x))=t(x)-r(x).
\]
注意上式左边若\(q(x)-p(x)\neq 0\),便有
\[
\mathrm{deg }\,g(x)(q(x)-p(x))\geq\mathrm{deg }\,g(x)>\mathrm{deg }\,(t(x)-r(x)),
\]
引出矛盾. 因此只可能\(p(x)=q(x),t(x)=r(x)\).
\end{proof}

\begin{corollary}\label{corollary:整除关于多项式的带余除法的充要条件}
设\(f(x),g(x)\in\mathbb{F}[x],g(x)\neq 0\),必存在唯一的\(q(x),r(x)\in\mathbb{F}[x]\),使得$f(x)=g(x)q(x)+r(x)$. 
则\(g(x)\mid f(x)\)的充要条件是\(r(x)=0\).
\end{corollary}

\begin{example}
设\(g(x)=ax + b\in\mathbb{F}[x]\)且\(a\neq0\),又\(f(x)\in\mathbb{F}[x]\),求证:\(g(x)\mid f(x)^2\)的充要条件是\(g(x)\mid f(x)\)。
\end{example}
\begin{proof}
充分性显然,只需证明必要性。

{\color{blue}证法一:}
设\(f(x)=g(x)q(x)+r\),则
\[
f(x)^2 = g(x)^2q(x)^2 + 2rg(x)q(x)+r^2.
\]
由\(g(x)\mid f(x)^2\)可得\(g(x)\mid r^2\),故\(r^2 = 0\),即\(r = 0\),从而\(g(x)\mid f(x)\)。

{\color{blue}证法二:}
由\hyperref[theorem:余数定理]{余数定理},\(f\left(-\frac{b}{a}\right)^2 = 0\),故\(f\left(-\frac{b}{a}\right)= 0\),从而\(g(x)\mid f(x)\)。
\end{proof}

\begin{example}
设 \(g(x)=ax^{2}+bx + c(abc\neq0)\), \(f(x)=x^{3}+px^{2}+qx + r\), 满足 \(g(x)\mid f(x)\), 求证:
\[
\frac{ap - b}{a}=\frac{aq - c}{b}=\frac{ar}{c}.
\]
\end{example}
\begin{proof}
用待定系数法, 设
\begin{align*}
x^{3}+px^{2}+qx + r=(ax^{2}+bx + c)(mx + n)
=amx^{3}+(an + bm)x^{2}+(bn + cm)x+cn.
\end{align*}
比较系数得
\[
am = 1,\ an + bm = p,\ bn + cm = q,\ cn = r.
\]
由此即可得到所需等式. 
\end{proof}

\subsection{凑项法}

“凑项法”是指在要证明的等式中添加若干项再减去若干项来证明结论的方法.

\begin{proposition}\label{proposition:n方差整除的充要条件}
\((x^{d}-a^{d}) \mid (x^{n}-a^{n})\) 的充要条件是 \(d\mid n\), 其中 \(a\neq0\).
\end{proposition}
\begin{proof}
$(\Leftarrow)$:由 \(d|n\) 可设 \(n = kd\),\(k\in \mathbb{N}_+\)。从而
\[
x^n - a^n=(x^d)^k-(a^d)^k=(x^d - a^d)(x^{d(k - 1)}+x^{d(k - 2)}a^d+\cdots +a^{d(k - 1)}).
\]
故 \((x^d - a^d)|(x^n - a^n)\)。

\((\Rightarrow)\):假设 \(d\nmid n\),则由带余除法可知,存在 \(q, r\in \mathbb{N}_+\) 且 \(0\leqslant r < d\),使得 \(n = qd + r\)。于是
\[
x^n - a^n=x^{dq + r}-a^{dq + r}=(x^{dq}-a^{dq})x^r+x^ra^{dq}-a^{dq + r}=(x^{dq}-a^{dq})x^r+a^{dq}(x^r - a^r).
\]
注意到 \((x^{dq}-a^{dq})|(x^d - a^d)\),但由 \(0\leqslant r < d\) 可知,\((x^d - a^d)\nmid (x^r - a^r)\)。故 \((x^d - a^d)\nmid (x^n - a^n)\) 矛盾!
\end{proof}

\begin{example}
设 \(f(x)=x^{3m}+x^{3n + 1}+x^{3p+2}\), 其中 \(m,n,p\) 为自然数, 又 \(g(x)=x^{2}+x + 1\), 求证: \(g(x)\mid f(x)\).
\end{example}
\begin{proof}
由\hyperref[proposition:n方差整除的充要条件]{命题\ref{proposition:n方差整除的充要条件}}可知,\((x^3 - 1)|(x^{3k} - 1)\),\(\forall k\in \mathbb{N}_+\)。又因为 \((x^2 + x + 1)|(x^3 - 1)\),所以 \((x^2 + x + 1)|(x^{3k} - 1)\),\(\forall k\in \mathbb{N}_+\)。注意到
\[
x^{3m}+x^{3n + 1}+x^{3p + 2}=(x^{3m}-1)+x(x^{3n}-1)+x^2(x^{3p}-1)+(x^2 + x + 1).
\]
再结合 \((x^2 + x + 1)|(x^{3m}-1)\),\((x^{3n}-1)\),\((x^{3p}-1)\) 可得\(g(x)|f(x)\)。
\end{proof}




\section{最大公因式与互素多项式}

\begin{definition}[最大公因式和互素]\label{definition:最大公因式和互素}
设\(f(x),g(x)\)是\(\mathbb{F}\)上的多项式,$d(x)$是\(\mathbb{F}\)上的首1多项式,若\(d(x)\)满足

(i)$\,\,d(x)$是\(f(x),g(x)\)的公因式,

(ii)对\(f(x),g(x)\)的任一公因式\(h(x)\),都有\(h(x)\mid d(x)\),

则称\(d(x)\)为\(f(x),g(x)\)的\textbf{最大公因式}(或称\(d(x)\)为\(f(x),g(x)\)的 g.c.d.),记为\(d(x)=(f(x),g(x))\).
特别地,若\(d(x)=1\),则称\(f(x),g(x)\)互素.
\end{definition}

\begin{proposition}\label{proposition:最大公因式的小结论}
\begin{enumerate}[(1)]
\item 若$\mathbb{F}$上的多项式$d_0(x)$(但不一定是首1多项式)也满足

(i)$\,\,d_0(x)$是\(f(x),g(x)\)的公因式,

(ii)对\(f(x),g(x)\)的任一公因式\(h(x)\),都有\(h(x)\mid d_0(x)\),

则$d_0(x)\sim d(x)$,即$d_0(x)$与$d(x)$相差一个非零常数倍.

\item 对$\forall a,b\in\mathbb{F}$,$(af(x),bg(x))=(f(x),g(x))=d(x)$.
\end{enumerate}
\end{proposition}
\begin{proof}
\begin{enumerate}[(1)]
\item 证明是显然的.

\item 证明是显然的.
\end{enumerate}
\end{proof}

\begin{definition}[最小公倍式]\label{definition:最小公倍式}
设\(f(x),g(x)\)是\(\mathbb{F}\)上的多项式,$m(x)$是\(\mathbb{F}\)上的首1多项式,若\(d(x)\)满足

(i)\(\,\,m(x)\)是\(f(x)\)与\(g(x)\)的公倍式,

(ii)对\(f(x)\)与\(g(x)\)的任一公倍式\(l(x)\)均有\(m(x)\mid l(x)\),

则称\(m(x)\)为\(f(x)\)与\(g(x)\)的\textbf{最小公倍式}(或称\(m(x)\)为\(f(x),g(x)\)的 l.c.m.),记为\(m(x)=[f(x),g(x)]\).
\end{definition}

\begin{proposition}\label{proposition:最小公倍式的小结论}
\begin{enumerate}[(1)]
\item 若$\mathbb{F}$上的多项式$m_0(x)$(但不一定是首1多项式)也满足

(i)\(\,\,m(x)\)是\(f(x)\)与\(g(x)\)的公倍式,

(ii)对\(f(x)\)与\(g(x)\)的任一公倍式\(l(x)\)均有\(m(x)\mid l(x)\),

则$m_0(x)\sim m(x)$,即$m_0(x)$与$m(x)$相差一个非零常数倍.

\item 对$\forall a,b\in\mathbb{F}$,$[af(x),bg(x)]=[f(x),g(x)]=m(x)$.
\end{enumerate}
\end{proposition}
\begin{proof}
\begin{enumerate}[(1)]
\item 证明是显然的.

\item 证明是显然的.
\end{enumerate}
\end{proof}


\begin{theorem}[最大公因式的必要条件]\label{theorem:最大公因式的必要条件}
设\(f(x),g(x)\)是\(\mathbb{F}\)上的多项式,\(d(x)\)是它们的最大公因式,则必存在\(\mathbb{F}\)上的多项式\(u(x),v(x)\),使得\[f(x)u(x)+g(x)v(x)=d(x).\]
\end{theorem}
\begin{remark}
设\(d(x)=f(x)u(x)+g(x)v(x)\),则\(d(x)\)不一定是\(f(x)\)和\(g(x)\)的最大公因式.
\end{remark}
\begin{proof}
若\(f(x)=0\),则显然\((f(x),g(x)) = g(x)\);若\(g(x)=0\),则\((f(x),g(x)) = f(x)\). 故不妨设\(f(x)\neq 0,g(x)\neq 0\). 由带余除法,我们有下列等式:
\begin{align*}
f(x)&=g(x)q_1(x)+r_1(x),\\
g(x)&=r_1(x)q_2(x)+r_2(x),\\
r_1(x)&=r_2(x)q_3(x)+r_3(x),\\
&\cdots\cdots\cdots\\
r_{s - 2}(x)&=r_{s - 1}(x)q_s(x)+r_s(x),\\
&\cdots\cdots\cdots
\end{align*}
余式的次数是严格递减的,因此经过有限步后,必有一个等式其余式为零. 不妨设\(r_{s + 1}(x)=0\),于是
\begin{align}
r_{s - 1}(x)=r_s(x)q_{s + 1}(x). \label{theorem5.6-5.3.1}
\end{align}

现在要证明\(r_s(x)\)即为\(f(x)\)与\(g(x)\)的最大公因式. 由上式知\(r_s(x)\mid r_{s - 1}(x)\),但
\begin{align}
r_{s - 2}(x)=r_{s - 1}(x)q_s(x)+r_s(x),\label{theorem5.6-5.3.2}   
\end{align}
因此\(r_s(x)\mid r_{s - 2}(x)\). 这样可一直推下去,得到\(r_s(x)\mid g(x),r_s(x)\mid f(x)\). 这表明\(r_s(x)\)是\(f(x)\)与\(g(x)\)的公因式. 又设\(h(x)\)是\(f(x)\)与\(g(x)\)的公因式,则\(h(x)\mid r_1(x)\),于是\(h(x)\mid r_2(x)\),不断往下推,容易看出有\(h(x)\mid r_s(x)\). 因此\(r_s(x)\)是最大公因式.

再证明\eqref{theorem5.6-5.3.1}式. 从\eqref{theorem5.6-5.3.2}式得
\begin{align}
r_s(x)=r_{s - 2}(x)-r_{s - 1}(x)q_s(x),\label{theorem5.6-5.3.3}    
\end{align}
但我们有
\begin{align}
r_{s - 3}(x)=r_{s - 2}(x)q_{s - 1}(x)+r_{s - 1}(x),\label{theorem5.6-5.3.4})    
\end{align}
从\eqref{theorem5.6-5.3.4}式中解出\(r_{s - 1}(x)\)代入\eqref{theorem5.6-5.3.3}式,得
\[
r_s(x)=r_{s - 2}(x)(1 + q_{s - 1}(x)q_s(x))-r_{s - 3}(x)q_s(x).
\]
用类似的方法逐步将\(r_i(x)\)用\(r_{i - 1}(x),r_{i - 2}(x)\)代入,最后得到
\[
r_s(x)=f(x)u(x)+g(x)v(x).
\]
显然\(u(x),v(x)\in\mathbb{K}[x]\).
\end{proof}

\begin{theorem}[最大公因式的充分条件]\label{theorem:最大公因式的充分条件}
设\(f(x),g(x),d(x)\)是\(\mathbb{F}\)上的多项式.若\(d(x)\mid f(x),d(x)\mid g(x)\)并且存在\(\mathbb{F}\)上的多项式$u(x),v(X)$,使得\(d(x)=f(x)u(x)+g(x)v(x)\),则 \(d(x)\)必是\(f(x)\)和\(g(x)\)的最大公因式.
\end{theorem}
\begin{proof}
如果同时\(d(x)\mid f(x),d(x)\mid g(x)\),则\(d(x)\)是\(f(x)\)和\(g(x)\)的公因式. 若\(h(x)\)也是\(f(x),g(x)\)的公因式,则由\(h(x)\mid f(x)\),\(h(x)\mid g(x)\)可推出\(h(x)\mid (f(x)u(x)+g(x)v(x)) = d(x)\),因此\(d(x)\)是最大公因式.
\end{proof}

\begin{example}
设\((f(x),g(x)) = d(x)\), 求证: 对任意的正整数\(n\),
\[
(f(x)^n,f(x)^{n - 1}g(x),\cdots,g(x)^n)=d(x)^n.
\] 
\end{example}
\begin{proof}
显然\(d(x)^n\)是\(f(x)^{n - k}g(x)^k(0\leq k\leq n)\)的公因式. 又假设
\[
f(x)u(x)+g(x)v(x)=d(x),
\]
两边同时\(n\)次方得到
\begin{align*}
f^n\left( x \right) u^n\left( x \right) +f^{n-1}\left( x \right) g\left( x \right) u^{n-1}\left( x \right) v\left( x \right) +\cdots +g^n\left( x \right) v^n\left( x \right) =d^n\left( x \right) .    
\end{align*}
于是由\hyperref[theorem:最大公因式的充分条件]{最大公因式的充分条件}可知\(d(x)^n\)是\(f(x)^{n - k}g(x)^k(0\leq k\leq n)\)的最大公因式.
\end{proof}

\begin{corollary}[次数不小于1的多项式互素的充要条件]\label{corollary:次数不小于1的多项式互素的充要条件}
设\(f(x),g(x)\)是次数不小于1的多项式互素的充要条件是必唯一地存在两个多项式\(u(x),v(x)\),使得
\[
f(x)u(x)+g(x)v(x)=1,
\]
且\(\text{deg }u(x)<\text{deg }g(x),\text{deg }v(x)<\text{deg }f(x)\).
\end{corollary}
\begin{proof}
充分性由\hyperref[theorem:多项式互素的充要条件]{多项式互素的充要条件}可直接得到.下面证明必要性.

先证存在性.
因为\(( f(x),g(x) ) = 1\)且\(\mathrm{deg}\,f(x),\mathrm{deg}\,g(x) > 1\),所以由多项式互素的充要条件可知,必存在非零多项式\(h(x),k(x)\),使得
\begin{align}
f(x)h(x)+g(x)k(x)=1. \label{corollary5.2-1.1} 
\end{align}
由带余除法可知,存在\(q(x),u(x)\),使得
\[
h(x)=g(x)q(x)+u(x),\quad \mathrm{deg}\,u(x)<\mathrm{deg}\,g(x).
\]
代入\eqref{corollary5.2-1.1}式可得
\[
f(x)[g(x)q(x)+u(x)]+g(x)k(x)=1.
\]
即有
\begin{align}
f(x)u(x)+g(x)[f(x)q(x)+k(x)] = 1. \label{corollary5.2-1.2}  
\end{align}
令\(v(x)=f(x)q(x)+k(x)\),则\(\mathrm{deg}\,v(x)<\mathrm{deg}\,f(x)\)。否则,若\(\mathrm{deg}\,v(x)\geqslant \mathrm{deg}\,f(x)\),则由\eqref{corollary5.2-1.2}式可知
\begin{align}
f(x)u(x)+g(x)v(x)=1. \label{corollary5.2-1.3}  
\end{align}
从而由\(\mathrm{deg}\,v(x)\geqslant \mathrm{deg}\,f(x)\)及\(\mathrm{deg}\,u(x)<\mathrm{deg}\,g(x)\)可得
\[
\mathrm{deg}\,(f(x)u(x))=\mathrm{deg}\,f(x)+\mathrm{deg}\,u(x)<\mathrm{deg}\,v(x)+\mathrm{deg}\,g(x)=\mathrm{deg}(g(x)v(x)).
\]
而由\eqref{corollary5.2-1.3}式可知\(\mathrm{deg}\left( f\left( x \right) u\left( x \right) \right) =\mathrm{deg}\left( 1-g\left( x \right) v\left( x \right) \right) =\mathrm{deg}\left( g\left( x \right) v\left( x \right) \right) \)矛盾!

再证唯一性,设另有\(u_1(x),v_1(x)\)适合条件,即
\[
f(x)u_1(x)+g(x)v_1(x)=f(x)u(x)+g(x)v(x)=1.
\]
从而
\[
f(x)(u(x)-u_1(x))=g(x)(v(x)-v_1(x)).
\]
上式表明\(g(x)\mid f(x)(u(x)-u_1(x))\),又由于\(( f(x),g(x) ) = 1\),因此\(g(x)\mid (u(x)-u_1(x))\)。而\(\mathrm{deg}\,(u(x)-u_1(x))<\mathrm{deg}\,g(x)\),故\(u(x)-u_1(x)=0\),即\(u(x)=u_1(x)\)。同理可得\(v(x)=v_1(x)\)。
\end{proof}

\begin{theorem}[多项式互素的充要条件]\label{theorem:多项式互素的充要条件}
设\(f(x),g(x)\)是\(\mathbb{F}\)上的多项式.则
\begin{enumerate}[(1)]
\item \((f(x),g(x)) = 1\)的充要条件是存在\(\mathbb{F}\)上的多项式\(u(x),v(x)\),使得$f(x)u(x)+g(x)v(x)=1.$
\item \((f(x),g(x)) = 1\)的充要条件是对任意给定的正整数\(m,n\), \((f(x)^m,g(x)^n)=1\).
\end{enumerate}
\end{theorem}
\begin{proof}
\begin{enumerate}
\item 必要性:由\hyperref[theorem:最大公因式的必要条件]{最大公因式的必要条件}立得.

充分性:设$(f(x),g(x))=d(x)$,则由$f(x)u(x)+g(x)v(x)=1$可知,$d(x)\mid 1$,因此$d(x)=1.$

\item 必要性由\hyperref[proposition:两两互素的多项式组的乘积也互素]{命题\ref{proposition:两两互素的多项式组的乘积也互素}}即得. 反过来, 若\(d(x)\neq 1\)是\(f(x)\)和\(g(x)\)的公因式, 则它也是\(f(x)^m\)和\(g(x)^n\)的公因式, 因此\(f(x)^m\)和\(g(x)^n\)不可能互素.
\end{enumerate}
\end{proof}

\begin{proposition}[互素多项式和最大公因式的基本性质]\label{proposition:互素多项式和最大公因式的基本性质}
设$f(x),g(x),f_1(x),f_2(x)\in \mathbb{K}[x]$,则
\begin{enumerate}[(1)]
\item 若\(f_1(x)\mid g(x), f_2(x)\mid g(x)\),且\((f_1(x), f_2(x)) = 1\),则\(f_1(x)f_2(x)\mid g(x)\).
\item  若\((f(x), g(x)) = 1\),且\(f(x)\mid g(x)h(x)\),则\(f(x)\mid h(x)\).
\item  若\((f(x), g(x)) = d(x), f(x)=f_1(x)d(x), g(x)=g_1(x)d(x)\),则\((f_1(x), g_1(x)) = 1\).
\item  若\((f(x), g(x)) = d(x)\),则\((t(x)f(x), t(x)g(x)) = t(x)d(x)\).
\item  若\((f_1(x), g(x)) = 1, (f_2(x), g(x)) = 1\),则\((f_1(x)f_2(x), g(x)) = 1\).

\item 若\((f(x),g(x))=1\),则\((f(x^m),g(x^m))=1\), 其中\(m\)为任一正整数.

\item  若\((f(x),g(x)) = 1\),则\((f(x)g(x),f(x)+g(x)) = 1\).
\end{enumerate}
\end{proposition}
\begin{proof}
\begin{enumerate}[(1)]
\item 由\hyperref[theorem:多项式互素的充要条件]{多项式互素的充要条件(1)}可知,存在\(u(x),v(x)\in\mathbb{K}[x]\),使
\[
f_1(x)u(x)+f_2(x)v(x)=1.
\]
设\(g(x)=f_1(x)s(x)=f_2(x)t(x)\),则
\begin{align*}
g(x)&=g(x)(f_1(x)u(x)+f_2(x)v(x))\\
&=f_2(x)t(x)f_1(x)u(x)+f_1(x)s(x)f_2(x)v(x)\\
&=f_1(x)f_2(x)(t(x)u(x)+s(x)v(x)),
\end{align*}
即\(f_1(x)f_2(x)\mid g(x)\).

\item 由\hyperref[theorem:theorem:多项式互素的充要条件]{多项式互素的充要条件}可知,存在\(u(x),v(x)\in\mathbb{K}[x]\),使
\[
f(x)u(x)+g(x)v(x)=1,
\]
则
\[
f(x)u(x)h(x)+g(x)v(x)h(x)=h(x).
\]
因上式左边可被\(f(x)\)整除,故\(f(x)\mid h(x)\).

\item 由\hyperref[theorem:theorem:多项式互素的充要条件]{多项式互素的充要条件}可知,存在\(u(x),v(x)\in\mathbb{K}[x]\),使
\[
f(x)u(x)+g(x)v(x)=d(x),
\]
即
\[
f_1(x)d(x)u(x)+g_1(x)d(x)v(x)=d(x),
\]
两边消去\(d(x)\)即得
\[
f_1(x)u(x)+g_1(x)v(x)=1,
\]
因此\(f_1(x),g_1(x)\)互素.

\item \(u(x),v(x)\in\mathbb{K}[x]\),使
\[
f(x)u(x)+g(x)v(x)=d(x),
\]
则
\[
t(x)f(x)u(x)+t(x)g(x)v(x)=t(x)d(x).
\]
因此,若\(h(x)\mid t(x)f(x),h(x)\mid t(x)g(x)\),则必有\(h(x)\mid t(x)d(x)\). 又\(t(x)d(x)\)是\(t(x)f(x),t(x)g(x)\)的公因式,因此\(t(x)d(x)\)是\(t(x)f(x)\)与\(t(x)g(x)\)的最大公因式.

\item 由\hyperref[theorem:theorem:多项式互素的充要条件]{多项式互素的充要条件}可知,存在\(u_1(x),v_1(x),u_2(x),v_2(x)\in\mathbb{K}[x]\),使
\begin{align*}
f_1(x)u_1(x)+g(x)v_1(x)&=1,\\
f_2(x)u_2(x)+g(x)v_2(x)&=1,
\end{align*}
将上两式两边分别相乘得
\begin{align*}
(f_1(x)f_2(x))u_1(x)u_2(x)+g(x)(v_1(x)f_2(x)u_2(x)
+g(x)v_1(x)v_2(x)+v_2(x)f_1(x)u_1(x))=1.
\end{align*}
这就是说\(f_1(x)f_2(x)\)和\(g(x)\)互素.

\item 因为\(f(x)\)和\(g(x)\)互素, 故存在多项式\(u(x),v(x)\), 使得
\[
f(x)u(x)+g(x)v(x)=1,
\]
从而有
\[
f(x^m)u(x^m)+g(x^m)v(x^m)=1,
\]
于是\(f(x^m)\)和\(g(x^m)\)互素.

\item 由\hyperref[theorem:多项式互素的充要条件]{互素多项式的充要条件(1)}可知,存在\(u(x),v(x)\),使得
\[
f(x)u(x)+g(x)v(x)=1.
\]
从而
\[
f(x)[u(x)-v(x)]+[f(x)+g(x)]v(x)=1.
\]
故由\hyperref[theorem:多项式互素的充要条件]{互素多项式的充要条件(1)}可知,\((f(x),f(x)+g(x)) = 1\)。同理可得,\((f(x),f(x)+g(x)) = 1\)。再由\hyperref[proposition:互素多项式和最大公因式的基本性质]{互素多项式和最大公因式的基本性质 (5)}即得\((f(x)g(x),f(x)+g(x)) = 1\).
\end{enumerate}
\end{proof}

\begin{theorem}[多个多项式的最大公因式的必要条件]\label{theorem:多个多项式的最大公因式的必要条件}
设\(d(x)\)是\(f_1(x),f_2(x),\cdots,f_m(x)\)的最大公因式,求证:必存在多项式\(g_1(x),g_2(x),\cdots,g_m(x)\),使得
\[
f_1(x)g_1(x)+f_2(x)g_2(x)+\cdots + f_m(x)g_m(x)=d(x).
\]
\end{theorem}
\begin{proof}
用数学归纳法. 对\(m = 2\),结论已成立. 设结论对\(m - 1\)成立. 设\(h(x)\)是\(f_1(x),f_2(x),\cdots,f_{m - 1}(x)\)的最大公因式,则有\(g_1(x),g_2(x),\cdots,g_{m - 1}(x)\),使得
\[
f_1(x)g_1(x)+f_2(x)g_2(x)+\cdots + f_{m - 1}(x)g_{m - 1}(x)=h(x).
\]
结合上式由条件可知\(d(x)\)是\(h(x)\)和\(f_m(x)\)的最大公因式,故存在\(u(x),v(x)\),使得
\[
h(x)u(x)+f_m(x)v(x)=d(x).
\]
将\(h(x)\)代入可得
\[
f_1(x)g_1(x)u(x)+f_2(x)g_2(x)u(x)+\cdots + f_{m - 1}(x)g_{m - 1}(x)u(x)+f_m(x)v(x)=d(x),
\]
即知结论成立.
\end{proof}

\begin{corollary}[多个多项式互素的充要条件]\label{corollary:多个多项式互素的充要条件}
数域\(\mathbb{F}\)上的多项式\(f_1(x),f_2(x),\cdots,f_m(x)\)互素的充要条件是存在\(\mathbb{F}\)上的多项式\(g_1(x),g_2(x),\cdots,g_m(x)\),使得
\[
f_1(x)g_1(x)+f_2(x)g_2(x)+\cdots + f_m(x)g_m(x)=1.
\]
\end{corollary}
\begin{proof}
必要性:由\hyperref[theorem:多个多项式的最大公因式的必要条件]{多个多项式的最大公因式的必要条件}立即得到.

充分性:设存在\(\mathbb{F}\)上的多项式\(g_1(x),g_2(x),\cdots,g_m(x)\),使得
\[
f_1(x)g_1(x)+f_2(x)g_2(x)+\cdots + f_m(x)g_m(x)=1.
\]
设\(d(x)\)是\(f_1(x),f_2(x),\cdots,f_m(x)\)的最大公因式,则由上式可知,$d(x)\mid 1$,从而$d(x)=1$.
\end{proof}

\begin{proposition}[两两互素的多项式组的乘积也互素]\label{proposition:两两互素的多项式组的乘积也互素}
设\(f_1(x),\cdots,f_m(x),g_1(x),\cdots,g_n(x)\)为多项式, 且
\[
(f_i(x),g_j(x)) = 1, 1\leq i\leq m; 1\leq j\leq n,
\]
求证:
\[
(f_1(x)f_2(x)\cdots f_m(x),g_1(x)g_2(x)\cdots g_n(x)) = 1.
\]
\end{proposition}
\begin{proof}
利用\hyperref[proposition:互素多项式和最大公因式的基本性质]{互素多项式和最大公因式的基本性质(5)}以及数学归纳法即得结论.
\end{proof}

\begin{corollary}\label{corollary:互素多项式n次方后仍互素}
设$f(x),g(x)$为多项式,若$(f(x),g(x))=1$,则$(f(x),g^n(x))=1$.  
\end{corollary}
\begin{proof}
在\hyperref[proposition:两两互素的多项式组的乘积也互素]{命题\ref{proposition:两两互素的多项式组的乘积也互素}(上一个命题)}中取$f_1(x)=f(x),f_i(x)=1(i=2,3\cdots,n)$,$g_j(x)=g(x)(j=1,2,\cdots,n)$即可得到结论.
\end{proof}

\begin{theorem}[中国剩余定理]\label{theorem:中国剩余定理}
设$g_1(x)$,$\cdots$,$g_n(x)$是两两互素的多项式,$r_1(x)$,$\cdots$,$r_n(x)$是$n$个多项式,则存在多项式$f(x)$,$q_1(x)$,$\cdots$,$q_n(x)$,使
\begin{align*}
f(x)=g_i(x)q_i(x)+r_i(x),i = 1,\cdots,n.  
\end{align*}
\end{theorem}
\begin{proof}
先证明存在多项式\(f_i(x)\), 使对任意的\(i\), 有
\[
f_i(x)=g_i(x)p_i(x)+1,\ g_j(x)\mid f_i(x)(j\neq i).
\]
一旦得证, 只需令\(f(x)=r_1(x)f_1(x)+\cdots+r_n(x)f_n(x)\)即可. 现构造\(f_1(x)\)如下. 因为\(g_1(x)\)和\(g_j(x)(j\neq 1)\)互素, 故存在\(u_j(x),v_j(x)\), 使\(g_1(x)u_j(x)+g_j(x)v_j(x)=1\). 令
\[
f_1(x)=g_2(x)v_2(x)\cdots g_n(x)v_n(x)=(1 - g_1(x)u_2(x))\cdots(1 - g_1(x)u_n(x)),
\]
显然\(f_1(x)\)符合要求. 同理可构造\(f_i(x)\). 
\end{proof}

\begin{proposition}[两个多项式的乘积与其最大公因式和最小公倍式的乘积相伴]\label{proposition:两个多项式的乘积与其最大公因式和最小公倍式的乘积相伴}
设\(f(x),g(x)\)是非零多项式,则
\[
f(x)g(x)\sim(f(x),g(x))[f(x),g(x)].
\]
\end{proposition}
\begin{proof}
{\color{blue}证法一:}
设\(d(x)=(f(x),g(x))\)且\(f(x)=f_0(x)d(x),g(x)=g_0(x)d(x)\),则由\hyperref[proposition:互素多项式和最大公因式的基本性质]{互素多项式和最大公因式的基本性质(3)}可知\(f_0(x),g_0(x)\)互素.设\(l(x)\)是\(f(x),g(x)\)的公倍式且
\[
l(x)=f(x)u(x)=g(x)v(x),
\]
则\(f_0(x)d(x)u(x)=g_0(x)d(x)v(x)\),消去\(d(x)\)得
\[
f_0(x)u(x)=g_0(x)v(x).
\]
上式表明$f_0(x)\mid g_0(x)v(x),g_0(x)\mid f_0(x)u(x)$,又因为\(f_0(x),g_0(x)\)互素,所以由\hyperref[proposition:互素多项式和最大公因式的基本性质]{互素多项式和最大公因式的基本性质(2)}可知,\(f_0(x)\mid v(x),g_0(x)\mid u(x)\). 设\(u(x)=g_0(x)p(x)\),则
\[
l(x)=f_0(x)d(x)g_0(x)p(x),
\]
即\(f_0(x)d(x)g_0(x)\mid l(x)\). 显然\(f_0(x)d(x)g_0(x)\)是\(f(x),g(x)\)的公倍式,因此由\hyperref[proposition:最大公因式的小结论]{命题\ref{proposition:最大公因式的小结论}(1)}可知
\[
\frac{f(x)g(x)}{d(x)}=f_0(x)d(x)g_0(x)\sim[f(x),g(x)].
\]
故由\hyperref[proposition:相伴多项式的基本性质]{相伴多项式的基本性质}可知
\[
f(x)g(x)\sim(f(x),g(x))[f(x),g(x)].
\]

{\color{blue}证法二:}
设 \(f(x), g(x)\) 的公共标准分解为
\[
f(x) = c_1 p_1(x)^{e_1}p_2(x)^{e_2} \cdots p_k(x)^{e_k}, \quad g(x) = c_2 p_1(x)^{f_1}p_2(x)^{f_2} \cdots p_k(x)^{f_k},
\]
其中 \(p_i(x)\) 为互不相同的首一不可约多项式,\(c, d\) 是非零常数,则
\[
d(x) = p_1(x)^{r_1}p_2(x)^{r_2} \cdots p_k(x)^{r_k}, \quad h(x) = p_1(x)^{s_1}p_2(x)^{s_2} \cdots p_k(x)^{s_k},
\]
其中 \(r_i = \min\{e_i, f_i\}, s_i = \max\{e_i, f_i\}\)。注意到
\[
f(x)g(x) = c_1c_2 p_1(x)^{e_1 + f_1}p_2(x)^{e_2 + f_2} \cdots p_k(x)^{e_k + f_k},
\]
并且
\[
(f(x), g(x)) = p_1(x)^{r_1}p_2(x)^{r_2} \cdots p_k(x)^{r_k}, \quad [f(x), g(x)] = p_1(x)^{s_1}p_2(x)^{s_2} \cdots p_k(x)^{s_k},
\]
其中 \(r_i = \min\{e_i, f_i\}, s_i = \max\{e_i, f_i\}\). 令 \(c = c_1c_2\), 则有
\[
f(x)g(x) = cd(x)h(x).
\]
\end{proof}

\begin{proposition}[最大公因式与最小公倍式在开方下不变]\label{proposition:两个多项式的最大公因式与最小公倍式的n次方就是它们n次方的最大公因式与最小公倍式}
设\((f(x),g(x)) = d(x), [f(x),g(x)] = h(x)\), 求证:
\[
(f(x)^n,g(x)^n)=d(x)^n, \ [f(x)^n,g(x)^n]=h(x)^n.
\]
\end{proposition}
\begin{remark}
不妨设\(f(x),g(x)\)都是首一多项式的原因:若$f(x),g(x)$的首项系数分别为$a,b$,则用$\frac{f(x)}{a},\frac{g(x)}{b}$代替,再结合\hyperref[proposition:最大公因式的小结论]{命题\ref{proposition:最大公因式的小结论}(2)}和\hyperref[proposition:最小公倍式的小结论]{命题\ref{proposition:最小公倍式的小结论}(2)}即可得到结论.
\end{remark}
\begin{proof}
{\color{blue}证法一:}
不妨设\(f(x),g(x)\)都是首1多项式, \(f(x)=f_1(x)d(x), g(x)=g_1(x)d(x)\), 则$f_1(x),g_1(x),d(x)$都是首1多项式.由\hyperref[proposition:互素多项式和最大公因式的基本性质]{互素多项式和最大公因式的基本性质(3)}可知\((f_1(x),g_1(x)) = 1\).由\hyperref[proposition:两个多项式的乘积与其最大公因式和最小公倍式的乘积相伴]{命题\ref{proposition:两个多项式的乘积与其最大公因式和最小公倍式的乘积相伴}}可知\(h(x)\sim f_1(x)g_1(x)d(x)\),又因为$h(x),f_1(x),g_1(x),d(x)$均为首1多项式,所以\(h(x)=f_1(x)g_1(x)d(x)\).由\hyperref[proposition:两两互素的多项式组的乘积也互素]{命题\ref{proposition:两两互素的多项式组的乘积也互素}}可知,\((f_1(x)^n,g_1(x)^n)=1\),从而由\hyperref[proposition:互素多项式和最大公因式的基本性质]{互素多项式和最大公因式的基本性质(4)}可知
\[
(f(x)^n,g(x)^n)=(f_1(x)^nd(x)^n,g_1(x)^nd(x)^n)=d(x)^n.
\]
由\hyperref[proposition:两个多项式的乘积与其最大公因式和最小公倍式的乘积相伴]{命题\ref{proposition:两个多项式的乘积与其最大公因式和最小公倍式的乘积相伴}}可知\(f(x)^ng(x)^n\sim (f(x)^n,g(x)^n)[f(x)^n,g(x)^n]\),又因为$f(x),g(x)$都是首1多项式,所以\(f(x)^ng(x)^n=(f(x)^n,g(x)^n)[f(x)^n,g(x)^n]=d(x)^n[f(x)^n,g(x)^n]\).于是可得
\[
[f(x)^n,g(x)^n]=f_1(x)^ng_1(x)^nd(x)^n=h(x)^n.
\]

{\color{blue}证法二:}设 \(f(x), g(x)\) 的公共标准分解为
\[
f(x) = cp_1(x)^{e_1}p_2(x)^{e_2} \cdots p_k(x)^{e_k}, \quad g(x) = dp_1(x)^{f_1}p_2(x)^{f_2} \cdots p_k(x)^{f_k},
\]
其中 \(p_i(x)\) 为互不相同的首一不可约多项式,\(c, d\) 是非零常数,则
\[
d(x) = p_1(x)^{r_1}p_2(x)^{r_2} \cdots p_k(x)^{r_k}, \quad h(x) = p_1(x)^{s_1}p_2(x)^{s_2} \cdots p_k(x)^{s_k},
\]
其中 \(r_i = \min\{e_i, f_i\}, s_i = \max\{e_i, f_i\}\). 注意到
\[
f(x)^n = c^n p_1(x)^{ne_1}p_2(x)^{ne_2} \cdots p_k(x)^{ne_k}, \quad g(x)^n = d^n p_1(x)^{nf_1}p_2(x)^{nf_2} \cdots p_k(x)^{nf_k},
\]
并且\(\min\{ne_i, nf_i\} = nr_i, \max\{ne_i, nf_i\} = ns_i\), 因此
\[
(f(x)^n, g(x)^n) = p_1(x)^{nr_1}p_2(x)^{nr_2} \cdots p_k(x)^{nr_k} = d(x)^n,
\]
\[
[f(x)^n, g(x)^n] = p_1(x)^{ns_1}p_2(x)^{ns_2} \cdots p_k(x)^{ns_k} = h(x)^n.
\]
\end{proof}

\begin{proposition}\label{proposition:n方差的最大公因式}
设\(f(x)=x^m - 1, g(x)=x^n - 1\), 求证: \((f(x),g(x))=x^d - 1\), 其中\(d\)是\(m,n\)的最大公因子.
\end{proposition}
\begin{proof}
{\color{blue}证法一:}
不妨设\(m\geq n, m = nq + r\), 先证明\((x^m - 1,x^n - 1)=(x^r - 1,x^n - 1)\). 假设\(d_1(x)=(x^m - 1,x^n - 1), d_2(x)=(x^r - 1,x^n - 1)\). 注意到
\[
x^m - 1=x^{nq + r}-1=x^r(x^{nq}-1)+(x^r - 1),
\]
\((x^n - 1)\mid(x^{nq}-1)\), 故\(d_1(x)\mid(x^r - 1)\), 从而\(d_1(x)\mid d_2(x)\). 从上式也可以看出\(d_2(x)\mid(x^m - 1)\), 从而\(d_2(x)\mid d_1(x)\), 因此\(d_1(x)=d_2(x)\). 又设\(n = q_1r + r_1\), 则\((x^m - 1,x^n - 1)=(x^n - 1,x^r - 1)=(x^r - 1,x^{r_1}-1)\). 再由辗转相除, 有某个\(r_{s - 1}=q_{s + 1}r_s\), 其中\(r_s = d\)是\(m,n\)的最大公因子, 于是\((x^m - 1,x^n - 1)=(x^{r_{s - 1}}-1,x^{r_s}-1)=x^d - 1\).

{\color{blue}证法二:}只需求出\(f(x),g(x)\)的公根. \(f(x)\)的根为
\[
\cos\frac{2k\pi}{m}+\mathrm{i}\sin\frac{2k\pi}{m}, 1\leq k\leq m,
\]
\(g(x)\)的根为
\[
\cos\frac{2k\pi}{n}+\mathrm{i}\sin\frac{2k\pi}{n}, 1\leq k\leq n,
\]
则公根为
\[
\cos\frac{2k\pi}{d}+\mathrm{i}\sin\frac{2k\pi}{d}, 1\leq k\leq d.
\]
这就是\(x^d - 1\)的全部根, 于是结论成立.
\end{proof}


\section{不可约多项式与因式分解}

\begin{definition}[不可约多项式的定义]\label{definition:不可约多项式的定义}
设\(f(x)\)是数域\(\mathbb{F}\)上的多项式,若\(f(x)\)可以分解为两个次数小于\(f(x)\)的\(\mathbb{F}\)上多项式之积,则称\(f(x)\)是\(\mathbb{F}\)上的可约多项式,否则称\(f(x)\)为\(\mathbb{F}\)上的不可约多项式.
\end{definition}

\begin{proposition}[不可约多项式的基本性质]\label{proposition:不可约多项式的基本性质}
\begin{enumerate}[(1)]
\item 设\(p(x)\)是数域\(\mathbb{K}\)上的不可约多项式, 则对\(\mathbb{K}\)上任一多项式\(f(x)\), 或者\(p(x)\mid f(x)\), 或者\((p(x),f(x)) = 1\).

\item 设\(p(x)\)是数域\(\mathbb{F}\)上的不可约多项式, \(f(x)\)是\(\mathbb{F}\)上的多项式. 证明: 若\(p(x)\)的某个复根\(a\)也是\(f(x)\)的根, 则\(p(x)\mid f(x)\). 特别地, \(p(x)\)的任一复根都是\(f(x)\)的根.
\end{enumerate}
\end{proposition}
\begin{note}
不可约多项式的基本性质(2)表明:不可约多项式也满足\hyperref[proposition:极小多项式的基本性质]{极小多项式的基本性质}.
\end{note}
\begin{proof}
\begin{enumerate}[(1)]
\item 设\(d(x)=(p(x),f(x))\). 因为\(p(x)\)不可约, 故\(f(x)\)的因式只能是非零常数多项式或\(cp(x)(c\neq 0)\), 从而或者\(d(x)=1\)或者\(d(x)=cp(x)\) (首一多项式), 故得结论.

\item 若\((p(x),f(x)) = 1\), 则存在\(\mathbb{F}\)上的多项式\(u(x),v(x)\), 使得\(p(x)u(x)+f(x)v(x)=1\). 令\(x = a\)可得\(1 = p(a)u(a)+f(a)v(a)=0\), 矛盾. 因此\(p(x)\)与\(f(x)\)不互素, 从而只能是\(p(x)\mid f(x)\), 结论得证. 
\end{enumerate}

\end{proof}

\begin{theorem}[不可约多项式的“素性”]\label{theorem:不可约多项式的“素性”}
设\(p(x)\)是数域\(\mathbb{F}\)上的非常数多项式,则\(p(x)\)为\(\mathbb{F}\)上不可约多项式的充要条件是对\(\mathbb{F}\)上任意适合\(p(x)\mid f(x)g(x)\)的多项式\(f(x)\)与\(g(x)\), 或者\(p(x)\mid f(x)\), 或者\(p(x)\mid g(x)\).
\end{theorem}
\begin{proof}
必要性:设\(p(x)\)是\(\mathbb{F}[x]\)中的不可约多项式,且\(p(x)\mid f(x)g(x)\)。若\(p(x)\mid f(x)\),则结论成立。
若\(p(x)\nmid f(x)\),则由定理可知\((p(x),f(x)) = 1\),从而由互素多项式与最大公因式的基本性质可知\(p(x)\mid g(x)\).

充分性:(反证法)假设\(p(x)\)可约,则必存在次数小于\(\text{deg}(p(x))\)的多项式\(f(x)\),\(g(x)\),使得\(p(x) = f(x)g(x)\).从而\(p(x) \mid f(x)g(x)\),于是由条件可知\(p(x) \mid f(x)\)或\(p(x) \mid g(x)\).因此\(\text{deg}(p(x)) \leq \text{deg}(f(x))\)或\(\text{deg}(g(x))\).这与\(\text{deg}(p(x)) > \text{deg}(f(x))\),\(\text{deg}(g(x))\)矛盾.
\end{proof}

\begin{corollary}\label{corollary:不可约多项式“素性”的推论}
设\(p(x)\)为不可约多项式且
\[
p(x)\mid f_1(x)f_2(x)\cdots f_m(x),
\]
则\(p(x)\)必可整除其中某个\(f_i(x)\).
\end{corollary}
\begin{proof}
由\hyperref[theorem:不可约多项式的“素性”]{不可约多项式的“素性”}归纳可得.
\end{proof}

\begin{proposition}\label{proposition:多项式可以写成不可约多项式的幂的充要条件}
设\(f(x)\)是数域\(\mathbb{F}\)上的非常数多项式, 求证: \(f(x)\)等于某个不可约多项式的幂的充要条件是对任意的非常数多项式\(g(x)\), 或者\(f(x)\)和\(g(x)\)互素, 或者\(f(x)\)可以整除\(g(x)\)的某个幂.
\end{proposition}
\begin{proof}
设\(f(x)=p(x)^k\), \(p(x)\)在\(\mathbb{F}\)上不可约,且\(f(x)\)和\(g(x)\)不互素, 则\(p(x)\)是\(f(x)\)和\(g(x)\)的公因式, 故\(f(x)\)可以整除\(g(x)^k\).

反之, 由\hyperref[theorem:因式分解定理]{因式分解定理},可设\(f(x)=p(x)^mh(x)\), \(p(x)\)在\(\mathbb{F}\)上不可约, \(\mathrm{deg }\,h(x)>0\), 且\(p(x)\)不能整除\(h(x)\), 则\(f(x)\mid h(x)\),故$f(x)$不和$h(x)$互素.由于$\mathrm{deg}\,h(x)<\mathrm{deg }\,f(x)$,因此$f(x)$也不能整除\(h(x)\),矛盾!
\end{proof}

\begin{definition}[代数数]\label{definition:代数数}
设\(u\)是复数域中某个数, 若\(u\)适合某个非零有理系数多项式 (或整系数多项式)\(f(x)=a_nx^n + a_{n - 1}x^{n - 1}+\cdots+a_1x + a_0\), 则称\(u\)是一个\textbf{代数数}.
\end{definition}

\begin{definition}[极小多项式(最小多项式)]\label{definition:极小多项式(最小多项式)}
对任一代数数\(u\), 存在唯一一个\(u\)适合的首一有理系数多项式\(g(x)\), 使得\(g(x)\)是\(u\)适合的所有非零有理系数多项式中次数最小者. 这样的\(g(x)\)称为\(u\)的\textbf{极小多项式}或\textbf{最小多项式}.
\end{definition}
\begin{proof}
现在证明这个定义是良定义的,只须证明对任一代数数所对应的极小多项式的存在性和唯一性.

先证存在性.在\(u\)适合的所有非零有理系数多项式构成的集合中 (由假设这个集合非空,否则$u$就不是一个代数数),由良序公理可知,存在一个次数最小的多项式,然后将其首一化, 即可得到\(u\)的极小多项式\(g(x)\). 

再证唯一性.为了证明极小多项式的唯一性, 我们先证明极小多项式的一个基本性质, 即极小多项式可以整除\(u\)适合的任一多项式\(f(x)\). 假设
\[
f(x)=g(x)q(x)+r(x), \mathrm{deg }r(x)<\mathrm{deg }g(x),
\]
则由\(f(u)=g(u)=0\)可知\(r(u)=0\). 若\(r(x)\neq 0\), 则\(u\)适合一个比\(g(x)\)的次数更小的多项式\(r(x)\), 这和\(g(x)\)是极小多项式矛盾. 因此\(r(x)=0\), 即\(g(x)\mid f(x)\). 设\(h(x)\)也是\(u\)的极小多项式, 则由上述性质可得\(g(x)\mid h(x), h(x)\mid g(x)\), 从而\(g(x)\)和\(h(x)\)只差一个非零常数, 又它们都是首一的, 故只能相等, 唯一性得证.
\end{proof}

\begin{proposition}[极小多项式的基本性质]\label{proposition:极小多项式的基本性质}
\begin{enumerate}[(1)]
\item 设$g(x)$为$u$的极小多项式,则$g(x)$一定整除\(u\)适合的任一多项式\(f(x)\).
\end{enumerate}
\end{proposition}
\begin{proof}
\begin{enumerate}[(1)]
\item 假设
\[
f(x)=g(x)q(x)+r(x), \mathrm{deg }r(x)<\mathrm{deg }g(x),
\]
则由\(f(u)=g(u)=0\)可知\(r(u)=0\). 若\(r(x)\neq 0\), 则\(u\)适合一个比\(g(x)\)的次数更小的多项式\(r(x)\), 这和\(g(x)\)是极小多项式矛盾. 因此\(r(x)=0\), 即\(g(x)\mid f(x)\). 
\end{enumerate}
\end{proof}

\begin{proposition}[极小多项式式的充要条件]\label{proposition:极小多项式式的充要条件}
设\(g(x)\)是一个\(u\)适合的首一有理系数多项式, 则\(g(x)\)是\(u\)的极小多项式的充要条件是\(g(x)\)是有理数域上的不可约多项式.
\end{proposition}
\begin{proof}
先证必要性. 若极小多项式\(g(x)\)在有理数域上可约, 则\(g(x)=g_1(x)g_2(x)\)可分解为两个比\(g(x)\)的次数更小的多项式的乘积. 由\(0 = g(u)=g_1(u)g_2(u)\)可知\(g_1(u)\)和\(g_2(u)\)中至少有一个等于零. 不妨设\(g_1(u)=0\), 则\(u\)适合一个比\(g(x)\)的次数更小的多项式\(g_1(x)\), 这和\(g(x)\)是极小多项式矛盾. 

再证充分性. 设\(g(x)\)是\(u\)适合的有理数域上的首一不可约多项式, \(h(x)\)是\(u\)的极小多项式. 由\hyperref[proposition:极小多项式的基本性质]{极小多项式的基本性质(1)}可知\(h(x)\mid g(x)\).因为$g(u)=h(u)=0$,所以$g(x)$和$h(x)$有公共根,从而$x-u$一定是$g(x),h(x)$的公因式,于是$g(x)$和$h(x)$不互素.又\(g(x)\)是不可约多项式,因此$g(x)\mid h(x)$.于是$g(x)\sim h(x)$,即\(g(x)\)和\(h(x)\)只差一个非零常数,而它们又都是首一的,故只能相等.因此\(g(x)\)就是\(u\)的极小多项式.
\end{proof}

\subsection{多项式的标准分解}

多项式的标准分解是证明某些问题的有力工具.

\begin{theorem}[因式分解定理]\label{theorem:因式分解定理}
设\(f(x)\)是数域\(\mathbb{K}\)上的多项式且\(\mathrm{deg }f(x)\geq 1\), 则

(1) \(f(x)\)可分解为有限个\(\mathbb{K}\)上的不可约多项式之积;

(2) 若
\begin{align}
f(x)=p_1(x)p_2(x)\cdots p_s(x)=q_1(x)q_2(x)\cdots q_t(x).\label{theorem5.12-5.4.1}
\end{align}
是\(f(x)\)的两个不可约分解, 即\(p_i(x),q_j(x)\)都是\(\mathbb{K}\)上的次数大于零的不可约多项式, 则\(s = t\), 且经过适当调换因式的次序以后, 有
\[
q_i(x)\sim p_i(x),\ i = 1,2,\cdots,s.
\]
\end{theorem}
\begin{note}
\begin{enumerate}
\item 这个定理表明, 任一多项式可唯一地分解为若干个不可约多项式之积. 这里唯一是在相伴意义下的唯一, 即相应的多项式可以差一个常数因子. 如果把分解式中相同或仅差一个常数的因式合并在一起, 就得到了一个 \textbf{标准分解式}:
\begin{align}
f(x)=cp_1(x)^{e_1}p_2(x)^{e_2}\cdots p_m(x)^{e_m}, \label{theorem5.12-5.4.2}
\end{align}
其中\(c\neq 0, p_i(x)\)是互异的首一不可约多项式, \(e_i\geq 1(i = 1,2,\cdots,m)\).

若\(e_i>1\ (e_i = 1)\), 我们称\eqref{theorem5.12-5.4.2}式中的因式\(p_i(x)\)为\(f(x)\)的\textbf{\(e_i\)重因式 (单因式)}. 显然这时\(p_i(x)^{e_i}\mid f(x)\), 但\(p_i(x)^{e_i + 1}\)不能整除\(f(x)\).

\item 设\(f(x),g(x)\)是\(\mathbb{K}\)上的两个多项式, 在它们的标准分解式中适当添加零次项,就能得到公共的标准分解. 故对\(\mathbb{K}\)上任意的两个多项式$f(x),g(X)$,都可以不妨设它们有如下的\textbf{公共的标准分解式}:
\[
f(x)=c_1p_1(x)^{e_1}p_2(x)^{e_2}\cdots p_n(x)^{e_n};
\]
\[
g(x)=c_2p_1(x)^{f_1}p_2(x)^{f_2}\cdots p_n(x)^{f_n},
\]
其中\(e_i\geq 0, f_i\geq 0(i = 1,2,\cdots,n)\).
\end{enumerate}
\end{note}
\begin{proof}
(1) 对多项式\(f(x)\)的次数用数学归纳法. 若\(\mathrm{deg }f(x)=1\), 结论显然成立. 设次数小于\(n\)的多项式都可以分解为\(\mathbb{K}\)上的不可约多项式之积而\(\mathrm{deg }f(x)=n\). 若\(f(x)\)不可约, 结论自然成立. 若\(f(x)\)可约, 则
\[
f(x)=f_1(x)f_2(x),
\]
其中\(f_1(x),f_2(x)\)的次数小于\(n\), 由归纳假设它们可以分解为有限个\(\mathbb{K}\)上的不可约多项式之积. 所有这些多项式之积就是\(f(x)\).

(2) 对\eqref{theorem5.12-5.4.1}式中的\(s\)用数学归纳法. 若\(s = 1\), 则\(f(x)=p_1(x)\), 因此\(f(x)\)是不可约多项式, 于是\(t = 1, q_1(x)=p_1(x)\). 现假设对不可约因式个数小于\(s\)的多项式结论正确. 由\eqref{theorem5.12-5.4.1}式, 有
\[
p_1(x)\mid q_1(x)q_2(x)\cdots q_t(x),
\]
由\hyperref[corollary:不可约多项式“素性”的推论]{推论\ref{corollary:不可约多项式“素性”的推论}}可知, 必存在某个\(i\), 不妨设\(i = 1\), 使
\[
p_1(x)\mid q_1(x).
\]
但是\(p_1(x),q_1(x)\)都是不可约多项式, 因此存在\(0\neq c_1\in\mathbb{K}\), 使
\[
q_1(x)=c_1p_1(x),
\]
此即\(p_1(x)\sim q_1(x)\). 将上式代入\eqref{theorem5.12-5.4.1}式并消去\(p_1(x)\), 得到
\[
p_2(x)\cdots p_s(x)=c_1q_2(x)\cdots q_t(x).
\]
这时左边为\(s - 1\)个不可约多项式之积, 由归纳假设, \(s - 1=t - 1\), 即\(s = t\). 另一方面, 存在\(0\neq c_i\in\mathbb{K}\), 使\(q_i(x)=c_ip_i(x)\).
\end{proof}

\begin{corollary}\label{corollary:标准分解式具有相同因子的多项式的最大公因式与最小公倍式}
设\(f(x),g(x)\)是\(\mathbb{K}\)上的两个多项式,不妨设它们有如下的公共的标准分解式:
\[
f(x)=c_1p_1(x)^{e_1}p_2(x)^{e_2}\cdots p_n(x)^{e_n};
\]
\[
g(x)=c_2p_1(x)^{f_1}p_2(x)^{f_2}\cdots p_n(x)^{f_n},
\]
其中\(e_i\geq 0, f_i\geq 0(i = 1,2,\cdots,n)\), 则\(f(x),g(x)\)的最大公因式
\[
(f(x),g(x))=p_1(x)^{k_1}p_2(x)^{k_2}\cdots p_n(x)^{k_n},
\]
其中\(k_i = \min\{e_i,f_i\}(i = 1,2,\cdots,n)\).

类似地, \(f(x),g(x)\)的最小公倍式
\[
[f(x),g(x)]=p_1(x)^{h_1}p_2(x)^{h_2}\cdots p_n(x)^{h_n},
\]
其中\(h_i = \max\{e_i,f_i\}(i = 1,2,\cdots,n)\).
\end{corollary}
\begin{proof}
利用最大公因式和最小公倍式的定义容易证明.
\end{proof}

\begin{proposition}[整除关系在平方下不变]\label{proposition:整除关系在平方下不变}
证明:\(g(x)^2 \mid f(x)^2\) 的充要条件是 \(g(x) \mid f(x)\).
\end{proposition}
\begin{proof}
充分性是显然的,只需证明必要性。设 \(f(x), g(x)\) 的公共标准分解为
\[
f(x) = cp_1(x)^{e_1}p_2(x)^{e_2} \cdots p_k(x)^{e_k}, \quad g(x) = dp_1(x)^{f_1}p_2(x)^{f_2} \cdots p_k(x)^{f_k},
\]
其中 \(p_i(x)\) 为互不相同的首一不可约多项式,\(c, d\) 是非零常数,则
\[
f(x)^2 = c^2 p_1(x)^{2e_1}p_2(x)^{2e_2} \cdots p_k(x)^{2e_k}, \quad g(x)^2 = d^2 p_1(x)^{2f_1}p_2(x)^{2f_2} \cdots p_k(x)^{2f_k}.
\]
若 \(g(x)^2 \mid f(x)^2\),则 \(2f_i \leq 2e_i\),从而 \(f_i \leq e_i (1 \leq i \leq k)\)。因此 \(g(x) \mid f(x)\).
\end{proof}

\section{多项式函数与根}

\begin{definition}[多项式的重根]\label{definition:多项式的重根}
设 $f(x) \in \mathbb{K}[x]$,$b \in \mathbb{K}$,若存在正整数 $k$,使 $(x - b)^k \mid f(x)$,但 $(x - b)^{k+1}$ 不能整除 $f(x)$,则称 $b$ 是 $f(x)$ 的一个 \textbf{$\boldsymbol{k}$ 重根}。若 $k = 1$,则称 $b$ 为\textbf{单根}。
\end{definition}

\begin{theorem}[多项式没有重因式的充要条件]\label{theorem:多项式没有重因式的充要条件}
数域\(\mathbb{K}\)上的多项式\(f(x)\)没有重因式的充分必要条件是\(f(x)\)与\(f'(x)\)互素.
\end{theorem}
\begin{proof}
设多项式\(p(x)\)是\(f(x)\)的\(m(m > 1)\)重因式, 则\(f(x)=p(x)^mg(x)\), 故
\[
f'(x)=mp(x)^{m - 1}p'(x)g(x)+p(x)^mg'(x).
\]
于是\(p(x)^{m - 1}\mid f'(x)\), 这表明\(f(x)\)与\(f'(x)\)有公因式\(p(x)^{m - 1}\).
反之, 若不可约多项式\(p(x)\)是\(f(x)\)的单因式, 可设\(f(x)=p(x)g(x), p(x)\)不能整除\(g(x)\). 于是
\[
f'(x)=p'(x)g(x)+p(x)g'(x).
\]
若\(p(x)\)是\(f'(x)\)的因式, 则\(p(x)\mid p'(x)g(x)\). 但\(p(x)\)不能整除\(g(x)\)且\(p(x)\)不可约, 故\(p(x)\mid p'(x)\). 而\(p'(x)\neq 0\)且\(\mathrm{deg }p'(x)<\mathrm{deg }p(x)\), 这是不可能的. 若\(f(x)\)无重因式, 则在\(f(x)\)的标准分解式\eqref{theorem5.12-5.4.2}中, \(e_i = 1\)对一切\(i = 1,2,\cdots,m\)成立, 于是\(p_i(x)\)都不能整除\(f'(x)\). 由于\(p_i(x)\)为不可约多项式, 故\((p_i(x),f'(x)) = 1\), 由\hyperref[proposition:互素多项式和最大公因式的基本性质]{互素多项式和最大公因式的基本性质(5)}可知
\[
(p_1(x)p_2(x)\cdots p_m(x),f'(x)) = 1,
\]
即\((f(x),f'(x)) = 1\).
\end{proof}

\begin{theorem}\label{theorem:多项式除去与其导数的最大公因式就能消去重因式}
设\(d(x)=(f(x),f'(x))\), 则\(f(x)/d(x)\)是一个没有重因式的多项式, 且这个多项式的不可约因式与\(f(x)\)的不可约因式相同 (不计重数).
\end{theorem}
\begin{proof}
设\(f(x)\)有如\eqref{theorem5.12-5.4.2}式的标准分解式, 则
\begin{align*}
f'(x)&=ce_1p_1(x)^{e_1 - 1}p_2(x)^{e_2}\cdots p_s(x)^{e_s}p_1'(x)\\
&+ce_2p_1(x)^{e_1}p_2(x)^{e_2 - 1}\cdots p_s(x)^{e_s}p_2'(x)\\
&+\cdots\\
&+ce_sp_1(x)^{e_1}p_2(x)^{e_2}\cdots p_s(x)^{e_s - 1}p_s'(x).\label{proposition5.12-5.4.4}
\end{align*}
因此\(p_1(x)^{e_1 - 1}p_2(x)^{e_2 - 1}\cdots p_s(x)^{e_s - 1}\)是\(f(x)\)与\(f'(x)\)的公因式. 注意到\(f(x)\)的因式一定具有\(p_1(x)^{k_1}p_2(x)^{k_2}\cdots p_s(x)^{k_s}\)的形状. 不妨设\(h(x)\)是\(f(x),f'(x)\)的公因式. 注意到\(p_1(x)^{e_1}\)可以整除\eqref{proposition5.12-5.4.4}式中右边除第一项外的所有项, 但不能整除第一项, 因此\(p_1(x)^{e_1}\)不能整除\(f'(x)\). 同理, \(p_i(x)^{e_i}\)不能整除\(f'(x)\). 由此我们不难看出
\[
h(x)\mid p_1(x)^{e_1 - 1}p_2(x)^{e_2 - 1}\cdots p_s(x)^{e_s - 1},
\]
即\(p_1(x)^{e_1 - 1}p_2(x)^{e_2 - 1}\cdots p_s(x)^{e_s - 1}=d(x)\). 显然\(f(x)/d(x)\)没有重因式且与\(f(x)\)含有相同的不可约因式.
\end{proof}

\begin{proposition}[多项式有k重根的充要条件]\label{proposition:多项式有k重根的充要条件}
求证:$a$ 是多项式 $f(x)$ 的 $k$ 重根的充要条件是:
\begin{align*}
f(a) = f'(a) = \cdots = f^{(k-1)}(a) = 0, \quad f^{(k)}(a) \neq 0.
\end{align*}
\end{proposition}
\begin{proof}
若 $a$ 是 $f(x)$ 的 $k$ 重根, 可设 $f(x) = (x - a)^k g(x)$, $g(x)$ 不含因式 $x - a$。通过对 $f(x)$ 求导可发现, $x - a$ 可整除 $f^{(j)}(x)$ ($1 \leq j \leq k - 1$)。因此
\begin{align*}
f(a) = f'(a) = \cdots = f^{(k-1)}(a) = 0.
\end{align*}
而 $f^{(k)}(a) = k! g(a) \neq 0$, 故必要性得证。

反之, 若 $a$ 是 $f(x)$ 的 $m$ 重根, 若 $m > k$, 则由必要性的证明可知, 将有 $f^{(k)}(a) = 0$, 这与已知矛盾。同样, 若 $m < k$, 则由必要性的证明可知, 将有 $f^{(m)}(a) \neq 0$, 这也与已知矛盾, 于是只能 $m = k$.
\end{proof}

\begin{proposition}\label{proposition:f有n重根的判断条件}
设 $\deg f(x) = n \geq 1$,若 $f'(x) \mid f(x)$,证明:$f(x)$ 有 $n$ 重根.
\end{proposition}
\begin{proof}
{\color{blue}证法一:}
设 $f(x) = \frac{1}{n}(x - a)f'(x)$,现证明 $a$ 是 $f(x)$ 的 $n$ 重根。假设 $a$ 是 $f(x)$ 的 $k$ 重根,$f(x) = (x - a)^k g(x)$,$k < n$ 且 $g(x)$ 不含因式 $x - a$,则
\begin{align*}
f'(x) = k(x - a)^{k-1}g(x) + (x - a)^k g'(x) = n(x - a)^{k-1}g(x).
\end{align*}
于是 $g(x) \mid (x - a)g'(x)$,而 $g(x)$ 与 $x - a$ 互素,故将有 $g(x) \mid g'(x)$。引出矛盾。

{\color{blue}证法二 :} 设 $f(x) = \frac{1}{n}(x - a)f'(x)$,则
\begin{align*}
\frac{f(x)}{(f(x), f'(x))} = b(x - a), \quad b \neq 0.
\end{align*}
由\hyperref[theorem:多项式除去与其导数的最大公因式就能消去重因式]{定理\ref{theorem:多项式除去与其导数的最大公因式就能消去重因式}}可知,$x - a$ 是 $f(x)$ 唯一的不可约因式,因此 $f(x) = b(x - a)^n$。
\end{proof}

\begin{proposition}
数域$\mathbb{F}$上任意一个不可约多项式在复数域$\mathbb{C}$中无重根.
\end{proposition}
\begin{proof}
设$f(x)$是$\mathbb{F}$上的不可约多项式,则deg$\,f(x)<$deg$\,f'(x)$.从而$f(x)\nmid f'(x)$,于是$(f(x),f'(x))=1$.故由\hyperref[theorem:多项式没有重因式的充要条件]{多项式没有重因式的充要条件}可知$f(x)$复数域$\mathbb{C}$中无重根.
\end{proof}

\begin{lemma}[次数不为1得到不可约多项式没有根]\label{lemma:次数不为1得到不可约多项式没有根}
设 $f(x)$ 是数域 $\mathbb{K}$ 上的不可约多项式且 $\deg f(x) \geq 2$,则 $f(x)$ 在 $\mathbb{K}$ 中没有根。
\end{lemma}
\begin{proof}
用反证法,设 $b \in \mathbb{K}$ 是 $f(x)$ 的根,由\hyperref[theorem:余数定理]{余数定理} 知 $(x - b) \mid f(x)$,即 $f(x) = (x - b)g(x)$ 可分解为两个低次多项式之积,这与 $f(x)$ 不可约矛盾。
\end{proof}

\begin{theorem}[多项式根的有限性]\label{theorem:多项式根的有限性}
设\(f(x)\)是数域\(\mathbb{F}\)上的\(n\)次多项式,则\(f(x)\)在\(\mathbb{F}\)中最多只有\(n\)个根.
\end{theorem}
\begin{note}
由\hyperref[proposition:多项式根的有限性]{命题\ref{proposition:多项式根的有限性}}可知,若一个$n$次多项式的根超过$n$个,则这个多项式一定恒为零.
\end{note}
\begin{proof}
将 $f(x)$ 作标准因式分解,则由\hyperref[lemma:次数不为1得到不可约多项式没有根]{次数不为1得到不可约多项式没有根}知 $f(x)$ 在 $\mathbb{K}$ 中根的个数等于该分解式中一次因式的个数,它不会超过 $n$。
\end{proof}

\begin{corollary}[两个多项式相等的判定准则]\label{corollary:两个多项式相等的判定准则}
设 $f(x)$ 与 $g(x)$ 是 $\mathbb{K}$ 上的次数不超过 $n$ 的两个多项式,若存在 $\mathbb{K}$ 上 $n + 1$ 个不同的数 $b_1, b_2, \ldots, b_{n+1}$,使
\begin{align*}
f(b_i) = g(b_i), \quad i = 1, 2, \ldots, n + 1,
\end{align*}
则 $f(x) = g(x)$.
\end{corollary}
\begin{proof}
作 $h(x) = f(x) - g(x)$,显然 $h(x)$ 次数不超过 $n$。但它有 $n + 1$ 个不同的根,因此只可能 $h(x) = 0$,即 $f(x) = g(x)$。
\end{proof}

\begin{example}
求证:$f(x) = \sin x$ 在实数域内不能表示为 $x$ 的多项式。
\end{example}
\begin{proof}
注意到 $f(x) = \sin x$ 在实数域内有无穷多个根,而任一非零多项式只能有有限个根,因此 $f(x) = \sin x$ 在实数域内不能表示为 $x$ 的多项式。
\end{proof}

\begin{example}
设 $f(x)$ 是数域 $\mathbb{F}$ 上的多项式,若对 $\mathbb{F}$ 中某个非零常数 $a$,有 $f(x + a) = f(x)$,求证:$f(x)$ 必是常数多项式。
\end{example}
\begin{proof}
假设 $f(x)$ 不是常数多项式,则 $f(x) - f(a)$ 也不是常数多项式,但由 $f(x + a) = f(x)$ 可知,$ka \ (k \in \mathbb{Z})$ 是 $f(x) - f(a)$ 的无穷多个根,矛盾。
\end{proof}

\begin{example}
设 $f(x)$ 是非常数多项式且 $f(x)$ 可以整除 $f(x^m) \ (m \in \mathbb{N}_+)$,求证:$f(x)$ 的根只能是 $0$ 或 $1$ 的某个方根。
\end{example}
\begin{proof}
将 $f(x)$ 看成复数域上的多项式,则 $f(x^m) = f(x)g(x)$。假设 $c$ 是 $f(x)$ 的一个复根,即 $f(c) = 0$,则 $f(c^m) = 0$,即 $c^m$ 也是 $f(x)$ 的根。由此可知 $c^m, c^{m^2}, c^{m^3}, \ldots$ 也都是 $f(x)$ 的根。由于 $f(x)$ 只有有限个不同的复根,故存在正整数 $k>t$,使得 $c^{m^k} = c^{m^t}$。因此若 $c \neq 0$,取$n=m^k-m^t\in \mathbb{N}_+$,则有 $c^n = 1$。
\end{proof}


\begin{theorem}[余数定理]\label{theorem:余数定理}
设\(f(x)\in\mathbb{F}[x], b\in\mathbb{F}\),则存在\(\mathbb{F}\)上的多项式\(g(x)\),使得
\[
f(x)=(x - b)g(x)+f(b).
\]
特别地,\(b\)是\(f(x)\)的根的充要条件是\((x - b)\mid f(x)\).
\end{theorem}
\begin{note}
利用余数定理可以实现求根与判断整除性之间的相互转换.
\end{note}
\begin{proof}
由带余除法知
\begin{align*}
f(x) = (x - b)g(x) + r(x),
\end{align*}
其中 $\deg r(x) < 1$,因此 $r(x)$ 为常数多项式。在 上式中用 $b$ 代替 $x$,即得 $r(x) = f(b)$。
\end{proof}

\begin{example}
设 $n$ 是奇数,求证:$(x + y)(y + z)(x + z)$ 可整除 $(x + y + z)^n - x^n - y^n - z^n$。
\end{example}
\begin{proof}
将多项式 $(x + y + z)^n - x^n - y^n - z^n$ 看成是未定元 $x$ 的多项式。当 $x = -y$ 时,$(x + y + z)^n - x^n - y^n - z^n = 0$,因此由\hyperref[theorem:余数定理]{余数定理}可知$x + y$ 是 $(x + y + z)^n - x^n - y^n - z^n$ 的因式。同理 $x + z, y + z$ 也是因式。又这 3 个因式互素,故$(x + y)(y + z)(x + z)$ 可整除 $(x + y + z)^n - x^n - y^n - z^n$。
\end{proof}

\begin{example}
设 $f(x)$ 是一个 $n$ 次多项式,若当 $k = 0, 1, \ldots, n$ 时有 $f(k) = \frac{k}{k + 1}$,求 $f(n + 1)$。
\end{example}
\begin{proof}
解 今 $g(x) = (x + 1)f(x) - x$,则 $0, 1, \ldots, n$ 是 $g(x)$ 的根,因此
\begin{align*}
g(x) = c x(x - 1)(x - 2) \cdots (x - n),
\end{align*}
即
\begin{align*}
(x + 1)f(x) - x = c x(x - 1)(x - 2) \cdots (x - n),
\end{align*}
其中 $c$ 是一个常数。令 $x = -1$,可求出 $c = \frac{(-1)^{n+1}}{(n + 1)!}$。从而
\begin{align*}
f(x) = \frac{1}{x + 1} \left( \frac{(-1)^{n+1} x(x - 1) \cdots (x - n)}{(n + 1)!} + x \right),
\end{align*}
故
\begin{align*}
f(n + 1) = \frac{1}{n + 2} \left( \frac{(-1)^{n+1} (n + 1)!}{(n + 1)!} + n + 1 \right).
\end{align*}
当 $n$ 是奇数时,$f(n + 1) = 1$;当 $n$ 是偶数时,$f(n + 1) = \frac{n}{n + 2}$。
\end{proof}

\begin{example}
设 $(x^4 + x^3 + x^2 + x + 1) \mid (x^3 f_1(x^5) + x^2 f_2(x^5) + x f_3(x^5) + f_4(x^5))$,这里 $f_i(x) \ (1 \leq i \leq 4)$ 都是实系数多项式,求证:$f_i(1) = 0 \ (1 \leq i \leq 4)$。
\end{example}
\begin{proof}
设 $\varepsilon_i \ (1 \leq i \leq 4)$ 是 1 的五次虚根,则 $\varepsilon_i \ (1 \leq i \leq 4)$ 都适合 $x^5 - 1$,从而由余数定理可知
\begin{align*}
x^5 - 1 = (x - 1)(x - \varepsilon_1)(x - \varepsilon_2)(x - \varepsilon_3)(x - \varepsilon_4) = (x - 1)(x^4 + x^3 + x^2 + x + 1)。
\end{align*}
故
\begin{align*}
x^4 + x^3 + x^2 + x + 1 = (x - \varepsilon_1)(x - \varepsilon_2)(x - \varepsilon_3)(x - \varepsilon_4)。
\end{align*}
因此 $\varepsilon_i \ (1 \leq i \leq 4)$ 都是 $x^4 + x^3 + x^2 + x + 1$ 的根.
由条件可得
\begin{align*}
\varepsilon_i^3 f_1(1) + \varepsilon_i^2 f_2(1) + \varepsilon_i f_3(1) + f_4(1) = 0 \quad (1 \leq i \leq 4)。
\end{align*}
这是一个由 4 个未知数、4 个方程式组成的线性方程组(将 $f_i(1)$ 看成是未知数),其系数行列式是一个 Vandermonde 行列式,显然其值不等于零,因此 $f_i(1) = 0$。□
\end{proof}

\section{复系数多项式}

\begin{theorem}[代数基本定理]\label{theorem:代数基本定理}
次数大于零的复数域上的一元多项式至少有一个复数根。
\end{theorem}
\begin{proof}
设复数域上的 $n$ 次多项式为
\begin{align}
f(z) = a_n z^n + a_{n-1} z^{n-1} + \cdots + a_1 z + a_0。 \label{theorem5.14-5.6.1}
\end{align}
首先证明,必存在一个复数 $z_0$,使对一切复数 $z$,有
\begin{align*}
|f(z)| \geq |f(z_0)|。
\end{align*}
令 $z = x + iy$,其中 $x, y$ 是实变量。展开 $f(x + iy)$ 并分开实部和虚部,则
\begin{align*}
f(z) = u(x, y) + iv(x, y),
\end{align*}
其中 $u(x, y)$ 及 $v(x, y)$ 为实系数二元多项式函数。又
\begin{align*}
|f(z)| = \sqrt{u(x, y)^2 + v(x, y)^2}
\end{align*}
是一个二元连续函数,但
\begin{align*}
|f(z)| &= |a_n z^n + a_{n-1} z^{n-1} + \cdots + a_0| \\
&\geq |a_n z^n| - |a_{n-1} z^{n-1} + \cdots + a_0| \\
&\geq |z|^n \left( |a_n| - \left| \frac{a_{n-1}}{z} \right| - \left| \frac{a_{n-2}}{z^2} \right| + \cdots + \left| \frac{a_0}{z^n} \right| \right),
\end{align*}
因此当 $|z| \to \infty$ 时,$|f(z)| \to \infty$。于是必存在一个实常数 $R$,当 $|z| \geq R$ 时,$|f(z)|$ 充分大,因此 $|f(z)|$ 的最小值必含于圆圈 $|z| \leq R$ 中。但这是平面上的一个闭区域,因此必存在 $z_0$ 使 $|f(z_0)|$ 为最小。

接下来要证明 $f(z_0) = 0$。用反证法,即若 $f(z_0) \neq 0$,则必可找到 $z_1$,使
$|f(z_1)| < |f(z_0)|$,这样就与 $|f(z_0)|$ 是最小值相矛盾。
将 $z = z_0 + h$ 代入 \eqref{theorem5.14-5.6.1}式便可得到一个关于 $h$ 的 $n$ 次多项式:
\begin{align}
f(z_0 + h) = b_n h^n + b_{n-1} h^{n-1} + \cdots + b_1 h + b_0. \label{theorem5.14-5.6.2}
\end{align}
当 $h = 0$ 时,$f(z_0) = b_0$,由假设 $f(z_0) \neq 0$,故
\begin{align*}
\frac{f(z_0 + h)}{f(z_0)} = \frac{b_n}{f(z_0)} h^n + \frac{b_{n-1}}{f(z_0)} h^{n-1} + \cdots + \frac{b_1}{f(z_0)} h + 1。
\end{align*}
$b_1, b_2, \ldots, b_n$ 中有些可能为零,但绝不全为零。设 $b_k$ 是第一个不为零的复数,则
\begin{align}
\frac{f(z_0 + h)}{f(z_0)} = 1 + c_k h^k + c_{k+1} h^{k+1} + \cdots + c_n h^n, \label{theorem5.14-5.6.3}
\end{align}
其中 $c_j = \frac{b_j}{f(z_0)}$。令 $d = \sqrt[k]{\frac{1}{|c_k|}}$,$h = ed$ 代入\eqref{theorem5.14-5.6.3}式得
\begin{align*}
\frac{f(z_0 + h)}{f(z_0)} = 1 - e^k + e^{k+1} (c_{k+1} d^{k+1} + c_{k+2} d^{k+2} e + \cdots)。
\end{align*}
取充分小的正实数 $e$(至少小于 1),使
\begin{align*}
e (|c_{k+1} d^{k+1}| + |c_{k+2} d^{k+2}| + \cdots) < \frac{1}{2},
\end{align*}
于是
\begin{align*}
\left| \frac{f(z_0 + h)}{f(z_0)} \right| &\leq |1 - e^k| + |e^{k+1} (c_{k+1} d^{k+1} + c_{k+2} d^{k+2} e + \cdots)| \\
&\leq 1 - e^k + e^{k+1} (|c_{k+1} d^{k+1}| + |c_{k+2} d^{k+2}| + \cdots) \\
&< 1 - e^k + \frac{1}{2} e^k \\
&= 1 - \frac{1}{2} e^k < 1。
\end{align*}
将这样的 $e$ 代入 $h = ed$,得
\begin{align*}
|f(z_0 + ed)| < |f(z_0)|。
\end{align*}
这就推出了矛盾。
\end{proof}

\begin{corollary}
\begin{enumerate}
\item 复数域上的一元 $n$ 次多项式恰有 $n$ 个复根(包括重根)。

\item 复数域上的不可约多项式都是一次多项式。

\item 复数域上的一元 $n$ 次多项式必可分解为一次因式的乘积。
\end{enumerate}
\end{corollary}

\begin{theorem}[Vieta定理]\label{theorem:Vieta定理}
若数域\(\mathbb{F}\)上的多项式\(f(x)=a_0x^n + a_1x^{n - 1}+\cdots + a_{n - 1}x + a_n\)在\(\mathbb{F}\)中有\(n\)个根\(x_1,x_2,\cdots,x_n\),则
\begin{align*}
&\,\,\,\, \sum_{i=1}^n{x_i}=x_1+x_2+\cdots +x_n=-\frac{a_1}{a_0},
\\
&\sum_{1\le i<j\le n}{x_ix_j}=x_1x_2+\cdots +x_1x_n+x_2x_3+\cdots +x_{n-1}x_n=\frac{a_2}{a_0},
\\
&\,\,\,\, \cdots \cdots \cdots \cdots 
\\
&\,\,\,\, x_1x_2\cdots x_n=(-1)^n\frac{a_n}{a_0}.
\end{align*}
\end{theorem}
\begin{proof}
$f(x) = a_0 (x - x_1)(x - x_2) \cdots (x - x_n)$,将这个式子的右边展开与 $f(x)$ 比较系数即得结论。
\end{proof}

\begin{example}
\begin{enumerate}[(1)]
\item 设三次方程 $x^3 + px^2 + qx + r = 0$ 的 3 个根成等差数列,求证:
\begin{align*}
2p^3 - 9pq + 27r = 0。
\end{align*}

\item 设三次方程 $x^3 + px^2 + qx + r = 0 \ (r \neq 0)$ 的 3 个根成等比数列,求证:
\begin{align*}
rp^3 = q^3。
\end{align*}

\item 设多项式 $x^3 + 3x^2 + mx + n$ 的 3 个根成等差数列,多项式 $x^3 - (m - 2)x^2 + (n - 3)x + 8$ 的 3 个根成等比数列,求 $m$ 和 $n$。
\end{enumerate}
\end{example}
\begin{proof}
\begin{enumerate}[(1)]
\item 设方程的 3 个根为 $c - d, c, c + d$,则由 Vieta 定理可得
\begin{align*}
\begin{cases}
3c = -p, \\
3c^2 - d^2 = q, \\
c(c^2 - d^2) = -r。
\end{cases}
\end{align*}
由此可得 $2p^3 - 9pq + 27r = 0。$

\item 设方程的 3 个根为 $\frac{c}{d}, c, cd$,则由 Vieta 定理可得
\begin{align*}
\begin{cases}
\frac{c}{d} + c + cd = -p, \\
\frac{c^2}{d} + c^2 + c^2 d = q, \\
\frac{c^3}{d} = -r。
\end{cases}
\end{align*}
由此可得 $rp^3 = q^3。$

\item 由(1)(2)可知,$m, n$ 应满足如下关系:
\begin{align*}
\begin{cases}
m = n + 2, \\
-8(m - 2)^3 = (n - 3)^3。
\end{cases}
\end{align*}
若 $n - 3 = -2(m - 2)$,则可联立求得 $m = 3, n = 1$。

若 $n - 3 = -2\omega(m - 2)$,其中 $\omega = -\frac{1}{2} + \frac{\sqrt{3}}{2}i$,则可联立求得 $m = 2 - \sqrt{3}i, n = -\sqrt{3}i$。

若 $n - 3 = -2\omega^2(m - 2)$,则可联立求得 $m = 2 + \sqrt{3}i, n = \sqrt{3}i。$
\end{enumerate}
\end{proof}

\begin{example}
设 $x_1, x_2, x_3$ 是三次方程 $x^3 + px^2 + qx + r = 0 \ (r \neq 0)$ 的 3 个根,求这 3 个根倒数的平方和。
\end{example}
\begin{proof}
由 Vieta 定理可得
\begin{align*}
\frac{1}{x_1^2} + \frac{1}{x_2^2} + \frac{1}{x_3^2} = \frac{(x_1 x_2 + x_1 x_3 + x_2 x_3)^2 - 2x_1 x_2 x_3 (x_1 + x_2 + x_3)}{x_1^2 x_2^2 x_3^2}= \frac{q^2 - 2pr}{r^2}。
\end{align*}
\end{proof}

\begin{example}
已知方程 $x^3 + px^2 + qx + r = 0$ 的 3 个根为 $x_1, x_2, x_3$,求一个三次方程使其根为 $x_1^3, x_2^3, x_3^3$。
\end{example}
\begin{note}
利用代数恒等式:$a^3+b^3+c^3=\left( a+b+c \right) ^3-3\left( a+b+c \right) \left( ab+bc+ac \right) +3abc$得到
\begin{align*}
x_{1}^{3}+x_{2}^{3}+x_{3}^{3}=\left( x_1+x_2+x_3 \right) ^3-3\left( x_1+x_2+x_3 \right) \left( x_1x_2+x_2x_3+x_1x_3 \right) +3x_1x_2x_3.
\\
x_{1}^{3}x_{2}^{3}+x_{1}^{3}x_{3}^{3}+x_{2}^{3}x_{3}^{3}=\left( x_1x_2+x_2x_3+x_1x_3 \right) ^3-3x_1x_2x_3\left( x_1+x_2+x_3 \right) \left( x_1x_2+x_2x_3+x_1x_3 \right) +3x_{1}^{3}x_{3}^{3}x_{2}^{3}x_{3}^{3}.
\end{align*}
即可由Vieta 定理得到结果.
\end{note}
\begin{proof}
由 Vieta 定理经计算可得
\begin{align*}
\begin{cases}
x_1^3 + x_2^3 + x_3^3 = -p^3 + 3pq - 3r, \\
x_1^3 x_2^3 + x_1^3 x_3^3 + x_2^3 x_3^3 = q^3 - 3pqr + 3r^2, \\
x_1^3 x_2^3 x_3^3 = -r^3。
\end{cases}
\end{align*}
因此,以 $x_1^3, x_2^3, x_3^3$ 为根的三次方程为
\begin{align*}
x^3 + (p^3 - 3pq + 3r)x^2 + (q^3 - 3pqr + 3r^2)x + r^3 = 0。□
\end{align*}
\end{proof}

\begin{example}
设多项式 $x^3 + px^2 + qx + r$ 的 3 个根都是实数,求证:$p^2 \geq 3q$。
\end{example}
\begin{proof}
设多项式的 3 个根为 $x_1, x_2, x_3$,由已知条件可知:
\begin{align*}
(x_1 - x_2)^2 + (x_2 - x_3)^2 + (x_1 - x_3)^2 \geq 0。
\end{align*}
用 Vieta 定理可计算出
\begin{align*}
&(x_1-x_2)^2+(x_2-x_3)^2+(x_1-x_3)^2=2\left( x_{1}^{2}+x_{2}^{2}+x_{3}^{2} \right) -2\left( x_1x_2+x_2x_3+x_1x_3 \right) 
\\
&=2\left( x_1+x_2+x_3 \right) ^2-6\left( x_1x_2+x_2x_3+x_1x_3 \right) =2(p^2-3q).
\end{align*}
因此结论为真。
\end{proof}

\begin{example}
设 $f(x) = a_n x^n + a_{n-1} x^{n-1} + \cdots + a_1 x + a_0$ 的 $n$ 个根 $x_1, x_2, \cdots, x_n$ 皆不等于零,求以 $\frac{1}{x_1}, \frac{1}{x_2}, \cdots, \frac{1}{x_n}$ 为根的多项式。
\end{example}
\begin{proof}
令
\begin{align*}
g(x) = a_0 x^n + a_1 x^{n-1} + \cdots + a_{n-1} x + a_n,
\end{align*}
则
\begin{align*}
x^n g\left(\frac{1}{x_i}\right) = a_0 + a_1 x_i + \cdots + a_{n-1} x_i^{n-1} + a_n x_i^n = f(x_i) = 0。
\end{align*}
因为 $x_i \neq 0$,故 $g\left(\frac{1}{x_i}\right) = 0$,即 $g(x)$ 的根为 $f(x)$ 根之倒数。
\end{proof}

\begin{example}
设 $f(x) = a_n x^n + a_{n-1} x^{n-1} + \cdots + a_1 x + a_0 \ (a_n \neq 0)$ 是数域 $\mathbb{F}$ 上的可约多项式,求证:多项式 $g(x) = a_0 x^n + a_1 x^{n-1} + \cdots + a_{n-1} x + a_n$ 在 $\mathbb{F}$ 上也可约。
\end{example}
\begin{proof}
设 $f(x) = p(x) q(x)$,其中 $\deg p(x) = m, \deg q(x) = n - m, 0 < m < n$,则
\begin{align*}
g(x) = x^n f\left(\frac{1}{x}\right) = x^n p\left(\frac{1}{x}\right) q\left(\frac{1}{x}\right) = \left(x^m p\left(\frac{1}{x}\right)\right) \left(x^{n-m} q\left(\frac{1}{x}\right)\right),
\end{align*}
因此 $g(x)$ 也可约.
\end{proof}


\section{实系数多项式}

\begin{theorem}[实系数多项式的复根成对出现]\label{theorem:实系数多项式的复根成对出现}
设 $f(x) = a_n x^n + a_{n-1} x^{n-1} + \cdots + a_1 x + a_0$ 是实系数多项式,若复数 $a + bi \ (b \neq 0)$ 是其根,则 $a - bi$ 也是它的根。
\end{theorem}
\begin{proof}
令 $z = a + bi$,其共轭复数为 $\overline{z} = a - bi$,则
\begin{align*}
f(\overline{z})=a_n\overline{z}^n+a_{n-1}\overline{z}^{n-1}+\cdots +a_1\overline{z}+a_0=\overline{a_nz^n+a_{n-1}z^{n-1}+\cdots +a_1z+a_0}=0.
\end{align*}
由此即得结论。
\end{proof}

\begin{corollary}
实数域上的不可约多项式为一次多项式或下列二次多项式:
\begin{align*}
ax^2 + bx + c, \quad \text{其中} \ b^2 - 4ac < 0。
\end{align*}
\end{corollary}
\begin{proof}
一次多项式显然为不可约。当 $b^2 - 4ac < 0$ 时,$ax^2 + bx + c$ 没有实根,故不可约。

反过来,任一高于二次的实系数多项式 $f(x)$ 如有实根,则 $f(x)$ 可约;如有一复根 $a + bi \ (b \neq 0)$,则 $a - bi$ 也是它的根,从而
\begin{align*}
(x - (a + bi))(x - (a - bi)) = x^2 - 2ax + (a^2 + b^2)
\end{align*}
是 $f(x)$ 的因式,故任一高于二次的实系数多项式$f(x)$都可约.从而我们只需考虑一次和二次多项式的情况.

(i)当$f(x)$为一次实系数多项式时,显然$f(x)$一定不可约.

(ii)当$f(x)$为二次多项式时,设$f(x)=ax^2+bx+c$,则当$f(x)$有实根,即$b^2-4ac\geq 0$时,设$f(x)$的两个实根分别为$x_1,x_2$,则$f(x)=(x-x_1)(x-x_2)$,此时$f(x)$可约.当$f(x)$无实根,即$b^2-4ac<0$时,此时$f(x)$在实数域上不可约.
\end{proof}

\begin{corollary}
实数域上的多项式$f(x)$必可分解为有限个一次因式及不可约二次因式的乘积.
\end{corollary}

\begin{proposition}\label{proposition:实系数多项式的判定条件}
设 $f(x)$ 是复数域上的多项式,若对任意的实数 $c$,$f(c)$ 总是实数,求证:$f(x)$ 是实系数多项式。
\end{proposition}
\begin{proof}
设 $f(x) = a_n x^n + a_{n-1} x^{n-1} + \cdots + a_1 x + a_0$,分别令 $x = 0, 1, 2, \cdots, n$,得到一个以 $a_n, a_{n-1}, \cdots, a_1, a_0$ 为未知数,由 $n + 1$ 个方程式组成的实系数线性方程组。该方程组的系数行列式是一个非零的 Vandermonde 行列式,故方程组必有唯一解,且解为实数。因此 $f(x)$ 是实系数多项式.
\end{proof}

\begin{example}
证明:奇数次实系数多项式必有实数根.
\end{example}
\begin{proof}
实系数多项式的虚根总是成对出现的,因此奇数次实系数多项式必有实数根.
\end{proof}

\begin{proposition}\label{proposition:实系数多项式的根的符号判定准则}
设 $f(x) = a_n x^n + a_{n-1} x^{n-1} + \cdots + a_1 x + a_0$ 是实系数多项式,求证:
\begin{enumerate}[(1)]
\item 若 $a_i \ (0 \leq i \leq n)$ 全是正数或全是负数,则 $f(x)$ 没有非负实根,即只有负实根.
\item 若 $(-1)^i a_i \ (0 \leq i \leq n)$ 全是正数或全是负数,则 $f(x)$ 没有非正实根,即只有正实根.
\item 若 $a_n > 0$ 且 $(-1)^{n-i} a_i > 0 \ (0 \leq i \leq n-1)$,则 $f(x)$ 没有非正实根,即只有正实根.;若 $a_n > 0$ 且 $(-1)^{n-i} a_i \geq 0 \ (0 \leq i \leq n-1)$,则 $f(x)$ 没有负实根,,即只有正实数和零能作为根.
\end{enumerate}
\end{proposition}
\begin{proof}
(1) 若 $a_i$ 全是正数且 $f(x)$ 有非负实根 $c \geq 0$,代入后可得
\begin{align*}
f(c) = a_n c^n + a_{n-1} c^{n-1} + \cdots + a_1 c + a_0 \geq a_0 > 0,
\end{align*}
这和 $c$ 是根矛盾,因此 $f(x)$ 没有非负实根。同理可证 $a_i$ 全是负数的情形。

(2) 和 (3) 同理可证。
\end{proof}

\begin{example}
求证:实系数方程 $x^3 + px^2 + qx + r = 0$ 的根的实部全是负数的充要条件是
\begin{align*}
p > 0, \quad r > 0, \quad pq > r。
\end{align*}
\end{example}
\begin{proof}
先证必要性:设原方程的 3 个根为 $x_1, x_2, x_3$,其中 $x_1$ 是实数根,$x_1 < 0$。另假设 $x_2 = a + bi, x_3 = a - bi, a < 0$,则由Vieta定理可得
\begin{align*}
p &= -(x_1 + x_2 + x_3) = -(x_1 + 2a) > 0, \\
r &= -x_1 x_2 x_3 = -x_1(a^2 + b^2) > 0。
\end{align*}
\begin{align*}
pq - r &= -(x_1 + 2a)(x_1 x_2 + x_1 x_3 + x_2 x_3) + x_1(a^2 + b^2) \\
&= -(x_1 + 2a)(2x_1 a + a^2 + b^2) + x_1(a^2 + b^2) \\
&= -2a((x_1 + a)^2 + b^2) > 0。
\end{align*}
又假设 $x_1, x_2, x_3$ 全是负实数,则显然 $p > 0, q > 0, r > 0$,而
\begin{align*}
pq - r &= -(x_1 + x_2 + x_3)(x_1 x_2 + x_1 x_3 + x_2 x_3) + x_1 x_2 x_3 \\
&= -((x_1 + x_2 + x_3)(x_1 x_2 + x_1 x_3 + x_2 x_3) + x_1 x_2 x_3) > 0。
\end{align*}
再证充分性:由 $p > 0, r > 0, pq - r > 0$ 可知 $q > 0$,若方程的根是实数,则由\hyperref[proposition:实系数多项式的根的符号判定准则]{命题\ref{proposition:实系数多项式的根的符号判定准则}(1)}可知,此根必是负数。现假设方程有根 $x_1 < 0, x_2 = a + bi, x_3 = a - bi$,因为
\begin{align*}
pq - r = -2a((x_1 + a)^2 + b^2) > 0,
\end{align*}
故得 $a < 0$,结论得证。
\end{proof}

\begin{example}
设 $\varepsilon$ 是 1 的 $n$ 次根:
\begin{align*}
\varepsilon = \cos \frac{2\pi}{n} + i \sin \frac{2\pi}{n},
\end{align*}
求证:$\varepsilon^{mi} \ (1 \leq i \leq n)$ 是 $x^n - 1 = 0$ 的全部根的充要条件是 $(m, n) = 1$。
\end{example}
\begin{proof}
若 $(m, n) = 1$,只要证明 $\varepsilon^{mi} \ (1 \leq i \leq n)$ 互不相同即可。若不然,有 $\varepsilon^{ms} = \varepsilon^{mt} \ (1 \leq s < t \leq n)$,便有 $\varepsilon^{m(t-s)} = 1$,$n \mid m(t-s)$。因为 $m, n$ 互素,故 $n \mid (t-s)$,而 $0 < t-s < n$,矛盾。

反之,若 $(m, n) = d > 1$,则 $\varepsilon^{m \frac{n}{d}} = \varepsilon^{n \frac{m}{d}} = 1$,而$\frac{n}{d}\in \mathbb{N}_+$且$\frac{n}{d}<n$,故$\varepsilon ^{mj}=\varepsilon ^{m\left( j-\frac{n}{d} \right) +m\frac{n}{d}}=\varepsilon ^{m\left( j-\frac{n}{d} \right)},j=\frac{n}{d}+1,\cdots ,n.$于是$\varepsilon ^{mi}\left( i=1,2,\cdots ,n \right) $只生成了$\frac{n}{d}<n$个不同根.而$x^n-1$有$n$个不同根,故$\varepsilon ^{mi}\left( i=1,2,\cdots ,n \right)$无法覆盖所有$x^n-1$的所有$n$个不同根.
从而 $\varepsilon^{mi} \ (1 \leq i \leq n)$ 不可能是 $x^n - 1 = 0$ 的全部根。
\end{proof}

\begin{proposition}
设\( f(x) \) 是实系数首一多项式且无实数根,求证:\( f(x) \) 可以表示为两个实系数多项式的平方和。
\end{proposition}
\begin{proof}
因为实系数多项式的虚根成对出现,故 \( f(x) \) 是偶数次多项式,不妨设它的根为
\[
x_1, x_2, \cdots, x_n; \overline{x_1}, \overline{x_2}, \cdots, \overline{x_n}.
\]
令
\[
u(x) = (x - x_1)(x - x_2) \cdots (x - x_n); \quad v(x) = (x - \overline{x_1})(x - \overline{x_2}) \cdots (x - \overline{x_n}),
\]
则 \( v(x) = \overline{u(x)} \),\( f(x) = u(x)v(x) \)。又将 \( u(x), v(x) \) 的实部和虚部分开,可设
\[
u(x) = g(x) + ih(x), \quad v(x) = g(x) - ih(x),
\]
即有
\[
f(x) = g(x)^2 + h(x)^2.
\]
\end{proof}



\section{有理系数多项式}

\begin{theorem}[整系数多项式有理根定理]\label{theorem:整系数多项式有理根定理}
设有 \( n \) 次整系数多项式
\begin{align}
f(x) = a_n x^n + a_{n-1} x^{n-1} + \cdots + a_1 x + a_0, \label{theorem156-5.7.1}
\end{align}
则有理数 \( \frac{q}{p} \) 是 \( f(x) \) 的根的必要条件是 \( p \mid a_n, q \mid a_0 \),其中 \( p, q \) 是互素的整数。
\end{theorem}
\begin{proof}
将 \( \frac{q}{p} \) 代入\eqref{theorem156-5.7.1}式得
\[
a_n \left( \frac{q}{p} \right)^n + a_{n-1} \left( \frac{q}{p} \right)^{n-1} + \cdots + a_1 \left( \frac{q}{p} \right) + a_0 = 0,
\]
将上式两边乘以 \( p^n \) 得
\[
a_n q^n + a_{n-1} q^{n-1} p + \cdots + a_1 q p^{n-1} + a_0 p^n = 0.
\]
从而
\begin{align*}
q\left( a_nq^{n-1}+a_{n-1}q^{n-2}p+\cdots +a_1p^{n-1} \right) =-a_0p^n.
\end{align*}
于是$q\mid a_0p^n$,又因为$(q,p)=1$,所以$q\mid a_0$.同理可得$p \mid a_n$.
\end{proof}

\begin{definition}[本原多项式]
设多项式
\[
f(x) = a_n x^n + a_{n-1} x^{n-1} + \cdots + a_1 x + a_0
\]
是整系数多项式,若 \( a_n, a_{n-1}, \cdots, a_1, a_0 \) 的最大公约数等于 1,则称 \( f(x) \) 为\textbf{本原多项式}.
\end{definition}

\begin{lemma}[Gauss引理]\label{lemma:Gauss引理}
两个本原多项式之积仍是本原多项式。
\end{lemma}
\begin{proof}
设
\[
f(x) = a_n x^n + a_{n-1} x^{n-1} + \cdots + a_1 x + a_0,
\]
\[
g(x) = b_m x^m + b_{m-1} x^{m-1} + \cdots + b_1 x + b_0
\]
是两个本原多项式。若
\[
f(x) g(x) = c_{m+n} x^{m+n} + c_{m+n-1} x^{m+n-1} + \cdots + c_1 x + c_0
\]
不是本原多项式,则 \( c_0, c_1, \cdots, c_{m+n} \) 必有一个公约素因子 \( p \)。因为 \( f(x) \) 是本原多项式,故 \( p \) 不能整除 \( f(x) \) 的所有系数,可设 \( p \mid a_0, p \mid a_1, \cdots, p \mid a_{i-1} \),但 \( p \) 不能整除 \( a_i \)。同理,可设 \( p \mid b_0, p \mid b_1, \cdots, p \mid b_{j-1} \),但 \( p \) 不能整除 \( b_j \)。注意到
\[
c_{i+j} = \cdots + a_{i-2} b_{j+2} + a_{i-1} b_{j+1} + a_i b_j + a_{i+1} b_{j-1} + \cdots,
\]
\( p \) 可整除 \( c_{i+j} \),\( p \) 也能整除右式除 \( a_i b_j \) 以外的所有项。但 \( p \) 不能整除 \( a_i \) 和 \( b_j \),故 \( p \) 不能整除 \( a_i b_j \),引出矛盾。
\end{proof}

\begin{theorem}\label{theorem:整系数多项式在有理数域上可约,则一定可以分解成两个次数较低的整系数多项式之积}
若整系数多项式 $f(x)$ 在有理数域上可约,则它必可分解为两个次数较低的整系数多项式之积。
\end{theorem}
\begin{proof}
假设整系数多项式 $f(x)$ 可以分解为两个次数较低的有理系数多项式之积:
\begin{align*}
f(x) = g(x)h(x),
\end{align*}
$g(x)$ 的各项系数为有理数,必有一个公分母记为 $c$,于是 $g(x) = \frac{1}{c}(cg(x))$,其中 $cg(x)$ 为整系数多项式。若把 $cg(x)$ 中所有系数的最大公因数 $d$ 提出来,则
\begin{align*}
g(x) = \frac{d}{c}\left(\frac{c}{d}g(x)\right),
\end{align*}
$\frac{c}{d}g(x)$ 是一个本原多项式。这表明 $g(x) = ag_1(x)$,$a$ 为有理数,$g_1(x)$ 为本原多项式。同理,$h(x) = bh_1(x)$,其中 $b$ 为有理数,$h_1(x)$ 为本原多项式。于是我们得到
\begin{align*}
f(x) = g(x)h(x) = abg_1(x)h_1(x).
\end{align*}
由\hyperref[lemma:Gauss引理]{Gauss引理}知,$g_1(x)h_1(x)$ 是本原多项式。若 $ab$ 不是一个整数,则 $abg_1(x)h_1(x)$ 将不是整系数多项式,这与 $f(x)$ 是整系数多项式相矛盾。因此 $ab$ 必须是整数,于是 $f(x)$ 可以分解为两个次数较小的整系数多项式之积。     
\end{proof}

\begin{definition}[整系数多项式在整数环上可约]
我们通常称一个整系数多项式$f(x)$\textbf{在整数环上可约},若它可以分解为两个次数较低的整系数多项式之积.
\end{definition}

\begin{proposition}\label{proposition:整数环上不可约一定在有理数域上不可约}
整系数多项式$f(x)$若在整数环上不可约,则在有理数域上也不可约.
\end{proposition}
\begin{proof}
由\hyperref[theorem:整系数多项式在有理数域上可约,则一定可以分解成两个次数较低的整系数多项式之积]{定理\ref{theorem:整系数多项式在有理数域上可约,则一定可以分解成两个次数较低的整系数多项式之积}}即得.
\end{proof}

\begin{example}
$f(x)$ 是次数大于零的首一整系数多项式,若 $f(0), f(1)$ 都是奇数,求证:$f(x)$ 没有有理根. 
\end{example}
\begin{proof}
若 $c$ 是偶数,则上述左边为奇数,不可能等于零。若 $c$ 是奇数,令 $c = 2b + 1$,其中 $b$ 是整数,可得
\begin{align*}
(2b + 1)^n + a_{n-1} (2b + 1)^{n-1} + \cdots + a_1 (2b + 1) + a_0 = 0.
\end{align*}
用二项式定理展开后将看到,上式左边是一个偶数加上 $1 + a_{n-1} + \cdots + a_1 + a_0$,故必是奇数,也不可能等于零。因此 $f(x)$ 没有有理根.
\end{proof}

\begin{proposition}\label{proposition:proposition:有理系数多项式的判定条件}
设 $f(x)$ 是实系数多项式,若对任意的有理数 $c$,$f(c)$ 总是有理数,求证:$f(x)$ 是有理系数多项式。
\end{proposition}
\begin{remark}
证明与\hyperref[proposition:实系数多项式的判定条件]{命题\ref{proposition:实系数多项式的判定条件}}
\end{remark}
\begin{proof}
设 $f(x) = a_n x^n + a_{n-1} x^{n-1} + \cdots + a_1 x + a_0$,分别令 $x = 0, 1, 2, \cdots, n$,得到一个以 $a_n, a_{n-1}, \cdots, a_1, a_0$ 为未知数,由 $n + 1$ 个方程式组成的实系数线性方程组。该方程组的系数行列式是一个非零的 Vandermonde 行列式,故方程组必有唯一解,且解为有理数。因此 $f(x)$ 是有理系数多项式.
\end{proof}

\begin{example}
设 $f(x)$ 是有理系数多项式,$a, b, c$ 是有理数,但 $\sqrt{c}$ 是无理数。求证:若 $a + b\sqrt{c}$ 是 $f(x)$ 的根,则 $a - b\sqrt{c}$ 也是 $f(x)$ 的根。
\end{example}
\begin{proof}
设 $f(x) = a_n x^n + a_{n-1} x^{n-1} + \cdots + a_1 x + a_0$,则
\begin{align*}
f(a + b\sqrt{c}) = a_n (a + b\sqrt{c})^n + a_{n-1} (a + b\sqrt{c})^{n-1} + \cdots + a_1 (a + b\sqrt{c}) + a_0 = 0。
\end{align*}
将 $(a + b\sqrt{c})^k$ 用二项式定理展开,可设
\begin{align*}
f(a + b\sqrt{c}) = A + B\sqrt{c} = 0,
\end{align*}
其中 $A, B$ 都是有理数。因为 $\sqrt{c}$ 是无理数,故 $A = B = 0$。因此
\begin{align*}
f(a - b\sqrt{c}) = A - B\sqrt{c} = 0,
\end{align*}
即 $a - b\sqrt{c}$ 也是 $f(x)$ 的根。
\end{proof}

\begin{example}
设 $f(x)$ 是有理系数多项式,$a, b, c, d$ 是有理数,但 $\sqrt{c}, \sqrt{d}, \sqrt{cd}$ 都是无理数。求证:若 $a\sqrt{c} + b\sqrt{d}$ 是 $f(x)$ 的根,则下列数也是 $f(x)$ 的根:
\begin{align*}
a\sqrt{c} - b\sqrt{d}, -a\sqrt{c} + b\sqrt{d}, -a\sqrt{c} - b\sqrt{d}.
\end{align*}
\end{example}
\begin{proof}
令
\begin{align*}
g(x) = (x - (a\sqrt{c} + b\sqrt{d}))(x - (a\sqrt{c} - b\sqrt{d}))(x - (-a\sqrt{c} + b\sqrt{d}))(x - (-a\sqrt{c} - b\sqrt{d})),
\end{align*}
则经计算可得
\begin{align*}
g(x) = x^4 - 2(a^2 c + b^2 d)x^2 + (a^2 c - b^2 d)^2.
\end{align*}
注意到 $g(x)$ 是一个有理数首一多项式,只要证明它不可约,便可由\hyperref[proposition:极小多项式式的充要条件]{极小多项式式的充要条件}得到 $g(x)$ 是 $a\sqrt{c} + b\sqrt{d}$ 的极小多项式,从而由\hyperref[proposition:极小多项式的基本性质]{极小多项式的基本性质}可知$g(x) \mid f(x)$,于是结论成立。显然 $g(x)$ 没有有理系数的一次因式,只要证明它没有有理系数的二次因式即可。经过简单的计算可知,在$g(x)$的一个一次因式中任取一个一次因式相乘都不是有理系数多项式,因此 $g(x)$ 没有有理系数的二次因式.
\end{proof}

\begin{example}
求以 $\sqrt{2} + \sqrt[3]{3}$ 为根的次数最小的首一有理系数多项式。
\end{example}
\begin{remark}
\hypertarget{f(x)的6个根的找法}{\textbf{确定f(x)的6个根的方法:}}原方程 $x - \sqrt{2} = \sqrt[3]{3}$ 的解为 $x = \sqrt{2} + \sqrt[3]{3}$。但三次方程 $y^3 = 3$ 的所有根为 $y = \sqrt[3]{3}, \sqrt[3]{3\omega}, \sqrt[3]{3\omega^2}$(其中 $\omega = -\frac{1}{2} + \frac{\sqrt{3}}{2}i$ 是三次单位根),因此原方程对应三个解:

\begin{align*}
x &= \sqrt{2} + \sqrt[3]{3}, \quad \sqrt{2} + \sqrt[3]{3\omega}, \quad \sqrt{2} + \sqrt[3]{3\omega^2}.
\end{align*}
在消去 $\sqrt{2}$ 的平方步骤中,方程 $x^3 + 6x - 3 = (3x^2 + 2)\sqrt{2}$ 的两边平方后,原方程中的 $\sqrt{2}$可以被替换为 $-\sqrt{2}$,从而产生另一组解:
\begin{align*}
x &= -\sqrt{2} + \sqrt[3]{3}, \quad -\sqrt{2} + \sqrt[3]{3\omega}, \quad -\sqrt{2} + \sqrt[3]{3\omega^2}.
\end{align*}
\end{remark}
\begin{solution}
本题即求 $\sqrt{2} + \sqrt[3]{3}$ 的极小多项式。令 $x - \sqrt{2} = \sqrt[3]{3}$,两边立方得到 $(x - \sqrt{2})^3 = 3$。整理可得 $x^3 + 6x - 3 = (3x^2 + 2) \sqrt{2}$,再两边平方可得,$\sqrt{2} + \sqrt[3]{3}$ 适合下列多项式:
\begin{align*}
f(x) = x^6 - 6x^4 - 6x^3 + 12x^2 - 36x + 1.
\end{align*}
由 $f(x)$ 的构造过程,\hyperlink{f(x)的6个根的找法}{不难看出} $f(x)$ 的 6 个根分别为 $\pm \sqrt{2} + \sqrt[3]{3}$,$\pm \sqrt{2} + \sqrt[3]{3}\omega$,$\pm \sqrt{2} + \sqrt[3]{3}\omega^2$。其中 $\omega = -\frac{1}{2} + \frac{\sqrt{3}}{2}i$。因此,我们有
\begin{align*}
f\left( x \right) =\left( x-\sqrt{2}-\sqrt[3]{3} \right) \left( x+\sqrt{2}-\sqrt[3]{3} \right) \left( x-\sqrt{2}-\sqrt[3]{3}\omega \right) \left( x+\sqrt{2}-\sqrt[3]{3}\omega \right) \left( x-\sqrt{2}-\sqrt[3]{3}\omega ^2 \right) \left( x+\sqrt{2}-\sqrt[3]{3}\omega ^2 \right) .
\end{align*}
通过简单的验证可知,任取 $f(x)$ 的 2 个一次因式相乘都不是有理系数多项式;任取 $f(x)$ 的 3 个一次因式相乘也都不是有理系数多项式,因此 $f(x)$ 是有理数域上的不可约多项式,从而由\hyperref[proposition:极小多项式式的充要条件]{极小多项式式的充要条件}可知,$f(x)$是 $\sqrt{2} + \sqrt[3]{3}$ 的极小多项式。
\end{solution}

\begin{example}
求证:有理系数多项式 $x^4 + px^2 + q$ 在有理数域上可约的充要条件是或者 $p^2 - 4q = k^2$,其中 $k$ 是一个有理数;或者 $q$ 是某个有理数的平方,且 $\pm 2\sqrt{q} - p$ 也是有理数的平方。
\end{example}
\begin{proof}
必要性:若多项式 $x^4 + px^2 + q$ 在有理数域上可约,考虑下列两种情况:

(1) $x^4 + px^2 + q$ 有有理数根 $t$,这时 $t^2$ 是 $x^2 + px + q$ 的有理根,因此其判别式 $p^2 - 4q$ 必是一个有理数的完全平方。

(2) $x^4 + px^2 + q$ 无有理数根,则$x^4 + px^2 + q$ 在有理数域上可分解为两个二次多项式的积。设 $x^4 + px^2 + q = (x^2 + ax + b)(x^2 + cx + d)$,展开后比较系数可得
\begin{align*}
\begin{cases}
a + c = 0, \\
ad + bc = 0.
\end{cases}
\end{align*}

若 $a = 0$,则 $c = 0$,这时将有 $p = b + d, \ q = bd$,因此 $p^2 - 4q = (b - d)^2$。若 $a \neq 0$,则 $b = d$,比较系数后可知 $p = 2b - a^2, \ q = b^2$,因此 $\pm 2\sqrt{q} - p = a^2$。

充分性:若 $p^2 - 4q = k^2$,则

\begin{align*}
x^4 + px^2 + q &= x^4 + px^2 + \frac{1}{4}(p + k)(p - k) = \left(x^2 + \frac{1}{2}(p + k)\right)\left(x^2 + \frac{1}{2}(p - k)\right).
\end{align*}
因此多项式可约。

若 $q = b^2, \ \pm 2\sqrt{q} - p = \pm 2b - p = a^2$,则 $p = -a^2 \pm 2b$。于是
\begin{align*}
x^4 + px^2 + q &= x^4 + (-a^2 \pm 2b)x^2 + b^2 = (x^2 \pm b)^2 - a^2x^2
\end{align*}
也可约。
\end{proof}


\begin{theorem}[Eisenstein 判别法]\label{theorem:Eisenstein 判别法}
设多项式
\begin{align*}
f(x) = a_n x^n + a_{n-1} x^{n-1} + \cdots + a_1 x + a_0
\end{align*}
是整系数多项式,$a_n \neq 0, n \geq 1, p$ 是一个素数。若 $p \mid a_i (i = 0, 1, \cdots, n-1)$,但 $p \nmid a_n$ 且 $p^2 \nmid a_0$,则 $f(x)$ 在有理数域上不可约。    
\end{theorem}
\begin{proof}
只需证明 $f(x)$ 在整数环上不可约即可。设 $f(x)$ 可分解为两个次数较低的整系数多项式之积:
\begin{align*}
f(x) = (b_m x^m + b_{m-1} x^{m-1} + \cdots + b_0)(c_t x^t + c_{t-1} x^{t-1} + \cdots + c_0),
\end{align*}
其中 $m + t = n$。显然 $a_0 = b_0 c_0, a_n = b_m c_t$。由假设 $p \mid a_0$,故 $p \mid b_0$ 或 $p \mid c_0$。又 $p^2 \nmid a_0$,故 $p$ 不能同时整除 $b_0$ 及 $c_0$。不妨设 $p \mid b_0$ 但 $p \nmid c_0$。又由假设,$p$ 不能整除 $a_n = b_m c_t$,故 $p$ 既不能整除 $b_m$ 又不能整除 $c_t$。因此不妨设 $p \mid b_0, p \mid b_1, \cdots, p \mid b_{j-1}$ 但 $p$ 不能整除 $b_j$,其中 $0 < j \leq m < n$。而
\begin{align*}
a_j = b_j c_0 + b_{j-1} c_1 + \cdots + b_0 c_j,
\end{align*}
根据假设,$p \mid a_j$,又 $p$ 可整除上述右端除 $b_j c_0$ 外的其余项,而不能整除 $b_j c_0$ 这一项,引出矛盾.
\end{proof}

\begin{example}
设 $p_1, \cdots, p_m$ 是 $m$ 个互不相同的素数,求证:对任意的 $n \geq 1$,下列多项式在有理数域上不可约:
\begin{align*}
f(x) = x^n - p_1 \cdots p_m.
\end{align*} 
\end{example}
\begin{proof}
用 Eisenstein 判别法即可证明.(取$p=p_i$即可)
\end{proof}

\begin{example}
证明:$x^8 + 1$ 在有理数域上不可约。
\end{example}
\begin{proof}
作代换 $x = y + 1$,得

\begin{align*}
x^8 + 1 &= (y + 1)^8 + 1 = y^8 + 8y^7 + 28y^6 + 56y^5 + 70y^4 + 56y^3 + 28y^2 + 8y + 2.
\end{align*}

显然 2 可整除除第一项外的所有系数,但 4 不能整除常数项。用 Eisenstein 判别法可知 $(y + 1)^8 + 1$ 不可约,故 $x^8 + 1$ 也不可约。
\end{proof}

\begin{example}
设 $f(x)$ 是有理系数多项式,已知 $\sqrt{2}$ 是 $f(x)$ 的根,证明:$\sqrt{2} \varepsilon, \sqrt{2} \varepsilon^2, \cdots, \sqrt{2} \varepsilon^{n-1}$ 也是 $f(x)$ 的根,其中 $\varepsilon = \cos \frac{2\pi}{n} + i \sin \frac{2\pi}{n}$ 是 1 的 $n$ 次根。  
\end{example}
\begin{proof}
显然 $\sqrt{2}$ 适合多项式 $x^n - 2$,由 Eisenstein 判别法可知,$x^n - 2$ 在有理数域上不可约,因此它是 $\sqrt{2}$ 的极小多项式。最后由\hyperref[proposition:极小多项式的基本性质]{极小多项式的基本性质}可得 $(x^n - 2) \mid f(x)$,从而结论得证.
\end{proof}

\begin{example}
设 $f(x)$ 是次数大于 1 的奇数次有理系数不可约多项式,求证:若 $x_1, x_2$ 是 $f(x)$ 在复数域内两个不同的根,则 $x_1 + x_2$ 必不是有理数。
\end{example}
\begin{proof}
不妨设 $f(x)$ 为首一多项式,我们用反证法来证明结论。设 $x_1 + x_2 = r$ 为有理数,则有理系数多项式 $f(x)$ 与 $f(r - x)$ 有公共根 $x_1$。因为 $f(x)$ 在有理数域上不可约,故 $f(x)$ 是 $x_1$ 的极小多项式,从而由极小多项式的基本性质可得 $f(x) \mid f(r - x)$。注意到 $f(x)$ 与 $f(r - x)$ 次数相同,首项系数相同,从而有 $f(r - x) = -f(x)$。令 $x = \frac{r}{2}$,则可得 $f\left(\frac{r}{2}\right) = 0$,即 $\frac{r}{2}$ 是 $f(x)$ 的一个有理根,这与 $f(x)$ 在有理数域上不可约相矛盾。
\end{proof}

\begin{example}
设 $f(x) = (x - a_1)(x - a_2) \cdots (x - a_n) - 1$,其中 $a_1, a_2, \cdots, a_n$ 是 $n$ 个不同的整数,求证:$f(x)$ 在有理数域上不可约。
\end{example}
\begin{proof}
由\hyperref[proposition:整数环上不可约一定在有理数域上不可约]{命题\ref{proposition:整数环上不可约一定在有理数域上不可约}}可知,只要证明 $f(x)$ 在整数环上不可约即可。用反证法,设 $f(x) = g(x)h(x)$,其中 $g(x), h(x)$ 都是次数小于 $n$ 的首一整数系数多项式。注意到
\begin{align*}
g(a_i)h(a_i) = -1,
\end{align*}
因为 $g(x), h(x)$ 是整数系数多项式,故 $g(a_i) = 1, h(a_i) = -1$ 或 $g(a_i) = -1, h(a_i) = 1$。无论是哪种情况,都有
\begin{align*}
g(a_i) + h(a_i) = 0, \quad 1 \leq i \leq n,
\end{align*}
即次数小于 $n$ 的多项式 $g(x) + h(x)$ 有 $n$ 个不同的根,故 $g(x) + h(x) = 0$。因此 $f(x) = -g(x)^2$,但 $f(x)$ 是首一多项式,而 $-g(x)^2$ 的首项系数为 $-1$,矛盾.
\end{proof}

\begin{example}
设 $f(x) = (x - a_1)^2(x - a_2)^2 \cdots (x - a_n)^2 + 1$,其中 $a_1, a_2, \cdots, a_n$ 是 $n$ 个不同的整数,求证:$f(x)$ 在有理数域上不可约。
\end{example}
\begin{proof}
由\hyperref[proposition:整数环上不可约一定在有理数域上不可约]{命题\ref{proposition:整数环上不可约一定在有理数域上不可约}}可知,只要证明 $f(x)$ 在整数环上不可约即可。用反证法,设 $f(x) = u(x)v(x)$,其中 $u(x), v(x)$ 都是次数小于 $2n$ 的首一整数系数多项式。注意到 $f(x)$ 没有实根,故
$u(x), v(x)$
也都没有实根,从而由实系数多项式虚根成对可知,$u(x), v(x)$ 作为实数域上的函数都恒大于零。由于 $f(x)$ 是 $2n$ 次多项式,故 $u(x)$ 和 $v(x)$ 的次数至少有一个不超过 $n$,不妨设 $u(x)$ 的次数不超过 $n$。

若 $u(x)$ 的次数小于 $n$,则由 $f(a_i) = 1$ 可得 $u(a_i)v(a_i) = 1$,因此 $u(a_i) = 1$。考虑非零多项式 $u(x) - 1$,由上面的分析可知它有 $n$ 个不同的根 $a_1, a_2, \cdots, a_n$,这与它的次数小于 $n$ 矛盾。

因此 $u(x)$ 只能是 $n$ 次首一多项式,于是 $v(x)$ 也是 $n$ 次首一多项式。另一方面,由于 $u(a_i)v(a_i) = 1$,故 $u(a_i) = v(a_i) = \pm 1 (1 \leq i \leq n)$。注意到 $u(x) - v(x)$ 的次数小于 $n$ 并且它有 $n$ 个不同的根 $a_1, a_2, \cdots, a_n$,因此 $u(x) = v(x)$ 或 $u(x) = -v(x)$。今设 $u(x) = v(x)$,则 $f(x) = u(x)^2 + 1$,即
\begin{align*}
(u(x) + h(x))(u(x) - h(x)) = 1.
\end{align*}
因为 $u(x), h(x)$ 都是整数系数多项式,故或者 $u(x) + h(x) = 1, u(x) - h(x) = 1$;或者 $u(x) + h(x) = -1, u(x) - h(x) = -1$,于是作差可得$h(x) = 0$,矛盾。因此结论得证。
\end{proof}



\section{多元多项式}

因式分解定理对多元多项式仍成立.(证明见抽象代数内容)

\begin{lemma}\label{lemma:多元多项式乘积的首项}
若 $f(x_1, x_2, \cdots, x_n)$ 及 $g(x_1, x_2, \cdots, x_n)$ 都是 $K$ 上非零的 $n$ 元多项式,则按字典排列法排列后乘积的首项等于 $f$ 的首项与 $g$ 的首项之积。
\end{lemma}
\begin{proof}
设 $ax_1^{i_1} x_2^{i_2} \cdots x_n^{i_n}$ 和 $bx_1^{j_1} x_2^{j_2} \cdots x_n^{j_n}$ 分别是 $f$ 和 $g$ 的首项(按字典排列法),它们的乘积为 $abx_1^{i_1 + j_1} x_2^{i_2 + j_2} \cdots x_n^{i_n + j_n}$。其他任意两个单项式 $cx_1^{k_1} x_2^{k_2} \cdots x_n^{k_n}$ 和 $dx_1^{r_1} x_2^{r_2} \cdots x_n^{r_n}$ 之积为 $cdx_1^{k_1 + r_1} x_2^{k_2 + r_2} \cdots x_n^{k_n + r_n}$。设 $i_1 = k_1, \cdots, i_{t-1} = k_{t-1}, i_t > k_t; j_1 = r_1, \cdots, j_{s-1} = r_{s-1}, j_s > r_s$。不妨设 $t \leq s$,显然

\begin{align*}
i_1 + j_1 = k_1 + r_1, \cdots, i_{t-1} + j_{t-1} = k_{t-1} + r_{t-1}, i_t + j_t > k_t + r_t.
\end{align*}

因此 $abx_1^{i_1 + j_1} x_2^{i_2 + j_2} \cdots x_n^{i_n + j_n}$ 先于 $cdx_1^{k_1 + r_1} x_2^{k_2 + r_2} \cdots x_n^{k_n + r_n}$。

同理可证明:$abx_1^{i_1 + j_1} x_2^{i_2 + j_2} \cdots x_n^{i_n + j_n}$ 先于 $adx_1^{i_1 + r_1} x_2^{i_2 + r_2} \cdots x_n^{i_n + r_n}$ 和 $cbx_1^{k_1 + j_1} x_2^{k_2 + j_2} \cdots x_n^{k_n + j_n}$。因此它确是 $fg$ 的首项。
\end{proof}

\begin{proposition}[多元多项式的整性]\label{proposition:多元多项式的整性}
若 $f(x_1, x_2, \cdots, x_n) \neq 0, g(x_1, x_2, \cdots, x_n) \neq 0$,则
\begin{align*}
f(x_1, x_2, \cdots, x_n)g(x_1, x_2, \cdots, x_n) \neq 0.
\end{align*}
\end{proposition}
\begin{proof}
由 $f$ 和 $g$ 的首项不为零及\hyperref[lemma:多元多项式乘积的首项]{引理\ref{lemma:多元多项式乘积的首项}}可知 $fg$ 的首项不为零,于是 $fg \neq 0$。
\end{proof}

\begin{corollary}
若 $h(x_1, x_2, \cdots, x_n) \neq 0$,且
\begin{align*}
f(x_1, x_2, \cdots, x_n)h(x_1, x_2, \cdots, x_n) &= g(x_1, x_2, \cdots, x_n)h(x_1, x_2, \cdots, x_n),
\end{align*}
则
\begin{align*}
f(x_1, x_2, \cdots, x_n) &= g(x_1, x_2, \cdots, x_n).
\end{align*}
\end{corollary}
\begin{proof}
由条件可得
\begin{align*}
\left[ f(x_1,x_2,\cdots ,x_n)-g(x_1,x_2,\cdots ,x_n) \right] h(x_1,x_2,\cdots ,x_n)=0.
\end{align*}
又因为$h(x_1,x_2,\cdots ,x_n)\ne 0,所以若 f(x_1,x_2,\cdots ,x_n)-g(x_1,x_2,\cdots ,x_n)\ne 0$,则由命题 5.25 可知
\begin{align*}
\left[ f(x_1,x_2,\cdots ,x_n)-g(x_1,x_2,\cdots ,x_n) \right] h(x_1,x_2,\cdots ,x_n)\ne 0
\end{align*}
矛盾!故$f(x_1,x_2,\cdots ,x_n)-g(x_1,x_2,\cdots ,x_n)=0$,即\[f(x_1,x_2,\cdots ,x_n)=g(x_1,x_2,\cdots ,x_n).\]
\end{proof}



\begin{definition}[齐次多项式]
若一个多项式 \(f(x_1,x_2,\cdots ,x_n)\) 的每个单项式都是 \(k\) 次式,则\textbf{称之为 \(\boldsymbol{k}\) 次齐次多项式或 \(\boldsymbol{k}\) 次型}.
\end{definition}

\begin{proposition}[齐次多项式的基本性质]
\begin{enumerate}[(1)]
\item 两个次数相同的齐次多项式之和若不为零,则必仍是同次齐次多项式。任意两个齐次多项式之积仍为齐次多项式。

\item 任一 \(n\) 元多项式均可表示为若干个齐次多项式之和,
\end{enumerate}
\end{proposition}
\begin{proof}
\begin{enumerate}[(1)]
\item 显然.

\item 这只需要将各次数相等的项放在一起即可。
\end{enumerate}
\end{proof}

\begin{lemma}[多元多项式的非零性]\label{lemma:多元多项式的非零性}
设 \(f(x_1,x_2,\cdots ,x_n)\) 是 \(K\) 上非零的 \(n\) 元多项式,则必存在 \(K\) 中的数 \(a_1,a_2,\cdots ,a_n\),使 \(f(a_1,a_2,\cdots ,a_n) \ne 0\)。
\end{lemma}
\begin{proof}
对未定元个数 \(n\) 用数学归纳法。当 \(n=1\) 时,多项式 \(f(x)\) 只有有限个零点,故总有 \(a \in K\) 使 \(f(a) \ne 0\)。现设对有 \(n-1\) 个未定元的多项式结论成立,将 \(f(x_1,x_2,\cdots ,x_n)\) 写成未定元 \(x_n\) 的多项式:
\begin{align*}
f(x_1,x_2,\cdots ,x_n) = b_0 + b_1 x_n + \cdots + b_m x_n^m,
\end{align*}
其中 \(b_i = b_i(x_1,x_2,\cdots ,x_{n-1})\) 是 \(n-1\) 元多项式。因为 \(f(x_1,x_2,\cdots ,x_n) \ne 0\),故可设 \(b_m \ne 0\)。由归纳假设,存在 \(a_1,\cdots ,a_{n-1} \in K\),使
\begin{align*}
b_m(a_1,\cdots ,a_{n-1}) \ne 0.
\end{align*}
因而
\begin{align*}
f(a_1,\cdots ,a_{n-1},x_n) = b_0(a_1,\cdots ,a_{n-1}) + b_1(a_1,\cdots ,a_{n-1}) x_n + \cdots + b_m(a_1,\cdots ,a_{n-1}) x_n^m
\end{align*}
是一个非零的以 \(x_n\) 为未定元的一元多项式,故存在 \(a_n \in K\),使
\begin{align*}
f(a_1,a_2,\cdots ,a_n) \ne 0.
\end{align*}
\end{proof}

\begin{proposition}[多元多项式相等的充要条件]\label{proposition:多元多项式相等的充要条件}
数域 \(K\) 上的两个 \(n\) 元多项式 \(f(x_1,x_2,\cdots ,x_n)\) 与 \(g(x_1,x_2,\cdots ,x_n)\) 相等的充分必要条件是对一切 \(a_1,a_2,\cdots ,a_n \in K\),均有
\begin{align*}
f(a_1,a_2,\cdots ,a_n) = g(a_1,a_2,\cdots ,a_n).
\end{align*}
\end{proposition}
\begin{proof}
只需证明充分性。作
\begin{align*}
h(x_1,x_2,\cdots ,x_n) = f(x_1,x_2,\cdots ,x_n) - g(x_1,x_2,\cdots ,x_n).
\end{align*}
若 \(h(x_1,x_2,\cdots ,x_n) \ne 0\),则由\hyperref[lemma:多元多项式的非零性]{多元多项式的非零性非零}可知必有 \(a_1,a_2,\cdots ,a_n \in K\),使
\begin{align*}
h(a_1,a_2,\cdots ,a_n) \ne 0,
\end{align*}
这与假设矛盾.
\end{proof}

\begin{example}
设 \(f(x_1,\cdots ,x_n), g(x_1,\cdots ,x_n) \ne 0\) 是 \(K\) 上的多元多项式。假设对一切使 \(g(a_1,\cdots ,a_n) \ne 0\) 的 \(a_1,\cdots ,a_n \in K\),均有 \(f(a_1,\cdots ,a_n) = 0\),求证:
\begin{align*}
f(x_1,\cdots ,x_n) = 0.
\end{align*}
\end{example}
\begin{proof}
用反证法,假设 \(f(x_1,\cdots ,x_n) \ne 0\),则由\hyperref[proposition:多元多项式的整性]{多元多项式的整性}可知
\begin{align*}
h(x_1,\cdots ,x_n) = f(x_1,\cdots ,x_n) g(x_1,\cdots ,x_n) \ne 0,
\end{align*}
于是由\hyperref[lemma:多元多项式的非零性]{元多项式的非零性}存在 \(a_1,\cdots ,a_n \in K\),使得 \(h(a_1,\cdots ,a_n) \ne 0\),从而\(f(a_1,\cdots ,a_n) \ne 0\) 并且 \(g(a_1,\cdots ,a_n) \ne 0\),这与条件矛盾.
\end{proof}

\begin{proposition}
设 $A(x_1, x_2, \cdots, x_m) = (a_{ij})$ 为 $n$ 阶方阵,其元素 $a_{ij} = a_{ij}(x_1, x_2, \cdots, x_m)$ 都是 $\mathbb{K}$ 上的多元多项式。设 $g(x_1, x_2, \cdots, x_m), h_i(x_1, x_2, \cdots, x_m) \neq 0 \ (1 \leq i \leq k)$都是 $\mathbb{K}$ 上的多元多项式,
\[
U = \left\{ (a_1, a_2, \cdots, a_m) \in \mathbb{K}^m \mid h_i(a_1, a_2, \cdots, a_m) \neq 0 \ (1 \leq i \leq k) \right\}.
\]
若对所有的 $(a_1, a_2, \cdots, a_m) \in U$,都成立
\[
|A(a_1, a_2, \cdots, a_m)| = g(a_1, a_2, \cdots, a_m),
\]
证明:$|A(x_1, x_2, \cdots, x_m)| = g(x_1, x_2, \cdots, x_m)$。
\end{proposition}
\begin{note}
这个命题告诉我们:在元素为多元多项式的文字行列式的求值过程中,在假设某些非零条件下成立的情形下得到的结果,其实就是所求行列式的值。因此,在求行列式的过程中,\textbf{可以暂不考虑未定元取特殊值的情形,而把主要精力放在一般的情形进行计算即可}.
\end{note}
\begin{proof}
用反证法,设 $(|A| - g)(x_1, x_2, \cdots, x_m) \neq 0$,则由\hyperref[lemma:多元多项式的整性]{多元多项式的整性}可知
\[
(|A| - g)h_1 \cdots h_k (x_1, x_2, \cdots, x_m) \neq 0,
\]
于是存在 $a_1, a_2, \cdots, a_m \in \mathbb{K}$,使得 $(|A| - g)h_1 \cdots h_k (a_1, a_2, \cdots, a_m) \neq 0$,从而 $h_i(a_1, a_2, \cdots, a_m) \neq 0 \ (1 \leq i \leq k)$,即 $(a_1, a_2, \cdots, a_m) \in U$,并且 $(|A| - g)(a_1, a_2, \cdots, a_m) \neq 0$,即 $|A(a_1, a_2, \cdots, a_m)| \neq g(a_1, a_2, \cdots, a_m)$,这与条件矛盾。
\end{proof}

\begin{theorem}[行列式的求根法]\label{theorem:行列式的求根法}
设 $A(x_1, x_2, \cdots, x_m) = (a_{ij})$ 为 $n$ 阶方阵,其元素 $a_{ij} = a_{ij}(x_1, \cdots, x_m)$ 都是 $\mathbb{K}$ 上的多元多项式,于是 $|A|$ 也是 $\mathbb{K}$ 上的多元多项式。若把 $x_1$ 看成未定元,则可将 $|A|$ 整理成关于 $x_1$ 的一元多项式:
\begin{align}
|A| &= c_0(x_2, \cdots, x_m)x_1^d + c_1(x_2, \cdots, x_m)x_1^{d-1} + \cdots + c_d(x_2, \cdots, x_m),\label{equation5.1-theorem546}
\end{align}
其中 $c_0(x_2, \cdots, x_m) \neq 0, d \geq 1$ 为次数。假设存在互异的多项式 $g_1(x_2, \cdots, x_m), \cdots, g_d(x_2, \cdots, x_m)$,使得当 $x_1 = g_i(x_2, \cdots, x_m) (1 \leq i \leq d)$ 时 $|A| = 0$,证明:
\begin{align*}
|A| &= c_0(x_2, \cdots, x_m)(x_1 - g_1(x_2, \cdots, x_m)) \cdots (x_1 - g_d(x_2, \cdots, x_m))。
\end{align*}
\end{theorem}
\begin{proof}
由假设 $0 = c_0(x_2, \cdots, x_m)g_1^d + \cdots + c_{d-1}(x_2, \cdots, x_m)g_1 + c_d(x_2, \cdots, x_m)$,故有
\begin{align*}
|A| &= c_0(x_2, \cdots, x_m)(x_1^d - g_1^d) + \cdots + c_{d-1}(x_2, \cdots, x_m)(x_1 - g_1)R_1(x_1, x_2, \cdots, x_m)。
\end{align*}
在上述式中令 $x_1 = g_2(x_2, \cdots, x_m)$,则有 $0 = (g_2 - g_1)R_1(g_2, x_2, \cdots, x_m)$。注意到 $g_2 - g_1 \neq 0$,故由多元多项式的整性可得 $R_1(g_2, x_2, \cdots, x_m) = 0$。再由相同的讨论可知,$x_1 - g_2$ 是 $R_1(x_1, x_2, \cdots, x_m)$ 的因式,从而
\begin{align*}
|A| &= (x_1 - g_1)(x_1 - g_2)R_2(x_1, x_2, \cdots, x_m)。
\end{align*}
不断地这样做下去,可得
\begin{align*}
|A| &= (x_1 - g_1) \cdots (x_1 - g_d)R_d(x_1, x_2, \cdots, x_m)。
\end{align*}
最后与\eqref{equation5.1-theorem546}式比较 $x_1$ 的首项系数可得 $R_d(x_1, x_2, \cdots, x_m) = c_0(x_2, \cdots, x_m)$.
\end{proof}


\section{互素多项式的应用}

\begin{proposition}
设 $f(x), g(x)$ 是数域 $\mathbb{K}$ 上的互素多项式,$A$ 是 $\mathbb{K}$ 上的 $n$ 阶方阵,满足 $f(A) = O$,证明:$g(A)$ 是可逆矩阵。
\end{proposition}
\begin{proof}
根据假设,存在 $\mathbb{K}$ 上的多项式 $u(x), v(x)$,使得
\begin{align*}
f(x)u(x) + g(x)v(x) = 1。
\end{align*}
在上述式中代入 $x = A$,可得恒等式
\begin{align*}
f(A)u(A) + g(A)v(A) = I_n。
\end{align*}
因为 $f(A) = O$,故有 $g(A)v(A) = I_n$,从而 $g(A)$ 是非零矩阵且 $g(A)^{-1} = v(A)$.
\end{proof}

\begin{proposition}
设 $f(x), g(x)$ 是数域 $\mathbb{K}$ 上的互素多项式,$A$ 是 $\mathbb{K}$ 上的 $n$ 阶方阵,证明:$f(A)g(A) = O$ 的充要条件是 $r(f(A)) + r(g(A)) = n$。
\end{proposition}
\begin{proof}
根据假设,存在 $\mathbb{K}$ 上的多项式 $u(x), v(x)$,使得
\begin{align*}
f(x)u(x) + g(x)v(x) = 1。
\end{align*}
在上述式中代入 $x = A$,可得恒等式
\begin{align*}
f(A)u(A) + g(A)v(A) = I_n。
\end{align*}
考虑如下分块矩阵的初等变换:
\begin{align*}
\begin{pmatrix}
f(A) & O \\
O & g(A)
\end{pmatrix}
&\to
\begin{pmatrix}
f(A) & f(A)u(A) \\
O & g(A)
\end{pmatrix}
\to
\begin{pmatrix}
f(A) & I_n \\
O & g(A)
\end{pmatrix}
\to
\begin{pmatrix}
f(A) & I_n \\
-f(A)g(A) & O
\end{pmatrix}
\to
\begin{pmatrix}
O & I_n \\
-f(A)g(A) & O
\end{pmatrix},
\end{align*}
故有 $r(f(A)) + r(g(A)) = r(f(A)g(A)) + n$,从而结论得证。
\end{proof}

\begin{proposition}
设 $f(x), g(x)$ 是数域 $\mathbb{K}$ 上的互素多项式,$\varphi$ 是 $\mathbb{K}$ 上 $n$ 维线性空间 $V$ 上的线性变换,满足 $f(\varphi)g(\varphi) = 0$,证明:$V = V_1 \oplus V_2$,其中 $V_1 = \text{Ker} f(\varphi), V_2 = \text{Ker} g(\varphi)$。
\end{proposition}
\begin{note}
这个命题告诉我们:多项式的互素因式分解可以诱导出空间的直和分解,从几何层面上看,这就是相似标准型理论原始的除法点.
\end{note}
\begin{proof}
根据假设,存在 $\mathbb{K}$ 上的多项式 $u(x), v(x)$,使得
\begin{align*}
f(x)u(x) + g(x)v(x) = 1。
\end{align*}
在上述式中代入 $x = \varphi$,可得恒等式
\begin{align*}
f(\varphi)u(\varphi) + g(\varphi)v(\varphi) = I_V。
\end{align*}
对任意的 $\alpha \in V$,由上述可得
\begin{align*}
\alpha = f(\varphi)u(\varphi)(\alpha) + g(\varphi)v(\varphi)(\alpha),
\end{align*}
注意到 
\begin{align*}
g\left( \varphi \right) \left( f\left( \varphi \right) u\left( \varphi \right) \left( \alpha \right) \right) =g\left( \varphi \right) f\left( \varphi \right) u\left( \varphi \right) \left( \alpha \right) =u\left( \varphi \right) f\left( \varphi \right) g\left( \varphi \right) \left( \alpha \right) =0,
\\
f\left( \varphi \right) \left( g\left( \varphi \right) u\left( \varphi \right) \left( \alpha \right) \right) =f\left( \varphi \right) g\left( \varphi \right) u\left( \varphi \right) \left( \alpha \right) =u\left( \varphi \right) f\left( \varphi \right) g\left( \varphi \right) \left( \alpha \right) =0.
\end{align*}
于是$f(\varphi)u(\varphi)(\alpha) \in \text{Ker} g(\varphi), g(\varphi)v(\varphi)(\alpha) \in \text{Ker} f(\varphi)$,故有 $V = V_1 + V_2$。任取 $\beta \in V_1 \cap V_2$,由上述可得
\begin{align*}
\beta = u(\varphi)f(\varphi)(\beta) + v(\varphi)g(\varphi)(\beta) = 0,
\end{align*}
故有 $V_1 \cap V_2 = 0$,因此 $V = V_1 \oplus V_2$。     
\end{proof}

\begin{example}
设 $\mathbb{Q}(\sqrt[n]{2}) = \{a_0 + a_1 \sqrt[n]{2} + a_2 \sqrt[4]{4} + \cdots + a_{n-1} \sqrt[n]{2^{n-1}} \mid a_i \in \mathbb{Q}, 0 \leq i \leq n - 1\}$,证明:$\mathbb{Q}(\sqrt[n]{2})$ 是一个数域,并求 $\mathbb{Q}(\sqrt[n]{2})$ 作为 $\mathbb{Q}$ 上线性空间的一组基。
\end{example}
\begin{proof}
设 $f(x) = x^n - 2$,由 Eisenstein 判别法可知 $f(x)$ 在 $\mathbb{Q}$ 上不可约,从而 $f(x)$ 是 $\sqrt[n]{2}$ 的极小多项式。我们先证明:$a_0 + a_1 \sqrt[n]{2} + \cdots + a_{n-1} \sqrt[n]{2^{n-1}} = 0$ 的充要条件是 $a_0 = a_1 = \cdots = a_{n-1} = 0$。充分性是显然的,现证必要性:令 $g(x) = a_0 + a_1 x + \cdots + a_{n-1} x^{n-1}$,则 $g(\sqrt[n]{2}) = 0$,由极小多项式的基本性质可得 $f(x) \mid g(x)$。因为 $g(x)$ 的次数小于 $n$,故只能是 $g(x) = 0$,即 $a_0 = a_1 = \cdots = a_{n-1} = 0$。

利用 $\sqrt[n]{2} = 2$ 容易验证,$\mathbb{Q}(\sqrt[n]{2})$ 中任意两个数的加法、减法和乘法都是封闭的。要证明 $\mathbb{Q}(\sqrt[n]{2})$ 是数域,只要证明除法或者取倒数封闭即可。任取 $\mathbb{Q}(\sqrt[n]{2})$ 中的非零数 $\alpha = a_0 + a_1 \sqrt[2]{2} + \cdots + a_{n-1} \sqrt[n]{2^{n-1}} \neq 0$,由上面的讨论可知 $a_0, a_1, \cdots, a_{n-1}$ 不全为零。令 $g(x) = a_0 + a_1 x + \cdots + a_{n-1} x^{n-1}$,则 $g(\sqrt[n]{2}) \neq 0$。因为 $f(x)$ 不可约且 $g(x) \neq 0$ 的次数小于 $n$,故 $f(x)$ 与 $g(x)$ 互素,由\hyperref[theorem:多项式互素的充要条件]{多项式互素的充要条件}可知,存在有理系数多项式 $u(x), v(x)$,使得
\begin{align*}
f(x)u(x) + g(x)v(x) = 1,
\end{align*}
在上述中代入 $x = \sqrt[n]{2}$,可得 $\sqrt[n]{2} v(\sqrt[n]{2}) = 1$,于是 $\alpha^{-1} = v(\sqrt[n]{2}) \in \mathbb{Q}(\sqrt[n]{2})$。因此,$\mathbb{Q}(\sqrt[n]{2})$ 是数域。

由 $\mathbb{Q}(\sqrt[n]{2})$ 的定义可知,$\mathbb{Q}(\sqrt[n]{2})$ 中任一元都是 $1, \sqrt[2]{2}, \cdots, \sqrt[n]{2^{n-1}}$ 的 $\mathbb{Q}$-线性组合;又由开始的讨论可知,$1, \sqrt[2]{2}, \cdots, \sqrt[n]{2^{n-1}}$ 是 $\mathbb{Q}$-线性无关的,因此它们构成了 $\mathbb{Q}(\sqrt[n]{2})$ 作为 $\mathbb{Q}$ 上线性空间的一组基。特别地,$\dim_{\mathbb{Q}} \mathbb{Q}(\sqrt[n]{2}) = n$。
\end{proof}

\begin{proposition}
设 $f(x) = x^n + a_1 x^{n-1} + \cdots + a_{n-1} x + a_n$ 是数域 $\mathbb{K}$ 上的不可约多项式,$\varphi$ 是 $\mathbb{K}$ 上 $n$ 维线性空间 $V$ 上的线性变换,$\alpha_1 \neq 0, \alpha_2, \cdots, \alpha_n$ 是 $V$ 中的向量,满足
\begin{align*}
\varphi (\alpha _1)=\alpha _2,\varphi (\alpha _2)=\alpha _3,\cdots ,\varphi (\alpha _{n-1})=\alpha _n,\varphi (\alpha _n)=-a_n\alpha _1-a_{n-1}\alpha _2-\cdots -a_1\alpha _n.
\end{align*}
证明:$\{\alpha_1, \alpha_2, \cdots, \alpha_n\}$ 是 $V$ 的一组基。
\end{proposition}
\begin{proof}
我们只要证明 $\alpha_1, \alpha_2, \cdots, \alpha_n$ 线性无关即可。用反证法,设存在不全为零的 $n$ 个数 $c_1, c_2, \cdots, c_n$,使得
\begin{align*}
c_1 \alpha_1 + c_2 \alpha_2 + \cdots + c_n \alpha_n = 0,
\end{align*}
则有
\begin{align*}
(c_1 I_V + c_2 \varphi + \cdots + c_n \varphi^{n-1})(\alpha_1) = 0。
\end{align*}
令 $g(x) = c_1 + c_2 x + \cdots + c_n x^{n-1}$,则 $g(x) \neq 0$ 且 $g(\varphi)(\alpha_1) = 0$。另一方面,由假设容易验证 $f(\varphi)(\alpha_1) = 0$。因为 $f(x)$ 不可约且 $g(x)$ 的次数小于 $n$,故 $f(x)$ 与 $g(x)$ 互素,从而存在 $\mathbb{K}$ 上的多项式 $u(x), v(x)$,使得
\begin{align*}
f(x)u(x) + g(x)v(x) = 1。
\end{align*}
在上述中代入 $x = \varphi$,可得恒等式
\begin{align*}
f(\varphi)u(\varphi) + g(\varphi)v(\varphi) = I_V。
\end{align*}
上式两边同时作用 $\alpha_1$ 可得
\begin{align*}
\alpha_1 = u(\varphi)f(\varphi)(\alpha_1) + v(\varphi)g(\varphi)(\alpha_1) = 0,
\end{align*}
这与条件$\alpha_1 \neq 0$ 矛盾,从而结论得证。
\end{proof}












\chapter{特征值}




















\end{document}

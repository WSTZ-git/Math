\documentclass[../../main.tex]{subfiles}
\graphicspath{{\subfix{../../image/}}} % 指定图片目录,后续可以直接使用图片文件名。

% 例如:
% \begin{figure}[H]
% \centering
% \includegraphics[scale=0.4]{图.png}
% \caption{}
% \label{figure:图}
% \end{figure}
% 注意:上述\label{}一定要放在\caption{}之后,否则引用图片序号会只会显示??.

\begin{document}

\section{Lebesgue积分和Riemann积分的关系}

\begin{definition}[Riemann积分相关定义]\label{definition:Riemann积分相关定义}
设 \( f(x) \) 是定义在 \( I = [a,b] \) 上的有界函数,\(\{ \Delta^{(n)} \}\) 是对 \([a,b]\) 所做的分划序列:

\[
\Delta^{(n)} : a = x_0^{(n)} < x_1^{(n)} < \cdots < x_{k_n}^{(n)} = b \quad (n = 1,2,\cdots),
\]

\[
|\Delta^{(n)}| = \max \{ x_i^{(n)} - x_{i - 1}^{(n)} : 1 \leqslant i \leqslant k_n \}, \quad \lim_{n \to \infty} |\Delta^{(n)}| = 0.
\]

对每个 \( i \) 以及 \( n \),若令

\[
M_i^{(n)} = \sup \{ f(x) : x_{i - 1}^{(n)} \leqslant x \leqslant x_i^{(n)} \},
\]

\[
m_i^{(n)} = \inf \{ f(x) : x_{i - 1}^{(n)} \leqslant x \leqslant x_i^{(n)} \},
\]

则关于 \( f(x) \) 的 Darboux 上、下积分,下述等式成立:

\[
\overline{\int_a^b} f(x) \, \mathrm{d}x = \lim_{n \to \infty} \sum_{i = 1}^{k_n} M_i^{(n)} (x_i^{(n)} - x_{i - 1}^{(n)}),
\]

\[
\underline{\int_a^b} f(x) \, \mathrm{d}x = \lim_{n \to \infty} \sum_{i = 1}^{k_n} m_i^{(n)} (x_i^{(n)} - x_{i - 1}^{(n)}).
\]
\end{definition}

\begin{lemma}\label{lemma:引理4.23}
设 \( f(x) \) 是定义在 \( I = [a,b] \) 上的有界函数,记 \( \omega(x) \) 是 \( f(x) \) 在 \( [a,b] \) 上的振幅(函数),则有
\begin{align*}
\int_I \omega(x) \, \mathrm{d}x &= \overline{\int_a^b} f(x) \, \mathrm{d}x - \underline{\int_a^b} f(x) \, \mathrm{d}x,
\end{align*}
其中左端是 \( \omega(x) \) 在 \( I \) 上的 Lebesgue 积分.
\end{lemma}
\begin{proof}
因为 \( f(x) \) 在 \( [a,b] \) 上是有界的,所以 \( \omega(x) \) 是 \( [a,b] \) 上的有界函数. 由\refpro{proposition:振幅函数可测}可知,\( \omega(x) \) 是 \( [a,b] \) 上的可测函数,因此 \( \omega \in L([a,b]) \).

对于\refdef{Riemann积分相关定义}所说的分划序列 \( \{\Delta^{(n)}\} \),作函数列
\[
\omega_{\Delta^{(n)}}(x) = 
\begin{cases} 
M_i^{(n)} - m_i^{(n)} &, x \in (x_{i-1}^{(n)}, x_i^{(n)}), \\
0 & ,x \text{ 是 } \Delta^{(n)} \text{ 的分点},
\end{cases}
\quad (i = 1,2,\cdots,k_n,\, n = 1,2,\cdots).
\]
\( E = \{ x \in [a,b] : x \text{ 是 } \Delta^{(n)} \, (n = 1,2,\cdots) \text{ 的分点} \} \).

显然 \( m(E) = 0 \),且有
\[
\lim_{n \to \infty} \omega_{\Delta^{(n)}}(x) = \omega(x), \quad x \in [a,b] \setminus E.
\]
现在记 \( A,B \) 各为 \( f(x) \) 在 \( [a,b] \) 上的上、下确界,由于对一切 \( n \),有 \( \omega_{\Delta^{(n)}}(x) \leq A - B \),故根据\hyperref[theorem:控制收敛定理]{控制收敛定理}(控制函数是常数函数)可知,
\[
\lim_{n \to \infty} \int_I \omega_{\Delta^{(n)}}(x) \, \mathrm{d}x = \int_I \omega(x) \, \mathrm{d}x.
\]

另一方面,因为
\begin{align*}
\int_I \omega_{\Delta^{(n)}}(x) \, \mathrm{d}x &= \sum_{i=1}^{k_n} (M_i^{(n)} - m_i^{(n)})(x_i^{(n)} - x_{i-1}^{(n)}) \\
&= \sum_{i=1}^{k_n} M_i^{(n)}(x_i^{(n)} - x_{i-1}^{(n)}) - \sum_{i=1}^{k_n} m_i^{(n)}(x_i^{(n)} - x_{i-1}^{(n)}),
\end{align*}
所以得到
\[
\int_I \omega(x) \, \mathrm{d}x = \lim_{n \to \infty} \int_I \omega_{\Delta^{(n)}}(x) \, \mathrm{d}x = \overline{\int_a^b} f(x) \, \mathrm{d}x - \underline{\int_a^b} f(x) \, \mathrm{d}x.
\]
\end{proof}

\begin{theorem}\label{theorem:定理4.24}
若 \( f(x) \) 是定义在 \( [a,b] \) 上的有界函数,则 \( f(x) \) 在 \( [a,b] \) 上 Riemann 可积的充分必要条件是 \( f(x) \) 在 \( [a,b] \) 上的不连续点集是零测集.
\end{theorem}
\begin{note}
上述定理指出,对于 \( [a,b] \) 上的有界函数而言,其 Riemann 可积性并非由该函数在不连续点处的性态所致,而是取决于它的不连续点集的测度.
\end{note}
\begin{proof}
必要性,若 \( f(x) \) 在 \( [a,b] \) 上是 Riemann 可积的,则 \( f(x) \) 的 Darboux 上、下积分相等,从而由\reflem{lemma:引理4.23}可知 \( \int_I \omega(x) \, \mathrm{d}x = 0 \). 因为 \( \omega(x) \geq 0 \),所以由\refcor{corollary:非负可测函数积分为0则几乎处处为0}可知\( \omega(x) = 0 \),\(\text{a. e.}\) \( x \in [a,b] \). 从而
\begin{align*}
\underset{\delta \rightarrow 0}{\lim}\underset{x\in B\left( x_0,\delta \right)}{\mathrm{sup}}|f(x)-f(x_0)|\leqslant \underset{\delta \rightarrow 0}{\lim}\underset{x',x'' \in B\left( x_0,\delta \right)}{\mathrm{sup}}|f\left( x' \right) -f\left( x'' \right) |=w(x_0)=0,\mathrm{a}.\mathrm{e}.\,x_0\in \left[ a,b \right] .
\end{align*}
这说明 \( f(x) \) 在 \( [a,b] \) 上是几乎处处连续的.

充分性,若 \( f(x) \) 在 \( [a,b] \) 上的不连续点集是零测集,则
\begin{align*}
&w(x_0)=\underset{\delta \rightarrow 0}{\lim}\underset{x‘,x'' \in B\left( x_0,\delta \right)}{\mathrm{sup}}|f\left( x‘\right) -f\left( x'' \right) |
\\
&\leqslant \underset{\delta \rightarrow 0}{\lim}\underset{x‘\in B\left( x_0,\delta \right)}{\mathrm{sup}}|f(x‘)-f(x_0)|+\underset{\delta \rightarrow 0}{\lim}\underset{x'' \in B\left( x_0,\delta \right)}{\mathrm{sup}}|f(x'' )-f(x_0)|
\\
&=0,\mathrm{a}.\mathrm{e}.\,x_0\in \left[ a,b \right] .
\end{align*}
因此\( f(x) \) 的振幅函数 \( \omega(x) \) 几乎处处等于零,从而由\reflem{lemma:引理4.23}可知
\[
\overline{\int_a^b} f(x) \, \mathrm{d}x - \underline{\int_a^b} f(x) \, \mathrm{d}x = \int_I \omega(x) \, \mathrm{d}x = 0,
\]
即 \( f(x) \) 的 Darboux 上、下积分相等,\( f(x) \) 在 \( [a,b] \) 上是 Riemann 可积的.
\end{proof}

\begin{theorem}\label{theorem:定理4.25}
若 \( f(x) \) 在 \( I = [a,b] \) 上是 Riemann 可积的,则 \( f(x) \) 在 \( [a,b] \) 上是 Lebesgue 可积的,且其积分值相同.
\end{theorem}
\begin{remark}
今后,为整合起见,对 \( f(x) \) 在 \( [a,b] \) 上的 Lebesgue 积分,也记为$\int_a^b f(x) \, \mathrm{d}x.$
\end{remark}
\begin{proof}
首先,根据题设以及\refthe{theorem:定理4.24},\( f(x) \) 在 \( [a,b] \) 上是几乎处处连续的. 因此 \( f(x) \) 是 \( [a,b] \) 上的有界可测函数,\( f \in L(I) \).

其次,对 \( [a,b] \) 的任一分划
\[
\Delta: a = x_0 < x_1 < \cdots < x_n = b,
\]
根据\hyperref[theorem:积分对定义域的可数可加性]{Lebesgue 积分对积分区域的可加性},我们有
\[
\int_I f(x) \, \mathrm{d}x = \sum_{i=1}^n \int_{[x_{i-1}, x_i]} f(x) \, \mathrm{d}x.
\]
记 \( M_i, m_i \) 分别为 \( f(x) \) 在 \( [x_{i-1}, x_i] \) 上的上、下确界,则得
\[
m_i (x_i - x_{i-1}) \leq \int_{[x_{i-1}, x_i]} f(x) \, \mathrm{d}x \leq M_i (x_i - x_{i-1})
\]
\( (i = 1,2,\cdots,n) \),从而可知
\[
\sum_{i=1}^n m_i (x_i - x_{i-1}) \leq \int_I f(x) \, \mathrm{d}x \leq \sum_{i=1}^n M_i (x_i - x_{i-1}).
\]
于是,在上式左、右两端对一切分划 \( \Delta \) 各取上、下确界,立即得到
\[
\int_I f(x) \, \mathrm{d}x = \overline{\int_a^b} f(x) \, \mathrm{d}x = \underline{\int_a^b} f(x) \, \mathrm{d}x.
\]
这说明 \( f(x) \) 在 \( [a,b] \) 上的 Lebesgue 积分与 Riemann 积分是相等的.
\end{proof}

\begin{proposition}
设 \( f(x) \) 是 \( \mathbb{R} \) 上的有界可测函数,且不恒为零. 若有
\[
f(x + y) = f(x) \cdot f(y) \quad (x, y \in \mathbb{R}),
\]
则 \( f(x) = \mathrm{e}^{\alpha x} (x \in \mathbb{R}) \).
\end{proposition}
\begin{proof}
由题设知 \( f(x) = f(x)f(0) \),故 \( f(0) = 1 \). 注意到 \( f(x) \not\equiv 0 (x \in \mathbb{R}) \),令 \( F(x) = \int_0^x f(t) \, \mathrm{d}t (x \in \mathbb{R}) \),且选 \( a \in \mathbb{R} \),使得 \( F(a) \neq 0 \),则有
\begin{align*}
F(x + a) - F(x) = \int_x^{x + a} f(t) \, \mathrm{d}t = \int_0^a f(x + t) \, \mathrm{d}t 
= \int_0^a f(x)f(t) \, \mathrm{d}t = f(x)F(a),
\end{align*}
\[
f(x) = \frac{F(x + a) - F(x)}{F(a)}.
\]
这说明 \( f(x) \) 是连续函数,因此 \( F \in C^{(1)}(\mathbb{R}) \),从而可得 \( f'(x + y) = f(x)f'(y) \). 取 \( y = 0 \),即得 \( f'(x) = f(x)f'(0) \). 记 \( \alpha = f'(0) \),可知 \( (f(x)\mathrm{e}^{-\alpha x})' \equiv 0 \),而 \( f(0) = 1 \),故又有 \( f(x)\mathrm{e}^{-\alpha x} \equiv 1 \),即得所证.
\end{proof}

\begin{theorem}
设 \( \{ E_k \} \) 是递增可测集列,其并集是 \( E \),又
\[
f \in L(E_k) \quad (k = 1, 2, \cdots).
\]
若极限 \( \lim_{k \to \infty} \int_{E_k} |f(x)| \, \mathrm{d}x \) 存在,则 \( f \in L(E) \),且有
\[
\int_E f(x) \, \mathrm{d}x = \lim_{k \to \infty} \int_{E_k} f(x) \, \mathrm{d}x.
\]
\end{theorem}
\begin{remark}
在上述定理中,特别当 \( E_k \) 是矩体 \( I_k \)(如 \( \mathbb{R} \) 中的 \( E_k = [0, k] \) \( (k = 1, 2, \cdots) \),\( E = [0, +\infty) \)),且 \( f(x) \) 在每个 \( I_k \) 上都是 Riemann 可积函数,以及条件
\[
\lim_{k \to \infty} \int_{I_k} |f(x)| \, \mathrm{d}x < +\infty
\]
成立时,我们就可以通过计算 Riemann 积分 \( \int_{I_k} f(x) \, \mathrm{d}x \) 而得到 Lebesgue 积分
\[
\int_E f(x) \, \mathrm{d}x = \lim_{k \to \infty} \int_{I_k} f(x) \, \mathrm{d}x
\]
的值.

还应指出的是,上述计算方法与 \( \{ I_k \} \) 的选择无关,只要保证它递增到并集 \( E \).
\end{remark}
\begin{proof}
因为 \( \{ |f(x)| \chi_{E_k}(x) \} \) 是非负渐升列,且有
\[
\lim_{k \to \infty} |f(x)| \chi_{E_k}(x) = |f(x)|, \quad x \in E,
\]
所以由\hyperref[theorem:Beppo Levi非负渐升列积分定理]{Beppo Levi 非负渐升列积分定理}可知
\begin{align*}
\int_E |f(x)| \, \mathrm{d}x = \lim_{k \to \infty} \int_E |f(x)| \chi_{E_k}(x) \, \mathrm{d}x = \lim_{k \to \infty} \int_{E_k} |f(x)| \, \mathrm{d}x < +\infty,
\end{align*}
即 \( f \in L(E) \). 又由于在 \( E \) 上有 \( (k = 1, 2, \cdots) \),
\[
\lim_{k \to \infty} f(x) \chi_{E_k}(x) = f(x), \quad |f(x) \chi_{E_k}(x)| \leqslant |f(x)|,
\]
故根据\hyperref[theorem:控制收敛定理]{控制收敛定理}可得
\[
\int_E{f(x)\,\mathrm{d}x}=\lim_{k\rightarrow \infty} \int_E{f(x)\chi _{E_k}\left( x \right) \,\mathrm{d}x}=\lim_{k\rightarrow \infty} \int_{E_k}{f(x)\,\mathrm{d}x}.
\]
\end{proof}

\begin{example}
设 \( f(x) = \frac{\sin x}{x} \),则它在 \( [0, +\infty) \) 上的反常积分为
\[
\int_0^{+\infty} \frac{\sin x}{x} \, \mathrm{d}x = \frac{\pi}{2}.
\]
但我们有
\[
\int_0^{+\infty} \left| \frac{\sin x}{x} \right| \, \mathrm{d}x = +\infty.
\]
这说明 \( f \notin L([0, +\infty)) \).
\end{example}
\begin{proof}

\end{proof}

\begin{example}
设 \( f(x) = x^{\alpha} \sin(1/x) \) \( (x \in [0, 1]) \),

(i) 若 \( \alpha \geq 0 \),则 \( f \in R([0, 1]) \);

(ii) 若 \( \alpha \geq -2 \),则 \( f(x) \) 在 \( [0, 1] \) 上的反常积分存在;

(iii) 若 \( \alpha > -1 \),则 \( f \in L([0, 1]) \).
\end{example}
\begin{proof}

\end{proof}

\begin{example}
求 \( I = \int_0^1 \frac{\ln x}{1 - x} \, \mathrm{d}x \).
\end{example}
\begin{solution}
由于当 \( 0 < x < 1 \) 时,有 \( -\frac{\ln x}{1 - x} = \sum_{n = 0}^{\infty} -x^n \ln x \),且
\begin{align*}
\int_0^1 x^n \ln x \, \mathrm{d}x = -\int_0^{+\infty} t e^{-(n + 1)t} \, dt = -\frac{1}{(n + 1)^2} \int_0^{+\infty} t e^{-t} \, dt = -\frac{1}{(n + 1)^2}.
\end{align*}
故得
\[
\int_0^1 \left( -\frac{\ln x}{1 - x} \right) \, \mathrm{d}x = \sum_{n = 0}^{\infty} \int_0^1 -x^n \ln x \, \mathrm{d}x = \sum_{n = 0}^{\infty} \frac{1}{(n + 1)^2} = \frac{\pi^2}{6}.
\]
由此可知 \( I = -\frac{\pi^2}{6} \).
\end{solution}
\begin{remark}
1. 设 \( f \in L(E) \),且 \( E = \bigcup_{n = 1}^{\infty} E_n \),\( E_i \cap E_j = \varnothing \) \( (i \neq j) \),其中每个 \( E_n \) 均为可测集,则
\[
\int_E f(x) \, \mathrm{d}x = \sum_{n = 1}^{\infty} \int_{E_n} f(x) \, \mathrm{d}x.
\]
但此结论对反常积分不一定真. 例如:对收敛级数 \( \sum_{n = 1}^{\infty} \frac{(-1)^n}{n} \)(非绝对收敛)以及 \( \alpha \neq -\ln 2 \),将 \( \sum_{n = 1}^{\infty} \frac{(-1)^n}{n} \) 的项作重新排列使新级数收敛到 \( \alpha \),并令
\[
E_n = [n - 1, n), \quad f(x) = \frac{(-1)^n}{n} \quad (n - 1 \leq x < n, n \in \mathbb{N}),
\]
我们有
\[
-\ln 2 = \int_0^{+\infty} f(x) \, \mathrm{d}x \neq \sum_{n = 1}^{\infty} \int_{E_n} f(x) \, \mathrm{d}x = \alpha.
\]

这说明在一种积分理论中,如果反常积分存在的函数总是可积的,那么此种积分理论就不具备对区域的可数可加性. 因此,我们不能期望有这样一种积分理论,它同时是反常积分和 Lebesgue 积分的推广. 如果放弃对积分区域可数可加性的要求,那么这种积分理论是存在的.

2. 对于定义在 \( [a, b] \) 上的函数 \( f(x) \),\( g(x) \),令 \( F(x) = \max_{[a, b]} \{ f(x), g(x) \} \). 若 \( f, g \in L([a, b]) \),则 \( F \in L([a, b]) \);若 \( f, g \in R([a, b]) \),则 \( F \in R([a, b]) \). 但若 \( f(x), g(x) \) 在 \( [a, b] \) 上反常可积,则 \( F(x) \) 在 \( [a, b] \) 上不一定反常可积.

3. 设 \( f \in R([a, b]) \),\( g(x) \) 在 \( [a, b] \) 上有界,且有
\[
m(\{ x \in [a, b] : f(x) \neq g(x) \}) = 0,
\]
但 \( g(x) \) 在 \( [a, b] \) 上不一定 Riemann 可积,例如
\[
f(x) = 1, \quad g(x) = 
\begin{cases} 
1, & x \in \mathbb{Q}, \\
0, & x \notin \mathbb{Q}
\end{cases} \quad (x \in [0, 1]).
\]

4. 设 \( f \in L([0, 1]) \) 且有界,不一定存在 \( g \in R([0, 1]) \),使得 \( g(x) = f(x) \),\(\text{a.e.}\, x \in [0, 1] \). 例如,取 \( [0, 1] \) 中一个无处稠密的正测集 \( E \),且令 \( f(x) = \chi_E(x) \) \( (0 \leq x \leq 1) \),则
\[
\int_0^1 f(x) \, \mathrm{d}x = m(E) > 0.
\]
此时,如果存在 \( g \in R([0, 1]) \),且有 \( g(x) = f(x) \),\(\text{a.e.}\, x \in [0, 1] \),那么点集 \( \{ x \in [0, 1] : g(x) = 0 \} \) 在 \( [0, 1] \) 中稠密,而使 \( \int_0^1 g(x) \, \mathrm{d}x = 0 \).

5. \( [a, b] \) 中存在零测集 \( E \),对于任意的 \( f \in R([a, b]) \),\( E \) 中必有 \( f(x) \) 的连续点.

证明 记 \( [a, b] \cap \mathbb{Q} = \{ r_n \} \),且作点集
\[
E_m = \bigcup_{n = 1}^{\infty} \left( r_n - 2^{-(n + m)}, r_n + 2^{-(n + m)} \right) \quad (m \in \mathbb{N}), \quad E = \bigcap_{m = 1}^{\infty} E_m,
\]
则 \( m(E_m) \leq 2^{-(n + 1)} \),且 \( m(E) = 0 \). 注意到 \( f(x) \) 的连续点集 \( \mathrm{cont}(f) \) 是稠密 \( G_{\delta} \) 集,且 \( m([a, b] \setminus \mathrm{cont}(f)) = 0 \). 根据 Baire 纲定理,可知 \( \mathrm{cont}(f) \) 是第二纲集.

\( [a, b] \setminus E_m \) \( (m \in \mathbb{N}) \) 是无处稠密集,\( E \) 是 \( G_{\delta} \) 集,\( [a, b] \setminus E = \bigcup_{m = 1}^{\infty} ([a, b] \setminus E_m) \) 是第一纲集. 从而得到 \( \mathrm{cont}(f) \not\subset [a, b] \setminus E \),即 \( \mathrm{cont}(f) \cap E \neq \varnothing \).
\end{remark}





































































































































\end{document}
\documentclass[../../main.tex]{subfiles}
\graphicspath{{\subfix{../../image/}}} % 指定图片目录,后续可以直接使用图片文件名。

% 例如:
% \begin{figure}[H]
% \centering
% \includegraphics{image-01.01}
% \caption{图片标题}
% \label{figure:image-01.01}
% \end{figure}
% 注意:上述\label{}一定要放在\caption{}之后,否则引用图片序号会只会显示??.

\begin{document}

\section{可积函数与连续函数的关系}

\begin{lemma}\label{lemma:引理4.3}
设 \( f(x) \) 在 \( E \) 上可积, 则对 \( \forall \varepsilon > 0 \), 都存在 \( E \) 上的简单函数 \( \varphi(x) \) 使得
\[
\int_E |f(x) - \varphi(x)|\mathrm{d}x < \varepsilon
\]
此时称 \( f(x) \) 可由 \( \varphi(x) \) \textbf{平均逼近}.
\end{lemma}
\begin{proof}
记 \( f(x) = f^+(x) - f^-(x) \). 由非负可测函数积分的定义(上确界的定义)知, 对 \( \forall \varepsilon > 0 \), 存在非负简单函数 \( \varphi^+ \leqslant f^+ \), \( \varphi^- \leqslant f^- \) 使得
\[
\int_E f^+(x)\mathrm{d}x - \int_E \varphi^+(x)\mathrm{d}x < \frac{\varepsilon}{2}
\]
\[
\int_E f^-(x)\mathrm{d}x - \int_E \varphi^-(x)\mathrm{d}x < \frac{\varepsilon}{2}
\]
令 \( \varphi = \varphi^+ - \varphi^- \), 则 \( \varphi(x) \) 是 \( E \) 上的简单函数, 且
\begin{align*}
\int_E{|f(x)}-\varphi (x)|\mathrm{d}x &=\int_E{\left| \left[ f^+\left( x \right) -f^-\left( x \right) \right] -\left[ \varphi ^+\left( x \right) -\varphi ^-\left( x \right) \right] \right|}
\\
&\leqslant \int_E{|f^+(x)}-\varphi ^+(x)|\mathrm{d}x +\int_E{|f^-(x)}-\varphi ^-(x)|\mathrm{d}x 
\\
&=\int_E{[f^+(x)}-\varphi ^+(x)]\mathrm{d}x +\int_E{[f^-(x)}-\varphi ^-(x)]\mathrm{d}x 
\\
&<\frac{\varepsilon}{2}+\frac{\varepsilon}{2}=\varepsilon .
\end{align*}
故引理得证.
\end{proof}

\begin{theorem}
若 \( f \in L(E) \),则对任给 \( \varepsilon > 0 \),存在 \( \mathbb{R}^n \) 上具有紧支集的连续函数 \( g(x) \),使得
\[
\int_E |f(x) - g(x)| \mathrm{d}x < \varepsilon.
\]
\end{theorem}
\begin{remark}
上述事实表明,若 \( f \in L(E) \),则对任给的 \( \varepsilon > 0 \),存在 \( f \) 的分解:
\[
f(x) = g(x) + [f(x) - g(x)] = f_1(x) + f_2(x), \quad x \in E,
\]
其中 \( f_1(x) \) 是 \( \mathbb{R}^n \) 上具有紧支集的连续函数,\( |f_2(x)| \) 在 \( E \) 上的积分小于 \( \varepsilon \).即\textbf{可积函数可以被$\mathbb{R}^n$上具有紧支集的可测简单函数逼近.}
\end{remark}
\begin{proof}
由于 \( f \in L(E) \),故由\reflem{lemma:引理4.3}可知,对任给的 \( \varepsilon > 0 \),存在 \( \mathbb{R}^n \) 上具有紧支集的可测简单函数 \( \varphi(x) \),使得
\[
\int_E |f(x) - \varphi(x)| \mathrm{d}x < \frac{\varepsilon}{2}.
\]
不妨设 \( |\varphi(x)| \leqslant M \),根据\refcor{corollary:推论3.19}可知,存在 \( \mathbb{R}^n \) 上具有紧支集的连续函数 \( g(x) \),使得 \( |g(x)| \leqslant M \)(\( x \in \mathbb{R}^n \)),且有
\[
m(\{ x \in E : |\varphi(x) - g(x)| > 0 \}) < \frac{\varepsilon}{4M},
\]
从而可得
\begin{align*}
\int_E{|\varphi (x)}-g(x)|\mathrm{d}x&=\int_{\{x\in E:|\varphi (x)-g(x)|>0\}}{|\varphi (x)}-g(x)|\mathrm{d}x+\int_{\{x\in E:|\varphi (x)-g(x)|=0\}}{|\varphi (x)}-g(x)|\mathrm{d}x
\\
&=\int_{\{x\in E:|\varphi (x)-g(x)|>0\}}{|\varphi (x)}-g(x)|\mathrm{d}x
\\
&\leqslant 2Mm(\{x:|\varphi (x)-g(x)|>0\})<\frac{\varepsilon}{2}.
\end{align*}
最后,我们有
\begin{align*}
\int_E |f(x) - g(x)| \mathrm{d}x \leqslant \int_E |f(x) - \varphi(x)| \mathrm{d}x + \int_E |\varphi(x) - g(x)| \mathrm{d}x < \frac{\varepsilon}{2} + \frac{\varepsilon}{2} = \varepsilon.
\end{align*}
\end{proof}

\begin{corollary}
设 \( f \in L(E) \),则存在 \( \mathbb{R}^n \) 上具有紧支集的连续函数列 \( \{ g_k(x) \} \),使得

(i)
\[
\lim_{k \to \infty} \int_E |f(x) - g_k(x)| \mathrm{d}x = 0;
\]

(ii)
\[
\lim_{k \to \infty} g_k(x) = f(x), \quad \text{a.e. } x \in E.
\]
\end{corollary}
\begin{proof}

\end{proof}

\begin{corollary}
设 \( f \in L([a,b]) \),则存在其支集在 \( (a,b) \) 内的连续函数列 \( \{ g_k(x) \} \),使得

(i)
\[
\lim_{k \to \infty} \int_{[a,b]} |f(x) - g_k(x)| \mathrm{d}x = 0;
\]

(ii)
\[
\lim_{k \to \infty} g_k(x) = f(x), \quad \text{a.e. } x \in [a,b].
\]
\end{corollary}
\begin{proof}

\end{proof}

\begin{example}
设 \( f \in L(\mathbb{R}^n) \).若对一切 \( \mathbb{R}^n \) 上具有紧支集的连续函数 \( \varphi(x) \),有
\[
\int_{\mathbb{R}^n} f(x)\varphi(x) \mathrm{d}x = 0,
\]
则 \( f(x) = 0 \),a. e. \( x \in \mathbb{R}^n \).
\end{example}
\begin{proof}
采用反证法.不妨假设 \( f(x) \) 在有界正测集 \( E \) 上有 \( 0 < f(x) \),则可作具有紧支集的连续函数列 \( \{ \varphi_k(x) \} \),使得
\[
\lim_{k \to \infty} \int_{\mathbb{R}^n} |\chi_E(x) - \varphi_k(x)| \mathrm{d}x = 0,
\]
\[
|\varphi_k(x)| \leqslant 1 \quad (k = 1,2,\cdots),
\]
\[
\lim_{k \to \infty} \varphi_k(x) = \chi_E(x), \quad \text{a. e. } x \in E.
\]
由于 \( |f(x)\varphi_k(x)| \leqslant |f(x)| \),\( x \in E \),故知
\begin{align*}
0 &< \int_E f(x) \mathrm{d}x = \int_{\mathbb{R}^n} f(x)\chi_E(x) \mathrm{d}x \\
&= \lim_{k \to \infty} \int_{\mathbb{R}^n} f(x)\varphi_k(x) \mathrm{d}x = 0,
\end{align*}
矛盾.
\end{proof}

\begin{example}
设 \( f \in L([a,b]) \).若对其支集在 \( (a,b) \) 内且可微的任一函数 \( \varphi(x) \),都有
\[
\int_{[a,b]} f(x)\varphi'(x) \mathrm{d}x = 0,
\]
则 \( f(x) = c \)(常数),a. e. \( x \in [a,b] \).
\end{example}
\begin{proof}
对任意的支集在 \( (a,b) \) 内的连续函数 \( g(x) \),作 \( h(x) \):支集在 \( (a,b) \) 内的连续函数,且满足 \( \int_{[a,b]} h(x) \mathrm{d}x = 1 \).令
\[
\varphi(x) = \int_{[a,x]} g(t) \mathrm{d}t - \int_{[a,x]} h(t) \mathrm{d}t \cdot \int_{[a,b]} g(t) \mathrm{d}t, \quad x \in [a,b],
\]
易知 \( \varphi(x) \) 的支集在 \( (a,b) \) 内,且有
\[
\varphi'(x) = g(x) - h(x)\int_{[a,b]} g(t) \mathrm{d}t, \quad x \in [a,b],
\]
从而由题设可得
\begin{align*}
0 &= \int_{[a,b]} f(x)\varphi'(x) \mathrm{d}x = \int_{[a,b]} f(x)\left( g(x) - h(x)\int_{[a,b]} g(t) \mathrm{d}t \right) \mathrm{d}x \\
&= \int_{[a,b]} f(x)g(x) \mathrm{d}x - \int_{[a,b]} f(x)h(x) \mathrm{d}x \cdot \int_{[a,b]} g(x) \mathrm{d}x \\
&= \int_{[a,b]} \left( f(x) - \int_{[a,b]} f(t)h(t) \mathrm{d}t \right) g(x) \mathrm{d}x.
\end{align*}
因此,我们有
\[
f(x) - \int_{[a,b]} f(t)h(t) \mathrm{d}t = 0, \quad \text{a. e. } x \in [a,b],
\]
即得所证.
\end{proof}


























































\end{document}
\documentclass[../../main.tex]{subfiles}
\graphicspath{{\subfix{../../image/}}} % 指定图片目录,后续可以直接使用图片文件名。

% 例如:
% \begin{figure}[H]
% \centering
% \includegraphics{image-01.01}
% \caption{图片标题}
% \label{figure:image-01.01}
% \end{figure}
% 注意:上述\label{}一定要放在\caption{}之后,否则引用图片序号会只会显示??.

\begin{document}

\section{一般可测函数的积分}

\subsection{积分的定义与初等性质}

\begin{definition}
设\(f(x)\)是\(E \subset \mathbf{R}^n\)上的可测函数.若积分
\[
\int_E f^+(x) \mathrm{d}x, \quad \int_E f^-(x) \mathrm{d}x
\]
中至少有一个是有限值,则称
\[
\int_E f(x) \mathrm{d}x = \int_E f^+(x) \mathrm{d}x - \int_E f^-(x) \mathrm{d}x
\]
为\(f(x)\)在\(E\)上的积分;当上式右端两个积分值皆为有限时,则称\(f(x)\)在\(E\)上是\textbf{可积的},或称\(f(x)\)是\(E\)上的\textbf{可积函数}.在\(E\)上可积的函数的全体记为\(L(E)\).
\end{definition}

\begin{theorem}\label{theorem:f与|f|的可积性等价}
若\(f(x)\)在$E$上可测,则\(f(x)\)在$E$上可积等价于\(|f(x)|\)在$E$上可积,且有
\begin{align*}
\left| \int_E f(x) \mathrm{d}x \right| \leqslant \int_E |f(x)| \mathrm{d}x. 
\end{align*}
\end{theorem}
\begin{proof}
由于等式
\[
\int_E |f(x)| \mathrm{d}x = \int_E f^+(x) \mathrm{d}x + \int_E f^-(x) \mathrm{d}x
\]
成立,故知在\(f(x)\)可测的条件下,\(f(x)\)的可积性与\(|f(x)|\)的可积性是等价的,且有
\begin{align*}
\left| \int_E{f(x)\mathrm{d}x} \right|=\left| \int_E{f^+(x)\mathrm{d}x}-\int_E{f^-(x)\mathrm{d}x} \right|\leqslant \int_E{f^+(x)\mathrm{d}x}+\int_E{f^-(x)\mathrm{d}x}=\int_E{|f(x)|\mathrm{d}x}.
\end{align*}
\end{proof}

\begin{theorem}\label{theorem:可积函数的基本性质}
\begin{enumerate}[(1)]
\item  若\(f(x)\)是\(E\)上的有界可测函数,且\(m(E) < +\infty\),则$f \in L(E).$

\item  若\(f \in L(E)\),则\(f(x)\)在\(E\)上是几乎处处有限的.

\item  若\(E \in \mathscr{M}\),且\(f(x) = 0\),\(\text{a. e.}\ x \in E\),则
\[
\int_E f(x) \mathrm{d}x = 0.
\]

\item \begin{enumerate}[(i)]
\item 若\(f(x)\)是\(E\)上的可测函数,\(g \in L(E)\),且\(|f(x)| \leqslant g(x)\),\(x \in E\)(\(g(x)\)称为\(f(x)\)的\textbf{控制函数}),则\(f \in L(E)\).

\item 若\(f \in L(E)\),\(e \subset E\)是可测集,则\(f \in L(e)\).
\end{enumerate}

\item \begin{enumerate}[(i)]
\item 设\(f \in L(\mathbf{R}^n)\),则
\[
\lim_{N \to \infty} \int_{\{x \in \mathbf{R}^n: |x| \geqslant N\}} |f(x)| \mathrm{d}x = 0,
\]
或说对任给\(\varepsilon > 0\),存在\(N\),使得
\[
\int_{\{x: |x| \geqslant N\}} |f(x)| \mathrm{d}x < \varepsilon.
\]

\item 若\(f \in L(E)\),且有\(E_N = \{x \in E: |x| \geqslant N\}\),则
\[
\lim_{N \to \infty} \int_{E \cap E_N} f(x) \mathrm{d}x =\lim_{N \to \infty} \int_{E_N} f(x) \mathrm{d}x = 0.
\]
\end{enumerate}
\end{enumerate}
\end{theorem}
\begin{proof}
\begin{enumerate}[(1)]
\item 不妨设\(|f(x)| \leqslant M\) \((x \in E)\),由于\(|f(x)|\)是\(E\)上的非负可测函数,故有
\[
\int_E |f(x)| \mathrm{d}x \leqslant \int_E M \mathrm{d}x = M m(E) < +\infty.
\]
因此由\refthe{theorem:f与|f|的可积性等价}可知$f \in L(E).$

\item 由$f\in L(E)$及\refthe{theorem:f与|f|的可积性等价}可知,非负可测函数$|f(x)|$在$E$上也可积.从而由\refthe{theorem:非负可积函数必几乎处处有限}可知,$|f(x)|$在$E$上几乎处处有限,即
\begin{align*}
m(\{x\in E:f(x)=\pm \infty\})=m(\{x\in E:|f(x)|=+\infty\})=0.
\end{align*}
故\(f(x)\)在\(E\)上是几乎处处有限的.

\item  因为\(|f(x)| = 0\),\(\text{a. e.}\),$x\in E$,且$|f(x)|$非负可测,所以由\refpro{theorem:对等的函数在相同可测集下的积分相等}可得
\[
\left| \int_E f(x) \mathrm{d}x \right| \leqslant \int_E |f(x)| \mathrm{d}x = 0.
\]
故$\int_E f(x) \mathrm{d}x = 0.$

\item \begin{enumerate}[(i)]
\item 由\hyperref[theorem:非负可测函数的性质]{非负可测函数的积分性质(1)}可知
\[
\int_E |f(x)| \mathrm{d}x \leqslant \int_E g(x) \mathrm{d}x < +\infty.
\]
故$|f|\in L(E)$,因此由\refthe{theorem:f与|f|的可积性等价}可知$f \in L(E).$

\item 若\(f \in L(E)\),\(e \subset E\)是可测集,则\hyperref[theorem:非负可测函数的性质]{非负可测函数的积分性质(1)(3)}可知
\begin{align*}
\int_e{\left| f\left( x \right) \right|\mathrm{d}x}=\int_E{\left| f\left( x \right) \right|\chi _e\left( x \right) \mathrm{d}x}=\int_E{\left| f\left( x \right) \chi _e\left( x \right) \right|\mathrm{d}x}\leqslant \int_E{\left| f\left( x \right) \right|\mathrm{d}x}<+\infty .
\end{align*}
故$|f|\in L(e)$,因此由\refthe{theorem:f与|f|的可积性等价}可知$f \in L(e).$
\end{enumerate}

\item \begin{enumerate}[(i)]
\item 记\(E_N = \{x \in \mathbf{R}^n: |x| \geqslant N\}\),则\(\{|f(x)| \chi_{E_N}(x)\}\)是非负可积函数渐降列,且有
\[
\lim_{N \to \infty} |f(x)| \chi_{E_N}(x) = 0, \quad x \in \mathbf{R}^n.
\]
由此可知
\begin{align*}
\lim_{N \to \infty} \int_{E_N} |f(x)| \mathrm{d}x = \lim_{N \to \infty} \int_{\mathbf{R}^n} |f(x)| \chi_{E_N}(x) \mathrm{d}x 
\xlongequal{\text{\refcor{corollary:渐降函数列积分定理}}} \int_{\mathbf{R}^n} \lim_{N \to \infty} |f(x)| \chi_{E_N}(x) \mathrm{d}x = 0.
\end{align*}

\item 由$f\in L(E)$及\hyperref[theorem:非负可测函数的性质]{非负可测函数的积分性质(1)(3)}可知
\begin{align*}
\int_{\mathbb{R} ^n}{\left| f\left( x \right) \chi _{E_N}\left( x \right) \right|\mathrm{d}x}=\int_{\mathbb{R} ^n}{\left| f\left( x \right) \right|\chi _{E_N}\left( x \right) \mathrm{d}x}\leqslant \int_{\mathbb{R} ^n}{\left| f\left( x \right) \right|\chi _E\left( x \right) \mathrm{d}x}=\int_E{\left| f\left( x \right) \right|\mathrm{d}x}<+\infty .
\end{align*}
因此\(f \cdot \chi_{E_N} \in L(\mathbf{R}^n)\).又$E_N\subset E\cap \left\{ x\in \mathbb{R} ^n:\left| x \right|\geqslant N \right\}$,故由\hyperref[theorem:非负可测函数的性质]{非负可测函数的积分性质(3)}及(i)可得
\begin{align*}
\underset{N\rightarrow \infty}{\lim}\int_{E\cap E_N}{f\left( x \right) \mathrm{d}x}=\underset{N\rightarrow \infty}{\lim}\int_{E_N}{f\left( x \right) \mathrm{d}x}=\underset{N\rightarrow \infty}{\lim}\int_{\left\{ x\in \mathbb{R} ^n:\left| x \right|\geqslant N \right\}}{f\left( x \right) \chi _{E_N}\left( x \right) \mathrm{d}x}=0.
\end{align*}
\end{enumerate}
\end{enumerate}
\end{proof}

\begin{theorem}[积分的线性性质]\label{theorem:积分的线性性质}
若 \( f,g \in L(E) \),\( C \in \mathbb{R} \),则
\begin{enumerate}
\item[(i)] \( \int_E Cf(x) \, \mathrm{d}x = C \int_E f(x) \, \mathrm{d}x \),进而$Cf\in L(E)$;
\item[(ii)] $f+g\in L(E)$且\( \int_E (f(x) + g(x)) \, \mathrm{d}x = \int_E f(x) \, \mathrm{d}x + \int_E g(x) \, \mathrm{d}x \).

\item[(iii)] 若 \( f \in L(E) \),\( g(x) \) 是 \( E \) 上的有界可测函数,则 \( f \cdot g \in L(E) \). 
\end{enumerate}
\end{theorem}
\begin{proof}
不妨假定$f,g$都是实值函数(即处处有限).
(i) 由公式
\begin{align}\label{eq:100.94}
f^+(x) = \frac{|f(x)| + f(x)}{2}, \quad f^-(x) = \frac{|f(x)| - f(x)}{2}
\end{align}
立即可知:当 \( C \geq 0 \) 时,\( (Cf)^+ = Cf^+ \),\( (Cf)^- = Cf^- \). 根据积分定义以及\hyperref[theorem:非负可测函数积分的线性性质]{非负可测函数积分的线性性质},可得
\begin{align*}
\int_E Cf(x) \, \mathrm{d}x &= \int_E Cf^+(x) \, \mathrm{d}x - \int_E Cf^-(x) \, \mathrm{d}x \\
&= C \left( \int_E f^+(x) \, \mathrm{d}x - \int_E f^-(x) \, \mathrm{d}x \right) = C \int_E f(x) \, \mathrm{d}x.
\end{align*}
当 \( C = -1 \) 时,由\eqref{eq:100.94}式可知\( (-f)^+ = f^- \),\( (-f)^- = f^+ \). 同理可得
\[
\int_E (-f(x)) \, \mathrm{d}x = \int_E f^-(x) \, \mathrm{d}x - \int_E f^+(x) \, \mathrm{d}x = - \int_E f(x) \, \mathrm{d}x.
\]
当 \( C < 0 \) 时,由\eqref{eq:100.94}式可知\( Cf(x) = -|C|f(x) \). 由上述结论可得
\begin{align*}
\int_E Cf(x) \, \mathrm{d}x &= \int_E -|C| f(x) \, \mathrm{d}x = - \int_E |C|f(x) \, \mathrm{d}x \\
&= -|C| \int_E f(x) \, \mathrm{d}x = C \int_E f(x) \, \mathrm{d}x.
\end{align*}
综上可得
\begin{align*}
\int_E{\left| Cf\left( x \right) \right|\mathrm{d}x}=\left| C \right|\int_E{\left| f\left( x \right) \right|\mathrm{d}x}<+\infty ,\forall C\in \mathbb{R} .
\end{align*}
故\( Cf(x)\in L(E) \).

(ii) 首先,由于有 \( |f(x) + g(x)| \leq |f(x)| + |g(x)| \),故可知 \( f + g \in L(E) \). 其次,注意到
\[
(f + g)^+ - (f + g)^- = f + g = f^+ - f^- + g^+ - g^-,
\]
进而
\[
(f + g)^+ + f^- + g^- = (f + g)^- + f^+ + g^+,
\]
从而由\hyperref[theorem:非负可测函数积分的线性性质]{非负可测函数积分的线性性质}得
\begin{align*}
\int_E (f + g)^+(x) \, \mathrm{d}x + \int_E f^-(x) \, \mathrm{d}x + \int_E g^-(x) \, \mathrm{d}x 
= \int_E (f + g)^-(x) \, \mathrm{d}x + \int_E f^+(x) \, \mathrm{d}x + \int_E g^+(x) \, \mathrm{d}x.
\end{align*}
因为式中每项积分值都是有限的,所以可移项且得到
\[
\int_E (f(x) + g(x)) \, \mathrm{d}x = \int_E f(x) \, \mathrm{d}x + \int_E g(x) \, \mathrm{d}x.
\]

(iii)注意到
\[
|f(x) \cdot g(x)| \leq |f(x)| \cdot \sup_{x \in E} |g(x)|, \quad x \in E.
\]
由$g$在$E$上有界,故$\sup_{x \in E} |g(x)|\in \mathbb{R}$.从而由(i)可得$|f(x)| \cdot \sup_{x \in E} |g(x)|\in L(E)$,于是再由\nrefthe{theorem:可积函数的基本性质}{(4)(i)}可知$f\cdot g\in L(E)$.
\end{proof}

\begin{corollary}
若 \( f \in L(E) \),且 \( f(x) = g(x) \),\(\text{a. e.}\ x \in E \),则
\[
\int_E f(x) \, \mathrm{d}x = \int_E g(x) \, \mathrm{d}x.
\]
\end{corollary}
\begin{note}
这个推论表明:\textbf{改变可测函数在零测集上的值,不会影响它的可积性与积分值.}
\end{note}
\begin{proof}
令$E_1=\left\{ x\in E:f\left( x \right) \ne g\left( x \right) \right\}$,$E_2=E\backslash E_1$,$m\left( E_1 \right) =0$,则
\begin{align*}
\int_E{f\left( x \right) \mathrm{d}x}&=\int_E{f^+\left( x \right) \mathrm{d}x}-\int_E{f^-\left( x \right) \mathrm{d}x}\xlongequal{\text{{\nrefthe{theorem:非负可测函数的性质}{(3)}}}}\int_E{f^+\left( x \right) \chi _E\left( x \right) \mathrm{d}x}-\int_E{f^-\left( x \right) \chi _E\left( x \right) \mathrm{d}x} \\
&=\int_E{f^+\left( x \right) \chi _{E_1\cup E_2}\left( x \right) \mathrm{d}x}-\int_E{f^-\left( x \right) \chi _{E_1\cup E_2}\left( x \right) \mathrm{d}x}
\\
&=\int_E{f^+\left( x \right) \left[ \chi _{E_1}\left( x \right) +\chi _{E_2}\left( x \right) \right] \mathrm{d}x}-\int_E{f^-\left( x \right) \left[ \chi _{E_1}\left( x \right) +\chi _{E_2}\left( x \right) \right] \mathrm{d}x} \\
&=\int_E{f^+\left( x \right) \chi _{E_1}\left( x \right) \mathrm{d}x}+\int_E{f^+\left( x \right) \chi _{E_2}\left( x \right) \mathrm{d}x}-\int_E{f^-\left( x \right) \chi _{E_1}\left( x \right) \mathrm{d}x}-\int_E{f^-\left( x \right) \chi _{E_2}\left( x \right) \mathrm{d}x} \\
&\xlongequal{\text{\nrefthe{theorem:非负可测函数的性质}{(3)}}}\int_{E_1}{f^+\left( x \right) \mathrm{d}x}+\int_{E_2}{f^+\left( x \right) \mathrm{d}x}-\int_{E_1}{f^-\left( x \right) \mathrm{d}x}-\int_{E_2}{f^-\left( x \right) \mathrm{d}x} \\
&\xlongequal{\text{\nrefthe{theorem:非负可测函数的性质}{(5)(ii)}}}\int_{E_2}{f^+\left( x \right) \mathrm{d}x}-\int_{E_2}{f^-\left( x \right) \mathrm{d}x}=\int_{E_2}{g^+\left( x \right) \mathrm{d}x}-\int_{E_2}{g^-\left( x \right) \mathrm{d}x} \\
&\xlongequal{\text{\nrefthe{theorem:非负可测函数的性质}{(5)(ii)}}}\int_{E_1}{g^+\left( x \right) \mathrm{d}x}+\int_{E_2}{g^+\left( x \right) \mathrm{d}x}-\int_{E_1}{g^-\left( x \right) \mathrm{d}x}-\int_{E_2}{g^-\left( x \right) \mathrm{d}x} \\
&\xlongequal{\text{\nrefthe{theorem:非负可测函数的性质}{(3)}}}\int_E{g^+\left( x \right) \chi _{E_1}\left( x \right) \mathrm{d}x}+\int_E{g^+\left( x \right) \chi _{E_2}\left( x \right) \mathrm{d}x}-\int_E{g^-\left( x \right) \chi _{E_1}\left( x \right) \mathrm{d}x}-\int_E{g^-\left( x \right) \chi _{E_2}\left( x \right) \mathrm{d}x} \\
&=\int_E{g^+\left( x \right) \left[ \chi _{E_1}\left( x \right) +\chi _{E_2}\left( x \right) \right] \mathrm{d}x}-\int_E{g^-\left( x \right) \left[ \chi _{E_1}\left( x \right) +\chi _{E_2}\left( x \right) \right] \mathrm{d}x} \\
&=\int_E{g^+\left( x \right) \chi _E\left( x \right) \mathrm{d}x}-\int_E{g^-\left( x \right) \chi _E\left( x \right) \mathrm{d}x}=\int_E{g^+\left( x \right) \mathrm{d}x}-\int_E{g^-\left( x \right) \mathrm{d}x} \\
&=\int_E{g\left( x \right) \mathrm{d}x}.
\end{align*}
\end{proof}

















































































































































































































































\end{document}
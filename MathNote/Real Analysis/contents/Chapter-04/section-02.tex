\documentclass[../../main.tex]{subfiles}
\graphicspath{{\subfix{../../image/}}} % 指定图片目录,后续可以直接使用图片文件名。

% 例如:
% \begin{figure}[H]
% \centering
% \includegraphics[scale=0.4]{图.png}
% \caption{}
% \label{figure:图}
% \end{figure}
% 注意:上述\label{}一定要放在\caption{}之后,否则引用图片序号会只会显示??.

\begin{document}

\section{一般可测函数的积分}

\subsection{积分的定义与初等性质}

\begin{definition}
设\(f(x)\)是\(E \subset \mathbf{R}^n\)上的可测函数.若积分
\[
\int_E f^+(x) \mathrm{d}x, \quad \int_E f^-(x) \mathrm{d}x
\]
中至少有一个是有限值,则称
\[
\int_E f(x) \mathrm{d}x = \int_E f^+(x) \mathrm{d}x - \int_E f^-(x) \mathrm{d}x
\]
为\(f(x)\)在\(E\)上的积分;当上式右端两个积分值皆为有限时,则称\(f(x)\)在\(E\)上是\textbf{可积的},或称\(f(x)\)是\(E\)上的\textbf{可积函数}.在\(E\)上可积的函数的全体记为\(L(E)\).
\end{definition}

\begin{theorem}\label{theorem:f与|f|的可积性等价}
若\(f(x)\)在$E$上可测,则\(f(x)\)在$E$上可积等价于\(|f(x)|\)在$E$上可积,且有
\begin{align*}
\left| \int_E f(x) \mathrm{d}x \right| \leqslant \int_E |f(x)| \mathrm{d}x. 
\end{align*}
\end{theorem}
\begin{proof}
由\hyperref[theorem:非负可测函数积分的线性性质]{非负可测函数积分的线性性质}可知
\[
\int_E{|f(x)|\mathrm{d}x}=\int_E{\left[ f^+\left( x \right) +f^-\left( x \right) \right] \mathrm{d}x}=\int_E{f^+(x)\mathrm{d}x}+\int_E{f^-(x)\mathrm{d}x}
\]
成立,故知在\(f(x)\)可测的条件下,\(f(x)\)的可积性与\(|f(x)|\)的可积性是等价的,且有
\begin{align*}
\left| \int_E{f(x)\mathrm{d}x} \right|=\left| \int_E{f^+(x)\mathrm{d}x}-\int_E{f^-(x)\mathrm{d}x} \right|\leqslant \int_E{f^+(x)\mathrm{d}x}+\int_E{f^-(x)\mathrm{d}x}=\int_E{|f(x)|\mathrm{d}x}.
\end{align*}
\end{proof}

\begin{theorem}[积分的基本性质]\label{theorem:积分的基本性质}
\begin{enumerate}[(1)]
\item  若\(f(x)\)是\(E\)上的有界可测函数,且\(m(E) < +\infty\),则$f \in L(E).$

\item  若\(f \in L(E)\),则\(f(x)\)在\(E\)上是几乎处处有限的.

\item  若\(E \in \mathscr{M}\),且\(f(x) = 0\),\(\text{a. e.}\ x \in E\),则$\int_E f(x) \mathrm{d}x = 0.$

\item \begin{enumerate}[(i)]
\item 若\(f(x)\)是\(E\)上的可测函数,\(g \in L(E)\),且\(|f(x)| \leqslant g(x)\),a.e. \(x \in E\)(\(g(x)\)称为\(f(x)\)的\textbf{控制函数}),则\(f \in L(E)\).

\item 若\(f \in L(E)\),\(e \subset E\)是可测集,则\(f \in L(e)\).
\end{enumerate}

\item 若 \( f(x) \leqslant g(x) \),a.e. \( x \in E \),则 \( \int_E f(x)\mathrm{d}x \leqslant \int_E g(x)\mathrm{d}x \).

\item \begin{enumerate}[(i)]
\item 设\(f \in L(\mathbf{R}^n)\),则
\[
\lim_{N \to \infty} \int_{\{x \in \mathbf{R}^n: |x| \geqslant N\}} |f(x)| \mathrm{d}x = 0,
\]
或说对任给\(\varepsilon > 0\),存在\(N\),使得
\[
\int_{\{x: |x| \geqslant N\}} |f(x)| \mathrm{d}x < \varepsilon.
\]

\item 若\(f \in L(E)\),且有\(E_N = \{x \in E: |x| \geqslant N\}\),则
\[
\lim_{N \to \infty} \int_{E \cap E_N} f(x) \mathrm{d}x =\lim_{N \to \infty} \int_{E_N} f(x) \mathrm{d}x = 0.
\]
\end{enumerate}
\end{enumerate}
\end{theorem}
\begin{remark}
(3)反过来并不成立,例如,$f\left( x \right) =\begin{cases}
1,&x\in \left[ 0,1 \right] ,\\
-1,&x\in \left( 1,2 \right] .\\
\end{cases}.$
\end{remark}
\begin{proof}
\begin{enumerate}[(1)]
\item 不妨设\(|f(x)| \leqslant M\) \((x \in E)\),由于\(|f(x)|\)是\(E\)上的非负可测函数,故有
\[
\int_E |f(x)| \mathrm{d}x \leqslant \int_E M \mathrm{d}x = M m(E) < +\infty.
\]
因此由\refthe{theorem:f与|f|的可积性等价}可知$f \in L(E).$

\item 由$f\in L(E)$及\refthe{theorem:f与|f|的可积性等价}可知,非负可测函数$|f(x)|$在$E$上也可积.从而由\refthe{theorem:非负可积函数必几乎处处有限}可知,$|f(x)|$在$E$上几乎处处有限,即
\begin{align*}
m(\{x\in E:f(x)=\pm \infty\})=m(\{x\in E:|f(x)|=+\infty\})=0.
\end{align*}
故\(f(x)\)在\(E\)上是几乎处处有限的.

\item 因为\(|f(x)| = 0\),\(\text{a. e.}\),$x\in E$,且$|f(x)|$非负可测,所以由\refpro{theorem:对等的函数在相同可测集下的积分相等}可得
\[
\left| \int_E f(x) \mathrm{d}x \right| \leqslant \int_E |f(x)| \mathrm{d}x = 0.
\]
故$\int_E f(x) \mathrm{d}x = 0.$

\item \begin{enumerate}[(i)]
\item 由\hyperref[theorem:非负可测函数积分的性质]{非负可测函数的积分性质(1)}可知
\[
\int_E |f(x)| \mathrm{d}x \leqslant \int_E g(x) \mathrm{d}x < +\infty.
\]
故$|f|\in L(E)$,因此由\refthe{theorem:f与|f|的可积性等价}可知$f \in L(E).$

\item 若\(f \in L(E)\),\(e \subset E\)是可测集,则\hyperref[theorem:非负可测函数积分的性质]{非负可测函数的积分性质(1)(3)}可知
\begin{align*}
\int_e{\left| f\left( x \right) \right|\mathrm{d}x}=\int_E{\left| f\left( x \right) \right|\chi _e\left( x \right) \mathrm{d}x}=\int_E{\left| f\left( x \right) \chi _e\left( x \right) \right|\mathrm{d}x}\leqslant \int_E{\left| f\left( x \right) \right|\mathrm{d}x}<+\infty .
\end{align*}
故$|f|\in L(e)$,因此由\refthe{theorem:f与|f|的可积性等价}可知$f \in L(e).$
\end{enumerate}

\item 因为\( f(x) \leqslant g(x) \),a.e. \( x \in E \),所以 \( f^+(x) \leqslant g^+(x) \),\( f^-(x) \geqslant g^-(x) \),a.e. \( x \in E \). 由\hyperref[theorem:非负可测函数积分的性质]{非负可测函数积分的性质(1)}可知
\[
\int_E f^+(x)\mathrm{d}x \leqslant \int_E g^+(x)\mathrm{d}x, \quad \int_E f^-(x)\mathrm{d}x \geqslant \int_E g^-(x)\mathrm{d}x
\]
从而
\begin{align*}
\int_E f(x)\mathrm{d}x = \int_E f^+(x)\mathrm{d}x - \int_E f^-(x)\mathrm{d}x 
\leqslant \int_E g^+(x)\mathrm{d}x - \int_E g^-(x)\mathrm{d}x = \int_E g(x)\mathrm{d}x
\end{align*}
故结论成立.

\item \begin{enumerate}[(i)]
\item 记\(E_N = \{x \in \mathbf{R}^n: |x| \geqslant N\}\),则\(\{|f(x)| \chi_{E_N}(x)\}\)是非负可积函数渐降列,且有
\[
\lim_{N \to \infty} |f(x)| \chi_{E_N}(x) = 0, \quad x \in \mathbf{R}^n.
\]
由此可知
\begin{align*}
\lim_{N \to \infty} \int_{E_N} |f(x)| \mathrm{d}x = \lim_{N \to \infty} \int_{\mathbf{R}^n} |f(x)| \chi_{E_N}(x) \mathrm{d}x 
\xlongequal{\text{\refcor{corollary:非负渐降函数列积分定理}}} \int_{\mathbf{R}^n} \lim_{N \to \infty} |f(x)| \chi_{E_N}(x) \mathrm{d}x = 0.
\end{align*}

\item 由$f\in L(E)$及\hyperref[theorem:非负可测函数积分的性质]{非负可测函数的积分性质(1)(3)}可知
\begin{align*}
\int_{\mathbb{R} ^n}{\left| f\left( x \right) \chi _{E_N}\left( x \right) \right|\mathrm{d}x}=\int_{\mathbb{R} ^n}{\left| f\left( x \right) \right|\chi _{E_N}\left( x \right) \mathrm{d}x}\leqslant \int_{\mathbb{R} ^n}{\left| f\left( x \right) \right|\chi _E\left( x \right) \mathrm{d}x}=\int_E{\left| f\left( x \right) \right|\mathrm{d}x}<+\infty .
\end{align*}
因此\(f \cdot \chi_{E_N} \in L(\mathbf{R}^n)\).又$E_N\subset E\cap \left\{ x\in \mathbb{R} ^n:\left| x \right|\geqslant N \right\}$,故由\hyperref[theorem:非负可测函数积分的性质]{非负可测函数的积分性质(3)}及(i)可得
\begin{align*}
\underset{N\rightarrow \infty}{\lim}\int_{E\cap E_N}{f\left( x \right) \mathrm{d}x}=\underset{N\rightarrow \infty}{\lim}\int_{E_N}{f\left( x \right) \mathrm{d}x}=\underset{N\rightarrow \infty}{\lim}\int_{\left\{ x\in \mathbb{R} ^n:\left| x \right|\geqslant N \right\}}{f\left( x \right) \chi _{E_N}\left( x \right) \mathrm{d}x}=0.
\end{align*}
\end{enumerate}
\end{enumerate}
\end{proof}

\begin{theorem}[积分的线性性质]\label{theorem:积分的线性性质}
若 \( f,g \in L(E) \),\( C \in \mathbb{R} \),则
\begin{enumerate}
\item[(i)] \( \int_E Cf(x) \, \mathrm{d}x = C \int_E f(x) \, \mathrm{d}x \),进而$Cf\in L(E)$;
\item[(ii)] $f+g\in L(E)$且\( \int_E (f(x) + g(x)) \, \mathrm{d}x = \int_E f(x) \, \mathrm{d}x + \int_E g(x) \, \mathrm{d}x \).

\item[(iii)] 若 \( f \in L(E) \),\( g(x) \) 是 \( E \) 上的有界可测函数,则 \( f \cdot g \in L(E) \). 
\end{enumerate}
\end{theorem}
\begin{remark}
不妨假定$f,g$都是实值函数(即处处有限)的原因:(i) 假设结论对处处有限的函数成立. 若$f$不是处处有限的函数, 则由$f \in L(E)$及\hyperref[theorem:积分的基本性质]{可积函数的基本性质(ii)}可知, 令$E_1 = \{x \in E : |f(x)| = +\infty\}$, 则$m(E_1) = 0$, 再令$E_2 = E \backslash E_1$, 则由假设可知
\begin{align}
\int_{E_2} Cf(x) \, \mathrm{d}x = C \int_{E_2} f(x) \, \mathrm{d}x. \label{100.99}
\end{align}
由\hyperref[theorem:theorem:非负可测函数积分的线性性质]{非负可测函数积分线性性质}及\nrefthe{theorem:非负可测函数积分的性质}{(3)}可得
\begin{align*}
\int_E Cf(x) \, \mathrm{d}x &= \int_E (Cf(x))^+ \, \mathrm{d}x - \int_E (Cf(x))^- \, \mathrm{d}x \\
&= \int_E (Cf(x))^+ \chi_{E_1 \cup E_2}(x) \, \mathrm{d}x - \int_E (Cf(x))^- \chi_{E_1 \cup E_2}(x) \, \mathrm{d}x \\
&= \int_E (Cf(x))^+ \chi_{E_1}(x) \, \mathrm{d}x + \int_E (Cf(x))^+ \chi_{E_2}(x) \, \mathrm{d}x - \int_E (Cf(x))^- \chi_{E_1}(x) \, \mathrm{d}x - \int_E (Cf(x))^- \chi_{E_2}(x) \, \mathrm{d}x \\
&= \int_{E_1} (Cf(x))^+ \, \mathrm{d}x + \int_{E_2} (Cf(x))^+ \, \mathrm{d}x - \int_{E_1} (Cf(x))^- \, \mathrm{d}x - \int_{E_2} (Cf(x))^- \, \mathrm{d}x \\
&= \int_{E_2} (Cf(x))^+ \, \mathrm{d}x - \int_{E_2} (Cf(x))^- \, \mathrm{d}x \overset{\text{\eqref{100.99}式}}{=} \int_{E_2} Cf(x) \, \mathrm{d}x \\
&= C \int_{E_2} f(x) \, \mathrm{d}x = C \int_{E_2} f^+(x) \, \mathrm{d}x - C \int_{E_2} f^-(x) \, \mathrm{d}x \\
&= C \left( \int_{E_1} f^+(x) \, \mathrm{d}x + \int_{E_2} f^+(x) \, \mathrm{d}x - \int_{E_1} f^-(x) \, \mathrm{d}x - \int_{E_2} f^-(x) \, \mathrm{d}x \right) \\
&= C \left( \int_E f^+(x) \chi_{E_1}(x) \, \mathrm{d}x + \int_E f^+(x) \chi_{E_2}(x) \, \mathrm{d}x - \int_E f^-(x) \chi_{E_1}(x) \, \mathrm{d}x - \int_E f^-(x) \chi_{E_2}(x) \, \mathrm{d}x \right) \\
&= C \left( \int_E f^+(x) \chi_{E_1 \cup E_2}(x) \, \mathrm{d}x - \int_E f^-(x) \chi_{E_1 \cup E_2}(x) \, \mathrm{d}x \right) \\
&= C \left( \int_E f^+(x) \, \mathrm{d}x - \int_E f^-(x) \, \mathrm{d}x \right) = C \int_E f(x) \, \mathrm{d}x.
\end{align*}
故对一般情况结论也成立.

(ii) 由(i)同理可证.
\end{remark}
\begin{proof}
不妨假定$f,g$都是实值函数(即处处有限).

(i) 由公式
\begin{align}\label{eq:100.94}
f^+(x) = \frac{|f(x)| + f(x)}{2}, \quad f^-(x) = \frac{|f(x)| - f(x)}{2}
\end{align}
立即可知:当 \( C \geq 0 \) 时,\( (Cf)^+ = Cf^+ \),\( (Cf)^- = Cf^- \). 根据积分定义以及\hyperref[theorem:非负可测函数积分的线性性质]{非负可测函数积分的线性性质},可得
\begin{align*}
\int_E Cf(x) \, \mathrm{d}x &= \int_E Cf^+(x) \, \mathrm{d}x - \int_E Cf^-(x) \, \mathrm{d}x \\
&= C \left( \int_E f^+(x) \, \mathrm{d}x - \int_E f^-(x) \, \mathrm{d}x \right) = C \int_E f(x) \, \mathrm{d}x.
\end{align*}
当 \( C = -1 \) 时,由\eqref{eq:100.94}式可知\( (-f)^+ = f^- \),\( (-f)^- = f^+ \). 同理可得
\[
\int_E (-f(x)) \, \mathrm{d}x = \int_E f^-(x) \, \mathrm{d}x - \int_E f^+(x) \, \mathrm{d}x = - \int_E f(x) \, \mathrm{d}x.
\]
当 \( C < 0 \) 时,由\eqref{eq:100.94}式可知\( Cf(x) = -|C|f(x) \). 由上述结论可得
\begin{align*}
\int_E Cf(x) \, \mathrm{d}x &= \int_E -|C| f(x) \, \mathrm{d}x = - \int_E |C|f(x) \, \mathrm{d}x \\
&= -|C| \int_E f(x) \, \mathrm{d}x = C \int_E f(x) \, \mathrm{d}x.
\end{align*}
综上可得
\begin{align*}
\int_E{\left| Cf\left( x \right) \right|\mathrm{d}x}=\left| C \right|\int_E{\left| f\left( x \right) \right|\mathrm{d}x}<+\infty ,\forall C\in \mathbb{R} .
\end{align*}
故\( Cf(x)\in L(E) \).

(ii) 首先,由于有 \( |f(x) + g(x)| \leq |f(x)| + |g(x)| \),故可知 \( f + g \in L(E) \). 其次,注意到
\[
(f + g)^+ - (f + g)^- = f + g = f^+ - f^- + g^+ - g^-,
\]
进而
\[
(f + g)^+ + f^- + g^- = (f + g)^- + f^+ + g^+,
\]
从而由\hyperref[theorem:非负可测函数积分的线性性质]{非负可测函数积分的线性性质}得
\begin{align*}
\int_E (f + g)^+(x) \, \mathrm{d}x + \int_E f^-(x) \, \mathrm{d}x + \int_E g^-(x) \, \mathrm{d}x 
= \int_E (f + g)^-(x) \, \mathrm{d}x + \int_E f^+(x) \, \mathrm{d}x + \int_E g^+(x) \, \mathrm{d}x.
\end{align*}
因为式中每项积分值都是有限的,所以可移项且得到
\[
\int_E (f(x) + g(x)) \, \mathrm{d}x = \int_E f(x) \, \mathrm{d}x + \int_E g(x) \, \mathrm{d}x.
\]

(iii)注意到
\[
|f(x) \cdot g(x)| \leq |f(x)| \cdot \sup_{x \in E} |g(x)|, \quad x \in E.
\]
由$g$在$E$上有界,故$\sup_{x \in E} |g(x)|\in \mathbb{R}$.从而由(i)可得$|f(x)| \cdot \sup_{x \in E} |g(x)|\in L(E)$,于是再由\nrefthe{theorem:积分的基本性质}{(4)(i)}可知$f\cdot g\in L(E)$.
\end{proof}

\begin{corollary}
若 \( f \in L(E) \),且 \( f(x) = g(x) \),\(\text{a. e.}\ x \in E \),则
\[
\int_E f(x) \, \mathrm{d}x = \int_E g(x) \, \mathrm{d}x.
\]
\end{corollary}
\begin{note}
这个推论表明:\textbf{改变可测函数在零测集上的值,不会影响它的可积性与积分值.}
\end{note}
\begin{proof}
令$E_1=\left\{ x\in E:f\left( x \right) \ne g\left( x \right) \right\}$,$E_2=E\backslash E_1$,$m\left( E_1 \right) =0$,则
\begin{align*}
\int_E{f\left( x \right) \mathrm{d}x}&=\int_E{f^+\left( x \right) \mathrm{d}x}-\int_E{f^-\left( x \right) \mathrm{d}x}\xlongequal{\text{{\nrefthe{theorem:非负可测函数积分的性质}{(3)}}}}\int_E{f^+\left( x \right) \chi _E\left( x \right) \mathrm{d}x}-\int_E{f^-\left( x \right) \chi _E\left( x \right) \mathrm{d}x} \\
&=\int_E{f^+\left( x \right) \chi _{E_1\cup E_2}\left( x \right) \mathrm{d}x}-\int_E{f^-\left( x \right) \chi _{E_1\cup E_2}\left( x \right) \mathrm{d}x}
\\
&=\int_E{f^+\left( x \right) \left[ \chi _{E_1}\left( x \right) +\chi _{E_2}\left( x \right) \right] \mathrm{d}x}-\int_E{f^-\left( x \right) \left[ \chi _{E_1}\left( x \right) +\chi _{E_2}\left( x \right) \right] \mathrm{d}x} \\
&=\int_E{f^+\left( x \right) \chi _{E_1}\left( x \right) \mathrm{d}x}+\int_E{f^+\left( x \right) \chi _{E_2}\left( x \right) \mathrm{d}x}-\int_E{f^-\left( x \right) \chi _{E_1}\left( x \right) \mathrm{d}x}-\int_E{f^-\left( x \right) \chi _{E_2}\left( x \right) \mathrm{d}x} \\
&\xlongequal{\text{\nrefthe{theorem:非负可测函数积分的性质}{(3)}}}\int_{E_1}{f^+\left( x \right) \mathrm{d}x}+\int_{E_2}{f^+\left( x \right) \mathrm{d}x}-\int_{E_1}{f^-\left( x \right) \mathrm{d}x}-\int_{E_2}{f^-\left( x \right) \mathrm{d}x} \\
&\xlongequal{\text{\nrefthe{theorem:非负可测函数积分的性质}{(5)(ii)}}}\int_{E_2}{f^+\left( x \right) \mathrm{d}x}-\int_{E_2}{f^-\left( x \right) \mathrm{d}x}=\int_{E_2}{g^+\left( x \right) \mathrm{d}x}-\int_{E_2}{g^-\left( x \right) \mathrm{d}x} \\
&\xlongequal{\text{\nrefthe{theorem:非负可测函数积分的性质}{(5)(ii)}}}\int_{E_1}{g^+\left( x \right) \mathrm{d}x}+\int_{E_2}{g^+\left( x \right) \mathrm{d}x}-\int_{E_1}{g^-\left( x \right) \mathrm{d}x}-\int_{E_2}{g^-\left( x \right) \mathrm{d}x} \\
&\xlongequal{\text{\nrefthe{theorem:非负可测函数积分的性质}{(3)}}}\int_E{g^+\left( x \right) \chi _{E_1}\left( x \right) \mathrm{d}x}+\int_E{g^+\left( x \right) \chi _{E_2}\left( x \right) \mathrm{d}x}-\int_E{g^-\left( x \right) \chi _{E_1}\left( x \right) \mathrm{d}x}-\int_E{g^-\left( x \right) \chi _{E_2}\left( x \right) \mathrm{d}x} \\
&=\int_E{g^+\left( x \right) \left[ \chi _{E_1}\left( x \right) +\chi _{E_2}\left( x \right) \right] \mathrm{d}x}-\int_E{g^-\left( x \right) \left[ \chi _{E_1}\left( x \right) +\chi _{E_2}\left( x \right) \right] \mathrm{d}x} \\
&=\int_E{g^+\left( x \right) \chi _E\left( x \right) \mathrm{d}x}-\int_E{g^-\left( x \right) \chi _E\left( x \right) \mathrm{d}x}=\int_E{g^+\left( x \right) \mathrm{d}x}-\int_E{g^-\left( x \right) \mathrm{d}x} \\
&=\int_E{g\left( x \right) \mathrm{d}x}.
\end{align*}
\end{proof}

\begin{example}
设 $f(x)$ 是 $[0,1]$ 上的可测函数, 且有
\[
\int_{[0,1]} |f(x)| \ln(1 + |f(x)|) \, \mathrm{d}x < +\infty,
\]
则 $f \in L([0,1])$.
\end{example}
\begin{proof}
为了阐明 $f \in L([0,1])$, 自然想到去寻求可积的控制函数. 题设告诉我们 $|f(x)| \ln(1 + |f(x)|)$ 是 $[0,1]$ 上的可积函数, 难道它能控制 $|f(x)|$ 吗? 显然, 这只是在 $\ln(1 + |f(x)|) \geqslant 1$ 或 $|f(x)| \geqslant \mathrm{e} - 1$ 时才行. 但注意到 $|f(x)| < \mathrm{e} - 1$ 时, 由于区间 $[0,1]$ 的测度是有限的, 故常数 $\mathrm{e} - 1$ 本身就是控制函数. 也就是说, 可在不同的定义区域寻求不同的控制函数.

为此, 作点集
\[
E_1 = \{ x \in [0,1] : |f(x)| \leqslant \mathrm{e} \}, \quad E_2 = [0,1] \setminus E_1,
\]
则我们有
\[
|f(x)| \leqslant \mathrm{e}, \quad x \in E_1;
\]
\[
|f(x)| \leqslant |f(x)| \ln(1 + |f(x)|), \quad x \in E_2.
\]
这就是说 $f \in L(E_1)$ 且 $f \in L(E_2)$, 从而
\[
f \in L(E_1 \cup E_2) = L([0,1]).
\]
\end{proof}

\begin{theorem}\label{theorem:递增可测函数列的积分和极限可交换}
设 $f \in L(E)$,$f_n \in L(E)$($n \in \mathbf{N}$)。若有
\[
\lim_{n \to \infty} f_n(x) = f(x) \ (x \in E), \quad f_n(x) \leqslant f_{n + 1}(x) \ (n \in \mathbf{N}, x \in E),
\]
则
\[
\lim_{n \to \infty} \int_E f_n(x) \, \mathrm{d}x = \int_E f(x) \, \mathrm{d}x.
\]
\end{theorem}
\begin{proof} 
令 $F_n(x) = f(x) - f_n(x)$($n \in \mathbf{N}, x \in E$),则 $\{ F_n(x) \}$ 是 $E$ 上非负渐降收敛于 $0$ 的可积函数列,从而由\hyperref[corollary:非负渐降函数列积分定理]{非负渐降函数列积分定理}可知
\begin{align*}
0 = \lim_{n \to \infty} \int_E F_n(x) \, \mathrm{d}x 
= \lim_{n \to \infty} \left( \int_E f(x) \, \mathrm{d}x - \int_E f_n(x) \, \mathrm{d}x \right) 
= \int_E f(x) \, \mathrm{d}x - \lim_{n \to \infty} \int_E f_n(x) \, \mathrm{d}x,
\end{align*}
即得所证。
\end{proof}

\begin{proposition}
设 $g \in L(E)$,$f_n \in L(E)$($n \in \mathbf{N}$)。若 $f_n(x) \geqslant g(x)$,a. e. $x \in E$,则
\[
\int_E \varliminf_{n \to \infty} f_n(x) \, \mathrm{d}x \leqslant \varliminf_{n \to \infty} \int_E f_n(x) \, \mathrm{d}x.
\]
\end{proposition}
\begin{proof}
根据\hyperref[lemma:Fatou引理]{Fatou引理},我们有
\begin{align*}
&\quad \quad \int_E{\underset{n\rightarrow \infty}{\underline{\lim }}\left[ f_n\left( x \right) -g\left( x \right) \right] \mathrm{d}x}\leqslant \underset{n\rightarrow \infty}{\underline{\lim }}\left( \int_E{\left[ f_n\left( x \right) -g\left( x \right) \right] \mathrm{d}x} \right) 
\\
&\Longleftrightarrow \int_E{\underset{n\rightarrow \infty}{\underline{\lim }}f_n\left( x \right) \mathrm{d}x}-\int_E{g_n\left( x \right) \mathrm{d}x}\leqslant \underset{n\rightarrow \infty}{\underline{\lim }}\int_E{f_n\left( x \right) \mathrm{d}x}-\int_E{g_n\left( x \right) \mathrm{d}x}
\\
&\Longleftrightarrow \int_E{\underset{n\rightarrow \infty}{\underline{\lim }}f_n\left( x \right) \mathrm{d}x}\leqslant \underset{n\rightarrow \infty}{\underline{\lim }}\int_E{f_n\left( x \right) \mathrm{d}x}.
\end{align*}
证毕.
\end{proof}

\begin{theorem}[Jensen不等式]\label{theorem:Jensen不等式}
设 $w(x)$ 是 $E \subset \mathbf{R}$ 上的正值可测函数,且
\[
\int_E w(x) \, \mathrm{d}x = 1;
\]
$\varphi(x)$ 是区间 $I = [a,b]$ 上的(下)凸函数;$f(x)$ 在 $E$ 上可测,且值域 $R(f) \subset I$。若 $fw \in L(E)$,则
\[
\varphi\left( \int_E f(x)w(x) \, \mathrm{d}x \right) \leqslant \int_E \varphi(f(x))w(x) \, \mathrm{d}x.
\]
\end{theorem}
\begin{remark}
因为$\varphi(x)$在$[a,b]$上下凸,所以由\refthe{Basis of Analytics-theorem:开区间下凸函数左右导数处处存在}可知$\varphi \in C([a,b)]$.从而由\refthe{theorem:连续函数复合可测函数也可测}可知$\varphi(f(x))$在$E$上也可测.
\end{remark}
\begin{proof}
注意到 $a \leqslant f(x) \leqslant b$,我们有
\[
a=\int_E{aw(x)\,\mathrm{d}x}\leqslant y_0=\int_E{f(x)w(x)\,\mathrm{d}x}\leqslant \int_E{bw(x)\,\mathrm{d}x}=b.
\]
故$y_0\in[a,b]$.

(i) 设 $y_0 \in (a,b)$,由 $\varphi(x)$ 之(下)凸性可知有
\[
\varphi(y) \geqslant \varphi(y_0) + k(y - y_0), \quad y \in [a,b].
\]
(其中由\refthe{Basis of Analytics-theorem:开区间下凸函数左右导数处处存在}及下凸函数的切线放缩可知$k=\varphi'_+(y_0)$)
以 $f(x)$ 代 $y$ 得
\[
\varphi(f(x)) \geqslant \varphi(y_0) + k(f(x) - y_0), \quad \text{a. e. } x \in E.
\]
在上式两端乘以 $w(x)$,并在 $E$ 上作积分,则
\begin{align*}
\int_E \varphi(f(x))w(x) \, \mathrm{d}x &\geqslant \int_E \varphi(y_0)w(x) \, \mathrm{d}x + k \int_E (f(x) - y_0)w(x) \, \mathrm{d}x \\
&= \varphi(y_0) + k \left( \int_E f(x)w(x) \, \mathrm{d}x - y_0 \right) \\
&= \varphi(y_0) = \varphi\left( \int_E f(x)w(x) \, \mathrm{d}x \right).
\end{align*}

(ii) 若 $y_0 = b$(或 $a$),易知此时有
\[
\int_E (b - f(x))w(x) \, \mathrm{d}x = 0,
\]
由\hyperref[theorem:非负可测函数积分的性质]{非负可测函数积分的性质(5)(i)}可知 $f(x) = b$,a. e. $x \in E$,从而
\begin{align*}
\int_E \varphi(f(x))w(x) \, \mathrm{d}x = \int_E \varphi(b)w(x) \, \mathrm{d}x 
= \varphi(b) \int_E w(x) \, \mathrm{d}x = \varphi(b) 
= \varphi\left( \int_E f(x)w(x) \, \mathrm{d}x \right).
\end{align*}
证毕.
\end{proof}
\begin{remark}
Jensen 不等式在 $\mathbf{R}^n$ 上也成立,只需将区间 $I$ 用凸集代替。下面是一个特例:

设 $E \subset \mathbf{R}$,且 $m(E) = 1$,$f(x)$ 在 $E$ 上正值可积,且记 $A = \int_E f(x) \, \mathrm{d}x$,则
\[
\sqrt{1 + A^2} \leqslant \int_E \sqrt{1 + f^2(x)} \, \mathrm{d}x \leqslant 1 + A.
\]
实际上,考查 $\varphi(x) = (1 + x^2)^{1/2}$,易知 $\varphi(x)$ 是(下)凸函数。根据 Jensen 不等式($w(x) \equiv 1$),有 $\left( A^2 \leqslant \int_E f^2(x) \, \mathrm{d}x \right)$.
\begin{align*}
\sqrt{1 + A^2} &\leqslant \left( 1 + \int_E f^2(x) \, \mathrm{d}x \right)^{1/2} = \left( \int_E (1 + f^2(x)) \, \mathrm{d}x \right)^{1/2} \\
&\leqslant \int_E \sqrt{1 + f^2(x)} \, \mathrm{d}x \leqslant \int_E (1 + f(x)) \, \mathrm{d}x = 1 + A.
\end{align*}
\end{remark}

\begin{theorem}[积分对定义域的可数可加性]\label{theorem:积分对定义域的可数可加性}
设 $E_k \in \mathscr{M}$($k = 1, 2, \cdots$),$E_i \cap E_j = \varnothing$($i \neq j$)。若 $f(x)$ 在 $E = \bigcup_{k=1}^{\infty} E_k$ 上可积,则
\begin{align*}
\int_E f(x) \, \mathrm{d}x = \sum_{k=1}^{\infty} \int_{E_k} f(x) \, \mathrm{d}x. 
\end{align*}
\end{theorem}
\begin{proof}
根据 $f \in L(E)$ 以及\hyperref[corollary:非负可测函数积分的可数可加性]{非负可测函数积分的可数可加性},我们有
\begin{align*}
\sum_{k=1}^{\infty} \int_{E_k} f^{\pm}(x) \, \mathrm{d}x = \int_E f^{\pm}(x) \, \mathrm{d}x \leqslant \int_E |f(x)| \, \mathrm{d}x < +\infty.
\end{align*}
从而可知
\begin{align*}
\sum_{k=1}^{\infty} \int_{E_k} f(x) \, \mathrm{d}x = \sum_{k=1}^{\infty} \left( \int_{E_k} f^+(x) \, \mathrm{d}x - \int_{E_k} f^-(x) \, \mathrm{d}x \right) 
= \int_E f^+(x) \, \mathrm{d}x - \int_E f^-(x) \, \mathrm{d}x = \int_E f(x) \, \mathrm{d}x.
\end{align*}
\end{proof}

\begin{theorem}[可积函数几乎处处为零的判别法]\label{theorem:可积函数几乎处处为零的一种判别法}
\begin{enumerate}[(1)]
\item 设 \( f(x) \) 为 \( E \) 上可测函数,且 \( \int_E |f(x)|\,\mathrm{d}x = 0 \),则 \( f(x) = 0 \),\(\text{a.e.}\, x \in E \).

\item 设函数 $f(x) \in L([a,b])$。若对任意的 $c \in [a,b]$,有$\int_{[a,c]} f(x) \, \mathrm{d}x = 0,$
则 $f(x) = 0$,a. e. $x \in [a,b]$。
\end{enumerate}
\end{theorem}
\begin{proof}
\begin{enumerate}[(1)]
\item 记 \( E_0 = \{ x \in E : |f(x)| > 0 \} \),下面证明 \( m(E_0) = 0 \). 由于 \( f \) 可测,则 \( |f| \) 可测. 令
\[
E_n = \left\{ x \in E : |f(x)| \geqslant \frac{1}{n} \right\}
\]

则 \( \{ E_n \} \) 是单调递增的可测集列,且 \( E_0 = \bigcup_{n = 1}^{\infty} E_n \). 由\hyperref[theorem:递增可测集列的测度运算]{递增可测集列的测度运算}可得
\[
m(E_0) = m\left( \bigcup_{n = 1}^{\infty} E_n \right) = m\left( \lim_{n \to \infty} E_n \right) = \lim_{n \to \infty} m(E_n).
\]
若 \( m(E_0) > 0 \),则对 \( m(E_0)/2 \),存在 \( n_0 \in \mathbb{N} \) 使得
\[
m(E_{n_0}) > m(E_0)/2.
\]
注意到,当 \( x \in E_{n_0} \) 时,有 \( |f(x)| \geqslant 1/n_0 \). 从而
\begin{align*}
\int_E |f(x)|\,\mathrm{d}x &\geqslant \int_{E_0} |f(x)|\,\mathrm{d}x \geqslant \int_{E_{n_0}} |f(x)|\,\mathrm{d}x \\
&\geqslant \int_{E_{n_0}} \frac{1}{n_0}\,\mathrm{d}x = \frac{1}{n_0} \cdot m(E_{n_0}) \\
&> \frac{m(E_0)}{2n_0} > 0
\end{align*}
矛盾,故 \( m(E_0) = 0 \). 因此,\( f(x) = 0 \),\(\text{a.e.}\, x \in E \).

\item 若结论不成立,则存在 $E \subset [a,b]$,$m(E) > 0$ 且 $f(x)$ 在 $E$ 上的值不等于零。不妨假定在 $E$ 上 $f(x) > 0$。由\refthe{theorem:定理2.13},可作闭集 $F$,$F \subset E$,且 $m(F) > 0$,并令 $G = (a,b) \setminus F$,则$G$为开集.于是由\hyperref[theorem:开集构造定理]{开集构造定理}可知,$G=\bigcup_{n=1}^{\infty}{\left( a_n,b_n \right)}$,其中 $\{ (a_n, b_n) \}$ 为开集 $G$ 的构成区间.由\hyperref[theorem:积分对定义域的可数可加性]{积分对定义域的可数可加性},我们有
\[
\int_G f(x) \, \mathrm{d}x + \int_F f(x) \, \mathrm{d}x = \int_a^b f(x) \, \mathrm{d}x = 0.
\]
因为 $\int_F f(x) \, \mathrm{d}x > 0$,所以
\[
\sum_{n \geqslant 1} \int_{[a_n, b_n]} f(x) \, \mathrm{d}x = \int_G f(x) \, \mathrm{d}x = -\int_F f(x) \, \mathrm{d}x > 0\neq 0,
\]
从而存在 $n_0$,使得
\[
\int_{[a_{n_0}, b_{n_0}]} f(x) \, \mathrm{d}x \neq 0.
\]
又由\hyperref[theorem:积分对定义域的可数可加性]{积分对定义域的可数可加性}可知
\begin{align*}
\int_{[a,b_{n_0}]}{f(x)\,\mathrm{d}x}=\int_{[a,a_{n_0}]\cup [a_{n_0},b_{n_0}]}{f(x)\,\mathrm{d}x}=\int_{[a,a_{n_0}]}{f(x)\,\mathrm{d}x}+\int_{[a_{n_0},b_{n_0}]}{f(x)\,\mathrm{d}x.}
\end{align*}
于是
\begin{align*}
\int_{[a,b_{n_0}]}{f(x)\,\mathrm{d}x}-\int_{[a,a_{n_0}]}{f(x)\,\mathrm{d}x}=\int_{[a_{n_0},b_{n_0}]}{f(x)\,\mathrm{d}x}\ne 0\Rightarrow \int_{[a,b_{n_0}]}{f(x)\,\mathrm{d}x}\ne \int_{[a,a_{n_0}]}{f(x)\,\mathrm{d}x}.
\end{align*}
由此可知
\[
\int_{[a, a_{n_0}]} f(x) \, \mathrm{d}x \neq 0 \quad \text{或} \quad \int_{[a, b_{n_0}]} f(x) \, \mathrm{d}x \neq 0.
\]
这与假设矛盾。
\end{enumerate}
\end{proof}

\begin{corollary}\label{corollary:非负可测函数积分为0则几乎处处为0}
设 \( f(x) \) 为 \( E \) 上非负可测函数,且 \( \int_E f(x)\,\mathrm{d}x = 0 \),则 \( f(x) = 0 \),\(\text{a.e.}\, x \in E \). 
\end{corollary}
\begin{proof}
由\nrefthe{theorem:可积函数几乎处处为零的一种判别法}{(1)}立得.
\end{proof}

\begin{proposition}
设 $g(x)$ 是 $E$ 上的可测函数。若对任意的 $f \in L(E)$,都有 $fg \in L(E)$,则除一个零测集 $Z$ 外,$g(x)$ 是 $E \setminus Z$ 上的有界函数。
\end{proposition}
\begin{remark}
比较\refpro{proposition:4.1例7}.
\end{remark}
\begin{proof}
如果结论不成立,那么一定存在自然数子列 $\{ k_i \}$,使得
\[
m(\{ x \in E : k_i \leqslant |g(x)| < k_{i+1} \}) = m(E_i) > 0 \quad (i = 1, 2, \cdots).
\]

现在作函数
\[
f(x) = 
\begin{cases} 
\dfrac{\text{sign}g(x)}{i^{1 + (1/2)} m(E_i)}, & x \in E_i, \\
0, & x \notin E_i 
\end{cases} \quad (i = 1, 2, \cdots).
\]
因为
\begin{align*}
\int_E |f(x)| \, \mathrm{d}x &= \sum_{i=1}^{\infty} \int_{E_i} |f(x)| \, \mathrm{d}x \\
&= \sum_{i=1}^{\infty} \frac{1}{i^{1 + (1/2)} m(E_i)} m(E_i) < +\infty,
\end{align*}
所以 $f \in L^1(E)$,但我们有
\[
\int_E f(x)g(x) \, \mathrm{d}x \geqslant \sum_{i=1}^{\infty} \frac{k_i}{i^{1 + (1/2)} m(E_i)} m(E_i) = +\infty,
\]
这说明 $fg \notin L(E)$,矛盾。
\end{proof}

\begin{theorem}[积分的绝对连续性]\label{theorem:积分的绝对连续性}
若 $f \in L(E)$,则对任给的 $\varepsilon > 0$,存在 $\delta > 0$,使得当 $E$ 中子集 $e$ 的测度 $m(e) < \delta$ 时,有
\begin{align*}
\left| \int_e f(x) \, \mathrm{d}x \right| \leqslant \int_e |f(x)| \, \mathrm{d}x < \varepsilon.
\end{align*}
\end{theorem}
\begin{proof}
不妨假定 $f(x) \geqslant 0$,否则用$|f(x)|$代替$f(x)$。根据\hyperref[theorem:简单函数逼近定理]{简单函数逼近定理}可知,存在非负简单可测函数渐升列$\{\varphi_n(x)\}$,使得$\lim_{n\to \infty}\varphi_n(x)=f(x).$
再由\hyperref[corollary:非负渐降函数列积分定理]{非负渐降函数列积分定理}可得
\begin{align*}
\underset{n\rightarrow \infty}{\lim}\int_E{\left( f\left( x \right) -\varphi _n\left( x \right) \right) \mathrm{d}x}=\,\int_E{\left( f\left( x \right) -\underset{n\rightarrow \infty}{\lim}\varphi _n\left( x \right) \right) \mathrm{d}x}=0.
\end{align*}
于是对于任给的 $\varepsilon > 0$,存在可测简单函数 $\varphi(x)$,$0 \leqslant \varphi(x) \leqslant f(x)$($x \in E$),使得
\[
\int_E (f(x) - \varphi(x)) \, \mathrm{d}x = \int_E f(x) \, \mathrm{d}x - \int_E \varphi(x) \, \mathrm{d}x < \frac{\varepsilon}{2}.
\]
现在设 $\varphi(x) \leqslant M$,取 $\delta = \varepsilon / (2M)$,则当 $e \subset E$,且 $m(e) < \delta$ 时,就有
\begin{align*}
\int_e f(x) \, \mathrm{d}x &= \int_e f(x) \, \mathrm{d}x - \int_e \varphi(x) \, \mathrm{d}x + \int_e \varphi(x) \, \mathrm{d}x \\
&\leqslant \int_E (f(x) - \varphi(x)) \, \mathrm{d}x + \int_e \varphi(x) \, \mathrm{d}x \\
&< \frac{\varepsilon}{2} + Mm(e) \leqslant \frac{\varepsilon}{2} + \frac{\varepsilon}{2} = \varepsilon.
\end{align*}
\end{proof}

\begin{corollary}
设 $f \in L(E)$($E \subset \mathbf{R}$),且
\[
0 < A = \int_E f(x) \, \mathrm{d}x < +\infty,
\]
则存在 $E$ 中可测子集 $e$,使得
\[
\int_e f(x) \, \mathrm{d}x = \frac{A}{3}.
\]
\end{corollary}
\begin{proof}
设 $E_t = E \cap (-\infty, t)$,$t \in \mathbf{R}$,并记
\[
g(t) = \int_{E_t} f(x) \, \mathrm{d}x,
\]
则由\hyperref[theorem:积分的绝对连续性]{积分的绝对连续性}可知,对任给的 $\varepsilon > 0$,存在 $\delta > 0$,只要 $|\Delta t| < \delta$,由\hyperref[theorem:积分对定义域的可数可加性]{积分对定义域的可数可加性},就有
\begin{align*}
|g(t+\Delta t)-g(t)|=\left| \int_{E\cap [t,t+\Delta t)}{f\left( x \right) \,\mathrm{d}x} \right|\leqslant \int_{E\cap [t,t+\Delta t)}{|f(x)|\,\mathrm{d}x}\leqslant \int_{[t,t+\Delta t)}{|f(x)|\,\mathrm{d}x}<\varepsilon .
\end{align*}
这说明 $g \in C(\mathbf{R})$。因为 $g(x)$ 是递增函数,且有
\[
\lim_{t\rightarrow -\infty} g(t)=g\left( -\infty \right) =\int_{\varnothing}{f\left( x \right) \mathrm{d}x}=0,\quad \lim_{t\rightarrow +\infty} g(t)=g\left( +\infty \right) =A,
\]
而 $0 < A/3 < A$,所以根据连续函数介值定理可知,存在 $t_0$:$-\infty < t_0 < +\infty$,使得 $g(t_0) = A/3$:
\[
g(t_0) = \int_{E \cap (-\infty, t_0)} f(x) \, \mathrm{d}x = \frac{A}{3}.
\]
令 $e = E \cap (-\infty, t_0)$,即得所证。
\end{proof}

\begin{theorem}[积分变量的平移变换定理]\label{theorem:积分变量的平移变换定理}
若 $f \in L(\mathbf{R}^n)$,则对任意的 $y_0 \in \mathbf{R}^n$,$f(x + y_0) \in L(\mathbf{R}^n)$,有
\begin{align*}
\int_{\mathbf{R}^n} f(x + y_0) \, \mathrm{d}x = \int_{\mathbf{R}^n} f(x) \, \mathrm{d}x.
\end{align*}
\end{theorem}
\begin{remark}
$E-\{y_0\}=\{x-y_0:x\in E\}$是\hyperref[definition:向量差集]{向量差集},不是集合的差.
\end{remark}
\begin{proof}
只需考虑 $f(x) \geqslant 0$ 的情形。首先看 $f(x)$ 是非负可测简单函数的情形:
\[
f(x) = \sum_{i=1}^k c_i \chi_{E_i}(x), \quad x \in \mathbf{R}^n.
\]
显然有
\[
f(x+y_0)=\sum_{i=1}^k{c_i\chi _{E_i}(x+y_0)}=\sum_{i=1}^k{c_i\chi _{E_i-\{y_0\}}(x)},
\]
它仍是非负可测简单函数.注意到$E-\{y_0\}=E+\{-y_0\}$,故由\hyperref[theorem:外测度的平移不变性]{外测度的平移不变性}知
\begin{align*}
\int_{\mathbf{R}^n} f(x + y_0) \, \mathrm{d}x = \sum_{i=1}^k c_i m(E_i - \{ y_0 \})
= \sum_{i=1}^k c_i m(E_i) = \int_{\mathbf{R}^n} f(x) \, \mathrm{d}x.
\end{align*}

其次,考虑一般非负可测函数 $f(x)$。此时根据\hyperref[theorem:简单函数逼近定理]{简单函数逼近定理}可知,存在非负可测简单函数渐升列 $\{ \varphi_k(x) \}$,使得 $\lim_{k \to \infty} \varphi_k(x) = f(x)$,$x \in \mathbf{R}^n$。显然,$\{ \varphi_k(x + y_0) \}$ 仍为渐升列,且有
\[
\lim_{k \to \infty} \varphi_k(x + y_0) = f(x + y_0), \quad x \in \mathbf{R}^n.
\]
从而先前的讨论及\hyperref[theorem:Beppo Levi非负渐升列积分定理]{Beppo Levi非负渐升列积分定理}可得
\begin{align*}
\int_{\mathbf{R}^n} f(x + y_0) \, \mathrm{d}x = \lim_{k \to \infty} \int_{\mathbf{R}^n} \varphi_k(x + y_0) \, \mathrm{d}x 
= \lim_{k \to \infty} \int_{\mathbf{R}^n} \varphi_k(x) \, \mathrm{d}x = \int_{\mathbf{R}^n} f(x) \, \mathrm{d}x.
\end{align*}
\end{proof}

\begin{example}
设 $f \in L([0, +\infty))$,则
\[
\lim_{n \to \infty} f(x + n) = 0, \quad \text{a. e. } x \in \mathbf{R}.
\]
\end{example}
\begin{proof}
因为 $f(x + n) = f(x + 1 + (n - 1))$,所以只需考查 $[0, 1]$ 中的点即可。为证此,又只需指出级数 $\sum_{n=1}^{\infty} |f(x + n)|$ 在 $[0, 1]$ 上几乎处处收敛即可。应用积分的手段,由于
\begin{align*}
\int_{[0, 1]} \sum_{n=1}^{\infty} |f(x + n)| \, \mathrm{d}x &= \sum_{n=1}^{\infty} \int_{[0, 1]} |f(x + n)| \, \mathrm{d}x \\
&= \sum_{n=1}^{\infty} \int_{[n, n + 1]} |f(x)| \, \mathrm{d}x = \int_{[1, \infty)} |f(x)| \, \mathrm{d}x < +\infty,
\end{align*}
可知 $\sum_{n=1}^{\infty} |f(x + n)|$ 作为 $x$ 的函数是在 $[0, 1]$ 上可积的,因而是几乎处处有限的,即级数是几乎处处收敛的。
\end{proof}

\begin{proposition}
设 $I \subset \mathbf{R}$ 是区间,$f \in L(I)$,$a \neq 0$,记 $J = \{ x / a : x \in I \}$,$g(x) = f(ax)$($x \in J$),则 $g \in L(J)$,且有
\[
\int_I f(x) \, \mathrm{d}x = |a| \int_J g(x) \, \mathrm{d}x.
\]

\end{proposition}
\begin{remark}
这只是积分变量替换的一个特殊情形.
\end{remark}
\begin{note}
对 $f \in L(\mathbf{R}^n)$,$a \in \mathbf{R} \setminus \{ 0 \}$,则
\[
\int_{\mathbf{R}^n} f(ax) \, \mathrm{d}x = \frac{1}{|a|^n} \int_{\mathbf{R}^n} f(x) \, \mathrm{d}x.
\]
\end{note}
\begin{proof}
(i) 若 $f(x) = \chi_E(x)$,$E$ 是 $I$ 中的可测集,则 $a^{-1}E \subset J$。由于 $\chi_E(ax) = \chi_{a^{-1}E}(x)$,故有
\[
\int_J g(x) \, \mathrm{d}x = \frac{1}{|a|} m(E) = \frac{1}{|a|} \int_I f(x) \, \mathrm{d}x.
\]
由此可知当 $f(x)$ 是简单可测函数时,结论也真。

(ii) 对 $f \in L(I)$,设简单可测函数列 $\{ \varphi_n(x) \}$,使得 $\varphi_n(x) \to f(x)$($n \to \infty$,$x \in I$),且 $|\varphi_n(x)| \leqslant |f(x)|$($n = 1, 2, \cdots, x \in I$),则令 $\psi_n(x) = \varphi_n(ax)$($x \in J$,$n = 1, 2, \cdots$),$\psi_n(x) \to g(x)$($n \to \infty$,$x \in J$),我们有
\begin{align*}
|a| \int_J g(x) \, \mathrm{d}x = |a| \lim_{n \to \infty} \int_J \psi_n(x) \, \mathrm{d}x = \lim_{n \to \infty} \int_I \varphi_n(x) \, \mathrm{d}x = \int_I f(x) \, \mathrm{d}x.
\end{align*}
\end{proof}



\subsection{控制收敛定理}

\begin{theorem}[控制收敛定理]\label{theorem:控制收敛定理}
设 $f_k \in L(E)$($k = 1, 2, \cdots$),且有
\begin{align}\label{eq:100.101}
\lim_{k \to \infty} f_k(x) = f(x), \quad \text{a. e. } x \in E.
\end{align}
若存在 $E$ 上的可积函数 $F(x)$,使得
\[
|f_k(x)| \leqslant F(x), \quad \text{a. e. } x \in E \ (k = 1, 2, \cdots),
\]
则
\[
\lim_{k \to \infty} \int_E f_k(x) \, \mathrm{d}x = \int_E f(x) \, \mathrm{d}x. 
\]
(通常称 $F(x)$ 为函数列 $\{ f_k(x) \}$ 的\textbf{控制函数}. )
\end{theorem}
\begin{proof}
显然,由\refcor{corollary:可测函数列的极限也可测}可知$f(x)$ 是 $E$ 上的可测函数,且由 $|f_k(x)| \leqslant F(x)$(a. e. $x \in E$)及\eqref{eq:100.101}式可知 $|f(x)| \leqslant F(x)$,a. e. $x \in E$. 因此,由\hyperref[theorem:积分的基本性质]{积分的基本性质(4)(i)}可知$f(x)$ 也是 $E$ 上的可积函数. 作函数列
\[
g_k(x) = |f_k(x) - f(x)| \quad (k = 1, 2, \cdots),
\]
则 $g_k \in L(E)$,且 $0 \leqslant g_k(x) \leqslant 2F(x)$,a. e. $x \in E$($k = 1, 2, \cdots$).注意到$\underset{k\rightarrow \infty}{\lim}g_k\left( x \right) =0$,显然$\lim_{k\to +\infty}g_k(x)$在$E$上也可积.

根据\hyperref[lemma:Fatou引理]{Fatou引理},我们有
\[
\int_E \lim_{k \to \infty} (2F(x) - g_k(x)) \, \mathrm{d}x \leqslant \varliminf_{k \to \infty} \int_E (2F(x) - g_k(x)) \, \mathrm{d}x.
\]
因为 $F(x)$,$\lim_{n\to \infty}g_n(x)$以及每个 $g_k(x)$都是可积的,所以由\hyperref[theorem:积分的线性性质]{积分的线性性质}可得
\[
\int_E 2F(x) \, \mathrm{d}x - \int_E \lim_{k \to \infty} g_k(x) \, \mathrm{d}x \leqslant \int_E 2F(x) \, \mathrm{d}x - \varlimsup_{k \to \infty} \int_E g_k(x) \, \mathrm{d}x.
\]
消去 $\int_E 2F(x) \, \mathrm{d}x$,并注意到 $\lim_{k \to \infty} g_k(x) = 0$,a. e. $x \in E$,可得
\[
\varlimsup_{k \to \infty} \int_E g_k(x) \, \mathrm{d}x = 0.
\]
又由\hyperref[theorem:积分的线性性质]{积分的线性性质}及\refthe{theorem:f与|f|的可积性等价}可知($k = 1, 2, \cdots$)
\begin{align*}
\left| \int_E f_k(x) \, \mathrm{d}x - \int_E f(x) \, \mathrm{d}x \right| = \left| \int_E (f_k(x) - f(x)) \, \mathrm{d}x \right| 
\leqslant \int_E g_k(x) \, \mathrm{d}x
\end{align*}
令$k\to \infty$,得
\begin{align*}
\underset{k\rightarrow \infty}{\overline{\lim }}\left| \int_E{f_k(x)\,\mathrm{d}x}-\int_E{f(x)\,\mathrm{d}x} \right|\leqslant \underset{k\rightarrow \infty}{\overline{\lim }}\int_E{g_k(x)\,\mathrm{d}x}=0.
\end{align*}
于是$$\underset{k\rightarrow \infty}{\lim}\left| \int_E{f_k(x)\,\mathrm{d}x}-\int_E{f(x)\,\mathrm{d}x} \right|=\underset{k\rightarrow \infty}{\overline{\lim }}\left| \int_E{f_k(x)\,\mathrm{d}x}-\int_E{f(x)\,\mathrm{d}x} \right|=0,$$
从而$\underset{k\rightarrow \infty}{\lim}\left[ \int_E{f_k(x)\,\mathrm{d}x}-\int_E{f(x)\,\mathrm{d}x} \right] =0$.因此$\underset{k\rightarrow \infty}{\lim}\int_E{f_k(x)\,\mathrm{d}x}=\int_E{f(x)\,\mathrm{d}x}.$
\end{proof}
\begin{remark}
注意(i)在上述定理的推演中,实际上证明了更强的结论:
\begin{align}
\lim_{k \to \infty} \int_E |f_k(x) - f(x)| \, \mathrm{d}x = 0. \label{eq:100.102}
\end{align}
今后,我们将称式\eqref{eq:100.102}为\(f_k(x)\)在\(E\)上依\(L^1\)的意义收敛于\(f(x)\).一般来说,$\underset{k\rightarrow \infty}{\lim}\int_E{f_k(x)\,\mathrm{d}x}=\int_E{f(x)\,\mathrm{d}x}$不能推出\eqref{eq:100.102}成立(在非负情形有例外).

此外,当式\eqref{eq:100.102}成立时,也不一定有
\[
\lim_{k \to \infty} f_k(x) = f(x), \quad \text{a. e. } x \in E.
\]
不过可以得出\(f_k(x)\)在\(E\)上依测度收敛于\(f(x)\)的结论.实际上,因为对任意的\(\sigma > 0\),记
\[
E_k(\sigma) = \{ x \in E : |f_k(x) - f(x)| > \sigma \},
\]
就有
\begin{align*}
\sigma m(E_k(\sigma)) &= \int_{E_k(\sigma)} \sigma \, \mathrm{d}x \leqslant \int_{E_k(\sigma)} |f_k(x) - f(x)| \, \mathrm{d}x \\
&\leqslant \int_E |f_k(x) - f(x)| \, \mathrm{d}x \to 0 \quad (k \to \infty),
\end{align*}
所以\(m(E_k(\sigma)) \to 0\) \((k \to \infty)\).

由此,进一步又可知,存在子列\(\{ f_{k_i}(x) \}\),使得
\[
\lim_{i \to \infty} f_{k_i}(x) = f(x), \quad \text{a. e. } x \in E.
\]

(ii)上述控制收敛定理的一个特例是有界收敛定理:

设\(\{ f_k(x) \}\)是\(E\)上的可测函数列,\(m(E) < +\infty\),且对\(x \in E\)有
\[
\lim_{k \to \infty} f_k(x) = f(x), \quad |f_k(x)| \leqslant M \quad (k = 1, 2, \cdots),
\]
则\(f \in L(E)\),且
\[
\lim_{k \to \infty} \int_E f_k(x) \, \mathrm{d}x = \int_E f(x) \, \mathrm{d}x.
\]
为阐明这一点,只需注意常数函数\(M\)就是\(E\)上的控制函数.
\end{remark}

\begin{example}
设 \( f_n \in C^{(1)}((a,b)) \)(\( n = 1,2,\cdots \)),且有
\[
\lim_{n \to \infty} f_n(x) = f(x), \quad \lim_{n \to \infty} f_n'(x) = F(x), \quad x \in (a,b).
\]
若 \( f'(x), F(x) \) 在 \( (a,b) \) 上连续,则 \( f'(x) = F(x), x \in (a,b) \).
\end{example}
\begin{proof}
只需指出在 \( (a,b) \) 的一个稠密子集上有 \( f'(x) = F(x) \) 即可.为此,任取 \( (a,b) \) 中的子区间 \( [c,d] \),且记
\[
E_n = \left\{ x \in [c,d] : |f_k'(x) - F(x)| \leqslant 1, k \geqslant n \right\},
\]
易知每个 \( E_n \) 皆闭集,且 \( [c,d] = \bigcup_{n=1}^{\infty} E_n \).从而根据 Baire 定理(定理 1.23)可知,存在 \( n_0 \) 以及区间 \( [c',d'] \),使得 \( E_{n_0} \supset [c',d'] \).由于
\[
|f_k'(x) - F(x)| \leqslant 1 \quad (k \geqslant n_0), \, x \in [c',d'],
\]
故知 \( k \geqslant n_0 \) 时,\( \{ f_k'(x) \} \) 在 \( [c',d'] \) 上一致有界.这样,由等式
\[
\int_{[c',x]} f_k'(t) dt = f_k(x) - f_k(c'), \quad c' < x < d'
\]
可知(有界收敛定理)
\[
\int_{[c',x]} F(t) dt = f(x) - f(c'), \quad c' < x < d'.
\]
在等式两端对 \( x \) 求导可得
\[
F(x) = f'(x), \quad c' < x < d',
\]
即得所证.
\end{proof}

\begin{theorem}[依测度收敛型控制收敛定理]\label{theorem:依测度收敛型控制收敛定理}
设 \( f_k \in L(\mathbb{R}^n) \)(\( k = 1,2,\cdots \)),且 \( f_k(x) \) 在 \( \mathbb{R}^n \) 上依测度收敛于 \( f(x) \).若存在 \( F \in L(\mathbb{R}^n) \),使得
\[
|f_k(x)| \leqslant F(x) \quad (k = 1,2,\cdots; \text{a. e. } x \in \mathbb{R}^n),
\]
则 \( f \in L(\mathbb{R}^n) \),且有
\[
\lim_{k \to \infty} \int_{\mathbb{R}^n} f_k(x) \mathrm{d}x = \int_{\mathbb{R}^n} f(x) \mathrm{d}x.
\]
\end{theorem}
\begin{remark}
也可由第三章的相关定理可以直接证明.
\end{remark}
\begin{proof}
(下述证法虽繁,但习之也不无益处.)设 \( \varepsilon \) 是任意给定的正数,则只需指出存在 \( K \),使得 \( k > K \) 时,有
\[
\int_{\mathbb{R}^n} |f_k(x) - f(x)| \mathrm{d}x < \varepsilon.
\]

首先,由题设知,存在 \( \{ f_{k_i}(x) \} \),使得
\[
\lim_{i \to \infty} f_{k_i}(x) = f(x), \quad \text{a. e. } x \in \mathbb{R}^n,
\]
从而在 \( |f_{k_i}(x)| \leqslant F(x) \) 中令 \( i \to \infty \),即知 \( f \in L(\mathbb{R}^n) \),且 \( |f(x)| \leqslant F(x) \),a. e. \( x \in \mathbb{R}^n \).

其次,把 \( \mathbb{R}^n \) 分解如下:

(i)由 \( F \in L(\mathbb{R}^n) \) 可知,存在 \( N \),使得
\[
\int_{\{ x : |x| \geqslant N \}} F(x) \mathrm{d}x < \frac{\varepsilon}{6}.
\]
自然同时对一切 \( k = 1,2,\cdots \),也有
\begin{align*}
\int_{\{ x : |x| \geqslant N \}} |f_k(x) - f(x)| \mathrm{d}x &\leqslant 2 \int_{\{ x : |x| \leqslant N \}} F(x) \mathrm{d}x < \frac{\varepsilon}{3}.
\end{align*}

(ii)由 \( F(x) \) 的\hyperref[theorem:积分的绝对连续性]{积分绝对连续性}可知,存在 \( \delta > 0 \),使得当 \( m(e) < \delta \) 时,有
\[
\int_{e} F(x) \mathrm{d}x < \frac{\varepsilon}{6}.
\]
自然,同时对一切 \( k = 1,2,\cdots \),也有
\[
\int_{e} |f_k(x) - f(x)| \mathrm{d}x \leqslant 2 \int_{e} F(x) \mathrm{d}x < \frac{\varepsilon}{3}.
\]

(iii)再看 \( B = B(0, N) \),记 \( m(B) = l \).由 \( f_k(x) \) 依测度收敛于 \( f(x) \) 可知,对 \( \varepsilon/(3l) \) 以及 \( \delta \),(记 \( E_k = \{ x \in B : |f_k(x) - f(x)| > \varepsilon/(3l) \} \))必存在 \( K \),当 \( k \geqslant K \) 时,有 \( m(E_k) < \delta \).

(iv)对 \( k \geqslant K \) 作分解
\begin{align*}
\int_{\mathbb{R}^n} |f_k(x) - f(x)| \mathrm{d}x &= \int_{\{ x : |x| \geqslant N \}} |f_k(x) - f(x)| \mathrm{d}x + \int_{B} |f_k(x) - f(x)| \mathrm{d}x \\
&< \frac{\varepsilon}{3} + \int_{E_k} |f_k(x) - f(x)| \mathrm{d}x + \int_{B \setminus E_k} |f_k(x) - f(x)| \mathrm{d}x \\
&< \frac{\varepsilon}{3} + \frac{\varepsilon}{3} + \int_{B \setminus E_k} \frac{\varepsilon}{3l} \mathrm{d}x \leqslant \frac{2\varepsilon}{3} + \frac{\varepsilon}{3l} \int_{B} \mathrm{d}x = \varepsilon.
\end{align*}
\end{proof}
\begin{remark}
对于 \( E \) 上依测度收敛于 \( f \in L(E) \) 的非负可积函数列 \( \{ f_k(x) \} \),若有
\[
\lim_{k \to \infty} \int_E f_k(x) \mathrm{d}x = \int_E f(x) \mathrm{d}x,
\]
则
\[
\lim_{k \to \infty} \int_E |f_k(x) - f(x)| \mathrm{d}x = 0.
\]

记 \( m_k(x) = \min_{x \in E} \{ f_k(x), f(x) \} \),\( M_k(x) = \max_{x \in E} \{ f_k(x), f(x) \} \),则对 \( \sigma > 0 \),由于 \( \{ x \in E : f(x) - m_k(x) > \sigma \} \subset \{ x \in E : |f_k(x) - f(x)| > \sigma \} \),故知 \( m_k(x) \) 在 \( E \) 上依测度收敛于 \( f(x) \).再注意到 \( 0 \leqslant m_k(x) \leqslant f(x) \)(\( x \in E \)),可得
\[
\lim_{k \to \infty} \int_E m_k(x) \mathrm{d}x = \int_E f(x) \mathrm{d}x.
\]
又因为 \( M_k(x) = f(x) + f_k(x) - m_k(x) \)(\( x \in E \)),所以有
\begin{align*}
\int_E M_k(x) \mathrm{d}x &= \int_E f(x) \mathrm{d}x + \int_E f_k(x) \mathrm{d}x - \int_E m_k(x) \mathrm{d}x \\
&\to \int_E f(x) \mathrm{d}x \quad (k \to \infty).
\end{align*}
最后,根据 \( |f_k(x) - f(x)| = M_k(x) - m_k(x) \),我们有
\[
\lim_{k \to \infty} \int_E |f_k(x) - f(x)| \mathrm{d}x = \lim_{k \to \infty} \int_E M_k(x) \mathrm{d}x - \lim_{k \to \infty} \int_E m_k(x) \mathrm{d}x = 0.
\]
\end{remark}

\begin{example}
$\int_{[0,1]} \frac{x\sin x}{1 + (nx)^\alpha} \mathrm{d}x = o\left( \frac{1}{n} \right) (n \to \infty, \alpha > 1).$
\end{example}
\begin{proof}
往证
\[
\int_{[0,1]} \frac{nx\sin x}{1 + (nx)^\alpha} \mathrm{d}x \to 0 \, (n \to \infty).
\]
令
\[
g(x) = 1 + (nx)^\alpha - nx^{3/2} \quad (g(0) = 1, g(1) = 1 + n^\alpha - n),
\]
易知,在 \(1 < \alpha \leqslant 3/2\) 且 \(n\) 充分大时,\(g(x)\) 在 \([0,1]\) 中有极值点,从而在 \(n\) 充分大时,不难得出
\[
0 < \frac{nx\sin x}{1 + (nx)^\alpha} \leqslant \frac{1}{\sqrt{x}} \quad (x \in [0,1]).
\]
根据控制收敛定理即得所证.
\end{proof}

\begin{example}
$\int_{[a, +\infty)} \frac{x\text{e}^{-n^2 x^2}}{1 + x^2} \mathrm{d}x = o\left( \frac{1}{n^2} \right) (n \to \infty, a > 0).$
\end{example}
\begin{proof}
只需指出
\[
I = \lim_{n \to \infty} \int_{[a, \infty)} \frac{n^2 x\text{e}^{-n^2 x^2}}{1 + x^2} \mathrm{d}x = 0
\]
即可.令 \(u = nx\),则
\[
I = \int_{[na, +\infty)} \frac{u\text{e}^{-u^2}}{1 + u^2 / n^2} \mathrm{d}u = \int_{[0, \infty)} \chi_{[na, +\infty)}(u) \frac{u\text{e}^{-u^2}}{1 + u^2 / n^2} \mathrm{d}u.
\]
注意到
\[
\lim_{n \to \infty} \chi_{[na, +\infty)}(u) (1 + u^2 / n^2)^{-1} u\text{e}^{-u^2} = 0,
\]
\[
0 \leqslant \chi_{[na, +\infty)}(u) (1 + u^2 / n^2)^{-1} u\text{e}^{-u^2} \leqslant u\text{e}^{-u^2} \quad (0 \leqslant u < +\infty),
\]
以及 \(u\text{e}^{-u^2}\) 在 \([0, +\infty)\) 上可积,故根据控制收敛定理即得所证.
\end{proof}

\begin{corollary}[逐项积分定理]\label{corollary:逐项积分定理}
设 \( f_k \in L(E) \)(\( k = 1,2,\cdots \)).若有
\[
\sum_{k=1}^{\infty} \int_E |f_k(x)| \mathrm{d}x < +\infty,
\]
则 \( \sum_{k=1}^{\infty} f_k(x) \) 在 \( E \) 上几乎处处收敛;若记其和函数为 \( f(x) \),则 \( f \in L(E) \),且有
\[
\sum_{k=1}^{\infty} \int_E f_k(x) \mathrm{d}x = \int_E f(x) \mathrm{d}x. 
\]
\end{corollary}
\begin{proof}
作函数 \( F(x) = \sum_{k=1}^{\infty} |f_k(x)| \),由\hyperref[theorem:非负可测函数的逐项积分定理]{非负可测函数的逐项积分定理}可知
\[
\int_E F(x) \mathrm{d}x = \sum_{k=1}^{\infty} \int_E |f_k(x)| \mathrm{d}x < +\infty,
\]
即 \( F \in L(E) \),从而 \( F(x) \) 在 \( E \) 上是几乎处处有限的.这说明级数 \( \sum_{k=1}^{\infty} f_k(x) \) 在 \( E \) 上几乎处处收敛.记其和函数为 \( f(x) \).由于
\[
|f(x)| \leqslant \sum_{k=1}^{\infty} |f_k(x)| = F(x), \quad \text{a. e. } x \in E,
\]
故 \( f \in L(E) \).

现在令 \( g_m(x) = \sum_{k=1}^{m} f_k(x) \)(\( m = 1,2,\cdots \)),则
\[
|g_m(x)| \leqslant \sum_{k=1}^{m} |f_k(x)| \leqslant F(x) \quad (m = 1,2,\cdots).
\]
于是由\hyperref[theorem:控制收敛定理]{控制收敛定理}可得
\begin{align*}
\int_E{f(x)\mathrm{d}x}=\int_E{\lim_{m\rightarrow \infty} g_m(x)\mathrm{d}x}=\lim_{m\rightarrow \infty} \int_E{g_m(x)\mathrm{d}x}=\underset{m\rightarrow \infty}{\lim}\int_E{\sum_{k=1}^m{f_k\left( x \right)}\mathrm{d}x}=\sum_{k=1}^{\infty}{\int_E{f_k(x)\mathrm{d}x}}.
\end{align*}
\end{proof}

\begin{theorem}[积分号下求导]\label{theorem:积分号下求导}
设 \( f(x,y) \) 是定义在 \( E \times (a,b) \) 上的函数,它作为 \( x \) 的函数在 \( E \) 上是可积的,作为 \( y \) 的函数在 \( (a,b) \) 上是可微的.若存在 \( F \in L(E) \),使得
\[
\left| \frac{\mathrm{d}}{\mathrm{d}y} f(x,y) \right| \leqslant F(x), \quad (x,y) \in E \times (a,b),
\]
则
\[
\frac{\mathrm{d}}{\mathrm{d}y} \int_E f(x,y) \mathrm{d}x = \int_E \frac{\mathrm{d}}{\mathrm{d}y} f(x,y) \mathrm{d}x. 
\]
\end{theorem}
\begin{proof}
任意取定 \( y \in (a,b) \) 以及 \( h_k \to 0 \)(\( k \to \infty \)),我们有
\[
\lim_{k \to \infty} \frac{f(x,y + h_k) - f(x,y)}{h_k} = \frac{\mathrm{d}}{\mathrm{d}y} f(x,y), \quad x \in E,
\]
而且当 \( k \) 充分大时,下式成立(可从微分中值定理考查):
\[
\left| \frac{f(x,y + h_k) - f(x,y)}{h_k} \right| \leqslant F(x), \quad x \in E.
\]
从而由\hyperref[theorem:控制收敛定理]{控制收敛定理}可得
\begin{align*}
\frac{\mathrm{d}}{\mathrm{d}y}\int_E{f(x,y)\mathrm{d}x}=\lim_{k\rightarrow \infty} \int_E{\frac{f(x,y+h_k)-f(x,y)}{h_k}\mathrm{d}x}=\lim_{k\rightarrow \infty} \frac{\int_E{f\left( x,y+h_k \right) \mathrm{d}x}-\int_E{f\left( x,y \right) \mathrm{d}x}}{h_k}=\int_E{\frac{\mathrm{d}}{\mathrm{d}y}f(x,y)\mathrm{d}x}.
\end{align*}
\end{proof}

\begin{example}
设 \( f(x), f_n(x) \)(\( n \in \mathbb{N} \))在 \( \mathbb{R} \) 上实值可积.若对 \( \mathbb{R} \) 中任一可测集 \( E \),有
\[
\lim_{n \to \infty} \int_E f_n(x) \mathrm{d}x = \int_E f(x) \mathrm{d}x,
\]
则 \( \varliminf_{n \to \infty} f_n(x) \leqslant f(x) \leqslant \varlimsup_{n \to \infty} f_n(x) \),a. e. \( x \in \mathbb{R} \).(由此可知,若存在子列 \( \{ f_{n_k}(x) \} \):\( \lim_{k \to \infty} f_{n_k}(x) = g(x) \),a. e. \( x \in \mathbb{R} \),则 \( f(x) = g(x) \),a. e. \( x \in \mathbb{R} \).)
\end{example}
\begin{proof}
作 \( g_n(x) = \sup_{k \geqslant n} \{ f_k(x) \} \),且令
\[
p(x) = \lim_{n \to \infty} g_n(x) = \varlimsup_{n \to \infty} f_n(x),
\]
则只需指出在 \( E(m(E) < +\infty) \) 上,有 \( f(x) \leqslant p(x) \),a. e. \( x \in E \) 即可.

(i)作点集 \( P = \{ x \in E : p(x) = -\infty \} \),\( P_n = \{ x \in E : g_n(x) < 0 \} \)(\( n \in \mathbb{N} \)),由 \( g_n(x) \) 递减收敛于 \( p(x) \),故 \( P = \bigcup_{n=1}^{\infty} P_n \).又对任意的闭集 \( F \subset P_n \),均有
\[
\int_F f(x) \mathrm{d}x = \lim_{k \to \infty} \int_F f_k(x) \mathrm{d}x \leqslant \lim_{k \to \infty} \int_F g_k(x) \mathrm{d}x = \int_F p(x) \mathrm{d}x,
\]
从而可知 \( f(x) \leqslant p(x) \),a. e. \( x \in P_n \),自然有 \( f(x) \leqslant p(x) \),a. e. \( x \in P \).注意到 \( p(x) = -\infty \)(\( x \in P \)),故 \( m(P) = 0 \).

(ii)若 \( p(x) = +\infty \),则易知 \( f(x) \leqslant p(x) \).

(iii)若 \( -\infty < p(x) < +\infty \),则令 \( Q_m = \{ x \in \mathbb{R} : -m \leqslant p(x) \leqslant m \} \),我们有 \( \mathbb{R} = \bigcup_{m=1}^{\infty} Q_m \).易知只需考查 \( Q_m \) 上 \( f(x) \) 与 \( p(x) \) 的大小.

作点集 \( S_n = \{ Q_m : g_n(x) - p(x) < 1 \} \),则 \( Q_m = \bigcup_{n=1}^{\infty} S_n \).由此又只需指出 \( f(x) \leqslant p(x) \)(\( x \in S_n \)).因为函数 \( p(x), g_n(x), g_{n+1}(x), \cdots \) 均一致有界,所以得到
\[
\lim_{n \to \infty} \int_F g_k(x) \mathrm{d}x = \int_F p(x) \mathrm{d}x \quad (F \subset S_n).
\]
注意到
\begin{align*}
\int_F f(x) \mathrm{d}x &= \lim_{k \to \infty} \int_F f_k(x) \mathrm{d}x \leqslant \lim_{k \to \infty} \int_F g_k(x) \mathrm{d}x \\
&= \int_F p(x) \mathrm{d}x \quad (F \in S_n).
\end{align*}
故得 \( f(x) \leqslant p(x) \),a. e. \( x \in S_n \).当然,此结论在 \( \mathbb{R} \) 上也真.

(iv)对于前一不等式,只需注意
\[
\varliminf_{n \to \infty} f_n(x) = -\varlimsup_{n \to \infty} (-f_n(x)).
\]
\end{proof}
\begin{remark}
设 \( f_n(x) = \mathrm{e}^{-nx} - 2\mathrm{e}^{-2nx} \)(\( n \in \mathbb{N} \)),则 \( f_n \in L([0, +\infty)) \)(\( n \in \mathbb{N} \)),但逐项积分等式不真:
\[
\int_{[0, +\infty)} \sum_{n=1}^{\infty} f_n(x) \mathrm{d}x \neq \sum_{n=1}^{\infty} \int_{[0, +\infty)} f_n(x) \mathrm{d}x.
\]
\end{remark}


































































\end{document}
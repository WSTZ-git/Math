\documentclass[../../main.tex]{subfiles}
\graphicspath{{\subfix{../../image/}}} % 指定图片目录,后续可以直接使用图片文件名。

% 例如:
% \begin{figure}[H]
% \centering
% \includegraphics[scale=0.4]{图.png}
% \caption{}
% \label{figure:图}
% \end{figure}
% 注意:上述\label{}一定要放在\caption{}之后,否则引用图片序号会只会显示??.

\begin{document}

\section{非负可测函数的积分}

\subsection{非负可测简单函数积分}

\begin{definition}
设 \(f(x)\) 是 \(\mathbb{R}^n\) 上的非负可测简单函数,它在点集 \(A_i\)(\(i = 1,2,\cdots,p\))上取值 \(c_i\)意味着:
\begin{align*}
f(x)=\sum_{i = 1}^{p}c_i\chi_{A_i}(x),\quad  \bigcup_{i = 1}^{p}A_i=\mathbb{R}^n,\quad A_i\cap A_j=\varnothing\ (i\neq j).
\end{align*}
若 \(E\in\mathscr{M}\),则定义 \(f(x)\) 在 \(E\) 上的\textbf{积分}为
\begin{align*}
\int_{E}f(x)\mathrm{d}x=\sum_{i = 1}^{p}c_im(E\cap A_i).
\end{align*}
这里积分符号下的 \(\mathrm{d}x\) 是 \(\mathbb{R}^n\) 上 Lebesgue 测度的标志.(注意,我们曾约定 \(0\cdot\infty = 0\).) 
\end{definition}
\begin{remark}
此外,由定义立即得知,$\int_{E}f(x)\mathrm{d}x$ 只与 \(f(x)\) 在 \(E\) 上的值有关. 
\end{remark}

\begin{example}
设在 \(\mathbb{R}\) 上定义函数(Dirchlet函数)
\begin{align*}
D(x)=\chi_{\mathbb{Q}}(x)=\begin{cases}
1, & x\text{ 是有理数},\\
0, & x\text{ 是无理数}.
\end{cases}
\end{align*}
我们有
\begin{align*}
\int_{(0,1)}{D(x)\mathrm{d}x}=1\cdot m\left( \left( 0,1 \right) \cap \mathbb{Q} \right) +0\cdot m\left( \left( 0,1 \right) \cap \left( \mathbb{R} \backslash \mathbb{Q} \right) \right) =0.
\end{align*} 
\end{example}

\begin{lemma}\label{lemma:简单函数的基本性质}
设 \( f(x)=\sum_{i = 1}^{n}a_i\chi_{E_i}(x), g(x)=\sum_{j = 1}^{m}b_j\chi_{F_j}(x) \) 为 \( E \) 上的简单函数,其中
\begin{align*}
E=\bigcup_{i=1}^{\mathrm{n}}{E_i}=\bigcup_{i=1}^m{F_i},\quad E_i\cap E_j=\varnothing ,\quad F_i\cap F_j=\varnothing .
\end{align*}
则它们可以表示为统一形式:
\[
f(x)=\sum_{i = 1}^{n}\sum_{j = 1}^{m}a_i\chi_{E_i\cap F_j}(x),\quad g(x)=\sum_{i = 1}^{n}\sum_{j = 1}^{m}b_j\chi_{E_i\cap F_j}(x)
\]
进而
\begin{align*}
f(x)\pm g(x)=\sum_{i = 1}^{n}\sum_{j = 1}^{m}(a_i\pm b_j)\chi_{E_i\cap F_j}(x).
\end{align*}
\end{lemma}
\begin{proof}
实际上,对于 \( E \) 的两种划分:\( E = \bigcup_{i = 1}^{n}E_i = \bigcup_{j = 1}^{m}F_j \),其中,\( E_i \) 互不相交,\( F_j \) 互不相交。我们有
\begin{gather*}
E_i\cap E=E_i\cap \bigcup_{j=1}^m{F_j}=\bigcup_{j=1}^m{\left( E_i\cap F_j \right)},i=1,2,\cdots ,n.
\\
E\cap F_j=\left( \bigcup_{i=1}^m{E_i} \right) \cap F_j=\bigcup_{i=1}^m{\left( E_i\cap F_j \right) ,}j=1,2,\cdots ,m.
\end{gather*}
显然\( (E_i\cap F_j)(i=1,2,\cdots,n,j=1,2,\cdots,m) \) 互不相交.从而
\begin{gather*}
f\left( x \right) =\sum_{i=1}^n{a_i\chi _{E_i}\left( x \right)}=\sum_{i=1}^n{a_i\chi _{\bigcup_{j=1}^m{\left( E_i\cap F_j \right)}}\left( x \right)}=\sum_{i=1}^n{a_i\sum_{j=1}^m{\chi _{E_i\cap F_j}\left( x \right)}}=\sum_{i=1}^n{\sum_{j=1}^m{a_i\chi _{E_i\cap F_j}\left( x \right)}}.
\\
g\left( x \right) =\sum_{j=1}^m{b_j\chi _{F_j}\left( x \right)}=\sum_{j=1}^m{b_j\chi _{\bigcup_{i=1}^m{\left( E_i\cap F_j \right)}}\left( x \right)}=\sum_{j=1}^m{b_j\sum_{i=1}^n{\chi _{E_i\cap F_j}\left( x \right)}}=\sum_{j=1}^m{\sum_{i=1}^n{b_j\chi _{E_i\cap F_j}\left( x \right)}}.
\end{gather*}
\end{proof}

\begin{theorem}[积分的线性性质]\label{theorem:简单函数积分的线性性质}
设 \(f(x),g(x)\) 是 \(\mathbb{R}^n\) 上的非负可测简单函数,\(f(x)\) 在点集 \(A_i\)(\(i = 1,2,\cdots,p\))上取值 \(a_i\)(\(i = 1,2,\cdots,p\)),\(g(x)\) 在点集 \(B_j\)(\(j = 1,2,\cdots,q\))上取值 \(b_j\)(\(j = 1,2,\cdots,q\)),\(E\in\mathscr{M}\),则有
\begin{enumerate}[(i)]
\item 若 \(C\) 是非负常数,则
\begin{align*}
\int_{E}Cf(x)\mathrm{d}x = C\int_{E}f(x)\mathrm{d}x;
\end{align*}
\item 
\begin{align*}
\int_{E}(f(x)+g(x))\mathrm{d}x=\int_{E}f(x)\mathrm{d}x+\int_{E}g(x)\mathrm{d}x.
\end{align*}
\end{enumerate}
\end{theorem}
\begin{proof}
(i) 可从定义直接得出.

(ii) 由\reflem{lemma:简单函数的基本性质}可知\(f(x)+g(x)\) 在 \(A_i\cap B_j\)(假定非空)上取值 \(a_i + b_j\),故有
\begin{align*}
\int_{E}(f(x)+g(x))\mathrm{d}x&=\sum_{i = 1}^{p}\sum_{j = 1}^{q}(a_i + b_j)m(E\cap A_i\cap B_j)\\
&=\sum_{i = 1}^{p}a_i\sum_{j = 1}^{q}m(E\cap A_i\cap B_j)+\sum_{j = 1}^{q}b_j\sum_{i = 1}^{p}m(E\cap A_i\cap B_j)\\
&=\sum_{i = 1}^{p}a_im(E\cap A_i)+\sum_{j = 1}^{q}b_jm(E\cap B_j)\\
&=\int_{E}f(x)\mathrm{d}x+\int_{E}g(x)\mathrm{d}x.
\end{align*} 
\end{proof}

\begin{theorem}\label{theorem:定理4.2}
若 \(\{E_k\}\) 是 \(\mathbb{R}^n\) 中的递增可测集列,\(f(x)\) 是 \(\mathbb{R}^n\) 上的非负可测简单函数,则
\begin{align*}
\int_{E}f(x)\mathrm{d}x=\lim_{k\rightarrow\infty}\int_{E_k}f(x)\mathrm{d}x,\quad E = \bigcup_{k = 1}^{\infty}E_k.
\end{align*}
\end{theorem}
\begin{proof}
设 \(f(x)\) 在 \(A_i\)(\(i = 1,2,\cdots,p\))上取值 \(c_i\)(\(i = 1,2,\cdots,p\)),则
\begin{align*}
\lim_{k\rightarrow \infty} \int_{E_k}{f(x)\mathrm{d}x}&=\lim_{k\rightarrow \infty} \sum_{i=1}^p{c_im\left( E_k\cap A_i \right)}=\sum_{i=1}^p{c_i\lim_{k\rightarrow \infty} m\left( E_k\cap A_i \right)}
\\
&\xlongequal{\text{\hyperref[theorem:递增可测集列的测度运算]{递增可测集列的测度运算}}}\sum_{i=1}^p{c_im\left( \lim_{k\rightarrow \infty} \left( E_k\cap A_i \right) \right)}=\sum_{i=1}^p{c_im\left( \bigcup_{k=1}^{\infty}{E_k\cap A_i} \right)}
\\
&=\sum_{i=1}^p{c_im\left( E\cap A_i \right)}=\int_E{f(x)\mathrm{d}x}.
\end{align*} 
\end{proof}



\subsection{非负可测函数的积分}

\begin{definition}
设 \(f(x)\) 是 \(E\subset\mathbb{R}^n\) 上的非负可测函数,定义 \(f(x)\) 在 \(E\) 上的积分为
\begin{align*}
&\int_{E}f(x)\mathrm{d}x=\sup_{\substack{h(x)\leqslant  f(x)\\x\in E}}\left\{\int_{E}h(x)\mathrm{d}x:h(x)\text{ 是 }\mathbb{R}^n\text{ 上的非负可测简单函数}\right\}\\
&=\sup\left\{ \int_E{h\left( x \right) \mathrm{d}x}:h\left( x \right) \text{是}\mathbb{R} ^n\text{上的非负可测简单函数且}h\left( x \right) \leqslant slant f\left( x \right) \left( x\in E \right) \right\},
\end{align*}
这里的积分可以是 \(+\infty\);若 \(\int_{E}f(x)\mathrm{d}x< +\infty\),则称 \(f(x)\) 在 \(E\) 上是\textbf{可积的},或称 \(f(x)\) 是 \(E\) 上的\textbf{可积函数}. 
\end{definition}

\begin{theorem}[非负可测函数积分的性质]\label{theorem:非负可测函数积分的性质}
\begin{enumerate}[(1)]
\item 设 \(f(x),g(x)\) 是 \(E\) 上的非负可测函数. 若 \(f(x)\leqslant slant g(x)\), a.e. \(x\in E\),则
\begin{align*}
\int_{E}f(x)\mathrm{d}x\leqslant slant \int_{E}g(x)\mathrm{d}x.
\end{align*}

\item 设$f(x)$在$E$上非负可测,我们有
\begin{enumerate}[(i)]
\item 若存在 \(E\) 上非负可积函数 \(F(x)\),使得
\[f(x)\leqslant slant F(x),\quad x\in E,\]
则 \(f(x)\) 在 \(E\) 上可积.

\item 若 \(f(x)\) 在 \(E\) 上有界,且 \(m(E)< +\infty\),则 \(f(x)\) 在 \(E\) 上可积.
\end{enumerate}

\item 若 \(f(x)\) 是 \(E\) 上的非负可测函数,\(A\) 是 \(E\) 中可测子集,则
\begin{align*}
\int_{A}f(x)\mathrm{d}x=\int_{E}f(x)\chi_{A}(x)\mathrm{d}x.
\end{align*}

\item 设$f(x)$是$E$上的非负可测函数,若$F\subset E$且$F$可测,则
\begin{align*}
\int_F{f\left( x \right) \mathrm{d}x}\leqslant slant \int_E{f\left( x \right) \mathrm{d}x}.
\end{align*}

\item \begin{enumerate}[(i)]
\item \(f(x)\) 在 \(E\) 上几乎处处等于零的充要条件是\(\int_{E}f(x)\mathrm{d}x = 0\).

\item 若 \(m(E)=0\),则 \(\int_{E}f(x)\mathrm{d}x = 0\).
\end{enumerate}
\end{enumerate}
\end{theorem}
\begin{remark}
\hypertarget{不妨设简单函数列保持不变的原因}{不妨设}$f_n(x)\leqslant slant g_n(x)$,a.e. $x\in E$的原因:因为
\begin{align}
\lim_{n\rightarrow \infty}f_n(x) = f(x),\quad \lim_{n\rightarrow \infty}g_n(x) = g(x),\quad \forall x\in E.\label{100.100}
\end{align}
并且$f(x) \leqslant slant g(x)$,a.e.$x\in E$,所以$f(x) \leqslant slant \lim_{n\rightarrow \infty}g_n(x)$,a.e.$x\in E$.从而由极限的保号性可知,存在$N_1\in \mathbb{N}$,使得对$\forall n_1\geqslant slant N_1$,都有
\begin{align*}
f(x) \leqslant slant g_{n_1}(x),\quad \text{a.e.}\,x\in E.
\end{align*}
对$\forall n_1\geqslant slant N_1$,由\eqref{100.100}式可知$\lim_{n\rightarrow \infty}f_n(x) \leqslant slant g_{n_1}(x)$,a.e.$x\in E$,于是再根据极限的保号性可知,存在$N_2\geqslant slant n_1$且$N_2\in \mathbb{N}$,使得
\begin{align*}
f_{N_2}(x) \leqslant slant g_{n_1}(x),\quad \text{a.e.}\,x\in E.
\end{align*}
又$\{ f_n \}$是单调递增函数列,故
\begin{align*}
f_{n_1}(x) \leqslant slant f_{n_1+1}(x) \leqslant slant \cdots \leqslant slant f_{N_2}(x) \leqslant slant g_{n_1}(x),\quad \text{a.e.}\,x\in E.
\end{align*}
因此再由$n_1$的任意性可知,对$\forall n\geqslant slant N_1$,都有
\begin{align*}
f_n(x) \leqslant slant g_n(x),\quad \text{a.e.}\,x\in E.
\end{align*}
故只需去掉$\{ f_n \}$,$\{ g_n \}$的前$N_1$项即可.新的函数列$\{ f_n \}$,$\{ g_n \}$满足非负可测递增,且
\begin{align*}
\lim_{n\rightarrow \infty}f_n(x) = f(x),\quad \lim_{n\rightarrow \infty}g_n(x) = g(x),\quad \forall x\in E.
\end{align*}
\begin{align*}
f_n(x) \leqslant slant g_n(x),\quad \text{a.e.}\,x\in E.
\end{align*}
\end{remark}
\begin{proof}
\begin{enumerate}[(1)]
\item 由\hyperref[theorem:简单函数逼近定理]{简单函数逼近定理}可知,存在非负可测简单函数渐升列$\{ f_n \}$和$\{ g_n \}$,使得
\begin{align*}
\lim_{n\rightarrow \infty}f_n(x) = f(x),\quad \lim_{n\rightarrow \infty}g_n(x) = g(x),\quad \forall x\in E.
\end{align*}
由于$f(x) \leqslant slant g(x)$,a.e.$x\in E$,故可\hyperlink{不妨设简单函数列保持不变的原因}{不妨设$f_n(x) \leqslant slant g_n(x)$,a.e.$x\in E$}.于是再结合\refthe{thoerem:定理4.2}可得
\begin{align*}
\int_E f(x) \mathrm{d}x = \int_E \lim_{n\rightarrow \infty}f_n(x) \mathrm{d}x = \lim_{n\rightarrow \infty}\int_E f_n(x) \mathrm{d}x
\leqslant slant \lim_{n\rightarrow \infty}\int_E g_n(x) \mathrm{d}x = \int_E \lim_{n\rightarrow \infty}g_n(x) \mathrm{d}x = \int_E g(x) \mathrm{d}x.
\end{align*}
故结论成立.

\item \begin{enumerate}[(i)]
\item 由 \(F(x)\) 在 \(E\) 上可积及 (1) 可知
\begin{align*}
\int_E{f(x)\mathrm{d}x}\leqslant slant \int_E{F(x)\mathrm{d}x}<+\infty,
\end{align*}
故 \(f(x)\) 在 \(E\) 上可积.

\item 由 \(f(x)\) 在 \(E\) 上有界可知,存在 \(M>0\),使得
\begin{align*}
f(x)\leqslant slant M,\quad \forall x\in E.
\end{align*}
从而由 (1) 可得
\begin{align*}
\int_E{f(x)\mathrm{d}x}\leqslant slant \int_E{M\mathrm{d}x}.
\end{align*}
由于常值函数也是简单函数,故
\begin{align*}
\int_E{M\mathrm{d}x}=M\cdot m(E)<+\infty.
\end{align*}
因此
\begin{align*}
\int_E{f(x)\mathrm{d}x}\leqslant slant \int_E{M\mathrm{d}x}<+\infty.
\end{align*}
故 \(f(x)\) 在 \(E\) 上可积.
\end{enumerate}

\item 设$h(x)$表示$\mathbb{R}^n$上的非负可测简单函数.任取$\int_A h_0(x)\,\mathrm{d}x \in \{\int_A h(x)\,\mathrm{d}x : h(x) \leqslant slant f(x), x \in A\}$,则$h_0(x)$为$A$上的非负可测简单函数,且
$h_0(x) \leqslant slant f(x) (x\in A)$。
令
\[
h_1(x) = 
\begin{cases} 
h_0(x), & x \in A, \\
0, & x \in E\backslash A,
\end{cases}
\]
显然$h_1(x)$是$E$上的非负简单可测函数,且
$h_1(x) \chi_A(x) = h_0(x) \leqslant slant f(x) \chi_A(x), x \in E$。
设$h_0(x)$在点集$A_i (i = 1,2,\cdots,p)$上的取值为$c_i (i = 1,2,\cdots,p)$,则
\[
\int_E h_1(x)\,\mathrm{d}x = \sum_{i=1}^p c_i m(E \cap A_i) + 0 \cdot m(E\backslash A) = \sum_{i=1}^p c_i m(E \cap A_i) = \int_A h_0(x)\,\mathrm{d}x。
\]
因此$\int_A h_0(x)\,\mathrm{d}x \in \{\int_A h(x)\,\mathrm{d}x : h(x) \chi_A(x) \leqslant slant f(x) \chi_A(x), x \in E\}$,故
\[
\{\int_A h(x)\,\mathrm{d}x : h(x) \leqslant slant f(x), x \in A\} \subset \{\int_A h(x)\,\mathrm{d}x : h(x) \chi_A(x) \leqslant slant f(x) \chi_A(x), x \in E\}。
\]

任取$\int_A h_0(x)\,\mathrm{d}x \in \{\int_A h(x)\,\mathrm{d}x : h(x) \chi_A(x) \leqslant slant f(x) \chi_A(x), x \in E\}$,则$h_0(x)$为$E$上的非负可测简单函数且
$h_0(x) \chi_A(x) \leqslant slant f(x) \chi_A(x), x \in E$。
令$h_1(x) = h_0(x), x \in A$,显然$h_1(x)$是$A$上的非负可测简单函数,且
$h_1(x) = h_0(x) \chi_A(x) \leqslant slant f(x) \chi_A(x) = f(x), x \in A$。
由$h_1(x) = h_0(x), x \in A$可知$h_1(x) \leqslant slant h_0(x) \leqslant slant h_1(x), x \in A$。从而由$(1)$可得
\[
\int_A h_1(x)\,\mathrm{d}x \leqslant slant \int_A h_0(x)\,\mathrm{d}x \leqslant slant \int_A h_1(x)\,\mathrm{d}x \Rightarrow \int_A h_0(x)\,\mathrm{d}x = \int_A h_1(x)\,\mathrm{d}x。
\]
因此$\int_A h_0(x)\,\mathrm{d}x \in \{\int_A h(x)\,\mathrm{d}x : h(x) \leqslant slant f(x), x \in A\}$,故
\[
\{\int_A h(x)\,\mathrm{d}x : h(x) \leqslant slant f(x), x \in A\} \supset \{\int_A h(x)\,\mathrm{d}x : h(x) \chi_A(x) \leqslant slant f(x) \chi_A(x), x \in E\}。
\]

综上可得
\begin{align}
\{\int_A h(x)\,\mathrm{d}x : h(x) \leqslant slant f(x), x \in A\} = \{\int_A h(x)\,\mathrm{d}x : h(x) \chi_A(x) \leqslant slant f(x) \chi_A(x), x \in E\}\label{equation:100.79}
\end{align}

任取$\int_A h_0(x)\,\mathrm{d}x \in \{\int_A h(x)\,\mathrm{d}x : h(x) \chi_A(x) \leqslant slant f(x) \chi_A(x), x \in E\}$,则$h_0(x)$为$E$上的非负可测简单函数,且
$h_0(x) \chi_A(x) \leqslant slant f(x) \chi_A(x), x \in E$。
令$h_1(x) = h_0(x) \chi_A(x), x \in E$,显然$h_1(x)$是$E$上的非负可测简单函数,且
$h_1(x) = h_0(x) \chi_A(x) \leqslant slant f(x) \chi_A(x), x \in E$。
设$h_0(x)$在点集$E_i (i = 1,2,\cdots,p)$上的取值为$c_i (i = 1,2,\cdots,p)$,则
\[
\bigcup_{i=1}^p (E_i\cap A) = (\bigcup_{i=1}^p E_i) \cap A = E\cap A = A,\quad (E_i\cap A) \cap (E_j\cap A) = \varnothing (i\ne j)。
\]
从而
\begin{align*}
\int_E h_1(x)\,\mathrm{d}x &= \int_E h_0(x) \chi_A(x)\,\mathrm{d}x = \int_E \sum_{i=1}^p c_i (\chi_{E_i}(x) \cdot \chi_A(x))\,\mathrm{d}x \\
&= \int_E \left[ \sum_{i=1}^p c_i \chi_{E_i\cap A}(x) + 0 \cdot \chi_{E\backslash A}(x) \right]\,\mathrm{d}x = \sum_{i=1}^p c_i m(E_i\cap A) + 0 \cdot m(E\backslash A) \\
&= \sum_{i=1}^p c_i m(E_i\cap A) = \int_A h_0(x)\,\mathrm{d}x。
\end{align*}
因此$\int_A h_0(x)\,\mathrm{d}x \in \{\int_E h(x)\,\mathrm{d}x : h(x) \leqslant slant f(x) \chi_A(x), x \in E\}$,故
\[
\{\int_A h(x)\,\mathrm{d}x : h(x) \chi_A(x) \leqslant slant f(x) \chi_A(x), x \in E\} \subset \{\int_E h(x)\,\mathrm{d}x : h(x) \leqslant slant f(x) \chi_A(x), x \in E\}。
\]

任取$\int_E h_0(x)\,\mathrm{d}x \in \{\int_E h(x)\,\mathrm{d}x : h(x) \leqslant slant f(x) \chi_A(x), x \in E\}$,则$h_0(x)$为$E$上的非负可测简单函数,且
\begin{align}
h_0(x) \leqslant slant f(x) \chi_A(x), x \in E。\label{100.81}
\end{align}
令$h_1(x) = h_0(x), x \in E$,显然$h_1(x)$是$E$上的非负可测简单函数,且
\begin{align*}
h_1(x) \chi_A(x) = h_0(x) \chi_A(x) \leqslant slant f(x) \chi_A(x) \cdot \chi_A(x) = f(x) \chi_{A\cap A}(x) = f(x) \chi_A(x), x \in E.
\end{align*}
又由\eqref{100.81}式可知
$h_0(x) = 0, x \in E\backslash A$。
于是可设$h_0(x)$在点集$E_i (i = 1,2,\cdots,p-1)$上取值为$c_i\ne 0 (i = 1,2,\cdots,p-1)$,在$E_p$上取值为$0$,则$E\backslash A\subset E_p$,从而
\begin{align*}
E_i\subset \bigcup_{i=1}^{p-1} E_i\subset A, i = 1,2,\cdots,p-1.
\end{align*}
进而$h_0(x) = \sum_{i=1}^{p-1} c_i \chi_{E_i}(x)$。于是
\[
\int_A h_1(x)\,\mathrm{d}x = \int_A h_0(x)\,\mathrm{d}x = \sum_{i=1}^{p-1} c_i m(A\cap E_i) = \sum_{i=1}^{p-1} c_i m(E_i) = \int_E h_0(x)\,\mathrm{d}x。
\]
因此$\int_E h_0(x)\,\mathrm{d}x \in \{\int_A h(x)\,\mathrm{d}x : h(x) \chi_A(x) \leqslant slant f(x) \chi_A(x), x \in E\}$,故
\[
\{\int_A h(x)\,\mathrm{d}x : h(x) \chi_A(x) \leqslant slant f(x) \chi_A(x), x \in E\} \supset \{\int_E h(x)\,\mathrm{d}x : h(x) \leqslant slant f(x) \chi_A(x), x \in E\}。
\]
综上可得
\begin{align}
\{\int_A h(x)\,\mathrm{d}x : h(x) \chi_A(x) \leqslant slant f(x) \chi_A(x), x \in E\} = \{\int_E h(x)\,\mathrm{d}x : h(x) \leqslant slant f(x) \chi_A(x), x \in E\}。\label{equation:100.80}
\end{align}
综上所述,由\eqref{equation:100.80}\eqref{equation:100.79}式,我们有
\begin{align*}
\int_{A}f(x)\mathrm{d}x=\sup_{\substack{h(x)\leqslant  f(x)\\x\in A}}\left\{\int_{A}h(x)\mathrm{d}x\right\}
=\sup_{\substack{h(x)\chi_{A}(x)\leqslant  f(x)\chi_{A}(x)\\x\in E}}\left\{\int_{A}h(x)\mathrm{d}x\right\}=\int_{E}f(x)\chi_{A}(x)\mathrm{d}x.
\end{align*}

\item 由(1)和(3)立得.

\item \begin{enumerate}[(i)]
\item {\heiti 必要性:}由必要性假设可知,存在零测集$Z\subset E$,使得
\begin{align*}
f(x)=0,x\in E\backslash Z.
\end{align*}
设$h(x)$为$E$上的非负可测简单函数,且$h(x)\leqslant slant f(x),x\in E$.从而
\begin{align*}
h(x)=0,x\in E\backslash Z.
\end{align*}
于是可设$h(x)$在点集$E_i (i = 1,2,\cdots,p-1)$上取值为$c_i\ne 0 (i = 1,2,\cdots,p-1)$,在$E_p$上取值为$0$,即
\[
h(x) = \sum_{i=1}^{p-1} c_i \chi_{E_i}(x),\quad E = \bigcup_{i=1}^p E_i,\quad E_i \cap E_j = \varnothing (i \ne j),\quad c_i \ne c_j (i \ne j)。
\]
从而
\[
E\backslash Z \subset E_p,\quad E_i \subset \bigcup_{i=1}^{p-1} E_i \subset Z (i = 1,2,\cdots,p-1)。
\]
又$Z$为零测集,故$m(E_i) = 0 (i = 1,2,\cdots,p-1)$。于是
\[
\int_E h(x)\,\mathrm{d}x = \sum_{i=1}^{p-1} c_i m(E_i) + 0 \cdot m(E_p) = 0。
\]
故
\[
\int_E f(x)\,\mathrm{d}x = \sup_{\substack{h(x) \leqslant slant f(x) \\ x \in E}} \left\{ \int_E h(x)\,\mathrm{d}x \right\} = 0。
\]

{\heiti 充分性:}记
\[E_k=\{x\in E:f(x)>1/k\},\]
由(1)和(4)可得
\[\frac{1}{k}m(E_k)=\int_{E_k}\frac{1}{k}\mathrm{d}x\leqslant \int_{E_k}f(x)\mathrm{d}x\leqslant \int_{E}f(x)\mathrm{d}x = 0,\]
故知 \(m(E_k)=0\)(\(k = 1,2,\cdots\)). 注意到
\[\{x\in E:f(x)>0\}=\bigcup_{k = 1}^{\infty}E_k,\]
立即得出 \(m(\{x\in E:f(x)>0\}) = 0\).

\item 设$h(x)$为$E$上的非负可测简单函数,且$h(x) \leqslant slant f(x), x\in E$。从而可设$h(x)$在点集$E_i (i = 1,2,\cdots,p)$上取值为$c_i (i = 1,2,\cdots,p)$,即
\[
h(x) = \sum_{i=1}^p c_i \chi_{E_i}(x),\quad E = \bigcup_{i=1}^p E_i,\quad E_i \cap E_j = \varnothing (i \ne j),\quad c_i \ne c_j (i \ne j)。
\]
注意到
\[
E_i \subset \bigcup_{i=1}^p E_i = E (i = 1,2,\cdots,p),
\]
又$E$为零测集,故$m(E_i) = 0 (i = 1,2,\cdots,p)$。于是
\[
\int_E h(x)\,\mathrm{d}x = \sum_{i=1}^p c_i m(E_i) = 0。
\]
故
\[
\int_E f(x)\,\mathrm{d}x = \sup_{\substack{h(x) \leqslant slant f(x) \\ x \in E}} \left\{ \int_E h(x)\,\mathrm{d}x \right\} = 0。
\]
\end{enumerate}
\end{enumerate} 
\end{proof}

\begin{theorem}\label{theorem:非负可积函数必几乎处处有限}
若$f(x)$是$E$上的非负可积函数,则$f(x)$在$E$上是几乎处处有限的。
\end{theorem}
\begin{proof}
令$E_k = \{x \in E: f(x) > k\}$,则由$f(x)$在$E$上可测可知$\{E_k\}$是递减可测集列,且
\[
\{x \in E: f(x) = +\infty\} = \bigcap_{k = 1}^{\infty} E_k.
\]
对于每个$k$,可得
\[
km(E_k) \leqslant slant \int_{E_k} f(x) \, \mathrm{d}x \leqslant slant \int_{E} f(x) \, \mathrm{d}x < +\infty,
\]
从而
\begin{align*}
m(E_k)\leqslant slant \frac{\int_{E} f(x)\, \mathrm{d}x}{k},\forall k\in \mathbb{N}.
\end{align*}
令$k\to \infty$得$\lim_{k \to \infty} m(E_k) = 0$。于是
\begin{align*}
&m(\{x\in E:f(x)=+\infty \}=m\left( \bigcap_{k=1}^{\infty}{E_k} \right) =m\left( \underset{k\rightarrow \infty}{\lim}E_k \right) 
\\
&\xlongequal{\text{\refpro{proposition:单调集列的上下限集相等都等于其极限集}}}m\left( \underset{k\rightarrow \infty}{\underline{\lim }}E_k \right) \overset{\text{\nrefthe{theorem:可测集的Fatou引理}{(2)}}}{\leqslant slant}\underset{k\rightarrow \infty}{\underline{\lim }}m\left( E_k \right) 
=\underset{k\rightarrow \infty}{\lim}m\left( E_k \right) =0.
\end{align*}
故$m(\{x\in E:\{f(x)\}=+\infty \}=0,$即$f(x)$在$E$上几乎处处有限.
\end{proof}

\begin{theorem}[Beppo Levi非负渐升列积分定理]\label{theorem:Beppo Levi非负渐升列积分定理}
设有定义在$E$上的非负可测函数渐升列:
\[
f_1(x) \leqslant slant f_2(x) \leqslant slant \cdots \leqslant slant f_k(x) \leqslant slant \cdots,
\]
且有$\lim_{k \to \infty} f_k(x) = f(x), x \in E$,则
\begin{align*}
\lim_{k \to \infty} \int_E f_k(x) \, \mathrm{d}x = \int_E f(x) \, \mathrm{d}x. 
\end{align*}
\end{theorem}
\begin{note}
这个\hyperref[theorem:Beppo Levi非负渐升列积分定理]{Beppo Levi非负渐升列积分定理}表明,对于非负可测函数渐升列来说,极限与积分的次序可以交换,即
\begin{align*}
\int_E \lim_{n \to \infty} f_n(x) \mathrm{d}\mu = \lim_{n \to \infty} \int_E f_n(x) \mathrm{d}\mu .
\end{align*}
此外,由\hyperref[theorem:简单函数逼近定理]{简单函数逼近定理}可知,非负可测函数是非负可测简单函数渐升列的极限,因而使得积分理论中的许多结果可直接从可测简单函数的积分性质得到.
\end{note}
\begin{proof}
由函数列$\{f_k(x)\}$的渐升性和$\lim_{k \to \infty} f_k(x) = f(x), x \in E$可知,$f(x)$是$E$上的非负可测函数,从而积分$\int_E f(x) \, \mathrm{d}x$有定义。由函数列$\{f_k(x)\}$的渐升性及\nrefthe{theorem:非负可测函数积分的性质}{(2)}可知
\[
\int_E f_k(x) \, \mathrm{d}x \leqslant slant \int_E f_{k + 1}(x) \, \mathrm{d}x \quad (k = 1,2,\cdots),
\]
所以根据单调有界定理可知$\lim_{k \to \infty} \int_E f_k(x) \, \mathrm{d}x$有定义,而且从函数列的渐升性可知
\begin{align}\label{equation:100.90}
\lim_{k \to \infty} \int_E f_k(x) \, \mathrm{d}x \leqslant slant \int_E f(x) \, \mathrm{d}x.
\end{align}
现在令$c$满足$0 < c < 1$,$h(x)$是$\mathbf{R}^n$上的任一非负可测简单函数,且$h(x) \leqslant slant f(x), x \in E$。记
\[
E_k = \{x \in E: f_k(x) \geqslant slant ch(x)\} \quad (k = 1,2,\cdots),
\]
则$\{E_k\}$是递增可测集列,且$\lim_{k \to \infty} E_k = \{x\in E:f(x)\geqslant slant ch(x)\}=E$。根据\refthe{theorem:定理4.2}可知
\[
\lim_{k \to \infty} c \int_{E_k} h(x) \, \mathrm{d}x = c \int_E h(x) \, \mathrm{d}x,
\]
于是从不等式
\[
\int_E f_k(x) \, \mathrm{d}x \geqslant slant \int_{E_k} f_k(x) \, \mathrm{d}x \geqslant slant \int_{E_k} ch(x) \, \mathrm{d}x = c \int_{E_k} h(x) \, \mathrm{d}x
\]
得到
\[
\lim_{k \to \infty} \int_E f_k(x) \, \mathrm{d}x \geqslant slant c \int_E h(x) \, \mathrm{d}x.
\]
在上式中令$c \to 1$,有
\[
\lim_{k \to \infty} \int_E f_k(x) \, \mathrm{d}x \geqslant slant \int_E h(x) \, \mathrm{d}x.
\]
依$f(x)$的积分定义即知
\begin{align}\label{equation:100.91}
\lim_{k \to \infty} \int_E f_k(x) \, \mathrm{d}x \geqslant slant \int_E f(x) \, \mathrm{d}x.
\end{align}
综上,由\eqref{equation:100.90}\eqref{equation:100.91}式可知结论成立.
\end{proof}

\begin{corollary}[非负渐降函数列积分定理]\label{corollary:非负渐降函数列积分定理}
设 \( \{ f_k(x) \} \) 是 \( E \) 上的非负可积函数渐降列,且有
\[
\lim_{k \to \infty} f_k(x) = f(x),\quad \text{a. e. } x \in E,
\]
则
\[
\lim_{k \to \infty} \int_E f_k(x) \, \mathrm{d}x = \int_E f(x) \, \mathrm{d}x.
\]
\end{corollary}
\begin{proof}
因为\( 0 \leqslant slant f(x) \leqslant slant f_1(x) \),所以由\nrefthe{theorem:非负可测函数积分的性质}{(2)(i)}可知\( f(x) \) 在 \( E \) 上可积. 记
\[
g_k(x) = f_1(x) - f_k(x) \quad (k = 1, 2, \cdots),
\]
则 \( \{ g_k(x) \} \) 是非负可积函数渐升列. 从而由\hyperref[theorem:Beppo Levi非负渐升列积分定理]{Beppo Levi非负渐升列积分定理}可得
\begin{align}
&\lim_{k\rightarrow \infty} \int_E{(f_1(x)}-f_k(x))\,\mathrm{d}x=\lim_{k\rightarrow \infty} \int_E{g_k(x)\,\mathrm{d}x}=\int_E{\lim_{k\rightarrow \infty} g_k(x)\,\mathrm{d}x} \nonumber
\\
&\xlongequal{\text{\refpro{theorem:对等的函数在相同可测集下的积分相等}}}\int_E{(f_1(x)}-f(x))\,\mathrm{d}x=\int_E{f_1\left( x \right) \,\mathrm{d}x}-\int_E{f\left( x \right) \,\mathrm{d}x}.\label{equation:91}
\end{align}
注意到 \( f_1(x) = (f_1(x) - f_k(x)) + f_k(x) \),于是由\hyperref[theorem:非负可测函数积分的线性性质]{非负可测函数积分的线性性质}我们有
\[
\int_E f_1(x) \, \mathrm{d}x = \int_E (f_1(x) - f_k(x)) \, \mathrm{d}x + \int_E f_k(x) \, \mathrm{d}x,
\]
进而
\[
\int_E (f_1(x) - f_k(x)) \, \mathrm{d}x = \int_E f_1(x) \, \mathrm{d}x - \int_E f_k(x) \, \mathrm{d}x.
\]
令$k\to \infty$,可得
\[
\lim_{k\rightarrow \infty} \int_E{(f_1(x)}-f_k(x))\,\mathrm{d}x=\lim_{k\rightarrow \infty} \left( \int_E{f_1(x)\,\mathrm{d}x}-\int_E{f_k(x)\,\mathrm{d}x} \right) =\int_E{f_1(x)\,\mathrm{d}x}-\lim_{k\rightarrow \infty} \int_E{f_k(x)\,\mathrm{d}x}.
\]
再结合\eqref{equation:91}式可得
\begin{align*}
\int_E{f_1(x)\,\mathrm{d}x}-\lim_{k\rightarrow \infty} \int_E{f_k(x)\,\mathrm{d}x}=\int_E f_1(x) - \int_E f(x) \, \mathrm{d}x.
\end{align*}
因为可积函数的积分值是有限的,所以从两端消去同值项,即得所证.
\end{proof}

\begin{theorem}[非负可测函数积分的线性性质]\label{theorem:非负可测函数积分的线性性质}
设 \( f(x), g(x) \) 是 \( E \) 上的非负可测函数,\( \alpha, \beta \) 是非负常数,则
\[
\int_E (\alpha f(x) + \beta g(x)) \, \mathrm{d}x = \alpha \int_E f(x) \, \mathrm{d}x + \beta \int_E g(x) \, \mathrm{d}x.
\]
\end{theorem}
\begin{proof}
由\hyperref[theorem:简单函数逼近定理]{简单函数逼近定理}可知,存在 \( \{ \varphi_k(x) \}, \{ \psi_k(x) \} \) 是非负可测简单函数渐升列,使得
\[
\lim_{k \to \infty} \varphi_k(x) = f(x), \quad \lim_{k \to \infty} \psi_k(x) = g(x), \quad x \in E,
\]
则 \( \{ \varphi_k(x) + \psi_k(x) \} \) 仍为非负可测简单函数渐升列,且有
\[
\lim_{k \to \infty} (\alpha\varphi_k(x) + \beta\psi_k(x)) = \alpha f(x) + \beta g(x), \quad x \in E.
\]
从而由\hyperref[theorem:简单函数积分的线性性质]{简单函数积分的线性性质}和\hyperref[theorem:Beppo Levi非负渐升列积分定理]{Beppo Levi非负渐升列积分定理}可知
\begin{align*}
\int_E{(\alpha f(x)}+\beta g(x))\,\mathrm{d}x&=\lim_{k\rightarrow \infty} \int_E{(\alpha \varphi _k(x)}+\beta \psi _k(x))\,\mathrm{d}x
\\
&=\alpha \lim_{k\rightarrow \infty} \int_E{\varphi _k(x)\,\mathrm{d}x}+\beta \lim_{k\rightarrow \infty} \int_E{\psi _k(x)\,\mathrm{d}x}
\\
&=\alpha \int_E{f(x)\,\mathrm{d}x}+\beta \int_E{g(x)\,\mathrm{d}x}.
\end{align*}
\end{proof}

\begin{theorem}\label{theorem:对等的函数在相同可测集下的积分相等}
设 \( f(x), g(x) \) 是 \( E \) 上的非负可测函数. 若 \( f(x) = g(x) \) a. e. \( x \in E \),则
\[
\int_E f(x) \, \mathrm{d}x = \int_E g(x) \, \mathrm{d}x.
\]
\end{theorem}
\begin{note}
这个命题表明:改变非负可测函数在零测集上的值,不会影响它的可积性与积分值.
\end{note}
\begin{proof}
令 \( E_1 = \{ x \in E : f(x) \neq g(x) \} \),\( E_2 = E \setminus E_1 \),\( m(E_1) = 0 \),则
\begin{align*}
\int_E{f(x)\,\mathrm{d}x}&\xlongequal{\text{\nrefthe{theorem:非负可测函数积分的性质}{(3)}}}\int_E{f(x)\chi _E(x)\,\mathrm{d}x}=\int_E{f(x)\chi _{E_1\cup E_2}(x)\,\mathrm{d}x}
\\
&=\int_E{f(x)[\chi _{E_1}(x)}+\chi _{E_2}(x)]\,\mathrm{d}x\xlongequal{\text{\nrefthe{theorem:非负可测函数积分的性质}{(3)}}}\int_{E_1}{f(x)\,\mathrm{d}x}+\int_{E_2}{f(x)\,\mathrm{d}x}
\\
&\xlongequal{\text{\nrefthe{theorem:非负可测函数积分的性质}{(5)(ii)}}}0+\int_{E_2}{f(x)\,\mathrm{d}x}=0+\int_{E_2}{g(x)\,\mathrm{d}x}
\\
&\xlongequal{\text{\nrefthe{theorem:非负可测函数积分的性质}{(5)(ii)}}}\int_{E_1}{g(x)\,\mathrm{d}x}+\int_{E_2}{g(x)\,\mathrm{d}x}=\int_E{g(x)\,\mathrm{d}x}.
\end{align*}
\end{proof}

\begin{theorem}[逐项积分定理]\label{theorem:非负可测函数的逐项积分定理}
若\(\{f_k(x)\}\)是\(E\)上的非负可测函数列,则
\begin{align*}
\int_E \sum_{k = 1}^{\infty} f_k(x) \mathrm{d}x = \sum_{k = 1}^{\infty} \int_E f_k(x) \mathrm{d}x.
\end{align*}
\end{theorem}
\begin{proof}
令\(S_m(x) = \sum_{k = 1}^{m} f_k(x)\),则\(\{S_m(x)\}\)是\(E\)上的非负可测函数渐升列,且
\(\lim_{m \to \infty} S_m(x) = \sum_{k = 1}^{\infty} f_k(x).\)
从而根据 \hyperref[theorem:Beppo Levi非负渐升列积分定理]{Beppo Levi 非负渐升列积分定理}以及\hyperref[theorem:非负可测函数积分的线性性质]{非负可测函数积分的线性性质},可知
\begin{align*}
\int_E{\sum_{k=1}^{\infty}{f_k(x)\mathrm{d}x}}=\int_E{\lim_{m\rightarrow \infty} S_m(x)\mathrm{d}x}=\lim_{m\rightarrow \infty} \int_E{S_m(x)\mathrm{d}x}=\lim_{m\rightarrow \infty} \sum_{k=1}^m{\int_E{f_k(x)\mathrm{d}x}}=\sum_{k=1}^{\infty}{\int_E{f_k(x)\mathrm{d}x}}
\end{align*}
\end{proof}

\begin{corollary}[非负可测函数积分的可数可加性]\label{corollary:非负可测函数积分的可数可加性}
设\(E_k \in \mathscr{M}(k = 1, 2, \cdots)\),\(E_i \cap E_j = \varnothing (i \neq j)\).若\(f(x)\)是\(E = \bigcup_{k = 1}^{\infty} E_k\)上的非负可测函数,则
\begin{align*}
\int_E f(x) \mathrm{d}x = \int_{\bigcup_{k = 1}^{\infty} E_k} f(x) \mathrm{d}x = \sum_{k = 1}^{\infty} \int_{E_k} f(x) \mathrm{d}x.
\end{align*}
\end{corollary}
\begin{note}
特别地,当\(f(x) \equiv 1\)时,上式就是测度的可数可加性. 从这里还可看到,通过点集的特征函数,积分与测度的问题是可以互相转化的. 
\end{note}
\begin{proof}
由\hyperref[theorem:非负可测函数的逐项积分定理]{逐项积分定理}可得
\begin{align*}
\sum_{k=1}^{\infty}{\int_{E_k}{f(x)\mathrm{d}x}}=\sum_{k=1}^{\infty}{\int_E{f(x)\chi _{E_k}(x)\mathrm{d}x}}=\int_E{f(x)\sum_{k=1}^{\infty}{\chi _{E_k}(x)\mathrm{d}x}}=\int_E{f(x)\chi _{\bigcup_{k=1}^{\infty}{E_k}}\left( x \right) \mathrm{d}x}=\int_E{f(x)\mathrm{d}x}.
\end{align*}
\end{proof}

\begin{example}
若\(E_1, E_2, \cdots, E_n\)是\([0, 1]\)中的可测集,\([0, 1]\)中每一点至少属于上述集合中的\(k\)个\((k \leqslant slant n)\),则在\(E_1, E_2, \cdots, E_n\)中必有一个点集的测度大于或等于\(k / n\).
\end{example}
\begin{proof}
因为当\(x \in [0, 1]\)时,有\(\sum_{i = 1}^{n} \chi_{E_i}(x) \geqslant slant k\),所以
\begin{align*}
\sum_{i = 1}^{n} m(E_i) &= \sum_{i = 1}^{n} \int_{[0, 1]} \chi_{E_i}(x) \mathrm{d}x = \int_{[0, 1]} \sum_{i = 1}^{n} \chi_{E_i}(x) \mathrm{d}x \geqslant slant k.
\end{align*}
若每一个\(m(E_i)\)皆小于\(k / n\),则
\[
\sum_{i = 1}^{n} m(E_i) < \frac{k}{n} \cdot n = k.
\]
这与前式矛盾,故存在\(i_0\),使得\(m(E_{i_0}) \geqslant slant k / n\). 
\end{proof}

\begin{theorem}[Fatou引理]\label{lemma:Fatou引理}
若\(\{f_k(x)\}\)是\(E\)上的非负可测函数列,则
\begin{align*}
\int_E \varliminf_{k \to \infty} f_k(x) \mathrm{d}x \leqslant slant \varliminf_{k \to \infty} \int_E f_k(x) \mathrm{d}x. 
\end{align*}
\end{theorem}
\begin{note}
\hyperref[lemma:Fatou引理]{Fatou 引理}常用于判断极限函数的可积性. 例如,当\(E\)上的非负可测函数列\(\{f_k(x)\}\)满足
\[
\int_E f_k(x) \mathrm{d}x \leqslant slant M \quad (k = 1, 2, \cdots)
\]
时,我们就得到
\[
\int_E \varliminf_{k \to \infty} f_k(x) \mathrm{d}x \leqslant slant \varliminf_{k \to \infty} \int_E f_k(x) \mathrm{d}x\leqslant slant M.
\] 
\end{note}
\begin{remark}
\hyperref[lemma:Fatou引理]{Fatou引理}的不等号是可能成立的,可见\refexa{example:Fatou引理不等号成立的例子}.
\end{remark}
\begin{proof}
令\(g_k(x) = \inf\{f_j(x): j \geqslant slant k\}\),我们有
\[
g_k(x) \leqslant slant g_{k + 1}(x) \quad (k = 1, 2, \cdots),
\]
而且得到
\[
\varliminf_{k \to \infty} f_k(x) = \lim_{k \to \infty} g_k(x), \quad x \in E,
\]
从而根据 \hyperref[theorem:Beppo Levi非负渐升列积分定理]{Beppo Levi 非负渐升列积分定理}可知,
\begin{align*}
\int_E \varliminf_{k \to \infty} f_k(x) \mathrm{d}x &= \int_E \lim_{k \to \infty} g_k(x) \mathrm{d}x = \lim_{k \to \infty} \int_E g_k(x) \mathrm{d}x \\
&= \varliminf_{k \to \infty} \int_E g_k(x) \mathrm{d}x \leqslant slant \varliminf_{k \to \infty} \int_E f_k(x) \mathrm{d}x.
\end{align*}
\end{proof}

\begin{example}\label{example:Fatou引理不等号成立的例子}
在\([0, 1]\)上作非负可测函数列:
\[
f_n(x)=
\begin{cases}
0, & x = 0, \\
n, & 0 < x < \frac{1}{n}, \quad (n = 1, 2, \cdots). \\
0, & \frac{1}{n} \leqslant slant x \leqslant slant 1
\end{cases}
\]
显然,\(\lim_{n \to \infty} f_n(x) = 0\)(\(x \in [0, 1]\)),因此我们有
\[
\int_{[0, 1]} \lim_{n \to \infty} f_n(x) \mathrm{d}x = 0 < 1 = \lim_{n \to \infty} \int_{[0, 1]} f_n(x) \mathrm{d}x.
\] 
\end{example}
\begin{note}
这个例题说明\hyperref[lemma:Fatou引理]{Fatou引理}的不等号是可能成立的.
\end{note}

\begin{theorem}\label{theorem:函数可积的充要条件1}
设\(f(x)\)是\(E\)上的几乎处处有限的非负可测函数,\(m(E) < +\infty\).在\([0, +\infty)\)上作如下划分:
\[
0 = y_0 < y_1 < \cdots < y_k < y_{k + 1} < \cdots \to \infty,
\]
其中\(y_{k + 1} - y_k < \delta\) \((k = 0, 1, \cdots)\).若令
\[
E_k = \{x \in E: y_k \leqslant slant f(x) < y_{k + 1}\} \quad (k = 0, 1, \cdots),
\]
则\(f(x)\)在\(E\)上是可积的当且仅当级数
\[
\sum_{k = 0}^{\infty} y_k m(E_k) < +\infty.
\]
此时有
\begin{align*}
\lim_{\delta \to 0} \sum_{k = 0}^{\infty} y_k m(E_k) = \int_E f(x) \mathrm{d}x.
\end{align*}
\end{theorem}
\begin{note}
由上述定理可知,对\(m(E) < +\infty\)以及\(E\)上的非负实值可测函数来说,它的可积性等价于
\[
\sum_{k = 1}^{\infty} k m(E_k) < +\infty,
\]
其中
\[
E_k = \{x \in E: k \leqslant slant f(x) < k + 1\} \quad (k = 1, 2, \cdots).
\]
但若把\(E_k\)换成\(\{x \in E: f(x) \geqslant slant k\}\),则还有\refthe{therorem:非负可测函数可积的充要条件2}.
\end{note}
\begin{proof}
必要性显然成立,下证充分性.
因为有不等式
\[
y_k m(E_k) \leqslant slant \int_{E_k} f(x) \mathrm{d}x \leqslant slant y_{k + 1} m(E_k),
\]
所以由\refcor{corollary:非负可测函数积分的可数可加性}得到
\begin{align*}
\sum_{k = 0}^{\infty} y_k m(E_k) &\leqslant slant \int_E f(x) \mathrm{d}x \leqslant slant \sum_{k = 0}^{\infty} y_{k + 1} m(E_k) \\
&= \sum_{k = 0}^{\infty} (y_{k + 1} - y_k) m(E_k) + \sum_{k = 0}^{\infty} y_k m(E_k) \\
&\leqslant slant \delta m(E) + \sum_{k = 0}^{\infty} y_k m(E_k).
\end{align*}
令$\delta \to 0$,由条件立即可知结论成立.
\end{proof}

\begin{theorem}\label{therorem:非负可测函数可积的充要条件2}
设\(E \subset \mathbf{R}\),\(m(E) < +\infty\),\(f(x)\)是\(E\)上的非负实值可测函数,则\(f(x)\)在\([0, +\infty)\)上可积的充分必要条件是
\[
\sum_{n = 0}^{\infty} m(\{x \in E: f(x) \geqslant slant n\}) < +\infty.
\]
\end{theorem}
\begin{proof}
{\heiti 必要性:} 只需注意到下式即可:
\begin{align*}
\sum_{n=0}^{\infty}{m\left( \left\{ x\in E:f\left( x \right) \geqslant slant n \right\} \right)}&=\sum_{n=0}^{\infty}{m\left( \bigcup_{k=n}^{\infty}{\left\{ x\in E:k\leqslant slant f\left( x \right) <k+1 \right\}} \right)}
\\
&=\sum_{n=0}^{\infty}{\sum_{k=n}^{\infty}{m\left( \left\{ x\in E:k\leqslant slant f\left( x \right) <k+1 \right\} \right)}}
\\
&\xlongequal{\text{\refthe{Basis of Analytics-关于无限和的Fubinin定理}}}\sum_{k=0}^{\infty}{\sum_{n=0}^k{m\left( \left\{ x\in E:k\leqslant slant f\left( x \right) <k+1 \right\} \right)}}
\\
&=\sum_{k=0}^{\infty}{\left( k+1 \right) m\left( \left\{ x\in E:k\leqslant slant f\left( x \right) <k+1 \right\} \right)}<+\infty .
\end{align*}
{\heiti 充分性:} 由\refcor{corollary:非负可测函数积分的可数可加性}可得
\begin{align*}
\int_E f(x)\mathrm{d}x &= \sum_{k=0}^{\infty} \int_{\{x\in E:k\leqslant slant f(x)<k+1\}} f(x)\mathrm{d}x
\leqslant slant \sum_{k=0}^{\infty} (k+1)m(\{x\in E:k\leqslant slant f(x)<k+1\}) \\
&= \sum_{k=0}^{\infty} \sum_{n=0}^k m\left( \left\{ x\in E:k\leqslant slant f(x) <k+1 \right\} \right)
\xlongequal{\text{\refthe{Basis of Analytics-关于无限和的Fubinin定理}}} \sum_{n=0}^{\infty} \sum_{k=n}^{\infty} m\left( \left\{ x\in E:k\leqslant slant f(x) <k+1 \right\} \right) \\
&= \sum_{n=0}^{\infty} m\left( \bigcup_{k=n}^{\infty} \left\{ x\in E:k\leqslant slant f(x) <k+1 \right\} \right)
= \sum_{n=0}^{\infty} m\left( \left\{ x\in E:f(x) \geqslant slant n \right\} \right) < +\infty.
\end{align*}
\end{proof}

\begin{example}
设\(f(x)\)是\([a, b]\)的上非负实值可测函数,则\(f^2(x)\)在\([a, b]\)上可积当且仅当
\[
\sum_{n = 1}^{\infty} n m(\{x \in [a, b]: f(x) \geqslant slant n\}) < +\infty.
\]
\end{example}
\begin{proof}
(i)首先,若\(f^2(x)\)在\([a, b]\)上可积,则易知\(f(x)\)在\([a, b]\)上可积. 若令
\[
E_n = \{x \in [a, b]: n \leqslant slant f(x) < n + 1\}, \quad n \in \mathbf{N},
\]
则\(\bigcup_{n = 0}^{\infty} E_n = b - a\),且有
\begin{align*}
\sum_{n = 0}^{\infty} n m(E_n) &\leqslant slant \sum_{n = 0}^{\infty} \int_{E_n} f(x) \mathrm{d}x = \int_{[a, b]} f(x) \mathrm{d}x \\
&\leqslant slant \sum_{n = 0}^{\infty} (n + 1)m(E_n) = \sum_{n = 0}^{\infty} n m(E_n) + (b - a),
\end{align*}
\begin{align*}
\sum_{n = 0}^{\infty} n^2 m(E_n) &\leqslant slant \sum_{n = 0}^{\infty} \int_{E_n} f^2(x) \mathrm{d}x \leqslant slant \sum_{n = 0}^{\infty} (n + 1)^2 m(E_n) \\
&= \sum_{n = 0}^{\infty} n^2 m(E_n) + 2 \sum_{n = 0}^{\infty} n m(E_n) + (b - a).
\end{align*}
这就是说,\(f^2(x)\)在\([a, b]\)上可积当且仅当
\[
\sum_{n = 0}^{\infty} n^2 m(E_n) < +\infty, \quad \sum_{n = 0}^{\infty} n m(E_n) < +\infty.
\]
(ii) 注意到等式
\begin{align*}
\frac{1}{2} \sum_{n = 1}^{\infty} n m(E_n) + \frac{1}{2} \sum_{n = 1}^{\infty} n^2 m(E_n) 
&= \sum_{n = 1}^{\infty} \frac{n(n + 1)}{2} m(E_n) = \sum_{n = 1}^{\infty} \sum_{k = 1}^{n} k m(E_n) \\
&= \sum_{k = 1}^{\infty} k \sum_{n = k}^{\infty} m(E_n) = \sum_{k = 1}^{\infty} k m(\{x \in [a, b]: f(x) \geqslant slant k\}),
\end{align*}
即得所证.
\end{proof}

\begin{proposition}\label{proposition:4.1例7}
设\(f(x)\)是\(E\)上的非负实值可测函数. 若对任意的正整数\(n\),均有\(m(\{x \in E: f(x)>n\})>0\),则存在非负可测函数\(g\),且$g$可积,使得\(fg\)不可积.
\end{proposition}
\begin{proof}
令\(E_n = \{x \in E: n \leqslant slant f(x) < n + 1\}\),由\(\sum_{n = 1}^{\infty} m(E_n) > 0\)可知,存在\(\{n_k\}\):\(m(E_{n_k})>0\) \((k \in \mathbf{N})\). 作函数
\[
g(x)=
\begin{cases}
(1 / k^2) \cdot m(E_{n_k}) &, x \in E_{n_k} (k \in \mathbf{N}), \\
0 & ,x \in \left(\bigcup_{k = 1}^{\infty} E_{n_k}\right)^c,
\end{cases}
\]
易知\(g \in L(E)\),且有
\[
\int_E g(x) f(x) \mathrm{d}x \geqslant slant \sum_{k = 1}^{\infty} \frac{n_k}{k^2} = +\infty \quad (\text{注意 } n_k \geqslant slant k).
\] 
\end{proof}





























\end{document}
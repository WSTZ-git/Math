\documentclass[../../main.tex]{subfiles}
\graphicspath{{\subfix{../../image/}}} % 指定图片目录,后续可以直接使用图片文件名。

% 例如:
% \begin{figure}[H]
% \centering
% \includegraphics[scale=0.4]{image-01.01}
% \caption{图片标题}
% \label{figure:image-01.01}
% \end{figure}
% 注意:上述\label{}一定要放在\caption{}之后,否则引用图片序号会只会显示??.

\begin{document}

\section{可测函数的定义及其性质}

为了论述的简便和统一,今后我们在谈到可测函数时允许函数取“值”\(\pm\infty\).(称 \(\mathbf{R}\cup\{+\infty\}\cup\{-\infty\}\) 为广义实数集.)现在先将有关 \(\pm\infty\) 的运算规则约定如下(注意,这里的 \(\pm\infty\) 不是指无穷大变量):

(i) \(-\infty < +\infty\),若 \(x \in \mathbf{R}\),则 \(-\infty < x < +\infty\);

(ii) 若 \(x \in \mathbf{R}\),则
\begin{align*}
x + (\pm\infty) &= (\pm\infty) + x = (\pm\infty) + (\pm\infty) = \pm\infty,\\
x - (\mp\infty) &= (\pm\infty) - (\mp\infty) = \pm\infty,\\
\pm(\pm\infty) &= +\infty, \quad \pm(\mp\infty) = -\infty,\\
|\pm\infty| &= +\infty;
\end{align*}

(iii) \(x \in \mathbf{R}\) 且 \(x \neq 0\) 的符号函数为
\[
\mathrm{sign}x = 
\begin{cases}
+1, & x > 0,\\
-1, & x < 0,
\end{cases}
\]
\[
x\cdot(\pm\infty) = \pm(\mathrm{sign}x)\infty,
\]
\[
(\pm\infty)(\pm\infty) = +\infty, \quad (\pm\infty)(\mp\infty) = -\infty,
\]
但是 \((\pm\infty) - (\pm\infty)\),\((\pm\infty) + (\mp\infty)\) 等是无意义的;

(iv) 特别约定 \(0\cdot(\pm\infty) = 0\).

注意,\(+\infty\) 经常简记为 \(\infty\). 

\begin{definition}[可测函数]
设 \(f(x)\) 是定义在可测集 \(E \subset \mathbb{R}^n\) 上的广义实值函数. 若对于任意的实数 \(t\),点集
\[
\{x \in E: f(x) > t\} \text{(或简写为} \{x: f(x) > t\} \text{或}f^{-1}((t,+\infty))\text{)}
\]
是可测集,则称 \(f(x)\) 是 \(E\) 上的\textbf{可测函数},或称 \(f(x)\) 在 \(E\) 上\textbf{可测}. 
\end{definition}

\begin{theorem}\label{theorem:定理3.1}
设 \(f(x)\) 是可测集 \(E\) 上的函数,\(D\) 是 \(\mathbf{R}\) 中的一个稠密集. 若对任意的 \(r \in D\),点集 \(\{x: f(x) > r\}\) 都是可测集,则对任意的 \(t \in \mathbf{R}\),点集 \(\{x: f(x) > t\}\) 也是可测集.进而$f(x)$在$E$上可测.
\end{theorem}
\begin{note}
这定理说明,今后,我们只需对$\mathbb{R}$中的一个稠密集中的元$r$,指出集合$\{x:f(x)>r\}$是可测集就可以得到$f(x)$是可测函数.
\end{note}
\begin{proof}
对任一实数 \(t\),选取 \(D\) 中的点列 \(\{r_k\}\),使得
\begin{align*}
r_k \geqslant t\ (k = 1,2,\cdots); \quad \lim_{k \to \infty}r_k = t.
\end{align*}
一方面,对$\forall x_0\in \{x:f(x)>t\}$,都有$f(x_0)>t=\lim_{k\to \infty}r_k$.于是由极限的保号性可知,存在$k_0\in \mathbb{N}$,使得$f(x_0)>r_{k_0}$.从而$x_0\in \{x:f(x)>r_{k_0}\}\subset \bigcup_{k = 1}^{\infty}\{x: f(x) > r_k\}$.
另一方面,对$\forall x_0\in \bigcup_{k = 1}^{\infty}\{x: f(x) > r_k\}$,都存在$k_0\in \mathbb{N}$,使得$x_0\in \{x:f(x)>r_{k_0}\}$.从而$f(x_0)>r_{k_0}\geq t$,于是$x_0\in \{x:f(x)>t\}$.故
\begin{align}
\{x: f(x) > t\} = \bigcup_{k = 1}^{\infty}\{x: f(x) > r_k\}. \label{eq:3.1}
\end{align}
因为每个点集 \(\{x: f(x) > r_k\}\) 都是可测集,所以 \(\{x: f(x) > t\}\) 是可测集. 
\end{proof}

\begin{proposition}
设 \(f(x)\) 是定义在区间 \([a,b]\) 上的单调函数,则 \(f(x)\) 是 \([a,b]\) 上的可测函数.
\end{proposition}
\begin{proof}
事实上,对于任意的 \(t \in \mathbf{R}\),点集 \(\{x \in [a,b]: f(x) > t\}\) 定属于下述三种情况之一:区间、单点集或空集. 从而可知
\[
\{x \in [a,b]: f(x) > t\}
\]
是可测集. 这说明 \(f(x)\) 是 \([a,b]\) 上的可测函数. 
\end{proof}

\begin{theorem}\label{theorem:定理3.2}
若 \(f(x)\) 是 \(E\) 上的可测函数,则下列等式皆成立并且其中左端的点集皆可测:

(i) \(\{x: f(x) \leqslant t\} = E \setminus \{x: f(x) > t\}\) \((t \in \mathbf{R})\);

(ii) \(\{x: f(x) \geqslant t\} = \bigcap_{k = 1}^{\infty}\left\{x: f(x) > t - \frac{1}{k}\right\}\) \((t \in \mathbf{R})\);

(iii) \(\{x: f(x) < t\} = E \setminus \{x: f(x) \geqslant t\}\) \((t \in \mathbf{R})\);

(iv) \(\{x: f(x) = t\} = \{x: f(x) \geqslant t\} \cap \{x: f(x) \leqslant t\}\) \((t \in \mathbf{R})\);

(v) \(\{x: f(x) < +\infty\} = \bigcup_{k = 1}^{\infty}\{x: f(x) < k\}\);

(vi) \(\{x: f(x) = +\infty\} = E \setminus \{x: f(x) < +\infty\}\);

(vii) \(\{x: f(x) > -\infty\} = \bigcup_{k = 1}^{\infty}\{x: f(x) > -k\}\);

(viii) \(\{x: f(x) = -\infty\} = E \setminus \{x: f(x) > -\infty\}\).
\end{theorem}
\begin{remark}
由于对任意的 \(t \in \mathbf{R}\),有
\begin{align*}
&\{x: f(x) > t\} = \bigcup_{k = 1}^{\infty}\left\{x: f(x) > t + \frac{1}{k}\right\}\\
&= E \setminus \{x: f(x) \leqslant t\} = E \setminus \bigcap_{k = 1}^{\infty}\left\{x: f(x) < t + \frac{1}{k}\right\},
\end{align*}
故定理中 (i),(ii) 与 (iii) 的左端点集的可测性均可当作 \(f(x)\) 可测性的定义. 
\end{remark}
\begin{proof}
由极限的保号性和保不等式性易证上述等式皆成立. 至于左端点集的可测性可阐明如下:

从可测性定义易推 (i),(ii) 与 (vii). 从 (ii) 可推出 (iii). 从 (i) 与 (ii) 可推出 (iv). 从 (iii) 可推出 (v). 从 (v) 可推出 (vi). 从 (vii) 可推出 (viii). 
\end{proof}

\begin{theorem}\label{theorem:定理3.3}
(1) 设 \(f(x)\) 是定义在 \(E\subset \mathbb{R}^n\) 上的广义实值函数,$E_n$为$E$的可测子集$(n=1,2,\cdots)$,且$E=\bigcup_{n=1}^{\infty}E_n$ ,则\(f(x)\) 在$E$上可测的充要条件是\(f(x)\)在每个$E_n$上可测;

(2) 若 \(f(x)\) 在 \(E\) 上可测,\(A\) 是 \(E\) 中可测集,则 \(f(x)\) 看做是定义在 \(A\) 上的函数在 \(A\) 上也是可测的.
\end{theorem}
\begin{proof}
(1) {\heiti 必要性:}对 \( \forall t \in \mathbb{R} \),注意到
\[
\{ x \in E_n : f(x) > t \} = E_n \cap \{ x \in E : f(x) > t \}, \, n = 1, 2, \cdots
\]
由 \( f(x) \) 在 \( E \) 上可测知,\( \{ x \in E : f(x) > t \} \) 可测。又 \( E_n \) 可测,故 \( \{ x \in E_n : f(x) > t \} \) 也可测。因此 \( f(x) \) 在每个 \( E_n \) 上可测。

{\heiti 充分性:}对 \( \forall t \in \mathbb{R} \),注意到
\[
\{ x \in E : f(x) > t \} = \bigcup_{n = 1}^{\infty} \{ x \in E_n : f(x) > t \}
\]
由 \( f(x) \) 在 \( E_n \)(\( n = 1, 2, \cdots \))上可测知,\( \{ x \in E_n : f(x) > t \} \)(\( n = 1, 2, \cdots \))都可测,从而 \( \bigcup_{n = 1}^{\infty} \{ x \in E_n : f(x) > t \} \) 可测,于是由上式可知 \( \{ x \in E : f(x) > t \} \) 也可测,故 \( f(x) \) 在 \( E \) 上可测。

(2) 只需注意等式
\[
\{x \in A: f(x) > t\} = A \cap \{x \in E: f(x) > t\}, \quad t \in \mathbb{R}.
\]
\end{proof}

\begin{proposition}
若 \(E \in \mathscr{M}\),则 \(\chi_E(x)\) 是 \(\mathbb{R}^n\) 上的可测函数. 
\end{proposition}
\begin{proof}
注意到对 \(\forall x \in E\),有 \(\chi_E(x) = 1\). 于是当 \(t < 1\) 时,有 \(\{x \in E: \chi_E(x) > t\} = E \in \mathscr{M}\);当 \(t \geqslant 1\) 时,有 \(\{x \in E: \chi_E(x) > t\} = \varnothing \in \mathscr{M}\). 故 \(\chi_E(x)\) 在 \(E\) 上可测.

又注意到对 \(\forall x \in \mathbb{R}^n \backslash E\),有 \(\chi_E(x) = 0\). 于是当 \(t < 0\) 时,有 \(\{x \in \mathbb{R}^n \backslash E: \chi_E(x) > t\} = \mathbb{R}^n \backslash E \in \mathscr{M}\);当 \(t \geqslant 0\) 时,有 \(\{x \in \mathbb{R}^n \backslash E: \chi_E(x) > t\} = \varnothing \in \mathscr{M}\). 故 \(\chi_E(x)\) 在 \(\mathbb{R}^n \backslash E\) 上也可测.

因此由\nrefthe{theorem:定理3.3}{(1)}可得 \(\chi_E(x)\) 在 \(E \cup (\mathbb{R}^n \backslash E) = \mathbb{R}^n\) 上可测.
\end{proof}

\begin{theorem}[可测函数的运算性质]\label{theorem:可测函数的运算性质}
\begin{enumerate}[(1)]
\item 若 \(f(x)\),\(g(x)\) 是 \(E\) 上的实值可测函数,则下列函数
\begin{align*}
(\text{i})cf(x)(c \in \mathbb{R});\quad (\text{ii}) f(x) + g(x); \quad (\text{iii})f(x)\cdot g(x)
\end{align*}
都是 \(E\) 上的可测函数.

\item 若 \(\{f_k(x)\}\) 是 \(E\) 上的可测函数列,则下列函数:
\begin{align*}
(\text{i}) \sup_{k \geqslant 1}\{f_k(x)\};\quad (\text{ii})\inf_{k \geqslant 1}\{f_k(x)\} ;\quad (\text{iii})\varlimsup_{k \to \infty} f_k(x) ;\quad (\text{iv}) \varliminf_{k \to \infty} f_k(x)
\end{align*}
都是 \(E\) 上的可测函数.
\end{enumerate}
\end{theorem}
\begin{remark}
(1)中所说的运算性质对于取广义实值的可测函数也是成立的.

已证$f(x),g(x)$在$\{x\in E:-\infty<f(x)<+\infty\}$上可测.对$\forall t\in \mathbb{R}$,注意到
\begin{gather*}
\{x\in  \{x\in E:f(x)=+\infty\}:f(x)>t\}=\{x\in E:f(x)=+\infty\},
\\
\{x\in \{x\in E:f(x)=-\infty\}:f(x)>t\}=\varnothing\in \mathscr{M}.
\end{gather*}
由\refthe{theorem:定理3.2}知$\{x\in E:f(x)=+\infty\}\in \mathscr{M}$.因此$f(x)$在$\{x\in E:f(x)=-\infty\}\cup\{x\in E:f(x)=+\infty\}$上可测.故再由\nrefthe{theorem:定理3.3}{(1)}可知$f(x)$在$\{x\in E:-\infty<f(x)<+\infty\}\cup \{x\in E:f(x)=-\infty\}\cup\{x\in E:f(x)=+\infty\}=E$上可测.

\end{remark}
\begin{proof}
\begin{enumerate}[(1)]
\item (i) 对于 \(t \in \mathbb{R}\),若 \(c > 0\),则由
\[
\{x: cf(x) > t\} = \{x: f(x) > c^{-1}t\}
\]
可知,\(cf(x)\) 在 \(E\) 上可测;若 \(c < 0\),则
\[
\{x: cf(x) > t\} = \{x: f(x) < c^{-1}t\}
\]
再由\refthe{theorem:定理3.2}可知,\(cf(x)\) 在 \(E\) 上可测;若 \(c = 0\),则 \(cf(x) = 0\) .于是当$t<0$时,有
$\{x:cf(x)>t\}=E\in \mathscr{M}$;当$t\geqslant 0$时,有$\{x:cf(x)>t\}=\varnothing\in \mathscr{M}$.故此时仍有\(cf(x)\) 在 \(E\) 上可测.

(ii) 因为有理数集至多可数,所以可设\(\{r_i\}\) 是全体有理数.对 \(t \in \mathbb{R}\),一方面,任取$x_0\in \{x: f(x) + g(x) > t\}$,则 \(f(x_0) + g(x_0) > t\),此即 \(f(x_0) > t - g(x_0)\),故由有理数集的稠密性可知,存在$i_0\in\mathbb{N}$,使得
\begin{align*}
f(x_0)>r_{i_0}>t-g(x_0).
\end{align*}
于是$$x_0\in \{x:f(x)>r_{i_0}\}\cap \{x: g(x) > t - r_{i_0}\}\subset \bigcup_{i = 1}^{\infty}(\{x: f(x) > r_i\} \cap \{x: g(x) > t - r_i\}).$$因此$\{x: f(x) + g(x) > t\} 
\subset \bigcup_{i = 1}^{\infty}(\{x: f(x) > r_i\} \cap \{x: g(x) > t - r_i\}).$

另一方面,任取$x_0\in  \bigcup_{i = 1}^{\infty}(\{x: f(x) > r_i\} \cap \{x: g(x) > t - r_i\})$,则存在$i_0\in \mathbb{N}$,使得
\[
x_0\in\{x:f(x)>r_{i_0}\}\cap \{x: g(x) > t - r_{i_0}\}.
\]
于是$f(x_0)+g(x_0)>r_{i_0}+t-r_{i_0}=t$.故$x_0\in \{x: f(x) + g(x) > t\} $.因此$\{x: f(x) + g(x) > t\} 
\supset \bigcup_{i = 1}^{\infty}(\{x: f(x) > r_i\} \cap \{x: g(x) > t - r_i\}).$

综上可知
\begin{align*}
\{x: f(x) + g(x) > t\} 
= \bigcup_{i = 1}^{\infty}(\{x: f(x) > r_i\} \cap \{x: g(x) > t - r_i\}),
\end{align*}
从而由 \(f(x),g(x)\) 在 \(E\) 上可测知 \(f(x) + g(x)\) 是 \(E\) 上的可测函数.

(iii) 首先,\(f^2(x)\) 在 \(E\) 上可测. 这是因为对于 \(t \in \mathbb{R}\),我们有
\begin{align*}
\{x: f^2(x) > t\} 
&= 
\begin{cases}
E, & t < 0,\\
\{x: f(x) > \sqrt{t}\} \cup \{x: f(x) < -\sqrt{t}\}, & t \geqslant 0.
\end{cases}
\end{align*}
于是由\refthe{theorem:定理3.2}可知,$f^2(x)$在$E$上可测.
又在 \(f(x)g(x) = \{[f(x) + g(x)]^2 - [f(x) - g(x)]^2\}/4\) 中,由(i)(ii)可知\(f(x) + g(x)\) 以及 \(f(x) + (-g(x))\) 都是 \(E\) 上可测函数,所以 \(f(x)\cdot g(x)\) 在 \(E\) 上可测.

\item (i) 对$\forall t \in \mathbf{R}$,显然有$\left\{x: \sup_{k \geqslant 1}\{f_k(x)\} > t\right\} \supset \bigcup_{k = 1}^{\infty}\{x: f_k(x) > t\}.$任取$x_0\in \left\{x: \sup_{k \geqslant 1}\{f_k(x)\} > t\right\}$,则$ \sup_{k \geqslant 1}\{f_k(x_0)\} > t$
于是由上确界的定义可知,存在$k_0\in \mathbb{N}$,使得$f_{k_0}(x_0)>t$.此即$$x_0\in \{x: f_{k_0}(x) > t\}\subset \bigcup_{k = 1}^{\infty}\{x: f_k(x) > t\}.$$
故$\left\{x: \sup_{k \geqslant 1}\{f_k(x)\} > t\right\} \subset \bigcup_{k = 1}^{\infty}\{x: f_k(x) > t\}.$
因此
\begin{align*}
\left\{x: \sup_{k \geqslant 1}\{f_k(x)\} > t\right\} = \bigcup_{k = 1}^{\infty}\{x: f_k(x) > t\},
\end{align*}
从而由$\{f_k(x)\}$在$E$上可测可知\(\sup_{k \geqslant 1}\{f_k(x)\}\) 是 \(E\) 上的可测函数.

(ii) 由于 \(\inf_{k \geqslant 1}\{f_k(x)\} = -\sup_{k \geqslant 1}\{-f_k(x)\}\),故可知 \(\inf_{k \geqslant 1}\{f_k(x)\}\) 在 \(E\) 上可测.

(iii) 只需注意到 \(\varlimsup_{k \to \infty} f_k(x) = \inf_{i \geqslant 1}\left(\sup_{k \geqslant i}[f_k(x)]\right)\) 即可.

(iv) 根据等式 \(\varliminf_{k \to \infty} f_k(x) = -\varlimsup_{k \to \infty}(-f_k(x))\) 可知,\(\varliminf_{k \to \infty} f_k(x)\) 是 \(E\) 上的可测函数. 
\end{enumerate}
\end{proof}

\begin{corollary}\label{corollary:可测函数列的极限也可测}
若 \(\{f_k(x)\}\) 是 \(E\) 上的可测函数列,且有
\begin{align*}
\lim_{k \to \infty}f_k(x) = f(x) (x \in E),
\end{align*}
则 \(f(x)\) 是 \(E\) 上的可测函数. 
\end{corollary}
\begin{proof}
只需注意到$f(x)=\lim_{k\to \infty}f_k(x)=\varlimsup_{k\to \infty}f_k(x)$,再由\hyperref[theorem:可测函数的运算性质]{可测函数的运算性质(2)}立得.
\end{proof}

\begin{definition}[函数的正部和负部]
设 \(f(x)\) 是定义在 \(E\) 上的广义实值函数,令
\[
f^+(x) = \max\{f(x),0\}, \quad f^-(x) = \max\{-f(x),0\},
\]
并分别称它们为 \(f(x)\) 的\textbf{正部}与\textbf{负部}.
\end{definition}

\begin{lemma}\label{lemma:函数的正部和负部的基本性质}
设 \(f(x)\) 是定义在 \(E\) 上的广义实值函数,则
\begin{enumerate}[(1)]
\item $f(x)=f^+(x)-f^-(x),\quad \forall x\in E.$

\item $f^+(x),f^-(x)\geqslant 0,\quad \forall x\in E.$

\item $\left| f\left( x \right) \right|=f^+\left( x \right) +f^-\left( x \right) ,\quad \forall x\in E.$

\item 若$f(x)\leqslant g(x),x\in E$,则$f^+(x)\leqslant g^+(x),f^-(x)\geqslant g^-(x),\forall x\in E.$
\end{enumerate}
\end{lemma}
\begin{proof}
证明都是显然的.
\end{proof}

\begin{theorem}
(1)\(f(x)\) 在 \(E\) 上可测的充要条件是\(f^+(x)\),\(f^-(x)\) 都是 \(E\) 上的可测函数.

(2)若 \(f(x)\) 在 \(E\) 上可测时,则\(|f(x)|\) 也在 \(E\) 上可测.
\end{theorem}
\begin{remark}
注意,(2)反之不然.
\end{remark}
\begin{proof}
(1)只需注意到$f(x) = f^+(x) - f^-(x)$即可.

(2)因为我们有
\begin{align*}
|f(x)| = f^+(x) + f^-(x), 
\end{align*}
所以当 \(f(x)\) 在 \(E\) 上可测时,由(1)可知\(|f(x)|\) 也在 \(E\) 上可测.
\end{proof}

\begin{proposition}
若 \(f(x,y)\) 是定义在 \(\mathbf{R}^2\) 上的实值函数,且对固定的 \(x \in \mathbf{R}\),\(f(x,y)\) 是 \(y \in \mathbf{R}\) 上的连续函数;对固定的 \(y \in \mathbf{R}\),\(f(x,y)\) 是 \(x \in \mathbf{R}\) 上的可测函数,则 \(f(x,y)\) 是 \(\mathbf{R}^2\) 上的可测函数.
\end{proposition}
\begin{proof}
对每个 \(n = 1,2,\cdots\),作函数
\[
f_n(x,y) = f\left(x,\frac{k}{n}\right), \frac{k - 1}{n} < y \leqslant \frac{k}{n} \ (k = 0, \pm 1, \pm 2,\cdots).
\]
因为对任意的 \(t \in \mathbf{R}\),显然有$\{(x,y) \in \mathbf{R}^2: f_n(x,y) < t\} 
\supset \bigcup_{k = -\infty}^{\infty}\left\{x \in \mathbf{R}: f\left(x,\frac{k}{n}\right) < t\right\} \times \left(\frac{k - 1}{n}, \frac{k}{n}\right].$任取$(x_0,y_0)\in \{(x,y) \in \mathbf{R}^2: f_n(x,y) < t\},$则$f_n(x_0,y_0)<t$.从而存在$k_0\in \mathbb{Z}$,使得$\frac{k_0-1}{n}<y_0 \leq \frac{k}{n}$,并且$f(x_0,\frac{k_0}{n})=f_n(x_0,y_0)<t$.于是
\begin{align*}
(x_0,y_0)\in\left\{x \in \mathbf{R}: f\left(x,\frac{k_0}{n}\right) < t\right\} \times \left(\frac{k_0 - 1}{n}, \frac{k_0}{n}\right] \subset \bigcup_{k = -\infty}^{\infty}\left\{x \in \mathbf{R}: f\left(x,\frac{k}{n}\right) < t\right\} \times \left(\frac{k - 1}{n}, \frac{k}{n}\right].
\end{align*}
因此$\{(x,y) \in \mathbf{R}^2: f_n(x,y) < t\} 
\subset \bigcup_{k = -\infty}^{\infty}\left\{x \in \mathbf{R}: f\left(x,\frac{k}{n}\right) < t\right\} \times \left(\frac{k - 1}{n}, \frac{k}{n}\right].$
故
\begin{align*}
\{(x,y) \in \mathbf{R}^2: f_n(x,y) < t\} 
&= \bigcup_{k = -\infty}^{\infty}\left\{x \in \mathbf{R}: f\left(x,\frac{k}{n}\right) < t\right\} \times \left(\frac{k - 1}{n}, \frac{k}{n}\right],
\end{align*}
所以由条件及\hyperref[theorem:可测集的性质]{可测集的性质(6)}可知\(f_n(x,y)\) 是 \(\mathbf{R}^2\) 上的可测函数. 而由题设易知
\[
\lim_{n \to \infty}f_n(x,y) = f(x,y), \quad (x,y) \in \mathbf{R}^2,
\]
再由\refcor{corollary:可测函数列的极限也可测}即得所证.
\end{proof}

\begin{proposition}[连续函数必可测]\label{proposition:连续函数必可测}
设 \(E \subset \mathbf{R}^n\) 是可测集. 若 \(f \in C(E)\),则 \(f(x)\) 是 \(E\) 上的可测函数. 
\end{proposition}
\begin{proof}
对$\forall t\in \mathbb{R}$,注意到
\begin{align*}
\left\{ x\in E:f\left( x \right) >t \right\} =f^{-1}\left( t,+\infty \right) ,
\end{align*}
显然$\left( t,+\infty \right)$是$\mathbb{R}$上的开集,又$f\in C\left( E \right)$,故$f^{-1}\left( t,+\infty \right)$是$E$上的开集。又因为Borel集都可测,所以$f^{-1}\left( t,+\infty \right)$也可测。因此$f\left( x \right)$在$E$上可测。
\end{proof}

\begin{example}
设在 \(\mathbb{R}\) 上定义Dirchlet函数
\begin{align*}
D(x)=\chi_{\mathbb{Q}}(x)=\begin{cases}
1, & x\text{ 是有理数},\\
0, & x\text{ 是无理数}.
\end{cases}
\end{align*}
证明$D(x)$在$\mathbb{R}$上可测.
\end{example}
\begin{proof}
对$\forall t\in \mathbb{R}$,都有
\begin{align*}
\left\{ x\in \mathbb{R} :D\left( x \right) >t \right\} =\begin{cases}
\varnothing &,t\geqslant 1\\
\mathbb{Q} &,t\in \left[ 0,1 \right)\\
\mathbb{R} &,t<0\\
\end{cases}.
\end{align*}
显然$\varnothing,\mathbb{Q},\mathbb{R}$都是可测集,故$D\left( x \right)$在$\mathbb{R}$上可测.
\end{proof}

\begin{example}
证明$f\left( x \right) =\begin{cases}
\frac{1}{x},x\in \left( 0,1 \right]\\
+\infty ,x=0\\
\end{cases}$是可测函数,并且对$[0,1]$上任意连续函数$g(x)$,都有$m(\{x\in [0,1]:f(x)\ne g(x)\})>0$.
\end{example}
\begin{proof}
对$\forall t\in \mathbb{R}$,都有
\begin{align*}
\left\{ x\in \left[ 0,1 \right] :f\left( x \right) >t \right\} &=\left\{ x=0:f\left( x \right) >t \right\} \cup \left\{ x\in \left( 0,1 \right] :f\left( x \right) >t \right\} =\left\{ 0 \right\} \cup \left\{ x\in \left( 0,1 \right] :\frac{1}{x}>t \right\} 
\\
&=\left\{ 0 \right\} \cup \left\{ x\in \left( 0,1 \right] :x<\frac{1}{t} \right\} =\begin{cases}
\left[ 0,\frac{1}{t} \right) ,t\geqslant 1\\
\left[ 0,1 \right] ,t<1\\
\end{cases}\in \mathscr{M} .
\end{align*}
故$f$是$\mathbb{R}$上的可测函数.设$g\in C\left[ 0,1 \right]$,则不妨设$g\left( x \right) <M$,$\forall x\in \left[ 0,1 \right]$,其中$M>0$.
于是由测度的单调性可得
\begin{align*}
m\left( \left\{ x\in \left[ 0,1 \right] :f\left( x \right) \ne g\left( x \right) \right\} \right) &\geqslant m\left( \left\{ x\in \left( 0,1 \right] :f\left( x \right) >g\left( x \right) \right\} \right) =m\left( \left\{ x\in \left( 0,1 \right] :\frac{1}{x}>g\left( x \right) \right\} \right) 
\\
&\geqslant m\left( \left\{ x\in \left( 0,1 \right] :\frac{1}{x}>M \right\} \right) =m\left( \left\{ x\in \left( 0,1 \right] :x<\frac{1}{M} \right\} \right) 
\\
&=m\left( \left( 0,\frac{1}{M} \right) \right) =\frac{1}{M}>0.
\end{align*}
\end{proof}

\begin{definition}
设有一个与集合$E \subset \mathbb{R}^n$中的点$x$有关的命题$P(x)$。若除了$E$中的一个零测集以外,$P(x)$皆为真,则称$P(x)$在$E$上\textbf{几乎处处}是真的,并简记为$P(x)$,a. e.$ x\in E$。
\end{definition}

\begin{definition}
设$f(x), g(x)$是定义在$E \subset \mathbb{R}^n$上的可测函数。若有
\begin{align*}
m(\{x \in E: f(x) \neq g(x)\}) = 0,
\end{align*}
则称$f(x)$与$g(x)$在$E$上\textbf{几乎处处相等},也称为$f(x)$与$g(x)$是\textbf{对等的},记为$$f(x) = g(x),\,\mathrm{a}.\mathrm{e}.\,x \in E.$$

设$f(x)$是定义在$E \subset \mathbb{R}^n$上的可测函数。若有
\begin{align*}
m(\{x \in E: |f(x)| = +\infty\}) = 0,
\end{align*}
则称$f(x)$在$E$上是\textbf{几乎处处有限的},并记为
\begin{align*}
|f(x)| < \infty,\,\mathrm{a}.\mathrm{e}.\,x \in E.
\end{align*}
\end{definition}
\begin{remark}
可测函数有界与有限的区别:$|f(x)| < +\infty$,a. e. $x \in E$与$|f(x)| < M$($M$是某个实数),a. e. $x \in E$是不同的。后者蕴含前者,但反之不然。此即\textbf{可测函数有界必有限,但有限不一定有界},例如\(E = (0, 1]\),\(f(x) = 1/x\)在\(E\)上每一点都有限,但\(f(x)\)在\(E\)上无界.如果$E=[0,1]$,则\(f(x) = 1/x\)在$x=0$处无限,即$f(0)=+\infty$,也即$\{x\in E:f(x)=+\infty\}={0}.$
\end{remark}

\begin{theorem}
设$f(x),g(x)$是定义在$E \subset \mathbb{R}^n$上的广义实值函数,$f(x)$是$E$上的可测函数。若$f(x) = g(x)$, a.e. $x \in E$,则$g(x)$在$E$上可测。
\end{theorem}
\begin{remark}
由这个定理可知,对一个可测函数来说,当改变它在零测集上的值时不会改变函数的可测性。 
\end{remark}
\begin{proof}
令$A = \{x: f(x) \neq g(x)\}$,则$m(A) = 0$且$E \setminus A$是可测集。对于$t \in \mathbb{R}$,我们有
\begin{align*}
\{x \in E: g(x) > t\} 
&= \{x \in E \setminus A: g(x) > t\} \cup \{x \in A: g(x) > t\} \\
&= \{x \in E \setminus A: f(x) > t\} \cup \{x \in A: g(x) > t\}.
\end{align*}
根据$f(x)$在$E$上的可测性可知,上式右端第一个点集是可测的,而第二个点集是零测集的子集仍是零测集,也是可测集.从而可知左端点集是可测的。
\end{proof}

\begin{proposition}[局部有界化]\label{proposition:局部有界化}
设$0 < m(A) < +\infty$,$f(x)$是$A \subset \mathbb{R}^n$上的非负可测函数,且有$0 < f(x) < +\infty$,a. e. $x \in A$,则对任给的$\delta$:$0 < \delta < m(A)$,存在$B \subset A$以及自然数$k_0$,使得
\begin{align*}
m(A \setminus B) < \delta, \quad \frac{1}{k_0} \leqslant f(x) \leqslant k_0, \quad x \in B.
\end{align*}
\end{proposition}
\begin{proof}
记$A_k = \{x \in A: 1/k \leqslant f(x) \leqslant k\}$($k = 1, 2, \cdots$),$Z_1 = \{x \in A: f(x) = 0\}$,$Z_2 = \{x \in A: f(x) = +\infty\}$,易知$m(Z_1) = m(Z_2) = 0$,且有
\begin{align*}
A = \left( \bigcup_{k = 1}^{\infty} A_k \right) \cup Z_1 \cup Z_2, A_k \subset A_{k + 1} \quad (k = 1, 2, \cdots).
\end{align*}
于是
\begin{align*}
m(A)=m\left( \bigcup_{k = 1}^{\infty} A_k \right)+m(Z_1)+m(Z_2)=m\left( \lim_{k \to \infty} A_k \right)\xlongequal{\text{\hyperref[theorem:递增可测集列的测度运算]{递增可测集列的测度运算}}}\lim_{k\to \infty}m(A_k).
\end{align*}
从而存在$k_0$,使得$m(A \setminus A_{k_0}) < \delta$。
取$B = A_{k_0}$,即得所证。 
\end{proof}

\begin{definition}[简单函数]
设$f(x)$是$E \subset \mathbb{R}^n$上的实值函数。若
\begin{align*}
\{y: y = f(x), x \in E\}
\end{align*}
是有限集,则称$f(x)$为$E$上的\textbf{简单函数}。
\end{definition}

\begin{theorem}\label{theorem:简单函数是有限个特征函数的线性组合}
$f(x)$是$E$上的简单函数,则可设
\begin{align*}
\{y:y=f(x),x\in E\}=\{c_1,c_2,\cdots,c_p\}.
\end{align*}
再令
\begin{align*}
E_i=\{x\in E:f(x)=c_i\},i=1,2,\cdots,p.
\end{align*}
于是
\begin{gather*}
E = \bigcup_{i = 1}^{p} E_i, \quad E_i \cap E_j = \varnothing, \quad i, j = 1, 2, \cdots, p. \\
f(x) = c_i, \quad x \in E_i,\quad i = 1, 2, \cdots, p.
\end{gather*}
故可将$f$记为
\begin{align*}
f(x) = \sum_{i = 1}^{p} c_i \chi_{E_i}(x),x \in E.
\end{align*}
从而\textbf{简单函数是有限个特征函数的线性组合}。特别地,当每个$E_i$是矩体(这里允许取无限大的矩体)时,称$f(x)$是\textbf{阶梯函数}。 
\end{theorem}

\begin{corollary}
$f(x)$是$E$上的简单函数的充要条件是$f(x)$可以写成有限个特征函数的线性组合.
\end{corollary}

\begin{proposition}\label{proposition:简单函数的性质}
\begin{enumerate}
\item 若$f(x),g(x)$是$E$上的简单函数,则$$f(x) \pm g(x),\quad f(x) \cdot g(x)$$
也是$E$上的简单函数.

\item 
\end{enumerate}
\end{proposition}
\begin{proof}
\begin{enumerate}
\item 由\refthe{theorem:简单函数是有限个特征函数的线性组合}易证.

\item 
\end{enumerate}
\end{proof}

\begin{definition}[可测简单函数]
设$f(x)$是$E$上的简单函数,则
\begin{align*}
f(x)= \sum_{i = 1}^{p} c_i \chi_{E_i}(x),x \in E.
\end{align*}
其中$E = \bigcup_{i = 1}^{p} E_i,  E_i \cap E_j = \varnothing, i, j = 1, 2, \cdots, p.$
若上式中的每个$E_i$都是可测集,则称$f(x)$是$E$上的\textbf{可测简单函数}。
\end{definition}

\begin{theorem}[简单函数逼近定理]\label{theorem:简单函数逼近定理}
(1) 若$f(x)$是$E$上的非负可测函数,则存在非负可测的简单函数渐升列:
$\varphi_k(x) \leqslant \varphi_{k + 1}(x)$,$k = 1, 2, \cdots$,
使得
\begin{align}
\lim_{k \to \infty} \varphi_k(x) = f(x), \quad x \in E; \label{eq:3.5}
\end{align}
(2) 若$f(x)$是$E$上的可测函数,则存在可测简单函数列$\{\varphi_k(x)\}$,使得$|\varphi_k(x)| \leqslant |f(x)|$,且有
$\lim_{k \to \infty} \varphi_k(x) = f(x)$,$x \in E$。
若$f(x)$还是有界的,则上述的收敛是一致的。
\end{theorem}
\begin{remark}
注意$\bigcup_{j=1}^{k2^k}E_{k, j}=\bigcup_{j=1}^{k2^k}\left\{x \in E: \frac{j - 1}{2^k} \leqslant f(x) < \frac{j}{2^k} \right\}=\{x\in E:0\leqslant f(x)<k\}.$
\end{remark}
\begin{note}
$\varphi_k(x)$随着$k$增大,对$[0,k]$区间的分割就越细.
\end{note}
\begin{proof}
(1) 对任意的自然数$k$,将$[0, k]$划分为$k2^k$等分,并记
\begin{align*}
E_{k, j} &= \left\{x \in E: \frac{j - 1}{2^k} \leqslant f(x) < \frac{j}{2^k} \right\}, \\
E_{k} &= \{x \in E: f(x) \geqslant k\}, \\
j &= 1, 2, \cdots, k2^k, \quad k = 1, 2, \cdots.
\end{align*}
作函数列
\begin{align*}
\varphi_k(x) &= 
\begin{cases}
\frac{j - 1}{2^k}, & x \in E_{k, j}, \\
k, & x \in E_{k},
\end{cases} \\
j &= 1, 2, \cdots, k2^k, \quad k = 1, 2, \cdots,
\end{align*}
且写成
\begin{align*}
\varphi_k(x) = k\chi_{E_{k}}(x) + \sum_{j = 1}^{k2^k} \frac{j - 1}{2^k} \chi_{E_{k, j}}(x), \quad x \in E.
\end{align*}
由\refthe{theorem:定理3.2}可知,每个$\varphi_k(x)$都是非负可测简单函数.现在考虑其单调性.对$\forall k\in \mathbb{N}$,固定$k$。对$\forall x_0\in E$,
\one 当$0\leqslant f\left( x_0 \right) <k + 1$时,即$x_0\in \left\{ x\in E:0\leqslant f\left( x \right) <k + 1 \right\} =\bigcup_{j = 1}^{(k + 1)2^{k + 1}}E_{k + 1,j}$,则存在$j_0\in \left\{ 1,2,\cdots,(k + 1)2^{k + 1} \right\}$,使得$x_0\in E_{k + 1,j_0}=\left\{ x\in E:\frac{j_0 - 1}{2^{k + 1}}\leqslant f\left( x \right) <\frac{j_0}{2^{k + 1}} \right\}$,即
\begin{align}
\frac{j_0 - 1}{2^{k + 1}}\leqslant f\left( x_0 \right) <\frac{j_0}{2^{k + 1}}, \quad \varphi _{k + 1}\left( x_0 \right) =\frac{j_0 - 1}{2^{k + 1}}. \label{eq:100.61}
\end{align}
(I) 当$j_0\in \left[ 1,k2^{k + 1} \right]$时,
(i) 当$j_0 - 1$为偶数时,则此时$j_0 + 1$也是偶数,从而$j_0 + 1\in \left[ 2,k2^{k + 1} \right]$。于是$\frac{j_0 + 1}{2}\in \left[ 1,k2^k \right] \cap \mathbb{N}$。又注意到
\begin{align*}
\frac{j_0 - 1}{2^{k + 1}}=\frac{\frac{j_0 - 1}{2}}{2^k}, \quad \frac{\frac{j_0 - 1}{2}+1}{2^k}=\frac{j_0 + 1}{2^{k + 1}}>\frac{j_0}{2^{k + 1}},
\end{align*}
从而
\begin{align*}
\frac{j_0 - 1}{2^{k + 1}}\leqslant f\left( x_0 \right) <\frac{j_0}{2^{k + 1}}<\frac{\frac{j_0 + 1}{2}}{2^k}.
\end{align*}
故此时就有$x_0\in \left\{ x\in E:\frac{\frac{j_0 - 1}{2}}{2^k}\leqslant f\left( x \right) <\frac{\frac{j_0 + 1}{2}}{2^k} \right\} =E_{k,\frac{j_0 + 1}{2}}$。因此再结合\eqref{eq:100.61}式可得
\begin{align*}
\varphi _k\left( x_0 \right) =\frac{\frac{j_0 - 1}{2}}{2^k}=\frac{j_0 - 1}{2^{k + 1}}=\varphi _{k + 1}\left( x_0 \right) \leqslant f\left( x_0 \right).
\end{align*}
(ii)当$j_0 - 1$为奇数时,则此时$j_0 - 2,j_0$都是偶数。再结合$j_0\in \left[ 1,k2^{k + 1} \right]$可知$j_0\in \left[ 2,k2^{k + 1} \right]$。于是$\frac{j_0}{2}\in \left[ 1,k2^k \right] \cap \mathbb{N}$。又注意到
\begin{align*}
\frac{\frac{j_0 - 2}{2}}{2^k}=\frac{j_0 - 2}{2^{k + 1}}<\frac{j_0 - 1}{2^{k + 1}}, \quad \frac{j_0}{2^{k + 1}}=\frac{\frac{j_0}{2}}{2^k}.
\end{align*}
从而
\begin{align*}
\frac{\frac{j_0 - 2}{2}}{2^k}<\frac{j_0 - 1}{2^{k + 1}}\leqslant f\left( x_0 \right) <\frac{j_0}{2^{k + 1}}=\frac{\frac{j_0}{2}}{2^k}.
\end{align*}
故此时就有$x_0\in \left\{ x\in E:\frac{\frac{j_0 - 2}{2}}{2^k}\leqslant f\left( x \right) <\frac{\frac{j_0}{2}}{2^k} \right\} =E_{k,\frac{j_0}{2}}$。因此再结合\eqref{eq:100.61}式可得
\begin{align*}
\varphi _k\left( x_0 \right) =\frac{\frac{j_0 - 2}{2}}{2^k}=\frac{j_0 - 2}{2^{k + 1}}<\frac{j_0 - 1}{2^{k + 1}}=\varphi _{k + 1}\left( x_0 \right) \leqslant f\left( x_0 \right).
\end{align*}
(II) 当$j_0\in \left[ k2^{k + 1}+1,(k + 1)2^{k + 1} \right]$时,则由\eqref{eq:100.61}式可知,此时有
\begin{align*}
k\leqslant \frac{j_0 - 1}{2^{k + 1}}=\varphi _{k + 1}\left( x_0 \right) \leqslant f\left( x_0 \right) <\frac{j_0}{2^{k + 1}}\leqslant k + 1.
\end{align*}
于是此时$x_0\in \left\{ x\in E:f\left( x \right) \geqslant k \right\} =E_k$,从而此时$\varphi _k\left( x_0 \right) =k$。故此时就有
\begin{align*}
\varphi _k\left( x_0 \right) =k\leqslant \frac{j_0 - 1}{2^{k + 1}}=\varphi _{k + 1}\left( x_0 \right) \leqslant f\left( x_0 \right) <\frac{j_0}{2^{k + 1}}\leqslant k + 1.
\end{align*}
\two 当$f\left( x_0 \right) \geqslant k + 1$时,则此时$x_0\in E_{k + 1}\subset E_k$。从而此时就有
\begin{align*}
\varphi _k\left( x_0 \right) =k<k + 1=\varphi _{k + 1}\left( x_0 \right) \leqslant f\left( x_0 \right).
\end{align*}
综上所述,我们有 
\begin{gather*}
\varphi_k(x) \leqslant \varphi_{k + 1}(x) \leqslant f(x), \quad \varphi_k(x) \leqslant k, \\
x \in E, \quad k = 1, 2, \cdots.
\end{gather*}
现在,对任意的$x_0 \in E$,\one 若$f(x_0) \leqslant M$,则对$\forall k > M$,都有$x_0\in \{x\in E:0\leqslant f(x)<k\}=\bigcup_{j=1}^{\infty}E_{k,j}$.从而存在$j_0\in \{1,2,\cdots,k2^k\},$使得$x_0\in E_{k,j_0}$,即
\begin{align*}
\frac{j_0-1}{2^k}\leqslant f(x_0)<\frac{j_0}{2^k},\quad \varphi_k(x_0)=\frac{j_0-1}{2^k}.
\end{align*}
于是
\begin{align*}
0 \leqslant f(x_0) - \varphi_k(x_0) \leqslant \frac{1}{2^k},
\end{align*}
令$k\to \infty$得$\lim_{k \to \infty} \varphi_k(x_0) = f(x_0).$

\two 若$f(x_0) = +\infty$,则对$\forall k\in \mathbb{N}$,都有$x_0\in E_k$.从而此时就有$\varphi_k(x_0) = k (k = 1, 2, \cdots).$令$k\to \infty$得$$\lim_{k \to \infty} \varphi_k(x_0) =+\infty = f(x_0).$$

综上,再由$x_0$的任意性可得
\begin{align*}
\lim_{k \to \infty} \varphi_k(x) = f(x), \quad x \in E.
\end{align*}

(2) 记$f(x) = f^+(x) - f^-(x)$。由(1)知存在可测简单函数列$\{\varphi_k^{(1)}(x)\}$及$\{\varphi_k^{(2)}(x)\}$,满足
\begin{align*}
\lim_{k \to \infty} \varphi_k^{(1)}(x) = f^+(x), \quad \lim_{k \to \infty} \varphi_k^{(2)}(x) = f^-(x), \quad x \in E.
\end{align*}
显然,$\varphi_k^{(1)}(x) - \varphi_k^{(2)}(x)$是可测简单函数,且有
\begin{align*}
\lim_{k \to \infty} [\varphi_k^{(1)}(x) - \varphi_k^{(2)}(x)] = f^+(x) - f^-(x) = f(x), \quad x \in E.
\end{align*}
若在$E$上有$|f(x)| \leqslant M$,则当$k > M$时,由(1)同理可知
\begin{align*}
\sup_{x\in E} |f^+(x) - \varphi_k^{(1)}(x)| \leqslant \frac{1}{2^k},\\
\sup_{x\in E} |f^-(x) - \varphi_k^{(2)}(x)| \leqslant \frac{1}{2^k}.
\end{align*}
于是
\begin{align*}
\underset{x\in E}{\mathrm{sup}}\left| f\left( x \right) -\left[ \varphi _{k}^{(1)}(x)-\varphi _{k}^{(2)}(x) \right] \right|&=\underset{x\in E}{\mathrm{sup}}\left| \left[ f^+\left( x \right) -\varphi _{k}^{\left( 1 \right)}\left( x \right) \right] -\left[ f^-\left( x \right) -\varphi _{k}^{\left( 2 \right)}\left( x \right) \right] \right|
\\
&\leqslant \underset{x\in E}{\mathrm{sup}}\left( \left| f^+\left( x \right) -\varphi _{k}^{\left( 1 \right)}\left( x \right) \right|+\left| f^-\left( x \right) -\varphi _{k}^{\left( 2 \right)}\left( x \right) \right| \right) 
\\
&\leqslant \underset{x\in E}{\mathrm{sup}}\left| f^+\left( x \right) -\varphi _{k}^{\left( 1 \right)}\left( x \right) \right|+\underset{k\rightarrow \infty}{\lim}\underset{x\in E}{\mathrm{sup}}\left| f^-\left( x \right) -\varphi _{k}^{\left( 2 \right)}\left( x \right) \right|
\\
&\leqslant \frac{1}{2^{k-1}}\rightarrow 0,k\rightarrow \infty .
\end{align*}
从而知$\varphi_k^{(1)}(x) - \varphi_k^{(2)}(x)$是一致收敛于$f(x)$的。 
\end{proof}

\begin{definition}
对于定义在$E \subset \mathbb{R}^n$上的函数$f(x)$,称点集
\begin{align*}
\{x: f(x) \neq 0\}
\end{align*}
的闭包为$f(x)$的\textbf{支(撑)集},记为$\mathrm{supp}(f)$。若$f(x)$的支集是有界(即支(撑)集是紧集)的,则称$f(x)$是\textbf{具有紧支集的函数}. 
\end{definition}

\begin{corollary}\label{corollary:简单函数逼近定理的加强形式}
\hyperref[theorem:简单函数逼近定理]{简单函数逼近定理}中所说的可测简单函数列中的每一个均可取成具有紧支集的函数.
\end{corollary}
\begin{proof}
对每个$k$,令$g_k(x) = \varphi_k(x)\chi_{B(0,k)}(x) (x \in E)$,则$g_k(x)$仍是可测简单函数且具有紧支集.

对$\forall x_0 \in E$,则存在$k_0$,使得当$k \geqslant k_0$时有$x_0 \in B(0,k)$。此时可得
\begin{align*}
\lim_{k \to \infty} g_k(x_0) = \lim_{k \to \infty} \varphi_k(x_0) = f(x_0).
\end{align*}
故再由$x_0$的任意性可得
\begin{align*}
\lim_{k \to \infty} g_k(x) = \lim_{k \to \infty} \varphi_k(x) = f(x),\quad \forall x\in E.
\end{align*} 
并且若$\varphi_k(x)\leqslant \varphi_{k+1}(x),\forall x\in E$,则当$x\in E\cap B(0,k) \subset E\cap B(0,k + 1)$时,则此时我们有
\begin{align*}
g_k(x) = \varphi_k(x) \leqslant \varphi_{k + 1}(x) = g_{k + 1}(x).
\end{align*}
当$x\notin E\cap B(0,k)$时,显然有
\begin{align*}
g_k(x) = 0 \leqslant g_{k + 1}(x).
\end{align*}
综上可得
\begin{align*}
g_k(x) \leqslant g_{k + 1}(x), \forall x\in E.
\end{align*} 
\end{proof}

\begin{theorem}
设$f(x)$是$\mathbb{R}$上的实值可测函数,则存在函数值都是有理数的函数列$\{f_n(x)\}$,使得$f_n(x)$在$\mathbb{R}$上一致收敛且递增于$f(x)$。
\end{theorem}
\begin{proof}
作$E_{k,n} = \{x \in \mathbb{R}: f(x) \in [k2^{-n}, (k + 1)2^{-n}), k \in \mathbb{Z}, n \in \mathbb{N}\}$,且令
\begin{align*}
f_n(x) = \sum_{k = -\infty}^{\infty} k2^{-n} \chi_{E_{k,n}}(x),
\end{align*}
则由\refthe{theorem:简单函数逼近定理}同理可证$0 \leqslant f(x) - f_n(x) \leqslant 2^{-n}$以及$f_n(x)$关于$n$递增.从而可得$f_n(x) \nearrow f(x) (n \to \infty)$即得所证。 
\end{proof}






































































\end{document}
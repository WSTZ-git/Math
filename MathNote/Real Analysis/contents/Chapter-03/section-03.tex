\documentclass[../../main.tex]{subfiles}
\graphicspath{{\subfix{../../image/}}} % 指定图片目录,后续可以直接使用图片文件名。

% 例如:
% \begin{figure}[H]
% \centering
% \includegraphics[scale=0.4]{图.png}
% \caption{}
% \label{figure:图}
% \end{figure}
% 注意:上述\label{}一定要放在\caption{}之后,否则引用图片序号会只会显示??.

\begin{document}

\section{可测函数与连续函数的关系}

\subsection{Lusin定理}

\begin{lemma}\label{lemma:简单函数在无交闭集连续则在这些闭集的并上也连续}
设\(F_1, \cdots, F_n\)为\(\mathbb{R}^n\)中互不相交的闭集,记\(F = \bigcup_{k = 1}^{n} F_k\),则定义在$F$上的任意简单函数\(f(x) = \sum_{k = 1}^{n} c_k \chi_{F_k}(x)\)都是\(F\)上的连续函数.
\end{lemma}
\begin{proof}
设\(x_0 \in F\),则存在\(k_0\in \{1,2,\cdots,n\}\),使得\(x_0 \in F_{k_0}\). 由于\(F_1, \cdots, F_n\)互不相交,故\(x_0 \notin \bigcup_{k \neq k_0} F_k\). 又\(\bigcup_{k \neq k_0} F_k\)闭,则由\nrefpro{proposition:点集间的距离的性质}{(3)}可知\(d(x_0, \bigcup_{k \neq k_0} F_k) > 0\). 对\(\forall \varepsilon > 0\),记\(\delta = d(x_0, \bigcup_{k \neq k_0} F_k)\). 则当\(x \in F\cap B(x,\delta)\)时,有
\begin{align*}
d\left( x,\bigcup_{k\ne k_0}{F_k} \right) \geqslant slant d\left( x_0,\bigcup_{k\ne k_0}{F_k} \right) -d\left( x,x_0 \right) =\delta -d\left( x,x_0 \right) >0.
\end{align*}
于是由\nrefpro{proposition:点集间的距离的性质}{(2)}可知$x\notin F\backslash \bigcup_{k\ne k_0}{F_k}$,故
\(x \in F_{k_0}\). 从而
\begin{align*}
|f(x) - f(x_0)| = |c_{k_0} - c_{k_0}| = 0 < \varepsilon
\end{align*}
因此,\(f\)在点\(x_0\)连续,由\(x_0\)的任意性,\(f\)在\(F\)上连续. 
\end{proof}

\begin{theorem}[Lusin(卢津)定理]\label{theorem:Lusin(卢津)定理}
若\(f(x)\)是\(E \subset \mathbb{R}^{n}\)上的几乎处处有限的可测函数,则对任给的\(\delta > 0\),存在\(E\)中的闭集\(F\),\(m(E \setminus F) < \delta\),使得\(f(x)\)是\(F\)上的连续函数.
\end{theorem}
\begin{remark}
上述\hyperref[theorem:Lusin(卢津)定理]{Lusin定理}的结论不能改为:\(f(x)\)是\(E \setminus Z\)上的连续函数,其中\(m(Z) = 0\)(\hyperref[theorem:Lusin(卢津)定理]{Lusin定理}也可不用\hyperref[theorem:Egorov定理]{Egorov定理}来证明,见美国数学月刊(1988)). 粗略地讲,\hyperref[theorem:Lusin(卢津)定理]{Lusin定理}是把可测函数的不连续性局部连续化了. 
\end{remark}
\begin{remark}
1.\hypertarget{不妨假定f是实值函数的原因}{不妨妨假定}\(f(x)\)是实值函数的原因:假设已证\(f(x)\)是实值函数的情形,令
\begin{align*}
E_1 = \{x \in E : |f(x)| < +\infty\},E_2 = \{x \in E : |f(x)| = +\infty\}.
\end{align*}
则\(E_1 \cap E_2 = \varnothing\),\(E = E_1 \cup E_2\)。
由假设可知,对\(\forall \delta > 0\),存在闭集\(F \subset E_1 \subset E\),\(m(E_1 \setminus F) < \delta\),使得\(f(x)\)是\(F\)上的连续函数。
又由\(f(x)\)在\(E\)上几乎处处有限可知\(m(E_2) = 0\)。进而
\begin{align*}
m(E \setminus F) = m((E_1 \cup E_2) \cap F^c) 
= m((E_1 \setminus F) \cap (E_2 \setminus F)) 
= m(E_1 \setminus F) + m(E_2 \setminus F) 
< \delta.
\end{align*}
从而原结论成立。

2.\hypertarget{不妨设f有界的原因}{不妨设$f(x)$是有界函数的原因:} 假设已证\(f(x)\)有界的情形,则当\(f(x)\)无界时,令\(g(x) = \frac{f(x)}{1 + |f(x)|}\),则
\begin{align*}
g(x) = \frac{f(x)}{1 + f(x)} = 1 - \frac{1}{1 + f(x)} \leqslant slant 1, \quad \forall x \in \{x \in E : f(x) \geqslant slant 0\}; \\
g(x) = \frac{f(x)}{1 - f(x)} = \frac{1}{1 - f(x)} - 1 \geqslant slant -1, \quad \forall x \in \{x \in E : f(x) < 0\}.
\end{align*}
从而\(|g(x)| < 1\),\(\forall x \in E\)。即\(g(x)\)有界。于是由假设可知,对\(\forall \delta > 0\),存在\(E\)中的闭集\(F\),\(m(E \setminus F) < \delta\),使得\(g(x)\)是\(F\)上的连续函数。

又注意到
\begin{align*}
f(x) = g(x)(1 + |f(x)|) = \frac{g(x)}{1 + |f(x)|} = \frac{g(x)}{1 - \frac{|f(x)|}{1 + |f(x)|}} = \frac{g(x)}{1 - |g(x)|},
\end{align*}
故由连续函数的性质可知,此时\(f(x)\)也是\(F\)上的连续函数。从而原结论成立。
\end{remark}
\begin{proof}
\hyperlink{不妨假定f是实值函数的原因}{不妨假定}\(f(x)\)是实值函数,这是因为$f(x)$几乎处处有限,从而
\begin{align*}
m(\{x \in E: |f(x)| = +\infty\}) = 0.
\end{align*}
(1)首先考虑\(f(x)\)是可测简单函数的情形:
\[f(x) = \sum_{i = 1}^{p} c_{i} \chi_{E_{i}}(x), x \in E = \bigcup_{i = 1}^{p} E_{i}, E_{i} \cap E_{j} = \varnothing (i \neq j).\]
此时,由\refthe{theorem:定理2.13}可知,对任给的\(\delta > 0\)以及每个\(E_{i}\),可作\(E_{i}\)中的闭集\(F_{i}\),使得
\[m(E_{i} \setminus F_{i}) < \frac{\delta}{p}, \quad i = 1,2,\cdots,p.\]
显然\(F_{1},F_{2},\cdots,F_{p}\)是互不相交的闭集,于是由\reflem{lemma:简单函数在无交闭集连续则在这些闭集的并上也连续}可知\(f(x)\)在\(F = \bigcup_{i = 1}^{p} F_{i}\)上连续.由\hyperref[theorem:闭集的运算性质]{闭集的运算性质}可知\(F\)也是闭集,且由\nrefthe{theorem:集族的并和交的基本性质}{(3)}有
\begin{align*}
m(E \setminus F) \leqslant slant m(\bigcup_{i=1}^{\infty}(E_i\backslash F_i))=\sum_{i = 1}^{p} m(E_{i} \setminus F_{i}) < \sum_{i = 1}^{p} \frac{\delta}{p} = \delta.
\end{align*}
(2)其次,考虑\(f(x)\)是一般可测函数的情形. 由于可作变换
\[g(x) = \frac{f(x)}{1 + |f(x)|} \quad \left(f(x) = \frac{g(x)}{1 - |g(x)|}\right),\]
故\hyperlink{不妨设f有界的原因}{不妨假定}\(f(x)\)是有界函数. 根据\hyperref[theorem:简单函数逼近定理]{简单函数逼近定理}可知,存在可测简单函数列\(\{\varphi_{k}(x)\}\)在\(E\)上一致收敛于\(f(x)\). 现在对任给的\(\delta > 0\)以及每个\(\varphi_{k}(x)\),由(1)可知存在\(E\)中的闭集\(F_{k}\):\(m(E \setminus F_{k}) < \frac{\delta}{2^{k}}\),使得\(\varphi_{k}(x)\)在\(F_{k}\)上连续. 令\(F = \bigcap_{k = 1}^{\infty} F_{k}\),则\(F \subset E\),又由\hyperref[theorem:闭集的运算性质]{闭集的运算性质}可知$F$为闭集.且有
\begin{align*}
m(E\setminus F)&=m(E\cap \left( \bigcap_{k=1}^{\infty}{F_k} \right) ^c)=m(E\cap \bigcup_{k=1}^{\infty}{F_{k}^{c}})
\\
&=m(\bigcup_{k=1}^{\infty}{\left( E\backslash F_k \right)})\leqslant slant \sum_{k=1}^{\infty}{m(E}\setminus F_k)<\delta .
\end{align*}
因为每个\(\varphi_{k}(x)\)在\(F\)上都是连续的,所以根据一致收敛性可知,\(f(x)\)在\(F\)上连续. 
\end{proof}

\begin{theorem}[Lusin定理的逆定理$\,$可测函数的又一定义]\label{theorem:Lusin定理的逆定理}
设 \( f(x) \) 为可测集 \( E \) 上几乎处处有限的实值函数, 若对 \( \forall\,\delta > 0 \), 存在闭集 \( F_{\delta} \subset E \), 使得 \( m(E - F_{\delta}) < \delta \), 且 \( f(x) \) 在 \( F_{\delta} \) 上连续, 则 \( f(x) \) 是 \( E \) 上的可测函数.
\end{theorem}
\begin{proof}
对每个 \( n \in \mathbb{N} \), 都存在闭集 \( F_{n} \subset E \) 使得 \( m(E - F_{n}) < 1/n \), 且 \( f(x) \) 在 \( F_{n} \) 上连续, 故 \( f(x) \) 在 \( F_{n} \) 上可测.
记 \( F = \bigcup_{n = 1}^{\infty} F_{n} \), 则 \( f(x) \) 在 \( F \) 上可测. 由于 \( F_{n} \subset F \), \( n = 1, 2, \cdots \), 故
\[
m(E - F) \leqslant slant m(E - F_{n}) < \frac{1}{n}, \quad n = 1, 2, \cdots
\]
令 \( n \to \infty \) 得, \( m(E - F) = 0 \). 从而 \( f(x) \) 在 \( E - F \) 上可测. 因此, \( f(x) \) 在 \( E = F \cup (E - F) \) 上可测.
\end{proof}

\begin{corollary}\label{corollary:推论3.19}
若\(f(x)\)是\(E \subset \mathbb{R}^{n}\)上几乎处处有限的可测函数,则对任给的\(\delta > 0\),存在\(\mathbb{R}^{n}\)上的一个连续函数\(g(x)\),使得
\begin{align*}
m(\{x \in E: f(x) \neq g(x)\}) < \delta; 
\end{align*}
若\(E\)还是有界集,则可使上述\(g(x)\)具有紧支集.
\end{corollary}
\begin{proof}
由\hyperref[theorem:Lusin(卢津)定理]{Lusin定理}可知,对任给的\(\delta > 0\),存在\(E\)中的闭集\(F\),\(m(E \setminus F) < \delta\)且\(f(x)\)是\(F\)上的连续函数,从而根据\hyperref[theorem:连续函数延拓定理]{连续函数延拓定理(2)},存在\(\mathbb{R}^{n}\)上的连续函数\(g(x)\),使得
\[f(x) = g(x), \quad x \in F.\]
因为\(\{x \in E: f(x) \neq g(x)\} \subset E \setminus F\),所以得
\[m(\{x \in E: f(x) \neq g(x)\}) \leqslant slant m(E \setminus F) < \delta.\]
若\(E\)是有界集,不妨设\(E \subset B(0, k)\),则作\(\mathbb{R}^{n}\)上的连续函数\(\varphi(x)\),\(0 \leqslant slant \varphi(x) \leqslant slant 1\),且满足($\varphi$在$B(0,k)\backslash F$中连续且端点连续连接)
\[\varphi(x) = 
\begin{cases}
1, & x \in F, \\
0, & x \notin B(0, k).
\end{cases}\]
从而将上述\(g(x)\)换成\(g(x) \cdot \varphi(x)\).令$A=\{x\in \mathbb{R}^n:g(x)\ne 0\}$,则$g(x)$的支集为$\overline{A}\subset B(0,k)$.于是$\overline{A}$为有界闭集,进而$g(x)$具有紧支集.
\end{proof}

\begin{corollary}\label{corollary:推论3.20}
若\(f(x)\)是\(E \subset \mathbb{R}^{n}\)上几乎处处有限的可测函数,则存在\(\mathbb{R}^{n}\)上的连续函数列\(\{g_{k}(x)\}\),使得
\begin{align*}
\lim_{k \to \infty} g_{k}(x) = f(x), \,\text{a.e. } x \in E. 
\end{align*}
\end{corollary}
\begin{proof}
由\refcor{corollary:推论3.19}可知,对于任意的趋于零的正数列\(\{\varepsilon_{k}\}\)与\(\{\delta_{k}\}\),存在\(\mathbb{R}^{n}\)上的连续函数列\(\{g_{k}(x)\}\),使得
\[m(\{x \in E: |f(x) - g_{k}(x)| \geqslant slant \varepsilon_{k}\}) < \delta_{k}, \quad k = 1,2,\cdots.\]
这说明\(\{g_{k}(x)\}\)在\(E\)上依测度收敛于\(f(x)\). 从而根据\hyperref[theorem:Riesz定理]{Riesz定理},可选子列\(\{g_{k_{i}}(x)\}\),使得
\[\lim_{i \to \infty} g_{k_{i}}(x) = f(x), \quad \text{a.e. } x \in E.\] 
\end{proof}
\begin{remark}
我们知道,\(\mathbb{R}\)上的Dirichlet函数
\[f(x)=
\begin{cases}
1, & x\text{ 是有理数},\\
0, & x\text{ 是无理数}
\end{cases}\]
可以表示为(双重指标)连续函数列的累次极限:
\begin{align*}
\lim_{n \to \infty}\lim_{k \to \infty}[\cos(n!2\pi x)]^{2k}=f(x),\quad x\in\mathbb{R}.
\end{align*}
然而,并不存在\(\mathbb{R}\)上的连续函数列\(\{g_{k}(x)\}\),使得
\begin{align*}
\lim_{k \to \infty}g_{k}(x)=f(x),\quad x\in\mathbb{R}.
\end{align*} 
\end{remark}

\begin{example}
若\(f(x)\)是\(\mathbb{R}\)上的实值可测函数,且对任意的\(x, y \in \mathbb{R}\),有\(f(x + y) = f(x) + f(y)\),则\(f(x)\)是连续函数.
\end{example}
\begin{proof}
因为\(f(x + h) - f(x) = f(h)\)以及\(f(0) = 0\),所以只需证明\(f(x)\)在\(x = 0\)处连续即可. 根据\hyperref[theorem:Lusin(卢津)定理]{Lusin定理},可作有界闭集\(F\):\(m(F)>0\),使得\(f(x)\)在\(F\)上(一致)连续,即对任意的\(\varepsilon > 0\),存在\(\delta_1 > 0\),有
\begin{align*}
|f(x) - f(y)| < \varepsilon, \quad |x - y| < \delta_1, \quad x, y \in F.
\end{align*}
现在研究\(F - F\). 由\hyperref[theorem:Steinhaus定理]{Steinhaus定理}知道,存在\(\delta_2 > 0\),使得
\begin{align*}
F - F \supset [-\delta_2, \delta_2].
\end{align*}
取\(\delta = \min\{\delta_1, \delta_2\}\),则当\(z \in [-\delta, \delta]\)时,由于存在\(x, y \in F\),使得\(z = x - y\),故可得
\begin{align*}
|f(z)| = |f(x - y)| = |f(x) - f(y)| < \varepsilon.
\end{align*}
这说明\(f(x)\)在\(x = 0\)处是连续的.
\end{proof}

\begin{example}
设\(f(x)\)是\(I = (a, b)\)上的实值可测函数. 若\(f(x)\)具有中值(下)凸性质:
\[f\left(\frac{x + y}{2}\right) \leqslant slant \frac{f(x) + f(y)}{2}, \quad x, y \in I,\]
则\(f \in C(I)\).
\end{example}
\begin{proof}
根据数学分析的理论易知,若\(f(x)\)是\(I\)上的有界函数,则\(f \in C(I)\).

对此,假定\(f(x)\)在\(x = x_0 \in I\)处不连续,且考查区间\([x_0 - 2\delta, x_0 + 2\delta] \subset I\),其中存在\(\{\xi_k\}\):
\[\xi_k \in (x_0 - \delta, x_0 + \delta), \quad f(\xi_k) \geqslant slant k \quad (k = 1,2,\cdots).\]
对于任意的\(x \in (\xi_k - \delta, \xi_k + \delta)\),显然有
\[x_0 - 2\delta \leqslant slant x \leqslant slant x_0 + 2\delta, \quad x_0 - 2\delta \leqslant slant x' \stackrel{\text{def}}{=} 2\xi_k - x \leqslant slant x_0 + 2\delta.\]
由\(2\xi_k = x' + x\)可知\(2f(\xi_k) \leqslant slant f(x) + f(x')\),从而必有\(f(x) \geqslant slant k\)或者\(f(x') \geqslant slant k\).
这说明
\[m(\{x \in (\xi_k - \delta, \xi_k + \delta) : f(x) \geqslant slant k\}) \geqslant slant \delta.\]
也就是说,对于任意大的自然数\(k\),均有
\begin{align*}
m(\{x_0 - 2\delta \leqslant slant x \leqslant slant x_0 + 2\delta : f(x) \geqslant slant k\}) \geqslant slant \delta.
\end{align*}
从而导致\(f(x_0) = +\infty\),矛盾,即得所证. 
\end{proof}



\subsection{复合函数的可测性}

\begin{lemma}\label{lemma:函数可测的充要条件1}
若\(f(x)\)是定义在\(\mathbb{R}^{n}\)上的实值函数,则\(f(x)\)在\(\mathbb{R}^{n}\)上可测的充分必要条件是,对于\(\mathbb{R}\)中的任一开集\(G\),\(f^{-1}(G)\)是可测集.
\end{lemma}
\begin{proof}
{\heiti 充分性:}对$\forall t\i \mathbb{R}$,显然$(t,+\infty)$可测,故由充分性的假设可知$f^{-1}((t,+\infty))$也可测,因此$f(x)$在$\mathbb{R}^n$上可测.

{\heiti 必要性:}由假设知\(f^{-1}((t, +\infty))\)是可测集,故知对任意的区间\((a, b) \subset \mathbb{R}\),点集
\begin{align*}
f^{-1}((a, b)) = f^{-1}((a, +\infty)) \setminus f^{-1}([b, +\infty))
\end{align*}
是可测的. 若\(G \subset \mathbb{R}\)是开集,则由\hyperref[theorem:开集构造定理]{开集构造定理(1)}可设\(G = \bigcup_{k \geqslant slant 1} (a_k, b_k)\),从而根据
\begin{align*}
f^{-1}(G) = \bigcup_{k \geqslant slant 1} f^{-1}(a_k, b_k)
\end{align*}
可知\(f^{-1}(G)\)是可测集.
\end{proof}

\begin{theorem}\label{theorem:连续函数复合可测函数也可测}
设\(f(x)\)是\(\mathbb{R}\)上的连续函数,\(g(x)\)是\(\mathbb{R}\)上的实值可测函数,则复合函数\(h(x) = f(g(x))\)是\(\mathbb{R}\)上的可测函数.
\end{theorem}
\begin{remark}
当\(f(x)\)是可测函数而\(g(x)\)是连续函数时,\(f(g(x))\)就不一定是可测函数(见\refexa{example:复合函数的可测性反例1}).
\end{remark}
\begin{proof}
由$f$的连续性可知,对任一开集\(G \subset \mathbb{R}\),都有\(f^{-1}(G)\)是开集.再根据\(g(x)\)的可测性,由\refcor{lemma:函数可测的充要条件1}可知\(g^{-1}(f^{-1}(G))\)是可测集. 这说明\(h(x) = f(g(x))\)是\(\mathbb{R}\)上的可测函数. 
\end{proof}

\begin{example}\label{example:复合函数的可测性反例1}
设\(\varPhi(x)\)是\([0, 1]\)上的 Cantor 函数,令
\[
\varPsi(x)=\frac{x+\varPhi(x)}{2},
\]
则\(\varPsi(x)\)是\([0, 1]\)上的严格递增的连续函数. 记\(C\)是\([0, 1]\)中的 Cantor 集,\(W\)是\(\varPsi(C)\)中的不可测子集.

现在令\(f(x)\)是点集\(\varPsi^{-1}(W)\)上的特征函数,作
\[
g(x)=\varPsi^{-1}(x),\quad x\in[0, 1].
\]
显然,\(f(x) = 0\),a.e. \(x \in [0, 1]\),\(g(x)\)是\([0, 1]\)上的严格递增的连续函数. 易知\(f(g(x))\)在\([0, 1]\)上不是可测函数.
\end{example}
\begin{remark}
该例说明,存在可测函数\(f(x)\),它有反函数\(f^{-1}(x)\),但\(f^{-1}(x)\)不可测. 
\end{remark}

\begin{theorem}
设\(T: \mathbb{R}^{n} \to \mathbb{R}^{n}\)是连续变换,当\(Z \subset \mathbb{R}^{n}\)且\(m(Z) = 0\)时,\(T^{-1}(Z)\)是零测集. 若\(f(x)\)是\(\mathbb{R}^{n}\)上的实值可测函数,则\(f(T(x))\)是\(\mathbb{R}^{n}\)上的可测函数.
\end{theorem}
\begin{proof}
设\(G\)是\(\mathbb{R}\)中的任一开集,由假设知道\(f^{-1}(G)\)是可测集. 不妨设\(f^{-1}(G) = H \setminus Z\),其中\(m(Z) = 0\),且\(H\)是\(G_{\delta}\)型集. 由假设可知\(T^{-1}(Z)\)是零测集以及\(T^{-1}(H)\)是\(G_{\delta}\)型集,故从等式
\begin{align*}
T^{-1}(f^{-1}(G)) = T^{-1}(H) \setminus T^{-1}(Z)
\end{align*}
立即得出\(T^{-1}(f^{-1}(G))\)是可测集. 这说明\(f(T(x))\)是\(\mathbb{R}^{n}\)上的可测函数.
\end{proof}

\begin{corollary}
设\(f(x)\)是\(\mathbb{R}^{n}\)的实值可测函数,\(T: \mathbb{R}^{n} \to \mathbb{R}^{n}\)是非奇异线性变换,则\(f(T(x))\)是\(\mathbb{R}^{n}\)上的可测函数.
\end{corollary}
\begin{proof}

\end{proof}

\begin{example}
若\(f(x)\)是\(\mathbb{R}^{n}\)上的可测函数,则\(f(x - y)\)是\(\mathbb{R}^{n} \times \mathbb{R}^{n}\)上的可测函数.
\end{example}
\begin{proof}
(i) 记\(F(x, y) = f(x)\),\((x, y) \in \mathbb{R}^{n} \times \mathbb{R}^{n}\),则因对\(t \in \mathbb{R}\),有
\begin{align*}
\{(x, y): F(x, y) > t\} = \{(x, y): f(x) > t, y \in \mathbb{R}^{n}\},
\end{align*}
所以\(F(x, y)\)是\(\mathbb{R}^{n} \times \mathbb{R}^{n}\)上的可测函数.

(ii) 作\(\mathbb{R}^{n} \times \mathbb{R}^{n}\)到\(\mathbb{R}^{n} \times \mathbb{R}^{n}\)的非奇异线性变换\(T\):
\[
\begin{cases}
x = \xi - \eta, \\
y = \xi + \eta,
\end{cases}
\quad (\xi, \eta) \in \mathbb{R}^{n} \times \mathbb{R}^{n}.
\]
易知在变换\(T\)下,\(F(x, y)\)变为\(F(\xi - \eta, \xi + \eta) = f(\xi - \eta)\),从而\(f(\xi - \eta)\)是\(\mathbb{R}^{n} \times \mathbb{R}^{n}\)上的可测函数.
\end{proof}

\begin{example}
设\(f(x)\)是\((0, +\infty)\)上的实值可测函数,令\(F(x, y) = f(y / x)\)(\(0 < x, y < +\infty\)),则\(F(x, y)\)是\((0, +\infty) \times (0, +\infty)\)上的二元可测函数.
\end{example}
\begin{proof}
令\(g(\theta) = f(\tan\theta)\),\(0 < \theta < \pi / 2\). 因为\(y = \tan x\)的反函数是绝对连续的,它把零测集变为零测集(见第五章),所以\(g(\theta)\)在\((0, \pi / 2)\)上可测. 从而对\(t \in \mathbb{R}\),点集
\begin{align*}
E = \{\theta: 0 < \theta < \pi / 2, g(\theta) > t\}
\end{align*}
是可测集. 又由于我们有
\begin{align*}
&\{(x, y): 0 < x < +\infty, 0 < y < +\infty, F(x, y) > t\} \\
=&\{(r\cos\theta, r\sin\theta): 0 < r < +\infty, \theta \in E\} = S_{E}(0, +\infty),
\end{align*}
故根据\refexa{example:极坐标变换与可测性}所述,即得所证.
\end{proof}

\begin{theorem}
设定义在\(\mathbb{R}^{2}\)上的函数\(f(x, y)\)满足:

(i) \(f(x, y)\)是单变量\(y \in \mathbb{R}\)的可测函数;

(ii) \(f(x, y)\)是单变量\(x \in \mathbb{R}\)的连续函数,

则对定义在\(\mathbb{R}\)上任一实值可测函数\(g(y)\),\(f[g(y), y]\)是\(\mathbb{R}\)上的可测函数.
\end{theorem}
\begin{proof}
对\(\mathbb{R}\)作如下的区间分割:\(\left[\frac{m - 1}{n}, \frac{m}{n}\right]\)(\(m \in \mathbb{Z}\),\(n \in \mathbb{N}\)),并对\((x, y) \in [(m - 1) / n, m / n] \times \mathbb{R}\),作函数列(凸线性组合)
\[
f_{n}(x, y) = n\left(\frac{m}{n} - x\right)f\left(\frac{m - 1}{n}, y\right) + n\left(x - \frac{m - 1}{n}\right)f\left(\frac{m}{n}, y\right),
\]
易知\(f_{n}(x, y)\)位于\(f((m - 1) / n, y)\)与\(f(m / n, y)\)之间.

因为对每点\((x, y)\),均存在区间列
\[
I_{k} = \left[\frac{m_{k} - 1}{n_{k}}, \frac{m_{k}}{n_{k}}\right] (k \in \mathbb{N}), \quad \bigcap_{k = 1}^{\infty} I_{k} = x.
\]
由(ii)可知\(\lim_{n \to \infty} f_{n}(x, y) = f(x, y)\),从而我们有
\[
\lim_{n \to \infty} f_{n}(g(y), y) = f(g(y), y),
\]
即可得证. 
\end{proof}

























\end{document}
\documentclass[../../main.tex]{subfiles}
\graphicspath{{\subfix{../../image/}}} % 指定图片目录,后续可以直接使用图片文件名。

% 例如:
% \begin{figure}[H]
% \centering
% \includegraphics{image-01.01}
% \caption{图片标题}
% \label{figure:image-01.01}
% \end{figure}
% 注意:上述\label{}一定要放在\caption{}之后,否则引用图片序号会只会显示??.

\begin{document}

\section{$\mathbb{R}^n$中点与点之间的距离$\cdot$点集的极限点}

\subsection{点集的直径、点的(球)邻域、矩体}

\begin{definition}[\(\mathbb{R}^n\)与\(\mathbb{R}^n\)中的运算]\label{definition:Rn与Rn中的运算}
  记一切有序数组\(x = (\xi_1,\xi_2,\cdots,\xi_n)\)的全体为\(\mathbb{R}^n\),其中\(\xi_i\in\mathbb{R}\)(\(i = 1,2,\cdots,n\))是实数,称\(\xi_i\)为\(x\)的第\(i\)个坐标,并定义运算如下:

(i) 加法:对于\(x = (\xi_1,\cdots,\xi_n)\)以及\(y = (\eta_1,\cdots,\eta_n)\),令
\[x + y = (\xi_1+\eta_1,\cdots,\xi_n+\eta_n);\]

(ii) 数乘:对于\(\lambda\in\mathbb{R}\),令\(\lambda x = (\lambda\xi_1,\cdots,\lambda\xi_n)\in\mathbb{R}^n\)。

在上述两种运算下构成一个向量空间。对于\(1\leqslant i\leqslant n\),记
\[e_i = (0,\cdots,0,1,0,\cdots,0),\]
其中除第\(i\)个坐标为\(1\),外其余皆为\(0\)。\(e_1,e_2,\cdots,e_i,\cdots,e_n\)组成\(\mathbb{R}^n\)的基底,从而\(\mathbb{R}^n\)是实数域上的\(n\)维向量空间,并称\(x = (\xi_1,\cdots,\xi_n)\)为\(\mathbb{R}^n\)中的\textbf{向量}或\textbf{点}.当每个\(\xi_i\)均为有理数时,\(x = (\xi_1,\cdots,\xi_n)\)称为\textbf{有理点}.
\end{definition}

\begin{definition}
设\(x = (\xi_1,\cdots,\xi_n)\in\mathbb{R}^n\),令
\[|x| = (\xi_1^2+\cdots+\xi_n^2)^{\frac{1}{2}},\]
称\(|x|\)为向量\(x\)的\textbf{模}或\textbf{长度}.
\end{definition}

\begin{proposition}[向量的模的性质]\label{proposition:向量的模的性质}
设\(x=(x_1,\cdots ,x_n),y=\left( y_1,\cdots ,y_n \right) \in \mathbb{R} ^n\),则

(i) \(|x|\geqslant0\),\(|x| = 0\)当且仅当\(x=(0,\cdots,0)\);

(ii) 对任意的\(a\in\mathbb{R}\),有\(|ax| = |a||x|\);

(iii) \(|x + y|\leqslant|x|+|y|\);

(iv) 设\(x = (\xi_1,\cdots,\xi_n)\),\(y = (\eta_1,\cdots,\eta_n)\),则有
\[(\xi_1\eta_1+\cdots+\xi_n\eta_n)^2\leqslant(\xi_1^2+\cdots+\xi_n^2)(\eta_1^2+\cdots+\eta_n^2).\]
\end{proposition}
\begin{proof}
  (i),(ii)的结论是明显的;(iii)是(iv)的推论.因此我们只证明(iv).

只需注意到函数
\[f(\lambda)=(\xi_1+\lambda\eta_1)^2+\cdots+(\xi_n+\lambda\eta_n)^2\]
是非负的(对一切\(\lambda\)),由\(\lambda\)的二次方程\(f(\lambda)\)的判别式小于或等于零即得。(iv)就是著名的Cauchy - Schwarz不等式。
\end{proof}

\begin{definition}[距离空间]\label{definition:距离空间}
  一般地说,设\(X\)是一个集合。若对\(X\)中任意两个元素\(x\)与\(y\),有一个确定的实数与之对应,记为\(d(x,y)\),它满足下述三条性质:

(i) \(d(x,y)\geqslant0\),\(d(x,y)=0\)当且仅当\(x = y\);

(ii) \(d(x,y)=d(y,x)\);

(iii) \(d(x,y)\leqslant d(x,z)+d(z,y)\),

则认为在\(X\)中定义了距离\(d\),并称\((X,d)\)为\textbf{距离空间}.
\end{definition}
\begin{note}
  因而\((\mathbb{R}^n,d)\)是一个距离空间,其中\(d(x,y)=|x - y|\)。
我们称\(\mathbb{R}^n\)为\textbf{\(\boldsymbol{n}\)维欧氏空间}。
\end{note}

\begin{definition}[点集的直径与有界集]\label{definition:点集的直径与有界集}
设\(E\)是\(\mathbb{R}^n\)中一些点形成的集合,令
\[\text{diam}(E)=\sup\{|x - y|:x,y\in E\},\]
称为点集\(E\)的\textbf{直径}。若\(\text{diam}(E)<+\infty\),则称\(E\)为\textbf{有界集}.
\end{definition}

\begin{proposition}[有界集的充要条件]\label{proposition:有界集的充要条件}
  \(E\)是有界集的充要条件是,存在\(M > 0\),使得\(\forall x\in E\)都满足\(|x|\leqslant M\)。
\end{proposition}
\begin{proof}
  由\hyperref[definition:点集的直径与有界集]{有界集的定义}易得.
\end{proof}

\begin{definition}[点的(球)邻域]\label{definition:点的(球)邻域}
  设\(x_0\in\mathbb{R}^n\),\(\delta>0\),称点集
\[\{x\in\mathbb{R}^n:|x - x_0|<\delta\}\]
为\(\mathbb{R}^n\)中以\(x_0\)为中心,以\(\delta\)为半径的\textbf{开球},也称为\(x_0\)的\textbf{(球)邻域},记为\(B(x_0,\delta)\),从而称
\[\{x\in\mathbb{R}^n:|x - x_0|\leqslant\delta\}\]
为\textbf{闭球},记为\(C(x_0,\delta)\)。\(\mathbb{R}^n\)中以\(x_0\)为中心,以\(\delta\)为半径的球面是
\[\{x\in\mathbb{R}^n:|x - x_0|=\delta\}.\]
\end{definition}

\begin{definition}[矩体]
设\(a_i,b_i(i = 1,2,\cdots,n)\)皆为实数,且\(a_i < b_i(i = 1,2,\cdots,n)\),称点集
\[\{x = (\xi_1,\xi_2,\cdots,\xi_n):a_i<\xi_i < b_i\ (i = 1,2,\cdots,n)\}\]
为\(\mathbb{R}^n\)中的\textbf{开矩体}(\(n = 2\)时为矩形,\(n = 1\)时为区间),即直积集
\[(a_1,b_1)\times\cdots\times(a_n,b_n).\]
类似地,\(\mathbb{R}^n\)中的\textbf{闭矩体}以及\textbf{半开闭矩体}就是直积集
\[[a_1,b_1]\times\cdots\times[a_n,b_n],\quad(a_1,b_1]\times\cdots\times(a_n,b_n],\]
称\(b_i - a_i(i = 1,2,\cdots,n)\)为\textbf{矩体的边长}。若各边长都相等,则称矩体为\textbf{方体}。

矩体也常用符号\(I,J\)等表示,其\textbf{体积}用\(|I|\),\(|J|\)等表示。
\end{definition}

\begin{proposition}[矩体的直径与体积]\label{proposition:矩体的直径与体积}
  若\(I=(a_1,b_1)\times\cdots\times(a_n,b_n)\),则
\[\text{diam}(I)=[(b_1 - a_1)^2+\cdots+(b_n - a_n)^2]^{\frac{1}{2}},\quad |I|=\prod_{i = 1}^{n}(b_i - a_i).\]
\end{proposition}

\begin{definition}
  设\(x_k\in\mathbb{R}^n(k = 1,2,\cdots)\)。若存在\(x\in\mathbb{R}^n\),使得
\[\lim_{k\rightarrow\infty}|x_k - x| = 0,\]
则称\(x_k(k = 1,2,\cdots)\)为\(\mathbb{R}^n\)中的\textbf{收敛(于\(\boldsymbol{x}\)的)点列},称\(x\)为它的\textbf{极限},并简记为
\[\lim_{k\rightarrow\infty}x_k = x.\]
\end{definition}

\begin{definition}[Cauchy列]\label{definition:Cauchy列}
  称\(\{x_k\}\)为\textbf{Cauchy列}或\textbf{基本列},若$\lim_{l,m\rightarrow\infty}|x_l - x_m| = 0$.即对任意\(\varepsilon>0\),存在\(N\),使得当\(k,l>N\)时,有
\[|x_k - x_l|<\varepsilon.\]
\end{definition}

\begin{theorem}
  \(x_k(k = 1,2,\cdots)\)是收敛列的充分必要条件是\(\{x_k\}\)为Cauchy列,即
\[\lim_{l,m\rightarrow\infty}|x_l - x_m| = 0.\]
\end{theorem}
\begin{proof}
  若令\(x_k=\{\xi_1^{(k)},\xi_2^{(k)},\cdots,\xi_n^{(k)}\}\),\(x = \{\xi_1,\xi_2,\cdots,\xi_n\}\),则由于不等式
\[|\xi_i^{(k)} - \xi_i|\leqslant|x_k - x|\leqslant|\xi_1^{(k)} - \xi_1|+\cdots+|\xi_n^{(k)} - \xi_n|\]
对一切\(k\)与\(i\)都成立。故可知\(x_k(k = 1,2,\cdots)\)收敛于\(x\)的充分必要条件是,对每个\(i\),实数列\(\{\xi_i^{(k)}\}\)都收敛于\(\xi_i\)。由此根据实数列收敛的Cauchy收敛准则可知结论成立.
\end{proof}

\subsection{点集的极限点}

\begin{definition}[极限点、导集与完全集]\label{definition:极限点、导集与完全集}
设\(E\subset\mathbb{R}^n\),\(x\in\mathbb{R}^n\)。若存在\(E\)中的互异点列\(\{x_k\}\),使得
\[\lim_{k\rightarrow\infty}|x_k - x| = 0,\]
则称\(x\)为\(E\)的\textbf{极限点}或\textbf{聚点}.\(E\)的极限点全体记为\(E'\),称为\(E\)的\textbf{导集}。

若$E = E'$,则$E$称为\textbf{完全集}.
\end{definition}
\begin{note}
显然,有限集是不存在极限点的.
\end{note}

\begin{theorem}[一个点是极限点的充要条件]\label{theorem:一个点是极限点的充要条件}
若\(E\subset\mathbb{R}^n\),则\(x\in E'\)当且仅当对任意的\(\delta>0\),有
\[(B(x,\delta)\setminus\{x\})\cap E\neq\varnothing.\]
\end{theorem}
\begin{proof}
  若\(x\in E'\),则存在\(E\)中的互异点列\(\{x_k\}\),使得
\[|x_k - x|\to 0\quad(k\to\infty),\]
从而对任意的\(\delta>0\),存在\(k_0\),当\(k\geqslant k_0\)时,有\(|x_k - x|<\delta\),即
\[x_k\in B(x,\delta)\quad(k\geqslant k_0).\]

反之,若对任意的\(\delta>0\),有\((B(x,\delta)\setminus\{x\})\cap E\neq\varnothing\),则令\(\delta_1 = 1\),可取\(x_1\in E\),\(x_1\neq x\)且\(|x - x_1|<1\)。令
\[\delta_2=\min\left(|x - x_1|,\frac{1}{2}\right),\]
可取\(x_2\in E\),\(x_2\neq x\)且\(|x - x_2|<\delta_2\)。继续这一过程,就可得到\(E\)中互异点列\(\{x_k\}\),使得\(|x - x_k|<\delta_k\),即
\[\lim_{k\to\infty}|x - x_k| = 0.\]
这说明\(x\in E'\)。
\end{proof}

\begin{definition}[孤立点]\label{definition:孤立点}
  设\(E\subset\mathbb{R}^n\)。若\(E\)中的点\(x\)不是\(E\)的极限点,即存在\(\delta>0\),使得
\[(B(x,\delta)\setminus\{x\})\cap E=\varnothing,\]
则称\(x\)为\(E\)的\textbf{孤立点},即\(x\in E\setminus E'\)。
\end{definition}

\begin{theorem}[导集的性质]\label{theorem:导集的性质}
设\(E_1,E_2\subset\mathbb{R}^n\),则\((E_1\cup E_2)' = E_1'\cup E_2'\)。
\end{theorem}
\begin{proof}
  因为\(E_1\subset E_1\cup E_2\),\(E_2\subset E_1\cup E_2\),所以
\[E_1'\subset (E_1\cup E_2)',\quad E_2'\subset (E_1\cup E_2)',\]
从而有\(E_1'\cup E_2'\subset (E_1\cup E_2)'\)。反之,若\(x\in (E_1\cup E_2)'\),则存在\(E_1\cup E_2\)中的互异点列\(\{x_k\}\),使得
\[\lim_{k\rightarrow\infty}x_k = x.\]
显然,在\(\{x_k\}\)中必有互异点列\(\{x_{k_i}\}\)属于\(E_1\)或属于\(E_2\),而且
\[\lim_{i\rightarrow\infty}x_{k_i} = x.\]
在\(\{x_{k_i}\}\subset E_1\)时,有\(x\in E_1'\),否则\(x\in E_2'\)。这说明
\[(E_1\cup E_2)'\subset E_1'\cup E_2'.\]
\end{proof}

\begin{theorem}[Bolzano - Weierstrass定理]\label{theorem:Bolzano - Weierstrass定理}
\(\mathbb{R}^n\)中任一有界无限点集\(E\)至少有一个极限点。
\end{theorem}
\begin{proof}
首先从\(E\)中取出互异点列\(\{x_k\}\)。显然,\(\{x_k\}\)仍是有界的,而且\(\{x_k\}\)的第\(i(i = 1,2,\cdots,n)\)个坐标所形成的实数列\(\{\xi_i^{(k)}\}\)是有界数列。其次,根据\(\mathbb{R}^1\)的Bolzano - Weierstrass定理可知,从\(\{x_k\}\)中可选出子列\(\{x_k^{(1)}\}\),使得\(\{x_k^{(1)}\}\)的第一个坐标形成的数列是收敛列;再考查\(\{x_k^{(1)}\}\)的第二个坐标形成的数列,同理可从中选出\(\{x_k^{(2)}\}\),使其第二个坐标形成的数列成为收敛列,此时其第一坐标数列仍为收敛列(注意,收敛数列的任一子列必收敛于同一极限),……至第\(n\)步,可得到\(\{x_k\}\)的子列\(\{x_k^{(n)}\}\),其一切坐标数列皆收敛,从而知\(\{x_k^{(n)}\}\)是收敛点列,设其极限为\(x\)。由于\(\{x_k^{(n)}\}\)是互异点列,故\(x\)为\(E\)的极限点。
\end{proof}




\end{document}
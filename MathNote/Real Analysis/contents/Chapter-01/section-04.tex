\documentclass[../../main.tex]{subfiles}
\graphicspath{{\subfix{../../image/}}} % 指定图片目录,后续可以直接使用图片文件名。

% 例如:
% \begin{figure}[H]
% \centering
% \includegraphics[scale=0.4]{image-01.01}
% \caption{图片标题}
% \label{figure:image-01.01}
% \end{figure}
% 注意:上述\label{}一定要放在\caption{}之后,否则引用图片序号会只会显示??.

\begin{document}

\section{$\mathbb{R}^n$中的基本点集:闭集$\cdot$开集$\cdot$Borel集$\cdot$Cantor集}

\subsection{闭集}

\begin{definition}[闭集与闭包]\label{definition:闭集与闭包}
设\(E\subset\mathbb{R}^n\)。若\(E\supset E'\)(即\(E\)包含\(E\)的一切极限点),则称\(E\)为\textbf{闭集}(这里规定空集为闭集)。记\(\overline{E}=E\cup E'\),并称\(\overline{E}\)为\(E\)的\textbf{闭包}(\(E\)为闭集就是\(E = \overline{E}\))。
\end{definition}

\begin{definition}[稠密子集]\label{definition:稠密子集}
  若\(A\subset B\)且\(\overline{A}=B\),则称\(A\)在\(B\)中\textbf{稠密},或称\(A\)是\(B\)的\textbf{稠密子集}.
\end{definition}

\begin{theorem}[闭集的运算性质]\label{theorem:闭集的运算性质}
  (i) 若\(F_1,F_2\)是\(\mathbb{R}^n\)中的闭集,则其并集\(F_1\cup F_2\)也是闭集,从而有限多个闭集的并集是闭集;

(ii) 若\(\{F_{\alpha}:\alpha\in I\}\)是\(\mathbb{R}^n\)中的一个闭集族,则其交集\(F = \bigcap_{\alpha\in I}F_{\alpha}\)是闭集。

(iii)设\(E_{\alpha}\subset\mathbb{R}^n(\alpha\in I)\),则
\[\bigcup_{\alpha\in I}\overline{E_{\alpha}}\subset\overline{\bigcup_{\alpha\in I}E_{\alpha}},\quad\overline{\bigcap_{\alpha\in I}E_{\alpha}}\subset\bigcap_{\alpha\in I}\overline{E_{\alpha}}.\]
\end{theorem}
\begin{remark}
  无穷多个闭集的并集不一定是闭集。例如,令
\[F_k=\left[\frac{1}{k + 1},\frac{1}{k}\right]\subset\mathbb{R}\quad(k = 1,2,\cdots),\]
则有\(\bigcup_{k = 1}^{\infty}F_k=(0,1]\)。此例还说明
\[\left[ 0,1 \right] =\overline{\bigcup_{k=1}^{\infty}{F_k}}\ne \bigcup_{k=1}^{\infty}{\overline{F_k}}=\left( 0,1 \right] .\]
\end{remark}
\begin{proof}
  (i) 从等式
\begin{align*}
\overline{F_1\cup F_2}&=(F_1\cup F_2)\cup (F_1\cup F_2)'\\
&=(F_1\cup F_2)\cup (F_1'\cup F_2')\\
&=(F_1\cup F_1')\cup (F_2\cup F_2')\\
&=\overline{F_1}\cup\overline{F_2}
\end{align*}
可知,若\(F_1,F_2\)为闭集,则\(\overline{F_1\cup F_2}=F_1\cup F_2\)。即\(F_1\cup F_2\)是闭集。

(ii) 因为对一切\(\alpha\in I\),有\(F\subset F_{\alpha}\),所以对一切\(\alpha\in I\),有\(\overline{F}\subset\overline{F_{\alpha}} = F_{\alpha}\),从而有
\[\overline{F}\subset\bigcap_{\alpha\in I}F_{\alpha}=F.\]
但\(F\subset\overline{F}\),故\(F=\overline{F}\)。这说明\(F\)是闭集。
\end{proof}

\begin{theorem}[Cantor闭集套定理]\label{theorem:Cantor闭集套定理}
若\(\{F_k\}\)是\(\mathbb{R}^n\)中的非空有界闭集列,且满足\(F_1\supset F_2\supset\cdots\supset F_k\supset\cdots\),则\(\bigcap_{k = 1}^{\infty}F_k\neq\varnothing\)。
\end{theorem}
\begin{proof}
  若在\(\{F_k\}\)中有无穷多个相同的集合,则存在自然数\(k_0\),当\(k\geqslant k_0\)时,有\(F_k = F_{k_0}\)。此时,\(\bigcap_{k = 1}^{\infty}F_k = F_{k_0}\neq\varnothing\)。现在不妨假定对一切\(k\),\(F_{k + 1}\)是\(F_k\)的真子集,即
\[F_k\setminus F_{k + 1}\neq\varnothing\quad (\text{一切 }k),\]
我们选取\(x_k\in F_k\setminus F_{k + 1}(k = 1,2,\cdots)\),则\(\{x_k\}\)是\(\mathbb{R}^n\)中的有界互异点列。根据Bolzano - Weierstrass定理可知,存在\(\{x_{k_i}\}\)以及\(x\in\mathbb{R}^n\),使得\(\lim_{i\rightarrow\infty}|x_{k_i}-x| = 0\)。由于每个\(F_k\)都是闭集,故知\(x\in F_k(k = 1,2,\cdots)\),即
\[x\in\bigcap_{k = 1}^{\infty}F_k.\]
\end{proof}

\subsection{开集}

\begin{definition}[开集]\label{definition:开集}
设\(G\subset\mathbb{R}^n\)。若\(G^c=\mathbb{R}^n\setminus G\)是闭集,则称\(G\)为\textbf{开集}。
\end{definition}
\begin{note}
由此定义立即可知,\(\mathbb{R}^n\)本身与空集\(\varnothing\)是开集;\(\mathbb{R}^n\)中的开矩体是开集;闭集的补集是开集。
\end{note}

\begin{theorem}[开集的运算性质]\label{theorem:开集的运算性质}
(i) 若\(\{G_{\alpha}:\alpha\in I\}\)是\(\mathbb{R}^n\)中的一个开集族,则其并集\(G = \bigcup_{\alpha\in I}G_{\alpha}\)是开集;

(ii) 若\(G_k(k = 1,2,\cdots,m)\)是\(\mathbb{R}^n\)中的开集,则其交集\(G=\bigcap_{k = 1}^{m}G_k\)是开集(有限个开集的交集是开集);

(iii) 若\(G\)是\(\mathbb{R}^n\)中的非空点集,则\(G\)是开集的充分必要条件是,对于\(G\)中任一点\(x\),存在\(\delta>0\),使得\(B(x,\delta)\subset G\)。
\end{theorem}
\begin{proof}
  (i) 由定义知\(G_{\alpha}^c(\alpha\in I)\)是闭集,且有\(G^c=\bigcap_{\alpha\in I}G_{\alpha}^c\)。根据闭集的性质可知\(G^c\)是闭集,即\(G\)是开集。

(ii) 由定义知\(G_k^c(k = 1,2,\cdots,m)\)是闭集,且有\(G^c=\bigcup_{k = 1}^{m}G_k^c\)。根据闭集的性质可知\(G^c\)是闭集,即\(G\)是开集。

(iii) 若\(G\)是开集且\(x\in G\),则由于\(G^c\)是闭集以及\(x\notin G^c\),可知存在\(\delta>0\),使得\(B(x,\delta)\subset G\)。

反之,若对\(G\)中的任一点\(x\),存在\(\delta>0\),使得\(B(x,\delta)\subset G\),则
\[B(x,\delta)\cap G^c=\varnothing,\]
从而\(x\)不是\(G^c\)的极限点,即\(G^c\)的极限点含于\(G^c\)。这说明\(G^c\)是闭集,即\(G\)是开集。
\end{proof}

\begin{definition}[内点与边界点]\label{definition:内点与边界点}
  设\(E\subset\mathbb{R}^n\)。对\(x\in E\),若存在\(\delta>0\),使得\(B(x,\delta)\subset E\),则称\(x\)为\(E\)的\textbf{内点}。\(E\)的内点全体记为\(\mathring{E}\),称为\(E\)的\textbf{内核}。若\(x\in\overline{E}\)但\(x\notin\mathring{E}\),则称\(x\)为\(E\)的\textbf{边界点}。边界点全体记为\(\partial E\)。
\end{definition}
\begin{note}
  显然,内核一定为开集。\hyperref[theorem:开集的运算性质]{开集的运算性质(iii)}说明开集就是集合中每个点都是内点的集合。
\end{note}

\begin{proposition}[振幅函数及其性质]\label{proposition:振幅函数可测}
设函数 \( f(x) \) 在 \( B(x_0, \delta_0) \) 上有定义. 令
\[
\omega_f(x_0) = \lim_{\delta \to 0} \sup \left\{ |f(x') - f(x'')| : x', x'' \in B(x_0, \delta) \right\},
\]
我们称 \( \omega_f(x_0) \) 为 \( f \) 在 \( x_0 \) 处的\textbf{振幅}. 若 \( G \) 是 \( \mathbb{R}^n \) 中的开集,且 \( f(x) \) 定义在 \( G \) 上,则对任意的 \( t \in \mathbb{R} \),点集
\[
H = \{ x \in G : \omega_f(x) < t \}
\]
是开集,进而$\omega_f$在$G$上可测.
\end{proposition}
\begin{proof}
不妨设 \( H \neq \varnothing \). 对于 \( H \) 中的任一点 \( x_0 \),因为 \( \omega_f(x_0) < t \),所以存在 \( \delta_0 > 0 \),使得 \( B(x_0, \delta_0) \subset G \),且有
\[
\sup \left\{ |f(x') - f(x'')| : x', x'' \in B(x_0, \delta_0) \right\} < t.
\]
现在对于 \( x \in B(x_0, \delta_0) \),可以选取 \( \delta_1 > 0 \),使得
\[
B(x, \delta_1) \subset B(x_0, \delta_0).
\]
显然有
\[
\sup \left\{ |f(x') - f(x'')| : x', x'' \in B(x, \delta_1) \right\} < t,
\]
从而可知 \( \omega_f(x) < t \),即
\[
B(x_0, \delta_0) \subset H.
\]
这说明 \( H \) 中的点都是内点,\( H \) 是开集.
\end{proof}

\begin{theorem}[开集构造定理]\label{theorem:开集构造定理}
(i) \(\mathbb{R}\)中的非空开集是可数个互不相交的开区间(这里也包括\((-\infty,a)\),\((b,+\infty)\)以及\((-\infty,+\infty)\))的并集;

(ii) \(\mathbb{R}^n\)中的非空开集\(G\)是可列个互不相交的半开闭方体的并集。
\end{theorem}
\begin{proof}
  (i) 设\(G\)是\(\mathbb{R}\)中的开集。对于\(G\)中的任一点\(a\),由于\(a\)是\(G\)的内点,故存在\(\delta>0\),使得\((a - \delta,a + \delta)\subset G\)。现在令
\[a'=\inf\{x:(x,a)\subset G\},\quad a''=\sup\{x:(a,x)\subset G\}\]
(这里\(a'\)可以是\(-\infty\),\(a''\)可以是\(+\infty\)),显然\(a'<a<a''\)且\((a',a'')\subset G\)。这是因为对区间\((a',a'')\)中的任一点\(z\),不妨设\(a'<z\leqslant a\),必存在\(x\),使得\(a'<x<z\)且\((x,a)\subset G\),即\(z\in G\)。我们称这样的开区间\((a',a'')\)为\(G\)(关于点\(a\))的{\heiti 构成区间}\(I_a\)。

如果\(I_a=(a',a'')\),\(I_b=(b',b'')\)是\(G\)的构成区间,那么可以证明它们或是重合的或是互不相交的。为此,不妨设\(a < b\)。若
\[I_a\cap I_b\neq\varnothing,\]
则有\(b'<a''\)。于是令\(\min\{a',b'\}=c\),\(\max\{a'',b''\}=d\),则有\((c,d)=(a',a'')\cup(b',b'')\)。取\(x\in I_a\cap I_b\),则\(I_x=(c,d)\)是构成区间,且
\[(c,d)=(a',a'')=(b',b'').\]

最后,我们知道\(\mathbb{R}\)中互不相交的区间族是可数的。

(ii) 首先将\(\mathbb{R}^n\)用格点(坐标皆为整数)分为可列个边长为\(1\)的半开闭方体,其全体记为\(\Gamma_0\)。再将\(\Gamma_0\)中每个方体的每一边二等分,则每个方体就可分为\(2^n\)个边长为\(\frac{1}{2}\)的半开闭方体,记\(\Gamma_0\)中如此做成的子方体的全体为\(\Gamma_1\)。继续按此方法二分下去,可得其所含方体越来越小的方体族组成的序列\(\{\Gamma_k\}\),这里\(\Gamma_k\)中每个方体的边长是\(2^{-k}\),且此方体是\(\Gamma_{k + 1}\)中相应的\(2^n\)个互不相交的方体的并集。我们称如此分成的方体为二进方体。

现在把\(\Gamma_0\)中凡含于\(G\)内的方体取出来,记其全体为\(H_0\)。再把\(\Gamma_1\)中含于
\[G\setminus\bigcup_{J\in H_0}J\]
(\(J\)表示半开闭二进方体)内的方体取出来,记其全体为\(H_1\)。依此类推,\(H_k\)为\(\Gamma_k\)中含于
\[G\setminus\bigcup_{i = 0}^{k - 1}\bigcup_{J\in H_i}J\]
内的方体的全体。显然,一切由\(H_k(k = 0,1,2,\cdots)\)中的方体构成的集合为可列的。因为\(G\)是开集,所以对任意的\(x\in G\),存在\(\delta>0\),使得\(B(x,\delta)\subset G\)。而\(\Gamma_k\)中的方体的直径当\(k\rightarrow\infty\)时是趋于零的,从而可知\(x\)最终必落入某个\(\Gamma_k\)中的方体。这说明
\[G=\bigcup_{k = 0}^{\infty}\bigcup_{J\in H_k}J,\quad J表示半开闭二进方体.\]

\(\mathbb{R}^n\)中的开集还有一个重要事实,即\(\mathbb{R}^n\)中存在由可列个开集构成的开集族\(\Gamma\),使得\(\mathbb{R}^n\)中任一开集均是\(\Gamma\)中某些开集的并集。事实上,\(\Gamma\)可取为
\[\left\{B\left(x,\frac{1}{k}\right):x是\mathbb{R}^n中的有理点,k是自然数\right\}.\]
首先,\(\Gamma\)是可列集。其次,对于\(\mathbb{R}^n\)中开集\(G\)的任一点\(x\),必存在\(\delta>0\),使得\(B(x,\delta)\subset G\)。现在取有理点\(x'\),使得\(d(x,x')<1/k\),其中\(k > 2/\delta\),从而有
\[x\in B(x',1/k)\subset B(x,\delta)\subset G,\]
显然,一切如此做成的\(B(x',1/k)\)的并集就是\(G\)。
\end{proof}

\begin{definition}[开覆盖]\label{definition:开覆盖}
设\(E\subset\mathbb{R}^n\),\(\Gamma\)是\(\mathbb{R}^n\)中的一个开集族。若对任意的\(x\in E\),存在\(G\in\Gamma\),使得\(x\in G\),则称\(\Gamma\)为\(E\)的一个\textbf{开覆盖}。

设\(\Gamma\)是\(E\)的一个开覆盖。若\(\Gamma'\subset\Gamma\)仍是\(E\)的一个开覆盖,则称\(\Gamma'\)为\(\Gamma\)(关于\(E\))的一个\textbf{子覆盖}.
\end{definition}

\begin{lemma}\label{lemma:任一开覆盖都含有一个可数子覆盖}
  \(\mathbb{R}^n\)中点集\(E\)的任一开覆盖\(\Gamma\)都含有一个可数子覆盖。
\end{lemma}

\begin{theorem}[Heine - Borel有限子覆盖定理]\label{theorem:Heine - Borel有限子覆盖定理}
\(\mathbb{R}^n\)中有界闭集的任一开覆盖均含有一个有限子覆盖。
\end{theorem}
\begin{remark}
  在上述定理中,有界的条件是不能缺的。例如,在\(\mathbb{R}^1\)中对自然数集作开覆盖\(\{(n-\frac{1}{2},n+\frac{1}{2})\}\)就不存在有限子覆盖。同样,闭集的条件也是不能缺的。例如,在\(\mathbb{R}\)中对点集\(\{1,\frac{1}{2},\cdots,\frac{1}{n},\cdots\}\)作开覆盖
\[\left\{\left(\frac{1}{n}-\frac{1}{2n},\frac{1}{n}+\frac{1}{2n}\right)\right\}\quad(n = 1,2,\cdots),\]
就不存在有限子覆盖。
\end{remark}
\begin{proof}
  设\(F\)是\(\mathbb{R}^n\)中的有界闭集,\(\Gamma\)是\(F\)的一个开覆盖。由\hyperref[lemma:任一开覆盖都含有一个可数子覆盖]{引理\ref{lemma:任一开覆盖都含有一个可数子覆盖}},可以假定\(\Gamma\)由可列个开集组成:
\[\Gamma=\{G_1,G_2,\cdots,G_i,\cdots\}.\]
令
\[H_k=\bigcup_{i = 1}^{k}G_i,\quad L_k = F\cap H_k^c\quad(k = 1,2,\cdots).\]
显然,\(H_k\)是开集,\(L_k\)是闭集且有\(L_k\supset L_{k + 1}(k = 1,2,\cdots)\)。分两种情况:

(i) 存在\(k_0\),使得\(L_{k_0}\)是空集,即\(H_{k_0}\)中不含\(F\)的点,从而知\(F\subset H_{k_0}\),定理得证;

(ii) 一切\(L_k\)皆非空集,则由\hyperref[theorem:Cantor闭集套定理]{Cantor闭集套定理}可知,存在点\(x_0\in L_k(k = 1,2,\cdots)\),即\(x_0\in F\)且\(x_0\in H_k^c(k = 1,2,\cdots)\)。这就是说\(F\)中存在点\(x_0\)不属于一切\(H_k\),与原设矛盾,故第(ii)种情况不存在。
\end{proof}

\begin{theorem}\label{theorem:Rn中的有界闭集是紧集}
设\(E\subset\mathbb{R}^n\)。若\(E\)的任一开覆盖都包含有限子覆盖,则\(E\)是有界闭集。
\end{theorem}
\begin{proof}
  设\(y\in E^c\),则对于每一个\(x\in E\),存在\(\delta_x>0\),使得
\[B(x,\delta_x)\cap B(y,\delta_x)=\varnothing.\]
显然,\(\{B(x,\delta_x):x\in E\}\)是\(E\)的一个开覆盖,由题设知存在有限子覆盖,设为
\[B(x_1,\delta_{x_1}),\quad\cdots,\quad B(x_m,\delta_{x_m}).\]
由此立即可知\(E\)是有界集。现在再令
\[\delta_0=\min\{\delta_{x_1},\cdots,\delta_{x_m}\},\]
则\(B(y,\delta_0)\cap E=\varnothing\),即\(y\notin E'\)。这说明\(E'\subset E\),即\(E\)是闭集。有界性显然。
\end{proof}

\begin{definition}[紧集]\label{definition:紧集}
  如果\(E\)的任一开覆盖均包含有限子覆盖,我们就称\(E\)为\textbf{紧集}.
\end{definition}
\begin{note}
  \hyperref[theorem:Heine - Borel有限子覆盖定理]{Heine - Borel有限子覆盖定理}和\hyperref[theorem:Rn中的有界闭集是紧集]{定理\ref{theorem:Rn中的有界闭集是紧集}}表明,\(\mathbb{R}^n\)中的紧集就是有界闭集.
\end{note}

\begin{definition}[实值函数的连续]\label{definition:实值函数的连续}
  设\(f(x)\)是定义在\(E\subset\mathbb{R}^n\)上的实值函数,\(x_0\in E\)。如果对任意的\(\varepsilon>0\),存在\(\delta>0\),使得当\(x\in E\cap B(x_0,\delta)\)时,有
\[|f(x)-f(x_0)|<\varepsilon,\]
则称\(f(x)\)在\(x = x_0\)处\textbf{连续},称\(x_0\)为\(f(x)\)的一个\textbf{连续点}(在\(x_0\notin E'\)的情形,即\(x_0\)是\(E\)的孤立点时,\(f(x)\)自然在\(x = x_0\)处连续)。若\(E\)中的任一点皆为\(f(x)\)的连续点,则称\(f(x)\)\textbf{在\(\boldsymbol{E}\)上连续}。记\(E\)上的连续函数之全体为\(C(E)\)。
\end{definition}

\begin{proposition}[在$\mathbb{R}^n$的紧集上连续的函数的性质]\label{proposition:在Rn的紧集上连续的函数的性质}
  设\(F\)是\(\mathbb{R}^n\)中的有界闭集,\(f\in C(F)\),则

(i) \(f(x)\)是\(F\)上的有界函数,即\(f(F)\)是\(\mathbb{R}\)中的有界集.

(ii) 存在\(x_0\in F\),\(y_0\in F\),使得
\[f(x_0)=\sup\{f(x):x\in F\},\quad f(y_0)=\inf\{f(x):x\in F\}.\]

(iii) \(f(x)\)在\(F\)上是一致连续的,即对任给的\(\varepsilon>0\),存在\(\delta>0\),当\(x',x''\in F\)且\(|x' - x''|<\delta\)时,有
\[|f(x') - f(x'')|<\varepsilon.\]
此外,若\(E\subset\mathbb{R}^n\)上的连续函数列\(\{f_k(x)\}\)一致收敛于\(f(x)\),则\(f(x)\)是\(E\)上的连续函数。
\end{proposition}


\subsection{Borel集}

\begin{definition}[$\boldsymbol{F}_{\boldsymbol{\sigma }},\boldsymbol{G}_{\boldsymbol{\delta }}$集]\label{definition:F_sigma,G_delta集}
若 $E \subset \mathbb{R}^n$ 是可数个闭集的并集,则称 $E$ 为 $F_{\sigma}$(型)集;

若 $E \subset \mathbb{R}^n$ 是可数个开集的交集,则称 $E$ 为 $G_{\delta}$(型)集。
\end{definition}
\begin{remark}
由定义可直接推知,$F_{\sigma}$ 集的补集是 $G_{\delta}$ 集;$G_{\delta}$ 集的补集是 $F_{\sigma}$ 集。
\end{remark}

\begin{example}
记 $\mathbb{R}^n$ 中全体有理点为 $\{r_k\}$,则有理点集
\begin{align*}
\bigcup_{k = 1}^{\infty} \{r_k\}
\end{align*}
为 $F_{\sigma}$ 集。 
\end{example}

\begin{example}[函数连续点的结构]
若$f(x)$是定义在开集$G\subset\mathbb{R}^n$上的实值函数,则$f(x)$的连续点集是$G_{\delta}$集.
\end{example}
\begin{proof}
令$\omega_f(x)$为$f(x)$在$x$点的振幅,易知$f(x)$在$x = x_0$处连续的充分必要条件是$\omega_f(x_0)=0$. 由此可知$f(x)$的连续点集可表示为
\begin{align*}
\bigcap_{k = 1}^{\infty}\left\{x\in G:\omega_f(x)<\frac{1}{k}\right\}.
\end{align*}
因为$\{x\in G:\omega_f(x)<1/k\}$是开集,所以$f(x)$的连续点集是$G_{\delta}$集. 
\end{proof}

\begin{definition}[$\sigma$-代数]
设$\Gamma$是由集合$X$的一些子集所构成的集合族且满足下述条件:

(i) $\varnothing\in\Gamma$;

(ii) 若$A\in\Gamma$,则$A^c\in\Gamma$;

(iii) 若$A_n\in\Gamma$ ($n = 1,2,\cdots$),则$\bigcup_{n = 1}^{\infty}A_n\in\Gamma$,

这时称$\Gamma$是($X$上的)一个\textbf{$\boldsymbol{\sigma}$-代数}.
\end{definition}

\begin{proposition}[$\sigma$-代数的基本性质]\label{proposition:σ-代数的基本性质}
若$\Gamma$是$X$上的一个$\sigma$-代数,则

(i) 若$A_n\in\Gamma$($n = 1,2,\cdots,m$),则$\bigcup_{n = 1}^{m}A_n\in\Gamma\quad ,\bigcap_{n = 1}^{m}A_n\in\Gamma, $;

(ii) 若$A_n\in\Gamma$($n = 1,2,\cdots$),则
\begin{align*}
\bigcap_{n = 1}^{\infty}A_n\in\Gamma, \quad \varlimsup_{n\to\infty}A_n\in\Gamma, \quad \varliminf_{n\to\infty}A_n\in\Gamma;
\end{align*}

(iii) 若$A,B\in\Gamma$,则$A\setminus B\in\Gamma$;

(iv) $X\in\Gamma$. 
\end{proposition}
\begin{proof}
由$\sigma$-代数的定义立得.
\end{proof}

\begin{definition}[生成$\sigma$-代数]
设$\Sigma$是集合$X$的一些子集所构成的集合族,考虑包含$\Sigma$的$\sigma$-代数$\Gamma$(即若$A\in\Sigma$,必有$A\in\Gamma$,这样的$\Gamma$是存在的,如$\mathcal{P}(X)$). 记包含$\Sigma$的最小$\sigma$-代数为$\Gamma(\Sigma)$. 也就是说,对任一包含$\Sigma$的$\sigma$-代数$\Gamma'$,若$A\in\Gamma(\Sigma)$,则$A\in\Gamma'$,称$\Gamma(\Sigma)$为由\textbf{$\Sigma$生成的$\sigma$-代数}.
\end{definition}

\begin{definition}[Borel集]\label{definition:Borel集}
由$\mathbb{R}^n$中一切开集构成的开集族所生成的$\sigma$-代数称为\textbf{Borel $\sigma$-代数},记为$\mathscr{B}$. $\mathscr{B}$中的元称为\textbf{Borel集}.
\end{definition}

\begin{proposition}[Borel集的基本性质]\label{proposition:Borel集的基本性质}
$\mathbb{R}^n$中的闭集、开集、$F_{\sigma}$集与$G_{\delta}$集皆为Borel集;

任一Borel集的补集是Borel集;Borel集列的并、交、上(下)限集皆为Borel集. 
\end{proposition}
\begin{note}
例如,$F_{\sigma\delta}$集(可数个$F_{\sigma}$集的交集)是Borel集.
\end{note}
\begin{proof}
证明是显然的.
\end{proof}

\begin{example}\label{example:例12}
设$f_k\in C(\mathbb{R}^n)$ ($k = 1,2,\cdots$),且有
\begin{align*}
\lim_{k\to\infty}f_k(x)=f(x), \quad x\in\mathbb{R}^n,
\end{align*}
则$f(x)$的连续点集
\begin{align*}
\bigcap_{m = 1}^{\infty}\bigcup_{k = 1}^{\infty}\mathring{E}_k\left(\frac{1}{m}\right)
\end{align*}
是$G_{\delta}$型集,其中$E_k(\varepsilon)=\{x\in\mathbb{R}^n: |f_k(x)-f(x)|\leqslant\varepsilon\}$.
\end{example}
\begin{proof}
(i) 设$x_0\in\mathbb{R}^n$是$f(x)$的连续点. 由题设知,对任意$\varepsilon>0$,存在$k_0$,使得$|f_{k_0}(x_0)-f(x_0)|<\varepsilon/3$,且存在$\delta>0$,使得
\begin{align*}
|f(x)-f(x_0)|<\varepsilon/3, \quad |f_{k_0}(x)-f_{k_0}(x_0)|<\varepsilon/3, \quad x\in U(x_0,\delta),
\end{align*}
从而对$x\in U(x_0,\delta)$,有
\begin{align*}
|f_{k_0}(x)-f(x)|<\varepsilon, \quad U(x_0,\delta)\subset\mathring{E}_{k_0}(\varepsilon).
\end{align*}
这说明$x_0\in\bigcup_{k = 1}^{\infty}\mathring{E}_k(\varepsilon)$. 又由$\varepsilon$的任意性,可推知
\begin{align*}
x_0\in\bigcap_{m = 1}^{\infty}\bigcup_{k = 1}^{\infty}\mathring{E}_k\left(\frac{1}{m}\right).
\end{align*}

(ii) 设$x_0\in\bigcap_{m = 1}^{\infty}\bigcup_{k = 1}^{\infty}\mathring{E}_k\left(\frac{1}{m}\right)$. 对$\varepsilon>0$,取$m>3/\varepsilon$. 由于$x_0\in\bigcup_{k = 1}^{\infty}\mathring{E}_k\left(\frac{1}{m}\right)$,故存在$k_0$,使得$x_0\in\mathring{E}_{k_0}\left(\frac{1}{m}\right)$,从而可得$U(x_0,\delta_0)\subset E_{k_0}\left(\frac{1}{m}\right)$,即
\begin{align*}
|f_{k_0}(x)-f(x)|\leqslant\frac{1}{m}<\frac{\varepsilon}{3}, \quad x\in U(x_0,\delta_0).
\end{align*}

注意到$f_{k_0}(x)$在$x = x_0$处连续,又有$\delta_1>0$,使得
\begin{align*}
|f_{k_0}(x)-f_{k_0}(x_0)|<\frac{\varepsilon}{3}, \quad x\in U(x_0,\delta_1).
\end{align*}

记$\delta=\min\{\delta_0,\delta_1\}$,则当$x\in U(x_0,\delta)$时,有$|f(x)-f(x_0)|<\varepsilon$. 这说明$f(x)$在$x = x_0$处连续. 
\end{proof}

\begin{definition}[Baire第一函数类]
称区间$I$上连续函数列的极限函数$f(x)$的全体为\textbf{Baire第一函数类},记为$f\in B_1(I)$.
\end{definition}

\begin{theorem}\label{theorem:Baire第一函数类的性质1}
若$f_n\in B_1(\mathbb{R})$,且$f_n(x)$在$\mathbb{R}$上一致收敛于$f(x)$,则$f\in B_1(\mathbb{R})$.
\end{theorem}
\begin{proof}
事实上,由题设知,对任意$k\in\mathbb{N}$,存在$n_k\in\mathbb{N}$,使得
\begin{align*}
|f_{n_k}(x)-f(x)|<1/2^{k + 1} \quad (x\in\mathbb{R}).
\end{align*}
这里不妨认定$n_1 < n_2 < \cdots < n_k < \cdots$. 考查$\sum_{k = 1}^{\infty}[f_{n_{k + 1}}(x)-f_{n_k}(x)]$,因为我们有
\begin{align*}
|f_{n_{k + 1}}(x)-f_{n_k}(x)| &\leqslant |f_{n_{k + 1}}(x)-f(x)|+|f_{n_k}(x)-f(x)| \\
&<\frac{1}{2^{k + 2}}+\frac{1}{2^{k + 1}}<\frac{1}{2^{k}} \quad (x\in\mathbb{R}),
\end{align*}
所以$g(x)\stackrel{\text{def}}{=}\sum_{k = 1}^{\infty}[f_{n_{k + 1}}(x)-f_{n_k}(x)]\in B_1(\mathbb{R})$. 显然有$g(x)=f(x)-f_{n_1}(x)$,即$f(x)=g(x)+f_{n_1}(x)$. 证毕.
\end{proof}

\begin{proposition}
$\mathbb{R}$中存在非$F_{\sigma\delta}$集、非$F_{\delta\delta\sigma}$集等等. 
\end{proposition}

\begin{theorem}[Baire定理]\label{theorem:Baire定理}
设$E\subset\mathbb{R}^n$是$F_{\sigma}$集,即$E = \bigcup_{k = 1}^{\infty}F_k$,$\,F_k$ ($k = 1,2,\cdots$) 是闭集. 若每个$F_k$皆无内点,则$E$也无内点.
\end{theorem}
\begin{proof}
若$E$有内点,设为$x_0$,则存在$\delta_0>0$,使$\overline{B}(x_0,\delta_0)\subset E$. 因为$F_1$是无内点的,所以必存在$x_1\in B(x_0,\delta_0)$,且有$x_1\notin F_1$. 又因为$F_1$是闭集,所以可以取到$\delta_1(0<\delta_1<1)$,使得
\begin{align*}
\overline{B}(x_1,\delta_1)\cap F_1=\varnothing,
\end{align*}
同时有$\overline{B}(x_1,\delta_1)\subset B(x_0,\delta_0)$. 再从$\overline{B}(x_1,\delta_1)$出发以类似的推理使用于$F_2$,则可得$\overline{B}(x_2,\delta_2)\cap F_2=\varnothing$,同时有$\overline{B}(x_2,\delta_2)\subset B(x_1,\delta_1)$,这里可以要求$0<\delta_2<1/2$. 继续这一过程,可得点列$\{x_k\}$与正数列$\{\delta_k\}$,使得对每个自然数$k$,有
\begin{align*}
\overline{B}(x_k,\delta_k)\subset B(x_{k - 1},\delta_{k - 1}), \quad \overline{B}(x_k,\delta_k)\cap F_k=\varnothing,
\end{align*}
其中$0<\delta_k<1/k$. 由于当$l>k$时,有$x_l\in B(x_k,\delta_k)$,故
\begin{align*}
|x_l - x_k|<\delta_k<\frac{1}{k}.
\end{align*}
这说明$\{x_k\}$是$\mathbb{R}^n$中的基本列(Cauchy列),从而是收敛列,即存在$x\in\mathbb{R}^n$,使得$\lim_{k\to\infty}|x_k - x| = 0$.

此外,从不等式
\begin{align*}
|x - x_k|\leqslant|x - x_l|+|x_l - x_k|<|x - x_l|+\delta_k, \quad l>k
\end{align*}
立即可知(令$l\to\infty$)$|x - x_k|\leqslant\delta_k$. 这说明$x\in\overline{B}(x_k,\delta_k)$,即对一切$k$,$x\notin F_k$. 这与$x\in E$发生矛盾. 
\end{proof}

\begin{example}\label{example:例13}
有理数集$\mathbb{Q}$不是$G_{\delta}$集.
\end{example}
\begin{proof}
事实上,令$\mathbb{Q}=\{r_k: k = 1,2,\cdots\}$,假定$\mathbb{Q}=\bigcap_{i = 1}^{\infty}G_i$,式中$G_i$ ($i = 1,2,\cdots$) 是开集,则有表示式
\begin{align*}
\mathbb{R}=(\mathbb{R}\setminus\mathbb{Q})\cup\mathbb{Q}=\left(\bigcup_{i = 1}^{\infty}G_i^c\right)\cup\left(\bigcup_{k = 1}^{\infty}\{r_k\}\right),
\end{align*}
这里的每个单点集$\{r_k\}$与$G_i^c$皆为闭集,而且从$\overline{G}_i=\mathbb{R}^1$可知每个$G_i^c$是无内点的. 这说明$\mathbb{R}$是可列个无内点之闭集的并集. 从而由Baire定理可知$\mathbb{R}$也无内点,这一矛盾说明$\mathbb{Q}$不是$G_{\delta}$集. 
\end{proof}

\begin{definition}
设$E\subset\mathbb{R}^n$. 若$\overline{E}=\mathbb{R}^n$,则称$E$为$\mathbb{R}^n$中的\textbf{稠密集};若$\mathring{\overline{E}}=\varnothing$,则称$E$为$\mathbb{R}^n$中的\textbf{无处稠密集};可数个无处稠密集的并集称为\textbf{贫集}或\textbf{第一纲集}. 不是第一纲集称为\textbf{第二纲集}.
\end{definition}

\begin{example}\label{example:例14}
设$\{G_k\}$是$\mathbb{R}^n$中的稠密开集列,则$G_0 = \bigcap_{k = 1}^{\infty}G_k$在$\mathbb{R}^n$中稠密.
\end{example}
\begin{proof}
只需指出对$\mathbb{R}^n$中任一闭球$\overline{B}=\overline{B}(x,\delta)$,均有$G_0\cap\overline{B}\neq\varnothing$即可. 采用反证法:假定存在闭球$\overline{B}_0=\overline{B}(x_0,\delta_0)$,使得$G_0\cap\overline{B}_0=\varnothing$,则易知
\begin{align*}
\mathbb{R}^n&=(G_0\cap\overline{B}_0)^c = G_0^c\cup(\overline{B}_0)^c,\\
\overline{B}_0&=\mathbb{R}^n\cap\overline{B}_0 = G_0^c\cap\overline{B}_0=\left(\bigcap_{k = 1}^{\infty}G_k\right)^c\cap\overline{B}_0=\bigcup_{k = 1}^{\infty}(G_k^c\cap\overline{B}_0).
\end{align*}
注意到$G_k^c$是无内点的闭集,故由\hyperref[theorem:Baire定理]{Baire定理}可知,$\overline{B}_0$也无内点,矛盾.
\end{proof}

\begin{example}\label{example:例15}
设$f_k\in C(\mathbb{R}^n)$ ($k = 1,2,\cdots$). 若$\lim_{k\to\infty}f_k(x)=f(x)$ ($x\in\mathbb{R}^n$),则$f(x)$的不连续点集为第一纲集.
\end{example}
\begin{proof}
注意到$f(x)$的连续点集的表示,只需指出(\refexa{example:例12})
\[
\left(G\left(\frac{1}{m}\right)\right)^c \quad \left(G\left(\frac{1}{m}\right)=\bigcup_{k = 1}^{\infty}\mathring{E}_k\left(\frac{1}{m}\right)\right)
\]
是第一纲集. 对$\varepsilon>0$,令
\[
F_k(\varepsilon)=\bigcap_{i = 1}^{\infty}\left\{x\in\mathbb{R}^n: |f_k(x)-f_{k + i}(x)|\leqslant\varepsilon\right\},
\]
易知$\mathbb{R}^n=\bigcup_{k = 1}^{\infty}F_k(\varepsilon)$,$F_k(\varepsilon)\subset E_k(\varepsilon)$,从而有
\[
\mathring{F}_k(\varepsilon)\subset\mathring{E}_k(\varepsilon)\subset G(\varepsilon), \quad \bigcup_{k = 1}^{\infty}\mathring{F}_k(\varepsilon)\subset G(\varepsilon).
\]
由此知
\begin{align*}
[G(\varepsilon)]^c&=\mathbb{R}^n\setminus G(\varepsilon)\subset\mathbb{R}^n\setminus\bigcup_{k = 1}^{\infty}\mathring{F}_k(\varepsilon)\\
&=\bigcup_{k = 1}^{\infty}F_k(\varepsilon)\setminus\bigcup_{k = 1}^{\infty}\mathring{F}_k(\varepsilon)\subset\bigcup_{k = 1}^{\infty}[F_k(\varepsilon)\setminus\mathring{F}_k(\varepsilon)]=\bigcup_{k = 1}^{\infty}\partial F_k(\varepsilon).
\end{align*}
因为$F_k(\varepsilon)$是闭集,所以$\partial F_k(\varepsilon)$是无处稠密集. 这说明$(G(\varepsilon))^c$是第一纲集. 
\end{proof}

\begin{example}\label{example:例16}
设$f\in C([0,1])$,且令
\[
f_1'(x)=f(x), f_2'(x)=f_1(x), \cdots, f_n'(x)=f_{n - 1}(x), \cdots.
\]
若对每一个$x\in[0,1]$,都存在自然数$k$,使得$f_k(x)=0$,则$f(x)\equiv0$.
\end{example}
\begin{proof}
只需指出$f(x)$在$[0,1]$中的一个稠密集上为$0$即可. 对此,我们在$[0,1]$中任取一个闭子区间$I$,并记
\[
F_k=\{x\in I: f_k(x)=0\} \quad (k = 1,2,\cdots).
\]
显然,每个$F_k$都是闭集,且$I=\bigcup_{k = 1}^{\infty}F_k$. 根据\hyperref[theorem:Baire定理]{Baire定理}可知,存在$F_{k_0}$,它包含一个区间$(\alpha,\beta)$. 因为在$(\alpha,\beta)$上$f_{k_0}(x)=0$,所以$f(x)=0$,$x\in(\alpha,\beta)$. 注意到$(\alpha,\beta)\subset I$,即得所证. 
\end{proof}



\subsection{Cantor(三分)集}

\begin{definition}
设$[0,1]\subset\mathbb{R}$,将$[0,1]$三等分,并移去中央三分开区间
\begin{align*}
I_{1,1}=\left(\frac{1}{3},\frac{2}{3}\right),
\end{align*}
记其留存部分为$F_1$,即
\begin{align*}
F_1=\left[0,\frac{1}{3}\right]\cup\left[\frac{2}{3},1\right]=F_{1,1}\cup F_{1,2};
\end{align*}
再将$F_1$中的区间$[0,1/3]$及$[2/3,1]$各三等分,并移去中央三分开区间
\begin{align*}
I_{2,1}=\left(\frac{1}{9},\frac{2}{9}\right) \quad \text{及} \quad I_{2,2}=\left(\frac{7}{9},\frac{8}{9}\right),
\end{align*}
记$F_1$中留存的部分为$F_2$,即
\begin{align*}
F_2&=\left[0,\frac{1}{9}\right]\cup\left[\frac{2}{9},\frac{1}{3}\right]\cup\left[\frac{2}{3},\frac{7}{9}\right]\cup\left[\frac{8}{9},1\right]\\
&=F_{2,1}\cup F_{2,2}\cup F_{2,3}\cup F_{2,4}.
\end{align*}
一般地说(归纳定义),设所得剩余部分为$F_n$,则将$F_n$中每个(互不相交)区间三等分,并移去中央三分开区间,记其留存部分为$F_{n + 1}$,如此等等. 从而我们得到集合列$\{F_n\}$,其中
\begin{align*}
F_n=F_{n,1}\cup F_{n,2}\cup\cdots\cup F_{n,2^n} \quad (n = 1,2,\cdots).
\end{align*}
作点集$C = \bigcap_{n = 1}^{\infty}F_n$,我们称$C$为\textbf{Cantor(三分)集}. 
\end{definition}

\begin{theorem}[Cantor集的基本性质]\label{theorem:Cantor集的基本性质}
\begin{enumerate}[(1)]
\item $C$是非空有界闭集,因此是紧集.

\item $C = C'$,即$C$为完全集.

\item $C$无内点.

\item Cantor集的基数是$c$.

\item $[0,1]\setminus C$的长度的总和为1.
\end{enumerate}
\end{theorem}
\begin{proof}
\begin{enumerate}[(1)]
\item 因为每个$F_n$都是非空有界闭集,而且$F_n\supset F_{n + 1}$,所以根据Cantor闭集套定理,可知$C$不是空集(实际上,$F_n$ ($n = 1,2,\cdots$) 中每个闭区间的端点都是没有被移去的,即都是$C$中的点). 显然,$C$是闭集.

\item 设$x\in C$,则$x\in F_n$ ($n = 1,2,\cdots$),即对每个$n$,$x$属于长度为$1/3^n$的$2^n$个闭区间中的一个. 于是,对任一$\delta>0$,存在$n$,满足$1/3^n<\delta$,使得$F_n$中包含$x$的闭区间含于$(x - \delta, x + \delta)$. 此闭区间有两个端点,它们是$C$中的点且总有一个不是$x$. 这就说明$x$是$C$的极限点,故得$C'\supset C$. 由(i)知$C = C'$.

\item 设$x\in C$,给定任一区间$(x - \delta, x + \delta)$,取$2/3^n<\delta$. 因为$x\in F_n$,所以$F_n$中必有某个长度为$1/3^n$的闭区间$F_{n,k}$含于$(x - \delta, x + \delta)$. 然而,在构造$C$集的第$n + 1$步时,将移去$F_{n,k}$的中央三分开区间. 这说明$(x - \delta, x + \delta)$不含于$C$.

\item 事实上,将$[0,1]$中的实数按三进位小数展开,则Cantor集中点$x$与下述三进位小数集的元
\begin{align*}
x=\sum_{i = 1}^{\infty}\frac{a_i}{3^i}, \quad a_i = 0,2
\end{align*}
一一对应. 从而知$C$为连续基数集(与$(0,1]$的二进位小数比较). 

\item 由$C$的定义可得
\begin{align*}
\sum_{n = 1}^{\infty}2^{n - 1}3^{-n}=1.
\end{align*}
\end{enumerate}
\end{proof}

\begin{definition}[类Cantor集]
设$\delta$是$(0,1)$内任意给定的数,考虑在$[0,1]$区间, 取$p=(1 + 2\delta)/\delta$,采用类似于Cantor集的构造过程:

第一步,移去长度为$1/p$的同心开区间;

第二步,在留存的两个闭区间的每一个中,又移去长度为$1/p^2$的同心开区间;

第三步,在留存的四个闭区间中再移去长度为$1/p^3$的同心区间. 继续此过程,可得一列移去的开区间,记其并集为$G$(开集),
我们称$C_p=[0,1]\setminus G$为\textbf{类Cantor集}(当$p = 3$时,$C_p$就是Cantor(三分)集).类Cantor集也称为\textbf{Harnack集}. 
\end{definition}
\begin{remark}
若要在$\mathbb{R}^n$的单位方体$[0,1]\times[0,1]\times\cdots\times[0,1]$中构造具有类似性质的集合,则只需取$C\times C\times\cdots\times C$($C$是$[0,1]$中的类Cantor集)即可.
\end{remark}

\begin{theorem}[类Cantor集的基本性质]\label{theorem:类Cantor集的基本性质}
\begin{enumerate}[(1)]
\item $G$的总长度为$\delta$($0<\delta<1$是任意给定的数)的稠密开集.

\item $C_p$是非空完全集,且没有内点.
\end{enumerate}
\end{theorem}
\begin{proof}
\begin{enumerate}[(1)]
\item 由$G$的定义可知$G$的总长度为
\begin{align*}
\sum_{n = 1}^{\infty}2^{n - 1}\left(\frac{1}{p}\right)^n=\frac{1}{p - 2}=\delta.
\end{align*}

\item 
\end{enumerate}
\end{proof}

\begin{definition}[Cantor函数]\label{definition:Cantor函数}
设 \(C\) 是 \([0,1]\) 中的 Cantor 集, 其中的点我们用三进位小数
\begin{align*}
x = 2\sum_{i = 1}^{\infty}\frac{\alpha_i}{3^i}, \quad \alpha_i = 0,1 \quad (i = 1,2,\cdots)
\end{align*}
来表示.

(i) 作定义在 \(C\) 上的函数 \(\varphi(x)\). 对于 \(x \in C\), 定义
\[
\varphi(x)=\varphi\left(2\sum_{i = 1}^{\infty}\frac{\alpha_i}{3^i}\right)=\sum_{i = 1}^{\infty}\frac{\alpha_i}{2^i}, \quad \alpha_i = 0,1 \quad (i = 1,2,\cdots).
\]

(ii) 作定义在 \([0,1]\) 上的 \(\varPhi(x)\). 对于 \(x \in [0,1]\), 定义
\[
\varPhi(x)=\sup\{\varphi(y):y\in C,y\leqslant x\}.
\]
我们称 \(\varPhi(x)\) 为 \textbf{Cantor函数}.
\end{definition}

\begin{theorem}[Cantor函数的性质]\label{theorem:Cantor函数的性质}
设$\varPhi(x)=\sup\{\varphi(y):y\in C,y\leqslant x\}$为Cantor函数, 则有下列性质:
\begin{enumerate}[(1)]
\item \(\varphi(C)=[0,1]\),即$\varphi$是满射.并且\(\varphi(x)\) 是 \(C\) 上的递增函数.

\item \(\varPhi(x)\) 是 \([0,1]\) 上的递增连续函数.此外, 在构造 Cantor 集的过程中所移去的每个中央三分开区间 \(I_{n,k}\) 上, \(\varPhi(x)\) 都是常数. 
\end{enumerate}
\end{theorem}
\begin{proof}
\begin{enumerate}[(1)]
\item 因为 \([0,1]\) 中的点可用二进位小数表示, 所以由$\varphi$的定义有 \(\varphi(C)=[0,1]\)..

下面证明$\varphi(x)$是$C$上的递增函数.设 \(\alpha_1,\alpha_2,\cdots,\beta_1,\beta_2,\cdots\) 是取 \(0\) 或 \(1\) 的数, 而且它们所表示的 \(C\) 中的数有下述关系:
\[
2\sum_{i = 1}^{\infty}\frac{\alpha_i}{3^i}<2\sum_{i = 1}^{\infty}\frac{\beta_i}{3^i}.
\]
若记 \(k = \min\{i:\alpha_i\neq\beta_i\}\), 则我们有
\begin{align*}
0&<\sum_{i = 1}^{\infty}\frac{\beta_i - \alpha_i}{3^i}=\frac{\beta_k - \alpha_k}{3^k}+\sum_{i > k}\frac{\beta_i - \alpha_i}{3^i}\\
&\leqslant\frac{\beta_k - \alpha_k}{3^k}+\sum_{i > k}\frac{2}{3^i}=\frac{\beta_k - \alpha_k + 1}{3^k}.
\end{align*}
由此可知 \((\alpha_k<\beta_k)\alpha_k = 0,\beta_k = 1\), 从而得到
\begin{align*}
\varphi\left(2\sum_{i = 1}^{\infty}\frac{\alpha_i}{3^i}\right)&=\sum_{i = 1}^{\infty}\frac{\alpha_i}{2^i}=\sum_{i = 1}^{k - 1}\frac{\alpha_i}{2^i}+\sum_{i = k}^{\infty}\frac{\alpha_i}{2^i}\\
&\leqslant\sum_{i = 1}^{k - 1}\frac{\beta_i}{2^i}+\sum_{i = k + 1}^{\infty}\frac{1}{2^i}=\sum_{i = 1}^{k - 1}\frac{\beta_i}{2^i}+\frac{1}{2^k}\\
&\leqslant\sum_{i = 1}^{k - 1}\frac{\beta_i}{2^i}+\sum_{i = k}^{\infty}\frac{\beta_i}{2^i}=\varphi\left(2\sum_{i = 1}^{\infty}\frac{\beta_i}{3^i}\right).
\end{align*}

\item 由(2)的结论及$\varPhi$的定义即得$\varPhi$的递增性.因为 \(\varPhi([0,1]) = [0,1]\), 所以由\refpro{Basis of Analytics-proposition:定义是区间的单调函数值域还是区间就必是连续函数}可知\(\varPhi(x)\) 是 \([0,1]\) 上的连续函数. 
\end{enumerate}
\end{proof}

\begin{example}
\(E\subset\mathbb{R}\) 是完全集当且仅当 \(E = \left(\bigcup_{n\geqslant 1}(a_n,b_n)\right)^c\),其中 \((a_i,b_i)\) 与 \((a_j,b_j)\) \((i\neq j)\) 无公共端点。
\end{example}
\begin{proof}
{\heiti 必要性:}若 \(E\) 是完全集,则 \(E\) 是闭集。从而 \(E^c\) 是开集,它是 \(E^c\) 内构成区间的并集。这些构成区间相互之间是没有公共端点的,否则 \(E\) 中就会有孤立点了,这是不可能的。

{\heiti 充分性:}首先,由题设知 \(E\) 是闭集。其次,对任意的 \(x\in E\),如果 \(x\notin E'\),那么存在 \(\delta>0\),使得 \((x - \delta,x + \delta)\cap E = \{x\}\)。这说明 \(x\) 是某两个开区间的端点,与假设矛盾。
\end{proof}

\begin{example}
设 \(E\subset\mathbb{R}^2\) 是完全集,则 \(E\) 是不可数集。
\end{example}
\begin{proof}
用反证法。假定 \(E = \{x_n\in\mathbb{R}^2: n = 1,2,\cdots\}\)。

(i) 选取 \(y_1\in E\setminus\{x_1\}\),则点 \(x_1\) 到 \(y_1\) 的距离大于 \(0\)。存在以 \(y_1\) 为中心的闭正方形 \(Q_1\),\(Q_1\cap E\) 是紧集。

(ii) 看 \(E\setminus\{x_2\}\)。因为 \(y_1\) 是 \(E\setminus\{x_2\}\) 的极限点,所以 \(\mathring{Q}_1\cap (E\setminus\{x_2\})\neq\varnothing\)。又取 \(y_2\in\mathring{Q}_1\cap (E\setminus\{x_2\})\),并作以 \(y_2\) 为中心的闭正方形 \(Q_2\):\(Q_2\subset Q_1\),\(x_1\notin Q_2\),\(x_2\notin Q_2\),可知 \((Q_1\cap E)\supset (Q_2\cap E)\) 是紧集。如此继续做下去,可得有界闭集套列 \(\{Q_n\cap E\}\):\((Q_{n - 1}\cap E)\supset (Q_n\cap E)\) \((n\in\mathbb{N})\),而且 \(x_1,x_2,\cdots,x_n\) 不在其内。我们有
\begin{align*}
\bigcap_{n = 1}^{\infty}(Q_n\cap E)=\varnothing,
\end{align*}
导致矛盾。 
\end{proof}

\begin{proposition}
任一非空完全集的基数均为 \(c\)。
\end{proposition}
\begin{proof}
证明见那汤松著《实变函数论》的上册,有高等教育出版社出版的中译本,1955 年.
\end{proof}

\begin{example}
设 \(E = \left\{x\in[0,1]: x = \sum_{n = 1}^{\infty}a_n/10^n, a_n = 2 \text{ 或 } 7\right\}\),我们有

(i) \(E\) 是闭集; 

(ii) \(\overline{E}=c\); 

(iii) \(E\) 在 \([0,1]\) 中不稠密。
\end{example}
\begin{proof}
(i) 若有 \(\{x_m\}\subset E\):\(x_m\rightarrow x (m\rightarrow\infty)\),则
\begin{align*}
x = \sum_{n = 1}^{\infty}b_n/10^n \quad (b_n = 0,1,2,\cdots,9).
\end{align*}
如果 \(|x_m - x|<10^{-p}\),那么在 \(x\in E\) 时,\(b_n = 2 \text{ 或 } 7 (n = 1,2,\cdots,p - 1)\)。这说明 \(E\) 是闭集。

(ii) 与 \(0\) 和 \(1\) 组成的数列类似,\(\overline{E}=c\)。

(iii) 注意到 \(E\cap(0.28,0.7)=\varnothing\),故 \(E\) 不是稠密集。 
\end{proof}
































\end{document}
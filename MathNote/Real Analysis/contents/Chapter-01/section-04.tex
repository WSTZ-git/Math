\documentclass[../../main.tex]{subfiles}
\graphicspath{{\subfix{../../image/}}} % 指定图片目录,后续可以直接使用图片文件名。

% 例如:
% \begin{figure}[h]
% \centering
% \includegraphics{image-01.01}
% \label{fig:image-01.01}
% \caption{图片标题}
% \end{figure}

\begin{document}

\section{$\mathbb{R}^n$中的基本点集:闭集$\cdot$开集$\cdot$Borel集$\cdot$Cantor集}

\subsection{闭集}

\begin{definition}[闭集与闭包]\label{definition:闭集与闭包}
设\(E\subset\mathbb{R}^n\)。若\(E\supset E'\)(即\(E\)包含\(E\)的一切极限点),则称\(E\)为\textbf{闭集}(这里规定空集为闭集)。记\(\overline{E}=E\cup E'\),并称\(\overline{E}\)为\(E\)的\textbf{闭包}(\(E\)为闭集就是\(E = \overline{E}\))。
\end{definition}

\begin{definition}[稠密子集]\label{definition:稠密子集}
  若\(A\subset B\)且\(\overline{A}=B\),则称\(A\)在\(B\)中\textbf{稠密},或称\(A\)是\(B\)的\textbf{稠密子集}.
\end{definition}

\begin{theorem}[闭集的运算性质]\label{theorem:闭集的运算性质}
  (i) 若\(F_1,F_2\)是\(\mathbb{R}^n\)中的闭集,则其并集\(F_1\cup F_2\)也是闭集,从而有限多个闭集的并集是闭集;

(ii) 若\(\{F_{\alpha}:\alpha\in I\}\)是\(\mathbb{R}^n\)中的一个闭集族,则其交集\(F = \bigcap_{\alpha\in I}F_{\alpha}\)是闭集。

(iii)设\(E_{\alpha}\subset\mathbb{R}^n(\alpha\in I)\),则
\[\bigcup_{\alpha\in I}\overline{E_{\alpha}}\subset\overline{\bigcup_{\alpha\in I}E_{\alpha}},\quad\overline{\bigcap_{\alpha\in I}E_{\alpha}}\subset\bigcap_{\alpha\in I}\overline{E_{\alpha}}.\]
\end{theorem}
\begin{remark}
  无穷多个闭集的并集不一定是闭集。例如,令
\[F_k=\left[\frac{1}{k + 1},\frac{1}{k}\right]\subset\mathbb{R}\quad(k = 1,2,\cdots),\]
则有\(\bigcup_{k = 1}^{\infty}F_k=(0,1]\)。此例还说明
\[\left[ 0,1 \right] =\overline{\bigcup_{k=1}^{\infty}{F_k}}\ne \bigcup_{k=1}^{\infty}{\overline{F_k}}=\left( 0,1 \right] .\]
\end{remark}
\begin{proof}
  (i) 从等式
\begin{align*}
\overline{F_1\cup F_2}&=(F_1\cup F_2)\cup (F_1\cup F_2)'\\
&=(F_1\cup F_2)\cup (F_1'\cup F_2')\\
&=(F_1\cup F_1')\cup (F_2\cup F_2')\\
&=\overline{F_1}\cup\overline{F_2}
\end{align*}
可知,若\(F_1,F_2\)为闭集,则\(\overline{F_1\cup F_2}=F_1\cup F_2\)。即\(F_1\cup F_2\)是闭集。

(ii) 因为对一切\(\alpha\in I\),有\(F\subset F_{\alpha}\),所以对一切\(\alpha\in I\),有\(\overline{F}\subset\overline{F_{\alpha}} = F_{\alpha}\),从而有
\[\overline{F}\subset\bigcap_{\alpha\in I}F_{\alpha}=F.\]
但\(F\subset\overline{F}\),故\(F=\overline{F}\)。这说明\(F\)是闭集。
\end{proof}

\begin{theorem}[Cantor闭集套定理]\label{theorem:Cantor闭集套定理}
若\(\{F_k\}\)是\(\mathbb{R}^n\)中的非空有界闭集列,且满足\(F_1\supset F_2\supset\cdots\supset F_k\supset\cdots\),则\(\bigcap_{k = 1}^{\infty}F_k\neq\varnothing\)。
\end{theorem}
\begin{proof}
  若在\(\{F_k\}\)中有无穷多个相同的集合,则存在自然数\(k_0\),当\(k\geqslant k_0\)时,有\(F_k = F_{k_0}\)。此时,\(\bigcap_{k = 1}^{\infty}F_k = F_{k_0}\neq\varnothing\)。现在不妨假定对一切\(k\),\(F_{k + 1}\)是\(F_k\)的真子集,即
\[F_k\setminus F_{k + 1}\neq\varnothing\quad (\text{一切 }k),\]
我们选取\(x_k\in F_k\setminus F_{k + 1}(k = 1,2,\cdots)\),则\(\{x_k\}\)是\(\mathbb{R}^n\)中的有界互异点列。根据Bolzano - Weierstrass定理可知,存在\(\{x_{k_i}\}\)以及\(x\in\mathbb{R}^n\),使得\(\lim_{i\rightarrow\infty}|x_{k_i}-x| = 0\)。由于每个\(F_k\)都是闭集,故知\(x\in F_k(k = 1,2,\cdots)\),即
\[x\in\bigcap_{k = 1}^{\infty}F_k.\]
\end{proof}

\subsection{开集}

\begin{definition}[开集]\label{definition:开集}
设\(G\subset\mathbb{R}^n\)。若\(G^c=\mathbb{R}^n\setminus G\)是闭集,则称\(G\)为\textbf{开集}。
\end{definition}
\begin{note}
  由此定义立即可知,\(\mathbb{R}^n\)本身与空集\(\varnothing\)是开集;\(\mathbb{R}^n\)中的开矩体是开集;闭集的补集是开集。
\end{note}

\begin{theorem}[开集的运算性质]\label{theorem:开集的运算性质}
(i) 若\(\{G_{\alpha}:\alpha\in I\}\)是\(\mathbb{R}^n\)中的一个开集族,则其并集\(G = \bigcup_{\alpha\in I}G_{\alpha}\)是开集;

(ii) 若\(G_k(k = 1,2,\cdots,m)\)是\(\mathbb{R}^n\)中的开集,则其交集\(G=\bigcap_{k = 1}^{m}G_k\)是开集(有限个开集的交集是开集);

(iii) 若\(G\)是\(\mathbb{R}^n\)中的非空点集,则\(G\)是开集的充分必要条件是,对于\(G\)中任一点\(x\),存在\(\delta>0\),使得\(B(x,\delta)\subset G\)。
\end{theorem}
\begin{proof}
  (i) 由定义知\(G_{\alpha}^c(\alpha\in I)\)是闭集,且有\(G^c=\bigcap_{\alpha\in I}G_{\alpha}^c\)。根据闭集的性质可知\(G^c\)是闭集,即\(G\)是开集。

(ii) 由定义知\(G_k^c(k = 1,2,\cdots,m)\)是闭集,且有\(G^c=\bigcup_{k = 1}^{m}G_k^c\)。根据闭集的性质可知\(G^c\)是闭集,即\(G\)是开集。

(iii) 若\(G\)是开集且\(x\in G\),则由于\(G^c\)是闭集以及\(x\notin G^c\),可知存在\(\delta>0\),使得\(B(x,\delta)\subset G\)。

反之,若对\(G\)中的任一点\(x\),存在\(\delta>0\),使得\(B(x,\delta)\subset G\),则
\[B(x,\delta)\cap G^c=\varnothing,\]
从而\(x\)不是\(G^c\)的极限点,即\(G^c\)的极限点含于\(G^c\)。这说明\(G^c\)是闭集,即\(G\)是开集。
\end{proof}

\begin{definition}[内点与边界点]\label{definition:内点与边界点}
  设\(E\subset\mathbb{R}^n\)。对\(x\in E\),若存在\(\delta>0\),使得\(B(x,\delta)\subset E\),则称\(x\)为\(E\)的\textbf{内点}。\(E\)的内点全体记为\(\mathring{E}\),称为\(E\)的\textbf{内核}。若\(x\in\overline{E}\)但\(x\notin\mathring{E}\),则称\(x\)为\(E\)的\textbf{边界点}。边界点全体记为\(\partial E\)。
\end{definition}
\begin{note}
  显然,内核一定为开集。\hyperref[theorem:开集的运算性质]{开集的运算性质(iii)}说明开集就是集合中每个点都是内点的集合。
\end{note}

\begin{theorem}[\(\mathbb{R}^n\)中的非空开集的性质]\label{theorem:Rn中的非空开集的性质}
(i) \(\mathbb{R}\)中的非空开集是可数个互不相交的开区间(这里也包括\((-\infty,a)\),\((b,+\infty)\)以及\((-\infty,+\infty)\))的并集;

(ii) \(\mathbb{R}^n\)中的非空开集\(G\)是可列个互不相交的半开闭方体的并集。
\end{theorem}
\begin{proof}
  (i) 设\(G\)是\(\mathbb{R}\)中的开集。对于\(G\)中的任一点\(a\),由于\(a\)是\(G\)的内点,故存在\(\delta>0\),使得\((a - \delta,a + \delta)\subset G\)。现在令
\[a'=\inf\{x:(x,a)\subset G\},\quad a''=\sup\{x:(a,x)\subset G\}\]
(这里\(a'\)可以是\(-\infty\),\(a''\)可以是\(+\infty\)),显然\(a'<a<a''\)且\((a',a'')\subset G\)。这是因为对区间\((a',a'')\)中的任一点\(z\),不妨设\(a'<z\leqslant a\),必存在\(x\),使得\(a'<x<z\)且\((x,a)\subset G\),即\(z\in G\)。我们称这样的开区间\((a',a'')\)为\(G\)(关于点\(a\))的构成区间\(I_a\)。

如果\(I_a=(a',a'')\),\(I_b=(b',b'')\)是\(G\)的构成区间,那么可以证明它们或是重合的或是互不相交的。为此,不妨设\(a < b\)。若
\[I_a\cap I_b\neq\varnothing,\]
则有\(b'<a''\)。于是令\(\min\{a',b'\}=c\),\(\max\{a'',b''\}=d\),则有\((c,d)=(a',a'')\cup(b',b'')\)。取\(x\in I_a\cap I_b\),则\(I_x=(c,d)\)是构成区间,且
\[(c,d)=(a',a'')=(b',b'').\]

最后,我们知道\(\mathbb{R}\)中互不相交的区间族是可数的。

(ii) 首先将\(\mathbb{R}^n\)用格点(坐标皆为整数)分为可列个边长为\(1\)的半开闭方体,其全体记为\(\Gamma_0\)。再将\(\Gamma_0\)中每个方体的每一边二等分,则每个方体就可分为\(2^n\)个边长为\(\frac{1}{2}\)的半开闭方体,记\(\Gamma_0\)中如此做成的子方体的全体为\(\Gamma_1\)。继续按此方法二分下去,可得其所含方体越来越小的方体族组成的序列\(\{\Gamma_k\}\),这里\(\Gamma_k\)中每个方体的边长是\(2^{-k}\),且此方体是\(\Gamma_{k + 1}\)中相应的\(2^n\)个互不相交的方体的并集。我们称如此分成的方体为二进方体。

现在把\(\Gamma_0\)中凡含于\(G\)内的方体取出来,记其全体为\(H_0\)。再把\(\Gamma_1\)中含于
\[G\setminus\bigcup_{J\in H_0}J\]
(\(J\)表示半开闭二进方体)内的方体取出来,记其全体为\(H_1\)。依此类推,\(H_k\)为\(\Gamma_k\)中含于
\[G\setminus\bigcup_{i = 0}^{k - 1}\bigcup_{J\in H_i}J\]
内的方体的全体。显然,一切由\(H_k(k = 0,1,2,\cdots)\)中的方体构成的集合为可列的。因为\(G\)是开集,所以对任意的\(x\in G\),存在\(\delta>0\),使得\(B(x,\delta)\subset G\)。而\(\Gamma_k\)中的方体的直径当\(k\rightarrow\infty\)时是趋于零的,从而可知\(x\)最终必落入某个\(\Gamma_k\)中的方体。这说明
\[G=\bigcup_{k = 0}^{\infty}\bigcup_{J\in H_k}J,\quad J表示半开闭二进方体.\]

\(\mathbb{R}^n\)中的开集还有一个重要事实,即\(\mathbb{R}^n\)中存在由可列个开集构成的开集族\(\Gamma\),使得\(\mathbb{R}^n\)中任一开集均是\(\Gamma\)中某些开集的并集。事实上,\(\Gamma\)可取为
\[\left\{B\left(x,\frac{1}{k}\right):x是\mathbb{R}^n中的有理点,k是自然数\right\}.\]
首先,\(\Gamma\)是可列集。其次,对于\(\mathbb{R}^n\)中开集\(G\)的任一点\(x\),必存在\(\delta>0\),使得\(B(x,\delta)\subset G\)。现在取有理点\(x'\),使得\(d(x,x')<1/k\),其中\(k > 2/\delta\),从而有
\[x\in B(x',1/k)\subset B(x,\delta)\subset G,\]
显然,一切如此做成的\(B(x',1/k)\)的并集就是\(G\)。
\end{proof}

\begin{definition}[开覆盖]\label{definition:开覆盖}
设\(E\subset\mathbb{R}^n\),\(\Gamma\)是\(\mathbb{R}^n\)中的一个开集族。若对任意的\(x\in E\),存在\(G\in\Gamma\),使得\(x\in G\),则称\(\Gamma\)为\(E\)的一个\textbf{开覆盖}。

设\(\Gamma\)是\(E\)的一个开覆盖。若\(\Gamma'\subset\Gamma\)仍是\(E\)的一个开覆盖,则称\(\Gamma'\)为\(\Gamma\)(关于\(E\))的一个\textbf{子覆盖}.
\end{definition}

\begin{lemma}\label{lemma:任一开覆盖都含有一个可数子覆盖}
  \(\mathbb{R}^n\)中点集\(E\)的任一开覆盖\(\Gamma\)都含有一个可数子覆盖。
\end{lemma}

\begin{theorem}[Heine - Borel有限子覆盖定理]\label{theorem:Heine - Borel有限子覆盖定理}
\(\mathbb{R}^n\)中有界闭集的任一开覆盖均含有一个有限子覆盖。
\end{theorem}
\begin{remark}
  在上述定理中,有界的条件是不能缺的。例如,在\(\mathbb{R}^1\)中对自然数集作开覆盖\(\{(n-\frac{1}{2},n+\frac{1}{2})\}\)就不存在有限子覆盖。同样,闭集的条件也是不能缺的。例如,在\(\mathbb{R}\)中对点集\(\{1,\frac{1}{2},\cdots,\frac{1}{n},\cdots\}\)作开覆盖
\[\left\{\left(\frac{1}{n}-\frac{1}{2n},\frac{1}{n}+\frac{1}{2n}\right)\right\}\quad(n = 1,2,\cdots),\]
就不存在有限子覆盖。
\end{remark}
\begin{proof}
  设\(F\)是\(\mathbb{R}^n\)中的有界闭集,\(\Gamma\)是\(F\)的一个开覆盖。由\hyperref[lemma:任一开覆盖都含有一个可数子覆盖]{引理\ref{lemma:任一开覆盖都含有一个可数子覆盖}},可以假定\(\Gamma\)由可列个开集组成:
\[\Gamma=\{G_1,G_2,\cdots,G_i,\cdots\}.\]
令
\[H_k=\bigcup_{i = 1}^{k}G_i,\quad L_k = F\cap H_k^c\quad(k = 1,2,\cdots).\]
显然,\(H_k\)是开集,\(L_k\)是闭集且有\(L_k\supset L_{k + 1}(k = 1,2,\cdots)\)。分两种情况:

(i) 存在\(k_0\),使得\(L_{k_0}\)是空集,即\(H_{k_0}\)中不含\(F\)的点,从而知\(F\subset H_{k_0}\),定理得证;

(ii) 一切\(L_k\)皆非空集,则由\hyperref[theorem:Cantor闭集套定理]{Cantor闭集套定理}可知,存在点\(x_0\in L_k(k = 1,2,\cdots)\),即\(x_0\in F\)且\(x_0\in H_k^c(k = 1,2,\cdots)\)。这就是说\(F\)中存在点\(x_0\)不属于一切\(H_k\),与原设矛盾,故第(ii)种情况不存在。
\end{proof}

\begin{theorem}\label{theorem:Rn中的有界闭集是紧集}
设\(E\subset\mathbb{R}^n\)。若\(E\)的任一开覆盖都包含有限子覆盖,则\(E\)是有界闭集。
\end{theorem}
\begin{proof}
  设\(y\in E^c\),则对于每一个\(x\in E\),存在\(\delta_x>0\),使得
\[B(x,\delta_x)\cap B(y,\delta_x)=\varnothing.\]
显然,\(\{B(x,\delta_x):x\in E\}\)是\(E\)的一个开覆盖,由题设知存在有限子覆盖,设为
\[B(x_1,\delta_{x_1}),\quad\cdots,\quad B(x_m,\delta_{x_m}).\]
由此立即可知\(E\)是有界集。现在再令
\[\delta_0=\min\{\delta_{x_1},\cdots,\delta_{x_m}\},\]
则\(B(y,\delta_0)\cap E=\varnothing\),即\(y\notin E'\)。这说明\(E'\subset E\),即\(E\)是闭集。有界性显然。
\end{proof}

\begin{definition}[紧集]\label{definition:紧集}
  如果\(E\)的任一开覆盖均包含有限子覆盖,我们就称\(E\)为\textbf{紧集}.
\end{definition}
\begin{note}
  \hyperref[theorem:Heine - Borel有限子覆盖定理]{Heine - Borel有限子覆盖定理}和\hyperref[theorem:Rn中的有界闭集是紧集]{定理\ref{theorem:Rn中的有界闭集是紧集}}表明,\(\mathbb{R}^n\)中的紧集就是有界闭集.
\end{note}

\begin{definition}[实值函数的连续]\label{definition:实值函数的连续}
  设\(f(x)\)是定义在\(E\subset\mathbb{R}^n\)上的实值函数,\(x_0\in E\)。如果对任意的\(\varepsilon>0\),存在\(\delta>0\),使得当\(x\in E\cap B(x_0,\delta)\)时,有
\[|f(x)-f(x_0)|<\varepsilon,\]
则称\(f(x)\)在\(x = x_0\)处\textbf{连续},称\(x_0\)为\(f(x)\)的一个\textbf{连续点}(在\(x_0\notin E'\)的情形,即\(x_0\)是\(E\)的孤立点时,\(f(x)\)自然在\(x = x_0\)处连续)。若\(E\)中的任一点皆为\(f(x)\)的连续点,则称\(f(x)\)\textbf{在\(\boldsymbol{E}\)上连续}。记\(E\)上的连续函数之全体为\(C(E)\)。
\end{definition}

\begin{proposition}[在$\mathbb{R}^n$的紧集上连续的函数的性质]\label{proposition:在Rn的紧集上连续的函数的性质}
  设\(F\)是\(\mathbb{R}^n\)中的有界闭集,\(f\in C(F)\),则

(i) \(f(x)\)是\(F\)上的有界函数,即\(f(F)\)是\(\mathbb{R}\)中的有界集.

(ii) 存在\(x_0\in F\),\(y_0\in F\),使得
\[f(x_0)=\sup\{f(x):x\in F\},\quad f(y_0)=\inf\{f(x):x\in F\}.\]

(iii) \(f(x)\)在\(F\)上是一致连续的,即对任给的\(\varepsilon>0\),存在\(\delta>0\),当\(x',x''\in F\)且\(|x' - x''|<\delta\)时,有
\[|f(x') - f(x'')|<\varepsilon.\]
此外,若\(E\subset\mathbb{R}^n\)上的连续函数列\(\{f_k(x)\}\)一致收敛于\(f(x)\),则\(f(x)\)是\(E\)上的连续函数。
\end{proposition}


\subsection{Borel集}

\begin{definition}[$\boldsymbol{F}_{\boldsymbol{\sigma }},\boldsymbol{G}_{\boldsymbol{\delta }}$集]\label{definition:F_sigma,G_delta集}
若 $E \subset \mathbb{R}^n$ 是可数个闭集的并集,则称 $E$ 为 $F_{\sigma}$(型)集;

若 $E \subset \mathbb{R}^n$ 是可数个开集的交集,则称 $E$ 为 $G_{\delta}$(型)集。
\end{definition}
\begin{remark}
由定义可直接推知,$F_{\sigma}$ 集的补集是 $G_{\delta}$ 集;$G_{\delta}$ 集的补集是 $F_{\sigma}$ 集。
\end{remark}

\begin{example}
记 $\mathbb{R}^n$ 中全体有理点为 $\{r_k\}$,则有理点集
\begin{align*}
\bigcup_{k = 1}^{\infty} \{r_k\}
\end{align*}
为 $F_{\sigma}$ 集。 
\end{example}



















\end{document}
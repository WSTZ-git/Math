\documentclass[../../main.tex]{subfiles}
\graphicspath{{\subfix{../../image/}}} % 指定图片目录,后续可以直接使用图片文件名。

% 例如:
% \begin{figure}[h]
% \centering
% \includegraphics{image-01.01}
% \caption{图片标题}
% \label{fig:image-01.01}
% \end{figure}
% 注意:上述\label{}一定要放在\caption{}之后,否则引用图片序号会只会显示??.

\begin{document}

\section{$\mathbb{R}^n$ 中开集、闭集及其性质}

\subsection{$n$ 维欧氏空间}

\begin{definition}
我们用 $\mathbb{R}^n$ 表示 $n$ 维欧氏空间, 即
\begin{align*}
\mathbb{R}^n = \{x = (\xi_1, \xi_2, \cdots, \xi_n) : \xi_i \in \mathbb{R}, i = 1, 2, \cdots, n\}。
\end{align*}
其中, $\xi_i$ 称为 $x$ 的第 $i$ 个\textbf{坐标}.
\end{definition}

\begin{definition}[$\mathbb{R}^n$中的加法与数乘]
设 $x = (\xi_1, \xi_2, \cdots, \xi_n)$, $y = (\eta_1, \eta_2, \cdots, \eta_n) \in \mathbb{R}^n$, $k \in \mathbb{R}$, 定义$\mathbb{R}^n$中的加法、数乘分别为
\begin{align*}
x + y &= (\xi_1 + \eta_1, \xi_2 + \eta_2, \cdots, \xi_n + \eta_n)\\
kx &= (k\xi_1, k\xi_2, \cdots, k\xi_n)
\end{align*}
\end{definition}
\begin{note}
不难证明$\mathbb{R}^n$在上述加法和数乘下构成线性空间.
\end{note}

\begin{definition}[$\mathbb{R}^n$中两点之间距离]
对于任意的 $x, y \in \mathbb{R}^n$,设 $x = (\xi_1, \xi_2, \cdots, \xi_n)$, $y = (\eta_1, \eta_2, \cdots, \eta_n) \in \mathbb{R}^n$, 定义
\begin{align*}
d(x, y) = \left(\sum_{i = 1}^{n}|\xi_i - \eta_i|^2\right)^{\frac{1}{2}}
\end{align*}
表示点 $x$ 到 $y$ 的\textbf{距离}. 通常记 $d(x, 0) = \|x\|$, 表示 $x$ 的\textbf{范数}, 若 $x \in \mathbb{R}^1$, 则 $\|x\|$ 即为 $x$ 的绝对值.
\end{definition}

\begin{proposition}[$\mathbb{R}^n$中两点之间距离的基本性质]
\begin{enumerate}[(1)]
\item (非负性) 对于任意的 $x, y \in \mathbb{R}^n$, 我们有$d(x, y) \geqslant 0$, 当且仅当 $x = y$时等号成立;
\item (对称性) 对于任意的 $x, y \in \mathbb{R}^n$, 我们有$d(x, y) = d(y, x)$;
\item (三角不等式) 对任意的 $x, y, z \in \mathbb{R}^n$, 都有
\begin{align*}
d(x, y) \leqslant d(x, z) + d(z, y)。
\end{align*}
\end{enumerate}
\end{proposition}
\begin{proof}
(1)和(2)的证明是显然的.下面证明(3).
设 $x=(x_1,x_2,\cdots,x_n),y=(y_1,y_2,\cdots,y_n),z=(z_1,z_2,\cdots,z_n)$,则
\begin{align*}
d(x,y) &=\sqrt{\sum_{i = 1}^n (x_i - y_i)^2}=\sqrt{\sum_{i = 1}^n (x_i - z_i + z_i - y_i)^2},\\
d(x,z) &=\sqrt{\sum_{i = 1}^n (x_i - z_i)^2},d(z,y)=\sqrt{\sum_{i = 1}^n (z_i - y_i)^2}.
\end{align*}
记 $x_i - z_i = a_i,z_i - y_i = b_i$,其中 $i = 1,2,\cdots,n$。则
\begin{gather}
d(x,z) =\sqrt{\sum_{i = 1}^n a_{i}^{2}},d(z,y)=\sqrt{\sum_{i = 1}^n b_{i}^{2}},\label{proposition0.1-1.1}\\
d(x,y) =\sqrt{\sum_{i = 1}^n (x_i - z_i + z_i - y_i)^2}=\sqrt{\sum_{i = 1}^n (a_i + b_i)^2}=\sqrt{\sum_{i = 1}^n a_{i}^{2}+\sum_{i = 1}^n b_{i}^{2}+2\sum_{i = 1}^n a_ib_i}.\label{proposition0.1-1.2}
\end{gather}
又由Cauchy - Schwarz不等式可得
\[
(\sum_{i = 1}^n a_ib_i)^2\leqslant (\sum_{i = 1}^n a_{i}^{2})(\sum_{i = 1}^n b_{i}^{2}).
\]
从而
\begin{align}
\sum_{i = 1}^n a_ib_i\leqslant \sqrt{(\sum_{i = 1}^n a_{i}^{2})(\sum_{i = 1}^n b_{i}^{2})}.\label{proposition0.1-1.3}
\end{align}
于是结合\eqref{proposition0.1-1.1}\eqref{proposition0.1-1.2}\eqref{proposition0.1-1.3}式可得
\begin{align*}
d(x,y) &=\sqrt{\sum_{i = 1}^n a_{i}^{2}+\sum_{i = 1}^n b_{i}^{2}+2\sum_{i = 1}^n a_ib_i}\leqslant \sqrt{\sum_{i = 1}^n a_{i}^{2}+\sum_{i = 1}^n b_{i}^{2}+2\sqrt{(\sum_{i = 1}^n a_{i}^{2})(\sum_{i = 1}^n b_{i}^{2})}}\\
&=\sqrt{[\sqrt{\sum_{i = 1}^n a_{i}^{2}}+\sqrt{\sum_{i = 1}^n b_{i}^{2}}]^2}=\sqrt{\sum_{i = 1}^n a_{i}^{2}}+\sqrt{\sum_{i = 1}^n b_{i}^{2}}=d(x,z)+d(z,y).
\end{align*} 
\end{proof}


\begin{definition}
设 $\{x_k\}$ 为 $\mathbb{R}^n$ 中的点列, 若存在 $x \in \mathbb{R}^n$ 使得
\begin{align*}
\lim_{k \to \infty} d(x_k, x) = 0
\end{align*}
则称 $\{x_k\}$ 收敛于 $x$, 记为 $\lim_{k \to \infty} x_k = x$ 或 $x_k \to x (k \to \infty)$.
\end{definition}

\begin{proposition}
设 $\{x_k\} \subset \mathbb{R}^n$, 则 $\{x_k\}$ 是收敛数列, 当且仅当
\begin{align*}
\lim_{i, j \to \infty} d(x_i, x_j) = 0
\end{align*} 
\end{proposition}
\begin{proof}
设 $x_k = (\xi_1^{(k)}, \xi_2^{(k)}, \cdots, \xi_n^{(k)})$, $x = (\xi_1, \xi_2, \cdots, \xi_n)$. 注意到
\begin{align*}
|\xi_i^{(k)} - \xi_i| \leqslant d(x_k, x) \leqslant \sum_{i = 1}^{n}|\xi_i^{(k)} - \xi_i|
\end{align*}
易证 $x_k \to x$, 当且仅当对每个坐标位置 $i$, 都有 $\xi_i^{(k)} \to \xi_i (k \to \infty)$. 再由柯西收敛准则即可得到证明.
\end{proof}

\subsection{$\mathbb{R}^n$ 中的开集及其性质} 

\begin{definition}[邻域、内点、内部和开集]
设 $x_0 \in \mathbb{R}^n$, $\varepsilon > 0$, 定义
\begin{align*}
U(x_0, \varepsilon) = \{x \in \mathbb{R}^n : d(x, x_0) < \varepsilon\}
\end{align*}
为 $x_0$ 的 $\varepsilon -$ \textbf{邻域}.

设 $A \subset \mathbb{R}^n$, $x \in A$, 若存在 $\varepsilon_0 > 0$ 使得 $U(x, \varepsilon_0) \subset A$, 则称 $x$ 为 $A$ 的\textbf{内点}. $A$ 的全体内点, 记为 $A^\circ$, 也称为 $A$ 的\textbf{内部}.

若 $A$ 中每个点都是 $A$ 的内点, 则称 $A$ 为\textbf{开集}.此即对$\forall x\in A,$都存在$r_x>0$,使得
\begin{align*}
U(x,r_x)\subset A.
\end{align*}
\end{definition}
\begin{note}
$(a, b)$, $(-\infty, a)$, $(a, +\infty)$ 都是 $\mathbb{R}$ 中的开集; 邻域 $U(x_0, r)$, 又称以 $x_0$ 为心、以 $r$ 为半径的开球, 是 $\mathbb{R}^n$ 中的开集; $A^\circ$ 也是开集.
\end{note}

\begin{proposition}[开集的性质]\label{proposition:开集的性质}
\begin{enumerate}[(1)]
\item $\varnothing$ 和 $\mathbb{R}^n$ 是开集;
\item 任意个开集的并集是开集;
\item 有限个开集的交集是开集.
\end{enumerate}
\end{proposition}
\begin{remark}
无限个开集的交集不一定是开集. 例如
\begin{align*}
\bigcap_{n = 1}^{\infty}\left(-\frac{1}{n}, \frac{1}{n}\right) = \{0\}.
\end{align*}
\end{remark}
\begin{proof}
\begin{enumerate}[(1)]
\item 显然.

\item 设 $\{G_\alpha : \alpha \in \varGamma\}$ 为一族开集. 任取 $x \in \bigcup_{\alpha \in \varGamma} G_\alpha$, 则存在 $\alpha_0 \in \varGamma$ 使得 $x \in G_{\alpha_0}$. 由于 $G_{\alpha_0}$ 是开集, 则存在 $\varepsilon_0 > 0$ 使得
\begin{align*}
U(x, \varepsilon_0) \subset G_{\alpha_0} \subset \bigcup_{\alpha \in \varGamma} G_\alpha
\end{align*}
故 $x$ 是 $\bigcup_{\alpha \in \varGamma} G_\alpha$ 的内点. 再由 $x$ 的任意性知 $\bigcup_{\alpha \in \varGamma} G_\alpha$ 是开集.

\item 设 $G_1, G_2, \cdots, G_n$ 为开集, 任取 $x \in \bigcap_{i = 1}^{n} G_i$, 则 $x \in G_i$, $i = 1, 2, \cdots, n$. 由于 $G_i$ 是开集, 故存在 $\varepsilon_i > 0$ 使得
\begin{align*}
U(x, \varepsilon_i) \subset G_i, \quad i = 1, 2, \cdots, n
\end{align*}
令 $\varepsilon = \min\{\varepsilon_1, \varepsilon_2, \cdots, \varepsilon_n\}$, 则 $\varepsilon > 0$ 且 $U(x, \varepsilon) \subset \bigcap_{i = 1}^{n} G_i$, 故 $x$ 是 $\bigcap_{i = 1}^{n} G_i$ 的内点. 再由 $x$ 的任意性知 $\bigcap_{i = 1}^{n} G_i$ 是开集.
\end{enumerate}
\end{proof} 

\begin{proposition}\label{proposition:集合的内部一定是开集}
设$A$为非空集合,则$A^\circ$ 为开集.
\end{proposition}
\begin{proof}
任取 $x \in A^\circ$, 则存在 $r > 0$ 使得 $U(x, r) \subset A$. 对 $\forall y \in U(x, r)$, 令 $\delta = r - d(x, y)$, 则易知 $U(y, \delta) \subset A$, 故 $y \in A^\circ$. 从而 $U(x, r) \subset A^\circ$, 因此, $A^\circ$ 是开集.
\end{proof}

\subsection{$\mathbb{R}^n$ 中的闭集及其性质}

\begin{definition}[聚点、极限点和孤立点]
设 $x_0 \in \mathbb{R}^n$, $A \subset \mathbb{R}^n$. 若对 $\forall \varepsilon > 0$, 都有
\begin{align*}
U(x_0, \varepsilon) \cap (A - \{x_0\}) \neq \varnothing.
\end{align*}
则称 $x_0$ 为 $A$ 的\textbf{聚点}或\textbf{极限点}. 

不是聚点, 即存在 $\varepsilon_0 > 0$ 使得
\begin{align*}
U(x_0, \varepsilon_0) \cap (A - \{x_0\}) = \varnothing
\end{align*}
则称 $x_0$ 为 $A$ 的\textbf{孤立点}.
\end{definition}
\begin{remark}
$\mathbb{R}^n$ 空间, 聚点 = 内点 + 边界点, 故聚点不一定属于 $A$, 比如边界点. 例如, $A = (0, 1)$, 则 $[0, 1]$ 都是 $A$ 的聚点.
\end{remark}

\begin{proposition}[Rn中聚点的等价条件]\label{proposition:Rn中聚点的等价条件}
设 $x_0 \in \mathbb{R}^n$, $A \subset \mathbb{R}^n$. 则$x_0$ 是 $A$ 的聚点等价于:
\begin{enumerate}[(1)]
\item 对 $\forall \varepsilon > 0$, 都有$U(x_0, \varepsilon) \cap (A - \{x_0\}) \neq \varnothing.$

\item 对 $\forall \varepsilon > 0$, $U(x, \varepsilon)$ 中含有无穷多 $A$ 中的点.
\end{enumerate}
\end{proposition}
\begin{proof}
(2)$\Leftarrow$(1)是显然的.下证
(1)$\Rightarrow$(2).

假设存在 $\varepsilon_0 > 0$, 使得
\begin{align*}
U(x_0, \varepsilon_0) \cap (A - \{x_0\}) = \{x_1, x_2, \cdots, x_n\}
\end{align*}
令
\begin{align*}
\delta = \min\{|x_0 - x_i| : i = 1, 2, \cdots, n\}
\end{align*}
则 $U(x_0, \delta)$ 中不含 $A$ 中异于 $x_0$ 的点, 这与 $x_0$ 是 $A$ 的聚点矛盾.
\end{proof}

\begin{definition}[导集、完全集、闭包和闭集]
设 $A \subset \mathbb{R}^n$, 则 $A$ 的聚点的全体, 称为 $A$ 的\textbf{导集}, 记为 $A'$. 若 $A' = A$, 则称 $A$ 为\textbf{完全集} (无孤立点).
$A - A'$ 中的点, 即为所有孤立点.

$A \cup A'$ 称为 $A$ 的\textbf{闭包}, 记为 $\overline{A}$.
开集的余集, 称为\textbf{闭集}.
\end{definition}
\begin{note}
例如, $[a, b]$, $(-\infty, a]$, $[a, +\infty)$ 都是 $\mathbb{R}$ 中的闭集; 以 $x_0$ 为心, 以 $r$ 为半径的闭球 $B(x_0, r) = \{x \in \mathbb{R}^n : d(x, x_0) \leqslant r\}$ 是 $\mathbb{R}^n$ 中的闭集; $A'$, $\overline{A}$ 也是闭集.
\end{note}

\begin{proposition}
\begin{enumerate}[(1)]
\item 若 $A \subset B$, 则 $A' \subset B'$;
\item $(A \cup B)' = A' \cup B'$.
\end{enumerate}
\end{proposition}
\begin{proof}
利用导集的定义容易验证.
\end{proof}

\begin{proposition}[闭集的性质]\label{proposition:闭集的性质}
\begin{enumerate}[(1)]
\item $\varnothing$ 和 $\mathbb{R}^n$ 是闭集;
\item 任意个闭集的交集是闭集;
\item 有限个闭集的并集是闭集.
\end{enumerate}
\end{proposition}
\begin{remark}
无限个闭集的并集不一定是闭集. 例如, $\bigcup_{n = 1}^{\infty} [1/n, 1] = (0, 1]$.
\end{remark}
\begin{proof}
由\refpro{proposition:开集的性质},闭集的定义以及\hyperref[Set Theory-theorem:De Morgan定律]{De Morgan定律},容易验证.
\end{proof}

\begin{proposition}\label{proposition:一个点在导集或闭包中的充要条件}
设 $A \subset \mathbb{R}^n$, 则

(1) $x \in A' \Leftrightarrow \exists \{x_n\} \subset (A - \{x\})$ 使得 $x_n \to x$;

(2) $x \in \overline{A} \Leftrightarrow \exists \{x_n\} \subset A$ 使得 $x_n \to x$.
\end{proposition}
\begin{proof}
(1) “$\Rightarrow$”. 若 $x \in A'$, 则对 $\forall \varepsilon > 0$, 都有 $U(x, \varepsilon) \cap (A - \{x\}) \neq \varnothing$. 特别地, 依次令 $\varepsilon_n = 1/n$, $n = 1, 2, \cdots$, 取 $x_n \in U(x, 1/n) \cap (A - \{x_0\})$, 则 $x_n \to x$.

“$\Leftarrow$”. 设 $\{x_n\} \subset (A - \{x\})$ 满足 $x_n \to x$. 由于 $x_n \to x$, 则对 $\forall \varepsilon > 0$, $\exists N \in \mathbb{N}$, 使得当 $n \geqslant N$ 时有 $d(x_n, x) < \varepsilon$, 即 $x_n \in U(x, \varepsilon)$. $\forall n \geqslant N$. 又 $x_n \neq x$, 故 $U(x, \varepsilon) \cap (A - \{x\}) \neq \varnothing$. 因此, $x \in A'$. 

(2) “$\Rightarrow$”. $\overline{A} = A \cup A'$. 若 $x \in A$, 令 $x_n = x$, $n \in \mathbb{N}$, 则 $x_n \to x$. 若 $x \in A'$, 由 (i) 知, 结论仍然成立.

“$\Leftarrow$”. 设 $\{x_n\} \subset A$ 满足 $x_n \to x$. 若 $\exists n_0 \in \mathbb{N}$, 使得 $x_{n_0} = x$, 则 $x \in A \subset \overline{A}$. 否则 $\forall n \in \mathbb{N}$, 都有 $x_n \neq x$, 则由 (i) 知, $x \in A' \subset \overline{A}$.
\end{proof}

\begin{theorem}
$A$ 为闭集 $\Leftrightarrow A' \subset A$.
\end{theorem}
\begin{proof}
“$\Rightarrow$”. 设 $A$ 为闭集, 则 $A^c$ 为开集. 任取 $x \in A'$, 往证 $x \in A$, 即 $x \notin A^c$. 若 $x \in A^c$, 由于 $A^c$ 是开集, 则存在 $\varepsilon_0 > 0$ 使得 $U(x, \varepsilon_0) \subset A^c$. 故 $U(x, \varepsilon_0) \cap A = \varnothing$, 从而 $U(x, \varepsilon_0) \cap (A - \{x\}) = \varnothing$. 这与 $x \in A'$ 矛盾.

“$\Leftarrow$”. 设 $A' \subset A$, 往证 $A$ 是闭集, 即 $A^c$ 是开集. 任取 $x \in A^c$, 由于 $A' \subset A$, 则 $x \notin A'$. 故 $\exists \varepsilon_0 > 0$ 使得 $U(x, \varepsilon_0) \cap (A - \{x\}) = \varnothing$, 从而
\begin{align*}
U(x, \varepsilon_0) \subset (A - \{x\})^c = (A \cap \{x\}^c)^c = A^c \cup \{x\} = A^c
\end{align*}
因此, $A^c$ 是开集.
\end{proof}

\begin{theorem}
$A$ 为闭集 $\Leftrightarrow A = \overline{A}$.
\end{theorem}
\begin{remark}
闭集 $A = \overline{A}$, 再由\hyperref[proposition:一个点在导集或闭包中的充要条件]{命题\ref{proposition:一个点在导集或闭包中的充要条件}(2)}知, 闭集中任一点都能找到闭集中的一个点列收敛到该点. 这也是闭集才具有的好的性质.
\end{remark}
\begin{proof}
“$\Rightarrow$”. $A$ 为闭集, 则 $A' \subset A$. 故 $\overline{A} = A \cup A' \subset A \subset \overline{A}$. 因此, $A = \overline{A}$.

“$\Leftarrow$”. 若 $A = \overline{A}$, 则 $A' \subset \overline{A} = A$, 故 $A$ 是闭集.
\end{proof}

\begin{theorem}
设$A$为一个非空集合,则$A'$ 为闭集.
\end{theorem}
\begin{proof}
只需证明 $(A')' \subset A'$. 设 $x \in (A')'$, 由\hyperref[proposition:一个点在导集或闭包中的充要条件]{命题\ref{proposition:一个点在导集或闭包中的充要条件}(1)}, 存在 $\{x_n\} \subset A' - \{x\}$ 使得 $x_n \to x$. 往证 $x \in A'$, 即存在 $\{y_n\} \subset A - \{x\}$ 使得 $y_n \to x$.

对于固定的 $n \in \mathbb{N}$, 由于 $x_n \in A'$, 则存在 $y_n \in A - \{x, x_n\}$ 使得 $d(y_n, x_n) < 1/n$. 于是
\begin{align*}
d(y_n, x) \leqslant d(y_n, x_n) + d(x_n, x) \to 0, \quad n \to \infty
\end{align*}
故 $y_n \to x$.
\end{proof}

\begin{definition}[连续映射]
设 $X$, $Y$ 为距离空间, 称映射 $f : X \to Y$ 在点 $x_0 \in X$ 处连续, 是指对 $\forall \varepsilon > 0$, 存在 $\delta > 0$ 使得当 $d(x, x_0) < \delta$ 时, 有 $d(f(x), f(x_0)) < \varepsilon$.

若 $f$ 在任意点 $x \in X$ 都连续, 则称 $f$ 为 $X$ 上的\textbf{连续映射}.
\end{definition}
\begin{remark}
$f$ 在 $x_0$ 点连续可等价地用集合语言描述如下:
\begin{align*}
\forall \varepsilon > 0, \exists \delta > 0, \text{ 使得 } f(U(x_0, \delta)) \subset U(f(x_0), \varepsilon)
\end{align*}
\end{remark}

\begin{theorem}[连续映射的充要条件]\label{theorem:连续映射的充要条件}
设 $f : X \to Y$ 是映射, 则下列条件等价:
\begin{enumerate}[(1)]
\item $f$ 连续;
\item $Y$ 的任一开集在 $f$ 下的原象是 $X$ 中的开集;
\item $Y$ 的任一闭集在 $f$ 下的原象是 $X$ 中的闭集.
\end{enumerate}
\end{theorem}
\begin{remark}
若上述定理的(2)换成 “$X$ 的任一开集在 $f$ 下的象是 $Y$ 中的开集”, 结论不一定成立. 因为连续映射在开集上的象未必是开集. 例如, $f(x) = |x|$, 则 $f$ 在开区间 $(-1, 1/2)$ 上连续, 但 $f$ 的象是 $[0, 1)$, 不是开集. 
\end{remark}
\begin{proof}
(1) $\Rightarrow$ (2). 设 $f$ 连续, $G \subset Y$ 为开集. 不妨设 $f^{-1}(G) \neq \varnothing$, 任取 $x_0 \in f^{-1}(G)$, 则 $f(x_0) \in G$. 由于 $G$ 是开集, 则 $\exists \varepsilon > 0$ 使得 $U(f(x_0), \varepsilon) \subset G$. 又 $f$ 连续, 则对上述 $\varepsilon$, $\exists \delta > 0$ 使得
\begin{align*}
f(U(x_0, \delta)) \subset U(f(x_0), \varepsilon) \subset G
\end{align*}
从而 $U(x_0, \delta) \subset f^{-1}(G)$. 这就证明了 $x_0$ 是 $f^{-1}(G)$ 的内点. 因此, $f^{-1}(G)$ 是开集.

(2) $\Rightarrow$ (1). 设 $x_0 \in X$, 对 $\forall \varepsilon > 0$, 都有 $U(f(x_0), \varepsilon)$ 是 $Y$ 中的开集, 从而由 (2) 知 $f^{-1}(U(f(x_0), \varepsilon))$ 是 $X$ 中的开集. 又 $x_0 \in f^{-1}(U(f(x_0), \varepsilon))$, 则存在 $\delta > 0$ 使得
\begin{align*}
U(x_0, \delta) \subset f^{-1}(U(f(x_0), \varepsilon))
\end{align*}
故 $f(U(x_0, \delta)) \subset U(f(x_0), \varepsilon)$, 从而 $f$ 在点 $x_0$ 连续, 再由 $x_0$ 的任意性知 $f$ 连续.

(2) $\Rightarrow$ (3). 设 $F \subset Y$ 为闭集, 则 $F^c$ 是开集, 故由 (2) 知 $f^{-1}(F^c)$ 是 $X$ 中的开集. 于是, $f^{-1}(F) = (f^{-1}(F^c))^c$ 是 $X$ 中的闭集. 

(3) $\Rightarrow$ (2) 类似.
\end{proof}





\end{document}
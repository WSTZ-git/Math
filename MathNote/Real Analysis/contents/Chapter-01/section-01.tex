\documentclass[../../main.tex]{subfiles}
\graphicspath{{\subfix{../../image/}}} % 指定图片目录,后续可以直接使用图片文件名。

% 例如:
% \begin{figure}[H]
% \centering
% \includegraphics{image-01.01}
% \caption{图片标题}
% \label{figure:image-01.01}
% \end{figure}
% 注意:上述\label{}一定要放在\caption{}之后,否则引用图片序号会只会显示??.

\begin{document}

\section{集合之间的运算}

\begin{theorem}\label{theorem:集合运算的基本性质}
设有集合\(A,B\)与\(C\),则

(i) 交换律:
\[A\cup B = B\cup A, \quad A\cap B = B\cap A;\]

(ii) 结合律:
\begin{align*}
  A\cup(B\cup C)&=(A\cup B)\cup C,\\
A\cap(B\cap C)&=(A\cap B)\cap C;
\end{align*}

(iii) 分配律:
\begin{align*}
A\cap(B\cup C)&=(A\cap B)\cup(A\cap C),\\
A\cup(B\cap C)&=(A\cup B)\cap(A\cup C).
\end{align*}
\end{theorem}

\begin{definition}[集族的并和交]\label{definition:集族的并和交}
设有集合族\(\{A_{\alpha}\}_{\alpha\in I}\),我们定义其并集与交集如下:
\begin{align*}
\bigcup_{\alpha\in I}A_{\alpha}&=\{x:\text{存在 }\alpha\in I,x\in A_{\alpha}\}=\{x:\exists \alpha \in I\,\,s.t\,\,\,x\in A_{\alpha}\},\\
\bigcap_{\alpha\in I}A_{\alpha}&=\{x:\text{对一切 }\alpha\in I,x\in A_{\alpha}\}=\{x:\forall \alpha \in I,x\in A_{\alpha}\}.
\end{align*}
\end{definition}

\begin{theorem}\label{theorem:集族的并和交的基本性质}
\begin{enumerate}[(1)]
\item 广义交换律和结合律:当一个集合族被分解(以任何方式)为许多子集合族时,那么先作子集合族中各集合的并集,然后再作各并集的并集,仍然得到原集合族的并,而且作并集时与原有的顺序无关。当然,对于交的运算也是如此。

\item 分配律:

(i) \(A\cap\left(\bigcup_{\alpha\in I}B_{\alpha}\right)=\bigcup_{\alpha\in I}(A\cap B_{\alpha})\);

(ii) \(A\cup\left(\bigcap_{\alpha\in I}B_{\alpha}\right)=\bigcap_{\alpha\in I}(A\cup B_{\alpha})\)。

\item \(\bigcup_{\alpha \in I} A_{\alpha} \setminus \bigcup_{\alpha \in I} B_{\alpha} \subset \bigcup_{\alpha \in I} (A_{\alpha} \setminus B_{\alpha}).\)

\item $\bigcup_{\alpha \in I}{\left( A_{\alpha}\cap B_{\alpha} \right)}\subset \bigcup_{\alpha \in I}{A_{\alpha}}\cap \bigcup_{\alpha \in I}{B_{\alpha}}.$
\end{enumerate}
\end{theorem}
\begin{enumerate}[(1)]
\item 

\item 

\item 对\(\forall x \in \bigcup_{\alpha \in I} A_{\alpha} \setminus \bigcup_{\alpha \in I} B_{\alpha}\),存在\(\alpha_x \in I\),使\(x \in A_{\alpha_x}\),并且\(x \notin B_{\alpha}, \forall \alpha \in I\)。从而\(x \in A_{\alpha_x} \setminus B_{\alpha_x} \subset \bigcup_{\alpha \in I} (A_{\alpha} \setminus B_{\alpha})\)。故\(\bigcup_{\alpha \in I} A_{\alpha} \setminus \bigcup_{\alpha \in I} B_{\alpha} \subset \bigcup_{\alpha \in I} (A_{\alpha} \setminus B_{\alpha})\)。

\item 对$\forall x\in \bigcup_{\alpha \in I}{\left( A_{\alpha}\cap B_{\alpha} \right)}$, 都存在$\alpha _x\in I$, 使得$x\in A_{\alpha _x}\cap B_{\alpha _x}$. 于是$x\in \bigcup_{\alpha \in I}{A_{\alpha}}$且$x\in \bigcup_{\alpha \in I}{B_{\alpha}}$, 即$x\in \bigcup_{\alpha \in I}{A_{\alpha}}\cap \bigcup_{\alpha \in I}{B_{\alpha}}$. 
\end{enumerate}


\begin{definition}
设$A,B$是两个集合,称$\{x:x\in A,x\notin B\}$为$A$与$B$的\textbf{差集},记作$A-B$或$A\setminus B$.

在上述定义中,当$B\subset A$时,称$A-B$为集合$B$相对于集合$A$的\textbf{补集}或\textbf{余集}.

通常,在我们讨论问题的范围内,所涉及的集合总是某个给定的“大”集合\(X\)的子集,我们称\(X\)为全集。此时,集合\(B\)相对于全集\(X\)的补集就简称为\(B\)的补集或余集,并记为\(B^c\)或\(\mathscr{C} B\),即
\[B^c = X- B.\]
今后,凡没有明显标出全集\(X\)时,都表示取补集运算的全集\(X\)预先已知,而所讨论的一切集合皆为其子集。于是\(B^c\)也记为
\[B^c = \{x\in X:x\notin B\}.\]
\end{definition}

\begin{proposition}[集合的差与补的基本性质]\label{proposition:集合的差与补的基本性质}
\begin{enumerate}[(1)]
\item \(A\cup A^c = X\),\(A\cap A^c = \varnothing\),\((A^c)^c = A\),\(X^c = \varnothing\),\(\varnothing^c = X\)。

\item\(A- B = A\cap B^c\)。

\item 若\(A\supset B\),则\(A^c\subset B^c\);若\(A\cap B = \varnothing\),则\(A\subset B^c\)。

\item $A-B^c=B-A^c$.
\end{enumerate}
\end{proposition}
\begin{proof}
\begin{enumerate}[(1)]
\item 

\item 

\item 

\item $x\in A-B^c\Longleftrightarrow x\in A\text{且}x\notin B^c\Longleftrightarrow x\in A\text{且}x\in B
\Longleftrightarrow x\in B\text{且}x\notin A^c\Longleftrightarrow x\in B-A^c.$
\end{enumerate}
\end{proof}


\begin{theorem}[De Morgan法则]\label{theorem:De Morgan法则}
  (i) \(\left(\bigcup_{\alpha\in I}A_{\alpha}\right)^c=\bigcap_{\alpha\in I}A_{\alpha}^c\); $\quad \quad$
(ii) \(\left(\bigcap_{\alpha\in I}A_{\alpha}\right)^c=\bigcup_{\alpha\in I}A_{\alpha}^c\)。  
\end{theorem}
\begin{proof}
  以(i)为例。若\(x\in\left(\bigcup_{\alpha\in I}A_{\alpha}\right)^c\),则\(x\notin\bigcup_{\alpha\in I}A_{\alpha}\),即对一切\(\alpha\in I\),有\(x\notin A_{\alpha}\)。这就是说,对一切\(\alpha\in I\),有\(x\in A_{\alpha}^c\)。故得\(x\in\bigcap_{\alpha\in I}A_{\alpha}^c\)。

反之,若\(x\in\bigcap_{\alpha\in I}A_{\alpha}^c\),则对一切\(\alpha\in I\),有\(x\in A_{\alpha}^c\),即对一切\(\alpha\in I\),有\(x\notin A_{\alpha}\)。这就是说,
\[x\notin\bigcup_{\alpha\in I}A_{\alpha}, \quad x\in\left(\bigcup_{\alpha\in I}A_{\alpha}\right)^c.\]
\end{proof}

\begin{definition}[集合的对称差]\label{definition:集合的对称差}
  设\(A,B\)为两个集合,称集合\((A\setminus B)\cup(B\setminus A)\)为\(A\)与\(B\)的\textbf{对称差集},记为\(A\triangle B\).
\end{definition}

\begin{proposition}[集合的对称差的基本性质]\label{proposition:集合的对称差的基本性质}
  (i) \(A\triangle\varnothing = A\),\(A\triangle A=\varnothing\),\(A\triangle A^c = X\),\(A\triangle X = A^c\)。

(ii) 交换律:\(A\triangle B = B\triangle A\)。

(iii) 结合律:\((A\triangle B)\triangle C = A\triangle(B\triangle C)\)。

(iv) 交与对称差满足分配律:
\[A\cap(B\triangle C)=(A\cap B)\triangle(A\cap C).\]

(v) \(A^c\triangle B^c = A\triangle B\);\(A = A\triangle B\)当且仅当\(B = \varnothing\)。

(vi) 对任意的集合\(A\)与\(B\),存在唯一的集合\(E\),使得\(E\triangle A = B\)(实际上\(E = B\triangle A\))。
\end{proposition}

\begin{definition}[递增、递减集合列]\label{definition:递增、递减集合列}
设\(\{A_k\}\)是一个集合列。若
\[A_1\supset A_2\supset\cdots\supset A_k\supset\cdots,\]
则称此集合列为\textbf{递减集合列},此时称其交集\(\bigcap_{k = 1}^{\infty}A_k\)为集合列\(\{A_k\}\)的极限集,记为\(\lim_{k\rightarrow\infty}A_k\);若\(\{A_k\}\)满足
\[A_1\subset A_2\subset\cdots\subset A_k\subset\cdots,\]
则称\(\{A_k\}\)为\textbf{递增集合列},此时称其并集\(\bigcup_{k = 1}^{\infty}A_k\)为\(\{A_k\}\)的极限集,记为\(\lim_{k\rightarrow\infty}A_k\)。
\end{definition}

\begin{proposition}\label{proposition:单调集合上下限的一般形式}
\begin{enumerate}
\item 当$\{A_k\}$为递减集合列时,$\lim_{k\to \infty}A_k=\bigcap_{k=1}^{\infty}A_k=\bigcap_{k=N}^{\infty}A_k$($\forall N\in \mathbb{N}$).

\item 当$\{A_k\}$为递增集合列时,$\lim_{k\to \infty}A_k=\bigcup_{k=1}^{\infty}A_k=\bigcup_{k=N}^{\infty}A_k$($\forall N\in \mathbb{N}$).
\end{enumerate}

\end{proposition}
\begin{proof}
\begin{enumerate}
\item 对$\forall N\in \mathbb{N}$,一方面,
由$\bigcap_{k=1}^{\infty}{A_k}=\bigcap_{k=1}^{N-1}{A_k}\cap \bigcap_{k=N}^{\infty}{A_k}$可知$\bigcap_{k=1}^{\infty}{A_k}\subset \bigcap_{k=N}^{\infty}{A_k}$.另一方面,由$\{A_k\}$为递减集合列可得
\begin{align*}
A_1\supset A_2\supset \cdots \supset A_{N-1}\supset A_k,\forall k=N,N+1,\cdots .
\end{align*}
因此$\bigcap_{k=1}^{N-1}{A_k}\supset \bigcap_{k=N}^{\infty}{A_k}$,故再根据$\bigcap_{k=1}^{\infty}{A_k}=\bigcap_{k=1}^{N-1}{A_k}\cap \bigcap_{k=N}^{\infty}{A_k}$可知$\bigcap_{k=1}^{\infty}{A_k}\supset \bigcap_{k=N}^{\infty}{A_k}$.

\item 对$\forall N\in \mathbb{N}$,一方面,
由$\bigcup_{k=1}^{\infty}{A_k}=\bigcup_{k=1}^{N-1}{A_k}\cup \bigcup_{k=N}^{\infty}{A_k}$可知$\bigcup_{k=1}^{\infty}{A_k}\supset \bigcup_{k=N}^{\infty}{A_k}$.另一方面,由$\{A_k\}$为递增集合列可得
\begin{align*}
A_1\subset A_2\subset \cdots \subset A_{N-1}\subset A_N.
\end{align*}
因此$\bigcup_{k=1}^{N-1}{A_k}\subset A_N \subset \bigcup_{k=N}^{\infty}{A_k}$,故再根据$\bigcup_{k=1}^{\infty}{A_k}=\bigcup_{k=1}^{N-1}{A_k}\cup \bigcup_{k=N}^{\infty}{A_k}$可知$\bigcup_{k=1}^{\infty}{A_k}\subset \bigcup_{k=N}^{\infty}{A_k}$.
\end{enumerate}
\end{proof}

\begin{definition}[上、下极限集]\label{definition:上、下极限集}
  设\(\{A_k\}\)是一集合列,令
\[B_j=\bigcup_{k = j}^{\infty}A_k\quad (j = 1,2,\cdots),\]
显然有\(B_j\supset B_{j + 1}(j = 1,2,\cdots)\)。我们称
\[\lim_{k\rightarrow\infty}B_k=\bigcap_{j = 1}^{\infty}B_j=\bigcap_{j = 1}^{\infty}\bigcup_{k = j}^{\infty}A_k\]
为集合列\(\{A_k\}\)的\textbf{上极限集},简称为\textbf{上限集},记为
\[\varlimsup_{k\rightarrow\infty}A_k=\bigcap_{j = 1}^{\infty}\bigcup_{k = j}^{\infty}A_k.\]

类似地,称集合\(\bigcup_{j = 1}^{\infty}\bigcap_{k = j}^{\infty}A_k\)为集合列\(\{A_k\}\)的\textbf{下极限集},简称为\textbf{下限集},记为
\[\varliminf_{k\rightarrow\infty}A_k=\bigcup_{j = 1}^{\infty}\bigcap_{k = j}^{\infty}A_k.\]

若上、下限集相等,则说\(\{A_k\}\)的极限集存在并等于上限集或下限集,记为\(\lim_{k\rightarrow\infty}A_k\)。
\end{definition}

\begin{proposition}[上、下极限集的性质]\label{proposition:上、下极限集的性质}
  设\(\{A_k\}\)是一集合列,$E$是一个集合则
\begin{align*}
  (i) E\setminus\varlimsup_{k\rightarrow\infty}A_k=\varliminf_{k\rightarrow\infty}(E\setminus A_k);\quad (ii) E\setminus\varliminf_{k\rightarrow\infty}A_k=\varlimsup_{k\rightarrow\infty}(E\setminus A_k).
\end{align*}
\end{proposition}

\begin{theorem}\label{theorem:上、下极限集的刻画}
  若\(\{A_k\}\)为一集合列,则
\begin{align*}
  &(\mathrm{i})\underset{k\rightarrow \infty}{\overline{\lim }}A_k=\bigcap_{j=1}^{\infty}{\bigcup_{k=j}^{\infty}{A_k}}=\{x:\text{对任一自然数} j,\text{存在} k(k\geqslant j),x\in A_k\}=\left\{ x:\forall j\in \mathbb{N} ,\exists k\geqslant j\text{且}k\in \mathbb{N} \,\,s.t.\,\,x\in A_k \right\}
\\
&(\mathrm{i}\mathrm{i})\varliminf_{k\rightarrow\infty}A_k=\bigcup_{j = 1}^{\infty}\bigcap_{k = j}^{\infty}A_k = \{x:\text{存在自然数 }j_0,\text{当 }k\geqslant j_0\text{ 时},x\in A_k\}=\left\{ x:\exists j_0\in \mathbb{N} ,\forall k\geqslant j_0\text{且}k\in \mathbb{N} ,x\in A_k \right\} 
\end{align*}
并且我们有
\[\varlimsup_{k\rightarrow\infty}A_k\supset\varliminf_{k\rightarrow\infty}A_k.\]
\end{theorem}
\begin{proof}
  以(ii)为例。若\(x\in\varliminf_{k\rightarrow\infty}A_k\),则存在自然数\(j_0\),使得
\[x\in\bigcap_{k = j_0}^{\infty}A_k,\]
从而当\(k\geqslant j_0\)时,有\(x\in A_k\)。反之,若存在自然数\(j_0\),当\(k\geqslant j_0\)时,有\(x\in A_k\),则得到
\[x\in\bigcap_{k = j_0}^{\infty}A_k.\]
由此可知\(x\in\bigcup_{j = 1}^{\infty}\bigcap_{k = j}^{\infty}A_k=\varliminf_{k\rightarrow\infty}A_k\)。

由$\left( \mathrm{i} \right) \left( \mathrm{ii} \right) $可知,\(\{A_k\}\)的上限集是由属于\(\{A_k\}\)中无穷多个集合的元素所形成的;\(\{A_k\}\)的下限集是由只不属于\(\{A_k\}\)中有限多个集合的元素所形成的。从而立即可知
\[\varlimsup_{k\rightarrow\infty}A_k\supset\varliminf_{k\rightarrow\infty}A_k.\]
\end{proof}

\begin{definition}[直积集]\label{definition:直积集}
设\(X,Y\)是两个集合,称一切有序“元素对”\((x,y)\)(其中\(x\in X,y\in Y\))形成的集合为\(X\)与\(Y\)的\textbf{直积集},记为\(X\times Y\),即
\[X\times Y = \{(x,y):x\in X,y\in Y\},\]
其中\((x,y)=(x',y')\)是指\(x = x',y = y'\)。\(X\times X\)也记为\(X^2\)。
\end{definition}




\end{document}
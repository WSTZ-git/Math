\documentclass[../../main.tex]{subfiles}
\graphicspath{{\subfix{../../image/}}} % 指定图片目录,后续可以直接使用图片文件名。

% 例如:
% \begin{figure}[H]
% \centering
% \includegraphics{image-01.01}
% \caption{图片标题}
% \label{figure:image-01.01}
% \end{figure}
% 注意:上述\label{}一定要放在\caption{}之后,否则引用图片序号会只会显示??.

\begin{document}

\section{点集间的距离}

\begin{definition}
设 \(x\in\mathbb{R}^n\),\(E\) 是 \(\mathbb{R}^n\) 中的非空点集,称
\begin{align*}
d(x,E)=\inf\{|x - y|: y\in E\}
\end{align*}
为点 \(x\) 到 \(E\) 的\textbf{距离};若 \(E_1\),\(E_2\) 是 \(\mathbb{R}^n\) 中的非空点集,称
\begin{align*}
d(E_1,E_2)=\inf\{|x - y|: x\in E_1, y\in E_2\}
\end{align*}
为 \(E_1\) 与 \(E_2\) 之间的\textbf{距离}。也可等价地定义为
\[
\inf\{d(x,E_2): x\in E_1\} \quad \text{或} \quad \inf\{d(E_1,y): y\in E_2\}.
\] 
\end{definition}

\begin{proposition}[点集间的距离的性质]\label{proposition:点集间的距离的性质}
\begin{enumerate}[(1)]
\item 设$E_1,E_2,\cdots,E_n,F$是$\mathbb{R}^n$中$n+1$个非空点集,则
\begin{align*}
d\left( F,\bigcup_{i=1}^n{E_i} \right) =\underset{i=1,2,\cdots ,n}{\min}\left\{ d\left( F,E_i \right) \right\}.
\end{align*}

\item 设$E_1,E_2,\cdots,E_n$是$\mathbb{R}^n$中$n$个非空点集,若$d(E_i,E_j)>0(i\ne j)$,则$E_i\cap E_j=\varnothing (i\ne j)$.

\item 设$E_1,E_2,\cdots,E_n$是$\mathbb{R}^n$中$n$个非空闭集,若$E_i\cap E_j=\varnothing (i\ne j)$,则$d(E_i,E_j)>0(i\ne j)$.
\end{enumerate}
\end{proposition}
\begin{remark}
若(3)中去掉$E_i$是闭集这个条件,则结论不一定成立(例如,两个开球相切).
\end{remark}
\begin{proof}
\begin{enumerate}[(1)]
\item 由$\bigcup_{i=1}^nE_i\supset E_i$($i=1,2,\cdots,n$)可知
\begin{align*}
\left\{(x,y)|x\in F,y\in\bigcup_{i=1}^nE_i\right\}\supset\left\{(x,y)|x\in F,y\in E_i\right\}\quad(i=1,2,\cdots,n).
\end{align*}
因此
\begin{align*}
d\left(F,\bigcup_{i=1}^nE_i\right)=\inf\left\{d(x,y)|x\in F,y\in\bigcup_{i=1}^nE_i\right\}\geqslant\inf\left\{d(x,y)|x\in F,y\in E_i\right\}=d(F,E_i)\quad(i=1,2,\cdots,n).
\end{align*}
故
\begin{align*}
d\left(F,\bigcup_{i=1}^nE_i\right)\geqslant\min_{i=1,2,\cdots,n}\left\{d(F,E_i)\right\}.
\end{align*}
对$\forall x\in F,y\in\bigcup_{i=1}^nE_i$,都存在$j\in\{1,2,\cdots,n\}$,使得$y\in E_j$。于是
\begin{align*}
d(x,y)\geqslant d(F,E_j)\geqslant\min_{i=1,2,\cdots,n}\left\{d(F,E_i)\right\}.
\end{align*}
故$\min_{i=1,2,\cdots,n}\left\{d(F,E_i)\right\}$是$\left\{d(x,y)|x\in F,y\in\bigcup_{i=1}^nE_i\right\}$的一个下界。因此
\begin{align*}
d\left(F,\bigcup_{i=1}^nE_i\right)=\inf\left\{d(x,y)|x\in F,y\in\bigcup_{i=1}^nE_i\right\}\geqslant\min_{i=1,2,\cdots,n}\left\{d(F,E_i)\right\}.
\end{align*}
综上,$d\left(F,\bigcup_{i=1}^nE_i\right)=\min_{i=1,2,\cdots,n}\left\{d(F,E_i)\right\}$。

\item 反证,假设存在$i\ne j$,使得$E_i\cap E_j\ne \varnothing,$则任取$x_0\in E_i\cap E_j$,又由$d(E_i,E_j)>0$可知,对$\forall x\in E_i,y\in E_j$,都有$d(x,y)\geq d(E_i,E_j)>0$.这与$d(x_0,x_0)=0,x_0\in E_i\cap E_j$矛盾!

\item 反证,假设存在$i\ne j$,使得$d(E_i,E_j)=0$.由$d(E_i,E_j) = \inf\{d(x,y) \mid x \in E_i, y \in E_j\}$及下确界的定义可知,对$\forall n \in \mathbb{N}$,存在$x_n \in E_i$,$y_n \in E_j$,使得$d(x_n,y_n) < \frac{1}{n}$。从而$\lim_{n \to \infty} d(x_n,y_n) = 0$,因此
\begin{align*}
\lim_{n \to \infty} x_n = \lim_{n \to \infty} y_n = c.
\end{align*}
再由$E_i$,$E_j$都是闭集可知$c \in E_i \cap E_j$,这与$E_i \cap E_j = \varnothing$矛盾!
\end{enumerate}
\end{proof}

\begin{example}
在 \(\mathbb{R}^2\) 中作点集
\begin{align*}
E_1 &= \{x = (\xi,\eta): -\infty < \xi < +\infty, \eta = 0\},\\
E_2 &= \{y = (\xi,\eta): \xi\cdot\eta = 1\},
\end{align*}
则 \(d(E_1,E_2) = 0\)。
\end{example}
\begin{proof}
事实上,当我们取 \(x = (\xi,0)\in E_1\) 且 \(y = (\xi,\eta)\in E_2\) 时,由
\begin{align*}
d(E_1,E_2) \leqslant d(x,y) = |\eta| = \frac{1}{|\xi|}
\end{align*}
可知,对任给的 \(\varepsilon > 0\),只需 \(|\xi|\) 充分大,就有 \(d(E_1,E_2) < \varepsilon\)。由此得
\begin{align*}
d(E_1,E_2) = 0.
\end{align*}

显然,若 \(x\in E\),则 \(d(x,E) = 0\)。但反之,若 \(d(x,E) = 0\),则 \(x\) 不一定属于 \(E\)。不过在 \(x\notin E\) 时,必有 \(x\in E'\)。 
\end{proof}

\begin{theorem}
若 \(F\subset\mathbb{R}^n\) 是非空闭集,且 \(x_0\in\mathbb{R}^n\),则存在 \(y_0\in F\),有
\[
|x_0 - y_0| = d(x_0,F).
\]
\end{theorem}
\begin{proof}
作闭球 \(\overline{B}=\overline{B}(x_0,\delta)\),使得 \(\overline{B}\cap F\) 不是空集。显然
\[
d(x_0,F)=d(x_0,\overline{B}\cap F).
\]
\(\overline{B}\cap F\) 是有界闭集,而 \(|x_0 - y|\) 看作定义在 \(\overline{B}\cap F\) 上的 \(y\) 的函数是连续的,故它在 \(\overline{B}\cap F\) 上达到最小值,即存在 \(y_0\in\overline{B}\cap F\),使得
\[
|x_0 - y_0| = \inf\{|x_0 - y|: y\in\overline{B}\cap F\},
\]
从而有 \(|x_0 - y_0| = d(x_0,F)\)。
\end{proof}

\begin{theorem}
若 \(E\) 是 \(\mathbb{R}^n\) 中非空点集,则 \(d(x,E)\) 作为 \(x\) 的函数在 \(\mathbb{R}^n\) 上是一致连续的。
\end{theorem}
\begin{proof}
考虑 \(\mathbb{R}^n\) 中的两点 \(x,y\)。根据 \(d(y,E)\) 的定义,对任给的 \(\varepsilon>0\),必存在 \(z\in E\),使得 \(|y - z|<d(y,E)+\varepsilon\),从而有
\begin{align*}
d(x,E)&\leqslant|x - z|\leqslant|x - y|+|y - z|\\
&<|x - y|+d(y,E)+\varepsilon.
\end{align*}
由 \(\varepsilon\) 的任意性可知
\[
d(x,E)-d(y,E)\leqslant|x - y|.
\]
同理可证 \(d(y,E)-d(x,E)\leqslant|x - y|\)。这说明
\[
|d(x,E)-d(y,E)|\leqslant|x - y|.
\]
\end{proof}

\begin{corollary}
若 \(F_1,F_2\) 是 \(\mathbb{R}^n\) 中的两个非空闭集且其中至少有一个是有界的,则存在 \(x_1\in F_1\),\(x_2\in F_2\),使得
\[
|x_1 - x_2| = d(F_1,F_2).
\] 
\end{corollary}

\begin{lemma}
若 \(F_1,F_2\) 是 \(\mathbb{R}^n\) 中两个互不相交的非空闭集,则存在 \(\mathbb{R}^n\) 上的连续函数 \(f(x)\),使得

(i) \(0\leqslant f(x)\leqslant 1\) \((x\in\mathbb{R}^n)\);

(ii) \(F_1 = \{x: f(x) = 1\}\),\(F_2 = \{x: f(x) = 0\}\)。
\end{lemma}
\begin{proof}
构造函数 \(f(x)\):
\begin{align*}
f(x)=\frac{d(x,F_2)}{d(x,F_1)+d(x,F_2)}, \quad x\in\mathbb{R}^n,
\end{align*}
它就是所求的函数。 
\end{proof}

\begin{theorem}[连续延拓定理]\label{theorem:连续函数延拓定理}
(1)若 \(F\) 是 \(\mathbb{R}^n\) 中的闭集,\(f(x)\) 是定义在 \(F\) 上的连续函数,且 \(|f(x)|\leqslant M\) \((x\in F)\),则存在 \(\mathbb{R}^n\) 上的连续函数 \(g(x)\) 满足
\[
|g(x)|\leqslant M, \quad g(x)=f(x), \quad x\in F.
\]

(2)若 \(F\) 是 \(\mathbb{R}^n\) 中的闭集,\(f(x)\) 是定义在 \(F\) 上的连续函数,则存在 \(\mathbb{R}^n\) 上的连续函数 \(g(x)\) 满足
\[
g(x)=f(x), \quad x\in F.
\]
\end{theorem}
\begin{remark}
\(\mathbb{R}^2\) 中存在由某些有理点构成的稠密集,其中任意两点的距离为无理数。 
\end{remark}
\begin{proof}
(1) 把 \(F\) 分成三个点集:
\begin{align*}
A&=\left\{x\in F:\frac{M}{3}\leqslant f(x)\leqslant M\right\},\\
B&=\left\{x\in F:-M\leqslant f(x)\leqslant\frac{-M}{3}\right\},\\
C&=\left\{x\in F:\frac{-M}{3}<f(x)<\frac{M}{3}\right\},
\end{align*}
并作函数
\[
g_1(x)=\frac{M}{3}\cdot\frac{d(x,B)-d(x,A)}{d(x,B)+d(x,A)}, \quad x\in\mathbb{R}^n.
\]
因为 \(A\) 与 \(B\) 是互不相交的闭集,所以 \(g_1(x)\) 处处有定义且在 \(\mathbb{R}^n\) 上处处连续。此外,还有
\[
|g_1(x)|\leqslant\frac{M}{3}, \quad x\in\mathbb{R}^n,
\]
\[
|f(x)-g_1(x)|\leqslant\frac{2}{3}M, \quad x\in F.
\]
再在 \(F\) 上来考查 \(f(x)-g_1(x)\)(相当于上述之 \(f(x)\)),并用类似的方法作 \(\mathbb{R}^n\) 上的连续函数 \(g_2(x)\)。此时由于 \(f(x)-g_1(x)\) 的界是 \(2M/3\),故 \(g_2(x)\) 应满足
\[
|g_2(x)|\leqslant\frac{1}{3}\cdot\frac{2M}{3}, \quad x\in\mathbb{R}^n,
\]
\[
|(f(x)-g_1(x))-g_2(x)|\leqslant\frac{2}{3}\cdot\frac{2M}{3}=\left(\frac{2}{3}\right)^2M, \quad x\in F.
\]

继续这一过程,可得在 \(\mathbb{R}^n\) 上的连续函数列 \(\{g_k(x)\}\),使得
\[
|g_k(x)|\leqslant\frac{1}{3}\cdot\left(\frac{2}{3}\right)^{k - 1}M, \quad x\in\mathbb{R}^n \quad (k = 1,2,\cdots),
\]
\[
\left|f(x)-\sum_{i = 1}^{k}g_i(x)\right|\leqslant\left(\frac{2}{3}\right)^kM, \quad x\in F \quad (k = 1,2,\cdots).
\]
上面的第一式表明 \(\sum_{k = 1}^{\infty}g_k(x)\) 是一致收敛的。若记其和函数为 \(g(x)\),则 \(g(x)\) 是 \(\mathbb{R}^n\) 上的连续函数。上面的第二式表明
\[
g(x)=\sum_{k = 1}^{\infty}g_k(x)=f(x), \quad x\in F.
\]
最后,对于任意的 \(x\in\mathbb{R}^n\),得到
\begin{align*}
|g(x)|&\leqslant\sum_{k = 1}^{\infty}|g_k(x)|\leqslant\frac{M}{3}\left(1+\frac{2}{3}+\left(\frac{2}{3}\right)^2+\cdots\right)\\
&\leqslant\frac{M}{3}\cdot\frac{1}{1 - \frac{2}{3}}=M.
\end{align*}

(2)令\(F(x)=\arctan f(x)\),则$|F(x)|\leqslant \frac{\pi}{2}(x\in F)$.于是由(1)可知,存在$\mathbb{R}^n$上的连续函数$G(x)$满足
\begin{align*}
|G(x)|\leqslant \frac{\pi}{2},\quad G(x)=F(x),\quad x\in F.
\end{align*}
取$g(x)=\tan G(x)$,则
\begin{align*}
f(x)=\tan F(x)=\tan G(x)=g(x),\quad x\in F.
\end{align*}
故结论得证.
\end{proof}



























\end{document}
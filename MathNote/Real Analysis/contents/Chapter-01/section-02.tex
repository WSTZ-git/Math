\documentclass[../../main.tex]{subfiles}
\graphicspath{{\subfix{../../image/}}} % 指定图片目录,后续可以直接使用图片文件名。

% 例如:
% \begin{figure}[H]
% \centering
% \includegraphics[scale=0.3]{image-01.01}
% \caption{图片标题}
% \label{figure:image-01.01}
% \end{figure}
% 注意:上述\label{}一定要放在\caption{}之后,否则引用图片序号会只会显示??.

\begin{document}

\section{映射与基数}

\begin{definition}[映射的像集]\label{definition:映射的像集}
  对于\(f:X\rightarrow Y\)以及\(A\subset X\),我们记
\[f(A)=\{y\in Y:x\in A,y = f(x)\},\]
并称\(f(A)\)为集合\(A\)在映射\(f\)下的\textbf{(映)像集}(\(f(\varnothing)=\varnothing\)).
\end{definition}

\begin{proposition}[映射的像集的基本性质]\label{proposition:映射的像集的基本性质}
  对于\(f:X\rightarrow Y\),我们有

  (i) \(f\left(\bigcup_{\alpha\in I}A_{\alpha}\right)=\bigcup_{\alpha\in I}f(A_{\alpha})\left( A_{\alpha}\in X,\alpha \in I \right) \);

(ii) \(f\left(\bigcap_{\alpha\in I}A_{\alpha}\right)\subset\bigcap_{\alpha\in I}f(A_{\alpha})\left( A_{\alpha}\in X,\alpha \in I \right) \)。
\end{proposition}

\begin{definition}[映射的原像集]\label{definition:映射的原像集}
  对于\(f:X\rightarrow Y\)以及\(B\subset Y\),我们记
\[f^{-1}(B)=\{x\in X:f(x)\in B\},\]
并称\(f^{-1}(B)\)为\(B\)关于\(f\)的\textbf{原像集}.
\end{definition}

\begin{proposition}[映射的原像集的基本性质]\label{proposition:映射的原像集的基本性质}
  对于\(f:X\rightarrow Y\),我们有

  (i) 若\(B_1\subset B_2\),则\(f^{-1}(B_1)\subset f^{-1}(B_2)\)(\(A\subset Y\));

(ii) \(f^{-1}\left(\bigcup_{\alpha\in I}B_{\alpha}\right)=\bigcup_{\alpha\in I}f^{-1}(B_{\alpha})\)(\(B_{\alpha}\subset Y,\alpha\in I\));

(iii) \(f^{-1}\left(\bigcap_{\alpha\in I}B_{\alpha}\right)=\bigcap_{\alpha\in I}f^{-1}(B_{\alpha})\)(\(B_{\alpha}\subset Y,\alpha\in I\));

(iv) \(f^{-1}(B^c)=(f^{-1}(B))^c\)(\(B\subset Y\))。
\end{proposition}

\begin{definition}[示性函数]\label{definition:示性函数}
  一般地,对于\(X\)中的子集\(A\),我们作
\[
\chi_A(x)=
\begin{cases}
1, & x\in A, \\
0, & x\in X\setminus A,
\end{cases}
\]
且称\(\chi_A:X\rightarrow\mathbb{R}\)是定义在\(X\)上的\(A\)的\textbf{特征函数}或\textbf{示性函数}.
\end{definition}

\begin{proposition}[示性函数的基本性质]\label{proposition:示性函数的基本性质}
  对于\(X\)中的子集\(A,B\),我们有
\begin{enumerate}[(i)]
  \item \(A\neq B\)等价于\(\chi_A\neq\chi_B\).

  \item \(A\subset B\)等价于\(\chi_A(x)\leqslant\chi_B(x)\).

  \item \(\chi_{A\cup B}(x)=\chi_A(x)+\chi_B(x)-\chi_{A\cap B}(x)\).

  \item \(\chi_{A\cap B}(x)=\chi_A(x)\cdot\chi_B(x)\).

  \item \(\chi_{A\setminus B}(x)=\chi_A(x)(1 - \chi_B(x))\).
  
  \item \(\chi_{A\triangle B}(x)=\vert\chi_A(x)-\chi_B(x)\vert\).
\end{enumerate}
\end{proposition}

\begin{definition}[幂集]\label{definition:幂集}
  设\(X\)是一个非空集合,由\(X\)的一切子集(包括\(\varnothing\),\(X\)自身)为元素形成的集合称为\(X\)的\textbf{幂集},记为\(\mathcal{P}(X)\)。
\end{definition}
\begin{note}
  例如,由\(n\)个元素形成的集合\(E\)之幂集\(\mathcal{P}(E)\)共有\(2^n\)个元素。
\end{note}

\begin{example}[单调映射的不动点]\label{example:单调映射的不动点}
设\(X\)是一个非空集合,且有\(f:\mathcal{P}(X)\to\mathcal{P}(X)\)。若对\(\mathcal{P}(X)\)中满足\(A\subset B\)的任意\(A,B\),必有\(f(A)\subset f(B)\),则存在\(T\subset\mathcal{P}(X)\),使得\(f(T)=T\)。
\end{example}
\begin{proof}
作集合\(S,T\):
\begin{align*}
S&=\{A:A\in\mathcal{P}(X)\text{ 且 }A\subset f(A)\},\\
T&=\bigcup_{A\in S}A(\in\mathcal{P}(X)),
\end{align*}
则有\(f(T)=T\).

事实上,因为由\(A\in S\)可知\(A\subset f(A)\),从而由\(A\subset T\)可得\(f(A)\subset f(T)\)。根据\(A\in S\)推出\(A\subset f(T)\),这就导致
\[
\bigcup_{A\in S}A\subset f(T),\quad T\subset f(T).
\]

另一方面,又从\(T\subset f(T)\)可知\(f(T)\subset f(f(T))\)。这说明\(f(T)\in S\),我们又有\(f(T)\subset T\)。
\end{proof}

\begin{definition}[集合之间的对等关系]\label{definition:集合之间的对等关系}
  设有集合\(A\)与\(B\)。若存在一个从\(A\)到\(B\)上的一一映射,则称集合\(A\)与\(B\)\textbf{对等},记为\(A\sim B\)。
\end{definition}

\begin{proposition}[对等关系的基本性质]\label{proposition:对等关系的基本性质}
设有集合\(A\)与\(B\),则

  (i) \(A\sim A\);

(ii) 若\(A\sim B\),则\(B\sim A\);

(iii) 若\(A\sim B\),\(B\sim C\),则\(A\sim C\)。
\end{proposition}

\begin{lemma}[映射分解定理]\label{lemma:映射分解定理}
若有\(f:X\rightarrow Y\),\(g:Y\rightarrow X\),则存在分解
\[X = A\cup A^{\sim}, \quad Y = B\cup B^{\sim},\]
其中\(f(A)=B\),\(g(B^{\sim})=A^{\sim}\),\(A\cap A^{\sim}=\varnothing\)以及\(B\cap B^{\sim}=\varnothing\)。
\end{lemma}
\begin{proof}
  对于\(X\)中的子集\(E\)(不妨假定\(Y\setminus f(E)\neq\varnothing\)),若满足
\[E\cap g(Y\setminus f(E)) = \varnothing,\]
则称\(E\)为\(X\)中的分离集。现将\(X\)中的分离集的全体记为\(\Gamma\),且作其并集
\[A = \bigcup_{E\in\Gamma}E.\]
我们有\(A\in\Gamma\)。事实上,对于任意的\(E\in\Gamma\),由于\(A\supset E\),故从
\[E\cap g(Y\setminus f(E)) = \varnothing\]
可知\(E\cap g(Y\setminus f(A)) = \varnothing\),从而有\(A\cap g(Y\setminus f(A)) = \varnothing\)。这说明\(A\)是\(X\)中的分离集且是\(\Gamma\)中最大元。

现在令\(f(A)=B\),\(Y\setminus B = B^{\sim}\)以及\(g(B^{\sim}) = A^{\sim}\)。首先知道
\[Y = B\cup B^{\sim}.\]
其次,由于\(A\cap A^{\sim} = \varnothing\),故又易得\(A\cup A^{\sim} = X\)。事实上,若不然,那么存在\(x_0\in X\),使得\(x_0\notin A\cup A^{\sim}\)。现在作\(A_0 = A\cup\{x_0\}\),我们有
\[B = f(A)\subset f(A_0), \quad B^{\sim}\supset Y\setminus f(A_0),\]
从而知\(A^{\sim}\supset g(Y\setminus f(A_0))\)。这就是说,\(A\)与\(g(Y\setminus f(A_0))\)不相交。由此可得
\[A_0\cap g(Y\setminus f(A_0)) = \varnothing.\]
这与\(A\)是\(\Gamma\)的最大元相矛盾。
\end{proof}

\begin{theorem}[Cantor - Bernstein定理]\label{theorem:Cantor - Bernstein定理}
若集合\(X\)与\(Y\)的某个真子集对等,\(Y\)与\(X\)的某个真子集对等,则\(X\sim Y\)。
\end{theorem}
\begin{note}
  特例:
设集合\(A,B,C\)满足下述关系:
\[C\subset A\subset B.\]
若\(B\sim C\),则\(B\sim A\)。
\end{note}
\begin{proof}
  由题设知存在单射\(f:X\rightarrow Y\)与单射\(g:Y\rightarrow X\),根据\hyperref[lemma:映射分解定理]{映射分解定理}知
\[X = A\cup A^{\sim}, \quad Y = B\cup B^{\sim}, \quad f(A) = B, \quad g(B^{\sim}) = A^{\sim}.\]
注意到这里的\(f:A\rightarrow B\)以及\(g^{-1}:A^{\sim}\rightarrow B^{\sim}\)是一一映射,因而可作\(X\)到\(Y\)上的一一映射\(F\):
\[F(x)=
\begin{cases}
f(x), & x\in A, \\
g^{-1}(x), & x\in A^{\sim}.
\end{cases}
\]
这说明\(X\sim Y\)。
\end{proof}

\begin{definition}[集合的基数(或势)]\label{definition:集合的基数(或势)}
  设\(A,B\)是两个集合,如果\(A\sim B\),那么我们就说\(A\)与\(B\)的\textbf{基数}(cardinal number)或\textbf{势}是相同的,记为\(\overline{\overline{A}}=\overline{\overline{B}}\)。
  可见,凡是互相对等的集合均具有相同的基数。
  
  如果用\(\alpha\)表示这一相同的基数,那么\(\overline{\overline{A}}=\alpha\)就表示\(A\)属于这一对等集合族。对于两个集合\(A\)与\(B\),记\(\overline{\overline{A}}=\alpha\),\(\overline{\overline{B}}=\beta\)。若\(A\)与\(B\)的一个子集对等,则称\(\alpha\)不大于\(\beta\),记为
\[\alpha\leqslant\beta.\]
若\(\alpha\leqslant\beta\)且\(\alpha\neq\beta\),则称\(\alpha\)小于\(\beta\)(或\(\beta\)大于\(\alpha\)),记为
\[\alpha<\beta\quad(\text{或}\beta>\alpha).\]
显然,若\(\alpha\leqslant\beta\)且\(\beta\leqslant\alpha\),则由\hyperref[theorem:Cantor - Bernstein定理]{Cantor - Bernstein定理}可知\(\alpha=\beta\).
\end{definition}

\begin{definition}[有限集与无限集]\label{definition:有限集与无限集}
设\(A\)是一个集合。如果存在自然数\(n\),使得\(A\sim\{1,2,\cdots,n\}\),则称\(A\)为\textbf{有限集},且用同一符号\(n\)记\(A\)的基数。由此可见,对于有限集来说,其基数可以看作集合中元素的数目。若一个集合不是有限集,则称为\textbf{无限集}.下面我们着重介绍无限集中若干重要且常见的基数。
\end{definition}

\begin{definition}[自然数集\(\mathbb{N}\)的基数·可列集]\label{definition:自然数集的基数·可列集}
记自然数集\(\mathbb{N}\)的基数为\(\aleph_0\)(读作阿列夫(Aleph,希伯来文)零)。若集合\(A\)的基数为\(\aleph_0\),则\(A\)叫作\textbf{可列集}.这是由于\(\mathbb{N}=\{1,2,\cdots,n,\cdots\}\),而\(A\sim\mathbb{N}\),故可将\(A\)中元素按一一对应关系以自然数次序排列起来,附以下标,就有
\[A = \{a_1,a_2,\cdots,a_n,\cdots\}.\]
\end{definition}

\begin{theorem}\label{theorem:任一无限集必包含一个可列子集}
任一无限集\(E\)必包含一个可列子集。
\end{theorem}
\begin{note}
  这个定理说明,在众多的无限集中,最小的基数是\(\aleph_0\)。
\end{note}
\begin{proof}
  任取\(E\)中一元,记为\(a_1\);再从\(E\setminus\{a_1\}\)中取一元,记为\(a_2\),\(\cdots\)。设已选出\(a_1,a_2,\cdots,a_n\)。因为\(E\)是无限集,所以
\[E\setminus\{a_1,a_2,\cdots,a_n\} \neq \varnothing.\]
于是又从\(E\setminus\{a_1,a_2,\cdots,a_n\}\)中可再选一元,记为\(a_{n + 1}\)。这样,我们就得到一个集合
\[\{a_1,a_2,\cdots,a_n,a_{n + 1},\cdots\}.\]
这是一个可列集且是\(E\)的子集。
\end{proof}

\begin{theorem}\label{theorem:无限集并上可列集基数不变}
设\(A\)是无限集且其基数为\(\alpha\)。若\(B\)是至多可列集,则\(A\cup B\)的基数仍为\(\alpha\)。
\end{theorem}
\begin{proof}
  不妨设\(B = \{b_1,b_2,\cdots\}\),\(A\cap B=\varnothing\),且
\[A = A_1\cup A_2, \quad A_1 = \{a_1,a_2,\cdots\}.\]
我们作映射\(f\)如下:
\begin{align*}
f(a_i)&=a_{2i}, \quad a_i\in A_1;\\
f(b_i)&=a_{2i - 1}, \quad b_i\in B;\\
f(x)&=x, \quad x\in A_2.
\end{align*}
显然,\(f\)是\(A\cup B\)到\(A\)上的一一映射。
\end{proof}

\begin{theorem}\label{theorem:无限集的充要条件}
集合\(A\)为无限集的充要条件是\(A\)与其某真子集对等。
\end{theorem}
\begin{proof}
因为有限集是不与其真子集对等的,所以充分性是成立的。现在取\(A\)中一个非空有限子集\(B\),则由\hyperref[theorem:无限集并上可列集基数不变]{定理\ref{theorem:无限集并上可列集基数不变}}立即可知
\begin{align*}
  \overline{\overline{A}}=\overline{\overline{\left( \left( A\setminus B \right) \cup B \right) }}=\overline{\overline{\left( A\setminus B \right) }}.
\end{align*}
故$A\sim \left( A\setminus B \right) $.
\end{proof}

\begin{theorem}
  \([0,1]=\{x: 0\leqslant x\leqslant1\}\)不是可数集.
\end{theorem}
\begin{proof}
  只需讨论\((0,1]\)。为此,采用二进位制小数表示法:
\[x = \sum_{n = 1}^{\infty}\frac{a_n}{2^n},\]
其中\(a_n\)等于\(0\)或\(1\),且在表示式中有无穷多个\(a_n\)等于\(1\)。显然,\((0,1]\)与全体二进位制小数一一对应。

若在上述表示式中把\(a_n = 0\)的项舍去,则得到\(x=\sum_{i = 1}^{\infty}2^{-n_i}\),这里的\(\{n_i\}\)是严格上升的自然数数列。再令
\[k_1 = n_1, \quad k_i = n_i - n_{i - 1}, \quad i = 2,3,\cdots,\]
则\(\{k_i\}\)是自然数子列。把由自然数构成的数列的全体记为\(\mathscr{H}\),则\((0,1]\)与\(\mathscr{H}\)一一对应。

现在假定\((0,1]\)是可数的,则\(\mathscr{H}\)是可数的,不妨将其全体排列如下:
\begin{align*}
&(k_1^{(1)},k_2^{(1)},\cdots,k_i^{(1)},\cdots),\\
&(k_1^{(2)},k_2^{(2)},\cdots,k_i^{(2)},\cdots),\\
&\cdots\cdots\cdots\cdots\cdots\cdots\cdots\cdots\\
&(k_1^{(i)},k_2^{(i)},\cdots,k_i^{(i)},\cdots),\\
&\cdots\cdots\cdots\cdots\cdots\cdots\cdots\cdots
\end{align*}
但这是不可能的,因为\((k_1^{(1)} + 1,k_2^{(2)} + 1,\cdots,k_i^{(i)} + 1,\cdots)\)属于\(\mathscr{H}\),而它并没有被排列出来。这说明\(\mathscr{H}\)是不可数的,也就是说\((0,1]\)是不可数集。
\end{proof}

\begin{definition}[$\mathbb{R}$的基数$\cdot$不可数集]\label{definition:R的基数与不可数集}
  我们称\((0,1]\)的基数为\textbf{连续基数},记为\(c\)(或\(\aleph_1\)).
\end{definition}
\begin{note}
  易知\(\overline{\overline{\mathbb{R}}}=c=\aleph_1\).
\end{note}

\begin{theorem}
  设有集合列\(\{A_k\}\)。若每个\(A_k\)的基数都是连续基数,则其并集\(\bigcup_{k = 1}^{\infty}A_k\)的基数是连续基数。
\end{theorem}
\begin{proof}
  不妨假定\(A_i\cap A_j=\varnothing(i\neq j)\),且\(A_k\sim[k,k + 1)\),我们有
\[\bigcup_{k = 1}^{\infty}A_k\sim[1,+\infty)\sim\mathbb{R}.\]
\end{proof}

\begin{theorem}[无最大基数定理]\label{theorem:无最大基数定理}
若\(A\)是非空集合,则\(A\)与其幂集\(\mathcal{P}(A)\)(由\(A\)的一切子集所构成的集合族)不对等。
\end{theorem}
\begin{note}
  易知集合\(A\)的基数小于其幂集\(\mathcal{P}(A)\)的基数.
\end{note}
\begin{proof}
  假定\(A\)与其幂集\(\mathcal{P}(A)\)对等,即存在一一映射\(f:A\rightarrow\mathcal{P}(A)\)。我们作集合
\[B = \{x\in A:x\notin f(x)\},\]
于是有\(y\in A\),使得\(f(y)=B\in\mathcal{P}(A)\)。现在分析一下\(y\)与\(B\)的关系:

(i) 若\(y\in B\),则由\(B\)的定义可知\(y\notin f(y)=B\);

(ii) 若\(y\notin B\),则由\(B\)的定义可知\(y\in f(y)=B\)。

这些矛盾说明\(A\)与\(\mathcal{P}(A)\)之间并不存在一一映射,即\(A\)与\(\mathcal{P}(A)\)并不是对等的。
\end{proof}




\end{document}
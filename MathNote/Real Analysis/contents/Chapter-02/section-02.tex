\documentclass[../../main.tex]{subfiles}
\graphicspath{{\subfix{../../image/}}} % 指定图片目录,后续可以直接使用图片文件名。

% 例如:
% \begin{figure}[H]
% \centering
% \includegraphics{image-01.01}
% \caption{图片标题}
% \label{figure:image-01.01}
% \end{figure}
% 注意:上述\label{}一定要放在\caption{}之后,否则引用图片序号会只会显示??.

\begin{document}

\section{可测集与测度}

\begin{definition}[可测集]
设 \(E \subset \mathbb{R}^n\). 若对任意的点集 \(T \subset \mathbb{R}^n\), 有
\begin{align*}
m^*(T)=m^*(T \cap E)+m^*(T \cap E^c), 
\end{align*}
则称 \(E\) 为 \textbf{Lebesgue 可测集}(或 \textbf{\(m^*\)-可测集})或$E$\textbf{可测}, 简称为\textbf{可测集}, 其中 \(T\) 称为\textbf{试验集}(这一定义可测集的等式也称为 \textbf{Carathéodory条件}). 可测集的全体称为\textbf{可测集类}, 简记为 \(\mathscr{M}\). 
\end{definition}

\begin{theorem}[集合可测的充要条件]\label{theorem:可测的充要条件}
设\(E\subset \mathbb{R}^n\), 则$E\in \mathscr{M}$的充要条件是对任一点集 \(T \subset \mathbb{R}^n\)且\(m^*(T)< + \infty\),都有
\begin{align}
m^*(T) \geq m^*(T \cap E)+m^*(T \cap E^c) \label{eq:2.3}
\end{align}
成立. 
\end{theorem}
\begin{remark}
往后经常利用这个定理的充分性来证明一个集合可测.但这个定理的必要性要弱于可测集的定义.
\end{remark}
\begin{proof}
必要性由可测集的定义立得.下证充分性.
由外测度的次可加性可得
\[
m^*(T)=m^*\left( T\cap \mathbb{R} ^n \right) =m^*\left( T\cap \left( E\cup E^c \right) \right) =m^*\left( \left( T\cap E \right) \cap \left( T\cup E^c \right) \right)  \leq m^*(T \cap E)+m^*(T \cap E^c)
\]
总是成立的. 又因为在 \(m^*(T)=\infty\) 时 \eqref{eq:2.3} 式总成立,故对任意的点集 \(T \subset \mathbb{R}^n\),都有
\begin{align*}
m^*(T)=m^*(T \cap E)+m^*(T \cap E^c), 
\end{align*}
即$E\in \mathscr{M}$.
\end{proof}

\begin{definition}[零测集]
外测度为零的点集称为\textbf{零测集}. 
\end{definition}
\begin{remark}
显然, \(\mathbb{R}^n\) 中由单个点组成的点集是零测集. 从而根据外测度的次可加性知道 \(\mathbb{R}^n\) 中的有理点集 \(\mathbb{Q}^n\) 是零测集.
\end{remark}

\begin{proposition}
\begin{enumerate}
\item 零测集的任一子集是零测集.

\item 零测集一定可测,即若 \(m^*(E)=0\), 则 \(E \in \mathscr{M}\). 
\end{enumerate}
\end{proposition}
\begin{proof}
\begin{enumerate}
\item 由外测度的单调性立得.

\item 事实上, 此时我们有
\begin{align*}
m^*(T \cap E)+m^*(T \cap E^c) \leq m^*(E)+m^*(T)=m^*(T).
\end{align*}
\end{enumerate}
\end{proof}

\begin{proposition}\label{proposition:外测度的可加性的条件}
若 \(E_1 \subset S\), \(E_2 \subset S^c\), \(S \in \mathscr{M}\), 则有
\begin{align*}
m^*(E_1 \cup E_2) = m^*(E_1) + m^*(E_2).
\end{align*}
\end{proposition}
\begin{remark}
这个命题表明:当两个集合由一个可测集分离开时, 其外测度就有可加性.
\end{remark}
\begin{proof}
事实上, 此时取试验集 \(T = E_1 \cup E_2\), 从 \(S\) 是可测集的定义得
\begin{align*}
m^*(E_1 \cup E_2) = m^*((E_1 \cup E_2) \cap S) + m^*((E_1 \cup E_2) \cap S^c) = m^*(E_1) + m^*(E_2).
\end{align*}
\end{proof}

\begin{corollary}
当 \(E_1\) 与 \(E_2\) 是互不相交的可测集时, 对任一集合 \(T\) 有
\begin{align*}
m^*(T \cap (E_1 \cup E_2)) = m^*(T \cap E_1) + m^*(T \cap E_2).
\end{align*} 
\end{corollary}
\begin{proof}
注意到$T\cap E_1\in E_1$,$T\cap E_2\in E_1^c$,而$E_1\in \mathscr{M}$,故由\refpro{proposition:外测度的可加性的条件}可知
\begin{align*}
m^*(T \cap (E_1 \cup E_2))=m^*((T\cap E_1)\cup (T\cap E_2)) = m^*(T \cap E_1) + m^*(T \cap E_2).
\end{align*} 
\end{proof}

\begin{theorem}[可测集的性质]\label{theorem:可测集的性质}
\begin{enumerate}[(1)]
\item \(\varnothing \in \mathscr{M}\).
\item 若 \(E \in \mathscr{M}\), 则 \(E^c \in \mathscr{M}\).
\item 若 \(E_1 \in \mathscr{M}\), \(E_2 \in \mathscr{M}\), 则 \(E_1 \cup E_2\), \(E_1 \cap E_2\) 以及 \(E_1 \setminus E_2\) 皆属于 \(\mathscr{M}\). (由此知, 可测集任何有限次取交、并运算后所得的集皆为可测集. )
\item 若 \(E_i \in \mathscr{M}\) (\(i = 1,2,\cdots\)), 则其并集也属于 \(\mathscr{M}\). 若进一步有 \(E_i \cap E_j = \varnothing\) (\(i \neq j\)), 则
\begin{align*}
m^*\left(\bigcup_{i = 1}^{\infty} E_i\right) = \sum_{i = 1}^{\infty} m^*(E_i),
\end{align*}
即 \(m^*\) 在 \(\mathscr{M}\) 上满足可数可加性(或称为 \(\sigma -\)可加性).
\end{enumerate}
\end{theorem}
\begin{proof}
\begin{enumerate}[(1)]
\item 显然成立.
\item 注意到 \((E^c)^c = E\), 从定义可立即得出结论.
\item 对于任一集 \(T \subset \mathbb{R}^n\), 根据集合分解(参阅\hyperref[figure:集合关系示意图1]{图\ref{figure:集合关系示意图1}})与外测度的次可加性, 我们有
\begin{align*}
m^*(T) &\leq m^*(T \cap (E_1 \cup E_2)) + m^*(T \cap (E_1 \cup E_2)^c) \\
&= m^*(T \cap (E_1 \cup E_2)) + m^*((T \cap E_1^c) \cap E_2^c) \\
&\leq m^*((T \cap E_1) \cap E_2) + m^*((T \cap E_1) \cap E_2^c) \\
&\quad + m^*((T \cap E_1^c) \cap E_2) + m^*((T \cap E_1^c) \cap E_2^c).
\end{align*}
又由 \(E_1\), \(E_2\) 的可测性知, 上式右端就是
\begin{align*}
m^*(T \cap E_1) + m^*(T \cap E_1^c) = m^*(T).
\end{align*}
这说明
\begin{align*}
m^*(T) = m^*(T \cap (E_1 \cup E_2)) + m^*(T \cap (E_1 \cup E_2)^c).
\end{align*}
也就是说 \(E_1 \cup E_2\) 是可测集.
\begin{figure}[H]
\centering
\includegraphics[scale=0.35]{集合关系示意图1}
\caption{}
\label{figure:集合关系示意图1}
\end{figure}
为证 \(E_1 \cap E_2\) 是可测集, 只需注意 \(E_1 \cap E_2 = (E_1^c \cup E_2^c)^c\) 即可. 又由 \(E_1 \setminus E_2 = E_1 \cap E_2^c\) 可知, \(E_1 \setminus E_2\) 是可测集. 
\item 首先, 设 \(E_1\), \(E_2\), \(\cdots\), \(E_i\), \(\cdots\) 皆互不相交, 并令
\[
S = \bigcup_{i = 1}^{\infty} E_i, \quad S_k = \bigcup_{i = 1}^k E_i, \quad k = 1,2,\cdots.
\]
由(3)知每个 \(S_k\) 都是可测集, 从而对任一集 \(T\), 我们有
\begin{align*}
m^*(T) &= m^*(T \cap S_k) + m^*(T \cap S_k^c) \\
&= m^*\left(\bigcup_{i = 1}^k (T \cap E_i)\right) + m^*(T \cap S_k^c) \\
&= \sum_{i = 1}^k m^*(T \cap E_i) + m^*(T \cap S_k^c).
\end{align*}
由于 \(T \cap S_k^c \supset T \cap S^c\), 可知
\begin{align*}
m^*(T) \geq \sum_{i = 1}^k m^*(T \cap E_i) + m^*(T \cap S^c).
\end{align*}
令 \(k \to \infty\), 就有
\begin{align*}
m^*(T) \geq \sum_{i = 1}^{\infty} m^*(T \cap E_i) + m^*(T \cap S^c).
\end{align*}
由此可得
\begin{align*}
m^*(T) \geq m^*(T \cap S) + m^*(T \cap S^c).
\end{align*}
这说明 \(S \in \mathscr{M}\).

此外, 在公式
\begin{align*}
m^*(T) \geq \sum_{i = 1}^{\infty} m^*(T \cap E_i) + m^*(T \cap S^c)
\end{align*}
中以 \(T \cap S\) 替换 \(T\), 则又可得
\begin{align*}
m^*(T \cap S) \geq \sum_{i = 1}^{\infty} m^*(T \cap E_i).
\end{align*}
但反向不等式总是成立的, 因而实际上有
\begin{align*}
m^*(T \cap S) = \sum_{i = 1}^{\infty} m^*(T \cap E_i).
\end{align*}
在这里再取 \(T\) 为全空间 \(\mathbb{R}^n\), 就可证明可数可加性质:
\begin{align*}
m^*(S) = m^*\left(\bigcup_{i = 1}^{\infty} E_i\right) = \sum_{i = 1}^{\infty} m^*(E_i).
\end{align*}

其次, 对于一般的可测集列 \(\{E_i\}\), 我们令
\[
S_1 = E_1, \quad S_k = E_k \setminus \left(\bigcup_{i = 1}^{k - 1} E_i\right), \quad k \geq 2,
\]
则 \(\{S_k\}\) 是互不相交的可测集列. 而由 \(\bigcup_{i = 1}^{\infty} E_i = \bigcup_{k = 1}^{\infty} S_k\) 可知, \(\bigcup_{i = 1}^{\infty} E_i\) 是可测集.
\end{enumerate} 
\end{proof}

\begin{corollary}\label{corollary:可测集类是sigma-代数}
$\mathscr{M}$是$\mathbb{R}^n$上的一个$\sigma$-代数.
\end{corollary}
\begin{proof}
由\hyperref[theorem:可测集的性质]{可测集的性质(1)(2)(4)}立得.
\end{proof}


\begin{proposition}\label{proposition:Cantor集可测且测度为0}
证明:Cantor集C是可测的,并且$m(C)=0$.
\end{proposition}
\begin{proof}
开区间是可测的. 由\hyperref[theorem:开集构造定理]{开集构造定理}, 我们知道 \(\mathbb{R}\) 中的开集是开区间的可数并, 因此也可测. 因此, 闭集也是可测的. 显然, 每个 \(C_n\) 都是闭集. 并且
\begin{align*}
C = \bigcap_{n = 1}^{\infty} C_n
\end{align*}
于是 \(C\) 也是闭集. 因此 \(C\) 是可测的.

下面, 我们用两种方法计算康托集的测度.

{\color{blue}法一:}根据我们的构造, \(C_{n + 1}\) 的测度刚好是去掉了 \(1/3\) 的 \(C_n\) 的测度. 换言之,
\begin{align*}
m(C_{n + 1}) = \left(1 - \frac{1}{3}\right)m(C_n) = \frac{2}{3}m(C_n)
\end{align*}
递归地, 对任意 \(n \in \mathbb{N}\), 我们有
\begin{align*}
m(C_n) = \left(\frac{2}{3}\right)^n m(C_0) = \left(\frac{2}{3}\right)^n
\end{align*}
注意到
\begin{align*}
m(C_0) = 1 < \infty
\end{align*}
因此由测度的第二单调收敛定理,
\begin{align*}
m(C) = m\left(\bigcap_{n = 1}^{\infty} C_n\right) = \lim_{n \to \infty} m(C_n) = \lim_{n \to \infty} \left(\frac{2}{3}\right)^n = 0
\end{align*}
此即得证.

{\color{blue}法二:}设 \(n \geq 2\). \(C_n\) 比 \(C_{n - 1}\) 减少了 \(2^{n - 1}\) 个区间, 每个区间长度为 \(\frac{1}{3^n}\). 因此 \(C_n\) 比 \(C_{n - 1}\) 减少的长度为
\begin{align*}
2^{n - 1}\frac{1}{3^n} = \frac{1}{3}\left(\frac{2}{3}\right)^{n - 1}
\end{align*}
总共减少的长度为
\begin{align*}
\sum_{n = 1}^{\infty} \frac{1}{3}\left(\frac{2}{3}\right)^{n - 1} = \frac{1}{3} \frac{1}{1 - \frac{2}{3}} = \frac{1}{3} \cdot 3 = 1
\end{align*}
因此
\begin{align*}
m(C) = 1 - 1 = 0.
\end{align*} 
\end{proof}

\begin{proposition}\label{proposition:可测集类的基数是2^c}
\(\mathscr{M}\) 的基数是 \(2^c\).
\end{proposition}
\begin{proof}
由\refpro{proposition:Cantor集可测且测度为0}可知Cantor集是零测集,不难推断 \(\mathscr{M}\) 的基数大于或等于 \(2^c\), 但 \(\mathscr{M}\) 的基数又不会超过 \(2^c\), 于是 \(\mathscr{M}\) 的基数实际上是 \(2^c\). 
\end{proof}

\begin{definition}[Lebesgue测度]
对于可测集 \(E\), 其外测度称为\textbf{测度}, 记为 \(m(E)\). 这就是通常所说的 \(\mathbb{R}^n\) 上的 \textbf{Lebesgue测度}.
\end{definition}

\begin{definition}[测度]
设 \(X\) 是非空集合, \(\mathscr{A}\) 是 \(X\) 的一些子集构成的 \(\sigma -\)代数. 若 \(\mu\) 是定义在 \(\mathscr{A}\) 上的一个集合函数, 且满足:
\begin{enumerate}[(i)]
\item \(0 \leq \mu(E) \leq +\infty\) (\(E \in \mathscr{A}\));
\item \(\mu(\varnothing)=0\);
\item \(\mu\) 在 \(\mathscr{A}\) 上是可数可加的,
\end{enumerate}
则称 \(\mu\) 是 \(\mathscr{A}\) 上的(非负)\textbf{测度}. \(\mathscr{A}\) 中的元素称为(\(\mu\))\textbf{可测集}, 有序组 \((X, \mathscr{A}, \mu)\) 称为\textbf{测度空间}. 
\end{definition}
\begin{remark}
由\refcor{corollary:可测集类是sigma-代数}可知$\mathscr{M}$就是$\mathbb{R}^n$上的一个$\sigma$-代数,故本节所建立的测度空间就是 \((\mathbb{R}^n, \mathscr{M}, m)\). 
\end{remark}








\end{document}
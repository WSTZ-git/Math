\documentclass[../../main.tex]{subfiles}
\graphicspath{{\subfix{../../image/}}} % 指定图片目录,后续可以直接使用图片文件名。

% 例如:
% \begin{figure}[h]
% \centering
% \includegraphics{image-01.01}
% \label{fig:image-01.01}
% \caption{图片标题}
% \end{figure}

\begin{document}

\section{Lebesgue可测集的$\sigma $代数}


\begin{definition}[可测]\label{definition:可测}
集合\(E\)称为在$\mathbb{R}$中是\textbf{可测的}或是$\mathbb{R}$中的一个\textbf{可测集},或称$E$满足卡拉西奥多里(Carathéodory)条件,若对任意集合\(A\),
\[m^*(A)=m^*(A\cap E)+m^*(A\cap E^C)=m^*(A\cap E)+m^*(A - E).\] 
\end{definition}

\begin{proposition}[可测的充要条件]\label{proposition:可测的充要条件1}
  设\(E\subset\mathbb{R}\),则\(E\)是可测集当且仅当对任意\(A\subset\mathbb{R}\)有
\[m^*(A)\geq m^*(A\cap E)+m^*(A - E).\]
\end{proposition}
\begin{proof}
  由\hyperref[definition:可测]{可测的定义}可知,我们只须证明小于等于号的关系恒成立.注意到\(A=(A\cap E)\cup(A - E)\),由于\hyperref[proposition:Lebesgue外测度的可数次可加性]{Lebesgue外测度的可数次可加性},我们有
  \[m^*(A)=m^*((A\cap E)\cup(A - E))\leq m^*(A\cap E)+m^*(A - E)\]
  此即得证. 
\end{proof}

\begin{proposition}[可测集的性质]\label{proposition:可测集的性质}
\begin{enumerate}[(1)]
  \item 空集与$\mathbb{R}$是可测的.
  
  \item 可测集的补是可测的.

  \item 任何外测度为零的集合是可测的。特别地,任何可数集是可测的。
 
  \item 可数个可测集的并是可测的.
\end{enumerate}
\end{proposition}
\begin{proof}
\begin{enumerate}[(1)]
  \item 由可测的定义易得.

  \item 由可测的定义易得.

  \item  令集合\(E\)的外测度为零.令\(A\)为任意集合。由于
  \[A\cap E\subseteq E\text{ 且 }A\cap E^C\subseteq A\]
  根据外测度的单调性,
  \[m^*(A\cap E)\leqslant m^*(E)=0\text{ 且 }m^*(A\cap E^C)\leqslant m^*(A)\]
  因此
  \begin{align*}
  m^*(A)&\geqslant m^*(A\cap E^C)=0 + m^*(A\cap E^C)\\
  &=m^*(A\cap E)+m^*(A\cap E^C)
  \end{align*}
  从而由\hyperref[proposition:可测的充要条件1]{可测的充要条件}可知,\(E\)是可测的。

  \item 
\end{enumerate}
\end{proof}




\end{document}
\documentclass[../../main.tex]{subfiles}
\graphicspath{{\subfix{../../image/}}} % 指定图片目录,后续可以直接使用图片文件名。

% 例如:
% \begin{figure}[H]
% \centering
% \includegraphics{image-01.01}
% \caption{图片标题}
% \label{figure:image-01.01}
% \end{figure}
% 注意:上述\label{}一定要放在\caption{}之后,否则引用图片序号会只会显示??.

\begin{document}

\section{点集的Lebesgue外测度}

\begin{definition}[Lebesgue外测度]
设 \(E \subset \mathbb{R}^n\). 若 \(\{I_k\}\) 是 \(\mathbb{R}^n\) 中的可数个开矩体, 且有
\begin{align*}
E \subset \bigcup_{k \geq 1} I_k,
\end{align*}
则称 \(\{I_k\}\) 为 \(E\) 的一个 \textbf{L-覆盖}(显然, 这样的覆盖有很多, 且每一个 \(L -\)覆盖 \(\{I_k\}\) 确定一个非负广义实值 \(\sum_{k \geq 1} |I_k|\) (可以是 \(+ \infty\), \(|I_k|\) 表示 \(I_k\) 的体积)). 称
\begin{align*}
m^*(E) = \inf \left\{ \sum_{k \geq 1} |I_k| : \{I_k\} \text{ 为 } E \text{ 的 } L-\text{覆盖} \right\}
\end{align*}
为点集 \(E\) 的 \textbf{Lebesgue 外测度}, 简称\textbf{外测度}.
\end{definition}
\begin{remark}
显然, 若 \(E\) 的任意的 \(L -\)覆盖 \(\{I_k\}\) 均有
\begin{align*}
\sum_{k \geq 1} |I_k| = + \infty,
\end{align*}
则 \(m^*(E) = + \infty\), 否则 \(m^*(E) < + \infty\). 
\end{remark}

\begin{theorem}[\(\mathbb{R}^n\) 中点集的外测度性质]\label{theorem:R^n 中点集的外测度性质}
\begin{enumerate}[(1)]
\item 非负性: \(m^*(E) \geq 0\), \(m^*(\varnothing)=0\);
\item 单调性: 若 \(E_1 \subset E_2\), 则 \(m^*(E_1) \leq m^*(E_2)\);
\item 次可加性: \(m^*\left(\bigcup_{k = 1}^{\infty} E_k\right) \leq \sum_{k = 1}^{\infty} m^*(E_k)\).
\end{enumerate}
\end{theorem}
\begin{proof}
\begin{enumerate}[(1)]
\item 这可从定义直接得出.
\item 这是因为 \(E_2\) 的任一个 \(L -\)覆盖都是 \(E_1\) 的 \(L -\)覆盖.
\item 不妨设 \(\sum_{k = 1}^{\infty} m^*(E_k) < + \infty\). 对任意的 \(\varepsilon > 0\) 以及每个自然数 \(k\), 存在 \(E_k\) 的 \(L -\)覆盖 \(\{I_{k, l}\}\), 使得
\begin{align*}
E_k \subset \bigcup_{l = 1}^{\infty} I_{k, l}, \quad \sum_{l = 1}^{\infty} |I_{k, l}| < m^*(E_k) + \frac{\varepsilon}{2^k}.
\end{align*}
由此可知
\begin{align*}
\bigcup_{k = 1}^{\infty} E_k \subset \bigcup_{k, l = 1}^{\infty} I_{k, l}, \quad \sum_{k, l = 1}^{\infty} |I_{k, l}| \leq \sum_{k = 1}^{\infty} m^*(E_k) + \varepsilon.
\end{align*}
显然, \(\{I_{k, l} : k, l = 1, 2, \cdots\}\) 是 \(\bigcup_{k = 1}^{\infty} E_k\) 的 \(L -\)覆盖, 从而有
\begin{align*}
m^*\left(\bigcup_{k = 1}^{\infty} E_k\right) \leq \sum_{k = 1}^{\infty} m^*(E_k) + \varepsilon.
\end{align*}
由 \(\varepsilon\) 的任意性可知结论成立.
\end{enumerate}
\end{proof}

\begin{proposition}\label{proposition-R^n中的单点集的外测度为零}
\(\mathbb{R}^n\) 中的单点集的外测度为零, 即 \(m^*(\{x_0\}) = 0\), \(x_0 \in \mathbb{R}^n\).  同理, \(\mathbb{R}^n\) 中的点集
\[
\left\{ x = (\xi_1, \xi_2, \cdots, \xi_{i - 1}, t_0, \xi_i, \cdots, \xi_n) : a_j \leq \xi_j \leq b_j, j \neq i \right\}
\]
(\(n - 1\) 维超平面块)的外测度也为零.
\end{proposition}
\begin{proof}
这是因为可作一开矩体 \(I\), 使得 \(x_0 \in I\) 且 \(|I|\) 可任意地小.
\end{proof}

\begin{corollary}
若 \(E \subset \mathbb{R}^n\) 为可数点集, 则 \(m^*(E) = 0\).
\end{corollary}
\begin{remark}
由此可知有理点集的外测度 \(m^*(\mathbb{Q}^n)=0\). 这里我们看到了一个虽然处处稠密但外测度为零的可列点集.
\end{remark}
\begin{proof}
由外测度的次可加性不难证明.
\end{proof}

\begin{proposition}\label{proposition:Cantor集的外测度是零}
\([0,1]\) 中的 Cantor 集 \(C\) 的外测度是零.
\end{proposition}
\begin{remark}
这个\refpro{proposition:Cantor集的外测度是零}说明外测度为零的点集不一定是可列集.
\end{remark}
\begin{proof}
事实上, 因为 \(C = \bigcap_{n = 1}^{\infty} F_n\), 其中的 \(F_n\) (在构造 \(C\) 的过程中第 \(n\) 步所留存下来的) 是 \(2^n\) 个长度为 \(3^{-n}\) 的闭区间的并集, 所以我们有
\begin{align*}
m^*(C) \leq m^*(F_n) \leq 2^n \cdot 3^{-n},
\end{align*}
从而得知 \(m^*(C)=0\). 
\end{proof}

\begin{proposition}
设 \(I\) 是 \(\mathbb{R}^n\) 中的开矩体, \(\overline{I}\) 是闭矩体, 则 \(m^*(I) = m^*(\overline{I}) = |I|\). 
\end{proposition}
\begin{proof}
对任给的 \(\varepsilon > 0\), 作一开矩体 \(J\), 使得 \(J \supset \overline{I}\) 且 \(|J| < |I| + \varepsilon\), 从而由外测度的单调性有
\begin{align*}
m^*(\overline{I}) \leq |J| < |I| + \varepsilon.
\end{align*}
由 \(\varepsilon\) 的任意性可知 \(m^*(\overline{I}) \leq |I|\). 现在设 \(\{I_k\}\) 是 \(\overline{I}\) 的任意的 \(L -\)覆盖, 则因为 \(\overline{I}\) 是有界闭集, 所以存在 \(\{I_k\}\) 的有限子覆盖
\[
\{I_{i_1}, I_{i_2}, \cdots, I_{i_l}\}, \quad \bigcup_{j = 1}^l I_{i_j} \supset \overline{I}.
\]
由外测度的单调性和次可加性可得
\begin{align*}
|I| \leq \sum_{j = 1}^l |I_{i_j}| \leq \sum_{k = 1}^{\infty} |I_k|,
\end{align*}
再由下确界是最大的下界可得 \(|I| \leq m^*(\overline{I})\), 从而我们有 \(m^*(\overline{I}) = |I|\). 

又因为$I\subset \overline{I}$,所以由外测度的单调性可得$m^*(I)\leq m^*(\overline{I})=|I|$.同理可证\(|I| \leq m^*(I)\),故$m^*(I)=|I|=m^*(\overline{I})$.
\end{proof}

\begin{lemma}
设 \(E \subset \mathbb{R}^n\) 以及 \(\delta > 0\). 令
\begin{align*}
m^*_{\delta}(E) = \inf\left\{ \sum_{k = 1}^{\infty} |I_k| : \bigcup_{k = 1}^{\infty} I_k \supset E, \text{每个开矩体 } I_k \text{ 的边长} < \delta \right\},
\end{align*}
则 \(m^*_{\delta}(E)=m^*(E)\).
\end{lemma}
\begin{proof}
显然有 \(m^*_{\delta}(E) \geq m^*(E)\). 为证明其反向不等式也成立,不妨设 \(m^*(E)< + \infty\). 由外测度的定义可知,对于任给的 \(\varepsilon > 0\),存在 \(E\) 的 \(L -\)覆盖 \(\{I_k\}\),使得
\begin{align*}
\sum_{k = 1}^{\infty} |I_k| \leq m^*(E) + \varepsilon.
\end{align*}
对于每个 \(k\),我们把 \(I_k\) 分割成 \(l(k)\) 个开矩体:
\[
I_{k,1}, I_{k,2}, \cdots, I_{k,l(k)},
\]
它们互不相交且每个开矩体的边长都小于 \(\delta/2\). 现在保持每个 \(I_{k,i}\) 的中心不动,边长扩大 \(\lambda(1 < \lambda < 2)\) 倍做出开矩体,并记为 \(\lambda I_{k,i}\),显然,对每个 \(k\),有
\begin{align*}
\bigcup_{i = 1}^{l(k)} \lambda I_{k,i} \supset I_k, \quad \sum_{i = 1}^{l(k)} |\lambda I_{k,i}| = \lambda^n \sum_{i = 1}^{l(k)} |I_{k,i}| = \lambda^n |I_k|.
\end{align*}
易知 \(\{\lambda I_{k,i} : i = 1,2,\cdots, l(k); k = 1,2,\cdots\}\) 是 \(E\) 的边长小于 \(\delta\) 的 \(L -\)覆盖,且有
\begin{align*}
\sum_{k = 1}^{\infty} \sum_{i = 1}^{l(k)} |\lambda I_{k,i}| = \lambda^n \sum_{k = 1}^{\infty} |I_k| \leq \lambda^n (m^*(E) + \varepsilon),
\end{align*}
从而可知 \(m^*_{\delta}(E) \leq \lambda^n (m^*(E) + \varepsilon)\). 令 \(\lambda \to 1\) 并注意到 \(\varepsilon\) 的任意性,我们得到 \(m^*_{\delta}(E) \leq m^*(E)\). 这说明 \(m^*_{\delta}(E)=m^*(E)\).
\end{proof}

\begin{theorem}
设 \(E_1, E_2\) 是 \(\mathbb{R}^n\) 中的两个点集. 若 \(d(E_1, E_2)>0\),则
\begin{align*}
m^*(E_1 \cup E_2) = m^*(E_1) + m^*(E_2).
\end{align*}
\end{theorem}
\begin{proof}
只需证明 \(m^*(E_1 \cup E_2) \geq m^*(E_1) + m^*(E_2)\) 即可. 为此,不妨设 \(m^*(E_1 \cup E_2)< + \infty\). 对任给的 \(\varepsilon > 0\),作 \(E_1 \cup E_2\) 的 \(L -\)覆盖 \(\{I_k\}\),使得
\begin{align*}
\sum_{k = 1}^{\infty} |I_k| < m^*(E_1 \cup E_2) + \varepsilon,
\end{align*}
其中 \(I_k\) 的边长都小于 \(d(E_1, E_2)/\sqrt{n}\). 现在将 \(\{I_k\}\) 分为如下两组:
\begin{align*}
(\text{i})J_{i_1}, J_{i_2}, \cdots,\bigcup_{k \geq 1} J_{i_k} \supset E_1 ; \quad (\text{ii})J_{l_1}, J_{l_2}, \cdots,\bigcup_{k \geq 1} J_{l_k} \supset E_2.
\end{align*}
且其中任一矩体皆不能同时含有 \(E_1\) 与 \(E_2\) 中的点,从而得
\begin{align*}
m^*(E_1 \cup E_2) + \varepsilon &> \sum_{k \geq 1} |I_k| = \sum_{k \geq 1} |J_{i_k}| + \sum_{k \geq 1} |J_{l_k}| \\
&\geq m^*(E_1) + m^*(E_2).
\end{align*}
再由 \(\varepsilon\) 的任意性可知 \(m^*(E_1 \cup E_2) \geq m^*(E_1) + m^*(E_2)\). 
\end{proof}

\begin{proposition}
设 \(E \subset [a,b]\), \(m^*(E)>0\), \(0 < c < m^*(E)\), 则存在 \(E\) 的子集 \(A\), 使得 \(m^*(A)=c\).
\end{proposition}
\begin{proof}
记 \(f(x)=m^*([a,x) \cap E)\), \(a \leq x \leq b\), 则 \(f(a)=0\), \(f(b)=m^*(E)\). 考查 \(x\) 与 \(x + \Delta x\). 不妨设 \(a \leq x < x + \Delta x \leq b\), 则由
\[
[a,x + \Delta x) \cap E = ([a,x) \cap E) \cup ([x,x + \Delta x) \cap E)
\]
可知 \(f(x + \Delta x) \leq f(x) + \Delta x\), 即
\begin{align*}
f(x + \Delta x) - f(x) \leq \Delta x.
\end{align*}
对 \(\Delta x < 0\) 也可证得类似不等式. 总之, 我们有
\[
|f(x + \Delta x) - f(x)| \leq |\Delta x|, \quad a \leq x \leq b.
\]
这说明 \(f \in C([a,b])\). 根据连续函数中值定理, 对 \(f(a) < c < f(b)\), 存在 \(\xi \in (a,b)\), 使得 \(f(\xi)=c\). 取 \(A = [a,\xi) \cap E\), 即得证.
\end{proof}

\begin{theorem}[外测度的平移不变性]\label{theorem:外测度的平移不变性}
设 \(E \subset \mathbb{R}^n\), \(x_0 \in \mathbb{R}^n\). 记 \(E + \{x_0\} = \{x + x_0, x \in E\}\), 则
\begin{align}
m^*(E + \{x_0\}) = m^*(E). \label{eq:2.1}
\end{align}
\end{theorem}
\begin{proof}
首先, 对于 \(\mathbb{R}^n\) 中的开矩体 \(I\), 易知 \(I + \{x_0\}\) 仍是一个开矩体且其相应边长均相等, \(|I| = |I + \{x_0\}|\). 其次, 对 \(E\) 的任意的 \(L -\)覆盖 \(\{I_k\}\), \(\{I_k + \{x_0\}\}\) 仍是 \(E + \{x_0\}\) 的 \(L -\)覆盖. 从而由
\begin{align*}
m^*(E + \{x_0\}) \leq \sum_{k = 1}^{\infty} |I_k + \{x_0\}| = \sum_{k = 1}^{\infty} |I_k|
\end{align*}
可知(对一切 \(L -\)覆盖取下确界)
\begin{align*}
m^*(E + \{x_0\}) \leq m^*(E).
\end{align*}
反之, 考虑对 \(E + x_0\) 作向量 \(-x_0\) 的平移, 可得原点集 \(E\). 同理又有
\begin{align*}
m^*(E) \leq m^*(E + \{x_0\}).
\end{align*} 
\end{proof}

\begin{theorem}[外测度的数乘]\label{theorem:外测度的数乘}
设 \(E \subset \mathbb{R}\), \(\lambda \in \mathbb{R}\), 记 \(\lambda E = \{ \lambda x : x \in E\}\), 则
\begin{align*}
m^*(\lambda E) = |\lambda| m^*(E).
\end{align*}
\end{theorem}
\begin{proof}
因为 \(E \subset \bigcup_{n \geq 1} (a_n, b_n)\) 等价于 \(\lambda E \subset \bigcup_{n \geq 1} \lambda (a_n, b_n)\), \(m^*([a_n, b_n]) = m^*((a_n, b_n))\), 且对任一区间 \((\alpha, \beta)\), 有
\begin{align*}
m^*(\lambda (\alpha, \beta)) = |\lambda| m^*((\alpha, \beta)) = |\lambda| (\beta - \alpha),
\end{align*}
所以按外测度定义可得 \(m^*(\lambda E) = |\lambda| m^*(E)\). 
\end{proof}

\begin{definition}[集合上的外测度]
设 \(X\) 是一个非空集合, \(\mu^*\) 是定义在幂集 \(\mathcal{P}(X)\) 上的一个取广义实值的集合函数, 且满足:
\begin{enumerate}[(i)]
\item \(\mu^*(\varnothing)=0\), \(\mu^*(E) \geq 0\) (\(E \subset X\));
\item 若 \(E_1, E_2 \subset X\), \(E_1 \subset E_2\), 则 \(\mu^*(E_1) \leq \mu^*(E_2)\);
\item 若 \(\{E_n\}\) 是 \(X\) 的子集列, 则有
\begin{align*}
\mu^*\left( \bigcup_{n = 1}^{\infty} E_n \right) \leq \sum_{n = 1}^{\infty} \mu^*(E_n),
\end{align*}
\end{enumerate}
那么称 \(\mu^*\) 是 \(X\) 上的一个\textbf{外测度}.

若 \((X, d)\) 是一个距离空间, 且其上的外测度 \(\mu^*\) 还满足\textbf{距离外测度性质}:当 \(d(E_1, E_2)>0\) 时, 有
\begin{align*}
\mu^*(E_1 \cup E_2) = \mu^*(E_1) + \mu^*(E_2),
\end{align*}
那么称 \(\mu^*\) 是 \(X\) 上的一个\textbf{距离外测度}(利用距离外测度性质可以证明开集的可测性). 
\end{definition}









\end{document}
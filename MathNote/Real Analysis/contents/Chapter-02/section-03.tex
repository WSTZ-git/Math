\documentclass[../../main.tex]{subfiles}
\graphicspath{{\subfix{../../image/}}} % 指定图片目录,后续可以直接使用图片文件名。

% 例如:
% \begin{figure}[H]
% \centering
% \includegraphics{image-01.01}
% \caption{图片标题}
% \label{figure:image-01.01}
% \end{figure}
% 注意:上述\label{}一定要放在\caption{}之后,否则引用图片序号会只会显示??.

\begin{document}

\section{可测集与Borel集的关系}

\begin{lemma}[Carathéodory引理]\label{lemma:卡拉西奥多里引理}
设$G\neq\mathbb{R}^n$是开集,$E\subset G$,则令
$E_k = \{x\in E:d(x,G^c)\geqslant 1/k\}\quad (k = 1,2,\cdots)$,
有
\begin{align*}
\lim_{k\to\infty}m^*(E_k)=m^*(E).
\end{align*}
\end{lemma}
\begin{proof}
(i) 易知$\{E_k\}$是递增列,且$\lim_{k\to\infty}E_k\subset E$. 又对$x\in E$,由于$x$是$G$的内点,因此$d(x,y)>0,\forall y\in G^c$,否则,存在$y_0\in G^c$,使得$d(x,y_0)=0$,从而$x=y_0\in G^c$矛盾!于是$$d(x,G^c)=\inf\{d(x,y)|y\in G^c\} \geq 0=\lim_{k\to \infty}\frac{1}{k}.$$
进而存在充分大的$k>0$,使得$d(x,G^c)\geqslant \frac{1}{k}$,即此时$x\in E_k$.

故当$k$充分大时,必有$x\in E_k$,这说明$E\subset\bigcup_{k = 1}^{\infty}E_k=\lim_{k\to\infty}E_k$. 从而可知
\begin{align*}
E=\lim_{k\to\infty}E_k=\bigcup_{k = 1}^{\infty}E_k.
\end{align*}

(ii) 由外测度的单调性可知$m^*(E_k)\leq m^*(E)(k=1,2,\cdots)$,从而$\lim_{k\to\infty}m^*(E_k)\leqslant m^*(E)$. 为证反向不等式,不妨假定$\lim_{k\to\infty}m^*(E_k)<+\infty$. 令
\begin{align*}
A_k = E_{k + 1}\setminus E_k=\left\{ x\in E:d\left( x,G^c \right) \in \left[ \frac{1}{k+1},\frac{1}{k} \right) \right\} (k = 1,2,\cdots),
\end{align*}
则
\begin{align*}
A_{2k}=\left\{ x\in E:d\left( x,G^c \right) \in \left[ \frac{1}{2k+1},\frac{1}{2k} \right) \right\} (k = 1,2,\cdots).
\end{align*}
对$\forall i,j\in \mathbb{N}$且$i\ne j$,不妨设$j>i$,则$j-i\geq 1$.任取$x\in A_{2i},y\in A_{2j}$,则
\begin{align*}
d(x,G^c)\in  \left[ \frac{1}{2i+1},\frac{1}{2i} \right),\quad d(y,G^c)\in  \left[ \frac{1}{2j+1},\frac{1}{2j} \right).
\end{align*}
再由三角不等式可知
\begin{align*}
d\left( x,y \right) \geqslant \left| d\left( x,G^c \right) -d\left( y,G^c \right) \right|\geqslant \frac{1}{2i+1}-\frac{1}{2j}=\frac{2\left( j-i \right) -1}{2j\left( 2i+1 \right)}>0.
\end{align*}
因此$d(A_{2i},A_{2j})\geqslant\frac{2\left( j-i \right) -1}{2j\left( 2i+1 \right)}>0$($i\ne j$). 再注意到$E_{2k}\supset\bigcup_{j = 1}^{k - 1}A_{2j}$,可得
\begin{align*}
m^*(E_{2k})\geqslant m^*\left(\bigcup_{j = 1}^{k - 1}A_{2j}\right)\xlongequal{\text{\refcor{corollary:距离大于零的两个点集的外测度满足可数可加性}}}\sum_{j = 1}^{k - 1}m^*(A_{2j}).
\end{align*}
这说明(令$k\to\infty$)
\begin{align*}
\sum_{j = 1}^{\infty}m^*(A_{2j})<+\infty.\quad \left(\text{类似地可知}\sum_{j = 1}^{\infty}m^*\left(A_{2j + 1}\right)<+\infty\right)
\end{align*}
因为对任意的$k$,我们有
\begin{align*}
E \xlongequal{\text{\refpro{proposition:单调集合上下限的一般形式}}}\bigcup_{j=2k}^{\infty}E_j = E_{2k}\cup\left(\bigcup_{j = k}^{\infty}A_{2j}\right)\cup\left(\bigcup_{j = k}^{\infty}A_{2j + 1}\right),
\end{align*}
所以对任意的$k$,就有
\begin{align*}
m^*(E)\leqslant m^*(E_{2k})+\sum_{j = k}^{\infty}m^*(A_{2j})+\sum_{j = k}^{\infty}m^*(A_{2j + 1}).
\end{align*}
现在,令$k\to\infty$,并注意上式右端后两项趋于零,因此又知
\begin{align*}
m^*(E)\leqslant\lim_{k\to\infty}m^*(E_k),
\end{align*}
即得所证。 
\end{proof}

\begin{theorem}\label{theorem:闭集的可测性}
非空闭集$F$是可测集.
\end{theorem}
\begin{proof}
对任一试验集$T$,由于$T\setminus F\subset F^c = G$是开集,故由\hyperref[lemma:卡拉西奥多里引理]{Carathéodory引理}知,存在$T\setminus F$中的集列$\{F_k\}$:
\begin{align*}
d(F_k,F)\geqslant 1/k>0 (k = 1,2,\cdots),\quad \lim_{k\to\infty}m^*(F_k)=m^*(T\setminus F).
\end{align*}
从而由外测度的单调性我们有(对任一试验集$T$)
\begin{align*}
m^*(T)\geqslant m^*[T\cap (F\cup F_k)]=m^*[(T\cap F)\cup F_k]\xlongequal{\text{\refcor{corollary:距离大于零的两个点集的外测度满足可数可加性}}}m^*(T\cap F)+m^*(F_k).
\end{align*}
再令$k\to\infty$,可得
\begin{align*}
m^*(T)\geqslant m^*(T\cap F)+m^*(T\setminus F)=m^*(T\cap F)+m^*(T\cap F^c).
\end{align*}
这说明$F$是可测集.

\end{proof}

\begin{corollary}\label{corollary:Borel集是可测集}
Borel集是可测集.
\end{corollary}
\begin{proof}
由\hyperref[theorem:闭集的可测性]{闭集的可测性}及\hyperref[theorem:可测集的性质]{可测集的性质(2)}可知开集是可测集. 又因为可测集类是一个$\sigma$-代数,所以由\hyperref[definition:Borel集]{Borel集的定义}可知可测集包含Borel$\,\sigma$-代数,故任一 Borel 集皆可测. 
\end{proof}

\begin{theorem}\label{theorem:定理2.13}
若$E\in\mathscr{M}$,则对任给的$\varepsilon > 0$,我们有
\begin{enumerate}[(i)]
\item 存在包含$E$的开集$G$,使得$m(G\setminus E)<\varepsilon$;
\item 存在含于$E$的闭集$F$,使得$m(E\setminus F)<\varepsilon$。
\end{enumerate}
\end{theorem}
\begin{proof}
\begin{enumerate}[(i)]
\item 首先考虑$m(E)<+\infty$的情形。由定义知,存在$E$的$L$-覆盖$\{I_k\}$,使得
\begin{align*}
\sum_{k = 1}^{\infty}|I_k|<m(E)+\varepsilon.
\end{align*}
令$G = \bigcup_{k = 1}^{\infty}I_k$,则$G$是包含$E$的开集,且
\begin{align*}
m\left( G \right) =m\left( \bigcup_{k=1}^{\infty}{I_k} \right) \leqslant \sum_{k=1}^{\infty}{m\left( I_k \right)}=\sum_{k=1}^{\infty}{\left| I_k \right|}<m(E)+\varepsilon .
\end{align*}
因为$m(E)<+\infty$,所以移项后再合并由\nrefthe{theorem:测度的基本性质}{(2)}得$m(G\setminus E)=m(G)-m(E)<\varepsilon$

其次讨论$m(E)$是$+\infty$的情形。令
\begin{align*}
E_k = E\cap B(0,k),\quad E=\bigcup_{k = 1}^{\infty}E_k,\quad k = 1,2,\cdots.
\end{align*}
因为$m(E_k)<\infty$($k = 1,2,\cdots$),所以对任给的$\varepsilon>0$,存在包含$E_k$的开集$G_k$,使得$m(G_k\setminus E_k)<\varepsilon/2^k$。现在作点集$G = \bigcup_{k = 1}^{\infty}G_k$,则$G\supset E$且为开集。由\nrefthe{theorem:集族的并和交的基本性质}{(3)}我们有
\begin{align*}
G\setminus E\subset\bigcup_{k = 1}^{\infty}(G_k\setminus E_k),
\end{align*}
从而得
\begin{align*}
m(G\setminus E)\leqslant\sum_{k = 1}^{\infty}m(G_k\setminus E_k)<\sum_{k = 1}^{\infty}\frac{\varepsilon}{2^k}=\varepsilon.
\end{align*}
\item 考虑$E^c$。由(i)可知,对任给的$\varepsilon>0$,存在包含$E^c$的开集$G$,使得$m(G\setminus E^c)<\varepsilon$。现在令$F = G^c$,显然$F$是闭集且$F\subset E$。由\nrefpro{proposition:集合的差与补的基本性质}{(4)}可知$E\setminus F = G\setminus E^c$,所以得到$m(E\setminus F)<\varepsilon$。
\end{enumerate}
\end{proof}

\begin{corollary}\label{corollary:集合可测的充要条件2}
$E\in \mathscr{M}$当且仅当对$\forall \varepsilon>0,$存在开集$G\supset E$、闭集$F\subset E$,使得$m(G\backslash F)<\varepsilon.$
\end{corollary}
\begin{proof}
{\heiti 必要性:}由\refthe{theorem:定理2.13}可知,存在开集$G$和闭集$F$,满足$F\subset E\subset G$,使得$m(E\backslash F),m(G\backslash E)<\frac{\varepsilon}{2}$.注意到
\begin{align*}
G\backslash F=(G\backslash E)\cup (E\backslash F),
\end{align*}
并且$(G\backslash E)\cap (E\backslash F)=\varnothing$,故由测度的可数可加性可得
\begin{align*}
m(G\backslash F)=m[(G\backslash E)\cup (E\backslash F)]=m(G\backslash E)+m(E\backslash F)<\varepsilon.
\end{align*}

{\heiti 充分性:}对$\forall \varepsilon>0$,由\refcor{corollary:Borel集是可测集}可知$G,F$都可测.任取一点集$T$,则由外测度的性质和测度的定义可得
\begin{align*}
&m^*\left( T\cap E \right) +m^*\left( T\cap E^c \right) \leqslant m^*\left( T\cap G \right) +m^*\left( T\cap F^c \right) =m^*(T\cap \left( \left( G\backslash F \right) \cup F \right) )+m^*\left( T\cap F^c \right) 
\\
&\leqslant m^*\left( \left( T\cap \left( G\backslash F \right) \right) \cup \left( T\cap F \right) \right) +m^*\left( T\cap F^c \right) \leqslant m^*\left( T\cap \left( G\backslash F \right) \right) +m^*\left( T\cap F \right) +m^*\left( T\cap F^c \right) 
\\
&\leqslant m^*\left( G\backslash F \right) +m^*\left( T \right) <m^*\left( T \right) +\varepsilon .
\end{align*}
由$\varepsilon$的任意性可知
\begin{align*}
m^*\left( T\cap E \right) +m^*\left( T\cup E^c \right) \leqslant m^*(T).
\end{align*}
故再由\refthe{theorem:可测的充要条件}可知$E$可测.
\end{proof}

\begin{theorem}\label{theorem:定理2.14}
若$E\in\mathscr{M}$,则
\begin{enumerate}[(i)]
\item $E = H\setminus Z_1$,$H$是$G_\delta$集,$m(Z_1)=0$;
\item $E = K\cup Z_2$,$K$是$F_\sigma$集,$m(Z_2)=0$。
\end{enumerate}
\end{theorem}
\begin{proof}
\begin{enumerate}[(i)]
\item 对于每个自然数$k$,由\nrefthe{theorem:定理2.13}{(i)}可知,存在包含$E$的开集$G_k$,使得$m(G_k\setminus E)<\frac{1}{k}$。现在作点集$H = \bigcap_{k = 1}^{\infty}G_k$,则$H$为$G_\delta$集且$E\subset H$。因为对一切$k$,都有
\begin{align*}
m(H\setminus E)\leqslant m(G_k\setminus E)<\frac{1}{k},
\end{align*}
所以令$k\to \infty$可得$m(H\setminus E)=0$。若令$H\setminus E = Z_1$,则得$E = H\setminus Z_1$。
\item 对于每个自然数$k$,由\nrefthe{theorem:定理2.13}{(ii)}可知,存在含于$E$的闭集$F_k$,使得$m(E\setminus F_k)<\frac{1}{k}$。现在作点集$K = \bigcup_{k = 1}^{\infty}F_k$,则$K$是$F_\sigma$集且$K\subset E$。因为对一切$k$,都有
\begin{align*}
m(E\setminus K)\leqslant m(E\setminus F_k)<\frac{1}{k},
\end{align*}
所以令$k\to \infty$可得$m(E\setminus K)=0$。若令$E\setminus K = Z_2$,则得$E = K\cup Z_2$。
\end{enumerate}
\end{proof} 

\begin{theorem}[外测度的正则性]\label{theorem:外测度的正则性}
若\(E \subset \mathbb{R}^n\),则存在包含\(E\)的\(G_\delta\)集\(H\),使得\(m(H)=m^*(E)\).
\end{theorem}
\begin{proof}
由外测度的定义和下确界的定义可知,对于每个自然数\(k\),存在包含\(E\)的开集\(G_k\),使得
\begin{align*}
m(G_k) \leqslant m^*(E) + \frac{1}{k}.
\end{align*}
现在作点集\(H = \bigcap_{k = 1}^{\infty}G_k\),则\(H\)是\(G_\delta\)集且\(H \supset E\)。因为
\begin{align*}
m^*(E) \leqslant m(H) \leqslant m(G_k) \leqslant m^*(E) + \frac{1}{k},
\end{align*}
所以令$k\to \infty$可得\(m(H)=m^*(E)\)。
\end{proof}

\begin{definition}[等测包与等测核]
\begin{enumerate}
\item 设$E\subset \mathbb{R}^n$,若存在包含$E$的可测集$H$,使得$m(H)=m^*(E)$.我们称如此的$H$为$E$的\textbf{等测包}.

\item 设$E\in \mathscr{M}$,若存在含于$E$的可测集$K$,使得$m(K)=m(E)$.我们称如此的$K$为$E$的\textbf{等测核}.
\end{enumerate}
\end{definition}
\begin{note}
由\hyperref[theorem:外测度的正则性]{外测度的正则性}可知上述定义的等测包(一定存在)是良定义的.
由\hyperref[theorem:定理2.14]{定理\ref{theorem:定理2.14}(ii)}可知上述定义的等测核(一定存在)是良定义的.
\end{note}
\begin{remark}
注意,若\(H\)是\(E\)的等测包且\(m^*(E)<\infty\),则有
\begin{align*}
m(H) - m^*(E) = 0,
\end{align*}
但\(m^*(H\setminus E)\)不一定等于零。不过可以证明\(H\setminus E\)的任一可测子集皆为零测集(见\refpro{proposition:等测包与原集合差的可测子集必为零测集})。
\end{remark}

\begin{proposition}\label{proposition:等测包与原集合差的可测子集必为零测集}
若\(H\)是\(E\)的等测包且\(m^*(E)<\infty\),则\(H\setminus E\)的任一可测子集皆为零测集。
\end{proposition}
\begin{proof}
设\(A\)为\(H\setminus E\)的可测子集,则由\(A\subset H\setminus E\)可知,\(A\subset H\)且\(A\cap E=\varnothing\)。又注意到\(E\subset H\),故\(E\subset H\setminus A\)。又因\(H\)可测,故\(H\setminus A\)也可测。从而由外测度的单调性可知
\begin{align}
m(H\setminus A)\geqslant m^*(E).\label{eq:100.59}
\end{align}
由\(H\setminus A\)可测得(\(H\)为试验集)
\begin{align*}
m(H)&=m^*(H)=m^*(H\cap(H\setminus A))+m^*(H\cap(H\setminus A)^c)\\
&=m(H\setminus A)+m^*(H\cap(H\cap A^c)^c)
\\
&=m(H\setminus A)+m^*(H\cap(H^c\cup A))\\
&=m(H\setminus A)+m(A).
\end{align*}
又由\(H\)为\(E\)的等测包可知\(m(H)=m^*(E)\),结合上式可得
\begin{align*}
m^*(E)=m(H\setminus A)+m(A).
\end{align*}
再结合\eqref{eq:100.59}式,有
\begin{align*}
m^*(E)\geqslant m^*(E)+m(A).
\end{align*}
移项得\(m(A)\leqslant0\)。故由测度的非负性可知\(m(A)=0\)。
\end{proof}

\begin{corollary}
设\(E_k \subset \mathbb{R}^n\)(\(k = 1,2,\cdots\)),则
\begin{align*}
m^*\left(\varliminf_{k \to \infty} E_k\right) \leqslant \varliminf_{k \to \infty} m^*(E_k).
\end{align*}
\end{corollary}
\begin{proof}
对每个\(E_k\)均作等测包\(H_k\):
\begin{align*}
H_k \supset E_k,\quad m(H_k) = m^*(E_k)\quad (k = 1,2,\cdots),
\end{align*}
则可得
\begin{align*}
m^*\left(\varliminf_{k \to \infty} E_k\right) \overset{\text{外测度的单调性}}{\leqslant} m\left(\varliminf_{k \to \infty} H_k\right) \overset{\text{\nrefthe{theorem:可测集的Fatou引理}{(2)}}}{\leqslant} \varliminf_{k \to \infty} m(H_k) = \varliminf_{k \to \infty} m^*(E_k).
\end{align*}
\end{proof}

\begin{corollary}
若\(\{E_k\}\)是递增集合列,则
\begin{align*}
\lim_{k \to \infty} m^*(E_k) = m^*\left(\lim_{k \to \infty} E_k\right).
\end{align*}
\end{corollary} 
\begin{proof}
记\(E=\lim_{k \to \infty}E_k=\bigcup_{k = 1}^{\infty}E_k\),则由\(\{E_k\}\)的递增性可知\(E_k\subset E\)(\(k = 1,2,\cdots\)),从而由外测度的单调性可得
\begin{align*}
m^*(E_k)\leqslant m^*(E),\quad k = 1,2,\cdots.
\end{align*}
令\(k \to \infty\),得\(\lim_{k \to \infty}m^*(E_k)\leqslant m^*(E)\)。若\(\lim_{k \to \infty}m^*(E_k)=+\infty\),则结论显然成立。故不妨设\(\lim_{k \to \infty}m^*(E_k)<+\infty\).

下证\(\lim_{k \to \infty}m^*(E_k)\geqslant m^*(E)\)。
对\(\forall k\in\mathbb{N}\),取\(E_k\)的等测包\(H_k\),则\(m(H_k)=m^*(E_k)\)。令\(F_k=\bigcap_{m = k}^{\infty}H_m\),则显然\(F_k\)可测,\(\{F_k\}\)递增,\(E_k\subset F_k\subset H_k\)。
再令\(F=\bigcup_{k = 1}^{\infty}F_k\),则\(F\)可测,\(E=\bigcup_{k = 1}^{\infty}E_k\subset \bigcup_{k = 1}^{\infty}F_k=F\)。于是由外测度的单调性及\hyperref[theorem:递增可测集列的测度运算]{递增可测集列的测度运算}可得
\begin{align}
m^*(E)\leqslant m(F)=m\left(\bigcup_{k = 1}^{\infty}F_k\right)=m\left(\lim_{k \to \infty}F_k\right)\xlongequal{\text{\hyperref[theorem:递增可测集列的测度运算]{递增可测集列的测度运算}}}\lim_{k \to \infty}m(F_k).\label{eq:100.60}
\end{align}
又由\(F_k\subset H_k\)和测度的单调性以及\(m(H_k)=m^*(E_k)\)可知
\begin{align}
\lim_{k \to \infty}m(F_k)\leqslant \lim_{k \to \infty}m(H_k)=\lim_{k \to \infty}m^*(E_k).\label{eq:100.61}
\end{align}
故结合\eqref{eq:100.60}\eqref{eq:100.61}式可得\(m^*(E)\leqslant \lim_{k \to \infty}m^*(E_k)\)。综上可得,\(m^*(E)=\lim_{k \to \infty}m^*(E_k)\)。 
\end{proof}

\begin{theorem}
若\(E \in \mathscr{M}\),\(x_0 \in \mathbb{R}^n\),则\((E + \{x_0\}) \in \mathscr{M}\)且
\begin{align*}
m(E + \{x_0\}) = m(E).
\end{align*}
\end{theorem}
\begin{proof}
由\refthe{theorem:定理2.14}可知
\begin{align*}
E = H \setminus Z,
\end{align*}
其中\(H = \bigcap_{k = 1}^{\infty} G_k\),每个\(G_k\)都是开集,\(m(Z) = 0\)。因为\(G_k + \{x_0\}\)是开集,所以
\begin{align*}
\bigcap_{k = 1}^{\infty} (G_k + \{x_0\})
\end{align*}
是可测集。根据\hyperref[theorem:外测度的平移不变性]{外测度的平移不变性},可知点集\(Z + \{x_0\}\)是零测集,于是从等式
\begin{align*}
E + \{x_0\}= (H + \{x_0\}) \setminus (Z + \{x_0\})
= \left(\bigcap_{k = 1}^{\infty} (G_k + \{x_0\}) \setminus (Z + \{x_0\})\right)
\end{align*}
立即可知\(E + \{x_0\} \in \mathscr{M}\)。再用\hyperref[theorem:外测度的平移不变性]{外测度的平移不变性}得到
\begin{align*}
m(E + \{x_0\}) = m(E).
\end{align*}
\end{proof}
\begin{remark}
一般地说,若在Borel \(\sigma\)-代数上定义了测度\(\mu\),且对紧集\(K\)有\(\mu(K)<+\infty\),则称\(\mu\)为\textbf{Borel测度}(显然,\(\mathbb{R}^n\)上的Lebesgue测度是一种Borel测度)。

可以证明:若\(\mu\)是\(\mathbb{R}^n\)上的平移不变的Borel测度,则存在常数\(\lambda\),使得对\(\mathbb{R}^n\)中每一个Borel集\(B\),均有
\begin{align*}
\mu(B)=\lambda m(B).
\end{align*}
这就是说,除了一个常数倍因子外,Lebesgue测度是\(\mathbb{R}^n\)上平移不变的唯一的Borel测度。 
\end{remark}

\begin{example}
作\([0,1]\)中的第二纲零测集\(E\)。
\end{example}
\begin{solution}
令\(\{r_n\}=[0,1]\cap\mathbb{Q}\),\(I_{n,k}=(r_n - 2^{-n - k},r_n + 2^{-n - k})\)(\(n,k\in\mathbb{N}\)),易知
\begin{align*}
m\left(\bigcup_{n = 1}^{\infty}I_{n,k}\right)\leqslant 2^{-k + 1},\quad
m\left(\bigcap_{k = 1}^{\infty}\bigcup_{n = 1}^{\infty}I_{n,k}\right)= 0.
\end{align*}
由于每个\([0,1]\setminus\bigcup_{n = 1}^{\infty}I_{n,k}\)(\(k\in\mathbb{N}\))均是无处稠密集,故可知\(E = \bigcap_{k = 1}^{\infty}\bigcup_{n = 1}^{\infty}I_{n,k}\)是第二纲集。
\end{solution}

\begin{example}
设\(A\subset\mathbb{R}\),且对\(x\in A\),存在无穷多个数组\((p,q)\)(\(p,q\in\mathbb{Z}\),\(q\geqslant1\)),使得\(\vert x - p/q\vert\leqslant1/q^3\),则\(m(A)=0\)
\end{example}
\begin{proof}
\begin{enumerate}[(i)]
\item 令\(B = [0,1]\cap A\),注意到\(x + n-(p + nq)/q=x - p/q\),故\(A=\bigcup_{n = -\infty}^{+\infty}(B+\{n\})\),从而只需指出\(m(B)=0\)。
\item 令\(I_{p,q}=\left[\frac{p}{q}-\frac{1}{q^3},\frac{p}{q}+\frac{1}{q^3}\right]\),则\(x\in I_{p,q}\)等价于
\begin{align}
qx-\frac{1}{q^2}\leqslant p\leqslant qx+\frac{1}{q^2}.\label{eq:2.7}
\end{align}
易知对\(q\geqslant2\)或\(q = 1\),在长度为\(2/q^2\)的区间中至多有一个或三个整数,故\(x\in B\)当且仅当\(x\)属于无穷多个\(B_q\):
\(B_q = [0,1]\cap\left(\bigcup_{p}I_{p,q}\right)\)。
从而又只需指出\(\sum_{q}m(B_q)<+\infty\)。
由\eqref{eq:2.7}式知,对整数\(q\),使\(I_{p,q}\cap[0,1]\neq\varnothing\)就是
\(-\frac{1}{q^2}\leqslant p\leqslant q+\frac{1}{q^2}\)。
在\(q\geqslant2\)时,这相当于\(0\leqslant p\leqslant q\)。因此,我们有
\(m(B_q)\leqslant 2(q + 1)/q^3\),
即得所证。
\end{enumerate} 
\end{proof}




















\end{document}
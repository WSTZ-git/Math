\documentclass[../../main.tex]{subfiles}
\graphicspath{{\subfix{../../image/}}} % 指定图片目录,后续可以直接使用图片文件名。

% 例如:
% \begin{figure}[H]
% \centering
% \includegraphics[scale=0.4]{图.png}
% \caption{}
% \label{figure:图}
% \end{figure}
% 注意:上述\label{}一定要放在\caption{}之后,否则引用图片序号会只会显示??.

\begin{document}

\section{连续变换与可测集}

\begin{theorem}[变换的基本性质]\label{theorem:变换的基本性质}
设变换 $T:\mathbb{R}^n\to\mathbb{R}^n$,则
\begin{enumerate}
\item $T\left( \bigcup_{\alpha \in I}{A_{\alpha}} \right) =\bigcup_{\alpha \in I}{T\left( A_{\alpha} \right)}$.

\item 
\end{enumerate}
\end{theorem}
\begin{proof}
\begin{enumerate}
\item 下面是转换后的LaTeX正文格式代码:

一方面, 对$\forall x\in T\left( \bigcup_{\alpha \in I}{A_{\alpha}} \right)$, 存在$y\in \bigcup_{\alpha \in I}{A_{\alpha}}$, 从而存在$\alpha _y\in I$, 使得$y\in A_{\alpha _y}$且$x=T\left( y \right)$. 于是$x=T\left( y \right) \subset T\left( A_{\alpha _y} \right) \subset \bigcup_{\alpha \in I}{T\left( A_{\alpha} \right)}$. 故$T\left( \bigcup_{\alpha \in I}{A_{\alpha}} \right) \subset \bigcup_{\alpha \in I}{T\left( A_{\alpha} \right)}$.

另一方面, 对$\forall x\in \bigcup_{\alpha \in I}{T\left( A_{\alpha} \right)}$, 都存在$\alpha _x\in I$, 使得$x\in T\left( A_{\alpha _x} \right)$. 于是存在$y\in A_{\alpha _x}$, 使得$x=T\left( y \right)$. 又因为$y\in A_{\alpha _x}\subset \bigcup_{\alpha \in I}{A_{\alpha}}$, 所以$x=T\left( y \right) \subset T\left( \bigcup_{\alpha \in I}{A_{\alpha}} \right)$. 故$T\left( \bigcup_{\alpha \in I}{A_{\alpha}} \right) \supset \bigcup_{\alpha \in I}{T\left( A_{\alpha} \right)}$.

\item 
\end{enumerate}
\end{proof}

\begin{definition}[连续变换]
设有变换 $T:\mathbb{R}^n\to\mathbb{R}^n$. 若对任一开集 $G\subset\mathbb{R}^n$, 逆(原)像集
\[
T^{-1}(G)\quad \text{即}\quad \{x\in\mathbb{R}^n:T(x)\in G\}
\]
是一个开集, 则称 $T$ 是从 $\mathbb{R}^n$ 到 $\mathbb{R}^n$ 的\textbf{连续变换}.
\end{definition}

\begin{theorem}[连续变换的充要条件]\label{theorem:连续变换的充要条件}
变换 $T:\mathbb{R}^n\to\mathbb{R}^n$ 是连续变换的充分必要条件是, 对任一点 $x\in\mathbb{R}^n$ 以及任意的 $\varepsilon>0$, 存在 $\delta>0$, 使得当 $|y - x|<\delta$ 时, 有
\begin{align}
|T(y) - T(x)|<\varepsilon.\label{eq:thm221}
\end{align}
\end{theorem}
\begin{proof}
{\heiti 必要性:} 对任一点 $x\in\mathbb{R}^n$ 以及任意的 $\varepsilon>0$, 有 $x$ 属于开集
\[
T^{-1}(B(T(x),\varepsilon)),
\]
从而存在 $\delta>0$, 使得
\[
B(x,\delta)\subset T^{-1}(B(T(x),\varepsilon)).
\]
这说明, 当 $|y - x|<\delta$ 时, 有 $y\in B(x,\delta)\subset T^{-1}(B(T(x),\varepsilon))$, 即
\[
|T(y) - T(x)|<\varepsilon.
\]

{\heiti 充分性:} 设 $G$ 是 $\mathbb{R}^n$ 中任一开集, 且 $T^{-1}(G)$ 不是空集, 则对任一点 $x\in T^{-1}(G)$, 有 $T(x)\in G$. 因此, 存在 $\varepsilon>0$, 使得 $B(T(x),\varepsilon)\subset G$. 根据充分性的假定, 对此 $\varepsilon>0$, 存在 $\delta>0$, 使得当 $|y - x|<\delta$ 时, 有
\[
|T(y) - T(x)|<\varepsilon,\quad \text{即}\quad T(y)\in B(T(x)\subset G,\varepsilon).
\]
也即$T(y)\in G,\forall y\in B(x,\delta)$.此即$T(B(x,\delta))\subset G$.
这就是说 $B(x,\delta)\subset T^{-1}(G)$, 即 $T^{-1}(G)$ 是开集. 
\end{proof}

\begin{proposition}
若 $T:\mathbb{R}^n\to\mathbb{R}^n$ 是线性变换, 则 $T$ 是连续变换.
\end{proposition}
\begin{proof}
令 $e_i$ ($i = 1,2,\cdots,n$) 是 $\mathbb{R}^n$ 中的一组基, 则对 $\mathbb{R}^n$ 中任意的 $x = (\xi_1,\xi_2,\cdots,\xi_n)$, 有
\begin{align*}
x = \xi_1e_1 + \xi_2e_2 + \cdots + \xi_ne_n.
\end{align*}
再令 $T(e_i) = x_i$ ($i = 1,2,\cdots,n$), 又有
\begin{align*}
T(x) = \xi_1x_1 + \xi_2x_2 + \cdots + \xi_nx_n.
\end{align*}
记 $M = \left(\sum_{i = 1}^{n}|x_i|^2\right)^{1/2}$ ,从而由Cauchy不等式可得
\begin{align*}
|T(x)|&\leqslant|\xi_1||x_1| + |\xi_2||x_2| + \cdots + |\xi_n||x_n|\\
&\leqslant\left(\sum_{i = 1}^{n}|x_i|^2\right)^{1/2}\left(\sum_{i = 1}^{n}|\xi_i|^2\right)^{1/2} = M|x|.
\end{align*}
由此可知
\begin{align*}
|T(y) - T(x)| = |T(y - x)|&\leqslant M|y - x|.
\end{align*}
再由\hyperref[theorem:连续变换的充要条件]{连续变换的充要条件}可知 $T$ 是连续变换. 
\end{proof}

\begin{theorem}
设 $T:\mathbb{R}^n\to\mathbb{R}^n$ 是连续变换. 若 $K$ 是 $\mathbb{R}^n$ 中的紧集, 则 $T(K)$ 是 $\mathbb{R}^n$ 中的紧集.
\end{theorem}
\begin{proof}
对于 $T(K)$ 的任一开覆盖族 $\{H_i\}$, 令 $G_i = T^{-1}(H_i)$, 则 $\{G_i\}$ 是 $K$ 的开覆盖族. 根据有限子覆盖定理可知, 在 $\{G_i\}$ 中存在 $G_{i_1},G_{i_2},\cdots,G_{i_k}$, 使得
\begin{align*}
K\subset\bigcup_{j = 1}^{k}G_{i_j}.
\end{align*}
从而得
\begin{align*}
T(K)\subset\bigcup_{j = 1}^{k}T(G_{i_j})\subset\bigcup_{j = 1}^{k}H_{i_j}.
\end{align*}
这说明 $T(K)$ 是 $\mathbb{R}^n$ 中的紧集.
\end{proof}

\begin{corollary}
设 $T:\mathbb{R}^n\to\mathbb{R}^n$ 是连续变换. 若 $E$ 是 $F_\sigma$ 集, 则 $T(E)$ 是 $F_\sigma$ 集.
\end{corollary}
\begin{proof}
由$E\subset \mathbb{R} ^n$是$F_{\sigma}$集, 故$E=\bigcup_{k=1}^{\infty}{E_k}$, 其中$E_k$都是闭集. 令
\begin{align*}
F_k=E_k\cap C\left( 0,k \right) \left( k=1,2,\cdots \right) ,\quad F=\bigcup_{k=1}^{\infty}{F_k}.
\end{align*}
显然$\left\{ F_k \right\}$是$\mathbb{R} ^n$中的递增紧集列, 并且
\begin{align*}
F=\bigcup_{k=1}^{\infty}{\left( E_k\cap C\left( 0,k \right) \right)}\subset \left( \bigcup_{k=1}^{\infty}{E_k} \right) \cap \left( \bigcup_{k=1}^{\infty}{C\left( 0,k \right)} \right) =E\cap \mathbb{R} ^n=E.
\end{align*}
于是
\begin{align*}
T\left( E \right) =T\left( F \right) =T\left( \bigcup_{k=1}^{\infty}{F_k} \right) =\bigcup_{k=1}^{\infty}{T\left( F_k \right)}.
\end{align*}
由定理可知$T\left( F_k \right)$都是$\mathbb{R} ^n$中的紧集, 进而$T\left( F_k \right)$都是闭集, 从而$\bigcup_{k=1}^{\infty}{T\left( F_k \right)}$也是闭集. 故$T\left( E \right)$是闭集, 结论得证.
\end{proof}

\begin{corollary}
设 $T:\mathbb{R}^n\to\mathbb{R}^n$ 是连续变换. 若对 $\mathbb{R}^n$ 中的任一零测集 $Z$, $T(Z)$ 必为零测集, 则对 $\mathbb{R}^n$ 中的任一可测集 $E$, $T(E)$ 必为可测集.
\end{corollary}
\begin{proof}
根据\nrefthe{theorem:定理2.14}{(ii)}, 有 $E = K\cup Z$, 其中 $K$ 是 $F_\sigma$ 集, $Z$ 是零测集. 因为
\begin{align*}
T(E) = T(K)\cup T(Z),
\end{align*}
而 $T(K)$ 是 $F_\sigma$ 集, $T(Z)$ 为零测集, 所以 $T(E)$ 是可测集.
\end{proof}

\begin{theorem}\label{theorem:集合的线性变换的像的外测度}
若 $T:\mathbb{R}^n\to\mathbb{R}^n$ 是非奇异线性变换, $E\subset\mathbb{R}^n$, 则
\begin{align}
m^*(T(E)) = |\det T|\cdot m^*(E).\label{eq:thm225}
\end{align}
\end{theorem}
\begin{remark}
在 $|\det T| = 0$ 时, $T$ 将 $\mathbb{R}^n$ 变为一个低维线性子空间, 显然其映像集是零测集, 我们有
\[
m(T(E)) = |\det T|\cdot m(E) = 0,\quad E\subset\mathbb{R}^n.
\] 
\end{remark}
\begin{proof}
记
\begin{align*}
I_0 &= \{x = (\xi_1,\xi_2,\cdots,\xi_n): 0\leqslant\xi_i < 1, 1\leqslant i\leqslant n\},\\
I &= \{x = (\xi_1,\xi_2,\cdots,\xi_n): 0\leqslant\xi_i < 2^{-k}, 1\leqslant i\leqslant n\}.
\end{align*}
显然, $I_0$ 是 $2^{nk}$ 个 $I$ 的平移集 $I + \{x_j\}$ ($j = 1,2,\cdots,2^{nk}$) 的并集, $T(I_0)$ 是 $2^{nk}$ 个
\[
T(I + \{x_j\}),\quad j = 1,2,\cdots,2^{nk}
\]
的并集, 而且有 (注意 $T^{-1}$ 是连续变换)
\[
m(T(I + \{x_j\})) = m(T(I)),\quad j = 1,2,\cdots,2^{nk}.
\]

现在假定 \eqref{eq:thm225} 式对于 $I_0$ 成立:
\begin{align}
m(T(I_0)) = |\det T|,\label{eq:thm225_1}
\end{align}
则
\[
|\det T| = 2^{nk}m(T(I)).
\]
因为 $m(I)=2^{-nk}$, 所以得到
\[
m(T(I)) = 2^{-nk}|\det T| = |\det T|m(I).
\]
这说明 \eqref{eq:thm225} 式对每个 $I$ 以及 $I$ 的平移集都成立, 从而可知 \eqref{eq:thm225} 式对可数个互不相交的任意二进方体的并集是成立的, 也就说明对任一开集 $G\subset\mathbb{R}^n$ \eqref{eq:thm225} 式均成立. 于是应用等测包的推理方法立即可知, 对一般点集 \eqref{eq:thm225} 式成立.

下面证明 \eqref{eq:thm225_1} 式成立. 大家知道 $T$ 至多可以表为如下几个初等变换的乘积:

(i)坐标 $\xi_1,\xi_2,\cdots,\xi_n$ 之间的交换;

(ii)$\xi_1\to\beta\xi_1,\xi_i\to\xi_i$ ($i = 2,3,\cdots,n$);

(iii)$\xi_1\to\xi_1 + \xi_2,\xi_i\to\xi_i$ ($i = 2,3,\cdots,n$).

在 (i) 的情形, 显然有 $|\det T| = 1$, $T(I_0)=I_0$. 从而可知 \eqref{eq:thm225_1} 式成立.

在 (ii) 的情形, 矩阵 $T$ 可由恒等矩阵在第一行乘以 $\beta$ 而得到, 此时有
\[
T(I_0) = \{x = (\xi_1,\xi_2,\cdots,\xi_n): 0\leqslant\xi_i < 1\ (i = 2,3,\cdots,n),\ 0\leqslant\xi_1 < \beta\ (\beta > 0), \beta < \xi_1\leqslant 0\ (\beta < 0)\}.
\]
从而可知 $m(T(I_0)) = |\beta|$, 即 \eqref{eq:thm225_1} 式成立.

在 (iii) 的情形, 此时 $\det T = 1$, 而且有
\[
T(I_0) = \{x = (\xi_1,\xi_2,\cdots,\xi_n): 0\leqslant\xi_i < 1\ (i\neq 1),\ 0\leqslant\xi_1 - \xi_2 < 1\}.
\]
记
\begin{align*}
A &= \{x = (\xi_1,\xi_2,\cdots,\xi_n)\in T(I_0): \xi_1 < 1\},\\
e_1 &= (1,0,\cdots,0),\quad B = T(I_0)\setminus A.
\end{align*}
我们有
\begin{align*}
A &= \{x = (\xi_1,\xi_2,\cdots,\xi_n)\in I_0: \xi_2 < \xi_1\},\\
B - e_1 &= \{x = (\xi_1,\xi_2,\cdots,\xi_n)\in I_0: \xi_1 < \xi_2\}.
\end{align*}
因此得到
\begin{align*}
m(T(I_0)) &= m(A) + m(B) = m(A) + m(B - e_1)\\
&= m(I_0) = 1 = \det T.
\end{align*}
这说明 \eqref{eq:thm225_1} 式对 $I_0$ 成立.

最后不妨设 $T = T_1\cdot T_2\cdot\cdots\cdot T_j$, 这里的每个 $T_j$ 均是 (i)~(iii) 情形之一, 从而由归纳法可知
\begin{align*}
m^*(T(E)) &= m(T_1(T_2(\cdots(T_j(E))\cdots)))\\
&= |\det T_1||\det T_2|\cdots|\det T_j|m^*(E)\\
&= |\det T|m^*(E).
\end{align*}
\end{proof}

\begin{corollary}
设 \(T:\mathbb{R}^n \to \mathbb{R}^n\) 是非奇异线性变换. 若 \(E \in \mathscr{M}\),则 \(T(E)\in\mathscr{M}\) 且有
\begin{align*}
m(T(E)) = |\det T|m(E).
\end{align*}
\end{corollary}
\begin{proof}
由\refthe{theorem:集合的线性变换的像的外测度}立得.
\end{proof}

\begin{example}
若 \(E \subset \mathbb{R}^2\) 是可测集,则将 \(E\) 作旋转变换后所成集为可测集,且测度不变.
\end{example}
\begin{proof}

\end{proof}

\begin{example}
\(\mathbb{R}^2\) 中三角形的测度等于它的面积.
\end{example}
\begin{proof}
显然,\(\mathbb{R}^2\) 中任一三角形都是可测集. 由于测度的平移不变性,故不妨假定三角形的一个顶点在原点. 记三角形为 \(T\),其面积记为 \(|T|\). 因为 \(m(T)=m(-T)\),所以经平移后可得 \(2m(T)=m(T)+m(-T)=m(P)\),其中 \(P\) 是平行四边形. 再将 \(P\) 中的子三角形作旋转或平移,可使 \(P\) 转换为矩形 \(Q\),且有 \(m(P)=m(Q)=|P| = 2|T|\),从而得 \(m(T)=|T|\).
\end{proof}

\begin{example}
圆盘 \(D = \{ (x,y): x^2 + y^2 \leqslant r^2\}\) 是 \(\mathbb{R}^2\) 中可测集,且 \(m(D)=\pi r^2\).
\end{example}
\begin{proof}
记 \(P_n\) 与 \(Q_n\) 为 \(D\) 的内接与外切正 \(n\) 边形,由 \(P_n\) 与 \(Q_n\) 的可测性易知 \(D\) 是可测集. 注意到 \(P_n \subset D \subset Q_n\),以及
\begin{align*}
m(P_n) &= \pi r^2 \frac{\sin(\pi/n)}{\pi/n} \cos\frac{\pi}{n} \to \pi r^2 \quad (n \to \infty),\\
m(Q_n) &= \pi r^2 \frac{\tan(\pi/n)}{\pi/n} \to \pi r^2 \quad (n \to \infty),
\end{align*}
可知 \(m(D)=\pi r^2\).
\end{proof}

\begin{example}\label{example:极坐标变换与可测性}
设 \(E \subset (-\pi, \pi]\),\(0 \leqslant a < b \leqslant +\infty\),令
\[
S_E = S_E(a,b)=\{(r\cos\theta, r\sin\theta): a < r < b, \theta \in E\}.
\]
大家知道,若 \(E = (\alpha, \beta)\),则 \(S_E\) 就是通常所说的扇形,其面积为 \((b^2 - a^2)(\beta - \alpha)/2\).

(1) 对于一般点集 \(E\),我们有
\[
m^*(S) \leqslant \frac{(b^2 - a^2)m^*(E)}{2}.
\]
(注意,这里 \(m^*(S)\) 是二维外测度,\(m^*(E)\) 是一维外测度. )

(2) 若 \(E \subset (-\pi, \pi]\) 是可测集,则 \(S\) 是可测集.
\end{example}
\begin{proof}
(1) (i) 设 \(b < +\infty\),此时,对任给 \(\varepsilon > 0\),存在开区间列 \(\{I_n\}\):\(\bigcup_{n = 1}^{\infty} I_n \supset E\),\(\sum_{n = 1}^{\infty}|I_n| < m^*(E) + \varepsilon\). 显然,\(\bigcup_{n = 1}^{\infty} S_{I_n} \supset S_E\),从而有
\begin{align*}
m^*(S_E) &\leqslant m^*\left(\bigcup_{n = 1}^{\infty} S_{I_n}\right) \leqslant \sum_{n = 1}^{\infty}m^*(S_{I_n})\\
&= (b^2 - a^2)\sum_{n = 1}^{\infty}|I_n|/2 \leqslant \frac{b^2 - a^2}{2}(m^*(E) + \varepsilon),
\end{align*}
由 \(\varepsilon\) 的任意性即得所证.

(ii) 设 \(b = +\infty\),\(m^*(E) = 0\). 此时,对 \(n \geqslant 1\),由 (i) 知
\[
m^*(S_E(a,n)) \leqslant \frac{(n^2 - a^2)m^*(E)}{2} = 0.
\]
从而得到
\[
m^*(S_E(a, +\infty)) = \lim_{n \to \infty}m^*(S_E(n)) = 0.
\]

(iii) 设 \(b = +\infty\),\(m^*(E) > 0\). 结论显然.

(2) 由于 \(S_E(a,b) = S_E(0, +\infty) \cap S_{(-\pi, \pi]}(a,b)\),故只需指出 \(S_E(0, +\infty)\) 可测即可.

设 \(I \subset (-\pi, \pi]\) 是开区间,记 \(T = S_I(a,b)\)(开环扇形),\(E^c = (-\pi, \pi] \setminus E\) 以及 \(S_E = S_E(0, +\infty)\),我们有
\begin{align*}
m^*(T \cap S_E) + m^*(T \cap S_{E^c}) 
&= m^*(S_{I \cap E}(a,b)) + m^*(S_{I \cap E^c}(a,b))\\
&\leqslant \frac{b^2 - a^2}{2}\{m^*(I \cap E) + m^*(I \cap E^c)\}\\
&= \frac{b^2 - a^2}{2}|I| = m(T) \text{(开环扇形面积)}.
\end{align*}

设 \(R\) 是一个开矩形,易知它可由互不相交的可列个开环扇形 \(T_n\) 组成,至多差一零测集(边界). 因此(注意,开环扇形可测)得到
\begin{align*}
m^*(R \cap S_E) + m^*(R \cap S_{E^c}) 
&\leqslant \sum_{n = 1}^{\infty}m^*(T_n \cap S_E) + \sum_{n = 1}^{\infty}m^*(T_n \cap S_{E^c})\\
&\leqslant \sum_{n = 1}^{\infty}m(T_n) = m\left(\bigcup_{n = 1}^{\infty} T_n\right) = m(R).
\end{align*}
这说明,对任一矩形 \(R\),有
\[
m(R) = m^*(R \cap S_E) + m^*(R \cap S_{E^c}).
\]
而 \(S_{E^c}\) 就是 \(S_E\) 的补集(除原点外),也就是说 \(S_E\) 是可测集. 
\end{proof}






































































\end{document}
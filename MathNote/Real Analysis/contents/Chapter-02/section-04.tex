\documentclass[../../main.tex]{subfiles}
\graphicspath{{\subfix{../../image/}}} % 指定图片目录,后续可以直接使用图片文件名。

% 例如:
% \begin{figure}[H]
% \centering
% \includegraphics{image-01.01}
% \caption{图片标题}
% \label{figure:image-01.01}
% \end{figure}
% 注意:上述\label{}一定要放在\caption{}之后,否则引用图片序号会只会显示??.

\begin{document}

\section{正测度集与矩体的关系}

\begin{theorem}\label{theorem:定理2.19}
设\(E\)是\(\mathbb{R}^n\)中的可测集,且\(m(E)>0\),\(0 < \lambda < 1\),则存在矩体\(I\),使得
\begin{align}
\lambda|I|<m(I\cap E).\label{eq:2.8}
\end{align}
\end{theorem}
\begin{remark}
上述定理告诉我们,任何一个正测集,其中总有一部分被一个矩体套住,使两者的测度差小于预先给定的正数\(\varepsilon\)。当然,这一测度差不一定能等于零。 
\end{remark}
\begin{proof}
{\heiti 情形I:}当\(m(E)<+\infty\)时,对于\(0 < \varepsilon < (\lambda^{-1}-1)m(E)\),作\(E\)的\(L\)-覆盖\(\{I_k\}\),使得
\begin{align*}
\sum_{k = 1}^{\infty}|I_k|<m(E)+\varepsilon.
\end{align*}
从而存在\(k_0\),使得\(\lambda|I_{k_0}|<m(I_{k_0}\cap E)\)。事实上,若对一切\(k\),有
\begin{align*}
\lambda|I_k|\geqslant m(I_k\cap E),
\end{align*}
则可得
\begin{align*}
m(E)=m(E\cap \bigcup_{k=1}^{\infty}I_k)\leqslant\sum_{k = 1}^{\infty}m(I_k\cap E)\leqslant\lambda\sum_{k = 1}^{\infty}|I_k|\leqslant\lambda(m(E)+\varepsilon)<m(E).
\end{align*}
这就导致\(m(E)<m(E)\),产生矛盾。

{\heiti 情形II:}当$m(E)=+\infty$时,由\nrefthe{theorem:定理2.13}{(ii)}可知,存在闭集$F\subset E$,使得$m(E\backslash F)<1$.记$H=E\backslash F$,则$m(H)<1$且$H\subset E$.于是由{\heiti 情形I}可知,存在矩体$I$,使得
\begin{align*}
\lambda |I|<m(I\cap H).
\end{align*}
再由$I\cap H\subset I\cap E$及测度的单调性可得
\begin{align*}
\lambda |I|<m(I\cap H)\leqslant m(I\cap E).
\end{align*}
故结论得证.
\end{proof}

\begin{example}
\([0,1]\)中存在正测集\(E\),使对\([0,1]\)中任一开区间\(I\),有
\begin{align*}
0 < m(E\cap I) < m(I).
\end{align*}
\end{example}
\begin{solution}
首先,在\([0,1]\)中作类Cantor集\(H_1\):\(m(H_1)=1/2\)。其次,在\([0,1]\)中\(H_1\)的邻接区间\(\{I_{1j}\}\)的每个\(I_{1j}\)内再作类Cantor集\(H_{1j}\):
\(m(H_{1j}) = |I_{1j}|/2^2\),并记\(H_2 = \bigcup_{j = 1}^{\infty}H_{1j}\)。然后,对\(H_1\cup H_2\)的邻接区间\(\{I_{2j}\}\)的每个\(I_{2j}\),又作类Cantor集\(H_{2j}\):\(m(H_{2j}) = |I_{2j}|/2^3\)。再记\(H_3 = \bigcup_{j = 1}^{\infty}H_{2j}\),依次继续进行,则得\(\{H_m\}\)。令\(E = \bigcup_{n = 1}^{\infty}H_n\),得证。 
\end{solution}

\begin{definition}[向量差集]\label{definition:向量差集}
设$A,B$为两个非空集合,定义$A,B$的\textbf{向量差集}为
\begin{align*}
A - B \stackrel{\text{def}}{=} \{x - y: x\in A,y \in B\}.
\end{align*}
\end{definition}

\begin{theorem}[Steinhaus定理]\label{theorem:Steinhaus定理}
设\(E\)是\(\mathbb{R}^n\)中的可测集,且\(m(E)>0\)。作(向量差)点集
\begin{align*}
E - E \stackrel{\text{def}}{=} \{x - y: x,y \in E\},
\end{align*}
则存在\(\delta_0>0\),使得\(E - E \supset B(0,\delta_0)\)。
\end{theorem}
\begin{proof}
取\(\lambda\)满足\(1 - 2^{-(n + 1)}<\lambda<1\)$(n\geq 2)$。由\refthe{theorem:定理2.19}可知,存在矩体\(I\),使得\(\lambda|I|<m(I\cap E)\)。现在记\(I\)的最短边长为\(\delta\),并作开矩体
\begin{align*}
J = \left\{x = (\xi_1,\xi_2,\cdots,\xi_n): |\xi_i|<\frac{\delta}{2}\ (i = 1,2,\cdots,n)\right\}.
\end{align*}
从而只需证明\(J\subset E - E\)即可(在$J$中任取一个以原点为中心的开球$B(0,\delta_0)$),也就是只要证明对每个\(x_0\in J\),点集\(E\cap I\)必与点集\((E\cap I)+\{x_0\}\)相交(此时任取$y\in (E\cap I)\cap((E\cap I)+\{x_0\})$,从而存在$z\in E\cap I$,使得$y=z+x_0$.也即存在\(y,z\in E\cap I\subset E\),使得\(y - z = x_0\))即可。因为\(J\)是以原点为中心,边长为\(\delta\)的开矩阵,所以\(I\)的平移矩体\(I+\{x_0\}\)仍含有\(I\)的中心,从而知
\begin{align*}
m(I\cap (I+\{x_0\}))>2^{-n}|I|(n\geq 2).
\end{align*}
结合上式,再由\nrefthe{theorem:测度的基本性质}{(4)}可得
\begin{align*}
m(I\cup (I+\{x_0\}))=|I|+m(I+\{x_0\})-m(I\cap (I+\{x_0\}))
<2|I|-2^{-n}|I|,
\end{align*}
即
\begin{align*}
m(I\cup (I+\{x_0\}))<2\lambda|I|.
\end{align*}
但由于\(E\cap I\)与\((E\cap I)+\{x_0\}\)有着相同的测度并且都大于\(\lambda|I|\),同时又都含于\(I\cup (I+\{x_0\})\)之中,故它们必定相交,否则其并集测度要大于\(2\lambda|I|\),从而引起矛盾。
\end{proof} 

\begin{proposition}
设有定义在 $\mathbb{R}$ 上的函数 $f(x)$, 满足
\begin{align*}
f(x + y) = f(x) + f(y),\quad x,y\in\mathbb{R},
\end{align*}
且在 $E\subset\mathbb{R}$ ($m(E)>0$) 上有界, 则 $f(x)=cx$ ($x\in\mathbb{R}$), 其中 $c = f(1)$.

\end{proposition}
\begin{proof}
(i)首先, 由题设知, 对 $r\in\mathbb{Q}$, 必有 $f(r)=rf(1)$.

(ii)其次, 由 $m(E)>0$ 可知, 存在区间 $I$: $I\subset E - E$. 不妨设 $|f(x)|\leqslant M$ ($x\in E$), 又对任意的 $x\in I$, 有 $x',x''\in E$, 使得 $x = x' - x''$, 则
\[
|f(x)| = |f(x') - f(x'')|\leqslant |f(x')| + |f(x'')|\leqslant 2M.
\]

记 $I = [a,b]$, 并考查 $[0,b - a]$. 若 $x\in [0,b - a]$, 则 $x + a\in [a,b]$. 从而由 $f(x) = f(x + a) - f(a)$ 可知, $|f(x)|\leqslant 4M$, $x\in [0,b - a]$. 记 $b - a = c$, 这说明
\begin{align*}
|f(x)|\leqslant 4M,\quad x\in [0,c].
\end{align*}
易知
\begin{align*}
|f(x)|\leqslant 4M,\quad x\in [-c,c].
\end{align*}

已知对任意的 $x\in\mathbb{R}$ 以及自然数 $n$, 均存在有理数 $r$, 使得 $|x - r|<c/n$, 因此我们得到
\begin{align*}
|f(x) - xf(1)|&=|f(x - r) + rf(1) - xf(1)|\\
&=|f(x - r) + (r - x)f(1)|\leqslant\frac{4M + c|f(1)|}{n}.
\end{align*}
根据 $n$ 的任意性 ($r$ 的任意性), 即得 $f(x)=xf(1)$. 
\end{proof}




















\end{document}
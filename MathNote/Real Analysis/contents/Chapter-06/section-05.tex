\documentclass[../../main.tex]{subfiles}
\graphicspath{{\subfix{../../image/}}} % 指定图片目录,后续可以直接使用图片文件名。

% 例如:
% \begin{figure}[H]
% \centering
% \includegraphics[scale=0.3]{image-01.01}
% \caption{图片标题}
% \label{figure:image-01.01}
% \end{figure}
% 注意:上述\label{}一定要放在\caption{}之后,否则引用图片序号会只会显示??.

\begin{document}

\section{卷积}






\begin{definition}[展缩函数]\label{definition:展缩函数}
设 \( K(x) \) 是定义在 \( \mathbb{R}^n \) 上的函数,\( \varepsilon > 0 \),令
\[
K_{\varepsilon}(x) = \varepsilon^{-n} K\left( \frac{x}{\varepsilon} \right) = \varepsilon^{-n} K\left( \frac{x_1}{\varepsilon}, \frac{x_2}{\varepsilon}, \cdots, \frac{x_n}{\varepsilon} \right),
\]
称 \( K_{\varepsilon}(x) \) 为 \( K(x) \) 的\textbf{展缩函数}.
\end{definition}



\end{document}
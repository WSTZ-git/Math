\documentclass[../../main.tex]{subfiles}
\graphicspath{{\subfix{../../image/}}} % 指定图片目录,后续可以直接使用图片文件名。

% 例如:
% \begin{figure}[H]
% \centering
% \includegraphics[scale=0.4]{图.png}
% \caption{}
% \label{figure:图}
% \end{figure}
% 注意:上述\label{}一定要放在\caption{}之后,否则引用图片序号会只会显示??.

\begin{document}

\section{常见问题}

\begin{example}
已知椭球面
$$
\Sigma_0:\frac{x^2}{a^2}+\frac{y^2}{b^2}+\frac{z^2}{c^2}=1,a>b,$$
的外切柱面$\Sigma_\varepsilon$($\varepsilon=1$或$-1$)平行于已知直线
$$
l_\varepsilon:\frac{x-2}{0}=\frac{y-1}{\varepsilon\sqrt{a^2-b^2}}=\frac{z-3}{c}.
$$
试求与$\Sigma_\varepsilon$交于一个圆周的平面的所有可能的法方向.

注:本题中的外切柱面指的是每一条直母线均与已知椭球面相切的柱面.
\end{example}
\begin{solution}
设$M_1(x_1,y_1,z_1)$为$\Sigma_0$与$\Sigma_\varepsilon$的切点,则由条件可设$\Sigma_\varepsilon$上过$M_1$点的直母线为
\begin{align}
l:\begin{cases}
x=x_1\\
y=y_1+\varepsilon \sqrt{a^2-b^2}t\\
z=z_1+ct
\end{cases}, \quad t\in \mathbb{R}. \label{eq::::-=--2389tj34g34g3432r221.4}
\end{align}
将上式与$\Sigma_0$的方程联立得
\begin{align}
\frac{x_1^2}{a^2}+\frac{(y_1+\varepsilon \sqrt{a^2-b^2}t)^2}{b^2}+\frac{(z_1+ct)^2}{c^2}=1. \label{eq::::-=--2389tj34g34g3432r221.1}
\end{align}
又因为$M_1\in \Sigma_0$,所以
\begin{align}
\frac{x_1^2}{a^2}+\frac{y_1^2}{b^2}+\frac{z_1^2}{c^2}=1. \label{eq::::-=--2389tj34g34g3432r221.2}
\end{align}
再联立\eqref{eq::::-=--2389tj34g34g3432r221.1}\eqref{eq::::-=--2389tj34g34g3432r221.2}式可得
\begin{align*}
\frac{a^2}{b^2}t^2+\left( \frac{2\varepsilon \sqrt{a^2-b^2}}{b^2}y_1+\frac{2}{c}z_1 \right) t=0.
\end{align*}
由$l$与$\Sigma_0$相切可知上述方程只有$t=0$这一个解,故
\begin{align}
\frac{2\varepsilon \sqrt{a^2-b^2}}{b^2}y_1+\frac{2}{c}z_1=0. \label{eq::::-=--2389tj34g34g3432r221.3}
\end{align}
再将$\Sigma_0$的方程与\eqref{eq::::-=--2389tj34g34g3432r221.3}\eqref{eq::::-=--2389tj34g34g3432r221.4}式联立消去$x_1,y_1,z_1,t$得
\begin{align*}
\Sigma_\varepsilon:\frac{x^2}{a^2}+\frac{(cy-\varepsilon \sqrt{a^2-b^2}z)^2}{a^2c^2}=1.
\end{align*}
再设与$\Sigma_\varepsilon$交于一个圆周的平面为
\begin{align*}
\pi:Ax+By+Cz=0.
\end{align*}
记$K=\sqrt{A^2+B^2+C^2}$,现在做如下正交变换
\begin{align*}
\left[ \begin{array}{c}
x'\\
y'\\
z'\\
\end{array} \right] =\left[ \begin{matrix}
\frac{AC}{K^2}&		\frac{BC}{K^2}&		\frac{-A^2-B^2}{K^2}\\
-\frac{B}{K}&		\frac{A}{K}&		0\\
\frac{A}{K}&		\frac{B}{K}&		\frac{C}{K}\\
\end{matrix} \right] \left[ \begin{array}{c}
x\\
y\\
z\\
\end{array} \right] 
\end{align*}
将$Oxyz$坐标系变为$Ox'y'z'$坐标系。在新坐标系$Ox'y'z'$下
\begin{align*}
\pi:z'=0.
\end{align*}
\begin{align*}
\Sigma_\varepsilon:\frac{\left( \frac{AC}{K^2}x' -\frac{B}{K}y' +\frac{A}{K}z' \right) ^2}{a^2}+\frac{1}{a^2c^2}\left[ c\left( \frac{BC}{K^2}x' +\frac{A}{K}y' +\frac{B}{K}z' \right) -\varepsilon \sqrt{a^2-b^2}\left( \frac{-A^2-B^2}{K^2}x' +\frac{C}{K}z'\right) \right] ^2=1.
\end{align*}
故
\begin{align*}
\pi \cap \Sigma_\varepsilon:\frac{\left( \frac{AC}{K^2}x' -\frac{B}{K}y' \right) ^2}{a^2}+\frac{1}{a^2c^2}\left[ c\left( \frac{BC}{K^2}x' +\frac{A}{K}y' \right) -\varepsilon \sqrt{a^2-b^2}\left( \frac{-A^2-B^2}{K^2}x'\right) \right] ^2=1.
\end{align*}
注意到交线$\pi \cap \Sigma_\varepsilon$为圆周的充要条件是上式中的$x'^2,y'^2$的系数相等且不为$0$,$x'y'$的系数为$0$。此即
\begin{align*}
\begin{cases}
\frac{2\varepsilon \sqrt{a^2-b^2}}{c}\cdot \frac{A}{C}\left( 1+\frac{\varepsilon \sqrt{a^2-b^2}}{c}\cdot \frac{B}{C} \right) =0\\
1+\frac{4(a^2-b^2)}{c^2}\cdot \frac{A^2}{C^2}=\left( 1+\frac{\varepsilon \sqrt{a^2-b^2}}{c}\cdot \frac{B}{C} \right) ^2\ne 0
\end{cases},
\end{align*}
上式也等价于
\begin{align*}
\begin{cases}
\frac{2\varepsilon \sqrt{a^2-b^2}}{c}\cdot A\left( C+\frac{\varepsilon \sqrt{a^2-b^2}}{c}\cdot B \right) =0\\
C^2+\frac{4(a^2-b^2)}{c^2}\cdot A^2=\left( C+\frac{\varepsilon \sqrt{a^2-b^2}}{c}\cdot B \right) ^2\ne 0
\end{cases},
\end{align*}
从而$A=0$,此时
\begin{align*}
C^2=\left( C+\frac{\varepsilon \sqrt{a^2-b^2}}{c}\cdot B \right) ^2&\Longleftrightarrow C+\frac{\varepsilon \sqrt{a^2-b^2}}{c}\cdot B=\pm C
\\
&\Longleftrightarrow \frac{\varepsilon \sqrt{a^2-b^2}}{c}\cdot B=0\text{或}-2C
\\
&\Longleftrightarrow B=0\text{或}\frac{B}{C}=-\frac{2c}{\varepsilon \sqrt{a^2-b^2}}.
\end{align*}
故$\pi$的法向量只可能为
$(0,0,1)$或$(0,-2c,\varepsilon \sqrt{a^2-b^2})$。

\end{solution}








\end{document}
\documentclass[../../main.tex]{subfiles}
\graphicspath{{\subfix{../../image/}}} % 指定图片目录,后续可以直接使用图片文件名。

% 例如:
% \begin{figure}[H]
% \centering
% \includegraphics[scale=0.4]{图.png}
% \caption{}
% \label{figure:图}
% \end{figure}
% 注意:上述\label{}一定要放在\caption{}之后,否则引用图片序号会只会显示??.

\begin{document}

\section{高等代数公理化定义}

\begin{theorem}[行列式的刻画]\label{theorem:行列式的刻画}
设$f$为从$n$阶方阵全体构成的集合到数集上的映射,使得对任意的$n$阶方阵$\boldsymbol{A}$,任意的指标$1\leqslant  i\leqslant  n$,以及任意的常数$c$,满足下列条件:

(1) 设$\boldsymbol{A}$的第$i$列是方阵$\boldsymbol{B}$和$\boldsymbol{C}$的第$i$列之和,且$\boldsymbol{A}$的其余列与$\boldsymbol{B}$和$\boldsymbol{C}$的对应列完全相同,则$f(\boldsymbol{A})=f(\boldsymbol{B})+f(\boldsymbol{C})$;

(2) 将$\boldsymbol{A}$的第$i$列乘以常数$c$得到方阵$\boldsymbol{B}$,则$f(\boldsymbol{B})=cf(\boldsymbol{A})$;

(3) 对换$\boldsymbol{A}$的任意两列得到方阵$\boldsymbol{B}$,则$f(\boldsymbol{B})= - f(\boldsymbol{A})$;

(4) $f(\boldsymbol{I}_n)=1$,其中$\boldsymbol{I}_n$是$n$阶单位阵.

求证:$f(\boldsymbol{A})=\vert \boldsymbol{A}\vert$.
\end{theorem}
\begin{note}
这个命题给出了\textbf{行列式的刻画}:在方阵\(n\)个列向量上的多重线性和反对称性,以及正规性(即单位矩阵处的取值为\(1\)),唯一确定了行列式这个函数.
\end{note}
\begin{proof}
设\(\boldsymbol{A} = (\boldsymbol{\alpha}_1,\boldsymbol{\alpha}_2,\cdots,\boldsymbol{\alpha}_n)\),其中\(\boldsymbol{\alpha}_k\)为\(\boldsymbol{A}\)的第\(k\)列,\(\boldsymbol{e}_1,\boldsymbol{e}_2,\cdots,\boldsymbol{e}_n\)为标准单位列向量,则
\begin{align*}
\boldsymbol{\alpha}_j = a_{1j}\boldsymbol{e}_1 + a_{2j}\boldsymbol{e}_2 + \cdots + a_{nj}\boldsymbol{e}_n = \sum_{k = 1}^{n}a_{kj}\boldsymbol{e}_k,j = 1,2,\cdots,n.
\end{align*}
从而由条件\((1)\)和\((2)\)可得
\begin{align*}
&f\left( \boldsymbol{A} \right) =f\left( \boldsymbol{\alpha }_1,\boldsymbol{\alpha }_2,\cdots ,\boldsymbol{\alpha }_n \right) =f\left( \sum_{k_1=1}^n{a_{k_11}\boldsymbol{e}_k},\boldsymbol{\alpha }_2,\cdots ,\boldsymbol{\alpha }_n \right) 
\\
&=a_{11}f\left( \boldsymbol{e}_1,\boldsymbol{\alpha }_2,\cdots ,\boldsymbol{\alpha }_n \right) +a_{21}f\left( \boldsymbol{e}_2,\boldsymbol{\alpha }_2,\cdots ,\boldsymbol{\alpha }_n \right) +\cdots +a_{n1}f\left( \boldsymbol{e}_n,\boldsymbol{\alpha }_2,\cdots ,\boldsymbol{\alpha }_n \right) 
\\
&=\sum_{k_1=1}^n{a_{k_11}f\left( \boldsymbol{e}_{k_1},\boldsymbol{\alpha }_2,\cdots ,\boldsymbol{\alpha }_n \right)}=\sum_{k_1=1}^n{a_{k_11}f\left( \boldsymbol{e}_{k_1},\sum_{k_2=1}^n{a_{k_22}\boldsymbol{e}_{k_2}},\cdots ,\boldsymbol{\alpha }_n \right)}
\\
&=\sum_{k_1=1}^n{a_{k_11}\left[ a_{12}f\left( \boldsymbol{e}_{k_1},\boldsymbol{e}_1,\cdots ,\boldsymbol{\alpha }_n \right) +a_{22}f\left( \boldsymbol{e}_{k_1},\boldsymbol{e}_2,\cdots ,\boldsymbol{\alpha }_n \right) +\cdots +a_{n2}f\left( \boldsymbol{e}_{k_1},\boldsymbol{e}_n,\cdots ,\boldsymbol{\alpha }_n \right) \right]}
\\
&=\sum_{k_1=1}^n{a_{k_11}\sum_{k_2=1}^n{a_{k_22}}f\left( \boldsymbol{e}_{k_1},\boldsymbol{e}_{k_2},\cdots ,\boldsymbol{\alpha }_n \right)}=\cdots =\sum_{k_1=1}^n{a_{k1}\sum_{k_2=1}^n{a_{k_22}}\cdots \sum_{k_n=1}^n{a_{k_nn}f\left( \boldsymbol{e}_{k_1},\boldsymbol{e}_{k_2},\cdots ,\boldsymbol{e}_{k_n} \right)}}
\\
&=\sum_{k_1=1}^n{\sum_{k_2=1}^n{\cdots \sum_{k_n=1}^n{a_{k1}a_{k_22}\cdots a_{k_nn}f\left( \boldsymbol{e}_{k_1},\boldsymbol{e}_{k_2},\cdots ,\boldsymbol{e}_{k_n} \right)}}}=\sum_{\left( k_1,k_2,\cdots ,k_n \right)}{a_{k_11}a_{k_22}\cdots a_{k_nn}f\left( \boldsymbol{e}_{k_1},\boldsymbol{e}_{k_2},\cdots ,\boldsymbol{e}_{k_n} \right)}.
\end{align*}
若\(k_i = k_j\),则\((\boldsymbol{e}_{k_1},\boldsymbol{e}_{k_2},\cdots,\boldsymbol{e}_{k_n})\)的第\(i\)列和第\(j\)列对换后仍然是\((\boldsymbol{e}_{k_1},\boldsymbol{e}_{k_2},\cdots,\boldsymbol{e}_{k_n})\).由条件\((3)\)可知,\(f(\boldsymbol{e}_{k_1},\boldsymbol{e}_{k_2},\cdots,\boldsymbol{e}_{k_n}) = -f(\boldsymbol{e}_{k_1},\boldsymbol{e}_{k_2},\cdots,\boldsymbol{e}_{k_n})\),于是\(f(\boldsymbol{e}_{k_1},\boldsymbol{e}_{k_2},\cdots,\boldsymbol{e}_{k_n}) = 0\).
因此在\(f(\boldsymbol{A})\)的表示式中,只剩下\(k_i\)(\(i = 1,2,\cdots,n\))互不相同的项.
通过\(\tau(k_1k_2\cdots k_n)\)次相邻对换可将\((\boldsymbol{e}_{k_1},\boldsymbol{e}_{k_2},\cdots,\boldsymbol{e}_{k_n})\)变成\((\boldsymbol{e}_1,\boldsymbol{e}_2,\cdots,\boldsymbol{e}_n) = \boldsymbol{I}_n\),
故由条件\((3)\)和\((4)\)可得
\begin{align*}
f(\boldsymbol{e}_{k_1},\boldsymbol{e}_{k_2},\cdots,\boldsymbol{e}_{k_n}) = (-1)^{\tau(k_1k_2\cdots k_n)}f(\boldsymbol{I}_n) = (-1)^{\tau(k_1k_2\cdots k_n)}.
\end{align*}
于是由行列式的组合定义可知
\begin{align*}
f(\boldsymbol{A}) = \sum_{(k_1,k_2,\cdots,k_n)}a_{k_11}a_{k_22}\cdots a_{k_nn}f(\boldsymbol{e}_{k_1},\boldsymbol{e}_{k_2},\cdots,\boldsymbol{e}_{k_n}) = \sum_{(k_1,k_2,\cdots,k_n)}(-1)^{\tau(k_1k_2\cdots k_n)}a_{k_11}a_{k_22}\cdots a_{k_nn} = |\boldsymbol{A}|.
\end{align*}

\end{proof}

\begin{theorem}[矩阵迹的刻画]\label{theorem:矩阵迹的刻画}
设 \( K \) 为数域, \( f: M_n(K) \to K \) 为一个映射, 且满足

(1) \( \forall \boldsymbol{A}, \boldsymbol{B} \in M_n(K), f(\boldsymbol{A} + \boldsymbol{B}) = f(\boldsymbol{A}) + f(\boldsymbol{B}) \);

(2) \( \forall k \in K, \boldsymbol{A} \in M_n(K), f(k\boldsymbol{A}) = kf(\boldsymbol{A}) \);

(3) \( \forall \boldsymbol{A}, \boldsymbol{B} \in M_n(K), f(\boldsymbol{AB}) = f(\boldsymbol{BA}) \);

(4) \( f(\boldsymbol{I}_n) = n \).

求证:\(f\)就是迹映射,即\(f(A)=\mathrm{tr}(A)\)对一切\(\mathbb{F}\)上\(n\)阶矩阵\(A\)成立.
\end{theorem}
\begin{note}
这个命题给出了迹的刻画,它告诉我们迹函数由线性、交换性和正规性(即单位矩阵处的取值为其阶数)唯一决定.
\end{note}
\begin{proof}
设\(E_{ij}\)是\(n\)阶基础矩阵.由(1)和(4),有
\[
n = f(I_n)=f(E_{11}+E_{22}+\cdots+E_{nn})=f(E_{11})+f(E_{22})+\cdots+f(E_{nn}).
\]
又由(3),有
\[
f(E_{ii})=f(E_{ij}E_{ji})=f(E_{ji}E_{ij})=f(E_{jj}),
\]
所以\(f(E_{ii}) = 1(1\leqslant  i\leqslant  n)\).另一方面,若\(i\neq j\),则
\[
f(E_{ij})=f(E_{i1}E_{1j})=f(E_{1j}E_{i1})=f(O)=f(0\cdot I_n)=0\cdot f(I_n)=0.
\]
设\(n\)阶矩阵\(A=(a_{ij})\),则
\[
f(A)=f\left(\sum_{i,j = 1}^{n}a_{ij}E_{ij}\right)=\sum_{i,j = 1}^{n}a_{ij}f(E_{ij})=\sum_{i = 1}^{n}a_{ii}=\mathrm{tr}(A).
\]

\end{proof}

\begin{theorem}
设$\phi: \mathbb{R}^{n \times n} \to \mathbb{R}^{n \times n}$满足

(i) $\phi(AB) = \phi(A)\phi(B), \forall A,B \in \mathbb{R}^{n \times n}$.

(ii)$\phi(0) = 0$, 且存在秩1矩阵$W$使得$\phi(W) \neq 0$.

证明
\begin{enumerate}[(1)]
\item 对任意的秩为$r$的矩阵$A,B$,都有$r(A)=r(B).$

\item 存在可逆矩阵$R \in \mathbb{R}^{n \times n}$使得
\begin{align}
\phi(E_{ij}) = R E_{ij} R^{-1}, \forall i,j = 1,2,\cdots,n. \label{eq:23.81}
\end{align}
\end{enumerate}
\end{theorem}
\begin{note}
本题对一切数域都成立.

证明的想法是类比相似矩阵的定理的证明. 
\end{note}
\begin{proof}
\begin{enumerate}[(1)]
\item 注意到设$A,B \in \mathbb{R}^{n \times n}$且秩一样, 则存在可逆矩阵$P,Q \in \mathbb{R}^{n \times n}$使得$PAQ = B$. 则
$\operatorname{rank}(\phi(B)) = \operatorname{rank}(\phi(PAQ)) = \operatorname{rank}(\phi(P)\phi(A)\phi(Q)) \leq \operatorname{rank}(\phi(A)).$
对称的有$\operatorname{rank}(\phi(A)) \leq \operatorname{rank}(\phi(B))$. 这就证明了$\phi(A),\phi(B)$秩是相同的.

\item 我们特定$R = (v_1,v_2,\cdots,v_n)$. 等式\eqref{eq:23.81}等价于
\begin{align*}
\phi (E_{ij})(v_1,v_2,\cdots ,v_n)=(v_1,v_2,\cdots ,v_n)E_{ij}=\left( 0,\cdots ,\underset{\text{第}j\text{列}}{\underbrace{v_i}},\cdots ,0 \right) .
\end{align*}

我们的目标转化为寻求一组基$v_i$使得
\begin{align}
\phi(E_{ij})v_k = \delta_{jk}v_i. \label{eq:23.82}
\end{align}

特别的, 在\eqref{eq:23.82}中考虑$k = j = k$, 则需要的是
$\phi(E_{ii})v_i = v_i, i = 1,2,\cdots,n.$

我们知道$E_{ij}E_{ki} = \delta_{jk}E_{ii}, \phi(E_{ii}^2) = \phi(E_{ii}) = \phi^2(E_{ii})$, 即$\phi(E_{ii})$是幂等矩阵. 注意到$E_{ii}$的秩为1, 于是有
$$\operatorname{rank}(\phi(E_{ii})) = \operatorname{rank}(\phi(W)) > 0,$$

即$\phi(E_{ii})$有特征值1. 现在就可以取$v_i$是$\phi(E_{ii})$属于特征值1的特征向量. 问题变为如此取的$v_i$是否满足\eqref{eq:23.82}?
如果$\sum_{i=1}^n c_i v_i = 0$, 则
$$\phi(E_{kk})\left(\sum_{i=1}^n c_i v_i\right) = 0 = c_k \phi(E_{kk})v_k = c_k v_k \implies c_k = 0,$$
即$v_i, i = 1,2,\cdots,n$线性无关.
注意到当$k \neq j$, 有
$$\phi(E_{ij})v_k = \phi(E_{ij})\phi(E_{kk})v_k = 0.$$
当$k = j$, 有
$$\phi(E_{ij})v_k = \phi(E_{ij})\phi(E_{kk})v_k = \phi(E_{ik})v_k,$$
这并没有用处, 因此我们需要改造$v_i$.

现在设$\phi(E_{ij})v_j = t_{i1}v_1 + t_{i2}v_2 + \cdots + t_{in}v_n$, 并取$k \neq i$, 则有
$$\phi(E_{kk})\phi(E_{ij})v_j = t_{ik}\phi(E_{kk})v_k = t_{ik}v_k = 0 \implies t_{ik} = 0.$$

于是可设$\phi(E_{ij})v_j = r_{ij}v_i$. 我们待定非0的$c_i \in \mathbb{R}$, 注意$c_i v_i$仍然是$\phi(E_{ii})$特征值1对应的特征向量, 所以之前推导的性质并没有改变. 现在期望$\phi(E_{ij})c_j v_j = c_i v_i$, 即$r_{ij} = \frac{c_i}{c_j}$. 那么$r_{ij}$能否表示为这种形式呢? 这诱使我们继续考虑$r_{ij}$的样子.
注意到
\begin{align*}
\phi(E_{il})\phi(E_{lj})v_j = \phi(E_{il})r_{lj}v_l = r_{il}r_{lj}v_i = r_{ij}v_i \implies r_{ij} = r_{il}r_{lj},
\end{align*}
即$r_{ij} = \frac{r_{ii}}{r_{ji}}$. 于是问题转化为是否存在$l \in \{1,2,\cdots,n\}$, 使得$r_{il} \neq 0, \forall i \in \{1,2,\cdots,n\}$.
事实上这一定存在, 若不然, 则对任何$l \in \{1,2,\cdots,n\}$, 存在$i_l \in \{1,2,\cdots,n\}$使得$r_{i_l l} = 0$. 于是对任何$i,j \in \{1,2,\cdots,n\}$, 我们有$r_{ij} = r_{i i_l} r_{i_l j} = 0$, 这不可能! 现在我们就证明了存在可逆矩阵$R \in \mathbb{R}^{n \times n}$使得\eqref{eq:23.81}成立.
\end{enumerate}

\end{proof}

\begin{theorem}[基础矩阵的刻画]\label{theorem:非零相似于基础矩阵}
设有 $n^2$ 个 $n$ 阶非零矩阵 $A_{ij}\ (1 \leqslant  i,j \leqslant  n)$,适合
\[
A_{ij}A_{jk} = A_{ik},\ A_{ij}A_{lk} = O\ (j \neq l).
\]
求证:存在可逆矩阵 $P$,使得对任意的 $i,j$,$P^{-1}A_{ij}P = E_{ij}$,其中 $E_{ij}$ 是基础矩阵。
\end{theorem}
\begin{proof}
因为 $A_{11} \neq O$,故存在 $\alpha$,使得 $A_{11}\alpha \neq 0$。令 $\alpha_1 = A_{11}\alpha$,由 $A_{11}A_{11} = A_{11}$ 可得 $A_{11}\alpha_1 = \alpha_1$。再令 $\alpha_i = A_{i1}\alpha_1$,由 $A_{i1}A_{i1} = A_{11}$ 可知 $\alpha_i \neq 0$。我们得到了 $n$ 个非零向量 $\alpha_1,\alpha_2,\cdots,\alpha_n$,由已知条件容易验证这 $n$ 个向量适合下列性质:
\[A_{ij}\alpha_j = \alpha_i,\ A_{ij}\alpha_k = 0\ (j \neq k)\]
由此不难证明这 $n$ 个向量线性无关。令 $P = (\alpha_1,\alpha_2,\cdots,\alpha_n)$,则 $P$ 是可逆矩阵,且
\[
A_{ij}P = (A_{ij}\alpha_1,A_{ij}\alpha_2,\cdots,A_{ij}\alpha_n) = (0,\cdots,0,\alpha_i,0,\cdots,0).
\]
其中上式中的 $\alpha_i$ 在第 $j$ 列。另一方面,有
\[
PE_{ij} = (\alpha_1,\alpha_2,\cdots,\alpha_n)E_{ij} = (0,\cdots,0,\alpha_i,0,\cdots,0).
\]
因此,对任意的 $i,j$,$A_{ij}P = PE_{ij}$,即 $P^{-1}A_{ij}P = E_{ij}$.

\end{proof}





\end{document}
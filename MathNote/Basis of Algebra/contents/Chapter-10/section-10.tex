\documentclass[../../main.tex]{subfiles}
\graphicspath{{\subfix{../../image/}}} % 指定图片目录,后续可以直接使用图片文件名。

% 例如:
% \begin{figure}[H]
% \centering
% \includegraphics[scale=0.4]{图.png}
% \caption{}
% \label{figure:图}
% \end{figure}
% 注意:上述\label{}一定要放在\caption{}之后,否则引用图片序号会只会显示??.

\begin{document}

\section{一些没分类的习题}

\begin{lemma}\label{lemma:引理1564561}
设 $u_1,\dots,u_m$ 和 $v_1,\dots,v_n$ 是两组给定的非零复数,若对任意正整数 $k$ 都有 $u_1^k+\dots+u_m^k=v_1^k+\dots+v_n^k$,则$u_1,\dots,u_m$ 是 $v_1,\dots,v_n$ 的一个排列,进而$m=n$.
\end{lemma}
\begin{proof}
反证,若 $u_1,\cdots ,u_m$ 不是 $v_1,\cdots ,v_n$ 的一个排列,则
若 $u_i=v_j$($i=1,2,\cdots ,m,j=1,2,\cdots ,n$),则将这样的 $u_{i}^{k},v_{j}^{k}$ 从上式中消去得到新式不妨设为
\begin{align}
u_{1}^{k}+\cdots +u_{m'}^{k}=v_{1}^{k}+\cdots +v_{n'}^{k},\forall k\in \mathbb{N} .\label{eq:104.1}
\end{align}
其中 $u_i\ne v_j$($i=1,2,\cdots ,m',j=1,2,\cdots ,n'$).

(i)当 $m=n$ 时,就有 $m' =n' \geqslant slant 1$,否则 $m' =n' =0$,即 $u_1,\cdots ,u_m$ 是 $v_1,\cdots ,v_n$ 的一个排列,矛盾!
(ii)当 $m\ne n$ 时,就有 $m' \geqslant slant 1$ 或 $n' \geqslant slant 1$.
无论(i)还是(ii),都有 $m' +n' \geqslant slant 1$.于是我们可以将 \eqref{eq:104.1} 式中相同的项合并得到新的等式不妨设为
\begin{align*}
c_1u_{1}^{k}+\cdots +c_{m''}u_{m''}^{k}=d_1v_{1}^{k}+\cdots +d_{n''}v_{n''}^{k},\forall k\in \mathbb{N} .
\end{align*}
其中 $u_i,v_j$ 两两互不相同,$c_i,d_j\in \mathbb{N} _1$.取 $k=1,2,\cdots ,m'' +n''$,就有
\begin{align*}
\begin{pmatrix}
1&		\cdots&		1&		1&		\cdots&		1\\
u_1&		\cdots&		u_{m''}&		v_1&		\cdots&		v_{n''}\\
\vdots&		&		\vdots&		\vdots&		&		\vdots\\
u_{1}^{m'' +n''}&		\cdots&		u_{m''}^{m'' +n''}&		v_{1}^{m'' +n''}&		\cdots&		v_{n''}^{m'' +n''}\\
\end{pmatrix} \begin{pmatrix}
c_1\\
\vdots\\
c_{m''}\\
-d_1\\
\vdots\\
-d_{n''}\\
\end{pmatrix} =0\Rightarrow c_1=\cdots =c_{m''}=d_1=\cdots =d_{n''}=0.
\end{align*}
这与 $c_i,d_j\in \mathbb{N} _1$ 矛盾!故 $u_1,\cdots ,u_m$ 就是 $v_1,\cdots ,v_n$ 的一个排列,进而 $m=n$.
\end{proof}

\begin{proposition}\label{proposition:命题24}
设 $A,B$ 是 $n$ 阶复方阵,且对任意正整数 $k$ 有 $$\mathrm{tr}((A+B)^k)=\mathrm{tr}(A^k)+\mathrm{tr}(B^k),$$记 $A$ 的全体特征值为 $\lambda_1,\cdots,\lambda_s,0,\cdots,0$,$B$ 全体特征值为 $\mu_1,\cdots,\mu_t,0,\cdots,0$,这里 $\lambda_i,\mu_j$ 都非零(可以相同),则 $A+B$ 的全体特征值为 $\lambda_1,\cdots,\lambda_s,\mu_1,\cdots,\mu_t,0,\cdots,0$.换句话说,$A+B$ 的全体非零特征值,就是把 $A$ 的和 $B$ 的全体非零特征值拼起来.
\end{proposition}
\begin{proof}
设 $A+B$ 的特征值为 $x_1,\cdots ,x_m,0,\cdots ,0$,则
\begin{align*}
\mathrm{tr}\left( \left( A+B \right) ^k \right) =\mathrm{tr}\left( A^k \right) +\mathrm{tr}\left( B^k \right) ,\forall k\in \mathbb{N} 
\Longleftrightarrow x_{1}^{k}+\cdots +x_{m}^{k}=\lambda _{1}^{k}+\cdots +\lambda _{s}^{k}+\mu _{1}^{k}+\cdots +\mu _{t}^{k},\forall k\in \mathbb{N} .
\end{align*}
于是由\reflem{lemma:引理1564561}可知,$x_1,\cdots ,x_m$ 就是 $\lambda _1,\cdots ,\lambda _s,\mu _1,\cdots ,\mu _t$ 的一个排列.这就完成了证明.
\end{proof}

\begin{example}\label{example:例题611}
设 $A,B$ 是实对称矩阵且对任意正整数 $k$ 有 $\mathrm{tr}\left( \left( A+B \right) ^k \right) =\mathrm{tr}\left( A^k \right) +\mathrm{tr}\left( B^k \right) $,证明:$AB=BA=0$.
\end{example}
\begin{proof}
记 $A$ 的全体特征值为 $\lambda_1,\cdots,\lambda_s,0,\cdots,0$,$B$ 全体特征值为 $\mu_1,\cdots,\mu_t,0,\cdots,0$,这里 $\lambda_i,\mu_j$ 都非零(可以相同),$s,t\in[0,n]\cap \mathbb{N}$.
由实对称矩阵的正交相似标准型可知,存在可逆矩阵 $U$,使得
\begin{align*}
B=U^T\begin{pmatrix}
D_0&		O\\
O&		O\\
\end{pmatrix} U,D_0=\begin{pmatrix}
\mu _1&		&		&		\\
&		\mu _2&		&		\\
&		&		\ddots&		\\
&		&		&		\mu _t\\
\end{pmatrix}.
\end{align*}
显然条件与结论在正交相似变换 $B\rightarrow U^TBU$ 下不变,故可以不妨设 $B=\begin{pmatrix}
D_0&		O\\
O&		O\\
\end{pmatrix}$,$A=\begin{pmatrix}
X&		Y\\
Z&		W\\
\end{pmatrix}$.于是
\begin{align*}
AB=O\Longleftrightarrow \begin{pmatrix}
X&		Y\\
Z&		W\\
\end{pmatrix} \begin{pmatrix}
D_0&		O\\
O&		O\\
\end{pmatrix} =O\Rightarrow XD_0=ZD_0=O\Rightarrow X=Z=O.
\end{align*}
又因为 $A$ 是对称阵,所以 $A=\begin{pmatrix}
O&		O\\
O&		W\\
\end{pmatrix}$.从而
\begin{align*}
BA = \begin{pmatrix}
D_0&		O\\
O&		O\\
\end{pmatrix}\begin{pmatrix}
O&		O\\
O&		W\\
\end{pmatrix} = O = AB.
\end{align*}
因此只需证明$AB=O$.
由\refpro{proposition:命题24}可知,$A+B$的全体特征值为$\lambda _1,\cdots ,\lambda _s,\mu _1,\cdots ,\mu _t,0,\cdots ,0.$显然 $s+t=\mathrm{r}\left( A+B \right) \leqslant slant n$.

(i)若 $s+t=n$,则利用正交相似标准型,不妨设
\begin{align*}
A=U^T\begin{pmatrix}
D_1&		O\\
O&		O\\
\end{pmatrix} U,B=\begin{pmatrix}
O&		O\\
O&		D_2\\
\end{pmatrix},D_1=\begin{pmatrix}
\lambda _1&		&		\\
&		\ddots&		\\
&		&		\lambda _s\\
\end{pmatrix},D_2=\begin{pmatrix}
\mu _1&		&		\\
&		\ddots&		\\
&		&		\mu _t\\
\end{pmatrix},U=\begin{pmatrix}
U_1&		U_2\\
U_3&		U_4\\
\end{pmatrix},
\end{align*}
其中 $U$ 是正交阵.从而
\begin{align*}
A+B=\begin{pmatrix}
U_{1}^{T}&		U_{2}^{T}\\
U_{3}^{T}&		U_{4}^{T}\\
\end{pmatrix} \begin{pmatrix}
D_1&		O\\
O&		O\\
\end{pmatrix} \begin{pmatrix}
U_1&		U_2\\
U_3&		U_4\\
\end{pmatrix} +\begin{pmatrix}
O&		O\\
O&		D_2\\
\end{pmatrix} =\begin{pmatrix}
U_{1}^{T}D_1U_1&		U_{1}^{T}D_1U_2\\
U_{2}^{T}D_1U_1&		U_{2}^{T}D_1U_2+D_2\\
\end{pmatrix}.
\end{align*}
注意到此时 $\left| A+B \right|=\left| D_1 \right|\left| D_2 \right|$,于是
\begin{align*}
\left| A+B \right|=\left| U_{1}^{T}D_1U_1 \right|\left| U_{2}^{T}D_1U_2+D_2-U_{2}^{T}D_1U_1U_{1}^{-1}D_{1}^{-1}U_{1}^{-T}U_{1}^{T}D_1U_2 \right|
\end{align*}
\begin{align*}
=\left| U_{1}^{T}D_1U_1 \right|\left| D_2 \right|=\left| U_{1}^{T}U_1 \right|\left| D_1 \right|\left| D_2 \right|=\left| D_1 \right|\left| D_2 \right|.
\end{align*}
由此可知 $\left| U_{1}^{T}U_1 \right|=1$.由 $U$ 是正交阵可知
\begin{align}
\begin{pmatrix}
U_{1}^{T}&		U_{2}^{T}\\
U_{3}^{T}&		U_{4}^{T}\\
\end{pmatrix} \begin{pmatrix}
U_1&		U_2\\
U_3&		U_4\\
\end{pmatrix} =I_n\Rightarrow U_{1}^{T}U_1+U_{3}^{T}U_3=I\Rightarrow \left| U_{1}^{T}U_1 \right|=\left| I-U_{3}^{T}U_3 \right|=1.\label{eq:104.3}
\end{align}
由\nrefpro{proposition:正定和半正定阵的判定准则}{(2)}可知 $U_{3}^{T}U_3$ 是半正定阵,因此设 $U_{3}^{T}U_3$ 的全体特征值为 $t_1,\cdots ,t_s\geqslant slant 0$,则由 \eqref{eq:104.3} 式可知
\begin{align*}
\left( 1-t_1 \right) \cdots \left( 1-t_s \right) =1\Rightarrow t_1=\cdots =t_s=0.
\end{align*}
于是由\hyperref[theorem:正定矩阵的充要条件]{半正定矩阵的合同标准型}知,存在可逆阵 $C$,使得
\begin{align*}
C^TU_{3}^{T}U_3C=O\Rightarrow U_{3}^{T}U_3=O \xLongrightarrow{\hyperref[corollary:线性方程组只有零解的充要条件]{\mathrm{r}\left( U_3 \right) =\mathrm{r}\left( U_{3}^{T}U_3 \right) =0}} U_3=O.
\end{align*}
同理可得 $U_2=O$.因此
\begin{align*}
A=\begin{pmatrix}
U_{1}^{T}&		\\
&		U_{4}^{T}\\
\end{pmatrix} \begin{pmatrix}
D_1&		\\
&		O\\
\end{pmatrix} \begin{pmatrix}
U_1&		\\
&		U_4\\
\end{pmatrix} =\begin{pmatrix}
U_{1}^{T}D_1U_1&		\\
&		O\\
\end{pmatrix},B=\begin{pmatrix}
O&		O\\
O&		D_2\\
\end{pmatrix}.
\end{align*}
故 $AB=BA=O$.

(ii)若 $s+t<n$,则由实对称阵可相似对角化,再结合特征值可知 $\mathrm{r}\left( A+B \right) =\mathrm{r}\left( A \right) +\mathrm{r}\left( B \right) $.由\nrefpro{proposition:矩阵秩的基本公式}{(6)}可知
\begin{align*}
\mathrm{r}\left( A+B \right) \leqslant slant \mathrm{r}\begin{pmatrix}
A\\
B\\
\end{pmatrix} \leqslant slant \mathrm{r}\left( A \right) +\mathrm{r}\left( B \right) \Longrightarrow \mathrm{r}\begin{pmatrix}
A\\
B\\
\end{pmatrix} =\mathrm{r}\left( A \right) +\mathrm{r}\left( B \right) =s+t.
\end{align*}
故 $\begin{pmatrix}
A\\
B\\
\end{pmatrix} X=O$ 的解空间 $V$ 的维数等于 $n-\left( s+t \right) $.取 $V$ 的一组标准正交基 $e_1,\cdots ,e_{n-\left( s+t \right)}$,并扩充为全空间的标准正交基 $e_1,\cdots ,e_{n-\left( s+t \right)},\cdots ,e_n$.于是可设
\begin{align*}
A\left( e_1,\cdots ,e_n \right) =\left( e_1,\cdots ,e_n \right) \begin{pmatrix}
O&		A_1\\
O&		A_2\\
\end{pmatrix},B\left( e_1,\cdots ,e_n \right) =\left( e_1,\cdots ,e_n \right) \begin{pmatrix}
O&		B_1\\
O&		B_2\\
\end{pmatrix}.
\end{align*}
其中 $A_2,B_2$ 都是 $s+t$ 阶方阵.又因为 $A,B$ 为对称阵,所以
\begin{align*}
A\left( e_1,\cdots ,e_n \right) =\left( e_1,\cdots ,e_n \right) \begin{pmatrix}
O&		O\\
O&		A_2\\
\end{pmatrix},
\end{align*}
\begin{align*}
B\left( e_1,\cdots ,e_n \right) =\left( e_1,\cdots ,e_n \right) \begin{pmatrix}
O&		O\\
O&		B_2\\
\end{pmatrix},
\end{align*}
\begin{align*}
\left( A+B \right) \left( e_1,\cdots ,e_n \right) =\left( e_1,\cdots ,e_n \right) \begin{pmatrix}
O&		O\\
O&		A_2+B_2\\
\end{pmatrix}.
\end{align*}
从而由 $A,B,A+B$ 的特征值可知,$A_2$ 的 $B_2$ 的全体特征值分别为 $\lambda _1,\cdots ,\lambda _t,0,\cdots ,0$ 和 $\mu _1,\cdots ,\mu _t,0,\cdots ,0$.$A_2+B_2$ 一定含有特征值 $\lambda _1,\cdots ,\lambda _t,\mu _1,\cdots ,\mu _t$.而 $A_2+B_2$ 就是 $s+t$ 阶方阵,故 $A_2+B_2$ 的全体特征值就是 $\lambda _1,\cdots ,\lambda _t,\mu _1,\cdots ,\mu _t$.
此时,由(i)同理可证 $A_2B_2=B_2A_2=O$.于是
\begin{align*}
AB\left( e_1,\cdots ,e_n \right) =\left( e_1,\cdots ,e_n \right) \begin{pmatrix}
O&		O\\
O&		A_2\\
\end{pmatrix} \begin{pmatrix}
O&		O\\
O&		B_2\\
\end{pmatrix} =\left( e_1,\cdots ,e_n \right) \begin{pmatrix}
O&		O\\
O&		A_2B_2\\
\end{pmatrix} =O\Rightarrow AB=O.
\end{align*}
这就完成了证明.
\end{proof}

\begin{corollary}
设 $A,B$ 是 $n$ 阶实对称矩阵,则
$
\left( \begin{matrix}
A+B&		\\
&		O\\
\end{matrix} \right) , \left( \begin{matrix}
A&		\\
&		B	\\
\end{matrix} \right)
$
相似当且仅当 $AB=BA=0$.
\end{corollary}
\begin{proof}
{\heiti 必要性:}设 $A$ 和 $B$ 的全部特征值分别为 $\lambda _1,\cdots ,\lambda _s,0,\cdots ,0$ 和 $\mu _1,\cdots ,\mu _t,0,\cdots ,0$($s,t\in [0,n],\lambda _i,\mu _j\ne 0$),则由 $\begin{pmatrix}
A+B&		\\
&		O\\
\end{pmatrix}$ 和 $\begin{pmatrix}
A&		\\
&		B		\\
\end{pmatrix}$ 相似可知,$A+B$ 的特征值是 $\lambda _1,\cdots ,\lambda _s,\mu _1,\cdots ,\mu _t,0,\cdots ,0$.从而存在可逆矩阵 $P$,使得
\begin{align*}
\begin{pmatrix}
A+B&		\\
&		O\\
\end{pmatrix} =\begin{pmatrix}
A&		\\
&		B		\\
\end{pmatrix} =P^{-1}\begin{scriptsize}
\begin{pmatrix}
\lambda _1&		&		&		&		&		&		&		&		\\
&		\ddots&		&		&		&		&		&		&		\\
&		&		\lambda _s&		&		&		&		&		&		\\
&		&		&		\mu _1&		&		&		&		&		\\
&		&		&		&		\ddots&		&		&		&		\\
&		&		&		&		&		\mu _t&		&		&		\\
&		&		&		&		&		&		0&		&		\\
&		&		&		&		&		&		&		\ddots&		\\
&		&		&		&		&		&		&		&		0\\
\end{pmatrix}\end{scriptsize} P.
\end{align*}
进而对 $\forall k\in \mathbb{N}$,都有
\begin{align*}
\begin{pmatrix}
\left( A+B \right) ^k&		\\
&		O\\
\end{pmatrix} =\begin{pmatrix}
A^k&				\\
&		B^k		\\
\end{pmatrix} =P^{-1}\begin{scriptsize}\begin{pmatrix}
\lambda _{1}^{k}&		&		&		&		&		&		&		&		\\
&		\ddots&		&		&		&		&		&		&		\\
&		&		\lambda _{s}^{k}&		&		&		&		&		&		\\
&		&		&		\mu _{1}^{k}&		&		&		&		&		\\
&		&		&		&		\ddots&		&		&		&		\\
&		&		&		&		&		\mu _{t}^{k}&		&		&		\\
&		&		&		&		&		&		0&		&		\\
&		&		&		&		&		&		&		\ddots&		\\
&		&		&		&		&		&		&		&		0\\
\end{pmatrix}\end{scriptsize} P.
\end{align*}
于是对 $\forall k\in \mathbb{N}$,都有
\begin{align*}
\mathrm{tr}\left( \left( A+B \right) ^k \right) =\mathrm{tr}\left( A^k \right) +\mathrm{tr}\left( B^k \right) =\lambda _{1}^{k}+\cdots +\lambda _{s}^{k}+\mu _{1}^{k}+\cdots +\mu _{t}^{k}.
\end{align*}
因此,再由\refexa{example:例题611}可知 $AB=BA=O$.

{\heiti 充分性:}设 $A$ 和 $B$ 的全部特征值分别为 $\lambda _1,\cdots ,\lambda _s,0,\cdots ,0$ 和 $\mu _1,\cdots ,\mu _t,0,\cdots ,0$($s,t\in [0,n],\lambda _i,\mu _j\ne 0$),因为 $A,B$ 为实对称阵且 $AB=BA$,所以由\refpro{proposition:一族两两可交换的复正规 (实对称) 矩阵可同时酉 (正交) 相似对角化}可知,存在正交阵 $P$,使得
\begin{align*}
P^{-1}AP=A_1,\quad P^{-1}BP=B_1,
\end{align*}
其中 $A_1$ 和 $B_1$ 都是对角阵,且主对角元分别是 $A$ 和 $B$ 的特征值.不妨设
\begin{align*}
A_1=\begin{pmatrix}
\lambda _1&		&		&		&		&		\\
&		\ddots&		&		&		&		\\
&		&		\lambda _s&		&		&		\\
&		&		&		0&		&		\\
&		&		&		&		\ddots&		\\
&		&		&		&		&		0\\
\end{pmatrix},
\end{align*}
由 $AB=BA=O$ 可知 $A_1B_1=O$.因此
\begin{align*}
B_1=\begin{scriptsize}\bordermatrix{%
&       &		&		s&		&		&		&		&		&		\cr
&       0&		&		&		&		&		&		&		&		\cr
&       &		\ddots&		&		&		&		&		&		&		\cr
s&       &		&		0&		&		&		&		&		&		\cr
&       &		&		&		\mu _1&		&		&		&		&		\cr
&       &		&		&		&		\ddots&		&		&		&		\cr
&       &		&		&		&		&		\mu _t&		&		&		\cr
&       &		&		&		&		&		&		0&		&		\cr
&       &		&		&		&		&		&		&		\ddots&		\cr
&       &		&		&		&		&		&		&		&		0\cr
}\end{scriptsize}.
\end{align*}
于是
\begin{align*}
P^{-1}\begin{pmatrix}
A+B&		\\
&		O\\
\end{pmatrix} P=\begin{pmatrix}
A_1+B_1&		\\
&		O\\
\end{pmatrix} =\begin{scriptsize}\begin{pmatrix}
\lambda _1&		&		&		&		&		&		&		&		\\
&		\ddots&		&		&		&		&		&		&		\\
&		&		\lambda _s&		&		&		&		&		&		\\
&		&		&		\mu _1&		&		&		&		&		\\
&		&		&		&		\ddots&		&		&		&		\\
&		&		&		&		&		\mu _t&		&		&		\\
&		&		&		&		&		&		0&		&		\\
&		&		&		&		&		&		&		\ddots&		\\
&		&		&		&		&		&		&		&		0\\
\end{pmatrix}\end{scriptsize},
\end{align*}
\begin{align*}
P^{-1}\begin{pmatrix}
A&		\\
&		B\\
\end{pmatrix} P=\begin{pmatrix}
A_1&		\\
&		B_1\\
\end{pmatrix} \sim \begin{scriptsize}\begin{pmatrix}
\lambda _1&		&		&		&		&		&		&		&		\\
&		\ddots&		&		&		&		&		&		&		\\
&		&		\lambda _s&		&		&		&		&		&		\\
&		&		&		\mu _1&		&		&		&		&		\\
&		&		&		&		\ddots&		&		&		&		\\
&		&		&		&		&		\mu _t&		&		&		\\
&		&		&		&		&		&		0&		&		\\
&		&		&		&		&		&		&		\ddots&		\\
&		&		&		&		&		&		&		&		0\\
\end{pmatrix}\end{scriptsize}.
\end{align*}
故 $\begin{pmatrix}
A+B&		\\
&		O\\
\end{pmatrix}$ 和 $\begin{pmatrix}
A&		\\
&		B\\
\end{pmatrix}$ 相似.
\end{proof}

\begin{example}
设 $A,B$ 是 $n$ 阶复方阵,若 $AB=O$,则对 $\forall k\in \mathbb{N}$,都有
\begin{align*}
\mathrm{tr}\left( \left( A+B \right) ^k \right) =\mathrm{tr}\left( A^k \right) +\mathrm{tr}\left( B^k \right) .
\end{align*}
\end{example}
\begin{proof}
不妨设 $B$ 既不是零矩阵也不是可逆矩阵,否则结论显然成立.
取 $\mathrm{Ker}A$ 的一组基 $\alpha _1,\cdots ,\alpha _r$,并扩充成 $\mathbb{C} ^n$ 的一组基 $\alpha _1,\alpha _2,\cdots ,\alpha _n$.从而可设
\begin{align*}
A\left( \alpha _1,\cdots ,\alpha _n \right) =\left( \alpha _1,\cdots ,\alpha _n \right) \begin{pmatrix}
O&		A_1\\
O&		A_2\\
\end{pmatrix}, \quad B\left( \alpha _1,\cdots ,\alpha _n \right) =\left( \alpha _1,\cdots ,\alpha _n \right) \begin{pmatrix}
B_1&		B_2\\
B_3&		B_4\\
\end{pmatrix},
\end{align*}
其中 $A_1,A_2$ 是列满秩矩阵.于是
\begin{align*}
&\quad \quad \,\, AB\left( \alpha _1,\cdots ,\alpha _n \right) =O\\
&\Longleftrightarrow \left( \alpha _1,\cdots ,\alpha _n \right) \begin{pmatrix}
O&		A_1\\
O&		A_2\\
\end{pmatrix} \begin{pmatrix}
B_1&		B_2\\
B_3&		B_4\\
\end{pmatrix} =O
\\
&\Longleftrightarrow \begin{pmatrix}
O&		A_1\\
O&		A_2\\
\end{pmatrix} \begin{pmatrix}
B_1&		B_2\\
B_3&		B_4\\
\end{pmatrix} =O
\\
&\Longrightarrow A_1B_3=O,A_1B_4=O,A_2B_3=O,A_2B_4=O
\\
&\Longrightarrow \begin{pmatrix}
A_1\\
A_2\\
\end{pmatrix} B_3=O,\begin{pmatrix}
A_1\\
A_2\\
\end{pmatrix} B_4=O.
\end{align*}
又因为 $\begin{pmatrix}
A_1\\
A_2\\
\end{pmatrix}$ 列满秩,所以上述方程只有零解,即 $B_3=B_4=O$.故
\begin{align*}
A\left( \alpha _1,\cdots ,\alpha _n \right) =\left( \alpha _1,\cdots ,\alpha _n \right) \begin{pmatrix}
O&		A_1\\
O&		A_2\\
\end{pmatrix}, \quad B\left( \alpha _1,\cdots ,\alpha _n \right) =\left( \alpha _1,\cdots ,\alpha _n \right) \begin{pmatrix}
B_1&		B_2\\
O&		O\\
\end{pmatrix}.
\end{align*}
因此 $A$ 与 $\begin{pmatrix}
O&		A_1\\
O&		A_2\\
\end{pmatrix}$ 相似,$B$ 与 $\begin{pmatrix}
B_1&		B_2\\
O&		O\\
\end{pmatrix}$ 相似.从而 $A+B$ 与 $\begin{pmatrix}
B_1&		A_1+B_2\\
O&		A_2\\
\end{pmatrix}$ 相似.进而对 $\forall k\in \mathbb{N}$,都有
\begin{align*}
A^k\sim \begin{pmatrix}
O&		A_1\\
O&		A_2\\
\end{pmatrix} ^k=\begin{pmatrix}
O&		*\\
O&		A_{2}^{k}\\
\end{pmatrix}, \quad B^k\sim \begin{pmatrix}
B_1&		B_2\\
O&		O\\
\end{pmatrix} ^k=\begin{pmatrix}
B_{1}^{k}&		*\\
O&		O\\
\end{pmatrix},
\end{align*}
\begin{align*}
\left( A+B \right) ^k\sim \begin{pmatrix}
B_1&		A_1+B_2\\
O&		A_2\\
\end{pmatrix} ^k=\begin{pmatrix}
B_{1}^{k}&		*\\
O&		A_{2}^{k}\\
\end{pmatrix}.
\end{align*}
故对 $\forall k\in \mathbb{N}$,都有
\begin{align*}
\mathrm{tr}\left( \left( A+B \right) ^k \right) =\mathrm{tr}\left( A_{2}^{k} \right) +\mathrm{tr}\left( B_{1}^{k} \right) =\mathrm{tr}\left( A^k \right) +\mathrm{tr}\left( B^k \right) .
\end{align*}
\end{proof}

\begin{example}
设 $n$ 阶方阵 $A=\begin{pmatrix}&&1&1\\&\begin{turn}{80}$\ddots$\end{turn}&1&\\1&\begin{turn}{80}$\ddots$\end{turn}&\\1&&\end{pmatrix}$,计算 $A$ 的特征值,并求出其相似标准型与对应的过渡矩阵。
\end{example}
\begin{proof}
答案见豌豆讲义.
\end{proof}

\begin{example}
设 $A$ 是 $n$ 阶复方阵,证明:$A\overline{A}$ 的特征值中,虚数和负实数(如果存在)都是成对出现的,进而成立:$|\lambda I + A\overline{A}| \geqslant  0, \forall \lambda \geqslant  0$.
\end{example}
\begin{proof}
对$A\overline{A}$的任一特征值$\lambda$,存在特征向量$\alpha$,使得$A\overline{A}\alpha =\lambda \alpha$.令$W=\mathrm{span}\left( \alpha ,A\overline{\alpha } \right)$,则$\dim W=1$或$2$.

若$\dim W=1$,则$A\overline{\alpha }=\mu \alpha$,$\mu \in \mathbb{C}$.将$\alpha$扩充为全空间的基$\alpha ,\alpha _2,\cdots ,\alpha _n$,这组基都取共轭后仍是全空间的基,则
\begin{align*}
A\left( \overline{\alpha },\overline{\alpha _2},\cdots ,\overline{\alpha _n} \right) =\left( \alpha ,\alpha _2,\cdots ,\alpha _n \right) \begin{pmatrix}
\mu&		*\\
0&		A_1\\
\end{pmatrix}.
\end{align*}
这里$A_1$是$n-1$阶方阵.记$P=\left( \alpha ,\alpha _2,\cdots ,\alpha _n \right)$是可逆阵,则$A=P\begin{pmatrix}
\mu&		*\\
O&		A_1\\
\end{pmatrix} \overline{P}^{-1}$,并且
\begin{align*}
A\overline{A}=P\begin{pmatrix}
\mu&		*\\
0&		A_1\\
\end{pmatrix} \overline{\begin{pmatrix}
\mu&		*\\
0&		A_1\\
\end{pmatrix} }P^{-1}=P\begin{pmatrix}
\left| \mu \right|^2&		*\\
0&		A_1\overline{A_1}\\
\end{pmatrix} P^{-1}.
\end{align*}
若$\dim W=2$,则$\alpha ,A\overline{\alpha }$线性无关,将$\alpha ,A\overline{\alpha }$扩充为全空间的基$\alpha ,A\overline{\alpha },\alpha _3,\cdots ,\alpha _n$,这组基都取共轭后仍是全空间的基.又注意到$A\overline{A}\alpha =\lambda \alpha$,故
\begin{align*}
A\left( \overline{\alpha },\overline{A}\alpha ,\alpha _3,\cdots ,\overline{\alpha _n} \right) =\left( \alpha ,A\overline{\alpha },\alpha _3,\cdots ,\alpha _n \right) \begin{pmatrix}
0&		\lambda&		*\\
1&		0&		*\\
0&		0&		A_1\\
\end{pmatrix}.
\end{align*}
这里$A_1$是$n-2$阶方阵.记$P=\left( \alpha ,A\overline{\alpha },\alpha _3,\cdots ,\alpha _n \right)$,则$A=P\begin{pmatrix}
0&		\lambda&		*\\
1&		0&		*\\
0&		0&		A_1\\
\end{pmatrix} P^{-1}$,并且
\begin{align*}
A\overline{A}=P\begin{pmatrix}
0&		\lambda&		*\\
1&		0&		*\\
0&		0&		A_1\\
\end{pmatrix} \overline{\begin{pmatrix}
0&		\lambda&		*\\
1&		0&		*\\
0&		0&		A_1\\
\end{pmatrix} }P^{-1}=P\begin{pmatrix}
\lambda&		0&		*\\
0&		\overline{\lambda }&		*\\
0&		0&		A_1\overline{A_1}\\
\end{pmatrix} P^{-1}.
\end{align*}

因此由数学归纳法可知,存在可逆复方阵$P$,使得$A=PD\overline{P}^{-1}$,这里$D$是分块上三角阵,对角矩块为$1$阶非负实数或形如$\begin{pmatrix}
0&		\lambda\\
1&		0\\
\end{pmatrix}$的二阶块.进而,$A\overline{A}$可相似上三角化,其中主对角元素即全体特征值满足:或者是非负实数,或者是成对出现的共轭虚数,或者是成对出现的负数.

设$A\overline{A}$的全体特征值为$x_1,\cdots ,x_n$,则$x_1,\cdots ,x_n$中的虚数和负实数(如果存在)都是成对出现的.于是由命题可知$\lambda I+A\overline{A}$的全体特征值为$\lambda +x_1,\cdots ,\lambda +x_n$,则
\begin{align*}
\left| \lambda I+A\overline{A} \right|=\prod_{i=1}^n{\left( \lambda +x_i \right)}\geqslant slant 0,\forall \lambda \geqslant slant 0.
\end{align*}
\end{proof}

\begin{example}
设实数 $a,b,c,d,e$ 满足 $a^2 + b^2 + c^2 = d^2 + e^2 = 1$,设 $A$ 是 $5 \times 3$ 矩阵,每行都是 $a,b,c$ 的排列,$B$ 是 $5 \times 2$ 矩阵,每行都是 $d,e$ 的排列,记矩阵 $M = (A,B)$,证明:$(\mathrm{tr(}M))^2\leqslant slant \left( 5+2\sqrt{6} \right) \mathrm{r(}M)$,并且 $M$ 有实特征值 $\lambda$ 满足 $|\lambda| \leqslant  \sqrt{2} + \sqrt{3}$.
\end{example}
\begin{proof}
显然有$|a|,|b|,|c|,|d|,|e|\leqslant slant 1$,则
\begin{align*}
(\mathrm{tr}(M))^2\leqslant slant 5^2=25.
\end{align*}
注意到$1\leqslant slant \mathrm{r}(M) \leqslant slant 5$,故当$\mathrm{r}(M) \geqslant slant 3$时,恒有
\begin{align*}
\left( 5+2\sqrt{6} \right) \mathrm{tr}(M)\geqslant slant 25\geqslant slant (\mathrm{tr}(M))^2.
\end{align*}
因此我们只需考虑$\mathrm{r}(M) =1,2$的情形.

(1)当$\mathrm{r}(M) =1$时,我们有
\begin{align*}
M=\begin{pmatrix}
a&		b&		c&		d&		e\\
k_1a&		k_1b&		k_1c&		k_1d&		k_1e\\
k_2a&		k_2b&		k_2c&		k_2d&		k_2e\\
k_3a&		k_3b&		k_3c&		k_3d&		k_3e\\
k_4a&		k_4b&		k_4c&		k_4d&		k_4e\\
\end{pmatrix},
\end{align*}
因为$k_ia,k_ib,k_ic$是$a,b,c$的排列,所以
\begin{align*}
k_{i}^{2}\left( a^2+b^2+c^2 \right) =a^2+b^2+c^2\Rightarrow k_i=\pm 1,i=1,2,3,4.
\end{align*}
再利用Cauchy不等式可得
\begin{align*}
\mathrm{tr}\left( M \right) \leqslant slant |a|+|b|+|c|+|d|+|e| |a|+|b|+|c|+\sqrt{2}.
\end{align*}
因此
\begin{align*}
\left( \mathrm{tr}\left( M \right) \right) ^2=5+2\sqrt{6}=\left( 5+2\sqrt{6} \right) \mathrm{tr}(M).
\end{align*}

(2)当$\mathrm{r}(M) =2$时,设$M=\begin{pmatrix}
U&		P\\
Q&		N\\
\end{pmatrix}$,其中$U$为$3$阶方阵,$N$为$2$阶方阵.由$d^2+e^2=1$可得$\mathrm{tr}(N) \leqslant slant 2$.

现在讨论$\mathrm{tr}(U)$.

(i)若$U$的主对角元互不相同,则不妨设$U$的主对角元分别为$a,b,c$.此时利用Cauchy不等式可得
\begin{align*}
\mathrm{tr}(U) \leqslant slant |a|+|b|+|c|\leqslant slant \sqrt{3}<\sqrt{6}.
\end{align*}

(ii)若$U$的主对角只出现$a,b,c$中两个,则不妨设$U$的主对角元分别为$a,a,b$.此时利用Cauchy不等式可得
\begin{align*}
\mathrm{tr}(U) \leqslant slant 2|a|+|b|\leqslant slant \sqrt{5}<\sqrt{6}.
\end{align*}

(iii)若$U$的主对角只出现$a,b,c$中1个,则不妨设$U$的主对角元分别为$a,a,a$.此时$M$只可能有4种不同的形式:
\begin{align*}
M_1=\begin{pmatrix}
a&		b&		c\\
b&		a&		c\\
b&		c&		a\\
\end{pmatrix},\quad M_2=\begin{pmatrix}
a&		b&		c\\
b&		a&		c\\
c&		b&		a\\
\end{pmatrix},\quad M_3=\begin{pmatrix}
a&		b&		c\\
c&		a&		b\\
c&		b&		a\\
\end{pmatrix},\quad M_4=\begin{pmatrix}
a&		b&		c\\
c&		a&		b\\
b&		c&		a\\
\end{pmatrix}.
\end{align*}
由$\mathrm{r}(M) =2$可知这4个矩阵行列式全为0,即
\begin{gather*}
|M_1|=|M_2|=|M_3|=a^3+b^2c+bc^2-ac^2-ab^2-abc=(a-b)(a-c)(a+b+c),
\\
|M_4|=a^3+b^3+c^3-3abc=\frac{1}{2}(a+b+c)\left[(a-b)^2+(b-c)^2+(c-a)^2\right].
\end{gather*}
再结合$a^2+b^2+c^2=1$,经计算可得无论$M$是上述哪个矩阵,要么$a=b$,要么$a=c$,要么$a+b+c=0$.

若$a=b$或$a=c$,则由$a^2+b^2+c^2=1$可知
\begin{align*}
2a^2+b^2=1\text{或}2a^2+c^2=1\Rightarrow 2a^2\leqslant slant 1\Rightarrow |a|\leqslant slant \frac{\sqrt{2}}{2}<\frac{\sqrt{6}}{3}.
\end{align*}

若$a+b+c=0$,则由$a^2+b^2+c^2=1$可知
\begin{align*}
a^2+b^2+(a+b)^2=1\Rightarrow a^2+ab+b^2=\frac{1}{2}\Rightarrow \left( \frac{a}{2}+b \right) ^2+\frac{3}{4}a^2=\frac{1}{2}
\end{align*}
\begin{align*}
\Rightarrow \frac{3}{4}a^2\leqslant slant \frac{1}{2}\Rightarrow |a|\leqslant slant \frac{\sqrt{6}}{3}.
\end{align*}
从而此时
\begin{align*}
\mathrm{tr}(M) =3a\leqslant slant 3|a|=\sqrt{6}.
\end{align*}
于是
\begin{align*}
\mathrm{tr}(M) =\mathrm{tr}(U) +\mathrm{tr}(N) \leqslant slant 2+\sqrt{6}.
\end{align*}
故
\begin{align*}
(\mathrm{tr}(M))^2\leqslant slant \left( 2+\sqrt{6} \right) ^2=10+4\sqrt{6}=\left( 5+2\sqrt{6} \right) \mathrm{r}(M).
\end{align*}
\end{proof}
















\end{document}
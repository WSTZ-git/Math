\documentclass[../../main.tex]{subfiles}
\graphicspath{{\subfix{../../image/}}} % 指定图片目录,后续可以直接使用图片文件名。

% 例如:
% \begin{figure}[H]
% \centering
% \includegraphics[scale=0.4]{图.png}
% \caption{}
% \label{figure:图}
% \end{figure}
% 注意:上述\label{}一定要放在\caption{}之后,否则引用图片序号会只会显示??.

\begin{document}

\section{覆盖定理}

\begin{theorem}[覆盖定理]\label{theorem:覆盖定理}
\begin{enumerate}
\item 数域上 \( n \) 维线性空间 \( V \) 不能分解为有限个 (大于等于 2 个) 非平凡子空间之并.  

\item 域 \( \mathbb{F} \) 上线性空间 \( V \) 不能分解为两个非平凡子空间之并. 

\item 域 \( \mathbb{F} \) 上无限维线性空间 \( V \) 不能分解为有限个 (大于等于 2 个) 非平凡子空间之并.  
\end{enumerate}
\end{theorem}
\begin{remark}
全空间和空集称作平方子空间,不是平凡子空间的线性空间称为非平凡子空间.
\end{remark}
\begin{note}
对于数域上的有限维线性空间, 我们给一个简单的方法, 比丘维声等更简单.  
\end{note}
\begin{proof}
\begin{enumerate}
\item 设$V=\bigcup_{i=1}^n{V_i}$,其中$n\geqslant 2$,$V_i$都是非平凡子空间.考虑 \( V \) 的基 \( e_1, e_2, \cdots, e_n \) 和一组向量  
\[
\alpha_i = e_1 + i e_2 + i^2 e_3 + \cdots + i^{n-1} e_n, \, i = 1, 2, \cdots.
\] 
从而
\begin{align*}
\left( \alpha _{i_1},\alpha _{i_2},\cdots ,\alpha _{i_n} \right) =\left( e_1,e_2,\cdots ,e_n \right) \left( \begin{matrix}
1&		1&		\cdots&		1\\
i_1&		i_2&		\cdots&		i_n\\
\vdots&		\vdots&		&		\vdots\\
i_{1}^{n-1}&		i_{2}^{n-1}&		\cdots&		i_{n}^{n-1}\\
\end{matrix} \right) ,i_1,\cdots ,i_n\in \mathbb{N} .
\end{align*}
由Vandermonde行列式的性质, 上面这组向量$\{\alpha_i\}_{i=1}^{\infty}$中任何 \( n \) 个都线性无关. 而其中必有无限个$\alpha_i$落入某个题设 \( V \) 的某个非平凡子空间$V_i$中, 从而这个子空间至少是\( n \)维的, 这和其是非平凡子空间矛盾!  

\item 设 \( V = V_1 \bigcup V_2 \), \( V_1, V_2 \) 都是 \( V \) 的非平凡不变子空间. 则存在 \( v_1 \in V_1 \setminus V_2, v_2 \in V_2 \setminus V_1 \). 注意到
\[
v_1 + v_2 \in V_1 \Rightarrow v_2 \in V_1, \, v_1 + v_2 \in V_2 \Rightarrow v_1 \in V_2,
\]  
从而  
\[
v_1 + v_2 \notin V_1 \bigcup V_2,
\]  
这就是一个矛盾! 因此我们证明了域 \( \mathbb{F} \) 上线性空间 \( V \) 不能分解为两个非平凡子空间之并.  

\item 若 \( V_i \subset V, i = 1, 2, \cdots, n \) 是非平凡子空间. 我们归纳证明  
\begin{align}
V \neq \bigcup_{i=1}^n V_i. \label{eq:::::---64486-23.21}
\end{align}
当 \( n = 1 \) 显然有 \(\eqref{eq:::::---64486-23.21}\) 成立. 假设对 \( n \) 有 \(\eqref{eq:::::---64486-23.21}\) 成立, 当 \( n + 1 \) 时, 由归纳假设取 \( \alpha \in V \setminus \left( \bigcup_{i=1}^n V_i \right) \).  

如果 \( \alpha \notin V_{n+1} \), 则 \(\eqref{eq:::::---64486-23.21}\) 对 \( n + 1 \) 已经成立.  
- 若 \( \alpha \in V_{n+1} \), 由 \( V_{n+1} \) 非平凡, 存在 \( \beta \in V \setminus V_{n+1} \). 注意到  
\[
S = \{ t\alpha + \beta : t \in \mathbb{F} \} \subset V \setminus V_{n+1},
\]  
这是因为若某个 \( t\alpha + \beta \in V_{n+1} \) 可推出 \( \beta \in V_{n+1} \) 而矛盾!. 现在若有 \( t_1\alpha + \beta, t_2\alpha + \beta \) 属于同一个 \( V_j, j \in \{1, 2, \cdots, n\} \), 则  
\[
t_1\alpha + \beta - (t_2\alpha + \beta) = (t_1 - t_2) \alpha \in V_j,
\]  
即 \( t_1 = t_2 \). 现在 \( S \) 中向量有无穷多个, 所以必然有一个不落在 \( \bigcup_{j=1}^n V_j \), 自然也不落在 \( \bigcup_{j=1}^{n+1} V_j \), 这就证明了 \(\eqref{eq:::::---64486-23.21}\) 对 \( n + 1 \) 也成立!.  
至此我们完成了证明.
\end{enumerate}

\end{proof}

\begin{example}
设 \( V \) 是有限维线性空间,\( A_1, A_2, \cdots, A_s \) 是 \( V \) 上两两不同线性变换,则 \( \exists \alpha \in V \),使 \( A_1\alpha, A_2\alpha, \cdots, A_s\alpha \) 两两不同.
\end{example}
\begin{proof}
考虑$\bigcup_{1\leqslant i<j\leqslant n}{\mathrm{Ker}\left( A_j-A_i \right)}$,显然$\mathrm{Ker}\left( A_j-A_i \right) \ne V,\varnothing$。故由\hyperref[theorem:覆盖定理]{覆盖定理}知,我们有$\bigcup_{1\leqslant i<j\leqslant n}{\mathrm{Ker}\left( A_j-A_i \right)}\ne V$。因此$\exists \alpha \in V\backslash \bigcup_{1\leqslant i<j\leqslant n}{\mathrm{Ker}\left( A_j-A_i \right)}$,此时
\begin{align*}
\left( A_j-A_i \right) \alpha \ne 0\Longleftrightarrow A_i\alpha \ne A_j\alpha ,1\leqslant i<j\leqslant n.
\end{align*}

\end{proof}

\begin{example}
设 \( T \) 是 \( \mathbb{R}^{n \times n} \) 上的线性变换且 \( T(AB) = T(A)T(B) \),\( T(AB) = T(B)T(A) \) 必有一个成立. 证明: 要么对所有 \( A, B \in \mathbb{R}^{n \times n} \) 都有 \( T(AB) = T(A)T(B) \),要么对所有 \( A, B \in \mathbb{R}^{n \times n} \) 都有 \( T(AB) = T(B)T(A) \).  
\end{example}
\begin{proof}
对$\forall A\in \mathbb{R}^{n\times n},$考虑  
\[
V_A = \left\{ B \in \mathbb{R}^{n \times n} : T(AB) = T(B)T(A) \right\};
\]  
\[
V_A' = \left\{ B \in \mathbb{R}^{n \times n} : T(AB) = T(A)T(B) \right\},
\]  
则有 \( \mathbb{R}^{n \times n} = V_A \bigcup V_A' \). 由\hyperref[theorem:覆盖定理]{覆盖定理},我们有 \( \mathbb{R}^{n \times n} = V_A \) 或者 \( \mathbb{R}^{n \times n} = V_A' \).  
再由$A$的任意性可知结论成立.

\end{proof}

\begin{definition}
设 \( V \) 是域 \( \mathbb{F} \) 上的有限维线性空间,\( v \in V \),\( A \) 是 \( V \) 上线性变换。记 \( m_v \in \mathbb{F}[x] \) 是 \( m_v(A)v = 0 \) 成立的最低次的首一多项式,我们叫 \( m_v \) 为\textbf{ \( \boldsymbol{v} \) 的极小多项式}。
\end{definition}

\begin{proposition}\label{proposition:向量的极小多项式的整性}
设 \( V \) 是域 \( \mathbb{F} \) 上的有限维线性空间,\( v \in V \),\( A \) 是 \( V \) 上线性变换,\( m_v \) 为 \( v \) 的极小多项式。若$m\in \mathbb{F}[x]$,且$m(A)v=0$,则必有$m_v\mid m.$
\end{proposition}
\begin{proof}
由带余除法可知,存在\( q,r\in \mathbb{F}[x] \),且\( \deg r<\deg m_v \),使得\( m=qm_v+r \)。从而
\begin{align*}
0=m(A)v=q(A)m_v(A)v+r(A)v=r(A)v.
\end{align*}
于是\( r\equiv 0 \),否则与\( m_v \)是\( v \)的极小多项式矛盾!故\( m_v|m \)。

\end{proof}

\begin{example}
设 \( V \) 是域 \( \mathbb{F} \) 上的有限维线性空间,\( v \in V \),\( A \) 是 \( V \) 上线性变换。记 \( m_v \in \mathbb{F}[x] \) 是 \( m_v(A)v = 0 \) 成立的最低次的首一多项式,我们叫 \( m_v \) 为 \( v \) 的极小多项式。证明:存在 \( v \in V \) 使得 \( v \) 的极小多项式恰好是 \( A \) 的极小多项式。
\end{example}
\begin{proof}
设 \( m \in \mathbb{F}[x] \) 是 \( A \) 的极小多项式,那么
\[
m(A)v = 0, \forall v \in V
\]
{\color{blue}证法一:}
反证,假设对$\forall v\in V$,都有$m_v(x)\ne m(x)$.
又由\refpro{proposition:向量的极小多项式的整性}知\( m_v|m \)。因此$m_v(x)(v\in V)$都是$m(x)$的首一真因式.
考虑 \( m \) 的所有首一真因式
\[
m_1, m_2, \cdots, m_N \in \mathbb{F}[x],N\in\mathbb{N},
\]
则对$\forall v\in V$,都有$m_v\in \{m_1,m_2,\cdots,m_N\}$.
考虑
\[
V_i \triangleq \{ v \in V : m_i(A)v = 0 \}, i = 1, 2, \cdots, N.
\]
对$\forall v_0\in V,$由于$m_{v_0}(x)$都是$m(x)$的首一真因式,故存在$i_0\in [1,N]\cap \mathbb{N}$,使得$m_{v_0}=m_{i_0}$.从而$m_{i_0}(A)v_0=0$,于是$v_0\in V_{i_0}.$因此$V\subset \bigcup_{i=1}^N V_i.$
故\( V = \bigcup_{i=1}^N V_i \),于是由\hyperref[theorem:覆盖定理]{覆盖定理},我们有某个 \( i \) 使得 \( V_i = V \),故$$m_i(A)v=0,\forall v\in V.$$
因此$m_i(A)=0.$
这意味着 \( m | m_i \),故 \( m = m_i\)。这与$m_i$是$m$的真因式矛盾!

\end{proof}

\begin{proposition}\label{proposition:无穷多个线性子空间的交等于有限个的交}
设 \( V \) 是域 \( \mathbb{F} \) 上的有限维线性空间,\(\{V_{\alpha}\}_{\alpha \in \Lambda}\) 是 \( V \) 的子空间. 证明存在 \(\alpha_1, \alpha_2, \cdots, \alpha_n \in \Lambda\) 使得
\[
\bigcap_{i = 1}^n V_{\alpha_i} = \bigcap_{\alpha \in \Lambda} V_{\alpha}
\]
\end{proposition}
\begin{proof}
若 \(|\Lambda| < \infty\),则已经得证明. 若 \(|\Lambda| = \infty\),取 \(V_{\alpha_1}\),若
\[
\dim V_{\alpha_1} = \dim \bigcap_{\alpha \in \Lambda} V_{\alpha}
\]
这已经得证.

若
\[
\dim V_{\alpha_1} > \dim \bigcap_{\alpha \in \Lambda} V_{\alpha}
\]
先证存在 \(\alpha_2 \in \Lambda \setminus \{\alpha_1\}\) ,使得
\[
V_{\alpha_1} \bigcap V_{\alpha_2} \subset V_{\alpha_1}.
\]
假设对任何 \(\alpha \in \Lambda \setminus \{\alpha_1\}\),我们有
\[
V_{\alpha_1} \bigcap V_{\alpha} = V_{\alpha_1}\Longrightarrow \bigcap_{\alpha \in \Lambda}{V_{\alpha}}=V_{\alpha _1}\cap \left( \bigcap_{\alpha \in \Lambda \backslash \left\{ \alpha _1 \right\}}{V_{\alpha}} \right) =V_{\alpha _1}.
\]
则
\[
\dim V_{\alpha_1} = \dim \bigcap_{\alpha \in \Lambda} V_{\alpha}
\]
矛盾!故存在 \(\alpha_2 \in \Lambda \setminus \{\alpha_1\}\) ,使得
\[
V_{\alpha_1} \bigcap V_{\alpha_2} \subset V_{\alpha_1}.
\]
从而
\[
\dim \left( V_{\alpha_1} \bigcap V_{\alpha_2} \right) < \dim \left( V_{\alpha_1} \right)
\]
此时要么
\[
\dim \left( V_{\alpha_1} \bigcap V_{\alpha_2} \right) > \dim \bigcap_{\alpha \in \Lambda} V_{\alpha}
\]
要么
\[
\dim \left( V_{\alpha_1} \bigcap V_{\alpha_2} \right) = \dim \bigcap_{\alpha \in \Lambda} V_{\alpha}
\]
因此我们可以继续讨论 \(\alpha_3 \in \Lambda \setminus \{\alpha_1, \alpha_2\}\),依次下去.可以得到一个严格递减正整数列$\left\{ \mathrm{dim}\bigcap_{i=1}^n{V_{\alpha _i}} \right\} $,满足
\[
\mathrm{dim}\bigcap_{\alpha \in \Lambda}{V_{\alpha _i}}\leqslant \mathrm{dim}\bigcap_{i=1}^m{V_{\alpha _i}}\leqslant \mathrm{dim}V_{\alpha _1},\mathrm{dim}\bigcap_{i=1}^m{V_{\alpha _i}}\in \mathbb{N},m\in \mathbb{N}.
\]
因此严格递减正整数列$\left\{ \mathrm{dim}\bigcap_{i=1}^n{V_{\alpha _i}} \right\}$只能有限项,因此这样的操作会经过有限次而停止.于是我们知道存在 \(\alpha_1, \alpha_2, \cdots, \alpha_n \in \Lambda\) 使得
\[
\bigcap_{i = 1}^n V_{\alpha_i} = \bigcap_{\alpha \in \Lambda} V_{\alpha}.
\]

\end{proof}

\begin{example}
设\( P \)为数域,\( \tau \)为\( P^{n \times n} \)上的线性变换,满足条件:对任意固定的\( A, B \in P^{n \times n} \),\( \tau(AB) = \tau(A)\tau(B) \)或\( \tau(AB) = \tau(B)\tau(A) \)至少有一个成立,证明:要么对所有的\( A, B \in P^{n \times n} \),\( \tau(AB) = \tau(A)\tau(B) \),要么对所有的\( A, B \in P^{n \times n} \),\( \tau(AB) = \tau(B)\tau(A) \).
\end{example}
\begin{proof}
对$\forall A\in P^{n\times n},$记
$$U_A\triangleq \left\{ X:\tau \left( AX \right) =\tau \left( A \right) \tau \left( X \right) \right\}, \quad U_{A}^{\prime}\triangleq \left\{ X:\tau \left( AX \right) =\tau \left( X \right) \tau \left( A \right) \right\},$$
则由条件知
$$P^{n\times n}=U_A\cup U_{A}^{\prime}.$$
从而$U_A,U_{A}^{\prime}$都不是非平凡子空间,否则与\hyperref[theorem:覆盖定理]{覆盖定理}矛盾!因此对$\forall A\in P^{n\times n},$要么$U_A=P^{n\times n},$要么$U_{A}^{\prime}=P^{n\times n}.$再记
$$W_1=\left\{ A:U_A=P^{n\times n} \right\}, \quad W_2=\left\{ A:U_{A}^{\prime}=P^{n\times n} \right\},$$
则
$$P^{n\times n}=W_1\cup W_2.$$
于是$W_1,W_2$都不是非平凡子空间,否则与\hyperref[theorem:覆盖定理]{覆盖定理}矛盾!故要么$W_1=P^{n\times n},$要么$W_2=P^{n\times n}.$即
\begin{align*}
\text{要么对}\forall A\in P^{n\times n},\text{都有}U_A=P^{n\times n};
\\
\text{要么对}\forall A\in P^{n\times n},\text{都有}U_{A}^{\prime}=P^{n\times n}.
\end{align*}
也即
\begin{align*}
\text{要么对}\forall A,B\in P^{n\times n},\text{都有}\tau \left( AB \right) =\tau \left( A \right) \tau \left( B \right) ;
\\
\text{要么对}\forall A,B\in P^{n\times n},\text{都有}\tau \left( AB \right) =\tau \left( B \right) \tau \left( A \right) .
\end{align*}
\end{proof}
















\end{document}
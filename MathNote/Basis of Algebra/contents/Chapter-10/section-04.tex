\documentclass[../../main.tex]{subfiles}
\graphicspath{{\subfix{../../image/}}} % 指定图片目录,后续可以直接使用图片文件名。

% 例如:
% \begin{figure}[H]
% \centering
% \includegraphics[scale=0.4]{图.png}
% \caption{}
% \label{figure:图}
% \end{figure}
% 注意:上述\label{}一定要放在\caption{}之后,否则引用图片序号会只会显示??.

\begin{document}

\section{高等代数中两个重要结论和思想}

\begin{definition}[函数组线性无关/相关]
设\(\{f_1(x), f_2(x), \ldots, f_n(x)\}\)是一组定义域为$D$的函数,如果不存在不全为零的标量\(c_1, c_2, \ldots, c_n\),使得 
\begin{align*}
c_1 f_1(x) + c_2 f_2(x) + \cdots + c_n f_n(x) = 0,\quad \forall x\in D.
\end{align*}
则称这组函数在\(I\)上是\textbf{线性无关的}.

反之,如果存在不全为零的\(c_i\)使得上式成立,则称这组函数是\textbf{线性相关的}。
\end{definition}

\begin{theorem}\label{theorem:高等代数的重要结论1}
设 \( X \) 是域 \( \mathbb{F} \) 上的线性空间, \( f_1, f_2, \cdots, f_n \in X^* \), 记 \( N = \bigcap_{i=1}^n \ker f_i \), 则下述条件等价
\begin{enumerate}[(1)]
\item \( f \in X^* \) 可被 \( f_1, f_2, \cdots, f_n \) 线性表出;

\item \( N \subset \ker f \).
\end{enumerate}
其中$X^*$是线性空间$X$的对偶空间.
\end{theorem}
\begin{proof}
(1)推(2)是显然的, 我们来看(2)推(1).

构造线性映射
\[
\pi: X \to \mathbb{F}^n, \mathbf{x} \mapsto (f_1(\mathbf{x}), f_2(\mathbf{x}), \cdots, f_n(\mathbf{x})).
\]
于是 \( \ker \pi = N \). 定义
\[
g: \pi(X) \to \mathbb{F}, \pi(\mathbf{x}) \mapsto f(\mathbf{x}).
\]
先证$g$的良定义. 若$\pi(x) = \pi(y)$,则$\pi(x - y) = 0$,从而$x - y \in \ker \pi = N$,于是$f(x - y) = 0$,即$f(x) = f(y)$. 故$g$是良定义的.

现在 \( g \) 线性延拓到 \( \mathbb{F}^n \) 上(补空间上定义为零映射), 取$\mathbb{F}^n$的标准基为$\{e_1,e_2,\cdots,e_n\}$,其中$e_i=(0,\cdots,1,\cdots,0)^T,i=1,2,\cdots,n$.
现在我们注意到
对$\forall x \in X$,都有
\begin{align*}
f(x) = g(\pi(x)) = g(f_1(x), f_2(x), \cdots, f_n(x)) = g\left( \sum_{i=1}^n f_i(x) e_i \right) = \sum_{i=1}^n g(e_i) f_i(x).
\end{align*}
这就证明了 \( f \) 可被 \( f_1, f_2, \cdots, f_n \) 线性表出.

\end{proof}

\begin{corollary}\label{corollary:对偶空间的基的充要条件}
\( n \) 维线性空间$X$的 \( n \) 个线性函数 \( f_1, f_2, \cdots, f_n \) 线性无关的充要条件是他们可分点, 即对任何 \( a \neq 0 \), 存在某个 \( k = 1, 2, \cdots, n \), 使得 \( f_k(a) \neq 0 \).也即\( \bigcap_{i=1}^n \ker f_i = \{0\} \).
\end{corollary}
\begin{proof}
注意到 \( n \) 个线性函数 \( f_1, f_2, \cdots, f_n \) 线性无关等价于\( f_1, f_2, \cdots, f_n \)是这个线性空间对偶空间$X^*$的一组基,从而等价于任意的$f\in X^*$都可被\( f_1, f_2, \cdots, f_n \)线性表出.又由\refthe{theorem:高等代数的重要结论1}知,任意的$f\in X^*$都可被\( f_1, f_2, \cdots, f_n \)线性表出的充要条件是
\begin{align}
\bigcap_{i=1}^n \ker f_i \subset \bigcap_{f \in X^*} \ker f,\label{eq::----123878999999999}
\end{align}
下证$\bigcap_{f \in X^*} \ker f = \{0\}$. 对$\forall \alpha \in \bigcap_{f \in X^*} \ker f$,若$\alpha \neq 0$,则$X^*$中存在线性函数
\begin{align*}
f_0: X \longrightarrow X,\ x \longmapsto 2x
\end{align*}
从而$f(\alpha) = 2\alpha \neq 0$. 这与$\alpha \in \ker f_0$矛盾!因此$\alpha = 0$,故$\bigcap_{f \in X^*} \ker f = \{0\}$.再由\eqref{eq::----123878999999999}式可知, \( f_1, f_2, \cdots, f_n \) 线性无关等价于 \( \bigcap_{i=1}^n \ker f_i = \{0\} \),即使$f_i$的像全为0的向量只能是0.这就完成了证明.

\end{proof}

\begin{example}
设 \( V \) 是有限维线性空间且 \( A \) 是 \( V \) 上的线性变换. 定义
\[
B: V^* \to V^*, g \mapsto g \circ A.
\]
证明 \( f, Bf, B^2f, \cdots, B^{n-1}f \) 构成 \( V^* \) 的基的充要条件是 \( A \) 的任一非 \( 0 \) 不变子空间都不是 \( \ker f \) 的子空间.
\end{example}
\begin{note}
联想\hyperref[theorem:循环子空间的基本性质]{循环子空间的基本性质}即\( \text{span}\{x, Ax, A^2x, \cdots\} \) 是包含 \( x \) 的最小 \( A- \) 不变子空间.
\end{note}
\begin{proof}
由\refcor{corollary:对偶空间的基的充要条件}知 \( f, Bf, B^2f, \cdots, B^{n-1}f \) 构成 \( V^* \) 的基等价于
\begin{align}
\bigcap_{k=0}^{n-1} \ker B^k f = \{0\}. \label{eq:::---12312323489779789}
\end{align}
于是由
\begin{align*}
&\quad \quad \,\,x\in \bigcap_{k=0}^{n-1}{\mathrm{ker}B^kf}=\bigcap_{k=0}^{n-1}{\mathrm{ker}\left( f\circ A^k \right)}
\\
&\Longleftrightarrow f(A^ix)=0,\forall i=0,1,2,\cdots ,n-1
\\
&\Longleftrightarrow A^ix\in \mathrm{ker}f,\forall i=0,1,2,\cdots ,n-1
\\
&\xLeftrightarrow{\text{\refthe{theorem:循环子空间的基}}}\mathrm{span}\{x,Ax,A^2x,\cdots \}=\mathrm{span}\{x,Ax,\cdots ,A^{n-1}x\}\subset \mathrm{ker}f
\\
&\xLeftrightarrow{\text{\hyperref[theorem:循环子空间的基本性质]{循环子空间的基本性质}}} \text{包含}x\text{的最小}A-\text{不变子空间}\subset \mathrm{ker}f.
\end{align*}
知 \eqref{eq:::---12312323489779789} 成立等价于只有$0\in \bigcap_{k=0}^{n-1}{\mathrm{ker}B^kf}$,等价于只有包含$0$的最小$A-$不变子空间(即$\{0\}$)含于$\ker f$,也等价于含于 \( \ker f \) 的$A-$不变子空间只能是 \( \{0\} \), 这就等价于 \( A \) 的任一非 \( \{0\} \) 不变子空间都不是 \( \ker f \) 的子空间.

\end{proof}

\begin{theorem}\label{theorem:函数线性无关的充要条件}
函数 \( f_1, f_2, \cdots, f_m \) 线性无关的充要条件是: 存在 \( m \) 个数 \( x_i, i = 1, 2, \cdots, m \), 使得
\begin{align}
\begin{vmatrix}
f_1(x_1) & f_2(x_1) & \cdots & f_m(x_1) \\
f_1(x_2) & f_2(x_2) & \cdots & f_m(x_2) \\
\vdots & \vdots & \ddots & \vdots \\
f_1(x_m) & f_2(x_m) & \cdots & f_m(x_m)
\end{vmatrix} \neq 0 \label{eq::::------89237878971249}
\end{align}
\end{theorem}
\begin{remark}
本题并未指出 \( f \) 定义域, 因此其定义域未必是数字.
\end{remark}
\begin{proof}
充分性: 设存在 \( m \) 个数 \( x_i, i = 1, 2, \cdots, m \), 使得 \eqref{eq::::------89237878971249} 成立. 考虑关于 \( c_1, c_2, \cdots, c_m \) 的齐次线性方程组
\[
\sum_{i=1}^m c_i f_i(x_j) = 0, j = 1, 2, \cdots, m.
\]
我们由 \eqref{eq::::------89237878971249} 知其系数矩阵可逆, 因此线性方程组只有 0 解, 这就证明了 \( f_1, f_2, \cdots, f_m \) 线性无关.

必要性: 假定 \( f_1, f_2, \cdots, f_m \) 线性无关, 不妨设定义域$D$有不少于 \( m \) 个不同的点,否则,不妨设定义域$D$内有$k\leqslant m$个点,则考虑关于 \( c_1, c_2, \cdots, c_m \) 的齐次线性方程组
\[
\sum_{i=1}^m c_i f_i(x_j) = 0, j = 1, 2, \cdots, k.
\]
因为$k\leqslant m$,所以上述方程有无穷多个解,可从中任取一组非零解$(c_1',c_2',\cdots,c_m')$,于是
\begin{align*}
\sum_{i=1}^m c_i' f_i(x) = 0, \forall x \in D.
\end{align*}
故\( f_1, f_2, \cdots, f_m \) 线性相关.

我们用归纳法, \( m = 1 \) 命题显然成立, 假设命题对不超过 \( m - 1 \) 时成立, 则考虑 \( m \) 时, 对 \( f_1, f_2, \cdots, f_{m - 1} \) 用归纳假设存在 \( x_i, i = 1, 2, \cdots, m - 1 \) 使得
\begin{align}\label{eq:::----3345215344}
\begin{vmatrix}
f_1(x_1) & f_2(x_1) & \cdots & f_{m - 1}(x_1) \\
f_1(x_2) & f_2(x_2) & \cdots & f_{m - 1}(x_2) \\
\vdots & \vdots & \ddots & \vdots \\
f_1(x_{m - 1}) & f_2(x_{m - 1}) & \cdots & f_{m - 1}(x_{m - 1})
\end{vmatrix} \neq 0,
\end{align}
因为 \( f_1, f_2, \cdots, f_m \) 线性无关,所以存在$x_m$使得关于$k_1,k_2,\cdots,k_m$的方程
\begin{align}\label{eq:::----12325243342412}
k_1f_1\left( x_m \right) +k_2f_2\left( x_m \right) +\cdots +k_mf_m\left( x_m \right) =0.
\end{align}
只有零解.将
\[
\begin{vmatrix}
f_1(x_1)&f_2(x_1)&\cdots&f_m(x_1)\\
f_1(x_2)&f_2(x_2)&\cdots&f_m(x_2)\\
\vdots&\vdots&\ddots&\vdots\\
f_1(x_m)&f_2(x_m)&\cdots&f_m(x_m)\\
\end{vmatrix}
\]
按最后一行展开得
\[
\begin{vmatrix}
f_1(x_1)&f_2(x_1)&\cdots&f_m(x_1)\\
f_1(x_2)&f_2(x_2)&\cdots&f_m(x_2)\\
\vdots&\vdots&\ddots&\vdots\\
f_1(x_m)&f_2(x_m)&\cdots&f_m(x_m)\\
\end{vmatrix}=a_1f_1(x_m)+a_2f_2(x_m)+\cdots+a_mf_m(x_m),
\]
其中$a_i$为$(m,i)$元的代数余子式.由\eqref{eq:::----3345215344}式可知
\[
a_m=\begin{vmatrix}
f_1(x_1)&f_2(x_1)&\cdots&f_m(x_1)\\
f_1(x_2)&f_2(x_2)&\cdots&f_m(x_2)\\
\vdots&\vdots&\ddots&\vdots\\
f_1(x_{m-1})&f_2(x_{m-1})&\cdots&f_{m-1}(x_{m-1})\\
\end{vmatrix}\ne 0.
\]
故由方程\eqref{eq:::----12325243342412}只有零解知
\[
a_1f_1(x_m)+a_2f_2(x_m)+\cdots+a_mf_m(x_m)\ne 0,
\]
即
\[
\begin{vmatrix}
f_1(x_1)&f_2(x_1)&\cdots&f_m(x_1)\\
f_1(x_2)&f_2(x_2)&\cdots&f_m(x_2)\\
\vdots&\vdots&\ddots&\vdots\\
f_1(x_m)&f_2(x_m)&\cdots&f_m(x_m)\\
\end{vmatrix}\ne 0.
\]
这就证明了存在 \( x_m \) 使得 \eqref{eq::::------89237878971249} 成立, 我们完成了证明.

\end{proof}

\begin{example}
设 \( X \subset C[0,1] \) 的有限维子空间, 证明 \( X \) 中函数列逐点收敛蕴含一致收敛.
\end{example}
\begin{note}
证明的想法是把函数列逐点收敛转化为系数的收敛, 从而一致收敛.
\end{note}
\begin{proof}
设 \( f_1, f_2, \cdots, f_m \) 是 \( X \) 的一组基, 我们知道存在 \( m \) 个数 \( x_i, i = 1, 2, \cdots, m \), 使得
\begin{align}
\begin{vmatrix}
f_1(x_1) & f_2(x_1) & \cdots & f_m(x_1) \\
f_1(x_2) & f_2(x_2) & \cdots & f_m(x_2) \\
\vdots & \vdots & \ddots & \vdots \\
f_1(x_m) & f_2(x_m) & \cdots & f_m(x_m)
\end{vmatrix} \neq 0. \label{eq:::---4423887789}
\end{align}
对函数列\(\{f^{(k)}\} \subset X, k = 1, 2, \cdots \), 我们知道有表示
\[
f^{(k)}(x) = \sum_{j=1}^m c_j^{(k)} f_j(x), \lim_{k \to \infty} f^{(k)}(x) = f(x), \forall x \in [0,1].
\]
于是由 \eqref{eq:::---4423887789}知
\[
\begin{pmatrix}
f_1(x_1) & f_2(x_1) & \cdots & f_m(x_1) \\
f_1(x_2) & f_2(x_2) & \cdots & f_m(x_2) \\
\vdots & \vdots & \ddots & \vdots \\
f_1(x_m) & f_2(x_m) & \cdots & f_m(x_m)
\end{pmatrix}
\begin{pmatrix}
c_1^{(k)} \\
c_2^{(k)} \\
\vdots \\
c_m^{(k)}
\end{pmatrix}
=
\begin{pmatrix}
f^{(k)}(x_1) \\
f^{(k)}(x_2) \\
\vdots \\
f^{(k)}(x_m)
\end{pmatrix}
\]
即
\[
\begin{pmatrix}
c_1^{(k)} \\
c_2^{(k)} \\
\vdots \\
c_m^{(k)}
\end{pmatrix}
=
\begin{pmatrix}
f_1(x_1) & f_2(x_1) & \cdots & f_m(x_1) \\
f_1(x_2) & f_2(x_2) & \cdots & f_m(x_2) \\
\vdots & \vdots & \ddots & \vdots \\
f_1(x_m) & f_2(x_m) & \cdots & f_m(x_m)
\end{pmatrix}^{-1}
\begin{pmatrix}
f^{(k)}(x_1) \\
f^{(k)}(x_2) \\
\vdots \\
f^{(k)}(x_m)
\end{pmatrix}
\]
因此存在 \( c_j \) 使得
\[
\lim_{k \to \infty} c_j^{(k)} = c_j, j = 1, 2, \cdots, m.
\]
现在就有
\[
\lim_{k \to \infty} f^{(k)}(x) = \sum_{j=1}^m c_j f_j(x), \forall x \in [0,1].
\]
又
\[
\begin{aligned}
\sup_{x \in [0,1]} \left| f^{(k)}(x) - f(x) \right| &\leqslant \sup_{x \in [0,1]} \sum_{j=1}^n \left| c_j^{(k)} - c_j \right| \cdot \left| f_j(x) \right| \\
&\leqslant \sup_{\substack{1 \leqslant j \leqslant m, \\ x \in [0,1]}} \left| f_j(x) \right| \cdot \sum_{j=1}^n \left| c_j^{(k)} - c_j \right| \to 0, k \to \infty.
\end{aligned}
\]
这就证明了 \( X \) 中函数列逐点收敛蕴含一致收敛.

\end{proof}

\begin{theorem}\label{theorem:线性变换互素分解对应核子空间直和分解}
设 $V$ 是域 $\mathbb{F}$ 上的线性空间,$A$ 是 $V$ 上线性变换,$f, f_i \in \mathbb{F}[x], i = 1,2,\cdots, s$ 满足 $f = f_1f_2\cdots f_s$ 并有 $f_1, f_2, \cdots, f_s$ 两两互素. 则
$$\ker f(A) = \ker f_1(A) \oplus \ker f_2(A) \oplus \cdots \oplus \ker f_s(A).$$
\end{theorem}
\begin{note}
这个定理最常见的情况是:当$f$为$A$的零化多项式时,就有$\ker f(A)=V$,此时利用这个定理就能得到一个全空间的直和分解.
\end{note}
\begin{proof}
当 $s = 2$, 此时由裴蜀等式得 $p_1, p_2 \in \mathbb{F}[x]$ 使得 $p_1f_1 + p_2f_2 = 1$. 于是对 $\alpha \in \ker f(A)$, 我们有
$$\alpha = p_1(A)f_1(A)\alpha + p_2(A)f_2(A)\alpha.$$
定义
$$\alpha_1 \triangleq p_2(A)f_2(A)\alpha,\quad \alpha_2 \triangleq p_1(A)f_1(A)\alpha.$$
现在
$$f_1(A)\alpha_1 = p_2(A)f_1(A)f_2(A)\alpha = p_2(A)f(A)\alpha = 0 \Rightarrow \alpha_1 \in \ker f_1(A);$$
$$f_2(A)\alpha_2 = p_1(A)f_1(A)f_2(A)\alpha = p_1(A)f(A)\alpha = 0 \Rightarrow \alpha_2 \in \ker f_2(A).$$
因此我们的确有
$$\ker f(A) = \ker f_1(A) + \ker f_2(A).$$
现在设 $\alpha \in \ker f_1(A) \cap \ker f_2(A)$, 于是 $\alpha = p_1(A)f_1(A)\alpha + p_2(A)f_2(A)\alpha = 0$, 即
$$\ker f(A) = \ker f_1(A) \oplus \ker f_2(A).$$
对 $s > 2$, 我们考虑归纳法. 假设命题对 $s - 1$ 已经成立, 设 $g(x) \triangleq f_2(x)f_3(x)\cdots f_s(x)$, 则由\refpro{proposition:两两互素的多项式组的乘积也互素}知$f_1, g$ 互素. 由 $s = 2$ 的结论知
$$\ker f(A) = \ker f_1(A) \oplus \ker g(A).$$
对 $g$ 用归纳假设得
$$\ker f(A) = \ker f_1(A) \oplus \ker f_2(A) \oplus \cdots \oplus \ker f_s(A).$$

\end{proof}

\begin{example}
设 $B_i \in \mathbb{C}^{n \times n}, i = 1, 2, \cdots, k$ 是幂等矩阵, $A = B_1 B_2 \cdots B_k$, 证明:
$$
r(I - A) \leqslant k(n - r(A)).
$$
\end{example}
\begin{proof}
将$A$,$B_i$都看作在$\mathbb{C}^n$上左乘诱导的线性变换.由$B_i$是幂等矩阵知
\begin{align*}
B^2 = B \Longleftrightarrow \ker (B_i^2 - B_i) = \mathbb{C}^n,\ i = 1,2,\cdots,k.
\end{align*}
由\refthe{theorem:线性变换互素分解对应核子空间直和分解}知
\begin{align*}
\mathbb{C}^n = \ker (B_i^2 - B_i) = \ker B_i \oplus \ker (I - B_i),\ i = 1,2,\cdots,k.
\end{align*}
再结合\hyperref[theorem:值域和核空间维数之和等于原像空间维数]{维数公式}可得
\begin{align*}
n &= \dim \ker B_i + \dim \ker (I - B_i) = n - \dim \mathrm{Im} B_i + n - \dim \mathrm{Im} (I - B_i) \\
&= n - r(B_i) + n - r(I - B_i),\ i = 1,2,\cdots,k.
\end{align*}
因此
\begin{align*}
r(I - B_i) + r(B_i) = n,\ i = 1,2,\cdots,k. 
\end{align*}
注意到
\begin{align*}
I - A = I - B_1B_2\cdots B_k = (I - B_1)B_2\cdots B_k + (I - B_2)B_3\cdots B_k + \cdots + (I - B_{k-1})B_k + I - B_k,
\end{align*}
故由\hyperref[proposition:矩阵秩的基本公式]{矩阵秩的基本公式(2)}知
\begin{align*}
r(A) = r(B_1B_2\cdots B_k) \leqslant r(B_i),\ i = 1,2,\cdots,k.
\end{align*}
从而再由\hyperref[proposition:矩阵秩的基本公式]{矩阵秩的基本公式(2)和(5)}知
\begin{align*}
r(I - A) &\leqslant r(I - B_1) + r(I - B_2) + \cdots + r(I - B_k) \\
&= kn - r(B_1) - r(B_2) - \cdots - r(B_k) \\
&\leqslant kn - kr(A) = k(n - r(A)).
\end{align*}

\end{proof}

\begin{example}
设 $V$ 是域 $\mathbb{F}$ 上的线性空间, $A$ 是 $V$ 上线性变换. 设 $f, f_i \in \mathbb{F}[x], i = 1, 2, \cdots, s$ 满足 $f = f_1f_2\cdots f_s$ 并有 $f_1, f_2, \cdots, f_s$ 两两互素, $f$ 是 $A$ 的零化多项式. 证明:
$$
V = \mathrm{Im} \frac{f(A)}{f_1(A)} \oplus \mathrm{Im} \frac{f(A)}{f_2(A)} \oplus \cdots \oplus \mathrm{Im} \frac{f(A)}{f_s(A)}.
$$
\end{example}
\begin{proof}
由\refthe{theorem:线性变换互素分解对应核子空间直和分解},我们有
$$
\ker f(A) = \ker f_1(A) \oplus \ker f_2(A) \oplus \cdots \oplus \ker f_s(A).
$$
设
$$
\alpha = \frac{f(A)}{f_i(A)} \beta \in \mathrm{Im} \frac{f(A)}{f_i(A)},
$$
我们有
$$
f_i(A) \alpha = f(A) \beta = 0 \implies \alpha \in \ker f_i(A),
$$
故$\mathrm{Im}\frac{f\left( A \right)}{f_i\left( A \right)}\subseteq \mathrm{ker}f_i\left( A \right) ,i=1,2,\cdots ,s.$
于是由\refpro{proposition:直和分解的每个子空间的和还是直和}知
$$
\mathrm{Im} \frac{f(A)}{f_1(A)} + \mathrm{Im} \frac{f(A)}{f_2(A)} + \cdots + \mathrm{Im} \frac{f(A)}{f_s(A)}
$$
是直和.

由\refpro{proposition:f除以两两互素多项式得到的多项式组互素}知 $\frac{f}{f_i}, i = 1, 2, \cdots, s$ 互素, 由裴蜀等式我们知道存在 $u_i \in \mathbb{F}[x], i = 1, 2, \cdots, s$ 使得
$$
\sum_{i=1}^s u_i \frac{f}{f_i} = 1.
$$
于是对 $\alpha \in V$, 我们有
$$
\alpha = \sum_{i=1}^s \frac{f(A)}{f_i(A)} (u_i(A) \alpha) \in \mathrm{Im} \frac{f(A)}{f_1(A)} \oplus \mathrm{Im} \frac{f(A)}{f_2(A)} \oplus \cdots \oplus \mathrm{Im} \frac{f(A)}{f_s(A)}.
$$
现在我们知道
$$
V = \mathrm{Im} \frac{f(A)}{f_1(A)} \oplus \mathrm{Im} \frac{f(A)}{f_2(A)} \oplus \cdots \oplus \mathrm{Im} \frac{f(A)}{f_s(A)}.
$$

\end{proof}




































\end{document}
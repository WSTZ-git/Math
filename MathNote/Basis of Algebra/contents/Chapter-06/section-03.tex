\documentclass[../../main.tex]{subfiles}
\graphicspath{{\subfix{../../image/}}} % 指定图片目录,后续可以直接使用图片文件名。

% 例如:
% \begin{figure}[H]
% \centering
% \includegraphics[scale=0.4]{图.png}
% \caption{}
% \label{figure:图}
% \end{figure}
% 注意:上述\label{}一定要放在\caption{}之后,否则引用图片序号会只会显示??.

\begin{document}

\section{特征值与特征向量}

\begin{definition}[线性变换的特征值和特征向量]
设$\varphi$是数域$\mathbb{K}$上线性空间$V$上的线性变换,若$\lambda_0 \in \mathbb{K}, x \in V$且$x \neq 0$,使
\begin{align*}
\varphi(x) = \lambda_0 x,
\end{align*}
则称$\lambda_0$是线性变换$\varphi$的一个\textbf{特征值},向量$x$称为$\varphi$关于特征值$\lambda_0$的\textbf{特征向量}.
\end{definition}
\begin{note}
显然$\varphi$的关于特征值$\lambda_0$的全体特征向量加上零向量构成$V$的子空间.
\end{note}

\begin{definition}[线性变换的特征子空间]
设 $\lambda_0$ 是线性空间 $V$ 上的线性变换 $\varphi$ 的特征值, 令
\begin{align*}
V_{\lambda_0} = \{\alpha \in V \mid \varphi(\alpha) = \lambda_0 \alpha\} 
= \{\alpha \in V \mid \alpha \text{是} \varphi \text{的属于} \lambda_0 \text{的特征向量}\} \cup \{0\},
\end{align*}
则显然$V_{\lambda_0}$ 是 $V$ 的子空间, 称为 $\varphi$ 的属于特征值 $\lambda_0$ 的\textbf{特征子空间}.
\end{definition}
\begin{note}
显然$V_{\lambda_0}$是$\varphi$的不变子空间.
\end{note}

\begin{definition}[矩阵的特征值和特征向量]
设$A$是数域$\mathbb{F}$上的$n$阶方阵,若存在$\lambda_0 \in \mathbb{F}$及$n$维非零列向量$\alpha$,使
\begin{align*}
A \alpha = \lambda_0 \alpha
\end{align*}
式成立,则称$\lambda_0$为矩阵$A$的一个\textbf{特征值},$\alpha$为$A$关于特征值$\lambda_0$的\textbf{特征向量}.
\end{definition}

\begin{definition}[矩阵的特征子空间]
设 $\lambda_0$ 是 $\mathbb{F}$ 上的 $n$ 阶矩阵 $A$ 的特征值, 令
\begin{align*}
V_{\lambda_0} = \{x \in \mathbb{F}^n \mid Ax = \lambda_0 x\} 
= \{x \in \mathbb{F}^n \mid x \text{是} A \text{的属于} \lambda_0 \text{的特征向量}\} \cup \{0\},
\end{align*}
则 $V_{\lambda_0}$ 是线性方程组 $(\lambda_0 I_n - A)x = 0$ 的解空间, 从而是 $\mathbb{F}^n$ 的子空间, 称为 $A$ 的属于特征值 $\lambda_0$ 的\textbf{特征子空间}。
\end{definition}

\begin{definition}[特征多项式]
设$A$是$n$阶方阵,称$|\lambda I_n - A|$为$A$的\textbf{特征多项式}.
\end{definition}

\begin{theorem}[特征值的和与积]\label{theorem:特征值的和与积}
矩阵$A$的$n$个特征值的和与积分别为
\begin{gather*}
\lambda_1 + \lambda_2 + \cdots + \lambda_n = \mathrm{tr}(A), \\
\lambda_1 \lambda_2 \cdots \lambda_n = |A|.
\end{gather*}
\end{theorem}
\begin{proof}
设
\begin{align*}
|\lambda I_n - A| &= \lambda^n + a_1 \lambda^{n-1} + \cdots + a_{n-1} \lambda + a_n \\
&= (\lambda - \lambda_1)(\lambda - \lambda_2) \cdots (\lambda - \lambda_n).
\end{align*}
由Vieta定理知$\lambda_1 + \lambda_2 + \cdots + \lambda_n = -a_1$,$\lambda_1 \lambda_2 \cdots \lambda_n = (-1)^n a_n$.由\hyperref[example:特征行列式写成多项式形式的系数]{例题\ref{example:特征行列式写成多项式形式的系数}}可知$a_1 = -(a_{11} + a_{22} + \cdots + a_{nn}) = -\mathrm{tr}(A)$,$a_n = (-1)^n |A|$.因此矩阵$A$的$n$个特征值的和与积分别为
\begin{align*}
\lambda_1 + \lambda_2 + \cdots + \lambda_n &= \mathrm{tr}(A), \\
\lambda_1 \lambda_2 \cdots \lambda_n &= |A|.
\end{align*}

\end{proof}

\begin{definition}[特征多项式]
设$\varphi$是线性空间$V$上的线性变换,$\varphi$在$V$的某组基下的表示矩阵为$A$,由\hyperref[theorem:相似矩阵有相同特征值]{相似矩阵有相同特征值}知$|\lambda I_n - A|$与基或表示矩阵的选取无关,称$|\lambda I_n - A|$为$\varphi$的\textbf{特征多项式},记为$|\lambda I_V - \varphi|$.
\end{definition}

\begin{theorem}[复方阵必相似于上三角阵]\label{theorem:复方阵必相似于上三角阵}
任何复方阵必复相似于一个上三角阵,并且对角元素都是其特征值.
\end{theorem}
\begin{remark}
一般数域$\mathbb{K}$上的矩阵未必相似于上三角阵.
\end{remark}
\begin{proof}
设$A$是$n$阶复方阵,现对$n$用数学归纳法.当$n=1$时结论显然成立.假设对$n-1$阶矩阵结论成立,现对$n$阶矩阵$A$来证明.设$\lambda_1$是$A$的一个特征值,则存在非零列向量$\alpha_1$,使
\begin{align*}
A \alpha_1 = \lambda_1 \alpha_1.
\end{align*}
将$\alpha_1$作为$C_n$的一个基向量,并扩展为$C_n$的一组基$\{\alpha_1, \alpha_2, \cdots, \alpha_n\}$.将这些基向量按照列分块方式拼成矩阵$P = (\alpha_1, \alpha_2, \cdots, \alpha_n)$,则$P$为$n$阶非异阵,且
\begin{align*}
AP &= A(\alpha_1, \alpha_2, \cdots, \alpha_n) = (A\alpha_1, A\alpha_2, \cdots, A\alpha_n) \\
&= (\alpha_1, \alpha_2, \cdots, \alpha_n) \begin{pmatrix}
\lambda_1 & * \\
O & A_1
\end{pmatrix},
\end{align*}
其中$A_1$是一个$n-1$阶方阵.注意到$P = (\alpha_1, \alpha_2, \cdots, \alpha_n)$非异,上式即为
\begin{align*}
P^{-1}AP = \begin{pmatrix}
\lambda_1 & * \\
O & A_1
\end{pmatrix}.
\end{align*}
因为$A_1$是一个$n-1$阶方阵,所以由归纳假设可知,存在$n-1$阶非异阵$Q$,使$Q^{-1}A_1Q$是一个上三角阵.令
\begin{align*}
R = \begin{pmatrix}
1 & O \\
O & Q
\end{pmatrix},
\end{align*}
则$R$是$n$阶非异阵,且
\begin{align*}
R^{-1}P^{-1}APR &= \begin{pmatrix}
1 & O \\
O & Q
\end{pmatrix}^{-1} \begin{pmatrix}
\lambda_1 & * \\
O & A_1
\end{pmatrix} \begin{pmatrix}
1 & O \\
O & Q
\end{pmatrix} \\
&= \begin{pmatrix}
1 & O \\
O & Q^{-1}
\end{pmatrix} \begin{pmatrix}
\lambda_1 & * \\
O & A_1
\end{pmatrix} \begin{pmatrix}
1 & O \\
O & Q
\end{pmatrix} \\
&= \begin{pmatrix}
\lambda_1 & * \\
O & Q^{-1}A_1Q
\end{pmatrix}.
\end{align*}
这是一个上三角阵,它与$A$相似,并且对角元素都是其特征值.

\end{proof}

\begin{corollary}\label{corollary:特征值全在矩阵元素的数域中则一定相似于上三角阵}
若数域$\mathbb{K}$上的$n$阶方阵$A$的特征值全在$\mathbb{K}$中,则存在$\mathbb{K}$上的非异阵$P$,使$P^{-1}AP$是一个上三角阵.
\end{corollary}
\begin{proof}
由\hyperref[theorem:复方阵必相似于上三角阵]{复方阵必相似于上三角阵的证明}类似可得.

\end{proof}

\begin{proposition}\label{proposition:线性变换或矩阵在复线性空间上至少存在一个特征值及与其特征向量}
\begin{enumerate}
\item 设$\varphi$为$n$维线性空间$V$上的线性变换,则$\varphi$在$V$上至少存在一个特征值$\lambda_0\in\mathbb{C}$及其特征向量$\alpha_0\in V$.

\item 设$A$为$n$阶复矩阵,则$A$在复数域上至少存在一个特征值$\lambda_0\in\mathbb{C}$及其特征向量$\alpha_0\in\mathbb{C}^n$.
\end{enumerate}
\end{proposition}
\begin{proof}
\begin{enumerate}
\item 任取\(V\)的一组基\(\{ e_1,e_2,\cdots ,e_n \}\), 设\(\varphi\)在这组基下的表示矩阵为\(A\), 由代数学基本定理可知, 特征多项式\(|\lambda I_V - \varphi| = |\lambda I_n - A|\)在复数域上至少有一个根\(\lambda_0\in \mathbb{C}\). 又由线性方程组理论可知, \((\lambda_0I_n - A)x = 0\)一定有非零解\((x_1,x_2,\cdots ,x_n)^T\), 即\(\lambda_0I_n\begin{pmatrix}
    x_1\\
    x_2\\
    \vdots\\
    x_n\\
    \end{pmatrix}=A\begin{pmatrix}
    x_1\\
    x_2\\
    \vdots\\
    x_n\\
    \end{pmatrix}\). 记\(\alpha_0 = (e_1,e_2,\cdots ,e_n)\begin{pmatrix}
    x_1\\
    x_2\\
    \vdots\\
    x_n\\
    \end{pmatrix}\), 则
    \begin{align*}
    &\varphi(\alpha_0)=\varphi(e_1,e_2,\cdots ,e_n)\begin{pmatrix}
    x_1\\
    x_2\\
    \vdots\\
    x_n\\
    \end{pmatrix}=(e_1,e_2,\cdots ,e_n)A\begin{pmatrix}
    x_1\\
    x_2\\
    \vdots\\
    x_n\\
    \end{pmatrix}\\
    &=(e_1,e_2,\cdots ,e_n)\cdot\lambda_0I_n\begin{pmatrix}
    x_1\\
    x_2\\
    \vdots\\
    x_n\\
    \end{pmatrix}=\lambda_0(e_1,e_2,\cdots ,e_n)\begin{pmatrix}
    x_1\\
    x_2\\
    \vdots\\
    x_n\\
    \end{pmatrix}=\lambda_0\alpha_0.
    \end{align*}
故$\varphi$在$V$上至少存在一个特征值$\lambda_0\in\mathbb{C}$及其特征向量$\alpha_0\in V$.

\item 由代数学基本定理可知, 特征多项式\(|\lambda I_n - A|\)在复数域上至少有一个根\(\lambda_0\in \mathbb{C}\).
又由线性方程组理论可知, \((\lambda_0I_n - A)x = 0\)一定有非零解\(\alpha_0\in \mathbb{C}^n\).故$A$在复数域上至少存在一个特征值$\lambda_0\in\mathbb{C}$及其特征向量$\alpha_0\in\mathbb{C}^n$.
\end{enumerate}

\end{proof}

\subsection{直接利用定义计算和证明}

\begin{example}\label{example:由矩阵诱导的线性变换的特征值与其诱导矩阵相同}
设 $V$ 是 $n$ 阶矩阵全体组成的线性空间, $\varphi$ 是 $V$ 上的线性变换: $\varphi(X) = AX$, 其中 $A$ 是一个 $n$ 阶矩阵。求证: $\varphi$ 和 $A$ 具有相同的特征值 (重数可能不同)。
\end{example}
\begin{proof}
设 $\lambda_0$ 是 $A$ 的特征值, $x_0$ 是对应的特征向量, 即 $Ax_0 = \lambda_0 x_0$。令 $X = (x_0, 0, \ldots, 0)$, 则 $\varphi(X) = AX = \lambda_0 X$ 且 $X \neq 0$, 因此 $\lambda_0$ 也是 $\varphi$ 的特征值.

反之, 设 $\lambda_0$ 是 $\varphi$ 的特征值, $X$ 是对应的特征向量, 即 $\varphi(X) = AX = \lambda_0 X$。令 $X = (x_1, x_2, \ldots, x_n)$ 为列分块, 设第 $i$ 个列向量 $x_i \neq 0$, 则 $Ax_i = \lambda_0 x_i$, 因此 $\lambda_0$ 也是 $A$ 的特征值.

\end{proof}

\begin{example}
设 $\lambda_1, \lambda_2$ 是矩阵 $A$ 的两个不同的特征值, $\alpha_1, \alpha_2$ 分别是 $\lambda_1, \lambda_2$ 的特征向量, 求证: $\alpha_1 + \alpha_2$ 必不是 $A$ 的特征向量。
\end{example}
\begin{proof}
用反证法, 设 $A(\alpha_1 + \alpha_2) = \mu (\alpha_1 + \alpha_2)$, 又
\begin{align*}
A(\alpha_1 + \alpha_2) = A\alpha_1 + A\alpha_2 
= \lambda_1 \alpha_1 + \lambda_2 \alpha_2,
\end{align*}
于是 $(\lambda_1 - \mu) \alpha_1 + (\lambda_2 - \mu) \alpha_2 = 0$。由于属于不同特征值的特征向量线性无关, 故有 $\lambda_1 = \mu, \lambda_2 = \mu$, 从而 $\lambda_1 = \lambda_2$, 引出矛盾.

\end{proof}

\begin{proposition}\label{proposition:线性变换不变子空间直和分解与特征值的关系}
设 $\varphi$ 是线性空间 $V$ 上的线性变换, $V$ 有一个直和分解:
\begin{align*}
V = V_1 \oplus V_2 \oplus \cdots \oplus V_m,
\end{align*}
其中 $V_i$ 都是 $\varphi$-不变子空间。
\begin{enumerate}[(1)]
\item 设 $\varphi$ 限制在 $V_i$ 上的特征多项式为 $f_i(\lambda)$, 求证: $\varphi$ 的特征多项式
\begin{align*}
f(\lambda) = f_1(\lambda) f_2(\lambda) \cdots f_m(\lambda).
\end{align*}

\item  设 $\lambda_0$ 是 $\varphi$ 的特征值, $V_0 = \{v \in V \mid \varphi(v) = \lambda_0 v\}$ 为特征子空间, $V_{i,0} = V_i \cap V_0 = \{v \in V_i \mid \varphi(v) = \lambda_0 v\}$, 求证:
\begin{align*}
V_0 = V_{1,0} \oplus V_{2,0} \oplus \cdots \oplus V_{m,0}.
\end{align*}
\end{enumerate}
\end{proposition}
\begin{proof}
\begin{enumerate}[(1)]
\item  取 $V_i$ 的一组基, 将它们拼成 $V$ 的一组基。记 $A_i$ 是 $\varphi$ 在 $V_i$ 上的限制在 $V_i$ 所取基下的表示矩阵, 则由\hyperref[theorem:在不变子空间基下的矩阵]{定理\ref{theorem:在不变子空间基下的矩阵}}可知 $\varphi$ 在 $V$ 的这组基下的表示矩阵为分块对角矩阵 $A = \operatorname{diag}(A_1, A_2, \cdots, A_m)$, 于是
\begin{align*}
f(\lambda) &= |\lambda I_n - A| = |\lambda I - A_1| |\lambda I - A_2| \cdots |\lambda I - A_m|,
\end{align*}
即 $f(\lambda) = f_1(\lambda) f_2(\lambda) \cdots f_m(\lambda)$.

\item 任取 $\alpha \in V_0$, 设 $\alpha = \alpha_1 + \alpha_2 + \cdots + \alpha_m$, 其中 $\alpha_i \in V_i$, 则
\begin{align*}
\varphi(\alpha_1) + \varphi(\alpha_2) + \cdots + \varphi(\alpha_m) = \varphi(\alpha) = \lambda_0 \alpha = \lambda_0 \alpha_1 + \lambda_0 \alpha_2 + \cdots + \lambda_0 \alpha_m.
\end{align*}
注意到 $\varphi(\alpha_i) \in V_i,\varphi(\alpha) = \lambda_0 \alpha\in V$, 故由\hyperref[theorem:直和的等价条件]{直和的等价条件(5)}可得 $\varphi(\alpha_i) = \lambda_0 \alpha_i$, 即 $\alpha_i \in V_{i,0}$, 从而 $V_0=V_{1,0}+V_{2,0}+\cdots +V_{m,0}$.注意到 $V_{i,0} \subseteq V_i$, 故
\begin{align*}
V_{i,0} \cap (V_{1,0} + \cdots + V_{i-1,0}) \subseteq V_i \cap (V_1 + \cdots + V_{i-1}) = \{0\}, \quad 2 \leqslant  i \leqslant  m,
\end{align*}
于是由\hyperref[theorem:直和的等价条件]{直和的等价条件(2)}可知上述为直和。
\end{enumerate}

\end{proof}

\begin{corollary}\label{corollary:分块准对角矩阵的任一特征值的代数重数与几何重数}
对分块对角矩阵 $A = \operatorname{diag}\{A_1, A_2, \cdots, A_m\}$ 的任一特征值$\lambda_0$, 其代数重数等于每个分块的代数重数之和, 其几何重数等于每个分块的几何重数之和.
\end{corollary}
\begin{proof}
将\hyperref[proposition:线性变换不变子空间直和分解与特征值的关系]{命题\ref{proposition:线性变换不变子空间直和分解与特征值的关系}}的条件和结论代数化之后,即可得到结论.

\end{proof}

\begin{proposition}[特征向量的延拓]\label{proposition:特征向量的延拓}
设 $n$ 阶分块对角矩阵 $A = \operatorname{diag}\{A_1, A_2, \cdots, A_m\}$, 其中 $A_i$ 是 $n_i$ 阶矩阵。
\begin{enumerate}[(1)]
\item 任取 $A_i$ 的特征值 $\lambda_i$ 及其特征向量 $x_i \in \mathbb{C}^{n_i}$, 求证: 可在 $x_i$ 的上下添加适当多的零, 得到非零向量 $\tilde{x}_i \in \mathbb{C}^n$, 使得 $A \tilde{x}_i = \lambda_i \tilde{x}_i$, 即 $\tilde{x}_i$ 是 $A$ 关于特征值 $\lambda_i$ 的特征向量, 称为 $x_i$ 的\textbf{延拓}.

\item 任取 $A$ 的特征值 $\lambda_0$, 并设 $\lambda_0$ 是 $A_{i_1}, \cdots, A_{i_r}$ 的特征值, 但不是其他 $A_j$ ($1 \leqslant  j \leqslant  m, j \neq i_1, \cdots, i_r$) 的特征值, 求证: $A$ 关于特征值 $\lambda_0$ 的特征子空间的一组基可取为 $A_{i_1} (1 \leqslant  k \leqslant  r)$ 关于特征值 $\lambda_0$ 的特征子空间的一组基的延拓的并集.
\end{enumerate} 
\end{proposition}
\begin{proof}
\begin{enumerate}[(1)]
\item 令 $\tilde{x}_i = (0, \cdots, 0, x_i, 0, \cdots, 0)'$, 即 $\tilde{x}_i$ 的第 $i$ 块为 $x_i$, 其余块均为 $0$, 显然 $\tilde{x}_i \neq 0$。容易验证 $A \tilde{x}_i = \lambda_i \tilde{x}_i$, 故结论成立。

\item 由\hyperref[proposition:线性变换不变子空间直和分解与特征值的关系]{命题\ref{proposition:线性变换不变子空间直和分解与特征值的关系}(2)}以及\hyperref[theorem:直和的等价条件]{直和的等价条件(5)}即得.
\end{enumerate}

\end{proof}

\begin{example}
设 $A$ 是 $n$ 阶整数矩阵, $p, q$ 为互素的整数且 $q > 1$。求证:矩阵方程 $Ax = \frac{p}{q} x$ 必无非零解。
\end{example}
\begin{proof}
用反证法。设上述矩阵方程有非零解, 则 $\frac{p}{q}$ 为 $A$ 的特征值, 即为特征多项式 $f(\lambda) = \lambda^n + a_1 \lambda^{n-1} + \cdots + a_{n-1} \lambda + a_n$ 的根。由于 $A$ 是整数矩阵, 故 $f(\lambda)$ 为整数系数多项式。由\hyperref[theorem:整数系数多项式有有理根的必要条件]{整数系数多项式有有理根的必要条件}可知 $q \mid 1$, 从而 $q = \pm 1$,于是$q\mid p$,这与$p,q$互素矛盾。

\end{proof}

\begin{example}
求下列 $n$ 阶矩阵的特征值:
\[
A = \begin{pmatrix}
0 & a & \cdots & a & a \\
b & 0 & \cdots & a & a \\
\vdots & \vdots & \ddots & \vdots & \vdots \\
b & b & \cdots & 0 & a \\
b & b & \cdots & b & 0
\end{pmatrix}.
\]
\end{example}
\begin{solution}
若 $a = 0$ 或 $b = 0$, 则 $A$ 是主对角元全为零的下三角或上三角矩阵, 故 $A$ 的特征值全为零. 设 $a \neq 0, b \neq 0$, 则由\hyperref[proposition:小拆分法经典例题]{命题\ref{proposition:小拆分法经典例题}}可知:
若 $a \neq b$, 则 $|\lambda I_n-A|=\frac{a(\lambda +b)^n-b(\lambda +a)^n}{a-b}$. 设 $\frac{b}{a}$ 的 $n$ 次方根为 $\omega_i (1 \leqslant  i \leqslant  n)$, 则
\begin{align*}
&|\lambda I_n-A|=\frac{a(\lambda +b)^n-b(\lambda +a)^n}{a-b}=0\Rightarrow \left( \frac{\lambda +b}{\lambda +a} \right) ^n=\frac{b}{a}
\\
&\Rightarrow \frac{\lambda +b}{\lambda +a}=w_i\left( 1\le i\le n \right) \Rightarrow \lambda =\frac{a\omega _i-b}{1-\omega _i}(1\le i\le n).
\end{align*}
从而$A$ 的特征值为 $\frac{a \omega_i - b}{1 - \omega_i} (1 \leqslant  i \leqslant  n)$.

若 $a = b$, 则 $|\lambda I_n - A| = (\lambda - (n - 1)a)(\lambda + a)^{n-1}$, 从而 $A$ 的特征值为 $(n - 1)a$ (1 重), $-a$ ($n - 1$ 重).

综上,容易验证当$a=b=0$或$ab\ne0$时,$A$有完全的特征向量系或有$n$个不同的特征值,从而此时$A$可对角化.若$A$可对角化也不难得到$a=b=0$或$ab\ne0$.故$A$可对角化的充分必要条件是$a=b=0$或$ab\ne0$.

\end{solution}

\subsection{正向利用矩阵的多项式}

\begin{definition}[矩阵多项式]
若$A$是一个$n$阶矩阵,$f(x) = a_m x^m + a_{m-1} x^{m-1} + \cdots + a_1 x + a_0$是一个多项式,记
\begin{align*}
f(A) = a_m A^m + a_{m-1} A^{m-1} + \cdots + a_1 A + a_0 I_n.
\end{align*}
\end{definition}

\begin{proposition}\label{proposition:矩阵多项式的特征值就是原特征值代入多项式得到的数}
设$n$阶矩阵$A$的全部特征值为$\lambda_1, \lambda_2, \cdots, \lambda_n$, $f(x)$是一个多项式,则$f(A)$的全部特征值为$f(\lambda_1), f(\lambda_2), \cdots, f(\lambda_n)$.
\end{proposition}
\begin{remark}
这个命题告诉我们:如果能够将一个复杂矩阵写成一个简单矩阵的多项式,那么就可以由简单矩阵的特征值得到复杂矩阵的特征值.
\end{remark}
\begin{proof}
因为任一$n$阶矩阵均复相似于上三角阵,可设
\begin{align*}
P^{-1}AP = \begin{pmatrix}
\lambda_1 & * & \cdots & * \\
0 & \lambda_2 & \cdots & * \\
\vdots & \vdots & \ddots & \vdots \\
0 & 0 & \cdots & \lambda_n
\end{pmatrix}.
\end{align*}
因为上三角阵的和、数乘及乘方仍是上三角阵,经计算不难得到
\begin{align*}
P^{-1}f(A)P &= f(P^{-1}AP) = \begin{pmatrix}
f(\lambda_1) & * & \cdots & * \\
0 & f(\lambda_2) & \cdots & * \\
\vdots & \vdots & \ddots & \vdots \\
0 & 0 & \cdots & f(\lambda_n)
\end{pmatrix}.
\end{align*}
因此$f(A)$的全部特征值为$f(\lambda_1), f(\lambda_2), \cdots, f(\lambda_n)$.

\end{proof}

\begin{example}
设 $n$ 阶矩阵 $A$ 的全体特征值为 $\lambda_1, \lambda_2, \cdots, \lambda_n$, 求 $2n$ 阶矩阵
\[
\begin{pmatrix}
A^2 & A \\
A^2 & A
\end{pmatrix}
\]
的全体特征值。
\end{example}
\begin{proof}
由\hyperref[proposition:对角相同分块矩阵行列式计算]{命题\ref{proposition:对角相同分块矩阵行列式计算}}可知
\begin{align*}
\left| \lambda I_{2n} - \begin{pmatrix}
A^2 & A \\
A^2 & A
\end{pmatrix} \right| = \left| \begin{pmatrix}
\lambda I_n - A & -A^2 \\
-A^2 & \lambda I_n - A
\end{pmatrix} \right| = |\lambda I_n - A - A^2||\lambda I_n - A + A^2|.
\end{align*}
由\hyperref[proposition:矩阵多项式的特征值就是原特征值代入多项式得到的数]{命题\ref{proposition:矩阵多项式的特征值就是原特征值代入多项式得到的数}}可知 $A + A^2$ 的全体特征值为 $\lambda_i + \lambda_i^2 (1 \leqslant  i \leqslant  n)$, $A - A^2$ 的全体特征值为 $\lambda_i - \lambda_i^2 (1 \leqslant  i \leqslant  n)$, 因此所求矩阵的全体特征值为
\begin{align*}
\lambda_1 + \lambda_1^2, \lambda_1 - \lambda_1^2, \lambda_2 + \lambda_2^2, \lambda_2 - \lambda_2^2, \cdots, \lambda_n + \lambda_n^2, \lambda_n - \lambda_n^2.
\end{align*}

\end{proof}

\begin{proposition}[循环矩阵的特征值]\label{proposition:循环矩阵的特征值}
证明下列循环矩阵的特征值:
\[
A = \begin{pmatrix}
a_1 & a_2 & a_3 & \cdots & a_n \\
a_n & a_1 & a_2 & \cdots & a_{n-1} \\
a_{n-1} & a_n & a_1 & \cdots & a_{n-2} \\
\vdots & \vdots & \vdots & \ddots & \vdots \\
a_2 & a_3 & a_4 & \cdots & a_1
\end{pmatrix}.
\]
的特征值为 $$f(1), f(\omega_1), \cdots, f(\omega_{n-1}),$$
其中
\begin{align*}
\omega_k = \cos \frac{2k\pi}{n} + i \sin \frac{2k\pi}{n}, \quad 0 \leqslant  k \leqslant  n - 1.
\end{align*}
\end{proposition}
\begin{proof}
设 $J = \begin{pmatrix}
O & I_{n-1} \\
1 & O
\end{pmatrix}$, $f(x) = a_1 + a_2 x + a_3 x^2 + \cdots + a_n x^{n-1}$, 则由\hyperref[proposition:循环矩阵的性质]{循环矩阵的性质2}可知 $A = f(J)$。经简单计算可得
\begin{align*}
|\lambda I_n-J|&=\left| \begin{matrix}
\lambda&		-1&		0&		\cdots&		0&		0\\
0&		\lambda&		-1&		\cdots&		0&		0\\
\vdots&		\vdots&		\vdots&		&		\vdots&		\vdots\\
0&		0&		0&		\cdots&		\lambda&		-1\\
-1&		0&		0&		\cdots&		0&		\lambda\\
\end{matrix} \right|\xlongequal{\text{按第一列展开}}\lambda \left| \begin{matrix}
\lambda&		-1&		\cdots&		0&		0\\
0&		\lambda&		\cdots&		0&		0\\
\vdots&		\vdots&		&		\vdots&		\vdots\\
0&		0&		\cdots&		\lambda&		-1\\
0&		0&		\cdots&		0&		\lambda\\
\end{matrix} \right|+\left( -1 \right) ^{n+1}\left( -1 \right) \left| \begin{matrix}
-1&		0&		\cdots&		0&		0\\
\lambda&		-1&		\cdots&		0&		0\\
\vdots&		\vdots&		&		\vdots&		\vdots\\
0&		0&		\cdots&		-1&		0\\
0&		0&		\cdots&		\lambda&		-1\\
\end{matrix} \right|
\\
&=\lambda ^n+\left( -1 \right) ^{n+2}\left( -1 \right) ^{n-1}=\lambda ^n-1,
\end{align*}
于是 $J$ 的特征值为
\begin{align*}
\omega_k = \cos \frac{2k\pi}{n} + i \sin \frac{2k\pi}{n}, \quad 0 \leqslant  k \leqslant  n - 1。
\end{align*}
因此 $A$ 的特征值为 $f(1), f(\omega_1), \cdots, f(\omega_{n-1})$。

\end{proof}

\begin{definition}[友矩阵]
\begin{align*}
\boldsymbol{A}=\left( \begin{matrix}
0&		0&		0&		\cdots&		0&		-a_0\\
1&		0&		0&		\cdots&		0&		-a_1\\
0&		1&		0&		\cdots&		0&		-a_2\\
\vdots&		\vdots&		\vdots&		&		\vdots&		\vdots\\
0&		0&		0&		\cdots&		0&		-a_{n-2}\\
0&		0&		0&		\cdots&		1&		-a_{n-1}\\
\end{matrix} \right) 
\end{align*}
称为多项式$f\left( x \right) =x^n+a_{n-1}x^{n-1}+\cdots +a_1x+a_0$的
\textbf{友矩阵}.
\end{definition}

\begin{proposition}[友矩阵的特征多项式及特征值]\label{proposition:多项式的友阵的特征多项式与特征值}
设首一多项式 $f(x) = x^n + a_{n-1} x^{n-1} + \cdots + a_1 x + a_0$, $f(x)$ 的友矩阵
\[
C = \begin{pmatrix}
0 & 0 & \cdots & 0 & -a_0 \\
1 & 0 & \cdots & 0 & -a_1 \\
0 & 1 & \cdots & 0 & -a_2 \\
\vdots & \vdots & \ddots & \vdots & \vdots \\
0 & 0 & \cdots & 1 & -a_{n-1}
\end{pmatrix}.
\]
\begin{enumerate}[(1)]
\item 求证:矩阵 $C$ 的特征多项式就是 $f(\lambda)$。

\item 设 $f(x)$ 的根为 $\lambda_1, \lambda_2, \cdots, \lambda_n$,$g(x)$ 为任一多项式,求以 $g(\lambda_1), g(\lambda_2), \cdots, g(\lambda_n)$ 为根的 $n$ 次多项式。 
\end{enumerate}
\end{proposition}
\begin{proof}
\begin{enumerate}[(1)]
\item \begin{align*}
&|xE-\boldsymbol{A}|=\left| \begin{matrix}
x&		0&		0&		\cdots&		0&		a_0\\
-1&		x&		0&		\cdots&		0&		a_1\\
0&		-1&		x&		\cdots&		0&		a_2\\
\vdots&		\vdots&		\vdots&		&		\vdots&		\vdots\\
0&		0&		0&		\cdots&		x&		a_{n-2}\\
0&		0&		0&		\cdots&		-1&		x+a_{n-1}\\
\end{matrix} \right|\xlongequal[i=n,n-1,\cdots ,2]{xr_i+r_{i-1}}\left| \begin{matrix}
0&		0&		0&		\cdots&		0&		x^n+a_{n-1}x^{n-1}+\cdots +a_1x+a_0\\
-1&		0&		0&		\cdots&		0&		x^{n-1}+a_{n-1}x^{n-2}+\cdots +a_2x+a_1\\
0&		-1&		0&		\cdots&		0&		x^{n-2}+a_{n-1}x^{n-3}+\cdots +a_3x+a_2\\
\vdots&		\vdots&		\vdots&		&		\vdots&		\vdots\\
0&		0&		0&		\cdots&		0&		x^2+a_{n-1}x+a_{n-2}\\
0&		0&		0&		\cdots&		-1&		x+a_{n-1}\\
\end{matrix} \right|
\\
&\xlongequal{\text{按第一行展开}}\left( x^n+a_{n-1}x^{n-1}+\cdots +a_1x+a \right) \left( -1 \right) ^{n+1}\left| \begin{matrix}
-1&		0&		0&		\cdots&		0\\
0&		-1&		0&		\cdots&		0\\
\vdots&		\vdots&		\vdots&		&		\vdots\\
0&		0&		0&		\cdots&		0\\
0&		0&		0&		\cdots&		-1\\
\end{matrix} \right|
\\
&=\left( x^n+a_{n-1}x^{n-1}+\cdots +a_1x+a \right) \left( -1 \right) ^{n+1}\left( -1 \right) ^{n-1}
\\
&=x^n+a_{n-1}x^{n-1}+\cdots +a_1x+a=f\left( x \right) .
\end{align*}

\item 由假设及(1)的结论可知 $\lambda_1, \lambda_2, \cdots, \lambda_n$ 是 $C$ 的全体特征值,故由\hyperref[proposition:矩阵多项式的特征值就是原特征值代入多项式得到的数]{命题\ref{proposition:矩阵多项式的特征值就是原特征值代入多项式得到的数}}可知$g(\lambda_1), g(\lambda_2), \cdots, g(\lambda_n)$ 是 $g(C)$ 的全体特征值,从而 $h(x) = |xI_n - g(C)|$ 即为所求的多项式。
\end{enumerate}

\end{proof}

\subsection{反向利用矩阵的多项式}

\begin{proposition}\label{proposition:矩阵适合的多项式其特征值也适合}
设$n$阶矩阵$A$适合一个多项式$g(x)$,即$g(A) = O$,则$A$的任一特征值$\lambda_0$也必适合$g(x)$,即$g(\lambda_0) = 0$.
\end{proposition}
\begin{proof}
{\color{blue}证法一:}
设$\alpha$是$A$关于特征值$\lambda_0$的特征向量,经简单计算得
\begin{align*}
g(\lambda_0) \alpha = g(A) \alpha = 0.
\end{align*}
而$\alpha \neq 0$,因此$g(\lambda_0) = 0$.

{\color{blue}证法二:}
设$A$的极小多项式为$m(x)$,则$m(x)\mid g(x)$,由\hyperref[proposition:极小多项式的性质]{极小多项式的性质(5)}及整除的传递性可知$(x-\lambda_0)\mid g(x)$,故$g(\lambda_0)=0.$

\end{proof}

\begin{proposition}[幂零矩阵关于特征值的充要条件]\label{proposition:幂零矩阵关于特征值的充要条件}
求证:$n$ 阶矩阵 $A$ 为幂零矩阵的充要条件是 $A$ 的特征值全为零。
\end{proposition}
\begin{proof}
若 $A$ 为幂零矩阵, 即存在正整数 $k$, 使得 $A^k = O$, 则由\hyperref[proposition:矩阵适合的多项式其特征值也适合]{命题\ref{proposition:矩阵适合的多项式其特征值也适合}}可知$A$ 的任一特征值 $\lambda_0$ 也适合 $x^k$, 于是 $\lambda_0 = 0$。

反之,
{\color{blue}证法一:}若 $A$ 的特征值全为零, 则存在可逆矩阵 $P$, 使得 $P^{-1}AP = B$ 为上三角矩阵且主对角元素全为零。由\hyperref[proposition:上三角阵性质]{上三角阵性质(1)}可知 $B^n = O$, 于是 $A^n = (PBP^{-1})^n = PB^nP^{-1} = O$, 即 $A$ 为幂零矩阵。

{\color{blue}证法二:}也可以利用 \hyperref[theorem:Cayley-Hamilton定理]{Cayley-Hamilton 定理}来证明, 由于 $A$ 的特征值全为零, 故其特征多项式为 $\lambda^n$, 从而 $A^n = O$.

\end{proof}

\begin{example}
设 \( V \) 是数域 \( \mathbb{F} \) 上的 \( n \) 阶方阵全体构成的线性空间, \( n \) 阶方阵
\[
P=\left( \begin{matrix}
0&		\cdots&		0&		1\\
0&		\cdots&		1&		0\\
\vdots&		&		\vdots&		\vdots\\
1&		0&		\cdots&		0\\
\end{matrix} \right) ,
\]
\( V \) 上的线性变换 \( \eta \) 定义为 \( \eta(X) = PX'P \). 试求 \( \eta \) 的全体特征值及其特征向量.
\end{example}
\begin{note}
任意$n$阶矩阵$A$左乘$P$相当于行倒排,右乘$P$矩阵相当于列倒排.
\end{note}
\begin{solution}
由 \( P = P', P^2 = I_n \) 容易验证 \( \eta^2(X) = P(PX'P)P = X \), 即 \( \eta^2 = I_V \), 于是 \( \eta \) 的特征值也适合多项式 \( x^2 - 1 \), 从而特征值只能是 \( \pm 1 \).

设 \( \eta(X_0) = PX_0'P = \pm X_0 \), 这等价于 \( (PX_0)' = \pm PX_0 \), 即 \( PX_0 \) 为对称矩阵或反对称矩阵. 

令 \( PX_0 = E_{ii}, E_{ij} + E_{ji} \) (对称矩阵空间的基向量), 容易证明 \( \eta \) 关于特征值 1 的线性无关的特征向量为 \( X_0 = PE_{ii} \ (1 \leqslant  i \leqslant  n), P(E_{ij} + E_{ji}) \ (1 \leqslant  i < j \leqslant  n) \).

令 \( PX_0 = E_{ij} - E_{ji} \) (反对称矩阵空间的基向量), 容易证明 \( \eta \) 关于特征值 \( -1 \) 的线性无关的特征向量为 \( X_0 = P(E_{ij} - E_{ji}) \ (1 \leqslant  i < j \leqslant  n) \). 注意到这些特征向量恰好构成 \( V \) 的一组基, 故 \( \eta \) 的特征值为 1 \( \frac{n(n+1)}{2} \) 重, \( -1 \ \frac{n(n-1)}{2} \) 重.

\end{solution}

\begin{example}
设 \( n \) 阶方阵 \( A \) 的每行每列只有一个元素非零, 并且那些非零元素为 1 或 -1, 证明: \( A \) 的特征值都是单位根.
\end{example}
\begin{proof}
设 \( S \) 为由每行每列只有一个元素非零, 并且那些非零元素为 1 或 -1 的所有 \( n \) 阶方阵构成的集合, 由排列组合可得 \( \overline{\overline{S}} = 2^n n! \), 即 \( S \) 是一个有限集合. 注意到矩阵 \( M \in S \) 当且仅当 \( M = P_1 P_2 \cdots P_r \), 其中 \( P_k \) 是初等矩阵 \( P_{ij} \) 或 \( P_i(-1) \), 因此对任意的 \( M, N \in S \), \( MN \in S \). 特别地, 由 \( A \in S \) 可知 \( A^k \in S \ (k \geqslant  1) \), 即 \( \{A, A^2, A^3, \cdots\} \subseteq S \), 于是存在正整数 \( k > l \), 使得 \( A^k = A^l \). 注意到 \( |A| = \pm 1 \), 故 \( A \) 可逆, 于是 \( A^{k-l} = I_n \), 从而 \( A \) 的特征值适合多项式 \( x^{k-l} - 1 \), 即为单位根.

\end{proof}

\begin{example}
设 \( A \) 是 \( n \) 阶实方阵, 又 \( I_n - A \) 的特征值的模长都小于 1, 求证: \( 0 < |A| < 2^n \).
\end{example}
\begin{proof}
设 \( A \) 的特征值为 \( \lambda_1, \cdots, \lambda_n \), 则 \( I_n - A \) 的特征值为 \( 1 - \lambda_1, \cdots, 1 - \lambda_n \).
由假设 \( |1 - \lambda_i| < 1 \), 若 \( \lambda_i \) 是实数, 则 \( 0 < \lambda_i < 2 \); 若 \( \lambda_i \) 是虚数, 则 \( \overline{\lambda_i} \) 也是 \( A \) 的特征值, 此时$1-\overline{\lambda_i}$也是$I_n-A$的特征值.从而$|1-\lambda_i|<1,|1-\overline{\lambda_i}|<1$,于是
\[|1-\lambda_i^2|=|(1-\lambda_i)(1-\overline{\lambda_i})|=|1-\lambda_i||1-\overline{\lambda_i}|<1.\]
因此$0<\lambda_i^2<2$,故此时$0<|\lambda_i|<\sqrt{2}$.

综上,无论$\lambda_i$是实数还是虚数,都有 \( 0 < |\lambda_i| < 2 \). 由于 \( |A| \) 等于所有特征值之积, 故 \( 0 < |A| < 2^n \).

\end{proof}

\begin{proposition}[逆矩阵的特征值]\label{proposition:逆矩阵的特征值}
设 $n$ 阶矩阵 $A$ 是可逆矩阵, 且 $A$ 的全部特征值为 $\lambda_1, \lambda_2, \cdots, \lambda_n$, 则 $A^{-1}$ 的全部特征值为 $\lambda_1^{-1}, \lambda_2^{-1}, \cdots, \lambda_n^{-1}$。
\end{proposition}
\begin{proof}
首先注意到 $A$ 是可逆矩阵, $\lambda_1 \lambda_2 \cdots \lambda_n = |A| \neq 0$, 因此每个 $\lambda_i \neq 0$ (事实上, $A$ 可逆的充分必要条件是它的特征值全不为零)。由\hyperref[theorem:复方阵必相似于上三角阵]{复方阵必相似于上三角阵}可设
\begin{align*}
P^{-1}AP = \begin{pmatrix}
\lambda_1 & * & \cdots & * \\
0 & \lambda_2 & \cdots & * \\
\vdots & \vdots & \ddots & \vdots \\
0 & 0 & \cdots & \lambda_n
\end{pmatrix}.
\end{align*}
因为\hyperref[proposition:上三角阵性质]{上三角矩阵的逆矩阵仍然是上三角矩阵}, 经过计算不难得到
\begin{align*}
P^{-1}A^{-1}P = (P^{-1}AP)^{-1} 
= \begin{pmatrix}
\lambda_1^{-1} & * & \cdots & * \\
0 & \lambda_2^{-1} & \cdots & * \\
\vdots & \vdots & \ddots & \vdots \\
0 & 0 & \cdots & \lambda_n^{-1}
\end{pmatrix}.
\end{align*}
因此 $A^{-1}$ 的全部特征值为 $\lambda_1^{-1}, \lambda_2^{-1}, \cdots, \lambda_n^{-1}$.

\end{proof}

\begin{proposition}[伴随矩阵的特征值]\label{proposition:伴随矩阵的特征值}
设 \( n \) 阶矩阵 \( A \) 的全体特征值为 \( \lambda_1, \lambda_2, \cdots, \lambda_n \),求证:\( A^* \) 的全体特征值为 
\[
\prod_{i \neq 1} \lambda_i, \prod_{i \neq 2} \lambda_i, \cdots, \prod_{i \neq n} \lambda_i.
\]
\end{proposition}
\begin{proof}
因为任一 \( n \) 阶矩阵均复相似于上三角矩阵,故可设
\[
P^{-1}AP = 
\begin{pmatrix}
\lambda_1 & * & \cdots & * \\
0 & \lambda_2 & \cdots & * \\
\vdots & \vdots & \ddots & \vdots \\
0 & 0 & \cdots & \lambda_n
\end{pmatrix}.
\]
注意到\hyperref[proposition:上三角阵性质]{上三角矩阵的伴随矩阵仍是上三角矩阵},经计算可得
\[
P^{-1}A^*P = P^*A^*(P^{-1})^* = (P^{-1}AP)^* = 
\begin{pmatrix}
\prod_{i \neq 1} \lambda_i & * & \cdots & * \\
0 & \prod_{i \neq 2} \lambda_i & \cdots & * \\
\vdots & \vdots & \ddots & \vdots \\
0 & 0 & \cdots & \prod_{i \neq n} \lambda_i
\end{pmatrix}.
\]
因此 \( A^* \) 的全部特征值为
\[
\prod_{i \neq 1} \lambda_i, \prod_{i \neq 2} \lambda_i, \cdots, \prod_{i \neq n} \lambda_i.
\]

\end{proof}

\subsection{特征值的降价公式}

\begin{theorem}[特征值的降价公式]\label{theorem:特征值的降价公式}
设 \( A \) 是 \( m \times n \) 矩阵,\( B \) 是 \( n \times m \) 矩阵,且 \( m \geqslant  n \)。求证:
\[
|\lambda I_m - AB| = \lambda^{m-n} |\lambda I_n - BA|.
\]
特别地,若 \( A, B \) 都是 \( n \) 阶矩阵,则 \( AB \) 与 \( BA \) 有相同的特征多项式.
\end{theorem}
\begin{note}
本质上就是打洞原理.
\end{note}
\begin{proof}
{\color{blue}证法一(打洞原理):} 当 \(\lambda \neq 0\) 时, 考虑下列分块矩阵:
\[
\begin{pmatrix}
\lambda I_m & A \\
B & I_n
\end{pmatrix},
\]
因为 \(\lambda I_m, I_n\) 都是可逆矩阵, 故由行列式的降阶公式可得
\[
|I_n| \cdot |\lambda I_m - A(I_n)^{-1}B| = |\lambda I_m| \cdot |I_n - B(\lambda I_m)^{-1}A|,
\]
即有
\[
|\lambda I_m - AB| = \lambda^{m-n} |\lambda I_n - BA|
\]
成立.

当 \(\lambda = 0\) 时, 若 \(m > n\), 则 \(r(AB) \leqslant  \min\{r(A), r(B)\} \leqslant  \min\{m, n\} = n < m\), 故 \(|-AB| = 0\), 结论成立;若 \(m = n\), 则 \(|-AB| = (-1)^n |A||B| = |-BA|\), 结论也成立。

事实上, \(\lambda = 0\) 的情形也可以用 \hyperref[theorem:Cauchy-Binet公式]{Cauchy-Binet 公式}来处理, 还可以通过\hyperref[proposition:摄动法基本命题]{摄动法}由 \(\lambda \neq 0\) 的情形来得到.

{\color{blue}证法二(相抵标准型):} 设 \(A\) 的秩等于 \(r\), 则存在 \(m\) 阶可逆矩阵 \(P\) 和 \(n\) 阶可逆矩阵 \(Q\), 使得
\[
PAQ = \begin{pmatrix}
I_r & O \\
O & O
\end{pmatrix}.
\]
令
\[
Q^{-1}BP ^{-1}= \begin{pmatrix}
B_{11} & B_{12} \\
B_{21} & B_{22}
\end{pmatrix},
\]
其中 \(B_{11}\) 是 \(r \times r\) 矩阵, 则
\[
PABP^{-1} = \begin{pmatrix}
B_{11} & B_{12} \\
O & O
\end{pmatrix}, \quad Q^{-1}BAQ = \begin{pmatrix}
B_{11} & O \\
B_{21} & O
\end{pmatrix}.
\]
因此
\[
|\lambda I_m - AB| = \left| \begin{pmatrix}
\lambda I_r - B_{11} & -B_{12} \\
O & \lambda I_{m-r}
\end{pmatrix} \right| = \lambda^{m-r} |\lambda I_r - B_{11}|,
\]
同理
\[
|\lambda I_n - BA| = \left| \begin{pmatrix}
\lambda I_r - B_{11} & O \\
-B_{21} & \lambda I_{n-r}
\end{pmatrix} \right| = \lambda^{n-r} |\lambda I_r - B_{11}|.
\]
比较上面两个式子即可得出结论。

{\color{blue}证法三(摄动法):} 先证明 \(m = n\) 的情形。若 \(A\) 可逆, 则 \(BA = A^{-1}(AB)A\), 即 \(AB\) 和 \(BA\) 相似, 因此它们的特征多项式相等。对于一般的方阵 \(A\), 可取到一列有理数 \(t_k \to 0\), 使得 \(t_k I_n + A\) 是可逆矩阵。由可逆情形的证明可得
\[
|\lambda I_n - (t_k I_n + A)B| = |\lambda I_n - B(t_k I_n + A)|.
\]
注意到上述两边的行列式都是 \( t_k \) 的多项式, 从而关于 \( t_k \) 连续. 上式两边同时取极限, 令 \( t_k \to 0 \), 即有
\(
|\lambda I_n - AB| = |\lambda I_n - BA|
\)
成立。

再证明 \( m > n \) 的情形。令
\[
C = \begin{pmatrix}
A & O
\end{pmatrix}, \quad D = \begin{pmatrix}
B \\
O
\end{pmatrix},
\]
其中 \( C, D \) 均为 \( m \times m \) 分块矩阵, 则
\[
CD = AB, \quad DC = \begin{pmatrix}
BA & O \\
O & O
\end{pmatrix}.
\]
因此由方阵的情形可得
\[
|\lambda I_m - AB| = |\lambda I_m - CD| = |\lambda I_m - DC| = \lambda^{m-n} |\lambda I_n - BA|.
\]

\end{proof}

\begin{example}
设 $\alpha$ 是 $n$ 维实列向量且 $\alpha' \alpha = 1$,试求矩阵 $I_n - 2 \alpha \alpha'$ 的特征值。
\end{example}
\begin{solution}
设 $A = I_n - 2 \alpha \alpha'$,则由\hyperref[theorem:特征值的降价公式]{特征值的降价公式}可得
\begin{align*}
| \lambda I_n - A | 
= | (\lambda - 1) I_n + 2 \alpha \alpha' | 
= (\lambda - 1)^{n-1} (\lambda - 1 + 2 \alpha' \alpha) 
= (\lambda - 1)^{n-1} (\lambda + 1).
\end{align*}
因此,矩阵$A$的特征值为1($n-1$)重,-1(1重).进一步,容易验证$A$有完全的特征向量系($| -I_n - A | $为零,但其$n-1$阶子式不为零),于是$A$可对角化.

\end{solution}

\begin{example}\label{example-6.10}
设 $A$ 为 $n$ 阶方阵, $\alpha, \beta$ 为 $n$ 维列向量, 试求矩阵 $A \alpha \beta'$ 的特征值。
\end{example}
\begin{solution}
设 $B = A \alpha \beta'$, 则由\hyperref[theorem:特征值的降价公式]{特征值的降价公式}可得
\begin{align*}
| \lambda I_n - B | 
= | I_n - (A \alpha) \beta' | 
= \lambda^{n-1} (\lambda - \beta' A \alpha).
\end{align*}
若 $\beta' A \alpha \neq 0$, 则 $B$ 的特征值为 $0$ ($n-1$ 重), $\beta' A \alpha$ ($1$ 重).进一步,容易验证此时$B$有完全的特征向量系,从而可对角化.
若 $\beta' A \alpha = 0$, 则 $B$ 的特征值为 $0$ ($n$ 重).

综上,容易验证\(B=\boldsymbol{A}\boldsymbol{\alpha}\boldsymbol{\beta}'\)可对角化的充要条件是\(\boldsymbol{\beta}'\boldsymbol{A}\boldsymbol{\alpha}\neq 0\)或\(\boldsymbol{A}\boldsymbol{\alpha}\boldsymbol{\beta}' = \boldsymbol{O}\).

\end{solution}

\begin{example}
设 $a_i \ (1 \leqslant  i \leqslant  n)$ 都是实数, 且 $a_1 + a_2 + \cdots + a_n = 0$, 试求下列矩阵的特征值:
\[
A = 
\begin{pmatrix}
a_1^2 & a_1 a_2 + 1 & \cdots & a_1 a_n + 1 \\
a_2 a_1 + 1 & a_2^2 & \cdots & a_2 a_n + 1 \\
\vdots & \vdots & \ddots & \vdots \\
a_n a_1 + 1 & a_n a_2 + 1 & \cdots & a_n^2
\end{pmatrix}.
\]
\end{example}
\begin{solution}
矩阵 $A$ 可以分解为 $A = -I_n + BC$, 其中
\[
B = 
\begin{pmatrix}
a_1 & 1 \\
a_2 & 1 \\
\vdots & \vdots \\
a_n & 1
\end{pmatrix}, \quad
C = 
\begin{pmatrix}
a_1 & a_2 & \cdots & a_n \\
1 & 1 & \cdots & 1
\end{pmatrix}.
\]
由\hyperref[theorem:特征值的降价公式]{特征值的降价公式}得
\begin{align*}
| \lambda I_n - A | 
&= | (\lambda + 1) I_n - BC | \\
&= (\lambda + 1)^{n-2} | (\lambda + 1) I_2 - CB |.
\end{align*}
注意到 $a_1 + a_2 + \cdots + a_n = 0$, 故有
\[
CB = 
\begin{pmatrix}
a_1^2 + a_2^2 + \cdots + a_n^2 & 0 \\
0 & n
\end{pmatrix}.
\]
因此 $A$ 的特征值为 $-1$ ($n-2$ 重), $n-1$, $a_1^2 + a_2^2 + \cdots + a_n^2 - 1$.进一步,若\(a_i\)全部为零, 则特征值\(-1\)和\(n - 1\)都有完全的特征向量系. 若\(\sum_{i = 1}^{n}a_i^2 = n\), 利用秩的降阶公式可得特征值\(-1\)和\(n - 1\)都有完全的特征向量系. 在剩余情况, 利用秩的降阶公式可得\(3\)个特征值都有完全的特征向量系. 因此, \(\boldsymbol{A}\)可对角化. 事实上, 即使去掉\(a_1 + a_2+\cdots + a_n = 0\)的条件, 也可以计算出\(\boldsymbol{A}\)的全体特征值的代数重数和几何重数, 从而得到\(\boldsymbol{A}\)可对角化. 这一结论的深层次背景是: \(\boldsymbol{A}\)是实对称矩阵, 从而可正交对角化.

\end{solution}

\begin{proposition}\label{proposition:AC=CB则A,B至少有r个相同特征值}
设 $A, B, C$ 分别是 $m \times m, n \times n, m \times n$ 矩阵, 满足: $AC = CB$, $\mathrm{r}(C) = r$. 求证: $A$ 和 $B$ 至少有 $r$ 个相同的特征值。
\end{proposition}
\begin{remark}
不妨设 $C = \begin{pmatrix}
I_r & O \\
O & O
\end{pmatrix}$ 的原因:假设当 $C = \begin{pmatrix}
I_r & O \\
O & O
\end{pmatrix}$ 时, 结论已经成立, 则对于一般的满足条件的矩阵 $C$, 由条件我们有
\[
r(C) = r, \quad AC = BC.
\]
由 $r(C) = r$ 可知, 存在可逆矩阵 $P, Q$, 使得
\[
PCQ = \begin{pmatrix}
I_r & O \\
O & O
\end{pmatrix}.
\]
从而对 $AC = BC$ 两边同时左乘 $P$, 右乘 $Q$ 得到
\[
(PAP^{-1})(PCQ) = (PCQ)(Q^{-1}BQ).
\]
于是由假设可知 $PAP^{-1}$ 和 $Q^{-1}BQ$ 都至少有 $r$ 个相同的特征值. 又因为\hyperref[theorem:相似矩阵有相同的特征多项式与特征值]{相似矩阵有相同的特征值}, 所以 $A, B$ 也至少有 $r$ 个相同的特征值. 故不妨设成立.
\end{remark}
\begin{proof}
设 $P$ 为 $m$ 阶非异阵, $Q$ 为 $n$ 阶非异阵, 使得
\[
PCQ = 
\begin{pmatrix}
I_r & O \\
O & O
\end{pmatrix}.
\]
注意到问题的条件和结论在相抵变换 $C \mapsto PCQ$, $A \mapsto PAP^{-1}$, $B \mapsto Q^{-1} BQ$ 下保持不变, 故不妨从一开始就假设 $C = 
\begin{pmatrix}
I_r & O \\
O & O
\end{pmatrix}$
是相抵标准型. 设
\[
A = 
\begin{pmatrix}
A_{11} & A_{12} \\
A_{21} & A_{22}
\end{pmatrix}, \quad
B = 
\begin{pmatrix}
B_{11} & B_{12} \\
B_{21} & B_{22}
\end{pmatrix}
\]
为对应的分块, 则
\[
AC = 
\begin{pmatrix}
A_{11} & O \\
A_{21} & O
\end{pmatrix}, \quad
CB = 
\begin{pmatrix}
B_{11} & B_{12} \\
O & O
\end{pmatrix}.
\]
由 $AC = CB$ 可得 $A_{11} = B_{11}, A_{21} = 0, B_{12} = 0$. 于是
\begin{align*}
| \lambda I_m - A | &= | \lambda I_r - A_{11} | \cdot | \lambda I_{m-r} - A_{22} |, \\
| \lambda I_n - B | &= | \lambda I_r - B_{11} | \cdot | \lambda I_{n-r} - B_{22} |.
\end{align*}
从而 $A, B$ 至少有 $r$ 个相同的特征值 (即 $A_{11} = B_{11}$ 的特征值).

\end{proof}

\subsection{特征值与特征多项式系数的关系}

\begin{proposition}[特征值与特征多项式系数的关系]\label{proposition:特征值与特征多项式系数的关系}
设 $n$ 阶矩阵 $A$ 的特征多项式为
\[
f(\lambda) = \lambda^n + a_1 \lambda^{n-1} + \cdots + a_{n-1} \lambda + a_n.
\]
求证: $a_r$ 等于 $(-1)^r$ 乘以 $A$ 的所有 $r$ 阶主子式之和, 即
\[
a_r = (-1)^r \sum_{1 \leqslant  i_1 < i_2 < \cdots < i_r \leqslant  n}  A
\begin{pmatrix}
i_1 & i_2 & \cdots & i_r \\
i_1 & i_2 & \cdots & i_r
\end{pmatrix}, \quad 1 \leqslant  r \leqslant  n.
\]
进一步, 若设 $A$ 的特征值为 $\lambda_1, \lambda_2, \cdots, \lambda_n$, 则
\[
\sum_{1 \leqslant  i_1 < i_2 < \cdots < i_r \leqslant  n} \lambda_{i_1} \lambda_{i_2} \cdots \lambda_{i_r} = \sum_{1 \leqslant  i_1 < i_2 < \cdots < i_r \leqslant  n} A
\begin{pmatrix}
i_1 & i_2 & \cdots & i_r \\
i_1 & i_2 & \cdots & i_r
\end{pmatrix}, \quad 1 \leqslant  r \leqslant  n.
\]
\end{proposition}
\begin{remark}
上述结论中最常用的是 $r = 1$ 和 $r = n$ 的情形:
\[
\lambda_1 + \lambda_2 + \cdots + \lambda_n = \mathrm{tr}(A), \quad \lambda_1 \lambda_2 \cdots \lambda_n = |A|.
\]
特别地,\textbf{ $A$ 是非异阵的充要条件是 $A$ 的特征值全不为零.} 因此,特征值的计算是判断矩阵是否非异阵的重要依据.
\end{remark}
\begin{proof}
第一种结论是\hyperref[corollary:特征多项式系数与矩阵子式的关系]{推论\ref{corollary:特征多项式系数与矩阵子式的关系}}. 
由\hyperref[theorem:Vieta定理]{Vieta 定理}可得
\begin{align*}
\sum_{1\le i_1<i_2<\cdots <i_r\le n}{\lambda _{i_1}\lambda _{i_2}}\cdots \lambda _{i_r}&=\left( -1 \right) ^ra_r=\left( -1 \right) ^r\left( -1 \right) ^r\sum_{1\le i_1<i_2<\cdots <i_r\le n}{A\left( \begin{matrix}
i_1&		i_2&		\cdots&		i_r\\
i_1&		i_2&		\cdots&		i_r\\
\end{matrix} \right)}
\\
&=\sum_{1\le i_1<i_2<\cdots <i_r\le n}{A\left( \begin{matrix}
i_1&		i_2&		\cdots&		i_r\\
i_1&		i_2&		\cdots&		i_r\\
\end{matrix} \right)}\quad ,1\le r\le n.
\end{align*}
因此第二种结论也成立.

\end{proof}

\begin{example}
设 \(n\) 阶方阵 \(A\) 满足 
\begin{align*}
A^2 - A - 3I_n = O,
\end{align*}
求证:\(A - 2I_n\) 是非奇异阵。
\end{example}
\begin{note}
用特征值判断矩阵非异性.
\end{note}
\begin{proof}
用反证法。设 \(A - 2I_n\) 为奇异阵,则 2 是 \(A\) 的特征值。注意到 \(A\) 适合 
\begin{align*}
f(x) = x^2 - x - 3,
\end{align*}
但特征值 2 却不适合 \(f(x)\),这与\hyperref[proposition:矩阵适合的多项式其特征值也适合]{命题\ref{proposition:矩阵适合的多项式其特征值也适合}}矛盾。

\end{proof}

\begin{example}
设 $P$ 是可逆矩阵, $B = PAP^{-1} - P^{-1}AP$, 求证: $B$ 的特征值之和为零.
\end{example}
\begin{proof}
由\hyperref[proposition:特征值与特征多项式系数的关系]{特征值与特征多项式系数的关系}可知,只要证 $\operatorname{tr}(B) = 0$ 即可. 由迹的线性和交换性即得
\begin{align*}
\operatorname{tr}(B) = \operatorname{tr}(PAP^{-1}) - \operatorname{tr}(P^{-1}AP) 
= \operatorname{tr}(A) - \operatorname{tr}(A) = 0.
\end{align*}

\end{proof}

\begin{example}
设 $n$ 阶实方阵 $A$ 的特征值全是实数, 且 $A$ 的一阶主子式之和与二阶主子式之和都等于零. 求证: $A$ 是幂零矩阵.
\end{example}
\begin{proof}
设 $A$ 的特征值为 $\lambda_1, \lambda_2, \dots, \lambda_n$, 由条件和\hyperref[proposition:特征值与特征多项式系数的关系]{特征值与特征多项式系数的关系}可知
\begin{align*}
\sum_{i=1}^n \lambda_i &= \lambda_1 + \lambda_2 + \cdots + \lambda_n = 0, \\
\sum_{1 \leqslant  i < j \leqslant  n} \lambda_i \lambda_j &= \lambda_1 \lambda_2 + \lambda_1 \lambda_3 + \cdots + \lambda_{n-1} \lambda_n = 0.
\end{align*}
则
\begin{align*}
\sum_{i=1}^n \lambda_i^2 &= \left( \sum_{i=1}^n \lambda_i \right)^2 - 2 \sum_{1 \leqslant  i < j \leqslant  n} \lambda_i \lambda_j = 0.
\end{align*}
由于 $\lambda_i$ 都是实数, 故 $\lambda_i = 0 \ (1 \leqslant  i \leqslant  n)$ 成立, 再由\hyperref[proposition:幂零矩阵关于特征值的充要条件]{命题\ref{proposition:幂零矩阵关于特征值的充要条件}}可知 $A$ 为幂零矩阵.

\end{proof}

\begin{example}
设 $n (n \geqslant  3)$ 阶非异实方阵 $A$ 的特征值都是实数, 且 $A$ 的 $n-1$ 阶主子式之和等于零. 证明: 存在 $A$ 的一个 $n-2$ 阶主子式, 其符号与 $|A|$ 的符号相反.
\end{example}
\begin{proof}
设 $A$ 的特征值为 $\lambda_1, \lambda_2, \dots, \lambda_n$, 由 $A$ 非异可知它们都是非零实数. 再由条件和例 6.24 可知
\begin{align}
\sum_{1 \leqslant  i_1 < i_2 < \cdots < i_{n-1} \leqslant  n} \lambda_{i_1} \lambda_{i_2} \cdots \lambda_{i_{n-1}} = 0.\label{example-0.16-6.1}
\end{align}
将\eqref{example-0.16-6.1}式左边除以 $|A| = \lambda_1 \lambda_2 \cdots \lambda_n$ 可得
\begin{align}
\sum_{i=1}^n \frac{1}{\lambda_i} = 0, \label{example-0.16-6.2}
\end{align}
将\eqref{example-0.16-6.2} 式左边平方, 并将平方项移到等式的右边可得
\begin{align}
\sum_{1 \leqslant  i < j \leqslant  n} \frac{1}{\lambda_i \lambda_j} = -\frac{1}{2} \left( \sum_{i=1}^n \frac{1}{\lambda_i} \right)^2 < 0, \label{example-0.16-6.3}
\end{align}
将\eqref{example-0.16-6.3}式两边同时乘以 $|A| = \lambda_1 \lambda_2 \cdots \lambda_n$ 可得
\begin{align}
\sum_{1 \leqslant  i_1 < i_2 < \cdots < i_{n-2} \leqslant  n} \lambda_{i_1} \lambda_{i_2} \cdots \lambda_{i_{n-2}} = -\frac{1}{2} \left( \sum_{i=1}^n \frac{1}{\lambda_i^2} \right) |A|. \label{example-0.16-6.4}
\end{align}
由 \eqref{example-0.16-6.4}式和\hyperref[proposition:特征值与特征多项式系数的关系]{特征值与特征多项式系数的关系}可得
\begin{align*}
\sum_{1 \leqslant  i_1 < i_2 < \cdots < i_{n-2} \leqslant  n} A \begin{pmatrix}
i_1 & i_2 & \cdots & i_{n-2} \\
i_1 & i_2 & \cdots & i_{n-2}
\end{pmatrix} = -\frac{1}{2} \left( \sum_{i=1}^n \frac{1}{\lambda_i^2} \right) |A|,
\end{align*}
于是 $A$ 的 $n-2$ 阶主子式之和与 $|A|$ 的符号相反, 从而至少存在 $A$ 的一个 $n-2$ 阶主子式, 其符号与 $|A|$ 的符号相反. 

\end{proof}

\begin{conclusion}
设$n$阶方阵$A$的特征值为$\lambda_1, \lambda_2, \cdots, \lambda_n$,则对任意的正整数$k$,$A^k$的特征值为$\lambda_1^k, \lambda_2^k, \cdots, \lambda_n^k$,于是特征值的$k$次幂和
\begin{align*}
s_k = \lambda_1^k + \lambda_2^k + \cdots + \lambda_n^k = \operatorname{tr}(A^k), \quad k \geqslant  1.
\end{align*}
若已知$n$阶方阵$A$的迹$\operatorname{tr}(A^k)(1\leqslant  k \leqslant  n)$,则由Newton公式可以计算出特征值的初等对称多项式
\begin{align*}
\sigma_r = \sum_{1 \leqslant  i_1 < i_2 < \cdots < i_r \leqslant  n} \lambda_{i_1} \lambda_{i_2} \cdots \lambda_{i_r}, \quad 1 \leqslant  r \leqslant  n,
\end{align*}
从而可以确定特征多项式的系数, 最后便可计算出$A$的所有特征值.
\end{conclusion}

\begin{example}
设$A$是$n$阶对合矩阵, 即$A^2 = I_n$, 证明: $n - \operatorname{tr}(A)$为偶数, 并且$\operatorname{tr}(A) = n$的充要条件是$A = I_n$.
\end{example}
\begin{proof}
由$A^2 = I_n$可知$A$的特征值也适合$x^2 - 1$, 从而只能是$\pm 1$. 设$A$的特征值为$1$($p$重), $-1$($q$重), 则$p + q = n$. 且$\operatorname{tr}(A) = p - q$, 于是$n - \operatorname{tr}(A) = 2q$为偶数. 若$A = I_n$, 则$\operatorname{tr}(A) = n$. 反之, 若$\operatorname{tr}(A) = n$, 则由上述讨论可知$p = n$, $q = 0$, 从而$-1$不是$A$的特征值,即$A + I_n$是非奇阵. 最后由$A^2=I_n$可得
\begin{align*}
(A - I_n)(A + I_n) = O\Rightarrow A-I_n=O\Rightarrow A = I_n.
\end{align*}

\end{proof}

\begin{example}
设4阶方阵$A$满足: $\operatorname{tr}(A^k) = k \ (1 \leqslant  k \leqslant  4)$, 试求$A$的行列式.
\end{example}
\begin{proof}
题目条件即为$s_k = k \ (1 \leqslant  k \leqslant  4)$, 要求$\lvert A \rvert = \sigma_4$. 根据Newton公式(白皮书这一部分还没看)
\begin{align*}
s_k - s_{k-1} \sigma_1 + \cdots + (-1)^k k \sigma_k = 0 \quad (1 \leqslant  k \leqslant  4)
\end{align*}
可依次算出$\sigma_1 = 1$, $\sigma_2 = -\frac{1}{2}$, $\sigma_3 = \frac{1}{6}$,$\sigma_4 = \frac{1}{24}.$
故$\lvert A \rvert = \frac{1}{24}$. 也可以直接利用例5.64(白皮书这一部分还没看)来计算$\sigma_4$.

\end{proof}

\begin{definition}[线性变换的迹]
线性变换的迹定义为它在任一组基下的表示矩阵的迹.
\end{definition}
\begin{note}
因为矩阵的迹在相似变换下保持不变,并且同一线性变换在不同基下的表示矩阵必相似,所以同一线性变换在任意一组基下的表示矩阵的迹都相同,故线性变换的迹是良定义的.
\end{note}

\begin{proposition}\label{proposition:线性变换在其特征子空间上的限制的表示矩阵及迹}
设$\varphi$为$n$维线性空间,$\lambda_0$为$\varphi$的一个特征值,$V_0$为特征值$\lambda_0$的特征子空间,则存在$V_0$上的一组基,使得$\varphi$在$V_0$上的限制$\varphi\mid_{V_0}$在这组基下的表示矩阵为$\mathrm{dim}V_0$阶的对角阵$\mathrm{diag}\left\{ \lambda _0,\lambda _0,\cdots ,\lambda _0 \right\}$,从而$\mathrm{tr}\left( \varphi |_{V_0} \right) =\lambda _0\mathrm{dim}V_0$.
\end{proposition}
\begin{remark}
因为线性变换的特征子空间一定是不变子空间,所以线性变换在其特征子空间上做限制后仍是线性变换,因此线性变换在其特征子空间上的限制是良定义的.
\end{remark}
\begin{proof}
设\(x_1\)是\(\varphi\)属于\(\lambda_0\)的特征向量, 将其扩充成\(V_0\)的一组基\(\{ x_1,x_2,\cdots ,x_r \}\), 则\(r = \dim V_0\).
注意到\(x_1,x_2,\cdots ,x_r\)也是\(\varphi|_{V_0}\)属于\(\lambda_0\)的特征向量, 从而
\begin{align*}
\varphi|_{V_0}( x_1,x_2,\cdots ,x_r )=(\lambda_0x_1,\lambda_0x_2,\cdots ,\lambda_0x_r)
=( x_1,x_2,\cdots ,x_r )\begin{pmatrix}
\lambda_0& & & \\
& \lambda_0& & \\
& & \ddots& \\
& & & \lambda_0\\
\end{pmatrix}.
\end{align*}
故\(\varphi|_{V_0}\)在\(\{ x_1,x_2,\cdots ,x_r \}\)下的表示矩阵为\(\dim V_0\)阶对角阵\(\mathrm{diag}\{ \lambda_0,\lambda_0,\cdots ,\lambda_0 \}\).

\end{proof}

\begin{proposition}\label{proposition:经典矩阵乘法可交换诱导的性质例题}
设 \(A, B, C\) 是 \(n\) 阶矩阵, 其中 \(C = AB - BA\). 若它们满足条件 $AC = CA$或 $BC = CB$, 求证: \(C\) 的特征值全为零.
\end{proposition}
\begin{remark}
若将条件减弱为 \(ABC = CAB, BAC = CBA\), 则上述结论不再成立. 原因如下:

如将条件减弱为如题所述, 则结论不再成立, 可参考下面的反例:
\[
A = \begin{pmatrix}
1 & 0 \\ 
0 & -1
\end{pmatrix}, \quad
B = \begin{pmatrix}
0 & \frac{1}{2} \\ 
\frac{1}{2} & 0
\end{pmatrix}, \quad
C = \begin{pmatrix}
0 & 1 \\ 
1 & 0
\end{pmatrix}.
\]
由计算可得 \(C = AB - BA, ABC = CAB, CBA = BAC\), 但 \(C\) 的特征值为 1 和 -1。
\end{remark}
\begin{proof}
{\color{blue}证法一:}
由 \(AC = CA\) 可知, 对任意的正整数 \(k\),
\begin{align*}
C^k &= C^{k-1}AB - C^{k-1}BA = A(C^{k-1}B) - (C^{k-1}B)A.
\end{align*}
由迹的线性和交换性可得 \(\operatorname{tr}(C^k) = 0 \ (k \geqslant  1)\), 再由\hyperref[theorem:幂零矩阵关于迹的充要条件]{幂零矩阵关于迹的充要条件}可知 \(C\) 为幂零矩阵, 从而 \(C\) 的特征值全为零。

{\color{blue}证法二:}将\(A,B,C\)看成是\(n\)维复列向量空间\(V\)上的线性变换. 任取\(C\)的特征值\(\lambda_{0}\)及其特征子空间\(V_{0}\), 由\(AC = CA\), \(BC = CB\)以及\hyperref[proposition:特征子空间互为不变子空间]{命题\ref{proposition:特征子空间互为不变子空间}}可知, \(V_{0}\)是\(A -\)不变子空间, 也是\(B -\)不变子空间. 将等式\(C = AB - BA\)两边的线性变换同时限制在\(V_{0}\)上, 可得\(V_{0}\)上线性变换的等式\(C|_{V_{0}}=A|_{V_{0}}B|_{V_{0}}-B|_{V_{0}}A|_{V_{0}}\). 两边同时取迹 , 由迹的线性和交换性及\hyperref[proposition:线性变换在其特征子空间上的限制的表示矩阵及迹]{命题\ref{proposition:线性变换在其特征子空间上的限制的表示矩阵及迹}}可知
\begin{align*}
\lambda_{0}\dim V_{0}=\mathrm{tr}(C|_{V_{0}})=\mathrm{tr}(A|_{V_{0}}B|_{V_{0}})-\mathrm{tr}(B|_{V_{0}}A|_{V_{0}})=0,
\end{align*}
从而\(\lambda_{0}=0\), 结论得证.

{\color{blue}证法三:}注意到问题的条件和结论在同时相似变换:$A \mapsto P^{-1}AP$, $B \mapsto P^{-1}BP$, $C \mapsto P^{-1}CP$ 下不改变,故不妨从一开始就假设 $C$ 为 Jordan 标准型。设 $C = \mathrm{diag}\{J_1, J_2, \cdots, J_k\}$,其中 $\lambda_1, \lambda_2 \cdots, \lambda_k$ 是 $C$ 的全体不同特征值,$J_i$ 是属于特征值 $\lambda_i$ 的所有 Jordan 块拼成的根子空间分块。由于 $J_i$ 的特征值为 $\lambda_i$,它们互不相同,又 $AC = CA$, $BC = CB$,故由\refexa{example:分块准对角阵的块之间两两没有公共特征值则与其可交换的矩阵也有同样分块}可知,$A = \mathrm{diag}\{A_1, A_2, \cdots, A_k\}$, $B = \mathrm{diag}\{B_1, B_2, \cdots, B_k\}$ 和 $C$ 一样也是分块对角矩阵。于是我们有 $J_i = A_iB_i - B_iA_i$,两边同取迹可得
\begin{align*}
n_i\lambda_i = \mathrm{tr}(J_i) = \mathrm{tr}(A_iB_i - B_iA_i) = \mathrm{tr}(A_iB_i) - \mathrm{tr}(B_iA_i) = 0,
\end{align*}
从而 $k = 1$ 且 $C$ 的特征值全为零。

\end{proof}
\begin{remark}
上述{\color{blue}证法二}中,\(A,B,C\)在不变子空间\(V_{0}\)上的限制只能理解成线性变换在不变子空间上的限制, 而不是矩阵在不变子空间上的限制. 
\end{remark}

\begin{corollary}\label{proposition:经典矩阵乘法可交换诱导的性质例题推论}
设 \(A, B, C\) 是 \(n\) 阶矩阵, 其中 \(C = AB - BA\). 若它们满足条件$AC = CA,$ 或 $BC = CB$, 求证: \(A,B,C\) 可同时上三角化.
\end{corollary}
\begin{proof}
对阶数进行归纳. 由\hyperref[proposition:经典矩阵乘法可交换诱导的性质例题]{命题\ref{proposition:经典矩阵乘法可交换诱导的性质例题}证法二}可知, \(C\)的特征值全为\(0\), 其特征子空间\(V_{0}\)满足
\begin{align*}
A|_{V_{0}}B|_{V_{0}} - B|_{V_{0}}A|_{V_{0}}=C|_{V_{0}} = 0,
\end{align*}
即\(A|_{V_{0}},B|_{V_{0}}\)乘法可交换. 由\hyperref[proposition:一般数域上乘法可交换诱导的性质]{命题\ref{proposition:一般数域上乘法可交换诱导的性质}}可知\(A|_{V_{0}},B|_{V_{0}}\)有公共的特征向量, 即存在\(0\neq e_{1} \in V_{0}\), 使得
\begin{align*}
Ae_{1}=A|_{V_{0}}(e_{1})=\lambda_{1}e_{1},\ Be_{1}=B|_{V_{0}}(e_{1})=\mu_{1}e_{1},\ Ce_{1}=0.
\end{align*}
余下的证明完全类似于\hyperref[proposition:特征值全在同一数域的矩阵可上三角化]{命题\ref{proposition:特征值全在同一数域的矩阵可上三角化}}的证明, 请读者自行补充相关的细节.

\end{proof}

\begin{example}
设$A,B$是$n$阶方阵,满足$A^2B + BA^2 = 2ABA$,证明:$AB - BA$是幂零矩阵。 
\end{example}
\begin{proof}
记$C = AB - BA$,则显然$\mathrm{tr}(C)=0$。由条件可知
\begin{align*}
A^2B + BA^2 = 2ABA \Longleftrightarrow A^2B - ABA = ABA - BA^2 
\Longleftrightarrow A(AB - BA) = (AB - BA)A \Longleftrightarrow AC = CA.
\end{align*}
从而由上式及矩阵迹的交换性可得,对$\forall k\in [1,n]\cap \mathbb{N}$,都有
\begin{align*}
\mathrm{tr}(C^k) = \mathrm{tr}(C^{k - 1}(AB - BA)) = \mathrm{tr}(C^{k - 1}AB) - \mathrm{tr}(C^{k - 1}BA) 
= \mathrm{tr}(AC^{k - 1}B) - \mathrm{tr}(AC^{k - 1}B) = 0.
\end{align*}
故由\refpro{theorem:幂零矩阵关于迹的充要条件}可知$C=AB-BA$是幂零矩阵.

\end{proof}



\subsection{特征值的估计}

\begin{theorem}[第一圆盘定理]\label{theorem:第一圆盘定理}
设\(A = (a_{ij})\)是\(n\)阶矩阵, 则\(A\)的特征值在复平面的下列圆盘中:
\begin{align*}
|z - a_{ii}| \leqslant  R_{i},\ 1 \leqslant  i \leqslant  n,
\end{align*}
其中\(R_{i} = |a_{i1}| + \cdots + |a_{i,i - 1}| + |a_{i,i + 1}| + \cdots + |a_{in}|\).
\end{theorem}
\begin{remark}
该定理又称为Gerschgorin圆盘第一定理, 即戈氏圆盘第一定理. 上述圆盘称为戈氏圆盘.
\end{remark}
\begin{proof}


\end{proof}

\begin{theorem}[第二圆盘定理]\label{theorem:第二圆盘定理}
若\(n\)阶矩阵\(A\)的\(n\)个戈氏圆盘分成若干个连通区域, 其中某个连通区域恰含\(k\)个戈氏圆盘, 则有且仅有\(k\)个特征值落在该连通区域内 (若两个圆盘重合应计算重数, 若特征值为重根也要计算重数).
\end{theorem}
\begin{proof}


\end{proof}

\begin{example}
如果圆盘定理中有一个连通分支由两个圆盘外切组成, 证明: 每个圆盘除去切点的区域不可能同时包含两个特征值.
\end{example}
\begin{proof}
设\(A = (a_{ij})\)为\(n\)阶矩阵, \(D_{i}:|z - a_{ii}| \leqslant  R_{i}(1 \leqslant  i \leqslant  n)\)是\(A\)的\(n\)个戈氏圆盘. 不妨设\(A\)的两个戈氏圆盘\(D_{1},D_{2}\)外切并组成一个连通分支. 令
\begin{align*}
A(t)= 
\begin{pmatrix}
a_{11} & ta_{12} & \cdots & ta_{1n}\\
ta_{21} & a_{22} & \cdots & ta_{2n}\\
\vdots & \vdots & & \vdots\\
ta_{n1} & ta_{n2} & \cdots & a_{nn}
\end{pmatrix},
\end{align*}
由\hyperref[theorem:第一圆盘定理]{第一圆盘定理}, \(A(t)\)的特征值落在下列圆盘中:
\begin{align*}
tD_{i}:|z - a_{ii}| \leqslant  tR_{i},\ 1 \leqslant  i \leqslant  n.
\end{align*}
由于当\(0 \leqslant  t < 1\)时, \(A(t)\)的特征值是关于\(t\)的连续函数, 故\(A(t)\)的特征值\(\lambda_{i}(t)\)从\(D_{i}\)的圆心开始, 始终在圆盘\(tD_{i}(1 \leqslant  i \leqslant  n)\)中连续变动. 注意此时\(tD_{1},tD_{2}\)不相交, 它们是两个连通分支, 于是特征值\(\lambda_{i}(t)\)落在\(tD_{i}(i = 1,2)\)中. 最后当\(t = 1\)时, \(A\)的特征值\(\lambda_{1}=\lambda_{1}(1)\)落在\(D_{1}\)中, 特征值\(\lambda_{2}=\lambda_{2}(1)\)落在\(D_{2}\)中. 因此, \(\lambda_{1},\lambda_{2}\)不可能同时落在\(D_{1}\)或\(D_{2}\)除去切点的区域中.

\end{proof}

\begin{proposition}[不可对角化矩阵的摄动]\label{proposition:不可对角化矩阵的摄动}
\begin{enumerate}[(1)]
\item 设 $A = (a_{ij})_{n\times n}$ 是 $n$ 阶复方阵,证明:存在一个关于所有矩阵元 $a_{ij}$ 的 $n^2$ 元多项式 $f(a_{ij}, 1\leqslant  i, j\leqslant  n)$,使得只要 $A$ 不可对角化,就有 $f(a_{ij}, 1\leqslant  i, j\leqslant  n)=0$.

\item 设 $A = (a_{ij})_{n\times n}$ 是 $n$ 阶复方阵,证明:存在 $n^2$ 个多项式 $p_{ij}(t)$ 以及 $\delta > 0$,满足 $p_{ij}(0)=0$ 并且对任意 $t\in(0, \delta)$,矩阵 $A(t)=A + (p_{ij}(t))_{n\times n}$ 都可对角化. 

\item 设\(A = (a_{ij})\)为\(n\)阶复方阵, 证明: 存在正数\(\delta\), 使得对任意的\(s \in (0,\delta)\), 下列矩阵$A(s)$均有\(n\)个不同的特征值,进而$A(s)$可对角化.
\begin{align*}
A(s)= 
\begin{pmatrix}
a_{11}+s & a_{12} & \cdots & a_{1n}\\
a_{21} & a_{22}+s^{2} & \cdots & a_{2n}\\
\vdots & \vdots & & \vdots\\
a_{n1} & a_{n2} & \cdots & a_{nn}+s^{n}
\end{pmatrix}.
\end{align*}
\end{enumerate}
\end{proposition}
\begin{remark}
(1)说明:\textbf{可对角化的矩阵远远多于不可对角化的.}因为形式上从 $f$ 里面可以反解一个 $a_{ij}$ 出来,也即只要一个 $A$ 不可对角化,就一定满足 $a_{nn} =$ 某个关于其余 $n^2 - 1$ 元的函数,显然让两个东西相等是没那么容易的(因为随便取,一般都不等),所以\textbf{不可对角化的矩阵很少且完全包含在一个曲面当中.}

(2)(3)则是给出了摄动的方法(不唯一),实现:\textbf{用可对角化的矩阵逼近任意一个矩阵.} 
\end{remark}
\begin{proof}
\begin{enumerate}[(1)]
\item 若$A$不可对角化,则$A$的极小多项式$m(x)$有重根,于是$A$的特征多项式$f(x) = |xI - A|$有重根,从而等价于$f(x)$的判别式$\Delta = 0$。注意到
\begin{align*}
f(x) = |xI - A| = x^n - \mathrm{tr}(A)x^{n-1} + \cdots + (-1)^n|A| = x^n + p_1(a_{ij}, 1 \le i, j \le n)x^{n-1} + \cdots + p_n(a_{ij}, 1 \le i, j \le n)。
\end{align*}
其中$p_k$都是$n^2$元多项式。设$x_1, \cdots, x_n$是$f(x) = 0$的根,则根据Vieta定理可知,$f(x)$的判别式$\Delta = \prod_{1 \le i < j \le n} (x_i - x_j)^2$是一个关于$p_1(a_{ij}, 1 \le i, j \le n), \cdots, p_n(a_{ij}, 1 \le i, j \le n)$的多项式,记为$F(a_{ij}, 1 \le i, j \le n)$。从而此时$F(a_{ij}, 1 \le i, j \le n) = 0$。故$F(a_{ij}, 1 \le i, j \le n)$就为所求多项式.

\item 因为任何复矩阵都可上三角化,所以存在可逆阵$G$,使得
\begin{align*}
A = G\begin{pmatrix}
\lambda_1 & & * \\
& \ddots & \\
& & \lambda_s
\end{pmatrix}G^{-1},
\end{align*}
其中$\lambda_1, \cdots, \lambda_n$为$A$的特征值。令$P(t) = G\begin{pmatrix}
c_1t & & \\
& \ddots & \\
& & c_nt
\end{pmatrix}G^{-1}$,其中$c_i$是互不相同的常数,则$P(0) = 0$。从而
\begin{align*}
A(t) = A + P(t) = G\begin{pmatrix}
\lambda_1 + c_1t & & * \\
& \ddots & \\
& & \lambda_n + c_nt
\end{pmatrix}G^{-1}.
\end{align*}
显然$A(t)$的特征值分别为$\lambda_1 + c_1t, \cdots, \lambda_n + c_st$。再设$A$有$s(\leqslant n)$个互不相同的特征值,分别记为$\lambda_1', \cdots, \lambda_s'$,则$\lambda_i \in \{\lambda_1', \cdots, \lambda_s'\}, i = 1, 2, \cdots, n$。
取$\delta = \frac{1}{2}\min_{1 \le i, j \le s}\left\{\left|\frac{\lambda_j' - \lambda_i'}{c_i - c_j}\right|\right\}$,则对$\forall t \in (0, \delta)$,都有
\begin{align*}
\lambda_i + c_it \ne \lambda_j + c_jt, \quad \forall i, j \in \{1, 2, \cdots, n\} \text{且} i \ne j.
\end{align*}
若$\lambda_i + c_it = \lambda_j + c_jt$,则此时$t = \frac{\lambda_j - \lambda_i}{c_i - c_j} > \delta$矛盾!故$P(t)$为所求矩阵.

\item 先证当\(s\)充分大时, \(A(s)\)有\(n\)个不同的特征值. 由第一圆盘定理, \(A(s)\)的特征值落在下列戈氏圆盘中:
\begin{align*}
D_{i}:|z - a_{ii}-s^{i}| \leqslant  R_{i}=\sum_{j = 1,j\neq i}^{n}|a_{ij}|,\ 1 \leqslant  i \leqslant  n.
\end{align*}
取\(s\)充分大, 使得\(s^{n} \gg s^{n - 1} \gg \cdots \gg s\). 注意到\(R_{i}\)的值固定, 故\(D_{i}\)的圆心之间的距离大于半径\(R_{i}\), 从而\(D_{i}\)互不相交, 各自构成了一个连通分支. 再由第二圆盘定理, 每个连通分支\(D_{i}\)中有且仅有一个特征值, 于是\(A(s)\)有\(n\)个不同的特征值.

设\(f_{s}(\lambda)=|\lambda I_{n}-A(s)|\)是\(A(s)\)的特征多项式, 则其判别式\(\Delta(f_{s}(\lambda))\)是关于\(s\)的多项式. 由前面的讨论可知, 当\(s\)充分大时, \(f_{s}(\lambda)\)无重根, 从而\(\Delta(f_{s}(\lambda)) \neq 0\), 即\(\Delta(f_{s}(\lambda))\)是关于\(s\)的非零多项式. 若\(\Delta(f_{s}(\lambda))\)的所有复根都是零, 则任取一个正数\(\delta\); 若\(\Delta(f_{s}(\lambda))\)的复根不全为零, 则可取\(\delta\)为\(\Delta(f_{s}(\lambda))\)的非零复根的模长的最小值. 于是对任意的\(s \in (0,\delta)\), \(s\)都不是\(\Delta(f_{s}(\lambda))\)的根, 即\(\Delta(f_{s}(\lambda)) \neq 0\), 从而\(f_{s}(\lambda)\)都无重根, 即\(A(s)\)都有\(n\)个不同的特征值. 
\end{enumerate}

\end{proof}

\begin{example}
设 $f: M_n(\mathbb{C}) \to M_n(\mathbb{C})$ 是线性映射,满足 $A$ 可逆当且仅当 $f(A)$ 可逆,证明:存在常数 $c$ 使得 $|f(A)| = c|A|$ 对任意 $A$ 恒成立. 
\end{example}
\begin{proof}
由条件可知,$|A| = 0$ 当且仅当 $|f(A)| = 0$。于是对 $\forall A \in M_n(\mathbb{C}), \forall \lambda \in \mathbb{C}$,都有
\begin{align}
|\lambda I - A| = 0 \Longleftrightarrow |f(\lambda I - A)| = |\lambda f(I) - f(A)| = 0. \label{100.42}
\end{align}
任取 $A \in M_n(\mathbb{C})$,设 $A$ 的特征值分别为 $\lambda_1, \cdots, \lambda_n$,则
当 $A$ 有 $n$ 个不同特征值时,即 $\lambda_1, \cdots, \lambda_n$ 互不相同,则
\begin{align*}
|\lambda I - A| = (\lambda - \lambda_1) \cdots (\lambda - \lambda_n).
\end{align*}
从而由 \eqref{100.42} 式可知
\begin{align*}
|f(\lambda_i I - A)| = |\lambda_i f(I) - f(A)| = 0, \quad i = 1, 2, \cdots, n.
\end{align*}
又因为 $\deg |\lambda f(I) - f(A)| \leqslant n$,所以存在 $c_A \ne 0$,使得
\begin{align}
|f(\lambda I - A)| = |\lambda f(I) - f(A)| = c_A (\lambda - \lambda_1) \cdots (\lambda - \lambda_n) = c_A |\lambda I - A|, \quad \forall \lambda \in \mathbb{C}. \label{100.43}
\end{align}
显然此时 $|\lambda f(I) - f(A)|$ 是 $n$ 次多项式,于是 $f(I)$ 可逆。否则,$f(I)$ 一定有零特征值,从而由矩阵的相抵标准型可知,存在可逆阵 $G$,使得
\begin{align*}
f(I) = G B G^{-1}, \quad \text{其中} \quad B = \begin{pmatrix}
0 & & * \\
& \ddots & \\
& & b
\end{pmatrix}.
\end{align*}
于是
\begin{align*}
|\lambda f(I) - f(A)| = |G (\lambda f(I) - f(A)) G^{-1}| = |\lambda B - G f(A) G^{-1}| = \begin{vmatrix}
b_1 & & * \\
& \ddots & \\
& & \lambda - b_n
\end{vmatrix}.
\end{align*}
故 $\deg |\lambda f(I) - f(A)| < n$,这与 $|\lambda f(I) - f(A)|$ 是 $n$ 次多项式矛盾!从而
\begin{align*}
|\lambda f(I) - f(A)| = |f(I)| |\lambda I - f(I)^{-1} f(A)| = c_A |\lambda I - A|, \quad \forall \lambda \in \mathbb{C}.
\end{align*}
比较上式等式两边多项式(关于 $\lambda$)的最高次项的系数即得 $c_A = |f(I)|$,因此 $c_A$ 与 $A$ 无关。再结合 \eqref{100.43} 式可得
\begin{align*}
|f(\lambda I - A)| = |\lambda f(I) - f(A)| = |f(I)| |\lambda I - A|, \quad \forall \lambda \in \mathbb{C}.
\end{align*}
令 $\lambda = 0$,则有 $|f(A)| = |f(I)| |A|$。

综上,对任何 $A \in M_n(\mathbb{C})$ 且 $A$ 具有 $n$ 个不同特征值的矩阵,都有 $|f(A)| = |f(I)| |A|$。

对一般的矩阵 $A_0$,令
\begin{align*}
A(s) = A_0 + D_s, \quad \text{其中} \quad D_s = \begin{pmatrix}
s & & \\
& \ddots & \\
& & s^n
\end{pmatrix},
\end{align*}
则由\nrefpro{proposition:不可对角化矩阵的摄动}{(3)}可知,存在 $\delta \in (0, 1)$,使得对 $\forall s \in (0, \delta)$,都有 $A(s)$ 可对角化,从而此时 $A(s)$ 由 $n$ 个不同的特征值。于是由上述讨论可知
\begin{align}
|f(A(s))| = |f(I)| |A(s)| \Longleftrightarrow |f(A_0 + D_s)| = |f(I)| |A_0 + D_s| \Longleftrightarrow |f(A_0)| + |f(D_s)| = |f(I)| |A_0 + D_s|. \label{100.44}
\end{align}
注意到此时 $s < 1$,并且 $D_s$ 的特征值为 $s, \cdots, s^n$ 互不相同。故由上述讨论可得
\begin{align}
|f(D_s)| = |D_s|. \label{100.45}
\end{align}
于是结合 \eqref{100.44} \eqref{100.45} 式可得
\begin{align*}
|f(A_0)| + |D_s| = |f(I)| |A_0 + D_s|, \quad \forall s \in (0, \delta).
\end{align*}
由于上式两边都是关于 $s$ 的多项式,令 $s \rightarrow 0$,可得
\begin{align*}
|f(A_0)| = |f(I)| |A_0|.
\end{align*}
故结论得证。

\end{proof}

\begin{example}
设矩阵 $A = (a_{ij})_{n\times n}$ 满足所有元素均非负且 $\sum_{i,j = 1}^{n} a_{ij} = 1$,证明:$|\det A| \leqslant  1$ 且如果取等,则所有特征值的模长均为 $1$. 
\end{example}
\begin{proof}
设$A$的特征值分别为$\lambda_1,\cdots,\lambda_n$,$D_i:|z-a_{ii}|\leqslant \sum_{j\ne i}a_{ij}(1\leqslant i\leqslant n)$是$A$的$n$个戈氏圆盘。由\hyperref[theorem:第一圆盘定理]{第一圆盘定理}可知$\lambda_1,\cdots,\lambda_n$都落在$\bigcup_{i=1}^n D_i$中。对$\forall \lambda \in \bigcup_{i=1}^n D_i$,都存在$k\in \{1,2,\cdots,n\}$,使得$\lambda \in D_k$,从而
\begin{align*}
|\lambda|\leqslant |\lambda -a_{kk}|+a_{kk}\leqslant \sum_{j\ne k}a_{kj}+a_{kk}\leqslant 1.
\end{align*}
因此$|\lambda_1|,\cdots,|\lambda_n|\leqslant 1$,故$\det A\leqslant |\lambda_1\cdots \lambda_n|\leqslant 1$,当且仅当$A$的所有特征值模长为$1$等号成立。

\end{proof}














































\end{document}
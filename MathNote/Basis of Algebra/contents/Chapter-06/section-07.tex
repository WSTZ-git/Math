\documentclass[../../main.tex]{subfiles}
\graphicspath{{\subfix{../../image/}}} % 指定图片目录,后续可以直接使用图片文件名。

% 例如:
% \begin{figure}[H]
% \centering
% \includegraphics[scale=0.3]{image-01.01}
% \caption{图片标题}
% \label{figure:image-01.01}
% \end{figure}
% 注意:上述\label{}一定要放在\caption{}之后,否则引用图片序号会只会显示??.

\begin{document}

\section{其他}

\begin{example}
设 $A$, $B$, $C$ 都是 $n$ 阶复方阵,记 $M = \begin{pmatrix}
A & B & C \\
C & A & B \\
B & C & A
\end{pmatrix}$,证明:$M$ 的特征值是 $A + B + C$, $A + wB + w^{2}C$, $A + w^{2}B + wC$ 的特征值构成的集合的并,这里 $w$ 是三次单位根,并集记重复. 
\end{example}
\begin{note}
观察到$M$矩阵与循环矩阵由类似结构,回忆\refpro{proposition:循环行列式计算公式}的证明过程,利用与\refpro{proposition:循环行列式计算公式}相同的方法构造相似矩阵与对应的过渡矩阵.
\end{note}
\begin{proof}
注意到
\begin{align*}
\begin{pmatrix}
A & B & C \\
C & A & B \\
B & C & A
\end{pmatrix}
\begin{pmatrix}
1 & 1 & 1 \\
1 & w & w^2 \\
1 & w^2 & w^4
\end{pmatrix}
=
\begin{pmatrix}
1 & 1 & 1 \\
1 & w & w^2 \\
1 & w^2 & w^4
\end{pmatrix}
\begin{pmatrix}
A+B+C & & \\
& A+wB+w^2C & \\
& & A+wB+w^4C
\end{pmatrix},
\end{align*}
即
\begin{align*}
\begin{pmatrix}
1 & 1 & 1 \\
1 & w & w^2 \\
1 & w^2 & w^4
\end{pmatrix}^{-1}
\begin{pmatrix}
A & B & C \\
C & A & B \\
B & C & A
\end{pmatrix}
\begin{pmatrix}
1 & 1 & 1 \\
1 & w & w^2 \\
1 & w^2 & w^4
\end{pmatrix}
=
\begin{pmatrix}
A+B+C & & \\
& A+wB+w^2C & \\
& & A+wB+w^4C
\end{pmatrix},
\end{align*}
故$\begin{pmatrix}
A & B & C \\
C & A & B \\
B & C & A
\end{pmatrix}$与$\begin{pmatrix}
A+B+C & & \\
& A+wB+w^2C & \\
& & A+wB+w^4C
\end{pmatrix}$相似. 因此结论得证.
\end{proof}

\begin{example}
设$A$是$n$阶方阵,$\lambda_1,\cdots,\lambda_k$是$A$的$k$个互不相同的特征值,$v_i$是属于特征值$\lambda_i$的特征向量,若$W$是$A$的一个不变子空间,且$w = c_1v_1 + \cdots + c_kv_k\in W$,这里$c_1,\cdots,c_k$全都非零,证明:所有$v_i$均在$W$中。 
\end{example}
\begin{proof}
$\forall w\in W$,都有$w = c_1v_1 + \cdots + c_kv_k$,
从而由$W$是$A$的不变子空间及$v_i$是属于特征值$\lambda_i$的特征向量可知
\begin{align*}
\alpha_1 &= Aw = \lambda_1c_1v_1 + \cdots + \lambda_kc_kv_k\in W, \\
\alpha_2 &= A^2w = \lambda_1^2c_1v_1 + \cdots + \lambda_k^2c_kv_k\in W, \\
&\cdots \cdots \cdots \cdots \\
\alpha_k &= A^{k - 1}w = \lambda_1^{k - 1}c_1v_1 + \cdots + \lambda_k^{k - 1}c_kv_k\in W.
\end{align*}
于是
\begin{align*}
\begin{pmatrix}
\alpha_1 \\
\alpha_2 \\
\vdots \\
\alpha_k
\end{pmatrix} = 
\begin{pmatrix}
1 & 1 & \cdots & 1 \\
\lambda_1 & \lambda_2 & \cdots & \lambda_k \\
\vdots & \vdots & & \vdots \\
\lambda_1^{k - 1} & \lambda_2^{k - 1} & \cdots & \lambda_k^{k - 1}
\end{pmatrix}
\begin{pmatrix}
c_1v_1 \\
c_2v_2 \\
\vdots \\
c_kv_k
\end{pmatrix}.
\end{align*}
利用Vandermonde行列式可知
\begin{align*}
\left| \begin{matrix}
1&		1&		\cdots&		1\\
\lambda _1&		\lambda _2&		\cdots&		\lambda _k\\
\vdots&		\vdots&		&		\vdots\\
\lambda _{1}^{k-1}&		\lambda _{2}^{k-1}&		\cdots&		\lambda _{k}^{k-1}\\
\end{matrix} \right|\ne 0\Rightarrow 
\begin{pmatrix}
1 & 1 & \cdots & 1 \\
\lambda_1 & \lambda_2 & \cdots & \lambda_k \\
\vdots & \vdots & & \vdots \\
\lambda_1^{k - 1} & \lambda_2^{k - 1} & \cdots & \lambda_k^{k - 1}
\end{pmatrix} \text{可逆},
\end{align*}
因此
\begin{align*}
\begin{pmatrix}
c_1v_1 \\
c_2v_2 \\
\vdots \\
c_kv_k
\end{pmatrix} = 
\begin{pmatrix}
\alpha_1 \\
\alpha_2 \\
\vdots \\
\alpha_k
\end{pmatrix} 
\begin{pmatrix}
1 & 1 & \cdots & 1 \\
\lambda_1 & \lambda_2 & \cdots & \lambda_k \\
\vdots & \vdots & & \vdots \\
\lambda_1^{k - 1} & \lambda_2^{k - 1} & \cdots & \lambda_k^{k - 1}
\end{pmatrix}^{-1}.
\end{align*}
进而对$\forall i\in \{1,2,\cdots,k\}$,都有$c_iv_i\in L(\alpha_1,\alpha_2,\cdots,\alpha_k)$,又$c_i\ne 0$,故$v_i\in L(\alpha_1,\alpha_2,\cdots,\alpha_k)$。又因为$\alpha_i\in W\ (1\le i\le k)$,所以$v_i\in W$.
\end{proof}

\begin{example}
设$A$是$d\times d$整数矩阵且满足$I + A + A^2 + \cdots + A^{100} = 0$,对任意正整数$n \leq 100$,证明:$A^n + A^{n + 1} + \cdots + A^{100}$的行列式为$1$。 
\end{example}
\begin{proof}
设$m(x) = x^{100}+\cdots + x + 1$,则$m(x)\in \mathbb{Q}[x]$且$m(x)$不可约。
再设$A$的极小多项式为$g(x)$,则由条件可知$g(x)\mid m(x)$,再由$m(x)$不可约可得$g(x)=m(x)$。
记$A$的不变因子分别为$d_1,\cdots,d_k$,其中$d_i\mid d_{i + 1}\ (1\leqslant i\leqslant k)$,并且$d_k = m(x)$。于是$d_i\mid m(x)$,而$m(x)$不可约,故$A$的不变因子为$m(x),\cdots,m(x)$(共有$k$个)。从而
\begin{align*}
|\lambda I - A| = (m(\lambda))^k.
\end{align*}
又因为$A$是$d$阶矩阵,所以$d = 100k$。再根据矩阵的有理标准型可知,存在可逆矩阵$P$,使得
\begin{align*}
PAP^{-1} = F = \begin{pmatrix}
F(m(x)) & & \\
& \ddots & \\
& & F(m(x))
\end{pmatrix}_{100s\times 100s},
\end{align*}
其中$F(m(x)) = \begin{pmatrix}
0 & 1 & \cdots & 0 \\
\vdots & \vdots & \ddots & \vdots \\
0 & 0 & \cdots & 1 \\
-1 & -1 & \cdots & -1
\end{pmatrix}_{100\times 100}$。
又因为条件和结论在线性变换$A\rightarrow PAP^{-1}=F$下不改变,故不妨设$A = F$。设$F(m(x))$的特征值分别为$\lambda_1,\cdots,\lambda_{100}$,则
\begin{align*}
|\lambda_i I - F(m(x))| = 1 + \lambda_i + \cdots + \lambda_i^{100} = 0,
\end{align*}
从而$\lambda_i$都是$1 + x + \cdots + x^{100} = 0$的根,故
\begin{align}
\lambda_k = e^{\frac{2k\pi \mathrm{i}}{101}}\ (1\leqslant k\leqslant 100). \label{100.50}
\end{align}
再根据Vieta定理可得
\begin{align*}
|F(m(x))| = \lambda_1\cdots \lambda_{100} = 1.
\end{align*}
从而$|F| = |F(m(x))|^s = 1$,并且$F$的特征值就是$\lambda_k = e^{\frac{2k\pi \mathrm{i}}{101}}\ (1\leqslant k\leqslant 100)$且每个特征值都是$100$重的。

注意到对$\forall n\in [1,100]\cap \mathbb{N}$,有
\begin{align*}
|F^n + F^{n + 1} + \cdots + F^{100}| = |F|^n|I + F + \cdots + F^{100 - n}| = |I + F + \cdots + F^{100 - n}|.
\end{align*}
记$k = 100 - n$,则$k\in [0,99]\cap \mathbb{N}$。因此只需证
\begin{align*}
|I + F + \cdots + F^k| = 1,\forall k\in [0,99]\cap \mathbb{N}.
\end{align*}
当$k = 0$时,结论显然成立。当$k\in [1,99]\cap \mathbb{N}$时,我们有
\begin{align}
|I + F + \cdots + F^k| = 1 &\Longleftrightarrow |(I + F + \cdots + F^k)(I - F)| = |I - F| \Longleftrightarrow |I - F^{k + 1}| = |I - F| \notag \\
&\Longleftrightarrow (1 - \lambda_1^{k + 1})(1 - \lambda_2^{k + 1})\cdots (1 - \lambda_k^{k + 1}) = (1 - \lambda_1)(1 - \lambda_2)\cdots (1 - \lambda_k). \label{100.51}
\end{align}
记$\varepsilon = e^{\frac{2\pi \mathrm{i}}{101}}$,则由\eqref{100.50}式可知上式最后一个等式等价于
\begin{align*}
(1 - \varepsilon^{k + 1})(1 - \varepsilon^{2(k + 1)})\cdots (1 - \varepsilon^{100(k + 1)}) = (1 - \varepsilon)(1 - \varepsilon^2)\cdots (1 - \varepsilon^{100}).
\end{align*}
由$(k + 1,100) = 1\ (1\leqslant k\leqslant 99)$及\refpro{Abstract Algebra-proposition:有限循环群的生成元的充要条件}可知,$\varepsilon^{k + 1}$是$101$阶循环群$\{1,\varepsilon,\cdots,\varepsilon^{100}\} = \langle \varepsilon \rangle$的一个生成元,因此
\begin{align*}
\{1,\varepsilon,\cdots,\varepsilon^{100}\} = \langle \varepsilon \rangle = \langle \varepsilon^{k + 1} \rangle = \{1,\varepsilon^{k + 1},\cdots,\varepsilon^{100(k + 1)}\}.
\end{align*}
从而
\begin{align*}
(1 - \varepsilon^{k + 1})(1 - \varepsilon^{2(k + 1)})\cdots (1 - \varepsilon^{100(k + 1)}) = (1 - \varepsilon)(1 - \varepsilon^2)\cdots (1 - \varepsilon^{100}).
\end{align*}
故再由\eqref{100.51}式,结论得证。 
\end{proof}

\begin{example}
设$A$为$n$阶方阵,满足$(A^T)^m = A^k$,其中$m,k$是不同的正整数,证明:$A$的特征值是零或者单位根。 
\end{example}
\begin{proof}
由条件可知
\begin{align*}
(A^T)^{m^2}&=A^{mk},\\
(A^T)^{mk}&=A^{k^2}.
\end{align*}
进而
\begin{align*}
A^{m^2}=(A^T)^{mk}=A^{k^2}.
\end{align*}
于是$A$的特征值$\lambda$都满足
\begin{align*}
\lambda^{m^2}=\lambda^{k^2}.
\end{align*}
故$\lambda$为$0$或单位根。 
\end{proof}

\begin{example}
设$A,B$是$n$阶方阵,满足$A^2B + BA^2 = 2ABA$,证明:$AB - BA$是幂零矩阵。 
\end{example}
\begin{proof}
记$C = AB - BA$,则显然$\mathrm{tr}(C)=0$。由条件可知
\begin{align*}
A^2B + BA^2 = 2ABA \Longleftrightarrow A^2B - ABA = ABA - BA^2 
\Longleftrightarrow A(AB - BA) = (AB - BA)A \Longleftrightarrow AC = CA.
\end{align*}
从而由上式及矩阵迹的交换性可得,对$\forall k\in [1,n]\cap \mathbb{N}$,都有
\begin{align*}
\mathrm{tr}(C^k) = \mathrm{tr}(C^{k - 1}(AB - BA)) = \mathrm{tr}(C^{k - 1}AB) - \mathrm{tr}(C^{k - 1}BA) 
= \mathrm{tr}(AC^{k - 1}B) - \mathrm{tr}(AC^{k - 1}B) = 0.
\end{align*}
故由\refpro{proposition:幂零矩阵关于迹的充要条件}可知$C=AB-BA$是幂零矩阵.
\end{proof}














\end{document}
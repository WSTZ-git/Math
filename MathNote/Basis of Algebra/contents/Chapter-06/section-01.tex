\documentclass[../../main.tex]{subfiles}
\graphicspath{{\subfix{../../image/}}} % 指定图片目录,后续可以直接使用图片文件名。

% 例如:
% \begin{figure}[h]
% \centering
% \includegraphics{image-01.01}
% \caption{图片标题}
% \label{fig:image-01.01}
% \end{figure}
% 注意:上述\label{}一定要放在\caption{}之后,否则引用图片序号会只会显示??.

\begin{document}

\section{矩阵的Kronecker积}

\begin{definition}[矩阵的Kronecker积]
设 $A=(a_{ij})$ 和 $B=(b_{ij})$ 分别是数域 $\mathbb{F}$ 上的 $m\times n$ 和 $k\times l$ 矩阵,它们的 \textbf{Kronecker 积} $A\otimes B$ 是 $\mathbb{F}$ 上的 $mk\times nl$ 矩阵:
\begin{align*}
A\otimes B = 
\begin{pmatrix}
a_{11}B & a_{12}B & \cdots & a_{1n}B \\
a_{21}B & a_{22}B & \cdots & a_{2n}B \\
\vdots & \vdots & & \vdots \\
a_{m1}B & a_{m2}B & \cdots & a_{mn}B
\end{pmatrix}
\end{align*}
\end{definition}

\begin{theorem}[矩阵的Kronecker积的基本性质]\label{theorem:矩阵的Kronecker积的基本性质}
证明矩阵的 Kronecker 积满足下列性质 (假设以下的矩阵加法和乘法都有意义):
\begin{enumerate}[(1)]
\item\label{矩阵的Kronecker积的基本性质(1)} $(A + B)\otimes C = A\otimes C + B\otimes C$, $A\otimes (B + C) = A\otimes B + A\otimes C$;
\item\label{矩阵的Kronecker积的基本性质(2)} $(kA)\otimes B = k(A\otimes B) = A\otimes (kB)$;
\item\label{矩阵的Kronecker积的基本性质(3)} $(A\otimes C)(B\otimes D)=(AB)\otimes (CD)$;
\item\label{矩阵的Kronecker积的基本性质(4)} $(A\otimes B)\otimes C = A\otimes (B\otimes C)$;
\item\label{矩阵的Kronecker积的基本性质(5)} $I_m\otimes I_n = I_{mn}$;
\item\label{矩阵的Kronecker积的基本性质(6)} $(A\otimes B)' = A'\otimes B'$;
\item\label{矩阵的Kronecker积的基本性质(7)} 若 $A$, $B$ 都是可逆矩阵,则 $A\otimes B$ 也是可逆矩阵,并且
\begin{align*}
    (A\otimes B)^{-1}=A^{-1}\otimes B^{-1};
\end{align*}
\item\label{矩阵的Kronecker积的基本性质(8)} 若 $A$ 是 $m$ 阶矩阵,$B$ 是 $n$ 阶矩阵,则 $|A\otimes B| = |A|^n|B|^m$;
\item\label{矩阵的Kronecker积的基本性质(9)} 若 $A$ 是 $m$ 阶矩阵,$B$ 是 $n$ 阶矩阵,则 $\mathrm{tr}(A\otimes B)=\mathrm{tr}(A)\cdot\mathrm{tr}(B)$.
\item\label{矩阵的Kronecker积的基本性质(10)} 设$A,B$均为上三角阵,且$A,B$的主对角元素分别依次为$a_1,\cdots,a_n$和$b_1,\cdots,b_m$,则$A\otimes B$仍是上三角阵,且$A\otimes B$的主对角元素依次为$a_1b_1,\cdots,a_1b_m,a_2b_1,\cdots,a_2b_m,\cdots,a_nb_1,\cdots,a_nb_m.$
\item\label{矩阵的Kronecker积的基本性质(11)} 设$A,B$均为对角阵,且$A,B$的主对角元素分别依次为$a_1,\cdots,a_n$和$b_1,\cdots,b_m$,则$A\otimes B$仍是对角阵,且$A\otimes B$的主对角元素依次为$a_1b_1,\cdots,a_1b_m,a_2b_1,\cdots,a_2b_m,\cdots,a_nb_1,\cdots,a_nb_m.$
\end{enumerate}
\end{theorem}
\begin{proof}
\begin{enumerate}[(1)]
\item 由 Kronecker 积的定义经简单计算即可验证.

\item 由 Kronecker 积的定义经简单计算即可验证.

\item 设 $A=(a_{ij})$ 是 $m\times p$ 矩阵,$B=(b_{ij})$ 是 $p\times n$ 矩阵,$C=(c_{ij})$ 是 $k\times q$ 矩阵,$D=(d_{ij})$ 是 $q\times l$ 矩阵. 由 Kronecker 积的定义以及分块矩阵的乘法可得
\begin{align*}
(A\otimes C)(B\otimes D)&=
\begin{pmatrix}
a_{11}C & a_{12}C & \cdots & a_{1p}C \\
a_{21}C & a_{22}C & \cdots & a_{2p}C \\
\vdots & \vdots & & \vdots \\
a_{m1}C & a_{m2}C & \cdots & a_{mp}C
\end{pmatrix}
\begin{pmatrix}
b_{11}D & b_{12}D & \cdots & b_{1n}D \\
b_{21}D & b_{22}D & \cdots & b_{2n}D \\
\vdots & \vdots & & \vdots \\
b_{p1}D & b_{p2}D & \cdots & b_{pn}D
\end{pmatrix}\\
&=
\begin{pmatrix}
\sum_{j = 1}^{p}a_{1j}b_{j1}CD & \sum_{j = 1}^{p}a_{1j}b_{j2}CD & \cdots & \sum_{j = 1}^{p}a_{1j}b_{jn}CD \\
\sum_{j = 1}^{p}a_{2j}b_{j1}CD & \sum_{j = 1}^{p}a_{2j}b_{j2}CD & \cdots & \sum_{j = 1}^{p}a_{2j}b_{jn}CD \\
\vdots & \vdots & & \vdots \\
\sum_{j = 1}^{p}a_{mj}b_{j1}CD & \sum_{j = 1}^{p}a_{mj}b_{j2}CD & \cdots & \sum_{j = 1}^{p}a_{mj}b_{jn}CD
\end{pmatrix}\\
&=(AB)\otimes (CD).
\end{align*}

\item 设 $A=(a_{ij})$, $B=(b_{ij})$ 和 $C=(c_{ij})$ 分别是 $m\times n$, $k\times l$ 和 $p\times q$ 矩阵,则经计算即可发现 $(A\otimes B)\otimes C$ 和 $A\otimes (B\otimes C)$ 都等于下面的 $mkp\times nlq$ 矩阵:
\[
\begin{pmatrix}
a_{11}b_{11}C & \cdots & a_{11}b_{1l}C & \cdots & a_{1n}b_{11}C & \cdots & a_{1n}b_{1l}C \\
\vdots & & \vdots & & \vdots & & \vdots \\
a_{11}b_{k1}C & \cdots & a_{11}b_{kl}C & \cdots & a_{1n}b_{k1}C & \cdots & a_{1n}b_{kl}C \\
\vdots & & \vdots & & \vdots & & \vdots \\
a_{m1}b_{11}C & \cdots & a_{m1}b_{1l}C & \cdots & a_{mn}b_{11}C & \cdots & a_{mn}b_{1l}C \\
\vdots & & \vdots & & \vdots & & \vdots \\
a_{m1}b_{k1}C & \cdots & a_{m1}b_{kl}C & \cdots & a_{mn}b_{k1}C & \cdots & a_{mn}b_{kl}C
\end{pmatrix}.
\]

\item 由 Kronecker 积的定义经简单计算即可验证.

\item 由 Kronecker 积的定义经简单计算即可验证.

\item 由 (3) 和 (5) 可得
\begin{align*}
(A\otimes B)(A^{-1}\otimes B^{-1})=(AA^{-1})\otimes (BB^{-1})=I_m\otimes I_n = I_{mn}.
\end{align*}

\item 由 Laplace 定理容易证明:
\begin{align*}
|A\otimes I_n| = |A|^n, \quad |I_m\otimes B| = |B|^m;
\end{align*}
再由 (3) 以及矩阵乘积的行列式等于行列式的乘积可得
\begin{align*}
|A\otimes B| = |(A\otimes I_n)(I_m\otimes B)| = |A\otimes I_n||I_m\otimes B| = |A|^n|B|^m. 
\end{align*}

\item 由 Kronecker 积的定义经简单计算即可验证.

\item 由 Kronecker 积的定义经简单计算即可验证.

\item 由 Kronecker 积的定义经简单计算即可验证.
\end{enumerate}
\end{proof}

\begin{proposition}[矩阵的Kronecker积的秩]\label{proposition:矩阵的Kronecker积的秩}
设 $A$, $B$ 分别为 $m\times n$, $k\times l$ 矩阵,求证: $\mathrm{r}(A\otimes B)=\mathrm{r}(A)\cdot\mathrm{r}(B)$.
\end{proposition}
\begin{proof}
设 $\mathrm{r}(A)=r$, $\mathrm{r}(B)=s$, $P$, $Q$, $R$, $S$ 为可逆矩阵,使得
\begin{align*}
PAQ = \begin{pmatrix}
I_r & O \\
O & O
\end{pmatrix}, \quad RBS = \begin{pmatrix}
I_s & O \\
O & O
\end{pmatrix},
\end{align*}
则由\hyperref[矩阵的Kronecker积的基本性质(7)]{性质 (7)} 可知 $P\otimes R$, $Q\otimes S$ 均非异,再由\hyperref[矩阵的Kronecker积的基本性质(3)]{性质 (3)}可得
\begin{align*}
(P\otimes R)(A\otimes B)(Q\otimes S)=(PAQ)\otimes (RBS)\sim\begin{pmatrix}
I_{rs} & O \\
O & O
\end{pmatrix},
\end{align*}
于是 $\mathrm{r}(A\otimes B)=rs=\mathrm{r}(A)\cdot\mathrm{r}(B)$.  
\end{proof}

\begin{corollary}\label{corollary:矩阵的Kronecker积关于行列满秩阵的推论}
设 $A$, $B$ 分别为 $m\times n$, $k\times l$ 矩阵,求证: $A\otimes B$ 是行满秩阵 (列满秩阵) 的充要条件是 $A$, $B$ 均为行满秩阵 (列满秩阵). 
\end{corollary}
\begin{proof}
由\hyperref[proposition:矩阵的Kronecker积的秩]{矩阵的Kronecker积的秩}可知
\begin{align*}
\mathrm{r}\left( A\otimes B \right) =\mathrm{r}\left( A \right) \cdot \mathrm{r}\left( B \right) .
\end{align*}
于是立得结论.
\end{proof}

\begin{proposition}[矩阵的Kronecker积的特征值]\label{proposition:矩阵的Kronecker积的特征值}
设 $A$, $B$ 分别是 $m$, $n$ 阶矩阵,$A$ 的特征值为 $\lambda_i$ ($1\leq i\leq m$),$B$ 的特征值为 $\mu_j$ ($1\leq j\leq n$),求证: $A\otimes B$ 的特征值为 $\lambda_i\mu_j$ ($1\leq i\leq m$; $1\leq j\leq n$).
\end{proposition}
\begin{proof}
由\refproposition{proposition:特征值全在同一数域的矩阵可上三角化}可知,存在 $m$ 阶可逆矩阵 $P$ 以及 $n$ 阶可逆矩阵 $Q$,使得
\begin{align*}
P^{-1}AP = 
\begin{pmatrix}
\lambda_1 & * & * & * \\
& \lambda_2 & * & * \\
& & \ddots & \vdots \\
& & & \lambda_m
\end{pmatrix}, \quad 
Q^{-1}BQ = 
\begin{pmatrix}
\mu_1 & * & * & * \\
& \mu_2 & * & * \\
& & \ddots & \vdots \\
& & & \mu_n
\end{pmatrix}.
\end{align*}
由\hyperref[矩阵的Kronecker积的基本性质(10)]{性质(10)}可知$(P^{-1}AP)\otimes (Q^{-1}BQ)$仍是上三角矩阵且 $(P^{-1}AP)\otimes (Q^{-1}BQ)$ 的主对角元素依次为
\[
\lambda_1\mu_1,\cdots,\lambda_1\mu_n,\lambda_2\mu_1,\cdots,\lambda_2\mu_n,\cdots,\lambda_m\mu_1\cdots,\lambda_m\mu_n.
\]
注意到 $(P^{-1}AP)\otimes (Q^{-1}BQ)=(P\otimes Q)^{-1}(A\otimes B)(P\otimes Q)$,因此$(P^{-1}AP)\otimes (Q^{-1}BQ)$和$A\otimes B$相似,又相似矩阵特征值相同,故结论得证.
\end{proof}

\begin{proposition}\label{proposition:线性变换AXB的特征值}
设 $A$, $B$ 分别为 $m$, $n$ 阶矩阵,$V$ 为 $m\times n$ 矩阵全体构成的线性空间,$V$ 上的线性变换 $\varphi$ 定义为: $\varphi(\boldsymbol{X}) = \boldsymbol{A}\boldsymbol{X}\boldsymbol{B}$. 设 $\boldsymbol{A}$ 的特征值为 $\lambda_i$ ($1\leq i\leq m$),$\boldsymbol{B}$ 的特征值为 $\mu_j$ ($1\leq j\leq n$). 求证: 线性变换 $\varphi$ 的特征值为 $\lambda_i\mu_j$ ($1\leq i\leq m$; $1\leq j\leq n$). 
\end{proposition}
\begin{remark}
本题是\refexample{example:由矩阵诱导的线性变换的特征值与其诱导矩阵相同}的推广.
\end{remark}
\begin{proof}
取 $V$ 的一组基为 $m\times n$ 基础矩阵:
\[
E_{11},\cdots,E_{1n},E_{21},\cdots,E_{2n},\cdots,E_{m1},\cdots,E_{mn},
\]
我们首先证明 $\varphi$ 在这组基下的表示矩阵为 $A\otimes B'$. 事实上,
\begin{align*}
\varphi(E_{ij}) = AE_{ij}B = Ae_{i}f_{j}'B = \sum_{k = 1}^{m}\sum_{l = 1}^{n}a_{ki}b_{jl}E_{kl},
\end{align*}
其中 $e_{i}$, $f_{j}$ 分别是 $m$, $n$ 维标准单位列向量, 故 $\varphi$ 的表示矩阵为
\begin{align*}
\begin{pmatrix}
a_{11}B' & a_{12}B' & \cdots & a_{1m}B' \\
a_{21}B' & a_{22}B' & \cdots & a_{2m}B' \\
\vdots & \vdots & & \vdots \\
a_{m1}B' & a_{m2}B' & \cdots & a_{mm}B'
\end{pmatrix}=A\otimes B'.
\end{align*}
注意到 $B'$ 与 $B$ 有相同的特征值, 故由\hyperref[proposition:矩阵的Kronecker积的特征值]{矩阵的Kronecker积的特征值}可知, $\varphi$ 的特征值为 $\lambda_{i}\mu_{j}$.
\end{proof}

\begin{example}
设 $A$, $B$ 分别为 $m$, $n$ 阶矩阵,$V$ 为 $m\times n$ 矩阵全体构成的线性空间,$V$ 上的线性变换 $\varphi$ 定义为: $\varphi(\boldsymbol{X}) = \boldsymbol{A}\boldsymbol{X}\boldsymbol{B}$. 证明: $\varphi$ 是线性自同构的充要条件是 $A$, $B$ 都是可逆矩阵.
\end{example}
\begin{remark}
例 4.16 作为本题的特例,我们已经给出了两种证法,其中证法 1 仍然可以适用于本题,证法 2 则需改用例 6.99 进行讨论,当然也可用第 4 章解题 13 进行统一的处理,请读者自行补充细节. 下面再给出两种证法.(这里的题目与题号都是指白皮书上的)
\end{remark}
\begin{proof}
{\color{blue}证法三:}由\refproposition{proposition:线性变换AXB的特征值}的证明过程可知,$\varphi$ 在基础矩阵这组基下的表示矩阵为 $A\otimes B'$,再由\hyperref[矩阵的Kronecker积的基本性质(8)]{性质 (8) }可知 $|A\otimes B'| = |A|^n|B|^m$,故 $\varphi$ 是自同构当且仅当表示矩阵 $A\otimes B'$ 是可逆矩阵,这也当且仅当 $A$, $B$ 都是可逆矩阵.

{\color{blue}证法四:}由\refproposition{proposition:线性变换AXB的特征值}可知,$\varphi$ 是自同构当且仅当 $\varphi$ 所有的特征值 $\lambda_i\mu_j\neq 0$,这当且仅当所有的 $\lambda_i\neq 0$ 以及所有的 $\mu_j\neq 0$,这也当且仅当 $A$, $B$ 都是可逆矩阵.
\end{proof}

\begin{example}
设 $A$, $B$ 分别为 $m$, $n$ 阶矩阵,$V$ 为 $m\times n$ 矩阵全体构成的线性空间,$V$ 上的线性变换 $\varphi$ 定义为: $\varphi(\boldsymbol{X}) = \boldsymbol{A}\boldsymbol{X}\boldsymbol{B}$. 证明: $\varphi$ 是幂零线性变换的充要条件是 $A$, $B$ 至少有一个是幂零矩阵.
\end{example}
\begin{proof}
先证充分性. 不妨设 $A$ 是幂零矩阵,即存在正整数 $k$,使得 $A^k = O$,则 $\varphi^k(\boldsymbol{X}) = A^k\boldsymbol{X}B^k = O$,即 $\varphi^k = 0$,于是 $\varphi$ 是幂零线性变换.

再证必要性. 我们考虑必要性的逆否命题.设 $A$, $B$ 都不是幂零矩阵,即对任意给定的正整数 $k$,$A^k\neq O$,$B^k\neq O$,只要证明 $\varphi^k\neq 0$ 即可. 我们给出以下 4 种证法.

{\color{blue}证法一:}
不妨设 $A^k$ 的第 $i$ 列非零,$B^k$ 的第 $j$ 行非零,即有列向量 $A^ke_i\neq \boldsymbol{0}$,行向量 $f_j'B^k\neq \boldsymbol{0}$,其中 $e_i$, $f_j$ 分别是 $m$, $n$ 维标准单位列向量,于是
\begin{align*}
\varphi^k(E_{ij}) = A^kE_{ij}B^k = A^ke_if_j'B^k = (A^ke_i)(f_j'B^k)\neq O.
\end{align*}

{\color{blue}证法二:}设 $P_i$, $Q_i$ 为可逆矩阵,使得 $P_1A^kQ_1 = \mathrm{diag}\{I_r, O\}$,$P_2B^kQ_2 = \mathrm{diag}\{I_s, O\}$,不妨设 $r\geq s\geq 1$,于是
\begin{align*}
\varphi^k(Q_1P_2) = P_1^{-1}\mathrm{diag}\{I_r, O\}\mathrm{diag}\{I_s, O\}Q_2^{-1} = P_1^{-1}\mathrm{diag}\{I_s, O\}Q_2^{-1}\neq O.
\end{align*}

{\color{blue}证法三:} 由\refproposition{proposition:线性变换AXB的特征值}的证明过程可知,$\varphi^k$ 在基础矩阵这组基下的表示矩阵为 $A^k\otimes (B^k)'$,再由 Kronecker 积的定义可知 $A^k\otimes (B^k)'\neq O$,于是 $\varphi^k\neq 0$.

{\color{blue}证法四:}由\refproposition{proposition:幂零矩阵关于特征值的充要条件}可知,$\varphi$ 是幂零线性变换当且仅当 $\varphi$ 的所有特征值都等于零. 由于 $A$, $B$ 都不是幂零矩阵,故 $A$ 的特征值 $\lambda_i$ 不全为零,$B$ 的特征值 $\mu_j$ 不全为零. 再由\refproposition{proposition:线性变换AXB的特征值}可知,$\varphi$ 的特征值 $\lambda_i\mu_j$ 也不全为零,从而$\varphi$ 不是幂零线性变换.
\end{proof}

\begin{proposition}\label{proposition:线性变换AX-XB的表示矩阵和特征值}
设 $A,B$ 分别为 $m,n$ 阶矩阵,$V$ 为 $m \times n$ 矩阵全体构成的线性空间,$V$ 上的线性变换 $\varphi$ 定义为:$\varphi(\boldsymbol{X}) = \boldsymbol{AX} - \boldsymbol{XB}$。设 $A$ 的特征值为 $\lambda_i (1 \leq i \leq m)$,$B$ 的特征值为 $\mu_j (1 \leq j \leq n)$。求证:线性变换 $\varphi$ 的特征值为 $\lambda_i - \mu_j (1 \leq i \leq m; 1 \leq j \leq n)$。
\end{proposition}
\begin{proof}
取 $V$ 的一组基为 $m \times n$ 基础矩阵:
$E_{11},\cdots, E_{1n}, E_{21},\cdots, E_{2n},\cdots, E_{m1},\cdots, E_{mn}$,
类似\refproposition{proposition:幂零矩阵关于特征值的充要条件}的讨论可得,$\varphi$ 在上述基下的表示矩阵为 $\boldsymbol{A} \otimes \boldsymbol{I}_n - \boldsymbol{I}_m \otimes \boldsymbol{B}'$。由\refproposition{proposition:特征值全在同一数域的矩阵可上三角化}可知,存在 $m$ 阶可逆矩阵 $\boldsymbol{P}$ 以及 $n$ 阶可逆矩阵 $\boldsymbol{Q}$,使得
\begin{align*}
\boldsymbol{P}^{-1}\boldsymbol{AP} &= 
\begin{pmatrix}
\lambda_1 & * & * & * \\
 & \lambda_2 & * & * \\
 & & \ddots & \vdots \\
 & & & \lambda_m
\end{pmatrix}, \quad
\boldsymbol{Q}^{-1}\boldsymbol{B}'\boldsymbol{Q} = 
\begin{pmatrix}
\mu_1 & * & * & * \\
 & \mu_2 & * & * \\
 & & \ddots & \vdots \\
 & & & \mu_n
\end{pmatrix}.
\end{align*}
注意到
\begin{align*}
(\boldsymbol{P} \otimes \boldsymbol{Q})^{-1}(\boldsymbol{A} \otimes \boldsymbol{I}_n - \boldsymbol{I}_m \otimes \boldsymbol{B}')(\boldsymbol{P} \otimes \boldsymbol{Q}) 
&= (\boldsymbol{P}^{-1}\boldsymbol{AP}) \otimes \boldsymbol{I}_n - \boldsymbol{I}_m \otimes (\boldsymbol{Q}^{-1}\boldsymbol{B}'\boldsymbol{Q})
\end{align*}
是一个上三角矩阵,其主对角元素依次为
$\lambda_1 - \mu_1,\cdots, \lambda_1 - \mu_n, \lambda_2 - \mu_1,\cdots, \lambda_2 - \mu_n,\cdots, \lambda_m - \mu_1,\cdots, \lambda_m - \mu_n$,
由此即得结论。
\end{proof}

\begin{example}
设 $A,B$ 分别为 $m,n$ 阶矩阵,$V$ 为 $m \times n$ 矩阵全体构成的线性空间,$V$ 上的线性变换 $\varphi$ 定义为:$\varphi(\boldsymbol{X}) = \boldsymbol{AX} - \boldsymbol{XB}$。证明:若 $A,B$ 都是幂零矩阵,则 $\varphi$ 是幂零线性变换。
\end{example}
\begin{proof}
因为 $A,B$ 都是幂零矩阵,所以它们的特征值都为零。由\refproposition{proposition:线性变换AX-XB的表示矩阵和特征值}可知,$\varphi$ 的特征值也都为零,于是 $\varphi$ 是幂零线性变换。(也可由矩阵的运算直接证明本题。)
\end{proof}

\begin{example}
设 $\boldsymbol{A} = (a_{ij})$ 是 $n$ 阶矩阵,$g(\lambda) = |\lambda\boldsymbol{I}_n + \boldsymbol{A}|$。求证:$n^2$ 阶矩阵
\[
\boldsymbol{B} = 
\begin{pmatrix}
a_{11}\boldsymbol{I}_n + \boldsymbol{A} & a_{12}\boldsymbol{I}_n & \cdots & a_{1n}\boldsymbol{I}_n \\
a_{21}\boldsymbol{I}_n & a_{22}\boldsymbol{I}_n + \boldsymbol{A} & \cdots & a_{2n}\boldsymbol{I}_n \\
\vdots & \vdots & & \vdots \\
a_{n1}\boldsymbol{I}_n & a_{n2}\boldsymbol{I}_n & \cdots & a_{nn}\boldsymbol{I}_n + \boldsymbol{A}
\end{pmatrix}
\]
是可逆矩阵的充要条件是 $g(\boldsymbol{A})$ 是可逆矩阵。
\end{example}
\begin{proof}
显然 $\boldsymbol{B} = \boldsymbol{A} \otimes \boldsymbol{I}_n + \boldsymbol{I}_n \otimes \boldsymbol{A}$。设 $\boldsymbol{A}$ 的全体特征值为 $\lambda_1, \lambda_2, \cdots, \lambda_n$,则 $g(\lambda) = (\lambda + \lambda_1)(\lambda + \lambda_2)\cdots(\lambda + \lambda_n)$。由\refproposition{proposition:特征值全在同一数域的矩阵可上三角化}可知,存在 $n$ 阶可逆矩阵 $\boldsymbol{P}$,使得
\begin{align*}
\boldsymbol{P}^{-1}\boldsymbol{AP} &= 
\begin{pmatrix}
\lambda_1 & * & * & * \\
 & \lambda_2 & * & * \\
 & & \ddots & \vdots \\
 & & & \lambda_n
\end{pmatrix}.
\end{align*}
注意到
\begin{align*}
(\boldsymbol{P} \otimes \boldsymbol{P})^{-1}\boldsymbol{B}(\boldsymbol{P} \otimes \boldsymbol{P}) &= (\boldsymbol{P}^{-1}\boldsymbol{AP}) \otimes \boldsymbol{I}_n + \boldsymbol{I}_n \otimes (\boldsymbol{P}^{-1}\boldsymbol{AP})
\end{align*}
是一个上三角矩阵,其主对角元素为 $\lambda_i + \lambda_j (1 \leq i, j \leq n)$,故
\begin{align*}
|\boldsymbol{B}| &= \prod_{i,j = 1}^{n} (\lambda_i + \lambda_j) = \prod_{i = 1}^{n} g(\lambda_i).
\end{align*}
因为 $g(\boldsymbol{A})$ 的特征值为 $g(\lambda_1), g(\lambda_2), \cdots, g(\lambda_n)$,所以 $|\boldsymbol{B}| = |g(\boldsymbol{A})|$,从而 $\boldsymbol{B}$ 是可逆矩阵等价于 $g(\boldsymbol{A})$ 是可逆矩阵。
\end{proof}













\end{document}
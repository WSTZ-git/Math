\documentclass[../../main.tex]{subfiles}
\graphicspath{{\subfix{../../image/}}} % 指定图片目录,后续可以直接使用图片文件名。

% 例如:
% \begin{figure}[h]
% \centering
% \includegraphics{image-01.01}
% \label{fig:image-01.01}
% \caption{图片标题}
% \end{figure}

\begin{document}

\section{极小多项式与Cayley-Hamilton定理}

\begin{proposition}\label{proposition:矩阵一定适合一个非零多项式}
数域$\mathbb{K}$上的$n$阶矩阵$A$一定适合数域$\mathbb{K}$上的一个非零多项式.
\end{proposition}
\begin{proof}
我们已经知道, 数域$\mathbb{K}$上的$n$阶矩阵全体组成了$\mathbb{K}$上的线性空间, 其维数等于$n^2$. 因此对任一$n$阶矩阵$A$, 下列$n^2 + 1$个矩阵必线性相关:
$A^{n^2}, A^{n^2-1}, \cdots, A, I_n$.

也就是说, 存在$\mathbb{K}$中不全为零的数$c_i (i = 0, 1, 2, \cdots, c_{n^2})$, 使
\begin{align*}
c_{n^2}A^{n^2} + c_{n^2-1}A^{n^2-1} + \cdots + c_1A + c_0I_n = O.
\end{align*}
这表明矩阵$A$适合数域$\mathbb{K}$上的一个非零多项式.
\end{proof}

\begin{definition}[矩阵的极小多项式]
若$n$阶矩阵$A$ (或$n$维线性空间$V$上的线性变换$\varphi$) 适合一个非零首一多项式$m(x)$, 且$m(x)$是$A$ (或$\varphi$) 所适合的非零多项式中次数最小者, 则称$m(x)$是$A$ (或$\varphi$) 的一个\textbf{极小多项式}或\textbf{最小多项式}.
\end{definition}
\begin{remark}
由\hyperref[proposition:矩阵一定适合一个非零多项式]{命题\ref{proposition:矩阵一定适合一个非零多项式}}可知矩阵$A$的极小多项式$m(x)$一定存在,故极小多项式是良定义的.
\end{remark}

\begin{theorem}[Cayley-Hamilton定理]\label{theorem:Cayley-Hamilton定理}
\begin{enumerate}
\item \textbf{代数形式:}设$A$是数域$\mathbb{K}$上的$n$阶矩阵, $f(x)$是$A$的特征多项式, 则$f(A) = O$.

\item \textbf{几何形式:}设$\varphi$是$n$维线性空间$V$上的线性变换, $f(x)$是$\varphi$的特征多项式, 则$f(\varphi) = O$.
\end{enumerate}
\end{theorem}
\begin{proof}
\begin{enumerate}
\item {\heiti 代数形式:}因为复数域是最大数域,所以可将$A$看作一个复矩阵.由\hyperref[theorem:复方阵必相似于上三角阵]{复方阵必相似于上三角阵}知$A$复相似于一个上三角阵, 也就是说存在的可逆矩阵$P$, 使$P^{-1}AP = B$是一个上三角阵, 其中$P$与$B$都是复矩阵,由\hyperref[theorem:相似矩阵有相同的特征多项式与特征值]{相似矩阵有相同特征多项式}可知$A$与$B$有相同的特征多项式$f(x)$. 记
\begin{align*}
f(x) = x^n + a_1 x^{n-1} + \cdots + a_n,
\end{align*}
则$f(B) = O$. 而
\begin{align*}
f(A) &= A^n + a_1 A^{n-1} + \cdots + a_n I_n \\
&= (PBP^{-1})^n + a_1 (PBP^{-1})^{n-1} + \cdots + a_n I_n \\
&= PB^n P^{-1} + a_1 PB^{n-1} P^{-1} + \cdots + a_n I_n \\
&= P(B^n + a_1 B^{n-1} + \cdots + a_n I_n)P^{-1} \\
&= P f(B) P^{-1} = O.
\end{align*}

\item {\heiti 几何形式:}设 $\{ e_1,e_2,\cdots ,e_n \}$ 是 $V$ 的一组标准基,$\varphi$ 在这组基下的矩阵为 $A$,  
则由 $f(x)$ 是 $\varphi$ 的特征多项式可知,$f(x)$ 也是 $A$ 的特征多项式。  
从而由代数形式的结论可知 $f(A) = 0$。  
于是对 $\forall \alpha \in V$,都存在 $k_1,k_2,\cdots ,k_n$,使得  
\begin{align*}
\alpha = k_1e_1+k_2e_2+\cdots +k_ne_n 
= \left( e_1,e_2,\cdots ,e_n \right) \begin{pmatrix}
k_1\\
k_2\\
\vdots\\
k_n
\end{pmatrix}.
\end{align*}
两边同时作用 $\varphi$ 得到  
\begin{align*}
&\varphi \left( \alpha \right) =k_1\varphi \left( e_1 \right) +k_2\varphi \left( e_2 \right) +\cdots +k_n\varphi \left( e_n \right) =\left( \varphi \left( e_1 \right) ,\varphi \left( e_2 \right) ,\cdots ,\varphi \left( e_n \right) \right) \left( \begin{array}{c}
k_1\\
k_2\\
\vdots\\
k_n\\
\end{array} \right) 
\\
&=\left( e_1,e_2,\cdots ,e_n \right) A\left( \begin{array}{c}
k_1\\
k_2\\
\vdots\\
k_n\\
\end{array} \right) =A\left( \begin{array}{c}
k_1\\
k_2\\
\vdots\\
k_n\\
\end{array} \right) =A\left( e_1,e_2,\cdots ,e_n \right) \left( \begin{array}{c}
k_1\\
k_2\\
\vdots\\
k_n\\
\end{array} \right) =A\alpha .
\end{align*}
因此 $f(\varphi)(\alpha) = f(A)(\alpha) = 0$。故由 $\alpha$ 的任意性可知 $f(\varphi) = O$.
\end{enumerate}
\end{proof}

\subsection{极小多项式的性质}

\begin{proposition}[极小多项式的性质]\label{proposition:极小多项式的性质}
\begin{enumerate}[(1)]
\item 若$f(x)$是$A$适合的一个多项式, 则$A$的极小多项式$m(x)$整除$f(x)$.

\item 任一$n$阶矩阵的极小多项式必唯一.

\item 相似的矩阵具有相同的极小多项式.

\item 矩阵及其转置有相同的极小多项式.

\item 设$m(x)$是$n$阶矩阵$A$的极小多项式, $\lambda_0$是$A$的特征值, 则$(x - \lambda_0) \mid m(x)$.

\item 设$A$是一个分块对角阵
\begin{align*}
A = \begin{pmatrix}
A_1 & & \\
& A_2 & \\
& & \ddots & \\
& & & A_k
\end{pmatrix},
\end{align*}
其中$A_i$都是方阵, 则$A$的极小多项式等于诸$A_i$的极小多项式之最小公倍式.
\end{enumerate}
\end{proposition}
\begin{note}
性质(5)告诉我们:\textbf{矩阵的特征值一定是其极小多项式的根}.
\end{note}
\begin{proof}
\begin{enumerate}[(1)]
\item 由多项式的带余除法知道
\begin{align*}
f(x) = m(x)q(x) + r(x),
\end{align*}
且$\deg r(x) < \deg m(x)$. 将$x = A$代入上式得$r(A) = O$, 若$r(x) \neq 0$, 则$A$适合一个比$m(x)$次数更小的非零多项式, 矛盾. 故$r(x) = 0$, 即$m(x) \mid f(x)$.

\item 若$m(x), g(x)$都是矩阵$A$的极小多项式, 则由\hyperref[proposition:极小多项式的性质]{矩阵极小多项式的性质(1)}知道$m(x)$能够整除$g(x), g(x)$也能够整除$m(x)$. 因此$m(x)$与$g(x)$只差一个常数因子, 又极小多项式必须首项系数为1, 故$g(x) = m(x)$.

\item 设矩阵$A$和$B$相似, 即存在可逆矩阵$P$, 使$B = P^{-1}AP$. 设$A, B$的极小多项式分别为$m(x), g(x)$, 注意到
\begin{align*}
m(B) = m(P^{-1}AP) = P^{-1}m(A)P = O,
\end{align*}
因此$g(x) \mid m(x)$. 同理, $m(x) \mid g(x)$, 故$m(x) = g(x)$.

\item 设\(\boldsymbol{A}\)的极小多项式是\(m(x)\), 转置\(\boldsymbol{A}'\)的极小多项式是\(n(x)\). 将\(m(\boldsymbol{A}) = \boldsymbol{O}\)转置可得\(m(\boldsymbol{A}') = \boldsymbol{O}\), 因此\(n(x)\mid m(x)\). 同理可证\(m(x)\mid n(x)\), 故\(m(x) = n(x)\).

\item 由$m(A) = O$及
\hyperref[proposition:矩阵适合的多项式其特征值也适合]{命题\ref{proposition:矩阵适合的多项式其特征值也适合}}可得$m(\lambda_0) = 0$, 再由\hyperref[theorem:余数定理]{余数定理}得$(x - \lambda_0) \mid m(x)$.

\item 设$A$的极小多项式为$m(x)$, $A_i$的极小多项式为$m_i(x)$, 诸$m_i(x)$的最小公倍式为$g(x)$, 则$g(A_i) = O$, 于是
\begin{align*}
g(A) = \begin{pmatrix}
g(A_1) & & \\
& g(A_2) & \\
& & \ddots & \\
& & & g(A_k)
\end{pmatrix} = O,
\end{align*}
从而$m(x) \mid g(x)$. 又因为
\begin{align*}
m(A) = \begin{pmatrix}
m(A_1) & & \\
& m(A_2) & \\
& & \ddots & \\
& & & m(A_k)
\end{pmatrix} = O,
\end{align*}
所以对每个$i$有$m(A_i) = O$, 从而$m_i(x) \mid m(x)$,即$m(x)$是$m_i(x)$的公倍式.又$g(x)$是诸$m_i(x)$的最小公倍式, 故$g(x) \mid m(x)$.综上所述,$m(x) = g(x)$.
\end{enumerate}
\end{proof}

\begin{proposition}\label{proposition:F[A]的维数等于A的极小多项式的次数}
设数域\(\mathbb{F}\)上的\(n\)阶矩阵\(\boldsymbol{A}\)的极小多项式为\(m(x)\), 求证: \(\mathbb{F}[\boldsymbol{A}] = \{f(\boldsymbol{A})|f(x)\in\mathbb{F}[x]\}\)是\(M_n(\mathbb{F})\)的子空间, 且\(\dim\mathbb{F}[\boldsymbol{A}] = \deg m(x)\).
\end{proposition}
\begin{proof}
容易验证\(\mathbb{F}[\boldsymbol{A}]\)在矩阵的加法和数乘下封闭, 从而是\(M_n(\mathbb{F})\)的子空间. 对任一\(f(x)\in\mathbb{F}[x]\),由多项式的带余除法可知,存在$q(x),r(x)\in \mathbb{F}[x]$,使得\(f(x) = m(x)q(x) + r(x)\), 其中\(\deg r(x)<\deg m(x) = d\), 于是\(f(\boldsymbol{A}) = m(\boldsymbol{A})q(\boldsymbol{A}) + r(\boldsymbol{A}) = r(\boldsymbol{A})\)是\(\boldsymbol{I}_n,\boldsymbol{A},\cdots,\boldsymbol{A}^{d - 1}\)的线性组合. 另一方面, 若设\(c_0,c_1,\cdots,c_{d - 1}\in\mathbb{F}\), 使得
\begin{align*}
c_0\boldsymbol{I}_n + c_1\boldsymbol{A}+\cdots + c_{d - 1}\boldsymbol{A}^{d - 1} = \boldsymbol{O},
\end{align*}
则\(\boldsymbol{A}\)适合多项式\(g(x) = c_{d - 1}x^{d - 1}+\cdots + c_1x + c_0\), 由\hyperref[proposition:极小多项式的性质]{矩阵极小多项式的性质(1)}可知$m(x)\mid g(x)$,又因为\(d-1=\deg g(x)<\deg m(x) = d\),所以\(g(x) = 0\), 即\(c_0 = c_1 = \cdots = c_{d - 1} = 0\), 于是\(\boldsymbol{I}_n,\boldsymbol{A},\cdots,\boldsymbol{A}^{d - 1}\)在\(\mathbb{F}\)上线性无关. 因此, \(\{\boldsymbol{I}_n,\boldsymbol{A},\cdots,\boldsymbol{A}^{d - 1}\}\)是\(\mathbb{F}[\boldsymbol{A}]\)的一组基, 特别地, \(\dim\mathbb{F}[\boldsymbol{A}] = d = \deg m(x)\). 
\end{proof}

\begin{proposition}\label{proposition:矩阵的极小多项式是其特征多项式的因式}
$n$阶矩阵$A$的极小多项式是其特征多项式的因式. 特别, $A$的极小多项式的次数不超过$n$.
\end{proposition}
\begin{proof}
设$A$的极小多项式和特征多项式分别为$m(x)$和$f(x)$,则
由\hyperref[theorem:Cayley-Hamilton定理]{Cayley-Hamilton定理}可知$f(A)=O$,于是再
由\hyperref[proposition:极小多项式的性质]{矩阵极小多项式的基本性质(1)}可知$m(x)\mid f(x)$.又因为特征多项式$f(x)$一定不是零多项式,所以$\deg m(x)\leq deg f(x)=n$.
\end{proof}

\begin{corollary}\label{corollary:极小多项式和特征多项式有相同的根(不计重数)}
$n$阶矩阵$A$的极小多项式和特征多项式有相同的根(不计重数).
\end{corollary}
\begin{proof}
设\(m(x)\)和\(f(x)\)分别是\(n\)阶矩阵\(\boldsymbol{A}\)的极小多项式和特征多项式,
由\hyperref[proposition:极小多项式的性质]{极小多项式的性质(5)}可知, \(f(x)\)的根 (即特征值) 都是\(m(x)\)的根. 又由\hyperref[proposition:矩阵的极小多项式是其特征多项式的因式]{推论\ref{proposition:矩阵的极小多项式是其特征多项式的因式}}可知, \(m(x)\mid f(x)\), 从而\(m(x)\)的根也都是\(f(x)\)的根. 因此若不计重数, \(m(x)\)和\(f(x)\)有相同的根. 
\end{proof}

\begin{example}
设\(m(x)\)和\(f(x)\)分别是\(n\)阶矩阵\(\boldsymbol{A}\)的极小多项式和特征多项式, 求证: \(f(x)\mid m(x)^n\).
\end{example}
\begin{proof}
由于\(n\)阶矩阵\(\boldsymbol{A}\)的特征值最多是\(n\)重的,因此设\(n\)阶矩阵\(\boldsymbol{A}\)的特征值为$x_i(1\leq i\leq n)$,即$f(x)$为$x_i(1\leq i\leq n)$,并且
\begin{align*}
f(x)=(x-x_1)(x-x_2)\cdots(x-x_n).
\end{align*}
又由\hyperref[corollary:极小多项式和特征多项式有相同的根(不计重数)]{推论\ref{corollary:极小多项式和特征多项式有相同的根(不计重数)}}可知$x_i(1\leq i\leq n)$也都是$m(x)$的根.从而由\hyperref[theorem:余数定理]{余数定理}可知$(x-x_i)\mid m(x),i=1,2,\cdots,n$.于是由\hyperref[proposition:整除的基本性质]{整除的基本性质(6)}归纳可得
\begin{align*}
(x-x_1)(x-x_2)\cdots(x-x_n)\mid m^n(x).
\end{align*}
即$f(x)\mid m^n(x)$.
\end{proof}

\begin{proposition}[常见矩阵的极小多项式]\label{proposition:常见矩阵的极小多项式}
\begin{enumerate}[(1)]
\item 若\(n\)阶矩阵\(\boldsymbol{A}\)有\(n\)个不同的特征值, 则极小多项式等于特征多项式.特别地,\(n\)阶基础循环矩阵的极小多项式等于\(x^n - 1\).

\item 设\(n\)阶矩阵\(\boldsymbol{A}\)可对角化, \(\lambda_1,\lambda_2,\cdots,\lambda_k\)是\(\boldsymbol{A}\)的全体不同的特征值, 则\(\boldsymbol{A}\)的极小多项式为$(x - \lambda_1)(x - \lambda_2)\cdots(x - \lambda_k)$. 

\item \(n\)阶幂零Jordan块的极小多项式是\(x^n\).

\item 设\(n(n > 1)\)阶矩阵\(\boldsymbol{A}\)的秩为\(1\), 求证: \(\boldsymbol{A}\)的极小多项式为\(x^2-\mathrm{tr}(\boldsymbol{A})x\).
\end{enumerate}
\end{proposition}
\begin{proof}
\begin{enumerate}[(1)]
\item 设$A$的极小多项式和特征多项式分别为$m(x)$和$f(x)$,$A$的$n$个不同的特征值为$\lambda_i(1\leq i\leq n)$,则$f(x)=(x-\lambda_1)\cdots(x-\lambda_n)$.
由\hyperref[corollary:极小多项式和特征多项式有相同的根(不计重数)]{推论\ref{corollary:极小多项式和特征多项式有相同的根(不计重数)}}可知,$\lambda_i(1\leq i\leq n)$也是$m(x)$的根.从而
\begin{align*}
(x-\lambda_1)\cdots(x-\lambda_n)\mid m(x).
\end{align*}
即$f(x)\mid m(x)$,又由\hyperref[proposition:矩阵的极小多项式是其特征多项式的因式]{推论\ref{proposition:矩阵的极小多项式是其特征多项式的因式}}可知$m(x)\mid f(x)$,故$m(x)=f(x)$.

\item 设\(\boldsymbol{A}\)的极小多项式为\(m(x)\). 由\(\boldsymbol{A}\)可对角化知存在可逆矩阵\(\boldsymbol{P}\), 使得
\begin{align*}
\boldsymbol{P}^{-1}\boldsymbol{A}\boldsymbol{P} = \boldsymbol{B} = \mathrm{diag}\{\boldsymbol{B}_1,\boldsymbol{B}_2,\cdots,\boldsymbol{B}_k\},
\end{align*}
其中\(\boldsymbol{B}_i = \lambda_i\boldsymbol{I}\)为纯量矩阵. 显然\(\boldsymbol{B}_i\)的极小多项式为\(x - \lambda_i\), 故由\hyperref[proposition:极小多项式的性质]{极小多项式的性质(3)和(6)}可得
\begin{align*}
m(x) = m(\boldsymbol{B})= [x - \lambda_1,x - \lambda_2,\cdots,x - \lambda_k] = (x - \lambda_1)(x - \lambda_2)\cdots(x - \lambda_k). 
\end{align*} 

\item 设\(n\)阶幂零Jordan块为$A$,则由\hyperref[proposition:幂零Jordan块]{命题\ref{proposition:幂零Jordan块}}可知$A^k\ne O(k=1,2,\cdots,n-1)$,但$A^n=O$.故\(n\)阶幂零Jordan块$A$的极小多项式为$x^n$.

\item 由\hyperref[proposition:矩阵可对角化迹必不为零]{命题\ref{proposition:矩阵可对角化迹必不为零}证法二}可知, \(\boldsymbol{A}\)适合多项式\(x^2 - \mathrm{tr}(\boldsymbol{A})x\). 显然\(\boldsymbol{A}\)不可能适合多项式\(x\). 若\(\boldsymbol{A}\)适合多项式\(x - \mathrm{tr}(\boldsymbol{A})\), 则\(\boldsymbol{A} = \mathrm{tr}(\boldsymbol{A})\boldsymbol{I}_n\)为纯量矩阵, 其秩等于\(0\)或\(n\), 这与\(\mathrm{r}(\boldsymbol{A}) = 1\)矛盾. 因此, \(\boldsymbol{A}\)的极小多项式为\(x^2 - \mathrm{tr}(\boldsymbol{A})x\). 
\end{enumerate}
\end{proof}

\begin{proposition}\label{proposition:两个没有公共特征值的矩阵的极小多项式与特征多项式的关系}
设\(f(x)\)和\(m(x)\)分别是\(m\)阶矩阵\(\boldsymbol{A}\)的特征多项式和极小多项式, \(g(x)\)和\(n(x)\)分别是\(n\)阶矩阵\(\boldsymbol{B}\)的特征多项式和极小多项式, 证明以下结论等价:
\begin{enumerate}[(1)]
\item \(\boldsymbol{A},\boldsymbol{B}\)没有公共的特征值;

\item \((f(x),g(x)) = 1\)或\((f(x),n(x)) = 1\)或\((m(x),g(x)) = 1\)或\((m(x),n(x)) = 1\);

\item \(f(\boldsymbol{B})\)或\(m(\boldsymbol{B})\)或\(g(\boldsymbol{A})\)或\(n(\boldsymbol{A})\)是可逆矩阵.
\end{enumerate}
\end{proposition}
\begin{proof}
(1) \(\Leftrightarrow\) (2): 由\hyperref[corollary:极小多项式和特征多项式有相同的根(不计重数)]{推论\ref{corollary:极小多项式和特征多项式有相同的根(不计重数)}}可知, (2) 中所有的条件都等价. 显然 (1) 与\((f(x),g(x)) = 1\)等价, 故 (1) 与 (2) 等价.

(2) \(\Rightarrow\) (3): 例如, 若\((f(x),n(x)) = 1\), 则存在\(u(x),v(x)\), 使得\(f(x)u(x) + n(x)v(x) = 1\). 将\(x = \boldsymbol{B}\)代入上式并注意到\(n(\boldsymbol{B}) = \boldsymbol{O}\), 故可得\(f(\boldsymbol{B})u(\boldsymbol{B}) = \boldsymbol{I}_n\), 这表明\(f(\boldsymbol{B})\)是可逆矩阵. 将\(x = \boldsymbol{A}\)代入上式并注意到\(f(\boldsymbol{A}) = \boldsymbol{O}\)(\hyperref[theorem:Cayley-Hamilton定理]{Cayley-Hamilton定理}), 故可得\(n(\boldsymbol{A})v(\boldsymbol{A}) = \boldsymbol{I}_n\), 这表明\(n(\boldsymbol{A})\)是可逆矩阵. 同理可证其他的情形.

(3) \(\Rightarrow\) (1): 设\(\lambda_1,\cdots,\lambda_m\)是\(\boldsymbol{A}\)的特征值, 则\(n(\lambda_1),\cdots,n(\lambda_m)\)是\(n(\boldsymbol{A})\)的特征值. 例如, 若\(n(\boldsymbol{A})\)是可逆矩阵, 则\(n(\lambda_i)\neq 0\),即$\lambda_i$都不是$n(x)$的根.由\hyperref[corollary:极小多项式和特征多项式有相同的根(不计重数)]{推论\ref{corollary:极小多项式和特征多项式有相同的根(不计重数)}}可知,$\lambda_i$都不是$g(x)$的根,即\(\lambda_1,\cdots,\lambda_m\)都不是\(\boldsymbol{B}\)的特征值, 从而\(\boldsymbol{A},\boldsymbol{B}\)没有公共的特征值. 同理可证其他的情形.
\end{proof}

\begin{proposition}\label{proposition:g(A)可逆与A的特征多项式与极小多项式的关系}
设 \(f(x)\) 和 \(m(x)\) 分别是 \(n\) 阶矩阵 \(\boldsymbol{A}\) 的特征多项式和极小多项式,\(g(x)\) 是一个多项式,求证:\(g(\boldsymbol{A})\) 是可逆矩阵的充要条件是 \((f(x),g(x)) = 1\) 或 \((m(x),g(x)) = 1\).
\end{proposition}
\begin{proof}
先证充分性,若 \((f(x),g(x)) = 1\),则存在多项式 \(u(x),v(x)\),使得
\begin{align*}
u(x)f(x)+v(x)g(x)&=1.
\end{align*}
又由\hyperref[theorem:Cayley-Hamilton定理]{Cayley-Hamilton定理}可知,\(f(\boldsymbol{A})=\boldsymbol{O}\)。从而将 \(x = \boldsymbol{A}\) 代入上式得 \(v(\boldsymbol{A})g(\boldsymbol{A})=\boldsymbol{I}_n\),故 \(g(\boldsymbol{A})\) 可逆。

若 \((m(x),g(x)) = 1\),则存在多项式 \(u(x),v(x)\),使得
\begin{align*}
u(x)m(x)+v(x)g(x)&=1.
\end{align*}
又注意到 \(m(\boldsymbol{A})=\boldsymbol{O}\)。从而将 \(x = \boldsymbol{A}\) 代入上式得 \(v(\boldsymbol{A})g(\boldsymbol{A})=\boldsymbol{I}_n\),故 \(g(\boldsymbol{A})\) 可逆。

再证必要性,设 \(\lambda_1,\lambda_2,\cdots,\lambda_m\) 为 \(\boldsymbol{A}\) 的所有特征值,则 \(g(\lambda_1),g(\lambda_2),\cdots,g(\lambda_m)\) 为 \(g(\boldsymbol{A})\) 的所有特征值。又因为 \(g(\boldsymbol{A})\) 可逆,所以其特征值 \(g(\lambda_i)\neq 0\)(\(i = 1,2,\cdots,m\)),即 \(\lambda_i\) 都不是 \(g(x)\) 的根。而由\hyperref[corollary:极小多项式和特征多项式有相同的根(不计重数)]{推论\ref{corollary:极小多项式和特征多项式有相同的根(不计重数)}}可知,\(\lambda_i\) 是 \(f(x),m(x)\) 的全部根。因此 \(f(x),m(x)\) 与 \(g(x)\) 没有公共根,故 \((f(x),g(x)) = 1\),\((m(x),g(x)) = 1\)。 
\end{proof}

\begin{proposition}\label{proposition:矩阵可逆充要条件极小多项式常数项非零}
证明:\(n\) 阶方阵 \(\boldsymbol{A}\) 为可逆矩阵的充要条件是 \(\boldsymbol{A}\) 的极小多项式的常数项不为零。
\end{proposition}
\begin{note}
也可利用\hyperref[corollary:极小多项式和特征多项式有相同的根(不计重数)]{推论\ref{corollary:极小多项式和特征多项式有相同的根(不计重数)}}和\hyperref[theorem:Vieta定理]{Vieta定理}来证明.
\end{note}
\begin{proof}
设 \(f(x)\) 和 \(m(x)\) 分别是 \(\boldsymbol{A}\) 的特征多项式和极小多项式,则 \(m(x)\mid f(x)\)。
若 \(\boldsymbol{A}\) 可逆,则 \(f(x)\) 的常数项 \((-1)^n|\boldsymbol{A}|\) 不等于零,因此 \(m(x)\) 的常数项也不为零。

反之,设 \(m(x)=x^m + b_{m - 1}x^{m - 1}+\cdots + b_0\),其中 \(b_0\neq 0\),则
\begin{align*}
m(\boldsymbol{A})&=\boldsymbol{A}^m + b_{m - 1}\boldsymbol{A}^{m - 1}+\cdots + b_0\boldsymbol{I}_n=\boldsymbol{O},
\end{align*}
于是
\begin{align*}
\boldsymbol{A}(\boldsymbol{A}^{m - 1}+b_{m - 1}\boldsymbol{A}^{m - 2}+\cdots + b_1\boldsymbol{I}_n)&=-b_0\boldsymbol{I}_n.
\end{align*}
由 \(b_0\neq 0\) 即知 \(\boldsymbol{A}\) 可逆。
\end{proof}

\subsection{Cayley-Hamilton定理的应用:逆矩阵和伴随矩阵的多项式表示}

\begin{proposition}\label{proposition:矩阵的逆可以用其多项式表示}
设 \(\boldsymbol{A}\) 是 \(n\) 阶可逆矩阵,求证:\(\boldsymbol{A}^{-1}=g(\boldsymbol{A})\),其中 \(g(x)\) 是一个 \(n - 1\) 次多项式。
\end{proposition}
\begin{proof}
设 \(f(x)=x^n + a_1x^{n - 1}+\cdots + a_{n - 1}x + a_n\) 是 \(\boldsymbol{A}\) 的特征多项式,因为 \(\boldsymbol{A}\) 可逆,故 \(a_n = (-1)^n|\boldsymbol{A}|\neq 0\)。由\hyperref[theorem:Cayley-Hamilton定理]{Cayley-Hamilton定理}可得 \(f(\boldsymbol{A})=\boldsymbol{O}\),于是
\begin{align*}
\boldsymbol{A}\left(-\frac{1}{a_n}(\boldsymbol{A}^{n - 1}+a_1\boldsymbol{A}^{n - 2}+\cdots + a_{n - 1}\boldsymbol{I}_n)\right)&=\boldsymbol{I}_n.
\end{align*}
因此
\begin{align*}
\boldsymbol{A}^{-1}&=-\frac{1}{a_n}(\boldsymbol{A}^{n - 1}+a_1\boldsymbol{A}^{n - 2}+\cdots + a_{n - 1}\boldsymbol{I}_n).
\end{align*} 
\end{proof}

\begin{proposition}\label{proposition:伴随矩阵可原矩阵的多项式表示}
设 \(\boldsymbol{A}\) 是 \(n\) 阶矩阵,求证:伴随矩阵 \(\boldsymbol{A}^* = h(\boldsymbol{A})\),其中 \(h(x)\) 是一个 \(n - 1\) 次多项式。
\end{proposition}
\begin{proof}
我们用摄动法来证明结论。设 \(f(x)=x^n + a_1x^{n - 1}+\cdots + a_{n - 1}x + a_n\) 是 \(\boldsymbol{A}\) 的特征多项式,其中 \(a_n = (-1)^n|\boldsymbol{A}|\)。若 \(\boldsymbol{A}\) 是可逆矩阵,则由\hyperref[proposition:矩阵的逆可以用其多项式表示]{命题\ref{proposition:矩阵的逆可以用其多项式表示}}可得
\begin{align*}
\boldsymbol{A}^*&=|\boldsymbol{A}|\boldsymbol{A}^{-1}=(-1)^{n - 1}(\boldsymbol{A}^{n - 1}+a_1\boldsymbol{A}^{n - 2}+\cdots + a_{n - 1}\boldsymbol{I}_n).
\end{align*}
令 \(h(x)=(-1)^{n - 1}(x^{n - 1}+a_1x^{n - 2}+\cdots + a_{n - 1})\),则 \(\boldsymbol{A}^* = h(\boldsymbol{A})\),并且 \(h(x)\) 的系数由特征多项式 \(f(x)\) 的系数唯一确定。

对于一般的方阵 \(\boldsymbol{A}\),可取到一列有理数 \(t_k\rightarrow 0\),使得 \(t_k\boldsymbol{I}_n+\boldsymbol{A}\) 为可逆矩阵。设
\begin{align*}
f_{t_k}(x)&=|x\boldsymbol{I}_n-(t_k\boldsymbol{I}_n+\boldsymbol{A})|=x^n + a_1(t_k)x^{n - 1}+\cdots + a_{n - 1}(t_k)x + a_n(t_k)
\end{align*}
为 \(t_k\boldsymbol{I}_n+\boldsymbol{A}\) 的特征多项式,则 \(a_i(t_k)\) 都是 \(t_k\) 的多项式且 \(a_i(0)=a_i\)(\(1\leq i\leq n\))。由可逆矩阵情形的证明可得
\begin{align*}
(t_k\boldsymbol{I}_n+\boldsymbol{A})^*&=(-1)^{n - 1}\left((t_k\boldsymbol{I}_n+\boldsymbol{A})^{n - 1}+a_1(t_k)(t_k\boldsymbol{I}_n+\boldsymbol{A})^{n - 2}+\cdots + a_{n - 1}(t_k)\boldsymbol{I}_n\right).
\end{align*}
注意到上式两边的矩阵中的元素都是 \(t_k\) 的多项式,从而关于 \(t_k\) 连续。上式两边同时取极限,令 \(t_k\rightarrow 0\),即得
\begin{align*}
\boldsymbol{A}^*&=(-1)^{n - 1}(\boldsymbol{A}^{n - 1}+a_1\boldsymbol{A}^{n - 2}+\cdots + a_{n - 1}\boldsymbol{I}_n).
\end{align*}
因此无论 \(\boldsymbol{A}\) 是否可逆,我们都有 \(\boldsymbol{A}^* = h(\boldsymbol{A})\) 成立.
\end{proof}

\subsection{Cayley - Hamilton 定理的应用:AX = XB型矩阵方程的求解及其应用}

\begin{proposition}\label{AX=XB相关命题1}
设 \(\boldsymbol{A}\) 为 \(m\) 阶矩阵,\(\boldsymbol{B}\) 为 \(n\) 阶矩阵,求证:若 \(\boldsymbol{A}\),\(\boldsymbol{B}\) 没有公共的特征值,则矩阵方程 \(\boldsymbol{AX = XB}\) 只有零解 \(\boldsymbol{X = O}\)。 
\end{proposition}
\begin{proof}
{\color{blue}证法一:}
设 \(f(\lambda)=|\lambda\boldsymbol{I}_m - \boldsymbol{A}|\) 为 \(\boldsymbol{A}\) 的特征多项式,则由\hyperref[theorem:Cayley-Hamilton定理]{Cayley-Hamilton定理}可知 \(f(\boldsymbol{A})=\boldsymbol{O}\),再由 \(\boldsymbol{AX = XB}\) 可得
\begin{align*}
\boldsymbol{O}&=f(\boldsymbol{A})\boldsymbol{X}=\boldsymbol{X}f(\boldsymbol{B}).
\end{align*}
因为 \(\boldsymbol{A}\),\(\boldsymbol{B}\) 没有公共的特征值,故由\hyperref[proposition:两个没有公共特征值的矩阵的极小多项式与特征多项式的关系]{命题\ref{proposition:两个没有公共特征值的矩阵的极小多项式与特征多项式的关系}}可知,\(f(\boldsymbol{B})\) 是可逆矩阵,从而由上式即得 \(\boldsymbol{X = O}\).

{\color{blue}证法二:}
任取矩阵方程的一个解 \(\boldsymbol{X = C}\),若 \(\boldsymbol{C\neq O}\),则 \(\mathrm{r}(\boldsymbol{C}) = r\geq 1\)。由\hyperref[example-6.12]{例题\ref{example-6.12}}可知,\(\boldsymbol{A}\),\(\boldsymbol{B}\) 至少有 \(r\) 个相同的特征值,这与 \(\boldsymbol{A}\),\(\boldsymbol{B}\) 没有公共的特征值相矛盾。因此 \(\boldsymbol{C = O}\),即矩阵方程只有零解。\(\square\) 
\end{proof}

\begin{example}
设 \(n\) 阶方阵 \(\boldsymbol{A}\),\(\boldsymbol{B}\) 的特征值全部大于零且满足 \(\boldsymbol{A}^2 = \boldsymbol{B}^2\),求证:\(\boldsymbol{A} = \boldsymbol{B}\)。
\end{example}
\begin{proof}
由 \(\boldsymbol{A}^2 = \boldsymbol{B}^2\) 可得 \(\boldsymbol{A}(\boldsymbol{A - B}) = (\boldsymbol{A - B})(-\boldsymbol{B})\),即 \(\boldsymbol{A - B}\) 是矩阵方程 \(\boldsymbol{AX = X(-B)}\) 的解。注意到 \(\boldsymbol{A}\) 的特征值全部大于零,\(-\boldsymbol{B}\) 的特征值全部小于零,故它们没有公共的特征值,由\hyperref[AX=XB相关命题1]{命题\ref{AX=XB相关命题1}}可得 \(\boldsymbol{A - B} = \boldsymbol{O}\),即 \(\boldsymbol{A} = \boldsymbol{B}\)。
\end{proof}

\begin{example}
设 \(\boldsymbol{A} = \mathrm{diag}\{\boldsymbol{A}_1,\boldsymbol{A}_2,\cdots,\boldsymbol{A}_m\}\) 为 \(n\) 阶分块对角矩阵,其中 \(\boldsymbol{A}_i\) 是 \(n_i\) 阶矩阵且两两没有公共的特征值。设 \(\boldsymbol{B}\) 是 \(n\) 阶矩阵,满足 \(\boldsymbol{AB = BA}\),求证:\(\boldsymbol{B} = \mathrm{diag}\{\boldsymbol{B}_1,\boldsymbol{B}_2,\cdots,\boldsymbol{B}_m\}\),其中 \(\boldsymbol{B}_i\) 也是 \(n_i\) 阶矩阵。
\end{example}
\begin{proof}
按照 \(\boldsymbol{A}\) 的分块方式对 \(\boldsymbol{B}\) 进行分块,可设 \(\boldsymbol{B} = (\boldsymbol{B}_{ij})\),其中 \(\boldsymbol{B}_{ij}\) 是 \(n_i\times n_j\) 矩阵。由 \(\boldsymbol{AB = BA}\) 可知,对任意的 \(i,j\),有 \(\boldsymbol{A}_i\boldsymbol{B}_{ij} = \boldsymbol{B}_{ij}\boldsymbol{A}_j\)。因为 \(\boldsymbol{A}_i\),\(\boldsymbol{A}_j\)(\(i\neq j\))没有公共的特征值,故由\hyperref[AX=XB相关命题1]{命题\ref{AX=XB相关命题1}}可得 \(\boldsymbol{B}_{ij} = \boldsymbol{O}\)(\(i\neq j\)),从而 \(\boldsymbol{B} = \mathrm{diag}\{\boldsymbol{B}_{11},\boldsymbol{B}_{22},\cdots,\boldsymbol{B}_{mm}\}\) 也是分块对角矩阵。
\end{proof}

\begin{proposition}\label{proposition:AX-XB相关命题1}
设 \(\boldsymbol{A}\),\(\boldsymbol{B}\) 分别为 \(m\),\(n\) 阶矩阵,\(V\) 为 \(m\times n\) 矩阵全体构成的线性空间,\(V\) 上的线性变换 \(\varphi\) 定义为:\(\varphi(\boldsymbol{X}) = \boldsymbol{AX - XB}\)。求证:\(\varphi\) 是线性自同构的充要条件是 \(\boldsymbol{A}\),\(\boldsymbol{B}\) 没有公共的特征值。此时,对任一 \(m\times n\) 矩阵 \(\boldsymbol{C}\),矩阵方程 \(\boldsymbol{AX - XB = C}\) 存在唯一解。
\end{proposition}
\begin{proof}
若 \(\boldsymbol{A}\),\(\boldsymbol{B}\) 没有公共的特征值,则由\hyperref[AX=XB相关命题1]{命题\ref{AX=XB相关命题1}}可知,$\varphi(X)=AX-XB=0$只有零解,即$\mathrm{Ker}\varphi= 0$.从而\(\varphi\) 是 \(V\) 上的单映射,从而是线性自同构。若 \(\boldsymbol{A}\),\(\boldsymbol{B}\) 有公共的特征值 \(\lambda_0\),则 \(\lambda_0\) 也是 \(\boldsymbol{B}'\) 的特征值。设 \(\boldsymbol{\alpha}\),\(\boldsymbol{\beta}\) 为对应的特征向量,即 \(\boldsymbol{A\alpha}=\lambda_0\boldsymbol{\alpha}\),\(\boldsymbol{B}'\boldsymbol{\beta}=\lambda_0\boldsymbol{\beta}\),则 \(\boldsymbol{\alpha\beta}'\neq\boldsymbol{O}\) 且
\begin{align*}
\varphi(\boldsymbol{\alpha\beta}')&=(\boldsymbol{A\alpha})\boldsymbol{\beta}' - \boldsymbol{\alpha}(\boldsymbol{B}'\boldsymbol{\beta})'=\lambda_0\boldsymbol{\alpha\beta}' - \lambda_0\boldsymbol{\alpha\beta}'=\boldsymbol{O},
\end{align*}
于是 \(\mathrm{Ker}\varphi\neq 0\),从而 \(\varphi\) 不是线性自同构。
\end{proof}

\begin{example}
设 \(n\) 阶实矩阵 \(\boldsymbol{A}\) 的所有特征值都是正实数,证明:对任一实对称矩阵 \(\boldsymbol{C}\),存在唯一的实对称矩阵 \(\boldsymbol{B}\),满足 \(\boldsymbol{A}'\boldsymbol{B} + \boldsymbol{B}\boldsymbol{A} = \boldsymbol{C}\)。
\end{example}
\begin{proof}
考虑矩阵方程 \(\boldsymbol{A}'\boldsymbol{X} - \boldsymbol{X}(-\boldsymbol{A}) = \boldsymbol{C}\),注意到 \(\boldsymbol{A}'\) 的特征值全部大于零,\(-\boldsymbol{A}\) 的特征值全部小于零,它们没有公共的特征值,故由\hyperref[proposition:AX-XB相关命题1]{命题\ref{proposition:AX-XB相关命题1}}可得上述矩阵方程存在唯一解 \(\boldsymbol{X = B}\)。容易验证 \(\boldsymbol{X = \overline{B}}\),\(\boldsymbol{B}'\) 也都是上述矩阵方程的解,故由解的唯一性可知 \(\boldsymbol{B = \overline{B}}\) 且 \(\boldsymbol{B = B'}\),即 \(\boldsymbol{B}\) 为实对称矩阵,结论得证。
\end{proof}


\subsection{Cayley-Hamilton定理的应用:特征多项式诱导的直和分解}

\begin{example}
设 $\varphi$ 是复线性空间 $V$ 上的线性变换,又有两个复系数多项式:
\[f(x)=x^m + a_1x^{m - 1}+\cdots + a_m, \quad g(x)=x^n + b_1x^{n - 1}+\cdots + b_n.\]
设 $\sigma = f(\varphi),\tau = g(\varphi)$,矩阵 $C$ 是 $f(x)$ 的友阵,即
\begin{align*}
C = \begin{pmatrix}
0 & 0 & 0 & \cdots & -a_m \\
1 & 0 & 0 & \cdots & -a_{m - 1} \\
0 & 1 & 0 & \cdots & -a_{m - 2} \\
\vdots & \vdots & \vdots & & \vdots \\
0 & 0 & 0 & \cdots & -a_1
\end{pmatrix}.
\end{align*}
若 $g(C)$ 是可逆矩阵,求证: $\mathrm{Ker}\,\sigma \tau =\mathrm{Ker}\,\sigma \oplus \mathrm{Ker}\,\tau $. 
\end{example}
\begin{note}
$(f(x),g(x))=1$之后的证明类似\hyperref[proposition:互素多项式诱导直和分解1]{命题\ref{proposition:互素多项式诱导直和分解1}}.
\end{note}
\begin{proof}
由\hyperref[proposition:多项式的友阵的特征多项式与特征值]{命题\ref{proposition:多项式的友阵的特征多项式与特征值}}可知$C$的特征多项式就是$f(x)$.由\hyperref[proposition:g(A)可逆与A的特征多项式与极小多项式的关系]{命题\ref{proposition:g(A)可逆与A的特征多项式与极小多项式的关系}}可知$(f(x),g(x))=1$.
由 $(f(x),g(x)) = 1$ 可知,存在多项式 $u(x),v(x)$,使得
\[u(x)f(x)+v(x)g(x)=1.\]
从而
\begin{align}
u(\varphi)f(\varphi)+v(\varphi)g(\varphi)=I_V. \label{example0.5-1.1}
\end{align}
于是对 $\forall \alpha\in\mathrm{Ker}\,\sigma\tau$,由 
\eqref{example0.5-1.1}式可得
\[\alpha = u(\varphi)f(\varphi)(\alpha)+v(\varphi)g(\varphi)(\alpha).\]
又因为 $\alpha\in\mathrm{Ker}\,\sigma\tau$,所以 $f(\varphi)g(\varphi)(\alpha)=g(\varphi)f(\varphi)(\alpha)=0$ . 因此 $u(\varphi)f(\varphi)(\alpha)\in\mathrm{Ker}\,g(\varphi)$,$v(\varphi)g(\varphi)(\alpha)\in\mathrm{Ker}\,f(\varphi)$ .
故有 $\mathrm{Ker}\,\sigma\tau=\mathrm{Ker}\,\sigma+\mathrm{Ker}\,\tau$ . 任取 $\beta\in\mathrm{Ker}\,\sigma\cap\mathrm{Ker}\,\tau$,则 $\sigma(\beta)=f(\varphi)(\beta)=0$,$\tau(\beta)=g(\varphi)(\beta)=0$ . 由\eqref{example0.5-1.1} 式可得
\[\beta = u(\varphi)f(\varphi)(\beta)+v(\varphi)g(\varphi)(\beta)=0.\]
故 $\mathrm{Ker}\,\sigma\cap\mathrm{Ker}\,\tau = 0$,因此 $\mathrm{Ker}\,\sigma\tau=\mathrm{Ker}\,\sigma\oplus\mathrm{Ker}\,\tau$ . 
\end{proof}

\begin{proposition}\label{proposition:Cayley-Hamilton定理诱导直和分解(互素多项式命题推广)}
设 $\varphi$ 是数域 $\mathbb{K}$ 上 $n$ 维线性空间 $V$ 上的线性变换,其特征多项式是 $f(\lambda)$ 且 $f(\lambda)=f_1(\lambda)f_2(\lambda)$,其中 $f_1(\lambda),f_2(\lambda)$ 是互素的首一多项式. 令 $V_1 = \mathrm{Ker}\,f_1(\varphi)$,$V_2 = \mathrm{Ker}\,f_2(\varphi)$,求证:
\begin{enumerate}[(1)]
\item  $V_1,V_2$ 是 $\varphi$-不变子空间且 $V = V_1\oplus V_2$;

\item $V_1 = \mathrm{Im}\,f_2(\varphi)$,$V_2 = \mathrm{Im}\,f_1(\varphi)$;

\item $\varphi|_{V_1}$ 的特征多项式是 $f_1(\lambda)$,$\varphi|_{V_2}$ 的特征多项式是 $f_2(\lambda)$.
\end{enumerate}
\end{proposition}
\begin{note}
这个命题是\hyperref[proposition:互素多项式诱导直和分解1]{命题\ref{proposition:互素多项式诱导直和分解1}}的推广.

这个命题的结论还可以进一步推广,例如不限定 $f(\lambda)$ 是 $\varphi$ 的特征多项式,而只要求 $\varphi$ 适合它 (比如 $\varphi$ 的极小多项式 $m(\lambda)$),则由完全相同的讨论可以证明此时对这个命题的 (1) 和 (2) 都成立. 特别地,如果考虑极小多项式的首一互素因式分解
\[m(\lambda)=m_1(\lambda)m_2(\lambda), V_1 = \mathrm{Ker}\,m_1(\varphi), V_2 = \mathrm{Ker}\,m_2(\varphi),\]
则由完全类似的讨论可以证明: $\varphi|_{V_i}$ 的极小多项式就是 $m_i(\lambda)$. 
\end{note}
\begin{remark}
(3)中$f_1(\lambda)=g_1(\lambda),f_2(\lambda)=g_2(\lambda)$的原因:由于 $f_i(\lambda)$ 与 $g_i(\lambda)$ 的根相同,且 $f_1(\lambda)$ 与 $f_2(\lambda)$ 没有公共根,因此不妨设
\begin{align}
f_1(\lambda)&=(\lambda - x_1)^{i_1}\cdots(\lambda - x_s)^{i_s}, \quad f_2(\lambda)=(\lambda - y_1)^{j_1}\cdots(\lambda - y_l)^{j_l},\label{0.13-1.1-1}
\\
g_1(\lambda)&=(\lambda - x_1)^{i_{1}^{\prime}}\cdots(\lambda - x_s)^{i_{s}^{\prime}}, \quad g_2(\lambda)=(\lambda - y_1)^{j_{1}^{\prime}}\cdots(\lambda - y_l)^{j_{l}^{\prime}}.\label{0.13-1.1-2}
\end{align}
其中 $x_1,\cdots,x_s,y_1,\cdots,y_l$ 互不相同. 则
\begin{align*}
f_1(\lambda)f_2(\lambda)&=[(\lambda - x_1)^{i_1}\cdots(\lambda - x_s)^{i_s}][(\lambda - y_1)^{j_1}\cdots(\lambda - y_l)^{j_l}],\\
g_1(\lambda)g_2(\lambda)&=[(\lambda - x_1)^{i_{1}^{\prime}}\cdots(\lambda - x_s)^{i_{s}^{\prime}}][(\lambda - y_1)^{j_{1}^{\prime}}\cdots(\lambda - y_l)^{j_{l}^{\prime}}].
\end{align*}
又由 $f(\lambda)=f_1(\lambda)f_2(\lambda)=g_1(\lambda)g_2(\lambda)$ 可得
\begin{align*}
[(\lambda - x_1)^{i_1}\cdots(\lambda - x_s)^{i_s}][(\lambda - y_1)^{j_1}\cdots(\lambda - y_l)^{j_l}]&=[(\lambda - x_1)^{i_{1}^{\prime}}\cdots(\lambda - x_s)^{i_{s}^{\prime}}][(\lambda - y_1)^{j_{1}^{\prime}}\cdots(\lambda - y_l)^{j_{l}^{\prime}}].
\end{align*}
比较上式两边的常数项可得
\begin{align*}
x_1^{i_1}\cdots x_s^{i_s}y_1^{j_1}\cdots y_l^{j_l}&=x_1^{i_{1}^{\prime}}\cdots x_s^{i_{s}^{\prime}}y_1^{j_{1}^{\prime}}\cdots y_l^{j_{l}^{\prime}}.
\end{align*}
又因为 $x_1,\cdots,x_s,y_1,\cdots,y_l$ 互不相同,所以
\[i_1 = i_{1}^{\prime},\cdots,i_s = i_{s}^{\prime},j_1 = j_{1}^{\prime},\cdots,j_l = j_{l}^{\prime}.\]
再由\eqref{0.13-1.1-1}和\eqref{0.13-1.1-2}式可知 $f_1(\lambda)=g_1(\lambda)$,$f_2(\lambda)=g_2(\lambda)$. 
\end{remark}
\begin{proof}
\begin{enumerate}[(1)]
\item  对 $\forall \alpha\in V_1$,都有 $f_1(\varphi)(\alpha)=0$ . 从而
\begin{align*}
f_1(\varphi)(\varphi(\alpha))&=(f_1(\varphi)\varphi)(\alpha)=(\varphi f_1(\varphi))(\alpha)=\varphi(f_1(\varphi)(\alpha))=\varphi(0)=0.
\end{align*}
故 $V_1$ 是 $\varphi$-不变子空间,同理可得 $V_2$ 也是 $\varphi$-不变子空间. 
由 \hyperref[theorem:Cayley-Hamilton定理]{Cayley - Hamilton 定理}可得 $f(\varphi)=f_1(\varphi)f_2(\varphi)=\mathbf{0}$,故由\hyperref[proposition:互素多项式诱导直和分解1]{命题\ref{proposition:互素多项式诱导直和分解1}}可知 $V = V_1\oplus V_2$.

\item 由 $f_1(\varphi)f_2(\varphi)=\mathbf{0}$ 可得 $\mathrm{Im}\,f_2(\varphi)\subseteq\mathrm{Ker}\,f_1(\varphi)=V_1$,$\mathrm{Im}\,f_1(\varphi)\subseteq\mathrm{Ker}\,f_2(\varphi)=V_2$. 因为 $V = V_1\oplus V_2$,故由维数公式可得
\begin{align*}
\dim\mathrm{Im}\,f_2(\varphi)&=\dim V - \dim\mathrm{Ker}\,f_2(\varphi)=\dim V - \dim V_2=\dim V_1,\\
\dim\mathrm{Im}\,f_1(\varphi)&=\dim V - \dim\mathrm{Ker}\,f_1(\varphi)=\dim V - \dim V_1=\dim V_2,
\end{align*}
从而 $V_1 = \mathrm{Im}\,f_2(\varphi)$,$V_2 = \mathrm{Im}\,f_1(\varphi)$.

\item 设 $\varphi|_{V_i}$ 的特征多项式为 $g_i(\lambda) (i = 1,2)$,则由\hyperref[proposition:线性变换不变子空间直和分解与特征值的关系]{命题\ref{proposition:线性变换不变子空间直和分解与特征值的关系}}可得
\begin{align}
f(\lambda)=f_1(\lambda)f_2(\lambda)=g_1(\lambda)g_2(\lambda). \label{example03003-6.8}
\end{align}
注意到 $f_i(\varphi|_{V_i})=f_i(\varphi)|_{V_i}=\mathbf{0}$,即 $\varphi|_{V_i}$ 适合多项式 $f_i(\lambda)$,因此 $\varphi|_{V_i}$ 的特征值也适合 $f_i(\lambda)$,即 $g_i(\lambda)$ 的根都是 $f_i(\lambda)$ 的根. 因为 $(f_1(\lambda),f_2(\lambda)) = 1$,故 $f_1(\lambda)$ 与 $f_2(\lambda)$ 没有公共根,从而由 $f_i(\lambda)$ 的首一性和\eqref{example03003-6.8} 式即得 $f_1(\lambda)=g_1(\lambda)$,$f_2(\lambda)=g_2(\lambda)$. 
\end{enumerate}
\end{proof}



\subsection{Cayley-Hamilton定理的其他应用}

\begin{example}
设 $\boldsymbol{A}$ 为 $n$ 阶矩阵,$\boldsymbol{C}$ 为 $k\times n$ 矩阵,且对任意的 $\lambda\in\mathbb{C}$,$\begin{pmatrix}
\boldsymbol{A}-\lambda\boldsymbol{I}_n \\
\boldsymbol{C}
\end{pmatrix}$ 均为列满秩阵. 证明: 对任意的 $\lambda\in\mathbb{C}$,$\begin{pmatrix}
\boldsymbol{C} \\
\boldsymbol{C}(\boldsymbol{A}-\lambda\boldsymbol{I}_n) \\
\boldsymbol{C}(\boldsymbol{A}-\lambda\boldsymbol{I}_n)^2 \\
\vdots \\
\boldsymbol{C}(\boldsymbol{A}-\lambda\boldsymbol{I}_n)^{n - 1}
\end{pmatrix}$ 均为列满秩阵.
\end{example}
\begin{proof}
由\hyperref[corollary:线性方程组只有零解的充要条件]{推论\ref{corollary:线性方程组只有零解的充要条件}}可知,对任意的 $\lambda\in\mathbb{C}$,下列线性方程组只有零解:
\begin{align}
\begin{cases}
(\boldsymbol{A}-\lambda\boldsymbol{I}_n)\boldsymbol{x}=\boldsymbol{0}, \\
\boldsymbol{C}\boldsymbol{x}=\boldsymbol{0}.
\end{cases} \label{example0.6-6.9}
\end{align}
而要证明结论,根据\hyperref[corollary:线性方程组只有零解的充要条件]{推论\ref{corollary:线性方程组只有零解的充要条件}}可知,只要证明对任意的 $\lambda\in\mathbb{C}$,下列线性方程组只有零解即可:
\begin{align}
\begin{cases}
\boldsymbol{C}\boldsymbol{x}=\boldsymbol{0}, \\
\boldsymbol{C}(\boldsymbol{A}-\lambda\boldsymbol{I}_n)\boldsymbol{x}=\boldsymbol{0}, \\
\boldsymbol{C}(\boldsymbol{A}-\lambda\boldsymbol{I}_n)^2\boldsymbol{x}=\boldsymbol{0}, \\
\cdots\cdots\cdots \\
\boldsymbol{C}(\boldsymbol{A}-\lambda\boldsymbol{I}_n)^{n - 1}\boldsymbol{x}=\boldsymbol{0}.
\end{cases} \label{example0.6-6.10}
\end{align}

任取 $\lambda_0\in\mathbb{C}$ 以及对应线性方程组 \eqref{example0.6-6.10}的任一解 $\boldsymbol{x}_0$,则由线性方程组 \eqref{example0.6-6.10}可得 $\boldsymbol{C}\boldsymbol{x}_0=\boldsymbol{0}$,$\boldsymbol{C}\boldsymbol{A}\boldsymbol{x}_0=\boldsymbol{0}$,$\cdots$,$\boldsymbol{C}\boldsymbol{A}^{n - 1}\boldsymbol{x}_0=\boldsymbol{0}$,因此对任意次数小于 $n$ 的多项式 $g(x)$,均有 $\boldsymbol{C}g(\boldsymbol{A})\boldsymbol{x}_0=\boldsymbol{0}$. 设
\[f(\lambda)=|\lambda\boldsymbol{I}_n - \boldsymbol{A}|=(\lambda - \lambda_1)(\lambda - \lambda_2)\cdots(\lambda - \lambda_n)\]
为 $\boldsymbol{A}$ 的特征多项式,则由 \hyperref[theorem:Cayley-Hamilton定理]{Cayley - Hamilton 定理}可得
\[(\boldsymbol{A}-\lambda_1\boldsymbol{I}_n)(\boldsymbol{A}-\lambda_2\boldsymbol{I}_n)\cdots(\boldsymbol{A}-\lambda_n\boldsymbol{I}_n)=\boldsymbol{O}.\]
因此 $\boldsymbol{y}=(\boldsymbol{A}-\lambda_2\boldsymbol{I}_n)\cdots(\boldsymbol{A}-\lambda_n\boldsymbol{I}_n)\boldsymbol{x}_0$ 既满足 $(\boldsymbol{A}-\lambda_1\boldsymbol{I}_n)\boldsymbol{y}=\boldsymbol{0}$,又满足 $\boldsymbol{C}\boldsymbol{y}=\boldsymbol{0}$,故由线性方程组\eqref{example0.6-6.9}只有零解可得 $\boldsymbol{y}=(\boldsymbol{A}-\lambda_2\boldsymbol{I}_n)\cdots(\boldsymbol{A}-\lambda_n\boldsymbol{I}_n)\boldsymbol{x}_0=\boldsymbol{0}$. 不断重复上述论证,最后可得 $\boldsymbol{x}_0=\boldsymbol{0}$,结论得证. 
\end{proof}

\begin{example}
设 $\boldsymbol{A}$ 是 $n$ 阶矩阵,$\boldsymbol{B}$ 是 $n\times m$ 矩阵,分块矩阵 $(\boldsymbol{B},\boldsymbol{A}\boldsymbol{B},\cdots,\boldsymbol{A}^{n - 2}\boldsymbol{B},\boldsymbol{A}^{n - 1}\boldsymbol{B})$ 的秩为 $r$. 证明: 存在 $n$ 阶可逆矩阵 $\boldsymbol{P}$,使得
\[
\boldsymbol{P}^{-1}\boldsymbol{A}\boldsymbol{P}=\begin{pmatrix}
\boldsymbol{A}_{11} & \boldsymbol{A}_{12} \\
\boldsymbol{O} & \boldsymbol{A}_{22}
\end{pmatrix}, \quad
\boldsymbol{P}^{-1}\boldsymbol{B}=\begin{pmatrix}
\boldsymbol{B}_1 \\
\boldsymbol{O}
\end{pmatrix},
\]
其中 $\boldsymbol{A}_{11}$ 是 $r$ 阶矩阵,$\boldsymbol{B}_1$ 是 $r\times m$ 矩阵.
\end{example}
\begin{remark}
\hypertarget{例题0.7容易验证的原因}{$\boldsymbol{A}\boldsymbol{\alpha}_i(1\leq i\leq r)$ 都是 $\boldsymbol{\alpha}_1,\boldsymbol{\alpha}_2,\cdots,\boldsymbol{\alpha}_r$ 的线性组合的原因:}因为 $\boldsymbol{\alpha}_1,\cdots,\boldsymbol{\alpha}_r$ 是 $(\boldsymbol{B},\boldsymbol{A}\boldsymbol{B},\cdots,\boldsymbol{A}^{n - 2}\boldsymbol{B},\boldsymbol{A}^{n - 1}\boldsymbol{B})$ 列向量的极大无关组,所以对 $\forall i\in\{1,2,\cdots,r\}$,都存在 $k\in\{0,1,\cdots,n - 1\}$,使得 $\boldsymbol{\alpha}_i$ 是 $\boldsymbol{A}^k\boldsymbol{B}$ 的某一列向量.

当 $\boldsymbol{\alpha}_i$ 是 $\boldsymbol{A}^k\boldsymbol{B}(0\leqslant k\leqslant n - 2)$ 的某一列向量时,则 $\boldsymbol{A}\boldsymbol{\alpha}_i$ 一定是 $\boldsymbol{A}^{k + 1}\boldsymbol{B}$ 的某一列向量,又由于 $1\leqslant k + 1\leqslant n - 1$,因此 $\boldsymbol{A}\boldsymbol{\alpha}_i$ 仍是 $(\boldsymbol{B},\boldsymbol{AB},\cdots,\boldsymbol{A}^{n - 2}\boldsymbol{B},\boldsymbol{A}^{n - 1}\boldsymbol{B})$ 的某一列向量,从而 $\boldsymbol{A}\boldsymbol{\alpha}_i$ 可由 $\boldsymbol{\alpha}_1,\cdots,\boldsymbol{\alpha}_r$ 线性表出.

当 $\boldsymbol{\alpha}_i$ 是 $\boldsymbol{A}^{n - 1}\boldsymbol{B}$ 的某一列向量时,则 $\boldsymbol{A}\boldsymbol{\alpha}_i$ 一定是 $\boldsymbol{A}^n\boldsymbol{B}$ 的某一列向量. 由\eqref{equation12312434-1.1}式可知
\begin{align*}
\boldsymbol{A}^n\boldsymbol{B}=-a_1\boldsymbol{A}^{n - 1}\boldsymbol{B}-\cdots - a_{n - 1}\boldsymbol{A}\boldsymbol{B}-a_n\boldsymbol{B}.
\end{align*}
而上式右边的每一个列向量都可以由 $\boldsymbol{\alpha}_1,\cdots,\boldsymbol{\alpha}_r$ 线性表出,于是 $\boldsymbol{A}^n\boldsymbol{B}$ 的每一个列向量都可以由 $\boldsymbol{\alpha}_1,\cdots,\boldsymbol{\alpha}_r$ 线性表出. 故 $\boldsymbol{A}\boldsymbol{\alpha}_i$ 也可以由 $\boldsymbol{\alpha}_1,\cdots,\boldsymbol{\alpha}_r$ 线性表出. 
\end{remark}
\begin{proof}
设 $(\boldsymbol{B},\boldsymbol{A}\boldsymbol{B},\cdots,\boldsymbol{A}^{n - 2}\boldsymbol{B},\boldsymbol{A}^{n - 1}\boldsymbol{B})$ 列向量的极大无关组为 $\boldsymbol{\alpha}_1,\boldsymbol{\alpha}_2,\cdots,\boldsymbol{\alpha}_r$,由基扩张定理可将其扩张为 $\mathbb{F}^n$ 的一组基 $\{\boldsymbol{\alpha}_1,\boldsymbol{\alpha}_2,\cdots,\boldsymbol{\alpha}_n\}$. 令 $\boldsymbol{P}=(\boldsymbol{\alpha}_1,\boldsymbol{\alpha}_2,\cdots,\boldsymbol{\alpha}_n)$,则 $\boldsymbol{P}$ 为可逆矩阵. 设 $\boldsymbol{A}$ 的特征多项式为 $f(\lambda)=\lambda^n + a_1\lambda^{n - 1}+\cdots + a_{n - 1}\lambda + a_n$,则由\hyperref[theorem:Cayley-Hamilton定理]{Cayley - Hamilton 定理}可得
\begin{align*}
f(\boldsymbol{A})=\boldsymbol{A}^n + a_1\boldsymbol{A}^{n - 1}+\cdots + a_{n - 1}\boldsymbol{A}+a_n\boldsymbol{I}_n=\boldsymbol{O},
\end{align*}
从而
\begin{align}\label{equation12312434-1.1}
\boldsymbol{A}^n\boldsymbol{B}=-a_1\boldsymbol{A}^{n - 1}\boldsymbol{B}-\cdots - a_{n - 1}\boldsymbol{A}\boldsymbol{B}-a_n\boldsymbol{B}.    
\end{align}
由上式\hyperlink{例题0.7容易验证的原因}{容易验证} $\boldsymbol{A}\boldsymbol{\alpha}_i(1\leq i\leq r)$ 都是 $\boldsymbol{\alpha}_1,\boldsymbol{\alpha}_2,\cdots,\boldsymbol{\alpha}_r$ 的线性组合,于是 $\boldsymbol{A}\boldsymbol{P}=\boldsymbol{P}\begin{pmatrix}
\boldsymbol{A}_{11} & \boldsymbol{A}_{12} \\
\boldsymbol{O} & \boldsymbol{A}_{22}
\end{pmatrix}$,即有 $\boldsymbol{P}^{-1}\boldsymbol{A}\boldsymbol{P}=\begin{pmatrix}
\boldsymbol{A}_{11} & \boldsymbol{A}_{12} \\
\boldsymbol{O} & \boldsymbol{A}_{22}
\end{pmatrix}$. 又 $\boldsymbol{B}$ 的列向量都是 $\boldsymbol{\alpha}_1,\boldsymbol{\alpha}_2,\cdots,\boldsymbol{\alpha}_r$ 的线性组合,于是 $\boldsymbol{B}=\boldsymbol{P}\begin{pmatrix}
\boldsymbol{B}_1 \\
\boldsymbol{O}
\end{pmatrix}$,即有 $\boldsymbol{P}^{-1}\boldsymbol{B}=\begin{pmatrix}
\boldsymbol{B}_1 \\
\boldsymbol{O}
\end{pmatrix}$.
\end{proof}

\begin{example}
设 $\boldsymbol{A}$ 是数域 $\mathbb{F}$ 上的 $n$ 阶矩阵,递归地定义矩阵序列 $\{\boldsymbol{A}_k\}_{k = 1}^{\infty}$:
\[
\boldsymbol{A}_1=\boldsymbol{A}, \quad
p_k=-\frac{1}{k}\mathrm{tr}(\boldsymbol{A}_k), \quad
\boldsymbol{A}_{k + 1}=\boldsymbol{A}(\boldsymbol{A}_k + p_k\boldsymbol{I}_n), \quad k = 1,2,\cdots.
\]
求证: $\boldsymbol{A}_{n + 1}=\boldsymbol{O}$.
\end{example}
\begin{proof}
设 $\boldsymbol{A}$ 的全体特征值为 $\lambda_1,\lambda_2,\cdots,\lambda_n$,它们的幂和记为 $s_k=\sum_{i = 1}^{n}\lambda_i^k=\mathrm{tr}(\boldsymbol{A}^k)$,它们的初等对称多项式记为 $\sigma_k$,则 $\boldsymbol{A}$ 的特征多项式为
\begin{align*}
f(\lambda)=\lambda^n - \sigma_1\lambda^{n - 1}+\cdots + (-1)^{n - 1}\sigma_{n - 1}\lambda + (-1)^n\sigma_n.
\end{align*}
下面用归纳法证明: $p_k = (-1)^k\sigma_k (1\leq k\leq n)$. $p_1 = -\mathrm{tr}(\boldsymbol{A}) = -\sigma_1$,结论成立. 假设小于等于 $k$ 时结论成立,则 $\boldsymbol{A}_{k + 1}=\boldsymbol{A}^{k + 1} - \sigma_1\boldsymbol{A}^k+\cdots + (-1)^k\sigma_k\boldsymbol{A}$. 由Newton公式可得
\begin{align*}
p_{k + 1}&=-\frac{1}{k + 1}\mathrm{tr}(\boldsymbol{A}_{k + 1})=-\frac{1}{k + 1}(s_{k + 1} - s_k\sigma_1+\cdots + (-1)^k s_1\sigma_k)=(-1)^{k + 1}\sigma_{k + 1},
\end{align*}
结论得证. 最后,由\hyperref[theorem:Cayley-Hamilton定理]{Cayley - Hamilton 定理}可得
\begin{align*}
\boldsymbol{A}_{n + 1}&=\boldsymbol{A}^{n + 1} - \sigma_1\boldsymbol{A}^n+\cdots + (-1)^n\sigma_n\boldsymbol{A}=f(\boldsymbol{A})\boldsymbol{A}=\boldsymbol{O}.
\end{align*} 
\end{proof}





\end{document}
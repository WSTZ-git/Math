\documentclass[../../main.tex]{subfiles}
\graphicspath{{\subfix{../../image/}}} % 指定图片目录,后续可以直接使用图片文件名。

% 例如:
% \begin{figure}[H]
% \centering
% \includegraphics[scale=0.4]{图.png}
% \caption{}
% \label{figure:图}
% \end{figure}
% 注意:上述\label{}一定要放在\caption{}之后,否则引用图片序号会只会显示??.

\begin{document}

\section{由乘法交换性诱导的同时性质}

\begin{proposition}[矩阵乘法可交换的基本性质]\label{proposition:矩阵乘法可交换的基本性质}
若两个矩阵或线性变换$A,B$乘法可交换,即\(AB = BA\), 则有\((AB)^{m}=A^{m}B^{m}\), \(f(A)g(B)=g(B)f(A)\)以及二项式定理
\begin{align*}
(A + B)^{m}=A^{m}+\mathrm{C}_{m}^{1}A^{m - 1}B+\cdots+\mathrm{C}_{m}^{m - 1}AB^{m - 1}+B^{m}
\end{align*}
等成立, 其中\(m \geqslant  1\), \(f(x),g(x)\)为多项式. 

特别地,一个矩阵或线性变换$A$一定与其自身可交换,从而也满足\(f(A)g(A)=g(A)f(A)\),其中\(f(x),g(x)\)为多项式.
\end{proposition}
\begin{proof}
证明是显然的.
\end{proof}

\subsection{特征子空间互为不变子空间}

\begin{proposition}[特征子空间互为不变子空间]\label{proposition:特征子空间互为不变子空间}
\begin{enumerate}
\item 设\(\varphi,\psi\)是复线性空间\(V\)上乘法可交换的线性变换, 即\(\varphi\psi = \psi\varphi\), 求证: \(\varphi\)的特征子空间是\(\psi\)的不变子空间, \(\psi\)的特征子空间是\(\varphi\)的不变子空间.

\item 若\(n\)阶复矩阵\(A,B\)乘法可交换, 即\(AB = BA\), 则\(A,B\)的特征子空间互为不变子空间.
\end{enumerate}
\end{proposition}
\begin{remark}
这个命题的结论对一般的数域是不成立的. 例如, \(A = I_{2}\), \(B=\begin{pmatrix}
0& -1\\
1& 0
\end{pmatrix}\), 显然\(A,B\)乘法可交换, 但它们在有理数域或实数域上没有公共的特征向量. 事实上, \(B\)在有理数域或实数域上都没有特征值 (它的特征值是\(\pm\mathrm{i}\)), 从而也没有特征向量, 所以更谈不上公共的特征向量了. 为了这个命题的结论推广到数域\(\mathbb{F}\)上, 我们必须假设\(A,B\)的特征值都在\(\mathbb{F}\)中.
\end{remark}
\begin{proof}
\begin{enumerate}
\item 由代数基本定理以及线性方程组的求解理论可知, \(n(n \geqslant  1)\)维复线性空间上的线性变换或\(n\)阶复矩阵至少有一个特征值和特征向量. 任取线性变换\(\varphi\)的一个特征值\(\lambda_{0}\), 设\(V_{0}\)是特征值\(\lambda_{0}\)的特征子空间, 则对任意的\(\alpha \in V_{0}\), 有
\begin{align*}
\varphi\psi(\alpha)=\psi\varphi(\alpha)=\psi(\lambda_{0}\alpha)=\lambda_{0}\psi(\alpha),
\end{align*}
即\(\psi(\alpha) \in V_{0}\), 因此\(V_{0}\)是\(\psi\)的不变子空间. 同理可证\(\psi\)的特征子空间是\(\varphi\)的不变子空间.

\item 
\end{enumerate}
\end{proof}

\begin{proposition}\label{proposition:线性变换与任意线性变换乘法可交换必是纯量变换}
设\(V\)为\(n\)维复线性空间, \(S\)是\(\mathcal{L}(V)\)的非空子集, 满足: \(S\)中的全体线性变换没有非平凡的公共不变子空间. 设线性变换\(\varphi\)与\(S\)中任一线性变换乘法均可交换, 证明: \(\varphi\)是纯量变换.
\end{proposition}
\begin{proof}
任取\(\varphi\)的特征值\(\lambda_{0}\)及其特征子空间\(V_{0}\). 任取\(\psi \in S\), 则\(\varphi\psi = \psi\varphi\), 由\hyperref[proposition:特征子空间互为不变子空间]{命题\ref{proposition:特征子空间互为不变子空间}}可知\(V_{0}\)是\(\psi -\)不变子空间, 从而是\(S\)中全体线性变换的公共不变子空间. 又\(V_{0} \neq 0\)(特征向量均非零), 故\(V_{0}=V\), 从而\(\varphi=\lambda_{0}I_{V}\)为纯量变换. 
\end{proof}

\subsection{有公共的特征向量}

\begin{proposition}\label{proposition:乘法可交换必有公共的特征向量}
\begin{enumerate}
\item 设\(\varphi,\psi\)是复线性空间\(V\)上乘法可交换的线性变换, 求证: \(\varphi,\psi\)至少有一个公共的(复)特征向量.

\item 若\(n\)阶复矩阵\(A,B\)乘法可交换, 即\(AB = BA\), 则\(A,B\)至少有一个公共的(复)特征向量. 
\end{enumerate}
\end{proposition}
\begin{remark}
这个命题的结论对一般的数域是不成立的. 例如, \(A = I_{2}\), \(B=\begin{pmatrix}
0& -1\\
1& 0
\end{pmatrix}\), 显然\(A,B\)乘法可交换, 但它们在有理数域或实数域上没有公共的特征向量. 事实上, \(B\)在有理数域或实数域上都没有特征值 (它的特征值是\(\pm\mathrm{i}\)), 从而也没有特征向量, 所以更谈不上公共的特征向量了. 为了这个命题的结论推广到数域\(\mathbb{F}\)上, 我们必须假设\(A,B\)的特征值都在\(\mathbb{F}\)中.
\end{remark}
\begin{proof}
\begin{enumerate}
\item 任取\(\varphi\)的特征值\(\lambda_{0}\)及其特征子空间\(V_{0}\), 由\hyperref[proposition:特征子空间互为不变子空间]{命题\ref{proposition:特征子空间互为不变子空间}}可知, \(V_{0}\)是\(\psi -\)不变子空间. 将线性变换\(\psi\)限制在\(V_{0}\)上, 由于\(V_{0}\)是维数大于零的复线性空间, 故由\hyperref[proposition:线性变换或矩阵在复线性空间上至少存在一个特征值及与其特征向量]{命题\ref{proposition:线性变换或矩阵在复线性空间上至少存在一个特征值及与其特征向量}}可知\(\psi|_{V_{0}}\)至少有一个特征值\(\mu_{0}\)及其特征向量\(\alpha \in V_{0}\), 从而\(\varphi(\alpha)=\lambda_{0}\alpha\), \(\psi(\alpha)=\mu_{0}\alpha\), 于是\(\alpha\)就是\(\varphi,\psi\)的公共特征向量. 

\item 
\end{enumerate}
\end{proof}

\begin{proposition}\label{proposition:一般数域上乘法可交换诱导的性质}
\begin{enumerate}
\item 设\(\varphi,\psi\)是数域\(\mathbb{F}\)上线性空间\(V\)上的乘法可交换的线性变换, 且\(\varphi,\psi\)的特征值都在\(\mathbb{F}\)中, 求证: \(\varphi,\psi\)的特征子空间互为不变子空间, 并且\(\varphi,\psi\)至少有一个公共的特征向量.

\item 若数域\(\mathbb{F}\)上的\(n\)阶矩阵\(A,B\)乘法可交换, 且它们的特征值都在\(\mathbb{F}\)中, 则\(A,B\)的特征子空间互为不变子空间, 并且\(A,B\)在\(\mathbb{F}^n\)中至少有一个公共的特征向量.
\end{enumerate}
\end{proposition}
\begin{proof}
\begin{enumerate}
\item 由线性方程组的求解理论可知, 若数域\(\mathbb{F}\)上的线性变换或\(\mathbb{F}\)上的矩阵在\(\mathbb{F}\)中有一个特征值, 则在\(\mathbb{F}\)上的线性空间或\(\mathbb{F}\)上的列向量空间中必存在对应的特征向量. 任取线性变换\(\varphi\)的一个特征值\(\lambda_{0} \in \mathbb{F}\), 设\(V_{0}\)是特征值\(\lambda_{0}\)的特征子空间, 则对任意的\(\alpha \in V_{0}\), 有
\begin{align*}
\varphi\psi(\alpha)=\psi\varphi(\alpha)=\psi(\lambda_{0}\alpha)=\lambda_{0}\psi(\alpha),
\end{align*}
即\(\psi(\alpha) \in V_{0}\), 因此\(V_{0}\)是\(\psi -\)不变子空间. 取\(V_{0}\)的一组基并扩张为\(V\)的一组基, 则\(\psi\)在这组基下的表示矩阵为分块对角矩阵\(\begin{pmatrix}
A&C\\
O&B
\end{pmatrix}\), 其中\(A\)是\(\psi|_{V_{0}}\)在给定基下的表示矩阵, 于是\(|\lambda I_{V}-\psi| = |\lambda I - A||\lambda I - B|\). 因为\(\psi\)的特征值都在\(\mathbb{F}\)中, 故\(A\)的特征值都在\(\mathbb{F}\)中, 于是\(\psi|_{V_{0}}\)的特征值都在\(\mathbb{F}\)中. 任取\(\psi|_{V_{0}}\)的一个特征值\(\mu_{0} \in \mathbb{F}\)及其特征向量\(\alpha \in V_{0}\), 则\(\varphi(\alpha)=\lambda_{0}\alpha\), \(\psi(\alpha)=\mu_{0}\alpha\), 于是\(\alpha\)就是\(\varphi,\psi\)的公共特征向量. 

\item 
\end{enumerate}
\end{proof}



\subsection{可同时相似上三角化}

\begin{proposition}[矩阵的上三角化]\label{proposition:特征值全在同一数域的矩阵可上三角化}
\begin{enumerate}
\item 设数域\(\mathbb{F}\)上的\(n\)阶矩阵\(A\)的特征值都在\(\mathbb{F}\)中, 求证: \(A\)在\(\mathbb{F}\)上可上三角化, 即存在\(\mathbb{F}\)上的可逆矩阵\(P\), 使得\(P^{-1}AP\)是上三角矩阵.
\item 设数域\(\mathbb{F}\)上线性空间\(V\)上的线性变换\(\varphi\)的特征值都在\(\mathbb{F}\)中, 则存在\(V\)的一组基, 使得\(\varphi\)在这组基下的表示矩阵是上三角矩阵. 
\end{enumerate}
\end{proposition}
\begin{proof}
\begin{enumerate}
\item 对阶数进行归纳. 当\(n = 1\)时结论显然成立, 设对\(n - 1\)阶矩阵结论成立, 现对\(n\)阶矩阵\(A\)进行证明. 设\(\lambda_{1} \in \mathbb{F}\)是\(A\)的一个特征值, 则由线性方程组的求解理论可知, 存在特征向量\(e_{1} \in \mathbb{F}^n\), 使得\(Ae_{1}=\lambda_{1}e_{1}\). 由基扩张定理, 可将\(e_{1}\)扩张为\(\mathbb{F}^n\)的一组基\(\{e_{1},e_{2},\cdots ,e_{n}\}\), 于是
\begin{align*}
(Ae_{1},Ae_{2},\cdots ,Ae_{n})=(e_{1},e_{2},\cdots ,e_{n})\begin{pmatrix}
\lambda_{1}& *\\
O&A_{1}
\end{pmatrix},
\end{align*}
其中\(A_{1}\)是\(\mathbb{F}\)上的\(n - 1\)阶矩阵. 令\(P=(e_{1},e_{2},\cdots ,e_{n})\), 则\(P\)是\(\mathbb{F}\)上的\(n\)阶可逆矩阵, 且由上式可得\(AP = P\begin{pmatrix}
\lambda_{1}& *\\
O&A_{1}
\end{pmatrix}\), 即\(P^{-1}AP=\begin{pmatrix}
\lambda_{1}& *\\
O&A_{1}
\end{pmatrix}\). 由此可得\(|\lambda I_{n}-A| = (\lambda - \lambda_{1})|\lambda I_{n - 1}-A_{1}|\), 又$A$的特征值全在$\mathbb{F}$中,从而\(A_{1}\)的特征值也全在\(\mathbb{F}\)中, 故由归纳假设, 存在\(\mathbb{F}\)上的\(n - 1\)阶可逆矩阵\(Q\), 使得\(Q^{-1}A_{1}Q\)是上三角矩阵. 令
\begin{align*}
R = P\begin{pmatrix}
1& O\\
O& Q
\end{pmatrix},
\end{align*}
则\(R\)是\(\mathbb{F}\)上的\(n\)阶可逆矩阵, 且
\begin{align*}
R^{-1}AR=\begin{pmatrix}
1& O\\
O& Q
\end{pmatrix}^{-1}\begin{pmatrix}
\lambda_{1}& *\\
O&A_{1}
\end{pmatrix}\begin{pmatrix}
1& O\\
O& Q
\end{pmatrix}=\begin{pmatrix}
\lambda_{1}& *\\
O&Q^{-1}A_{1}Q
\end{pmatrix}
\end{align*}
是上三角矩阵. 

\item 
\end{enumerate}
\end{proof}

\begin{proposition}\label{proposition:乘法可交换诱导同时上三角化}
\begin{enumerate}
\item 设\(A,B\)是数域\(\mathbb{F}\)上的\(n\)阶矩阵, 满足: \(AB = BA\)且\(A,B\)的特征值都在\(\mathbb{F}\)中, 求证: \(A,B\)在\(\mathbb{F}\)上可同时上三角化, 即存在\(\mathbb{F}\)上的可逆矩阵\(P\), 使得\(P^{-1}AP\)和\(P^{-1}BP\)都是上三角矩阵.

\item 设数域\(\mathbb{F}\)上线性空间\(V\)上的线性变换\(\varphi,\psi\)乘法可交换, 且它们的特征值都在\(\mathbb{F}\)中, 则存在\(V\)的一组基, 使得\(\varphi,\psi\)在这组基下的表示矩阵都是上三角矩阵. 
\end{enumerate}
\end{proposition}
\begin{proof}
\begin{enumerate}
\item 对阶数进行归纳. 当\(n = 1\)时结论显然成立, 设对\(n - 1\)阶矩阵结论成立, 现对\(n\)阶矩阵进行证明. 因为\(AB = BA\)且\(A,B\)的特征值都在\(\mathbb{F}\)中, 故由\hyperref[proposition:一般数域上乘法可交换诱导的性质]{命题\ref{proposition:一般数域上乘法可交换诱导的性质}}可知, \(A,B\)有公共的特征向量\(e_{1} \in \mathbb{F}^n\), 不妨设
\begin{align*}
Ae_{1}=\lambda_{1}e_{1},\ Be_{1}=\mu_{1}e_{1},
\end{align*}
其中\(\lambda_{1},\mu_{1} \in \mathbb{F}\)分别是\(A,B\)的特征值. 由基扩张定理, 可将\(e_{1}\)扩张为\(\mathbb{F}^n\)的一组基\(\{e_{1},e_{2},\cdots ,e_{n}\}\). 令\(P=(e_{1},e_{2},\cdots ,e_{n})\), 则\(P\)是\(\mathbb{F}\)上的\(n\)阶可逆矩阵, 从而有
\begin{align}\label{proposition-07-1.1}
\begin{aligned}
A\left( e_1,e_2,\cdots ,e_n \right) =\left( e_1,e_2,\cdots ,e_n \right) \left( \begin{matrix}
\lambda _1&		*\\
O&		A_1\\
\end{matrix} \right) \Leftrightarrow AP=P\left( \begin{matrix}
\lambda _1&		*\\
O&		A_1\\
\end{matrix} \right) \Leftrightarrow P^{-1}AP=\left( \begin{matrix}
\lambda _1&		*\\
O&		A_1\\
\end{matrix} \right) ,
\\
B\left( e_1,e_2,\cdots ,e_n \right) =\left( e_1,e_2,\cdots ,e_n \right) \left( \begin{matrix}
\lambda _1&		*\\
O&		B_1\\
\end{matrix} \right) \Leftrightarrow BP=P\left( \begin{matrix}
\lambda _1&		*\\
O&		B_1\\
\end{matrix} \right) \Leftrightarrow P^{-1}BP=\left( \begin{matrix}
\lambda _1&		*\\
O&		B_1\\
\end{matrix} \right) .
\end{aligned}
\end{align}
其中\(A_{1},B_{1}\)是\(\mathbb{F}\)上的\(n - 1\)阶矩阵. 由\(AB = BA\)及\eqref{proposition-07-1.1}式可得到
\begin{align*}
&\quad\,\, \left( P^{-1}AP \right) \left( P^{-1}BP \right) =P^{-1}ABP=P^{-1}BAP=\left( P^{-1}BP \right) \left( P^{-1}AP \right) 
\\
&\Leftrightarrow \left( \begin{matrix}
\lambda _1&		*\\
O&		A_1\\
\end{matrix} \right) \left( \begin{matrix}
\lambda _1&		*\\
O&		B_1\\
\end{matrix} \right) =\left( \begin{matrix}
\lambda _1&		*\\
O&		B_1\\
\end{matrix} \right) \left( \begin{matrix}
\lambda _1&		*\\
O&		A_1\\
\end{matrix} \right) 
\\
&\Leftrightarrow \left( \begin{matrix}
\lambda _1&		*\\
O&		A_1B_1\\
\end{matrix} \right) =\left( \begin{matrix}
\lambda _1&		*\\
O&		B_1A_1\\
\end{matrix} \right) 
\end{align*}
从而\(A_{1}B_{1} = B_{1}A_{1}\).又由\eqref{proposition-07-1.1}式可得
\begin{align*}
\left| \lambda I_n-A \right|=\left| \lambda -\lambda _1 \right|\left| \lambda I_{n-1}-A_1 \right|,\quad \left| \lambda I_n-B \right|=\left| \lambda -\lambda _1 \right|\left| \lambda I_{n-1}-B_1 \right|.
\end{align*}
因此\(A_{1},B_{1}\)的特征值也是$A,B$的特征值.又由于$A,B$的特征值都在\(\mathbb{F}\)中,故\(A_{1},B_{1}\)的特征值都在\(\mathbb{F}\)中. 故由归纳假设, 存在\(\mathbb{F}\)上的\(n - 1\)阶可逆矩阵\(Q\), 使得\(Q^{-1}A_{1}Q\)和\(Q^{-1}B_{1}Q\)都是上三角矩阵. 令
\begin{align*}
R = P\begin{pmatrix}
1& O\\
O& Q
\end{pmatrix},
\end{align*}
则\(R\)是\(\mathbb{F}\)上的\(n\)阶可逆矩阵, 且
\begin{align*}
R^{-1}AR&=\begin{pmatrix}
1& O\\
O& Q
\end{pmatrix}^{-1}\begin{pmatrix}
\lambda_{1}& *\\
O&A_{1}
\end{pmatrix}\begin{pmatrix}
1& O\\
O& Q
\end{pmatrix}=\begin{pmatrix}
\lambda_{1}& *\\
O&Q^{-1}A_{1}Q
\end{pmatrix},\\
R^{-1}BR&=\begin{pmatrix}
1& O\\
O& Q
\end{pmatrix}^{-1}\begin{pmatrix}
\mu_{1}& *\\
O&B_{1}
\end{pmatrix}\begin{pmatrix}
1& O\\
O& Q
\end{pmatrix}=\begin{pmatrix}
\mu_{1}& *\\
O&Q^{-1}B_{1}Q
\end{pmatrix}
\end{align*}
都是上三角矩阵.

\item 
\end{enumerate}
\end{proof}

\begin{proposition}[一族两两可交换的一般域上的矩阵可同时上三角化]\label{proposition:一族两两可交换的一般域上的矩阵可同时上三角化}
给定域 $\mathbb{F}$ 和指标集 $\Lambda$,设 $A_\lambda \in \mathbb{F}^{n \times n}, \lambda \in \Lambda$ 且两两可交换且特征值都属于 $\mathbb{F}$,则存在可逆矩阵 $P \in \mathbb{F}^{n \times n}$,使得
\[P^{-1}A_\lambda P \text{ 是上三角矩阵}, \forall \lambda \in \Lambda.\]
\end{proposition}
\begin{note}
证明的想法是对有限的量归纳, 即矩阵降阶. 本结果将综合运用几何方法和矩阵方法.
\end{note}
\begin{remark}
因为数量矩阵的特征子空间就是全空间,将其限制在特征子空间上,维数并未下降,所以需要分类讨论.
\end{remark}
\begin{proof}
$\mathbf{Step}\,\,\mathbf{1}$若$\forall \lambda \in \Lambda$,都有$A_{\lambda}$是数量矩阵,此时结论显然成立.

$\mathbf{Step}\,\,\mathbf{2}$任取一个非数量矩阵$A_1\in \mathbb{F} ^{n\times n}$,再任取$A_1$的一个特征子空间$V_1$,则$1\leqslant \dim V_1<n$,否则$A_1$就是纯量阵.由\refpro{proposition:特征子空间互为不变子空间}可知,$V_1$是$A_{\lambda}-$不变子空间,$\forall \lambda \in \Lambda$.因此可考虑线性变换$A_{\lambda}|_{V_1},\lambda \in \Lambda$.下对矩阵阶数进行归纳证明.

当$n=1$时,结论显然成立.假设命题对小于等于$n - 1$的情况都成立,考虑$n$的情形.

注意到$A_{\lambda}|_{V_1},\lambda \in \Lambda$两两乘法可交换,故由归纳假设可知,存在$V_1$的一组基,使$A_{\lambda}|_{V_1}$在这组基下有上三角表示矩阵$\widetilde{A}_{\lambda},\lambda \in \Lambda$.将这组基扩充为$V$的一组基,于是在新的基下,$A_{\lambda}\left( \lambda \in \Lambda \right)$有表示矩阵$\begin{pmatrix}
\widetilde{A}_{\lambda}&		\widetilde{B}_{\lambda}\\
&		\widetilde{C}_{\lambda}\\
\end{pmatrix},\lambda \in \Lambda$.

又由于$A_{\lambda},\lambda \in \Lambda$两两乘法可交换,故对$\forall \lambda ,\mu \in \Lambda$,有
\[
\begin{pmatrix}
\widetilde{A}_{\lambda}&		\widetilde{B}_{\lambda}\\
&		\widetilde{C}_{\lambda}\\
\end{pmatrix} \begin{pmatrix}
\widetilde{A}_{\mu}&		\widetilde{B}_{\mu}\\
&		\widetilde{C}_{\mu}\\
\end{pmatrix} =\begin{pmatrix}
\widetilde{A}_{\mu}&		\widetilde{B}_{\mu}\\
&		\widetilde{C}_{\mu}\\
\end{pmatrix} \begin{pmatrix}
\widetilde{A}_{\lambda}&		\widetilde{B}_{\lambda}\\
&		\widetilde{C}_{\lambda}\\
\end{pmatrix} 
\]
\[
\Longleftrightarrow \begin{pmatrix}
\widetilde{A}_{\lambda}\widetilde{A}_{\mu}&		*\\
&		\widetilde{C}_{\lambda}\widetilde{C}_{\mu}\\
\end{pmatrix} =\begin{pmatrix}
\widetilde{A}_{\mu}\widetilde{A}_{\lambda}&		*\\
&		\widetilde{C}_{\mu}\widetilde{C}_{\lambda}\\
\end{pmatrix} 
\]
即$\widetilde{A}_{\lambda},\widetilde{C}_{\lambda},\lambda \in \Lambda$两两乘法可交换.从而由归纳假设可知,对$\forall \lambda \in \Lambda$,存在可逆阵$\widetilde{P}_{\lambda}$,使得$(\widetilde{P}_{\lambda})^{-1}\widetilde{C}_{\lambda}\widetilde{P}_{\lambda}$是上三角阵.取$P=\begin{pmatrix}
I&		O\\
O&		\widetilde{P}_{\lambda}\\
\end{pmatrix} \in \mathbb{F} ^{n\times n}$,则此时对$\forall \lambda \in \Lambda$,就有
\[
P^{-1}A_{\lambda}P=\begin{pmatrix}
\widetilde{A}_{\lambda}&		\widetilde{B}_{\lambda}\\
&		(\widetilde{P}_{\lambda})^{-1}\widetilde{C}_{\lambda}\widetilde{P}_{\lambda}\\
\end{pmatrix}.
\]
而$\widetilde{A}_{\lambda}$,$(\widetilde{P}_{\lambda})^{-1}\widetilde{C}_{\lambda}\widetilde{P}_{\lambda}$都是上三角阵,故$P^{-1}A_{\lambda}P$也是上三角阵.因此由数学归纳法可知,结论成立.
\end{proof}

\begin{proposition}[一族两两可交换的复数(实数)域上的矩阵可同时酉(正交)上三角化]\label{proposition:一族两两可交换的复数(实数)域上的矩阵可同时酉(正交)上三角化}
\begin{enumerate}
\item 给定指标集 $\Lambda$, 设 $A_{\lambda} \in \mathbb{C}^{n \times n}, \lambda \in \Lambda$ 且两两可交换, 则存在酉矩阵 $P \in \mathbb{C}^{n \times n}$, 使得
\(
P^{-1}A_{\lambda}P
\)
是上三角矩阵, $\forall \lambda \in \Lambda$.

\item 给定指标集 $\Lambda$, 设 $A_{\lambda} \in \mathbb{R}^{n \times n}, \lambda \in \Lambda$ 且两两可交换且特征值都是实数. 则存在正交矩阵 $P \in \mathbb{R}^{n \times n}$, 使得
\(
P^{-1}A_{\lambda}P
\)
是上三角矩阵, $\forall \lambda \in \Lambda$.
\end{enumerate} 
\end{proposition}
\begin{note}
证明的想法是对有限的量归纳, 即矩阵降阶. 本结果将综合运用几何方法和矩阵方法.
\end{note}
\begin{remark}
因为数量矩阵的特征子空间就是全空间,将其限制在特征子空间上,维数并未下降,所以需要分类讨论.
\end{remark}
\begin{proof}
\begin{enumerate}
\item 设 $V = \mathbb{C}^n$ 且 $A_{\lambda}$ 是 $V$ 上线性变换.

$\mathbf{Step}\,\,\mathbf{1}$若 $A_{\lambda}$ 都是数量矩阵, 则结果已经成立.

$\mathbf{Step}\,\,\mathbf{2}$ 取某个非数量矩阵 $A_1$ 和一个特征子空间 $V_1$ 且 $1 \leqslant  \dim V_1 < n$. 由交换性知 $V_1$ 是所有 $A_{\lambda}$ 不变子空间, 因此 $A_{\lambda}|_{V_1}, \lambda \in \Lambda$ 也是一族更低维度的两两可交换的矩阵. 于是我们就将维度降了下去, 从而可以使用归纳法来完成证明. 即:

当 $n = 1$, 命题显然成立, 假设命题对 小于等于$n - 1$ 都成立, 当 $n$ 时, 由归纳假设, $A_{\lambda}|_{V_1}, \lambda \in \Lambda$ 也是一族两两可交换的矩阵, 从而存在 $V_1$ 的一族标准正交基, 使得在这组基下 $A_{\lambda}|_{V_1}$ 有上三角的表示矩阵 $\widetilde{A}_{\lambda}, \lambda \in \Lambda$. 将这组基扩充到 $\mathbb{C}^n$ 使得构成一组标准正交基, 则在新的标准正交基下, $A_{\lambda}$ 有表示矩阵 $\begin{pmatrix} \widetilde{A}_{\lambda} & \widetilde{B}_{\lambda} \\ 0 & \widetilde{C}_{\lambda} \end{pmatrix}, \lambda \in \Lambda$. 由
\[
\begin{pmatrix} \widetilde{A}_{\lambda} & \widetilde{B}_{\lambda} \\ 0 & \widetilde{C}_{\lambda} \end{pmatrix} \begin{pmatrix} \widetilde{A}_{\mu} & \widetilde{B}_{\mu} \\ 0 & \widetilde{C}_{\mu} \end{pmatrix} = \begin{pmatrix} \widetilde{A}_{\mu} & \widetilde{B}_{\mu} \\ 0 & \widetilde{C}_{\mu} \end{pmatrix} \begin{pmatrix} \widetilde{A}_{\lambda} & \widetilde{B}_{\lambda} \\ 0 & \widetilde{C}_{\lambda} \end{pmatrix},\mu,\lambda \in \Lambda.
\]
知 $\widetilde{C}_{\lambda}, \lambda \in \Lambda$ 也是两两可交换的矩阵. 因此存在酉矩阵 $\widetilde{P}$, 使得每一个 $\begin{pmatrix} \widetilde{P} \end{pmatrix}^{-1} \widetilde{C}_{\lambda} \widetilde{P}$ 都是上三角的. 然后我们取酉矩阵 $P = \begin{pmatrix} E & 0 \\ 0 & \widetilde{P} \end{pmatrix} \in \mathbb{C}^{n \times n}$, 就有 $P^{-1}A_{\lambda}P = \begin{pmatrix} \widetilde{A}_{\lambda} & \widetilde{B}_{\lambda} \\ 0 & \begin{pmatrix} \widetilde{P} \end{pmatrix}^{-1} \widetilde{C}_{\lambda} \widetilde{P} \end{pmatrix}, \forall \lambda \in \Lambda$ 都是上三角矩阵, 我们完成了证明.

\item 设 $V = \mathbb{R}^n$ 且 $A_{\lambda}$ 是 $V$ 上线性变换.

$\mathbf{Step}\,\,\mathbf{1}$ 若 $A_{\lambda}$ 都是数量矩阵, 则结果已经成立.

$\mathbf{Step}\,\,\mathbf{2}$  取某个非数量矩阵 $A_1$ 和一个特征子空间 $V_1$ 且 $1 \leqslant  \dim V_1 < n$. 由交换性知 $V_1$ 是所有 $A_{\lambda}$ 不变子空间, 因此 $A_{\lambda}|_{V_1}, \lambda \in \Lambda$ 也是一族更低维度的两两可交换的矩阵. 于是我们就将维度降了下去, 从而可以使用归纳法来完成证明. 即:

当 $n = 1$, 命题显然成立, 假设命题对 小于等于$n - 1$ 都成立, 当 $n$ 时, 由归纳假设, $A_{\lambda}|_{V_1}, \lambda \in \Lambda$ 也是一族两两可交换的矩阵, 从而存在 $V_1$ 的一族标准正交基, 使得在这组基下 $A_{\lambda}|_{V_1}$ 有上三角的表示矩阵 $\widetilde{A}_{\lambda}, \lambda \in \Lambda$. 将这组基扩充到 $\mathbb{R}^n$ 使得构成一组标准正交基, 则在新的标准正交基下, $A_{\lambda}$ 有表示矩阵 $\begin{pmatrix} \widetilde{A}_{\lambda} & \widetilde{B}_{\lambda} \\ 0 & \widetilde{C}_{\lambda} \end{pmatrix}, \lambda \in \Lambda$. 由
\[
\begin{pmatrix} \widetilde{A}_{\lambda} & \widetilde{B}_{\lambda} \\ 0 & \widetilde{C}_{\lambda} \end{pmatrix} \begin{pmatrix} \widetilde{A}_{\mu} & \widetilde{B}_{\mu} \\ 0 & \widetilde{C}_{\mu} \end{pmatrix} = \begin{pmatrix} \widetilde{A}_{\mu} & \widetilde{B}_{\mu} \\ 0 & \widetilde{C}_{\mu} \end{pmatrix} \begin{pmatrix} \widetilde{A}_{\lambda} & \widetilde{B}_{\lambda} \\ 0 & \widetilde{C}_{\lambda} \end{pmatrix}
\]
知 $\widetilde{C}_{\lambda}, \lambda \in \Lambda$ 也是两两可交换的矩阵. 因此存在正交矩阵 $\widetilde{P}$, 使得每一个 $\begin{pmatrix} \widetilde{P} \end{pmatrix}^{-1} \widetilde{C}_{\lambda} \widetilde{P}$ 都是上三角的. 然后我们取正交矩阵 $P = \begin{pmatrix} E & 0 \\ 0 & \widetilde{P} \end{pmatrix} \in \mathbb{R}^{n \times n}$, 就有 $P^{-1}A_{\lambda}P = \begin{pmatrix} \widetilde{A}_{\lambda} & \widetilde{B}_{\lambda} \\ 0 & \begin{pmatrix} \widetilde{P} \end{pmatrix}^{-1} \widetilde{C}_{\lambda} \widetilde{P} \end{pmatrix}, \forall \lambda \in \Lambda$ 都是上三角矩阵, 我们完成了证明.
\end{enumerate}
\end{proof}


\subsection{可同时相似对角化}

\begin{proposition}\label{proposition:乘法可交换诱导同时对角化}
\begin{enumerate}
\item 设\(\varphi,\psi\)是数域\(\mathbb{F}\)上\(n\)维线性空间\(V\)上的线性变换, 满足: \(\varphi\psi = \psi\varphi\)且\(\varphi,\psi\)都可对角化, 求证: \(\varphi,\psi\)可同时对角化, 即存在\(V\)的一组基, 使得\(\varphi,\psi\)在这组基下的表示矩阵都是对角矩阵.

\item 设\(A,B\)是数域\(\mathbb{F}\)上的\(n\)阶矩阵, 满足: \(AB = BA\)且\(A,B\)都在\(\mathbb{F}\)上可对角化, 则\(A,B\)在\(\mathbb{F}\)上可同时对角化, 即存在\(\mathbb{F}\)上的可逆矩阵\(P\), 使得\(P^{-1}AP\)和\(P^{-1}BP\)都是对角矩阵.
\end{enumerate}
\end{proposition}
\begin{proof}
\begin{enumerate}
\item 对空间维数进行归纳. 当\(n = 1\)时结论显然成立, 设对维数小于\(n\)的线性空间结论成立, 现对\(n\)维线性空间进行证明. 设\(\varphi\)的全体不同特征值为\(\lambda_{1},\cdots ,\lambda_{s} \in \mathbb{F}\), 对应的特征子空间分别为\(V_{1},\cdots ,V_{s}\), 则由\(\varphi\)可对角化可知
\begin{align*}
V = V_{1}\oplus\cdots\oplus V_{s}.
\end{align*}
若\(s = 1\), 则\(\varphi=\lambda_{1}I_{V}\)为纯量变换, 此时只要取\(V\)的一组基, 使得\(\psi\)在这组基下的表示矩阵为对角矩阵, 则\(\varphi\)在这组基下的表示矩阵为\(\lambda_{1}I_{n}\), 结论成立. 若\(s > 1\), 则\(\dim V_{i}< n\). 注意到\(\varphi\psi = \psi\varphi\)且\(\varphi,\psi\)的特征值都在\(\mathbb{F}\)中, 由\hyperref[proposition:特征子空间互为不变子空间]{命题\ref{proposition:特征子空间互为不变子空间}}可知\(V_{i}\)都是\(\psi -\)不变子空间. 考虑线性变换的限制\(\varphi|_{V_{i}},\psi|_{V_{i}}\): 它们乘法可交换, 且由可对角化线性变换的性质可知它们都可对角化, 故由归纳假设可知, \(\varphi|_{V_{i}},\psi|_{V_{i}}\)可同时对角化, 即存在\(V_{i}\)的一组基, 使得\(\varphi|_{V_{i}},\psi|_{V_{i}}\)在这组基下的表示矩阵都是对角矩阵. 将\(V_{i}\)的基拼成\(V\)的一组基, 则\(\varphi,\psi\)在这组基下的表示矩阵都是对角矩阵, 即\(\varphi,\psi\)可同时对角化.

\item 
\end{enumerate}
\end{proof}

\begin{lemma}\label{lemma:可对角化矩阵限制后仍可对角化}
任何域上的可对角化矩阵限制到不变子空间上仍然是可对角化矩阵.
\end{lemma}
\begin{proof}
注意到矩阵可对角化等价于极小多项式可以分解为一次式的积. 设 \( A \) 的极小多项式是 \( p \), \( W \) 是矩阵 \( A \) 一个不变子空间且 \( p_W \) 是 \( A|_W \) 极小多项式. 注意到 \( p(A|_W) = 0 \), 故 \( p_W | p \). 从而 \( p_W \) 也是一次式的积, 故 \( A|_W \) 可对角化.
\end{proof}

\begin{proposition}[一族两两可交换的可对角化矩阵可同时相似对角化]\label{proposition:一族两两可交换的可对角化矩阵可同时相似对角化}
给定域 \( \mathbb{F} \), 设 \( A_\lambda \in \mathbb{F}^{n \times n}, \lambda \in \Lambda \) 且两两可交换. 若每一个 \( A_\lambda, \lambda \in \Lambda \) 都可以在 \( \mathbb{F} \) 上相似对角化, 则存在可逆矩阵 \( P \in \mathbb{F}^{n \times n} \), 使得
\[
P^{-1}A_\lambda P \text{ 是对角矩阵}, \forall \lambda \in \Lambda.
\]
\end{proposition}
\begin{proof}
设 \( V = \mathbb{F}^n \) 且 \( A_\lambda \) 是 \( V \) 上线性变换.

$\mathbf{Step}\,\,\mathbf{1}$  若 \( A_\lambda \) 都是数量矩阵, 则结果已经成立.

$\mathbf{Step}\,\,\mathbf{2}$  取某个非数量矩阵 \( A_1 \), 于是有 \( V = \bigoplus_{i=1}^s V_i \), 这里 \( s \geqslant  2 \) 且 \( V_i \) 是属于 \( A_1 \) 不同特征值的特征子空间. 显然由交换性, 对每一个 \( i = 1,2,\cdots,s \), \( V_i \) 是所有 \( A_\lambda \) 不变子空间, 且 \( A_\lambda|_{V_i}, \lambda \in \Lambda \) 是一族两两可交换的矩阵, 由\reflem{lemma:可对角化矩阵限制后仍可对角化}知它们也是可对角化的. 注意到 \( 1 \leqslant  \dim V_i < n, i = 1,2,\cdots,s \), 所以我们的维度降下去了, 因此可使用归纳法.

当 \( n = 1 \), 命题显然成立, 假设命题对 \( \leqslant  n - 1 \) 都成立, 当 \( n \) 时, 由归纳假设, 对每一个 \( i = 1,2,\cdots,s \), 存在 \( V_i \) 的一个基使得 \( A_\lambda|_{V_i}, \lambda \in \Lambda \) 在这个基下表示矩阵是对角矩阵. 于是把这些基合起来构成一个新的基, 我们就得到在这个新的基下 \( A_\lambda, \lambda \in \Lambda \) 都是对角矩阵.
\end{proof}

\begin{proposition}[一族两两可交换的复正规 (实对称) 矩阵可同时酉 (正交) 相似对角化]\label{proposition:一族两两可交换的复正规 (实对称) 矩阵可同时酉 (正交) 相似对角化}
\begin{enumerate}
\item 给定指标集 \(\Lambda\), 设 \(A_{\lambda} \in \mathbb{C}^{n \times n}, \lambda \in \Lambda\) 且两两可交换且复正规, 则存在酉矩阵 \(P \in \mathbb{C}^{n \times n}\), 使得
\[
P^{-1}A_{\lambda}P \text{ 是对角矩阵, } \forall \lambda \in \Lambda.
\]

\item 给定指标集 \(\Lambda\), 设 \(A_{\lambda} \in \mathbb{R}^{n \times n}, \lambda \in \Lambda\) 且两两可交换且实对称, 则存在正交矩阵 \(P \in \mathbb{R}^{n \times n}\), 使得
\[
P^{-1}A_{\lambda}P \text{ 是对角矩阵, } \forall \lambda \in \Lambda.
\]
\end{enumerate}
\end{proposition}
\begin{note}
注意到 \((Ax, y) = (x, A^T y)\), \((Ax, y) = (x, A^* y)\) 在全空间成立则在子空间也成立, 所以 \(A^T|_V = (A|_V)^T\), \(A^*|_V = (A|_V)^*\), 所以一个实对称变换限制在不变子空间也是实对称的, 一个复正规变换限制在不变子空间也是复正规的.
\end{note}
\begin{proof}
\begin{enumerate}
\item 设 \(V = \mathbb{C}^n\) 且 \(A_{\lambda}\) 是 \(V\) 上线性变换.

$\mathbf{Step}\,\,\mathbf{1}$若 \(A_{\lambda}\) 都是数量矩阵, 则结果已经成立.

$\mathbf{Step}\,\,\mathbf{2}$ 取某个非数量矩阵 \(A_1\), 于是有 \(V = \bigoplus_{i=1}^s V_i\), 这里 \(s \geqslant 2\) 且 \(V_i\) 是属于 \(A_1\) 不同特征值的特征子空间. 显然由交换性, 对每一个 \(i = 1, 2, \cdots, s\), \(V_i\) 是所有 \(A_{\lambda}\) 不变子空间,从而$V_i$两两正交, 且 \(A_{\lambda}|_{V_i}, \lambda \in \Lambda\) 是一族两两可交换的正规矩阵, 于是可酉对角化的. 注意到 \(1 \leqslant \dim V_i < n, i = 1, 2, \cdots, s\). 所以我们的维度降下去了, 因此可使用归纳法.

当 \(n = 1\), 命题显然成立, 假设命题对 \(\leqslant n - 1\) 都成立, 当 \(n\) 时, 由归纳假设, 对每一个 \(i = 1, 2, \cdots, s\), 存在 \(V_i\) 的一个标准正交基使得 \(A_{\lambda}|_{V_i}, \lambda \in \Lambda\) 在这个基下表示矩阵是对角矩阵. 由于$V_i$两两正交,于是把这些基合起来构成一个新的正交基, 我们就得到在这个新的基下 \(A_{\lambda}, \lambda \in \Lambda\) 都是对角矩阵.

\item 设 \(V = \mathbb{R}^n\) 且 \(A_{\lambda}\) 是 \(V\) 上线性变换.

$\mathbf{Step}\,\,\mathbf{1}$若 \(A_{\lambda}\) 都是数量矩阵, 则结果已经成立.

$\mathbf{Step}\,\,\mathbf{2}$ 取某个非数量矩阵 \(A_1\), 于是有 \(V = \bigoplus_{i=1}^s V_i\), 这里 \(s \geqslant 2\) 且 \(V_i\) 是属于 \(A_1\) 不同特征值的特征子空间. 显然由交换性, 对每一个 \(i = 1, 2, \cdots, s\), \(V_i\) 是所有 \(A_{\lambda}\) 不变子空间,从而$V_i$两两正交, 且 \(A_{\lambda}|_{V_i}, \lambda \in \Lambda\) 是一族两两可交换的实对称矩阵, 由\reflem{lemma:可对角化矩阵限制后仍可对角化}知它们也是可正交相似对角化的. 注意到 \(1 \leqslant \dim V_i < n, i = 1, 2, \cdots, s\), 所以我们的维度降下去了, 因此可使用归纳法.

当 \(n = 1\), 命题显然成立, 假设命题对 \(\leqslant n - 1\) 都成立, 当 \(n\) 时, 由归纳假设, 对每一个 \(i = 1, 2, \cdots, s\), 存在 \(V_i\) 的一个标准正交基使得 \(A_{\lambda}|_{V_i}, \lambda \in \Lambda\) 在这个基下表示矩阵是对角矩阵. 由于$V_i$两两正交,于是把这些基合起来构成一个新的正交基, 我们就得到在这个新的基下 \(A_{\lambda}, \lambda \in \Lambda\) 都是对角矩阵.
\end{enumerate}
\end{proof}

\begin{lemma}[根子空间维数]\label{lemma:根子空间维数}
设 \( A \in \mathbb{F}^{n \times n} \) 且 \( \mathbb{F} \) 是域. 若 \( \lambda \in \mathbb{F} \) 是 \( A \) 的 \( r \) 重特征值, 则我们有
\begin{align}
\dim \mathrm{Ker} \left( \lambda E - A \right)^m = r, \forall m \geqslant r. \label{eq::::------23.7}
\end{align}
\end{lemma}
\begin{proof}
注意到
\[
\dim \mathrm{Ker} \left( \lambda E - A \right)^r = n - \mathrm{rank} \left( \left( \lambda E - A \right)^r \right).
\]
而秩不随域扩张而改变, 故不妨设 \( \mathbb{F} \) 是代数闭域. 则不妨设 \( A \) 为 Jordan 标准型
\[
\begin{pmatrix}
J_{n_1} \left( \lambda \right) & & & \\
& \ddots & & \\
& & J_{n_s} \left( \lambda \right) & \\
& & & J
\end{pmatrix}, \quad \sum_{i=1}^s n_i = r,
\]
这里 \( J \) 没有特征值 \( \lambda \). 于是直接计算有
\begin{gather*}
\left( A - \lambda E \right)^r = \begin{pmatrix}
J_{n_1}^r \left( 0 \right) & & & \\
& \ddots & & \\
& & J_{n_s}^r \left( 0 \right) & \\
& & & \left( J - \lambda E_{n-r} \right)^r
\end{pmatrix} = \begin{pmatrix}
0 & & & \\
& \ddots & & \\
& & 0 & \\
& & & \left( J - \lambda E_{n-r} \right)^r
\end{pmatrix}, 
\\
\det \left( J - \lambda E_{n-r} \right)^r \neq 0.
\end{gather*}
现在知 \( \mathrm{rank} \left( A - \lambda E \right)^r = n - r \), 这就得到了 \eqref{eq::::------23.7} 的 \( m = r \) 的情况. 注意到当 \( m > r \) 时, 上述矩阵 \( 0 \) 块在次方时不会继续丢失阶数,而$\left( A - \lambda E \right)^m$仍可逆, 因此秩不会再次减少, 所以 \eqref{eq::::------23.7} 对任何 \( m \geqslant r \) 都成立, 这就完成了证明.
\end{proof}

\begin{theorem}[公共特征值]\label{theorem:公共特征值}
给定域 \( \mathbb{F} \), 指标集 \( \Lambda \) 和 \( A_\lambda \in \mathbb{F}^{n \times n}, \lambda \in \Lambda \).

1. 若 \( A_\lambda, \lambda \in \Lambda \) 两两可交换;

2. 若 \( \mathbb{F} = \mathbb{R}, n \) 为奇数.

则 \( A_\lambda \) 在 \( \mathbb{F} \) 上有公共特征向量.
\end{theorem}
\begin{note}
某种角度上说, 存在公共特征向量和可同时相似上三角化是等价的.

1. 由\hyperref[proposition:一族两两可交换的一般域上的矩阵可同时上三角化]{一族两两可交换的一般域上的矩阵可同时上三角化}, 不妨设 \( A_\lambda, \lambda \in \Lambda \) 都是上三角矩阵, 注意观察第一列知他们有公共的特征向量 \( e_1 \).

2. 证明的关键在于奇数次实多项式必有实根, 从而奇数矩阵必有实特征值.
\end{note}
\begin{proof} 
$\mathbf{Step}\,\,\mathbf{1}$ 取 \( \{A_\lambda\}_{\lambda \in \Lambda} \) 的极大无关组 \( A_1, A_2, \cdots, A_k \). 如果能找到 \( A_1, A_2, \cdots, A_k \) 的公共特征向量 \( \alpha \in \mathbb{R}^n \), 则对任何 \( A_\lambda = \sum_{i=1}^k c_i A_i \), 都有
\[
A_\lambda \alpha = \sum_{i=1}^k c_i A_i \alpha = \left( \sum_{i=1}^k c_i \lambda_i \right) \alpha,
\]
这里 \( \lambda_i \) 是 \( A_i, 1 \leqslant i \leqslant k \) 关于特征向量 \( \alpha \) 的特征值. 于是问题变成只需要对有限个矩阵讨论.

$\mathbf{Step}\,\,\mathbf{2}$ 对于两两可交换的实矩阵 \( A_1, A_2, \cdots, A_k \). 取 \( \{A_\lambda\}_{\lambda \in \Lambda} \) 的极大无关组 \( A_1, A_2, \cdots, A_k \). 设 \( \lambda \in \mathbb{R} \) 是 \( A_1 \) 的奇数重特征值, 记重数为 \( r \in \mathbb{N} \) 并设 \( W = \mathrm{Ker} \left( (\lambda E - A_1)^r \right) \). 由\reflem{lemma:根子空间维数}知 \( \dim W = r \) 且显然有 \( A_j|_W, j = 2, 3, \cdots, k \) 两两可交换.

注意到 \( r = n \) 是可能的, 真正降下去的是矩阵个数, 所以我们对矩阵个数归纳. 命题对 \( k = 1 \) 显然成立, 假设命题对 \( \leqslant k - 1 \) 成立, 当 \( k \) 时我们运用归纳假设知 \( A_j|_W, j = 2, 3, \cdots, k \) 有公共特征向量, 即存在实数 \( \lambda_i, 2 \leqslant i \leqslant k \), 使得
\[
W' = \bigcap_{i=2}^k \mathrm{Ker} \left( \lambda_i I - A_i|_W \right) \neq \{0\}.
\]

$\mathbf{Step}\,\,\mathbf{3}$ 由可交换性知 \( W' \) 是 \( A_1 \) 不变子空间, 所以断言 \( A_1|_{W'} \) 在 \( \mathbb{C} \) 上特征值只有 \( \lambda \). 事实上由$\mathbf{Step}\,\,\mathbf{1}$,显然 \( W' \subset W \). 又设在复数域上 \( A_1 \beta = \mu \beta, \beta \neq 0, \mu \in \mathbb{C} \), 则
\[
0 = (\lambda E - A_1)^r \beta = (\lambda E - A_1)^{r - 1} (\lambda \beta - A_1 \beta) = (\lambda - \mu)(\lambda E - A_1)^{r - 1} \beta = \cdots = (\lambda - \mu)^r \beta,
\]
故 \( \mu = \lambda \).

$\mathbf{Step}\,\,\mathbf{4}$ 取 \( W' \) 中 \( A_1 \) 一个特征向量 \( \gamma \), 则 \( \gamma \) 是 \( A_1, A_2, \cdots, A_k \) 公共特征向量, 这就完成了证明.
\end{proof}

\begin{lemma}[Netwon公式]\label{lemma:Netwon公式}
设
\[
s_j = \sum_{i=1}^n x_i^j, \quad \sigma_j = \sum_{1 \leqslant i_1 < i_2 < \cdots < i_j \leqslant n} x_{i_1}x_{i_2} \cdots x_{i_j}, j = 1, 2, \cdots, n,
\]

我们有
\[
s_k - \sigma_1 s_{k - 1} + \sigma_2 s_{k - 2} + \cdots + (-1)^{k - 1} \sigma_{k - 1} s_1 + (-1)^k k \sigma_k = 0, 1 \leqslant k \leqslant n;
\]
\[
s_k - \sigma_1 s_{k - 1} + \sigma_2 s_{k - 2} + \cdots + (-1)^{n - 1} \sigma_{n - 1} s_{k - n + 1} + (-1)^n \sigma_n s_{k - n} = 0, k > n.
\]
\end{lemma}
\begin{proof}

\end{proof}

\begin{theorem}\label{theorem:任意次迹零就是幂零矩阵}
对域 \( \mathbb{F} \), 设 \( C \in \mathbb{F}^{n \times n} \) 且 \( \mathrm{tr}(C^k) = 0, \forall k \in \mathbb{N} \), 则 \( C \) 是幂零矩阵.
\end{theorem}
\begin{proof}
不妨设 \( \mathbb{F} \) 是代数闭的, 设 \( C \) 的全部特征值是 \( \lambda_1, \lambda_2, \cdots, \lambda_n \), 这里可能相同. 则
\[
\sum_{i=1}^n \lambda_i^k = 0, \forall k \in \mathbb{N},
\]
即 \( \lambda_1, \lambda_2, \cdots, \lambda_n \) 的一切幂和为 0. 由\hyperref[lemma:Netwon公式]{Netwon公式}, 我们有 \( \lambda_1, \lambda_2, \cdots, \lambda_n \) 构成的所有初等对称多项式为 0. 由Vieta定理, 我们有 \( \lambda_1, \lambda_2, \cdots, \lambda_n \) 都是 \( \lambda^n = 0 \) 的根, 这就证明了 \( C \) 的全部特征值都是 0, 故 \( C \) 是幂零矩阵.
\end{proof}

\begin{example}
给定数域 \( \mathbb{F} \), \( A, B \in \mathbb{F}^{n \times n} \), \( A, B \) 特征值都在 \( \mathbb{F} \) 中.
\begin{enumerate}
\item 若 \( AB = 0 \), 证明存在可逆 \( P \in \mathbb{F}^{n \times n} \) 使得 \( P^{-1}AP, P^{-1}BP \) 都是上三角矩阵.

\item 若 \( \mathrm{rank}(AB - BA) \leqslant 1 \), 证明存在可逆 \( P \in \mathbb{F}^{n \times n} \) 使得 \( P^{-1}AP, P^{-1}BP \) 都是上三角矩阵.

\item 若存在 \( a, b \in \mathbb{F} \) 使得 \( AB - BA = aA + bB \), 证明存在可逆 \( P \in \mathbb{F}^{n \times n} \) 使得 \( P^{-1}AP, P^{-1}BP \) 都是上三角矩阵.
\end{enumerate}
\end{example}
\begin{note}
此类问题,我们可以先找一个公共的非平凡的不变子空间(或公共特征向量),从而可以从这个不变子空间中取一组基,再扩充为全空间的基.于是原矩阵在这组基下的表示矩阵就可同时上三角化,并且表示矩阵的$(1,1)$元可以确定,接下来只需考虑表示矩阵的其余的$n-1$阶矩阵即可,这样就对原矩阵进行了降阶.最后再利用数学归纳法就能立刻得到证明.
\end{note}
\begin{proof}
\begin{enumerate}
\item 若 \( A \) 可逆, 则 \( B = 0 \). 此时因为 \( AB = BA = 0 \). 运用\hyperref[proposition:一族两两可交换的一般域上的矩阵可同时上三角化]{可交换矩阵可同时上三角化}即证, 当然, 由\refthe{theorem:公共特征值}, 此时 \( A, B \) 有公共特征向量. 若 \( A \) 不可逆, 于是我们不难证明 \( \mathrm{Ker} A \) 是 \( B \) 的不变子空间(当然也是$A$的不变子空间), 所以存在 \( B|_{\mathrm{Ker} A} \) 的特征向量 \( \alpha \in \mathrm{Ker} A \). 此时 \( \alpha \) 也是 \( A \) 的特征向量 (特征值为 0), 于是 \( A, B \) 总有公共特征向量.

那么我们运用归纳法, 当 \( n = 1 \) 显然, 假定命题对 \( n - 1 \) 时成立, 当 \( n \) 时, 设 \( \alpha \) 是 \( A, B \) 公共特征向量, 将其扩充为 \( \mathbb{F}^n \) 一组基, 在这组基下, \( A, B \) 分别有表示矩阵
\[
\widetilde{A} = \begin{pmatrix} c_1 & \beta^T \\ 0 & A_{n - 1} \end{pmatrix}, \widetilde{B} = \begin{pmatrix} c_2 & \gamma^T \\ 0 & B_{n - 1} \end{pmatrix}, c_1, c_2 \in \mathbb{F}.
\]
由 \( AB = 0 \) 知 \( A_{n - 1}B_{n - 1} = 0 \). 故可对 \( A_{n - 1}, B_{n - 1} \) 用归纳假设, 即存在可逆 \( P_{n - 1} \in \mathbb{F}^{(n - 1) \times (n - 1)} \) 使得 $P_{n - 1}^{-1}A_{n - 1}P_{n - 1}$, $P_{n - 1}^{-1}$$B_{n - 1}$$P_{n - 1}$ 是上三角矩阵. 于是取 \( P = \begin{pmatrix} 1 & 0 \\ 0 & P_{n - 1} \end{pmatrix} \in \mathbb{F}^{n \times n} \), 就有
\[
P^{-1}\widetilde{A}P = \begin{pmatrix} c_1 & \beta^T P_{n - 1} \\ 0 & P_{n - 1}^{-1}A_{n - 1}P_{n - 1} \end{pmatrix}, P^{-1}\widetilde{B}P = \begin{pmatrix} c_2 & \gamma^T P_{n - 1} \\ 0 & P_{n - 1}^{-1}B_{n - 1}P_{n - 1} \end{pmatrix}
\]
都是上三角矩阵. 这就证明了存在可逆 \( P \in \mathbb{F}^{n \times n} \) 使得 \( P^{-1}AP, P^{-1}BP \) 都是上三角矩阵.

\item 若 \( \mathrm{rank}(AB - BA) = 0 \), 这是\hyperref[proposition:一族两两可交换的一般域上的矩阵可同时上三角化]{可交换矩阵可同时上三角化}的直接结果. 若 \( \mathrm{rank}(AB - BA) = 1 \), 我们假定 \( A \) 是不可逆的, 否则用 \( A - \lambda I, \lambda \in \mathbb{F} \)(这里$\lambda$是此时$A$的特征值) 代替 \( A \). 下证 \( \mathrm{Ker} A, \mathrm{Im} A \) 必有一个是 \( B \) 的不变子空间.

$\mathbf{Step}\,\,\mathbf{1}$ 若 \( \mathrm{Ker} A \subset \mathrm{Ker}(AB - BA) \), 则设 \( x \in \mathrm{Ker} A \), 我们有 \( ABx = BAx = 0 \), 这就证明了 \( \mathrm{Ker} A \) 是 \( B \) 不变子空间.

$\mathbf{Step}\,\,\mathbf{2}$ 若 \( \mathrm{Ker} A \not\subset \mathrm{Ker}(AB - BA) \) 不成立, 取 \( x \in \mathrm{Ker} A \) 使得 \( y = (AB - BA)x \neq 0 \). 注意到 \( \mathrm{rank}(AB - BA) = 1 \), 故 \( \mathrm{Im}(AB - BA) = \mathrm{span}\{y\} \). 但是
\[
y = (AB - BA)x = ABx - BAx = ABx \in \mathrm{Im} A,
\]
这就说明 \( \mathrm{Im}(AB - BA) \subset \mathrm{Im} A \). 于是对 \( u = Av \in \mathrm{Im} A \), 我们有
\[
Bu = BAv = ABv - (AB - BA)v \in \mathrm{Im} A,
\]
从而确实了 \( \mathrm{Im} A \) 是 \( B \) 不变子空间.

注意到 \( A, B \) 有一个公共非平凡不变子空间. 所以取这个不变子空间一组基扩充为全空间的基, 则 \( A, B \) 有表示矩阵
\[
\widetilde{A} = \begin{pmatrix} A_1 & A_3 \\ 0 & A_2 \end{pmatrix}, \widetilde{B} = \begin{pmatrix} B_1 & B_3 \\ 0 & B_2 \end{pmatrix}.
\]
注意到
\begin{align*}
r(AB - BA) &= r\begin{pmatrix} A_1B_1 - B_1A_1 & * \\ 0 & A_2B_2 - B_2A_2 \end{pmatrix}
\\
&\stackrel{\text{\hyperref[矩阵秩的基本公式3]{矩阵秩的基本公式(3)}}}{\geqslant} r\begin{pmatrix} A_1B_1 - B_1A_1 & 0 \\ 0 & A_2B_2 - B_2A_2 \end{pmatrix} 
\\
&= r(A_1B_1 - B_1A_1) + r(A_2B_2 - B_2A_2),
\end{align*}
我们有
\[
r(A_1B_1 - B_1A_1) \leqslant 1, r(A_2B_2 - B_2A_2) \leqslant 1,
\]
现在阶降下去了, 类似我们前面的归纳证明我们知道 \( A, B \) 可同时相似上三角化.

\item 如果 \( a = b = 0 \), 那么这是\hyperref[proposition:一族两两可交换的一般域上的矩阵可同时上三角化]{可交换矩阵可同时上三角化}的直接结果. 不失一般性, 不妨设 \( a \neq 0 \). 我们可以用 \( \frac{1}{a}B \) 代替 \( B \) 来不妨设 \( a = 1 \). 那么记 \( C = AB - BA = A + bB \), 则成立等式 \( CB - BC = C \). 显然只需证明 \( C, B \) 可同时相似上三角化即可, 因为此时 \( A \) 也就被相似上角化. 注意到
\[
\mathrm{tr}(CB - BC) = \mathrm{tr}(C) = 0, \mathrm{tr}(CCB - CBC) = \mathrm{tr}(C^2) = 0,
\]
\[
\mathrm{tr}(CCCB - CCBC) = \mathrm{tr}(C^3) = 0, \mathrm{tr}(CCCCB - CCCBC) = \mathrm{tr}(C^4) = 0,
\]
\[
\cdots\cdots\cdots\cdots
\]
于是我们证明了 \( \mathrm{tr}(C^k) = 0, \forall k \in \mathbb{N} \). 由\refthe{theorem:任意次迹零就是幂零矩阵}, 我们知道 \( C \) 是幂零矩阵.

假设 \( B, C \) 没有公共特征向量且 \( Bx = \lambda x, x \neq 0, \lambda \in \mathbb{F} \), 则 \( Cx \neq 0 \) 且
\[
CBx - BCx = Cx \Rightarrow (B - (\lambda - 1)I)Cx = 0,
\]
因此 \( B \) 有特征值 \( \lambda - 1 \). 这样 \( B \) 也应该有特征值 \( \lambda - 2 \), 如此下去 \( B \) 会有无穷多个特征值, 这不可能! 所以 \( B, C \) 应该有公共特征向量 \( \alpha \). 将 \( \alpha \) 扩充为全空间的一组基, \( B, C \) 在这组基下有表示矩阵
\[
\widetilde{B} = \begin{pmatrix} \lambda & * \\ 0 & B_1 \end{pmatrix}, \widetilde{C} = \begin{pmatrix} 0 & * \\ 0 & C_1 \end{pmatrix}, \lambda \in \mathbb{F}, C_1 \in \mathbb{F}^{(n - 1) \times (n - 1)},
\]
则 \( C_1B_1 - B_1C_1 = C_1 \). 这样阶数就降下去了. 所以可以类似前面归纳完成证明.
\end{enumerate}
\end{proof}

\begin{lemma}[Hilbert矩阵逆矩阵元素和]\label{lemma:Hilbert矩阵逆矩阵元素和}
设
\[
H = 
\begin{pmatrix}
1 & \frac{1}{2} & \cdots & \frac{1}{n} \\
\frac{1}{2} & \frac{1}{3} & \cdots & \frac{1}{n + 1} \\
\vdots & \vdots & \ddots & \vdots \\
\frac{1}{n} & \frac{1}{n + 1} & \cdots & \frac{1}{2n - 1}
\end{pmatrix}
\]
, 这里 $H$ 是 $n$ 阶矩阵. 证明 $H^{-1}$ 所有元素和为 $n^{2}$.
\end{lemma}
\begin{note}
这里涉及经典技巧, 即分解 $J = AH + HB$, 需要积累.
\end{note}
\begin{proof}
取 $n$ 阶矩阵
\[
J = 
\begin{pmatrix}
1 & 1 & \cdots & 1 \\
1 & 1 & \cdots & 1 \\
\vdots & \vdots & \ddots & \vdots \\
1 & 1 & \cdots & 1
\end{pmatrix}
,
A = 
\begin{pmatrix}
1 & 0 & \cdots & 0 \\
0 & 2 & \cdots & 0 \\
\vdots & \vdots & \ddots & \vdots \\
0 & 0 & \cdots & n
\end{pmatrix}
,
B = 
\begin{pmatrix}
0 & 0 & \cdots & 0 \\
0 & 1 & \cdots & 0 \\
\vdots & \vdots & \ddots & \vdots \\
0 & 0 & \cdots & n - 1
\end{pmatrix}
\]
,
则 $AH + HB = J = \mathbf{1}_{n}\mathbf{1}_{n}^{T}$. 于是
\begin{align*}
\mathbf{1}_{n}^{T}H^{-1}\mathbf{1}_{n} &= \mathrm{tr}\left(\mathbf{1}_{n}^{T}H^{-1}\mathbf{1}_{n}\right) = \mathrm{tr}\left(H^{-1}\mathbf{1}_{n}\mathbf{1}_{n}^{T}\right) = \mathrm{tr}\left(H^{-1}J\right) = \mathrm{tr}\left(H^{-1}(AH + HB)\right) \\
&= \mathrm{tr}\left(H^{-1}AH + B\right) = \mathrm{tr}(A) + \mathrm{tr}(B) = \frac{n(n + 1)}{2} + \frac{n(n - 1)}{2} = n^{2}.
\end{align*} 
\end{proof}





\subsection{个数的推广}

\begin{proposition}\label{proposition:多个矩阵乘法可交换必存在一个公共特征向量}
设数域\(\mathbb{F}\)上的\(n\)阶矩阵\(A_{1},A_{2},\cdots ,A_{m}\)两两乘法可交换, 且它们的特征值都在\(\mathbb{F}\)中, 求证: 它们在\(\mathbb{F}^n\)中至少有一个公共的特征向量.
\end{proposition}
\begin{proof}
对\(m\)进行归纳, \(m = 2\)时就是\hyperref[proposition:乘法可交换必有公共的特征向量]{命题\ref{proposition:乘法可交换必有公共的特征向量}}. 设矩阵个数小于\(m\)时结论成立, 现证\(m\)个矩阵的情形. 将所有的\(A_{i}\)都看成是列向量空间\(\mathbb{F}^n\)上的线性变换, 任取\(A_{1}\)的一个特征值\(\lambda_{1} \in \mathbb{F}\)及其特征子空间\(V_{1}\subseteq \mathbb{F}^n\). 注意到\(A_{1}A_{i}=A_{i}A_{1}\), 故由\hyperref[proposition:特征子空间互为不变子空间]{命题\ref{proposition:特征子空间互为不变子空间}}可知, \(V_{1}\)是\(A_{2},\cdots ,A_{m}\)的不变子空间. 将\(A_{2},\cdots ,A_{m}\)限制在\(V_{1}\)上, 它们仍然两两乘法可交换且特征值都在\(\mathbb{F}\)中, 故由归纳假设可得\(A_{2}|_{V_{1}},\cdots ,A_{m}|_{V_{1}}\)有公共的特征向量\(\alpha \in V_{1}\). 注意到\(\alpha\)也是\(A_{1}\)的特征向量, 于是\(\alpha\)是\(A_{1},A_{2},\cdots ,A_{m}\)的公共特征向量.
\end{proof}

\begin{proposition}\label{proposition:多个矩阵乘法可交换诱导同时上三角化}
设数域\(\mathbb{F}\)上的\(n\)阶矩阵\(A_{1},A_{2},\cdots ,A_{m}\)两两乘法可交换, 且它们的特征值都在\(\mathbb{F}\)中, 求证: 它们在\(\mathbb{F}\)上可同时上三角化, 即存在\(\mathbb{F}\)上的可逆矩阵\(P\), 使得\(P^{-1}A_{i}P(1\leqslant  i\leqslant  m)\)都是上三角矩阵.
\end{proposition}
\begin{proof}
完全类似于\hyperref[proposition:乘法可交换诱导同时上三角化]{命题\ref{proposition:乘法可交换诱导同时上三角化}}的证明, 其中利用\hyperref[proposition:多个矩阵乘法可交换必存在一个公共特征向量]{命题\ref{proposition:多个矩阵乘法可交换必存在一个公共特征向量}}得到\(A_{1},A_{2},\cdots ,A_{m}\)的公共特征向量, 请读者自行补充相关的细节.
\end{proof}

\begin{proposition}\label{proposition:多个矩阵乘法可交换诱导同时对角化}
设数域\(\mathbb{F}\)上的\(n\)阶矩阵\(A_{1},A_{2},\cdots ,A_{m}\)两两乘法可交换, 且它们都在\(\mathbb{F}\)上可对角化, 求证: 它们在\(\mathbb{F}\)上可同时对角化, 即存在\(\mathbb{F}\)上的可逆矩阵\(P\), 使得\(P^{-1}A_{i}P(1\leqslant  i\leqslant  m)\)都是对角矩阵.
\end{proposition}
\begin{proof}
若\(A_{i}\)都是纯量矩阵, 则结论显然成立. 以下不妨设\(A_{1}\)不是纯量矩阵, 余下的证明完全类似于\hyperref[proposition:乘法可交换诱导同时对角化]{命题\ref{proposition:乘法可交换诱导同时对角化}}的证明, 请读者自行补充相关的细节.
\end{proof}

\begin{example}
设\(A,B\)都是\(n\)阶矩阵且\(AB = BA\). 若\(A\)是幂零矩阵, 求证: \(|A + B| = |B|\).
\end{example}
\begin{proof}
{\color{blue}证法一:} 由\hyperref[proposition:乘法可交换诱导同时上三角化]{命题\ref{proposition:乘法可交换诱导同时上三角化}}可知, \(A,B\)可同时上三角化, 即存在可逆矩阵\(P\), 使得\(P^{-1}AP\)和\(P^{-1}BP\)都是上三角矩阵. 因为上三角矩阵的主对角元是矩阵的特征值, 而幂零矩阵的特征值全为零, 所以\(|P^{-1}AP + P^{-1}BP| = |P^{-1}BP|\), 即有\(|A + B| = |B|\).
    
{\color{blue}证法二:} 先假设\(B\)是可逆矩阵, 则\(|A + B| = |I_{n}+AB^{-1}||B|\), 只要证明\(|I_{n}+AB^{-1}| = 1\)即可. 由\(AB = BA\)可知\(AB^{-1}=B^{-1}A\), 再由\(A\)是幂零矩阵容易验证\(AB^{-1}\)也是幂零矩阵, 从而其特征值全为零. 因此\(I_{n}+AB^{-1}\)的特征值全为\(1\), 故\(|I_{n}+AB^{-1}| = 1\). 
    
对于一般的矩阵\(B\), 可取到一列有理数\(t_{k}\to 0\), 使得\(t_{k}I_{n}+B\)是可逆矩阵. 由可逆情形的证明可得\(|A + t_{k}I_{n}+B| = |t_{k}I_{n}+B|\). 注意到上式两边都是\(t_{k}\)的多项式, 从而关于\(t_{k}\)连续. 将上式两边同时取极限, 令\(t_{k}\to 0\), 即得结论.
\end{proof}









\end{document}
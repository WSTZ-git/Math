\documentclass[../../main.tex]{subfiles}
\graphicspath{{\subfix{../../image/}}} % 指定图片目录,后续可以直接使用图片文件名。

% 例如:
% \begin{figure}[h]
% \centering
% \includegraphics{image-01.01}
% \label{fig:image-01.01}
% \caption{图片标题}
% \end{figure}

\begin{document}

\section{由乘法交换性诱导的同时性质}

\begin{proposition}[矩阵乘法可交换的基本性质]\label{proposition:矩阵乘法可交换的基本性质}
若两个矩阵或线性变换$A,B$乘法可交换,即\(AB = BA\), 则有\((AB)^{m}=A^{m}B^{m}\), \(f(A)g(B)=g(B)f(A)\)以及二项式定理
\begin{align*}
(A + B)^{m}=A^{m}+\mathrm{C}_{m}^{1}A^{m - 1}B+\cdots+\mathrm{C}_{m}^{m - 1}AB^{m - 1}+B^{m}
\end{align*}
等成立, 其中\(m \geq 1\), \(f(x),g(x)\)为多项式. 

特别地,一个矩阵或线性变换$A$一定与其自身可交换,从而也满足\(f(A)g(A)=g(A)f(A)\),其中\(f(x),g(x)\)为多项式.
\end{proposition}
\begin{proof}
证明是显然的.
\end{proof}

\subsection{特征子空间互为不变子空间}

\begin{proposition}[特征子空间互为不变子空间]\label{proposition:特征子空间互为不变子空间}
\begin{enumerate}
\item 设\(\varphi,\psi\)是复线性空间\(V\)上乘法可交换的线性变换, 即\(\varphi\psi = \psi\varphi\), 求证: \(\varphi\)的特征子空间是\(\psi\)的不变子空间, \(\psi\)的特征子空间是\(\varphi\)的不变子空间.

\item 若\(n\)阶复矩阵\(A,B\)乘法可交换, 即\(AB = BA\), 则\(A,B\)的特征子空间互为不变子空间.
\end{enumerate}
\end{proposition}
\begin{remark}
这个命题的结论对一般的数域是不成立的. 例如, \(A = I_{2}\), \(B=\begin{pmatrix}
0& -1\\
1& 0
\end{pmatrix}\), 显然\(A,B\)乘法可交换, 但它们在有理数域或实数域上没有公共的特征向量. 事实上, \(B\)在有理数域或实数域上都没有特征值 (它的特征值是\(\pm\mathrm{i}\)), 从而也没有特征向量, 所以更谈不上公共的特征向量了. 为了这个命题的结论推广到数域\(\mathbb{F}\)上, 我们必须假设\(A,B\)的特征值都在\(\mathbb{F}\)中.
\end{remark}
\begin{proof}
\begin{enumerate}
\item 由代数基本定理以及线性方程组的求解理论可知, \(n(n \geq 1)\)维复线性空间上的线性变换或\(n\)阶复矩阵至少有一个特征值和特征向量. 任取线性变换\(\varphi\)的一个特征值\(\lambda_{0}\), 设\(V_{0}\)是特征值\(\lambda_{0}\)的特征子空间, 则对任意的\(\alpha \in V_{0}\), 有
\begin{align*}
\varphi\psi(\alpha)=\psi\varphi(\alpha)=\psi(\lambda_{0}\alpha)=\lambda_{0}\psi(\alpha),
\end{align*}
即\(\psi(\alpha) \in V_{0}\), 因此\(V_{0}\)是\(\psi\)的不变子空间. 同理可证\(\psi\)的特征子空间是\(\varphi\)的不变子空间.

\item 
\end{enumerate}
\end{proof}

\begin{proposition}\label{proposition:线性变换与任意线性变换乘法可交换必是纯量变换}
设\(V\)为\(n\)维复线性空间, \(S\)是\(\mathcal{L}(V)\)的非空子集, 满足: \(S\)中的全体线性变换没有非平凡的公共不变子空间. 设线性变换\(\varphi\)与\(S\)中任一线性变换乘法均可交换, 证明: \(\varphi\)是纯量变换.
\end{proposition}
\begin{proof}
任取\(\varphi\)的特征值\(\lambda_{0}\)及其特征子空间\(V_{0}\). 任取\(\psi \in S\), 则\(\varphi\psi = \psi\varphi\), 由\hyperref[proposition:特征子空间互为不变子空间]{命题\ref{proposition:特征子空间互为不变子空间}}可知\(V_{0}\)是\(\psi -\)不变子空间, 从而是\(S\)中全体线性变换的公共不变子空间. 又\(V_{0} \neq 0\)(特征向量均非零), 故\(V_{0}=V\), 从而\(\varphi=\lambda_{0}I_{V}\)为纯量变换. 
\end{proof}

\subsection{有公共的特征向量}

\begin{proposition}\label{proposition:乘法可交换必有公共的特征向量}
\begin{enumerate}
\item 设\(\varphi,\psi\)是复线性空间\(V\)上乘法可交换的线性变换, 求证: \(\varphi,\psi\)至少有一个公共的(复)特征向量.

\item 若\(n\)阶复矩阵\(A,B\)乘法可交换, 即\(AB = BA\), 则\(A,B\)至少有一个公共的(复)特征向量. 
\end{enumerate}
\end{proposition}
\begin{remark}
这个命题的结论对一般的数域是不成立的. 例如, \(A = I_{2}\), \(B=\begin{pmatrix}
0& -1\\
1& 0
\end{pmatrix}\), 显然\(A,B\)乘法可交换, 但它们在有理数域或实数域上没有公共的特征向量. 事实上, \(B\)在有理数域或实数域上都没有特征值 (它的特征值是\(\pm\mathrm{i}\)), 从而也没有特征向量, 所以更谈不上公共的特征向量了. 为了这个命题的结论推广到数域\(\mathbb{F}\)上, 我们必须假设\(A,B\)的特征值都在\(\mathbb{F}\)中.
\end{remark}
\begin{proof}
\begin{enumerate}
\item 任取\(\varphi\)的特征值\(\lambda_{0}\)及其特征子空间\(V_{0}\), 由\hyperref[proposition:特征子空间互为不变子空间]{命题\ref{proposition:特征子空间互为不变子空间}}可知, \(V_{0}\)是\(\psi -\)不变子空间. 将线性变换\(\psi\)限制在\(V_{0}\)上, 由于\(V_{0}\)是维数大于零的复线性空间, 故由\hyperref[proposition:线性变换或矩阵在复线性空间上至少存在一个特征值及与其特征向量]{命题\ref{proposition:线性变换或矩阵在复线性空间上至少存在一个特征值及与其特征向量}}可知\(\psi|_{V_{0}}\)至少有一个特征值\(\mu_{0}\)及其特征向量\(\alpha \in V_{0}\), 从而\(\varphi(\alpha)=\lambda_{0}\alpha\), \(\psi(\alpha)=\mu_{0}\alpha\), 于是\(\alpha\)就是\(\varphi,\psi\)的公共特征向量. 

\item 
\end{enumerate}
\end{proof}

\begin{proposition}\label{proposition:一般数域上乘法可交换诱导的性质}
\begin{enumerate}
\item 设\(\varphi,\psi\)是数域\(\mathbb{F}\)上线性空间\(V\)上的乘法可交换的线性变换, 且\(\varphi,\psi\)的特征值都在\(\mathbb{F}\)中, 求证: \(\varphi,\psi\)的特征子空间互为不变子空间, 并且\(\varphi,\psi\)至少有一个公共的特征向量.

\item 若数域\(\mathbb{F}\)上的\(n\)阶矩阵\(A,B\)乘法可交换, 且它们的特征值都在\(\mathbb{F}\)中, 则\(A,B\)的特征子空间互为不变子空间, 并且\(A,B\)在\(\mathbb{F}^n\)中至少有一个公共的特征向量.
\end{enumerate}
\end{proposition}
\begin{proof}
\begin{enumerate}
\item 由线性方程组的求解理论可知, 若数域\(\mathbb{F}\)上的线性变换或\(\mathbb{F}\)上的矩阵在\(\mathbb{F}\)中有一个特征值, 则在\(\mathbb{F}\)上的线性空间或\(\mathbb{F}\)上的列向量空间中必存在对应的特征向量. 任取线性变换\(\varphi\)的一个特征值\(\lambda_{0} \in \mathbb{F}\), 设\(V_{0}\)是特征值\(\lambda_{0}\)的特征子空间, 则对任意的\(\alpha \in V_{0}\), 有
\begin{align*}
\varphi\psi(\alpha)=\psi\varphi(\alpha)=\psi(\lambda_{0}\alpha)=\lambda_{0}\psi(\alpha),
\end{align*}
即\(\psi(\alpha) \in V_{0}\), 因此\(V_{0}\)是\(\psi -\)不变子空间. 取\(V_{0}\)的一组基并扩张为\(V\)的一组基, 则\(\psi\)在这组基下的表示矩阵为分块对角矩阵\(\begin{pmatrix}
A&C\\
O&B
\end{pmatrix}\), 其中\(A\)是\(\psi|_{V_{0}}\)在给定基下的表示矩阵, 于是\(|\lambda I_{V}-\psi| = |\lambda I - A||\lambda I - B|\). 因为\(\psi\)的特征值都在\(\mathbb{F}\)中, 故\(A\)的特征值都在\(\mathbb{F}\)中, 于是\(\psi|_{V_{0}}\)的特征值都在\(\mathbb{F}\)中. 任取\(\psi|_{V_{0}}\)的一个特征值\(\mu_{0} \in \mathbb{F}\)及其特征向量\(\alpha \in V_{0}\), 则\(\varphi(\alpha)=\lambda_{0}\alpha\), \(\psi(\alpha)=\mu_{0}\alpha\), 于是\(\alpha\)就是\(\varphi,\psi\)的公共特征向量. 

\item 
\end{enumerate}
\end{proof}



\subsection{可同时上三角化}

\begin{proposition}\label{proposition:特征值全在同一数域的矩阵可上三角化}
\begin{enumerate}
\item 设数域\(\mathbb{F}\)上的\(n\)阶矩阵\(A\)的特征值都在\(\mathbb{F}\)中, 求证: \(A\)在\(\mathbb{F}\)上可上三角化, 即存在\(\mathbb{F}\)上的可逆矩阵\(P\), 使得\(P^{-1}AP\)是上三角矩阵.
\item 设数域\(\mathbb{F}\)上线性空间\(V\)上的线性变换\(\varphi\)的特征值都在\(\mathbb{F}\)中, 则存在\(V\)的一组基, 使得\(\varphi\)在这组基下的表示矩阵是上三角矩阵. 
\end{enumerate}
\end{proposition}
\begin{proof}
\begin{enumerate}
\item 对阶数进行归纳. 当\(n = 1\)时结论显然成立, 设对\(n - 1\)阶矩阵结论成立, 现对\(n\)阶矩阵\(A\)进行证明. 设\(\lambda_{1} \in \mathbb{F}\)是\(A\)的一个特征值, 则由线性方程组的求解理论可知, 存在特征向量\(e_{1} \in \mathbb{F}^n\), 使得\(Ae_{1}=\lambda_{1}e_{1}\). 由基扩张定理, 可将\(e_{1}\)扩张为\(\mathbb{F}^n\)的一组基\(\{e_{1},e_{2},\cdots ,e_{n}\}\), 于是
\begin{align*}
(Ae_{1},Ae_{2},\cdots ,Ae_{n})=(e_{1},e_{2},\cdots ,e_{n})\begin{pmatrix}
\lambda_{1}& *\\
O&A_{1}
\end{pmatrix},
\end{align*}
其中\(A_{1}\)是\(\mathbb{F}\)上的\(n - 1\)阶矩阵. 令\(P=(e_{1},e_{2},\cdots ,e_{n})\), 则\(P\)是\(\mathbb{F}\)上的\(n\)阶可逆矩阵, 且由上式可得\(AP = P\begin{pmatrix}
\lambda_{1}& *\\
O&A_{1}
\end{pmatrix}\), 即\(P^{-1}AP=\begin{pmatrix}
\lambda_{1}& *\\
O&A_{1}
\end{pmatrix}\). 由此可得\(|\lambda I_{n}-A| = (\lambda - \lambda_{1})|\lambda I_{n - 1}-A_{1}|\), 又$A$的特征值全在$\mathbb{F}$中,从而\(A_{1}\)的特征值也全在\(\mathbb{F}\)中, 故由归纳假设, 存在\(\mathbb{F}\)上的\(n - 1\)阶可逆矩阵\(Q\), 使得\(Q^{-1}A_{1}Q\)是上三角矩阵. 令
\begin{align*}
R = P\begin{pmatrix}
1& O\\
O& Q
\end{pmatrix},
\end{align*}
则\(R\)是\(\mathbb{F}\)上的\(n\)阶可逆矩阵, 且
\begin{align*}
R^{-1}AR=\begin{pmatrix}
1& O\\
O& Q
\end{pmatrix}^{-1}\begin{pmatrix}
\lambda_{1}& *\\
O&A_{1}
\end{pmatrix}\begin{pmatrix}
1& O\\
O& Q
\end{pmatrix}=\begin{pmatrix}
\lambda_{1}& *\\
O&Q^{-1}A_{1}Q
\end{pmatrix}
\end{align*}
是上三角矩阵. 

\item 
\end{enumerate}
\end{proof}

\begin{proposition}\label{proposition:乘法可交换诱导同时上三角化}
\begin{enumerate}
\item 设\(A,B\)是数域\(\mathbb{F}\)上的\(n\)阶矩阵, 满足: \(AB = BA\)且\(A,B\)的特征值都在\(\mathbb{F}\)中, 求证: \(A,B\)在\(\mathbb{F}\)上可同时上三角化, 即存在\(\mathbb{F}\)上的可逆矩阵\(P\), 使得\(P^{-1}AP\)和\(P^{-1}BP\)都是上三角矩阵.

\item 设数域\(\mathbb{F}\)上线性空间\(V\)上的线性变换\(\varphi,\psi\)乘法可交换, 且它们的特征值都在\(\mathbb{F}\)中, 则存在\(V\)的一组基, 使得\(\varphi,\psi\)在这组基下的表示矩阵都是上三角矩阵. 
\end{enumerate}
\end{proposition}
\begin{proof}
\begin{enumerate}
\item 对阶数进行归纳. 当\(n = 1\)时结论显然成立, 设对\(n - 1\)阶矩阵结论成立, 现对\(n\)阶矩阵进行证明. 因为\(AB = BA\)且\(A,B\)的特征值都在\(\mathbb{F}\)中, 故由\hyperref[proposition:一般数域上乘法可交换诱导的性质]{命题\ref{proposition:一般数域上乘法可交换诱导的性质}}可知, \(A,B\)有公共的特征向量\(e_{1} \in \mathbb{F}^n\), 不妨设
\begin{align*}
Ae_{1}=\lambda_{1}e_{1},\ Be_{1}=\mu_{1}e_{1},
\end{align*}
其中\(\lambda_{1},\mu_{1} \in \mathbb{F}\)分别是\(A,B\)的特征值. 由基扩张定理, 可将\(e_{1}\)扩张为\(\mathbb{F}^n\)的一组基\(\{e_{1},e_{2},\cdots ,e_{n}\}\). 令\(P=(e_{1},e_{2},\cdots ,e_{n})\), 则\(P\)是\(\mathbb{F}\)上的\(n\)阶可逆矩阵, 从而有
\begin{align}\label{proposition-07-1.1}
\begin{aligned}
A\left( e_1,e_2,\cdots ,e_n \right) =\left( e_1,e_2,\cdots ,e_n \right) \left( \begin{matrix}
    \lambda _1&		*\\
    O&		A_1\\
\end{matrix} \right) \Leftrightarrow AP=P\left( \begin{matrix}
    \lambda _1&		*\\
    O&		A_1\\
\end{matrix} \right) \Leftrightarrow P^{-1}AP=\left( \begin{matrix}
    \lambda _1&		*\\
    O&		A_1\\
\end{matrix} \right) ,
\\
B\left( e_1,e_2,\cdots ,e_n \right) =\left( e_1,e_2,\cdots ,e_n \right) \left( \begin{matrix}
    \lambda _1&		*\\
    O&		B_1\\
\end{matrix} \right) \Leftrightarrow BP=P\left( \begin{matrix}
    \lambda _1&		*\\
    O&		B_1\\
\end{matrix} \right) \Leftrightarrow P^{-1}BP=\left( \begin{matrix}
    \lambda _1&		*\\
    O&		B_1\\
\end{matrix} \right) .
\end{aligned}
\end{align}
其中\(A_{1},B_{1}\)是\(\mathbb{F}\)上的\(n - 1\)阶矩阵. 由\(AB = BA\)及\eqref{proposition-07-1.1}式可得到
\begin{align*}
&\quad\,\, \left( P^{-1}AP \right) \left( P^{-1}BP \right) =P^{-1}ABP=P^{-1}BAP=\left( P^{-1}BP \right) \left( P^{-1}AP \right) 
\\
&\Leftrightarrow \left( \begin{matrix}
\lambda _1&		*\\
O&		A_1\\
\end{matrix} \right) \left( \begin{matrix}
\lambda _1&		*\\
O&		B_1\\
\end{matrix} \right) =\left( \begin{matrix}
\lambda _1&		*\\
O&		B_1\\
\end{matrix} \right) \left( \begin{matrix}
\lambda _1&		*\\
O&		A_1\\
\end{matrix} \right) 
\\
&\Leftrightarrow \left( \begin{matrix}
\lambda _1&		*\\
O&		A_1B_1\\
\end{matrix} \right) =\left( \begin{matrix}
\lambda _1&		*\\
O&		B_1A_1\\
\end{matrix} \right) 
\end{align*}
从而\(A_{1}B_{1} = B_{1}A_{1}\).又由\eqref{proposition-07-1.1}式可得
\begin{align*}
\left| \lambda I_n-A \right|=\left| \lambda -\lambda _1 \right|\left| \lambda I_{n-1}-A_1 \right|,\quad \left| \lambda I_n-B \right|=\left| \lambda -\lambda _1 \right|\left| \lambda I_{n-1}-B_1 \right|.
\end{align*}
因此\(A_{1},B_{1}\)的特征值也是$A,B$的特征值.又由于$A,B$的特征值都在\(\mathbb{F}\)中,故\(A_{1},B_{1}\)的特征值都在\(\mathbb{F}\)中. 故由归纳假设, 存在\(\mathbb{F}\)上的\(n - 1\)阶可逆矩阵\(Q\), 使得\(Q^{-1}A_{1}Q\)和\(Q^{-1}B_{1}Q\)都是上三角矩阵. 令
\begin{align*}
R = P\begin{pmatrix}
1& O\\
O& Q
\end{pmatrix},
\end{align*}
则\(R\)是\(\mathbb{F}\)上的\(n\)阶可逆矩阵, 且
\begin{align*}
R^{-1}AR&=\begin{pmatrix}
1& O\\
O& Q
\end{pmatrix}^{-1}\begin{pmatrix}
\lambda_{1}& *\\
O&A_{1}
\end{pmatrix}\begin{pmatrix}
1& O\\
O& Q
\end{pmatrix}=\begin{pmatrix}
\lambda_{1}& *\\
O&Q^{-1}A_{1}Q
\end{pmatrix},\\
R^{-1}BR&=\begin{pmatrix}
1& O\\
O& Q
\end{pmatrix}^{-1}\begin{pmatrix}
\mu_{1}& *\\
O&B_{1}
\end{pmatrix}\begin{pmatrix}
1& O\\
O& Q
\end{pmatrix}=\begin{pmatrix}
\mu_{1}& *\\
O&Q^{-1}B_{1}Q
\end{pmatrix}
\end{align*}
都是上三角矩阵.

\item 
\end{enumerate}
\end{proof}


\subsection{可同时对角化}

\begin{proposition}\label{proposition:乘法可交换诱导同时对角化}
\begin{enumerate}
\item 设\(\varphi,\psi\)是数域\(\mathbb{F}\)上\(n\)维线性空间\(V\)上的线性变换, 满足: \(\varphi\psi = \psi\varphi\)且\(\varphi,\psi\)都可对角化, 求证: \(\varphi,\psi\)可同时对角化, 即存在\(V\)的一组基, 使得\(\varphi,\psi\)在这组基下的表示矩阵都是对角矩阵.

\item 设\(A,B\)是数域\(\mathbb{F}\)上的\(n\)阶矩阵, 满足: \(AB = BA\)且\(A,B\)都在\(\mathbb{F}\)上可对角化, 则\(A,B\)在\(\mathbb{F}\)上可同时对角化, 即存在\(\mathbb{F}\)上的可逆矩阵\(P\), 使得\(P^{-1}AP\)和\(P^{-1}BP\)都是对角矩阵.
\end{enumerate}
\end{proposition}
\begin{proof}
\begin{enumerate}
\item 对空间维数进行归纳. 当\(n = 1\)时结论显然成立, 设对维数小于\(n\)的线性空间结论成立, 现对\(n\)维线性空间进行证明. 设\(\varphi\)的全体不同特征值为\(\lambda_{1},\cdots ,\lambda_{s} \in \mathbb{F}\), 对应的特征子空间分别为\(V_{1},\cdots ,V_{s}\), 则由\(\varphi\)可对角化可知
\begin{align*}
V = V_{1}\oplus\cdots\oplus V_{s}.
\end{align*}
若\(s = 1\), 则\(\varphi=\lambda_{1}I_{V}\)为纯量变换, 此时只要取\(V\)的一组基, 使得\(\psi\)在这组基下的表示矩阵为对角矩阵, 则\(\varphi\)在这组基下的表示矩阵为\(\lambda_{1}I_{n}\), 结论成立. 若\(s > 1\), 则\(\dim V_{i}< n\). 注意到\(\varphi\psi = \psi\varphi\)且\(\varphi,\psi\)的特征值都在\(\mathbb{F}\)中, 由\hyperref[proposition:特征子空间互为不变子空间]{命题\ref{proposition:特征子空间互为不变子空间}}可知\(V_{i}\)都是\(\psi -\)不变子空间. 考虑线性变换的限制\(\varphi|_{V_{i}},\psi|_{V_{i}}\): 它们乘法可交换, 且由可对角化线性变换的性质可知它们都可对角化, 故由归纳假设可知, \(\varphi|_{V_{i}},\psi|_{V_{i}}\)可同时对角化, 即存在\(V_{i}\)的一组基, 使得\(\varphi|_{V_{i}},\psi|_{V_{i}}\)在这组基下的表示矩阵都是对角矩阵. 将\(V_{i}\)的基拼成\(V\)的一组基, 则\(\varphi,\psi\)在这组基下的表示矩阵都是对角矩阵, 即\(\varphi,\psi\)可同时对角化.

\item 
\end{enumerate}
\end{proof}


\subsection{个数的推广}

\begin{proposition}\label{proposition:多个矩阵乘法可交换必存在一个公共特征向量}
设数域\(\mathbb{F}\)上的\(n\)阶矩阵\(A_{1},A_{2},\cdots ,A_{m}\)两两乘法可交换, 且它们的特征值都在\(\mathbb{F}\)中, 求证: 它们在\(\mathbb{F}^n\)中至少有一个公共的特征向量.
\end{proposition}
\begin{proof}
对\(m\)进行归纳, \(m = 2\)时就是\hyperref[proposition:乘法可交换必有公共的特征向量]{命题\ref{proposition:乘法可交换必有公共的特征向量}}. 设矩阵个数小于\(m\)时结论成立, 现证\(m\)个矩阵的情形. 将所有的\(A_{i}\)都看成是列向量空间\(\mathbb{F}^n\)上的线性变换, 任取\(A_{1}\)的一个特征值\(\lambda_{1} \in \mathbb{F}\)及其特征子空间\(V_{1}\subseteq \mathbb{F}^n\). 注意到\(A_{1}A_{i}=A_{i}A_{1}\), 故由\hyperref[proposition:特征子空间互为不变子空间]{命题\ref{proposition:特征子空间互为不变子空间}}可知, \(V_{1}\)是\(A_{2},\cdots ,A_{m}\)的不变子空间. 将\(A_{2},\cdots ,A_{m}\)限制在\(V_{1}\)上, 它们仍然两两乘法可交换且特征值都在\(\mathbb{F}\)中, 故由归纳假设可得\(A_{2}|_{V_{1}},\cdots ,A_{m}|_{V_{1}}\)有公共的特征向量\(\alpha \in V_{1}\). 注意到\(\alpha\)也是\(A_{1}\)的特征向量, 于是\(\alpha\)是\(A_{1},A_{2},\cdots ,A_{m}\)的公共特征向量.
\end{proof}

\begin{proposition}\label{proposition:多个矩阵乘法可交换诱导同时上三角化}
设数域\(\mathbb{F}\)上的\(n\)阶矩阵\(A_{1},A_{2},\cdots ,A_{m}\)两两乘法可交换, 且它们的特征值都在\(\mathbb{F}\)中, 求证: 它们在\(\mathbb{F}\)上可同时上三角化, 即存在\(\mathbb{F}\)上的可逆矩阵\(P\), 使得\(P^{-1}A_{i}P(1\leq i\leq m)\)都是上三角矩阵.
\end{proposition}
\begin{proof}
完全类似于\hyperref[proposition:乘法可交换诱导同时上三角化]{命题\ref{proposition:乘法可交换诱导同时上三角化}}的证明, 其中利用\hyperref[proposition:多个矩阵乘法可交换必存在一个公共特征向量]{命题\ref{proposition:多个矩阵乘法可交换必存在一个公共特征向量}}得到\(A_{1},A_{2},\cdots ,A_{m}\)的公共特征向量, 请读者自行补充相关的细节.
\end{proof}

\begin{proposition}\label{proposition:多个矩阵乘法可交换诱导同时对角化}
设数域\(\mathbb{F}\)上的\(n\)阶矩阵\(A_{1},A_{2},\cdots ,A_{m}\)两两乘法可交换, 且它们都在\(\mathbb{F}\)上可对角化, 求证: 它们在\(\mathbb{F}\)上可同时对角化, 即存在\(\mathbb{F}\)上的可逆矩阵\(P\), 使得\(P^{-1}A_{i}P(1\leq i\leq m)\)都是对角矩阵.
\end{proposition}
\begin{proof}
若\(A_{i}\)都是纯量矩阵, 则结论显然成立. 以下不妨设\(A_{1}\)不是纯量矩阵, 余下的证明完全类似于\hyperref[proposition:乘法可交换诱导同时对角化]{命题\ref{proposition:乘法可交换诱导同时对角化}}的证明, 请读者自行补充相关的细节.
\end{proof}

\begin{example}
设\(A,B\)都是\(n\)阶矩阵且\(AB = BA\). 若\(A\)是幂零矩阵, 求证: \(|A + B| = |B|\).
\end{example}
\begin{proof}
{\color{blue}证法一:} 由\hyperref[proposition:乘法可交换诱导同时上三角化]{命题\ref{proposition:乘法可交换诱导同时上三角化}}可知, \(A,B\)可同时上三角化, 即存在可逆矩阵\(P\), 使得\(P^{-1}AP\)和\(P^{-1}BP\)都是上三角矩阵. 因为上三角矩阵的主对角元是矩阵的特征值, 而幂零矩阵的特征值全为零, 所以\(|P^{-1}AP + P^{-1}BP| = |P^{-1}BP|\), 即有\(|A + B| = |B|\).
    
{\color{blue}证法二:} 先假设\(B\)是可逆矩阵, 则\(|A + B| = |I_{n}+AB^{-1}||B|\), 只要证明\(|I_{n}+AB^{-1}| = 1\)即可. 由\(AB = BA\)可知\(AB^{-1}=B^{-1}A\), 再由\(A\)是幂零矩阵容易验证\(AB^{-1}\)也是幂零矩阵, 从而其特征值全为零. 因此\(I_{n}+AB^{-1}\)的特征值全为\(1\), 故\(|I_{n}+AB^{-1}| = 1\). 
    
对于一般的矩阵\(B\), 可取到一列有理数\(t_{k}\to 0\), 使得\(t_{k}I_{n}+B\)是可逆矩阵. 由可逆情形的证明可得\(|A + t_{k}I_{n}+B| = |t_{k}I_{n}+B|\). 注意到上式两边都是\(t_{k}\)的多项式, 从而关于\(t_{k}\)连续. 将上式两边同时取极限, 令\(t_{k}\to 0\), 即得结论.
\end{proof}







\end{document}
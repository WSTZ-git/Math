\documentclass[../../main.tex]{subfiles}
\graphicspath{{\subfix{../../image/}}} % 指定图片目录,后续可以直接使用图片文件名。

% 例如:
% \begin{figure}[H]
% \centering
% \includegraphics[scale=0.4]{图.png}
% \caption{}
% \label{figure:图}
% \end{figure}
% 注意:上述\label{}一定要放在\caption{}之后,否则引用图片序号会只会显示??.

\begin{document}

\section{一般数域上的Jordan标准型}

遇到数域$\mathbb{K}$上的相似问题,一般来说,可以有3种处理方法. 

第一种方法是先将问题转化成几何语言,再利用线性变换理论进行研究;

第二种方法是先将问题转化成代数语言,再把数域$\mathbb{K}$上的矩阵自然地看成是复矩阵进行研究,最后利用高等代数中若干概念在基域扩张下的不变性(例如,\hyperref[corollary:矩阵的相似关系在基域扩张下不变]{相似在基域扩张下的不变性}) 将所得结果返回到数域$\mathbb{K}$上;

第三种方法是利用一般数域上基于初等因子的相似标准型理论对问题进行研究. 

\begin{proposition}\label{proposition:数域K上关于核与值域的直和分解的充要条件}
设$V$是数域$\mathbb{K}$上的$n$维线性空间,$\varphi$是$V$上秩小于$n$的线性变换,求证:$V=\mathrm{Ker}\varphi\oplus\mathrm{Im}\varphi$的充要条件是$0$是$\varphi$的极小多项式的单根.
\end{proposition}
\begin{note}
分析:当$\mathbb{K}=\mathbb{C}$时,可以利用Jordan标准型理论进行证明. 若特征值$0$是$\varphi$的极小多项式的单根,则可设$\varphi$的初等因子组为$\lambda,\cdots,\lambda,(\lambda - \lambda_1)^{r_1},\cdots,(\lambda - \lambda_s)^{r_s}$,其中$\lambda_1,\cdots,\lambda_s$是非零特征值, 且有$k$个$\lambda$. 因此, 存在$V$的一组基$e_1,\cdots,e_k,e_{k + 1},\cdots,e_n$,使得$\varphi$在这组基下的表示矩阵为$\mathrm{diag}\{0,\cdots,0,J_{r_1}(\lambda_1),\cdots,J_{r_s}(\lambda_s)\}$.容易验证$\mathrm{Ker}\varphi = L(e_1,\cdots,e_k)$,$\mathrm{Im}\varphi = L(e_{k + 1},\cdots,e_n)$,于是$V=\mathrm{Ker}\varphi\oplus\mathrm{Im}\varphi$.反之,若$V=\mathrm{Ker}\varphi\oplus\mathrm{Im}\varphi$,则$\mathrm{Ker}\varphi\cap\mathrm{Im}\varphi = 0$,由\refpro{proposition:线性变换可对角化的几何充要条件1}的充分性的证明可知,$\varphi$关于特征值$0$的Jordan块都是一阶的,因此$0$是$\varphi$的极小多项式的单根. 然而,当$\mathbb{K}\neq\mathbb{C}$时,上述讨论就不再适用了,并且本题的结论也不能简单地延拓到复数域上,因为$V=\mathrm{Ker}\varphi\oplus\mathrm{Im}\varphi$是数域$\mathbb{K}$上线性空间的直和分解,一般并不能看成是复数域上线性空间的直和分解. 接下去让我们来看前两种方法是如何巧妙地解决问题的.
\end{note}
\begin{proof}
{\color{blue}证法一:}若$V=\mathrm{Ker}\varphi\oplus\mathrm{Im}\varphi$,则由\refpro{proposition:像空间和核空间的直和分解}可知, $\mathrm{Ker}\varphi=\mathrm{Ker}\varphi^2=\cdots$. 设$\varphi$的极小多项式$m(\lambda)=\lambda^kg(\lambda)$,其中$g(0)\neq 0$,我们来证明$k = 1$. 用反证法,假设$k\geqslant 2$,则对任意的$\alpha\in V$,有$\varphi^kg(\varphi)(\alpha)=0$,从而$g(\varphi)(\alpha)\in\mathrm{Ker}\varphi^k=\mathrm{Ker}\varphi$,于是$\varphi g(\varphi)(\alpha)=0$对任意的$\alpha\in V$成立, 即$\varphi g(\varphi)=0$,因此$\varphi$适合多项式$\lambda g(\lambda)$,其次数比极小多项式的次数还小,这就导出了矛盾. 反之,设$\varphi$的极小多项式$m(\lambda)=\lambda g(\lambda)$,其中$g(0)\neq 0$,则由\refpro{corollary:矩阵适合多项式诱导的结论}可知,$V = V_1\oplus V_2$,其中$V_1=\mathrm{Ker}\varphi=\mathrm{Im}g(\varphi)$,$V_2=\mathrm{Ker}g(\varphi)=\mathrm{Im}\varphi$,于是$V=\mathrm{Ker}\varphi\oplus\mathrm{Im}\varphi$.

{\color{blue}证法二:}由\refpro{proposition:像空间和核空间的直和分解}可知,$V=\mathrm{Ker}\varphi\oplus\mathrm{Im}\varphi$当且仅当$\mathrm{r}(\varphi)=\mathrm{r}(\varphi^2)$,因此我们只要证明: $\mathrm{r}(\varphi)=\mathrm{r}(\varphi^2)$当且仅当$0$是$\varphi$的极小多项式的单根. 任取$\varphi$在某组基下的表示矩阵$A$,则上述问题的代数版本是: $\mathrm{r}(A)=\mathrm{r}(A^2)$当且仅当$0$是$A$的极小多项式的单根. 注意到数域$\mathbb{K}$上的矩阵可自然地看成是复矩阵,并且矩阵的秩和极小多项式在基域扩张下不改变,因此我们可以把$A$当作复矩阵进行证明 (即本题分析中的讨论,其中用\refpro{proposition:线性变换可对角化的代数充要条件2}替代\refpro{proposition:线性变换可对角化的几何充要条件1}的引用),具体细节请读者自行完成.

{\color{blue}证法三:}
设$\varphi$在$\mathbb{K}$上的初等因子为$\lambda^{r_1},\cdots,\lambda^{r_k},P_1(\lambda)^{e_1},\cdots,P_t(\lambda)^{e_t}$,其中$P_1(\lambda),\cdots,P_t(\lambda)$是$\mathbb{K}$上常数项非零的不可约多项式,则由\refthe{theorem:基于初等因子组的有理标准型}或\refpro{proposition:广义标准型理论相关结论1}可知,存在$V$的一组基$\{e_1,e_2,\cdots,e_n\}$,使得$\varphi$在这组基下的表示矩阵为
\begin{align*}
&\mathrm{diag}\{F(\lambda^{r_1}),\cdots,F(\lambda^{r_k}),F(P_1(\lambda)^{e_1}),\cdots,F(P_t(\lambda)^{e_t})\}\text{ 或 }\\
&\mathrm{diag}\{J_{r_1}(0),\cdots,J_{r_k}(0),J_{e_1}(P_1(\lambda)),\cdots,J_{e_t}(P_t(\lambda))\}.
\end{align*}
若特征值$0$是$\varphi$的极小多项式的单根, 则$r_1 = \cdots = r_k = 1$,容易验证$\mathrm{Ker}\varphi = L(e_1,\cdots,e_k)$,$\mathrm{Im}\varphi = L(e_{k + 1},\cdots,e_n)$,从而$V = \mathrm{Ker}\varphi\oplus\mathrm{Im}\varphi$. 反之,若$V = \mathrm{Ker}\varphi\oplus\mathrm{Im}\varphi$,则$\mathrm{Ker}\varphi\cap\mathrm{Im}\varphi = 0$. 若存在某个$r_i > 1$,比如说$r_1 > 1$,则由\refpro{proposition:线性变换可对角化的几何充要条件1}的充分性完全类似的证明可知,$0\neq e_1\in\mathrm{Ker}\varphi\cap\mathrm{Im}\varphi$,这就推出了矛盾. 因此,$r_1 = \cdots = r_k = 1$,从而$0$是$\varphi$的极小多项式的单根. 

\end{proof}

\begin{proposition}\label{proposition:数域K上矩阵的相似(幂零和可逆)分解}
设$A$是数域$\mathbb{K}$上的$n$阶矩阵,求证: $A$相似于$\mathrm{diag}\{B,C\}$,其中$B$是$\mathbb{K}$上的幂零矩阵,$C$是$\mathbb{K}$上的可逆矩阵.
\end{proposition}
\begin{remark}
分析: 本题是\refpro{proposition:矩阵相似于分块幂零,可逆块}的推广, 即将复数域上的结论推广到数域$\mathbb{K}$上. 不过,\refpro{proposition:矩阵相似于分块幂零,可逆块}的证明利用了Jordan标准型理论,显然在数域$\mathbb{K}$上不再适用. 通常当我们考虑线性变换的问题时,数域都是事先给定的,从而在讨论的过程中不会涉及数域的问题. 因此我们可用第一种方法来处理本题,即把代数问题转化成几何问题,然后再用线性变换理论加以解决. 本题的几何版本为: 设$V$是数域$\mathbb{K}$上的$n$维线性空间,$\varphi$是$V$上的线性变换,证明: $V = V_1\oplus V_2$,其中$V_1,V_2$都是$\varphi$-不变子空间,且$\varphi|_{V_1}$是幂零线性变换,$\varphi|_{V_2}$是可逆线性变换. 我们可用两种几何方法来证明这一结论.
\end{remark}
\begin{proof}
{\color{blue}证法一:}设$\varphi$的特征多项式为$f(\lambda)=\lambda^kg(\lambda)$,其中$0\leqslant k\leqslant n$,$g(0)\neq 0$. 注意到$(\lambda^k,g(\lambda)) = 1$,故由\refpro{proposition:Cayley-Hamilton定理诱导直和分解(互素多项式命题推广)}可知,$V = V_1\oplus V_2$,其中$V_1=\mathrm{Ker}\varphi^k$,$V_2=\mathrm{Ker}g(\varphi)$,并且$\varphi|_{V_1}$的特征多项式是$\lambda^k$,$\varphi|_{V_2}$的特征多项式是$g(\lambda)$. 因此,$\varphi|_{V_1}$是幂零线性变换,且由$\varphi|_{V_2}$的行列式值为$(-1)^{n - k}g(0)\neq 0$可知,$\varphi|_{V_2}$是可逆线性变换.

{\color{blue}证法二:}由\refpro{proposition:线性映射像空间与和空间等式链}可知,存在整数$m\in[0,n]$,使得
\[V=\mathrm{Ker}\varphi^m\oplus\mathrm{Im}\varphi^m,\quad \mathrm{Ker}\varphi^m=\mathrm{Ker}\varphi^{m + 1}=\cdots,\quad \mathrm{Im}\varphi^m=\mathrm{Im}\varphi^{m + 1}=\cdots.\]
令$V_1=\mathrm{Ker}\varphi^m$,$V_2=\mathrm{Im}\varphi^m$,则$V = V_1\oplus V_2$. 因为$V_1=\mathrm{Ker}\varphi^m$,所以$\varphi|_{V_1}$适合多项式$\lambda^m$,从而它是幂零线性变换. 因为$\varphi|_{V_2}$的像空间是$\varphi(\mathrm{Im}\varphi^m)=\mathrm{Im}\varphi^{m + 1}=\mathrm{Im}\varphi^m$,所以$\varphi|_{V_2}$是满映射,从而它是可逆线性变换. 

{\color{blue}证法三:}
设$A$在$\mathbb{K}$上的初等因子为$\lambda^{r_1},\cdots,\lambda^{r_k},P_1(\lambda)^{e_1},\cdots,P_t(\lambda)^{e_t}$,其中$P_1(\lambda),\cdots,P_t(\lambda)$是$\mathbb{K}$上常数项非零的不可约多项式,则由\refthe{theorem:基于初等因子组的有理标准型}或\refpro{proposition:广义标准型理论相关结论1}可知,$A$在$\mathbb{K}$上相似于分块对角矩阵
\begin{align*}
&\mathrm{diag}\{F(\lambda^{r_1}),\cdots,F(\lambda^{r_k}),F(P_1(\lambda)^{e_1}),\cdots,F(P_t(\lambda)^{e_t})\}\text{ 或 }\\
&\mathrm{diag}\{J_{r_1}(0),\cdots,J_{r_k}(0),J_{e_1}(P_1(\lambda)),\cdots,J_{e_t}(P_t(\lambda))\}.
\end{align*}
令$B = \mathrm{diag}\{F(\lambda^{r_1}),\cdots,F(\lambda^{r_k})\}$或$\mathrm{diag}\{J_{r_1}(0),\cdots,J_{r_k}(0)\}$,$C = \mathrm{diag}\{F(P_1(\lambda)^{e_1}),\cdots,F(P_t(\lambda)^{e_t})\}$或$\mathrm{diag}$ $\{$ $J_{e_1}$ $(P_1(\lambda))$,$\cdots$ ,$J_{e_t}$ $(P_t(\lambda))$ $\}$,则由每个$F(\lambda^{r_i})$或$J_{r_i}(0)$都幂零可知$B$是幂零矩阵,由每个$F(P_j(\lambda)^{e_j})$或$J_{e_j}(P_j(\lambda))$的行列式的绝对值为$P_j(0)^{e_j}\neq 0$可知$C$是可逆矩阵,因此结论成立.

\end{proof}

\vspace{0.5cm}

上面只是比较简单的两道例题,如果希望能更一般地处理数域$\mathbb{K}$上的相似问题,那么我们可以运用数域$\mathbb{K}$上基于初等因子的相似标准型理论. 事实上,\refthe{theorem:基于初等因子组的有理标准型}已经给出\textbf{数域$\mathbb{K}$上基于初等因子的有理标准型},接下去我们将给出数域$\mathbb{K}$上基于初等因子的Jordan标准型. 这一理论跟之前阐述的数域$\mathbb{K}$上基于不变因子的有理标准型理论和复数域上的Jordan标准型理论之间有着密切的联系,无论是从引入的方法,还是从最终的结论来看,这一理论都是前面两种理论的自然延续和推广,因此不妨称之为\textbf{实数域上的广义Jordan标准型理论}. 

\vspace{0.5cm}

\begin{definition}
用$C_m$表示第$(m,1)$元素为$1$,其他元素全为零的$m$阶矩阵:
\begin{align*}
C_m = 
\begin{pmatrix}
0 & 0 & 0 & \cdots & 0 \\
0 & 0 & 0 & \cdots & 0 \\
\vdots & \vdots & \vdots & & \vdots \\
0 & 0 & 0 & \cdots & 0 \\
1 & 0 & 0 & \cdots & 0
\end{pmatrix}.
\end{align*}
\end{definition}

\begin{proposition}\label{proposition:广义标准型相关结论}
设$P(\lambda)=\lambda^m + a_1\lambda^{m - 1}+\cdots + a_{m - 1}\lambda + a_m$是$\mathbb{K}$上的首一不可约多项式,$e$是正整数,证明下列矩阵的不变因子组均为$1,\cdots,1,P(\lambda)^e$:
\begin{enumerate}[(1)]
\item $J_e(P(\lambda)) = 
\begin{pmatrix}
F(P(\lambda)) & I_m & O & \cdots & O & O \\
O & F(P(\lambda)) & I_m & \cdots & O & O \\
\vdots & \vdots & \vdots & & \vdots & \vdots \\
O & O & O & \cdots & F(P(\lambda)) & I_m \\
O & O & O & \cdots & O & F(P(\lambda))
\end{pmatrix}$

\item $\widetilde{J}_e(P(\lambda)) = 
\begin{pmatrix}
F(P(\lambda)) & C_m & O & \cdots & O & O \\
O & F(P(\lambda)) & C_m & \cdots & O & O \\
\vdots & \vdots & \vdots & & \vdots & \vdots \\
O & O & O & \cdots & F(P(\lambda)) & C_m \\
O & O & O & \cdots & O & F(P(\lambda))
\end{pmatrix}.$
\end{enumerate}
\end{proposition}
\begin{proof}
(1) 由有理标准型理论可知, $F(P(\lambda))$的特征多项式和极小多项式都是$P(\lambda)$,故$J_e(P(\lambda))$的特征多项式为$P(\lambda)^e$,从而$J_e(P(\lambda))$的极小多项式为$P(\lambda)^l$,其中$1\leqslant l\leqslant e$. 下面验证$J_e(P(\lambda))$不适合$P(\lambda)^{e - 1}$,从而$J_e(P(\lambda))$的极小多项式必为$P(\lambda)^e$. 以下简记$g(\lambda)=P(\lambda)^{e - 1}$,$F = F(P(\lambda))$,则通过分块矩阵的计算可得
\begin{align*}
g(J_e(P(\lambda))) = 
\begin{pmatrix}
g(F) & \frac{1}{1!}g'(F) & \frac{1}{2!}g^{(2)}(F) & \cdots & \frac{1}{(e - 1)!}g^{(e - 1)}(F) \\
 & g(F) & \frac{1}{1!}g'(F) & \cdots & \frac{1}{(e - 2)!}g^{(e - 2)}(F) \\
 & & g(F) & \cdots & \frac{1}{(e - 3)!}g^{(e - 3)}(F) \\
 & & & \ddots & \vdots \\
 & & & & g(F)
\end{pmatrix}.
\end{align*}
由Cayley - Hamilton定理可得$P(F)=O$,从而$g^{(i)}(F)=O (0\leqslant i\leqslant e - 2)$,但$g^{(e - 1)}(F)=(e - 1)!P'(F)^{e - 1}$. 由于$P(\lambda)$是不可约多项式,故$(P(\lambda),P'(\lambda)) = 1$,进一步有$(P(\lambda),P'(\lambda)^{e - 1}) = 1$,从而由\refpro{proposition:g(A)可逆与A的特征多项式与极小多项式的关系}可知,$P'(F)^{e - 1}$是可逆矩阵,于是$\frac{1}{(e - 1)!}g^{(e - 1)}(F)=P'(F)^{e - 1}\neq O$,即有$g(J_e(P(\lambda)))\neq O$. 因此$J_e(P(\lambda))$的极小多项式为$P(\lambda)^e$,其不变因子组为$1,\cdots,1,P(\lambda)^e$.

(2) 我们来计算$\widetilde{J}_e(P(\lambda))$的$me - 1$阶行列式因子, 注意到特征矩阵$\lambda I-\widetilde{J}_e(P(\lambda))$的前$me - 1$行、后$me - 1$列构成的$me - 1$阶子式是一个主对角元全为$-1$的下三角行列式,其值为$(-1)^{me - 1}$,故$\widetilde{J}_e(P(\lambda))$的$me - 1$阶行列式因子为$1$. 又$\widetilde{J}_e(P(\lambda))$的$me$阶行列式因子为$P(\lambda)^e$,故其行列式因子组为$1,\cdots,1,P(\lambda)^e$,从而不变因子组也为$1,\cdots,1,P(\lambda)^e$.

\end{proof}

\begin{proposition}\label{proposition:广义标准型理论相关结论1}
设$A$是$\mathbb{K}$上的$n$阶矩阵,它在$\mathbb{K}$上的初等因子组为$P_1(\lambda)^{e_1},P_2(\lambda)^{e_2},\cdots,P_t(\lambda)^{e_t}$,其中$P_i(\lambda)$是$\mathbb{K}$上的首一不可约多项式,$e_i\geqslant 1, 1\leqslant i\leqslant t$,证明$A$在$\mathbb{K}$上相似于下列分块对角矩阵:
\begin{enumerate}[(1)]
\item $J = \mathrm{diag}\{J_{e_1}(P_1(\lambda)),J_{e_2}(P_2(\lambda)),\cdots,J_{e_t}(P_t(\lambda))\};$

\item $\widetilde{J} = \mathrm{diag}\{\widetilde{J}_{e_1}(P_1(\lambda)),\widetilde{J}_{e_2}(P_2(\lambda)),\cdots,\widetilde{J}_{e_t}(P_t(\lambda))\}.$
\end{enumerate}
\end{proposition}
\begin{remark}
这个\refpro{proposition:广义标准型理论相关结论1}中的$J$和$\widetilde{J}$均称为数域$\mathbb{K}$上基于初等因子的实数域上的广义Jordan标准型. 当$\mathbb{K}=\mathbb{C}$时,注意到不可约多项式都是一次的,故可设$P(\lambda)=\lambda - \lambda_0$,则\refpro{proposition:广义标准型相关结论}中的广义Jordan块$J_e(P(\lambda))$和$\widetilde{J}_e(P(\lambda))$都变成了复数域上的Jordan块$J_e(\lambda_0)$,实数域上的广义Jordan标准型$J$和$\widetilde{J}$都变成了复数域上的Jordan标准型. $J$和$\widetilde{J}$之间的区别只是形式上的,即对每个广义Jordan块而言,其上次对角线上的矩阵一个是单位矩阵$I_m$,一个是矩阵$C_m$. 从本质上看,这两种实数域上的广义Jordan标准型其实是一致的,只不过在一些具体问题的讨论中,各有各的用途而已. 
\end{remark}
\begin{proof}
将$\lambda I - J$和$\lambda I - \widetilde{J}$按照每个分块依次进行$\lambda$-矩阵的初等变换,由\hyperref[theorem:lambda-矩阵和初等因子的基本性质]{$\lambda$-矩阵和初等因子的基本性质}可知,上述两个矩阵都相抵于
\begin{align*}
\mathrm{diag}\{1,\cdots,1,P_1(\lambda)^{e_1};1,\cdots,1,P_2(\lambda)^{e_2};\cdots;1,\cdots,1,P_t(\lambda)^{e_t}\}.
\end{align*}
由\hyperref[theorem:lambda-矩阵和初等因子的基本性质]{$\lambda$-矩阵和初等因子的基本性质(2)}可知,$J$和$\widetilde{J}$的初等因子组都是$P_1(\lambda)^{e_1},P_2(\lambda)^{e_2},\cdots,P_t(\lambda)^{e_t}$,即它们与$A$在$\mathbb{K}$上有相同的初等因子组,因此它们与$A$在$\mathbb{K}$上相似. 

\end{proof}

\begin{theorem}[实数域上的广义Jordan标准型]\label{theorem:实数域上的广义Jordan标准型}
设 $A$ 是实数域$\mathbb{R}$上的 $n$ 阶矩阵,证明 $A$ 在实数域$\mathbb{R}$上相似于下列分块对角矩阵:

(1) $J = \mathrm{diag}\{J_{r_1}(\lambda_1),\cdots,J_{r_k}(\lambda_k),J_{s_1}(a_1,b_1),\cdots,J_{s_l}(a_l,b_l)\}$;

(2) $\widetilde{J} = \mathrm{diag}\{J_{r_1}(\lambda_1),\cdots,J_{r_k}(\lambda_k),\widetilde{J}_{s_1}(a_1,b_1),\cdots,\widetilde{J}_{s_l}(a_l,b_l)\}$,

其中 $\lambda_1,\cdots,\lambda_k,a_1,b_1,\cdots,a_l,b_l$ 都是实数,$b_1,\cdots,b_l$ 都非零,并且$\lambda_j$都是$A$的实特征值,$a_j\pm \mathrm{i}b_j$都是$A$的复特征值,$J_{r_i}(\lambda_i)$ 表示以 $\lambda_i$ 为特征值的通常意义下的 Jordan 块,并且
\begin{gather*}
c_j=-2a_j,d_j=a_j^2+b_j^2,
\\
R_j = \left( \begin{matrix}
a_j&		b_j\\
-b_j&		a_j\\
\end{matrix} \right) \text{或}\left( \begin{matrix}
0&		-d_j\\
1&		-c_j\\
\end{matrix} \right) \text{或}\left( \begin{matrix}
0&		1\\
-d_j&		-c_j\\
\end{matrix} \right) 
,C_2 = \begin{pmatrix}0 & 0 \\ 1 & 0\end{pmatrix},
\\
J_{s_j}(a_j,b_j) = 
\begin{pmatrix}
R_j & I_2 & & \\
& R_j & I_2 & \\
& & \ddots & \ddots \\
& & & R_j & I_2 \\
& & & & R_j
\end{pmatrix}, \quad
\widetilde{J}_{s_j}(a_j,b_j) = 
\begin{pmatrix}
R_j & C_2 & & \\
& R_j & C_2 & \\
& & \ddots & \ddots \\
& & & R_j & C_2 \\
& & & & R_j
\end{pmatrix}.
\end{gather*}
从而$J_{r_j}(\lambda_j)$的特征多项式等于极小多项式等于$(x-\lambda_j)^{r_j}$.

$R_j$的特征多项式都等于极小多项式都等于
\begin{align*}
x^2+c_jx+d_j=(x-a_j)^2+b_j^2=\left( x-a_j-\mathrm{i}b_j \right) \left( x-a_j+\mathrm{i}b_j \right).
\end{align*}

$J_{s_j}(a_j,b_j)$和$\widetilde{J}_{s_j}(a_j,b_j)$的特征多项式都等于极小多项式都等于
\begin{align*}
(x^2+c_jx+d_j)^{\frac{s_j}{2}}=((x-a_j)^2+b_{j}^{2})^{\frac{s_j}{2}}=\left[ \left( x-a_j-\mathrm{i}b_j \right) \left( x-a_j+\mathrm{i}b_j \right) \right] ^{\frac{s_j}{2}}.
\end{align*}
\end{theorem}
\begin{note}
将定理中的$I_2$换成$kI_2(\forall k\in \mathbb{R}\text{且}k\ne 0)$,$C_2$换成$kC_2(\forall k\in \mathbb{R}\text{且}k\ne 0)$结论仍成立.因为替换后得到的$J'_{s_j}(a_j,b_j),\widetilde{J}'_{s_j}(a_j,b_j)$的特征值多项式不变,并且特征值对应的几何重数也不变,所以仍与原来的$J_{s_j}(a_j,b_j),\widetilde{J}_{s_j}(a_j,b_j)$相似.
\end{note}
\begin{proof}
注意到实数域上的不可约多项式是一次多项式或者是判别式小于零的二次多项式,故可设 $A$ 的初等因子组为 $(\lambda - \lambda_1)^{r_1},\cdots,(\lambda - \lambda_k)^{r_k},((\lambda - a_1)^2 + b_1^2)^{s_1},\cdots,((\lambda - a_l)^2 + b_l^2)^{s_l}$,其中 $\lambda_1,\cdots,\lambda_k,a_1,b_1,\cdots,a_l,b_l$ 都是实数,且 $b_1,\cdots,b_l$ 都非零。

(1) 由\hyperref[proposition:广义标准型理论相关结论1]{命题\ref{proposition:广义标准型理论相关结论1}(1)}可知,$A$ 实相似于 $\mathrm{diag}\{J_{r_1}(\lambda_1),\cdots,J_{r_k}(\lambda_k),J_{s_1}((\lambda - a_1)^2 + b_1^2),\cdots,J_{s_l}((\lambda - a_l)^2 + b_l^2)\}$,注意到 
$$
F((\lambda - a_j)^2 + b_j^2) = \begin{pmatrix}0 & 1 \\ -(a_j^2 + b_j^2) & 2a_j\end{pmatrix}=\left( \begin{matrix}
0&		-d_j\\
1&		-c_j\\
\end{matrix} \right) =\left( \begin{matrix}
0&		1\\
-d_j&		-c_j\\
\end{matrix} \right) 
$$
与 $R_j = \begin{pmatrix}a_j & b_j \\ -b_j & a_j\end{pmatrix}$ 有相同的特征值 $a_j\pm \mathrm{i}b_j$,故它们在复数域上,从而也在实数域上相似。因为 $J_{s_j}((\lambda - a_j)^2 + b_j^2)$ 的上次对角线都是 $I_2$,所以不难把这种相似关系扩张到整个广义 Jordan 块上,从而 $J_{s_j}((\lambda - a_j)^2 + b_j^2)$ 实相似于 $J_{s_j}(a_j,b_j)$,于是 $A$ 实相似于 $J$。

(2) 因为 $\widetilde{J}_{s_j}((\lambda - a_j)^2 + b_j^2)$ 的上次对角线都是 $C_2$,所以用\hyperref[proposition:广义标准型理论相关结论1]{命题\ref{proposition:广义标准型理论相关结论1}(2)}很难推出第二个结论,这里我们采用直接计算 $\widetilde{J}_{s_j}(a_j,b_j)$ 的不变因子组的方法来证明。注意到 $\lambda I - \widetilde{J}_{s_j}(a_j,b_j)$ 右上方的 $2s_j - 1$ 阶子式等于 $(-1)^{2s_j - 1}b_j^{s_j}\neq 0$,故 $\widetilde{J}_{s_j}(a_j,b_j)$ 的 $2s_j - 1$ 阶行列式因子为 $1$,于是其行列式因子组和不变因子组均为 $1,\cdots,1,((\lambda - a_j)^2 + b_j^2)^{s_j}$。由 $\lambda$-矩阵的初等变换以及\hyperref[theorem:lambda-矩阵和初等因子的基本性质]{$\lambda$-矩阵和初等因子的基本性质(2)}可知,$A$ 和 $\widetilde{J}$ 在实数域上有相同的初等因子组,从而它们在实数域上相似。

关于特征多项式和极小多项式的结论见\reflem{lemma:Jordan矩阵的初等因子组}.

\end{proof}

\begin{theorem}[数域$\mathbb{K}$上的Jordan-Chevalley分解]\label{theorem:数K上的Jordan-Chevalley分解}
设$A$是数域$\mathbb{K}$上的$n$阶矩阵,证明存在$\mathbb{K}$上的$n$阶矩阵$B,C$,使得$A = B + C$,且满足:

(1) $B$在复数域上可对角化; 

(2) $C$是幂零矩阵; 

(3) $BC = CB$,
并且满足上述条件的分解一定是唯一的.
\end{theorem}
\begin{remark}
类似于复数域上的Jordan--Chevalley分解,我们还可以证明对于上述分解$A = B + C$,存在$\mathbb{K}$上的多项式$f(x)$,使得$B = f(A)$. 不过,由于此证明涉及抽象代数中域的扩张等相关知识点,故在这里就不作详细的展开了. 
\end{remark}
\begin{proof}
设$A$在$\mathbb{K}$上的初等因子组为$P_1(\lambda)^{e_1},P_2(\lambda)^{e_2},\cdots,P_t(\lambda)^{e_t}$,其中$P_i(\lambda)$是$\mathbb{K}$上的首一不可约多项式,$e_i\geqslant 1, 1\leqslant i\leqslant t$. 由\hyperref[proposition:广义标准型理论相关结论1]{命题\ref{proposition:广义标准型理论相关结论1}}可知,存在$\mathbb{K}$上的可逆矩阵$P$,使得
\begin{align*}
P^{-1}AP = J=\mathrm{diag}\{J_{e_1}(P_1(\lambda)),J_{e_2}(P_2(\lambda)),\cdots,J_{e_t}(P_t(\lambda))\}.
\end{align*}
我们先对广义Jordan块$J_{e_i}(P_i(\lambda))$来证明结论,为方便起见,记$F_i = F(P_i(\lambda))$. 由于$P_i(\lambda)$在$\mathbb{K}$上不可约,故$(P_i(\lambda),P_i'(\lambda)) = 1$,从而$P_i(\lambda)$在复数域上无重根,于是$F_i$在复数域上可对角化. 令
\begin{align*}
M_i = 
\begin{pmatrix}
F_i & O & O & \cdots & O & O \\
O & F_i & O & \cdots & O & O \\
\vdots & \vdots & \vdots & & \vdots & \vdots \\
O & O & O & \cdots & F_i & O \\
O & O & O & \cdots & O & F_i
\end{pmatrix}, \quad 
N_i = 
\begin{pmatrix}
O & I & O & \cdots & O & O \\
O & O & I & \cdots & O & O \\
\vdots & \vdots & \vdots & & \vdots & \vdots \\
O & O & O & \cdots & O & I \\
O & O & O & \cdots & O & O
\end{pmatrix},
\end{align*}
则容易验证$J_{e_i}(P_i(\lambda)) = M_i + N_i$,$M_i$复可对角化,$N_i$幂零,$M_iN_i = N_iM_i$. 再令$M = \mathrm{diag}\{M_1,\cdots,M_t\}$,$N $ $=$ $\mathrm{diag}$ $\{$ $N_1$,$\cdots$,$N_t$ $\}$,则$J = M + N$,$M$复可对角化,$N$幂零,$MN = NM$. 最后令$B = PMP^{-1}$,$C = PNP^{-1}$,则$B,C$是$\mathbb{K}$上的矩阵,$A = B + C$,$B$复可对角化,$C$幂零,$BC = CB$. 我们也可将$A,B,C$看成是复数域上的矩阵,由复数域上的Jordan - Chevalley分解定理的唯一性可知,满足上述条件的分解一定是唯一的. 

\end{proof}








\end{document}
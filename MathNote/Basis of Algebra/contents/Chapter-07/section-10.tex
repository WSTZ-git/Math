\documentclass[../../main.tex]{subfiles}
\graphicspath{{\subfix{../../image/}}} % 指定图片目录,后续可以直接使用图片文件名。

% 例如:
% \begin{figure}[H]
% \centering
% \includegraphics[scale=0.4]{图.png}
% \caption{}
% \label{figure:图}
% \end{figure}
% 注意:上述\label{}一定要放在\caption{}之后,否则引用图片序号会只会显示??.

\begin{document}

\section{矩阵函数}

\begin{definition}[复方阵幂级数]
设有 $n$ 阶复方阵序列 $\{A_p\}$:
\begin{align*}
A_p = \begin{pmatrix}
a_{11}^{(p)} & \cdots & a_{1n}^{(p)} \\
\vdots & & \vdots \\
a_{n1}^{(p)} & \cdots & a_{nn}^{(p)}
\end{pmatrix},
\end{align*}
$B = (b_{ij})$ 是一个同阶方阵,若对每个 $(i,j)$,序列 $\{a_{ij}^{(p)}\}$ 均收敛于 $b_{ij}$,即
\begin{align*}
\lim_{p \to \infty} a_{ij}^{(p)} = b_{ij},
\end{align*}
则称矩阵序列 $\{A_p\}$ 收敛于 $B$,记为
\begin{align*}
\lim_{p \to \infty} A_p = B.
\end{align*}
否则称 $\{A_p\}$ 发散.

设
\begin{align*}
f(z) = a_0 + a_1z + a_2z^2 + \cdots
\end{align*}
是一个幂级数,记
\begin{align*}
f_p(z) = a_0 + a_1z + a_2z^2 + \cdots + a_pz^p
\end{align*}
是其部分和. 若矩阵序列 $\{f_p(A)\}$ 收敛于 $B$,则称矩阵级数
\begin{align*}
f(A) = a_0I + a_1A + a_2A^2 + \cdots .
\end{align*}
收敛,极限为 $B$,记为 $f(A) = B$. 否则称 $f(A)$ 发散. 用变量矩阵 $X$ 代替 $A$,便可定义矩阵幂级数
\begin{align*}
f(X) = a_0I + a_1X + a_2X^2 + \cdots. 
\end{align*} 
\end{definition}

\begin{theorem}[复矩阵极限的性质]\label{theorem:复矩阵极限的性质}
设$n$ 阶复方阵序列$\{A_p\}$和$\{B_p\}$.
\begin{align*}
A_p = \begin{pmatrix}
a_{11}^{(p)} & \cdots & a_{1n}^{(p)} \\
\vdots & & \vdots \\
a_{n1}^{(p)} & \cdots & a_{nn}^{(p)}
\end{pmatrix},\quad B_p = \begin{pmatrix}
b_{11}^{(p)} & \cdots & b_{1n}^{(p)} \\
\vdots & & \vdots \\
b_{n1}^{(p)} & \cdots & b_{nn}^{(p)}
\end{pmatrix},
\end{align*}
$f$是任意一个复系数多项式,即$f\in \mathbb{C}[x]$.我们有
\begin{enumerate}[(1)]
\item $\underset{p\rightarrow +\infty}{\lim}\left( A_p\pm B_p \right) =\underset{p\rightarrow +\infty}{\lim}A_p\pm \underset{p\rightarrow +\infty}{\lim}B_p;$

\item $\underset{p\rightarrow +\infty}{\lim}\left( A_pB_p \right) =\underset{p\rightarrow +\infty}{\lim}A_p\cdot \underset{p\rightarrow +\infty}{\lim}B_p;$

\item $
\underset{p\rightarrow +\infty}{\lim}f\left( A_p \right) =f\left( \underset{p\rightarrow +\infty}{\lim}A_p \right) ,$特别地,$\left| \lambda I-\underset{p\rightarrow +\infty}{\lim}A_p \right|=\underset{p\rightarrow +\infty}{\lim}\left| \lambda I-A_p \right|.$
\end{enumerate}
\end{theorem}
\begin{proof}
\begin{enumerate}[(1)]
\item 由极限的四则运算性质可知
\begin{align*}
\underset{p\rightarrow \mathrm{+}\infty}{\lim}\left( a_{ij}^{\left( p \right)}\pm b_{ij}^{\left( p \right)} \right) =\underset{p\rightarrow \mathrm{+}\infty}{\lim}a_{ij}^{\left( p \right)}\pm \underset{p\rightarrow \mathrm{+}\infty}{\lim}b_{ij}^{\left( p \right)},\quad \forall i,j\in \left\{ 1,2,\cdots ,n \right\} .
\end{align*}
因此$\underset{p\rightarrow +\infty}{\lim}\left( A_p\pm B_p \right) =\underset{p\rightarrow +\infty}{\lim}A_p\pm \underset{p\rightarrow +\infty}{\lim}B_p.$

\item 由极限的四则运算性质可知
\begin{align*}
\underset{p\rightarrow \mathrm{+}\infty}{\lim}\sum_{i,j=1}^n{\sum_{k=1}^n{a_{ik}^{\left( p \right)}b_{kj}^{\left( p \right)}}}=\sum_{i,j=1}^n{\sum_{k=1}^n{\left( \underset{p\rightarrow \mathrm{+}\infty}{\lim}a_{ik}^{\left( p \right)}\cdot \underset{p\rightarrow \mathrm{+}\infty}{\lim}b_{kj}^{\left( p \right)} \right)}},\quad \forall i,j\in \left\{ 1,2,\cdots ,n \right\} .
\end{align*}
因此$\underset{p\rightarrow +\infty}{\lim}\left( A_p B_p \right) =\underset{p\rightarrow +\infty}{\lim}A_p\cdot \underset{p\rightarrow +\infty}{\lim}B_p.$

\item 由(1)(2)的结论立得.特别地,记$\lambda I-A_p=\left( c_{ij}^{\left( p \right)} \right) ,$则由行列式的组合定义可知
\begin{align*}
\left| \lambda I-\underset{p\rightarrow +\infty}{\lim}A_p \right|&=\left| \underset{p\rightarrow +\infty}{\lim}\left( \lambda I-A_p \right) \right|=\sum_{1\leqslant i_1,i_2,\cdots ,i_n\leqslant n}{\left[ \left( -1 \right) ^{\tau \left( i_1i_2\cdots i_n \right)}\underset{p\rightarrow +\infty}{\lim}c_{1i_1}^{\left( p \right)}\underset{p\rightarrow +\infty}{\lim}c_{1i_2}^{\left( p \right)}\cdots \underset{p\rightarrow +\infty}{\lim}c_{1i_n}^{\left( p \right)} \right]}
\\
&=\underset{p\rightarrow +\infty}{\lim}\sum_{1\leqslant i_1,i_2,\cdots ,i_n\leqslant n}{\left[ \left( -1 \right) ^{\tau \left( i_1i_2\cdots i_n \right)}c_{1i_1}^{\left( p \right)}c_{1i_2}^{\left( p \right)}\cdots c_{1i_n}^{\left( p \right)} \right]}=\underset{p\rightarrow +\infty}{\lim}\left| \lambda I-A_p \right|.
\end{align*}
\end{enumerate}

\end{proof}

\begin{proposition}\label{proposition:矩阵方幂极限存在的充要条件及极限矩阵}
设$A$是$n$阶矩阵,证明:$\lim_{k \to \infty} A^k$存在的充要条件是$A$的特征值的模长小于$1$, 或者特征值等于$1$并且$A$关于特征值$1$的Jordan块都是一阶的. 此时, 极限矩阵为
\begin{align*}
\lim_{k \to \infty} A^k = P\mathrm{diag}\{1,\cdots,1,0,\cdots,0\}P^{-1},
\end{align*}
其中$1$的个数等于$A$的特征值$1$的代数重数.
\end{proposition}
\begin{proof}
设$P$为非异阵, 使得$P^{-1}AP = J=\mathrm{diag}\{J_{r_1}(\lambda_1),J_{r_2}(\lambda_2),\cdots,J_{r_s}(\lambda_s)\}$为$A$的Jordan标准型, 则
\begin{align*}
A^k = PJ^kP^{-1}=P\mathrm{diag}\{J_{r_1}(\lambda_1)^k,J_{r_2}(\lambda_2)^k,\cdots,J_{r_s}(\lambda_s)^k\}P^{-1},
\end{align*}
因此$\lim_{k \to \infty} A^k$存在当且仅当$\lim_{k \to \infty} J_{r_i}(\lambda_i)^k (1\leqslant i \leqslant s)$都存在. 不妨取$k > n$, 由\nrefpro{proposition:Jordan块的性质}{(5)}计算可得Jordan块$J_{r_i}(\lambda_i)$的$k$次幂为
\begin{align*}
J_{r_i}(\lambda_i)^k = 
\begin{pmatrix}
\lambda_i^k & \mathrm{C}_{k}^{1}\lambda_i^{k - 1} & \mathrm{C}_{k}^{2}\lambda_i^{k - 2} & \cdots & \mathrm{C}_{k}^{r_i - 1}\lambda_i^{k - r_i + 1} \\
 & \lambda_i^k & \mathrm{C}_{k}^{1}\lambda_i^{k - 1} & \cdots & \mathrm{C}_{k}^{r_i - 2}\lambda_i^{k - r_i + 2} \\
 & & \lambda_i^k & \cdots & \mathrm{C}_{k}^{r_i - 3}\lambda_i^{k - r_i + 3} \\
 & & & \ddots & \vdots \\
 & & & & \lambda_i^k
\end{pmatrix},
\end{align*}
故当$|\lambda_i|\geqslant 1$且$\lambda_i\neq 1$时, $\lim_{k \to \infty} \lambda_i^k$发散; 当$\lambda_i = 1$且$r_i\geqslant 2$时, $\lim_{k \to \infty} \mathrm{C}_{k}^{1}\lambda_i^{k - 1}$发散; 当$\lambda_i = 1$且$r_1 = 1$时, $\lim_{k \to \infty} J_{r_i}(\lambda_i)^k = J_1(1)$; 当$|\lambda_i| < 1$时, $\lim_{k \to \infty} J_{r_i}(\lambda_i)^k = O$. 因此, $\lim_{k \to \infty} A^k$存在的充要条件是$A$的特征值的模长小于$1$, 或者特征值等于$1$并且$A$关于特征值$1$的Jordan块都是一阶的. 此时, 极限矩阵$\lim_{k \to \infty} A^k = $ $P$ $\mathrm{diag}$ $\{$ $1,\cdots,1$,$0$,$\cdots$,$0$ $\}$ $P^{-1}$, 其中$1$的个数等于$A$的特征值$1$的代数重数.

\end{proof}

\begin{theorem}\label{theorem:方阵幂级数收敛的条件}
设\(f(z)=\sum_{i = 0}^{\infty}a_{i}z^{i}\)是复幂级数, 则
\begin{enumerate}[(1)]
\item 方阵幂级数\(f(X)\)收敛的充分必要条件是对任一非异阵\(P\),\(f(P^{-1}XP)\)都收敛, 这时
\begin{align*}
f(P^{-1}XP) = P^{-1}f(X)P;
\end{align*}

\item 若\(X = \mathrm{diag}\{X_1,\cdots,X_k\}\), 则\(f(X)\)收敛的充分必要条件是\(f(X_1),\cdots,f(X_k)\)都收敛, 这时
\begin{align*}
f(X) = \mathrm{diag}\{f(X_1),\cdots,f(X_k)\};
\end{align*}

\item 若\(f(z)\)的收敛半径为\(r\),\(J_0\)是特征值为\(\lambda_0\)的\(n\)阶 Jordan 块
\begin{align*}
J_0 = \begin{pmatrix}
\lambda_0 & 1 & & \\
 & \lambda_0 & 1 & \\
 & & \ddots & \ddots \\
 & & & \ddots & 1 \\
 & & & & \lambda_0
\end{pmatrix},
\end{align*}
则当\(|\lambda_0| < r\)时\(f(J_0)\)收敛, 且
\begin{align}\label{equation:::::7.8.6}
f(J_0) = \begin{pmatrix}
f(\lambda_0) & \frac{1}{1!}f'(\lambda_0) & \frac{1}{2!}f^{(2)}(\lambda_0) & \cdots & \frac{1}{(n - 1)!}f^{(n - 1)}(\lambda_0) \\
 & f(\lambda_0) & \frac{1}{1!}f'(\lambda_0) & \cdots & \frac{1}{(n - 2)!}f^{(n - 2)}(\lambda_0) \\
 & & f(\lambda_0) & \cdots & \frac{1}{(n - 3)!}f^{(n - 3)}(\lambda_0) \\
 & & & \ddots & \vdots \\
 & & & & f(\lambda_0)
\end{pmatrix}.
\end{align} 
\end{enumerate}
\end{theorem}
\begin{proof}
设\(f_p(z)=a_0 + a_1z + a_2z^2 + \cdots + a_pz^p\)是\(f(z)\)前\(p + 1\)项的部分和.
\begin{enumerate}[(1)]
\item 注意到\(f_p(z)\)是多项式, 从而有
\begin{align*}
f_p(P^{-1}XP) = P^{-1}f_p(X)P.
\end{align*}
由于\(n\)阶矩阵序列的收敛等价于\(n^2\)个数值序列的收敛, 故
\begin{align*}
f(P^{-1}XP) &= \lim_{p \to \infty} f_p(P^{-1}XP) = \lim_{p \to \infty} P^{-1}f_p(X)P \\
&= P^{-1}(\lim_{p \to \infty} f_p(X))P = P^{-1}f(X)P.
\end{align*}

\item  注意到\(f_p(z)\)是多项式, 从而有
\begin{align*}
f_p(X) = f_p(\mathrm{diag}\{X_1,\cdots,X_k\}) = \mathrm{diag}\{f_p(X_1),\cdots,f_p(X_k)\}.
\end{align*}
由于分块矩阵序列的收敛等价于每个分块的矩阵序列的收敛, 故
\begin{align*}
f(X) &= \lim_{p \to \infty} f_p(X) = \lim_{p \to \infty} \mathrm{diag}\{f_p(X_1),\cdots,f_p(X_k)\} \\
&= \mathrm{diag}\{\lim_{p \to \infty} f_p(X_1),\cdots,\lim_{p \to \infty} f_p(X_k)\} = \mathrm{diag}\{f(X_1),\cdots,f(X_k)\}.
\end{align*}

\item  由\nrefpro{proposition:Jordan块的性质}{(5)}可知
\begin{align*}
f_p(J_0)= \begin{pmatrix}
f_p(\lambda_0) & \frac{1}{1!}f_p'(\lambda_0) & \frac{1}{2!}f_p^{(2)}(\lambda_0) & \cdots & \frac{1}{(n - 1)!}f_p^{(n - 1)}(\lambda_0) \\
 & f_p(\lambda_0) & \frac{1}{1!}f_p'(\lambda_0) & \cdots & \frac{1}{(n - 2)!}f_p^{(n - 2)}(\lambda_0) \\
 & & f_p(\lambda_0) & \cdots & \frac{1}{(n - 3)!}f_p^{(n - 3)}(\lambda_0) \\
 & & & \ddots & \vdots \\
 & & & & f_p(\lambda_0)
\end{pmatrix}. 
\end{align*}
令\(p \to \infty\), 由矩阵序列收敛与\(n^2\)个数值序列收敛的等价性和幂级数的相关性质即得结论.
\end{enumerate}

\end{proof}

\begin{theorem}\label{theorem:复方阵幂级数的谱半径}
设\(f(z)\)是复幂级数,收敛半径为\(r\). 设\(A\)是\(n\)阶复方阵,特征值为\(\lambda_1,\lambda_2,\cdots,\lambda_n\),定义\(A\)的\textbf{谱半径}
\begin{align*}
\rho(A) = \max_{1 \leqslant  i \leqslant  n}|\lambda_i|.
\end{align*}
\begin{enumerate}[(1)]
\item 若\(\rho(A) < r\),则\(f(A)\)收敛;

\item 若\(\rho(A) > r\),则\(f(A)\)发散;

\item 若\(\rho(A) = r\),则\(f(A)\)收敛的充分必要条件是:对每一模长等于\(r\)的特征值\(\lambda_j\),若\(A\)的属于\(\lambda_j\)的初等因子中最高幂为\(n_j\)次,则\(n_j\)个数值级数
\begin{align}
f(\lambda_j), f'(\lambda_j), \cdots, f^{(n_j - 1)}(\lambda_j) .\label{equation::::--7.8.6}
\end{align}
都收敛;

\item 若\(f(A)\)收敛,则\(f(A)\)的特征值为
\begin{align*}
f(\lambda_1), f(\lambda_2), \cdots, f(\lambda_n).
\end{align*} 
\end{enumerate}
\end{theorem}
\begin{proof}
\begin{enumerate}[(1)]
\item 设\(A\)的 Jordan 标准型为\(J = \mathrm{diag}\{J_1,J_2,\cdots,J_k\}\). 显然\(f(A)\)的收敛性等价于所有\(f(J_i)\)(\(i = 1,\cdots,k\))的收敛性. 由\refthe{theorem:方阵幂级数收敛的条件}即知(1)成立.

\item 若某一个\(|\lambda_j| > r\),则\(f(\lambda_j)\)发散,因此\(f(J_j)\)发散,故\(f(A)\)发散,这就证明了(2).

\item 当\(\rho(A) = r\)时,对\(|\lambda_i| < r\)的\(J_i\),\(f(J_i)\)收敛. 对\(|\lambda_j| = r\)的特征值\(\lambda_j\),注意到\(f(z)\)的任意次导数的收敛半径仍为\(r\),又初等因子\((\lambda - \lambda_j)^{n_j}\)对应的 Jordan 块为\(n_j\)阶,从\eqref{equation:::::7.8.6}式即可知道\(f(J_j)\)的收敛性等价于\eqref{equation::::--7.8.6}式中\(n_j\)个级数的收敛性.

\item 最后若\(f(A)\)收敛,则\(f(A)\)与\(f(J)\)有相同的特征值,即为\(f(\lambda_1),f(\lambda_2),\cdots,f(\lambda_n)\).
\end{enumerate}

\end{proof}

\begin{definition}\label{definition:矩阵函数}
于是对一切方阵,定义
\begin{align*}
\mathrm{e}^{\boldsymbol{A}}&= \boldsymbol{I}+\frac{1}{1!}\boldsymbol{A}+\frac{1}{2!}\boldsymbol{A}^{2}+\frac{1}{3!}\boldsymbol{A}^{3}+\cdots,\\
\sin\boldsymbol{A}&= \boldsymbol{A}-\frac{1}{3!}\boldsymbol{A}^{3}+\frac{1}{5!}\boldsymbol{A}^{5}-\frac{1}{7!}\boldsymbol{A}^{7}+\cdots,\\
\cos\boldsymbol{A}&= \boldsymbol{I}-\frac{1}{2!}\boldsymbol{A}^{2}+\frac{1}{4!}\boldsymbol{A}^{4}-\frac{1}{6!}\boldsymbol{A}^{6}+\cdots
\end{align*}
都有意义. 若 $\boldsymbol{A}$ 所有特征值的模长都小于 1,则 
\[
\ln(\boldsymbol{I}+\boldsymbol{A})=\boldsymbol{A}-\frac{1}{2}\boldsymbol{A}^{2}+\frac{1}{3}\boldsymbol{A}^{3}-\frac{1}{4}\boldsymbol{A}^{4}+\cdots
\]
也有意义. 同理还可以定义幂函数、双曲函数等. 
\end{definition}
\begin{remark}
由复分析知道:
\begin{align*}
&\mathrm{e}^{z}= 1+\frac{1}{1!}z+\frac{1}{2!}z^{2}+\frac{1}{3!}z^{3}+\cdots,\\
&\sin z= z-\frac{1}{3!}z^{3}+\frac{1}{5!}z^{5}-\frac{1}{7!}z^{7}+\cdots,\\
&\cos z= 1-\frac{1}{2!}z^{2}+\frac{1}{4!}z^{4}-\frac{1}{6!}z^{6}+\cdots,\\
&\ln(1 + z)=z-\frac{1}{2}z^{2}+\frac{1}{3}z^{3}-\frac{1}{4}z^{4}+\cdots.
\end{align*}
前 3 个级数在整个复平面上收敛,而 $\ln(1 + z)$ 的收敛半径为 1.于是由\refthe{theorem:复方阵幂级数的谱半径}可知$\mathrm{e}^{\boldsymbol{A}},\sin \boldsymbol{A},\cos \boldsymbol{A},\ln{\boldsymbol{A}}$都收敛,从而都有意义.
故上述定义是良定义的.
\end{remark}

\begin{theorem}\label{theorem:矩阵三角函数的一般表达式}
证明:$\cos A = \frac{e^{\mathrm{i}A}+e^{-\mathrm{i}A}}{2}$,$\quad$ $\sin A = \frac{e^{\mathrm{i}A}-e^{-\mathrm{i}A}}{2\mathrm{i}}$.
\end{theorem}
\begin{proof}
由\refdef{definition:矩阵函数}可知
\begin{align*}
e^{\mathrm{i}A}&=I+\frac{1}{1!}\mathrm{i}A-\frac{1}{2!}A^2-\frac{1}{3!}\mathrm{i}A^3-\cdots +\frac{\left( -1 \right) ^k}{2k!}A^{2k}+\frac{\left( -1 \right) ^k}{\left( 2k+1 \right) !}\mathrm{i}A^{2k+1}+\cdots,\\
e^{-\mathrm{i}A}&=I-\frac{1}{1!}\mathrm{i}A-\frac{1}{2!}A^2+\frac{1}{3!}\mathrm{i}A^3-\cdots +\frac{\left( -1 \right) ^k}{2k!}A^{2k}+\frac{\left( -1 \right) ^{k+1}}{\left( 2k+1 \right) !}\mathrm{i}A^{2k+1}+\cdots.
\end{align*}
从而
\begin{align*}
\frac{e^{\mathrm{i}A}+e^{-\mathrm{i}A}}{2}&=I-\frac{1}{2!}A^2+\cdots +\frac{\left( -1 \right) ^k}{2k!}A^{2k}+\cdots =\cos A,\\
\frac{e^{\mathrm{i}A}-e^{-\mathrm{i}A}}{2\mathrm{i}}&=\frac{1}{1!}A-\frac{1}{3!}A^3+\cdots +\frac{\left( -1 \right) ^k}{\left( 2k+1 \right) !}A^{2k+1}\cdots =\sin A.
\end{align*}

\end{proof}

\begin{proposition}\label{proposition:若A,B乘法可交换,则e^A,e^B也乘法可交换}
求证:若 $n$ 阶矩阵 $A$, $B$ 乘法可交换,则 $\mathrm{e}^A\cdot\mathrm{e}^B = \mathrm{e}^B\cdot\mathrm{e}^A$。
\end{proposition}
\begin{remark}
对一般来说对矩阵 $\boldsymbol{A},\boldsymbol{B}$,下面的等式并不一定成立:
\begin{align*}
\mathrm{e}^{\boldsymbol{A}}\cdot\mathrm{e}^{\boldsymbol{B}}=\mathrm{e}^{\boldsymbol{A}+\boldsymbol{B}}=\mathrm{e}^{\boldsymbol{B}}\cdot\mathrm{e}^{\boldsymbol{A}}.
\end{align*}
如对
\[
\boldsymbol{A}=\begin{pmatrix}
1 & 1 \\
0 & 0
\end{pmatrix}, \boldsymbol{B}=\begin{pmatrix}
0 & 0 \\
1 & 1
\end{pmatrix},
\]
通过计算不难验证 $AB = A$, $BA = B$,并且
\begin{align*}
\mathrm{e}^A &= 
\begin{pmatrix}
\mathrm{e} & \mathrm{e} - 1 \\
0 & 1
\end{pmatrix}, \quad
\mathrm{e}^B = 
\begin{pmatrix}
1 & 0 \\
\mathrm{e} - 1 & \mathrm{e}
\end{pmatrix}, \quad
\mathrm{e}^{A + B} = 
\begin{pmatrix}
\frac{\mathrm{e}^2 + 1}{2} & \frac{\mathrm{e}^2 - 1}{2} \\
\frac{\mathrm{e}^2 - 1}{2} & \frac{\mathrm{e}^2 + 1}{2}
\end{pmatrix}
\end{align*} 
故
$\mathrm{e}^{\boldsymbol{A}}\cdot\mathrm{e}^{\boldsymbol{B}}\neq\mathrm{e}^{\boldsymbol{A}+\boldsymbol{B}}$. 
\end{remark}
\begin{proof}
设 $f(z)=\mathrm{e}^z$,并且 $f_p(z)=1+\frac{1}{1!}z+\frac{1}{2!}z^2+\cdots+\frac{1}{p!}z^p$ 为 $f(z)$ 的部分和,因为 $f(z)$ 的收敛半径为 $+\infty$,所以对任一矩阵 $A$,$\lim_{p\to\infty}f_p(A)=f(A)=\mathrm{e}^A$。由于 $AB = BA$,故对任意的正整数 $p$, $q$,成立 $f_p(A)f_q(B)=f_q(B)f_p(A)$。先固定 $p$,令 $q\to\infty$,则可得
\begin{align*}
f_p(A)f(B) &= f_p(A)\left(\lim_{q\to\infty}f_q(B)\right)=\lim_{q\to\infty}\left(f_p(A)f_q(B)\right)=\lim_{q\to\infty}\left(f_q(B)f_p(A)\right)\\
&=\left(\lim_{q\to\infty}f_q(B)\right)f_p(A)=f(B)f_p(A)
\end{align*}
同理,再对上式令 $p\to\infty$,则可得 $f(A)f(B)=f(B)f(A)$,即结论成立。 

\end{proof}

\begin{corollary}\label{corollary:矩阵幂级数乘法可交换}
若 $f(z)$, $g(z)$ 是两个收敛半径都是 $+\infty$ 的复幂级数,则对任意乘法可交换的 $A$, $B$,均有 $f(A)g(B)=g(B)f(A)$。 
\end{corollary}
\begin{proof}
由\refpro{proposition:若A,B乘法可交换,则e^A,e^B也乘法可交换}类似的讨论可证明.

\end{proof}

\begin{definition}[矩阵的范数]\label{definition:矩阵的范数}
设 $A=(a_{ij})$ 是 $n$ 阶复矩阵,定义 $A$ 的\textbf{范数}为其所有元素模长的平方和的算术平方根,即 $\|A\| = \sqrt{\sum_{i,j = 1}^{n}|a_{ij}|^2}$.
\end{definition}

\begin{proposition}[矩阵的范数的基本性质]\label{proposition:矩阵的范数的基本性质}
设 $A=(a_{ij})$ 是 $n$ 阶复矩阵,$B=(b_{ij})$ 也是 $n$ 阶复矩阵,求证:

(1) $\|A\|\geqslant  0$,等号成立当且仅当 $A = O$;

(2) $\|A + B\|\leqslant  \|A\| + \|B\|$;

(3) $\|AB\|\leqslant  \|A\|\cdot\|B\|$.
\end{proposition}
\begin{proof}
(1) 显然成立。

(2)注意到
\begin{align*}
\left\| A+B \right\| ^2&=\sum_{i,j=1}^n{\left| a_{ij}+b_{ij} \right|^2}=\sum_{i,j=1}^n{\left( \left| a_{ij} \right|^2+\left| b_{ij} \right|^2+2\left| a_{ij}b_{ij} \right| \right)}\\
&=\sum_{i,j=1}^n{\left| a_{ij} \right|^2}+\sum_{i,j=1}^n{\left| b_{ij} \right|^2}+2\sum_{i,j=1}^n{\left| a_{ij}b_{ij} \right|}
\end{align*}
于是
\begin{align*}
\left( \left\| A \right\| +\left\| B \right\| \right) ^2&=\left\| A \right\| ^2+\left\| B \right\| ^2+2\left\| A \right\| \cdot \left\| B \right\| \\
&=\sum_{i,j=1}^n{\left| a_{ij} \right|^2}+\sum_{i,j=1}^n{\left| b_{ij} \right|^2}+2\sqrt{\left( \sum_{i,j=1}^n{\left| a_{ij} \right|^2} \right) \left( \sum_{i,j=1}^n{\left| b_{ij} \right|^2} \right)}\\
&\overset{\text{Cauchy-Schwarz不等式}}{\geqslant}\sum_{i,j=1}^n{\left| a_{ij} \right|^2}+\sum_{i,j=1}^n{\left| b_{ij} \right|^2}+2\sqrt{\left( \sum_{i,j=1}^n{\left| a_{ij}b_{ij} \right|} \right) ^2}\\
&=\sum_{i,j=1}^n{\left| a_{ij} \right|^2}+\sum_{i,j=1}^n{\left| b_{ij} \right|^2}+2\sum_{i,j=1}^n{\left| a_{ij}b_{ij} \right|}\\
&=\left\| A+B \right\| ^2.
\end{align*}
故结论得证.

(3) 注意到 $\|AB\|^2 = \sum_{i,j = 1}^{n}\left|\sum_{k = 1}^{n}a_{ik}b_{kj}\right|^2$,$\|A\|^2\cdot\|B\|^2 = \left(\sum_{i,k = 1}^{n}|a_{ik}|^2\right)\cdot\left(\sum_{k,j = 1}^{n}|b_{kj}|^2\right)$,任取$i,j\in \left\{ 1,2,\cdots ,n \right\}$,固定$i$和$j$,由Cauchy-Schwarz不等式可知
\begin{align*}
\left| \sum_{k=1}^n{a_{ik}b_{kj}} \right|^2\leqslant \left( \sum_{k=1}^n{a_{ik}^{2}} \right) \cdot \left( \sum_{k=1}^n{b_{kj}^{2}} \right) .
\end{align*}
从而先对$i$求和可得
\begin{align*}
\sum_{i=1}^n{\left| \sum_{k=1}^n{a_{ik}b_{kj}} \right|^2}\leqslant \sum_{i=1}^n{\left( \sum_{k=1}^n{a_{ik}^{2}} \right) \cdot \left( \sum_{k=1}^n{b_{kj}^{2}} \right)}=\left( \sum_{k=1}^n{b_{kj}^{2}} \right) \left( \sum_{i=1}^n{\left( \sum_{k=1}^n{a_{ik}^{2}} \right)} \right) .
\end{align*}
再对$j$求和可得
\begin{align*}
\sum_{i,j=1}^n{\left| \sum_{k=1}^n{a_{ik}b_{kj}} \right|^2}&\leqslant \sum_{j=1}^n{\left[ \left( \sum_{k=1}^n{b_{kj}^{2}} \right) \left( \sum_{i=1}^n{\left( \sum_{k=1}^n{a_{ik}^{2}} \right)} \right) \right]}\\
&=\left( \sum_{i=1}^n{\left( \sum_{k=1}^n{a_{ik}^{2}} \right)} \right) \left( \sum_{j=1}^n{\left( \sum_{k=1}^n{b_{kj}^{2}} \right)} \right) \\
&=\left( \sum_{i,k=1}^n{|a_{ik}|^2} \right) \cdot \left( \sum_{k,j=1}^n{|b_{kj}|^2} \right) .
\end{align*}
由此即得结论.

\end{proof}

\begin{proposition}\label{proposition:矩阵乘法可交换必成立e^A乘e^B=e^A+B}
如果 $\boldsymbol{A}$ 与 $\boldsymbol{B}$ 乘法可交换,即 $\boldsymbol{A}\boldsymbol{B}=\boldsymbol{B}\boldsymbol{A}$,则 $\mathrm{e}^{\boldsymbol{A}}\cdot\mathrm{e}^{\boldsymbol{B}}=\mathrm{e}^{\boldsymbol{A}+\boldsymbol{B}}$ 必成立.
\end{proposition}
\begin{remark}
利用这个命题也可给出\refpro{proposition:若A,B乘法可交换,则e^A,e^B也乘法可交换}的另一证明。 
\end{remark}
\begin{proof}
设 $f(z)=\mathrm{e}^z$,并且 $f_p(z)=1+\frac{1}{1!}z+\frac{1}{2!}z^2+\cdots+\frac{1}{p!}z^p$ 为 $f(z)$ 的部分和。注意到 $AB = BA$,经简单的计算可知,$f_p(A)f_p(B)$ 展开后的单项包含 $f_p(A + B)$ 展开后的所有单项,且剩余单项可表示为 $\frac{A^i}{i!}\frac{B^j}{j!}$ 的形式,其中 $i + j>p$,故由\hyperref[proposition:矩阵的范数的基本性质]{矩阵的范数的基本性质(3)}可得
\begin{align*}
\parallel f_p(A)f_p(B)-f_p(A+B)\parallel &\le \sum_{k>p}{\left( \sum_{i+j=k}{\frac{\parallel A\parallel ^i}{i!}\frac{\parallel B\parallel ^j}{j!}} \right)}=\sum_{k>p}{\left( \sum_{i=0}^k{\frac{\parallel A\parallel ^i}{i!}\frac{\parallel B\parallel ^{k-i}}{\left( k-i \right) !}} \right)}
\\
&=\sum_{k>p}{\left( \sum_{i=0}^k{\frac{k!}{i!\left( k-i \right) !}\frac{\parallel A\parallel ^i\parallel B\parallel ^{k-i}}{k!}} \right)}=\sum_{k>p}{\left( \sum_{i=0}^k{\mathrm{C}_{k}^{i}\frac{\parallel A\parallel ^i\parallel B\parallel ^{k-i}}{k!}} \right)}
\\
&=\sum_{k>p}{\frac{(\parallel A\parallel +\parallel B\parallel )^k}{k!}}.
\end{align*}
由于数项级数 $\sum_{k = 0}^{\infty}\frac{1}{k!}(\|A\|+\|B\|)^k$ 收敛到 $\mathrm{e}^{\|A\|+\|B\|}$,故当 $p$ 充分大时,上式右边趋于零。令 $p\to\infty$,则由上式即得 $\|f(A)f(B)-f(A + B)\| = 0$,再次由\hyperref[proposition:矩阵的范数的基本性质]{矩阵的范数的基本性质(1)}可得 $\mathrm{e}^A\cdot\mathrm{e}^B = \mathrm{e}^{A + B}$.

\end{proof}

\begin{corollary}\label{corollary:矩阵幂级数的绝对收敛性保证了Cauchy乘积的收敛性}
若矩阵幂级数 $\mathrm{e}^A$ 绝对收敛,则矩阵级数的 Cauchy 乘积
\begin{align*}
\mathrm{e}^{A + B}=\sum_{p = 0}^{\infty}\left(\sum_{i + j = p}\frac{A^i}{i!}\frac{B^j}{j!}\right)
\end{align*}
收敛到
\begin{align*}
\left(\sum_{i = 0}^{\infty}\frac{A^i}{i!}\right)\cdot\left(\sum_{j = 0}^{\infty}\frac{B^j}{j!}\right)=\mathrm{e}^A\cdot\mathrm{e}^B.
\end{align*}
\end{corollary}
\begin{remark}
注意矩阵级数的Cauchy积有
\begin{align*}
\mathrm{e}^{A+B}&=\sum_{p=0}^{\infty}{\frac{\left( A+B \right) ^p}{p!}}=\sum_{p=0}^{\infty}{\left( \sum_{i=0}^p{\mathrm{C}_{p}^{i}\frac{A^iB^{p-i}}{p!}} \right)}
\\
&=\sum_{p=0}^{\infty}{\left( \sum_{i=0}^p{\frac{p!}{i!\left( p-i \right) !}\frac{A^iB^{p-i}}{p!}} \right)}=\sum_{p=0}^{\infty}{\left( \sum_{i=0}^p{\frac{A^i}{i!}\frac{B^{p-i}}{\left( p-i \right)}} \right)}
\\
&=\sum_{p=0}^{\infty}{\left( \sum_{i+j=p}{\frac{A^i}{i!}\frac{B^j}{j!}} \right)}.
\end{align*}
\end{remark}
\begin{proof}
由\refpro{proposition:矩阵乘法可交换必成立e^A乘e^B=e^A+B}立得.

\end{proof}

\begin{proposition}\label{proposition:计算e^tA}
设$t$ 是一个数值变量,$\boldsymbol{A}$ 是一个 $n$ 阶复方阵.
$\boldsymbol{P}^{-1}\boldsymbol{A}\boldsymbol{P}=\boldsymbol{J}=\mathrm{diag}\{\boldsymbol{J}_1,\cdots,\boldsymbol{J}_k\}$ 是 $\boldsymbol{A}$ 的 Jordan 标准型,$\boldsymbol{J}_i$ 是特征值为 $\lambda_i$ 的 $r$ 阶 Jordan 块,则
\begin{align*}
\mathrm{e}^{t\boldsymbol{A}}=\boldsymbol{P}\mathrm{e}^{t\boldsymbol{J}}\boldsymbol{P}^{-1},
\end{align*}
其中
\begin{align*}
\mathrm{e}^{t\boldsymbol{J}}=\left( \begin{matrix}
\mathrm{e}^{t\boldsymbol{J}_1}&		&		\\
&		\ddots&		\\
&		&		\mathrm{e}^{t\boldsymbol{J}_k}\\
\end{matrix} \right) ,\quad \mathrm{e}^{t\boldsymbol{J}_i}=\mathrm{e}^{t\lambda _i}\left( \begin{matrix}
1&		t&		\frac{1}{2!}t^2&		\frac{1}{3!}t^3&		\cdots&		\frac{1}{(r-1)!}t^{r-1}\\
&		1&		t&		\frac{1}{2!}t^2&		\cdots&		\frac{1}{(r-2)!}t^{r-2}\\
&		&		1&		t&		\cdots&		\frac{1}{(r-3)!}t^{r-3}\\
&		&		&		\ddots&		\ddots&		\vdots\\
&		&		&		&		\ddots&		t\\
&		&		&		&		&		1\\
\end{matrix} \right) .
\end{align*}
\end{proposition}
\begin{proof}
{\color{blue}证法一:}
若令 $f(z)=\mathrm{e}^{tz}$,则由\refthe{theorem:方阵幂级数收敛的条件}即得 $f(\boldsymbol{A}) = \mathrm{e}^{t\boldsymbol{A}}$ 的计算结果. 

{\color{blue}证法二:}
注意到
\begin{align*}
\boldsymbol{J}_i=\lambda_i\boldsymbol{I}+\boldsymbol{N},
\end{align*}
其中 $\boldsymbol{N}$ 是 $r$ 阶基础幂零阵,即
\begin{align*}
\boldsymbol{N}^r=\boldsymbol{O},\quad \boldsymbol{N}=\left( \begin{matrix}
0&		1&		&		&		\\
&		0&		1&		&		\\
&		&		\ddots&		\ddots&		\\
&		&		&		\ddots&		1\\
&		&		&		&		0\\
\end{matrix} \right) .
\end{align*}
于是
\begin{align*}
\mathrm{e}^{\boldsymbol{N}}=\boldsymbol{I}+\boldsymbol{N}+\frac{1}{2!}\boldsymbol{N}^2+\frac{1}{3!}\boldsymbol{N}^3+\cdots+\frac{1}{(r - 1)!}\boldsymbol{N}^{r - 1}.
\end{align*}
因为 $(\lambda_i\boldsymbol{I})\boldsymbol{N}=\boldsymbol{N}(\lambda_i\boldsymbol{I})$,故由\refpro{proposition:矩阵乘法可交换必成立e^A乘e^B=e^A+B}可知
\begin{align*}
\mathrm{e}^{\boldsymbol{J}_i}&=\mathrm{e}^{\lambda_i\boldsymbol{I}+\boldsymbol{N}}=\mathrm{e}^{\lambda_i\boldsymbol{I}}\cdot\mathrm{e}^{\boldsymbol{N}}=\mathrm{e}^{\lambda_i}\cdot\mathrm{e}^{\boldsymbol{N}}\\
&=\mathrm{e}^{\lambda_i}\boldsymbol{I}+\mathrm{e}^{\lambda_i}\boldsymbol{N}+\frac{1}{2!}\mathrm{e}^{\lambda_i}\boldsymbol{N}^2+\cdots+\frac{1}{(r - 1)!}\mathrm{e}^{\lambda_i}\boldsymbol{N}^{r - 1}.
\end{align*}
同理
\begin{align*}
\mathrm{e}^{t\boldsymbol{J}_i}&=\mathrm{e}^{t\lambda_i}\cdot\mathrm{e}^{t\boldsymbol{N}}=\mathrm{e}^{t\lambda _i}\left[ t\boldsymbol{I}+t\boldsymbol{N}+\frac{t}{2!}\boldsymbol{N}^2+\frac{t}{3!}\boldsymbol{N}^3+\cdots +\frac{t}{(r-1)!}\boldsymbol{N}^{r-1} \right] 
\\
&=\mathrm{e}^{t\lambda_i}
\begin{pmatrix}
1 & t & \frac{1}{2!}t^2 & \frac{1}{3!}t^3 & \cdots & \frac{1}{(r - 1)!}t^{r - 1} \\
 & 1 & t & \frac{1}{2!}t^2 & \cdots & \frac{1}{(r - 2)!}t^{r - 2} \\
 & & 1 & t & \cdots & \frac{1}{(r - 3)!}t^{r - 3} \\
 & & & \ddots & \ddots & \vdots \\
 & & & & \ddots & t \\
 & & & & & 1
\end{pmatrix}.
\end{align*}
又注意到
\begin{align*}
\mathrm{e}^{t\boldsymbol{A}}&=\mathrm{e}^{\boldsymbol{P}(t\boldsymbol{J})\boldsymbol{P}^{-1}}=\boldsymbol{P}\mathrm{e}^{t\boldsymbol{J}}\boldsymbol{P}^{-1},\\
\mathrm{e}^{t\boldsymbol{J}}&=\begin{pmatrix}
\mathrm{e}^{t\boldsymbol{J}_1} & & \\
 & \ddots & \\
 & & \mathrm{e}^{t\boldsymbol{J}_k}
\end{pmatrix}.
\end{align*}
于是将 $\mathrm{e}^{t\boldsymbol{J}_i}$ 的式子代入上面的式子即可求出 $\mathrm{e}^{t\boldsymbol{A}}$. 

\end{proof}

\begin{proposition}[矩阵三角函数的性质]\label{proposition:矩阵三角函数的性质}
设$A$是$n$阶矩阵,求证:
\begin{enumerate}[(1)]
\item  $\sin^2 A+\cos^2 A = I_n$.

\item $\sin 2A = 2\sin A\cos A$.
\end{enumerate}
\end{proposition}
\begin{proof}
由\refthe{theorem:矩阵三角函数的一般表达式}可知
\begin{align}\label{equation--::1}
\cos A = \frac{1}{2}(e^{\mathrm{i}A}+e^{-\mathrm{i}A}),
\quad
\sin A = \frac{1}{2\mathrm{i}}(e^{\mathrm{i}A}-e^{-\mathrm{i}A}).
\end{align}
由\refpro{proposition:若A,B乘法可交换,则e^A,e^B也乘法可交换}和\refpro{proposition:矩阵乘法可交换必成立e^A乘e^B=e^A+B}可知
\begin{align}\label{equation--::2}
e^{\mathrm{i}A}e^{-\mathrm{i}A}=e^{-\mathrm{i}A}e^{\mathrm{i}A}=e^{\mathrm{i}A - \mathrm{i}A}=I_n,(e^{\mathrm{i}A})^2 = e^{2\mathrm{i}A},(e^{-\mathrm{i}A})^2 = e^{-2\mathrm{i}A}.
\end{align}
\begin{enumerate}[(1)]
\item 由\eqref{equation--::1}和\eqref{equation--::2}式可得
\begin{align*}
\sin^2 A+\cos^2 A &= \frac{1}{4}(e^{2\mathrm{i}A}+2I_n + e^{-2\mathrm{i}A}) - \frac{1}{4}(e^{2\mathrm{i}A}-2I_n + e^{-2\mathrm{i}A}) = I_n.
\end{align*}

\item 由\eqref{equation--::1}和\eqref{equation--::2}式可得
\begin{align*}
2\sin A\cos A=\frac{1}{2}\left( e^{\mathrm{i}A}-e^{-\mathrm{i}A} \right) \left( e^{\mathrm{i}A}+e^{-\mathrm{i}A} \right) =\frac{1}{2}\left( e^{\mathrm{i}2A}-e^{-\mathrm{i}2A} \right) =\sin 2A.    
\end{align*}
\end{enumerate}

\end{proof}

\begin{example}
计算 $\sin(e^{cI})$ 及 $\cos(e^{cI})$, 其中 $c$ 是非零常数.
\end{example}
\begin{solution}
由指数矩阵函数的定义可得
\begin{align*}
e^{cI}&=I+\frac{1}{1!}(cI)+\frac{1}{2!}(cI)^2+\frac{1}{3!}(cI)^3+\cdots\\
&=\left( 1+\frac{1}{1!}c+\frac{1}{2!}c^2+\frac{1}{3!}c^3+\cdots \right) I = e^cI.
\end{align*}
因此
\begin{align*}
\sin(e^{cI})&=\sin(e^cI)=e^cI-\frac{1}{3!}(e^cI)^3+\frac{1}{5!}(e^cI)^5-\frac{1}{7!}(e^cI)^7+\cdots\\
&=\left( e^c-\frac{1}{3!}(e^c)^3+\frac{1}{5!}(e^c)^5-\frac{1}{7!}(e^c)^7+\cdots \right) I = (\sin e^c)I,\\
\cos(e^{cI})&=\cos(e^cI)=I-\frac{1}{2!}(e^cI)^2+\frac{1}{4!}(e^cI)^4-\frac{1}{6!}(e^cI)^6+\cdots\\
&=\left( 1-\frac{1}{2!}(e^c)^2+\frac{1}{4!}(e^c)^4-\frac{1}{6!}(e^c)^6+\cdots \right) I = (\cos e^c)I. 
\end{align*} 

\end{solution}

\begin{proposition}\label{proposition:矩阵函数e^A的行列式}
设$A$是$n$阶方阵,证明:$e^A$的行列式为$e^{\mathrm{tr}(A)}$.
\end{proposition}
\begin{proof}
设$A$的特征值为$\lambda_1,\lambda_2,\cdots,\lambda_n$, 则由\refpro{proposition:矩阵多项式的特征值就是原特征值代入多项式得到的数}可知$e^A$的特征值为$e^{\lambda_1},e^{\lambda_2},\cdots,e^{\lambda_n}$, 因此
\begin{align*}
|e^A| = e^{\lambda_1}e^{\lambda_2}\cdots e^{\lambda_n}=e^{\lambda_1 + \lambda_2 + \cdots + \lambda_n}=e^{\mathrm{tr}(A)}.
\end{align*} 

\end{proof}

\begin{corollary}
求证: 对任一$n$阶方阵$A$,$e^A$总是非异阵.
\end{corollary}
\begin{proof}
由\refpro{proposition:矩阵函数e^A的行列式}可知$|e^A| = e^{\mathrm{tr}(A)}\neq 0$, 从而$e^A$非异. 也可由\refpro{proposition:矩阵乘法可交换必成立e^A乘e^B=e^A+B}得到
\begin{align*}
e^Ae^{-A}=e^{A - A}=I_n,
\end{align*}
于是$e^A$非异且$(e^A)^{-1}=e^{-A}$. 

\end{proof}







\end{document}
\documentclass[../../main.tex]{subfiles}
\graphicspath{{\subfix{../../image/}}} % 指定图片目录,后续可以直接使用图片文件名。

% 例如:
% \begin{figure}[h]
% \centering
% \includegraphics{image-01.01}
% \caption{图片标题}
% \label{fig:image-01.01}
% \end{figure}
% 注意:上述\label{}一定要放在\caption{}之后,否则引用图片序号会只会显示??.

\begin{document}

\section{矩阵函数}

\begin{lemma}\label{lemma:利用Jordan标准型计算复矩阵的多项式}
设$A\in M_n(\mathbb{C})$,则存在\(P\)是非异阵,使
\begin{align*}
P^{-1}AP = J = \mathrm{diag}\{J_1,J_2,\cdots,J_k\}
\end{align*}
是\(A\)的 Jordan 标准型,其中$J_i$是$A$的特征值$\lambda_i$的$r$阶Jordan块.
若$f(x)=a_0+a_1x+\cdots+a_px^p,$则
\begin{align*}
f(A)=P\mathrm{diag}\left\{ f\left( J_1 \right) ,f\left( J_2 \right) ,\cdots ,f\left( J_k \right) \right\} P^{-1},
\end{align*}
其中
\begin{align*}
f(J_i)= \begin{pmatrix}
f(\lambda_i) & \frac{1}{1!}f'(\lambda_i) & \frac{1}{2!}f^{(2)}(\lambda_i) & \cdots & \frac{1}{(r - 1)!}f^{(r - 1)}(\lambda_i) \\
 & f(\lambda_i) & \frac{1}{1!}f'(\lambda_i) & \cdots & \frac{1}{(r - 2)!}f^{(r - 2)}(\lambda_i) \\
 & & f(\lambda_i) & \cdots & \frac{1}{(r - 3)!}f^{(r - 3)}(\lambda_i) \\
 & & & \ddots & \vdots \\
 & & & & f(\lambda_i)
\end{pmatrix}. 
\end{align*}
\end{lemma}
\begin{proof}
注意到
\begin{align*}
J^m = \mathrm{diag}\{J_1^m,J_2^m,\cdots,J_k^m\}.
\end{align*}
又
\begin{align*}
A^m = (PJP^{-1})^m = PJ^mP^{-1},
\end{align*}
因此要计算\(f(A)\),只需计算出\(J_i^m\)即可. 
利用二项式定理和数学归纳法不难证明
\begin{align*}
J_{i}^{m}=\left[ \lambda _iI_r+\left( \begin{matrix}
0&		1&		&		&		\\
&		0&		1&		&		\\
&		&		0&		\ddots&		\\
&		&		&		\ddots&		1\\
&		&		&		&		0\\
\end{matrix} \right) \right] ^m=\begin{pmatrix}
\lambda_i^m & \mathrm{C}_m^1\lambda_i^{m - 1} & \mathrm{C}_m^2\lambda_i^{m - 2} & \cdots & \cdots \\
& \lambda_i^m & \mathrm{C}_m^1\lambda_i^{m - 1} & \cdots & \cdots \\
& & \lambda_i^m & \cdots & \cdots \\
& & & \ddots & \vdots \\
& & & & \lambda_i^m
\end{pmatrix}.
\end{align*}
则不难算出
\begin{align*}
f(J_i)= \begin{pmatrix}
f(\lambda_i) & \frac{1}{1!}f'(\lambda_i) & \frac{1}{2!}f^{(2)}(\lambda_i) & \cdots & \frac{1}{(r - 1)!}f^{(r - 1)}(\lambda_i) \\
 & f(\lambda_i) & \frac{1}{1!}f'(\lambda_i) & \cdots & \frac{1}{(r - 2)!}f^{(r - 2)}(\lambda_i) \\
 & & f(\lambda_i) & \cdots & \frac{1}{(r - 3)!}f^{(r - 3)}(\lambda_i) \\
 & & & \ddots & \vdots \\
 & & & & f(\lambda_i)
\end{pmatrix}. 
\end{align*}
再由
\begin{align*}
f(A) &= f(PJP^{-1}) = Pf(J)P^{-1} \\
&= P f(\mathrm{diag}\{J_1,J_2,\cdots,J_k\})P^{-1} \\
&= P \mathrm{diag}\{f(J_1),f(J_2),\cdots,f(J_k)\}P^{-1},
\end{align*}
即可计算出\(f(A)\). 
\end{proof}

\begin{definition}[复方阵幂级数]
设有 $n$ 阶复方阵序列 $\{A_p\}$:
\begin{align*}
A_p = \begin{pmatrix}
a_{11}^{(p)} & \cdots & a_{1n}^{(p)} \\
\vdots & & \vdots \\
a_{n1}^{(p)} & \cdots & a_{nn}^{(p)}
\end{pmatrix},
\end{align*}
$B = (b_{ij})$ 是一个同阶方阵,若对每个 $(i,j)$,序列 $\{a_{ij}^{(p)}\}$ 均收敛于 $b_{ij}$,即
\begin{align*}
\lim_{p \to \infty} a_{ij}^{(p)} = b_{ij},
\end{align*}
则称矩阵序列 $\{A_p\}$ 收敛于 $B$,记为
\begin{align*}
\lim_{p \to \infty} A_p = B.
\end{align*}
否则称 $\{A_p\}$ 发散.

设
\begin{align*}
f(z) = a_0 + a_1z + a_2z^2 + \cdots
\end{align*}
是一个幂级数,记
\begin{align*}
f_p(z) = a_0 + a_1z + a_2z^2 + \cdots + a_pz^p
\end{align*}
是其部分和. 若矩阵序列 $\{f_p(A)\}$ 收敛于 $B$,则称矩阵级数
\begin{align*}
f(A) = a_0I + a_1A + a_2A^2 + \cdots .
\end{align*}
收敛,极限为 $B$,记为 $f(A) = B$. 否则称 $f(A)$ 发散. 用变量矩阵 $X$ 代替 $A$,便可定义矩阵幂级数
\begin{align*}
f(X) = a_0I + a_1X + a_2X^2 + \cdots. 
\end{align*} 
\end{definition}

\begin{theorem}\label{theorem:方阵幂级数收敛的条件}
设\(f(z)=\sum_{i = 0}^{\infty}a_{i}z^{i}\)是复幂级数, 则
\begin{enumerate}[(1)]
\item 方阵幂级数\(f(X)\)收敛的充分必要条件是对任一非异阵\(P\),\(f(P^{-1}XP)\)都收敛, 这时
\begin{align*}
f(P^{-1}XP) = P^{-1}f(X)P;
\end{align*}

\item 若\(X = \mathrm{diag}\{X_1,\cdots,X_k\}\), 则\(f(X)\)收敛的充分必要条件是\(f(X_1),\cdots,f(X_k)\)都收敛, 这时
\begin{align*}
f(X) = \mathrm{diag}\{f(X_1),\cdots,f(X_k)\};
\end{align*}

\item 若\(f(z)\)的收敛半径为\(r\),\(J_0\)是特征值为\(\lambda_0\)的\(n\)阶 Jordan 块
\begin{align*}
J_0 = \begin{pmatrix}
\lambda_0 & 1 & & \\
 & \lambda_0 & 1 & \\
 & & \ddots & \ddots \\
 & & & \ddots & 1 \\
 & & & & \lambda_0
\end{pmatrix},
\end{align*}
则当\(|\lambda_0| < r\)时\(f(J_0)\)收敛, 且
\begin{align}\label{equation:::::7.8.6}
f(J_0) = \begin{pmatrix}
f(\lambda_0) & \frac{1}{1!}f'(\lambda_0) & \frac{1}{2!}f^{(2)}(\lambda_0) & \cdots & \frac{1}{(n - 1)!}f^{(n - 1)}(\lambda_0) \\
 & f(\lambda_0) & \frac{1}{1!}f'(\lambda_0) & \cdots & \frac{1}{(n - 2)!}f^{(n - 2)}(\lambda_0) \\
 & & f(\lambda_0) & \cdots & \frac{1}{(n - 3)!}f^{(n - 3)}(\lambda_0) \\
 & & & \ddots & \vdots \\
 & & & & f(\lambda_0)
\end{pmatrix}.
\end{align} 
\end{enumerate}
\end{theorem}
\begin{proof}
设\(f_p(z)=a_0 + a_1z + a_2z^2 + \cdots + a_pz^p\)是\(f(z)\)前\(p + 1\)项的部分和.
\begin{enumerate}[(1)]
\item 注意到\(f_p(z)\)是多项式, 从而有
\begin{align*}
f_p(P^{-1}XP) = P^{-1}f_p(X)P.
\end{align*}
由于\(n\)阶矩阵序列的收敛等价于\(n^2\)个数值序列的收敛, 故
\begin{align*}
f(P^{-1}XP) &= \lim_{p \to \infty} f_p(P^{-1}XP) = \lim_{p \to \infty} P^{-1}f_p(X)P \\
&= P^{-1}(\lim_{p \to \infty} f_p(X))P = P^{-1}f(X)P.
\end{align*}

\item  注意到\(f_p(z)\)是多项式, 从而有
\begin{align*}
f_p(X) = f_p(\mathrm{diag}\{X_1,\cdots,X_k\}) = \mathrm{diag}\{f_p(X_1),\cdots,f_p(X_k)\}.
\end{align*}
由于分块矩阵序列的收敛等价于每个分块的矩阵序列的收敛, 故
\begin{align*}
f(X) &= \lim_{p \to \infty} f_p(X) = \lim_{p \to \infty} \mathrm{diag}\{f_p(X_1),\cdots,f_p(X_k)\} \\
&= \mathrm{diag}\{\lim_{p \to \infty} f_p(X_1),\cdots,\lim_{p \to \infty} f_p(X_k)\} = \mathrm{diag}\{f(X_1),\cdots,f(X_k)\}.
\end{align*}

\item  由\reflemma{lemma:利用Jordan标准型计算复矩阵的多项式}可知
\begin{align*}
f_p(J_0)= \begin{pmatrix}
f_p(\lambda_0) & \frac{1}{1!}f_p'(\lambda_0) & \frac{1}{2!}f_p^{(2)}(\lambda_0) & \cdots & \frac{1}{(n - 1)!}f_p^{(n - 1)}(\lambda_0) \\
 & f_p(\lambda_0) & \frac{1}{1!}f_p'(\lambda_0) & \cdots & \frac{1}{(n - 2)!}f_p^{(n - 2)}(\lambda_0) \\
 & & f_p(\lambda_0) & \cdots & \frac{1}{(n - 3)!}f_p^{(n - 3)}(\lambda_0) \\
 & & & \ddots & \vdots \\
 & & & & f_p(\lambda_0)
\end{pmatrix}. 
\end{align*}
令\(p \to \infty\), 由矩阵序列收敛与\(n^2\)个数值序列收敛的等价性和幂级数的相关性质即得结论.
\end{enumerate}
\end{proof}

\begin{theorem}\label{theorem:复方阵幂级数的谱半径}
设\(f(z)\)是复幂级数,收敛半径为\(r\). 设\(A\)是\(n\)阶复方阵,特征值为\(\lambda_1,\lambda_2,\cdots,\lambda_n\),定义\(A\)的\textbf{谱半径}
\begin{align*}
\rho(A) = \max_{1 \leq i \leq n}|\lambda_i|.
\end{align*}
\begin{enumerate}[(1)]
\item 若\(\rho(A) < r\),则\(f(A)\)收敛;

\item 若\(\rho(A) > r\),则\(f(A)\)发散;

\item 若\(\rho(A) = r\),则\(f(A)\)收敛的充分必要条件是:对每一模长等于\(r\)的特征值\(\lambda_j\),若\(A\)的属于\(\lambda_j\)的初等因子中最高幂为\(n_j\)次,则\(n_j\)个数值级数
\begin{align}
f(\lambda_j), f'(\lambda_j), \cdots, f^{(n_j - 1)}(\lambda_j) .\label{equation::::--7.8.6}
\end{align}
都收敛;

\item 若\(f(A)\)收敛,则\(f(A)\)的特征值为
\begin{align*}
f(\lambda_1), f(\lambda_2), \cdots, f(\lambda_n).
\end{align*} 
\end{enumerate}
\end{theorem}
\begin{proof}
\begin{enumerate}[(1)]
\item 设\(A\)的 Jordan 标准型为\(J = \mathrm{diag}\{J_1,J_2,\cdots,J_k\}\). 显然\(f(A)\)的收敛性等价于所有\(f(J_i)\)(\(i = 1,\cdots,k\))的收敛性. 由\reftheorem{theorem:方阵幂级数收敛的条件}即知(1)成立.

\item 若某一个\(|\lambda_j| > r\),则\(f(\lambda_j)\)发散,因此\(f(J_j)\)发散,故\(f(A)\)发散,这就证明了(2).

\item 当\(\rho(A) = r\)时,对\(|\lambda_i| < r\)的\(J_i\),\(f(J_i)\)收敛. 对\(|\lambda_j| = r\)的特征值\(\lambda_j\),注意到\(f(z)\)的任意次导数的收敛半径仍为\(r\),又初等因子\((\lambda - \lambda_j)^{n_j}\)对应的 Jordan 块为\(n_j\)阶,从\eqref{equation:::::7.8.6}式即可知道\(f(J_j)\)的收敛性等价于\eqref{equation::::--7.8.6}式中\(n_j\)个级数的收敛性.

\item 最后若\(f(A)\)收敛,则\(f(A)\)与\(f(J)\)有相同的特征值,即为\(f(\lambda_1),f(\lambda_2),\cdots,f(\lambda_n)\).
\end{enumerate}
\end{proof}

\begin{definition}\label{definition:矩阵函数}
于是对一切方阵,定义
\begin{align*}
\mathrm{e}^{\boldsymbol{A}}&= \boldsymbol{I}+\frac{1}{1!}\boldsymbol{A}+\frac{1}{2!}\boldsymbol{A}^{2}+\frac{1}{3!}\boldsymbol{A}^{3}+\cdots,\\
\sin\boldsymbol{A}&= \boldsymbol{A}-\frac{1}{3!}\boldsymbol{A}^{3}+\frac{1}{5!}\boldsymbol{A}^{5}-\frac{1}{7!}\boldsymbol{A}^{7}+\cdots,\\
\cos\boldsymbol{A}&= \boldsymbol{I}-\frac{1}{2!}\boldsymbol{A}^{2}+\frac{1}{4!}\boldsymbol{A}^{4}-\frac{1}{6!}\boldsymbol{A}^{6}+\cdots
\end{align*}
都有意义. 若 $\boldsymbol{A}$ 所有特征值的模长都小于 1,则 
\[
\ln(\boldsymbol{I}+\boldsymbol{A})=\boldsymbol{A}-\frac{1}{2}\boldsymbol{A}^{2}+\frac{1}{3}\boldsymbol{A}^{3}-\frac{1}{4}\boldsymbol{A}^{4}+\cdots
\]
也有意义. 同理还可以定义幂函数、双曲函数等. 
\end{definition}
\begin{remark}
由复分析知道:
\begin{align*}
&\mathrm{e}^{z}= 1+\frac{1}{1!}z+\frac{1}{2!}z^{2}+\frac{1}{3!}z^{3}+\cdots,\\
&\sin z= z-\frac{1}{3!}z^{3}+\frac{1}{5!}z^{5}-\frac{1}{7!}z^{7}+\cdots,\\
&\cos z= 1-\frac{1}{2!}z^{2}+\frac{1}{4!}z^{4}-\frac{1}{6!}z^{6}+\cdots,\\
&\ln(1 + z)=z-\frac{1}{2}z^{2}+\frac{1}{3}z^{3}-\frac{1}{4}z^{4}+\cdots.
\end{align*}
前 3 个级数在整个复平面上收敛,而 $\ln(1 + z)$ 的收敛半径为 1.于是由\reftheorem{theorem:复方阵幂级数的谱半径}可知$\mathrm{e}^{\boldsymbol{A}},\sin \boldsymbol{A},\cos \boldsymbol{A},\ln{\boldsymbol{A}}$都收敛,从而都有意义.
故上述定义是良定义的.
\end{remark}

\begin{proposition}\label{proposition:矩阵乘法可交换必成立e^A乘e^B=e^A+B}
如果 $\boldsymbol{A}$ 与 $\boldsymbol{B}$ 乘法可交换,即 $\boldsymbol{A}\boldsymbol{B}=\boldsymbol{B}\boldsymbol{A}$,则 $\mathrm{e}^{\boldsymbol{A}}\cdot\mathrm{e}^{\boldsymbol{B}}=\mathrm{e}^{\boldsymbol{A}+\boldsymbol{B}}$ 必成立.
\end{proposition}
\begin{remark}
对一般来说对矩阵 $\boldsymbol{A},\boldsymbol{B}$,下面的等式并不一定成立:
\begin{align*}
\mathrm{e}^{\boldsymbol{A}}\cdot\mathrm{e}^{\boldsymbol{B}}=\mathrm{e}^{\boldsymbol{A}+\boldsymbol{B}}.
\end{align*}
如对
\[
\boldsymbol{A}=\begin{pmatrix}
1 & 1 \\
0 & 0
\end{pmatrix}, \boldsymbol{B}=\begin{pmatrix}
0 & 0 \\
1 & 1
\end{pmatrix},
\]
不难验证 $\mathrm{e}^{\boldsymbol{A}}\cdot\mathrm{e}^{\boldsymbol{B}}\neq\mathrm{e}^{\boldsymbol{A}+\boldsymbol{B}}$. 
\end{remark}
\begin{proof}

\end{proof}

\begin{proposition}\label{proposition:计算e^tA}
设$t$ 是一个数值变量,$\boldsymbol{A}$ 是一个 $n$ 阶复方阵.
$\boldsymbol{P}^{-1}\boldsymbol{A}\boldsymbol{P}=\boldsymbol{J}=\mathrm{diag}\{\boldsymbol{J}_1,\cdots,\boldsymbol{J}_k\}$ 是 $\boldsymbol{A}$ 的 Jordan 标准型,$\boldsymbol{J}_i$ 是特征值为 $\lambda_i$ 的 $r$ 阶 Jordan 块,则
\begin{align*}
\mathrm{e}^{t\boldsymbol{A}}=\boldsymbol{P}\mathrm{e}^{t\boldsymbol{J}}\boldsymbol{P}^{-1},
\end{align*}
其中
\begin{align*}
\mathrm{e}^{t\boldsymbol{J}}=\left( \begin{matrix}
\mathrm{e}^{t\boldsymbol{J}_1}&		&		\\
&		\ddots&		\\
&		&		\mathrm{e}^{t\boldsymbol{J}_k}\\
\end{matrix} \right) ,\quad \mathrm{e}^{t\boldsymbol{J}_i}=\mathrm{e}^{t\lambda _i}\left( \begin{matrix}
1&		t&		\frac{1}{2!}t^2&		\frac{1}{3!}t^3&		\cdots&		\frac{1}{(r-1)!}t^{r-1}\\
&		1&		t&		\frac{1}{2!}t^2&		\cdots&		\frac{1}{(r-2)!}t^{r-2}\\
&		&		1&		t&		\cdots&		\frac{1}{(r-3)!}t^{r-3}\\
&		&		&		\ddots&		\ddots&		\vdots\\
&		&		&		&		\ddots&		t\\
&		&		&		&		&		1\\
\end{matrix} \right) .
\end{align*}
\end{proposition}
\begin{proof}
{\color{blue}证法一:}
若令 $f(z)=\mathrm{e}^{tz}$,则由\reftheorem{theorem:方阵幂级数收敛的条件}即得 $f(\boldsymbol{A}) = \mathrm{e}^{t\boldsymbol{A}}$ 的计算结果. 

{\color{blue}证法二:}
注意到
\begin{align*}
\boldsymbol{J}_i=\lambda_i\boldsymbol{I}+\boldsymbol{N},
\end{align*}
其中 $\boldsymbol{N}$ 是 $r$ 阶基础幂零阵,即
\begin{align*}
\boldsymbol{N}^r=\boldsymbol{O},\quad \boldsymbol{N}=\left( \begin{matrix}
0&		1&		&		&		\\
&		0&		1&		&		\\
&		&		\ddots&		\ddots&		\\
&		&		&		\ddots&		1\\
&		&		&		&		0\\
\end{matrix} \right) .
\end{align*}
于是
\begin{align*}
\mathrm{e}^{\boldsymbol{N}}=\boldsymbol{I}+\boldsymbol{N}+\frac{1}{2!}\boldsymbol{N}^2+\frac{1}{3!}\boldsymbol{N}^3+\cdots+\frac{1}{(r - 1)!}\boldsymbol{N}^{r - 1}.
\end{align*}
因为 $(\lambda_i\boldsymbol{I})\boldsymbol{N}=\boldsymbol{N}(\lambda_i\boldsymbol{I})$,故由\refproposition{proposition:矩阵乘法可交换必成立e^A乘e^B=e^A+B}可知
\begin{align*}
\mathrm{e}^{\boldsymbol{J}_i}&=\mathrm{e}^{\lambda_i\boldsymbol{I}+\boldsymbol{N}}=\mathrm{e}^{\lambda_i\boldsymbol{I}}\cdot\mathrm{e}^{\boldsymbol{N}}=\mathrm{e}^{\lambda_i}\cdot\mathrm{e}^{\boldsymbol{N}}\\
&=\mathrm{e}^{\lambda_i}\boldsymbol{I}+\mathrm{e}^{\lambda_i}\boldsymbol{N}+\frac{1}{2!}\mathrm{e}^{\lambda_i}\boldsymbol{N}^2+\cdots+\frac{1}{(r - 1)!}\mathrm{e}^{\lambda_i}\boldsymbol{N}^{r - 1}.
\end{align*}
同理
\begin{align*}
\mathrm{e}^{t\boldsymbol{J}_i}&=\mathrm{e}^{t\lambda_i}\cdot\mathrm{e}^{t\boldsymbol{N}}=\mathrm{e}^{t\lambda _i}\left[ t\boldsymbol{I}+t\boldsymbol{N}+\frac{t}{2!}\boldsymbol{N}^2+\frac{t}{3!}\boldsymbol{N}^3+\cdots +\frac{t}{(r-1)!}\boldsymbol{N}^{r-1} \right] 
\\
&=\mathrm{e}^{t\lambda_i}
\begin{pmatrix}
1 & t & \frac{1}{2!}t^2 & \frac{1}{3!}t^3 & \cdots & \frac{1}{(r - 1)!}t^{r - 1} \\
 & 1 & t & \frac{1}{2!}t^2 & \cdots & \frac{1}{(r - 2)!}t^{r - 2} \\
 & & 1 & t & \cdots & \frac{1}{(r - 3)!}t^{r - 3} \\
 & & & \ddots & \ddots & \vdots \\
 & & & & \ddots & t \\
 & & & & & 1
\end{pmatrix}.
\end{align*}
又注意到
\begin{align*}
\mathrm{e}^{t\boldsymbol{A}}&=\mathrm{e}^{\boldsymbol{P}(t\boldsymbol{J})\boldsymbol{P}^{-1}}=\boldsymbol{P}\mathrm{e}^{t\boldsymbol{J}}\boldsymbol{P}^{-1},\\
\mathrm{e}^{t\boldsymbol{J}}&=\begin{pmatrix}
\mathrm{e}^{t\boldsymbol{J}_1} & & \\
 & \ddots & \\
 & & \mathrm{e}^{t\boldsymbol{J}_k}
\end{pmatrix}.
\end{align*}
于是将 $\mathrm{e}^{t\boldsymbol{J}_i}$ 的式子代入上面的式子即可求出 $\mathrm{e}^{t\boldsymbol{A}}$. 
\end{proof}











\end{document}
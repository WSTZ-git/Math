\documentclass[../../main.tex]{subfiles}
\graphicspath{{\subfix{../../image/}}} % 指定图片目录,后续可以直接使用图片文件名。

% 例如:
% \begin{figure}[h]
% \centering
% \includegraphics{image-01.01}
% \caption{图片标题}
% \label{fig:image-01.01}
% \end{figure}
% 注意:上述\label{}一定要放在\caption{}之后,否则引用图片序号会只会显示??.

\begin{document}

\section{乘法交换性诱导的多项式表示}

\begin{theorem}\label{theorem:极小多项式等于特征多项式的充要条件}
设 $\varphi$ 是数域 $\mathbb{K}$ 上 $n$ 维线性空间 $V$ 上的线性变换,则对 $V$ 上任一与 $\varphi$ 乘法可交换的线性变换 $\psi$,都存在不超过 $n - 1$ 次的多项式 $g(x)\in\mathbb{K}[x]$,使得 $\psi = g(\varphi)$ 成立的充要条件是 $\varphi$ 的极小多项式等于其特征多项式。
\end{theorem}
\begin{remark}
本题充分性证明的关键点是:$V = C(\varphi,e_1)$ 是一个循环空间,循环向量 $e_1$ 经过 $\varphi$ 的 $n - 1$ 次作用,生成了 $V$ 的一组基 $\{e_1,e_2,\cdots,e_n\}$。因此,只要验证了 $\psi$ 和 $g(\varphi)$ 在循环向量 $e_1$ 上的取值相同,那么由 $\varphi,\psi$ 的乘法交换性可知 $\psi$ 和 $g(\varphi)$ 在上述基上的取值也相同,从而它们必相等.
\end{remark}
\begin{proof}
{\heiti 充分性:}设 $\varphi$ 的极小多项式等于其特征多项式 $f(\lambda)=\lambda^n + a_1\lambda^{n - 1}+\cdots + a_{n - 1}\lambda + a_n$,则 $\varphi$ 只有一个非常数不变因子。由\hyperref[theorem:有理标准型核心定理]{有理标准型理论},存在 $V$ 的一组基 $\{e_1,e_2,\cdots,e_n\}$,使得 $\varphi$ 在这组基下的表示矩阵为友阵
\begin{align*}
C(f(\lambda))=\begin{pmatrix}
0 & 0 & \cdots & 0 & -a_n \\
1 & 0 & \cdots & 0 & -a_{n - 1} \\
0 & 1 & \cdots & 0 & -a_{n - 2} \\
\vdots & \vdots & & \vdots & \vdots \\
0 & 0 & \cdots & 1 & -a_1
\end{pmatrix}
\end{align*}
即有
\begin{align*}
\varphi(e_1)=e_2,\varphi(e_2)=e_3,\cdots,\varphi(e_{n - 1})=e_n,\varphi(e_n)=-a_ne_1 - a_{n - 1}e_2 - \cdots - a_1e_n
\end{align*}
任取 $V$ 上满足 $\varphi\psi = \psi\varphi$ 的线性变换 $\psi$,设
\begin{align}
\psi(e_1)=b_ne_1 + b_{n - 1}e_2 + \cdots + b_1e_n\label{equation:::::::7.7}
\end{align}
令 $g(x)=b_1x^{n - 1}+\cdots + b_{n - 1}x + b_n$,我们来证明:$\psi = g(\varphi)$。首先由 $e_k=\varphi^{k - 1}(e_1)(k\geq 2)$ 以及 \eqref{equation:::::::7.7} 式可知 $\psi(e_1)=g(\varphi)(e_1)$ 成立。其次由 $\varphi,\psi$ 乘法可交换,故对任意的 $e_k(k\geq 2)$ 有
\begin{align*}
\psi(e_k)&=\psi(\varphi^{k - 1}(e_1))=\varphi^{k - 1}(\psi(e_1))=\varphi^{k - 1}(g(\varphi)(e_1))\\
&=g(\varphi)(\varphi^{k - 1}(e_1))=g(\varphi)(e_k)
\end{align*}
最后,注意到 $\psi$ 与 $g(\varphi)$ 在基向量 $\{e_1,e_2,\cdots,e_n\}$ 上的取值都相等,故由线性扩张定理可知 $\psi = g(\varphi)$ 成立。

{\heiti 必要性:}设 $\varphi$ 的不变因子组为 $1,\cdots,1,d_1(\lambda),\cdots,d_k(\lambda)$,其中 $d_i(\lambda)$ 为非常数首一多项式,$d_i(\lambda)\mid d_{i + 1}(\lambda) (1\leq i\leq k - 1)$,则 $\varphi$ 的有理标准型 $\boldsymbol{F}=\mathrm{diag}\{\boldsymbol{F}_1,\boldsymbol{F}_2,\cdots,\boldsymbol{F}_k\}$,其中 $\boldsymbol{F}_i = \boldsymbol{F}(d_i(\lambda))$ 为 $n_i$ 阶矩阵。若 $\varphi$ 的极小多项式不等于其特征多项式,则 $k\geq 2$。构造分块对角矩阵
\begin{align*}
\boldsymbol{B}=\mathrm{diag}\{\boldsymbol{I}_{n_1},\boldsymbol{O}_{n_2},\cdots,\boldsymbol{O}_{n_k}\}
\end{align*}
显然 $\boldsymbol{BF}=\boldsymbol{FB}$。用反证法,若存在多项式 $g(x)$,使得 $\boldsymbol{B}=g(\boldsymbol{F})$,即
\begin{align*}
\boldsymbol{B}=\mathrm{diag}\{g(\boldsymbol{F}_1),g(\boldsymbol{F}_2),\cdots,g(\boldsymbol{F}_k)\}
\end{align*}
则 $g(\boldsymbol{F}_1)=\boldsymbol{I}_{n_1}$,$g(\boldsymbol{F}_i)=\boldsymbol{O}(i\geq 2)$。由\reflemma{lemma:Frobenius标准型矩阵的极小多项式和不变因子}可知 $d_k(\lambda)$ 是 $\boldsymbol{F}_k$ 的极小多项式(也是特征多项式),故 $d_k(\lambda)\mid g(\lambda)$,从而 $d_1(\lambda)\mid g(\lambda)$,于是 $g(\boldsymbol{F}_1)=\boldsymbol{O}$,矛盾!因此 $\boldsymbol{B}$ 不能表示为 $\boldsymbol{F}$ 的多项式,从而由 $\boldsymbol{B}$ 定义的线性变换 $\psi$ 与$\varphi$可交换,但是不能表示为$\varphi$的多项式,矛盾!
\end{proof}

\begin{corollary}\label{corollary:循环子空间的另一种刻画}
设 $\varphi$ 是数域 $\mathbb{K}$ 上 $n$ 维线性空间 $V$ 上的线性变换,$\mathbb{K}[\varphi]=\{f(\varphi)\mid f(x)\in\mathbb{K}[x]\}$,$C(\varphi)=\{\psi\in\mathcal{L}(V)\mid\varphi\psi = \psi\varphi\}$,则 $V$ 是关于 $\varphi$ 的循环空间的充要条件是 $C(\varphi)=\mathbb{K}[\varphi]$。此时,$C(\varphi)$ 的一组基为 $\{\boldsymbol{I}_V,\varphi,\cdots,\varphi^{n - 1}\}$。 
\end{corollary}
\begin{note}
\reftheorem{theorem:循环子空间的刻画}证明了:线性变换 $\varphi$ 的极小多项式等于其特征多项式当且仅当 $V$ 是关于 $\varphi$ 的循环空间。因此,作为\reftheorem{theorem:极小多项式等于特征多项式的充要条件}的推论,我们给出了循环空间的另一刻画.
\end{note}
\begin{proof}
由\reftheorem{theorem:极小多项式等于特征多项式的充要条件}及$f(\varphi)$全都与$\varphi$可交换以及\reftheorem{theorem:循环子空间的刻画}立得.
\end{proof}

\begin{example}
设数域 $\mathbb{K}$ 上的 $n$ 阶矩阵 $A$ 的特征多项式 $f(\lambda)=P_1(\lambda)P_2(\lambda)\cdots P_k(\lambda)$,其中 $P_i(\lambda) (1\leq i\leq k)$ 是 $\mathbb{K}$ 上互异的首一不可约多项式。设 $\mathbb{K}$ 上的 $n$ 阶矩阵 $B$ 满足 $AB = BA$,求证:存在 $\mathbb{K}$ 上次数不超过 $n - 1$ 的多项式 $f(x)$,使得 $B = f(A)$。
\end{example}
\begin{proof}
由\reftheorem{theorem:循环子空间的刻画}可知,$\mathbb{K}^n$ 是关于 $A$ 的循环空间,再由\reftheorem{theorem:极小多项式等于特征多项式的充要条件}即得结论。
\end{proof}

\begin{example}
设 $A$ 是数域 $\mathbb{K}$ 上的 $2$ 阶矩阵,试求 $C(A)=\{X\in M_2(\mathbb{K})\mid AX = XA\}$。
\end{example}
\begin{proof}
若 $A$ 的极小多项式等于特征多项式,则由\reftheorem{theorem:极小多项式等于特征多项式的充要条件}可知 $C(A)=\mathbb{K}[A]$。若极小多项式不等于特征多项式,则极小多项式必为一次多项式 $x - c$,从而 $A = c\boldsymbol{I}_2$,于是 $C(A)=M_2(\mathbb{K})$。
\end{proof}

\begin{example}
设数域 $\mathbb{K}$ 上的 $n$ 阶矩阵
\begin{align*}
A=\begin{pmatrix}
a_1 & b_1 & 0 & \cdots & 0 \\
& \ddots & \ddots & \ddots & \vdots \\
& & \ddots & \ddots & 0 \\
& * & & \ddots & b_{n - 1} \\
& & & & a_n
\end{pmatrix}
\end{align*}
其中 $b_1,\cdots,b_{n - 1}$ 均不为零。记 $C(A)=\{X\in M_n(\mathbb{K})\mid AX = XA\}$,证明:线性空间 $C(A)$ 的一组基为 $\{\boldsymbol{I}_n,A,\cdots,A^{n - 1}\}$。
\end{example}
\begin{proof}
题目中的 $A$ 是类下三角矩阵,上次对角元全部非零,比如 Frobenius 块、Jordan 块和三对角矩阵都满足这样的特点。考虑特征矩阵 $\lambda\boldsymbol{I}_n - A$ 的前 $n - 1$ 行、后 $n - 1$ 列构成的下三角行列式,其值为 $(-1)^{n - 1}b_1\cdots b_{n - 1}\neq 0$,于是$A$的第$n-1$个行列式因子为$d_{n-1}=1$,故 $A$ 的行列式因子组为 $1,\cdots,1,f(\lambda)$,从而 $\mathbb{K}^n$ 是关于 $A$ 的循环空间,再由\reftheorem{theorem:极小多项式等于特征多项式的充要条件}即得结论.
\end{proof}

\begin{proposition}\label{proposition:可以将分块对角矩阵的多项式表示问题归结为对每一分块的讨论的条件}
设有 $n$ 阶分块对角矩阵
\begin{align*}
A=\begin{pmatrix}
\boldsymbol{A}_1 & & \\
& \ddots & \\
& & \boldsymbol{A}_k
\end{pmatrix}, \quad B=\begin{pmatrix}
\boldsymbol{B}_1 & & \\
& \ddots & \\
& & \boldsymbol{B}_k
\end{pmatrix}
\end{align*}
其中 $\boldsymbol{A}_i$ 和 $\boldsymbol{B}_i$ 是同阶方阵。设 $\boldsymbol{A}_i$ 适合非零多项式 $g_i(x)$,且 $g_i(x) (1\leq i\leq k)$ 两两互素。求证:若对每个 $i$,存在多项式 $f_i(x)$,使得 $\boldsymbol{B}_i = f_i(\boldsymbol{A}_i)$,则必存在次数不超过 $n - 1$ 的多项式 $f(x)$,使得 $B = f(A)$。
\end{proposition}
\begin{note}
这个命题告诉我们,在什么条件下可以将分块对角矩阵的多项式表示问题归结为对每一分块的讨论。
\end{note}
\begin{proof}
因为 $g_i(x)$ 两两互素,故由\hyperref[theorem:中国剩余定理]{中国剩余定理}可知,存在多项式 $h(x)$ 满足 $h(x)=g_i(x)q_i(x)+f_i(x)$。将 $x = \boldsymbol{A}_i$ 代入上式,可得 $h(\boldsymbol{A}_i)=f_i(\boldsymbol{A}_i)=\boldsymbol{B}_i$,从而
\begin{align*}
h(A)=\mathrm{diag}\{h(\boldsymbol{A}_1),\cdots,h(\boldsymbol{A}_k)\}=\mathrm{diag}\{\boldsymbol{B}_1,\cdots,\boldsymbol{B}_k\}=B
\end{align*}
设 $A$ 的特征多项式为 $g(x)$,作带余除法 $h(x)=g(x)q(x)+f(x)$,其中 $\deg f(x)<n$。将 $x = A$ 代入上式,则由 Cayley - Hamilton 定理可得 $B = h(A)=f(A)$。
\end{proof}

\begin{lemma}\label{lemma:向量乘积为零的结论}
设$\alpha,\beta$均为列向量,若$\alpha \beta'=O$,则$\alpha=0$或$\beta=0$.
\end{lemma}
\begin{proof}
证明是显然的(反证或者直接写出$\alpha \beta'$的矩阵都可以).
\end{proof}

\begin{proposition}\label{proposition:与秩为n-1的矩阵乘法可交换的矩阵也可被其多项式表示}
设 $n$ 阶矩阵 $A$ 的秩等于 $n - 1$,$B$ 是同阶非零矩阵且 $AB = BA = O$,求证:存在次数不超过 $n - 1$ 的多项式 $f(x)$,使得 $B = f(A)$。
\end{proposition}
\begin{remark}
对适合 $AB = BA$ 的矩阵,由于 $AB = BA$ 当且仅当 $(P^{-1}AP)(P^{-1}BP)=(P^{-1}BP)(P^{-1}AP)$,因此我们可以通过同时相似变换,把问题归结为其中一个矩阵是相似标准型(或分块对角型矩阵)的情形来证明。
\end{remark}
\begin{proof}
{\color{blue}证法一:}
由于题目的条件和结论在同时相似变换:$A\mapsto P^{-1}AP$,$B\mapsto P^{-1}BP$ 下保持不变,故不妨从一开始就假设 $A$ 为 Jordan 标准型。因为 $\mathrm{r}(A)=n - 1$,从而线性方程$Ax=O$的解空间维数为$n-r(A)=1$,故 $A$ 关于特征值 $0$ 的几何重数为 $1$,从而属于特征值 $0$ 的 Jordan 块只有一个,记为 $J_0$;将属于其他非零特征值的 Jordan 块合在一起,记为 $J_1$,于是 $A = \mathrm{diag}\{J_0,J_1\}$。设 $B=\begin{pmatrix}
B_{11} & B_{12} \\
B_{21} & B_{22}
\end{pmatrix}$ 为相应的分块,则由 $AB = BA = O$ 可得 $B_{12},B_{21},B_{22}$ 都是零矩阵,于是 $B = \mathrm{diag}\{B_{11},O\}$ 且 $J_0B_{11}=B_{11}J_0$。由于 $J_0$ 是幂零矩阵而 $J_1$ 是可逆矩阵,并且$J_0$的特征值只有零,$J_1$的特征值全非零,故 $J_0$ 的特征多项式 $g_0(x)$ 和 $J_1$ 的特征多项式 $g_1(x)$ 互素;又由\hyperref[proposition:Jordan块的性质]{Jordan块的性质(5)}可知,存在多项式 $f_0(x)$,使得 $B_{11}=f_0(J_0)$;再取 $f_1(x)=0$,则 $O = f_1(J_1)$;最后由\refproposition{proposition:可以将分块对角矩阵的多项式表示问题归结为对每一分块的讨论的条件}即得结论。

{\color{blue}证法二:}
也可以用线性方程组的求解理论和极小多项式来做。由于 $\mathrm{r}(A)=n - 1$,故线性方程组 $Ax = 0$ 解空间的维数为 $1$,再由 $AB = O$ 可知,$B$ 的列向量都是解空间的向量,从而它们成比例,于是 $\mathrm{r}(B)=1$。由\reflemma{lemma:秩1矩阵的列向量分解}可设 $B = \alpha\beta'$,其中 $\alpha,\beta$ 为 $n$ 维非零列向量,由 $AB = O$ 可推出 $A\alpha = 0$,即 $\alpha$ 是线性方程组 $Ax = 0$ 的基础解系。同理,由 $BA = O$ 可推出 $\beta$ 是 $A'x = 0$ 的基础解系。设 $m(x)$ 是 $A$ 的极小多项式,由于 $A$ 不是可逆矩阵,故由\refproposition{proposition:矩阵可逆充要条件极小多项式常数项非零}可知$m(x)$ 的常数项等于零,从而存在次数比$m(x)$低一次的多项式$g(x)$,使得$m(x)=xg(x)$,于是 $Ag(A)=O$ 但 $g(A)\neq O$,否则,若$g(A)=O$,则这与$m(x)$是$A$的极小多项式矛盾! 由类似于矩阵 $B$ 的讨论可得,$g(A)$ 也可以写为 $g(A)=\eta\xi'$,其中 $\eta$ 是 $Ax = 0$ 的解,故 $\eta = k\alpha$。同理可得 $\xi = t\beta$,于是 $g(A)=ktB$,从而 $B = \frac{1}{kt}g(A)$ 可表示为 $A$ 的次数不超过 $n - 1$ 的多项式。
\end{proof}












\end{document}
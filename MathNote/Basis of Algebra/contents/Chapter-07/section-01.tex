\documentclass[../../main.tex]{subfiles}
\graphicspath{{\subfix{../../image/}}} % 指定图片目录,后续可以直接使用图片文件名。

% 例如:
% \begin{figure}[h]
% \centering
% \includegraphics{image-01.01}
% \caption{图片标题}
% \label{fig:image-01.01}
% \end{figure}
% 注意:上述\label{}一定要放在\caption{}之后,否则引用图片序号会只会显示??.

\begin{document}

\section{多项式矩阵}

\begin{definition}[$\lambda$-矩阵]
一般地, 下列形式的矩阵:
\begin{align*}
\boldsymbol{A}(\lambda)=
\begin{pmatrix}
a_{11}(\lambda) & a_{12}(\lambda) & \cdots & a_{1n}(\lambda) \\
a_{21}(\lambda) & a_{22}(\lambda) & \cdots & a_{2n}(\lambda) \\
\vdots & \vdots & & \vdots \\
a_{m1}(\lambda) & a_{m2}(\lambda) & \cdots & a_{mn}(\lambda)
\end{pmatrix},
\end{align*}
其中 $a_{ij}(\lambda)$ 是以 $\lambda$ 为未定元的数域 $\mathbb{K}$ 上的多项式, 称为\textbf{多项式矩阵}, 或 $\boldsymbol{\lambda}$\textbf{-矩阵}. $\lambda$-矩阵的加法、数乘及乘法与数域上的矩阵运算一样, 只需在运算过程中将数的运算代之于多项式即可.
\end{definition}

\begin{definition}[$\lambda$-矩阵的初等变换]
对 $\lambda$-矩阵 $A(\lambda)$ 施行的下列 3 种变换称为 $\lambda$-矩阵的初等行变换:
\begin{enumerate}[(1)]
\item 将 $A(\lambda)$ 的两行对换;

\item 将 $A(\lambda)$ 的第 $i$ 行乘以 $\mathbb{K}$ 中的非零常数 $c$;

\item 将 $A(\lambda)$ 的第 $i$ 行乘以 $\mathbb{K}$ 上的多项式 $f(\lambda)$ 后加到第 $j$ 行上去.
\end{enumerate}

同理我们可以定义 3 种 $\lambda$-矩阵的初等列变换.
\end{definition}

\begin{definition}[$\lambda$-矩阵的相抵]
若 $A(\lambda),B(\lambda)$ 是同阶 $\lambda$-矩阵且 $A(\lambda)$ 经过 $\lambda$-矩阵的初等变换后可变为 $B(\lambda)$, 则称 $A(\lambda)$ 与 $B(\lambda)$\textbf{相抵}.

与数字矩阵一样, $\lambda$-矩阵的相抵关系也是一种等价关系, 即
\begin{enumerate}[(1)]
\item $A(\lambda)$ 与自身相抵;

\item 若 $A(\lambda)$ 与 $B(\lambda)$ 相抵, 则 $B(\lambda)$ 与 $A(\lambda)$ 相抵;

\item 若 $A(\lambda)$ 与 $B(\lambda)$ 相抵, $B(\lambda)$ 与 $C(\lambda)$ 相抵, 则 $A(\lambda)$ 与 $C(\lambda)$ 相抵.
\end{enumerate}
\end{definition}
\begin{note}
$\lambda$-矩阵的相抵关系也是一种等价关系的证明与数域上相同,类似易证.
\end{note}

\begin{definition}[初等$\lambda$-矩阵]
下列 3 种矩阵称为初等 $\lambda$-矩阵:
\begin{enumerate}[(1)]
\item 将 $n$ 阶单位阵的第 $i$ 行与第 $j$ 行对换, 记为 $\boldsymbol{P}_{ij}$;

\item 将 $n$ 阶单位阵的第 $i$ 行乘以非零常数 $c$, 记为 $\boldsymbol{P}_{i}(c)$;

\item 将 $n$ 阶单位阵的第 $i$ 行乘以多项式 $f(\lambda)$ 后加到第 $j$ 行上去得到的矩阵, 记为 $\boldsymbol{T}_{ij}(f(\lambda))$.
\end{enumerate}
\end{definition}
\begin{remark}
第一类与第二类初等 $\lambda$-矩阵与数域上的第一类与第二类初等矩阵没有什么区别. 第三类初等 $\lambda$-矩阵的形状如下:
\begin{align*}
\boldsymbol{T}_{ij}(f(\lambda))=
\begin{pmatrix}
1 & & & & & \\
& \ddots & & & & \\
& & 1 & & & \\
& & \vdots & \ddots & & \\
& & f(\lambda) & \cdots & 1 & \\
& & & & & \ddots & \\
& & & & & & 1
\end{pmatrix}.
\end{align*} 
\end{remark}

\begin{theorem}
对 $\lambda$-矩阵 $\boldsymbol{A}(\lambda)$ 施行第 $k$ ($k = 1,2,3$) 类初等行 (列) 变换等于用第 $k$ 类初等 $\lambda$-矩阵左 (右) 乘以 $\boldsymbol{A}(\lambda)$. 
\end{theorem}
\begin{remark}
下列 $\lambda$-矩阵的变换不是 $\lambda$-矩阵的初等变换:
\[
\begin{pmatrix}
1 & 1 \\
0 & 1
\end{pmatrix}
\rightarrow
\begin{pmatrix}
\lambda & \lambda \\
0 & 1
\end{pmatrix}.
\]
这是因为前面一个矩阵的第一行乘以 $\lambda$ 不是 $\lambda$-矩阵的初等变换. 同理下面的变换需第一行乘以 $\lambda^{-1}$, 因此也不是 $\lambda$-矩阵的初等变换:
\[
\begin{pmatrix}
\lambda & 0 \\
0 & 1
\end{pmatrix}
\rightarrow
\begin{pmatrix}
1 & 0 \\
0 & 1
\end{pmatrix}.
\] 
\end{remark}
\begin{proof}
证明是显然的.
\end{proof}

\begin{definition}[可逆$\lambda$-矩阵]
若 $\boldsymbol{A}(\lambda),\boldsymbol{B}(\lambda)$ 都是 $n$ 阶 $\lambda$-矩阵, 且
\begin{align*}
\boldsymbol{A}(\lambda)\boldsymbol{B}(\lambda)=\boldsymbol{B}(\lambda)\boldsymbol{A}(\lambda)=\boldsymbol{I}_n,
\end{align*}
则称 $\boldsymbol{B}(\lambda)$ 是 $\boldsymbol{A}(\lambda)$ 的逆 $\lambda$-矩阵. 这时称 $\boldsymbol{A}(\lambda)$ 为\textbf{可逆$\lambda$-矩阵}, 在不引起混淆的情形下, 有时简称为\textbf{可逆阵}. 
\end{definition}
\begin{note}
容易证明,\textbf{有限个可逆 $\lambda$-矩阵之积仍是可逆 $\lambda$-矩阵}, 而\textbf{初等 $\lambda$-矩阵都是可逆 $\lambda$-矩阵}, 因此\textbf{有限个初等 $\lambda$-矩阵之积也是可逆的}. $\lambda$-矩阵.  
\end{note}
\begin{remark}
注意不要将数字矩阵中的一些结论随意搬到 $\lambda$-矩阵上. 比如下面的 $\lambda$-矩阵的行列式不为零, 但它不是可逆 $\lambda$-矩阵:
\[
\begin{pmatrix}
\lambda & 0 \\
0 & 1
\end{pmatrix}.
\]
这是因为矩阵
\[
\begin{pmatrix}
\lambda^{-1} & 0 \\
0 & 1
\end{pmatrix}
\]
不是 $\lambda$-矩阵之故. 
\end{remark}

\begin{lemma}
设 $\boldsymbol{M}(\lambda)$ 是一个 $n$ 阶 $\lambda$-矩阵, 则 $\boldsymbol{M}(\lambda)$ 可以化为如下形状:
\begin{align*}
\boldsymbol{M}(\lambda)=\boldsymbol{M}_m\lambda^m+\boldsymbol{M}_{m - 1}\lambda^{m - 1}+\cdots+\boldsymbol{M}_0,
\end{align*}
其中 $\boldsymbol{M}_i$ 为数域 $\mathbb{K}$ 上的 $n$ 阶数字矩阵. 因此, 一个多项式矩阵可以化为系数为矩阵的多项式, 反之亦然. 
\end{lemma}
\begin{proof}
证明是显然的.
\end{proof}

\begin{lemma}\label{lemma:lambda-矩阵的性质1}
设 $\boldsymbol{M}(\lambda)$ 与 $\boldsymbol{N}(\lambda)$ 是两个 $n$ 阶 $\lambda$-矩阵且都不等于零. 又设 $\boldsymbol{B}$ 为 $n$ 阶数字矩阵, 则必存在 $\lambda$-矩阵 $\boldsymbol{Q}(\lambda)$ 及 $\boldsymbol{S}(\lambda)$ 和数字矩阵 $\boldsymbol{R}$ 及 $\boldsymbol{T}$, 使下式成立:
\begin{align}
\boldsymbol{M}(\lambda)&=(\lambda\boldsymbol{I}-\boldsymbol{B})\boldsymbol{Q}(\lambda)+\boldsymbol{R}, \label{lemma0.2-7.1.1}\\
\boldsymbol{N}(\lambda)&=\boldsymbol{S}(\lambda)(\lambda\boldsymbol{I}-\boldsymbol{B})+\boldsymbol{T}. \label{lemma0.2-7.1.2}
\end{align}
\end{lemma}
\begin{proof}
将 $\boldsymbol{M}(\lambda)$ 写为
\begin{align*}
\boldsymbol{M}(\lambda)=\boldsymbol{M}_m\lambda^m+\boldsymbol{M}_{m - 1}\lambda^{m - 1}+\cdots+\boldsymbol{M}_0,
\end{align*}
其中 $\boldsymbol{M}_m\neq\boldsymbol{O}$. 可对 $m$ 用归纳法, 若 $m = 0$, 则已适合要求 (取 $\boldsymbol{Q}(\lambda)=\boldsymbol{O}$). 现设对小于 $m$ 次的矩阵多项式,\eqref{lemma0.2-7.1.1}式成立. 令
\begin{align*}
\boldsymbol{Q}_1(\lambda)=\boldsymbol{M}_m\lambda^{m - 1},
\end{align*}
则
\begin{align}
\boldsymbol{M}(\lambda)-(\lambda\boldsymbol{I}-\boldsymbol{B})\boldsymbol{Q}_1(\lambda)&=(\boldsymbol{B}\boldsymbol{M}_m+\boldsymbol{M}_{m - 1})\lambda^{m - 1}+\cdots+\boldsymbol{M}_0. \label{lemma0.2-7.1.3}
\end{align}
上式是一个次数小于 $m$ 的矩阵多项式, 由归纳假设得
\begin{align*}
\boldsymbol{M}(\lambda)-(\lambda\boldsymbol{I}-\boldsymbol{B})\boldsymbol{Q}_1(\lambda)=(\lambda\boldsymbol{I}-\boldsymbol{B})\boldsymbol{Q}_2(\lambda)+\boldsymbol{R}.
\end{align*}
于是
\begin{align*}
\boldsymbol{M}(\lambda)=(\lambda\boldsymbol{I}-\boldsymbol{B})[\boldsymbol{Q}_1(\lambda)+\boldsymbol{Q}_2(\lambda)]+\boldsymbol{R}.
\end{align*}
令 $\boldsymbol{Q}(\lambda)=\boldsymbol{Q}_1(\lambda)+\boldsymbol{Q}_2(\lambda)$ 即得\eqref{lemma0.2-7.1.1}式. 同理可证 \eqref{lemma0.2-7.1.2}式. 
\end{proof}

\begin{theorem}\label{theorem:矩阵相似的充要条件是对应的lamda-矩阵相抵}
设 $\boldsymbol{A},\boldsymbol{B}$ 是数域 $\mathbb{K}$ 上的矩阵, 则 $\boldsymbol{A}$ 与 $\boldsymbol{B}$ 相似的充分必要条件是 $\lambda$-矩阵 $\lambda\boldsymbol{I}-\boldsymbol{A}$ 与 $\lambda\boldsymbol{I}-\boldsymbol{B}$ 相抵.
\end{theorem}
\begin{proof}
若 $\boldsymbol{A}$ 与 $\boldsymbol{B}$ 相似, 则存在 $\mathbb{K}$ 上的非异阵 $\boldsymbol{P}$, 使 $\boldsymbol{B}=\boldsymbol{P}^{-1}\boldsymbol{A}\boldsymbol{P}$, 于是
\begin{align*}
\boldsymbol{P}^{-1}(\lambda\boldsymbol{I}-\boldsymbol{A})\boldsymbol{P}&=\lambda\boldsymbol{I}-\boldsymbol{P}^{-1}\boldsymbol{A}\boldsymbol{P}=\lambda\boldsymbol{I}-\boldsymbol{B}. 
\end{align*}
把 $\boldsymbol{P}$ 看成是常数 $\lambda$-矩阵, 上式表明 $\lambda\boldsymbol{I}-\boldsymbol{A}$ 与 $\lambda\boldsymbol{I}-\boldsymbol{B}$ 相抵.

反过来, 若 $\lambda\boldsymbol{I}-\boldsymbol{A}$ 与 $\lambda\boldsymbol{I}-\boldsymbol{B}$ 相抵, 即存在 $\boldsymbol{M}(\lambda)$ 及 $\boldsymbol{N}(\lambda)$, 使
\begin{align}
\boldsymbol{M}(\lambda)(\lambda\boldsymbol{I}-\boldsymbol{A})\boldsymbol{N}(\lambda)&=\lambda\boldsymbol{I}-\boldsymbol{B}, \label{theorem1213-7.1.5}
\end{align}
其中 $\boldsymbol{M}(\lambda)$ 与 $\boldsymbol{N}(\lambda)$ 都是有限个初等矩阵之积, 因而都是可逆阵. 因此可将 \eqref{theorem1213-7.1.5}式写为
\begin{align}
\boldsymbol{M}(\lambda)(\lambda\boldsymbol{I}-\boldsymbol{A})&=(\lambda\boldsymbol{I}-\boldsymbol{B})\boldsymbol{N}(\lambda)^{-1}, \label{theorem1213-7.1.6}
\end{align}
由\hyperref[lemma:lambda-矩阵的性质1]{引理\ref{lemma:lambda-矩阵的性质1}}可设
\begin{align*}
\boldsymbol{M}(\lambda)&=(\lambda\boldsymbol{I}-\boldsymbol{B})\boldsymbol{Q}(\lambda)+\boldsymbol{R},
\end{align*}
代入 \eqref{theorem1213-7.1.6}式经整理得
\begin{align*}
\boldsymbol{R}(\lambda\boldsymbol{I}-\boldsymbol{A})&=(\lambda\boldsymbol{I}-\boldsymbol{B})[\boldsymbol{N}(\lambda)^{-1}-\boldsymbol{Q}(\lambda)(\lambda\boldsymbol{I}-\boldsymbol{A})].
\end{align*}
上式左边是次数小于等于 1 的矩阵多项式, 因此上式右边中括号内的矩阵多项式的次数必须小于等于零, 也即必是一个常数矩阵, 设为 $\boldsymbol{P}$. 于是
\begin{align}
\boldsymbol{R}(\lambda\boldsymbol{I}-\boldsymbol{A})&=(\lambda\boldsymbol{I}-\boldsymbol{B})\boldsymbol{P}. \label{theorem1213-7.1.9}
\end{align}
\eqref{theorem1213-7.1.9}式又可整理为
\begin{align*}
(\boldsymbol{R}-\boldsymbol{P})\lambda&=\boldsymbol{R}\boldsymbol{A}-\boldsymbol{B}\boldsymbol{P}.
\end{align*}
再次比较次数得 $\boldsymbol{R}=\boldsymbol{P}, \boldsymbol{R}\boldsymbol{A}=\boldsymbol{B}\boldsymbol{P}$. 现只需证明 $\boldsymbol{P}$ 是一个非异阵即可. 由假设
\begin{align*}
\boldsymbol{P}&=\boldsymbol{N}(\lambda)^{-1}-\boldsymbol{Q}(\lambda)(\lambda\boldsymbol{I}-\boldsymbol{A}),
\end{align*}
将上式两边右乘 $\boldsymbol{N}(\lambda)$ 并移项得
\begin{align*}
\boldsymbol{P}\boldsymbol{N}(\lambda)+\boldsymbol{Q}(\lambda)(\lambda\boldsymbol{I}-\boldsymbol{A})\boldsymbol{N}(\lambda)&=\boldsymbol{I}.
\end{align*}
但由\eqref{theorem1213-7.1.5}式可得
\begin{align*}
(\lambda\boldsymbol{I}-\boldsymbol{A})\boldsymbol{N}(\lambda)&=\boldsymbol{M}(\lambda)^{-1}(\lambda\boldsymbol{I}-\boldsymbol{B}),
\end{align*}
因此
\begin{align}
\boldsymbol{P}\boldsymbol{N}(\lambda)+\boldsymbol{Q}(\lambda)\boldsymbol{M}(\lambda)^{-1}(\lambda\boldsymbol{I}-\boldsymbol{B})&=\boldsymbol{I}. \label{theorem1213-7.1.10}
\end{align}
再由\hyperref[lemma:lambda-矩阵的性质1]{引理\ref{lemma:lambda-矩阵的性质1}}可设
\begin{align*}
\boldsymbol{N}(\lambda)&=\boldsymbol{S}(\lambda)(\lambda\boldsymbol{I}-\boldsymbol{B})+\boldsymbol{T},
\end{align*}
代入\eqref{theorem1213-7.1.10}式并整理得
\begin{align*}
[\boldsymbol{P}\boldsymbol{S}(\lambda)+\boldsymbol{Q}(\lambda)\boldsymbol{M}(\lambda)^{-1}](\lambda\boldsymbol{I}-\boldsymbol{B})&=\boldsymbol{I}-\boldsymbol{P}\boldsymbol{T}.
\end{align*}
上式右边是次数小于等于零的矩阵多项式, 因此上式左边中括号内的矩阵多项式必须为零, 从而 $\boldsymbol{P}\boldsymbol{T}=\boldsymbol{I}$, 即 $\boldsymbol{P}$ 是非异阵.
\end{proof}




















\end{document}
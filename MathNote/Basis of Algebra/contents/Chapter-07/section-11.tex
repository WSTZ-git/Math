\documentclass[../../main.tex]{subfiles}
\graphicspath{{\subfix{../../image/}}} % 指定图片目录,后续可以直接使用图片文件名。

% 例如:
% \begin{figure}[H]
% \centering
% \includegraphics[scale=0.4]{图.png}
% \caption{}
% \label{figure:图}
% \end{figure}
% 注意:上述\label{}一定要放在\caption{}之后,否则引用图片序号会只会显示??.

\begin{document}

\section{有理标准型的几何与应用}

回顾\hyperref[section:有理标准型]{有理标准型和循环子空间相关理论}.

\begin{proposition}\label{proposition:矩阵有不同特征值则特征多项式和极小多项式一定相等}
设 $n$ 阶矩阵 $A$ 有 $n$ 个不同的特征值,求证:$A$ 的特征多项式和极小多项式相等。
\end{proposition}
\begin{proof}
{\color{blue}证法一:}
设 $A$ 的 $n$ 个不同的特征值为 $\lambda_1,\lambda_2,\cdots,\lambda_n$,则由\refcor{corollary:极小多项式和特征多项式有相同的根(不计重数)}可知,特征多项式 $f(\lambda)$ 和极小多项式 $m(\lambda)$ 有相同的根(不计重数),因此 
\begin{align*}
f(\lambda)=m(\lambda)=(\lambda - \lambda_1)(\lambda - \lambda_2)\cdots(\lambda - \lambda_n)
\end{align*}

{\color{blue}证法二:}
由于 $A$ 有 $n$ 个不同的特征值,故 $A$ 相似于对角矩阵。又因为相似矩阵有相同的特征多项式和极小多项式,所以只要对对角矩阵证明此结论即可。设 $A = \mathrm{diag}\{\lambda_1,\lambda_2,\cdots,\lambda_n\}$,则 $\lambda I_n - A = \mathrm{diag}\{\lambda - \lambda_1,\lambda - \lambda_2,\cdots,\lambda - \lambda_n\}$,这是一个主对角元素两两互素的对角矩阵,由\hyperref[theorem:lambda-矩阵和初等因子的基本性质]{$\lambda$-矩阵和初等因子的基本性质(1)}以及数学归纳法可知其法式为 $\mathrm{diag}\{1,\cdots,1,(\lambda - \lambda_1)(\lambda - \lambda_2)\cdots(\lambda - \lambda_n)\}$。因此,$A$ 的特征多项式和极小多项式相等.
\end{proof}

\begin{proposition}\label{proposition:不同特征值对应的特征向量之和恰好构成循环向量}
设$n$阶矩阵$A$有$n$个不同的特征值,且特征值 $\lambda_i$ 对应的特征向量为 $\alpha_i$,由\refcor{corollary:线性变换属于不同特征值的特征向量必线性无关}可知$\{\alpha_1,\cdots,\alpha_n\}$ 为 $\mathbb{C}^n$ 的一组基。则$\alpha = \alpha_1 + \cdots + \alpha_n$ 是 $A$ 的循环空间 $\mathbb{C}^n$ 的循环向量,即$\mathbb{C}^n = L(\alpha,A\alpha,\cdots,A^{n - 1}\alpha)=C(A,\alpha)$ 为循环空间,$\alpha$ 是循环向量。
\end{proposition}
\begin{proof}
事实上,由 $A^k\alpha = \lambda_1^k\alpha_1 + \cdots + \lambda_n^k\alpha_n$,利用 Vandermonde 行列式容易证明 $\{\alpha,A\alpha,\cdots,A^{n - 1}\alpha\}$ 是 $\mathbb{C}^n$ 的一组基,从而 $\mathbb{C}^n = L(\alpha,A\alpha,\cdots,A^{n - 1}\alpha)=C(A,\alpha)$ 为循环空间,$\alpha$ 是循环向量。
\end{proof}

\begin{proposition}\label{proposition:特征多项式在数域K上的不可约分解则在复数域上可对角化}
设数域 $\mathbb{K}$ 上的 $n$ 阶矩阵 $A$ 的特征多项式 $f(\lambda)=P_1(\lambda)P_2(\lambda)\cdots P_k(\lambda)$,其中 $P_i(\lambda) (1\leq i\leq k)$ 是 $\mathbb{K}$ 上互异的首一不可约多项式。求证:$A$ 的有理标准型只有一个 Frobenius 块,并且 $A$ 在复数域上可对角化。
\end{proposition}
\begin{remark}
我们也可以利用\refthe{theorem:循环子空间的刻画}和初等因子证明这个命题。若利用不变因子在基域扩张下的不变性,则这个命题也可由\refpro{proposition:矩阵有不同特征值则特征多项式和极小多项式一定相等}得到。
\end{remark}
\begin{proof}
设 $A$ 的不变因子组为 $d_1(\lambda),d_2(\lambda),\cdots,d_n(\lambda)$,其中$d_{i}(\lambda)\mid d_{i+1}(\lambda),i=1,2,\cdots,n-1$,则有
\begin{align}\label{equation--proposition0.3--1.1.0}
f(\lambda)=P_1(\lambda)P_2(\lambda)\cdots P_k(\lambda)=d_1(\lambda)d_2(\lambda)\cdots d_n(\lambda)
\end{align}
由于 $P_i(\lambda)$ 是不可约多项式,故存在某个 $j$,使得 $P_i(\lambda)\mid d_j(\lambda)$,否则,由\hyperref[proposition:不可约多项式的基本性质]{不可约多项式的基本性质(1)}可知$(P_i(\lambda)$,$d_j$ $(\lambda))$ $=1$,$j=1,2,\cdots,n$.再由\hyperref[proposition:互素多项式和最大公因式的基本性质]{互素多项式和最大公因式的基本性质(5)}可知$(P_i(\lambda),d_1(\lambda)d_2(\lambda)\cdots d_n(\lambda))=1$,这与\eqref{equation--proposition0.3--1.1.0}矛盾!
从而 $P_i(\lambda)\mid d_n(\lambda) (1\leq i\leq k)$。由\hyperref[proposition:互素多项式和最大公因式的基本性质]{互素多项式和最大公因式的基本性质(1)}可知,$P_1(\lambda)P_2(\lambda)\cdots P_k(\lambda)\mid d_n(\lambda)$,因此只能是 $d_1(\lambda)=\cdots = d_{n - 1}(\lambda)=1$,$d_n(\lambda)=f(\lambda)$,从而 $A$ 的有理标准型只有一个 Frobenius 块。由于特征多项式 $f(\lambda)=P_1(\lambda)P_2(\lambda)\cdots P_k(\lambda)$ 在 $\mathbb{K}$ 上无重因式,故 $(f(\lambda),f'(\lambda)) = 1$,从而 $f(\lambda)$ 在复数域上无重根,即 $A$ 有 $n$ 个不同的特征值,于是 $A$ 在复数域上可对角化。
\end{proof}

\begin{corollary}
设数域 $\mathbb{K}$ 上的 $n$ 阶矩阵 $A$ 的特征多项式 $f(\lambda)=P_1(\lambda)P_2(\lambda)\cdots P_k(\lambda)$,其中 $P_i(\lambda) (1\leq i\leq k)$ 是 $\mathbb{K}$ 上互异的首一不可约多项式.并且 $\alpha_i$ 为线性方程组 $P_i(A)x = 0$ 的非零解,则 $\alpha=\alpha_1 + \cdots + \alpha_k$ 是 $A$ 的循环空间 $\mathbb{K}^n$ 的循环向量。
\end{corollary}
\begin{proof}
由\refpro{proposition:特征多项式在数域K上的不可约分解则在复数域上可对角化}及\refthe{theorem:不变子空间是循环子空间的充要条件}可知$\mathbb{K}^n$就是一个循环空间.(未完成证明)
\end{proof}

\begin{proposition}\label{proposition:特征多项式是不可约多项式的刻画}
设 $\varphi$ 是数域 $\mathbb{K}$ 上 $n$ 维线性空间 $V$ 上的线性变换,$\varphi$ 的特征多项式为 $f(\lambda)$,证明以下 3 个结论等价:
\begin{enumerate}[(1)]
\item $V$ 只有平凡的 $\varphi$-不变子空间;

\item $V$ 中任一非零向量都是循环向量,使 $V$ 成为循环空间;

\item $f(\lambda)$ 是 $\mathbb{K}$ 上的不可约多项式。
\end{enumerate}
\end{proposition}
\begin{proof}
$(1)\Rightarrow(2)$:任取 $V$ 中非零向量 $\alpha$,则循环子空间 $C(\varphi,\alpha)$ 是非零 $\varphi$-不变子空间。由于 $V$ 只有平凡的 $\varphi$-不变子空间,故 $C(\varphi,\alpha)=V$,即 $V$ 中任一非零向量都是循环向量,使 $V$ 成为循环空间。

$(2)\Rightarrow(3)$:用反证法,假设 $f(\lambda)=g(\lambda)h(\lambda)$,其中 $g(\lambda),h(\lambda)$ 是 $\mathbb{K}$ 上次数小于 $n$ 的首一多项式。由 Cayley - Hamilton 定理可知 $\mathbf{0}=f(\varphi)=g(\varphi)h(\varphi)$,故由\hyperref[proposition:无限维线性空间的可逆线性变换充要条件1]{命题\ref{proposition:无限维线性空间的可逆线性变换充要条件1}(1)}的逆否命题可知 $g(\varphi),h(\varphi)$ 中至少有一个是奇异(不可逆/非双射)线性变换,不妨设为 $g(\varphi)$,由\refcor{proposition:像和原像空间维数相同时线性同构的充要条件}可知 $\mathrm{Ker}g(\varphi)\neq 0$。任取 $\mathrm{Ker}g(\varphi)$ 中的非零向量 $\alpha$,设 $\deg g(\lambda)=r$,则不妨设
\begin{align*}
g(\varphi) = a_r \varphi^r + a_{r-1} \varphi^{r-1} + \cdots + a_1, \quad \text{其中} \ a_r \neq 0.
\end{align*}
由$\alpha \in \ker g(\varphi)$可知
\begin{align*}
g(\varphi)(\alpha) = a_r \varphi^r(\alpha) + a_{r-1} \varphi^{r-1}(\alpha) + \cdots + a_1 \alpha = 0.
\end{align*}
于是
\begin{align}
\varphi^r(\alpha) = -\frac{a_{r-1}}{a_r} \varphi^{r-1}(\alpha) - \cdots - \frac{a_1}{a_r} \alpha. \label{equation--:_:_:_:_::0.1}
\end{align}
假设对$k \geq r$且$k \in \mathbb{N}$,成立$\varphi^k(\alpha)$可由$\{ \alpha, \varphi(\alpha), \cdots, \varphi^{r-1}(\alpha) \}$线性表示,则对\eqref{equation--:_:_:_:_::0.1}式两边同时作用$\varphi^{k-r+1}$可得
\begin{align*}
\varphi^{k+1}(\alpha) = -\frac{a_{r-1}}{a_r} \varphi^k(\alpha) - \cdots - \frac{a_1}{a_r} \varphi^{k-r+1}(\alpha).
\end{align*}
于是由归纳假设可知,$\varphi^{k+1}(\alpha)$可由$\{ \alpha, \varphi(\alpha), \cdots, \varphi^{r-1}(\alpha) \}$线性表示. 故由数学归纳法可得,对$\forall k \geq r$且$k \in \mathbb{N}$,都有$\varphi^k(\alpha)$可由$\{ \alpha, \varphi(\alpha), \cdots, \varphi^{r-1}(\alpha) \}$线性表示.因此
$C(\varphi ,\alpha )=L(\alpha ,\varphi (\alpha ),\cdots \cdots )=L(\alpha ,\varphi (\alpha ),\cdots ,\varphi ^{r-1}(\alpha ))$,其维数 $\leq r < n$,故 $C(\varphi,\alpha)\neq V$,这与 $V$ 中任一非零向量都是循环向量矛盾!

$(3)\Rightarrow(1)$:用反证法,假设存在非平凡的 $\varphi$-不变子空间 $U$,$\dim U = r$,则 $\varphi$ 在一组基下的表示矩阵为分块上三角矩阵 $M = \begin{pmatrix}
\boldsymbol{A} & \boldsymbol{C} \\
\boldsymbol{O} & \boldsymbol{B}
\end{pmatrix}$,其中 $\boldsymbol{A}$ 是 $\varphi|_U$ 的表示矩阵。于是特征多项式
\begin{align*}
f(\lambda)=|\lambda\boldsymbol{I}_V - \varphi|=|\lambda\boldsymbol{I}_n - M|=|\lambda\boldsymbol{I}_r - \boldsymbol{A}|\cdot|\lambda\boldsymbol{I}_{n - r} - \boldsymbol{B}|.
\end{align*}
是两个低次多项式的乘积,这与 $f(\lambda)$ 的不可约性矛盾!
\end{proof}

\begin{proposition}\label{proposition:极小多项式是不可约多项式的刻画}
设 $\varphi$ 是数域 $\mathbb{K}$ 上 $n$ 维线性空间 $V$ 上的线性变换,$\varphi$ 的极小多项式为 $m(\lambda)$。证明:$m(\lambda)$ 是 $\mathbb{K}$ 上的不可约多项式的充要条件是 $V$ 的任一非零 $\varphi$-不变子空间 $U$ 必为如下形式:
\begin{align*}
U = C(\varphi,\alpha_1)\oplus C(\varphi,\alpha_2)\oplus\cdots\oplus C(\varphi,\alpha_k)
\end{align*}
并且 $\varphi|_{C(\varphi,\alpha_i)}$ 的极小多项式都是 $m(\lambda)$。此时,$\varphi|_U$ 的极小多项式也是 $m(\lambda)$。
\end{proposition}
\begin{proof}
{\heiti 必要性:}设 $\varphi|_U$ 的极小多项式为 $n(\lambda)$,则 $m(\varphi|_U)=m(\varphi)|_U = \mathbf{0}$,从而 $n(\lambda)\mid m(\lambda)$。因为 $m(\lambda)$ 不可约,所以 $n(\lambda)=m(\lambda)$。又由于 $\varphi|_U$ 的所有不变因子都要整除 $m(\lambda)$ 且 $m(\lambda)$ 不可约,故所有的非常数不变因子都等于 $m(\lambda)$。最后,由\hyperref[theorem:有理标准型的几何意义]{有理标准型的几何意义}即得
\begin{align*}
U = C(\varphi,\alpha_1)\oplus C(\varphi,\alpha_2)\oplus\cdots\oplus C(\varphi,\alpha_k).
\end{align*}
并且\(\varphi|_{C(\varphi,\alpha_i)}\) 在基 \(\{\alpha_i,\varphi(\alpha_i),\cdots,\varphi^{r_i - 1}(\alpha_i)\}\) 下的表示矩阵就是友阵 \(C(m(\lambda))\),其中 \(r_i = \dim C(\varphi,\alpha_i)\)。于是\(\varphi|_{C(\varphi,\alpha_i)}\)的极小多项式就是其表示矩阵\(C(m(\lambda))\)的极小多项式.又由\reflem{lemma:Frobenius标准型矩阵的极小多项式和不变因子}可知,\(C(m(\lambda))\)的极小多项式就是$m(\lambda)$,并且$n(\lambda)=m(\lambda)$,
故结论得证.

{\heiti 充分性:}用反证法,设 $m(\lambda)=g(\lambda)h(\lambda)$,其中 $g(\lambda),h(\lambda)$ 是 $\mathbb{K}$ 上次数小于 $m(\lambda)$ 次数的首一多项式,则 $\mathbf{0}=m(\varphi)=g(\varphi)h(\varphi)$,故由\hyperref[proposition:无限维线性空间的可逆线性变换充要条件1]{命题\ref{proposition:无限维线性空间的可逆线性变换充要条件1}(1)}的逆否命题可知 $g(\varphi),h(\varphi)$ 中至少有一个是奇异线性变换,不妨设为 $g(\varphi)$,于是由\refcor{proposition:像和原像空间维数相同时线性同构的充要条件}可知 $\mathrm{Ker}g(\varphi)\neq 0$。任取 $\mathrm{Ker}g(\varphi)$ 中的非零向量 $\alpha$,得到循环子空间 $U = C(\varphi,\alpha)$,


由 $g(\varphi)(\alpha)=\mathbf{0}$可知,对$\forall k\in \mathbb{N}$,都有$\varphi^k(g(\varphi)(\alpha))=0$.从而对$\forall \beta \in U=C(\varphi,\alpha),$存在不全为零的$a_i$使得
\begin{align*}
g(\varphi)(\beta)&=g\left( a_1\alpha +a_2\varphi \left( \alpha \right) +a_3\varphi ^2\left( \alpha \right) +\cdots \right) 
\\
&=a_1g\left( \alpha \right) +a_2g\left( \varphi \left( \alpha \right) \right) +a_3g\left( \varphi ^2\left( \alpha \right) \right) +\cdots 
\\
&=a_1g\left( \alpha \right) +a_2\varphi \left( g\left( \alpha \right) \right) +a_3\varphi ^2\left( g\left( \alpha \right) \right) +\cdots \\
&=0+0+0+\cdots=0.
\end{align*}
因此$g(\varphi|_U)=g(\varphi)|_U = \mathbf{0}$,于是 $\varphi|_U$ 的极小多项式$m(\lambda)$整除 $g(\lambda)$,从而其次数 $\leq\deg g(\lambda)<\deg m(\lambda)$,这与条件矛盾!
\end{proof}

\begin{theorem}[基于初等因子组的有理标准型]\label{theorem:基于初等因子组的有理标准型}
设数域 $\mathbb{K}$ 上的 $n$ 阶矩阵 $A$ 的初等因子组为 $P_1(\lambda)^{r_1},P_2(\lambda)^{r_2},\cdots,P_k(\lambda)^{r_k}$,证明:$A$ 相似于分块对角矩阵
\begin{align*}
\widetilde{\boldsymbol{F}}&=\mathrm{diag}\{\boldsymbol{F}(P_1(\lambda)^{r_1}),\boldsymbol{F}(P_2(\lambda)^{r_2}),\cdots,\boldsymbol{F}(P_k(\lambda)^{r_k})\}\\
\widetilde{\boldsymbol{C}}&=\mathrm{diag}\{\boldsymbol{C}(P_1(\lambda)^{r_1}),\boldsymbol{C}(P_2(\lambda)^{r_2}),\cdots,\boldsymbol{C}(P_k(\lambda)^{r_k})\}
\end{align*}
称为 $A$ 的\textbf{基于初等因子组的有理标准型}。
\end{theorem}
\begin{proof}
由 Frobenius 块和友阵的性质可知,$\lambda\boldsymbol{I}_n - \widetilde{\boldsymbol{F}}$ 和 $\lambda\boldsymbol{I}_n - \widetilde{\boldsymbol{C}}$ 都相抵于
\begin{align*}
\mathrm{diag}\{1,\cdots,1,P_1(\lambda)^{r_1};1,\cdots,1,P_2(\lambda)^{r_2};\cdots;1,\cdots,1,P_k(\lambda)^{r_k}\}
\end{align*}
再由\hyperref[theorem:lambda-矩阵和初等因子的基本性质]{$\lambda$-矩阵和初等因子的基本性质(2)}可知,$\widetilde{\boldsymbol{F}},\widetilde{\boldsymbol{C}}$ 与 $A$ 有相同的初等因子组,从而它们相似。
\end{proof}

\begin{theorem}\label{theorem:基于初等因子组的循环子空间之和分解}
设 $\varphi$ 是数域 $\mathbb{K}$ 上 $n$ 维线性空间 $V$ 上的线性变换,$\varphi$ 的初等因子组为 $P_1(\lambda)^{r_1},P_2(\lambda)^{r_2},\cdots,P_k(\lambda)^{r_k}$。证明:存在 $\alpha_1,\alpha_2,\cdots,\alpha_k\in V$,使得
\begin{align*}
V = C(\varphi,\alpha_1)\oplus C(\varphi,\alpha_2)\oplus\cdots\oplus C(\varphi,\alpha_k)
\end{align*}
\end{theorem}
\begin{proof}
由\hyperref[theorem:基于初等因子组的有理标准型]{基于初等因子组的有理标准型}和\refthe{theorem:不变子空间是循环子空间的充要条件}即得。
\end{proof}

\begin{example}
求证:存在 $n$ 阶实方阵 $A$,满足 $A^2 + 2A + 5I_n = O$ 的充要条件是 $n$ 为偶数。
当 $n\geq 4$ 时,验证满足上述条件的矩阵 $A$ 有无限个不变子空间。
\end{example}
\begin{proof}
{\heiti 必要性:}注意到 $A$ 适合多项式 $g(\lambda)=\lambda^2 + 2\lambda + 5$,故 $A$ 的极小多项式 $m(\lambda)\mid g(\lambda)$,又因为 $g(\lambda)$ 在实数域上不可约,故只能是 $m(\lambda)=g(\lambda)$。同理可证 $A$ 所有的非常数不变因子都等于 $g(\lambda)$,从而 $A$ 的不变因子组为 $1,\cdots,1,g(\lambda),\cdots,g(\lambda)$($k$ 个 $g(\lambda)$)。因此 $A$ 的特征多项式 $f(\lambda)=g(\lambda)^k$,于是 $n = \deg f(\lambda)=2k$ 为偶数。

{\heiti 充分性:}设 $n = 2k$ 为偶数,则由必要性的证明可知,$A$ 的不变因子组为 $1,\cdots,1,g(\lambda),\cdots,g(\lambda)$($k$ 个 $g(\lambda)$)。可用有理标准型构造满足条件的矩阵:
\begin{align}\label{equation----Akkk}
A=\mathrm{diag}\left\{\begin{pmatrix}
0 & -5 \\
1 & -2
\end{pmatrix},\cdots,\begin{pmatrix}
0 & -5 \\
1 & -2
\end{pmatrix}\right\} \text{($k$ 个二阶方阵)}
\end{align}
当 $n\geq 4$ 时,设 $\{e_1,e_2,e_3,e_4\}$ 是前 4 个标准单位列向量,则容易验证\eqref{equation----Akkk}式中的矩阵$A$满足$Ae_1=e_2,Ae_3=e_4$,于是构造循环子空间 $\{C_l := C(A,e_1 + le_3)=L(e_1 + le_3,e_2 + le_4),l\in\mathbb{R}\}$ ,进一步容易验证循环子空间 $\{C_l := C(A,e_1 + le_3)=L(e_1 + le_3,e_2 + le_4),l\in\mathbb{R}\}$ 是两两互异的 $A$-不变子空间,故 $A$ 有无限个不变子空间。
\end{proof}

\begin{proposition}\label{proposition:极小多项式的次数小于等于矩阵的秩加1}
设 $A$ 是数域 $\mathbb{K}$ 上的 $n$ 阶方阵,求证:$A$ 的极小多项式的次数小于等于 $\mathrm{r}(A)+1$。
\end{proposition}
\begin{proof}
{\color{blue}证法一:}
设 $A$ 的不变因子组为 $1,\cdots,1,d_1(\lambda),\cdots,d_k(\lambda)$,则极小多项式 $m(\lambda)=d_k(\lambda)$,并且由\refthe{theorem:有理标准型核心定理}可知$A$ 相似于 $\boldsymbol{F}$ $=$ $\mathrm{diag}$ $\{$ $\boldsymbol{F}$ $(d_1(\lambda))$,$\cdots$ ,$\boldsymbol{F}$ $(d_k(\lambda))$ $\}$.设 $\deg d_k(\lambda)=r$,若 $d_k(0)\neq 0$,则由\hyperref[lemma:Frobenius标准型矩阵的极小多项式和不变因子]{Frobenius块的基本性质(1)}可知 $\boldsymbol{F}(d_k(\lambda))$ 非异;若 $d_k(0)=0$,则由\hyperref[lemma:Frobenius标准型矩阵的极小多项式和不变因子]{Frobenius块的基本性质(1)}可知 $\boldsymbol{F}(d_k(\lambda))$ 奇异且右上角的 $r - 1$ 阶子式非零,从而秩为 $r - 1$。因此,$\mathrm{r}(A)=\mathrm{r}(\boldsymbol{F})\geq\mathrm{r}(\boldsymbol{F}(d_k(\lambda)))\geq r - 1=\deg d_k(\lambda)-1$.

{\color{blue}证法二:}
从 $A$ 的极小多项式 $m(\lambda)$ 分离出来的初等因子中,形如 $\lambda^r$ 的初等因子至多只有 1 个,对应于零特征值的 Jordan 块 $J_r(0)$,其余的初等因子对应于非零特征值的 Jordan 块。因此 $\mathrm{r}(A)$ 大于等于这些 Jordan 块秩的和,后者等于 $\mathrm{deg}\,m(\lambda) - 1$ 或 $\mathrm{deg}\,m(\lambda)$.
\end{proof}

\begin{proposition}\label{proposition:特征值对应的特征矩阵的秩与不变因子组的关系}
设数域 $\mathbb{K}$ 上的 $n$ 阶矩阵 $A$ 的不变因子组是 $1,\cdots,1,d_1(\lambda),\cdots,d_k(\lambda)$,其中 $d_i(\lambda)$ 是非常数首一多项式,$d_i(\lambda)\mid d_{i + 1}(\lambda) (1\leq i\leq k - 1)$。求证:对 $A$ 的任一特征值 $\lambda_0$,
\begin{align*}
\mathrm{r}(\lambda_0\boldsymbol{I}_n - A)=n - \sum_{i = 1}^{k}\delta_{d_i(\lambda_0),0}
\end{align*}
其中记号 $\delta_{a,b}$ 表示:若 $a = b$,取值为 $1$;若 $a\neq b$,取值为 $0$。
\end{proposition}
\begin{proof}
{\color{blue}证法一:}
设 $\deg d_i(\lambda)=r_i$,则由\refthe{theorem:有理标准型核心定理}可知$A$ 相似于 $\boldsymbol{F}=\mathrm{diag}\{\boldsymbol{F}(d_1(\lambda)),\cdots,\boldsymbol{F}(d_k(\lambda))\}$,而相似矩阵有相同的特征多项式,故$\lambda_0I_n-A=\lambda_0I_n-F$.由\hyperref[lemma:Frobenius标准型矩阵的极小多项式和不变因子]{Frobenius块的基本性质(2)}可知$|\lambda_0\boldsymbol{I}_{r_i}-\boldsymbol{F}(d_i(\lambda))|=d_i(\lambda_0)$。若 $d_i(\lambda_0)\neq 0$,则 $\lambda_0\boldsymbol{I}_{r_i}-\boldsymbol{F}(d_i(\lambda))$ 非异;若 $d_i(\lambda_0)=0$,则 $\lambda_0\boldsymbol{I}_{r_i}-\boldsymbol{F}(d_i(\lambda))$ 奇异且右上角的 $r_i - 1$ 阶子式非零,从而秩为 $r_i - 1$。因此,
\begin{align*}
\mathrm{r}(\lambda_0\boldsymbol{I}_n - A)&=\mathrm{r}(\lambda_0\boldsymbol{I}_n - \boldsymbol{F})=\sum_{i = 1}^{k}\mathrm{r}(\lambda_0\boldsymbol{I}_{r_i}-\boldsymbol{F}(d_i(\lambda)))\\
&=\sum_{i = 1}^{k}(r_i - \delta_{d_i(\lambda_0),0})=n - \sum_{i = 1}^{k}\delta_{d_i(\lambda_0),0}
\end{align*}

{\color{blue}证法二:}
由\refthe{theorem:有理标准型核心定理}可知存在可逆 $\lambda$-矩阵 $\boldsymbol{P}(\lambda),\boldsymbol{Q}(\lambda)$,使得
\begin{align*}
\boldsymbol{P}(\lambda)(\lambda\boldsymbol{I}_n - A)\boldsymbol{Q}(\lambda)=\mathrm{diag}\{1,\cdots,1,d_1(\lambda),\cdots,d_k(\lambda)\}
\end{align*}
在上式中令 $\lambda = \lambda_0$,注意到 $\boldsymbol{P}(\lambda_0),\boldsymbol{Q}(\lambda_0)$ 是 $\mathbb{K}$ 上的可逆矩阵,故 $\lambda_0\boldsymbol{I}_n - A$ 相抵于 $\mathrm{diag}\{1,\cdots,1,d_1(\lambda_0),\cdots,d_k(\lambda_0)\}$,于是 $\mathrm{r}(\lambda_0\boldsymbol{I}_n - A)$ 等于 $n$ 减去等于零的 $d_i(\lambda_0)$ 的个数,从而结论得证。
\end{proof}

\begin{proposition}\label{proposition:迹零阵必相似于主对角全为零的矩阵}
\begin{enumerate}[(1)]
\item 设 $A$ 是数域 $\mathbb{K}$ 上的 $n$ 阶矩阵,求证:$A$ 相似于一个 $\mathbb{K}$ 上主对角元全为零的矩阵的充要条件是$\mathrm{tr}(A)=0$.

\item 设 $A$ 是数域 $\mathbb{R}$ 上的 $n$ 阶矩阵,求证:$A$ 正交相似于一个 $\mathbb{R}$ 上主对角元全为零的矩阵的充要条件是$\mathrm{tr}(A)=0$.

\item 设 $C$ 是数域 $\mathbb{K}$ 上的 $n$ 阶矩阵,求证:存在 $\mathbb{K}$ 上的 $n$ 阶矩阵 $A,B$,使得 $AB - BA = C$ 的充要条件是 $\mathrm{tr}(C)=0$。
\end{enumerate}
\end{proposition}
\begin{proof}
\begin{enumerate}[(1)]
\item 必要性:由矩阵的迹是矩阵的相似不变量,故$\mathrm{tr}(A)=0$.

充分性:对阶数进行归纳。当 $n = 1$ 时,$A = O$,结论显然成立。设阶数小于 $n$ 时结论成立,现证 $n$ 阶的情形。由于题目的条件和结论在相似关系下不改变,故不妨从一开始就假设 $A$ 是有理标准型
\begin{align*}
\boldsymbol{F}=\mathrm{diag}\{\boldsymbol{F}(d_1(\lambda)),\cdots,\boldsymbol{F}(d_k(\lambda))\}
\end{align*}
其中 $d_i(\lambda)$ 是 $A$ 的非常数不变因子,$d_i(\lambda)\mid d_{i + 1}(\lambda) (1\leq i\leq k - 1)$,$\deg d_i(\lambda)=r_i$。若 $r_i$ 都为 $1$,则 $d_1(\lambda)=\cdots = d_n(\lambda)=\lambda - c$,从而 $A = c\boldsymbol{I}_n$。又 $\mathrm{tr}(A)=0$,故 $c = 0$,从而 $A = O$,结论成立。以下假设存在某个 $r_i>1$,将第 $(1,1)$ 分块与第 $(i,i)$ 分块对换,这是一个相似变换,此时矩阵的第 $(1,1)$ 元为零,故不妨设 $A$ 的第 $(1,1)$ 元为零。注意到矩阵 $A=\begin{pmatrix}
0 & \boldsymbol{\alpha}' \\
\boldsymbol{\beta} & \boldsymbol{B}
\end{pmatrix}$,其中 $\boldsymbol{\alpha},\boldsymbol{\beta}\in\mathbb{K}^{n - 1}$,$\boldsymbol{B}\in M_{n - 1}(\mathbb{K})$,$\mathrm{tr}(\boldsymbol{B})=0$。由归纳假设,存在 $\mathbb{K}$ 上的 $n - 1$ 阶非异阵 $\boldsymbol{Q}$,使得 $\boldsymbol{Q}^{-1}\boldsymbol{B}\boldsymbol{Q}$ 的主对角元全为零,令 $\boldsymbol{P}=\begin{pmatrix}
1 & \boldsymbol{O} \\
\boldsymbol{O} & \boldsymbol{Q}
\end{pmatrix}$ 为 $\mathbb{K}$ 上的 $n$ 阶非异阵,则 $\boldsymbol{P}^{-1}\boldsymbol{A}\boldsymbol{P}=\begin{pmatrix}
0 & \boldsymbol{\alpha}'\boldsymbol{Q} \\
\boldsymbol{Q}^{-1}\boldsymbol{\beta} & \boldsymbol{Q}^{-1}\boldsymbol{B}\boldsymbol{Q}
\end{pmatrix}$ 的主对角元全为零,结论得证。

\begin{remark}
(1)的充分性也可以利用矩阵的初等变换进行归纳证明(见清疏讲义).
\end{remark}

\item 必要性:若矩阵 \( A \) 正交相似于主对角元全为零的矩阵,则由相似变换不改变迹可知, \(\mathrm{tr}(A) = 0\)。

充分性:对阶数\(n\)进行归纳。当\(n = 1\)时,结论显然成立。假设对任意阶数小于\(n\)的实迹零矩阵结论成立。考虑\(n\)阶实迹零矩阵\(A\),令\(T=\frac{A + A^\prime}{2}\),\(S=\frac{A - A^\prime}{2}\),则\(A = T + S\),并且\(T\)为实对称阵,\(S\)为实反称阵。

(i)若\(T = O\),则\(A = S\)是实反称矩阵,而实反称矩阵主对角元均为\(0\),结论显然成立。

(ii)若\(T\ne 0\),则存在正交矩阵\(P\),使得
\begin{align*}
P^{-1}TP=\left( \begin{matrix}
I_p& & \\
& -I_q& \\
& & O\\
\end{matrix} \right),
\end{align*}
其中\(p,q\)分别为\(T\)的正、负惯性指数。由\(T\ne O\)可知\(\text{r}(T)=p + q>0\),从而
\begin{align*}
p - q=\mathrm{tr}(P^{-1}TP)=\mathrm{tr}(T)=\mathrm{tr}(A)=0\Rightarrow p = q=\frac{\mathrm{r}(T)}{2}>0.
\end{align*}
于是\(p,q\geqslant 1\)。故
\begin{align*}
P_{p + 1,p}^{-1}P_{p + 1,2}^{-1}(P^{-1}TP)P_{p + 1,2}P_{p + 1,p}=\left( \begin{matrix}
T_1& & & \\
& I_{p - 1}& & \\
& & I_{q - 1}& \\
& & & O\\
\end{matrix} \right),
\end{align*}
其中\(T_1=\left( \begin{matrix}
1&0\\
0& -1\\
\end{matrix} \right)\),记\(T_2=\left( \begin{matrix}
I_{p - 1}& & \\
& I_{q - 1}& \\
& & O\\
\end{matrix} \right)\),\(P_0 = PP_{p + 1,2}P_{p + 1,p}\),则\(P_0\)为正交矩阵,并且
\begin{align*}
P_{0}^{-1}TP_0=\left( \begin{matrix}
T_1& \\
& T_2\\
\end{matrix} \right).
\end{align*}
注意到\(T_1\)为\(2\)阶迹零阵,\(T_2\)为\(n - 2\)阶迹零阵。从而由归纳假设可得,存在正交矩阵\(P_1,P_2\),使得
\begin{align*}
P_{1}^{-1}T_1P_1=T_{1}^{\prime},\quad P_{2}^{-1}T_2P_2=T_{2}^{\prime},
\end{align*}
其中\(T_{1}^{\prime},T_{2}^{\prime}\)都是主对角元全为\(0\)的矩阵。故令\(Q = P_0\left( \begin{matrix}
P_1& \\
& P_2\\
\end{matrix} \right)\),则\(Q\)也是正交矩阵,并且
\begin{align*}
Q^{-1}TQ=\left( \begin{matrix}
T_{1}^{\prime}& \\
& T_{2}^{\prime}\\
\end{matrix} \right).
\end{align*}
因此\(Q^{-1}TQ\)的主对角元全都为\(0\)。又由\(S\)为反称矩阵可知\((Q^{-1}SQ)^\prime = -Q^{-1}SQ\),故\(Q^{-1}SQ\)也为反称矩阵,从而\(Q^{-1}SQ\)的主对角元也全都为\(0\)。于是\(Q^{-1}AQ = Q^{-1}TQ + Q^{-1}SQ\)的主对角元也全都为\(0\).

\item 必要性由矩阵迹的线性和交换性即得,下证充分性。

由于题目的条件和结论在同时相似变换 $A\mapsto P^{-1}AP$,$B\mapsto P^{-1}BP$,$C\mapsto P^{-1}CP$ 下不改变,故由\nrefpro{proposition:迹零阵必相似于主对角全为零的矩阵}{(1)}不妨从一开始就假设 $C=(c_{ij})$ 的主对角元 $c_{ii}=0 (1\leq i\leq n)$。取定 $A = \mathrm{diag}\{\lambda_1,\lambda_2,\cdots,\lambda_n\}$ 为 $\mathbb{K}$ 上的主对角元互异的对角矩阵。设 $B=(x_{ij})$,则 $AB - BA = C$ 等价于方程 $\lambda_ix_{ij}-\lambda_jx_{ij}=c_{ij}$。当 $i = j$ 时,上式恒成立,故 $x_{ii}$ 可任取。当 $i\neq j$ 时,$x_{ij}=\frac{c_{ij}}{\lambda_i - \lambda_j}$ 被唯一确定。因此,一定存在 $\mathbb{K}$ 上的矩阵 $A,B$,使得 $AB - BA = C$ 成立.
\end{enumerate}
\end{proof}

\begin{corollary}\label{corollary:AB-BA的维数等于迹零矩阵的维数}
$\mathrm{dim}\left\{ AB-BA|A,B\in \mathbb{C} ^{n\times n} \right\} =\mathrm{dim}\left\{ A|\mathrm{tr}\left( A \right) =0 \right\} =n^2-1.$
\end{corollary}
\begin{proof}
由\nrefpro{proposition:迹零阵必相似于主对角全为零的矩阵}{(3)}可知,只须证$\mathrm{dim}\left\{ A|\mathrm{tr}\left( A \right) =0 \right\} =n^2-1.$

{\color{blue} 证法一:}
注意到$\{ A|\mathrm{tr}(A) =0 \}$的一组基为  
$E_{ij}(i\ne j),E_{ii}-E_{i+1,i+1}(i=1,2,\cdots,n-1)$.  
这组基中共含有$n^2-1$个线性无关的矩阵.  

{\color{blue} 证法二:}将矩阵的迹看作$\mathbb{C} ^{n\times n}\rightarrow \mathbb{C}$的一个线性映射,记作$\mathrm{tr}$.于是  
$\mathrm{dim}\{ A|\mathrm{tr}(A) =0 \} = \mathrm{dim}\,\mathrm{Ker}\,\mathrm{tr} = \mathrm{dim}\mathbb{C} ^{n\times n} - \mathrm{dim}\,\mathrm{Im}\,\mathrm{tr} = n^2 - \mathrm{dim}\mathbb{C} = n^2-1$.
\end{proof}

\begin{proposition}
设$\mathrm{tr}(A)=0$,$f(x)$是多项式且$\deg f(x)\geq 1$,证明:存在可逆方阵$X,Y$使得$A = f(XY)-f(YX)$.
\end{proposition}
\begin{proof}
由\nrefpro{corollary:AB-BA的维数等于迹零矩阵的维数}{(1)}可知,存在可逆矩阵$P$,使得
\begin{align*}
A = P\Lambda P^{-1},
\end{align*}
其中$\Lambda = (b_{ij})_{n\times n}$且$b_{ii} = 0\ (i = 1,2,\cdots,n)$。注意到条件与结论在线性变换$A \mapsto P\Lambda P^{-1}$下不改变,因此不妨设$A = \Lambda$,则取$Y = \begin{pmatrix}
1 & & & \\
& 2 & & \\
& & \ddots & \\
& & & n
\end{pmatrix}$,设$X = (x_{ij})$,记矩阵$U = XY$,$A_1 = AY$,则$A_1$也是主对角元全为零的矩阵,设$A_1 = (a_{ij})_{n\times n}$,其中$a_{ii} = 0\ (i = 1,2,\cdots,n)$。从而
\begin{align}
&A = f(XY) - f(YX) \Longleftrightarrow A = f(U) - f(YUY^{-1}) = f(U) - Yf(U)Y^{-1} \notag \\
&\Longleftrightarrow AY = f(U)Y - Yf(U) \Longleftrightarrow A_1 = f(U)Y - Yf(U). \label{100.49}
\end{align}
由\nrefpro{corollary:AB-BA的维数等于迹零矩阵的维数}{(3)}可知,存在$M_0 \in \mathbb{C}^{n\times n}$,使得$A_1 = M_0Y - YM_0$。令$M_1 = M_0 + \lambda Y$,$\forall \lambda \in \mathbb{C}$,则
\begin{align}
A_1 = M_1Y - YM_1. \label{100.48}
\end{align}
设$M_0 = (x_{ij})_{n\times n}$,则$M_1 = \begin{pmatrix}
x_{11} + \lambda & x_{12} & \cdots & x_{1n} \\
x_{21} & x_{22} + 2\lambda & \cdots & x_{2n} \\
\vdots & \vdots & \ddots & \vdots \\
x_{n1} & x_{n2} & \cdots & x_{nn} + n\lambda
\end{pmatrix}$。从而可设$D_k: |z - x_{kk} - k\lambda| \leqslant \sum_{j\ne k} x_{kj}\ (1 \leqslant k \leqslant n)$为$M_1$的$n$个戈氏圆盘。取$\lambda$充分大,使得
\begin{align}
|x_{kk} - k\lambda - f(0)| > \sum_{j\ne k} x_{kj}\ (1 \leqslant k \leqslant n), \label{100.46}
\end{align}
并且$D_1, \cdots, D_n$互不相交,从而此时$D_i\ (1 \leqslant i \leqslant n)$各自构成一个连通分支。于是由第二圆盘定理可知,每个连通分支$D_i$中有且仅有一个特征值,进而$M_1$此时有$n$个不同的特征值,故此时$M_1$可对角化。于是存在可逆阵$G$,使得
\begin{align*}
M_1 = G\begin{pmatrix}
\lambda_1 & & & \\
& \lambda_2 & & \\
& & \ddots & \\
& & & \lambda_n
\end{pmatrix}G^{-1},
\end{align*}
其中$\lambda_1, \cdots, \lambda_n$为$M_1$的所有特征值。取$u_i \in \mathbb{C}$为$f(x) = \lambda_i\ (1 \leqslant i \leqslant n)$的根,又由\eqref{100.46}式可知,$f(0)$落在$\bigcup_{i=1}^n D_i$之外,故$M_1$的所有特征值$\lambda_1, \cdots, \lambda_n \ne f(0)$。从而$u_i \ne 0\ (1 \leqslant i \leqslant n)$。于是令$U = G\begin{pmatrix}
u_1 & & & \\
& u_2 & & \\
& & \ddots & \\
& & & u_n
\end{pmatrix}G^{-1}$,则
\begin{align*}
f(U) = G\begin{pmatrix}
f(u_1) & & & \\
& f(u_2) & & \\
& & \ddots & \\
& & & f(u_n)
\end{pmatrix}G^{-1} = G\begin{pmatrix}
\lambda_1 & & & \\
& \lambda_2 & & \\
& & \ddots & \\
& & & \lambda_n
\end{pmatrix}G^{-1} = M_1.
\end{align*}
再结合\eqref{100.48}式可知$A_1 = f(U)Y - Yf(U)$。令$X = UY^{-1}$,则再根据\eqref{100.49}式就可知
\begin{align*}
A = f(XY) - f(YX).
\end{align*}
故结论得证。
\end{proof}










\end{document}
\documentclass[../../main.tex]{subfiles}
\graphicspath{{\subfix{../../image/}}} % 指定图片目录,后续可以直接使用图片文件名。

% 例如:
% \begin{figure}[h]
% \centering
% \includegraphics{image-01.01}
% \caption{图片标题}
% \label{fig:image-01.01}
% \end{figure}
% 注意:上述\label{}一定要放在\caption{}之后,否则引用图片序号会只会显示??.

\begin{document}

\section{Jordan标准型}

\begin{lemma}\label{lemma:Jordan块的初等因子组}
$r$ 阶矩阵
\[
J = \begin{pmatrix}
\lambda_0 & 1 & & & \\
& \lambda_0 & 1 & & \\
& & \ddots & \ddots & \\
& & & \ddots & 1 \\
& & & & \lambda_0
\end{pmatrix}
\]
的初等因子组为 $(\lambda - \lambda_0)^r$。
\end{lemma}
\begin{proof}
显然 $J$ 的特征多项式为 $(\lambda - \lambda_0)^r$。对任一小于 $r$ 的正整数 $k$,$\lambda I - J$ 总有一个 $k$ 阶子式,其值等于 $(-1)^k$,因此 $J$ 的行列式因子为
\begin{align}
1,\cdots,1,(\lambda - \lambda_0)^r.
\label{equation:7.6.1ewer}
\end{align}
\eqref{equation:7.6.1ewer}式也是 $J$ 的不变因子组,故 $J$ 的初等因子组只有一个多项式 $(\lambda - \lambda_0)^r$。
\end{proof}

\begin{lemma}\label{lemma:初等因子组等于准素因子组}
设特征矩阵 $\lambda I - A$ 经过初等变换化为下列对角阵:
\begin{align}\label{equation7629}
\left( \begin{matrix}
f_1(\lambda )&		&		&		\\
&		f_2(\lambda )&		&		\\
&		&		\ddots&		\\
&		&		&		f_n(\lambda )\\
\end{matrix} \right) ,
\end{align}
其中 $f_i(\lambda)\ (i = 1,\cdots,n)$ 为非零首一多项式。将 $f_i(\lambda)$ 作不可约分解,若 $(\lambda - \lambda_0)^k$ 能整除 $f_i(\lambda)$,但 $(\lambda - \lambda_0)^{k + 1}$ 不能整除 $f_i(\lambda)$,就称 $(\lambda - \lambda_0)^k$ 是 $f_i(\lambda)$ 的一个\textbf{准素因子},则矩阵 $A$ 的初等因子组等于所有 $f_i(\lambda)$ 的准素因子组。
\end{lemma}
\begin{remark}
这个引理给出了求矩阵初等因子组的另外一个方法,它可以不必先求不变因子组而直接用初等变换把特征矩阵化为对角阵,再分解主对角线上的多项式即可。另外,这个引理的结论及其证明在一般的数域 $\mathbb{K}$ 上也成立。 
\end{remark}
\begin{proof}
第一步,先证明下列事实:

若 $f_i(\lambda), f_j(\lambda)\ (i\neq j)$ 的最大公因式和最小公倍式分别为 $g(\lambda), h(\lambda)$,则
\[
\mathrm{diag}\{f_1(\lambda),\cdots,f_i(\lambda),\cdots,f_j(\lambda),\cdots,f_n(\lambda)\}
\]
经过初等变换可以变为
\[
\mathrm{diag}\{f_1(\lambda),\cdots,g(\lambda),\cdots,h(\lambda),\cdots,f_n(\lambda)\},
\]
且这两个对角阵具有相同的准素因子组。

不失一般性,令 $i = 1, j = 2$。因为 $(f_1(\lambda), f_2(\lambda)) = g(\lambda)$,所以存在 $u(\lambda), v(\lambda)$,使
\[
f_1(\lambda)u(\lambda)+f_2(\lambda)v(\lambda)=g(\lambda).
\]
又令 $f_1(\lambda)=g(\lambda)q(\lambda)$,$f_2(\lambda)=g(\lambda)q'(\lambda)$.则 $h(\lambda)=g(\lambda)q(\lambda)q'(\lambda)=f_2(\lambda)q(\lambda)$。对\eqref{equation7629}式作下列初等变换:
\begin{gather*}
\left( \begin{matrix}
f_1(\lambda )&		&		&		\\
&		f_2(\lambda )&		&		\\
&		&		\ddots&		\\
&		&		&		f_n(\lambda )\\
\end{matrix} \right) \xrightarrow{u(\lambda )\cdot r_1+r_2}\left( \begin{matrix}
f_1(\lambda )&		0&		&		\\
f_1(\lambda )u(\lambda )&		f_2(\lambda )&		&		\\
&		&		\ddots&		\\
&		&		&		f_n(\lambda )\\
\end{matrix} \right) \xrightarrow{v(\lambda )j_2+j_1}
\\
\left( \begin{matrix}
f_1(\lambda )&		&		&		\\
g(\lambda )&		f_2(\lambda )&		&		\\
&		&		\ddots&		\\
&		&		&		f_n(\lambda )\\
\end{matrix} \right) \xrightarrow{-q(\lambda )r_2+r_1}\left( \begin{matrix}
0&		-h(\lambda )&		&		\\
g(\lambda )&		f_2(\lambda )&		&		\\
&		&		\ddots&		\\
&		&		&		f_n(\lambda )\\
\end{matrix} \right) \underset{\left( -1 \right) \cdot r_2}{\xrightarrow{r_2\longleftrightarrow r_1}}
\\
\left( \begin{matrix}
g(\lambda )&		f_2\left( \lambda \right)&		&		\\
&		h(\lambda )&		&		\\
&		&		\ddots&		\\
&		&		&		f_n(\lambda )\\
\end{matrix} \right) \xrightarrow{-q\prime (\lambda )j_1+j_2}\left( \begin{matrix}
g(\lambda )&		&		&		\\
&		h(\lambda )&		&		\\
&		&		\ddots&		\\
&		&		&		f_n(\lambda )\\
\end{matrix} \right) .
\end{gather*}
现来考察 $g(\lambda)$ 与 $h(\lambda)$ 的准素因子。将 $f_1(\lambda), f_2(\lambda)$ 作标准因式分解,其分解式不妨设为
\begin{align*}
f_1(\lambda)&=(\lambda - \lambda_1)^{c_1}(\lambda - \lambda_2)^{c_2}\cdots(\lambda - \lambda_t)^{c_t},\\
f_2(\lambda)&=(\lambda - \lambda_1)^{d_1}(\lambda - \lambda_2)^{d_2}\cdots(\lambda - \lambda_t)^{d_t},
\end{align*}
其中 $c_i, d_i$ 为非负整数。令
\[
e_i = \max\{c_i, d_i\},\quad k_i = \min\{c_i, d_i\},
\]
则
\begin{align*}
g(\lambda)&=(\lambda - \lambda_1)^{k_1}(\lambda - \lambda_2)^{k_2}\cdots(\lambda - \lambda_t)^{k_t},\\
h(\lambda)&=(\lambda - \lambda_1)^{e_1}(\lambda - \lambda_2)^{e_2}\cdots(\lambda - \lambda_t)^{e_t}.
\end{align*}
不难看出 $g(\lambda), h(\lambda)$ 的准素因子组与 $f_1(\lambda), f_2(\lambda)$ 的准素因子组相同。 

第二步证明 \eqref{equation7629}式所示矩阵的法式可通过上述变换得到。

先将第 $(1,1)$ 位置的元素依次和第 $(2,2)$ 位置,$\cdots$,第 $(n,n)$ 位置的元素进行上述变换,此时第 $(1,1)$ 元素的所有一次因式的幂都是最小的;再将第 $(2,2)$ 位置的元素依次和第 $(3,3)$ 位置,$\cdots$,第 $(n,n)$ 位置的元素进行上述变换;$\cdots$;最后将第 $(n - 1,n - 1)$ 位置的元素和第 $(n,n)$ 位置的元素进行上述变换。可以看出,最后得到的对角阵就是 \eqref{equation7629}式所示矩阵的法式。注意到在每一次变换的过程中,准素因子组都保持不变,这就证明了结论。
\end{proof}

\begin{example}
设 $\lambda I - A$ 经过初等变换后化为下列对角阵:
\[
\begin{pmatrix}
1 & & & & \\
& (\lambda - 1)^2(\lambda + 2) & & & \\
& & \lambda + 2 & & \\
& & & 1 & \\
& & & & \lambda - 1
\end{pmatrix},
\]
求 $A$ 的初等因子组。
\end{example}
\begin{solution}
由\reflemma{lemma:初等因子组等于准素因子组} 知,$A$ 的初等因子组为 $\lambda - 1,(\lambda - 1)^2,\lambda + 2,\lambda + 2$。
\end{solution}

\begin{lemma}\label{lemma:Jordan矩阵的初等因子组}
设 $J$ 是分块对角阵:
\begin{align*}
\begin{pmatrix}
J_1 & & & \\
& J_2 & & \\
& & \ddots & \\
& & & J_k
\end{pmatrix},
\end{align*}
其中每个 $J_i$ 都是形如\reflemma{lemma:Jordan块的初等因子组}中的矩阵,$J_i$ 的初等因子组为 $(\lambda - \lambda_i)^{r_i}$,则 $J$ 的初等因子组为
\begin{align*}
(\lambda - \lambda_1)^{r_1}, (\lambda - \lambda_2)^{r_2}, \cdots, (\lambda - \lambda_k)^{r_k}.
\end{align*}
\end{lemma}
\begin{proof}
$\lambda I - J$ 是一个分块对角 $\lambda$-矩阵。由于对分块对角阵中某一块施行初等变换时其余各块保持不变,故由\reflemma{lemma:Jordan块的初等因子组}及\refproposition{proposition:特征矩阵的法式和不变因子}知,$\lambda I - J$ 相抵于下列分块对角阵:
\begin{align*}
H = \begin{pmatrix}
H_1 & & & \\
& H_2 & & \\
& & \ddots & \\
& & & H_k
\end{pmatrix},
\end{align*}
其中 $H_i = \mathrm{diag}\{1,\cdots,1,(\lambda - \lambda_i)^{r_i}\}$。再由\reflemma{lemma:初等因子组等于准素因子组}即得结论。
\end{proof}

\begin{theorem}[Jordan标准型]\label{theorem:Jordan标准型}
设 $A$ 是复数域上的矩阵且 $A$ 的初等因子组为
\[
(\lambda - \lambda_1)^{r_1}, (\lambda - \lambda_2)^{r_2}, \cdots, (\lambda - \lambda_k)^{r_k},
\]
则 $A$ 相似于分块对角阵:
\begin{align}
J = \begin{pmatrix}
J_1 & & & \\
& J_2 & & \\
& & \ddots & \\
& & & J_k
\end{pmatrix},
\label{equation-1237.6.4}
\end{align}
其中 $J_i$ 为 $r_i$ 阶矩阵,且
\begin{align*}
J_i = \begin{pmatrix}
\lambda_i & 1 & & & \\
& \lambda_i & 1 & & \\
& & \ddots & \ddots & \\
& & & \ddots & 1 \\
& & & & \lambda_i
\end{pmatrix}.
\end{align*}
\eqref{equation-1237.6.4}式中的矩阵 $J$ 称为 $A$ 的\textbf{ Jordan 标准型},每个 $J_i$ 称为 $A$ 的一个 \textbf{ Jordan 块}。 
\end{theorem}
\begin{remark}
由\reflemma{lemma:Jordan块的初等因子组}可以看出,若交换任意两个 Jordan 块的位置,得到的矩阵与原来的矩阵仍有相同的初等因子组,它们仍相似。因此矩阵 $A$ 的 Jordan 标准型中 Jordan 块的排列可以是任意的。但是,由于每个初等因子唯一确定了一个 Jordan 块,故若不计 Jordan 块的排列次序,则矩阵的 Jordan 标准型是唯一确定的。 
\end{remark}
\begin{proof}
由\reftheorem{theorem:矩阵相似的充分必要条件是它们有相同的初等因子组}知,$A$ 与 $J$ 有相同的初等因子组,因此 $A$ 与 $J$ 相似。
\end{proof}

\begin{theorem}
设 $\varphi$ 是复数域上线性空间 $V$ 上的线性变换,则必存在 $V$ 的一组基,使得 $\varphi$ 在这组基下的表示矩阵为 (7.6.4) 式所示的 Jordan 标准型。
\end{theorem}
\begin{proof}

\end{proof}

\begin{corollary}
设 $A$ 是 $n$ 阶复矩阵,则下列结论等价:

(1) $A$ 可对角化;

(2) $A$ 的极小多项式无重根;

(3) $A$ 的初等因子都是一次多项式。
\end{corollary}
\begin{proof}
(1) $\Rightarrow$ (2):由\hyperref[theorem:可对角化的判定条件]{可对角化的判定条件(5)}的结论即得。

(2) $\Rightarrow$ (3):设 $A$ 的极小多项式 $m(\lambda)$ 无重根。由于 $m(\lambda)$ 是 $A$ 的最后一个不变因子,故 $A$ 的所有不变因子都无重根,从而 $A$ 的初等因子都是一次多项式。

(3) $\Rightarrow$ (1):设 $A$ 的初等因子组为 $\lambda - \lambda_1,\lambda - \lambda_2,\cdots,\lambda - \lambda_n$,则由\reftheorem{theorem:Jordan标准型}知,$A$ 相似于对角阵 $\mathrm{diag}\{\lambda_1,\lambda_2,\cdots,\lambda_n\}$,即 $A$ 可对角化。
\end{proof}

\begin{corollary}\label{corollary:矩阵可对角化的几何叙述}
设 $\varphi$ 是复线性空间 $V$ 上的线性变换,则 $\varphi$ 可对角化当且仅当 $\varphi$ 的极小多项式无重根,当且仅当 $\varphi$ 的初等因子都是一次多项式。
\end{corollary}
\begin{proof}

\end{proof}

\begin{corollary}\label{corollary:线性变换可对角化则其在不变子空间的限制上也可对角化}
设 $\varphi$ 是复线性空间 $V$ 上的线性变换,$V_0$ 是 $\varphi$ 的不变子空间。若 $\varphi$ 可对角化,则 $\varphi$ 在 $V_0$ 上的限制也可对角化。
\end{corollary}
\begin{proof}
设 $\varphi,\varphi|_{V_0}$ 的极小多项式分别为 $g(\lambda),h(\lambda)$,则由\refcorollary{corollary:矩阵可对角化的几何叙述}知,$g(\lambda)$ 无重根。又 $g(\varphi|_{V_0}) = g(\varphi)|_{V_0} = \mathbf{0}$,故 $h(\lambda)\mid g(\lambda)$,于是 $h(\lambda)$ 也无重根,再次由\refcorollary{corollary:矩阵可对角化的几何叙述}知,$\varphi|_{V_0}$ 可对角化。
\end{proof}

\begin{corollary}\label{corollary:线性变换可对角化关于在直和分解限制的充要条件}
设 $\varphi$ 是复线性空间 $V$ 上的线性变换,且 $V = V_1\oplus V_2\oplus\cdots\oplus V_k$,其中每个 $V_i$ 都是 $\varphi$ 的不变子空间,则 $\varphi$ 可对角化的充分必要条件是 $\varphi$ 在每个 $V_i$ 上的限制都可对角化。
\end{corollary}
\begin{proof}
必要性由\refcorollary{corollary:线性变换可对角化则其在不变子空间的限制上也可对角化}即得,下证充分性。若 $\varphi$ 在每个 $V_i$ 上的限制都可对角化,则由定义存在 $V_i$ 的一组基,使得 $\varphi|_{V_i}$ 在这组基下的表示矩阵是对角阵。再由\reftheorem{theorem:在直和的基下的矩}知 $V_i$ 的一组基可以拼成 $V$ 的一组基,因此 $\varphi$ 在这组基下的表示阵是对角阵,即 $\varphi$ 可对角化。
\end{proof}

\begin{corollary}
设 $A$ 是数域 $\mathbb{K}$ 上的矩阵,如果 $A$ 的特征值全在 $\mathbb{K}$ 中,则 $A$ 在 $\mathbb{K}$ 上相似于其 Jordan 标准型。
\end{corollary}
\begin{proof}
由于 $A$ 的特征值全在 $\mathbb{K}$ 中,故 $A$ 的 Jordan 标准型 $J$ 实际上是 $\mathbb{K}$ 上的矩阵。因为 $A$ 在复数域上相似于 $J$,由\refcorollary{corollary:线性变换可对角化关于在直和分解限制的充要条件} 知,$A$ 在 $\mathbb{K}$ 上也相似于 $J$。
\end{proof}

\begin{example}
设 $A$ 是 7 阶矩阵,其初等因子组为
\[
\lambda - 1, (\lambda - 1)^3, (\lambda + 1)^2, \lambda - 2,
\]
求其 Jordan 标准型。
\end{example}
\begin{solution}
$A$ 的 Jordan 标准型为
\[
J = \begin{pmatrix}
1 & & & & & & \\
& 1 & 1 & 0 & & & \\
& 0 & 1 & 1 & & & \\
& 0 & 0 & 1 & & & \\
& & & & -1 & 1 & \\
& & & & 0 & -1 & \\
& & & & & & 2
\end{pmatrix},
\]
$J$ 含有 4 个 Jordan 块。
\end{solution}

\begin{example}
设复数域上的四维线性空间 $V$ 上的线性变换 $\varphi$ 在一组基 $\{e_1,e_2,e_3,e_4\}$ 下的表示矩阵为
\[
A = \begin{pmatrix}
3 & 1 & 0 & 0 \\
-4 & -1 & 0 & 0 \\
6 & 1 & 2 & 1 \\
-14 & -5 & -1 & 0
\end{pmatrix},
\]
求 $V$ 的一组基,使 $\varphi$ 在这组基下的表示矩阵为 Jordan 标准型,并求出从原来的基到新基的过渡矩阵。
\end{example}
\begin{solution}
用初等变换把 $\lambda I - A$ 化为对角 $\lambda$-矩阵并求出它的初等因子组为
\[
(\lambda - 1)^2, (\lambda - 1)^2.
\]
因此,$A$ 的 Jordan 标准型为
\[
J = \begin{pmatrix}
1 & 1 & & \\
0 & 1 & & \\
& & 1 & 1 \\
& & 0 & 1
\end{pmatrix}.
\]
设矩阵 $P$ 是从 $\{e_1,e_2,e_3,e_4\}$ 到新基的过渡矩阵,则
\[
P^{-1}AP = J,
\]
此即
\begin{align}
AP = PJ.\label{equation----7.6.6}
\end{align}
设 $P = (\alpha_1,\alpha_2,\alpha_3,\alpha_4)$,其中 $\alpha_i$ 是四维列向量,代入\eqref{equation----7.6.6}式得
\[
(A\alpha_1,A\alpha_2,A\alpha_3,A\alpha_4) = (\alpha_1,\alpha_2,\alpha_3,\alpha_4)\begin{pmatrix}
1 & 1 & & \\
0 & 1 & & \\
& & 1 & 1 \\
& & 0 & 1
\end{pmatrix},
\]
化成方程组为
\begin{align*}
(A - I)\alpha_1 &= \mathbf{0},\\
(A - I)\alpha_2 &= \alpha_1,\\
(A - I)\alpha_3 &= \mathbf{0},\\
(A - I)\alpha_4 &= \alpha_3.
\end{align*}
由于 $\alpha_1,\alpha_3$ 都是 $A$ 的属于特征值 1 的特征向量,故 $\alpha_2,\alpha_4$ 称为属于特征值 1 的广义特征向量。我们可取方程组 $(A - I)x = \mathbf{0}$ 的两个线性无关的解分别作为 $\alpha_1,\alpha_3$ (注意不能取线性相关的两个解,因为 $P$ 是非异阵),然后再分别求出 $\alpha_2,\alpha_4$ (注意诸 $\alpha_i$ 的解可能不唯一,只需取比较简单的一组解) 即可。经计算可得
\[
\alpha_1 = \begin{pmatrix}
1 \\
-2 \\
1 \\
-5
\end{pmatrix}, \quad \alpha_2 = \begin{pmatrix}
0 \\
1 \\
0 \\
0
\end{pmatrix}, \quad \alpha_3 = \begin{pmatrix}
0 \\
0 \\
1 \\
-1
\end{pmatrix}, \quad \alpha_4 = \begin{pmatrix}
0 \\
0 \\
0 \\
1
\end{pmatrix}.
\]
于是
\[
P = \begin{pmatrix}
1 & 0 & 0 & 0 \\
-2 & 1 & 0 & 0 \\
1 & 0 & 1 & 0 \\
-5 & 0 & -1 & 1
\end{pmatrix}.
\]
因此新基为
\[
\{e_1 - 2e_2 + e_3 - 5e_4, e_2, e_3 - e_4, e_4\}.
\] 
\end{solution}
















\end{document}
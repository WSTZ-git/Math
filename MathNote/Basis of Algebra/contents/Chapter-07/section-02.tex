\documentclass[../../main.tex]{subfiles}
\graphicspath{{\subfix{../../image/}}} % 指定图片目录,后续可以直接使用图片文件名。

% 例如:
% \begin{figure}[h]
% \centering
% \includegraphics{image-01.01}
% \caption{图片标题}
% \label{fig:image-01.01}
% \end{figure}
% 注意:上述\label{}一定要放在\caption{}之后,否则引用图片序号会只会显示??.

\begin{document}

\section{矩阵的法式}

\begin{lemma}\label{lemma:lambda矩阵一定相抵于b_11整除所有元素的矩阵}
设$A(\lambda)=(a_{ij}(\lambda))_{m\times n}$是任一非零$\lambda$-矩阵,则$A(\lambda)$必相抵于这样的一个$\lambda$-矩阵$B(\lambda)=(b_{ij}(\lambda))_{m\times n}$,其中$b_{11}(\lambda)\neq 0$且$b_{11}(\lambda)$可整除$B(\lambda)$中的任一元素$b_{ij}(\lambda)$.
\end{lemma}
\begin{proof}
设$k = \min\{\deg a_{ij}(\lambda)\mid a_{ij}(\lambda)\neq 0, 1\leq i\leq m; 1\leq j\leq n\}$,我们对$k$用数学归纳法。首先,经行对换及列对换可将$A(\lambda)$的第$(1,1)$元素变成次数最低的非零多项式,因此不妨设$a_{11}(\lambda)\neq 0$且$\deg a_{11}(\lambda)=k$。若$k = 0$,则$a_{11}(\lambda)$是一个非零常数,结论显然成立。假设对非零元素次数的最小值小于$k$的任一$\lambda$-矩阵,引理的结论成立,现考虑非零元素次数的最小值等于$k$的$\lambda$-矩阵$A(\lambda)$。若$a_{11}(\lambda)$可整除所有的$a_{ij}(\lambda)$,则结论已成立。若否,设在第一列中有元素$a_{i1}(\lambda)$不能被$a_{11}(\lambda)$整除,作带余除法:
\begin{align*}
a_{i1}(\lambda)=a_{11}(\lambda)q(\lambda)+r(\lambda).
\end{align*}
用$-q(\lambda)$乘以第一行加到第$i$行上,第$(i,1)$元素就变为$r(\lambda)$。注意到$r(\lambda)\neq 0$且$\deg r(\lambda)<\deg a_{11}(\lambda)=k$,由归纳假设即知结论成立。

同样的方法可施于第一行。因此我们不妨设$a_{11}(\lambda)$可整除第一行及第一列。这时,设$a_{21}(\lambda)=a_{11}(\lambda)g(\lambda)$。将第一行乘以$-g(\lambda)$加到第二行上,则第$(2,1)$元素变为零。用同样的方法可消去第一行、第一列除$a_{11}(\lambda)$以外的所有元素,于是$A(\lambda)$经初等变换后变成下列形状:
\begin{align*}
\begin{pmatrix}
a_{11}(\lambda)&0&\cdots&0\\
0&a_{22}'(\lambda)&\cdots&a_{2n}'(\lambda)\\
\vdots&\vdots&\ddots&\vdots\\
0&a_{m2}'(\lambda)&\cdots&a_{mn}'(\lambda)
\end{pmatrix}.
\end{align*}
这时,若$a_{11}(\lambda)$可整除所有其他元素,则结论已成立。若否,比如$a_{11}(\lambda)$不能整除$a_{ij}'(\lambda)$,则将第$i$行加到第一行上去,这时在第一行又出现了一元素$a_{ij}'(\lambda)$,它不能被$a_{11}(\lambda)$整除。重复上面的做法,通过归纳假设即可得到结论.
\end{proof}

\begin{theorem}\label{theorem:lambda矩阵相抵于特殊的对角阵}
设$A(\lambda)$是一个$n$阶$\lambda$-矩阵,则$A(\lambda)$相抵于对角阵
\begin{align}
\mathrm{diag}\{d_1(\lambda),d_2(\lambda),\cdots,d_r(\lambda);0,\cdots,0\},\label{theorem0.1-7.1.1}
\end{align}
其中$d_i(\lambda)$是非零首一多项式且$d_i(\lambda)\mid d_{i + 1}(\lambda)$ ($i = 1,2,\cdots,r - 1$)。我们
称上式中的对角$\lambda$-矩阵为$A(\lambda)$的\textbf{法式}或\textbf{相抵标准型}. 
\end{theorem}
\begin{proof}
对$n$用数学归纳法,当$n = 1$时结论显然,现设$A(\lambda)$是$n$阶$\lambda$-矩阵。由\hyperref[lemma:lambda矩阵一定相抵于b_11整除所有元素的矩阵]{引理\ref{lemma:lambda矩阵一定相抵于b_11整除所有元素的矩阵}}可知$A(\lambda)$相抵于$n$阶$\lambda$-矩阵$B(\lambda)=(b_{ij}(\lambda))$,其中$b_{11}(\lambda)\mid b_{ij}(\lambda)$对一切$i,j$成立。因此,将$B(\lambda)$的第一行乘以$\lambda$的某个多项式加到第二行上去便可消去$b_{21}(\lambda)$。同理可依次消去第一列除$b_{11}(\lambda)$以外的所有元素。再用类似方法消去第一行其余元素。这样便得到了一个矩阵:
\begin{align*}
\begin{pmatrix}
b_{11}(\lambda)&0&\cdots&0\\
0&b_{22}'(\lambda)&\cdots&b_{2n}'(\lambda)\\
\vdots&\vdots&\ddots&\vdots\\
0&b_{n2}'(\lambda)&\cdots&b_{nn}'(\lambda)
\end{pmatrix}.
\end{align*}
不难看出,这时$b_{11}(\lambda)$仍可整除所有的$b_{ij}'(\lambda)$。设$c$为$b_{11}(\lambda)$的首项系数,记$d_1(\lambda)=c^{-1}b_{11}(\lambda)$,设$\overline{B}(\lambda)$为上面的矩阵中右下方的$n - 1$阶$\lambda$-矩阵,则由归纳假设可知存在$P(\lambda)$及$Q(\lambda)$,使
\begin{align*}
P(\lambda)\overline{B}(\lambda)Q(\lambda)=\mathrm{diag}\{d_2(\lambda),\cdots,d_r(\lambda);0,\cdots,0\},
\end{align*}
且$d_i(\lambda)\mid d_{i + 1}(\lambda)$ ($i = 2,\cdots,r - 1$),其中$P(\lambda)$与$Q(\lambda)$可写成为有限个$n - 1$阶初等$\lambda$-矩阵之积。因此
\begin{align*}
\begin{pmatrix}
1&O\\
O&P(\lambda)
\end{pmatrix}
\begin{pmatrix}
d_1(\lambda)&O\\
O&\overline{B}(\lambda)
\end{pmatrix}
\begin{pmatrix}
1&O\\
O&Q(\lambda)
\end{pmatrix}
=\mathrm{diag}\{d_1(\lambda),d_2(\lambda),\cdots,d_r(\lambda);0,\cdots,0\},
\end{align*}
且
\begin{align*}
\begin{pmatrix}
1&O\\
O&P(\lambda)
\end{pmatrix},
\begin{pmatrix}
1&O\\
O&Q(\lambda)
\end{pmatrix}
\end{align*}
可写成有限个$n$阶初等$\lambda$-矩阵之积。于是只需证明$d_1(\lambda)\mid d_2(\lambda)$即可。但这点很容易看出,事实上由于$\overline{B}(\lambda)$中的任一元素均可被$d_1(\lambda)$整除,因此$P(\lambda)\overline{B}(\lambda)Q(\lambda)$中的任一元素也可被$d_1(\lambda)$整除,这就证明了定理。
\end{proof}
\begin{remark}
我们上面对$n$阶$\lambda$-矩阵证明了它必相抵于一个对角阵。事实上,对长方$\lambda$-矩阵,结论也同样成立,证明也类似。\eqref{theorem0.1-7.1.1}式中的$r$通常称为$A(\lambda)$的秩。但要注意即使某个$n$阶$\lambda$-矩阵的秩等于$n$,它也未必是可逆$\lambda$-矩阵.
\end{remark}

\begin{corollary}
任一$n$阶可逆$\lambda$-矩阵都可表示为有限个初等$\lambda$-矩阵之积.
\end{corollary}
\begin{proof}
由\hyperref[theorem:lambda矩阵相抵于特殊的对角阵]{定理\ref{theorem:lambda矩阵相抵于特殊的对角阵}},存在$P(\lambda),Q(\lambda)$,使可逆阵$A(\lambda)$适合
\begin{align*}
P(\lambda)A(\lambda)Q(\lambda)=\mathrm{diag}\{d_1(\lambda),d_2(\lambda),\cdots,d_r(\lambda);0,\cdots,0\},
\end{align*}
其中$P(\lambda),Q(\lambda)$为有限个初等$\lambda$-矩阵之积。因为上式左边是个可逆阵,故右边的矩阵也可逆,从而$r = n$。注意一个对角$\lambda$-矩阵要可逆必须$d_1(\lambda),d_2(\lambda),\cdots,d_n(\lambda)$皆为非零常数,又它们都是首一多项式,故只能是$d_1(\lambda)=d_2(\lambda)=\cdots=d_n(\lambda)=1$,于是
\begin{align*}
A(\lambda)=P(\lambda)^{-1}Q(\lambda)^{-1}.
\end{align*}
因为初等$\lambda$-矩阵的逆仍是初等$\lambda$-矩阵,故$P(\lambda)^{-1}$与$Q(\lambda)^{-1}$都是有限个初等$\lambda$-矩阵之积,从而$A(\lambda)$也是有限个初等$\lambda$-矩阵之积. 
\end{proof}

\begin{corollary}\label{corollary:特征矩阵的法式}
设$A$是数域$\mathbb{K}$上的$n$阶矩阵,则$A$的特征矩阵$\lambda I_n - A$必相抵于
\begin{align*}
\mathrm{diag}\{1,\cdots,1,d_1(\lambda),\cdots,d_m(\lambda)\},
\end{align*}
其中$d_i(\lambda)\mid d_{i + 1}(\lambda)$ ($i = 1,2,\cdots,m - 1$)。我们称上式中的对角$\lambda$-矩阵为$A(\lambda)$的特征矩阵$\lambda-A$的\textbf{法式}或\textbf{相抵标准型}. 
\end{corollary}
\begin{proof}
由\hyperref[theorem:lambda矩阵相抵于特殊的对角阵]{定理\ref{theorem:lambda矩阵相抵于特殊的对角阵}},存在$P(\lambda),Q(\lambda)$,使
\begin{align*}
P(\lambda)(\lambda I_n - A)Q(\lambda)=\mathrm{diag}\{d_1(\lambda),d_2(\lambda),\cdots,d_r(\lambda);0,\cdots,0\},
\end{align*}
其中$P(\lambda),Q(\lambda)$为有限个初等$\lambda$-矩阵之积。根据$\lambda$-矩阵初等变换的定义以及行列式的性质可得,上式左边的行列式等于$c|\lambda I_n - A|$,其中$c$是一个非零常数,从而上式右边的行列式不为零,故$r = n$。把$d_i(\lambda)$中的常数多项式写出来(因是首一多项式,故为常数$1$),即得结论.
\end{proof}

\begin{example}
求$\lambda I - A$的法式,其中
\[A = \begin{pmatrix}
0 & 1 & -1 \\
3 & -2 & 0 \\
-1 & 1 & -1
\end{pmatrix}.\]
\end{example}
\begin{solution}
\begin{align*}
\lambda I - A &= 
\begin{pmatrix}
\lambda & -1 & 1 \\
-3 & \lambda + 2 & 0 \\
1 & -1 & \lambda + 1
\end{pmatrix}
\xrightarrow{r_1\leftrightarrow r_3}
\begin{pmatrix}
1 & -1 & \lambda + 1 \\
-3 & \lambda + 2 & 0 \\
\lambda & -1 & 1
\end{pmatrix}\\
&\xrightarrow{3r_1+r_2,-\lambda r_1+r_3}
\begin{pmatrix}
1 & -1 & \lambda + 1 \\
0 & \lambda - 1 & 3\lambda + 3 \\
0 & \lambda - 1 & -\lambda^2 - \lambda + 1
\end{pmatrix}
\xrightarrow{j_1+j_2,-(\lambda +1)j_1+j_3}
\begin{pmatrix}
1 & 0 & 0 \\
0 & \lambda - 1 & 3\lambda + 3 \\
0 & \lambda - 1 & -\lambda^2 - \lambda + 1
\end{pmatrix}\\
&\xrightarrow{-3j_2+j_3}
\begin{pmatrix}
1 & 0 & 0 \\
0 & \lambda - 1 & 6 \\
0 & \lambda - 1 & -\lambda^2 - 4\lambda + 4
\end{pmatrix}
\xrightarrow{j_2\leftrightarrow j_3}
\begin{pmatrix}
1 & 0 & 0 \\
0 & 6 & \lambda - 1 \\
0 & -\lambda^2 - 4\lambda + 4 & \lambda - 1
\end{pmatrix}\\
&\xrightarrow{6j_3}
\begin{pmatrix}
1 & 0 & 0 \\
0 & 6 & 6(\lambda - 1) \\
0 & -\lambda^2 - 4\lambda + 4 & 6(\lambda - 1)
\end{pmatrix}
\xrightarrow{-(\lambda-1)j_2+j_3}
\begin{pmatrix}
1 & 0 & 0 \\
0 & 6 & 0 \\
0 & -\lambda^2 - 4\lambda + 4 & (\lambda - 1)(\lambda^2 + 4\lambda + 2)
\end{pmatrix}\\
&\xrightarrow{\frac{1}{6}j_2}
\begin{pmatrix}
1 & 0 & 0 \\
0 & 1 & 0 \\
0 & -\lambda^2 - 4\lambda + 4 & (\lambda - 1)(\lambda^2 + 4\lambda + 2)
\end{pmatrix}
\xrightarrow{-(-\lambda^2 - 4\lambda + 4)r_2+r_3}
\begin{pmatrix}
1 & 0 & 0 \\
0 & 1 & 0 \\
0 & 0 & (\lambda - 1)(\lambda^2 + 4\lambda + 2)
\end{pmatrix}.
\end{align*} 
\end{solution}











\end{document}
\documentclass[../../main.tex]{subfiles}
\graphicspath{{\subfix{../../image/}}} % 指定图片目录,后续可以直接使用图片文件名。

% 例如:
% \begin{figure}[H]
% \centering
% \includegraphics[scale=0.3]{image-01.01}
% \caption{图片标题}
% \label{figure:image-01.01}
% \end{figure}
% 注意:上述\label{}一定要放在\caption{}之后,否则引用图片序号会只会显示??.

\begin{document}

\section{不变因子}

\begin{definition}[$k$阶行列式因子]
设 $A(\lambda)$ 是 $n$ 阶 $\lambda$-矩阵,$k$ 是小于等于 $n$ 的正整数。如果 $A(\lambda)$ 有一个 $k$ 阶子式不为零,则定义 $A(\lambda)$ 的 \textbf{$k$ 阶行列式因子} $D_k(\lambda)$ 为 $A(\lambda)$ 的所有 $k$ 阶子式的最大公因式(首一多项式)。如果 $A(\lambda)$ 的所有 $k$ 阶子式都等于零,则定义 $A(\lambda)$ 的 $k$ 阶行列式因子 $D_k(\lambda)$ 为零。 
\end{definition}

\begin{lemma}\label{lemma:i阶行列式因子整除i+1阶行列式因子}
设 $D_1(\lambda), D_2(\lambda), \cdots, D_r(\lambda)$ 是 $A(\lambda)$ 的非零行列式因子,则
\begin{align*}
D_i(\lambda)\mid D_{i + 1}(\lambda),\quad i = 1, 2, \cdots, r - 1.
\end{align*}
\end{lemma}
\begin{proof}
设 $A_{i + 1}$ 是 $A(\lambda)$ 的任一 $i + 1$ 阶子式,即在 $A(\lambda)$ 中任意取出 $i + 1$ 行及 $i + 1$ 列组成的行列式。将这个行列式按某一行展开,则它的每一个展开项都是一个多项式与一个 $i$ 阶子式的乘积。由于 $D_i(\lambda)$ 是所有 $i$ 阶子式的公因子,因此 $D_i(\lambda)\mid A_{i + 1}$。而 $D_{i + 1}(\lambda)$ 是所有 $i + 1$ 阶子式的最大公因子,因此 $D_i(\lambda)\mid D_{i + 1}(\lambda)$ 对一切 $i = 1, 2, \cdots, r - 1$ 成立。 
\end{proof}

\begin{definition}[不变因子]
设 $D_1(\lambda), D_2(\lambda), \cdots, D_r(\lambda)$ 是 $\lambda$-矩阵 $A(\lambda)$ 的非零行列式因子,则
\begin{align*}
g_1(\lambda)&=D_1(\lambda),\\
g_2(\lambda)&=D_2(\lambda)/D_1(\lambda),\\
&\cdots\\
g_r(\lambda)&=D_r(\lambda)/D_{r - 1}(\lambda)
\end{align*}
称为 $A(\lambda)$ 的\textbf{不变因子}。 
\end{definition}
\begin{note}
由不变因子和行列式因子的定义可知,不变因子和行列式因子相互唯一确定.
\end{note}
\begin{remark}
以后特征矩阵 $\lambda I - A$ 的行列式因子和不变因子均简称为 $A$ 的行列式因子和不变因子。 
\end{remark}

\begin{proposition}\label{proposition:相抵标准型的行列式因子和不变因子}
求下列矩阵的行列式因子和不变因子:
\[
A(\lambda)=\begin{pmatrix}
d_1(\lambda) & & & & \\
& \ddots & & & \\
& & d_r(\lambda) & & \\
& & & 0 & \\
& & & & \ddots & \\
& & & & & 0
\end{pmatrix},
\]
其中 $d_i(\lambda)$ 为非零首一多项式且 $d_i(\lambda)\mid d_{i + 1}(\lambda)\ (i = 1, 2, \cdots, r - 1)$。
\end{proposition}
\begin{solution}
$A(\lambda)$ 的非零行列式因子为
\begin{align*}
D_1(\lambda)&=d_1(\lambda),\\
D_2(\lambda)&=d_1(\lambda)d_2(\lambda),\\
&\cdots\\
D_r(\lambda)&=d_1(\lambda)d_2(\lambda)\cdots d_r(\lambda).
\end{align*}
根据不变因子的定义可知$A(\lambda)$的不变因子分别为:$d_1(\lambda),d_2(\lambda),\cdots,d_n(\lambda).$
\end{solution}

\begin{theorem}\label{theorem:相抵的lamda矩阵有相同的行列式因子和不变因子}
相抵的 $\lambda$-矩阵有相同的行列式因子,从而有相同的不变因子。
\end{theorem}
\begin{proof}
我们只需证明行列式因子在三类初等变换下不改变就可以了。对第一类初等变换,交换 $\lambda$-矩阵 $A(\lambda)$ 的任意两行(列),显然 $A(\lambda)$ 的 $i$ 阶子式最多改变一个符号,因此行列式因子不改变。

对第二类初等变换,$A(\lambda)$ 的 $i$ 阶子式与变换后矩阵的 $i$ 阶子式最多差一个非零常数,因此行列式因子也不改变。

对第三类初等变换,记变换后的矩阵为 $B(\lambda)$,则 $B(\lambda)$ 与 $A(\lambda)$ 的 $i$ 阶子式可能出现以下 3 种情形:子式完全相同;$B(\lambda)$ 子式中的某一行(列)等于 $A(\lambda)$ 中相应子式的同一行(列)加上该子式中某一行(列)与某个多项式之积;$B(\lambda)$ 子式中的某一行(列)等于 $A(\lambda)$ 中相应子式的同一行(列)加上不在该子式中的某一行(列)与某个多项式之积。在前面两种情形,行列式的值不改变,因此不影响行列式因子。现在来讨论第三种情形。设 $B_i$ 为 $B(\lambda)$ 的 $i$ 阶子式,相应的 $A(\lambda)$ 的 $i$ 阶子式记为 $A_i$,则由行列式的性质得
\begin{align*}
B_i = A_i + f(\lambda)\widetilde{A}_i,
\end{align*}
其中 $\widetilde{A}_i$ 由 $A(\lambda)$ 中的 $i$ 行与 $i$ 列组成,因此它与 $A(\lambda)$ 的某个 $i$ 阶子式最多差一个符号。$f(\lambda)$ 是乘以某一行(列)的那个多项式,于是 $A(\lambda)$ 的行列式因子 $D_i(\lambda)\mid A_i$,$D_i(\lambda)\mid\widetilde{A}_i$,故 $D_i(\lambda)\mid B_i$。这说明,$D_i(\lambda)$ 可整除 $B(\lambda)$ 的所有 $i$ 阶子式,因此 $D_i(\lambda)$ 可整除 $B(\lambda)$ 的 $i$ 阶行列式因子 $\widetilde{D}_i(\lambda)$。但 $B(\lambda)$ 也可用第三类初等变换变成 $A(\lambda)$,于是 $\widetilde{D}_i(\lambda)\mid D_i(\lambda)$。由于 $D_i(\lambda)$ 及 $\widetilde{D}_i(\lambda)$ 都是首一多项式,因此必有 $D_i(\lambda)=\widetilde{D}_i(\lambda)$.
\end{proof}

\begin{corollary}\label{corollary:不变因子和法式相互唯一确定}
设$n$ 阶 $\lambda$-矩阵 $A(\lambda)$ 的法式为
\begin{align*}
\varLambda=\mathrm{diag}\{d_1(\lambda),d_2(\lambda),\cdots,d_r(\lambda);0,\cdots,0\},
\end{align*}
其中 $d_i(\lambda)$ 是非零首一多项式且 $d_i(\lambda)$ $\mid$ $ d_{i + 1}(\lambda)$ ($i$ $=$ $1,$ $2,$ $\cdots,$ $r - 1$),则 $A(\lambda)$ 的不变因子为 $d_1(\lambda),d_2(\lambda),\cdots,d_r(\lambda)$特别地,\textbf{法式和不变因子之间相互唯一确定}。
\end{corollary}
\begin{proof}
首先,由\refthe{theorem:相抵的lamda矩阵有相同的行列式因子和不变因子}可知,$A(\lambda)$ 与 $\varLambda$ 有相同的不变因子。再由\refpro{proposition:相抵标准型的行列式因子和不变因子}可知,$\varLambda$ 的不变因子为 $d_1(\lambda),$ $d_2(\lambda),$ $\cdots,$ $d_r(\lambda)$,从而它们也是 $A(\lambda)$ 的不变因子。故$A(\lambda)$的法式可以唯一确定其不变因子.

接着,设$A(\lambda)$ 的不变因子为 $d_1(\lambda),d_2(\lambda),\cdots,d_r(\lambda)$,由\refthe{theorem:lambda矩阵相抵于特殊的对角阵},可设$A(\lambda)$相抵于对角阵
\begin{align}
B(\lambda)=\mathrm{diag}\{d_1'(\lambda),d_2'(\lambda),\cdots,d_r'(\lambda);0,\cdots,0\},
\end{align}
其中$d_i'(\lambda)$是非零首一多项式且$d_i'(\lambda)\mid d_{i + 1}'(\lambda)$ ($i = 1,2,\cdots,r - 1$)。再由\refpro{proposition:相抵标准型的行列式因子和不变因子}可知,$B(\lambda)$的不变因子为 $d_1'(\lambda),$ $d_2'(\lambda),$ $\cdots,$ $d_r'(\lambda)$.由\refthe{theorem:相抵的lamda矩阵有相同的行列式因子和不变因子}可知,$A(\lambda)$和$B(\lambda)$的不变因子相同,故根据$d_i(\lambda),d'_i(\lambda)$的整除关系,我们就有
\begin{align*}
&d_1\left( \lambda \right) =d_1'(\lambda ),
\\
&d_2\left( \lambda \right) =d_2' (\lambda ),
\\
&\cdots \cdots \cdots \cdots 
\\
&d_r\left( \lambda \right) =d_{r}^{\prime}\left( \lambda \right) .
\end{align*}
因此$A(\lambda)$ 的相抵于对角阵
\begin{align*}
\varLambda=\mathrm{diag}\{d_1(\lambda),d_2(\lambda),\cdots,d_r(\lambda);0,\cdots,0\},
\end{align*}
其中 $d_i(\lambda)$ 是非零首一多项式且 $d_i(\lambda)$ $\mid$ $ d_{i + 1}(\lambda)$ ($i$ $=$ $1,$ $2,$ $\cdots,$ $r - 1$).上式也就是$A(\lambda)$ 的法式.故$A(\lambda)$的不变因子可以唯一确定其法式.
\end{proof}

\begin{corollary}\label{corollary:相抵当且仅当它们有相同的法式}
设 $A(\lambda), B(\lambda)$ 为 $n$ 阶 $\lambda$-矩阵,则 $A(\lambda)$ 与 $B(\lambda)$ 相抵当且仅当它们有相同的法式。
\end{corollary}
\begin{proof}
若 $A(\lambda)$ 与 $B(\lambda)$ 有相同的法式,显然它们相抵。若 $A(\lambda)$ 与 $B(\lambda)$ 相抵,由\refthe{theorem:相抵的lamda矩阵有相同的行列式因子和不变因子} 知 $A(\lambda)$ 与 $B(\lambda)$ 有相同的不变因子,从而由\refcor{corollary:不变因子和法式相互唯一确定}可知, $A(\lambda)$ 与 $B(\lambda)$ 有相同的法式。
\end{proof}

\begin{corollary}\label{corollary:矩阵法式与初等变换的选取无关}
$n$ 阶 $\lambda$-矩阵 $A(\lambda)$ 的法式与初等变换的选取无关。
\end{corollary}
\begin{proof}
设 $\varLambda_1, \varLambda_2$ 是 $A(\lambda)$ 通过不同的初等变换得到的两个法式,则 $\varLambda_1$ 与 $\varLambda_2$ 相抵,由\refcor{corollary:相抵当且仅当它们有相同的法式}可得 $\varLambda_1 = \varLambda_2$。
\end{proof}

\begin{theorem}\label{theorem:矩阵相似关于不变因子和行列式因子的充要条件}
数域 $\mathbb{K}$ 上 $n$ 阶矩阵 $A$ 与 $B$ 相似的充分必要条件是它们的特征矩阵 $\lambda I - A$ 与 $\lambda I - B$ 具有相同的行列式因子或不变因子。
\end{theorem}
\begin{proof}
显然不变因子与行列式因子之间相互唯一确定。再由\refthe{theorem:矩阵相似的充要条件是对应的lamda-矩阵相抵}、\refcor{corollary:相抵当且仅当它们有相同的法式} 及\refcor{corollary:不变因子和法式相互唯一确定}即得结论。
\end{proof}

\begin{corollary}\label{corollary:矩阵的相似关系在基域扩张下不变}
设 $\mathbb{F}\subseteq\mathbb{K}$ 是两个数域,$A, B$ 是 $\mathbb{F}$ 上的两个矩阵,则 $A$ 与 $B$ 在 $\mathbb{F}$ 上相似的充分必要条件是它们在 $\mathbb{K}$ 上相似。
\end{corollary}
\begin{note}
这个推论告诉我们:\textbf{矩阵的相似关系在基域扩张下不变}。事实上,这个推论的证明过程也说明:\textbf{矩阵的不变因子在基域扩张下也不变}。 此即即\textbf{矩阵的相似关系与数域无关.}
\end{note}
\begin{proof}
若 $A$ 与 $B$ 在 $\mathbb{F}$ 上相似,由于 $\mathbb{F}\subseteq\mathbb{K}$,它们当然在 $\mathbb{K}$ 上也相似。反之,若 $A$ 与 $B$ 在 $\mathbb{K}$ 上相似,则 $\lambda I - A$ 与 $\lambda I - B$ 在 $\mathbb{K}$ 上有相同的不变因子,也就是说它们有相同的法式。由\refcor{corollary:矩阵法式与初等变换的选取无关}可知,求法式与初等变换的选取无关。注意到 $\lambda I - A$ 与 $\lambda I - B$ 是数域 $\mathbb{F}$ 上的 $\lambda$-矩阵,故可用 $\mathbb{F}$ 上 $\lambda$-矩阵的初等变换就能将它们变成法式,其中只涉及 $\mathbb{F}$ 中数的加、减、乘、除运算以及 $\mathbb{F}$ 上的多项式的加、减、乘、数乘运算,最后得到法式中的不变因子 $d_i(\lambda)$ 仍是 $\mathbb{F}$ 上的多项式。这就是说存在 $\mathbb{F}$ 上的可逆 $\lambda$-矩阵 $P(\lambda), Q(\lambda), M(\lambda), N(\lambda)$,使
\begin{align*}
P(\lambda)(\lambda I - A)Q(\lambda)=M(\lambda)(\lambda I - B)N(\lambda)=\mathrm{diag}\{d_1(\lambda),\cdots,d_n(\lambda)\},
\end{align*}
从而
\begin{align*}
M(\lambda)^{-1}P(\lambda)(\lambda I - A)Q(\lambda)N(\lambda)^{-1}=\lambda I - B,
\end{align*}
即 $\lambda I - A$ 与 $\lambda I - B$ 在 $\mathbb{F}$ 上相抵,由\refthe{theorem:矩阵相似的充要条件是对应的lamda-矩阵相抵} 可得 $A$ 与 $B$ 在 $\mathbb{F}$ 上相似。
\end{proof}




\end{document}
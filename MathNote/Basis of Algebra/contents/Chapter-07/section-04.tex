\documentclass[../../main.tex]{subfiles}
\graphicspath{{\subfix{../../image/}}} % 指定图片目录,后续可以直接使用图片文件名。

% 例如:
% \begin{figure}[h]
% \centering
% \includegraphics{image-01.01}
% \caption{图片标题}
% \label{fig:image-01.01}
% \end{figure}
% 注意:上述\label{}一定要放在\caption{}之后,否则引用图片序号会只会显示??.

\begin{document}

\section{有理标准型}

\begin{proposition}\label{proposition:特征矩阵的法式和不变因子}
矩阵 $A$ 的特征矩阵 $\lambda I - A$ 的法式为
\begin{align*}
\mathrm{diag}\{1,\cdots,1,d_1(\lambda),\cdots,d_k(\lambda)\},
\end{align*}
其中 $d_i(\lambda)$ 为非常数首一多项式且 $d_i(\lambda)\mid d_{i + 1}(\lambda)\ (i = 1, 2, \cdots, k - 1)$,则 $A$ 的不变因子就是
\[
1,\cdots,1,d_1(\lambda),\cdots,d_k(\lambda).
\] 
\end{proposition}
\begin{proof}
由\refcorollary{corollary:特征矩阵的法式}可知,矩阵 $A$ 的特征矩阵 $\lambda I - A$ 的法式为
\begin{align*}
\mathrm{diag}\{1,\cdots,1,d_1(\lambda),\cdots,d_k(\lambda)\},
\end{align*}
其中 $d_i(\lambda)$ 为非常数首一多项式且 $d_i(\lambda)\mid d_{i + 1}(\lambda)\ (i = 1, 2, \cdots, k - 1)$,则根据不变因子的定义可知 ,$A$ 的不变因子就是
\[
1,\cdots,1,d_1(\lambda),\cdots,d_k(\lambda).
\] 
\end{proof}


\begin{lemma}\label{lemma:Frobenius标准型矩阵的极小多项式和不变因子}
设 $r$ 阶矩阵
\[
F = \begin{pmatrix}
0 & 1 & 0 & \cdots & 0 \\
0 & 0 & 1 & \cdots & 0 \\
\vdots & \vdots & \vdots & & \vdots \\
0 & 0 & 0 & \cdots & 1 \\
-a_r & -a_{r - 1} & -a_{r - 2} & \cdots & -a_1
\end{pmatrix},
\]
则

(1) $F$ 的行列式因子为
\begin{align}
1,\cdots,1,f(\lambda),\label{(7.4.1)}
\end{align}
其中共有 $r - 1$ 个 $1$,$\,f(\lambda)=\lambda^r + a_1\lambda^{r - 1}+\cdots + a_r$,$\,F$ 的不变因子也由\eqref{(7.4.1)}式给出,$F$ 的不变因子分别为:
\begin{align*}
1,\cdots,1,f(\lambda).
\end{align*}
进而,$\lambda I-F$相抵于$\mathrm{diag}\{1,\cdots,1,f(\lambda)\}.$

(2) $F$ 的极小多项式等于 $f(\lambda)$。
\end{lemma}
\begin{proof}
(1) $F$ 的 $r$ 阶行列式因子就是它的特征多项式,由\refproposition{proposition:多项式的友阵的特征多项式与特征值}得
\[
|\lambda I - F| = \lambda^r + a_1\lambda^{r - 1}+\cdots + a_r.
\]
对任一 $k < r$,$\lambda I - F$ 总有一个 $k$ 阶子式其值等于 $(-1)^k$,故 $D_k(\lambda)=1$。又由\refcorollary{corollary:不变因子和法式相互唯一确定}可知,$\lambda I-F$的法式为$\mathrm{diag}\{1,\cdots,1,f(\lambda)\}.$故$\lambda I-F$相抵于$\mathrm{diag}\{1,\cdots,1,f(\lambda)\}.$

(2) 因为 $F$ 的特征多项式为 $f(\lambda)$,所以 $F$ 适合多项式 $f(\lambda)$。设 $e_i\ (i = 1, 2, \cdots, r)$ 是 $r$ 维标准单位行向量,则不难算出:
\[
e_1F = e_2,\quad e_1F^2 =e_2F= e_3,\quad \cdots,\quad e_1F^{r - 1} =e_{r-1}F= e_r.
\]
显然,$e_1, e_1F,\cdots, e_1F^{r - 1}$ 是一组线性无关的向量,从而任取$g(x)\in P_{r-1}[x]$且$g(x)$非零,则存在一组不全为零的数$a_1$,$a_2$,$\cdots$, $a_r$,使得
\begin{align*}
g(x)=a_1x^{r-1}+a_2x^{r-2}+\cdots+a_r.
\end{align*}
于是将$F$代入上式,再在等式两边同乘$e_1$得到
\begin{align*}
e_1g(F)=a_1e_1F^{r-1}+a_2e_1F^{r-2}+\cdots +a_re_1F.
\end{align*}
又因为$e_1, e_1F,\cdots, e_1F^{r - 1}$ 是一组线性无关的向量,且$a_1$,$a_2$,$\cdots$, $a_r$不全为零,所以$e_1g(F)\ne 0.$即$g(F)$的第一行不为零,故$g(F)\ne O$.
因此 $F$ 不可能适合一个次数不超过 $r - 1$ 的非零多项式,从而 $F$ 的极小多项式就是 $f(\lambda)$。
\end{proof}

\begin{lemma}\label{lemma:对角元素的置换不改变相抵性}
设 $\lambda$-矩阵 $A(\lambda)$ 相抵于对角 $\lambda$-矩阵
\begin{align}
\mathrm{diag}\{d_1(\lambda),d_2(\lambda),\cdots,d_n(\lambda)\},
\label{eq:::7.4.2}
\end{align}
$\lambda$-矩阵 $B(\lambda)$ 相抵于对角 $\lambda$-矩阵
\begin{align}
\mathrm{diag}\{d_1'(\lambda),d_2'(\lambda),\cdots,d_n'(\lambda)\},
\label{eq:::7.4.3}
\end{align}
且 $d_1'(\lambda),d_2'(\lambda),\cdots,d_n'(\lambda)$ 是 $d_1(\lambda),d_2(\lambda),\cdots,d_n(\lambda)$ 的一个置换(即若不计次序,这两组多项式完全相同),则 $A(\lambda)$ 相抵于 $B(\lambda)$。
\end{lemma}
\begin{proof}
利用初等行对换及初等列对换即可将\eqref{eq:::7.4.2}式变成\eqref{eq:::7.4.3}式,因此\eqref{eq:::7.4.2} 式所示的矩阵与 \eqref{eq:::7.4.3}式所示的矩阵相抵,从而 $A(\lambda)$ 与 $B(\lambda)$ 相抵。
\end{proof}

\begin{theorem}
设 $A$ 是数域 $\mathbb{K}$ 上的 $n$ 阶方阵,$A$ 的不变因子组为
\[
1,\cdots,1,d_1(\lambda),\cdots,d_k(\lambda),
\]
其中 $\deg d_i(\lambda)=m_i\geq1$,则 $A$ 相似于下列分块对角阵:
\begin{align}\label{equation-矩阵的有理标准型}
F = \begin{pmatrix}
F_1 & & & \\
& F_2 & & \\
& & \ddots & \\
& & & F_k
\end{pmatrix},
\end{align}
其中 $F_i$ 的阶等于 $m_i$,且 $F_i$ 是形如\reflemma{lemma:Frobenius标准型矩阵的极小多项式和不变因子} 中的矩阵,$F_i$ 的最后一行由 $d_i(\lambda)$ 的系数(除首项系数之外)的负值组成。此即,设$d_i=\lambda ^{m_i}+a_{1i}\lambda ^{m_i-1}+\cdots +a_{m_i,i}$,则
\begin{align*}
F=\left( \begin{matrix}
0&		1&		0&		\cdots&		0\\
0&		0&		1&		\cdots&		0\\
\vdots&		\vdots&		\vdots&		&		\vdots\\
0&		0&		0&		\cdots&		1\\
-a_{m_i,i}&		-a_{m_i-1,i}&		-a_{m_i-2,i}&		\cdots&		-a_{1i}\\
\end{matrix} \right) .
\end{align*}
\eqref{equation-矩阵的有理标准型}式称为矩阵 $A$ 的\textbf{有理标准型}或 \textbf{Frobenius 标准型},每个 $F_i$ 称为 \textbf{Frobenius 块}。 
\end{theorem}
\begin{proof}
注意到 $\lambda I - A$ 的第 $n$ 个行列式因子就是 $A$ 的特征多项式 $|\lambda I - A|$,再由不变因子的定义可知:
\begin{align*}
|\lambda I - A| = d_1(\lambda)d_2(\lambda)\cdots d_k(\lambda).
\end{align*}
而 $|\lambda I - A|$ 是一个 $n$ 次多项式,因此 $m_1 + m_2 + \cdots + m_k = n$。一方面,$\lambda I - A$ 的法式为
\[
\mathrm{diag}\{1,\cdots,1,d_1(\lambda),d_2(\lambda),\cdots,d_k(\lambda)\},
\]
其中有 $n - k$ 个 $1$。另一方面,对 $\lambda I - F$ 的每个分块都施以 $\lambda$-矩阵的初等变换,由\reflemma{lemma:Frobenius标准型矩阵的极小多项式和不变因子}可知,$\lambda I - F$ 相抵于如下对角阵:
\begin{align}
\mathrm{diag}\{1,\cdots,1,d_1(\lambda);1,\cdots,1,d_2(\lambda);\cdots;1,\cdots,1,d_k(\lambda)\},
\label{equation::7.4.5}
\end{align}
其中每个 $d_i(\lambda)$ 前各有 $m_i - 1$ 个 $1$,从而共有 $\sum_{i = 1}^{k}(m_i - 1)=n - k$ 个 $1$。因此 \eqref{equation::7.4.5} 式所示的矩阵与 $\lambda I - A$ 的法式只相差主对角线上元素的置换,由\reflemma{lemma:对角元素的置换不改变相抵性}可得 $\lambda I - A$ 与 $\lambda I - F$ 相抵,从而 $A$ 与 $F$ 相似。
\end{proof}

\begin{example}
设 6 阶矩阵 $A$ 的不变因子为
\[
1,1,1,\lambda - 1,(\lambda - 1)^2,(\lambda - 1)^2(\lambda + 1),
\]
则 $A$ 的有理标准型为
\[
\begin{pmatrix}
1 & & & & & \\
& 0 & 1 & & & \\
& -1 & 2 & & & \\
& & & 0 & 1 & 0 \\
& & & 0 & 0 & 1 \\
& & & -1 & 1 & 1
\end{pmatrix}.
\] 
\end{example}

\begin{theorem}\label{theorem:极小多项式与不变因子的关系}
设数域 $\mathbb{K}$ 上的 $n$ 阶矩阵 $A$ 的不变因子为
\[
1,\cdots,1,d_1(\lambda),\cdots,d_k(\lambda),
\]
其中 $d_i(\lambda)\mid d_{i + 1}(\lambda)\ (i = 1,\cdots,k - 1)$,则 $A$ 的极小多项式 $m(\lambda)=d_k(\lambda)$。
\end{theorem}
\begin{proof}
设 $A$ 的有理标准型为
\begin{align*}
F = \begin{pmatrix}
F_1 & & & \\
& F_2 & & \\
& & \ddots & \\
& & & F_k
\end{pmatrix}.
\end{align*}
因为相似矩阵有相同的极小多项式,故只需证明 $F$ 的极小多项式是 $d_k(\lambda)$ 即可。但 $F$ 是分块对角阵,由\hyperref[proposition:极小多项式的性质]{极小多项式的性质(6)}知 $F$ 的极小多项式是诸 $F_i$ 极小多项式的最小公倍式。又由\reflemma{lemma:Frobenius标准型矩阵的极小多项式和不变因子} 知 $F_i$ 的极小多项式为 $d_i(\lambda)$。因为 $d_i(\lambda)\mid d_{i + 1}(\lambda)$,故诸 $d_i(\lambda)$ 的最小公倍式等于 $d_k(\lambda)$。
\end{proof}

\begin{example}
下面两个 4 阶矩阵
\[
A = \begin{pmatrix}
0 & 0 & 0 & 0 \\
0 & 0 & 0 & 0 \\
0 & 0 & 0 & 1 \\
0 & 0 & 0 & 0
\end{pmatrix}, \quad
B = \begin{pmatrix}
0 & 1 & 0 & 0 \\
0 & 0 & 0 & 0 \\
0 & 0 & 0 & 1 \\
0 & 0 & 0 & 0
\end{pmatrix}
\]
的不变因子分别为 $A$:$1,\lambda,\lambda,\lambda^2$ 和 $B$:$1,1,\lambda^2,\lambda^2$。它们的特征多项式和极小多项式分别相等,但它们不相似。 
\end{example}






\end{document}
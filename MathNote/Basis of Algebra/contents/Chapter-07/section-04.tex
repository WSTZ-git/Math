\documentclass[../../main.tex]{subfiles}
\graphicspath{{\subfix{../../image/}}} % 指定图片目录,后续可以直接使用图片文件名。

% 例如:
% \begin{figure}[h]
% \centering
% \includegraphics{image-01.01}
% \caption{图片标题}
% \label{fig:image-01.01}
% \end{figure}
% 注意:上述\label{}一定要放在\caption{}之后,否则引用图片序号会只会显示??.

\begin{document}

\section{有理标准型}\label{section:有理标准型}

\begin{proposition}\label{proposition:特征矩阵的法式和不变因子}
设矩阵 $A$ 的特征矩阵 $\lambda I - A$ 的法式为
\begin{align*}
\mathrm{diag}\{1,\cdots,1,d_1(\lambda),\cdots,d_k(\lambda)\},
\end{align*}
其中 $d_i(\lambda)$ 为非常数首一多项式且 $d_i(\lambda)\mid d_{i + 1}(\lambda)\ (i = 1, 2, \cdots, k - 1)$,则 $A$ 的不变因子就是
\[
1,\cdots,1,d_1(\lambda),\cdots,d_k(\lambda).
\] 
\end{proposition}
\begin{proof}
由\refcor{corollary:特征矩阵的法式}可知,矩阵 $A$ 的特征矩阵 $\lambda I - A$ 的法式为
\begin{align*}
\mathrm{diag}\{1,\cdots,1,d_1(\lambda),\cdots,d_k(\lambda)\},
\end{align*}
其中 $d_i(\lambda)$ 为非常数首一多项式且 $d_i(\lambda)\mid d_{i + 1}(\lambda)\ (i = 1, 2, \cdots, k - 1)$,则根据不变因子的定义可知 ,$A$ 的不变因子就是
\[
1,\cdots,1,d_1(\lambda),\cdots,d_k(\lambda).
\] 
\end{proof}

\begin{lemma}[Frobenius块的基本性质]\label{lemma:Frobenius标准型矩阵的极小多项式和不变因子}
设 $r$ 阶矩阵
\[
F=F(f(\lambda)) = \begin{pmatrix}
0 & 1 & 0 & \cdots & 0 \\
0 & 0 & 1 & \cdots & 0 \\
\vdots & \vdots & \vdots & & \vdots \\
0 & 0 & 0 & \cdots & 1 \\
-a_r & -a_{r - 1} & -a_{r - 2} & \cdots & -a_1
\end{pmatrix},
\]
其中$\,f(\lambda)=\lambda^r + a_1\lambda^{r - 1}+\cdots + a_r$,则
\begin{enumerate}[(1)]
\item $|F(f(\lambda))|=(-1)^{r+2}a_r=(-1)^{r+2}f(0).$

\item $F=F(f(\lambda))$的特征多项式$|\lambda I-F|$为$f(\lambda)$.

\item $F$ 的行列式因子为
\begin{align}
1,\cdots,1,f(\lambda),\label{(7.4.1)}
\end{align}
其中共有 $r - 1$ 个 $1$,$\,f(\lambda)=\lambda^r + a_1\lambda^{r - 1}+\cdots + a_r$,$\,F$ 的不变因子也由\eqref{(7.4.1)}式给出,$F$ 的不变因子分别为:
\begin{align*}
1,\cdots,1,f(\lambda).
\end{align*}
进而,$\lambda I-F$相抵于$\mathrm{diag}\{1,\cdots,1,f(\lambda)\}.$

\item $F$ 的极小多项式等于 $f(\lambda)$。
\end{enumerate}
\end{lemma}
\begin{proof}
\begin{enumerate}[(1)]
\item 注意到$f(0)=a_r$,于是就有
\begin{align*}
\left| F(f(\lambda )) \right|=\left| \begin{matrix}
0&		1&		0&		\cdots&		0\\
0&		0&		1&		\cdots&		0\\
\vdots&		\vdots&		\vdots&		&		\vdots\\
0&		0&		0&		\cdots&		1\\
-a_r&		-a_{r-1}&		-a_{r-2}&		\cdots&		-a_1\\
\end{matrix} \right|\xlongequal{\text{按第一列展开}}\left( -1 \right) ^{r+2}a_r=\left( -1 \right) ^{r+2}f\left( 0 \right) .
\end{align*}

\item 由\hyperref[proposition:多项式的友阵的特征多项式与特征值]{命题\ref{proposition:多项式的友阵的特征多项式与特征值}(1)}同理可得得
\[
|\lambda I - F| = \lambda^r + a_1\lambda^{r - 1}+\cdots + a_r=f(\lambda).
\]

\item $F$ 的 $r$ 阶行列式因子就是它的特征多项式,由\hyperref[proposition:多项式的友阵的特征多项式与特征值]{命题\ref{proposition:多项式的友阵的特征多项式与特征值}(1)}同理可得得
\[
|\lambda I - F| = \lambda^r + a_1\lambda^{r - 1}+\cdots + a_r=f(\lambda).
\]
对任一 $k < r$,$\lambda I - F$ 总有一个 $k$ 阶子式其值等于 $(-1)^k$,故 $D_k(\lambda)=1$。又由\refcor{corollary:不变因子和法式相互唯一确定}可知,$\lambda I-F$的法式为$\mathrm{diag}\{1,\cdots,1,f(\lambda)\}.$故$\lambda I-F$相抵于$\mathrm{diag}\{1,\cdots,1,f(\lambda)\}.$

\item 因为 $F$ 的特征多项式为 $f(\lambda)$,所以 $F$ 适合多项式 $f(\lambda)$。设 $e_i\ (i = 1, 2, \cdots, r)$ 是 $r$ 维标准单位行向量,则不难算出:
\[
e_1F = e_2,\quad e_1F^2 =e_2F= e_3,\quad \cdots,\quad e_1F^{r - 1} =e_{r-1}F= e_r.
\]
显然,$e_1, e_1F,\cdots, e_1F^{r - 1}$ 是一组线性无关的向量,从而任取$g(x)\in P_{r-1}[x]$且$g(x)$非零,则存在一组不全为零的数$a_1$,$a_2$,$\cdots$, $a_r$,使得
\begin{align*}
g(x)=a_1x^{r-1}+a_2x^{r-2}+\cdots+a_r.
\end{align*}
于是将$F$代入上式,再在等式两边同乘$e_1$得到
\begin{align*}
e_1g(F)=a_1e_1F^{r-1}+a_2e_1F^{r-2}+\cdots +a_re_1F.
\end{align*}
又因为$e_1, e_1F,\cdots, e_1F^{r - 1}$ 是一组线性无关的向量,且$a_1$,$a_2$,$\cdots$, $a_r$不全为零,所以$e_1g(F)\ne 0.$即$g(F)$的第一行不为零,故$g(F)\ne O$.
因此 $F$ 不可能适合一个次数不超过 $r - 1$ 的非零多项式,从而 $F$ 的极小多项式就是 $f(\lambda)$。
\end{enumerate}
\end{proof}

\begin{lemma}\label{lemma:对角元素的置换不改变相抵性}
设 $\lambda$-矩阵 $A(\lambda)$ 相抵于对角 $\lambda$-矩阵
\begin{align}
\mathrm{diag}\{d_1(\lambda),d_2(\lambda),\cdots,d_n(\lambda)\},
\label{eq:::7.4.2}
\end{align}
$\lambda$-矩阵 $B(\lambda)$ 相抵于对角 $\lambda$-矩阵
\begin{align}
\mathrm{diag}\{d_1'(\lambda),d_2'(\lambda),\cdots,d_n'(\lambda)\},
\label{eq:::7.4.3}
\end{align}
且 $d_1'(\lambda),d_2'(\lambda),\cdots,d_n'(\lambda)$ 是 $d_1(\lambda),d_2(\lambda),\cdots,d_n(\lambda)$ 的一个置换(即若不计次序,这两组多项式完全相同),则 $A(\lambda)$ 相抵于 $B(\lambda)$。
\end{lemma}
\begin{proof}
利用初等行对换及初等列对换即可将\eqref{eq:::7.4.2}式变成\eqref{eq:::7.4.3}式,因此\eqref{eq:::7.4.2} 式所示的矩阵与 \eqref{eq:::7.4.3}式所示的矩阵相抵,从而 $A(\lambda)$ 与 $B(\lambda)$ 相抵。
\end{proof}

\begin{theorem}[有理标准型/Frobenius标准型]\label{theorem:有理标准型核心定理}
设 $A$ 是数域 $\mathbb{K}$ 上的 $n$ 阶方阵,$A$ 的不变因子组为
\[
1,\cdots,1,d_1(\lambda),\cdots,d_k(\lambda),
\]
其中 $\deg d_i(\lambda)=m_i\geq1$,则 $A$ 相似于下列分块对角阵:
\begin{align}\label{equation-矩阵的有理标准型}
F = \begin{pmatrix}
F_1 & & & \\
& F_2 & & \\
& & \ddots & \\
& & & F_k
\end{pmatrix},
\end{align}
其中 $F_i$ 的阶等于 $m_i$,且 $F_i$ 是形如\reflem{lemma:Frobenius标准型矩阵的极小多项式和不变因子} 中的矩阵,$F_i$ 的最后一行由 $d_i(\lambda)$ 的系数(除首项系数之外)的负值组成。此即,设$d_i=\lambda ^{m_i}+a_{1i}\lambda ^{m_i-1}+\cdots +a_{m_i,i}$,则
\begin{align*}
F_i=F(d_i(\lambda))=\left( \begin{matrix}
0&		1&		0&		\cdots&		0\\
0&		0&		1&		\cdots&		0\\
\vdots&		\vdots&		\vdots&		&		\vdots\\
0&		0&		0&		\cdots&		1\\
-a_{m_i,i}&		-a_{m_i-1,i}&		-a_{m_i-2,i}&		\cdots&		-a_{1i}\\
\end{matrix} \right) .
\end{align*}
\eqref{equation-矩阵的有理标准型}式称为矩阵 $A$ 的\textbf{有理标准型}或 \textbf{Frobenius 标准型},每个 $F_i$ 称为 \textbf{Frobenius 块}。

进而,$A$也相似于下列分块对角阵:
\begin{align*}
C=\left( \begin{matrix}
C_1&		&		&		\\
&		C_2&		&		\\
&		&		\ddots&		\\
&		&		&		C_k\\
\end{matrix} \right) ,
\end{align*}
其中$C_i$的阶等于$m_i$,$C_i$就是上述$F_i$的转置,即
\begin{align*}
C_i=C(d_i(\lambda ))=\left( \begin{matrix}
0&		0&		\cdots&		0&		-a_{m_i,i}\\
1&		0&		\cdots&		0&		-a_{m_i-1,i}\\
0&		1&		\cdots&		0&		-a_{m_i-2,i}\\
\vdots&		\vdots&		&		\vdots&		\vdots\\
0&		0&		\cdots&		1&		-a_{1i}\\
\end{matrix} \right) .
\end{align*}
\end{theorem}
\begin{proof}
注意到 $\lambda I - A$ 的第 $n$ 个行列式因子就是 $A$ 的特征多项式 $|\lambda I - A|$,再由不变因子的定义可知:
\begin{align*}
|\lambda I - A| = d_1(\lambda)d_2(\lambda)\cdots d_k(\lambda).
\end{align*}
而 $|\lambda I - A|$ 是一个 $n$ 次多项式,因此 $m_1 + m_2 + \cdots + m_k = n$。一方面,$\lambda I - A$ 的法式为
\[
\mathrm{diag}\{1,\cdots,1,d_1(\lambda),d_2(\lambda),\cdots,d_k(\lambda)\},
\]
其中有 $n - k$ 个 $1$。另一方面,对 $\lambda I - F$ 的每个分块都施以 $\lambda$-矩阵的初等变换,由\reflem{lemma:Frobenius标准型矩阵的极小多项式和不变因子}可知,$\lambda I - F$ 相抵于如下对角阵:
\begin{align}
\mathrm{diag}\{1,\cdots,1,d_1(\lambda);1,\cdots,1,d_2(\lambda);\cdots;1,\cdots,1,d_k(\lambda)\},
\label{equation::7.4.5}
\end{align}
其中每个 $d_i(\lambda)$ 前各有 $m_i - 1$ 个 $1$,从而共有 $\sum_{i = 1}^{k}(m_i - 1)=n - k$ 个 $1$。因此 \eqref{equation::7.4.5} 式所示的矩阵与 $\lambda I - A$ 的法式只相差主对角线上元素的置换,由\reflem{lemma:对角元素的置换不改变相抵性}可得 $\lambda I - A$ 与 $\lambda I - F$ 相抵,从而 $A$ 与 $F$ 相似。

又因为\hyperref[proposition:lambda-矩阵一定与其转置相似]{矩阵与其自身的转置相似},所以矩阵$A$也相似于$F$的转置,即$C$.
\end{proof}

\begin{example}
设 6 阶矩阵 $A$ 的不变因子为
\[
1,1,1,\lambda - 1,(\lambda - 1)^2,(\lambda - 1)^2(\lambda + 1),
\]
则 $A$ 的有理标准型为
\[
\begin{pmatrix}
1 & & & & & \\
& 0 & 1 & & & \\
& -1 & 2 & & & \\
& & & 0 & 1 & 0 \\
& & & 0 & 0 & 1 \\
& & & -1 & 1 & 1
\end{pmatrix}.
\] 
\end{example}

\begin{theorem}\label{theorem:极小多项式与不变因子的关系}
设数域 $\mathbb{K}$ 上的 $n$ 阶矩阵 $A$ 的不变因子为
\[
1,\cdots,1,d_1(\lambda),\cdots,d_k(\lambda),
\]
其中 $d_i(\lambda)\mid d_{i + 1}(\lambda)\ (i = 1,\cdots,k - 1)$,则 $A$ 的特征多项式为$d_1(\lambda)d_2(\lambda)\cdots d_k(\lambda)$,极小多项式为 $m(\lambda)=d_k(\lambda)$.
\end{theorem}
\begin{proof}
首先证明特征多项式,根据不变因子和行列式因子的定义可知
\begin{align*}
1\cdot 1\cdot d_1(\lambda)d_2(\lambda)\cdots d_k(\lambda)=D_n(\lambda).
\end{align*}
其中$D_n(\lambda)$为$\lambda I_n-A$的$n$阶行列式因子,即$A$的特征多项式$|\lambda I_n-A|$.
因此
\begin{align*}
|\lambda I_n-A|=d_1(\lambda)d_2(\lambda)\cdots d_k(\lambda).
\end{align*}
然后证明极小多项式,设 $A$ 的有理标准型为
\begin{align*}
F = \begin{pmatrix}
F_1 & & & \\
& F_2 & & \\
& & \ddots & \\
& & & F_k
\end{pmatrix}.
\end{align*}
因为相似矩阵有相同的极小多项式,故只需证明 $F$ 的极小多项式是 $d_k(\lambda)$ 即可。但 $F$ 是分块对角阵,由\hyperref[proposition:极小多项式的性质]{极小多项式的性质(6)}知 $F$ 的极小多项式是诸 $F_i$ 极小多项式的最小公倍式。又由\reflem{lemma:Frobenius标准型矩阵的极小多项式和不变因子} 知 $F_i$ 的极小多项式为 $d_i(\lambda)$。因为 $d_i(\lambda)\mid d_{i + 1}(\lambda)$,故诸 $d_i(\lambda)$ 的最小公倍式等于 $d_k(\lambda)$。
\end{proof}

\begin{example}
下面两个 4 阶矩阵
\[
A = \begin{pmatrix}
0 & 0 & 0 & 0 \\
0 & 0 & 0 & 0 \\
0 & 0 & 0 & 1 \\
0 & 0 & 0 & 0
\end{pmatrix}, \quad
B = \begin{pmatrix}
0 & 1 & 0 & 0 \\
0 & 0 & 0 & 0 \\
0 & 0 & 0 & 1 \\
0 & 0 & 0 & 0
\end{pmatrix}
\]
的不变因子分别为 $A$:$1,\lambda,\lambda,\lambda^2$ 和 $B$:$1,1,\lambda^2,\lambda^2$。它们的特征多项式和极小多项式分别相等,但它们不相似。 
\end{example}

\begin{definition}[循环子空间]\label{definition:循环子空间和循环空间}
设 \(V\) 是数域 \(\mathbb{K}\) 上的 \(n\) 维线性空间,\(\varphi\) 是 \(V\) 上的线性变换。设 \(\mathbf{0}\neq\alpha\in V\),则 \(U = L(\alpha,\varphi(\alpha),\varphi^2(\alpha),\cdots)\) 称为 \(V\) 的\textbf{循环子空间},记为 \(U = C(\varphi,\alpha)\),\(\alpha\) 称为 \(U\) 的\textbf{循环向量}。
若 \(U = V\),则称 \(V\) 为\textbf{循环空间}。 
\end{definition}

\begin{theorem}[循环子空间的基本性质]\label{theorem:循环子空间的基本性质}
设 \(V\) 是数域 \(\mathbb{K}\) 上的 \(n\) 维线性空间,\(\varphi\) 是 \(V\) 上的线性变换,\(\mathbf{0}\neq\alpha\in V\), \(U = C(\varphi,\alpha)\) 为循环子空间,则循环子空间 \(U\) 是 \(V\) 的 \(\varphi\)-不变子空间,并且是包含 \(\alpha\) 的最小 \(\varphi\)-不变子空间。
\end{theorem}
\begin{proof}
\(U\) 是 \(V\) 的 \(\varphi\)-不变子空间是显然的.下证\(U\) 是包含 \(\alpha\) 的最小 \(\varphi\)-不变子空间。

设$\alpha \in W$,且$W$为$\varphi$-不变子空间,则由数学归纳法易知
\begin{align*}
\alpha,\varphi^k(\alpha)\in W,\forall k\in \mathbb{N}_1.
\end{align*}
于是
\begin{align*}
U=L(\alpha,\varphi(\alpha),\cdots)\in W.
\end{align*}
\end{proof}

\begin{theorem}\label{theorem:循环子空间的基}
设 \(V\) 是数域 \(\mathbb{K}\) 上的 \(n\) 维线性空间,\(\varphi\) 是 \(V\) 上的线性变换,\(\mathbf{0}\neq\alpha\in V\), \(U = C(\varphi,\alpha)\) 为循环子空间,若 \(\dim U = r\),求证:\(\{\alpha,\varphi(\alpha),\cdots,\varphi^{r - 1}(\alpha)\}\) 是 \(U\) 的一组基。
\end{theorem}
\begin{proof}
设 \(m = \max\{k\in\mathbb{Z}^+\mid\alpha,\varphi(\alpha),\cdots,\varphi^{k - 1}(\alpha)\text{ 线性无关}\}\),则显然$1\leq m\leq r$,故$m$是良定义的.于是由\refpro{proposition:线性无关向量组的命题1}和数学归纳法容易验证:对任意的 \(k\geq m\),\(\varphi^k(\alpha)\) 都是 \(\alpha,\varphi(\alpha),\cdots,\varphi^{m - 1}(\alpha)\) 的线性组合,于是 \(\{\alpha,\varphi(\alpha),\cdots,\varphi^{m - 1}(\alpha)\}\) 是 \(U\) 的一组基,从而 \(m = \dim U = r\)。 
\end{proof}

\begin{theorem}\label{theorem:不变子空间是循环子空间的充要条件}
设 \(U\) 是 \(V\) 的 \(\varphi\)-不变子空间,求证:\(U\) 为循环子空间的充要条件是 \(\varphi|_U\) 在 \(U\) 的某组基下的表示矩阵为某个首一多项式的友阵.
\end{theorem}
\begin{proof}
{\heiti 充分性:}设 \(\varphi|_U\) 在 \(U\) 的一组基 \(\{e_1,e_2,\cdots,e_r\}\) 下的表示矩阵是友阵 \(C(d(\lambda))\),其中 \(d(\lambda)=\lambda^r + a_1\lambda^{r - 1}+\cdots + a_{r - 1}\lambda + a_r\),则由\hyperref[proposition:多项式的友矩和Frobenius块]{友阵的定义}可知 \(\varphi(e_i)=e_{i + 1}(1\leq i\leq r - 1)\),\(\varphi(e_r)=-\sum_{i = 1}^{r}a_{r - i + 1}e_i\)。因此 \(e_i=\varphi^{i - 1}(e_1)(2\leq i\leq r)\),\(U = L(e_1,e_2,\cdots,e_r)=C(\varphi,e_1)\) 为循环子空间。

{\heiti 必要性:}设 \(U = C(\varphi,\alpha)\) 是 \(r\) 维循环子空间,则由\refthe{theorem:循环子空间的基}可知,\(\{\alpha,\varphi(\alpha),\cdots,\varphi^{r - 1}(\alpha)\}\) 是 \(U\) 的一组基。设 
\[
\varphi^r(\alpha)=-a_r\alpha - a_{r - 1}\varphi(\alpha)-\cdots - a_1\varphi^{r - 1}(\alpha)
\]
令 \(d(\lambda)=\lambda^r + a_1\lambda^{r - 1}+\cdots + a_{r - 1}\lambda + a_r\),容易验证:\(\varphi|_U\) 在基 \(\{\alpha,\varphi(\alpha),\cdots,\varphi^{r - 1}(\alpha)\}\) 下的表示矩阵就是友阵 \(C(d(\lambda))\)。
\end{proof}

\begin{theorem}[有理标准型的几何意义]\label{theorem:有理标准型的几何意义}
设 \(V\) 是数域 \(\mathbb{K}\) 上的 \(n\) 维线性空间,\(\varphi\) 是 \(V\) 上的线性变换, 且 \(\varphi\) 的不变因子组是 \(1,\cdots,1,d_1(\lambda),\cdots,d_k(\lambda)\),其中 \(d_i(\lambda)\) 是非常数首一多项式,$d_i(\lambda)$ $\mid$ $d_{i + 1}(\lambda)$ $(1$ $\leq$ $i\leq$  $k - 1)$,则\(V\) 存在一个循环子空间的直和分解:
\begin{align}
V = C(\varphi,\alpha_1)\oplus C(\varphi,\alpha_2)\oplus\cdots\oplus C(\varphi,\alpha_k)\label{equation---:::::7.6}
\end{align}
使得 \(\varphi|_{C(\varphi,\alpha_i)}\) 在基 \(\{\alpha_i,\varphi(\alpha_i),\cdots,\varphi^{r_i - 1}(\alpha_i)\}\) 下的表示矩阵就是友阵 \(C(d_i(\lambda))\),其中 \(r_i = \dim C(\varphi,\alpha_i)\)。
\end{theorem}
\begin{note}
线性变换 \(\varphi\) 的有理标准型诱导的 \(V\) 的上述循环子空间直和分解 \eqref{equation---:::::7.6}就是有理标准型的几何意义。 
\end{note}
\begin{proof}
由\refthe{theorem:有理标准型核心定理}可知,存在 \(V\) 的一组基$\{e_1,e_2,\cdots,e_n\}$,使得 \(\varphi\) 在这组基下的表示矩阵为
\[
C=\mathrm{diag}\{C(d_1(\lambda)),C(d_2(\lambda)),\cdots,C(d_k(\lambda))\}
\]
其中\(\varphi|_{L(e_{i1},\cdots,e_{ir_i})}\) 的表示矩阵就是友阵 \(C(d_i(\lambda))\),\(i = 1,2,\cdots,k\)。再结合\refthe{theorem:不变子空间是循环子空间的充要条件}的讨论可知,\(L(e_{i1},\cdots,e_{ir_i})\) 就是一个循环子空间。于是任取 \(\alpha_i\in L(e_{i1},\cdots,e_{ir_i})\) 作为循环向量,则
\[
C(\varphi,\alpha_i)=L(e_{i1},\cdots,e_{ir_i})=L(\alpha_i,\varphi(\alpha_i),\cdots,\varphi^{r_i - 1}(\alpha_i))
\]
其中 \(\dim C(\varphi,\alpha_i)=r_i\)。

综上可知,此时 \(V\) 存在一个循环子空间的直和分解:
\begin{align*}
V = C(\varphi,\alpha_1)\oplus C(\varphi,\alpha_2)\oplus\cdots\oplus C(\varphi,\alpha_k)
\end{align*}
使得 \(\varphi|_{C(\varphi,\alpha_i)}\) 在基 \(\{\alpha_i,\varphi(\alpha_i),\cdots,\varphi^{r_i - 1}(\alpha_i)\}\) 下的表示矩阵就是友阵 \(C(d_i(\lambda))\),其中 \(r_i = \dim C(\varphi,\alpha_i)\)。
\end{proof}

\begin{theorem}[循环子空间的刻画]\label{theorem:循环子空间的刻画}
设 \(\varphi\) 是数域 \(\mathbb{K}\) 上 \(n\) 维线性空间 \(V\) 上的线性变换,\(\varphi\) 的特征多项式和极小多项式分别为 \(f(\lambda)\) 和 \(m(\lambda)\),证明以下 4 个结论等价:
\begin{enumerate}[(1)]
\item \(\varphi\) 的行列式因子组或不变因子组为 \(1,\cdots,1,f(\lambda)\);

\item \(\varphi\) 的初等因子组为 \(P_1(\lambda)^{r_1},P_2(\lambda)^{r_2},\cdots,P_k(\lambda)^{r_k}\),其中 \(P_i(\lambda)\) 是 \(\mathbb{K}\) 上互异的首一不可约多项式,\(r_i\geq1\),\(1\leq i\leq k\); 

\item \(\varphi\) 的极小多项式 \(m(\lambda)\) 等于特征多项式 \(f(\lambda)\); 

\item \(V\) 是关于线性变换 \(\varphi\) 的循环空间。
\end{enumerate}
\end{theorem}
\begin{proof}
\((1)\Leftrightarrow(2)\):由不变因子和初等因子之间的相互转换即得。

\((1)\Leftrightarrow(3)\):由极小多项式等于最大的不变因子,以及所有不变因子的乘积等于特征多项式即得。 

\((1)\Leftrightarrow(4)\):若 \(V\) 是循环空间,则由\refthe{theorem:不变子空间是循环子空间的充要条件}可知,\(\varphi\) 在某组基下的表示矩阵是友阵 \(C(g(\lambda))\),再由友阵的性质(\reflem{lemma:Frobenius标准型矩阵的极小多项式和不变因子})可知,\(\varphi\) 的行列式因子组和不变因子组均为 \(1,\cdots,1,g(\lambda)=f(\lambda)\)。若 \(\varphi\) 的不变因子组为 \(1,\cdots,1,f(\lambda)\),则由有理标准型的几何意义(\refthe{theorem:有理标准型的几何意义})可知,\(V\) 是循环空间.
\end{proof}





\end{document}
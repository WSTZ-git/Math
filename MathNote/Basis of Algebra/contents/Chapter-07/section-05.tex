\documentclass[../../main.tex]{subfiles}
\graphicspath{{\subfix{../../image/}}} % 指定图片目录,后续可以直接使用图片文件名。

% 例如:
% \begin{figure}[h]
% \centering
% \includegraphics{image-01.01}
% \caption{图片标题}
% \label{fig:image-01.01}
% \end{figure}
% 注意:上述\label{}一定要放在\caption{}之后,否则引用图片序号会只会显示??.

\begin{document}

\section{可对角化的判断(二)}

\begin{proposition}\label{proposition:可对角化的补充}
设\(\varphi\)是\(n\)维复线性空间\(V\)上的线性变换, 求证: \(\varphi\)可对角化的充要条件是对\(\varphi\)的任一特征值\(\lambda_{0}\), 下列条件之一成立:
\begin{enumerate}[(1)]
\item \(V = \mathrm{Ker}(\varphi-\lambda_{0}\boldsymbol{I}_{V})+\mathrm{Im}(\varphi-\lambda_{0}\boldsymbol{I}_{V})\);

\item \(V = \mathrm{Ker}(\varphi-\lambda_{0}\boldsymbol{I}_{V})\oplus\mathrm{Im}(\varphi-\lambda_{0}\boldsymbol{I}_{V})\);

\item \(\mathrm{Ker}(\varphi-\lambda_{0}\boldsymbol{I}_{V})\cap\mathrm{Im}(\varphi-\lambda_{0}\boldsymbol{I}_{V}) = 0\);

\item \(\mathrm{dim}\mathrm{Ker}(\varphi-\lambda_{0}\boldsymbol{I}_{V})=\mathrm{dim}\mathrm{Ker}(\varphi-\lambda_{0}\boldsymbol{I}_{V})^{2}\);

\item \(\mathrm{Ker}(\varphi-\lambda_{0}\boldsymbol{I}_{V})=\mathrm{Ker}(\varphi-\lambda_{0}\boldsymbol{I}_{V})^{2}=\mathrm{Ker}(\varphi-\lambda_{0}\boldsymbol{I}_{V})^{3}=\cdots\);

\item \(\mathrm{r}(\varphi-\lambda_{0}\boldsymbol{I}_{V})=\mathrm{r}((\varphi-\lambda_{0}\boldsymbol{I}_{V})^{2})\);

\item \(\mathrm{Im}(\varphi-\lambda_{0}\boldsymbol{I}_{V})=\mathrm{Im}(\varphi-\lambda_{0}\boldsymbol{I}_{V})^{2}=\mathrm{Im}(\varphi-\lambda_{0}\boldsymbol{I}_{V})^{3}=\cdots\);

\item \(\mathrm{Ker}(\varphi-\lambda_{0}\boldsymbol{I}_{V})\)存在\(\varphi\)-不变补空间, 即存在\(\varphi\)-不变子空间\(U\), 使得\(V = \mathrm{Ker}(\varphi-\lambda_{0}\boldsymbol{I}_{V})\oplus U\);

\item \(\mathrm{Im}(\varphi-\lambda_{0}\boldsymbol{I}_{V})\)存在\(\varphi\)-不变补空间, 即存在\(\varphi\)-不变子空间\(W\), 使得\(V = \mathrm{Im}(\varphi-\lambda_{0}\boldsymbol{I}_{V})\oplus W\).
\end{enumerate}
\end{proposition}
\begin{note}
由例4.36可知条件(1)~(9)是相互等价的, 因此本题的结论由例7.40 (与条件(3)对应) 或例7.41 (与条件(6)对应) 即得(这里的题号对应白皮书上的题号). 事实上, 对充分性而言, 我们还可以从其他条件出发来证明\(\varphi\)可对角化, 下面是\(3\)种证法.
\end{note}
\begin{proof}
{\color{blue}证法一:}对任一特征值\(\lambda_{0}\), 由\(\mathrm{Ker}(\varphi-\lambda_{0}\boldsymbol{I}_{V})=\mathrm{Ker}(\varphi-\lambda_{0}\boldsymbol{I}_{V})^{2}=\cdots=\mathrm{Ker}(\varphi-\lambda_{0}\boldsymbol{I}_{V})^{n}\), 取维数之后可得特征值\(\lambda_{0}\)的几何重数等于代数重数, 从而\(\varphi\)有完全的特征向量系, 于是\(\varphi\)可对角化.
    
{\color{blue}证法二:} 对任一特征值\(\lambda_{0}\), 由\(\mathrm{Ker}(\varphi-\lambda_{0}\boldsymbol{I}_{V})=\mathrm{Ker}(\varphi-\lambda_{0}\boldsymbol{I}_{V})^{2}=\cdots=\mathrm{Ker}(\varphi-\lambda_{0}\boldsymbol{I}_{V})^{n}\)可知, 特征子空间等于根子空间, 再由根子空间的直和分解可知, 全空间等于特征子空间的直和, 从而\(\varphi\)可对角化.

{\color{blue}证法三:} 设\(\varphi\)的全体不同特征值为\(\lambda_{1},\lambda_{2},\cdots,\lambda_{k}\), 特征多项式\(f(\lambda)=(\lambda-\lambda_{1})^{m_{1}}(\lambda-\lambda_{2})^{m_{2}}\cdots(\lambda-\lambda_{k})^{m_{k}}\), 则对任意的\(\boldsymbol{\alpha}\in V\), 由Cayley - Hamilton定理可得
\begin{align*}
(\varphi-\lambda_{1}\boldsymbol{I}_{V})^{m_{1}}(\varphi-\lambda_{2}\boldsymbol{I}_{V})^{m_{2}}\cdots(\varphi-\lambda_{k}\boldsymbol{I}_{V})^{m_{k}}(\boldsymbol{\alpha}) = \boldsymbol{0},
\end{align*}
即有\((\varphi-\lambda_{2}\boldsymbol{I}_{V})^{m_{2}}\cdots(\varphi-\lambda_{k}\boldsymbol{I}_{V})^{m_{k}}(\boldsymbol{\alpha})\in\mathrm{Ker}(\varphi-\lambda_{1}\boldsymbol{I}_{V})^{m_{1}}=\mathrm{Ker}(\varphi-\lambda_{1}\boldsymbol{I}_{V})\), 从而
\begin{align*}
(\varphi-\lambda_{1}\boldsymbol{I}_{V})(\varphi-\lambda_{2}\boldsymbol{I}_{V})^{m_{2}}\cdots(\varphi-\lambda_{k}\boldsymbol{I}_{V})^{m_{k}}(\boldsymbol{\alpha}) = \boldsymbol{0}.
\end{align*}
不断这样做下去, 最终可得对任意的\(\boldsymbol{\alpha}\in V\), 总有
\begin{align*}
(\varphi-\lambda_{1}\boldsymbol{I}_{V})(\varphi-\lambda_{2}\boldsymbol{I}_{V})\cdots(\varphi-\lambda_{k}\boldsymbol{I}_{V})(\boldsymbol{\alpha}) = \boldsymbol{0},
\end{align*}
即\(\varphi\)适合多项式\(g(\lambda)=(\lambda-\lambda_{1})(\lambda-\lambda_{2})\cdots(\lambda-\lambda_{k})\), 从而\(\varphi\)可对角化. 
\end{proof}









\end{document}
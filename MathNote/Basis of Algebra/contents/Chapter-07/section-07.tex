\documentclass[../../main.tex]{subfiles}
\graphicspath{{\subfix{../../image/}}} % 指定图片目录,后续可以直接使用图片文件名。

% 例如:
% \begin{figure}[H]
% \centering
% \includegraphics[scale=0.3]{image-01.01}
% \caption{图片标题}
% \label{figure:image-01.01}
% \end{figure}
% 注意:上述\label{}一定要放在\caption{}之后,否则引用图片序号会只会显示??.

\begin{document}

\section{矩阵相似的全系不变量}

\subsection{矩阵相似的判定准则之一:特征矩阵相抵}

回顾\refthe{theorem:矩阵相似的充要条件是对应的lamda-矩阵相抵}中矩阵相似的充要条件.

\begin{proposition}\label{proposition:对角元素任意排列仍相似}
设 $A,B$ 是数域 $\mathbb{F}$ 上的 $n$ 阶矩阵,$\lambda I_n - A$ 相抵于 $\mathrm{diag}\{f_1(\lambda), f_2(\lambda), \cdots, f_n(\lambda)\}$,$\lambda I_n - B$ 相抵于 $\mathrm{diag}$ $\{ $ $f_{i_1}$ $(\lambda)$, $f_{i_2}$ $(\lambda)$, $\cdots$, $f_{i_n}$ $(\lambda)$ $\}$ ,其中 $f_{i_1}$ $(\lambda)$ , $f_{i_2}$ $(\lambda)$, $\cdots$, $f_{i_n}$ $(\lambda)$ 是 $f_1(\lambda), f_2(\lambda), \cdots, f_n(\lambda)$ 的一个排列。求证:$A$ 与 $B$ 相似。
\end{proposition}
\begin{proof}
对换 $\lambda$-矩阵 $\mathrm{diag}\{f_1(\lambda), f_2(\lambda), \cdots, f_n(\lambda)\}$ 的第 $i,j$ 行,再对换第 $i,j$ 列,可将 $f_i(\lambda)$ 与 $f_j(\lambda)$ 互换位置。由于任一排列都可由若干次对换实现,故 $\mathrm{diag}\{f_1(\lambda), f_2(\lambda), \cdots, f_n(\lambda)\}$ 相抵于 $\mathrm{diag}\{f_{i_1}(\lambda), f_{i_2}(\lambda), \cdots, f_{i_n}(\lambda)\}$,于是 $\lambda I_n - A$ 相抵于 $\lambda I_n - B$,从而 $A$ 与 $B$ 相似。
\end{proof}

\begin{example}
设 $n$ 阶方阵 $A,B,C,D$ 中 $A,C$ 可逆,求证:存在可逆矩阵 $P,Q$,使得 $A = PCQ$,$B = PDQ$ 的充要条件是 $\lambda A - B$ 与 $\lambda C - D$ 相抵。
\end{example}
\begin{proof}
必要性由 $\lambda A - B = P(\lambda C - D)Q$ 即得。下证充分性。

设 $\lambda A - B$ 与 $\lambda C - D$ 相抵,则由 $A,C$ 可逆知,$\lambda I_n - A^{-1}B$ 与 $\lambda I_n - C^{-1}D$ 相抵,于是 $A^{-1}B$ 与 $C^{-1}D$ 相似。设 $Q$ 为可逆矩阵,使得 $A^{-1}B = Q^{-1}(C^{-1}D)Q$,令 $P = AQ^{-1}C^{-1}$,则 $P$ 可逆且 $A = PCQ$,$B = PDQ$。
\end{proof}



\subsection{矩阵相似的判定准则二:有相同的行列式因子组}

回顾\refthe{theorem:矩阵相似关于不变因子和行列式因子的充要条件}中矩阵相似的充要条件和$\lambda$-矩阵的行列式因子\hyperref[lemma:i阶行列式因子整除i+1阶行列式因子]{相关定义和性质}.

\begin{proposition}[矩阵必与其转置相似]\label{proposition:lambda-矩阵一定与其转置相似}
求证:任一 $n$ 阶矩阵 $A$ 都与它的转置 $A'$ 相似。从而$A$和$A'$有完全一样的特征多项式和特征值.
\end{proposition}
\begin{proof}
{\color{blue} 证法一:}
注意到 $(\lambda I_n - A)' = \lambda I_n - A'$,并且行列式的值在转置下不改变,行列式的所有$k(k=1,2,\cdots,n)$阶子式构成的集合在转置下也不改变,故 $\lambda I_n - A$ 和 $\lambda I_n - A'$ 有相同的行列式因子组,从而 $A$ 和 $A'$ 相似。

{\color{blue} 证法二:}不妨设
\[
P^{-1}AP=\begin{pmatrix}
J_{n_1}(a_1) & & & \\
& J_{n_2}(a_2) & & \\
& & \ddots & \\
& & & J_{n_k}(a_k)
\end{pmatrix},
\]
其中\(P\)为\(n\)阶可逆阵。记\(H_i=\begin{pmatrix}
& & & & 1 \\
& & & \begin{turn}{80}$\ddots$\end{turn} & \\
& & \begin{turn}{80}$\ddots$\end{turn} & & \\
& 1 & & & \\
1 & & & &
\end{pmatrix}_{i\times i},i = 1,2,\cdots,k\)。显然\(H_i\)都可逆且\(H_{i}^{-1}=H_i\)。从而
\begin{align*}
A^T&\sim P^TA^T\left( P^{-1} \right) ^T=\left( \begin{matrix}
J_{n_1}^{T}\left( a_1 \right)&		&		&		\\
&		J_{n_2}^{T}\left( a_2 \right)&		&		\\
&		&		\ddots&		\\
&		&		&		J_{n_k}^{T}\left( a_k \right)\\
\end{matrix} \right) 
\\
&\xlongequal{\text{\reflem{lemma:Jordan块的常见分解}}}\left( \begin{matrix}
H_1J_{n_1}\left( a_1 \right) H_1&		&		&		\\
&		H_2J_{n_2}\left( a_2 \right) H_2&		&		\\
&		&		\ddots&		\\
&		&		&		H_kJ_{n_k}\left( a_k \right) H_k\\
\end{matrix} \right) 
\\
&=\left( \begin{matrix}
H_1&		&		&		\\
&		H_2&		&		\\
&		&		\ddots&		\\
&		&		&		H_k\\
\end{matrix} \right) \left( \begin{matrix}
J_{n_1}\left( a_1 \right)&		&		&		\\
&		J_{n_2}\left( a_2 \right)&		&		\\
&		&		\ddots&		\\
&		&		&		J_{n_k}\left( a_k \right)\\
\end{matrix} \right) \left( \begin{matrix}
H_1&		&		&		\\
&		H_2&		&		\\
&		&		\ddots&		\\
&		&		&		H_k\\
\end{matrix} \right) 
\\
&\sim \left( \begin{matrix}
J_{n_1}\left( a_1 \right)&		&		&		\\
&		J_{n_2}\left( a_2 \right)&		&		\\
&		&		\ddots&		\\
&		&		&		J_{n_k}\left( a_k \right)\\
\end{matrix} \right) \sim A.
\end{align*}
\end{proof}

\begin{proposition}\label{proposition:A(a,b)矩阵均相似}
求证:对任意的 $b \neq 0$,$n$ 阶方阵 $\boldsymbol{A}(a,b)$ 均相互相似:
\[
\boldsymbol{A}(a,b) = 
\begin{pmatrix}
a & b & \cdots & b & b \\
 & a & \ddots & \ddots & b \\
 & & \ddots & \ddots & \vdots \\
 & & & a & b \\
 & & & & a
\end{pmatrix}
\]
\end{proposition}
\begin{proof}
只要证明对任意的 $b \neq 0$,$\boldsymbol{A}(a,b)$ 的行列式因子组都一样即可。显然 $D_n(\lambda) = (\lambda - a)^n$。$\lambda \boldsymbol{I}_n - \boldsymbol{A}(a,b)$ 的前 $n - 1$ 行、前 $n - 1$ 列构成的子式,其值为 $(\lambda - a)^{n - 1}$;$\lambda \boldsymbol{I}_n - \boldsymbol{A}(a,b)$ 的前 $n - 1$ 行、后 $n - 1$ 列构成的子式,其值设为 $g(\lambda)$。注意到 $g(a)$ 是 $n - 1$ 阶上三角行列式,主对角元素全为 $-b$,从而 $g(a) = (-b)^{n - 1} \neq 0$。因此 $(\lambda - a)^{n - 1}$ 与 $g(\lambda)$ 没有公共根,故 $((\lambda - a)^{n - 1}, g(\lambda)) = 1$,于是 $D_{n - 1}(\lambda) = 1$,从而 $\boldsymbol{A}(a,b)$ 的行列式因子组为 $1, \cdots, 1, (\lambda - a)^n$,结论得证.
\end{proof}
\begin{remark}
\begin{enumerate}[(1)]
\item 在上(下)三角矩阵(如 Jordan 块)或类上(下)三角矩阵(如友阵或 Frobenius 块)中,若上(下)次对角线上的元素全部非零,可以尝试计算行列式因子组。对一般的矩阵(如数字矩阵),不建议计算行列式因子组,推荐使用 $\lambda$-矩阵的初等变换计算法式,得到不变因子组。


\item 注意到 $\boldsymbol{A}(a,0) = a\boldsymbol{I}_n$ 的行列式因子组为 $D_i(\lambda) = (\lambda - a)^i (1 \leq i \leq n)$。因此,在求相似标准型的过程中,注意千万不能使用摄动法! 
\end{enumerate}
\end{remark}



\subsection{矩阵相似的判定准则三:有相同的不变因子组}

回顾\refthe{theorem:极小多项式与不变因子的关系}可知,所有不变因子的乘积等于特征多项式,整除关系下最大的那个不变因子等于极小多项式。因此,确定特征多项式和极小多项式可帮助确定不变因子组。

\begin{proposition}[同阶幂零阵必相似]\label{proposition:同阶幂零阵必相似}
设 \(A\) 是 \(n\) 阶 \(n\) 次幂零矩阵,即 \(A^n = O\) 但 \(A^{n - 1}\neq O\)。若 \(B\) 也是 \(n\) 阶 \(n\) 次幂零矩阵,求证:\(A\) 相似于 \(B\)。
\end{proposition}
\begin{proof}
显然 \(A\) 的极小多项式为 \(\lambda^n\),故 \(A\) 的不变因子组是 \(1,\cdots,1,\lambda^n\)。同理 \(B\) 的不变因子组也是 \(1,\cdots,1,\lambda^n\),因此 \(A\) 和 \(B\) 相似。
\end{proof}

\begin{proposition}\label{proposition:纯量阵关于不变因子和行列式因子的等价条件}
设 \(A\) 为 \(n\) 阶矩阵,证明以下 3 个结论等价:
\begin{enumerate}[(1)]
\item  \(A = cI_n\),其中 \(c\) 为常数;

\item  \(A\) 的 \(n - 1\) 阶行列式因子是一个 \(n - 1\) 次多项式;

\item \(A\) 的不变因子组中无常数。
\end{enumerate}
\end{proposition}
\begin{proof}
\((1)\Rightarrow(2)\):显然成立。

\((2)\Rightarrow(3)\):由于 \(A\) 的 \(n\) 阶行列式因子 \(D_n(\lambda)\) 是一个 \(n\) 次多项式,故 \(A\) 的最后一个不变因子 \(d_n(\lambda)=D_n(\lambda)/D_{n - 1}(\lambda)\) 是一个一次多项式,设为 \(\lambda - c\)。因为其他不变因子都要整除 \(d_n(\lambda)\),并且所有不变因子的乘积等于 \(n\) 阶行列式因子 \(D_n(\lambda)\),故 \(A\) 的不变因子组只能是 \(\lambda - c,\lambda - c,\cdots,\lambda - c\)。 

\((3)\Rightarrow(1)\):设 \(A\) 的不变因子组为 \(d_1(\lambda),d_2(\lambda),\cdots,d_n(\lambda)\),则 \(\deg d_i(\lambda)\geq1\)。注意到 \(d_1(\lambda)d_2(\lambda)\cdots d_n(\lambda)=D_n(\lambda)\) 的次数为 \(n\),因此$\deg d_i(\lambda)=1$.又\(d_i(\lambda)\mid d_n(\lambda)\),故只能是 \(d_1(\lambda)=d_2(\lambda)=\cdots=d_n(\lambda)=\lambda - c\)。因此 \(A\) 与 \(cI_n\) 有相同不变因子组,从而它们相似,即存在可逆矩阵 \(P\),使得 \(A = P^{-1}(cI_n)P = cI_n\)。另外,也可以利用 \(A\) 的极小多项式等于 \(\lambda - c\) 或 \(A\) 的 Jordan 标准型来证明。
\end{proof}

\begin{proposition}\label{example:特征值全为1矩阵的任意幂次于原矩阵相似}
设 \(n\) 阶矩阵 \(A\) 的特征值全为 1,求证:对任意的正整数 \(k\),\(A^k\) 与 \(A\) 相似。
\end{proposition}
\begin{remark}
{\color{blue}证法一}是用 “三段论法” 和极小多项式来证明的 (当然用行列式因子和几何重数替代也可以); 后面利用\refpro{proposition:矩阵相似的充要条件(关于秩的充要条件)}给出了第二种证法; 而\refpro{proposition:Jordan块的m次方的Jordan标准型} (当 $a = \pm 1$ 时) 给出了第三种证法. 
\end{remark}
\begin{proof}
{\color{blue}证法一:}
由 \(A\) 的特征值全为 1 可知 \(A^k\) 的特征值也全为 1。设 \(P\) 为可逆矩阵,使得 $P^{-1}$ $A$ $P$ $=$ $J$ $=$ $\mathrm{diag}$ $\{J_{r_1}(1)$,$\cdots$,$J_{r_s}(1)\}$ 为 Jordan 标准型。由于 \(P^{-1}A^kP=(P^{-1}AP)^k = J^k\),故只要证明 \(J^k\) 与 \(J\) 相似即可。又因为 \(J^k=\mathrm{diag}\{J_{r_1}(1)^k,\cdots,J_{r_s}(1)^k\}\),故问题可进一步归结到每个 Jordan 块,即只要证明 \(J_{r_i}(1)^k\) 与 \(J_{r_i}(1)\) 相似即可。因此不妨设 \(J = J_n(1)\) 只有一个 Jordan 块,则 \(J = I_n+J_0\),其中 \(J_0 = J_n(0)\) 是特征值为 0 的 \(n\) 阶 Jordan 块。注意到
\begin{align*}
J^k=(I_n + J_0)^k
=I_n + \mathrm{C}_k^1J_0 + \mathrm{C}_k^2J_0^2+\cdots+J_0^k.
\end{align*}
故 \(J^k\) 是一个上三角矩阵,其主对角线上的元素全为 1,上次对角线上的元素全为 \(k\),从而它的特征多项式为 \((\lambda - 1)^n\)。为了确定它的极小多项式,我们可进行如下计算:
\begin{align*}
(J^k - I_n)^{n - 1}=(\mathrm{C}_k^1J_0 + \mathrm{C}_k^2J_0^2+\cdots+J_0^k)^{n - 1}
=k^{n - 1}J_0^{n - 1}\neq O.
\end{align*}
于是 \(J^k\) 的极小多项式为 \((\lambda - 1)^n\),其不变因子组为 \(1,\cdots,1,(\lambda - 1)^n\)。因此 \(J^k\) 与 \(J\) 有相同的不变因子,从而 \(J^k\) 与 \(J\) 相似。

{\color{blue}证法二:}
显然 $A^k$ 的特征值也全为 $1$. 注意到
\begin{align*}
(A^k - I_n)^l = (A - I_n)^l(A^{k - 1} + A^{k - 2} + \cdots + I_n)^l, \quad l \geq 1.
\end{align*}
由于 $A^{k - 1} + A^{k - 2} + \cdots + I_n$ 的特征值全为 $k$, 故为可逆矩阵, 从而 $\mathrm{r}((A^k - I_n)^l) = \mathrm{r}((A - I_n)^l)$ 对任意的正整数 $l$ 都成立. 由\refpro{proposition:矩阵相似的充要条件(关于秩的充要条件)}可知, $A^k$ 与 $A$ 相似. 
\end{proof}

\begin{proposition}\label{example:特征值全为正负1的矩阵的逆与其自身相似}
设 \(n\) 阶矩阵 \(A\) 的特征值全为 1 或 -1,求证:\(A^{-1}\) 与 \(A\) 相似。
\end{proposition}
\begin{remark}
{\color{blue}证法一}是用 “三段论法” 和极小多项式来证明的 (当然用行列式因子和几何重数替代也可以); 后面利用\refpro{proposition:矩阵相似的充要条件(关于秩的充要条件)}给出了第二种证法; 而\refpro{proposition:Jordan块的m次方的Jordan标准型} (当 $a = \pm 1$ 时) 给出了第三种证法. 
\end{remark}
\begin{proof}
{\color{blue}证法一:}
设 \(P\) 为可逆矩阵,使得 \(P^{-1}AP = J=\mathrm{diag}\{J_{r_1}(\lambda_1),\cdots,J_{r_s}(\lambda_s)\}\) 为 Jordan 标准型,其中 \(\lambda_i = \pm1\)。由于 \(P^{-1}A^{-1}P=(P^{-1}AP)^{-1}=J^{-1}\),故只要证明 \(J^{-1}\) 与 \(J\) 相似即可。又因为 \(J^{-1}=\mathrm{diag}\{J_{r_1}(\lambda_1)^{-1},\cdots,J_{r_s}(\lambda_s)^{-1}\}\),故问题可进一步归结到每个 Jordan 块,即只要证明 \(J_{r_i}(\lambda_i)^{-1}\) 与 \(J_{r_i}(\lambda_i)\) 相似即可。因此不妨设 \(J = J_n(\lambda_0)\) 只有一个 Jordan 块,则 \(J = \lambda_0I_n+J_0\),其中 \(\lambda_0 = \pm1\),\(J_0 = J_n(0)\) 是特征值为 0 的 \(n\) 阶 Jordan 块。注意到
\begin{align*}
\lambda_0^nI_n=(\lambda_0I_n)^n-(-J_0)^n
=(\lambda_0I_n + J_0)(\lambda_0^{n - 1}I_n-\lambda_0^{n - 2}J_0+\cdots+(-1)^{n - 1}J_0^{n - 1}).
\end{align*}
以及 \(\lambda_0^{-1}=\lambda_0\),故可得
\begin{align*}
J^{-1}=(\lambda_0I_n + J_0)^{-1}=\lambda_0I_n-\lambda_0^2J_0+\cdots+(-1)^{n - 1}\lambda_0^nJ_0^{n - 1}.
\end{align*}
因此 \(J^{-1}\) 是一个上三角矩阵,其主对角线上的元素全为 \(\lambda_0\),上次对角线上的元素全为 \(-\lambda_0^2\),从而它的特征多项式为 \((\lambda - \lambda_0)^n\)。为了确定它的极小多项式,我们可进行如下计算:
\begin{align*}
(J^{-1}-\lambda_0I)^{n - 1}=(-\lambda_0^2J_0+\cdots+(-1)^{n - 1}\lambda_0^nJ_0^{n - 1})^{n - 1}=(-1)^{n - 1}J_0^{n - 1}\neq O
\end{align*}
于是 \(J^{-1}\) 的极小多项式为 \((\lambda - \lambda_0)^n\),其不变因子组为 \(1,\cdots,1,(\lambda - \lambda_0)^n\)。因此 \(J^{-1}\) 与 \(J\) 有相同的不变因子组,从而 \(J^{-1}\) 与 \(J\) 相似。

{\color{blue}证法二:}
显然 $A^{-1}$ 的特征值也全为 $1$ 或 $-1$. 设 $\lambda_0 = \pm 1$, 则由 $A$ 可逆以及 $(A^{-1} - \lambda_0 I_n)^l = (-\lambda_0)^lA^{-l}(A - \lambda_0 I_n)^l$ 可得 $\mathrm{r}((A^{-1} - \lambda_0 I_n)^l) = \mathrm{r}((A - \lambda_0 I_n)^l)$ 对任意的正整数 $l$ 都成立. 由\refpro{proposition:矩阵相似的充要条件(关于秩的充要条件)}可知, $A^{-1}$ 与 $A$ 相似.
\end{proof}

\subsection{矩阵相似的判定准则四:有相同的初等因子组}

\begin{definition}[准素因子]\label{definition:一般数域上的的准素因子}
设 \(f(\lambda)\) 为数域 \(\mathbb{K}\) 上的多项式,\(p(\lambda)\) 是 \(\mathbb{K}\) 上的首一不可约多项式,若存在正整数 \(k\),使得 \(p(\lambda)^k\mid f(\lambda)\),但 \(p(\lambda)^{k + 1}\nmid f(\lambda)\),则称 \(p(\lambda)^k\) 为 \(f(\lambda)\) 的一个\textbf{准素因子}。所有$f(\lambda)$的准素因子称为$f(\lambda)$的\textbf{准素因子组}.

事实上,若设 \(f(\lambda)\) 在 \(\mathbb{K}\) 上的标准因式分解为
\[
f(\lambda)=cP_1(\lambda)^{e_1}P_2(\lambda)^{e_2}\cdots P_t(\lambda)^{e_t}
\]
其中 \(c\) 为非零常数,\(P_i(\lambda)\) 为互异的首一不可约多项式,\(e_i>0(1\leq i\leq t)\),则 \(f(\lambda)\) 的所有准素因子为 \(P_1(\lambda)^{e_1},P_2(\lambda)^{e_2},\cdots,P_t(\lambda)^{e_t}\)。
\end{definition}

\begin{theorem}[$\lambda$-矩阵和初等因子的基本性质]\label{theorem:lambda-矩阵和初等因子的基本性质}
\begin{enumerate}[(1)]
\item 设 \(f(\lambda),g(\lambda)\) 是数域 \(\mathbb{K}\) 上的首一多项式,\(d(\lambda)=(f(\lambda),g(\lambda))\),\(m(\lambda)=[f(\lambda),g(\lambda)]\) 分别是 \(f(\lambda)\) 和 \(g(\lambda)\) 的最大公因式和最小公倍式,证明下列 \(\lambda\)-矩阵相抵:
\[
\begin{pmatrix}
f(\lambda) & 0 \\
0 & g(\lambda)
\end{pmatrix}, 
\begin{pmatrix}
g(\lambda) & 0 \\
0 & f(\lambda)
\end{pmatrix}, 
\begin{pmatrix}
d(\lambda) & 0 \\
0 & m(\lambda)
\end{pmatrix}
\]

\item 设 \(A\) 是数域 \(\mathbb{K}\) 上的 \(n\) 阶矩阵,其特征矩阵 \(\lambda I_n - A\) 经过初等变换可化为对角矩阵 \(\mathrm{diag}\{f_1(\lambda),f_2(\lambda),\cdots,f_n(\lambda)\}\),其中 \(f_i(\lambda)\) 是 \(\mathbb{K}\) 上的首一多项式。求证:矩阵 \(A\) 的初等因子组等于所有 \(f_i(\lambda)\) 的准素因子组。
\end{enumerate}
\end{theorem}
\begin{note}
由(2)可知,矩阵 \(A\) 的初等因子组就是 \(A\) 的所有不变因子的准素因子组。实际上,(2)就是\reflem{lemma:初等因子组等于准素因子组}的一个推广.
\end{note}
\begin{proof}
\begin{enumerate}[(1)]
\item 由已知,存在多项式 \(u(\lambda),v(\lambda)\),使得 \(f(\lambda)u(\lambda)+g(\lambda)v(\lambda)=d(\lambda)\)。设 \(f(\lambda)=d(\lambda)h(\lambda)\),则 \(m(\lambda)=g(\lambda)h(\lambda)\)。作下列 \(\lambda\)-矩阵的初等变换:
\begin{align*}
\begin{pmatrix}
f(\lambda) & 0 \\
0 & g(\lambda)
\end{pmatrix}&\to
\begin{pmatrix}
f(\lambda) & 0 \\
f(\lambda)u(\lambda) & g(\lambda)
\end{pmatrix}\to
\begin{pmatrix}
f(\lambda) & 0 \\
f(\lambda)u(\lambda)+g(\lambda)v(\lambda) & g(\lambda)
\end{pmatrix}
=
\begin{pmatrix}
f(\lambda) & 0 \\
d(\lambda) & g(\lambda)
\end{pmatrix}
\\
&\to
\begin{pmatrix}
0 & -g(\lambda)h(\lambda) \\
d(\lambda) & g(\lambda)
\end{pmatrix}\to
\begin{pmatrix}
0 & g(\lambda)h(\lambda) \\
d(\lambda) & 0
\end{pmatrix}\to
\begin{pmatrix}
d(\lambda) & 0 \\
0 & m(\lambda)
\end{pmatrix}.
\end{align*}
另一结论同理可得.

\item 对任意的 \(i < j\),以下操作记为 \(O(i,j)\):设 \(d(\lambda)=(f_i(\lambda),f_j(\lambda))\),\(m(\lambda)=[f_i(\lambda),f_j(\lambda)]\) 分别是 \(f_i(\lambda)\) 和 \(f_j(\lambda)\) 的最大公因式和最小公倍式,则用 \(d(\lambda)\) 替代 \(f_i(\lambda)\),用 \(m(\lambda)\) 替代 \(f_j(\lambda)\)。我们先证明,操作 \(O(i,j)\) 可通过 \(\lambda\)-矩阵的初等变换来实现,并且前后两个对角矩阵,即 $\mathrm{diag}$ $\{$ $f_1(\lambda)$,$\cdots$,$f_i(\lambda)$,$\cdots$,$f_j(\lambda)$,$\cdots$,$f_n(\lambda)$ $\}$ 与 $\mathrm{diag}$ $\{$ $f_1(\lambda)$,$\cdots$, $d(\lambda)$, $\cdots$, $m(\lambda)$, $\cdots$, $f_n(\lambda)\}$ 有相同的准素因子组。

由(1)即知 \(O(i,j)\) 是 \(\lambda\)-矩阵的相抵变换。设 \(f_i(\lambda),f_j(\lambda)\) 的公共因式分解为
\[
f_i(\lambda)=P_1(\lambda)^{e_{i1}}P_2(\lambda)^{e_{i2}}\cdots P_t(\lambda)^{e_{it}}, \quad f_j(\lambda)=P_1(\lambda)^{e_{j1}}P_2(\lambda)^{e_{j2}}\cdots P_t(\lambda)^{e_{jt}}
\]
其中 \(P_i(\lambda)\) 为互异的首一不可约多项式,\(e_{ik}\geq0\),\(e_{jk}\geq0(1\leq k\leq t)\),令 \(r_k = \min\{e_{ik},e_{jk}\}\),\(s_k = \max\{e_{ik},e_{jk}\}\),则有
\[
d(\lambda)=P_1(\lambda)^{r_1}P_2(\lambda)^{r_2}\cdots P_t(\lambda)^{r_t}, \quad m(\lambda)=P_1(\lambda)^{s_1}P_2(\lambda)^{s_2}\cdots P_t(\lambda)^{s_t}
\]
显然 \(\{f_i(\lambda),f_j(\lambda)\}\) 和 \(\{d(\lambda),m(\lambda)\}\) 有相同的准素因子组,因此 \(O(i,j)\) 操作前后的两个对角矩阵也有相同的准素因子组。

对对角矩阵 \(\mathrm{diag}\{f_1(\lambda),f_2(\lambda),\cdots,f_n(\lambda)\}\) 依次实施操作 \(O(1,j)(2\leq j\leq n)\),则得到对角矩阵的第 \((1,1)\) 元素的所有不可约因式的幂在主对角元素中都是最小的;然后依次操作 \(O(2,j)(3\leq j\leq n)\);\(\cdots\);最后操作 \(O(n - 1,n)\),可得一个对角矩阵 \(\varLambda=\mathrm{diag}\{d_1(\lambda),d_2(\lambda),\cdots,d_n(\lambda)\}\)。由操作的性质可知,\(\varLambda\) 满足 \(d_i(\lambda)\mid d_{i + 1}(\lambda)(1\leq i\leq n - 1)\),因此 \(\varLambda\) 就是矩阵 \(A\) 的法式。又因为对角矩阵 \(\mathrm{diag}\{f_1(\lambda),f_2(\lambda),\cdots,f_n(\lambda)\}\) 与法式有相同的准素因子组,故所有 \(f_i(\lambda)\) 的准素因子组就是矩阵 \(A\) 的初等因子组。
\end{enumerate}
\end{proof}

\begin{proposition}
设 \(A = \mathrm{diag}\{A_1,A_2,\cdots,A_k\}\) 为分块对角矩阵,求证:\(A\) 的初等因子组等于 \(A_i(1\leq i\leq k)\) 的初等因子组的无交并集。又若交换各块的位置,则所得的矩阵仍和 \(A\) 相似。
\end{proposition}
\begin{proof}
\end{proof}
显然 \(\lambda I - A\) 也是一个分块对角矩阵,用 \(\lambda\)-矩阵的初等变换将每一块化为法式,则由\hyperref[theorem:lambda-矩阵和初等因子的基本性质]{$\lambda$-矩阵和初等因子的基本性质(2)}可知,\(A\) 的初等因子组就是所有各块的初等因子组的无交并集。又交换 \(A\) 的各块并不改变 \(A\) 的初等因子组,因此所得之矩阵仍和 \(A\) 相似。






\end{document}
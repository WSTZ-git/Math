\documentclass[../../main.tex]{subfiles}
\graphicspath{{\subfix{../../image/}}} % 指定图片目录,后续可以直接使用图片文件名。

% 例如:
% \begin{figure}[H]
% \centering
% \includegraphics[scale=0.4]{图.png}
% \caption{}
% \label{figure:图}
% \end{figure}
% 注意:上述\label{}一定要放在\caption{}之后,否则引用图片序号会只会显示??.

\begin{document}

\section{初等因子}

\begin{definition}[初等因子]
设 $d_1(\lambda),d_2(\lambda),\cdots,d_k(\lambda)$ 是数域 $\mathbb{K}$ 上矩阵 $A$ 的非常数不变因子,由于 $d_i(\lambda)\mid d_{i + 1}(\lambda)$,因此可以在 $\mathbb{K}$ 上把 $d_i(\lambda)$ 分解成不可约因式之积:
\begin{gather}\label{equation---7.5.1}
\begin{aligned}
d_1(\lambda)&=p_1(\lambda)^{e_{11}}p_2(\lambda)^{e_{12}}\cdots p_t(\lambda)^{e_{1t}},\\
d_2(\lambda)&=p_1(\lambda)^{e_{21}}p_2(\lambda)^{e_{22}}\cdots p_t(\lambda)^{e_{2t}},\\
&\cdots\cdots\cdots\cdots\\
d_k(\lambda)&=p_1(\lambda)^{e_{k1}}p_2(\lambda)^{e_{k2}}\cdots p_t(\lambda)^{e_{kt}},
\end{aligned}
\end{gather}
其中 $e_{ij}$ 是非负整数 (注意 $e_{ij}$ 可以为零!),并且
\[
e_{1j}\leqslant  e_{2j}\leqslant \cdots\leqslant  e_{kj},\quad j = 1, 2, \cdots, t.
\]
若\eqref{equation---7.5.1}式中的 $e_{ij}>0$,则称 $p_j(\lambda)^{e_{ij}}$ 为 $A$ 的一个\textbf{初等因子},$A$ 的全体初等因子称为 $A$ 的\textbf{初等因子组}。 
\end{definition}

\begin{proposition}\label{proposition:初等因子组与不变因子组相互唯一确定}
矩阵$A$的初等因子组与不变因子组相互唯一确定.
\end{proposition}
\begin{proof}
由因式分解的唯一性可知 $A$ 的初等因子被 $A$ 的不变因子唯一确定。

反过来,若给定一组初等因子 $p_j(\lambda)^{e_{ij}}$,适当增加一些 $1$ (表示为 $p_j(\lambda)^{e_{ij}}$,其中 $e_{ij}=0$),则可将这组初等因子按不可约因式的降幂排列如下:
\begin{align}\label{equation-----7.5.2}
\begin{aligned}
&p_1(\lambda)^{e_{k1}},p_1(\lambda)^{e_{k - 1,1}},\cdots,p_1(\lambda)^{e_{11}},\\
&p_2(\lambda)^{e_{k2}},p_2(\lambda)^{e_{k - 1,2}},\cdots,p_2(\lambda)^{e_{12}},\\
&\quad \quad \cdots\cdots\cdots\cdots\\
&p_t(\lambda)^{e_{kt}},p_t(\lambda)^{e_{k - 1,t}},\cdots,p_t(\lambda)^{e_{1t}},
\end{aligned}
\end{align}
令
\begin{align*}
&d_k(\lambda)=p_1(\lambda)^{e_{k1}}p_2(\lambda)^{e_{k2}}\cdots p_t(\lambda)^{e_{kt}},\\
&d_{k - 1}(\lambda)=p_1(\lambda)^{e_{k - 1,1}}p_2(\lambda)^{e_{k - 1,2}}\cdots p_t(\lambda)^{e_{k - 1,t}},\\
&\quad \quad \cdots\cdots\cdots\cdots\\
&d_1(\lambda)=p_1(\lambda)^{e_{11}}p_2(\lambda)^{e_{12}}\cdots p_t(\lambda)^{e_{1t}},
\end{align*}
则 $d_i(\lambda)\mid d_{i + 1}(\lambda)\ (i = 1,\cdots,k - 1)$,且 $d_1(\lambda),\cdots,d_k(\lambda)$ 的初等因子组就如 \eqref{equation-----7.5.2}所示。因此,给定 $A$ 的初等因子组,我们可唯一地确定 $A$ 的不变因子组。这一事实表明,$A$ 的\textbf{不变因子组与初等因子组在讨论矩阵相似关系中的作用是相同的}。 

\end{proof}


\begin{theorem}\label{theorem:矩阵相似的充分必要条件是它们有相同的初等因子组}
数域 $\mathbb{K}$ 上的两个矩阵 $A$ 与 $B$ 相似的充分必要条件是它们有相同的初等因子组,即矩阵的初等因子组是矩阵相似关系的全系不变量。 
\end{theorem}
\begin{proof}
由\refthe{theorem:矩阵相似关于不变因子和行列式因子的充要条件}可知,矩阵$A$和$B$相似等价于$A$和$B$有相同的不变因子.又由\refpro{proposition:初等因子组与不变因子组相互唯一确定}可知,$A$和$B$有相同的不变因子等价于它们有相同的初等因子组.故矩阵 $A$ 与 $B$ 相似的充分必要条件是它们有相同的初等因子组.

\end{proof}

\begin{example}
设 9 阶矩阵 $A$ 的不变因子组为
\[
1,\cdots,1,(\lambda - 1)(\lambda^2 + 1),(\lambda - 1)^2(\lambda^2 + 1)(\lambda^2 - 2),
\]
试分别在有理数域、实数域和复数域上求 $A$ 的初等因子组。
\end{example}
\begin{solution}
$A$ 在有理数域上的初等因子组为
\[
\lambda - 1,(\lambda - 1)^2,\lambda^2 + 1,\lambda^2 + 1,\lambda^2 - 2.
\]
$A$ 在实数域上的初等因子组为
\[
\lambda - 1,(\lambda - 1)^2,\lambda^2 + 1,\lambda^2 + 1,\lambda + \sqrt{2},\lambda - \sqrt{2}.
\]
$A$ 在复数域上的初等因子组为
\[
\lambda - 1,(\lambda - 1)^2,\lambda + \mathrm{i},\lambda + \mathrm{i},\lambda - \mathrm{i},\lambda - \mathrm{i},\lambda + \sqrt{2},\lambda - \sqrt{2}.
\]

\end{solution}

\begin{example}
设 $A$ 是一个 10 阶矩阵,它的初等因子组为
\[
\lambda - 1,\lambda - 1,(\lambda - 1)^2,(\lambda + 1)^2,(\lambda + 1)^3,\lambda - 2.
\]
求 $A$ 的不变因子组。
\end{example}
\begin{solution}
将上述多项式按不可约因式的降幂排列:
\begin{align*}
\begin{matrix}
(\lambda -1)^2,&		\lambda -1,&		\lambda -1;\\
(\lambda +1)^3,&		(\lambda +1)^2,&		1;\\
\lambda -2,&		1,&		1.\\
\end{matrix}
\end{align*}
于是
\[
d_3(\lambda)=(\lambda - 1)^2(\lambda + 1)^3(\lambda - 2),\quad d_2(\lambda)=(\lambda - 1)(\lambda + 1)^2,\quad d_1(\lambda)=\lambda - 1.
\]
从而 $A$ 的不变因子组为
\[
1,\cdots,1,\lambda - 1,(\lambda - 1)(\lambda + 1)^2,(\lambda - 1)^2(\lambda + 1)^3(\lambda - 2),
\]
其中有 7 个 $1$。 

\end{solution}



















\end{document}
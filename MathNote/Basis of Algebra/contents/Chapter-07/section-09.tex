\documentclass[../../main.tex]{subfiles}
\graphicspath{{\subfix{../../image/}}} % 指定图片目录,后续可以直接使用图片文件名。

% 例如:
% \begin{figure}[H]
% \centering
% \includegraphics{image-01.01}
% \caption{图片标题}
% \label{figure:image-01.01}
% \end{figure}
% 注意:上述\label{}一定要放在\caption{}之后,否则引用图片序号会只会显示??.

\begin{document}

\section{Jordan标准型的进一步讨论和应用}

\begin{theorem}\label{theorem:关于一个特征值的Jordan块的个数等于其度数,阶数直和等于其重数}
线性变换 $\varphi$ 的特征值 $\lambda_1$ 的度数等于 $\varphi$ 的 Jordan 标准型中属于特征值 $\lambda_1$ 的 Jordan 块的个数,$\lambda_1$ 的重数等于所有属于特征值 $\lambda_1$ 的 Jordan 块的阶数之和。 
\end{theorem}
\begin{proof}
设 $V$ 是 $n$ 维复线性空间,$\varphi$ 是 $V$ 上的线性变换。设 $\varphi$ 的初等因子组为
\begin{align}
(\lambda - \lambda_1)^{r_1}, (\lambda - \lambda_2)^{r_2}, \cdots, (\lambda - \lambda_k)^{r_k}, \label{equation--7.7.1}
\end{align}
\refthe{theorem:线性变换一定有一个Jordan矩阵作为其表示矩阵}告诉我们,存在 $V$ 的一组基 $\{e_{11}, e_{12}, \cdots, e_{1r_1}; e_{21}, e_{22}, \cdots, e_{2r_2}; \cdots; e_{k1}, e_{k2}, \cdots, e_{kr_k}\}$,使得 $\varphi$ 在这组基下的表示矩阵为
\begin{align*}
\boldsymbol{J} = 
\begin{pmatrix}
\boldsymbol{J}_1 & & & \\
 & \boldsymbol{J}_2 & & \\
 & & \ddots & \\
 & & & \boldsymbol{J}_k
\end{pmatrix}.
\end{align*}
上式中每个 $\boldsymbol{J}_i$ 是相应于初等因子 $(\lambda - \lambda_i)^{r_i}$ 的 Jordan 块,其阶正好为 $r_i$。令 $V_i$ 是由基向量 $e_{i1}, e_{i2}, \cdots, e_{ir_i}$ 生成的子空间,则
\begin{gather}\label{equation--7.7.2}
\begin{aligned}
\varphi(e_{i1}) &= \lambda_i e_{i1}, 
 \\
\varphi(e_{i2}) &= e_{i1} + \lambda_i e_{i2},  \\
&\cdots\cdots\cdots\cdots \\
\varphi(e_{ir_i}) &= e_{i,r_i - 1} + \lambda_i e_{ir_i}.
\end{aligned}
\end{gather}
这表明 $\varphi(V_i) \subseteq V_i$,即 $V_i (i = 1, 2, \cdots, k)$ 是 $\varphi$ 的不变子空间。显然我们有
\[
V = V_1 \oplus V_2 \oplus \cdots \oplus V_k.
\]

线性变换 $\varphi$ 限制在 $V_1$ 上 (仍记为 $\varphi$) 便成为 $V_1$ 上的线性变换。这个线性变换在基 $\{e_{11}, e_{12}, \cdots, e_{1r_1}\}$ 下的表示矩阵为
\begin{align*}
\boldsymbol{J}_1 = 
\begin{pmatrix}
\lambda_1 & 1 & & & \\
 & \lambda_1 & 1 & & \\
 & & \ddots & \ddots & \\
 & & & \ddots & 1 \\
 & & & & \lambda_1
\end{pmatrix}.
\end{align*}
注意到 $\boldsymbol{J}_1$ 的特征值全为 $\lambda_1$,并且 $\lambda_1\boldsymbol{I} - \boldsymbol{J}_1$ 的秩等于 $r_1 - 1$,故 $\boldsymbol{J}_1$ 只有一个线性无关的特征向量,不妨选为 $e_{11}$。显然 $e_{11}$ 也是 $\varphi$ 作为 $V$ 上线性变换关于特征值 $\lambda_1$ 的特征向量。不失一般性,不妨设在 $\varphi$ 的初等因子组即\eqref{equation--7.7.1} 式中
\[
\lambda_1 = \lambda_2 = \cdots = \lambda_s, \quad \lambda_i \neq \lambda_1 (i = s + 1, \cdots, k),
\]
则 $\boldsymbol{J}_1, \cdots, \boldsymbol{J}_s$ 都以 $\lambda_1$ 为特征值,且{\heiti 相应于每一块有且只有一个线性无关的特征向量}。相应的特征向量可取为
\begin{align}
e_{11}, e_{21}, \cdots, e_{s1}, \label{equation--7.7.3}
\end{align}
显然这是 $s$ 个线性无关的特征向量。如果 $\lambda_i \neq \lambda_1$,则容易看出 $\mathrm{r}(\lambda_1\boldsymbol{I} - \boldsymbol{J}_i) = r_i$,于是
\begin{align*}
\mathrm{r}(\lambda_1\boldsymbol{I} - \boldsymbol{J}) &= \sum_{i = 1}^{k} \mathrm{r}(\lambda_1\boldsymbol{I} - \boldsymbol{J}_i) = (r_1 - 1) + \cdots + (r_s - 1) + r_{s + 1} + \cdots + r_k = n - s.
\end{align*}
因此 $\varphi$ 关于特征值 $\lambda_1$ 的特征子空间 $V_{\lambda_1}$ 的维数等于 $n - \mathrm{r}(\lambda_1\boldsymbol{I} - \boldsymbol{J}) = s$,从而特征子空间 $V_{\lambda_1}$ 以\eqref{equation--7.7.3}式中的向量为一组基。又 $\lambda_1$ 是 $\varphi$ 的 $r_1 + r_2 + \cdots + r_s$ 重特征值,因此 $\lambda_1$ 的重数与度数之差等于
\[
(r_1 + r_2 + \cdots + r_s) - s.
\]
\end{proof}

\begin{corollary}\label{corollary:每个Jordan块都有且仅有一个线性无关的特征向量}
线性变化(矩阵)的Jordan标准型中属于特征值$\lambda_0$的每一个Jordan块都有且仅有一个线性无关的特征向量.
\end{corollary}
\begin{proof}
由上述\refthe{theorem:关于一个特征值的Jordan块的个数等于其度数,阶数直和等于其重数}的证明立得.
\end{proof}

\begin{definition}[根子空间]
设 $\lambda_0$ 是 $n$ 维复线性空间 $V$ 上线性变换 $\varphi$ 的特征值,则
\begin{align*}
R(\lambda_0) = \{\boldsymbol{v} \in V \mid (\varphi - \lambda_0\boldsymbol{I})^n(\boldsymbol{v}) = \boldsymbol{0}\}
\end{align*}
构成了 $V$ 的一个子空间,称为属于特征值 $\lambda_0$ 的\textbf{根子空间}。 
\end{definition}

\begin{theorem}\label{theorem:Jordan标准型的几何意义}
设 $\varphi$ 是 $n$ 维复线性空间 $V$ 上的线性变换。
\begin{enumerate}[(1)]
\item 若 $\varphi$ 的初等因子组为
\[
(\lambda - \lambda_1)^{r_1}, (\lambda - \lambda_2)^{r_2}, \cdots, (\lambda - \lambda_k)^{r_k},
\]
则 $V$ 可分解为 $k$ 个不变子空间的直和:
\begin{align}
V = V_1 \oplus V_2 \oplus \cdots \oplus V_k, \label{equation--7.7.5}
\end{align}
其中 $V_i$ 是维数等于 $r_i$ 的关于 $\varphi - \lambda_i\boldsymbol{I}$ 的循环子空间;

\item 若 $\lambda_1, \cdots, \lambda_s$ 是 $\varphi$ 的全体不同特征值,则 $V$ 可分解为 $s$ 个不变子空间的直和:
\begin{align*}
V = R(\lambda_1) \oplus R(\lambda_2) \oplus \cdots \oplus R(\lambda_s), 
\end{align*}
其中 $R(\lambda_i)$ 是 $\lambda_i$ 的根子空间,$R(\lambda_i)$ 的维数等于 $\lambda_i$ 的重数,且每个 $R(\lambda_i)$ 又可分解为\eqref{equation--7.7.5} 式中若干个 $V_j$ 的直和。 
\end{enumerate} 
\end{theorem}
\begin{proof}
在\refthe{theorem:关于一个特征值的Jordan块的个数等于其度数,阶数直和等于其重数}的证明的基础上,现在再来看 $\boldsymbol{J}_1$ 所对应的子空间 $V_1$,由 \eqref{equation--7.7.2}式中诸等式可知
\[
(\varphi - \lambda_1\boldsymbol{I})(e_{1r_1}) = e_{1,r_1 - 1}, \cdots, (\varphi - \lambda_1\boldsymbol{I})(e_{12}) = e_{11}, (\varphi - \lambda_1\boldsymbol{I})(e_{11}) = \boldsymbol{0},
\]
因此,若记 $\boldsymbol{\alpha} = e_{1r_1}, \psi = \varphi - \lambda_1\boldsymbol{I}$,则
\begin{align*}
\psi(\boldsymbol{\alpha}) = e_{1,r_1 - 1},  \psi^2(\boldsymbol{\alpha}) = e_{1,r_1 - 2},  \cdots,  \psi^{r_1 - 1}(\boldsymbol{\alpha}) = e_{11},  \psi^{r_1}(\boldsymbol{\alpha}) = \boldsymbol{0}.
\end{align*}
也就是说
\[
\{\boldsymbol{\alpha}, \psi(\boldsymbol{\alpha}), \psi^2(\boldsymbol{\alpha}), \cdots, \psi^{r_1 - 1}(\boldsymbol{\alpha})\}
\]
构成了 $V_1$ 的一组基。

上面的事实说明,每个 Jordan 块 $\boldsymbol{J}_i$ 对应的子空间 $V_i$ 是一个循环子空间。把属于同一个特征值,比如属于 $\lambda_1$ 的所有循环子空间加起来构成 $V$ 的一个子空间:
\[
R(\lambda_1) = V_1 \oplus \cdots \oplus V_s.
\]
若 $\boldsymbol{v} \in R(\lambda_1)$,则不难算出 $(\varphi - \lambda_1\boldsymbol{I})^s(\boldsymbol{v}) = \boldsymbol{0}$,其中
\[
s = \dim R(\lambda_1) = r_1 + \cdots + r_s.
\]
事实上,我们可以证明
\begin{align}\label{equation--7.7.4}
R(\lambda_1) = \{\boldsymbol{v} \in V \mid (\varphi - \lambda_1\boldsymbol{I})^n(\boldsymbol{v}) = \boldsymbol{0}\}.
\end{align}
为证明 \eqref{equation--7.7.4}式成立,设 $U = \{\boldsymbol{v} \in V \mid (\varphi - \lambda_1\boldsymbol{I})^n(\boldsymbol{v}) = \boldsymbol{0}\}$,则由上面的分析知道,$R(\lambda_1) \subseteq U$。另一方面,任取 $\boldsymbol{v} \in U$,设 $\boldsymbol{v} = \boldsymbol{v}_1 + \boldsymbol{v}_2$,其中 $\boldsymbol{v}_1 \in R(\lambda_1)$,$\boldsymbol{v}_2 \in V_{s + 1} \oplus \cdots \oplus V_k$。因为 $(\lambda - \lambda_1)^n$ 与 $(\lambda - \lambda_{s + 1})^n \cdots (\lambda - \lambda_k)^n$ 互素,故存在多项式 $p(\lambda), q(\lambda)$,使
\[
(\lambda - \lambda_1)^n p(\lambda) + (\lambda - \lambda_{s + 1})^n \cdots (\lambda - \lambda_k)^n q(\lambda) = 1.
\]
将 $\lambda = \varphi$ 代入上式并作用在 $\boldsymbol{v}$ 上可得
\begin{align*}
\boldsymbol{v} &= p(\varphi)(\varphi - \lambda_1\boldsymbol{I})^n(\boldsymbol{v}) + q(\varphi)(\varphi - \lambda_{s + 1}\boldsymbol{I})^n \cdots (\varphi - \lambda_k\boldsymbol{I})^n(\boldsymbol{v}) \\
&= q(\varphi)(\varphi - \lambda_{s + 1}\boldsymbol{I})^n \cdots (\varphi - \lambda_k\boldsymbol{I})^n(\boldsymbol{v}_1) + q(\varphi)(\varphi - \lambda_{s + 1}\boldsymbol{I})^n \cdots (\varphi - \lambda_k\boldsymbol{I})^n(\boldsymbol{v}_2) \\
&= q(\varphi)(\varphi - \lambda_{s + 1}\boldsymbol{I})^n \cdots (\varphi - \lambda_k\boldsymbol{I})^n(\boldsymbol{v}_1) \in R(\lambda_1).
\end{align*}
这就证明了 \eqref{equation--7.7.4}式。

上面的结果表明:特征值 $\lambda_0$ 的根子空间可表示为若干个循环子空间的直和,每个循环子空间对应于一个 Jordan 块。
虽然我们前面的讨论是对特征值 $\lambda_1$ 进行的,其实对任一特征值 $\lambda_i$ 均适用。 
\end{proof}

\begin{proposition}
证明:复数域上的方阵 $\boldsymbol{A}$ 必可分解为两个对称阵的乘积。
\end{proposition}
\begin{proof}
设 $\boldsymbol{P}$ 是非异阵且使 $\boldsymbol{P}^{-1}\boldsymbol{AP} = \boldsymbol{J}$ 为 $\boldsymbol{A}$ 的 Jordan 标准型,于是 $\boldsymbol{A} = \boldsymbol{PJP}^{-1}$。设 $\boldsymbol{J}_i$ 是 $\boldsymbol{J}$ 的第 $i$ 个 Jordan 块,则
\begin{align*}
\boldsymbol{J}_i=\left( \begin{matrix}
\lambda _i&		1&		&		&		\\
&		\lambda _i&		1&		&		\\
&		&		\ddots&		\ddots&		\\
&		&		&		\ddots&		1\\
&		&		&		&		\lambda _i\\
\end{matrix} \right) =\left( \begin{matrix}
&		&		&		1&		\lambda _i\\
&		&		1&		\lambda _i&		\\
&		\begin{turn}{80}$\ddots$\end{turn}&		\begin{turn}{80}$\ddots$\end{turn}&		&		\\
1&		\begin{turn}{80}$\ddots$\end{turn}&		&		&		\\
\lambda _i&		&		&		&		\\
\end{matrix} \right) \left( \begin{matrix}
&		&		&		&		1\\
&		&		&		1&		\\
&		&		\begin{turn}{80}$\ddots$\end{turn}&		&		\\
&		1&		&		&		\\
1&		&		&		&		\\
\end{matrix} \right) ,
\end{align*}
即 $\boldsymbol{J}_i$ 可分解为两个对称阵之积。因此 $\boldsymbol{J}$ 也可以分解为两个对称阵之积,记为 $\boldsymbol{S}_1, \boldsymbol{S}_2$,于是
\[
\boldsymbol{A} = \boldsymbol{PJP}^{-1} = \boldsymbol{PS}_1\boldsymbol{S}_2\boldsymbol{P}^{-1} = (\boldsymbol{PS}_1\boldsymbol{P}')(\boldsymbol{P}^{-1})'\boldsymbol{S}_2\boldsymbol{P}^{-1}. 
\] 
显然$\boldsymbol{PS}_1\boldsymbol{P}'$和$(\boldsymbol{P}^{-1})'\boldsymbol{S}_2\boldsymbol{P}^{-1}$都是对称矩阵,
故$\boldsymbol{A}$ 必可分解为两个对称阵的乘积。
\end{proof}

\begin{example}
已知
\[
\boldsymbol{A} = 
\begin{pmatrix}
3 & 1 & 0 & 0 \\
-4 & -1 & 0 & 0 \\
6 & 1 & 2 & 1 \\
-14 & -5 & -1 & 0
\end{pmatrix},
\] 
计算$A^k.$
\end{example}
\begin{solution}
用初等变换把 $\lambda\boldsymbol{I} - \boldsymbol{A}$ 化为对角 $\lambda$-矩阵并求出它的初等因子组为
\[
(\lambda - 1)^2, (\lambda - 1)^2.
\]
因此,$\boldsymbol{A}$ 的 Jordan 标准型为
\begin{align*}
\boldsymbol{J} = 
\begin{pmatrix}
1 & 1 & & \\
0 & 1 & & \\
 & & 1 & 1 \\
 & & 0 & 1
\end{pmatrix}.
\end{align*} 
因为
\begin{align*}
\boldsymbol{P}^{-1}\boldsymbol{A}^k\boldsymbol{P} = (\boldsymbol{P}^{-1}\boldsymbol{AP})^k = \boldsymbol{J}^k,
\end{align*}
故先计算 $\boldsymbol{J}^k$。注意 $\boldsymbol{J}$ 是分块对角阵,它的 $k$ 次方等于将各对角块 $k$ 次方,因此
\begin{align*}
\boldsymbol{J}^k = 
\begin{pmatrix}
1 & 1 & & \\
0 & 1 & & \\
    & & 1 & 1 \\
    & & 0 & 1
\end{pmatrix}^k
=
\begin{pmatrix}
1 & k & & \\
0 & 1 & & \\
    & & 1 & k \\
    & & 0 & 1
\end{pmatrix},
\end{align*}
\begin{align*}
\boldsymbol{A}^k = \boldsymbol{PJ}^k\boldsymbol{P}^{-1} 
&=
\begin{pmatrix}
1 & 0 & 0 & 0 \\
-2 & 1 & 0 & 0 \\
1 & 0 & 1 & 0 \\
-5 & 0 & -1 & 1
\end{pmatrix}
\begin{pmatrix}
1 & k & & \\
0 & 1 & & \\
    & & 1 & k \\
    & & 0 & 1
\end{pmatrix}
\begin{pmatrix}
1 & 0 & 0 & 0 \\
-2 & 1 & 0 & 0 \\
1 & 0 & 1 & 0 \\
-5 & 0 & -1 & 1
\end{pmatrix}^{-1} \\
&=
\begin{pmatrix}
2k + 1 & k & 0 & 0 \\
-4k & -2k + 1 & 0 & 0 \\
6k & k & k + 1 & k \\
-14k & -5k & -k & -k + 1
\end{pmatrix}.
\end{align*}
\end{solution}

\begin{theorem}[Jordan-Chevalley分解]\label{theorem:Jordan-Chevalley分解}
设 $\boldsymbol{A}$ 是 $n$ 阶复矩阵,则 $\boldsymbol{A}$ 可分解为 $\boldsymbol{A} = \boldsymbol{B} + \boldsymbol{C}$,其中 $\boldsymbol{B}, \boldsymbol{C}$ 适合下面条件:

(1) $\boldsymbol{B}$ 是一个可对角化矩阵;

(2) $\boldsymbol{C}$ 是一个幂零阵;

(3) $\boldsymbol{BC} = \boldsymbol{CB}$;

(4) $\boldsymbol{B}, \boldsymbol{C}$ 均可表示为 $\boldsymbol{A}$ 的多项式。

不仅如此,上述满足条件 (1)(3) 的分解是唯一的(即只要满足条件(1)(3)的分解就是唯一的)。进而,上述满足条件(1)(2)(3)(4)的分解也是唯一的.
\end{theorem}
\begin{proof}
先对 $\boldsymbol{A}$ 的 Jordan 标准型 $\boldsymbol{J}$ 证明结论。设 $\boldsymbol{A}$ 的全体不同特征值为 $\lambda_1, \lambda_2, \cdots, \lambda_s$ 且
\begin{align*}
\boldsymbol{J} = 
\begin{pmatrix}
\boldsymbol{J}_1 & & & \\
 & \boldsymbol{J}_2 & & \\
 & & \ddots & \\
 & & & \boldsymbol{J}_s
\end{pmatrix},
\end{align*}
其中 $\boldsymbol{J}_i$ 是属于特征值 $\lambda_i$ 的根子空间对应的块,其阶设为 $m_i$。显然对每个 $i$ 均有 $\boldsymbol{J}_i = \boldsymbol{M}_i + \boldsymbol{N}_i$,其中 $\boldsymbol{M}_i = \lambda_i\boldsymbol{I}$ 是对角阵,$\boldsymbol{N}_i$ 是幂零阵且 $\boldsymbol{M}_i\boldsymbol{N}_i = \boldsymbol{N}_i\boldsymbol{M}_i$。令
\[
\boldsymbol{M} = 
\begin{pmatrix}
\boldsymbol{M}_1 & & & \\
 & \boldsymbol{M}_2 & & \\
 & & \ddots & \\
 & & & \boldsymbol{M}_s
\end{pmatrix}, \quad
\boldsymbol{N} = 
\begin{pmatrix}
\boldsymbol{N}_1 & & & \\
 & \boldsymbol{N}_2 & & \\
 & & \ddots & \\
 & & & \boldsymbol{N}_s
\end{pmatrix},
\]
则 $\boldsymbol{J} = \boldsymbol{M} + \boldsymbol{N}, \boldsymbol{MN} = \boldsymbol{NM}, \boldsymbol{M}$ 是对角阵,$\boldsymbol{N}$ 是幂零阵。

因为 $(\boldsymbol{J}_i - \lambda_i\boldsymbol{I})^{m_i} = \boldsymbol{O}$,所以 $\boldsymbol{J}_i$ 适合多项式 $(\lambda - \lambda_i)^{m_i}$。而 $\lambda_i$ 互不相同,因此多项式 $(\lambda - \lambda_1)^{m_1}, (\lambda - \lambda_2)^{m_2}, \cdots, (\lambda - \lambda_s)^{m_s}$ 两两互素。由\hyperref[theorem:中国剩余定理]{中国剩余定理},存在多项式 $g(\lambda)$ 满足条件
\[
g(\lambda) = h_i(\lambda)(\lambda - \lambda_i)^{m_i} + \lambda_i,
\]
对所有 $i = 1, 2, \cdots, s$ 成立 (这里 $h_i(\lambda)$ 也是多项式)。代入 $\boldsymbol{J}_i$ 得到
\[
g(\boldsymbol{J}_i) = h_i(\boldsymbol{J}_i)(\boldsymbol{J}_i - \lambda_i\boldsymbol{I})^{m_i} + \lambda_i\boldsymbol{I} = \lambda_i\boldsymbol{I} = \boldsymbol{M}_i.
\]
于是
\begin{align*}
g(\boldsymbol{J}) = 
\begin{pmatrix}
g(\boldsymbol{J}_1) & & & \\
 & g(\boldsymbol{J}_2) & & \\
 & & \ddots & \\
 & & & g(\boldsymbol{J}_s)
\end{pmatrix}
=
\begin{pmatrix}
\boldsymbol{M}_1 & & & \\
 & \boldsymbol{M}_2 & & \\
 & & \ddots & \\
 & & & \boldsymbol{M}_s
\end{pmatrix}
= \boldsymbol{M}.
\end{align*}
又因为 $\boldsymbol{N} = \boldsymbol{J} - \boldsymbol{M} = \boldsymbol{J} - g(\boldsymbol{J})$,所以 $\boldsymbol{N}$ 也是 $\boldsymbol{J}$ 的多项式。

现考虑一般情形,设 $\boldsymbol{P}^{-1}\boldsymbol{AP} = \boldsymbol{J}$,则 $\boldsymbol{A} = \boldsymbol{PJP}^{-1} = \boldsymbol{P}(\boldsymbol{M} + \boldsymbol{N})\boldsymbol{P}^{-1}$。令 $\boldsymbol{B} = \boldsymbol{PMP}^{-1}, \boldsymbol{C} = \boldsymbol{PNP}^{-1}$,则 $\boldsymbol{B}$ 是可对角化矩阵,$\boldsymbol{C}$ 是幂零阵,$\boldsymbol{BC} = \boldsymbol{CB}$ 并且
\[
g(\boldsymbol{A}) = g(\boldsymbol{PJP}^{-1}) = \boldsymbol{P}g(\boldsymbol{J})\boldsymbol{P}^{-1} = \boldsymbol{PMP}^{-1} = \boldsymbol{B},
\]
从而 $\boldsymbol{C} = \boldsymbol{A} - g(\boldsymbol{A})$。

最后证明唯一性。假设 $\boldsymbol{A}$ 有另一满足条件 (1)~(3) 的分解 $\boldsymbol{A} = \boldsymbol{B}_1 + \boldsymbol{C}_1$,则 $\boldsymbol{B} - \boldsymbol{B}_1 = \boldsymbol{C}_1 - \boldsymbol{C}$。由 $\boldsymbol{B}_1\boldsymbol{C}_1 = \boldsymbol{C}_1\boldsymbol{B}_1$ 不难验证 $\boldsymbol{AB}_1 = \boldsymbol{B}_1\boldsymbol{A}, \boldsymbol{AC}_1 = \boldsymbol{C}_1\boldsymbol{A}$。因为 $\boldsymbol{B} = g(\boldsymbol{A})$,故 $\boldsymbol{BB}_1 = \boldsymbol{B}_1\boldsymbol{B}$。同理 $\boldsymbol{CC}_1 = \boldsymbol{C}_1\boldsymbol{C}$。设 $\boldsymbol{C}^r = \boldsymbol{O}, \boldsymbol{C}_1^t = \boldsymbol{O}$,用二项式定理即知 $(\boldsymbol{C}_1 - \boldsymbol{C})^{r + t} = \boldsymbol{O}$。于是
\[
(\boldsymbol{B} - \boldsymbol{B}_1)^{r + t} = (\boldsymbol{C}_1 - \boldsymbol{C})^{r + t} = \boldsymbol{O}.
\]
因为 $\boldsymbol{BB}_1 = \boldsymbol{B}_1\boldsymbol{B}$,它们都是可对角化矩阵,由\refpro{proposition:乘法可交换诱导同时对角化}知它们可同时对角化,即存在可逆阵 $\boldsymbol{Q}$,使 $\boldsymbol{Q}^{-1}\boldsymbol{BQ}$ 和 $\boldsymbol{Q}^{-1}\boldsymbol{B}_1\boldsymbol{Q}$ 都是对角阵。注意到
\[
(\boldsymbol{Q}^{-1}\boldsymbol{BQ} - \boldsymbol{Q}^{-1}\boldsymbol{B}_1\boldsymbol{Q})^{r + t} = \left(\boldsymbol{Q}^{-1}(\boldsymbol{B} - \boldsymbol{B}_1)\boldsymbol{Q}\right)^{r + t} = \boldsymbol{Q}^{-1}(\boldsymbol{B} - \boldsymbol{B}_1)^{r + t}\boldsymbol{Q} = \boldsymbol{O},
\]
两个对角阵之差仍是一个对角阵,这个差的幂要等于零矩阵,则这两个矩阵必相等,由此即得 $\boldsymbol{B} = \boldsymbol{B}_1$,从而 $\boldsymbol{C} = \boldsymbol{C}_1$。
\end{proof}



























































































































\end{document}
\documentclass[../../main.tex]{subfiles}
\graphicspath{{\subfix{../../image/}}} % 指定图片目录,后续可以直接使用图片文件名。

% 例如:
% \begin{figure}[H]
% \centering
% \includegraphics[scale=0.3]{image-01.01}
% \caption{图片标题}
% \label{figure:image-01.01}
% \end{figure}
% 注意:上述\label{}一定要放在\caption{}之后,否则引用图片序号会只会显示??.

\begin{document}

\section{矩阵与二次型}

\subsection{用矩阵方法来讨论二次型问题}

\begin{lemma}\label{lemma:一般二次型的矩阵形式}
设$A$是$n$阶实对称矩阵,则
\begin{align*}
f\left( x \right) =x' Ax+2a\beta ' x+a^2c=\left( \begin{matrix}
x'&		a\\
\end{matrix} \right) \left( \begin{matrix}
A&		\beta\\
\beta '&		c\\
\end{matrix} \right) \left( \begin{array}{c}
x\\
a\\
\end{array} \right) .
\end{align*}
\end{lemma}
\begin{remark}
上述$f(x)$仍是一个二次型,只不过有1个变量恒为常数而已.
\end{remark}
\begin{proof}
由矩阵乘法易证.
\end{proof}

\begin{example}
设$\boldsymbol{A}$是$n$阶正定实对称矩阵, 求证: 函数$f(\boldsymbol{x})=\boldsymbol{x}'\boldsymbol{A}\boldsymbol{x}+2\boldsymbol{\beta}'\boldsymbol{x}+c$的极小值等于$c - \boldsymbol{\beta}'\boldsymbol{A}^{-1}\boldsymbol{\beta}$, 其中$\boldsymbol{\beta}=(b_1, \cdots, b_n)', b_i$和$c$都是实数.
\end{example}
\begin{proof}
注意到
\[
f(\boldsymbol{x})=(\boldsymbol{x}'\ 1)\begin{pmatrix}
\boldsymbol{A} & \boldsymbol{\beta} \\
\boldsymbol{\beta}' & c
\end{pmatrix}\begin{pmatrix}
\boldsymbol{x} \\
1
\end{pmatrix},
\]
因为$\boldsymbol{A}$可逆, 故可作如下对称分块初等变换:
\begin{align*}
\begin{pmatrix}
\boldsymbol{I}_n & \boldsymbol{O} \\
-\boldsymbol{\beta}'\boldsymbol{A}^{-1} & 1
\end{pmatrix}
\begin{pmatrix}
\boldsymbol{A} & \boldsymbol{\beta} \\
\boldsymbol{\beta}' & c
\end{pmatrix}
\begin{pmatrix}
\boldsymbol{I}_n & -\boldsymbol{A}^{-1}\boldsymbol{\beta} \\
\boldsymbol{O} & 1
\end{pmatrix}
=
\begin{pmatrix}
\boldsymbol{A} & \boldsymbol{O} \\
\boldsymbol{O} & c - \boldsymbol{\beta}'\boldsymbol{A}^{-1}\boldsymbol{\beta}
\end{pmatrix}.
\end{align*}
由$\begin{pmatrix}
\boldsymbol{x} \\
1
\end{pmatrix}=\begin{pmatrix}
\boldsymbol{I}_n & -\boldsymbol{A}^{-1}\boldsymbol{\beta} \\
\boldsymbol{O} & 1
\end{pmatrix}\begin{pmatrix}
\boldsymbol{y} \\
1
\end{pmatrix}$可解出$\boldsymbol{y}=\boldsymbol{x}+\boldsymbol{A}^{-1}\boldsymbol{\beta}$, 于是
\[
f(\boldsymbol{x})=(\boldsymbol{y}'\ 1)\begin{pmatrix}
\boldsymbol{A} & \boldsymbol{O} \\
\boldsymbol{O} & c - \boldsymbol{\beta}'\boldsymbol{A}^{-1}\boldsymbol{\beta}
\end{pmatrix}\begin{pmatrix}
\boldsymbol{y} \\
1
\end{pmatrix}=\boldsymbol{y}'\boldsymbol{A}\boldsymbol{y}+c - \boldsymbol{\beta}'\boldsymbol{A}^{-1}\boldsymbol{\beta}\geq c - \boldsymbol{\beta}'\boldsymbol{A}^{-1}\boldsymbol{\beta}.
\]
因此, 当$\boldsymbol{x}=-\boldsymbol{A}^{-1}\boldsymbol{\beta}$时, $f(\boldsymbol{x})$取到极小值$c - \boldsymbol{\beta}'\boldsymbol{A}^{-1}\boldsymbol{\beta}$. 
\end{proof}

\begin{proposition}\label{proposition:和的平方和的半正定型的秩等于其系数矩阵的秩}
设实二次型
\[f(x_1,x_2,\cdots,x_n)=\sum_{i = 1}^{k}(a_{i1}x_1 + a_{i2}x_2 + \cdots + a_{in}x_n)^2\]
其中 \(a_{ij}\) 都是实数, 求证 \(f\) 是半正定型且 \(f\) 的秩等于下列矩阵的秩:
\[
\boldsymbol{A} = \begin{pmatrix}
a_{11} & a_{12} & \cdots & a_{1n} \\
a_{21} & a_{22} & \cdots & a_{2n} \\
\vdots & \vdots & & \vdots \\
a_{k1} & a_{k2} & \cdots & a_{kn}
\end{pmatrix}
\]
\end{proposition}
\begin{proof}
\(f\) 的半正定性由定义即得. 注意到 \(f(\boldsymbol{x}) = (\boldsymbol{A}\boldsymbol{x})'(\boldsymbol{A}\boldsymbol{x}) = \boldsymbol{x}'(\boldsymbol{A}'\boldsymbol{A})\boldsymbol{x}\), 故 \(f\) 的相伴矩阵为 \(\boldsymbol{A}'\boldsymbol{A}\), 于是\hyperref[proposition:r(AA')=r(A)]{命题\ref{proposition:r(AA')=r(A)}(1)}可知, \(\mathrm{r}(f)=\mathrm{r}(\boldsymbol{A}'\boldsymbol{A})=\mathrm{r}(\boldsymbol{A})\). 
\end{proof}

\subsection{用二次型方法来讨论矩阵问题}

\begin{proposition}\label{proposition:正定实对称阵的Hadamard积也正定}
设 \(\boldsymbol{A}=(a_{ij})\), \(\boldsymbol{B}=(b_{ij})\) 都是 \(n\) 阶正定实对称矩阵, 求证: \(\boldsymbol{A}\), \(\boldsymbol{B}\) 的 Hadamard 乘积 \(\boldsymbol{H}=\boldsymbol{A}\circ\boldsymbol{B}=(a_{ij}b_{ij})\) 也是正定阵.
\end{proposition}
\begin{proof}
因为 \(\boldsymbol{B}\) 是正定阵, 故由\refpro{proposition:正定和半正定阵的判定准则}可知, 存在可逆实矩阵 \(\boldsymbol{C}\), 使得 \(\boldsymbol{B}=\boldsymbol{C}'\boldsymbol{C}\). 设 \(\boldsymbol{C}=(c_{ij})\), 则 \(b_{ij}=\sum_{k = 1}^{n}c_{ki}c_{kj}\). 作二次型
\begin{align*}
f(\boldsymbol{x})&=\boldsymbol{x}'\boldsymbol{H}\boldsymbol{x}=\sum_{i,j = 1}^{n}a_{ij}b_{ij}x_ix_j=\sum_{i,j = 1}^{n}\left(\sum_{k = 1}^{n}a_{ij}(c_{ki}c_{kj})\right)x_ix_j\\
&=\sum_{k = 1}^{n}\left(\sum_{i,j = 1}^{n}a_{ij}(c_{ki}x_i)(c_{kj}x_j)\right)=\sum_{k = 1}^{n}\boldsymbol{y}_k'\boldsymbol{A}\boldsymbol{y}_k
\end{align*}
其中 \(\boldsymbol{y}_k=(c_{k1}x_1,c_{k2}x_2,\cdots,c_{kn}x_n)'\). 因为 \(\boldsymbol{C}\) 可逆, 所以当 \(\boldsymbol{x}\neq\boldsymbol{0}\) 时, 至少有一个 \(\boldsymbol{y}_k\neq\boldsymbol{0}\), 因此由 \(\boldsymbol{A}\) 的正定性可得 \(f(\boldsymbol{x})>0\), 于是 \(f\) 是正定型, 从而 \(\boldsymbol{H}\) 是正定阵.
\end{proof}

\begin{example}
求下列实二次型的标准型:
\begin{enumerate}
\item \(f(x_1,x_2,\cdots,x_n)=\sum_{i,j = 1}^{n}\max\{i,j\}x_ix_j\);
\item \(f(x_1,x_2,\cdots,x_n)=\sum_{i,j = 1}^{n}|i - j|x_ix_j\).
\end{enumerate}
\end{example}
\begin{solution}
\begin{enumerate}
\item \(f\) 的系数矩阵是 \(\boldsymbol{A}=(a_{ij})\), 其中 \(a_{ij}=\max\{i,j\}\). 由\refexa{example--经典行列式1}可知, \(\boldsymbol{A}\) 的第 \(k\) 个顺序主子式 \(|\boldsymbol{A}_k|=(-1)^{k - 1}k(1\leq k\leq n)\), 再由\refpro{proposition:利用顺序主子式计算二次型的标准型}可知, \(f\) 的规范标准型为 \(y_1^2 - y_2^2 - \cdots - y_n^2\).
\item \(f\) 的系数矩阵是 \(\boldsymbol{A}=(a_{ij})\), 其中 \(a_{ij}=|i - j|\). 由于 \(a_{11}=0\), 故先做对称初等变换: 将 \(\boldsymbol{A}\) 的第二行加到第一行上, 再将第二列加到第一列上, 得到的矩阵记为 \(\boldsymbol{B}=(b_{ij})\), 其中 \(b_{11}=2\), 即 \(|\boldsymbol{B}_1| = 2\). 由于第三类初等变换不改变行列式的值, 故由\refexa{example--经典行列式1}可知, \(\boldsymbol{B}\) 的第 \(k\) 个顺序主子式 \(|\boldsymbol{B}_k|=(-1)^{k - 1}(k - 1)2^{k - 2}(2\leq k\leq n)\), 再由\refpro{proposition:利用顺序主子式计算二次型的标准型}可知, \(f\) 的规范标准型为 \(y_1^2 - y_2^2 - \cdots - y_n^2\). 
\end{enumerate} 
\end{solution}


\begin{proposition}\label{proposition:可逆实对称与实反称阵可交换则相加也可逆}
设 \(\boldsymbol{A}\) 是 \(n\) 阶可逆实对称矩阵, \(\boldsymbol{S}\) 是 \(n\) 阶实反对称矩阵且 \(\boldsymbol{A}\boldsymbol{S}=\boldsymbol{S}\boldsymbol{A}\), 求证: \(\boldsymbol{A}+\boldsymbol{S}\) 是可逆矩阵.
\end{proposition}
\begin{proof}
{\color{blue}证法一:}对任一 \(n\) 维非零实列向量 \(\boldsymbol{\alpha}\), 我们有
\begin{align*}
\boldsymbol{\alpha}'(\boldsymbol{A}+\boldsymbol{S})'(\boldsymbol{A}+\boldsymbol{S})\boldsymbol{\alpha}&=\boldsymbol{\alpha}'(\boldsymbol{A}'\boldsymbol{A}+\boldsymbol{A}'\boldsymbol{S}+\boldsymbol{S}'\boldsymbol{A}+\boldsymbol{S}'\boldsymbol{S})\boldsymbol{\alpha}\\
&=\boldsymbol{\alpha}'(\boldsymbol{A}'\boldsymbol{A})\boldsymbol{\alpha}+\boldsymbol{\alpha}'(\boldsymbol{A}'\boldsymbol{S}+\boldsymbol{S}'\boldsymbol{A})\boldsymbol{\alpha}+\boldsymbol{\alpha}'(\boldsymbol{S}'\boldsymbol{S})\boldsymbol{\alpha}
\end{align*}
由于 \(\boldsymbol{A}'\boldsymbol{S}+\boldsymbol{S}'\boldsymbol{A}=\boldsymbol{A}\boldsymbol{S}-\boldsymbol{S}\boldsymbol{A}=\boldsymbol{O}\), 故上式等于 \(\boldsymbol{\alpha}'(\boldsymbol{A}'\boldsymbol{A})\boldsymbol{\alpha}+\boldsymbol{\alpha}'(\boldsymbol{S}'\boldsymbol{S})\boldsymbol{\alpha}\). 由\refpro{proposition:正定和半正定阵的判定准则} 可知, \(\boldsymbol{A}'\boldsymbol{A}\) 是正定阵, \(\boldsymbol{S}'\boldsymbol{S}\) 是半正定阵, 所以上式总大于零, 即 \((\boldsymbol{A}+\boldsymbol{S})'(\boldsymbol{A}+\boldsymbol{S})\) 是正定阵, 于是 \(|\boldsymbol{A}+\boldsymbol{S}|^2>0\), 从而 \(\boldsymbol{A}+\boldsymbol{S}\) 是可逆矩阵.

{\color{blue}证法二:}由于 \(\boldsymbol{A}+\boldsymbol{S}=\boldsymbol{A}(\boldsymbol{I}_n+\boldsymbol{A}^{-1}\boldsymbol{S})\), 故只要证明 \(\boldsymbol{I}_n+\boldsymbol{A}^{-1}\boldsymbol{S}\) 可逆即可. 由 \(\boldsymbol{A}\boldsymbol{S}=\boldsymbol{S}\boldsymbol{A}\) 可知 \(\boldsymbol{A}^{-1}\boldsymbol{S}=\boldsymbol{S}\boldsymbol{A}^{-1}\), 于是
\[
(\boldsymbol{A}^{-1}\boldsymbol{S})'=\boldsymbol{S}'(\boldsymbol{A}^{-1})'=\boldsymbol{S}'(\boldsymbol{A}')^{-1}=-\boldsymbol{S}\boldsymbol{A}^{-1}=-\boldsymbol{A}^{-1}\boldsymbol{S}
\]
即 \(\boldsymbol{A}^{-1}\boldsymbol{S}\) 是实反对称矩阵, 最后由\hyperref[proposition:实对称与实反称矩阵性质]{命题\ref{proposition:实对称与实反称矩阵性质}(2)}即得结论.
\end{proof}







































































\end{document}
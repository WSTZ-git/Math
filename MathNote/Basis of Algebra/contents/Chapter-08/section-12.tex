\documentclass[../../main.tex]{subfiles}
\graphicspath{{\subfix{../../image/}}} % 指定图片目录,后续可以直接使用图片文件名。

% 例如:
% \begin{figure}[h]
% \centering
% \includegraphics{image-01.01}
% \caption{图片标题}
% \label{fig:image-01.01}
% \end{figure}
% 注意:上述\label{}一定要放在\caption{}之后,否则引用图片序号会只会显示??.

\begin{document}

\section{半正定型与半正定阵}

\begin{proposition}\label{proposition:半正定阵的判定准则123}
设 \(\boldsymbol{A}\) 是 \(n\) 阶实对称矩阵,则 \(\boldsymbol{A}\) 是半正定阵的充要条件是以下条件之一:
\begin{enumerate}[(1)]
\item \(\boldsymbol{A}\) 合同于 \(\begin{pmatrix}\boldsymbol{I}_r&\boldsymbol{O}\\\boldsymbol{O}&\boldsymbol{O}\end{pmatrix}\);

\item 存在实矩阵 \(\boldsymbol{C}\),使得 \(\boldsymbol{A}=\boldsymbol{C}'\boldsymbol{C}\);

\item \(\boldsymbol{A}\) 的所有主子式全大于等于零;

\item \(\boldsymbol{A}\) 的所有特征值全大于等于零。
\end{enumerate}
\end{proposition}
\begin{proof}
\begin{enumerate}[(1)]
\item 参考\reftheorem{theorem:正定矩阵的充要条件}.

\item 参考\hyperref[proposition:正定和半正定阵的判定准则]{命题\ref{proposition:正定和半正定阵的判定准则}(2)}.

\item 参考.

\item 参考第九章相关内容.
\end{enumerate}
\end{proof}

\begin{proposition}\label{proposition:半正定阵的相关性质}
设 \(\boldsymbol{A},\boldsymbol{B}\) 都是 \(n\) 阶半正定实对称矩阵, \(c\) 是非负实数, 求证:
\begin{enumerate}[(1)]
\item \(\boldsymbol{A}^*,\boldsymbol{A}+\boldsymbol{B},c\boldsymbol{A}\) 都是半正定阵;
\item 若 \(\boldsymbol{D}\) 是实矩阵, 则 \(\boldsymbol{D}'\boldsymbol{A}\boldsymbol{D}\) 也是半正定阵.
\end{enumerate}
\end{proposition}
\begin{proof}
\begin{enumerate}[(1)]
\item 因为 \(\boldsymbol{A}\) 半正定, 故存在实矩阵 \(\boldsymbol{C}\), 使得 \(\boldsymbol{A}=\boldsymbol{C}'\boldsymbol{C}\), 于是 \(\boldsymbol{A}^*=\boldsymbol{C}^*(\boldsymbol{C}')^*=\boldsymbol{C}^*(\boldsymbol{C}^*)'\) 是半正定阵. 对任一非零实列向量 \(\boldsymbol{\alpha}\), 有
\[\boldsymbol{\alpha}'(\boldsymbol{A}+\boldsymbol{B})\boldsymbol{\alpha}=\boldsymbol{\alpha}'\boldsymbol{A}\boldsymbol{\alpha}+\boldsymbol{\alpha}'\boldsymbol{B}\boldsymbol{\alpha}\geq0,\quad\boldsymbol{\alpha}'(c\boldsymbol{A})\boldsymbol{\alpha}=c\boldsymbol{\alpha}'\boldsymbol{A}\boldsymbol{\alpha}\geq0\]
因此 \(\boldsymbol{A}+\boldsymbol{B},c\boldsymbol{A}\) 都是半正定阵.
\item 采用 (1) 的记号, 则 \(\boldsymbol{D}'\boldsymbol{A}\boldsymbol{D}=\boldsymbol{D}'\boldsymbol{C}'\boldsymbol{C}\boldsymbol{D}=(\boldsymbol{C}\boldsymbol{D})'(\boldsymbol{C}\boldsymbol{D})\) 也是半正定阵.
\end{enumerate} 
\end{proof}

\subsection{{\heiti 性质1(极限性质)}半正定阵是正定阵的极限}

\begin{proposition}\label{proposition:半正定阵关于摄动的充要条件}
\(n\) 阶实对称矩阵 \(\boldsymbol{A}\) 是半正定阵的充要条件是对任意的正实数 \(t\), \(\boldsymbol{A}+t\boldsymbol{I}_n\) 都是正定阵.
\end{proposition}
\begin{remark}
这个\refproposition{proposition:半正定阵关于摄动的充要条件}告诉我们: \textbf{半正定阵是一列正定阵的极限}, 称之为\textbf{半正定阵的极限性质}. 因此我们可以利用极限性质和摄动法将半正定阵的问题转化成正定阵的问题来研究.
\end{remark}
\begin{proof}
先证必要性. 对任一非零实列向量 \(\boldsymbol{\alpha}\), 有
\[\boldsymbol{\alpha}'(\boldsymbol{A}+t\boldsymbol{I}_n)\boldsymbol{\alpha}=\boldsymbol{\alpha}'\boldsymbol{A}\boldsymbol{\alpha}+t\boldsymbol{\alpha}'\boldsymbol{\alpha}\]
因为 \(\boldsymbol{A}\) 半正定, 故 \(\boldsymbol{\alpha}'\boldsymbol{A}\boldsymbol{\alpha}\geq0\). 又 \(t>0\) 且 \(\boldsymbol{\alpha}'\boldsymbol{\alpha}>0\), 从而 \(\boldsymbol{\alpha}'(\boldsymbol{A}+t\boldsymbol{I}_n)\boldsymbol{\alpha}>0\), 因此 \(\boldsymbol{A}+t\boldsymbol{I}_n\) 是正定阵.

再证充分性. 由假设对任一非零实列向量 \(\boldsymbol{\alpha}\) 和正实数 \(t\), 有
\[\boldsymbol{\alpha}'(\boldsymbol{A}+t\boldsymbol{I}_n)\boldsymbol{\alpha}=\boldsymbol{\alpha}'\boldsymbol{A}\boldsymbol{\alpha}+t\boldsymbol{\alpha}'\boldsymbol{\alpha}>0\]
令 \(t\to0^+\), 上式两边同取极限可得 \(\boldsymbol{\alpha}'\boldsymbol{A}\boldsymbol{\alpha}\geq0\), 即 \(\boldsymbol{A}\) 是半正定阵. 
\end{proof}

\begin{proposition}\label{proposition:半正定阵关于顺序主子式的性质}
\(n\) 阶实对称矩阵 \(\boldsymbol{A}\) 是半正定阵的充要条件是 \(\boldsymbol{A}\) 的所有主子式全大于等于零.
\end{proposition}
\begin{remark}
我们不能用顺序主子式的非负性来推出半正定性, 这一点和正定阵不同. 例如, 矩阵 \(\boldsymbol{A}=\text{diag}\{1,0,-1\}\) 的顺序主子式都非负, 但 \(\boldsymbol{A}\) 却不是半正定阵. 
\end{remark}
\begin{proof}
必要性由\hyperref[proposition:正定阵的性质]{命题\ref{proposition:正定阵的性质}(1)} 和\hyperref[proposition:正定和半正定阵的判定准则]{命题\ref{proposition:正定和半正定阵的判定准则}(2)}即得, 下证充分性. 由\refproposition{proposition:直接计算两个矩阵和的行列式}可得
\[|\boldsymbol{A}+t\boldsymbol{I}_n|=t^n + c_1t^{n - 1}+\cdots + c_{n - 1}t + c_n\]
其中 \(c_i\) 是 \(\boldsymbol{A}\) 的所有 \(i\) 阶主子式之和. 由假设可知 \(c_i\geq0(1\leq i\leq n)\), 故对任意的正实数 \(t\), 我们总有 \(|\boldsymbol{A}+t\boldsymbol{I}_n|>0\). 设 \(\boldsymbol{A}_k(1\leq k\leq n)\) 是 \(\boldsymbol{A}\) 的 \(n\) 个顺序主子阵, 则 \(\boldsymbol{A}_k\) 的主子式也是 \(\boldsymbol{A}\) 的主子式, 从而 \(\boldsymbol{A}_k\) 的所有主子式全大于等于零, 根据上面的讨论同理 可知, 对任意的正实数 \(t\), 我们总有 \(|\boldsymbol{A}_k + t\boldsymbol{I}_k|>0\). 注意到 \(|\boldsymbol{A}_k + t\boldsymbol{I}_k|(1\leq k\leq n)\) 是 \(\boldsymbol{A}+t\boldsymbol{I}_n\) 的 \(n\) 个顺序主子式, 故由上面的讨论可知, 对任意的正实数 \(t\), \(\boldsymbol{A}+t\boldsymbol{I}_n\) 都是正定阵, 再由\refproposition{proposition:半正定阵关于摄动的充要条件} 即得 \(\boldsymbol{A}\) 为半正定阵. 
\end{proof}

\begin{proposition}\label{proposition:半正定阵的Hadamard乘积性质}
设 \(\boldsymbol{A}=(a_{ij}),\boldsymbol{B}=(b_{ij})\) 都是 \(n\) 阶半正定实对称矩阵, 求证: \(\boldsymbol{A},\boldsymbol{B}\) 的 Hadamard 乘积 \(\boldsymbol{H}=\boldsymbol{A}\circ\boldsymbol{B}=(a_{ij}b_{ij})\) 也是半正定阵.
\end{proposition}
\begin{proof}
{\color{blue}证法一:}设 \(\boldsymbol{B}=\boldsymbol{C}'\boldsymbol{C}\), 其中 \(\boldsymbol{C}\) 为实矩阵, 剩余的证明与\refproposition{proposition:正定实对称阵的Hadamard积也正定}完全类似.

{\color{blue}证法二:} 由于对任意的正实数 \(t\), \(\boldsymbol{A}+t\boldsymbol{I}_n,\boldsymbol{B}+t\boldsymbol{I}_n\) 都是正定阵, 故由\refproposition{proposition:正定实对称阵的Hadamard积也正定}可知 \((\boldsymbol{A}+t\boldsymbol{I}_n)\circ(\boldsymbol{B}+t\boldsymbol{I}_n)\) 为正定阵. 令 \(t\to0^+\), 即得 \(\boldsymbol{A}\circ\boldsymbol{B}\) 为半正定阵.
\end{proof}

\begin{proposition}\label{proposition:半正定阵的行列式的相关的不等式}
设 \(\boldsymbol{A}\) 是 \(n\) 阶半正定实对称矩阵, \(\boldsymbol{S}\) 是 \(n\) 阶实反对称矩阵, 求证:
\[|\boldsymbol{A}+\boldsymbol{S}|\geq|\boldsymbol{A}|+|\boldsymbol{S}|\geq|\boldsymbol{A}|\geq0\]
\end{proposition}
\begin{proof}
对任意的正实数 \(t\), \(\boldsymbol{A}+t\boldsymbol{I}_n\) 为正定阵, 故由\refproposition{proposition:A+S的行列式的相关结论}可得
\begin{align*}
|\boldsymbol{A}+t\boldsymbol{I}_n+\boldsymbol{S}|&\geq|\boldsymbol{A}+t\boldsymbol{I}_n|+|\boldsymbol{S}|\\
&\geq|\boldsymbol{A}+t\boldsymbol{I}_n|>0
\end{align*}
令\(t\to0^+\),即得结论.
\end{proof}

\begin{definition}[亚半正定]
设 \(\boldsymbol{M}\) 为 \(n\) 阶实矩阵, 若对任意的实列向量 \(\boldsymbol{\alpha}\), 总有 \(\boldsymbol{\alpha}'\boldsymbol{M}\boldsymbol{\alpha}\geq0\), 则称 \(\boldsymbol{M}\) 是\textbf{亚半正定阵}. 
\end{definition}

\begin{theorem}\label{theorem:亚半正定阵的等价条件}
证明下列结论等价:
\begin{enumerate}[(1)]
\item \(\boldsymbol{M}\) 是亚半正定阵;
\item \(\boldsymbol{M}+\boldsymbol{M}'\) 是半正定阵;
\item \(\boldsymbol{M}=\boldsymbol{A}+\boldsymbol{S}\), 其中 \(\boldsymbol{A}\) 是半正定实对称矩阵, \(\boldsymbol{S}\) 是实反对称矩阵.
\end{enumerate} 
\end{theorem}
\begin{remark}
由\refproposition{proposition:半正定阵的行列式的相关的不等式}可知,亚半正定阵 \(\boldsymbol{M}\) 天然满足 \(|\boldsymbol{M}| = |\boldsymbol{A}+\boldsymbol{S}|\geq0\).
\end{remark}
\begin{proof}
与\reftheorem{theorem:亚正定阵的等价条件}的证明完全类似.
\end{proof}

\begin{proposition}\label{proposition:对应矩阵的行列式小于零的二次型必存在小于零的一点}
设 \(f(\boldsymbol{x})=\boldsymbol{x}'\boldsymbol{A}\boldsymbol{x}\) 是 \(n\) 元实二次型,\(n\) 阶实矩阵 \(\boldsymbol{A}\) 未必对称且 \(|\boldsymbol{A}|<0\),求证:必存在一组实数 \(a_1,a_2,\cdots,a_n\),使得 \(f(a_1,a_2,\cdots,a_n)<0\)。
\end{proposition}
\begin{remark}
这个命题是\refproposition{proposition:系数阵的行列式小于零的二次型必存在小于零的一点}的延拓.
\end{remark}
\begin{proof}
用反证法证明。若对任意的 \(\boldsymbol{x}\in\mathbb{R}^n\),\(f(\boldsymbol{x})=\boldsymbol{x}'\boldsymbol{A}\boldsymbol{x}\geq0\),则 \(\boldsymbol{A}\) 是亚半正定阵。由\refproposition{proposition:半正定阵的行列式的相关的不等式}和\reftheorem{theorem:亚半正定阵的等价条件}可知 \(|\boldsymbol{A}|\geq0\),这与假设矛盾。
\end{proof}

\begin{proposition}\label{proposition:半正定阵行列式与主对角乘积的不等式}
设 \(\boldsymbol{A}=(a_{ij})\) 是 \(n\) 阶半正定实对称矩阵,求证:\(|\boldsymbol{A}|\leq a_{11}a_{22}\cdots a_{nn}\),且等号成立当且仅当或者存在某个 \(a_{ii}=0\) 或者 \(\boldsymbol{A}\) 是对角矩阵。 
\end{proposition}
\begin{proof}
对任意的正实数 \(t\),\(\boldsymbol{A}+t\boldsymbol{I}_n\) 为正定阵,故由\refcorollary{corollary:正定阵的行列式的相关不等式}可得
\begin{align*}
|\boldsymbol{A}+t\boldsymbol{I}_n|&\leq(a_{11}+t)(a_{22}+t)\cdots(a_{nn}+t)
\end{align*}
令 \(t\to0^+\),即得不等式。若 \(\boldsymbol{A}\) 是非正定的半正定阵,则由\refcorollary{corollary:可逆的半正定阵必是正定阵}可知 \(|\boldsymbol{A}| = 0\),此时等号成立当且仅当存在某个 \(a_{ii}=0\);若 \(\boldsymbol{A}\) 是正定阵,则由\refcorollary{corollary:正定阵的行列式的相关不等式}可知等号成立当且仅当 \(\boldsymbol{A}\) 是对角矩阵。
\end{proof}

\begin{proposition}\label{proposition:半正定阵的迹与行列式相关不等式}
设 \(\boldsymbol{A},\boldsymbol{B}\) 都是 \(n\) 阶半正定实对称矩阵,求证:\(\frac{1}{n}\mathrm{tr}(\boldsymbol{A}\boldsymbol{B})\geq|\boldsymbol{A}|^{\frac{1}{n}}|\boldsymbol{B}|^{\frac{1}{n}}\),并求等号成立的充要条件。
\end{proposition}
\begin{proof}
设 \(\boldsymbol{C}\) 为 \(n\) 阶实矩阵,使得 \(\boldsymbol{B}=\boldsymbol{C}'\boldsymbol{C}\),则由\hyperref[proposition:半正定阵的相关性质]{命题\ref{proposition:半正定阵的相关性质}(2)}可知 \(\boldsymbol{C}\boldsymbol{A}\boldsymbol{C}'=(a_{ij})\) 仍为半正定阵。注意到 \(\mathrm{tr}(\boldsymbol{A}\boldsymbol{B})=\mathrm{tr}(\boldsymbol{A}\boldsymbol{C}'\boldsymbol{C})=\mathrm{tr}(\boldsymbol{C}\boldsymbol{A}\boldsymbol{C}')=\sum_{i = 1}^{n}a_{ii}\),故由\refproposition{proposition:半正定阵行列式与主对角乘积的不等式}和基本不等式可得
\begin{align*}
|\boldsymbol{A}|^{\frac{1}{n}}|\boldsymbol{B}|^{\frac{1}{n}}=|\boldsymbol{A}|^{\frac{1}{n}}|\boldsymbol{C}'\boldsymbol{C}|^{\frac{1}{n}}=|\boldsymbol{C}\boldsymbol{A}\boldsymbol{C}'|^{\frac{1}{n}}\leq(a_{11}a_{22}\cdots a_{nn})^{\frac{1}{n}}\leq\frac{1}{n}\sum_{i = 1}^{n}a_{ii}=\frac{1}{n}\mathrm{tr}(\boldsymbol{A}\boldsymbol{B})
\end{align*}
由\refproposition{proposition:半正定阵行列式与主对角乘积的不等式}等号成立的充要条件和幂平均不等式等号成立的充要条件($a_{ii}$全相等)可得,上述不等式等号成立的充要条件是以下两种情形之一成立:

(1)存在某个$a_{ii}=0$且$a_{kk}$都相等,即\(a_{11}=a_{22}=\cdots=a_{nn}=0\),此时 \(\mathrm{tr}(\boldsymbol{A}\boldsymbol{B}) = 0\),故由\refproposition{proposition:A为实半正定阵AB=O的充要条件为tr(AB)=0}可知 \(\boldsymbol{A}\boldsymbol{B}=\boldsymbol{O}\);

(2)$\boldsymbol{C}\boldsymbol{A}\boldsymbol{C}'$是对角矩阵且$a_{kk}$都相等,即\(a_{11}=a_{22}=\cdots=a_{nn}=a>0\)且$a_ij=0,i\ne j$,此时
\(\boldsymbol{C}\boldsymbol{A}\boldsymbol{C}'=a\boldsymbol{I}_n\).并且此时\(|\boldsymbol{A}|^{\frac{1}{n}}|\boldsymbol{B}|^{\frac{1}{n}}=\mathrm{tr}(\boldsymbol{A}\boldsymbol{B})=\mathrm{tr}(\boldsymbol{A}\boldsymbol{C}'\boldsymbol{C})=\mathrm{tr}(\boldsymbol{C}\boldsymbol{A}\boldsymbol{C}')=\sum_{i = 1}^{n}a_{ii}=na>0\),因此$|A|,|B|\ne0$,于是此时$A,B$都可逆,从而$C$也可逆.故 \(\boldsymbol{AB}=\boldsymbol{AC}' \boldsymbol{C}=\boldsymbol{C}^{-1}\boldsymbol{CAC}' \boldsymbol{C}=\boldsymbol{C}^{-1}a\boldsymbol{I}_n\boldsymbol{C}=a\boldsymbol{I}_n\)。

综上所述,等号成立的充要条件是 \(\boldsymbol{A}\boldsymbol{B}=k\boldsymbol{I}_n\),其中 \(k\geq0\)。
\end{proof}

\subsection{性质2 若主对角元为零,则同行同列的所有元素都为零}

\begin{proposition}\label{proposition:若主对角元为零,则同行同列的所有元素都为零}
设 \(\boldsymbol{A}=(a_{ij})\) 为 \(n\) 阶半正定实对称矩阵,求证:若 \(a_{ii}=0\),则 \(\boldsymbol{A}\) 的第 \(i\) 行和第 \(i\) 列的所有元素都等于零。
\end{proposition}
\begin{proof}
任取 \(j\neq i\),考虑 \(\boldsymbol{A}\) 的第 \(i,j\) 行和列构成的主子式,由\refproposition{proposition:半正定阵关于顺序主子式的性质}可得
\[
\begin{vmatrix}
a_{ii}&a_{ij}\\
a_{ji}&a_{jj}
\end{vmatrix}=a_{ii}a_{jj}-a_{ij}a_{ji}=-a_{ij}^2\geq0
\]
从而 \(a_{ij}=a_{ji}=0(j\neq i)\),结论得证。
\end{proof}

































\end{document}
\documentclass[../../main.tex]{subfiles}
\graphicspath{{\subfix{../../image/}}} % 指定图片目录,后续可以直接使用图片文件名。

% 例如:
% \begin{figure}[h]
% \centering
% \includegraphics{image-01.01}
% \caption{图片标题}
% \label{fig:image-01.01}
% \end{figure}
% 注意:上述\label{}一定要放在\caption{}之后,否则引用图片序号会只会显示??.

\begin{document}

\section{矩阵与二次型}

\subsection{用矩阵方法来讨论二次型问题}

\begin{lemma}\label{lemma:一般二次型的矩阵形式}
设$A$是$n$阶实对称矩阵,则
\begin{align*}
f\left( x \right) =x' Ax+2a\beta ' x+a^2c=\left( \begin{matrix}
x'&		a\\
\end{matrix} \right) \left( \begin{matrix}
A&		\beta\\
\beta '&		c\\
\end{matrix} \right) \left( \begin{array}{c}
x\\
a\\
\end{array} \right) .
\end{align*}
\end{lemma}
\begin{remark}
上述$f(x)$仍是一个二次型,只不过有1个变量恒为常数而已.
\end{remark}
\begin{proof}
由矩阵乘法易证.
\end{proof}

\begin{example}
设$\boldsymbol{A}$是$n$阶正定实对称矩阵, 求证: 函数$f(\boldsymbol{x})=\boldsymbol{x}'\boldsymbol{A}\boldsymbol{x}+2\boldsymbol{\beta}'\boldsymbol{x}+c$的极小值等于$c - \boldsymbol{\beta}'\boldsymbol{A}^{-1}\boldsymbol{\beta}$, 其中$\boldsymbol{\beta}=(b_1, \cdots, b_n)', b_i$和$c$都是实数.
\end{example}
\begin{proof}
注意到
\[
f(\boldsymbol{x})=(\boldsymbol{x}'\ 1)\begin{pmatrix}
\boldsymbol{A} & \boldsymbol{\beta} \\
\boldsymbol{\beta}' & c
\end{pmatrix}\begin{pmatrix}
\boldsymbol{x} \\
1
\end{pmatrix},
\]
因为$\boldsymbol{A}$可逆, 故可作如下对称分块初等变换:
\begin{align*}
\begin{pmatrix}
\boldsymbol{I}_n & \boldsymbol{O} \\
-\boldsymbol{\beta}'\boldsymbol{A}^{-1} & 1
\end{pmatrix}
\begin{pmatrix}
\boldsymbol{A} & \boldsymbol{\beta} \\
\boldsymbol{\beta}' & c
\end{pmatrix}
\begin{pmatrix}
\boldsymbol{I}_n & -\boldsymbol{A}^{-1}\boldsymbol{\beta} \\
\boldsymbol{O} & 1
\end{pmatrix}
=
\begin{pmatrix}
\boldsymbol{A} & \boldsymbol{O} \\
\boldsymbol{O} & c - \boldsymbol{\beta}'\boldsymbol{A}^{-1}\boldsymbol{\beta}
\end{pmatrix}.
\end{align*}
由$\begin{pmatrix}
\boldsymbol{x} \\
1
\end{pmatrix}=\begin{pmatrix}
\boldsymbol{I}_n & -\boldsymbol{A}^{-1}\boldsymbol{\beta} \\
\boldsymbol{O} & 1
\end{pmatrix}\begin{pmatrix}
\boldsymbol{y} \\
1
\end{pmatrix}$可解出$\boldsymbol{y}=\boldsymbol{x}+\boldsymbol{A}^{-1}\boldsymbol{\beta}$, 于是
\[
f(\boldsymbol{x})=(\boldsymbol{y}'\ 1)\begin{pmatrix}
\boldsymbol{A} & \boldsymbol{O} \\
\boldsymbol{O} & c - \boldsymbol{\beta}'\boldsymbol{A}^{-1}\boldsymbol{\beta}
\end{pmatrix}\begin{pmatrix}
\boldsymbol{y} \\
1
\end{pmatrix}=\boldsymbol{y}'\boldsymbol{A}\boldsymbol{y}+c - \boldsymbol{\beta}'\boldsymbol{A}^{-1}\boldsymbol{\beta}\geq c - \boldsymbol{\beta}'\boldsymbol{A}^{-1}\boldsymbol{\beta}.
\]
因此, 当$\boldsymbol{x}=-\boldsymbol{A}^{-1}\boldsymbol{\beta}$时, $f(\boldsymbol{x})$取到极小值$c - \boldsymbol{\beta}'\boldsymbol{A}^{-1}\boldsymbol{\beta}$. 
\end{proof}


\end{document}
\documentclass[../../main.tex]{subfiles}
\graphicspath{{\subfix{../../image/}}} % 指定图片目录,后续可以直接使用图片文件名。

% 例如:
% \begin{figure}[h]
% \centering
% \includegraphics{image-01.01}
% \caption{图片标题}
% \label{fig:image-01.01}
% \end{figure}
% 注意:上述\label{}一定要放在\caption{}之后,否则引用图片序号会只会显示??.

\begin{document}

\section{多变元二次型的计算}

\begin{example}
化下列实二次型为标准型:
\[f(x_1,x_2,\cdots,x_n)=x_1x_2 + x_2x_3 + \cdots + x_{n - 1}x_n\]
\end{example}
\begin{solution}
{\color{blue}解法一:}

{\color{blue}解法二:}为了方便起见, 不妨考虑 \(2f(x_1,x_2,\cdots,x_n)\) 的系数矩阵
\begin{align*}
\boldsymbol{A} = \begin{pmatrix}
0 & 1 & & & \\
1 & 0 & 1 & & \\
& 1 & \ddots & \ddots & \\
& & \ddots & \ddots & 1 \\
& & & 1 & 0
\end{pmatrix}
\end{align*}
记 \(\boldsymbol{S}_n = \boldsymbol{A}(n\geq2)\), \(\boldsymbol{S}_1\) 为一阶零矩阵, \(\boldsymbol{C}\) 为 \(2\times(n - 2)\) 矩阵, 其中第 \((2,1)\) 元素为 1, 其他元素为 0. 对 \(\boldsymbol{S}_n\) 进行如下分块, 并利用非异阵 \(\boldsymbol{S}_2\) 对称地消去同行同列的矩阵 \(\boldsymbol{C},\boldsymbol{C}'\), 经计算可知 \(\boldsymbol{C}'\boldsymbol{S}_2^{-1}\boldsymbol{C} = \boldsymbol{C}'\boldsymbol{S}_2\boldsymbol{C} = \boldsymbol{O}\), 故 \(\boldsymbol{S}_n\) 合同于下列分块对角矩阵:
\begin{align*}
\boldsymbol{S}_n = \begin{pmatrix}
\boldsymbol{S}_2 & \boldsymbol{C} \\
\boldsymbol{C}' & \boldsymbol{S}_{n - 2}
\end{pmatrix} \to
\begin{pmatrix}
\boldsymbol{S}_2 & \boldsymbol{O} \\
\boldsymbol{O} & \boldsymbol{S}_{n - 2} - \boldsymbol{C}'\boldsymbol{S}_2^{-1}\boldsymbol{C}
\end{pmatrix} \to
\begin{pmatrix}
\boldsymbol{S}_2 & \boldsymbol{O} \\
\boldsymbol{O} & \boldsymbol{S}_{n - 2}
\end{pmatrix}
\end{align*}
因此, 由数学归纳法可知,当 \(n = 2k\) 时, \(\boldsymbol{A}\) 合同于 \(\mathrm{diag}\{\boldsymbol{S}_2,\cdots,\boldsymbol{S}_2\}\), 其中有 \(k\) 个 \(\boldsymbol{S}_2\); 当 \(n = 2k + 1\) 时, \(\boldsymbol{A}\) 合同于 \(\mathrm{diag}\{\boldsymbol{S}_2,\cdots,\boldsymbol{S}_2,\boldsymbol{S}_1\}\), 其中有 \(k\) 个 \(\boldsymbol{S}_2\). 注意到 \(\boldsymbol{S}_2\) 合同于 \(\mathrm{diag}\{1, - 1\}\), 故当 \(n = 2k\) 时, \(f\) 的规范标准型为 \(y_1^2 - y_2^2 + \cdots + y_{n - 1}^2 - y_n^2\); 当 \(n = 2k + 1\) 时, \(f\) 的规范标准型为 \(y_1^2 - y_2^2 + \cdots + y_{n - 2}^2 - y_{n - 1}^2\). 
\end{solution}

\begin{example}
化下列实二次型为标准型:
\[f(x_1,x_2,\cdots,x_n)=\sum_{i = 1}^{n}x_i^2+\sum_{1\leq i < j\leq n}x_ix_j\]
\end{example}
\begin{solution}
{\color{blue}解法一:}

{\color{blue}解法二:}

{\color{blue}解法三:}为了方便起见, 不妨考虑 \(2f(x_1,x_2,\cdots,x_n)\) 的系数矩阵
\begin{align*}
\boldsymbol{A} = \begin{pmatrix}
2 & 1 & 1 & \cdots & 1 \\
1 & 2 & 1 & \cdots & 1 \\
1 & 1 & 2 & \cdots & 1 \\
\vdots & \vdots & \vdots & & \vdots \\
1 & 1 & 1 & \cdots & 2
\end{pmatrix}
\end{align*}
注意到 \(\boldsymbol{A}\) 的第 \(k\) 个顺序主子式 \(|\boldsymbol{A}_k|\) 的每行元素之和都为 \(k + 1\), 故用求和法可求出 \(|\boldsymbol{A}_k| = k + 1(1\leq k\leq n)\), 于是 \(\boldsymbol{A}\) 为正定阵. 因此 \(f(x_1,x_2,\cdots,x_n)\) 为正定型, 其规范标准型为 \(y_1^2 + y_2^2 + \cdots + y_n^2\). 
\end{solution}

\begin{example}\label{example:例8.34}
化下列实二次型为标准型:
\[f(x_1,x_2,\cdots,x_n)=\sum_{i = 1}^{n}(x_i - s)^2, \quad s = \frac{1}{n}(x_1 + x_2 + \cdots + x_n)\]
\end{example}
\begin{proof}
{\color{blue}解法一:}

{\color{blue}解法二:}令 \(y_i = x_i - s(1\leq i\leq n)\), 用矩阵表示就是 \(\boldsymbol{y} = \boldsymbol{A}\boldsymbol{x}\), 其中
\begin{align*}
\boldsymbol{A} = \begin{pmatrix}
\frac{n - 1}{n} & -\frac{1}{n} & \cdots & -\frac{1}{n} \\
-\frac{1}{n} & \frac{n - 1}{n} & \cdots & -\frac{1}{n} \\
\vdots & \vdots & & \vdots \\
-\frac{1}{n} & -\frac{1}{n} & \cdots & \frac{n - 1}{n}
\end{pmatrix}
\end{align*}
注意到 \(\boldsymbol{A}\) 的第 \(k\) 个顺序主子式 \(|\boldsymbol{A}_k|\) 的每行元素之和都为 \((n - k)/n\), 故用求和法可求出 \(|\boldsymbol{A}_k| = (n - k)/n(1\leq k\leq n)\), 因此 \(\boldsymbol{A}\) 的秩等于 \(n - 1\). 由\refpro{proposition:和的平方和的半正定型的秩等于其系数矩阵的秩}可知 \(\mathrm{r}(f)=\mathrm{r}(\boldsymbol{A}) = n - 1\), 于是半正定型 \(f\) 的正惯性指数等于 \(n - 1\), 其规范标准型为 \(z_1^2 + z_2^2 + \cdots + z_{n - 1}^2\).

{\color{blue}解法三:}
显然 $f(\boldsymbol{x})$ 是半正定型,并且由线性方程组$Ax=O$可以解得 $\mathrm{Ker} f(\boldsymbol{x})=\{(c,c,\cdots,c)\mid c\in\mathbb{R}\}$ 的维数等于 $1$(实际上,$\dim \mathrm{Ker} f(\boldsymbol{x})=\mathrm{r}(A)=1$),故由\refpro{proposition:和的平方和的半正定型的秩等于其系数矩阵的秩}可知 $f(\boldsymbol{x})$ 的规范标准型为 $y_1^2 + y_2^2+\cdots + y_{n - 1}^2$.  
\end{proof}

\begin{example}\label{example:例8.35}
化下列实二次型为标准型, 其中 \(a_i\) 都是实数:
\[f(x_1,x_2,\cdots,x_n)=(x_1 - a_1x_2)^2+(x_2 - a_2x_3)^2+\cdots+(x_{n - 1} - a_{n - 1}x_n)^2+(x_n - a_nx_1)^2\]
\end{example}
\begin{proof}
{\color{blue}解法一:}

{\color{blue}解法二:}令 \(y_i = x_i - a_ix_{i + 1}(1\leq i\leq n - 1)\), \(y_n = x_n - a_nx_1\), 用矩阵表示就是 \(\boldsymbol{y} = \boldsymbol{A}\boldsymbol{x}\), 其中
\begin{align*}
\boldsymbol{A} = \begin{pmatrix}
1 & -a_1 & & & \\
& 1 & -a_2 & & \\
& & 1 & \ddots & \\
& & & \ddots & -a_{n - 1} \\
-a_n & & & & 1
\end{pmatrix}
\end{align*}
经计算可知 \(|\boldsymbol{A}| = 1 - a_1a_2\cdots a_n\) 并且 \(\boldsymbol{A}\) 的左上角的 \(n - 1\) 阶子式等于 1, 于是由\refpro{proposition:和的平方和的半正定型的秩等于其系数矩阵的秩}可知, 当 \(a_1a_2\cdots a_n = 1\) 时, \(\mathrm{r}(f)=\mathrm{r}(\boldsymbol{A}) = n - 1\), \(f\) 的规范标准型为 \(z_1^2 + z_2^2 + \cdots + z_{n - 1}^2\); 当 \(a_1a_2\cdots a_n\neq1\) 时, \(\mathrm{r}(f)=\mathrm{r}(\boldsymbol{A}) = n\), \(f\) 的规范标准型为 \(z_1^2 + z_2^2 + \cdots + z_n^2\).

{\color{blue}解法三:}
显然 $f(\boldsymbol{x})$ 是半正定型.解线性方程组$Ax=O$可得,当 $a_1a_2\cdots a_n\neq1$ 时,$\mathrm{Ker} f(\boldsymbol{x}) = \{\boldsymbol{0}\}$,故 $f(\boldsymbol{x})$ 的规范标准型为 $y_1^2 + y_2^2+\cdots + y_n^2$;当 $a_1a_2\cdots a_n = 1$ 时,$\mathrm{Ker} f(\boldsymbol{x})=\{(c,a_2\cdots a_n c,\cdots,a_n c)\mid c\in\mathbb{R}\}$ 的维数等于 $1$(实际上,$\dim \mathrm{Ker} f(\boldsymbol{x})=\mathrm{r}(A)=1$),故由\refpro{proposition:利用核空间求半正定型的规范标准型}可知$f(\boldsymbol{x})$ 的规范标准型为 $y_1^2 + y_2^2+\cdots + y_{n - 1}^2$.
\end{proof}







\end{document}
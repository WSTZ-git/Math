\documentclass[../../main.tex]{subfiles}
\graphicspath{{\subfix{../../image/}}} % 指定图片目录,后续可以直接使用图片文件名。

% 例如:
% \begin{figure}[H]
% \centering
% \includegraphics[scale=0.4]{图.png}
% \caption{}
% \label{figure:图}
% \end{figure}
% 注意:上述\label{}一定要放在\caption{}之后,否则引用图片序号会只会显示??.

\begin{document}

\section{正定型与正定矩阵初步}

\begin{definition}
设 $f(x_1,x_2,\cdots,x_n)=\boldsymbol{x}'\boldsymbol{A}\boldsymbol{x}$ 是 $n$ 元实二次型,$\boldsymbol{A}$ 是相伴矩阵.

(1) 若对任意 $n$ 维非零列向量 $\boldsymbol{\alpha}$ 均有 $\boldsymbol{\alpha}'\boldsymbol{A}\boldsymbol{\alpha}>0$,则称 $f$ 是\textbf{正定二次} (简称\textbf{正定型}),矩阵 $\boldsymbol{A}$ 称为\textbf{正定矩阵} (简称\textbf{正定阵});

(2) 若对任意 $n$ 维非零列向量 $\boldsymbol{\alpha}$ 均有 $\boldsymbol{\alpha}'\boldsymbol{A}\boldsymbol{\alpha}<0$,则称 $f$ 是\textbf{负定二次型} (简称\textbf{负定型}),矩阵 $\boldsymbol{A}$ 称为\textbf{负定矩阵} (简称\textbf{负定阵});

(3) 若对任意 $n$ 维非零列向量 $\boldsymbol{\alpha}$ 均有 $\boldsymbol{\alpha}'\boldsymbol{A}\boldsymbol{\alpha}\geqslant 0$,则称 $f$ 是\textbf{半正定二次型} (简称\textbf{半正定型}),矩阵 $\boldsymbol{A}$ 称为\textbf{半正定矩阵} (简称\textbf{半正定阵});

(4) 若对任意 $n$ 维非零列向量 $\boldsymbol{\alpha}$ 均有 $\boldsymbol{\alpha}'\boldsymbol{A}\boldsymbol{\alpha}\leqslant 0$,则称 $f$ 是\textbf{半负定二次型} (简称\textbf{半负定型}),矩阵 $\boldsymbol{A}$ 称为\textbf{半负定矩阵} (简称\textbf{半负定阵});

(5) 若存在 $\boldsymbol{\alpha}$,使 $\boldsymbol{\alpha}'\boldsymbol{A}\boldsymbol{\alpha}>0$;又存在 $\boldsymbol{\beta}$,使 $\boldsymbol{\beta}'\boldsymbol{A}\boldsymbol{\beta}<0$,则称 $f$ 是\textbf{不定型}.
\end{definition}
\begin{remark}
显然
\begin{align*}
f(x_1,x_2,\cdots,x_n)=x_1^2 + x_2^2 + \cdots + x_n^2
\end{align*}
是正定型,而
\begin{align*}
f(x_1,x_2,\cdots,x_n)=-x_1^2 - x_2^2 - \cdots - x_n^2
\end{align*}
是负定型. 
\end{remark}

\begin{theorem}\label{theorem:正定型的充要条件}
设 $f(x_1,x_2,\cdots,x_n)$ 是 $n$ 元实二次型,则 
\begin{enumerate}[(1)]
\item $f$ 是正定型的充分必要条件是 $f$ 的正惯性指数等于 $n$;

\item $f$ 是负定型的充分必要条件是 $f$ 的负惯性指数等于 $n$; 

\item $f$ 是半正定型的充分必要条件是 $f$ 的正惯性指数等于 $f$ 的秩 $r$; 

\item $f$ 是半负定型的充分必要条件是 $f$ 的负惯性指数等于 $f$ 的秩 $r$.
\end{enumerate}
\end{theorem}
\begin{proof}
只证明(1),其余结论的证明类似. 

若 $f$ 的正惯性指数等于 $n$,则 $f$ 可化为下列标准型:
\begin{align*}
f = y_1^2 + y_2^2 + \cdots + y_n^2,
\end{align*}
显然 $f$ 是正定型. 反之,若 $f$ 是正定型,如果 $f$ 的正惯性指数 $p < n$,则 $f$ 可化为如下标准型:
\begin{align}
f = y_1^2 + \cdots + y_p^2 - c_{p + 1}y_{p + 1}^2 - \cdots - c_ny_n^2,\label{eq:8.4.1}
\end{align}
其中 $c_j\geqslant  0 (j = p + 1,\cdots,n)$. 这时令 $b_1 = \cdots = b_p = 0$, $b_{p + 1} = \cdots = b_n = 1$,则 $b_1,b_2,\cdots,b_n$ 不全为零. 假设这时 $\boldsymbol{x}=\boldsymbol{C}\boldsymbol{y}$,其中
\[
\boldsymbol{x}=\begin{pmatrix}
x_1\\
x_2\\
\vdots\\
x_n
\end{pmatrix},\ \boldsymbol{y}=\begin{pmatrix}
y_1\\
y_2\\
\vdots\\
y_n
\end{pmatrix},
\]
$\boldsymbol{C}$ 是非异阵,则从 $y_i = b_i (i = 1,\cdots,n)$ 可得 $x_i = a_i (i = 1,\cdots,n)$ 是一组不全为零的实数. 于是
\[
f(a_1,a_2,\cdots,a_n)\leqslant  0,
\]
这与 $f$ 是正定型矛盾. 

\end{proof}

\begin{theorem}[(半)正定/负定阵的合同标准型]\label{theorem:正定矩阵的充要条件}
\begin{enumerate}[(1)]
\item $n$ 阶实对称阵 $\boldsymbol{A}$ 是正定阵当且仅当它合同于单位阵 $\boldsymbol{I}_n$;

\item $\boldsymbol{A}$ 是负定阵当且仅当它合同于 $-\boldsymbol{I}_n$;

\item $\boldsymbol{A}$ 是半正定阵当且仅当 $\boldsymbol{A}$ 合同于下列对角阵:
\[
\begin{pmatrix}
\boldsymbol{I}_r & \boldsymbol{O}\\
\boldsymbol{O} & \boldsymbol{O}
\end{pmatrix};
\]

\item $\boldsymbol{A}$ 是半负定阵当且仅当 $\boldsymbol{A}$ 合同于下列对角阵:
\[
\begin{pmatrix}
-\boldsymbol{I}_r & \boldsymbol{O}\\
\boldsymbol{O} & \boldsymbol{O}
\end{pmatrix}.
\] 
\end{enumerate}
\end{theorem}
\begin{proof}
由\refthe{theorem:正定型的充要条件}及\refthe{theorem:实二次型的规范标准型}可立即得到证明.

\end{proof}

\begin{corollary}\label{corollary:可逆的半正定阵必是正定阵}
若矩阵$A$是半正定阵,则$A$是正定阵的充要条件是$A$可逆(非异).
\end{corollary}
\begin{proof}
必要性显然,下证充分性.
由\refthe{theorem:正定矩阵的充要条件}可知$A$合同于$\left( \begin{matrix}
I_r&		O\\
O&		O\\
\end{matrix} \right) $,其中$r$是$A$的秩.又因为$A$可逆,所以$r=n$,故$A$与$I_n$合同.会因此再由\refthe{theorem:正定矩阵的充要条件}可知$A$是正定阵.

\end{proof}

\begin{definition}[顺序主子式]
设 $\boldsymbol{A}=(a_{ij})$ 是 $n$ 阶矩阵,$\boldsymbol{A}$ 的 $n$ 个子式:
\[
\begin{vmatrix}
a_{11} & a_{12} & \cdots & a_{1k}\\
a_{21} & a_{22} & \cdots & a_{2k}\\
\vdots & \vdots & & \vdots\\
a_{k1} & a_{k2} & \cdots & a_{kk}
\end{vmatrix}\ (k = 1,2,\cdots,n)
\]
称为 $\boldsymbol{A}$ 的\textbf{顺序主子式}.
\end{definition}

\begin{theorem}\label{theorem:实正定阵的充要条件}
$n$ 阶实对称阵 $\boldsymbol{A}$ 是正定阵的充分必要条件是它的 $n$ 个顺序主子式全大于零.
\end{theorem}
\begin{proof}
先证必要性. 设 $n$ 阶实对称阵 $\boldsymbol{A}=(a_{ij})$ 为正定阵,则对应的实二次型
\begin{align*}
f(x_1,x_2,\cdots,x_n)=\sum_{j = 1}^n\sum_{i = 1}^n a_{ij}x_ix_j
\end{align*}
为正定型. 令
\[
f_k(x_1,x_2,\cdots,x_k)=\sum_{j = 1}^k\sum_{i = 1}^k a_{ij}x_ix_j,
\]
则对任意一组不全为零的实数 $c_1,c_2,\cdots,c_k$,有
\[
f_k(c_1,c_2,\cdots,c_k)=f(c_1,c_2,\cdots,c_k,0,\cdots,0)>0,
\]
因此 $f_k$ 是一个正定二次型,从而它的相伴矩阵 $\boldsymbol{A}_k$ (由 $\boldsymbol{A}$ 的前 $k$ 行及前 $k$ 列组成) 是一个正定阵. 由于 $\boldsymbol{A}_k$ 合同于 $\boldsymbol{I}_k$,故存在 $k$ 阶非异阵 $\boldsymbol{B}$,使
\[
\boldsymbol{B}'\boldsymbol{A}_k\boldsymbol{B}=\boldsymbol{I}_k,
\]
于是
\[
\det(\boldsymbol{B}'\boldsymbol{A}_k\boldsymbol{B})=\det(\boldsymbol{B})^2\det(\boldsymbol{A}_k)=1,
\]
即有 $\det(\boldsymbol{A}_k)>0 (k = 1,2,\cdots,n)$.

再证充分性. 对 $\boldsymbol{A}$ 的阶数进行归纳. 当 $n = 1$ 时,$\boldsymbol{A}=(a)$,$a>0$,于是 $f = ax_1^2$ 是正定型,从而 $\boldsymbol{A}$ 是正定阵. 设结论对 $n - 1$ 成立,现证明对 $n$ 阶实对称阵 $\boldsymbol{A}$,若它的 $n$ 个顺序主子式全大于零,则 $\boldsymbol{A}$ 必是正定阵. 记 $\boldsymbol{A}_{n - 1}$ 是 $\boldsymbol{A}$ 的 $n - 1$ 阶顺序主子式所在的矩阵,则 $\boldsymbol{A}$ 可写为
\[
\begin{pmatrix}
\boldsymbol{A}_{n - 1} & \boldsymbol{\alpha}\\
\boldsymbol{\alpha}' & a_{nn}
\end{pmatrix}.
\]
因为 $\boldsymbol{A}$ 的顺序主子式全大于零,故 $\boldsymbol{A}_{n - 1}$ 的顺序主子式也全大于零,由归纳假设,$\boldsymbol{A}_{n - 1}$ 是正定阵. 于是 $\boldsymbol{A}_{n - 1}$ 合同于 $n - 1$ 阶单位阵,即存在 $n - 1$ 阶非异阵 $\boldsymbol{B}$,使
\[
\boldsymbol{B}'\boldsymbol{A}_{n - 1}\boldsymbol{B}=\boldsymbol{I}_{n - 1}.
\]
令 $\boldsymbol{C}$ 是下列分块矩阵:
\[
\boldsymbol{C}=\begin{pmatrix}
\boldsymbol{B} & \boldsymbol{O}\\
\boldsymbol{O} & 1
\end{pmatrix},
\]
则
\[
\boldsymbol{C}'\boldsymbol{A}\boldsymbol{C}=\begin{pmatrix}
\boldsymbol{B}' & \boldsymbol{O}\\
\boldsymbol{O} & 1
\end{pmatrix}
\begin{pmatrix}
\boldsymbol{A}_{n - 1} & \boldsymbol{\alpha}\\
\boldsymbol{\alpha}' & a_{nn}
\end{pmatrix}
\begin{pmatrix}
\boldsymbol{B} & \boldsymbol{O}\\
\boldsymbol{O} & 1
\end{pmatrix}=
\begin{pmatrix}
\boldsymbol{I}_{n - 1} & \boldsymbol{B}'\boldsymbol{\alpha}\\
\boldsymbol{\alpha}'\boldsymbol{B} & a_{nn}
\end{pmatrix}.
\]
这是一个实对称阵,其形式为
\[
\boldsymbol{C}'\boldsymbol{A}\boldsymbol{C}=\begin{pmatrix}
1 & \cdots & 0 & c_1\\
\vdots & & \vdots & \vdots\\
0 & \cdots & 1 & c_{n - 1}\\
c_1 & \cdots & c_{n - 1} & a_{nn}
\end{pmatrix},
\]
用第三类初等行及列变换可将上述矩阵化为对角阵. 这相当于对 $\boldsymbol{C}'\boldsymbol{A}\boldsymbol{C}$ 右乘一个非异阵 $\boldsymbol{Q}$ 后,再左乘 $\boldsymbol{Q}'$ 得到一个对角阵,亦即 $\boldsymbol{Q}'\boldsymbol{C}'\boldsymbol{A}\boldsymbol{C}\boldsymbol{Q}$ 等于
\[
\mathrm{diag}\{1,\cdots,1,c\}.
\]
由于 $|\boldsymbol{A}|>0$,故 $c>0$,这就证明了 $\boldsymbol{A}$ 是一个正定阵. 

\end{proof}

\begin{proposition}\label{proposition:正定阵的性质}
\begin{enumerate}[(1)]
\item 若 $\boldsymbol{A}$ 是正定阵,则$\boldsymbol{A}$ 的任一 $k$ 阶主子阵,即由 $\boldsymbol{A}$ 的第 $i_1,i_2,\cdots,i_k$ 行及 $\boldsymbol{A}$ 的第 $i_1,i_2,\cdots,i_k$ 列交点上元素组成的矩阵,必是正定阵;

若 $\boldsymbol{A}$ 是半正定阵,则半正定阵$A$的任一$k$阶主子阵也是半正定阵.

\item 若 $\boldsymbol{A}$ 是正定阵,则$\boldsymbol{A}$ 的所有主子式全大于零,特别,$\boldsymbol{A}$ 的主对角元素全大于零;

\item 若 $\boldsymbol{A}$ 是正定阵,则$\boldsymbol{A}$ 中绝对值最大的元素仅在主对角线上.
\end{enumerate} 
\end{proposition}
\begin{remark}
(2) 用的是变量代换,但是它和矩阵的合同变换(对换行与列)是等价。 
\end{remark}
\begin{proof}
\begin{enumerate}[(1)]
\item 设 $\boldsymbol{A}_k$ 是矩阵 $\boldsymbol{A}$ 的第 $k$ 个顺序主子式所在的矩阵,则 $\boldsymbol{A}_k$ 是实对称阵且其顺序主子式都大于零,因此 $\boldsymbol{A}_k$ 是正定阵.
经过若干次行对换以及相同的列对换,我们不难将 $\boldsymbol{A}$ 的第 $i_1,i_2,\cdots,i_k$ 行及 $i_1,i_2,\cdots,i_k$ 列分别换成第 $1$,$2$,$\cdots$,$k$ 行和第 $1,2$,$\cdots$,$k$ 列. 利用上面的结论即知正定阵的任一$k$阶主子阵是正定阵.

由于$A$的任一$k$阶主子阵的任一主子式都是$A$的主子式,故由\refpro{proposition:半正定阵关于顺序主子式的性质}知结论成立.

\item {\color{blue}证法一:}由(1)的结论,$\boldsymbol{A}$的所有主子式都是正定阵.又$\boldsymbol{A}$的每个主子式都是其自身的顺序主子式,故由\refthe{theorem:实正定阵的充要条件}可知$\boldsymbol{A}$的所有主子式都大于零.因此(2)成立.

{\color{blue}证法二:}设\(M\)是\(A\)的第\(i_1,\cdots,i_k\)行和列交点上的元素组成的主子式。设\(i_{k + 1}<\cdots < i_n\)是\([1,n]\)中去掉\(i_1,\cdots,i_k\)后剩余的指标,对二次型\(f(\boldsymbol{x})=\boldsymbol{x}'A\boldsymbol{x}\)作如下可逆线性变换:
\begin{align*}
y_1&=x_{i_1},\quad\cdots,\quad y_k = x_{i_k},\quad y_j = x_{i_j}\ (k + 1\leqslant  j\leqslant  n).
\end{align*}
于是\(f(\boldsymbol{x})=\boldsymbol{y}'B\boldsymbol{y}\),且\(B\)的第\(k\)个顺序主子式就是\(M\),因为\(B\)正定,故有\(M > 0\)。

\item 用反证法. 假设 $a_{ij}(i\neq j)$ 是 $\boldsymbol{A}$ 的绝对值最大的元素. 根据 (1),我们只需证明由第 $i,j$ 行与第 $i,j$ 列交点上元素组成的矩阵不是正定阵即可. 考虑矩阵 (由 $\boldsymbol{A}$ 的对称性不妨设 $i < j$)
\[
\begin{pmatrix}
a_{ii} & a_{ij}\\
a_{ji} & a_{jj}
\end{pmatrix},
\]
注意到 $a_{ij} = a_{ji}$ 且 $|a_{ij}|\geqslant  a_{ii}$, $|a_{ij}|\geqslant  a_{jj}$,上述矩阵的行列式值 $a_{ii}a_{jj}-a_{ij}^2\leqslant  0$,所以这个矩阵一定不是正定阵. 
\end{enumerate}

\end{proof}











\end{document}
\documentclass[../../main.tex]{subfiles}
\graphicspath{{\subfix{../../image/}}} % 指定图片目录,后续可以直接使用图片文件名。

% 例如:
% \begin{figure}[H]
% \centering
% \includegraphics[scale=0.4]{图.png}
% \caption{}
% \label{figure:图}
% \end{figure}
% 注意:上述\label{}一定要放在\caption{}之后,否则引用图片序号会只会显示??.

\begin{document}

\section{正定型与正定阵}

\subsection{正定型与正定阵}

\begin{proposition}[正定阵的判定准则]\label{proposition:正定阵的判定准则}
设 $A$ 是 $n$ 阶实对称矩阵,则 $A$ 是正定阵的充要条件是以下条件之一:
\begin{enumerate}[(1)]
\item $A$ 合同于单位矩阵 $I_n$ ;

\item 存在非异实矩阵 $C$,使得 $A = C'C$ ;

\item $A$ 的 $n$ 个顺序主子式全大于零;

\item $A$ 的所有主子式全大于零 ;

\item $A$ 的所有特征值全大于零。
\end{enumerate} 
\end{proposition}
\begin{remark}
(2)实际上给出了一种利用非异实阵构造正定阵的方式,即若$A$是非异实矩阵,则$A'A$必是正定阵.
\end{remark}
\begin{proof}
\begin{enumerate}[(1)]
\item 参考\refthe{theorem:正定矩阵的充要条件}.

\item 参考\refpro{proposition:正定和半正定阵的判定准则}.

\item 参考\refthe{theorem:实正定阵的充要条件}.

\item 参考\refpro{proposition:正定阵的性质}和\refthe{theorem:实正定阵的充要条件}.

\item 参考\refcor{corollary:二次型式(半)正定型(负定型)的充要条件}.
\end{enumerate}

\end{proof}

\begin{proposition}\label{proposition:A+S的行列式的相关结论}
设 \(\boldsymbol{A}\) 是 \(n\) 阶正定实对称矩阵, \(\boldsymbol{S}\) 是 \(n\) 阶实反对称矩阵, 求证:
\begin{enumerate}[(1)]
\item \(|\boldsymbol{A}+\boldsymbol{S}|\geqslant |\boldsymbol{A}|+|\boldsymbol{S}|\), 且等号成立当且仅当 \(n\leqslant 2\) 或当 \(n\geqslant 3\) 时, \(\boldsymbol{S}=\boldsymbol{O}\).
\item \(|\boldsymbol{A}+\boldsymbol{S}|\geqslant |\boldsymbol{A}|\), 且等号成立当且仅当 \(\boldsymbol{S}=\boldsymbol{O}\).
\end{enumerate}
\end{proposition}
\begin{proof}
{\color{blue}证法一:}
设 \(\boldsymbol{C}\) 为非异实矩阵, 使得 \(\boldsymbol{C}'\boldsymbol{A}\boldsymbol{C}=\boldsymbol{I}_n\). 注意到问题的条件和结论在同时合同变换 \(\boldsymbol{A}\mapsto\boldsymbol{C}'\boldsymbol{A}\boldsymbol{C}\), \(\boldsymbol{S}\mapsto\boldsymbol{C}'\boldsymbol{S}\boldsymbol{C}\) 下不改变(不等式两边同乘$|C|^2$), 故不妨从一开始就假设 \(\boldsymbol{A}=\boldsymbol{I}_n\) 为合同标准型, 从而由\refpro{proposition:反称矩阵A的I_n+A的行列式相关结论}即得结论. 

{\color{blue}证法二:}由\hyperref[proposition:正定阵与实反称阵可同时合同对角化-例9.114]{正定阵与实反称阵可同时合同对角化}可知,存在可逆矩阵 $C$, 使得
\begin{align*}
C'AC &= I_n,\quad C'SC = \mathrm{diag}\left\{\begin{pmatrix}0 & b_1\\ -b_1 & 0\end{pmatrix},\dots,\begin{pmatrix}0 & b_r\\ -b_r & 0\end{pmatrix},0,\dots,0\right\},
\end{align*}
其中 $b_1,\dots,b_r$ 是非零实数.因此我们有
\begin{align*}
|C'||A + S||C| &= |C'AC + C'SC| = \left| \operatorname{diag}\left\{ \begin{pmatrix} 1 & b_1 \\ -b_1 & 1 \end{pmatrix}, \cdots, \begin{pmatrix} 1 & b_r \\ -b_r & 1 \end{pmatrix}, 1, \cdots, 1 \right\} \right| \\
&= (1 + b_1^2)(1 + b_2^2) \cdots (1 + b_r^2) \geqslant 1 = |C'||A||C|,
\end{align*}
且等号成立的充要条件是\( r = 0 \),这也等价于\( C'SC = O \),即\( S = O \).

\end{proof}

\begin{example}
设$A,B \in \mathbb{R}^{n \times n}$都是反对称矩阵且$A$可逆. 证明:
$$|A^2 - B| \geqslant |A|^2.$$
\end{example}
\begin{proof}
由$A$是实反称阵且$A$可逆知
\begin{align*}
A=-A^T\Longrightarrow |A|=(-1)^n|A|\Longrightarrow n\text{为偶数}.
\end{align*}
由\refpro{proposition:实对称(反称)阵的特征值}知$A$的特征值都是纯虚数,从而$A^2$的特征值都是负数,故$A^2$负定.
因此存在可逆实阵$C$,使得$A^2=-C^TC$.于是
\begin{align*}
|A^2-B|&=|-C^TC-B|=(-1)^n|C^TC+B|\\
&=|C|^2|I_n+(C^{-1})^TBC^{-1}|\\
&=|A|^2|I_n+(C^{-1})^TBC^{-1}|.
\end{align*}
因为$B$是实反称阵,所以由\refpro{proposition:实对称(反称)阵的特征值}知$B$的特征值只能是$0$或纯虚数.又共轭复特征值成对出现,故可不妨设$B$的全体特征值为$0,\cdots,0,\pm a_1\mathrm{i},\cdots,\pm a_k\mathrm{i}$($a_i\in\mathbb{R}$),则
\begin{align*}
|I_n+(C^{-1})^TBC^{-1}|=\prod_{i=1}^k(1+a_i^2)\geqslant1.
\end{align*}
故
\begin{align*}
|A^2-B|=|A|^2|I_n+(C^{-1})^TBC^{-1}|\geqslant|A|^2.
\end{align*}
\end{proof}

\begin{example}
设$A$为实方阵,$A + A^T$为正定矩阵,但$A \ne A^T$,证明:
$$|A + A^T| < |2A|.$$
\end{example}
\begin{proof}
记$R=A+A^T,S=A-A^T$,则$R$是正定阵,$S$是实反称阵. 注意到
\begin{align*}
2A=A+A^T+A-A^T=R+S,
\end{align*}
故只需证
\begin{align*}
|R|<|R+S|.
\end{align*}
由\refpro{proposition:正定和半正定阵的判定准则}知,存在可逆阵$C$,使得$R=C^TC$. 从而
\begin{align*}
|R+S|=|C^TC+S|=|C^TC||I+(C^{-1})^TSC^{-1}|=|R||I+(C^{-1})^TSC^{-1}|.
\end{align*}
因为$S$为实反称阵,所以$(C^{-1})^TSC^{-1}$也为实反称阵. 由\hyperref[proposition:实对称(反称)阵的特征值]{实反称阵的特征值全为$0$或纯虚数},且共轭复特征值成对出现,故可设$I+(C^{-1})^TSC^{-1}$的全体特征值为
\begin{align*}
1,\cdots,1,1\pm a_1\mathrm{i},\cdots,1\pm a_k\mathrm{i},
\end{align*}
其中$a_i\in\mathbb{R}$. 从而
\begin{align*}
|R+S|=|R||I+(C^{-1})^TSC^{-1}|=|R|\prod_{i=1}^k(1+a_i^2)>|R|.
\end{align*}

\end{proof}

\begin{proposition}\label{proposition:正定阵的性质123}
设 \(\boldsymbol{A},\boldsymbol{B}\) 都是 \(n\) 阶正定实对称矩阵, \(c\) 是正实数, 求证:
\begin{enumerate}[(1)]
\item \(\boldsymbol{A}^{-1},\boldsymbol{A}^*,\boldsymbol{A}+\boldsymbol{B},c\boldsymbol{A}\) 都是正定阵;
\item 若 \(\boldsymbol{D}\) 是非异实矩阵, 则 \(\boldsymbol{D}'\boldsymbol{A}\boldsymbol{D}\) 是正定阵;
\item 若 \(\boldsymbol{A}-\boldsymbol{B}\) 是正定阵, 则 \(\boldsymbol{B}^{-1}-\boldsymbol{A}^{-1}\) 也是正定阵.
\end{enumerate}
\end{proposition}
\begin{proof}
\begin{enumerate}[(1)]
\item 利用\hyperref[proposition:正定阵的判定准则]{正定阵的判定准则(2)},由已知存在非异实矩阵 \(\boldsymbol{C}\), 使得 \(\boldsymbol{A}=\boldsymbol{C}'\boldsymbol{C}\), 从而 \(\boldsymbol{A}^{-1}=(\boldsymbol{C}'\boldsymbol{C})^{-1}=\boldsymbol{C}^{-1}(\boldsymbol{C}')^{-1}=\boldsymbol{C}^{-1}(\boldsymbol{C}^{-1})'\), 故 \(\boldsymbol{A}^{-1}\) 是正定阵. 又 \(\boldsymbol{A}^*=(\boldsymbol{C}'\boldsymbol{C})^*=\boldsymbol{C}^*(\boldsymbol{C}')^*=\boldsymbol{C}^*(\boldsymbol{C}^*)'\), 故 \(\boldsymbol{A}^*\) 是正定阵. 对任一非零实列向量 \(\boldsymbol{\alpha}\), \(\boldsymbol{\alpha}'(\boldsymbol{A}+\boldsymbol{B})\boldsymbol{\alpha}=\boldsymbol{\alpha}'\boldsymbol{A}\boldsymbol{\alpha}+\boldsymbol{\alpha}'\boldsymbol{B}\boldsymbol{\alpha}>0\), 从而 \(\boldsymbol{A}+\boldsymbol{B}\) 是正定阵. 注意到, 若 \(\boldsymbol{A}\) 是正定阵, 即使 \(\boldsymbol{B}\) 只是半正定阵, 通过上述方法也能推出 \(\boldsymbol{A}+\boldsymbol{B}\) 是正定阵. 同理可证 \(c\boldsymbol{A}\) 也是正定阵.
\item 由 (1) 相同的记号可得 \(\boldsymbol{D}'\boldsymbol{A}\boldsymbol{D}=\boldsymbol{D}'\boldsymbol{C}'\boldsymbol{C}\boldsymbol{D}=(\boldsymbol{C}\boldsymbol{D})'(\boldsymbol{C}\boldsymbol{D})\), 因为 \(\boldsymbol{C}\boldsymbol{D}\) 是可逆矩阵, 故 \(\boldsymbol{D}'\boldsymbol{A}\boldsymbol{D}\) 是正定阵. 
\item 由\refpro{proposition:B^-1-A^-1用A和B表示}可知 \(\boldsymbol{B}^{-1}-\boldsymbol{A}^{-1}=(\boldsymbol{B}+\boldsymbol{B}(\boldsymbol{A}-\boldsymbol{B})^{-1}\boldsymbol{B})^{-1}\), 再由 (1) 和 (2) 即得 \(\boldsymbol{B}^{-1}-\boldsymbol{A}^{-1}\) 是正定阵. 
\end{enumerate}

\end{proof}

\begin{proposition}\label{proposition:第8章解答题6}
设\(A\)是\(m\)阶正定实对称矩阵,\(B\)是\(m\times n\)实矩阵. 求证:\(B'AB\)是正定阵的充要条件是\(\mathrm{r}(B)=n\).
\end{proposition}
\begin{proof}
由\(A\)的正定性可知,存在可逆阵$C$,使得$A=C'C$,从而\(B'AB=B'C'CB=(CB)'(CB)\)且$CB$是实矩阵,故$B'AB$至少是半正定的,并且\(\boldsymbol{x}'(B'AB)\boldsymbol{x}=(B\boldsymbol{x})'A(B\boldsymbol{x}) = 0\)当且仅当\(B\boldsymbol{x}=\boldsymbol{0}\). 因此,\(B'AB\)是正定阵当且仅当\(B\boldsymbol{x}=\boldsymbol{0}\)只有零解,再由线性方程组的求解理论可知,这也当且仅当\(\mathrm{r}(B)=n\). 

\end{proof}

\begin{proposition}\label{proposition:A正定则H正定}
设 \(\boldsymbol{A}\) 为 \(n\) 阶正定实对称矩阵, \(n\) 维实列向量 \(\boldsymbol{\alpha},\boldsymbol{\beta}\) 满足 \(\boldsymbol{\alpha}'\boldsymbol{\beta}>0\), 求证: \(\boldsymbol{H}=\boldsymbol{A}-\frac{\boldsymbol{A}\boldsymbol{\beta}\boldsymbol{\beta}'\boldsymbol{A}}{\boldsymbol{\beta}'\boldsymbol{A}\boldsymbol{\beta}}+\frac{\boldsymbol{\alpha}\boldsymbol{\alpha}'}{\boldsymbol{\alpha}'\boldsymbol{\beta}}\) 是正定阵.
\end{proposition}
\begin{proof}
根据定义只要证明对任一实列向量 \(\boldsymbol{x}\), 均有 \(\boldsymbol{x}'\boldsymbol{H}\boldsymbol{x}\geqslant 0\), 且等号成立当且仅当 \(\boldsymbol{x}=\boldsymbol{0}\) 即可. 一方面, 由 \(\boldsymbol{\alpha}'\boldsymbol{\beta}>0\) 可知, \(\frac{\boldsymbol{x}'(\boldsymbol{\alpha}\boldsymbol{\alpha}')\boldsymbol{x}}{\boldsymbol{\alpha}'\boldsymbol{\beta}}=\frac{(\boldsymbol{\alpha}'\boldsymbol{x})^2}{\boldsymbol{\alpha}'\boldsymbol{\beta}}\geqslant 0\), 等号成立当且仅当 \(\boldsymbol{\alpha}'\boldsymbol{x}=0\). 另一方面, 由 \(\boldsymbol{A}\) 正定可知, 存在非异实矩阵 \(\boldsymbol{C}\), 使得 \(\boldsymbol{A}=\boldsymbol{C}'\boldsymbol{C}\). 设 \(\boldsymbol{C}\boldsymbol{\beta}=(b_1,b_2,\cdots,b_n)'\), \(\boldsymbol{C}\boldsymbol{x}=(x_1,x_2,\cdots,x_n)'\), 则由 Cauchy - Schwarz 不等式可知
\begin{align*}
&\boldsymbol{x}'\boldsymbol{A}\boldsymbol{x}-\frac{\boldsymbol{x}'\boldsymbol{A}\boldsymbol{\beta}\boldsymbol{\beta}'\boldsymbol{A}\boldsymbol{x}}{\boldsymbol{\beta}'\boldsymbol{A}\boldsymbol{\beta}}=(\boldsymbol{C}\boldsymbol{x})'(\boldsymbol{C}\boldsymbol{x})-\frac{(\boldsymbol{C}\boldsymbol{x})'(\boldsymbol{C}\boldsymbol{\beta})(\boldsymbol{C}\boldsymbol{\beta})'(\boldsymbol{C}\boldsymbol{x})}{(\boldsymbol{C}\boldsymbol{\beta})'(\boldsymbol{C}\boldsymbol{\beta})}\\
&=\left(\sum_{i = 1}^{n}b_i^2\right)^{-1}\left(\left(\sum_{i = 1}^{n}b_i^2\right)\left(\sum_{i = 1}^{n}x_i^2\right)-\left(\sum_{i = 1}^{n}b_ix_i\right)^2\right)\geqslant 0
\end{align*}
等号成立当且仅当 \(b_i\) 与 \(x_i\) 成比例, 即存在实数 \(k\), 使得 \(\boldsymbol{C}\boldsymbol{x}=k\boldsymbol{C}\boldsymbol{\beta}\), 即 \(\boldsymbol{x}=k\boldsymbol{\beta}\). 由上述计算可得 \(\boldsymbol{x}'\boldsymbol{H}\boldsymbol{x}\geqslant 0\), 且等号成立当且仅当 \(\boldsymbol{\alpha}'\boldsymbol{x}=0\) 且 \(\boldsymbol{x}=k\boldsymbol{\beta}\), 再由 \(\boldsymbol{\alpha}'\boldsymbol{\beta}>0\) 可得 \(k = 0\), 从而 \(\boldsymbol{x}=\boldsymbol{0}\), 结论得证. 

\end{proof}

\begin{example}\label{example:例8.48}
求证: 下列 \(n\) 阶实对称矩阵 \(\boldsymbol{A}=(a_{ij})\) 都是正定阵, 其中
\begin{enumerate}[(1)]
\item \(a_{ij}=\frac{1}{i + j}\);
\item \(a_{ij}=\frac{1}{i + j - 1}\);
\item \(a_{ij}=\frac{1}{i + j + 1}\).
\end{enumerate}
\end{example}
\begin{proof}
\begin{enumerate}[(1)]
\item 注意到 \(\boldsymbol{A}\) 的 \(n\) 个顺序主子式都是具有相同形状的 Cauchy 行列式, 故要证明它们全大于零, 只要证明 \(\boldsymbol{A}\) 的行列式大于零即可. 对$A$的所有$m$阶顺序主子式,在\hyperref[Cauchy行列式]{Cauchy行列式}中, 令 \(a_i = b_i = i(1\leqslant  i\leqslant  m)\), 则由\hyperref[Cauchy行列式]{Cauchy行列式}可得
\begin{align*}
|A|=\left| \begin{matrix}
(1+1)^{-1}&		(1+2)^{-1}&		\cdots&		(1+m)^{-1}\\
(2+1)^{-1}&		(2+2)^{-1}&		\cdots&		(2+m)^{-1}\\
\vdots&		\vdots&		&		\vdots\\
(m+1)^{-1}&		(m+2)^{-1}&		\cdots&		(m+m)^{-1}\\
\end{matrix} \right|=\frac{\prod\limits_{1\leqslant i<j\leqslant m}{\left( j-i \right) ^2}}{\prod\limits_{1\leqslant i<j\leqslant m}{\left( i+j \right)}}>0.
\end{align*}
故$A$是正定矩阵.

\item 考虑$A$的所有$m$阶顺序主子式,在\hyperref[Cauchy行列式]{Cauchy行列式}中, 令 \(a_i = b_i = i-\frac{1}{2}(1\leqslant  i\leqslant  m)\), 则由\hyperref[Cauchy行列式]{Cauchy行列式}可得
\begin{align*}
|A|=\left| \begin{matrix}
(1-\frac{1}{2}+1-\frac{1}{2})^{-1}&		(1-\frac{1}{2}+2-\frac{1}{2})^{-1}&		\cdots&		(1-\frac{1}{2}+m-\frac{1}{2})^{-1}\\
(2-\frac{1}{2}+1-\frac{1}{2})^{-1}&		(2-\frac{1}{2}+2-\frac{1}{2})^{-1}&		\cdots&		(2-\frac{1}{2}+m-\frac{1}{2})^{-1}\\
\vdots&		\vdots&		&		\vdots\\
(m-\frac{1}{2}+1-\frac{1}{2})^{-1}&		(m-\frac{1}{2}+2-\frac{1}{2})^{-1}&		\cdots&		(m-\frac{1}{2}+m-\frac{1}{2})^{-1}\\
\end{matrix} \right|=\frac{\prod\limits_{1\leqslant i<j\leqslant m}{\left( j-i \right) ^2}}{\prod\limits_{1\leqslant i<j\leqslant m}{\left( i+j-1 \right)}}>0.
\end{align*}
故\(\boldsymbol{A}\) 为正定阵.

\item 考虑$A$的所有$m$阶顺序主子式,在\hyperref[Cauchy行列式]{Cauchy行列式}中, 令 \(a_i = b_i = i+\frac{1}{2}(1\leqslant  i\leqslant  m)\), 则由\hyperref[Cauchy行列式]{Cauchy行列式}可得
\begin{align*}
|A|=\left| \begin{matrix}
(1+\frac{1}{2}+1+\frac{1}{2})^{-1}&		(1+\frac{1}{2}+2+\frac{1}{2})^{-1}&		\cdots&		(1+\frac{1}{2}+m+\frac{1}{2})^{-1}\\
(2+\frac{1}{2}+1+\frac{1}{2})^{-1}&		(2+\frac{1}{2}+2+\frac{1}{2})^{-1}&		\cdots&		(2+\frac{1}{2}+m+\frac{1}{2})^{-1}\\
\vdots&		\vdots&		&		\vdots\\
(m+\frac{1}{2}+1+\frac{1}{2})^{-1}&		(m+\frac{1}{2}+2+\frac{1}{2})^{-1}&		\cdots&		(m+\frac{1}{2}+m+\frac{1}{2})^{-1}\\
\end{matrix} \right|=\frac{\prod\limits_{1\leqslant i<j\leqslant m}{\left( j-i \right) ^2}}{\prod\limits_{1\leqslant i<j\leqslant m}{\left( i+j+1 \right)}}>0.
\end{align*}
故\(\boldsymbol{A}\) 为正定阵.
\end{enumerate} 

\end{proof}

\begin{proposition}\label{proposition:主对角元全大于零的严格对角占优阵必正定}
设 \(\boldsymbol{A}\) 是 \(n\) 阶实对称矩阵, 求证: 若 \(\boldsymbol{A}\) 是主对角元全大于零的严格对角占优阵, 则 \(\boldsymbol{A}\) 是正定阵.
\end{proposition}
\begin{proof}
注意到 \(\boldsymbol{A}\) 的 \(n\) 个顺序主子阵仍然是主对角元全大于零的严格对角占优阵, 故要证明 \(\boldsymbol{A}\) 的 \(n\) 个顺序主子式全大于零, 只要证明 \(\boldsymbol{A}\) 的行列式大于零即可, 而这由\refpro{proposition:更严格对角占优阵行列式必大于零}即得, 因此 \(\boldsymbol{A}\) 是正定阵. 

\end{proof}

\begin{proposition}\label{proposition:例8.50}
设 \(A\) 是 \(n\) 阶实对称矩阵,求证:必存在正实数 \(k\),使得对任一 \(n\) 维实列向量 \(\alpha\),总有
\begin{align*}
-k\alpha'\alpha \leqslant  \alpha'A\alpha \leqslant  k\alpha'\alpha
\end{align*}
\end{proposition}
\begin{proof}
设 \(A = (a_{ij})\),我们总可以取到充分大的正实数 \(k\),使得
\begin{align*}
k \pm a_{ii} > \sum_{j = 1,j\neq i}^{n}|a_{ij}|,\quad 1\leqslant  i \leqslant  n
\end{align*}
即 \(kI_n \pm A\) 是主对角元全大于零的严格对角占优阵,由\refpro{proposition:主对角元全大于零的严格对角占优阵必正定}可得 \(kI_n \pm A\) 为正定阵,从而对任一 \(n\) 维实列向量 \(\alpha\),总有 \(\alpha'(kI_n \pm A)\alpha \geqslant  0\),从而结论得证。 

\end{proof}

\begin{proposition}
设 \(\boldsymbol{\alpha},\boldsymbol{\beta}\) 为 \(n\) 维非零实列向量, 求证: \(\boldsymbol{\alpha}'\boldsymbol{\beta}>0\) 成立的充要条件是存在 \(n\) 阶正定实对称矩阵 \(\boldsymbol{A}\), 使得 \(\boldsymbol{\alpha}=\boldsymbol{A}\boldsymbol{\beta}\).
\end{proposition}
\begin{proof}
先证充分性. 若存在 \(n\) 阶正定实对称矩阵 \(\boldsymbol{A}\), 使得 \(\boldsymbol{\alpha}=\boldsymbol{A}\boldsymbol{\beta}\), 则 \(\boldsymbol{\alpha}'\boldsymbol{\beta}=(\boldsymbol{A}\boldsymbol{\beta})'\boldsymbol{\beta}=\boldsymbol{\beta}'\boldsymbol{A}\boldsymbol{\beta}>0\). 下面用两种方法来证明必要性.

{\color{blue}证法一:}注意到问题的条件和结论在矩阵变换 \(\boldsymbol{A}\mapsto\boldsymbol{C}'\boldsymbol{A}\boldsymbol{C}\), \(\boldsymbol{\alpha}\mapsto\boldsymbol{C}'\boldsymbol{\alpha}\), \(\boldsymbol{\beta}\mapsto\boldsymbol{C}^{-1}\boldsymbol{\beta}\) 下不改变, 故不妨从一开始就假设 \(\boldsymbol{\beta}=\boldsymbol{e}_n=(0,\cdots,0,1)'\) (这等价于将原来的 \(\boldsymbol{\beta}\) 放在非异阵 \(\boldsymbol{C}\) 的最后一列), \(\boldsymbol{\alpha}=(a_1,\cdots,a_{n - 1},a_n)'\), 则 \(\boldsymbol{\alpha}'\boldsymbol{\beta}>0\) 等价于 \(a_n>0\). 设 \(\boldsymbol{A}=\begin{pmatrix}t\boldsymbol{I}_{n - 1}&\boldsymbol{\alpha}_{n - 1}\\\boldsymbol{\alpha}_{n - 1}'&a_n\end{pmatrix}\), 其中 \(\boldsymbol{\alpha}_{n - 1}=(a_1,\cdots,a_{n - 1})'\) 且 \(t\gg0\), 则由行列式的降阶公式可得
\[|\boldsymbol{A}|=|t\boldsymbol{I}_{n - 1}|(a_n-\boldsymbol{\alpha}_{n - 1}'(t\boldsymbol{I}_{n - 1})^{-1}\boldsymbol{\alpha}_{n - 1})=t^{n - 2}(a_nt - a_1^2-\cdots - a_{n - 1}^2)>0\]
又 \(\boldsymbol{A}\) 的前 \(n - 1\) 个顺序主子式都大于零, 故 \(\boldsymbol{A}\) 为正定阵且满足 \(\boldsymbol{\alpha}=\boldsymbol{A}\boldsymbol{e}_n=\boldsymbol{A}\boldsymbol{\beta}\).

{\color{blue}证法二:} 设 \(\boldsymbol{A}=\boldsymbol{I}_n-\frac{\boldsymbol{\beta}\boldsymbol{\beta}'}{\boldsymbol{\beta}'\boldsymbol{\beta}}+\frac{\boldsymbol{\alpha}\boldsymbol{\alpha}'}{\boldsymbol{\alpha}'\boldsymbol{\beta}}\), 则由\refpro{proposition:A正定则H正定}可知 \(\boldsymbol{A}\) 为正定阵. 不难验证 \(\boldsymbol{A}\boldsymbol{\beta}=\boldsymbol{\alpha}\) 成立, 故结论得证.

\end{proof}

\begin{proposition}\label{proposition:两个正定阵和非异阵的关系}
设 \(\boldsymbol{A},\boldsymbol{B}\) 是 \(n\) 阶实矩阵, 使得 \(\boldsymbol{A}'\boldsymbol{B}'+\boldsymbol{B}\boldsymbol{A}\) 是正定阵, 求证: \(\boldsymbol{A},\boldsymbol{B}\) 都是非异阵.
\end{proposition}
\begin{proof}
用反证法证明. 若 \(\boldsymbol{A}\) 为奇异阵, 则存在非零实列向量 \(\boldsymbol{\alpha}\), 使得 \(\boldsymbol{A}\boldsymbol{\alpha}=\boldsymbol{0}\). 将正定阵 \(\boldsymbol{A}'\boldsymbol{B}'+\boldsymbol{B}\boldsymbol{A}\) 左乘 \(\boldsymbol{\alpha}'\), 右乘 \(\boldsymbol{\alpha}\) 可得
\begin{align*}
0<\boldsymbol{\alpha}'(\boldsymbol{A}'\boldsymbol{B}'+\boldsymbol{B}\boldsymbol{A})\boldsymbol{\alpha}=(\boldsymbol{A}\boldsymbol{\alpha})'(\boldsymbol{B}'\boldsymbol{\alpha})+(\boldsymbol{B}'\boldsymbol{\alpha})'(\boldsymbol{A}\boldsymbol{\alpha})= 0
\end{align*}
这就导出了矛盾. 同理可证 \(\boldsymbol{B}\) 也是非异阵. 

\end{proof}

\begin{example}
设 \(\boldsymbol{A},\boldsymbol{B},\boldsymbol{C}\) 都是 \(n\) 阶正定实对称矩阵, \(g(t)=|t^2\boldsymbol{A}+t\boldsymbol{B}+\boldsymbol{C}|\) 是关于 \(t\) 的多项式, 求证: \(g(t)\) 所有复根的实部都小于零.
\end{example}
\begin{remark}
若 \(\boldsymbol{A}\) 是正定实对称矩阵, 则 \(\boldsymbol{A}\) 合同于单位矩阵 \(\boldsymbol{I}_n\), 即存在非异实矩阵 \(\boldsymbol{C}\), 使得 \(\boldsymbol{A}=\boldsymbol{C}'\boldsymbol{I}_n\boldsymbol{C}\). 因为 \(\boldsymbol{C}\) 是实矩阵, 故可把上式中的 \(\boldsymbol{C}'\) 改写成 \(\overline{\boldsymbol{C}}'\), 从而 \(\boldsymbol{A}\) 复相合于 \(\boldsymbol{I}_n\), 于是 \(\boldsymbol{A}\) 也是正定 Hermite 矩阵. 因此在处理实矩阵问题的过程中, 如果遇到了复特征值和复特征向量, 那么可以自然地把正定实对称矩阵看成是一种特殊的正定 Hermite 矩阵, 从而其正定性可延拓到复数域上. 
\end{remark}
\begin{proof}
任取 \(g(t)\) 的一个复根 \(t_0\), 则 \(|t_0^2\boldsymbol{A}+t_0\boldsymbol{B}+\boldsymbol{C}| = 0\), 故存在非零复列向量 \(\boldsymbol{\alpha}\), 使得 \((t_0^2\boldsymbol{A}+t_0\boldsymbol{B}+\boldsymbol{C})\boldsymbol{\alpha}=\boldsymbol{0}\). 将上述等式左乘 \(\overline{\boldsymbol{\alpha}}'\), 可得
\[(\overline{\boldsymbol{\alpha}}'\boldsymbol{A}\boldsymbol{\alpha})t_0^2+(\overline{\boldsymbol{\alpha}}'\boldsymbol{B}\boldsymbol{\alpha})t_0+(\overline{\boldsymbol{\alpha}}'\boldsymbol{C}\boldsymbol{\alpha}) = 0\]
注意到 \(\boldsymbol{A},\boldsymbol{B},\boldsymbol{C}\) 也是正定 Hermite 矩阵, 故 \(a=\overline{\boldsymbol{\alpha}}'\boldsymbol{A}\boldsymbol{\alpha}>0\), \(b=\overline{\boldsymbol{\alpha}}'\boldsymbol{B}\boldsymbol{\alpha}>0\), \(c=\overline{\boldsymbol{\alpha}}'\boldsymbol{C}\boldsymbol{\alpha}>0\), 并且 \(t_0\) 是二次方程 \(at^2 + bt + c = 0\) 的根. 若 \(t_0\) 是实根, 则 \(t_0<0\), 否则将由 \(t_0\geqslant 0\) 得到 \(at_0^2+bt_0 + c\geqslant  c>0\), 这就推出了矛盾. 若 \(t_0\) 是虚根, 则 \(t_0\) 的实部为 \(-\frac{b}{2a}<0\), 结论得证. 

\end{proof}

\begin{proposition}\label{proposition:关于正定阵的行列式的不等式}
设 \(\boldsymbol{A}=(a_{ij})\) 是 \(n\) 阶正定实对称矩阵, \(P_{n - 1}\) 是 \(\boldsymbol{A}\) 的第 \(n - 1\) 个顺序主子式, 求证: \(|\boldsymbol{A}|\leqslant  a_{nn}P_{n - 1}\).
\end{proposition}
\begin{proof}
{\color{blue}证法一:}
设 \(\boldsymbol{A}=\begin{pmatrix}\boldsymbol{A}_{n - 1}&\boldsymbol{\alpha}\\\boldsymbol{\alpha}'&a_{nn}\end{pmatrix}\), 用第三类分块初等变换求得
\begin{align*}
|\boldsymbol{A}|&=\begin{vmatrix}\boldsymbol{A}_{n - 1}&\boldsymbol{\alpha}\\\boldsymbol{\alpha}'&a_{nn}\end{vmatrix}=\begin{vmatrix}\boldsymbol{A}_{n - 1}&\boldsymbol{\alpha}\\\boldsymbol{O}&a_{nn}-\boldsymbol{\alpha}'\boldsymbol{A}_{n - 1}^{-1}\boldsymbol{\alpha}\end{vmatrix}=(a_{nn}-\boldsymbol{\alpha}'\boldsymbol{A}_{n - 1}^{-1}\boldsymbol{\alpha})|\boldsymbol{A}_{n - 1}|
\end{align*}
因为 \(\boldsymbol{A}\) 正定, 所以 \(\boldsymbol{A}_{n - 1}\) 也正定, 从而 \(\boldsymbol{A}_{n - 1}^{-1}\) 也正定, 于是 \(\boldsymbol{\alpha}'\boldsymbol{A}_{n - 1}^{-1}\boldsymbol{\alpha}\geqslant 0\). 因此
\[|\boldsymbol{A}|=(a_{nn}-\boldsymbol{\alpha}'\boldsymbol{A}_{n - 1}^{-1}\boldsymbol{\alpha})|\boldsymbol{A}_{n - 1}|\leqslant  a_{nn}|\boldsymbol{A}_{n - 1}|=a_{nn}P_{n - 1}\]

{\color{blue}证法二:}由行列式性质,有
\[
|\boldsymbol{A}|=\begin{vmatrix}
a_{11}&\cdots&a_{1,n - 1}&a_{1n}\\
a_{21}&\cdots&a_{2,n - 1}&a_{2n}\\
\vdots&\vdots&\vdots&\vdots\\
a_{n - 1,1}&\cdots&a_{n - 1,n - 1}&a_{n - 1,n}\\
a_{n1}&\cdots&a_{n,n - 1}&0
\end{vmatrix}+\begin{vmatrix}
a_{11}&\cdots&a_{1,n - 1}&0\\
a_{21}&\cdots&a_{2,n - 1}&0\\
\vdots&\vdots&\vdots&\vdots\\
a_{n - 1,1}&\cdots&a_{n - 1,n - 1}&0\\
a_{n1}&\cdots&a_{n,n - 1}&a_{nn}
\end{vmatrix}
\]
令
\[
g(x_1,x_2,\cdots,x_{n - 1})=\begin{vmatrix}
a_{11}&\cdots&a_{1,n - 1}&x_1\\
a_{21}&\cdots&a_{2,n - 1}&x_2\\
\vdots&\vdots&\vdots\\
a_{n - 1,1}&\cdots&a_{n - 1,n - 1}&x_{n - 1}\\
x_1&\cdots&x_{n - 1}&0
\end{vmatrix}
\]
则
\[
|\boldsymbol{A}| = g(a_{1n},a_{2n},\cdots,a_{n - 1,n})+a_{nn}P_{n - 1}
\]
因为 \(\boldsymbol{A}\) 的第 \(n - 1\) 个顺序主子阵是正定阵,故由\refpro{proposition:利用正定阵构造负定二次型}可知 \(g(a_{1n},a_{2n},\cdots,a_{n - 1,n})\leqslant 0\),从而 \(|\boldsymbol{A}|\leqslant  a_{nn}P_{n - 1}\).

\end{proof}

\begin{corollary}\label{corollary:正定阵的行列式的相关不等式}
设 \(\boldsymbol{A}=(a_{ij})\) 是 \(n\) 阶正定实对称矩阵, 求证: \(|\boldsymbol{A}|\leqslant  a_{11}a_{22}\cdots a_{nn}\), 且等号成立当且仅当 \(\boldsymbol{A}\) 是对角矩阵.
\end{corollary}
\begin{proof}
设 \(\boldsymbol{A}=\begin{pmatrix}\boldsymbol{A}_{n - 1}&\boldsymbol{\alpha}\\\boldsymbol{\alpha}'&a_{nn}\end{pmatrix}\),则由\refpro{proposition:关于正定阵的行列式的不等式}可得 \(|\boldsymbol{A}|\leqslant  a_{nn}P_{n - 1}\), 且等号成立当且仅当 \(\boldsymbol{\alpha}=\boldsymbol{0}\).又$A_{n-1}$也是正定阵,故不断迭代下去,可得
\[|\boldsymbol{A}|\leqslant  a_{nn}P_{n - 1}\leqslant  a_{n - 1,n - 1}a_{nn}P_{n - 2}\leqslant \cdots\leqslant  a_{11}a_{22}\cdots a_{nn}\]
且等号成立当且仅当 \(\boldsymbol{A}\) 是对角矩阵. 

\end{proof}

\begin{proposition}[Fischer不等式]\label{proposition:Fischer不等式}
设$A \in \mathbb{R}^{p \times p}, D \in \mathbb{R}^{q \times q}$,考虑半正定矩阵$M = \begin{pmatrix} A & B \\ B^T & D \end{pmatrix}$,则
\begin{align*}
|M| \leqslant |A||D|.
\end{align*}
且等号成立当且仅当$B=O.$
\end{proposition}
\begin{proof}
{\color{blue}证法一:}
因为$M$半正定,所以所有主子式非负,因此$A,D$半正定。考虑$M + tI_{p+q}, A + tI_p, B + tI_q, t > 0$并让$t \to 0^+$可以不妨设$A,D$都是正定的。
利用正定矩阵存在正定平方根,注意到
\begin{align*}
W = \begin{pmatrix} A^{-1/2} & 0 \\ 0 & D^{-1/2} \end{pmatrix} \begin{pmatrix} A & B \\ B^T & D \end{pmatrix} \begin{pmatrix} A^{-1/2} & 0 \\ 0 & D^{-1/2} \end{pmatrix} = \begin{pmatrix} I_p & A^{-1/2}BD^{-1/2} \\ D^{-1/2}B^T A^{-1/2} & I_q \end{pmatrix},
\end{align*}
我们有
\begin{align*}
\frac{\det M}{\det A \cdot \det D} = \det W \overset{\text{均值不等式}}{\underset{\text{特征值大于0}}{\leqslant}} \left( \frac{\text{tr}(W)}{p+q} \right)^{p+q} = 1,
\end{align*}
现在就有
\begin{align*}
\det M \leqslant \det A \cdot \det D.
\end{align*}
且等号成立当且仅当$B=O$.

{\color{blue}证法二:}
因为$M$半正定,所以所有主子式非负,因此$A,D$半正定。考虑$M + tI_{p+q}, A + tI_p, B + tI_q, t > 0$并让$t \to 0^+$可以不妨设$A,D$都是正定的。从而存在非异实矩阵 \(C_1,C_2\), 使得 \(C_1'AC_1=I_r\), \(C_2'DC_2=I_{n - r}\). 令 \(C=\text{diag}\{C_1,C_2\}\), 则
\begin{align*}
C'MC&=\begin{pmatrix}C_1'AC_1&C_1'BC_2\\C_2'B'C_1&C_2'DC_2\end{pmatrix}=\begin{pmatrix}I_r&C_1'BC_2\\C_2'B'C_1&I_{n - r}\end{pmatrix}
\end{align*}
仍是正定阵. 由\refcor{corollary:正定阵的行列式的相关不等式}可得 \(|C'MC|\leqslant 1\), 且等号成立当且仅当 \(C_1'BC_2=O\), 即 \(|M|\leqslant |C|^{-2}=|C_1|^{-2}|C_2|^{-2}=|A||D|\), 且等号成立当且仅当 \(B=O\). 

{\color{blue}证法三:}因为$M$半正定,所以所有主子式非负,因此$A,D$半正定。考虑$M + tI_{p+q}, A + tI_p, B + tI_q, t > 0$并让$t \to 0^+$可以不妨设$A,D$都是正定的。从而可对题中矩阵进行下列对称分块初等变换:
\begin{align*}
\begin{pmatrix}A&B\\B'&D\end{pmatrix}\to\begin{pmatrix}A&B\\O&D - B'A^{-1}B\end{pmatrix}\to\begin{pmatrix}A&O\\O&D - B'A^{-1}B\end{pmatrix},
\end{align*}
得到的矩阵仍正定,从而\(D - B'A^{-1}B\)是正定阵. 因为第三类分块初等变换不改变行列式的值,故
\begin{align*}
\begin{vmatrix}A&B\\B'&D\end{vmatrix}=\vert A\vert\vert D - B'A^{-1}B\vert.
\end{align*}
注意到\(D=(D - B'A^{-1}B)+B'A^{-1}B\),其中\(B'A^{-1}B\)是半正定阵,故由\refpro{proposition:例9.76}可得
\begin{align*}
\vert D\vert\geqslant \vert D - B'A^{-1}B\vert+\vert B'A^{-1}B\vert\geqslant \vert D - B'A^{-1}B\vert,
\end{align*}
上述不等式的两个等号都成立当且仅当\(B'A^{-1}B = O\). 由\(O = B'A^{-1}B=(A^{-\frac{1}{2}}B)'(A^{-\frac{1}{2}}B)\)取迹后可得\(A^{-\frac{1}{2}}B = O\),从而\(B = O\),于是上述不等式的两个等号都成立当且仅当\(B = O\). 综上所述,我们有
\begin{align*}
\begin{vmatrix}A&B\\B'&D\end{vmatrix}=\vert A\vert\vert D - B'A^{-1}B\vert\leqslant \vert A\vert\vert D\vert,
\end{align*}
等号成立当且仅当\(B = O\). 

\end{proof}

\begin{proposition}
设 \(\boldsymbol{A}\) 是 \(n\) 阶实矩阵, \(\boldsymbol{A}=(\boldsymbol{B},C)\) 是 \(\boldsymbol{A}\) 的一个分块, 其中 \(\boldsymbol{B}\) 是 \(\boldsymbol{A}\) 的前 \(k\) 列组成的矩阵, \(C\) 是 \(\boldsymbol{A}\) 的后 \(n - k\) 列组成的矩阵. 求证:
\[|\boldsymbol{A}|^2\leqslant |\boldsymbol{B}'\boldsymbol{B}||C'C|\]
\end{proposition}
\begin{proof}
若 \(\boldsymbol{A}\) 不是可逆矩阵, 则 \(|\boldsymbol{A}| = 0\), 从而由\hyperref[proposition:正定和半正定阵的判定准则]{命题\ref{proposition:正定和半正定阵的判定准则}(2)}可知 \(\boldsymbol{B}'\boldsymbol{B},\boldsymbol{C}'\boldsymbol{C}\) 都是半正定阵, 故由\hyperref[proposition:正定和半正定阵的判定准则]{命题\ref{proposition:正定和半正定阵的判定准则}(2)}可得 \(|\boldsymbol{B}'\boldsymbol{B}|\geqslant 0\), \(|\boldsymbol{C}'\boldsymbol{C}|\geqslant 0\), 从而上式显然成立. 现设 \(\boldsymbol{A}\) 是可逆矩阵, 则由\hyperref[proposition:正定和半正定阵的判定准则]{命题\ref{proposition:正定和半正定阵的判定准则}(1)}可知
\begin{align*}
\boldsymbol{A}' \boldsymbol{A}=\left( \begin{array}{c}
\boldsymbol{B}'\\
\boldsymbol{C}'\\
\end{array} \right) \left( \begin{matrix}
\boldsymbol{B}&		\boldsymbol{C}\\
\end{matrix} \right) =\left( \begin{matrix}
\boldsymbol{B}' \boldsymbol{B}&		\boldsymbol{B}' \boldsymbol{C}\\
\boldsymbol{C}' \boldsymbol{B}&		\boldsymbol{C}' \boldsymbol{C}\\
\end{matrix} \right) .
\end{align*}
是正定阵, 再由\hyperref[proposition:Fischer不等式]{Fischer不等式}即得结论. 

\end{proof}

\begin{definition}[亚正定阵]
设 \(\boldsymbol{M}\) 为 \(n\) 阶实矩阵, 若对任意的非零实列向量 \(\boldsymbol{\alpha}\), 总有 \(\boldsymbol{\alpha}'\boldsymbol{M}\boldsymbol{\alpha}>0\), 则称 \(\boldsymbol{M}\) 是\textbf{亚正定阵}. 
\end{definition}

\begin{proposition}\label{proposition:亚正定矩阵的所有特征值的实部都大于零}
亚正定矩阵 \(\boldsymbol{A}\) 的所有特征值的实部都大于零.
\end{proposition}
\begin{proof}
设 \(\lambda_0 = a + b\mathrm{i}\) 是 \(\boldsymbol{A}\) 的特征值, \(\boldsymbol{\eta}\) 是属于 \(\lambda_0\) 的特征向量. 将 \(\boldsymbol{\eta}\) 的实部和虚部分开, 记为 \(\boldsymbol{\eta}=\boldsymbol{\alpha}+\mathrm{i}\boldsymbol{\beta}\), 则 \(\boldsymbol{A}(\boldsymbol{\alpha}+\mathrm{i}\boldsymbol{\beta})=(a + b\mathrm{i})(\boldsymbol{\alpha}+\mathrm{i}\boldsymbol{\beta})\). 分开实部和虚部可得 \(\boldsymbol{A}\boldsymbol{\alpha}=a\boldsymbol{\alpha}-b\boldsymbol{\beta}\), \(\boldsymbol{A}\boldsymbol{\beta}=b\boldsymbol{\alpha}+a\boldsymbol{\beta}\), 于是 \(\boldsymbol{\alpha}'\boldsymbol{A}\boldsymbol{\alpha}=a\boldsymbol{\alpha}'\boldsymbol{\alpha}-b\boldsymbol{\alpha}'\boldsymbol{\beta}\), \(\boldsymbol{\beta}'\boldsymbol{A}\boldsymbol{\beta}=b\boldsymbol{\beta}'\boldsymbol{\alpha}+a\boldsymbol{\beta}'\boldsymbol{\beta}\). 因此再由$\boldsymbol{A}$为亚正定矩阵可得
\begin{align*}
\boldsymbol{\alpha}'\boldsymbol{A}\boldsymbol{\alpha}+\boldsymbol{\beta}'\boldsymbol{A}\boldsymbol{\beta}&=a(\boldsymbol{\alpha}'\boldsymbol{\alpha}+\boldsymbol{\beta}'\boldsymbol{\beta})>0.
\end{align*} 
又$\boldsymbol{\alpha}'\boldsymbol{\alpha}+\boldsymbol{\beta}'\boldsymbol{\beta}\geqslant 0$,故$a>0.$

\end{proof}

\begin{theorem}\label{theorem:亚正定阵的等价条件}
证明下列结论等价:
\begin{enumerate}[(1)]
\item \(\boldsymbol{M}\) 是亚正定阵;
\item \(\boldsymbol{M}+\boldsymbol{M}'\) 是正定阵;
\item \(\boldsymbol{M}=\boldsymbol{A}+\boldsymbol{S}\), 其中 \(\boldsymbol{A}\) 是正定实对称矩阵, \(\boldsymbol{S}\) 是实反对称矩阵.
\end{enumerate}
\end{theorem}
\begin{remark}
\refpro{proposition:亚正定矩阵的所有特征值的实部都大于零}告诉我们: 亚正定阵 \(\boldsymbol{M}\) 的特征值的实部都大于零, 由此可得 \(\boldsymbol{M}\) 的行列式值大于零. 事实上, 这一结论还可以由\refpro{proposition:A+S的行列式的相关结论}得到, 即 \(|\boldsymbol{M}|=|\boldsymbol{A}+\boldsymbol{S}|\geqslant |\boldsymbol{A}|>0\). 另外, 这一结论还能给出\refpro{proposition:两个正定阵和非异阵的关系}的证法 2, 即由 \(\boldsymbol{B}\boldsymbol{A}+(\boldsymbol{B}\boldsymbol{A})'\) 正定可知 \(\boldsymbol{B}\boldsymbol{A}\) 亚正定, 从而 \(|\boldsymbol{B}\boldsymbol{A}|>0\), 于是 \(\boldsymbol{A},\boldsymbol{B}\) 都是非异阵. 
\end{remark}
\begin{proof}
\begin{enumerate}
\item \((1)\Rightarrow(2)\): 将 \(\boldsymbol{\alpha}'\boldsymbol{M}\boldsymbol{\alpha}>0\) 转置后可得 \(\boldsymbol{\alpha}'\boldsymbol{M}'\boldsymbol{\alpha}>0\), 再将两式相加后可得 \(\boldsymbol{\alpha}'(\boldsymbol{M}+\boldsymbol{M}')\boldsymbol{\alpha}>0\) 对任意的非零实列向量 \(\boldsymbol{\alpha}\) 都成立, 因此 \(\boldsymbol{M}+\boldsymbol{M}'\) 是正定阵.
\item \((2)\Rightarrow(3)\): 令 \(\boldsymbol{A}=\frac{1}{2}(\boldsymbol{M}+\boldsymbol{M}')\) 为 \(\boldsymbol{M}\) 的对称化, \(\boldsymbol{S}=\frac{1}{2}(\boldsymbol{M}-\boldsymbol{M}')\) 为 \(\boldsymbol{M}\) 的反对称化, 则结论成立.
\item \((3)\Rightarrow(1)\): 由\refpro{proposition:反对称阵的刻画}可知, 对任意的非零实列向量 \(\boldsymbol{\alpha}\), 总有 \(\boldsymbol{\alpha}'\boldsymbol{M}\boldsymbol{\alpha}=\boldsymbol{\alpha}'\boldsymbol{A}\boldsymbol{\alpha}+\boldsymbol{\alpha}'\boldsymbol{S}\boldsymbol{\alpha}=\boldsymbol{\alpha}'\boldsymbol{A}\boldsymbol{\alpha}>0\), 即 \(\boldsymbol{M}\) 为亚正定阵. 
\end{enumerate}

\end{proof}

\subsection{负定型与负定阵}

\begin{theorem}\label{theorem:负定阵或半负定阵的充要条件(与正定差一个负号)}
设 \(f(x_1,x_2,\cdots,x_n)\) 是实二次型, \(\boldsymbol{A}\) 是相伴的实对称矩阵, 则

(1)\(f\) 是负定型或半负定型当且仅当 \(-f\) 是正定型或半正定型;

(2)\(\boldsymbol{A}\) 是负定阵或半负定阵当且仅当 \(-\boldsymbol{A}\) 是正定阵或半正定阵.
\end{theorem}
\begin{remark}
由这个\refthe{theorem:负定阵或半负定阵的充要条件(与正定差一个负号)}可知,负定型或半负定型 (负定阵或半负定阵) 的问题通常都可以转化成正定型或半正定型 (正定阵或半正定阵) 的问题来研究. 
\end{remark}
\begin{proof}
由(半)正定、负定型(阵)的定义易得.

\end{proof}

\begin{lemma}\label{lemma:(半)正/负定阵/型的等价条件}
设 \(f(x_1,x_2,\cdots,x_n)\) 是实二次型, \(\boldsymbol{A}\) 是相伴的实对称矩阵,$\boldsymbol{x}=(x_1,x_2,\cdots,x_n)'$,则

(1)$f$是正定型的充要条件是若$f(\boldsymbol{x})\leqslant  0$,则$\boldsymbol{x}=0$.

(2)$f$是半正定型的充要条件是若$f(\boldsymbol{x})< 0$,则$\boldsymbol{x}=0$.

(3)$f$是负定型的充要条件是若$f(\boldsymbol{x})\geqslant  0$,则$\boldsymbol{x}=0$.

(4)$f$是半负定型的充要条件是若$f(\boldsymbol{x})> 0$,则$\boldsymbol{x}=0$.
\end{lemma}
\begin{proof}
证明是显然的.

\end{proof}

\begin{proposition}\label{proposition:负定阵的判定准则1}
设 \(\boldsymbol{A}\) 是 \(n\) 阶实对称矩阵, \(P_1,P_2,\cdots,P_n\) 是 \(\boldsymbol{A}\) 的 \(n\) 个顺序主子式, 求证 \(\boldsymbol{A}\) 负定的充要条件是:
\[P_1<0,\ P_2>0,\ \cdots,\ (-1)^nP_n>0\]
\end{proposition}
\begin{proof}
\(\boldsymbol{A}\) 负定当且仅当 \(-\boldsymbol{A}\) 正定, 由\hyperref[theorem:实正定阵的充要条件]{正定阵的顺序主子式判定法}即得结论. 

\end{proof}

\begin{proposition}
设$\boldsymbol{A}$为$n$阶实对称矩阵, 求证:

(1) $\boldsymbol{A}$是负定阵的充要条件是存在$n$阶非异实矩阵$\boldsymbol{C}$, 使得$\boldsymbol{A}=-\boldsymbol{C}'\boldsymbol{C}$.

(2) $\boldsymbol{A}$是半负定阵的充要条件是存在$n$阶实矩阵$\boldsymbol{C}$, 使得$\boldsymbol{A}=-\boldsymbol{C}'\boldsymbol{C}$. 特别地, $|\boldsymbol{A}| = (-1)^n|\boldsymbol{C}|^2$.
\end{proposition}
\begin{proof}
由\refpro{proposition:正定和半正定阵的判定准则}和\refthe{theorem:负定阵或半负定阵的充要条件(与正定差一个负号)}立得.

\end{proof}

\begin{proposition}\label{proposition:负定阵的相关性质}
设 \(\boldsymbol{A}\) 是 \(n\) 阶负定实对称矩阵, 求证: \(\boldsymbol{A}^{-1}\) 也是负定阵; 当 \(n\) 为偶数时, \(\boldsymbol{A}^*\) 是负定阵, 当 \(n\) 为奇数时, \(\boldsymbol{A}^*\) 是正定阵.
\end{proposition}
\begin{proof}
因为 \(\boldsymbol{A}\) 负定,故存在非异实矩阵 \(\boldsymbol{C}\), 使得\(\boldsymbol{A}=-\boldsymbol{C}'\boldsymbol{C}\). 于是 \(\boldsymbol{A}^{-1}=-\boldsymbol{C}^{-1}(\boldsymbol{C}')^{-1}=-\boldsymbol{C}^{-1}(\boldsymbol{C}^{-1})'\)也是负定阵; 由\hyperref[proposition:伴随矩阵的性质]{伴随矩阵的性质2}可得 \(\boldsymbol{A}^*=(-1)^{n - 1}\boldsymbol{C}^*(\boldsymbol{C}')^*=(-1)^{n - 1}\boldsymbol{C}^*(\boldsymbol{C}^*)'\), 故当 \(n\) 为偶数时, \(\boldsymbol{A}^*=-\boldsymbol{C}^*(\boldsymbol{C}^*)'\) 是负定阵; 当 \(n\) 为奇数时, \(\boldsymbol{A}^*=\boldsymbol{C}^*(\boldsymbol{C}^*)'\) 是正定阵.

\end{proof}

\begin{proposition}\label{proposition:利用正定阵构造负定二次型}
设有实二次型 \(f(x_1,x_2,\cdots,x_n)=\boldsymbol{x}'\boldsymbol{A}\boldsymbol{x}\), 其中 \(\boldsymbol{A}=(a_{ij})\) 是 \(n\) 阶正定实对称矩阵, 求证下列实二次型是负定型:
\[g(x_1,x_2,\cdots,x_n)=\begin{vmatrix}
a_{11}&a_{12}&\cdots&a_{1n}&x_1\\
a_{21}&a_{22}&\cdots&a_{2n}&x_2\\
\vdots&\vdots&\ddots&\vdots&\vdots\\
a_{n1}&a_{n2}&\cdots&a_{nn}&x_n\\
x_1&x_2&\cdots&x_n&0
\end{vmatrix}=\left| \begin{matrix}
	\boldsymbol{A}&		\boldsymbol{x}\\
	\boldsymbol{x}'&		0\\
\end{matrix} \right|.
\]
\end{proposition}
\begin{proof}
{\color{blue}证法一:}由\refpro{根据行列式代数余子式构造行列式}可得
\[g(x_1,x_2,\cdots,x_n)=-\sum_{i = 1}^{n}\sum_{j = 1}^{n}A_{ij}x_ix_j=-\boldsymbol{x}'\boldsymbol{A}^*\boldsymbol{x}\]
其中 \(A_{ij}\) 是元素 \(a_{ij}\) 的代数余子式, \(\boldsymbol{A}^*\) 是 \(\boldsymbol{A}\) 的伴随矩阵. 因为 \(\boldsymbol{A}\) 正定, 故由\refpro{proposition:正定阵的性质123}可知 \(\boldsymbol{A}^*\) 也正定, 从而 \(g\) 为负定型.

{\color{blue}证法二:} 因为 \(\boldsymbol{A}\) 正定, 所以 \(|\boldsymbol{A}|>0\), 故由降阶公式可得
\[g(x_1,x_2,\cdots,x_n)=|\boldsymbol{A}|(0 - \boldsymbol{x}'\boldsymbol{A}^{-1}\boldsymbol{x})=-|\boldsymbol{A}|(\boldsymbol{x}'\boldsymbol{A}^{-1}\boldsymbol{x})\]
再由\refpro{proposition:正定阵的性质123}可知 \(\boldsymbol{A}^{-1}\) 也正定, 即 \(\boldsymbol{x}'\boldsymbol{A}^{-1}\boldsymbol{x}\) 是正定型, 从而 \(g\) 为负定型.

{\color{blue}证法三:}设 $\boldsymbol{\alpha}=(a_1,a_2,\cdots,a_n)'$ 为实列向量,要证 $g$ 是负定型,等价地只要证明:若 $g(\boldsymbol{\alpha})\geqslant 0$,则 $\boldsymbol{\alpha}=\boldsymbol{0}$ 即可。作 $n + 1$ 变元二次型 $h(\boldsymbol{y})=\boldsymbol{y}'B\boldsymbol{y}$,其中 $B = \begin{pmatrix}
\boldsymbol{A} & \boldsymbol{\alpha} \\
\boldsymbol{\alpha}' & 0
\end{pmatrix}$,则 $|B| = g(\boldsymbol{\alpha})\geqslant 0$。又已知 $\boldsymbol{A}$ 正定,因此 $B$ 的前 $n$ 个顺序主子式为正数。由\refpro{proposition:利用顺序主子式计算二次型的标准型}可知,$h$ 是半正定型,从而 $B$ 是半正定阵。注意到 $B$ 的第 $(n + 1,n + 1)$ 元素为零,故由\refpro{proposition:若主对角元为零,则同行同列的所有元素都为零}可知 $\boldsymbol{\alpha}=\boldsymbol{0}$。

\end{proof}






\end{document}
\documentclass[../../main.tex]{subfiles}
\graphicspath{{\subfix{../../image/}}} % 指定图片目录,后续可以直接使用图片文件名。

% 例如:
% \begin{figure}[h]
% \centering
% \includegraphics{image-01.01}
% \caption{图片标题}
% \label{fig:image-01.01}
% \end{figure}
% 注意:上述\label{}一定要放在\caption{}之后,否则引用图片序号会只会显示??.

\begin{document}

\section{对称初等变换与矩阵合同}

回顾\reflemma{lemma:初等合同变换}中的对称初等变换.

\begin{proposition}\label{proposition:分块对称准对角阵正交相似于其任意排列}
设\(\mathrm{diag}\{A_1,A_2,\cdots,A_m\}\)是分块对角矩阵,其中\(A_i\)都是对称矩阵,求证:\(\mathrm{diag}\{A_1,A_2,\cdots,A_m\}\)正交相似于\(\mathrm{diag}\{A_{i_1},A_{i_2},\cdots,A_{i_m}\}\),其中\(A_{i_1},A_{i_2},\cdots,A_{i_m}\)是\(A_1,A_2,\cdots,A_m\)的一个排列。
\end{proposition}
\begin{proof}
对换分块对角矩阵的第\(i,j\)分块行,再对换第\(i,j\)分块列,这是一个正交变换,变换的结果是将第\((i,i)\)分块和第\((j,j)\)分块对换了位置。又任意一个排列都可以通过若干次对换来实现,因此两个分块对角矩阵$\mathrm{diag}$ $\{$ $A_1$,$A_2$,$\cdots$,$A_m$ $\}$和$\mathrm{diag}$ $\{$ $A_{i_1}$,$A_{i_2}$,$\cdots$,$A_{i_m}\}$正交相似。
\end{proof}

\begin{proposition}
求证:\(n\)阶实对称矩阵\(A\)是正定阵的充要条件是\(A\)的前\(n - 1\)个顺序主子式的代数余子式以及第\(n\)个顺序主子式全大于零。
\end{proposition}
\begin{proof}
将\(A\)的第\(i\)行和第\(n - i + 1\)行对换,再将第\(i\)列和第\(n - i + 1\)列对换\((1\leq i\leq n)\),得到的矩阵记为\(B\),则\(B\)和\(A\)合同。设$A=(a_{ij})_{n\times n}$,则有
\begin{align*}
A=
\begin{pmatrix}
a_{11} & a_{12} & \cdots & a_{1n} \\
a_{21} & a_{22} & \cdots & a_{2n} \\
\vdots & \vdots & \ddots & \vdots \\
a_{n1} & a_{n2} & \cdots & a_{nn}
\end{pmatrix},\quad
B=
\begin{pmatrix}
a_{nn} & a_{n,n-1} & \cdots & a_{n1} \\
a_{n-1,n} & a_{n-1,n-1} & \cdots & a_{n-1,1} \\
\vdots & \vdots & \ddots & \vdots \\
a_{1n} & a_{1,n-1} & \cdots & a_{11}
\end{pmatrix}.
\end{align*}
注意到\(B\)的\(n\)个顺序主子式按副对角线翻转(顺/逆时针旋转$180^{\circ}$)后就是\(A\)的前\(n - 1\)个顺序主子式的代数余子式以及第\(n\)个顺序主子式,又由\refproposition{pro:行列式计算常识}可知,\(B\)的\(n\)个顺序主子式等于\(A\)的前\(n - 1\)个顺序主子式的代数余子式以及第\(n\)个顺序主子式.

故\(A\)是正定阵当且仅当\(B\)是正定阵,这当且仅当\(B\)的\(n\)个顺序主子式全大于零,这也当且仅当\(A\)的前\(n - 1\)个顺序主子式的代数余子式以及第\(n\)个顺序主子式全大于零。
\end{proof}

\begin{proposition}\label{proposition:分块准对角阵合同于每个块的合同阵}
设有分块对称矩阵:
\begin{align*}
A = \begin{pmatrix}
A_1 & O \\
O & A_2
\end{pmatrix},
\end{align*}
假设\(A_1\)合同于\(B_1\),\(A_2\)合同于\(B_2\),求证:\(A\)合同于分块对称矩阵
\begin{align*}
B = \begin{pmatrix}
B_1 & O \\
O & B_2
\end{pmatrix}.
\end{align*}
\end{proposition} 
\begin{proof}
设\(C_1,C_2\)为非异阵,使得\(C_1'A_1C_1 = B_1\),\(C_2'A_2C_2 = B_2\),令
\begin{align*}
C = \begin{pmatrix}
C_1 & O \\
O & C_2
\end{pmatrix},
\end{align*}
则\(C\)为非异阵,使得\(C'AC = B\)。
\end{proof}

\begin{proposition}\label{proposition:分块准对角阵的正负惯性指数}
设分块实对称矩阵\(M = \begin{pmatrix}
A & O \\
O & B
\end{pmatrix}\),用\(p(A),q(A)\)分别表示\(A\)的正负惯性指数,求证:
\begin{align*}
p(M)=p(A)+p(B),\quad q(M)=q(A)+q(B).
\end{align*}
\end{proposition} 
\begin{proof}
由实对称矩阵的合同标准型可知,\(A\)合同于\(\mathrm{diag}\{I_{p(A)}, -I_{q(A)},O\}\),\(B\)合同于\(\mathrm{diag}\{I_{p(B)}, -I_{q(B)},O\}\),因此由\refproposition{proposition:分块对称准对角阵正交相似于其任意排列}和\refproposition{proposition:分块准对角阵合同于每个块的合同阵}可知,\(M = \mathrm{diag}\{A,B\}\)合同于\(\mathrm{diag}\{I_{p(A)+p(B)}, -I_{q(A)+q(B)},O\}\),从而结论得证。
\end{proof}

\begin{proposition}[正负惯性指数的降阶公式]\label{proposition:正负惯性指数的降阶公式}
设分块实对称矩阵\(M = \begin{pmatrix}
A & C \\
C' & B
\end{pmatrix}\),其中\(A,B\)都可逆,求证:
\begin{align*}
p(M)=p(A)+p(B - C'A^{-1}C)&=p(B)+p(A - CB^{-1}C'),\\
q(M)=q(A)+q(B - C'A^{-1}C)&=q(B)+q(A - CB^{-1}C').
\end{align*}
其中$p(X),q(X)$分别表示矩阵$X$的正惯性指数和负惯性指数.
\end{proposition}
\begin{proof}
先将\(M\)的第一分块行左乘\(-C'A^{-1}\)加到第二分块行上,再将第一分块列右乘\((-C'A^{-1})'=-A^{-1}C\)加到第二分块列上,可得如下合同变换:
\begin{align*}
M = \begin{pmatrix}
A & C \\
C' & B
\end{pmatrix}
&\rightarrow\begin{pmatrix}
A & C \\
O & B - C'A^{-1}C
\end{pmatrix}
\rightarrow\begin{pmatrix}
A & O \\
O & B - C'A^{-1}C
\end{pmatrix}.
\end{align*}
另一种对称分块初等变换是,先将\(M\)的第二分块行左乘\(-CB^{-1}\)加到第一分块行上,再将第二分块列右乘\((-CB^{-1})'=-B^{-1}C'\)加到第一分块列上,可得合同变换:
\begin{align*}
M = \begin{pmatrix}
A & C \\
C' & B
\end{pmatrix}
&\rightarrow\begin{pmatrix}
A - CB^{-1}C' & O \\
C' & B
\end{pmatrix}
\rightarrow\begin{pmatrix}
A - CB^{-1}C' & O \\
O & B
\end{pmatrix}.
\end{align*}
因此\(\begin{pmatrix}
A & O \\
O & B - C'A^{-1}C
\end{pmatrix}\)合同于\(\begin{pmatrix}
A - CB^{-1}C' & O \\
O & B
\end{pmatrix}\),再由\refproposition{proposition:分块准对角阵的正负惯性指数}即得结论。
\end{proof} 

\begin{proposition}
设\(\boldsymbol{\alpha}\)是\(n\)维实列向量且\(\boldsymbol{\alpha}'\boldsymbol{\alpha}=1\),证明:矩阵\(I_n - 2\boldsymbol{\alpha}\boldsymbol{\alpha}'\)的正惯性指数等于\(n - 1\),负惯性指数等于\(1\)。
\end{proposition}
\begin{proof}
构造分块对称矩阵\(M = \begin{pmatrix}
I_n & \sqrt{2}\boldsymbol{\alpha} \\
\sqrt{2}\boldsymbol{\alpha}' & 1
\end{pmatrix}\),由\hyperref[proposition:正负惯性指数的降阶公式]{正负惯性指数的降阶公式}可知,\(I_n - 2\boldsymbol{\alpha}\boldsymbol{\alpha}'\)的正惯性指数等于\(n - 1\),负惯性指数等于\(1\)。
\end{proof}

\begin{proposition}
求\(n(n\geq2)\)阶实对称矩阵\(A\)的正负惯性指数,其中\(a_i\)均为实数:
\[
A = \begin{pmatrix}
a_1^2 & a_1a_2 + 1 & \cdots & a_1a_n + 1 \\
a_2a_1 + 1 & a_2^2 & \cdots & a_2a_n + 1 \\
\vdots & \vdots & & \vdots \\
a_na_1 + 1 & a_na_2 + 1 & \cdots & a_n^2
\end{pmatrix}.
\]
\end{proposition} 
\begin{proof}
构造分块对称矩阵
\[
M = \begin{pmatrix}
-I_n & B \\
B' & -I_2
\end{pmatrix}, \text{其中} B' = \begin{pmatrix}
a_1 & a_2 & \cdots & a_n \\
1 & 1 & \cdots & 1
\end{pmatrix},
\]
又注意到$A=-I_n-B(-I_2)^{-1}B'$,则由\hyperref[proposition:打洞原理]{打洞原理}可知
\begin{align*}
\left| A \right|=\left| -I_n-B(-I_2)^{-1}B' \right|=\left| M \right|=\left( -1 \right) ^n\left| -I_2-B' (-I_n)^{-1}B \right|=\left( -1 \right) ^n\left| \begin{matrix}
\sum_{i=1}^n{a_{i}^{2}}-1&		\sum_{i=1}^n{a_i}\\
\sum_{i=1}^n{a_i}&		n-1\\
\end{matrix} \right|.
\end{align*}
令$C\triangleq -I_2-B' (-I_n)^{-1}B=\left( \begin{matrix}
\sum_{i=1}^n{a_{i}^{2}}-1&		\sum_{i=1}^n{a_i}\\
\sum_{i=1}^n{a_i}&		n-1\\
\end{matrix} \right) $,
注意到
\begin{align*}
|C|=\left( n-1 \right) \left( \sum_{i=1}^n{a_{i}^{2}}-1 \right) -\left( \sum_{i=1}^n{a_i} \right) ^2,\quad |C|=(-1)^n|A|.
\end{align*}
从而$C$经过对称初等变换可得
\begin{align*}
C&=\left( \begin{matrix}
\sum_{i=1}^n{a_{i}^{2}}-1&		\sum_{i=1}^n{a_i}\\
\sum_{i=1}^n{a_i}&		n-1\\
\end{matrix} \right) \underrightarrow{-\frac{\sum\limits_{i=1}^n{a_i}}{n-1}r_2+r_1}\left( \begin{matrix}
\left( \sum_{i=1}^n{a_{i}^{2}}-1 \right) -\frac{\left( \sum\limits_{i=1}^n{a_i} \right) ^2}{n-1}&		0\\
\sum_{i=1}^n{a_i}&		n-1\\
\end{matrix} \right) 
\\
&\underrightarrow{-\frac{\sum\limits_{i=1}^n{a_i}}{n-1}j_2+j_1}\left( \begin{matrix}
\left( \sum_{i=1}^n{a_{i}^{2}}-1 \right) -\frac{\left( \sum\limits_{i=1}^n{a_i} \right) ^2}{n-1}&		0\\
0&		n-1\\
\end{matrix} \right) =\left( \begin{matrix}
\frac{\left| C \right|}{n-1}&		0\\
0&		n-1\\
\end{matrix} \right) .
\end{align*}
故当\(\vert C\vert>0\)时,\(p(C) = 2\),\(q(C) = 0\);当\(\vert C\vert = 0\)时,\(p(C) = 1\),\(q(C) = 0\);当\(\vert C\vert<0\)时,\(p(C) = 1\),\(q(C) = 1\)。再由\hyperref[proposition:正负惯性指数的降阶公式]{正负惯性指数的降阶公式}可知
\begin{align*}
p\left( M \right) =p\left( -I_2 \right) +p\left( -I_n-B(-I_2)^{-1}B\prime \right) =p\left( -I_n \right) +p\left( -I_2-B\prime \left( -I_n \right) B \right) \Leftrightarrow p\left( A \right) =p\left( C \right) ,
\\
q\left( M \right) =q\left( -I_2 \right) +q\left( -I_n-B(-I_2)^{-1}B\prime \right) =q\left( -I_n \right) +q\left( -I_2-B\prime \left( -I_n \right) B \right) \Leftrightarrow q\left( A \right) =q\left( C \right) .
\end{align*}
因此
当\((-1)^n\vert A\vert=|C|>0\)时,\(p(A) = 2\),\(q(A) = n - 2\);当\(\vert A\vert =|C|= 0\)时,\(p(A) = 1\),\(q(A) = n - 2\);当\((-1)^n\vert A\vert=|C|<0\)时,\(p(A) = 1\),\(q(A) = n - 1\)。
\end{proof}

\begin{example}
设\(A\)是\(n\)阶可逆实矩阵,\(B = \begin{pmatrix}
O & A \\
A' & O
\end{pmatrix}\),求\(B\)的正负惯性指数。
\end{example}
\begin{proof}
将\(B\)的第一分块行左乘\(A^{-1}\),再将第一分块列右乘\((A^{-1})'=(A')^{-1}\),于是\(B\)合同于\(C = \begin{pmatrix}
O & I_n \\
I_n & O
\end{pmatrix}\)。将\(C\)的第二分块行加到第一分块行上,再将第二分块列加到第一分块列上,于是\(C\)合同于\(D = \begin{pmatrix}
2I_n & I_n \\
I_n & O
\end{pmatrix}\)。将\(D\)的第一分块行左乘\(-\frac{1}{2}I_n\)加到第二分块行上,再将第一分块列右乘\(-\frac{1}{2}I_n\)加到第二分块列上,于是\(D\)合同于\(\begin{pmatrix}
2I_n & O \\
O & -\frac{1}{2}I_n
\end{pmatrix}\),因此\(B\)的正负惯性指数都等于\(n\)。
\end{proof}

\begin{example}
设\(A\)是\(n\)阶正定实对称矩阵,求证:\(B = \begin{pmatrix}
A & -I_n \\
-I_n & A^{-1}
\end{pmatrix}\)是半正定阵。
\end{example} 
\begin{proof}
将\(B\)的第一分块行左乘\(A^{-1}\)加到第二分块行上,再将第一分块列右乘\(A^{-1}\)加到第二分块列上,于是\(B\)合同于\(\begin{pmatrix}
A & O \\
O & O
\end{pmatrix}\),这是一个半正定矩阵。
\end{proof}























































































\end{document}
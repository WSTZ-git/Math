\documentclass[../../main.tex]{subfiles}
\graphicspath{{\subfix{../../image/}}} % 指定图片目录,后续可以直接使用图片文件名。

% 例如:
% \begin{figure}[H]
% \centering
% \includegraphics[scale=0.4]{图.png}
% \caption{}
% \label{figure:图}
% \end{figure}
% 注意:上述\label{}一定要放在\caption{}之后,否则引用图片序号会只会显示??.

\begin{document}

\section{归纳法的应用}

数学归纳法是讨论二次型与相关矩阵问题的常用方法之一. 注意到\hyperref[proposition:正负惯性指数的降阶公式]{正负惯性指数的降阶公式}的证明过程展示了这样一种方法,例如有一个分块对称矩阵$M = \begin{pmatrix}
\boldsymbol{A} & \boldsymbol{C} \\
\boldsymbol{C}' & \boldsymbol{B}
\end{pmatrix}$,其中$\boldsymbol{A}$是可逆矩阵,则通过对称分块初等变换可用$\boldsymbol{A}$同时消去$\boldsymbol{C}$与$\boldsymbol{C}'$,从而得到分块对角矩阵$\begin{pmatrix}
\boldsymbol{A} & \boldsymbol{O} \\
\boldsymbol{O} & \boldsymbol{B - C'A^{-1}C}
\end{pmatrix}$. 此时矩阵$\boldsymbol{A}, \boldsymbol{B - C'A^{-1}C}$的阶都比$M$的阶低,如果问题的条件和结论在合同关系下不改变,则上述过程就是运用归纳法的基础. 事实上,正定阵的判定准则之一,即\hyperref[theorem:实正定阵的充要条件]{实对称矩阵$\boldsymbol{A}$是正定阵的充要条件是$\boldsymbol{A}$的顺序主子式全大于零},就是通过上述方法证明的. 

\begin{proposition}\label{proposition:正定阵的3个充要条件}
证明下列关于$n$阶实对称矩阵$\boldsymbol{A}=(a_{ij})$的命题等价:

(1) $\boldsymbol{A}$是正定阵;

(2) 存在主对角元全等于 1 的上三角矩阵$\boldsymbol{B}$和主对角元全为正数的对角矩阵$\boldsymbol{D}$, 使得$\boldsymbol{A}=\boldsymbol{B}'\boldsymbol{D}\boldsymbol{B}$;

(3)(\textbf{正定阵的Cholesky分解}) 存在唯一的主对角元全为正数的上三角矩阵$\boldsymbol{C}$, 使得$\boldsymbol{A}=\boldsymbol{C}'\boldsymbol{C}$.
\end{proposition}
\begin{remark}
设$\boldsymbol{C}=(c_{ij})$为主对角元全为正数的上三角矩阵, 使得$\boldsymbol{A}=\boldsymbol{C}'\boldsymbol{C}$, 则$c_{11}c_{1j}=a_{1j}$, 从而$c_{11}=\sqrt{a_{11}}>0$, $c_{1j}=\frac{a_{1j}}{\sqrt{a_{11}}}(2\leq j\leq n)$, 即$\boldsymbol{C}$的第一行元素被唯一确定. 同理不断地讨论下去, 可得这样的$\boldsymbol{C}$存在并被正定阵$\boldsymbol{A}$唯一确定. 因为$\boldsymbol{S}$是由$\boldsymbol{C}$的主对角元构成的对角矩阵, 故由$\boldsymbol{C}$的唯一性可得$\boldsymbol{S}$的唯一性, 从而可得$\boldsymbol{D}=\boldsymbol{S}^2$以及$\boldsymbol{B}=\boldsymbol{S}^{-1}\boldsymbol{C}$的唯一性. 因此,这个\refpro{proposition:正定阵的3个充要条件}中关于正定阵$\boldsymbol{A}$的两种分解 (2) 和 (3) 都是存在且唯一的, 其中分解 (3) 通常称为正定阵$\boldsymbol{A}$的 Cholesky 分解. 另外, 上述两种分解也有非常重要的几何意义, 它们与 Gram - Schmidt 正交化方法密切相关. 
\end{remark}
\begin{remark}
事实上,\hyperref[proposition:正定阵的3个充要条件]{正定阵的 Cholesky 分解}和非异阵的 $QR$ 分解从某种意义上看是等价的。{\color{blue}证法三}即是由非异阵的 $QR$ 分解推出\hyperref[proposition:正定阵的3个充要条件]{正定阵的 Cholesky 分解}。反之,对任一非异实矩阵 $A$,由\nrefpro{proposition:正定阵的判定准则}{(2)}可知$A'A$ 是正定阵,设 $A'A = R'R$ 是 \hyperref[proposition:正定阵的3个充要条件]{正定阵的 Cholesky 分解},其中 $R$ 是主对角元全大于零的上三角矩阵。令 $Q = AR^{-1}$,则
\[
Q'Q=(AR^{-1})'(AR^{-1})=(R')^{-1}(A'A)R^{-1}=(R')^{-1}(R'R)R^{-1}=I_n,
\]
即 $Q$ 是正交矩阵,从而 $A = QR$ 是 $QR$ 分解。从几何的层面上看,上述两种矩阵分解都等价于 Gram - Schmidt 正交化和标准化过程,所以它们之间的等价性是自然的。
\end{remark}
\begin{proof}
{\color{blue}证法一:}
(1) $\Rightarrow$ (2): 只要证明存在主对角元全为 1 的上三角矩阵$\boldsymbol{T}$, 使得$\boldsymbol{T}'\boldsymbol{A}\boldsymbol{T}=\boldsymbol{D}$是正定对角矩阵即可. 因为一旦得证,由\hyperref[proposition:上三角阵性质]{上三角阵性质}可知$\boldsymbol{B}=\boldsymbol{T}^{-1}$也是主对角元全为 1 的上三角矩阵, 并且$\boldsymbol{A}=\boldsymbol{B}'\boldsymbol{D}\boldsymbol{B}$. 对阶数$n$进行归纳, 当$n = 1$时结论显然成立. 假设对$n - 1$阶正定阵结论成立, 现证明$n$阶正定阵的情形. 设$\boldsymbol{A}=\begin{pmatrix}
\boldsymbol{A}_{n - 1} & \boldsymbol{\alpha} \\
\boldsymbol{\alpha}' & a_{nn}
\end{pmatrix}$, 其中$\boldsymbol{A}_{n - 1}$是$n - 1$阶矩阵,$\boldsymbol{\alpha}$是$n - 1$维列向量. 因为$\boldsymbol{A}$正定, 所以$\boldsymbol{A}_{n - 1}$是$n - 1$阶正定阵, 从而是可逆矩阵. 考虑如下对称分块初等变换:
\begin{align*}
\begin{pmatrix}
\boldsymbol{I}_{n - 1} & \boldsymbol{O} \\
-\boldsymbol{\alpha}'\boldsymbol{A}_{n - 1}^{-1} & 1
\end{pmatrix}
\begin{pmatrix}
\boldsymbol{A}_{n - 1} & \boldsymbol{\alpha} \\
\boldsymbol{\alpha}' & a_{nn}
\end{pmatrix}
\begin{pmatrix}
\boldsymbol{I}_{n - 1} & -\boldsymbol{A}_{n - 1}^{-1}\boldsymbol{\alpha} \\
\boldsymbol{O} & 1
\end{pmatrix}
=
\begin{pmatrix}
\boldsymbol{A}_{n - 1} & \boldsymbol{O} \\
\boldsymbol{O} & a_{nn}-\boldsymbol{\alpha}'\boldsymbol{A}_{n - 1}^{-1}\boldsymbol{\alpha}
\end{pmatrix}
\end{align*}
由$\boldsymbol{A}$的正定性及\hyperref[proposition:正定阵的性质]{命题\ref{proposition:正定阵的性质}(2)}可得$a_{nn}-\boldsymbol{\alpha}'\boldsymbol{A}_{n - 1}^{-1}\boldsymbol{\alpha}>0$. 再由归纳假设, 存在主对角元全为 1 的$n - 1$阶上三角矩阵$\boldsymbol{T}_{n - 1}$, 使得$\boldsymbol{T}_{n - 1}'\boldsymbol{A}_{n - 1}\boldsymbol{T}_{n - 1}=\boldsymbol{D}_{n - 1}$是$n - 1$阶正定对角矩阵. 令
\[
\boldsymbol{T}=\begin{pmatrix}
\boldsymbol{I}_{n - 1} & -\boldsymbol{A}_{n - 1}^{-1}\boldsymbol{\alpha} \\
\boldsymbol{O} & 1
\end{pmatrix}
\begin{pmatrix}
\boldsymbol{T}_{n - 1} & \boldsymbol{O} \\
\boldsymbol{O} & 1
\end{pmatrix}
\]
则$\boldsymbol{T}$是一个主对角元全为 1 的$n$阶上三角矩阵, 使得
\[
\boldsymbol{T}'\boldsymbol{A}\boldsymbol{T}=\begin{pmatrix}
\boldsymbol{D}_{n - 1} & \boldsymbol{O} \\
\boldsymbol{O} & a_{nn}-\boldsymbol{\alpha}'\boldsymbol{A}_{n - 1}^{-1}\boldsymbol{\alpha}
\end{pmatrix}
\]
是$n$阶正定对角矩阵.

(2) $\Rightarrow$ (3): 由 (2) 可设$\boldsymbol{D}=\text{diag}\{d_1, d_2, \cdots, d_n\}$, 则$A=B'DB$,其中$B$为主对角元全为1的上三角矩阵.令$s_i = \sqrt{d_i}>0$,
\[
\boldsymbol{S}=\text{diag}\{s_1, s_2, \cdots, s_n\}.
\]
设$\boldsymbol{C}=\boldsymbol{S}\boldsymbol{B}$, 则$\boldsymbol{A}=\boldsymbol{C}'\boldsymbol{C}$. 显然$\boldsymbol{C}=\boldsymbol{S}\boldsymbol{B}$是主对角元全为正数的上三角矩阵.

下证唯一性.设 \(B,C\) 均为主对角元全为正数的上三角阵,且 \(A = C'C = B'B\),令 \(N = BC^{-1}\),则
\begin{align}
N'N = \left( C^{-1} \right)'B'BC^{-1} = \left( C^{-1} \right)'C'CC^{-1} = I. \label{9.36-1.1}
\end{align}
因此 \(N\) 是正交矩阵。因为 \(B,C\) 均为主对角元全为正数的上三角阵,所以由上三角阵的性质可知,\(N,N^{-1},C^{-1}\) 也是主对角元全为正数的上三角阵,从而 \(N'\) 是下三角阵。
又由 \(N\) 是正交矩阵可知 \(N' = N^{-1}\) 也是下三角阵,故 \(N' = N^{-1}\) 既是上三角阵也是下三角阵,即 \(N' = N^{-1}\) 是主对角阵,从而 \(N\) 就是主对角元全为正数的主对角阵。
再由 \eqref{9.36-1.1} 式可知,\(N\) 的主对角元全为 1,因此 \(N = I\)。故 \(BC^{-1} = N = I\),即 \(B = C\)。

(3) $\Rightarrow$ (1): 这时$\boldsymbol{A}=\boldsymbol{C}'\boldsymbol{I}_n\boldsymbol{C}$, 故$\boldsymbol{A}$和$\boldsymbol{I}_n$合同, 从而$\boldsymbol{A}$正定.

{\color{blue}证法二(矩阵的QR分解):}
因为半正定阵 $A$ 是正定阵当且仅当 $A$ 是可逆矩阵,所以由可逆性和\refpro{proposition:例8.78}的结论就能推出\refpro{proposition:正定阵的3个充要条件}的结论.
\end{proof}

\begin{proposition}\label{proposition:利用顺序主子式计算二次型的标准型}
设$f(\boldsymbol{x}) = \boldsymbol{x}'\boldsymbol{A}\boldsymbol{x}$是实二次型, 相伴矩阵$\boldsymbol{A}$的前$n - 1$个顺序主子式$P_1, \cdots, P_{n - 1}$非零, 求证: 经过可逆线性变换$f$可化为下列标准型:
\[
f = P_1y_1^2 + \frac{P_2}{P_1}y_2^2 + \cdots + \frac{P_n}{P_{n - 1}}y_n^2,
\]
其中$P_n = |\boldsymbol{A}|$.
\end{proposition}
\begin{proof}
对$n$用归纳法. 当$n = 1$时结论显然成立, 假设结论对$n - 1$成立. 设
\[
\boldsymbol{A}=\begin{pmatrix}
\boldsymbol{A}_{n - 1} & \boldsymbol{\alpha} \\
\boldsymbol{\alpha}' & a_{nn}
\end{pmatrix},
\]
由于$|\boldsymbol{A}_{n - 1}| = P_{n - 1}\neq 0$, 故可对$\boldsymbol{A}$进行下列对称分块初等变换:
\begin{align*}
\boldsymbol{A}=\begin{pmatrix}
\boldsymbol{A}_{n - 1} & \boldsymbol{\alpha} \\
\boldsymbol{\alpha}' & a_{nn}
\end{pmatrix}
\rightarrow
\begin{pmatrix}
\boldsymbol{A}_{n - 1} & \boldsymbol{\alpha} \\
\boldsymbol{O} & a_{nn}-\boldsymbol{\alpha}'\boldsymbol{A}_{n - 1}^{-1}\boldsymbol{\alpha}
\end{pmatrix}
\rightarrow
\begin{pmatrix}
\boldsymbol{A}_{n - 1} & \boldsymbol{O} \\
\boldsymbol{O} & a_{nn}-\boldsymbol{\alpha}'\boldsymbol{A}_{n - 1}^{-1}\boldsymbol{\alpha}
\end{pmatrix}
=\boldsymbol{B},
\end{align*}
显然这是一个合同变换. 又因为第三类分块初等变换不改变行列式的值, 故
\[
|\boldsymbol{A}| = |\boldsymbol{A}_{n - 1}|(a_{nn}-\boldsymbol{\alpha}'\boldsymbol{A}_{n - 1}^{-1}\boldsymbol{\alpha}),
\]
即
\[
a_{nn}-\boldsymbol{\alpha}'\boldsymbol{A}_{n - 1}^{-1}\boldsymbol{\alpha}=\frac{P_n}{P_{n - 1}}.
\]
由归纳假设, 存在可逆矩阵$\boldsymbol{M}$, 使得
\[
\boldsymbol{M}'\boldsymbol{A}_{n - 1}\boldsymbol{M}=\text{diag}\left\{P_1, \frac{P_2}{P_1}, \cdots, \frac{P_{n - 1}}{P_{n - 2}}\right\}.
\]
作矩阵$\boldsymbol{C}=\begin{pmatrix}
\boldsymbol{M} & \boldsymbol{O} \\
\boldsymbol{O} & 1
\end{pmatrix}$, 则
\[
\boldsymbol{C}'\boldsymbol{B}\boldsymbol{C}=\text{diag}\left\{P_1, \frac{P_2}{P_1}, \cdots, \frac{P_n}{P_{n - 1}}\right\}. 
\] 
\end{proof}

\begin{example}
设$\boldsymbol{A}$为$n$阶正定实对称矩阵且非主对角元都是负数, 求证: $\boldsymbol{A}^{-1}$的每个元素都是正数.
\end{example}
\begin{proof}
对阶数$n$进行归纳. 当$n = 1$时结论显然成立, 设结论对$n - 1$阶成立, 现证明$n$阶的情形. 设$\boldsymbol{A}=\begin{pmatrix}
\boldsymbol{A}_{n - 1} & \boldsymbol{\alpha} \\
\boldsymbol{\alpha}' & a_{nn}
\end{pmatrix}$, 其中$\boldsymbol{A}_{n - 1}$是$\boldsymbol{A}$的第$n - 1$个顺序主子阵, 从而$\boldsymbol{A}_{n - 1}$是$n - 1$阶正定实对称矩阵且非主对角元都是负数, 故由归纳假设可知$\boldsymbol{A}_{n - 1}^{-1}$的每个元素都是正数. 利用分块初等变换可求出
\[
\boldsymbol{A}^{-1}=\begin{pmatrix}
\boldsymbol{A}_{n - 1}^{-1}+d_n^{-1}\boldsymbol{A}_{n - 1}^{-1}\boldsymbol{\alpha}\boldsymbol{\alpha}'\boldsymbol{A}_{n - 1}^{-1} & -d_n^{-1}\boldsymbol{A}_{n - 1}^{-1}\boldsymbol{\alpha} \\
-d_n^{-1}\boldsymbol{\alpha}'\boldsymbol{A}_{n - 1}^{-1} & d_n^{-1}
\end{pmatrix},
\]
其中$d_n = a_{nn}-\boldsymbol{\alpha}'\boldsymbol{A}_{n - 1}^{-1}\boldsymbol{\alpha}$.由\hyperref[proposition:打洞原理]{打洞原理}可知
\begin{align*}
|\boldsymbol{A}|=\left| \boldsymbol{A}_{n-1} \right|\left| a_{nn}-\boldsymbol{\alpha }\prime \boldsymbol{A}_{n-1}^{-1}\boldsymbol{\alpha } \right|=d_n\Rightarrow d_n=|\boldsymbol{A}|/|\boldsymbol{A}_{n-1}|>0.
\end{align*}
又注意到$\boldsymbol{A}_{n - 1}^{-1}$的每个元素都是正数, 且$\boldsymbol{\alpha}$的每个元素都是负数, 故$\boldsymbol{A}^{-1}$的每个元素都是正数. 
\end{proof}

\begin{example}
设$\boldsymbol{A}=(a_{ij})$是$n$阶正定实对称矩阵, 其逆阵$\boldsymbol{A}^{-1}=(b_{ij})$, 求证: $a_{ii}b_{ii}\geq 1$, 且等号成立当且仅当$\boldsymbol{A}$的第$i$行和列的所有元素除了$a_{ii}$之外全为零.
\end{example}
\begin{proof}
对换$\boldsymbol{A}$的第$i,n$行和列, 可将$a_{ii}$换到第$(n,n)$位置, 这相当于合同变换$\boldsymbol{P}_{in}\boldsymbol{A}\boldsymbol{P}_{in}$. 此时$(\boldsymbol{P}_{in}\boldsymbol{A}\boldsymbol{P}_{in})^{-1}=\boldsymbol{P}_{in}\boldsymbol{A}^{-1}\boldsymbol{P}_{in}$, 即对换了$\boldsymbol{A}^{-1}$的第$i,n$行和列, $b_{ii}$也换到了第$(n,n)$位置. 因此不失一般性, 只需证明$a_{nn}b_{nn}\geq 1$, 且等号成立当且仅当$\boldsymbol{A}$的$n$行和列的所有元素除了$a_{nn}$之外全为零即可. 

利用数学归纳法,对阶数$n$进行归纳. 当$n = 1$时结论显然成立, 设结论对$n - 1$阶成立, 现证明$n$阶的情形. 设$\boldsymbol{A}=\begin{pmatrix}
\boldsymbol{A}_{n - 1} & \boldsymbol{\alpha} \\
\boldsymbol{\alpha}' & a_{nn}
\end{pmatrix}$, 其中$\boldsymbol{A}_{n - 1}$是$\boldsymbol{A}$的第$n - 1$个顺序主子阵, 从而$\boldsymbol{A}_{n - 1}$是$n - 1$阶正定实对称矩阵且非主对角元都是负数, 故由归纳假设可知$\boldsymbol{A}_{n - 1}^{-1}$的每个元素都是正数. 利用分块初等变换可求出
\[
\boldsymbol{A}^{-1}=\begin{pmatrix}
\boldsymbol{A}_{n - 1}^{-1}+d_n^{-1}\boldsymbol{A}_{n - 1}^{-1}\boldsymbol{\alpha}\boldsymbol{\alpha}'\boldsymbol{A}_{n - 1}^{-1} & -d_n^{-1}\boldsymbol{A}_{n - 1}^{-1}\boldsymbol{\alpha} \\
-d_n^{-1}\boldsymbol{\alpha}'\boldsymbol{A}_{n - 1}^{-1} & d_n^{-1}
\end{pmatrix},
\]
其中$d_n = a_{nn}-\boldsymbol{\alpha}'\boldsymbol{A}_{n - 1}^{-1}\boldsymbol{\alpha}$.由此可得$b_{nn}=d_n^{-1}$, 再由$\boldsymbol{A}_{n - 1}$的正定性及\refpro{proposition:A+S的行列式的相关结论}可知$A_{n-1}^{-1}$也正定,于是
\[
b_{nn}^{-1}=d_n=a_{nn}-\boldsymbol{\alpha}'\boldsymbol{A}_{n - 1}^{-1}\boldsymbol{\alpha}\leq a_{nn},
\]
即有$a_{nn}b_{nn}\geq 1$, 且等号成立当且仅当$\boldsymbol{\alpha}=\boldsymbol{0}$. 
\end{proof}

\begin{theorem}[反对称矩阵的合同标准型]\label{theorem:反对称矩阵的合同标准型}
设$\boldsymbol{A}$是$n$阶反对称矩阵, 则$\boldsymbol{A}$必合同于下列形状的分块矩阵:
\begin{align}
\text{diag}\{\boldsymbol{S}, \cdots, \boldsymbol{S}, 0, \cdots, 0\}, \label{eq:8.2}
\end{align}
其中$\boldsymbol{S}=\begin{pmatrix}
0 & 1 \\
-1 & 0
\end{pmatrix}$. 我们称\eqref{eq:8.2}式为反对称矩阵$A$的\textbf{合同标准型}.特别地, 反对称矩阵$\boldsymbol{A}$的秩必为偶数$2r$, 其中$r$是$\boldsymbol{S}$在$\boldsymbol{A}$的上述合同标准型中的个数.
\end{theorem}
\begin{remark}
本例题给出了\refpro{proposition:反对称阵的秩必为偶数}的另一证明. 注意到在本题的证明中, 我们采用的是跨度为 2 的数学归纳法(第二数学归纳法), 故在起始步骤时需要验证$n = 1, 2$这两种情形, 但我们不难发现$n = 2$情形的证明完全包含在归纳过程的证明中, 因此可以用$n = 0, 1$的情形作为起始步骤. 需要注意的是, $n = 0$并不意味着存在零阶矩阵, 而只是说明归纳过程已经完全结束. 后面遇到跨度为 2 的数学归纳法, 我们通常都采用上述约定. 
\end{remark}
\begin{proof}
对阶数$n$进行归纳. 当$n = 0, 1$时结论显然成立, 假设结论对阶数小于$n$的反对称矩阵成立. 现有$n$阶反对称矩阵$\boldsymbol{A}$, 若$\boldsymbol{A}=\boldsymbol{O}$, 结论已成立, 故设$\boldsymbol{A}\neq\boldsymbol{O}$. 由于反对称矩阵的主对角元全为零, 故可设$\boldsymbol{A}$的$(i, j)$元素$a_{ij}\neq 0 (i < j)$, 此时$\boldsymbol{A}$的$(j, i)$元素为$-a_{ij}$. 于是利用矩阵的初等对称变换,对换$\boldsymbol{A}$的第一行与第$i$行, 再对换第一列与第$i$列; 对换第二行与第$j$行, 再对换第二列与第$j$列; 然后将第一行乘以$\frac{1}{a_{ij}}$, 第一列乘以$\frac{1}{a_{ij}}$; 最后得到$\boldsymbol{A}$合同于下列形状的矩阵:
\[
\boldsymbol{M}=\begin{pmatrix}
\boldsymbol{S} & \boldsymbol{B} \\
-\boldsymbol{B}' & \boldsymbol{A}_{n - 2}
\end{pmatrix},
\]
其中$\boldsymbol{A}_{n - 2}$是$n - 2$阶反对称矩阵,$S=\left( \begin{matrix}
0&		1\\
-1&		0\\
\end{matrix} \right) $. 显然$\boldsymbol{S}$是可逆反对称矩阵, 对$\boldsymbol{M}$作下列对称分块初等变换: 第一分块行左乘$\boldsymbol{B}'\boldsymbol{S}^{-1}$加到第二分块行上, 再将第一分块列右乘$(\boldsymbol{B}'\boldsymbol{S}^{-1})'=-\boldsymbol{S}^{-1}\boldsymbol{B}$加到第二分块列上, 于是$\boldsymbol{A}$合同于下列矩阵:
\[
\boldsymbol{N}=\begin{pmatrix}
\boldsymbol{S} & \boldsymbol{O} \\
\boldsymbol{O} & \boldsymbol{A}_{n - 2}+\boldsymbol{B}'\boldsymbol{S}^{-1}\boldsymbol{B}
\end{pmatrix}.
\]
注意到$\boldsymbol{A}_{n - 2}+\boldsymbol{B}'\boldsymbol{S}^{-1}\boldsymbol{B}$是$n - 2$阶反对称矩阵, 故由归纳假设它合同于\eqref{eq:8.2}式形状的矩阵, 因此分块对角矩阵$\boldsymbol{N}$也合同于\eqref{eq:8.2}式形状的矩阵, 结论得证.

由\refpro{proposition:反对称阵的秩必为偶数}可知反称矩阵$\boldsymbol{A}$的秩必为偶数$2r$,从而反称矩阵$\boldsymbol{A}$的秩等于其合同标准型的秩等于$2r$.设$\boldsymbol{A}$的上述合同标准型中$\boldsymbol{S}$的个数为$k$,则由$S$的秩为2及\drefpro{proposition:矩阵秩的基本公式}{矩阵秩的基本公式3}可得
\begin{align*}
\mathrm{r}\left( \boldsymbol{A} \right) =k\mathrm{r}\left( \boldsymbol{S} \right) =2k=2r.
\end{align*}
故$k=r$.
\end{proof}

\begin{proposition}\label{proposition:实反称矩阵的行列式必非负}
求证: $n$阶实反对称矩阵$\boldsymbol{A}$的行列式值总是非负实数.
\end{proposition}
\begin{proof}
由\refthe{theorem:反对称矩阵的合同标准型}可知, 存在非异实矩阵$\boldsymbol{C}$, 使得
\[
\boldsymbol{C}'\boldsymbol{A}\boldsymbol{C}=\text{diag}\{\boldsymbol{S}, \cdots, \boldsymbol{S}, 0, \cdots, 0\},
\]
其中$\boldsymbol{S}=\begin{pmatrix}
0 & 1 \\
-1 & 0
\end{pmatrix}$. 若$\boldsymbol{A}$是奇异阵, 则$|\boldsymbol{A}| = 0$, 结论显然成立. 若$\boldsymbol{A}$是非异阵, 则由上式可得$|\boldsymbol{A}|\cdot|\boldsymbol{C}|^2 = |\boldsymbol{S}|^{\frac{n}{2}} = 1$, 从而$|\boldsymbol{A}|>0$. 
\end{proof}

\begin{proposition}\label{proposition:反称矩阵A的I_n+A的行列式相关结论}
设$\boldsymbol{A}$为$n$阶实反对称矩阵, 求证:

(1) $|\boldsymbol{I}_n+\boldsymbol{A}|\geq 1 + |\boldsymbol{A}|$, 且等号成立当且仅当$n\leq 2$或当$n\geq 3$时, $\boldsymbol{A}=\boldsymbol{O}$.

(2) $|\boldsymbol{I}_n+\boldsymbol{A}|\geq 1$, 且等号成立当且仅当$\boldsymbol{A}=\boldsymbol{O}$.
\end{proposition}
\begin{proof}
(1) 由\refpro{proposition:直接计算两个矩阵和的行列式}可知
\[
|\boldsymbol{I}_n+\boldsymbol{A}| = |\boldsymbol{I}_n|+|\boldsymbol{A}|+\sum_{1\leq k\leq n - 1}\left(\sum_{1\leq i_1 < i_2 < \cdots < i_k\leq n} \boldsymbol{A}\begin{pmatrix}
i_1 & i_2 & \cdots & i_k \\
i_1 & i_2 & \cdots & i_k
\end{pmatrix}\right).
\]
注意到$\boldsymbol{A}\begin{pmatrix}
i_1 & i_2 & \cdots & i_k \\
i_1 & i_2 & \cdots & i_k
\end{pmatrix}$是$k$阶实反对称行列式, 故由\refpro{proposition:实反称矩阵的行列式必非负}可知其值大于等于零, 于是$|\boldsymbol{I}_n+\boldsymbol{A}|\geq 1 + |\boldsymbol{A}|$成立. 当$n\leq 2$时, 容易验证不等式的等号成立. 当$n\geq 3$时, 若不等式的等号成立, 则必有
\[
\boldsymbol{A}\begin{pmatrix}
i & j \\
i & j
\end{pmatrix}=\begin{vmatrix}
0 & a_{ij} \\
-a_{ij} & 0
\end{vmatrix}=a_{ij}^2 = 0,
\]
即有$a_{ij} = 0 (1\leq i < j\leq n)$, 从而$\boldsymbol{A}=\boldsymbol{O}$.

(2) 同理可证, 细节留给读者完成. 
\end{proof}


















\end{document}
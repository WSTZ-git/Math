\documentclass[../../main.tex]{subfiles}
\graphicspath{{\subfix{../../image/}}} % 指定图片目录,后续可以直接使用图片文件名。

% 例如:
% \begin{figure}[H]
% \centering
% \includegraphics[scale=0.4]{image-01.01}
% \caption{图片标题}
% \label{figure:image-01.01}
% \end{figure}
% 注意:上述\label{}一定要放在\caption{}之后,否则引用图片序号会只会显示??.

\begin{document}

\section{二次型的化简和矩阵的合同}

\begin{definition}[二次型]\label{definition:二次型}
设 $f$ 是数域 $\mathbb{K}$ 上的 $n$ 元二次齐次多项式:
\begin{align}\label{equation-definition:二次型}
f(x_1,x_2,\cdots,x_n)&=a_{11}x_1^2 + 2a_{12}x_1x_2+\cdots + 2a_{1n}x_1x_n\\
&+a_{22}x_2^2+\cdots + 2a_{2n}x_2x_n+\cdots + a_{nn}x_n^2,
\end{align}
称 $f$ 为数域 $\mathbb{K}$ 上的 $n$ 元二次型,简称\textbf{二次型}. 
\end{definition}

\begin{definition}[二次型与矩阵的相伴]\label{definition:二次型与矩阵的相伴}
用矩阵的乘法我们可以把\eqref{equation-definition:二次型} 式写成矩阵相乘的形式:
\begin{align}
f(x_1,x_2,\cdots,x_n)=\boldsymbol{x}'\boldsymbol{A}\boldsymbol{x},\label{eq:8.1.4}
\end{align}
其中
\[
\boldsymbol{A}=\begin{pmatrix}
a_{11} & a_{12} & \cdots & a_{1n} \\
a_{21} & a_{22} & \cdots & a_{2n} \\
\vdots & \vdots & & \vdots \\
a_{n1} & a_{n2} & \cdots & a_{nn}
\end{pmatrix}, \boldsymbol{x}=\begin{pmatrix}
x_1 \\
x_2 \\
\vdots \\
x_n
\end{pmatrix}.
\]
在矩阵 $\boldsymbol{A}$ 中,$a_{ij}=a_{ji}$ 对一切 $i,j$ 成立,也就是说矩阵 $\boldsymbol{A}$ 是一个对称阵. 由此可知,给定数域 $\mathbb{K}$ 上的一个 $n$ 元二次型,我们就得到了 $\mathbb{K}$ 上的一个 $n$ 阶对称阵 $\boldsymbol{A}$,称为该二次型的\textbf{相伴矩阵或系数矩阵}.

反过来,若给定 $\mathbb{K}$ 上的一个 $n$ 阶对称阵 $\boldsymbol{A}$,则由 \eqref{eq:8.1.4} 式,我们可以得到 $\mathbb{K}$ 上的一个二次型,称为对称阵 $\boldsymbol{A}$ 的\textbf{相伴二次型}.
\end{definition}

\begin{theorem}\label{theorem:二次型与其相伴矩阵一一对应}
证明:二次型与其相伴矩阵一一对应.此即
\begin{enumerate}[(1)]
\item (一个对称矩阵对应唯一一个二次型)设$\boldsymbol{A}=\boldsymbol{B}$都是对称矩阵,则$f = \boldsymbol{x}'\boldsymbol{A}\boldsymbol{x}=\boldsymbol{x}'\boldsymbol{B}\boldsymbol{x}$.

\item (一个二次型对应唯一一个系数矩阵)设 $f = \boldsymbol{x}'\boldsymbol{A}\boldsymbol{x}=\boldsymbol{x}'\boldsymbol{B}\boldsymbol{x}$,其中$\boldsymbol{A},\boldsymbol{B}$都是对称矩阵,则有$\boldsymbol{A}=\boldsymbol{B}$. 
\end{enumerate}
\end{theorem}
\begin{remark}
事实上,如果我们不限制矩阵是对称阵,则系数矩阵将不唯一,这样会给用矩阵方法研究二次型带来困难.
\end{remark}
\begin{proof}
\begin{enumerate}[(1)]
\item 由二次型的定义显然得证.

\item 由$f = \boldsymbol{x}'\boldsymbol{A}\boldsymbol{x}=\boldsymbol{x}'\boldsymbol{B}\boldsymbol{x}$可知$\boldsymbol{x}'\boldsymbol{A-B}\boldsymbol{x}=\boldsymbol{O}$.于是只需证$\boldsymbol{A-B}=\boldsymbol{O}$即可.又$\boldsymbol{A-B}$仍是对称阵.
这等价于证明下面的结论: 设 $\boldsymbol{A}=(a_{ij})$ 是 $n$ 阶对称阵,若 $\boldsymbol{\alpha}'\boldsymbol{A}\boldsymbol{\alpha}=0$ 对一切 $\boldsymbol{\alpha}$ 成立,则 $\boldsymbol{A}=\boldsymbol{O}$. 令 $\boldsymbol{\alpha}=\boldsymbol{e}_i$ 是 $n$ 维标准单位列向量,则 $a_{ii}=\boldsymbol{e}_i'\boldsymbol{A}\boldsymbol{e}_i = 0$. 再令 $\boldsymbol{\alpha}=\boldsymbol{e}_i+\boldsymbol{e}_j (i\neq j)$,则
\begin{align*}
0&=(\boldsymbol{e}_i+\boldsymbol{e}_j)'\boldsymbol{A}(\boldsymbol{e}_i+\boldsymbol{e}_j)=\boldsymbol{e}_i'\boldsymbol{A}\boldsymbol{e}_i+\boldsymbol{e}_j'\boldsymbol{A}\boldsymbol{e}_j+\boldsymbol{e}_i'\boldsymbol{A}\boldsymbol{e}_j+\boldsymbol{e}_j'\boldsymbol{A}\boldsymbol{e}_i=a_{ij}+a_{ji},
\end{align*}
因为 $a_{ij}=a_{ji}$,故 $a_{ij}=0 (i\neq j)$,于是 $\boldsymbol{A}=\boldsymbol{O}$. 这表明用对称阵来表示二次型时,系数矩阵是唯一的. 
\end{enumerate}
\end{proof}

\begin{definition}[矩阵的合同关系]\label{definition:矩阵的合同关系}
设 $\boldsymbol{A},\boldsymbol{B}$ 是数域 $\mathbb{K}$ 上的 $n$ 阶矩阵,若存在 $n$ 阶非异阵 $\boldsymbol{C}$,使
\begin{align*}
\boldsymbol{B}=\boldsymbol{C}'\boldsymbol{A}\boldsymbol{C},
\end{align*}
则称 $\boldsymbol{B}$ 与 $\boldsymbol{A}$ 是\textbf{合同的},或称 $\boldsymbol{B}$ 与 $\boldsymbol{A}$ 具有\textbf{合同关系}.
\end{definition}

\begin{theorem}
矩阵的合同关系是一个等价关系.
\end{theorem}
\begin{proof}
\begin{enumerate}
\item 任一矩阵 $\boldsymbol{A}$ 与自己合同,因为 $\boldsymbol{A}=\boldsymbol{I}'\boldsymbol{A}\boldsymbol{I}$;
\item 若 $\boldsymbol{B}$ 与 $\boldsymbol{A}$ 合同,则 $\boldsymbol{A}$ 与 $\boldsymbol{B}$ 合同. 这是因为若 $\boldsymbol{B}=\boldsymbol{C}'\boldsymbol{A}\boldsymbol{C}$,则 $\boldsymbol{A}=(\boldsymbol{C}')^{-1}\boldsymbol{B}\boldsymbol{C}^{-1}=(\boldsymbol{C}^{-1})'\boldsymbol{B}\boldsymbol{C}^{-1}$;
\item 若 $\boldsymbol{B}$ 与 $\boldsymbol{A}$ 合同,$\boldsymbol{D}$ 与 $\boldsymbol{B}$ 合同,则 $\boldsymbol{D}$ 与 $\boldsymbol{A}$ 合同. 事实上,若 $\boldsymbol{B}=\boldsymbol{C}'\boldsymbol{A}\boldsymbol{C}$,$\boldsymbol{D}=\boldsymbol{H}'\boldsymbol{B}\boldsymbol{H}$,则 $\boldsymbol{D}=\boldsymbol{H}'\boldsymbol{C}'\boldsymbol{A}\boldsymbol{C}\boldsymbol{H}=(\boldsymbol{C}\boldsymbol{H})'\boldsymbol{A}(\boldsymbol{C}\boldsymbol{H})$.
\end{enumerate} 
\end{proof}

\begin{lemma}[初等合同变换]\label{lemma:初等合同变换}
\begin{enumerate}
\item 对称阵 $\boldsymbol{A}$ 的下列的\textbf{初等对称变换}都是合同变换:
\begin{enumerate}[(1)]
\item 对换 $\boldsymbol{A}$ 的第 $i$ 行与第 $j$ 行,再对换第 $i$ 列与第 $j$ 列;
\item 将非零常数 $k$ 乘以 $\boldsymbol{A}$ 的第 $i$ 行,再将 $k$ 乘以第 $i$ 列;
\item 将 $\boldsymbol{A}$ 的第 $i$ 行乘以 $k$ 加到第 $j$ 行上,再将第 $i$ 列乘以 $k$ 加到第 $j$ 列上.
\end{enumerate}

\item 分块对称矩阵\(A\)的下列\textbf{对称分块初等变换}都是合同变换:
\begin{enumerate}[(1)]
\item 对换\(A\)的第\(i\)分块行和第\(j\)分块行,再对换第\(i\)分块列和第\(j\)分块列;
\item 将\(A\)的第\(i\)分块行左乘可逆矩阵\(M\),再将第\(i\)分块列右乘\(M'\);
\item 将\(A\)的第\(i\)分块行左乘矩阵\(M\)加到第\(j\)分块行上,再将第\(i\)分块列右乘\(M'\)加到第\(j\)分块列上。
\end{enumerate}
\end{enumerate}
\end{lemma}
\begin{proof}
\begin{enumerate}
\item 上述初等对称变换相当于将一个初等矩阵左乘以 $\boldsymbol{A}$ 后再将这个初等矩阵的转置右乘之,因此是合同变换.此即
\begin{enumerate}[(1)]
\item $\text{对换}\boldsymbol{A}\text{的第}i\text{行与第}j\text{行},\text{再对换第}i\text{列与第}j\text{列} \iff A\rightarrow P_{ij}AP_{ij}'=P_{ij}AP_{ij};$

\item $\text{将非零常数}k\text{乘以}\boldsymbol{A}\text{的第}i\text{行},\text{再将}k\text{乘以第}i\text{列}\Longleftrightarrow A\rightarrow P_i\left( k \right) AP_{i}\left( k \right)= P_i\left( k \right) AP_{i}\left( k \right)^{\prime};$

\item $\text{将}\boldsymbol{A}\text{的第}i\text{行乘以}k\text{加到第}j\text{行上},\text{再将第}i\text{列乘以}k\text{加到第}j\text{列上}\Longleftrightarrow A\rightarrow P_{ij}\left( k \right) AP_{ij}\left( k \right)^{\prime} .$
\end{enumerate}

\item 上述对称分块初等变换相当于将一个初等分块矩阵左乘以 $\boldsymbol{A}$ 后再将这个初等分块矩阵的转置右乘之,因此是合同变换.此即
\begin{enumerate}[(1)]
\item 对换\(A\)的第\(i\)分块行和第\(j\)分块行,再对换第\(i\)分块列和第\(j\)分块列$\iff A\rightarrow P_{ij}AP_{ij}'=P_{ij}AP_{ij};$

\item 将\(A\)的第\(i\)分块行左乘可逆矩阵\(M\),再将第\(i\)分块列右乘\(M'\)$\Longleftrightarrow A\rightarrow P_i\left( M \right) AP_{i}\left( M \right)= P_i\left( M \right) AP_{i}\left( M \right)^{\prime};$

\item 将\(A\)的第\(i\)分块行左乘矩阵\(M\)加到第\(j\)分块行上,再将第\(i\)分块列右乘\(M'\)加到第\(j\)分块列上

$\Longleftrightarrow A\rightarrow P_{ij}\left( M \right) AP_{ij}\left( M \right)^{\prime} .$
\end{enumerate}
\end{enumerate}
\end{proof}

\begin{lemma}\label{lemma:非零对称阵合同于(1,1)元素不为零的矩阵}
设 $\boldsymbol{A}$ 是数域 $\mathbb{K}$ 上的非零对称阵,则必存在非异阵 $\boldsymbol{C}$,使 $\boldsymbol{C}'\boldsymbol{A}\boldsymbol{C}$ 的第 $(1,1)$ 元素不等于零.
\end{lemma}
\begin{proof}
若 $a_{11} = 0$,而 $a_{ii}\neq 0$,则将 $\boldsymbol{A}$ 的第一行与第 $i$ 行对换,再将第一列与第 $i$ 列对换,得到的矩阵的第 $(1,1)$ 元素不为零. 根据上述引理,这样得到的矩阵和原矩阵合同.

若所有的 $a_{ii}=0 (i = 1,2,\cdots,n)$,设 $a_{ij}\neq 0 (i\neq j)$,将 $\boldsymbol{A}$ 的第 $j$ 行加到第 $i$ 行上,再将第 $j$ 列加到第 $i$ 列上. 因为 $\boldsymbol{A}$ 是对称阵,$a_{ij}=a_{ji}\neq 0$,于是第 $(i,i)$ 元素是 $2a_{ij}\neq 0$,再用前面的办法使第 $(1,1)$ 元素不等于零. 根据上述引理,这样得到的矩阵和原矩阵仍合同,这就证明了结论. 
\end{proof}

\begin{theorem}[对称矩阵的合同标准型]\label{theorem:对称矩阵的合同标准型}
设 $\boldsymbol{A}$ 是数域 $\mathbb{K}$ 上的 $n$ 阶对称阵,则必存在 $\mathbb{K}$ 上的 $n$ 阶非异阵 $\boldsymbol{C}$,使 $\boldsymbol{C}'\boldsymbol{A}\boldsymbol{C}$ 为对角阵.进而
\begin{align*}
\boldsymbol{C}' \boldsymbol{AC}=\mathrm{diag}\left\{ d_1,d_2,\cdots ,d_r,0,\cdots ,0 \right\} .
\end{align*}
其中$r=\mathrm{r}\left( \boldsymbol{C}' \boldsymbol{AC} \right)=\mathrm{r}\left( \boldsymbol{A} \right) .$,$d_i\ne 0,$ $i=1,2,\cdots,r.$即秩 $r$ 是矩阵合同关系下的一个不变量.
\end{theorem}
\begin{proof}
由\reflem{lemma:非零对称阵合同于(1,1)元素不为零的矩阵},不妨设 $\boldsymbol{A}=(a_{ij})$ 中 $a_{11}\neq 0$. 若 $a_{i1}\neq 0$,则可将第一行乘以 $-a_{11}^{-1}a_{i1}$ 加到第 $i$ 行上,再将第一列乘以 $-a_{11}^{-1}a_{i1}$ 加到第 $i$ 列上. 由于 $a_{i1}=a_{1i}$,故得到的矩阵的第 $(1,i)$ 元素及第 $(i,1)$ 元素均等于零. 由\hyperref[lemma:初等合同变换]{初等合同变换}可知,新得到的矩阵与 $\boldsymbol{A}$ 是合同的. 依次这样做下去,可把 $\boldsymbol{A}$ 的第一行与第一列除 $a_{11}$ 外的元素都消去,于是 $\boldsymbol{A}$ 合同于下列矩阵:
\[
\begin{pmatrix}
a_{11} & 0 & 0 & \cdots & 0 \\
0 & b_{22} & b_{23} & \cdots & b_{2n} \\
0 & b_{32} & b_{33} & \cdots & b_{3n} \\
\vdots & \vdots & \vdots & & \vdots \\
0 & b_{n2} & b_{n3} & \cdots & b_{nn}
\end{pmatrix}.
\]
上式右下角是一个 $n - 1$ 阶对称阵,记为 $\boldsymbol{A}_1$. 因此由归纳假设,存在 $n - 1$ 阶非异阵 $\boldsymbol{D}$,使 $\boldsymbol{D}'\boldsymbol{A}_1\boldsymbol{D}$ 为对角阵,于是
\begin{align*}
\begin{pmatrix}
1 & \boldsymbol{O} \\
\boldsymbol{O} & \boldsymbol{D}'
\end{pmatrix}
\begin{pmatrix}
a_{11} & \boldsymbol{O} \\
\boldsymbol{O} & \boldsymbol{A}_1
\end{pmatrix}
\begin{pmatrix}
1 & \boldsymbol{O} \\
\boldsymbol{O} & \boldsymbol{D}
\end{pmatrix}
=
\begin{pmatrix}
a_{11} & \boldsymbol{O} \\
\boldsymbol{O} & \boldsymbol{D}'\boldsymbol{A}_1\boldsymbol{D}
\end{pmatrix}
\end{align*}
是一个对角阵. 显然
\[
\begin{pmatrix}
1 & \boldsymbol{O} \\
\boldsymbol{O} & \boldsymbol{D}'
\end{pmatrix}
=
\begin{pmatrix}
1 & \boldsymbol{O} \\
\boldsymbol{O} & \boldsymbol{D}
\end{pmatrix}',
\]
因此 $\boldsymbol{A}$ 合同于对角阵.
\end{proof}




















\end{document}
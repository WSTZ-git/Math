\documentclass[../../main.tex]{subfiles}
\graphicspath{{\subfix{../../image/}}} % 指定图片目录,后续可以直接使用图片文件名。

% 例如:
% \begin{figure}[h]
% \centering
% \includegraphics{image-01.01}
% \caption{图片标题}
% \label{fig:image-01.01}
% \end{figure}
% 注意:上述\label{}一定要放在\caption{}之后,否则引用图片序号会只会显示??.

\begin{document}

\section{合同标准型的应用}

引进标准型的目的是为了简化问题的讨论. 应用\hyperref[theorem:对称矩阵的合同标准型]{对称矩阵的合同标准型} (复相合标准型) 可以简化二次型和对称矩阵 (Hermite 型和 Hermite 矩阵) 有关问题的讨论. 其方法是先对标准型证明所需结论, 若结论在合同 (复相合) 变换下不变, 就可以过渡到一般的情形. 这种做法和相抵标准型、相似标准型是完全类似的.

\begin{proposition}[对称矩阵的秩1分解]\label{proposition:对称矩阵的秩1分解}
秩等于$r$的对称矩阵可以表成$r$个秩等于1的对称矩阵之和.
\end{proposition}
\begin{remark}
这里没说是在哪个数域上的对称矩阵,因此我们应该考虑一般数域.只能使用\hyperref[theorem:对称矩阵的合同标准型]{对称矩阵的合同标准型}.
\end{remark}
\begin{proof}
设$\boldsymbol{C}$是可逆矩阵, 使得
\[
\boldsymbol{C}'\boldsymbol{A}\boldsymbol{C}=\text{diag}\{a_1, \cdots, a_r, 0, \cdots, 0\},
\]
其中$a_i\neq 0 (1\leq i\leq r)$, 则
\[
\boldsymbol{A}=(\boldsymbol{C}^{-1})'a_1\boldsymbol{E}_{11}\boldsymbol{C}^{-1}+\cdots+(\boldsymbol{C}^{-1})'a_r\boldsymbol{E}_{rr}\boldsymbol{C}^{-1},
\]
其中$\boldsymbol{E}_{ii}$是第$(i, i)$元素为 1, 其余元素全为 0 的基础矩阵, 从而每个$(\boldsymbol{C}^{-1})'a_i\boldsymbol{E}_{ii}\boldsymbol{C}^{-1}$都是秩等于 1 的对称矩阵. 
\end{proof}

\begin{proposition}\label{proposition:复对称阵都可分解成相同秩的复矩阵与其转置的乘积}
设$\boldsymbol{A}$为$n$阶复对称矩阵且秩等于$r$, 求证: $\boldsymbol{A}$可分解为$\boldsymbol{A}=\boldsymbol{T}'\boldsymbol{T}$, 其中$\boldsymbol{T}$是秩等于$r$的$n$阶复矩阵.
\end{proposition}
\begin{proof}
$\boldsymbol{A}$合同于对角矩阵, 即存在可逆矩阵$\boldsymbol{C}$, 使得
\[
\boldsymbol{A}=\boldsymbol{C}'\text{diag}\{c_1, \cdots, c_r, 0, \cdots, 0\}\boldsymbol{C},
\]
其中$c_i\neq 0 (1\leq i\leq r)$. 令$d_i = \sqrt{c_i}$(取定一个平方根即可),
\[
\boldsymbol{D}=\text{diag}\{d_1, \cdots, d_r, 0, \cdots, 0\},
\]
则$\boldsymbol{A}=(\boldsymbol{D}\boldsymbol{C})'(\boldsymbol{D}\boldsymbol{C})$. 令$\boldsymbol{T}=\boldsymbol{D}\boldsymbol{C}$即得结论. 
\end{proof}

\begin{proposition}
求证: 任一$n$阶复矩阵$\boldsymbol{A}$都相似于一个复对称矩阵.
\end{proposition}
\begin{proof}
由\refpro{proposition:复矩阵的对称可逆分解}可得$\boldsymbol{A}=\boldsymbol{B}\boldsymbol{C}$, 其中$\boldsymbol{B}, \boldsymbol{C}$都是复对称矩阵, 并且可以随意指定$\boldsymbol{B}, \boldsymbol{C}$中的一个为非异阵. 不妨设$\boldsymbol{C}$是非异阵, 则由\refpro{proposition:复对称阵都可分解成相同秩的复矩阵与其转置的乘积}可得$\boldsymbol{C}=\boldsymbol{T}'\boldsymbol{T}$, 其中$\boldsymbol{T}$是非异复矩阵. 于是$\boldsymbol{A}=\boldsymbol{B}\boldsymbol{C}=\boldsymbol{B}\boldsymbol{T}'\boldsymbol{T}$相似于$\boldsymbol{T}(\boldsymbol{B}\boldsymbol{T}'\boldsymbol{T})\boldsymbol{T}^{-1}=\boldsymbol{T}\boldsymbol{B}\boldsymbol{T}'$, 这是一个复对称矩阵. 
\end{proof}

\begin{proposition}\label{proposition:实对称阵和逆的二次型有相同的正负惯性指数}
设实二次型$f$和$g$的系数矩阵分别是$\boldsymbol{A}$和$\boldsymbol{A}^{-1}$, 求证: $f$和$g$有相同的正负惯性指数.
\end{proposition}
\begin{proof}
设$\boldsymbol{C}'\boldsymbol{A}\boldsymbol{C}=\text{diag}\{a_1, a_2, \cdots, a_n\}$, 则
\[
\boldsymbol{C}^{-1}\boldsymbol{A}^{-1}(\boldsymbol{C}^{-1})'=(\boldsymbol{C}'\boldsymbol{A}\boldsymbol{C})^{-1}=\text{diag}\{a_1^{-1}, a_2^{-1}, \cdots, a_n^{-1}\}.
\]
因为$a_i$和$a_i^{-1}$有相同的正负性, 所以$\boldsymbol{A}$和$\boldsymbol{A}^{-1}$有相同的正负惯性指数.
\end{proof}

\begin{proposition}\label{proposition:系数阵的行列式小于零的二次型必存在小于零的一点}
设$f$是$n$元实二次型, 其系数矩阵$\boldsymbol{A}$满足$|\boldsymbol{A}|<0$, 求证: 必存在一组实数$a_1, a_2, \cdots, a_n$, 使得
\[
f(a_1, a_2, \cdots, a_n)<0.
\]
\end{proposition}
\begin{proof}
设$\boldsymbol{C}$是可逆矩阵, 使得$\boldsymbol{C}'\boldsymbol{A}\boldsymbol{C}=\boldsymbol{B}$为对角矩阵. 注意到$|\boldsymbol{A}||\boldsymbol{C}|^2 = |\boldsymbol{B}|$, 故$|\boldsymbol{B}|<0$. 因为对调对角矩阵的主对角元后得到的矩阵和原矩阵合同, 故不失一般性, 可设$\boldsymbol{B}$的主对角元前$r$个为负, 后$n - r$个为正, 于是$r$必是奇数. 作$n$维列向量$\boldsymbol{\alpha}=(1, \cdots, 1, 0, \cdots, 0)'$, 其中有$r$个 1. 又令$(a_1, a_2, \cdots, a_n)'=\boldsymbol{C}\boldsymbol{\alpha}$, 则$f(a_1, a_2, \cdots, a_n)=(\boldsymbol{C}\boldsymbol{\alpha})'\boldsymbol{A}(\boldsymbol{C}\boldsymbol{\alpha})=\boldsymbol{\alpha}'\boldsymbol{B}\boldsymbol{\alpha}<0$. 

也可用反证法来证明, 若结论不成立, 则$f$是半正定型, 从而$\boldsymbol{A}$是半正定阵, 于是$|\boldsymbol{A}|\geq 0$, 矛盾!
\end{proof}

\begin{proposition}\label{proposition:实二次型只在原点处为零必是正定或负定型}
如果实二次型$f(x_1, x_2, \cdots, x_n)$仅在$x_1 = x_2 = \cdots = x_n = 0$时为零, 证明: $f$必是正定型或负定型.
\end{proposition}
\begin{proof}
设$f$的正负惯性指数分别为$p,q$, 秩为$r$, 我们分情况来讨论.

若$f$是不定型, 即$p > 0$且$q > 0$, 则存在可逆线性变换$\boldsymbol{x}=\boldsymbol{C}\boldsymbol{y}$, 使得$f$可化简为如下规范标准型:
\[
f = y_1^2 + \cdots + y_p^2 - y_{p + 1}^2 - \cdots - y_r^2.
\]
取$\boldsymbol{y}=(b_1, b_2, \cdots, b_n)'$, 其中$b_1 = b_{p + 1} = 1$, 其他$b_i$全为零, 则$\boldsymbol{x}=\boldsymbol{C}\boldsymbol{y}=(a_1, a_2, \cdots, a_n)'$是一个非零列向量, 但$f(a_1, a_2, \cdots, a_n) = 0$, 这与假设矛盾, 所以$f$不是不定型.

若$f$是半正定型, 但非正定型, 即$p = r < n$, 则存在可逆线性变换$\boldsymbol{x}=\boldsymbol{C}\boldsymbol{y}$, 使得$f$可化简为如下规范标准型:
\[
f = y_1^2 + \cdots + y_r^2.
\]
取$\boldsymbol{y}=(b_1, b_2, \cdots, b_n)'$, 其中$b_n = 1$, 其他$b_i$全为零, 则$\boldsymbol{x}=\boldsymbol{C}\boldsymbol{y}=(a_1, a_2, \cdots, a_n)'$是一个非零列向量, 但$f(a_1, a_2, \cdots, a_n) = 0$, 这与假设矛盾, 所以$f$不是非正定型的半正定型. 同理可证$f$也不是非负定型的半负定型.

综上所述,$f$必是正定型或负定型. 
\end{proof}

\begin{proposition}\label{proposition:半正定阵的伴随也半正定}
设$\boldsymbol{A}$为$n$阶实对称矩阵, 若$\boldsymbol{A}$半正定, 求证: $\boldsymbol{A}^*$也半正定.
\end{proposition}
\begin{proof}
因为$\boldsymbol{A}$半正定, 故存在非异阵$\boldsymbol{C}$, 使得
\[
\boldsymbol{C}'\boldsymbol{A}\boldsymbol{C}=\begin{pmatrix}
\boldsymbol{I}_r & \boldsymbol{O} \\
\boldsymbol{O} & \boldsymbol{O}
\end{pmatrix}.
\]
若$r = n$, 则$\boldsymbol{A}$是正定阵, 上式两边同取伴随可得$\boldsymbol{C}^*\boldsymbol{A}^*(\boldsymbol{C}^*)'=\boldsymbol{I}_n^* = \boldsymbol{I}_n$, 故$\boldsymbol{A}^*$也是正定阵. 若$r = n - 1$, 则上式两边同取伴随可得
\[
\boldsymbol{C}^*\boldsymbol{A}^*(\boldsymbol{C}^*)'=\begin{pmatrix}
\boldsymbol{I}_{n - 1} & \boldsymbol{O} \\
\boldsymbol{O} & 0
\end{pmatrix}^*=\begin{pmatrix}
\boldsymbol{O} & \boldsymbol{O} \\
\boldsymbol{O} & 1
\end{pmatrix},
\]
因此$\boldsymbol{A}^*$的正惯性指数为 1, 秩也为 1, 从而是半正定阵. 若$r < n - 1$, 则由\hyperref[theorem:伴随矩阵的秩]{定理\ref{theorem:伴随矩阵的秩}}可知$\boldsymbol{A}^*=\boldsymbol{O}$, 结论自然成立.
\end{proof}

\begin{proposition}[正定和半正定阵的判定准则之一]\label{proposition:正定和半正定阵的判定准则}
设$\boldsymbol{A}$为$n$阶实对称矩阵, 求证:

(1) $\boldsymbol{A}$是正定阵的充要条件是存在$n$阶非异实矩阵$\boldsymbol{C}$, 使得$\boldsymbol{A}=\boldsymbol{C}'\boldsymbol{C}$.

(2) $\boldsymbol{A}$是半正定阵的充要条件是存在$n$阶实矩阵$\boldsymbol{C}$, 使得$\boldsymbol{A}=\boldsymbol{C}'\boldsymbol{C}$. 特别地, $|\boldsymbol{A}| = |\boldsymbol{C}|^2\geq 0$.
\end{proposition}
\begin{proof}
(1) 由\refthe{theorem:正定矩阵的充要条件}可知, $\boldsymbol{A}$是正定阵当且仅当$\boldsymbol{A}$合同于$\boldsymbol{I}_n$, 即存在非异实矩阵$\boldsymbol{C}$, 使得$\boldsymbol{A}=\boldsymbol{C}'\boldsymbol{I}_n\boldsymbol{C}=\boldsymbol{C}'\boldsymbol{C}$.

(2) 由\refthe{theorem:正定矩阵的充要条件}可知, $\boldsymbol{A}$是半正定阵当且仅当$\boldsymbol{A}$合同于$\text{diag}\{\boldsymbol{I}_r, \boldsymbol{O}\}$, 即存在非异实矩阵$\boldsymbol{B}$, 使得$\boldsymbol{A}=\boldsymbol{B}'\text{diag}\{\boldsymbol{I}_r, \boldsymbol{O}\}\boldsymbol{B}$. 令$\boldsymbol{C}=\text{diag}\{\boldsymbol{I}_r, \boldsymbol{O}\}\boldsymbol{B}$, 则$\boldsymbol{A}=\boldsymbol{C}'\boldsymbol{C}$. 反之, 若$\boldsymbol{A}=\boldsymbol{C}'\boldsymbol{C}$, 其中$\boldsymbol{C}$是实矩阵, 则对任一$n$维实列向量$\boldsymbol{\alpha}$, $\boldsymbol{\alpha}'\boldsymbol{A}\boldsymbol{\alpha}=\boldsymbol{\alpha}'\boldsymbol{C}'\boldsymbol{C}\boldsymbol{\alpha}=(\boldsymbol{C}\boldsymbol{\alpha})'(\boldsymbol{C}\boldsymbol{\alpha})\geq 0$, 由定义可知$\boldsymbol{A}$为半正定阵.
\end{proof}

\begin{lemma}\label{lemma:矩阵二次型技巧}
设$A,B,P,Q$都是$n$阶矩阵,$\alpha,\beta$为$n$维列向量,则
\begin{align*}
\alpha ' AP\alpha +\beta ' B\beta -2\alpha ' \beta =\left( \begin{matrix}
\alpha '&		\beta '\\
\end{matrix} \right) \left( \begin{matrix}
A&		-I_n\\
-I_n&		B\\
\end{matrix} \right) \left( \begin{array}{c}
\alpha\\
\beta\\
\end{array} \right) ,
\\
P' AP+Q' BQ-2P' Q=\left( \begin{matrix}
P'&		Q'\\
\end{matrix} \right) \left( \begin{matrix}
A&		-I_n\\
-I_n&		B\\
\end{matrix} \right) \left( \begin{array}{c}
P\\
Q\\
\end{array} \right) .
\end{align*}
\end{lemma}
\begin{proof}
由矩阵乘法显然成立.
\end{proof}

\begin{example}
设$\boldsymbol{A}$为$n$阶正定实对称矩阵,$\boldsymbol{\alpha}, \boldsymbol{\beta}$为$n$维实列向量, 证明: $\boldsymbol{\alpha}'\boldsymbol{A}\boldsymbol{\alpha}+\boldsymbol{\beta}'\boldsymbol{A}^{-1}\boldsymbol{\beta}\geq 2\boldsymbol{\alpha}'\boldsymbol{\beta}$, 且等号成立的充要条件是$\boldsymbol{A}\boldsymbol{\alpha}=\boldsymbol{\beta}$.
\end{example}
\begin{proof}
{\color{blue}证法一:}
由\refpro{proposition:正定和半正定阵的判定准则}可设$\boldsymbol{A}=\boldsymbol{C}'\boldsymbol{C}$, 其中$\boldsymbol{C}$为非异实矩阵, 则$\boldsymbol{A}^{-1}=\boldsymbol{C}^{-1}(\boldsymbol{C}')^{-1}$. 再设$\boldsymbol{C}\boldsymbol{\alpha}=(a_1, a_2, \cdots, a_n)', (\boldsymbol{C}')^{-1}\boldsymbol{\beta}=(b_1, b_2, \cdots, b_n)'$为$n$维实列向量, 则
\begin{align*}
\boldsymbol{\alpha}'\boldsymbol{A}\boldsymbol{\alpha}+\boldsymbol{\beta}'\boldsymbol{A}^{-1}\boldsymbol{\beta} 
&= \boldsymbol{\alpha}'\boldsymbol{C}'\boldsymbol{C}\boldsymbol{\alpha}+\boldsymbol{\beta}'\boldsymbol{C}^{-1}(\boldsymbol{C}')^{-1}\boldsymbol{\beta}\\
&= (\boldsymbol{C}\boldsymbol{\alpha})'(\boldsymbol{C}\boldsymbol{\alpha})+((\boldsymbol{C}')^{-1}\boldsymbol{\beta})'((\boldsymbol{C}')^{-1}\boldsymbol{\beta})\\
&= \sum_{i = 1}^n (a_i^2 + b_i^2)\geq 2\sum_{i = 1}^n a_ib_i = 2(\boldsymbol{C}\boldsymbol{\alpha})'((\boldsymbol{C}')^{-1}\boldsymbol{\beta}) = 2\boldsymbol{\alpha}'\boldsymbol{\beta},
\end{align*}
等号成立的充要条件是$a_i = b_i (1\leq i\leq n)$, 即$\boldsymbol{C}\boldsymbol{\alpha}=(\boldsymbol{C}')^{-1}\boldsymbol{\beta}$, 也即$\boldsymbol{A}\boldsymbol{\alpha}=\boldsymbol{\beta}$. 

{\color{blue}证法二:}
将要证的不等式整理为
\begin{align*}
\begin{pmatrix}
\boldsymbol{\alpha}' & \boldsymbol{\beta}'
\end{pmatrix}
\begin{pmatrix}
\boldsymbol{A} & -\boldsymbol{I}_n \\
-\boldsymbol{I}_n & \boldsymbol{A}^{-1}
\end{pmatrix}
\begin{pmatrix}
\boldsymbol{\alpha} \\
\boldsymbol{\beta}
\end{pmatrix} \geq 0,
\end{align*}
这等价于证明 $\begin{pmatrix}
\boldsymbol{A} & -\boldsymbol{I}_n \\
-\boldsymbol{I}_n & \boldsymbol{A}^{-1}
\end{pmatrix}$ 是半正定阵,而这就是\refpro{proposition:正定阵构造半正定阵}的结论. 由\refpro{proposition:半正/负定阵关于线性方程的充要条件}可知,上述不等式的等号成立当且仅当 $\begin{pmatrix}
\boldsymbol{A} & -\boldsymbol{I}_n \\
-\boldsymbol{I}_n & \boldsymbol{A}^{-1}
\end{pmatrix}
\begin{pmatrix}
\boldsymbol{\alpha} \\
\boldsymbol{\beta}
\end{pmatrix}=\boldsymbol{0}$,即当且仅当 $\boldsymbol{A}\boldsymbol{\alpha}=\boldsymbol{\beta}$.

{\color{blue}证法三:}
设 \(P\) 是正交矩阵,使得 \(A = P'\mathrm{diag}\{\lambda_1,\lambda_2,\cdots,\lambda_n\}P\),其中 \(\lambda_i > 0\) 是 \(A\) 的特征值,则
\[
A^{-1} = P'\mathrm{diag}\{\lambda_1^{-1},\lambda_2^{-1},\cdots,\lambda_n^{-1}\}P
\]
设 \(\Lambda = \mathrm{diag}\{\lambda_1,\lambda_2,\cdots,\lambda_n\}\),\(P\alpha = (a_1,a_2,\cdots,a_n)'\),\(P\beta = (b_1,b_2,\cdots,b_n)'\),则
\begin{align*}
\alpha'A\alpha + \beta'A^{-1}\beta &= (P\alpha)'\Lambda(P\alpha) + (P\beta)'\Lambda^{-1}(P\beta)\\
&= (\lambda_1a_1^2 + \lambda_1^{-1}b_1^2) + (\lambda_2a_2^2 + \lambda_2^{-1}b_2^2) + \cdots + (\lambda_na_n^2 + \lambda_n^{-1}b_n^2)\\
&\geq 2a_1b_1 + 2a_2b_2 + \cdots + 2a_nb_n = 2(P\alpha)'(P\beta) = 2\alpha'\beta
\end{align*}
等号成立当且仅当 \(\lambda_ia_i = b_i\ (1\leq i \leq n)\),即 \(\Lambda(P\alpha) = (P\beta)\),也即 \((P'\Lambda P)\alpha = \beta\),从而当且仅当 \(A\alpha = \beta\) 成立。 
\end{proof}

\begin{example}
设$\boldsymbol{A}$为$n$阶正定实对称矩阵,$\boldsymbol{\alpha}, \boldsymbol{\beta}$为$n$维实列向量, 证明: $(\boldsymbol{\alpha}'\boldsymbol{\beta})^2\leq (\boldsymbol{\alpha}'\boldsymbol{A}\boldsymbol{\alpha})(\boldsymbol{\beta}'\boldsymbol{A}^{-1}\boldsymbol{\beta})$, 且等号成立的充要条件是$\boldsymbol{A}\boldsymbol{\alpha}$与$\boldsymbol{\beta}$成比例.
\end{example}
\begin{proof}
由\refpro{proposition:正定和半正定阵的判定准则}可设$\boldsymbol{A}=\boldsymbol{C}'\boldsymbol{C}$, 其中$\boldsymbol{C}$为非异实矩阵, 则$\boldsymbol{A}^{-1}=\boldsymbol{C}^{-1}(\boldsymbol{C}')^{-1}$. 再设$\boldsymbol{C}\boldsymbol{\alpha}=(a_1, a_2, \cdots, a_n)', (\boldsymbol{C}')^{-1}\boldsymbol{\beta}=(b_1, b_2, \cdots, b_n)'$为$n$维实列向量, 则由 Cauchy - Schwarz 不等式可得
\begin{align*}
(\boldsymbol{\alpha}'\boldsymbol{\beta})^2 
&= ((\boldsymbol{C}\boldsymbol{\alpha})'((\boldsymbol{C}')^{-1}\boldsymbol{\beta}))^2 = \left(\sum_{i = 1}^n a_ib_i\right)^2\leq \left(\sum_{i = 1}^n a_i^2\right)\left(\sum_{i = 1}^n b_i^2\right)\\
&= ((\boldsymbol{C}\boldsymbol{\alpha})'(\boldsymbol{C}\boldsymbol{\alpha}))((( \boldsymbol{C}')^{-1}\boldsymbol{\beta})'((\boldsymbol{C}')^{-1}\boldsymbol{\beta})) = (\boldsymbol{\alpha}'\boldsymbol{A}\boldsymbol{\alpha})(\boldsymbol{\beta}'\boldsymbol{A}^{-1}\boldsymbol{\beta}),
\end{align*}
等号成立的充要条件是$a_i$与$b_i$对应成比例, 即$\boldsymbol{C}\boldsymbol{\alpha}$与$(\boldsymbol{C}')^{-1}\boldsymbol{\beta}$成比例, 也即$\boldsymbol{A}\boldsymbol{\alpha}$与$\boldsymbol{\beta}$成比例. 
\end{proof}

\begin{proposition}\label{proposition:正定和半正定阵关于迹的判定准则}
设$\boldsymbol{A}$为$n$阶实对称矩阵, 证明:

(1) 若$\boldsymbol{A}$可逆, 则$\boldsymbol{A}$为正定阵的充要条件是对任意的$n$阶正定实对称矩阵$\boldsymbol{B}$, $\text{tr}(\boldsymbol{A}\boldsymbol{B})>0$;

(2) $\boldsymbol{A}$为半正定阵的充要条件是对任意的$n$阶半正定实对称矩阵$\boldsymbol{B}$, $\text{tr}(\boldsymbol{A}\boldsymbol{B})\geq 0$.
\end{proposition}
\begin{remark}
特别地,若$\boldsymbol{A}$为正定阵,则$\text{tr}(\boldsymbol{A})>0$;

若$\boldsymbol{A}$为半定阵,则$\text{tr}(\boldsymbol{A})\geq 0$.
\end{remark}
\begin{proof}
{\color{blue}证法一:}
(1) 先证必要性. 由\refpro{proposition:正定和半正定阵的判定准则}可设$\boldsymbol{A}=\boldsymbol{C}'\boldsymbol{C}$, 其中$\boldsymbol{C}$为非异实矩阵, 则由迹的交换性可得$\text{tr}(\boldsymbol{A}\boldsymbol{B}) = \text{tr}(\boldsymbol{C}'\boldsymbol{C}\boldsymbol{B}) = \text{tr}(\boldsymbol{C}\boldsymbol{B}\boldsymbol{C}')$. 由$\boldsymbol{B}$的正定性可知$\boldsymbol{C}\boldsymbol{B}\boldsymbol{C}'$为正定阵, 又由于\hyperref[proposition:正定阵的性质]{命题\ref{proposition:正定阵的性质}(2)},故$\text{tr}(\boldsymbol{A}\boldsymbol{B}) = \text{tr}(\boldsymbol{C}\boldsymbol{B}\boldsymbol{C}')>0$.

再证充分性. 用反证法, 若可逆实对称矩阵$\boldsymbol{A}$不正定, 则存在非异实矩阵$\boldsymbol{C}$, 使得$\boldsymbol{A}=\boldsymbol{C}'\text{diag}\{\boldsymbol{I}_p, -\boldsymbol{I}_q\}\boldsymbol{C}$, 其中负惯性指数$q > 0$. 令$\boldsymbol{B}=\boldsymbol{C}^{-1}\text{diag}\{\boldsymbol{I}_p, c\boldsymbol{I}_q\}(\boldsymbol{C}^{-1})'$, 其中正数$c > p/q$, 则$\boldsymbol{B}$是正定实对称矩阵, 且
\[
\text{tr}(\boldsymbol{A}\boldsymbol{B}) = \text{tr}\left(\boldsymbol{C}'\text{diag}\{\boldsymbol{I}_p, -c\boldsymbol{I}_q\}(\boldsymbol{C}')^{-1}\right) = \text{tr}(\text{diag}\{\boldsymbol{I}_p, -c\boldsymbol{I}_q\}) = p - cq < 0,
\]
这与假设矛盾!

(2) 由(1)同理可证. 

{\color{blue}证法二:}
(2) 先证必要性. 设 \(\boldsymbol{A}=(a_{ij}),\boldsymbol{B}=(b_{ij})\) 为半正定阵, 则由\refpro{proposition:半正定阵的Hadamard乘积性质}可知 \(\boldsymbol{A}\circ\boldsymbol{B}=(a_{ij}b_{ij})\) 也为半正定阵, 于是
\[\mathrm{tr(}\boldsymbol{AB})=\sum_{i=1}^n{\sum_{k=1}^n{a_{ik}b_{ki}}}=\sum_{i=1}^n{\sum_{k=1}^n{a_{ki}b_{ki}}}=\sum_{i,j=1}^n{a_{ij}b_{ij}}=\boldsymbol{\alpha }'(\boldsymbol{A}\circ \boldsymbol{B})\boldsymbol{\alpha }\ge 0.\]
其中 \(\boldsymbol{\alpha}=(1,1,\cdots,1)'\). 再证充分性. 令 \(\boldsymbol{x}=(x_1,x_2,\cdots,x_n)'\in\mathbb{R}^n\), 则 \(\boldsymbol{B}=\boldsymbol{x}\boldsymbol{x}'=(x_ix_j)\) 为半正定阵, 于是
\[\mathrm{tr}(\boldsymbol{A}\boldsymbol{B})=\sum_{i,j = 1}^{n}a_{ij}x_ix_j=\boldsymbol{x}'\boldsymbol{A}\boldsymbol{x}\geq0\]
由 \(\boldsymbol{x}\) 的任意性即得 \(\boldsymbol{A}\) 为半正定阵.

(1) 的必要性与 (2) 的必要性的证明完全类似, 下证充分性. 对任意的半正定阵 \(\boldsymbol{B}\) 和任意的正实数 \(t\), \(\boldsymbol{B}+t\boldsymbol{I}_n\) 为正定阵, 从而 \(\mathrm{tr}(\boldsymbol{A}(\boldsymbol{B}+t\boldsymbol{I}_n))>0\). 令 \(t\to0^+\), 可得 \(\mathrm{tr}(\boldsymbol{A}\boldsymbol{B})\geq0\), 于是由 (2) 的结论可知 \(\boldsymbol{A}\) 为半正定阵, 又 \(\boldsymbol{A}\) 可逆, 故由\refcor{corollary:可逆的半正定阵必是正定阵}可知 \(\boldsymbol{A}\) 为正定阵.

{\color{blue}证法三:}
(1) 先证必要性. 若 \(A\) 为正定阵,则由\refpro{proposition:例9.64}{(3)}可知,\(AB\) 的特征值全大于零,从而 \(\mathrm{tr}(AB)>0\). 再证充分性. 用反证法,设 \(A\) 不是正定阵,则由 \(A\) 可知 \(A\) 至少有一个特征值小于零,不妨设 \(\lambda_1 < 0\). 设 \(P\) 为正交矩阵,使得
\[
P'AP = \mathrm{diag}\{\lambda_1,\lambda_2,\cdots,\lambda_n\}
\]
令 \(B = P\mathrm{diag}\{N,1,\cdots,1\}P'\),其中 \(N\) 是充分大的正实数,则 \(B\) 为正定阵,且
\begin{align*}
\mathrm{tr}(AB) = \mathrm{tr}\left((P'AP)(P'BP)\right) = N\lambda_1 + \lambda_2 + \cdots + \lambda_n < 0
\end{align*}
这与假设矛盾. 因此 \(A\) 必为正定阵.

(2) 利用\nrefpro{proposition:例9.64}{(2)} 即可证明必要性,而充分性的证明与 (1) 完全类似. 
\end{proof}

\begin{proposition}\label{proposition:A为实半正定阵AB=O的充要条件为tr(AB)=0}
设$\boldsymbol{A}, \boldsymbol{B}$都是$n$阶半正定实对称矩阵, 证明: $\boldsymbol{A}\boldsymbol{B}=\boldsymbol{O}$的充要条件是$\text{tr}(\boldsymbol{A}\boldsymbol{B}) = 0$.
\end{proposition}
\begin{proof}
{\color{blue}证法一:}
必要性显然, 下证充分性. 由\refpro{proposition:正定和半正定阵的判定准则}可设$\boldsymbol{A}=\boldsymbol{C}'\boldsymbol{C}$, $\boldsymbol{B}=\boldsymbol{D}\boldsymbol{D}'$, 其中$\boldsymbol{C}, \boldsymbol{D}$是$n$阶实矩阵, 则由迹的交换性可得
\begin{align*}
0 = \text{tr}(\boldsymbol{A}\boldsymbol{B}) = \text{tr}(\boldsymbol{C}'\boldsymbol{C}\boldsymbol{D}\boldsymbol{D}') = \text{tr}(\boldsymbol{D}'\boldsymbol{C}'\boldsymbol{C}\boldsymbol{D}) = \text{tr}\left((\boldsymbol{C}\boldsymbol{D})'(\boldsymbol{C}\boldsymbol{D})\right),
\end{align*}
再由\hyperref[proposition:零矩阵的充要条件]{迹的正定性}可知$\boldsymbol{C}\boldsymbol{D}=\boldsymbol{O}$, 于是$\boldsymbol{A}\boldsymbol{B}=\boldsymbol{C}'(\boldsymbol{C}\boldsymbol{D})\boldsymbol{D}'=\boldsymbol{O}$. 

{\color{blue}证法二:}
必要性是显然的,下证充分性. 注意到 \(0 = \mathrm{tr}(AB) = \mathrm{tr}(B^{\frac{1}{2}}AB^{\frac{1}{2}})\),并且 \(B^{\frac{1}{2}}AB^{\frac{1}{2}}\) 为半正定阵,故 \(B^{\frac{1}{2}}AB^{\frac{1}{2}}\) 的主对角元或特征值全为零. 由\refpro{proposition:若主对角元为零,则同行同列的所有元素都为零}或实对称矩阵的正交相似标准型可知
\begin{align*}
O = B^{\frac{1}{2}}AB^{\frac{1}{2}} = (A^{\frac{1}{2}}B^{\frac{1}{2}})'(A^{\frac{1}{2}}B^{\frac{1}{2}})
\end{align*}
于是 \(A^{\frac{1}{2}}B^{\frac{1}{2}} = O\),从而 \(AB = A^{\frac{1}{2}}(A^{\frac{1}{2}}B^{\frac{1}{2}})B^{\frac{1}{2}} = O\). 
\end{proof}







\end{document}
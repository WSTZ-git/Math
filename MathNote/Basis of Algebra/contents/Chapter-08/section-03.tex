\documentclass[../../main.tex]{subfiles}
\graphicspath{{\subfix{../../image/}}} % 指定图片目录,后续可以直接使用图片文件名。

% 例如:
% \begin{figure}[H]
% \centering
% \includegraphics[scale=0.3]{image-01.01}
% \caption{图片标题}
% \label{figure:image-01.01}
% \end{figure}
% 注意:上述\label{}一定要放在\caption{}之后,否则引用图片序号会只会显示??.

\begin{document}

\section{惯性定理}

\subsection{实二次型}

\begin{theorem}[实二次型的规范标准型]\label{theorem:实二次型的规范标准型}
设实二次型
\begin{align*}
f(x_1,x_2,\cdots,x_n)=d_1x_1^2 + d_2x_2^2 + \cdots + d_nx_n^2,
\end{align*}
这个实二次型对应的系数矩阵为$\boldsymbol{A}$ 是$n$ 阶实对称阵,则$\boldsymbol{A}$一定合同于下列对角阵:
\begin{align}
\mathrm{diag}\{1,\cdots,1;-1,\cdots,-1;0,\cdots,0\},\label{eq:8.3.3}
\end{align}
其中有 $p$ 个 $1$,$q$ 个 $-1$,$n - r$ 个零.进而,$f$一定可作变量替换得到
\begin{align}
y_1^2 + \cdots + y_p^2 - y_{p + 1}^2 - \cdots - y_r^2.\label{eq:8.3.2}
\end{align}
我们将\eqref{eq:8.3.2}式中的二次型称为$f$的\textbf{规范标准型}.
\end{theorem}
\begin{proof}
由\refthe{theorem:对称矩阵的合同标准型},任意一个实对称阵 $\boldsymbol{A}$ 必合同于一个对角阵:
\begin{align*}
\boldsymbol{C}'\boldsymbol{A}\boldsymbol{C}=\mathrm{diag}\{d_1,d_2,\cdots,d_r,0,\cdots,0\},
\end{align*}
其中 $d_i\neq 0 (i = 1,\cdots,r)$. 注意到 $\boldsymbol{C}$ 是可逆阵,故 $r = \mathrm{r}(\boldsymbol{C}'\boldsymbol{A}\boldsymbol{C})=\mathrm{r}(\boldsymbol{A})$,即秩 $r$ 是矩阵合同关系下的一个不变量. 于是我们不妨设实对称阵已具有下列对角阵的形状:
\[
\boldsymbol{A}=\mathrm{diag}\{d_1,d_2,\cdots,d_r,0,\cdots,0\}.
\]

由\hyperref[lemma:初等合同变换]{初等合同变换}不难知道,任意调换 $\boldsymbol{A}$ 的主对角线上的元素得到的矩阵仍与 $\boldsymbol{A}$ 合同. 因此我们可把零放在一起,把正项与负项放在一起,即可设 $d_1>0,\cdots,d_p>0$; $d_{p + 1}<0,\cdots,d_r<0$. $\boldsymbol{A}$ 所代表的二次型为
\begin{align}
f(x_1,x_2,\cdots,x_n)=d_1x_1^2 + d_2x_2^2 + \cdots + d_rx_r^2.\label{eq:8.3.1}
\end{align}
再对上述二次型作变量替换,令
\[
\begin{cases}
y_1 = \sqrt{d_1}x_1,\cdots,y_p = \sqrt{d_p}x_p;\\
y_{p + 1} = \sqrt{-d_{p + 1}}x_{p + 1},\cdots,y_r = \sqrt{-d_r}x_r;\\
y_j = x_j (j = r + 1,\cdots,n),
\end{cases}
\]
则 \eqref{eq:8.3.1} 式变为
\begin{align*}
f = y_1^2 + \cdots + y_p^2 - y_{p + 1}^2 - \cdots - y_r^2.
\end{align*}
这一事实等价于说 $\boldsymbol{A}$ 合同于下列对角阵:
\begin{align*}
\mathrm{diag}\{1,\cdots,1;-1,\cdots,-1;0,\cdots,0\}.
\end{align*}
其中有 $p$ 个 $1$,$q$ 个 $-1$,$n - r$ 个零. 
\end{proof}

\begin{definition}
设 $f(x_1,x_2,\cdots,x_n)$ 是一个实二次型,若它能化为形如 \eqref{eq:8.3.2}式的形状,则称 $r$ 是该二次型的秩,$p$ 是它的\textbf{正惯性指数},$q = r - p$ 是它的\textbf{负惯性指数},$s = p - q$ 称为 $f$ 的\textbf{符号差}.
\end{definition}
\begin{remark}
显然,若已知秩 $r$ 与符号差 $s$,则 $p = \frac{1}{2}(r + s)$, $q = \frac{1}{2}(r - s)$. 事实上,在 $p,q,r,s$ 中只需知道其中两个数,其余两个数也就知道了. 由于实对称阵与实二次型之间的等价关系,我们将实二次型的秩、惯性指数及符号差也称为相应的实对称阵的秩、惯性指数及符号差.
\end{remark}

\begin{theorem}[惯性定理]\label{theorem:惯性定理}
证明\eqref{eq:8.3.2}式中的数$p$及$q=r-p$是两个合同不变量.
这等价于证明下面的结论.

设 $f(x_1,x_2,\cdots,x_n)$ 是一个 $n$ 元实二次型,且 $f$ 可化为两个标准型:
\begin{align*}
c_1y_1^2+\cdots + c_py_p^2 - c_{p + 1}y_{p + 1}^2 - \cdots - c_ry_r^2,\\
d_1z_1^2+\cdots + d_kz_k^2 - d_{k + 1}z_{k + 1}^2 - \cdots - d_rz_r^2,
\end{align*}
其中 $c_i>0$, $d_i>0$,则必有 $p = k$.
\end{theorem}
\begin{proof}
用反证法,设 $p>k$. 由前面的说明不妨设 $c_i$ 及 $d_i$ 均为 $1$,因此
\begin{align}
y_1^2+\cdots + y_p^2 - y_{p + 1}^2 - \cdots - y_r^2 = z_1^2+\cdots + z_k^2 - z_{k + 1}^2 - \cdots - z_r^2.\label{eq:8.3.4}
\end{align}
又设
\[
\boldsymbol{x}=\boldsymbol{B}\boldsymbol{y},\ \boldsymbol{x}=\boldsymbol{C}\boldsymbol{z},
\]
其中
\[
\boldsymbol{x}=\begin{pmatrix}
x_1\\
x_2\\
\vdots\\
x_n
\end{pmatrix},\ \boldsymbol{y}=\begin{pmatrix}
y_1\\
y_2\\
\vdots\\
y_n
\end{pmatrix},\ \boldsymbol{z}=\begin{pmatrix}
z_1\\
z_2\\
\vdots\\
z_n
\end{pmatrix},
\]
于是 $\boldsymbol{z}=\boldsymbol{C}^{-1}\boldsymbol{B}\boldsymbol{y}$. 令
\[
\boldsymbol{C}^{-1}\boldsymbol{B}=\begin{pmatrix}
c_{11} & c_{12} & \cdots & c_{1n}\\
c_{21} & c_{22} & \cdots & c_{2n}\\
\vdots & \vdots & & \vdots\\
c_{n1} & c_{n2} & \cdots & c_{nn}
\end{pmatrix},
\]
则
\[
\begin{cases}
z_1 = c_{11}y_1 + c_{12}y_2 + \cdots + c_{1n}y_n,\\
z_2 = c_{21}y_1 + c_{22}y_2 + \cdots + c_{2n}y_n,\\
\cdots\cdots\cdots\\
z_n = c_{n1}y_1 + c_{n2}y_2 + \cdots + c_{nn}y_n.
\end{cases}
\]
因为 $p>k$,故齐次线性方程组
\[
\begin{cases}
c_{11}y_1 + c_{12}y_2 + \cdots + c_{1n}y_n = 0,\\
\cdots\cdots\cdots\\
c_{k1}y_1 + c_{k2}y_2 + \cdots + c_{kn}y_n = 0,\\
y_{p + 1} = 0,\\
\cdots\cdots\cdots\\
y_n = 0
\end{cases}
\]
必有非零解 ($n$ 个未知数,$n - (p - k)$ 个方程式). 令其中一个非零解为 $y_1 = a_1$, $\cdots$, $y_p = a_p$, $y_{p + 1} = 0$, $\cdots$, $y_n = 0$,把这组解代入 \eqref{eq:8.3.4} 式左边得到
\[
a_1^2+\cdots + a_p^2>0.
\]
但这时 $z_1 = \cdots = z_k = 0$,故 \eqref{eq:8.3.4} 式右边将小于等于零,引出了矛盾. 同理可证 $p<k$ 也不可能.
\end{proof}

\begin{theorem}\label{theorem:秩与符号差(或正负惯性指数)都是实对称阵在合同关系下的全系不变量}
秩与符号差 (或正负惯性指数) 是实对称阵在合同关系下的全系不变量.
\end{theorem}
\begin{proof}
由\hyperref[theorem:惯性定理]{惯性定理}知道,秩 $r$ 与符号差 $s$ 是实对称阵合同关系的不变量. 反之,若 $n$ 阶实对称阵 $\boldsymbol{A},\boldsymbol{B}$ 的秩都为 $r$,符号差都是 $s$,则它们都合同于
\[
\mathrm{diag}\{1,\cdots,1;-1,\cdots,-1;0,\cdots,0\},
\]
其中有 $p = \frac{1}{2}(r + s)$ 个 $1$,$q = \frac{1}{2}(r - s)$ 个 $-1$ 及 $n - r$ 个零,因此 $\boldsymbol{A}$ 与 $\boldsymbol{B}$ 合同. 对正负惯性指数的结论也同样成立. 
\end{proof}


\subsection{复二次型}

\begin{theorem}[复二次型的规范标准型]\label{theorem:复二次型的规范标准型}
设复二次型
\begin{align*}
f(x_1,x_2,\cdots,x_n)=d_1x_1^2 + d_2x_2^2 + \cdots + d_rx_r^2,
\end{align*}
这个复二次型对应的系数矩阵为$\boldsymbol{A}$ 是$n$ 阶复对称阵,则$\boldsymbol{A}$一定合同于下列对角阵:
\begin{align}
\mathrm{diag}\{1,\cdots,1;0,\cdots,0\},\label{eq:::8.3.3}
\end{align}
其中有 $r$ 个 $1$,并且$r=\mathrm{r}\left( A \right)$.进而,$f$一定可作变量替换得到
\begin{align}
z_1^2 + z_2^2 + \cdots + z_r^2.\label{eq:::8.3.2}
\end{align}
我们将\eqref{eq:::8.3.2}式中的二次型称为$f$的\textbf{规范标准型}.
\end{theorem}
\begin{proof}
因为复二次型
\begin{align*}
f(x_1,x_2,\cdots,x_n)=d_1x_1^2 + d_2x_2^2 + \cdots + d_rx_r^2
\end{align*}
必可化为
\begin{align*}
z_1^2 + z_2^2 + \cdots + z_r^2,
\end{align*}
其中 $z_i = \sqrt{d_i}x_i (i = 1,2,\cdots,r)$, $z_j = x_j (j = r + 1,\cdots,n)$. 所以结论得证.故复对称阵的合同关系只有一个全系不变量,那就是秩 $r$. 
\end{proof}














\end{document}
\documentclass[../../main.tex]{subfiles}
\graphicspath{{\subfix{../../image/}}} % 指定图片目录,后续可以直接使用图片文件名。

% 例如:
% \begin{figure}[h]
% \centering
% \includegraphics{image-01.01}
% \caption{图片标题}
% \label{fig:image-01.01}
% \end{figure}
% 注意:上述\label{}一定要放在\caption{}之后,否则引用图片序号会只会显示??.

\begin{document}

\section{整除与带余除法}

\begin{definition}[整除的定义]\label{definition:整除的定义}
设\(f(x),g(x)\)是\(\mathbb{F}\)上的多项式,若存在\(\mathbb{F}\)上的多项式\(h(x)\),使得
\[
f(x)=g(x)h(x),
\]
则称\(g(x)\)是\(f(x)\)的因式,或称\(g(x)\)可整除\(f(x)\)(也称\(f(x)\)可被\(g(x)\)整除),记为\(g(x)\mid f(x)\).
\end{definition}

\begin{proposition}[整除的基本性质]\label{proposition:整除的基本性质}
设\(f(x),g(x),h(x)\in\mathbb{K}[x],0\neq c\in\mathbb{K}\),则
\begin{enumerate}[(1)]
\item 若\(f(x)\mid g(x)\),则\(cf(x)\mid g(x)\),因此非零常数多项式\(c\)是任一非零多项式的因式;

\item  \(f(x)\mid f(x)\);

\item 若\(f(x)\mid g(x),g(x)\mid h(x)\),则\(f(x)\mid h(x)\);

\item 若\(f(x)\mid g(x),f(x)\mid h(x)\),则对任意的多项式\(u(x),v(x)\),有
\[
f(x)\mid g(x)u(x)+h(x)v(x);
\]

\item 设\(f(x)\mid g(x),g(x)\mid f(x)\)且\(f(x),g(x)\)都是非零多项式,则存在\(\mathbb{K}\)中非零元\(c\),使
\[
f(x)=cg(x).
\]

\item 若$g_1(x)\mid f(x),g_2(x)\mid f(x)$,则$g_1(x)g_2(x)\mid f^2(x)$.
\end{enumerate}
\end{proposition}
\begin{proof}
\begin{enumerate}[(1)]
\item 若\(g(x)=f(x)p(x)\),则
\[
g(x)=(cf(x))(c^{-1}p(x)).
\]
此即\(cf(x)\mid g(x)\).

特别地,任取$a\in \mathbb{K}$,令$g(x)=a$,则$a\mid a$,从而$ca \mid a$,故$c$是$a$的因式.

\item 显然.

\item 若\(g(x)=f(x)p(x),h(x)=g(x)q(x)\),则
\[
h(x)=(f(x)p(x))q(x)=f(x)(p(x)q(x)).
\]

\item 若\(g(x)=f(x)p(x),h(x)=f(x)q(x)\),则
\[
g(x)u(x)+h(x)v(x)=f(x)(p(x)u(x)+q(x)v(x)).
\]

\item 设\(g(x)=f(x)p(x),f(x)=g(x)q(x)\),则
\[
f(x)=f(x)(p(x)q(x)).
\]
由此即得
\[
\mathrm{deg }f(x)=\mathrm{deg }f(x)+\mathrm{deg}(p(x)q(x)),
\]
从而
\[
\mathrm{deg}(p(x)q(x)) = 0,
\]
于是
\[
\mathrm{deg }p(x)=\mathrm{deg }q(x)=0.
\]
因此\(p(x)\)及\(q(x)\)均为非零常数多项式, 即\(f(x)\)和\(g(x)\)相差一个非零常数倍.

\item 由\(g_1(x),g_2(x)\mid f(x)\)可知, 存在多项式\(h_1(x),h_2(x)\), 使得
\begin{align*}
f(x) = g_1(x)h_1(x) = g_2(x)h_2(x).
\end{align*}
从而\(f^2(x) = g_1(x)g_2(x)h_1(x)h_2(x)\), 故\(g_1(x)g_2(x)\mid f^2(x)\). 
\end{enumerate}
\end{proof}

\begin{definition}[相伴多项式]\label{definition:}
若\(f(x)\mid g(x),g(x)\mid f(x)\)且\(f(x),g(x)\)都是非零多项式,则$f(x),g(x)$(即可以互相整除的两个多项式)称为\textbf{相伴多项式},记为\(f(x)\sim g(x)\).
\end{definition}
\begin{note}
由\hyperref[proposition:整除的基本性质]{整除的基本性质(5)}可知,相伴的多项式只相差一个非零常数倍.
\end{note}

\begin{proposition}[相伴多项式的基本性质]\label{proposition:相伴多项式的基本性质}
若$f(x)\sim g(x)$,则任意的多项式$u(x)$都有$f(x)u(x)\sim g(x)u(x)$.
\end{proposition}
\begin{proof}
由$f(x)\sim g(x)$及\hyperref[proposition:整除的基本性质]{整除的基本性质(4)}可知,任意的多项式$u(x)$都有$f(x)u(x)\mid g(x)u(x)$,$g(x)u(x)\mid f(x)u(x)$.故$f(x)u(x)\sim g(x)u(x)$.
\end{proof}

\begin{theorem}[多项式的带余除法]\label{theorem:多项式的带余除法}
设\(f(x),g(x)\in\mathbb{F}[x],g(x)\neq 0\),则必存在唯一的\(q(x),r(x)\in\mathbb{F}[x]\),使得
\[
f(x)=g(x)q(x)+r(x),
\]
且\(\text{deg }\,r(x)<\text{deg }\,g(x)\).
\end{theorem}
\begin{proof}
若\(\mathrm{deg }\,f(x)<\mathrm{deg }\,g(x)\),只需令\(q(x)=0,r(x)=f(x)\)即可.
现设\(\mathrm{deg }\,f(x)\geq\mathrm{deg }\,g(x)\),对\(f(x)\)的次数用数学归纳法. 若\(\mathrm{deg }\,f(x)=0\),则\(\mathrm{deg }\,g(x)=0\). 因此可设\(f(x)=a,g(x)=b(a\neq 0,b\neq 0)\). 这时令\(q(x)=ab^{-1},r(x)=0\)即可. 作为归纳假设,我们设结论对小于\(n\)次的多项式均成立. 设
\[
f(x)=a_nx^n + a_{n - 1}x^{n - 1}+\cdots+a_1x + a_0, a_n\neq 0,
\]
\[
g(x)=b_mx^m + b_{m - 1}x^{m - 1}+\cdots+b_1x + b_0, b_m\neq 0,
\]
由于\(n\geq m\),可令
\[
f_1(x)=f(x)-a_nb_m^{-1}x^{n - m}g(x),
\]
则\(\mathrm{deg }\,f_1(x)<n\). 由归纳假设,有
\[
f_1(x)=g(x)q_1(x)+r(x),
\]
且\(\mathrm{deg }\,r(x)<\mathrm{deg }\,g(x)\),于是
\[
f(x)-a_nb_m^{-1}x^{n - m}g(x)=g(x)q_1(x)+r(x).
\]
因此
\[
f(x)=g(x)(a_nb_m^{-1}x^{n - m}+q_1(x))+r(x).
\]
令
\[
q(x)=a_nb_m^{-1}x^{n - m}+q_1(x),
\]
即得$f(x)=g(x)q(x)+r(x)$.

再证明唯一性. 设另有\(p(x),t(x)\),使
\[
f(x)=g(x)p(x)+t(x),
\]
且\(\mathrm{deg }\,t(x)<\mathrm{deg }\,g(x)\),则
\[
g(x)(q(x)-p(x))=t(x)-r(x).
\]
注意上式左边若\(q(x)-p(x)\neq 0\),便有
\[
\mathrm{deg }\,g(x)(q(x)-p(x))\geq\mathrm{deg }\,g(x)>\mathrm{deg }\,(t(x)-r(x)),
\]
引出矛盾. 因此只可能\(p(x)=q(x),t(x)=r(x)\).
\end{proof}

\begin{corollary}\label{corollary:整除关于多项式的带余除法的充要条件}
设\(f(x),g(x)\in\mathbb{F}[x],g(x)\neq 0\),必存在唯一的\(q(x),r(x)\in\mathbb{F}[x]\),使得$f(x)=g(x)q(x)+r(x)$. 
则\(g(x)\mid f(x)\)的充要条件是\(r(x)=0\).
\end{corollary}

\begin{example}
设\(g(x)=ax + b\in\mathbb{F}[x]\)且\(a\neq0\),又\(f(x)\in\mathbb{F}[x]\),求证:\(g(x)\mid f(x)^2\)的充要条件是\(g(x)\mid f(x)\)。
\end{example}
\begin{proof}
充分性显然,只需证明必要性。

{\color{blue}证法一:}
设\(f(x)=g(x)q(x)+r\),则
\[
f(x)^2 = g(x)^2q(x)^2 + 2rg(x)q(x)+r^2.
\]
由\(g(x)\mid f(x)^2\)可得\(g(x)\mid r^2\),故\(r^2 = 0\),即\(r = 0\),从而\(g(x)\mid f(x)\)。

{\color{blue}证法二:}
由\hyperref[theorem:余数定理]{余数定理},\(f\left(-\frac{b}{a}\right)^2 = 0\),故\(f\left(-\frac{b}{a}\right)= 0\),从而\(g(x)\mid f(x)\)。
\end{proof}

\begin{example}
设 \(g(x)=ax^{2}+bx + c(abc\neq0)\), \(f(x)=x^{3}+px^{2}+qx + r\), 满足 \(g(x)\mid f(x)\), 求证:
\[
\frac{ap - b}{a}=\frac{aq - c}{b}=\frac{ar}{c}.
\]
\end{example}
\begin{proof}
用待定系数法, 设
\begin{align*}
x^{3}+px^{2}+qx + r=(ax^{2}+bx + c)(mx + n)
=amx^{3}+(an + bm)x^{2}+(bn + cm)x+cn.
\end{align*}
比较系数得
\[
am = 1,\ an + bm = p,\ bn + cm = q,\ cn = r.
\]
由此即可得到所需等式. 
\end{proof}

\subsection{凑项法}

“凑项法”是指在要证明的等式中添加若干项再减去若干项来证明结论的方法.

\begin{proposition}\label{proposition:n方差整除的充要条件}
\((x^{d}-a^{d}) \mid (x^{n}-a^{n})\) 的充要条件是 \(d\mid n\), 其中 \(a\neq0\).
\end{proposition}
\begin{proof}
$(\Leftarrow)$:由 \(d|n\) 可设 \(n = kd\),\(k\in \mathbb{N}_+\)。从而
\[
x^n - a^n=(x^d)^k-(a^d)^k=(x^d - a^d)(x^{d(k - 1)}+x^{d(k - 2)}a^d+\cdots +a^{d(k - 1)}).
\]
故 \((x^d - a^d)|(x^n - a^n)\)。

\((\Rightarrow)\):假设 \(d\nmid n\),则由带余除法可知,存在 \(q, r\in \mathbb{N}_+\) 且 \(0\leqslant r < d\),使得 \(n = qd + r\)。于是
\[
x^n - a^n=x^{dq + r}-a^{dq + r}=(x^{dq}-a^{dq})x^r+x^ra^{dq}-a^{dq + r}=(x^{dq}-a^{dq})x^r+a^{dq}(x^r - a^r).
\]
注意到 \((x^{dq}-a^{dq})|(x^d - a^d)\),但由 \(0\leqslant r < d\) 可知,\((x^d - a^d)\nmid (x^r - a^r)\)。故 \((x^d - a^d)\nmid (x^n - a^n)\) 矛盾!
\end{proof}

\begin{example}
设 \(f(x)=x^{3m}+x^{3n + 1}+x^{3p+2}\), 其中 \(m,n,p\) 为自然数, 又 \(g(x)=x^{2}+x + 1\), 求证: \(g(x)\mid f(x)\).
\end{example}
\begin{proof}
由\hyperref[proposition:n方差整除的充要条件]{命题\ref{proposition:n方差整除的充要条件}}可知,\((x^3 - 1)|(x^{3k} - 1)\),\(\forall k\in \mathbb{N}_+\)。又因为 \((x^2 + x + 1)|(x^3 - 1)\),所以 \((x^2 + x + 1)|(x^{3k} - 1)\),\(\forall k\in \mathbb{N}_+\)。注意到
\[
x^{3m}+x^{3n + 1}+x^{3p + 2}=(x^{3m}-1)+x(x^{3n}-1)+x^2(x^{3p}-1)+(x^2 + x + 1).
\]
再结合 \((x^2 + x + 1)|(x^{3m}-1)\),\((x^{3n}-1)\),\((x^{3p}-1)\) 可得\(g(x)|f(x)\)。
\end{proof}


\end{document}
\documentclass[../../main.tex]{subfiles}
\graphicspath{{\subfix{../../image/}}} % 指定图片目录,后续可以直接使用图片文件名。

% 例如:
% \begin{figure}[H]
% \centering
% \includegraphics[scale=0.4]{image-01.01}
% \caption{图片标题}
% \label{figure:image-01.01}
% \end{figure}
% 注意:上述\label{}一定要放在\caption{}之后,否则引用图片序号会只会显示??.

\begin{document}

\section{有理系数多项式}

\begin{theorem}[整数系数多项式有有理根的必要条件]\label{theorem:整数系数多项式有有理根的必要条件}
设有 \( n \) 次整系数多项式
\begin{align}
f(x) = a_n x^n + a_{n-1} x^{n-1} + \cdots + a_1 x + a_0, \label{theorem156-5.7.1}
\end{align}
则有理数 \( \frac{q}{p} \) 是 \( f(x) \) 的根的必要条件是 \( p \mid a_n, q \mid a_0 \),其中 \( p, q \) 是互素的整数。
\end{theorem}
\begin{proof}
将 \( \frac{q}{p} \) 代入\eqref{theorem156-5.7.1}式得
\[
a_n \left( \frac{q}{p} \right)^n + a_{n-1} \left( \frac{q}{p} \right)^{n-1} + \cdots + a_1 \left( \frac{q}{p} \right) + a_0 = 0,
\]
将上式两边乘以 \( p^n \) 得
\[
a_n q^n + a_{n-1} q^{n-1} p + \cdots + a_1 q p^{n-1} + a_0 p^n = 0.
\]
从而
\begin{align*}
q\left( a_nq^{n-1}+a_{n-1}q^{n-2}p+\cdots +a_1p^{n-1} \right) =-a_0p^n.
\end{align*}
于是$q\mid a_0p^n$,又因为$(q,p)=1$,所以$q\mid a_0$.同理可得$p \mid a_n$.
\end{proof}

\begin{definition}[本原多项式]
设多项式
\[
f(x) = a_n x^n + a_{n-1} x^{n-1} + \cdots + a_1 x + a_0
\]
是整系数多项式,若 \( a_n, a_{n-1}, \cdots, a_1, a_0 \) 的最大公约数等于 1,则称 \( f(x) \) 为\textbf{本原多项式}.
\end{definition}

\begin{lemma}[Gauss引理]\label{lemma:Gauss引理}
两个本原多项式之积仍是本原多项式。
\end{lemma}
\begin{proof}
设
\[
f(x) = a_n x^n + a_{n-1} x^{n-1} + \cdots + a_1 x + a_0,
\]
\[
g(x) = b_m x^m + b_{m-1} x^{m-1} + \cdots + b_1 x + b_0
\]
是两个本原多项式。若
\[
f(x) g(x) = c_{m+n} x^{m+n} + c_{m+n-1} x^{m+n-1} + \cdots + c_1 x + c_0
\]
不是本原多项式,则 \( c_0, c_1, \cdots, c_{m+n} \) 必有一个公约素因子 \( p \)。因为 \( f(x) \) 是本原多项式,故 \( p \) 不能整除 \( f(x) \) 的所有系数,可设 \( p \mid a_0, p \mid a_1, \cdots, p \mid a_{i-1} \),但 \( p \) 不能整除 \( a_i \)。同理,可设 \( p \mid b_0, p \mid b_1, \cdots, p \mid b_{j-1} \),但 \( p \) 不能整除 \( b_j \)。注意到
\[
c_{i+j} = \cdots + a_{i-2} b_{j+2} + a_{i-1} b_{j+1} + a_i b_j + a_{i+1} b_{j-1} + \cdots,
\]
\( p \) 可整除 \( c_{i+j} \),\( p \) 也能整除右式除 \( a_i b_j \) 以外的所有项。但 \( p \) 不能整除 \( a_i \) 和 \( b_j \),故 \( p \) 不能整除 \( a_i b_j \),引出矛盾。
\end{proof}

\begin{theorem}\label{theorem:整系数多项式在有理数域上可约,则一定可以分解成两个次数较低的整系数多项式之积}
若整系数多项式 $f(x)$ 在有理数域上可约,则它必可分解为两个次数较低的整系数多项式之积。
\end{theorem}
\begin{proof}
假设整系数多项式 $f(x)$ 可以分解为两个次数较低的有理系数多项式之积:
\begin{align*}
f(x) = g(x)h(x),
\end{align*}
$g(x)$ 的各项系数为有理数,必有一个公分母记为 $c$,于是 $g(x) = \frac{1}{c}(cg(x))$,其中 $cg(x)$ 为整系数多项式。若把 $cg(x)$ 中所有系数的最大公因数 $d$ 提出来,则
\begin{align*}
g(x) = \frac{d}{c}\left(\frac{c}{d}g(x)\right),
\end{align*}
$\frac{c}{d}g(x)$ 是一个本原多项式。这表明 $g(x) = ag_1(x)$,$a$ 为有理数,$g_1(x)$ 为本原多项式。同理,$h(x) = bh_1(x)$,其中 $b$ 为有理数,$h_1(x)$ 为本原多项式。于是我们得到
\begin{align*}
f(x) = g(x)h(x) = abg_1(x)h_1(x).
\end{align*}
由\hyperref[lemma:Gauss引理]{Gauss引理}知,$g_1(x)h_1(x)$ 是本原多项式。若 $ab$ 不是一个整数,则 $abg_1(x)h_1(x)$ 将不是整系数多项式,这与 $f(x)$ 是整系数多项式相矛盾。因此 $ab$ 必须是整数,于是 $f(x)$ 可以分解为两个次数较小的整系数多项式之积。     
\end{proof}

\begin{definition}[整系数多项式在整数环上可约]
我们通常称一个整系数多项式$f(x)$\textbf{在整数环上可约},若它可以分解为两个次数较低的整系数多项式之积.
\end{definition}

\begin{proposition}\label{proposition:整数环上不可约一定在有理数域上不可约}
整系数多项式$f(x)$若在整数环上不可约,则在有理数域上也不可约.
\end{proposition}
\begin{proof}
由\hyperref[theorem:整系数多项式在有理数域上可约,则一定可以分解成两个次数较低的整系数多项式之积]{定理\ref{theorem:整系数多项式在有理数域上可约,则一定可以分解成两个次数较低的整系数多项式之积}}即得.
\end{proof}

\begin{example}
$f(x)$ 是次数大于零的首一整系数多项式,若 $f(0), f(1)$ 都是奇数,求证:$f(x)$ 没有有理根. 
\end{example}
\begin{proof}
若 $c$ 是偶数,则上述左边为奇数,不可能等于零。若 $c$ 是奇数,令 $c = 2b + 1$,其中 $b$ 是整数,可得
\begin{align*}
(2b + 1)^n + a_{n-1} (2b + 1)^{n-1} + \cdots + a_1 (2b + 1) + a_0 = 0.
\end{align*}
用二项式定理展开后将看到,上式左边是一个偶数加上 $1 + a_{n-1} + \cdots + a_1 + a_0$,故必是奇数,也不可能等于零。因此 $f(x)$ 没有有理根.
\end{proof}

\begin{proposition}\label{proposition:proposition:有理系数多项式的判定条件}
设 $f(x)$ 是实系数多项式,若对任意的有理数 $c$,$f(c)$ 总是有理数,求证:$f(x)$ 是有理系数多项式。
\end{proposition}
\begin{remark}
证明与\hyperref[proposition:实系数多项式的判定条件]{命题\ref{proposition:实系数多项式的判定条件}}
\end{remark}
\begin{proof}
设 $f(x) = a_n x^n + a_{n-1} x^{n-1} + \cdots + a_1 x + a_0$,分别令 $x = 0, 1, 2, \cdots, n$,得到一个以 $a_n, a_{n-1}, \cdots, a_1, a_0$ 为未知数,由 $n + 1$ 个方程式组成的实系数线性方程组。该方程组的系数行列式是一个非零的 Vandermonde 行列式,故方程组必有唯一解,且解为有理数。因此 $f(x)$ 是有理系数多项式.
\end{proof}

\begin{example}
设 $f(x)$ 是有理系数多项式,$a, b, c$ 是有理数,但 $\sqrt{c}$ 是无理数。求证:若 $a + b\sqrt{c}$ 是 $f(x)$ 的根,则 $a - b\sqrt{c}$ 也是 $f(x)$ 的根。
\end{example}
\begin{proof}
设 $f(x) = a_n x^n + a_{n-1} x^{n-1} + \cdots + a_1 x + a_0$,则
\begin{align*}
f(a + b\sqrt{c}) = a_n (a + b\sqrt{c})^n + a_{n-1} (a + b\sqrt{c})^{n-1} + \cdots + a_1 (a + b\sqrt{c}) + a_0 = 0。
\end{align*}
将 $(a + b\sqrt{c})^k$ 用二项式定理展开,可设
\begin{align*}
f(a + b\sqrt{c}) = A + B\sqrt{c} = 0,
\end{align*}
其中 $A, B$ 都是有理数。因为 $\sqrt{c}$ 是无理数,故 $A = B = 0$。因此
\begin{align*}
f(a - b\sqrt{c}) = A - B\sqrt{c} = 0,
\end{align*}
即 $a - b\sqrt{c}$ 也是 $f(x)$ 的根。
\end{proof}

\begin{example}
设 $f(x)$ 是有理系数多项式,$a, b, c, d$ 是有理数,但 $\sqrt{c}, \sqrt{d}, \sqrt{cd}$ 都是无理数。求证:若 $a\sqrt{c} + b\sqrt{d}$ 是 $f(x)$ 的根,则下列数也是 $f(x)$ 的根:
\begin{align*}
a\sqrt{c} - b\sqrt{d}, -a\sqrt{c} + b\sqrt{d}, -a\sqrt{c} - b\sqrt{d}.
\end{align*}
\end{example}
\begin{proof}
令
\begin{align*}
g(x) = (x - (a\sqrt{c} + b\sqrt{d}))(x - (a\sqrt{c} - b\sqrt{d}))(x - (-a\sqrt{c} + b\sqrt{d}))(x - (-a\sqrt{c} - b\sqrt{d})),
\end{align*}
则经计算可得
\begin{align*}
g(x) = x^4 - 2(a^2 c + b^2 d)x^2 + (a^2 c - b^2 d)^2.
\end{align*}
注意到 $g(x)$ 是一个有理数首一多项式,只要证明它不可约,便可由\hyperref[proposition:极小多项式式的充要条件]{极小多项式式的充要条件}得到 $g(x)$ 是 $a\sqrt{c} + b\sqrt{d}$ 的极小多项式,从而由\hyperref[proposition:极小多项式的基本性质]{极小多项式的基本性质}可知$g(x) \mid f(x)$,于是结论成立。显然 $g(x)$ 没有有理系数的一次因式,只要证明它没有有理系数的二次因式即可。经过简单的计算可知,在$g(x)$的一个一次因式中任取一个一次因式相乘都不是有理系数多项式,因此 $g(x)$ 没有有理系数的二次因式.
\end{proof}

\begin{example}
求以 $\sqrt{2} + \sqrt[3]{3}$ 为根的次数最小的首一有理系数多项式。
\end{example}
\begin{remark}
\hypertarget{f(x)的6个根的找法}{\textbf{确定f(x)的6个根的方法:}}原方程 $x - \sqrt{2} = \sqrt[3]{3}$ 的解为 $x = \sqrt{2} + \sqrt[3]{3}$。但三次方程 $y^3 = 3$ 的所有根为 $y = \sqrt[3]{3}, \sqrt[3]{3\omega}, \sqrt[3]{3\omega^2}$(其中 $\omega = -\frac{1}{2} + \frac{\sqrt{3}}{2}i$ 是三次单位根),因此原方程对应三个解:

\begin{align*}
x &= \sqrt{2} + \sqrt[3]{3}, \quad \sqrt{2} + \sqrt[3]{3\omega}, \quad \sqrt{2} + \sqrt[3]{3\omega^2}.
\end{align*}
在消去 $\sqrt{2}$ 的平方步骤中,方程 $x^3 + 6x - 3 = (3x^2 + 2)\sqrt{2}$ 的两边平方后,原方程中的 $\sqrt{2}$可以被替换为 $-\sqrt{2}$,从而产生另一组解:
\begin{align*}
x &= -\sqrt{2} + \sqrt[3]{3}, \quad -\sqrt{2} + \sqrt[3]{3\omega}, \quad -\sqrt{2} + \sqrt[3]{3\omega^2}.
\end{align*}
\end{remark}
\begin{solution}
本题即求 $\sqrt{2} + \sqrt[3]{3}$ 的极小多项式。令 $x - \sqrt{2} = \sqrt[3]{3}$,两边立方得到 $(x - \sqrt{2})^3 = 3$。整理可得 $x^3 + 6x - 3 = (3x^2 + 2) \sqrt{2}$,再两边平方可得,$\sqrt{2} + \sqrt[3]{3}$ 适合下列多项式:
\begin{align*}
f(x) = x^6 - 6x^4 - 6x^3 + 12x^2 - 36x + 1.
\end{align*}
由 $f(x)$ 的构造过程,\hyperlink{f(x)的6个根的找法}{不难看出} $f(x)$ 的 6 个根分别为 $\pm \sqrt{2} + \sqrt[3]{3}$,$\pm \sqrt{2} + \sqrt[3]{3}\omega$,$\pm \sqrt{2} + \sqrt[3]{3}\omega^2$。其中 $\omega = -\frac{1}{2} + \frac{\sqrt{3}}{2}i$。因此,我们有
\begin{align*}
f\left( x \right) =\left( x-\sqrt{2}-\sqrt[3]{3} \right) \left( x+\sqrt{2}-\sqrt[3]{3} \right) \left( x-\sqrt{2}-\sqrt[3]{3}\omega \right) \left( x+\sqrt{2}-\sqrt[3]{3}\omega \right) \left( x-\sqrt{2}-\sqrt[3]{3}\omega ^2 \right) \left( x+\sqrt{2}-\sqrt[3]{3}\omega ^2 \right) .
\end{align*}
通过简单的验证可知,任取 $f(x)$ 的 2 个一次因式相乘都不是有理系数多项式;任取 $f(x)$ 的 3 个一次因式相乘也都不是有理系数多项式,因此 $f(x)$ 是有理数域上的不可约多项式,从而由\hyperref[proposition:极小多项式式的充要条件]{极小多项式式的充要条件}可知,$f(x)$是 $\sqrt{2} + \sqrt[3]{3}$ 的极小多项式。
\end{solution}

\begin{example}
求证:有理系数多项式 $x^4 + px^2 + q$ 在有理数域上可约的充要条件是或者 $p^2 - 4q = k^2$,其中 $k$ 是一个有理数;或者 $q$ 是某个有理数的平方,且 $\pm 2\sqrt{q} - p$ 也是有理数的平方。
\end{example}
\begin{proof}
必要性:若多项式 $x^4 + px^2 + q$ 在有理数域上可约,考虑下列两种情况:

(1) $x^4 + px^2 + q$ 有有理数根 $t$,这时 $t^2$ 是 $x^2 + px + q$ 的有理根,因此其判别式 $p^2 - 4q$ 必是一个有理数的完全平方。

(2) $x^4 + px^2 + q$ 无有理数根,则$x^4 + px^2 + q$ 在有理数域上可分解为两个二次多项式的积。设 $x^4 + px^2 + q = (x^2 + ax + b)(x^2 + cx + d)$,展开后比较系数可得
\begin{align*}
\begin{cases}
a + c = 0, \\
ad + bc = 0.
\end{cases}
\end{align*}

若 $a = 0$,则 $c = 0$,这时将有 $p = b + d, \ q = bd$,因此 $p^2 - 4q = (b - d)^2$。若 $a \neq 0$,则 $b = d$,比较系数后可知 $p = 2b - a^2, \ q = b^2$,因此 $\pm 2\sqrt{q} - p = a^2$。

充分性:若 $p^2 - 4q = k^2$,则

\begin{align*}
x^4 + px^2 + q &= x^4 + px^2 + \frac{1}{4}(p + k)(p - k) = \left(x^2 + \frac{1}{2}(p + k)\right)\left(x^2 + \frac{1}{2}(p - k)\right).
\end{align*}
因此多项式可约。

若 $q = b^2, \ \pm 2\sqrt{q} - p = \pm 2b - p = a^2$,则 $p = -a^2 \pm 2b$。于是
\begin{align*}
x^4 + px^2 + q &= x^4 + (-a^2 \pm 2b)x^2 + b^2 = (x^2 \pm b)^2 - a^2x^2
\end{align*}
也可约。
\end{proof}


\begin{theorem}[Eisenstein 判别法]\label{theorem:Eisenstein 判别法}
设多项式
\begin{align*}
f(x) = a_n x^n + a_{n-1} x^{n-1} + \cdots + a_1 x + a_0
\end{align*}
是整系数多项式,$a_n \neq 0, n \geq 1, p$ 是一个素数。若 $p \mid a_i (i = 0, 1, \cdots, n-1)$,但 $p \nmid a_n$ 且 $p^2 \nmid a_0$,则 $f(x)$ 在有理数域上不可约。    
\end{theorem}
\begin{proof}
只需证明 $f(x)$ 在整数环上不可约即可。设 $f(x)$ 可分解为两个次数较低的整系数多项式之积:
\begin{align*}
f(x) = (b_m x^m + b_{m-1} x^{m-1} + \cdots + b_0)(c_t x^t + c_{t-1} x^{t-1} + \cdots + c_0),
\end{align*}
其中 $m + t = n$。显然 $a_0 = b_0 c_0, a_n = b_m c_t$。由假设 $p \mid a_0$,故 $p \mid b_0$ 或 $p \mid c_0$。又 $p^2 \nmid a_0$,故 $p$ 不能同时整除 $b_0$ 及 $c_0$。不妨设 $p \mid b_0$ 但 $p \nmid c_0$。又由假设,$p$ 不能整除 $a_n = b_m c_t$,故 $p$ 既不能整除 $b_m$ 又不能整除 $c_t$。因此不妨设 $p \mid b_0, p \mid b_1, \cdots, p \mid b_{j-1}$ 但 $p$ 不能整除 $b_j$,其中 $0 < j \leq m < n$。而
\begin{align*}
a_j = b_j c_0 + b_{j-1} c_1 + \cdots + b_0 c_j,
\end{align*}
根据假设,$p \mid a_j$,又 $p$ 可整除上述右端除 $b_j c_0$ 外的其余项,而不能整除 $b_j c_0$ 这一项,引出矛盾.
\end{proof}

\begin{example}
设 $p_1, \cdots, p_m$ 是 $m$ 个互不相同的素数,求证:对任意的 $n \geq 1$,下列多项式在有理数域上不可约:
\begin{align*}
f(x) = x^n - p_1 \cdots p_m.
\end{align*} 
\end{example}
\begin{proof}
用 Eisenstein 判别法即可证明.(取$p=p_i$即可)
\end{proof}

\begin{example}
证明:$x^8 + 1$ 在有理数域上不可约。
\end{example}
\begin{proof}
作代换 $x = y + 1$,得

\begin{align*}
x^8 + 1 &= (y + 1)^8 + 1 = y^8 + 8y^7 + 28y^6 + 56y^5 + 70y^4 + 56y^3 + 28y^2 + 8y + 2.
\end{align*}

显然 2 可整除除第一项外的所有系数,但 4 不能整除常数项。用 Eisenstein 判别法可知 $(y + 1)^8 + 1$ 不可约,故 $x^8 + 1$ 也不可约。
\end{proof}

\begin{example}
设 $f(x)$ 是有理系数多项式,已知 $\sqrt{2}$ 是 $f(x)$ 的根,证明:$\sqrt{2} \varepsilon, \sqrt{2} \varepsilon^2, \cdots, \sqrt{2} \varepsilon^{n-1}$ 也是 $f(x)$ 的根,其中 $\varepsilon = \cos \frac{2\pi}{n} + i \sin \frac{2\pi}{n}$ 是 1 的 $n$ 次根。  
\end{example}
\begin{proof}
显然 $\sqrt{2}$ 适合多项式 $x^n - 2$,由 Eisenstein 判别法可知,$x^n - 2$ 在有理数域上不可约,因此它是 $\sqrt{2}$ 的极小多项式。最后由\hyperref[proposition:极小多项式的基本性质]{极小多项式的基本性质}可得 $(x^n - 2) \mid f(x)$,从而结论得证.
\end{proof}

\begin{example}
设 $f(x)$ 是次数大于 1 的奇数次有理系数不可约多项式,求证:若 $x_1, x_2$ 是 $f(x)$ 在复数域内两个不同的根,则 $x_1 + x_2$ 必不是有理数。
\end{example}
\begin{proof}
不妨设 $f(x)$ 为首一多项式,我们用反证法来证明结论。设 $x_1 + x_2 = r$ 为有理数,则有理系数多项式 $f(x)$ 与 $f(r - x)$ 有公共根 $x_1$。因为 $f(x)$ 在有理数域上不可约,故 $f(x)$ 是 $x_1$ 的极小多项式,从而由极小多项式的基本性质可得 $f(x) \mid f(r - x)$。注意到 $f(x)$ 与 $f(r - x)$ 次数相同,首项系数相同,从而有 $f(r - x) = -f(x)$。令 $x = \frac{r}{2}$,则可得 $f\left(\frac{r}{2}\right) = 0$,即 $\frac{r}{2}$ 是 $f(x)$ 的一个有理根,这与 $f(x)$ 在有理数域上不可约相矛盾。
\end{proof}

\begin{example}
设 $f(x) = (x - a_1)(x - a_2) \cdots (x - a_n) - 1$,其中 $a_1, a_2, \cdots, a_n$ 是 $n$ 个不同的整数,求证:$f(x)$ 在有理数域上不可约。
\end{example}
\begin{proof}
由\hyperref[proposition:整数环上不可约一定在有理数域上不可约]{命题\ref{proposition:整数环上不可约一定在有理数域上不可约}}可知,只要证明 $f(x)$ 在整数环上不可约即可。用反证法,设 $f(x) = g(x)h(x)$,其中 $g(x), h(x)$ 都是次数小于 $n$ 的首一整数系数多项式。注意到
\begin{align*}
g(a_i)h(a_i) = -1,
\end{align*}
因为 $g(x), h(x)$ 是整数系数多项式,故 $g(a_i) = 1, h(a_i) = -1$ 或 $g(a_i) = -1, h(a_i) = 1$。无论是哪种情况,都有
\begin{align*}
g(a_i) + h(a_i) = 0, \quad 1 \leq i \leq n,
\end{align*}
即次数小于 $n$ 的多项式 $g(x) + h(x)$ 有 $n$ 个不同的根,故 $g(x) + h(x) = 0$。因此 $f(x) = -g(x)^2$,但 $f(x)$ 是首一多项式,而 $-g(x)^2$ 的首项系数为 $-1$,矛盾.
\end{proof}

\begin{example}
设 $f(x) = (x - a_1)^2(x - a_2)^2 \cdots (x - a_n)^2 + 1$,其中 $a_1, a_2, \cdots, a_n$ 是 $n$ 个不同的整数,求证:$f(x)$ 在有理数域上不可约。
\end{example}
\begin{proof}
由\hyperref[proposition:整数环上不可约一定在有理数域上不可约]{命题\ref{proposition:整数环上不可约一定在有理数域上不可约}}可知,只要证明 $f(x)$ 在整数环上不可约即可。用反证法,设 $f(x) = u(x)v(x)$,其中 $u(x), v(x)$ 都是次数小于 $2n$ 的首一整数系数多项式。注意到 $f(x)$ 没有实根,故
$u(x), v(x)$
也都没有实根,从而由实系数多项式虚根成对可知,$u(x), v(x)$ 作为实数域上的函数都恒大于零。由于 $f(x)$ 是 $2n$ 次多项式,故 $u(x)$ 和 $v(x)$ 的次数至少有一个不超过 $n$,不妨设 $u(x)$ 的次数不超过 $n$。

若 $u(x)$ 的次数小于 $n$,则由 $f(a_i) = 1$ 可得 $u(a_i)v(a_i) = 1$,因此 $u(a_i) = 1$。考虑非零多项式 $u(x) - 1$,由上面的分析可知它有 $n$ 个不同的根 $a_1, a_2, \cdots, a_n$,这与它的次数小于 $n$ 矛盾。

因此 $u(x)$ 只能是 $n$ 次首一多项式,于是 $v(x)$ 也是 $n$ 次首一多项式。另一方面,由于 $u(a_i)v(a_i) = 1$,故 $u(a_i) = v(a_i) = \pm 1 (1 \leq i \leq n)$。注意到 $u(x) - v(x)$ 的次数小于 $n$ 并且它有 $n$ 个不同的根 $a_1, a_2, \cdots, a_n$,因此 $u(x) = v(x)$ 或 $u(x) = -v(x)$。今设 $u(x) = v(x)$,则 $f(x) = u(x)^2 + 1$,即
\begin{align*}
(u(x) + h(x))(u(x) - h(x)) = 1.
\end{align*}
因为 $u(x), h(x)$ 都是整数系数多项式,故或者 $u(x) + h(x) = 1, u(x) - h(x) = 1$;或者 $u(x) + h(x) = -1, u(x) - h(x) = -1$,于是作差可得$h(x) = 0$,矛盾。因此结论得证。
\end{proof}


\end{document}
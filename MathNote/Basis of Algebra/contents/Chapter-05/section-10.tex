\documentclass[../../main.tex]{subfiles}
\graphicspath{{\subfix{../../image/}}} % 指定图片目录,后续可以直接使用图片文件名。

% 例如:
% \begin{figure}[H]
% \centering
% \includegraphics[scale=0.4]{图.png}
% \caption{}
% \label{figure:图}
% \end{figure}
% 注意:上述\label{}一定要放在\caption{}之后,否则引用图片序号会只会显示??.

\begin{document}

\section{互素多项式的应用}

\begin{proposition}
设 $f(x), g(x)$ 是数域 $\mathbb{K}$ 上的互素多项式,$A$ 是 $\mathbb{K}$ 上的 $n$ 阶方阵,满足 $f(A) = O$,证明:$g(A)$ 是可逆矩阵。
\end{proposition}
\begin{proof}
根据假设,存在 $\mathbb{K}$ 上的多项式 $u(x), v(x)$,使得
\begin{align*}
f(x)u(x) + g(x)v(x) = 1。
\end{align*}
在上述式中代入 $x = A$,可得恒等式
\begin{align*}
f(A)u(A) + g(A)v(A) = I_n。
\end{align*}
因为 $f(A) = O$,故有 $g(A)v(A) = I_n$,从而 $g(A)$ 是非零矩阵且 $g(A)^{-1} = v(A)$.

\end{proof}

\begin{proposition}
设 $f(x), g(x)$ 是数域 $\mathbb{K}$ 上的互素多项式,$A$ 是 $\mathbb{K}$ 上的 $n$ 阶方阵,证明:$f(A)g(A) = O$ 的充要条件是 $r(f(A)) + r(g(A)) = n$。
\end{proposition}
\begin{proof}
根据假设,存在 $\mathbb{K}$ 上的多项式 $u(x), v(x)$,使得
\begin{align*}
f(x)u(x) + g(x)v(x) = 1。
\end{align*}
在上述式中代入 $x = A$,可得恒等式
\begin{align*}
f(A)u(A) + g(A)v(A) = I_n。
\end{align*}
考虑如下分块矩阵的初等变换:
\begin{align*}
\begin{pmatrix}
f(A) & O \\
O & g(A)
\end{pmatrix}
&\to
\begin{pmatrix}
f(A) & f(A)u(A) \\
O & g(A)
\end{pmatrix}
\to
\begin{pmatrix}
f(A) & I_n \\
O & g(A)
\end{pmatrix}
\to
\begin{pmatrix}
f(A) & I_n \\
-f(A)g(A) & O
\end{pmatrix}
\to
\begin{pmatrix}
O & I_n \\
-f(A)g(A) & O
\end{pmatrix},
\end{align*}
故有 $r(f(A)) + r(g(A)) = r(f(A)g(A)) + n$,从而结论得证。

\end{proof}

\begin{proposition}\label{proposition:互素多项式诱导直和分解1}
设 $f(x), g(x)$ 是数域 $\mathbb{K}$ 上的互素多项式,$\varphi$ 是 $\mathbb{K}$ 上 $n$ 维线性空间 $V$ 上的线性变换,满足 $f(\varphi)g(\varphi) = 0$,证明:$V = V_1 \oplus V_2$,其中 $V_1 = \text{Ker} f(\varphi), V_2 = \text{Ker} g(\varphi)$。
\end{proposition}
\begin{note}
这个命题告诉我们:多项式的互素因式分解可以诱导出空间的直和分解,从几何层面上看,这就是相似标准型理论原始的出发点.
\end{note}
\begin{proof}
根据假设,存在 $\mathbb{K}$ 上的多项式 $u(x), v(x)$,使得
\begin{align*}
f(x)u(x) + g(x)v(x) = 1。
\end{align*}
在上述式中代入 $x = \varphi$,可得恒等式
\begin{align*}
f(\varphi)u(\varphi) + g(\varphi)v(\varphi) = I_V。
\end{align*}
对任意的 $\alpha \in V$,由上述可得
\begin{align*}
\alpha = f(\varphi)u(\varphi)(\alpha) + g(\varphi)v(\varphi)(\alpha),
\end{align*}
注意到 
\begin{align*}
g\left( \varphi \right) \left( f\left( \varphi \right) u\left( \varphi \right) \left( \alpha \right) \right) =g\left( \varphi \right) f\left( \varphi \right) u\left( \varphi \right) \left( \alpha \right) =u\left( \varphi \right) f\left( \varphi \right) g\left( \varphi \right) \left( \alpha \right) =0,
\\
f\left( \varphi \right) \left( g\left( \varphi \right) u\left( \varphi \right) \left( \alpha \right) \right) =f\left( \varphi \right) g\left( \varphi \right) u\left( \varphi \right) \left( \alpha \right) =u\left( \varphi \right) f\left( \varphi \right) g\left( \varphi \right) \left( \alpha \right) =0.
\end{align*}
于是$f(\varphi)u(\varphi)(\alpha) \in \text{Ker} g(\varphi), g(\varphi)v(\varphi)(\alpha) \in \text{Ker} f(\varphi)$,故有 $V = V_1 + V_2$。任取 $\beta \in V_1 \cap V_2$,由上述可得
\begin{align*}
\beta = u(\varphi)f(\varphi)(\beta) + v(\varphi)g(\varphi)(\beta) = 0,
\end{align*}
故有 $V_1 \cap V_2 = 0$,因此 $V = V_1 \oplus V_2$。     

\end{proof}

\begin{example}
设 $\mathbb{Q}(\sqrt[n]{2}) = \{a_0 + a_1 \sqrt[n]{2} + a_2 \sqrt[4]{4} + \cdots + a_{n-1} \sqrt[n]{2^{n-1}} \mid a_i \in \mathbb{Q}, 0 \leqslant  i \leqslant  n - 1\}$,证明:$\mathbb{Q}(\sqrt[n]{2})$ 是一个数域,并求 $\mathbb{Q}(\sqrt[n]{2})$ 作为 $\mathbb{Q}$ 上线性空间的一组基。
\end{example}
\begin{proof}
设 $f(x) = x^n - 2$,由 Eisenstein 判别法可知 $f(x)$ 在 $\mathbb{Q}$ 上不可约,从而 $f(x)$ 是 $\sqrt[n]{2}$ 的极小多项式。我们先证明:$a_0 + a_1 \sqrt[n]{2} + \cdots + a_{n-1} \sqrt[n]{2^{n-1}} = 0$ 的充要条件是 $a_0 = a_1 = \cdots = a_{n-1} = 0$。充分性是显然的,现证必要性:令 $g(x) = a_0 + a_1 x + \cdots + a_{n-1} x^{n-1}$,则 $g(\sqrt[n]{2}) = 0$,由极小多项式的基本性质可得 $f(x) \mid g(x)$。因为 $g(x)$ 的次数小于 $n$,故只能是 $g(x) = 0$,即 $a_0 = a_1 = \cdots = a_{n-1} = 0$。

利用 $\sqrt[n]{2} = 2$ 容易验证,$\mathbb{Q}(\sqrt[n]{2})$ 中任意两个数的加法、减法和乘法都是封闭的。要证明 $\mathbb{Q}(\sqrt[n]{2})$ 是数域,只要证明除法或者取倒数封闭即可。任取 $\mathbb{Q}(\sqrt[n]{2})$ 中的非零数 $\alpha = a_0 + a_1 \sqrt[2]{2} + \cdots + a_{n-1} \sqrt[n]{2^{n-1}} \neq 0$,由上面的讨论可知 $a_0, a_1, \cdots, a_{n-1}$ 不全为零。令 $g(x) = a_0 + a_1 x + \cdots + a_{n-1} x^{n-1}$,则 $g(\sqrt[n]{2}) \neq 0$。因为 $f(x)$ 不可约且 $g(x) \neq 0$ 的次数小于 $n$,故 $f(x)$ 与 $g(x)$ 互素,由\hyperref[theorem:多项式互素的充要条件]{多项式互素的充要条件}可知,存在有理系数多项式 $u(x), v(x)$,使得
\begin{align*}
f(x)u(x) + g(x)v(x) = 1,
\end{align*}
在上述中代入 $x = \sqrt[n]{2}$,可得 $\sqrt[n]{2} v(\sqrt[n]{2}) = 1$,于是 $\alpha^{-1} = v(\sqrt[n]{2}) \in \mathbb{Q}(\sqrt[n]{2})$。因此,$\mathbb{Q}(\sqrt[n]{2})$ 是数域。

由 $\mathbb{Q}(\sqrt[n]{2})$ 的定义可知,$\mathbb{Q}(\sqrt[n]{2})$ 中任一元都是 $1, \sqrt[2]{2}, \cdots, \sqrt[n]{2^{n-1}}$ 的 $\mathbb{Q}$-线性组合;又由开始的讨论可知,$1, \sqrt[2]{2}, \cdots, \sqrt[n]{2^{n-1}}$ 是 $\mathbb{Q}$-线性无关的,因此它们构成了 $\mathbb{Q}(\sqrt[n]{2})$ 作为 $\mathbb{Q}$ 上线性空间的一组基。特别地,$\dim_{\mathbb{Q}} \mathbb{Q}(\sqrt[n]{2}) = n$。

\end{proof}

\begin{proposition}\label{proposition:不可约多项式与向量组线性无关}
设 $f(x) = x^n + a_1 x^{n-1} + \cdots + a_{n-1} x + a_n$ 是数域 $\mathbb{K}$ 上的不可约多项式,$\varphi$ 是 $\mathbb{K}$ 上 $n$ 维线性空间 $V$ 上的线性变换,$\alpha_1 \neq 0, \alpha_2, \cdots, \alpha_n$ 是 $V$ 中的向量,满足
\begin{align*}
\varphi (\alpha _1)=\alpha _2,\varphi (\alpha _2)=\alpha _3,\cdots ,\varphi (\alpha _{n-1})=\alpha _n,\varphi (\alpha _n)=-a_n\alpha _1-a_{n-1}\alpha _2-\cdots -a_1\alpha _n.
\end{align*}
证明:$\{\alpha_1, \alpha_2, \cdots, \alpha_n\}$ 是 $V$ 的一组基。
\end{proposition}
\begin{proof}
我们只要证明 $\alpha_1, \alpha_2, \cdots, \alpha_n$ 线性无关即可。用反证法,设存在不全为零的 $n$ 个数 $c_1, c_2, \cdots, c_n$,使得
\begin{align*}
c_1 \alpha_1 + c_2 \alpha_2 + \cdots + c_n \alpha_n = 0,
\end{align*}
则有
\begin{align*}
(c_1 I_V + c_2 \varphi + \cdots + c_n \varphi^{n-1})(\alpha_1) = 0。
\end{align*}
令 $g(x) = c_1 + c_2 x + \cdots + c_n x^{n-1}$,则 $g(x) \neq 0$ 且 $g(\varphi)(\alpha_1) = 0$。另一方面,由假设容易验证 $f(\varphi)(\alpha_1) = 0$。因为 $f(x)$ 不可约且 $g(x)$ 的次数小于 $n$,故 $f(x)$ 与 $g(x)$ 互素,从而存在 $\mathbb{K}$ 上的多项式 $u(x), v(x)$,使得
\begin{align*}
f(x)u(x) + g(x)v(x) = 1。
\end{align*}
在上述中代入 $x = \varphi$,可得恒等式
\begin{align*}
f(\varphi)u(\varphi) + g(\varphi)v(\varphi) = I_V。
\end{align*}
上式两边同时作用 $\alpha_1$ 可得
\begin{align*}
\alpha_1 = u(\varphi)f(\varphi)(\alpha_1) + v(\varphi)g(\varphi)(\alpha_1) = 0,
\end{align*}
\begin{example}\label{example:有理数域上线性变换与向量组线性无关的命题}
设$\alpha, \beta, \gamma \in \mathbb{Q}^3$且$\alpha \neq 0$. 若$A \in \mathbb{Q}^{3 \times 3}$满足
\begin{align*}
A\alpha =\beta ,A\beta =-\gamma ,A\gamma =\alpha -\beta \Longleftrightarrow A\alpha =\beta ,A\beta =\gamma ,A\gamma =\alpha +\beta .
\end{align*}
证明:$\alpha, \beta, \gamma$在$\mathbb{Q}$上线性无关.
\end{example}
\begin{note}
由\refpro{proposition:不可约多项式与向量组线性无关}可立得.
\end{note}
\begin{proof}
若$\alpha, \beta$在$\mathbb{Q}$上线性相关,则存在$k\in \mathbb{Q}$,使得$\beta =k\alpha$。从而由条件可得
\begin{align*}
A\alpha =\beta =k\alpha \Longrightarrow A\beta =kA\alpha =k^2\alpha =-\gamma 
\Longrightarrow A\gamma =\alpha -\beta =\left( 1-k \right) \alpha =-k^2A\alpha =-k^3\alpha .
\end{align*}
于是就有$\left( k^3-k+1 \right) \alpha =0$。又$\alpha \ne 0$,故$k^3-k+1=0$。但这个方程没有有理根,矛盾!故$\alpha, \beta$在$\mathbb{Q}$上线性无关。

若$\alpha, \beta, \gamma$在$\mathbb{Q}$上线性相关,则存在$a,b\in \mathbb{Q}$,使得$\gamma =a\alpha +b\beta$。由条件可得
\begin{align*}
A\gamma =\alpha -\beta =aA\alpha +bA\beta =a\beta -b\gamma =a\beta -b\left( a\alpha +b\beta \right) =-ab\alpha +\left( a+b^2 \right) \beta .
\end{align*}
因此
\begin{align*}
ab=1,\quad a+b^2=-1\Longrightarrow a+\frac{1}{a^2}=-1\Longrightarrow a^3-a^2+1=0,
\end{align*}
而上式无有理根,矛盾!故$\alpha, \beta, \gamma$在$\mathbb{Q}$上线性无关。

\end{proof}

\begin{example}
设$V$是$\mathbb{Q}$上4维空间,$\varphi$是$V$上的线性变换. 若
$$\alpha_i \in V, i = 1,2,3,4,5.$$
且满足
\begin{gather*}
\alpha _1\ne 0,\quad \alpha _4\ne \alpha _1+\alpha _2,\quad \varphi \alpha _1=\alpha _2,\quad \varphi \alpha _2=\alpha _3,
\\
\varphi \alpha _3=\alpha _1+\alpha _2,\quad \varphi \alpha _4=\alpha _5,\quad \varphi \alpha _5=\alpha _3+\alpha _4.
\end{gather*}
求$\det \varphi$.
\end{example}
\begin{proof}
由\refexa{example:有理数域上线性变换与向量组线性无关的命题}或\refpro{proposition:不可约多项式与向量组线性无关}可知
$\alpha_1,\alpha_2,\alpha_3$线性无关.若$\alpha_4\in \mathrm{span}\{\alpha_1,\alpha_2,\alpha_3\}$,设
$\alpha_4=a\alpha_1+b\alpha_2+c\alpha_3,\quad a,b,c\in \mathbb{Q}.$
由条件可得
\begin{align*}
\varphi \alpha_4=\alpha_5\Longrightarrow \varphi^2\alpha_4=\varphi \alpha_5=\alpha_3+\alpha_4=a\alpha_1+b\alpha_2+(c+1)\alpha_3,
\end{align*}
\begin{align*}
\varphi^2\alpha_4=\varphi^2(a\alpha_1+b\alpha_2+c\alpha_3)=\varphi\left(c\alpha_1+(a+c)\alpha_2+b\alpha_3\right)=b\alpha_1+(b+c)\alpha_2+(a+c)\alpha_3.
\end{align*}
从而
\begin{align*}
a=b\quad ,b=b+c,\quad c+1=a+c\Longrightarrow a=b=1,\quad c=0.
\end{align*}
故$\alpha_4=\alpha_1+\alpha_2$与条件矛盾!因此$\alpha_1,\alpha_2,\alpha_3,\alpha_4$线性无关.又因为$V$是$\mathbb{Q}$上$4$维空间,所以$\alpha_1,\alpha_2,\alpha_3,\alpha_4$就是$V$的一组基.从而
\begin{align*}
\varphi\left( \alpha_1,\alpha_2,\alpha_3,\alpha_4 \right)=\left( \alpha_1,\alpha_2,\alpha_3,\alpha_4 \right)\begin{pmatrix}
0&		0&		1&		x\\
1&		0&		1&		y\\
0&		1&		0&		z\\
0&		0&		0&		m\\
\end{pmatrix},
\end{align*}
其中$x,y,z,m\in \mathbb{Q}$.由条件可知
\begin{align*}
\varphi^2\alpha_4=\varphi \alpha_5=\alpha_3+\alpha_4.
\end{align*}
于是$\varphi^2\alpha_4$的在基$\alpha_1,\alpha_2,\alpha_3,\alpha_4$下的坐标就是
\begin{align*}
\begin{pmatrix}
0&		0&		1&		x\\
1&		0&		1&		y\\
0&		1&		0&		z\\
0&		0&		0&		m\\
\end{pmatrix}^2 \left( \begin{array}{c}
0\\
0\\
0\\
1\\
\end{array} \right) =\left( \begin{array}{c}
0\\
0\\
1\\
1\\
\end{array} \right) \Longleftrightarrow \left( \begin{array}{c}
z+xm\\
x+z+ym\\
y+zm\\
m^2\\
\end{array} \right) =\left( \begin{array}{c}
0\\
0\\
1\\
1\\
\end{array} \right) .
\end{align*}
解得$(x,y,z,m)^T=(-1,0,1,1)^T$或$(1,2,1,-1)^T$.故
\begin{align*}
\det \varphi=\begin{vmatrix}
0&		0&		1&		-1\\
1&		0&		1&		0\\
0&		1&		0&		1\\
0&		0&		0&		1\\
\end{vmatrix}=-1\text{或}\begin{vmatrix}
0&		0&		1&		1\\
1&		0&		1&		2\\
0&		1&		0&		1\\
0&		0&		0&		-1\\
\end{vmatrix}=1.
\end{align*}

\end{proof}这与条件$\alpha_1 \neq 0$ 矛盾,从而结论得证。

\end{proof}




\end{document}
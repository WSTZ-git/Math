\documentclass[../../main.tex]{subfiles}
\graphicspath{{\subfix{../../image/}}} % 指定图片目录,后续可以直接使用图片文件名。

% 例如:
% \begin{figure}[h]
% \centering
% \includegraphics{image-01.01}
% \caption{图片标题}
% \label{fig:image-01.01}
% \end{figure}
% 注意:上述\label{}一定要放在\caption{}之后,否则引用图片序号会只会显示??.

\begin{document}

\section{实系数多项式}

\begin{theorem}[实系数多项式的复根成对出现]\label{theorem:实系数多项式的复根成对出现}
设 $f(x) = a_n x^n + a_{n-1} x^{n-1} + \cdots + a_1 x + a_0$ 是实系数多项式,若复数 $a + bi \ (b \neq 0)$ 是其根,则 $a - bi$ 也是它的根。
\end{theorem}
\begin{proof}
令 $z = a + bi$,其共轭复数为 $\overline{z} = a - bi$,则
\begin{align*}
f(\overline{z})=a_n\overline{z}^n+a_{n-1}\overline{z}^{n-1}+\cdots +a_1\overline{z}+a_0=\overline{a_nz^n+a_{n-1}z^{n-1}+\cdots +a_1z+a_0}=0.
\end{align*}
由此即得结论。
\end{proof}

\begin{corollary}
实数域上的不可约多项式为一次多项式或下列二次多项式:
\begin{align*}
ax^2 + bx + c, \quad \text{其中} \ b^2 - 4ac < 0。
\end{align*}
\end{corollary}
\begin{proof}
一次多项式显然为不可约。当 $b^2 - 4ac < 0$ 时,$ax^2 + bx + c$ 没有实根,故不可约。

反过来,任一高于二次的实系数多项式 $f(x)$ 如有实根,则 $f(x)$ 可约;如有一复根 $a + bi \ (b \neq 0)$,则 $a - bi$ 也是它的根,从而
\begin{align*}
(x - (a + bi))(x - (a - bi)) = x^2 - 2ax + (a^2 + b^2)
\end{align*}
是 $f(x)$ 的因式,故任一高于二次的实系数多项式$f(x)$都可约.从而我们只需考虑一次和二次多项式的情况.

(i)当$f(x)$为一次实系数多项式时,显然$f(x)$一定不可约.

(ii)当$f(x)$为二次多项式时,设$f(x)=ax^2+bx+c$,则当$f(x)$有实根,即$b^2-4ac\geq 0$时,设$f(x)$的两个实根分别为$x_1,x_2$,则$f(x)=(x-x_1)(x-x_2)$,此时$f(x)$可约.当$f(x)$无实根,即$b^2-4ac<0$时,此时$f(x)$在实数域上不可约.
\end{proof}

\begin{corollary}
实数域上的多项式$f(x)$必可分解为有限个一次因式及不可约二次因式的乘积.
\end{corollary}

\begin{proposition}\label{proposition:实系数多项式的判定条件}
设 $f(x)$ 是复数域上的多项式,若对任意的实数 $c$,$f(c)$ 总是实数,求证:$f(x)$ 是实系数多项式。
\end{proposition}
\begin{proof}
设 $f(x) = a_n x^n + a_{n-1} x^{n-1} + \cdots + a_1 x + a_0$,分别令 $x = 0, 1, 2, \cdots, n$,得到一个以 $a_n, a_{n-1}, \cdots, a_1, a_0$ 为未知数,由 $n + 1$ 个方程式组成的实系数线性方程组。该方程组的系数行列式是一个非零的 Vandermonde 行列式,故方程组必有唯一解,且解为实数。因此 $f(x)$ 是实系数多项式.
\end{proof}

\begin{example}
证明:奇数次实系数多项式必有实数根.
\end{example}
\begin{proof}
实系数多项式的虚根总是成对出现的,因此奇数次实系数多项式必有实数根.
\end{proof}

\begin{proposition}\label{proposition:实系数多项式的根的符号判定准则}
设 $f(x) = a_n x^n + a_{n-1} x^{n-1} + \cdots + a_1 x + a_0$ 是实系数多项式,求证:
\begin{enumerate}[(1)]
\item 若 $a_i \ (0 \leq i \leq n)$ 全是正数或全是负数,则 $f(x)$ 没有非负实根,即只有负实根.
\item 若 $(-1)^i a_i \ (0 \leq i \leq n)$ 全是正数或全是负数,则 $f(x)$ 没有非正实根,即只有正实根.
\item 若 $a_n > 0$ 且 $(-1)^{n-i} a_i > 0 \ (0 \leq i \leq n-1)$,则 $f(x)$ 没有非正实根,即只有正实根.;若 $a_n > 0$ 且 $(-1)^{n-i} a_i \geq 0 \ (0 \leq i \leq n-1)$,则 $f(x)$ 没有负实根,,即只有正实数和零能作为根.
\end{enumerate}
\end{proposition}
\begin{proof}
(1) 若 $a_i$ 全是正数且 $f(x)$ 有非负实根 $c \geq 0$,代入后可得
\begin{align*}
f(c) = a_n c^n + a_{n-1} c^{n-1} + \cdots + a_1 c + a_0 \geq a_0 > 0,
\end{align*}
这和 $c$ 是根矛盾,因此 $f(x)$ 没有非负实根。同理可证 $a_i$ 全是负数的情形。

(2) 和 (3) 同理可证。
\end{proof}

\begin{example}
求证:实系数方程 $x^3 + px^2 + qx + r = 0$ 的根的实部全是负数的充要条件是
\begin{align*}
p > 0, \quad r > 0, \quad pq > r。
\end{align*}
\end{example}
\begin{proof}
先证必要性:设原方程的 3 个根为 $x_1, x_2, x_3$,其中 $x_1$ 是实数根,$x_1 < 0$。另假设 $x_2 = a + bi, x_3 = a - bi, a < 0$,则由Vieta定理可得
\begin{align*}
p &= -(x_1 + x_2 + x_3) = -(x_1 + 2a) > 0, \\
r &= -x_1 x_2 x_3 = -x_1(a^2 + b^2) > 0。
\end{align*}
\begin{align*}
pq - r &= -(x_1 + 2a)(x_1 x_2 + x_1 x_3 + x_2 x_3) + x_1(a^2 + b^2) \\
&= -(x_1 + 2a)(2x_1 a + a^2 + b^2) + x_1(a^2 + b^2) \\
&= -2a((x_1 + a)^2 + b^2) > 0。
\end{align*}
又假设 $x_1, x_2, x_3$ 全是负实数,则显然 $p > 0, q > 0, r > 0$,而
\begin{align*}
pq - r &= -(x_1 + x_2 + x_3)(x_1 x_2 + x_1 x_3 + x_2 x_3) + x_1 x_2 x_3 \\
&= -((x_1 + x_2 + x_3)(x_1 x_2 + x_1 x_3 + x_2 x_3) + x_1 x_2 x_3) > 0。
\end{align*}
再证充分性:由 $p > 0, r > 0, pq - r > 0$ 可知 $q > 0$,若方程的根是实数,则由\hyperref[proposition:实系数多项式的根的符号判定准则]{命题\ref{proposition:实系数多项式的根的符号判定准则}(1)}可知,此根必是负数。现假设方程有根 $x_1 < 0, x_2 = a + bi, x_3 = a - bi$,因为
\begin{align*}
pq - r = -2a((x_1 + a)^2 + b^2) > 0,
\end{align*}
故得 $a < 0$,结论得证。
\end{proof}

\begin{example}
设 $\varepsilon$ 是 1 的 $n$ 次根:
\begin{align*}
\varepsilon = \cos \frac{2\pi}{n} + i \sin \frac{2\pi}{n},
\end{align*}
求证:$\varepsilon^{mi} \ (1 \leq i \leq n)$ 是 $x^n - 1 = 0$ 的全部根的充要条件是 $(m, n) = 1$。
\end{example}
\begin{proof}
若 $(m, n) = 1$,只要证明 $\varepsilon^{mi} \ (1 \leq i \leq n)$ 互不相同即可。若不然,有 $\varepsilon^{ms} = \varepsilon^{mt} \ (1 \leq s < t \leq n)$,便有 $\varepsilon^{m(t-s)} = 1$,$n \mid m(t-s)$。因为 $m, n$ 互素,故 $n \mid (t-s)$,而 $0 < t-s < n$,矛盾。

反之,若 $(m, n) = d > 1$,则 $\varepsilon^{m \frac{n}{d}} = \varepsilon^{n \frac{m}{d}} = 1$,而$\frac{n}{d}\in \mathbb{N}_+$且$\frac{n}{d}<n$,故$\varepsilon ^{mj}=\varepsilon ^{m\left( j-\frac{n}{d} \right) +m\frac{n}{d}}=\varepsilon ^{m\left( j-\frac{n}{d} \right)},j=\frac{n}{d}+1,\cdots ,n.$于是$\varepsilon ^{mi}\left( i=1,2,\cdots ,n \right) $只生成了$\frac{n}{d}<n$个不同根.而$x^n-1$有$n$个不同根,故$\varepsilon ^{mi}\left( i=1,2,\cdots ,n \right)$无法覆盖所有$x^n-1$的所有$n$个不同根.
从而 $\varepsilon^{mi} \ (1 \leq i \leq n)$ 不可能是 $x^n - 1 = 0$ 的全部根。
\end{proof}

\begin{proposition}
设\( f(x) \) 是实系数首一多项式且无实数根,求证:\( f(x) \) 可以表示为两个实系数多项式的平方和。
\end{proposition}
\begin{proof}
因为实系数多项式的虚根成对出现,故 \( f(x) \) 是偶数次多项式,不妨设它的根为
\[
x_1, x_2, \cdots, x_n; \overline{x_1}, \overline{x_2}, \cdots, \overline{x_n}.
\]
令
\[
u(x) = (x - x_1)(x - x_2) \cdots (x - x_n); \quad v(x) = (x - \overline{x_1})(x - \overline{x_2}) \cdots (x - \overline{x_n}),
\]
则 \( v(x) = \overline{u(x)} \),\( f(x) = u(x)v(x) \)。又将 \( u(x), v(x) \) 的实部和虚部分开,可设
\[
u(x) = g(x) + ih(x), \quad v(x) = g(x) - ih(x),
\]
即有
\[
f(x) = g(x)^2 + h(x)^2.
\]
\end{proof}


\end{document}
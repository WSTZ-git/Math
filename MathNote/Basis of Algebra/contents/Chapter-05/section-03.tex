\documentclass[../../main.tex]{subfiles}
\graphicspath{{\subfix{../../image/}}} % 指定图片目录,后续可以直接使用图片文件名。

% 例如:
% \begin{figure}[H]
% \centering
% \includegraphics[scale=0.4]{图.png}
% \caption{}
% \label{figure:图}
% \end{figure}
% 注意:上述\label{}一定要放在\caption{}之后,否则引用图片序号会只会显示??.

\begin{document}

\section{最大公因式与互素多项式}

\begin{definition}[最大公因式和互素]\label{definition:最大公因式和互素}
设\(f(x),g(x)\)是\(\mathbb{F}\)上的多项式,$d(x)$是\(\mathbb{F}\)上的首1多项式,若\(d(x)\)满足

(i)$\,\,d(x)$是\(f(x),g(x)\)的公因式,

(ii)对\(f(x),g(x)\)的任一公因式\(h(x)\),都有\(h(x)\mid d(x)\),

则称\(d(x)\)为\(f(x),g(x)\)的\textbf{最大公因式}(或称\(d(x)\)为\(f(x),g(x)\)的 g.c.d.),记为\(d(x)=(f(x),g(x))\).
特别地,若\(d(x)=1\),则称\(f(x),g(x)\)互素.
\end{definition}

\begin{proposition}\label{proposition:最大公因式的小结论}
\begin{enumerate}[(1)]
\item 若$\mathbb{F}$上的多项式$d_0(x)$(但不一定是首1多项式)也满足

(i)$\,\,d_0(x)$是\(f(x),g(x)\)的公因式,

(ii)对\(f(x),g(x)\)的任一公因式\(h(x)\),都有\(h(x)\mid d_0(x)\),

则$d_0(x)\sim d(x)$,即$d_0(x)$与$d(x)$相差一个非零常数倍.

\item 对$\forall a,b\in\mathbb{F}$,$(af(x),bg(x))=(f(x),g(x))=d(x)$.
\end{enumerate}
\end{proposition}
\begin{proof}
\begin{enumerate}[(1)]
\item 证明是显然的.

\item 证明是显然的.
\end{enumerate}

\end{proof}

\begin{definition}[最小公倍式]\label{definition:最小公倍式}
设\(f(x),g(x)\)是\(\mathbb{F}\)上的多项式,$m(x)$是\(\mathbb{F}\)上的首1多项式,若\(d(x)\)满足

(i)\(\,\,m(x)\)是\(f(x)\)与\(g(x)\)的公倍式,

(ii)对\(f(x)\)与\(g(x)\)的任一公倍式\(l(x)\)均有\(m(x)\mid l(x)\),

则称\(m(x)\)为\(f(x)\)与\(g(x)\)的\textbf{最小公倍式}(或称\(m(x)\)为\(f(x),g(x)\)的 l.c.m.),记为\(m(x)=[f(x),g(x)]\).
\end{definition}

\begin{proposition}\label{proposition:最小公倍式的小结论}
\begin{enumerate}[(1)]
\item 若$\mathbb{F}$上的多项式$m_0(x)$(但不一定是首1多项式)也满足

(i)\(\,\,m(x)\)是\(f(x)\)与\(g(x)\)的公倍式,

(ii)对\(f(x)\)与\(g(x)\)的任一公倍式\(l(x)\)均有\(m(x)\mid l(x)\),

则$m_0(x)\sim m(x)$,即$m_0(x)$与$m(x)$相差一个非零常数倍.

\item 对$\forall a,b\in\mathbb{F}$,$[af(x),bg(x)]=[f(x),g(x)]=m(x)$.
\end{enumerate}
\end{proposition}
\begin{proof}
\begin{enumerate}[(1)]
\item 证明是显然的.

\item 证明是显然的.
\end{enumerate}

\end{proof}


\begin{theorem}[最大公因式的必要条件]\label{theorem:最大公因式的必要条件}
设\(f(x),g(x)\)是\(\mathbb{F}\)上的多项式,\(d(x)\)是它们的最大公因式,则必存在\(\mathbb{F}\)上的多项式\(u(x),v(x)\),使得\[f(x)u(x)+g(x)v(x)=d(x).\]
\end{theorem}
\begin{remark}
设\(d(x)=f(x)u(x)+g(x)v(x)\),则\(d(x)\)不一定是\(f(x)\)和\(g(x)\)的最大公因式.
\end{remark}
\begin{proof}
若\(f(x)=0\),则显然\((f(x),g(x)) = g(x)\);若\(g(x)=0\),则\((f(x),g(x)) = f(x)\). 故不妨设\(f(x)\neq 0,g(x)\neq 0\). 由带余除法,我们有下列等式:
\begin{align*}
f(x)&=g(x)q_1(x)+r_1(x),\\
g(x)&=r_1(x)q_2(x)+r_2(x),\\
r_1(x)&=r_2(x)q_3(x)+r_3(x),\\
&\cdots\cdots\cdots\\
r_{s - 2}(x)&=r_{s - 1}(x)q_s(x)+r_s(x),\\
&\cdots\cdots\cdots
\end{align*}
余式的次数是严格递减的,因此经过有限步后,必有一个等式其余式为零. 不妨设\(r_{s + 1}(x)=0\),于是
\begin{align}
r_{s - 1}(x)=r_s(x)q_{s + 1}(x). \label{theorem5.6-5.3.1}
\end{align}

现在要证明\(r_s(x)\)即为\(f(x)\)与\(g(x)\)的最大公因式. 由上式知\(r_s(x)\mid r_{s - 1}(x)\),但
\begin{align}
r_{s - 2}(x)=r_{s - 1}(x)q_s(x)+r_s(x),\label{theorem5.6-5.3.2}   
\end{align}
因此\(r_s(x)\mid r_{s - 2}(x)\). 这样可一直推下去,得到\(r_s(x)\mid g(x),r_s(x)\mid f(x)\). 这表明\(r_s(x)\)是\(f(x)\)与\(g(x)\)的公因式. 又设\(h(x)\)是\(f(x)\)与\(g(x)\)的公因式,则\(h(x)\mid r_1(x)\),于是\(h(x)\mid r_2(x)\),不断往下推,容易看出有\(h(x)\mid r_s(x)\). 因此\(r_s(x)\)是最大公因式.

再证明\eqref{theorem5.6-5.3.1}式. 从\eqref{theorem5.6-5.3.2}式得
\begin{align}
r_s(x)=r_{s - 2}(x)-r_{s - 1}(x)q_s(x),\label{theorem5.6-5.3.3}    
\end{align}
但我们有
\begin{align}
r_{s - 3}(x)=r_{s - 2}(x)q_{s - 1}(x)+r_{s - 1}(x),\label{theorem5.6-5.3.4})    
\end{align}
从\eqref{theorem5.6-5.3.4}式中解出\(r_{s - 1}(x)\)代入\eqref{theorem5.6-5.3.3}式,得
\[
r_s(x)=r_{s - 2}(x)(1 + q_{s - 1}(x)q_s(x))-r_{s - 3}(x)q_s(x).
\]
用类似的方法逐步将\(r_i(x)\)用\(r_{i - 1}(x),r_{i - 2}(x)\)代入,最后得到
\[
r_s(x)=f(x)u(x)+g(x)v(x).
\]
显然\(u(x),v(x)\in\mathbb{K}[x]\).

\end{proof}

\begin{theorem}[最大公因式的充分条件]\label{theorem:最大公因式的充分条件}
设\(f(x),g(x),d(x)\)是\(\mathbb{F}\)上的多项式.若\(d(x)\mid f(x),d(x)\mid g(x)\)并且存在\(\mathbb{F}\)上的多项式$u(x),v(x)$,使得\(d(x)=f(x)u(x)+g(x)v(x)\),则 \(d(x)\)必是\(f(x)\)和\(g(x)\)的最大公因式.
\end{theorem}
\begin{proof}
如果同时\(d(x)\mid f(x),d(x)\mid g(x)\),则\(d(x)\)是\(f(x)\)和\(g(x)\)的公因式. 若\(h(x)\)也是\(f(x),g(x)\)的公因式,则由\(h(x)\mid f(x)\),\(h(x)\mid g(x)\)可推出\(h(x)\mid (f(x)u(x)+g(x)v(x)) = d(x)\),因此\(d(x)\)是最大公因式.

\end{proof}

\begin{example}
设\((f(x),g(x)) = d(x)\), 求证: 对任意的正整数\(n\),
\[
(f(x)^n,f(x)^{n - 1}g(x),\cdots,g(x)^n)=d(x)^n.
\] 
\end{example}
\begin{proof}
显然\(d(x)^n\)是\(f(x)^{n - k}g(x)^k(0\leqslant  k\leqslant  n)\)的公因式. 又假设
\[
f(x)u(x)+g(x)v(x)=d(x),
\]
两边同时\(n\)次方得到
\begin{align*}
f^n\left( x \right) u^n\left( x \right) +f^{n-1}\left( x \right) g\left( x \right) u^{n-1}\left( x \right) v\left( x \right) +\cdots +g^n\left( x \right) v^n\left( x \right) =d^n\left( x \right) .    
\end{align*}
于是由\hyperref[theorem:最大公因式的充分条件]{最大公因式的充分条件}可知\(d(x)^n\)是\(f(x)^{n - k}g(x)^k(0\leqslant  k\leqslant  n)\)的最大公因式.

\end{proof}

\begin{corollary}[次数不小于1的多项式互素的充要条件]\label{corollary:次数不小于1的多项式互素的充要条件}
设\(f(x),g(x)\)是次数不小于1的多项式互素的充要条件是必唯一地存在两个多项式\(u(x),v(x)\),使得
\[
f(x)u(x)+g(x)v(x)=1,
\]
且\(\text{deg }u(x)<\text{deg }g(x),\text{deg }v(x)<\text{deg }f(x)\).
\end{corollary}
\begin{proof}
充分性由\hyperref[theorem:多项式互素的充要条件]{多项式互素的充要条件}可直接得到.下面证明必要性.

先证存在性.
因为\(( f(x),g(x) ) = 1\)且\(\mathrm{deg}\,f(x),\mathrm{deg}\,g(x) > 1\),所以由多项式互素的充要条件可知,必存在非零多项式\(h(x),k(x)\),使得
\begin{align}
f(x)h(x)+g(x)k(x)=1. \label{corollary5.2-1.1} 
\end{align}
由带余除法可知,存在\(q(x),u(x)\),使得
\[
h(x)=g(x)q(x)+u(x),\quad \mathrm{deg}\,u(x)<\mathrm{deg}\,g(x).
\]
代入\eqref{corollary5.2-1.1}式可得
\[
f(x)[g(x)q(x)+u(x)]+g(x)k(x)=1.
\]
即有
\begin{align}
f(x)u(x)+g(x)[f(x)q(x)+k(x)] = 1. \label{corollary5.2-1.2}  
\end{align}
令\(v(x)=f(x)q(x)+k(x)\),则\(\mathrm{deg}\,v(x)<\mathrm{deg}\,f(x)\)。否则,若\(\mathrm{deg}\,v(x)\geqslant \mathrm{deg}\,f(x)\),则由\eqref{corollary5.2-1.2}式可知
\begin{align}
f(x)u(x)+g(x)v(x)=1. \label{corollary5.2-1.3}  
\end{align}
从而由\(\mathrm{deg}\,v(x)\geqslant \mathrm{deg}\,f(x)\)及\(\mathrm{deg}\,u(x)<\mathrm{deg}\,g(x)\)可得
\[
\mathrm{deg}\,(f(x)u(x))=\mathrm{deg}\,f(x)+\mathrm{deg}\,u(x)<\mathrm{deg}\,v(x)+\mathrm{deg}\,g(x)=\mathrm{deg}(g(x)v(x)).
\]
而由\eqref{corollary5.2-1.3}式可知\(\mathrm{deg}\left( f\left( x \right) u\left( x \right) \right) =\mathrm{deg}\left( 1-g\left( x \right) v\left( x \right) \right) =\mathrm{deg}\left( g\left( x \right) v\left( x \right) \right) \)矛盾!

再证唯一性,设另有\(u_1(x),v_1(x)\)适合条件,即
\[
f(x)u_1(x)+g(x)v_1(x)=f(x)u(x)+g(x)v(x)=1.
\]
从而
\[
f(x)(u(x)-u_1(x))=g(x)(v(x)-v_1(x)).
\]
上式表明\(g(x)\mid f(x)(u(x)-u_1(x))\),又由于\(( f(x),g(x) ) = 1\),因此\(g(x)\mid (u(x)-u_1(x))\)。而\(\mathrm{deg}\,(u(x)-u_1(x))<\mathrm{deg}\,g(x)\),故\(u(x)-u_1(x)=0\),即\(u(x)=u_1(x)\)。同理可得\(v(x)=v_1(x)\)。

\end{proof}

\begin{theorem}[多项式互素的充要条件]\label{theorem:多项式互素的充要条件}
设\(f(x),g(x)\)是\(\mathbb{F}\)上的多项式.则
\begin{enumerate}[(1)]
\item \((f(x),g(x)) = 1\)的充要条件是存在\(\mathbb{F}\)上的多项式\(u(x),v(x)\),使得$f(x)u(x)+g(x)v(x)=1.$
\item \((f(x),g(x)) = 1\)的充要条件是对任意给定的正整数\(m,n\), \((f(x)^m,g(x)^n)=1\).
\end{enumerate}
\end{theorem}
\begin{proof}
\begin{enumerate}
\item 必要性:由\hyperref[theorem:最大公因式的必要条件]{最大公因式的必要条件}立得.

充分性:设$(f(x),g(x))=d(x)$,则由$f(x)u(x)+g(x)v(x)=1$可知,$d(x)\mid 1$,因此$d(x)=1.$

\item 必要性由\hyperref[proposition:两两互素的多项式组的乘积也互素]{命题\ref{proposition:两两互素的多项式组的乘积也互素}}即得. 反过来, 若\(d(x)\neq 1\)是\(f(x)\)和\(g(x)\)的公因式, 则它也是\(f(x)^m\)和\(g(x)^n\)的公因式, 因此\(f(x)^m\)和\(g(x)^n\)不可能互素.
\end{enumerate}

\end{proof}

\begin{proposition}[互素多项式和最大公因式的基本性质]\label{proposition:互素多项式和最大公因式的基本性质}
设$f(x),g(x),f_1(x),f_2(x)\in \mathbb{K}[x]$,则
\begin{enumerate}[(1)]
\item 若\(f_1(x)\mid g(x), f_2(x)\mid g(x)\),且\((f_1(x), f_2(x)) = 1\),则\(f_1(x)f_2(x)\mid g(x)\).
\item  若\((f(x), g(x)) = 1\),且\(f(x)\mid g(x)h(x)\),则\(f(x)\mid h(x)\).
\item  若\((f(x), g(x)) = d(x), f(x)=f_1(x)d(x), g(x)=g_1(x)d(x)\),则\((f_1(x), g_1(x)) = 1\).
\item  若\((f(x), g(x)) = d(x)\),则\((t(x)f(x), t(x)g(x)) = t(x)d(x)\).
\item\label{proposition:互素多项式和最大公因式的基本性质6}  若\((f_1(x), g(x)) = 1, (f_2(x), g(x)) = 1\),则\((f_1(x)f_2(x), g(x)) = 1\).

\item 若\((f(x),g(x))=1\),则\((f(x^m),g(x^m))=1\), 其中\(m\)为任一正整数.

\item  若\((f(x),g(x)) = 1\),则\((f(x)g(x),f(x)+g(x)) = 1\).
\end{enumerate}
\end{proposition}
\begin{proof}
\begin{enumerate}[(1)]
\item 由\hyperref[theorem:多项式互素的充要条件]{多项式互素的充要条件(1)}可知,存在\(u(x),v(x)\in\mathbb{K}[x]\),使
\[
f_1(x)u(x)+f_2(x)v(x)=1.
\]
设\(g(x)=f_1(x)s(x)=f_2(x)t(x)\),则
\begin{align*}
g(x)&=g(x)(f_1(x)u(x)+f_2(x)v(x))\\
&=f_2(x)t(x)f_1(x)u(x)+f_1(x)s(x)f_2(x)v(x)\\
&=f_1(x)f_2(x)(t(x)u(x)+s(x)v(x)),
\end{align*}
即\(f_1(x)f_2(x)\mid g(x)\).

\item 由\hyperref[theorem:theorem:多项式互素的充要条件]{多项式互素的充要条件}可知,存在\(u(x),v(x)\in\mathbb{K}[x]\),使
\[
f(x)u(x)+g(x)v(x)=1,
\]
则
\[
f(x)u(x)h(x)+g(x)v(x)h(x)=h(x).
\]
因上式左边可被\(f(x)\)整除,故\(f(x)\mid h(x)\).

\item 由\hyperref[theorem:theorem:多项式互素的充要条件]{多项式互素的充要条件}可知,存在\(u(x),v(x)\in\mathbb{K}[x]\),使
\[
f(x)u(x)+g(x)v(x)=d(x),
\]
即
\[
f_1(x)d(x)u(x)+g_1(x)d(x)v(x)=d(x),
\]
两边消去\(d(x)\)即得
\[
f_1(x)u(x)+g_1(x)v(x)=1,
\]
因此\(f_1(x),g_1(x)\)互素.

\item \(u(x),v(x)\in\mathbb{K}[x]\),使
\[
f(x)u(x)+g(x)v(x)=d(x),
\]
则
\[
t(x)f(x)u(x)+t(x)g(x)v(x)=t(x)d(x).
\]
因此,若\(h(x)\mid t(x)f(x),h(x)\mid t(x)g(x)\),则必有\(h(x)\mid t(x)d(x)\). 又\(t(x)d(x)\)是\(t(x)f(x),t(x)g(x)\)的公因式,因此\(t(x)d(x)\)是\(t(x)f(x)\)与\(t(x)g(x)\)的最大公因式.

\item 由\hyperref[theorem:theorem:多项式互素的充要条件]{多项式互素的充要条件}可知,存在\(u_1(x),v_1(x),u_2(x),v_2(x)\in\mathbb{K}[x]\),使
\begin{align*}
f_1(x)u_1(x)+g(x)v_1(x)&=1,\\
f_2(x)u_2(x)+g(x)v_2(x)&=1,
\end{align*}
将上两式两边分别相乘得
\begin{align*}
(f_1(x)f_2(x))u_1(x)u_2(x)+g(x)(v_1(x)f_2(x)u_2(x)
+g(x)v_1(x)v_2(x)+v_2(x)f_1(x)u_1(x))=1.
\end{align*}
这就是说\(f_1(x)f_2(x)\)和\(g(x)\)互素.

\item 因为\(f(x)\)和\(g(x)\)互素, 故存在多项式\(u(x),v(x)\), 使得
\[
f(x)u(x)+g(x)v(x)=1,
\]
从而有
\[
f(x^m)u(x^m)+g(x^m)v(x^m)=1,
\]
于是\(f(x^m)\)和\(g(x^m)\)互素.

\item 由\hyperref[theorem:多项式互素的充要条件]{互素多项式的充要条件(1)}可知,存在\(u(x),v(x)\),使得
\[
f(x)u(x)+g(x)v(x)=1.
\]
从而
\[
f(x)[u(x)-v(x)]+[f(x)+g(x)]v(x)=1.
\]
故由\hyperref[theorem:多项式互素的充要条件]{互素多项式的充要条件(1)}可知,\((f(x),f(x)+g(x)) = 1\)。同理可得,\((g(x),f(x)+g(x)) = 1\)。再由\hyperref[proposition:互素多项式和最大公因式的基本性质]{互素多项式和最大公因式的基本性质 (5)}即得\((f(x)g(x),f(x)+g(x)) = 1\).
\end{enumerate}

\end{proof}

\begin{theorem}[多个多项式的最大公因式的必要条件]\label{theorem:多个多项式的最大公因式的必要条件}
设\(d(x)\)是\(f_1(x),f_2(x),\cdots,f_m(x)\)的最大公因式,求证:必存在多项式\(g_1(x),g_2(x),\cdots,g_m(x)\),使得
\[
f_1(x)g_1(x)+f_2(x)g_2(x)+\cdots + f_m(x)g_m(x)=d(x).
\]
\end{theorem}
\begin{proof}
用数学归纳法. 对\(m = 2\),结论已成立. 设结论对\(m - 1\)成立. 设\(h(x)\)是\(f_1(x),f_2(x),\cdots,f_{m - 1}(x)\)的最大公因式,则有\(g_1(x),g_2(x),\cdots,g_{m - 1}(x)\),使得
\[
f_1(x)g_1(x)+f_2(x)g_2(x)+\cdots + f_{m - 1}(x)g_{m - 1}(x)=h(x).
\]
结合上式由条件可知\(d(x)\)是\(h(x)\)和\(f_m(x)\)的最大公因式,故存在\(u(x),v(x)\),使得
\[
h(x)u(x)+f_m(x)v(x)=d(x).
\]
将\(h(x)\)代入可得
\[
f_1(x)g_1(x)u(x)+f_2(x)g_2(x)u(x)+\cdots + f_{m - 1}(x)g_{m - 1}(x)u(x)+f_m(x)v(x)=d(x),
\]
即知结论成立.

\end{proof}

\begin{corollary}[多个多项式互素的充要条件]\label{corollary:多个多项式互素的充要条件}
数域\(\mathbb{F}\)上的多项式\(f_1(x),f_2(x),\cdots,f_m(x)\)互素的充要条件是存在\(\mathbb{F}\)上的多项式\(g_1(x),g_2(x),\cdots,g_m(x)\),使得
\[
f_1(x)g_1(x)+f_2(x)g_2(x)+\cdots + f_m(x)g_m(x)=1.
\]
\end{corollary}
\begin{proof}
必要性:由\hyperref[theorem:多个多项式的最大公因式的必要条件]{多个多项式的最大公因式的必要条件}立即得到.

充分性:设存在\(\mathbb{F}\)上的多项式\(g_1(x),g_2(x),\cdots,g_m(x)\),使得
\[
f_1(x)g_1(x)+f_2(x)g_2(x)+\cdots + f_m(x)g_m(x)=1.
\]
设\(d(x)\)是\(f_1(x),f_2(x),\cdots,f_m(x)\)的最大公因式,则由上式可知,$d(x)\mid 1$,从而$d(x)=1$.

\end{proof}

\begin{proposition}[两两互素的多项式组的乘积也互素]\label{proposition:两两互素的多项式组的乘积也互素}
设\(f_1(x),\cdots,f_m(x),g_1(x),\cdots,g_n(x)\)为多项式, 且
\[
(f_i(x),g_j(x)) = 1, 1\leqslant  i\leqslant  m; 1\leqslant  j\leqslant  n,
\]
求证:
\[
(f_1(x)f_2(x)\cdots f_m(x),g_1(x)g_2(x)\cdots g_n(x)) = 1.
\]
\end{proposition}
\begin{proof}
利用\hyperref[proposition:互素多项式和最大公因式的基本性质]{互素多项式和最大公因式的基本性质(5)}以及数学归纳法即得结论.

\end{proof}

\begin{corollary}\label{corollary:互素多项式n次方后仍互素}
设$f(x),g(x)$为多项式,若$(f(x),g(x))=1$,则$(f(x),g^n(x))=1$.  
\end{corollary}
\begin{proof}
在\hyperref[proposition:两两互素的多项式组的乘积也互素]{命题\ref{proposition:两两互素的多项式组的乘积也互素}(上一个命题)}中取$f_1(x)=f(x),f_i(x)=1(i=2,3\cdots,n)$,$g_j(x)=g(x)(j=1,2,\cdots,n)$即可得到结论.

\end{proof}

\begin{proposition}\label{proposition:f除以两两互素多项式得到的多项式组互素}
设$f,f_i\in \mathbb{F}[x]$,$i=1,2,\cdots,n$满足$f=f_1f_2\cdots f_n$,并有$f_i$,$i=1,2,\cdots,n$两两互素,则
\begin{align*}
\gcd\left( \frac{f}{f_1},\frac{f}{f_2},\cdots ,\frac{f}{f_n} \right) =1.
\end{align*}
\end{proposition}
\begin{proof}
假设
\begin{align*}
\gcd\left( \frac{f}{f_1},\frac{f}{f_2},\cdots ,\frac{f}{f_n} \right) =\gcd\left( \prod_{i\ne 1}{f_i},\prod_{i\ne 2}{f_i},\cdots ,\prod_{i\ne n}{f_i} \right) =g(x)\ne 1,
\end{align*}
则由\hyperref[theorem:因式分解定理]{因式分解定理}知,存在数域$\mathbb{F}$上的不可约多项式$p(x)$,使得$p(x)|g(x)$.从而
\begin{align*}
p|\prod_{i\ne 1}{f_i},\prod_{i\ne 2}{f_i},\cdots ,\prod_{i\ne n}{f_i}.
\end{align*}
由\refcor{corollary:不可约多项式“素性”的推论}可知
\begin{gather}
p|\prod_{i\ne 1}{f_i}\Longrightarrow p\left( x \right) \text{必可整除}f_2,f_3\cdots ,f_n\text{中的某一个}\Longrightarrow \text{存在}k_1\in \left\{ 1,2,\cdots ,n \right\}\backslash \{1\} ,\text{使得}p|f_{k_1},\label{eq:::--102.114} 
\\
p|\prod_{i\ne 2}{f_i}\Longrightarrow p\left( x \right) \text{必可整除}f_1,f_3\cdots ,f_n\text{中的某一个}\Longrightarrow \text{存在}k_2\in \left\{ 1,2,\cdots ,n \right\}\backslash \{2\} ,\text{使得}p|f_{k_2},
\nonumber \\
\cdots \cdots \cdots \cdots \cdots \cdots \cdots \cdots 
\nonumber \\
p|\prod_{i\ne n}{f_i}\Longrightarrow p\left( x \right) \text{必可整除}f_1,f_2\cdots ,f_{n-1}\text{中的某一个}\Longrightarrow \text{存在}k_n\in \left\{ 1,2,\cdots ,n \right\}\backslash \{n\}  ,\text{使得}p|f_{k_n},\nonumber
\end{gather}
因此$p|f_{k_1},f_{k_2},\cdots,f_{k_n}$.从而存在$j,l\in \left\{ 1,2,\cdots,n \right\}$,使得$k_j\ne k_l$.否则,若$k_1=k_2=\cdots=k_n$,则$k_1\ne 1,2,\cdots,n$.这与\eqref{eq:::--102.114}式矛盾!于是$p|f_{k_j},f_{k_l}$,即$\gcd(f_{k_j},f_{k_l})=p(x)\ne 1$,这与$f_{k_j},f_{k_l}$互素矛盾!故$\gcd\left( \frac{f}{f_1},\frac{f}{f_2},\cdots ,\frac{f}{f_n} \right) =1$.


\end{proof}

\begin{proposition}[两个多项式的乘积与其最大公因式和最小公倍式的乘积相伴]\label{proposition:两个多项式的乘积与其最大公因式和最小公倍式的乘积相伴}
设\(f(x),g(x)\)是非零多项式,则
\[
f(x)g(x)\sim(f(x),g(x))[f(x),g(x)].
\]
\end{proposition}
\begin{proof}
{\color{blue}证法一:}
设\(d(x)=(f(x),g(x))\)且\(f(x)=f_0(x)d(x),g(x)=g_0(x)d(x)\),则由\hyperref[proposition:互素多项式和最大公因式的基本性质]{互素多项式和最大公因式的基本性质(3)}可知\(f_0(x),g_0(x)\)互素.设\(l(x)\)是\(f(x),g(x)\)的公倍式且
\[
l(x)=f(x)u(x)=g(x)v(x),
\]
则\(f_0(x)d(x)u(x)=g_0(x)d(x)v(x)\),消去\(d(x)\)得
\[
f_0(x)u(x)=g_0(x)v(x).
\]
上式表明$f_0(x)\mid g_0(x)v(x),g_0(x)\mid f_0(x)u(x)$,又因为\(f_0(x),g_0(x)\)互素,所以由\hyperref[proposition:互素多项式和最大公因式的基本性质]{互素多项式和最大公因式的基本性质(2)}可知,\(f_0(x)\mid v(x),g_0(x)\mid u(x)\). 设\(u(x)=g_0(x)p(x)\),则
\[
l(x)=f_0(x)d(x)g_0(x)p(x),
\]
即\(f_0(x)d(x)g_0(x)\mid l(x)\). 显然\(f_0(x)d(x)g_0(x)\)是\(f(x),g(x)\)的公倍式,因此由\hyperref[proposition:最大公因式的小结论]{命题\ref{proposition:最大公因式的小结论}(1)}可知
\[
\frac{f(x)g(x)}{d(x)}=f_0(x)d(x)g_0(x)\sim[f(x),g(x)].
\]
故由\hyperref[proposition:相伴多项式的基本性质]{相伴多项式的基本性质}可知
\[
f(x)g(x)\sim(f(x),g(x))[f(x),g(x)].
\]

{\color{blue}证法二:}
设 \(f(x), g(x)\) 的公共标准分解为
\[
f(x) = c_1 p_1(x)^{e_1}p_2(x)^{e_2} \cdots p_k(x)^{e_k}, \quad g(x) = c_2 p_1(x)^{f_1}p_2(x)^{f_2} \cdots p_k(x)^{f_k},
\]
其中 \(p_i(x)\) 为互不相同的首一不可约多项式,\(c, d\) 是非零常数,则
\[
d(x) = p_1(x)^{r_1}p_2(x)^{r_2} \cdots p_k(x)^{r_k}, \quad h(x) = p_1(x)^{s_1}p_2(x)^{s_2} \cdots p_k(x)^{s_k},
\]
其中 \(r_i = \min\{e_i, f_i\}, s_i = \max\{e_i, f_i\}\)。注意到
\[
f(x)g(x) = c_1c_2 p_1(x)^{e_1 + f_1}p_2(x)^{e_2 + f_2} \cdots p_k(x)^{e_k + f_k},
\]
并且
\[
(f(x), g(x)) = p_1(x)^{r_1}p_2(x)^{r_2} \cdots p_k(x)^{r_k}, \quad [f(x), g(x)] = p_1(x)^{s_1}p_2(x)^{s_2} \cdots p_k(x)^{s_k},
\]
其中 \(r_i = \min\{e_i, f_i\}, s_i = \max\{e_i, f_i\}\). 令 \(c = c_1c_2\), 则有
\[
f(x)g(x) = cd(x)h(x).
\]

\end{proof}

\begin{proposition}[最大公因式与最小公倍式在开方下不变]\label{proposition:两个多项式的最大公因式与最小公倍式的n次方就是它们n次方的最大公因式与最小公倍式}
设\((f(x),g(x)) = d(x), [f(x),g(x)] = h(x)\), 求证:
\[
(f(x)^n,g(x)^n)=d(x)^n, \ [f(x)^n,g(x)^n]=h(x)^n.
\]
\end{proposition}
\begin{remark}
不妨设\(f(x),g(x)\)都是首一多项式的原因:若$f(x),g(x)$的首项系数分别为$a,b$,则用$\frac{f(x)}{a},\frac{g(x)}{b}$代替,再结合\hyperref[proposition:最大公因式的小结论]{命题\ref{proposition:最大公因式的小结论}(2)}和\hyperref[proposition:最小公倍式的小结论]{命题\ref{proposition:最小公倍式的小结论}(2)}即可得到结论.
\end{remark}
\begin{proof}
{\color{blue}证法一:}
不妨设\(f(x),g(x)\)都是首1多项式, \(f(x)=f_1(x)d(x), g(x)=g_1(x)d(x)\), 则$f_1(x),g_1(x),d(x)$都是首1多项式.由\hyperref[proposition:互素多项式和最大公因式的基本性质]{互素多项式和最大公因式的基本性质(3)}可知\((f_1(x),g_1(x)) = 1\).由\hyperref[proposition:两个多项式的乘积与其最大公因式和最小公倍式的乘积相伴]{命题\ref{proposition:两个多项式的乘积与其最大公因式和最小公倍式的乘积相伴}}可知\(h(x)\sim f_1(x)g_1(x)d(x)\),又因为$h(x),f_1(x),g_1(x),d(x)$均为首1多项式,所以\(h(x)=f_1(x)g_1(x)d(x)\).由\hyperref[proposition:两两互素的多项式组的乘积也互素]{命题\ref{proposition:两两互素的多项式组的乘积也互素}}可知,\((f_1(x)^n,g_1(x)^n)=1\),从而由\hyperref[proposition:互素多项式和最大公因式的基本性质]{互素多项式和最大公因式的基本性质(4)}可知
\[
(f(x)^n,g(x)^n)=(f_1(x)^nd(x)^n,g_1(x)^nd(x)^n)=d(x)^n.
\]
由\hyperref[proposition:两个多项式的乘积与其最大公因式和最小公倍式的乘积相伴]{命题\ref{proposition:两个多项式的乘积与其最大公因式和最小公倍式的乘积相伴}}可知\(f(x)^ng(x)^n\sim (f(x)^n,g(x)^n)[f(x)^n,g(x)^n]\),又因为$f(x),g(x)$都是首1多项式,所以\(f(x)^ng(x)^n=(f(x)^n,g(x)^n)[f(x)^n,g(x)^n]=d(x)^n[f(x)^n,g(x)^n]\).于是可得
\[
[f(x)^n,g(x)^n]=f_1(x)^ng_1(x)^nd(x)^n=h(x)^n.
\]

{\color{blue}证法二:}设 \(f(x), g(x)\) 的公共标准分解为
\[
f(x) = cp_1(x)^{e_1}p_2(x)^{e_2} \cdots p_k(x)^{e_k}, \quad g(x) = dp_1(x)^{f_1}p_2(x)^{f_2} \cdots p_k(x)^{f_k},
\]
其中 \(p_i(x)\) 为互不相同的首一不可约多项式,\(c, d\) 是非零常数,则
\[
d(x) = p_1(x)^{r_1}p_2(x)^{r_2} \cdots p_k(x)^{r_k}, \quad h(x) = p_1(x)^{s_1}p_2(x)^{s_2} \cdots p_k(x)^{s_k},
\]
其中 \(r_i = \min\{e_i, f_i\}, s_i = \max\{e_i, f_i\}\). 注意到
\[
f(x)^n = c^n p_1(x)^{ne_1}p_2(x)^{ne_2} \cdots p_k(x)^{ne_k}, \quad g(x)^n = d^n p_1(x)^{nf_1}p_2(x)^{nf_2} \cdots p_k(x)^{nf_k},
\]
并且\(\min\{ne_i, nf_i\} = nr_i, \max\{ne_i, nf_i\} = ns_i\), 因此
\[
(f(x)^n, g(x)^n) = p_1(x)^{nr_1}p_2(x)^{nr_2} \cdots p_k(x)^{nr_k} = d(x)^n,
\]
\[
[f(x)^n, g(x)^n] = p_1(x)^{ns_1}p_2(x)^{ns_2} \cdots p_k(x)^{ns_k} = h(x)^n.
\]

\end{proof}

\begin{proposition}\label{proposition:n方差的最大公因式}
设\(f(x)=x^m - 1, g(x)=x^n - 1\), 求证: \((f(x),g(x))=x^d - 1\), 其中\(d\)是\(m,n\)的最大公因子.
\end{proposition}
\begin{proof}
{\color{blue}证法一:}
不妨设\(m\geqslant  n, m = nq + r\), 先证明\((x^m - 1,x^n - 1)=(x^r - 1,x^n - 1)\). 假设\(d_1(x)=(x^m - 1,x^n - 1), d_2(x)=(x^r - 1,x^n - 1)\). 注意到
\[
x^m - 1=x^{nq + r}-1=x^r(x^{nq}-1)+(x^r - 1),
\]
\((x^n - 1)\mid(x^{nq}-1)\), 故\(d_1(x)\mid(x^r - 1)\), 从而\(d_1(x)\mid d_2(x)\). 从上式也可以看出\(d_2(x)\mid(x^m - 1)\), 从而\(d_2(x)\mid d_1(x)\), 因此\(d_1(x)=d_2(x)\). 又设\(n = q_1r + r_1\), 则\((x^m - 1,x^n - 1)=(x^n - 1,x^r - 1)=(x^r - 1,x^{r_1}-1)\). 再由辗转相除, 有某个\(r_{s - 1}=q_{s + 1}r_s\), 其中\(r_s = d\)是\(m,n\)的最大公因子, 于是\((x^m - 1,x^n - 1)=(x^{r_{s - 1}}-1,x^{r_s}-1)=x^d - 1\).

{\color{blue}证法二:}只需求出\(f(x),g(x)\)的公根. \(f(x)\)的根为
\[
\cos\frac{2k\pi}{m}+\mathrm{i}\sin\frac{2k\pi}{m}, 1\leqslant  k\leqslant  m,
\]
\(g(x)\)的根为
\[
\cos\frac{2k\pi}{n}+\mathrm{i}\sin\frac{2k\pi}{n}, 1\leqslant  k\leqslant  n,
\]
则公根为
\[
\cos\frac{2k\pi}{d}+\mathrm{i}\sin\frac{2k\pi}{d}, 1\leqslant  k\leqslant  d.
\]
这就是\(x^d - 1\)的全部根, 于是结论成立.

\end{proof}


\end{document}
\documentclass[../../main.tex]{subfiles}
\graphicspath{{\subfix{../../image/}}} % 指定图片目录,后续可以直接使用图片文件名。

% 例如:
% \begin{figure}[H]
% \centering
% \includegraphics[scale=0.4]{图.png}
% \caption{}
% \label{figure:图}
% \end{figure}
% 注意:上述\label{}一定要放在\caption{}之后,否则引用图片序号会只会显示??.

\begin{document}

\section{多项式函数与根}

\begin{definition}[多项式的重根]\label{definition:多项式的重根}
设 $f(x) \in \mathbb{K}[x]$,$b \in \mathbb{K}$,若存在正整数 $k$,使 $(x - b)^k \mid f(x)$,但 $(x - b)^{k+1}$ 不能整除 $f(x)$,则称 $b$ 是 $f(x)$ 的一个 \textbf{$\boldsymbol{k}$ 重根}。若 $k = 1$,则称 $b$ 为\textbf{单根}。
\end{definition}

\begin{theorem}[多项式没有重因式的充要条件]\label{theorem:多项式没有重因式的充要条件}
数域\(\mathbb{K}\)上的多项式\(f(x)\)没有重因式的充分必要条件是\(f(x)\)与\(f'(x)\)互素.
\end{theorem}
\begin{proof}
设多项式\(p(x)\)是\(f(x)\)的\(m(m > 1)\)重因式, 则\(f(x)=p(x)^mg(x)\), 故
\[
f'(x)=mp(x)^{m - 1}p'(x)g(x)+p(x)^mg'(x).
\]
于是\(p(x)^{m - 1}\mid f'(x)\), 这表明\(f(x)\)与\(f'(x)\)有公因式\(p(x)^{m - 1}\).
反之, 若不可约多项式\(p(x)\)是\(f(x)\)的单因式, 可设\(f(x)=p(x)g(x), p(x)\)不能整除\(g(x)\). 于是
\[
f'(x)=p'(x)g(x)+p(x)g'(x).
\]
若\(p(x)\)是\(f'(x)\)的因式, 则\(p(x)\mid p'(x)g(x)\). 但\(p(x)\)不能整除\(g(x)\)且\(p(x)\)不可约, 故\(p(x)\mid p'(x)\). 而\(p'(x)\neq 0\)且\(\mathrm{deg }p'(x)<\mathrm{deg }p(x)\), 这是不可能的. 若\(f(x)\)无重因式, 则在\(f(x)\)的标准分解式\eqref{theorem5.12-5.4.2}中, \(e_i = 1\)对一切\(i = 1,2,\cdots,m\)成立, 于是\(p_i(x)\)都不能整除\(f'(x)\). 由于\(p_i(x)\)为不可约多项式, 故\((p_i(x),f'(x)) = 1\), 由\hyperref[proposition:互素多项式和最大公因式的基本性质]{互素多项式和最大公因式的基本性质(5)}可知
\[
(p_1(x)p_2(x)\cdots p_m(x),f'(x)) = 1,
\]
即\((f(x),f'(x)) = 1\).
\end{proof}

\begin{theorem}\label{theorem:多项式除去与其导数的最大公因式就能消去重因式}
设\(d(x)=(f(x),f'(x))\), 则\(f(x)/d(x)\)是一个没有重因式的多项式, 且这个多项式的不可约因式与\(f(x)\)的不可约因式相同 (不计重数).
\end{theorem}
\begin{proof}
设\(f(x)\)有如\eqref{theorem5.12-5.4.2}式的标准分解式, 则
\begin{align*}
f'(x)&=ce_1p_1(x)^{e_1 - 1}p_2(x)^{e_2}\cdots p_s(x)^{e_s}p_1'(x)\\
&+ce_2p_1(x)^{e_1}p_2(x)^{e_2 - 1}\cdots p_s(x)^{e_s}p_2'(x)\\
&+\cdots\\
&+ce_sp_1(x)^{e_1}p_2(x)^{e_2}\cdots p_s(x)^{e_s - 1}p_s'(x).\label{proposition5.12-5.4.4}
\end{align*}
因此\(p_1(x)^{e_1 - 1}p_2(x)^{e_2 - 1}\cdots p_s(x)^{e_s - 1}\)是\(f(x)\)与\(f'(x)\)的公因式. 注意到\(f(x)\)的因式一定具有\(p_1(x)^{k_1}p_2(x)^{k_2}\cdots p_s(x)^{k_s}\)的形状. 不妨设\(h(x)\)是\(f(x),f'(x)\)的公因式. 注意到\(p_1(x)^{e_1}\)可以整除\eqref{proposition5.12-5.4.4}式中右边除第一项外的所有项, 但不能整除第一项, 因此\(p_1(x)^{e_1}\)不能整除\(f'(x)\). 同理, \(p_i(x)^{e_i}\)不能整除\(f'(x)\). 由此我们不难看出
\[
h(x)\mid p_1(x)^{e_1 - 1}p_2(x)^{e_2 - 1}\cdots p_s(x)^{e_s - 1},
\]
即\(p_1(x)^{e_1 - 1}p_2(x)^{e_2 - 1}\cdots p_s(x)^{e_s - 1}=d(x)\). 显然\(f(x)/d(x)\)没有重因式且与\(f(x)\)含有相同的不可约因式.
\end{proof}

\begin{proposition}[多项式有k重根的充要条件]\label{proposition:多项式有k重根的充要条件}
求证:$a$ 是多项式 $f(x)$ 的 $k$ 重根的充要条件是:
\begin{align*}
f(a) = f'(a) = \cdots = f^{(k-1)}(a) = 0, \quad f^{(k)}(a) \neq 0.
\end{align*}
\end{proposition}
\begin{proof}
若 $a$ 是 $f(x)$ 的 $k$ 重根, 可设 $f(x) = (x - a)^k g(x)$, $g(x)$ 不含因式 $x - a$。通过对 $f(x)$ 求导可发现, $x - a$ 可整除 $f^{(j)}(x)$ ($1 \leqslant  j \leqslant  k - 1$)。因此
\begin{align*}
f(a) = f'(a) = \cdots = f^{(k-1)}(a) = 0.
\end{align*}
而 $f^{(k)}(a) = k! g(a) \neq 0$, 故必要性得证。

反之, 若 $a$ 是 $f(x)$ 的 $m$ 重根, 若 $m > k$, 则由必要性的证明可知, 将有 $f^{(k)}(a) = 0$, 这与已知矛盾。同样, 若 $m < k$, 则由必要性的证明可知, 将有 $f^{(m)}(a) \neq 0$, 这也与已知矛盾, 于是只能 $m = k$.
\end{proof}

\begin{proposition}\label{proposition:f有n重根的判断条件}
设 $\deg f(x) = n \geqslant  1$,若 $f'(x) \mid f(x)$,证明:$f(x)$ 有 $n$ 重根.
\end{proposition}
\begin{proof}
{\color{blue}证法一:}
设 $f(x) = \frac{1}{n}(x - a)f'(x)$,现证明 $a$ 是 $f(x)$ 的 $n$ 重根。假设 $a$ 是 $f(x)$ 的 $k$ 重根,$f(x) = (x - a)^k g(x)$,$k < n$ 且 $g(x)$ 不含因式 $x - a$,则
\begin{align*}
f'(x) = k(x - a)^{k-1}g(x) + (x - a)^k g'(x) = n(x - a)^{k-1}g(x).
\end{align*}
于是 $g(x) \mid (x - a)g'(x)$,而 $g(x)$ 与 $x - a$ 互素,故将有 $g(x) \mid g'(x)$。引出矛盾。

{\color{blue}证法二 :} 设 $f(x) = \frac{1}{n}(x - a)f'(x)$,则
\begin{align*}
\frac{f(x)}{(f(x), f'(x))} = b(x - a), \quad b \neq 0.
\end{align*}
由\hyperref[theorem:多项式除去与其导数的最大公因式就能消去重因式]{定理\ref{theorem:多项式除去与其导数的最大公因式就能消去重因式}}可知,$x - a$ 是 $f(x)$ 唯一的不可约因式,因此 $f(x) = b(x - a)^n$。
\end{proof}

\begin{proposition}
数域$\mathbb{F}$上任意一个不可约多项式在复数域$\mathbb{C}$中无重根.
\end{proposition}
\begin{proof}
设$f(x)$是$\mathbb{F}$上的不可约多项式,则deg$\,f(x)<$deg$\,f'(x)$.从而$f(x)\nmid f'(x)$,于是$(f(x),f'(x))=1$.故由\hyperref[theorem:多项式没有重因式的充要条件]{多项式没有重因式的充要条件}可知$f(x)$复数域$\mathbb{C}$中无重根.
\end{proof}

\begin{lemma}[次数不为1得到不可约多项式没有根]\label{lemma:次数不为1得到不可约多项式没有根}
设 $f(x)$ 是数域 $\mathbb{K}$ 上的不可约多项式且 $\deg f(x) \geqslant  2$,则 $f(x)$ 在 $\mathbb{K}$ 中没有根。
\end{lemma}
\begin{proof}
用反证法,设 $b \in \mathbb{K}$ 是 $f(x)$ 的根,由\hyperref[theorem:余数定理]{余数定理} 知 $(x - b) \mid f(x)$,即 $f(x) = (x - b)g(x)$ 可分解为两个低次多项式之积,这与 $f(x)$ 不可约矛盾。
\end{proof}

\begin{theorem}[多项式根的有限性]\label{theorem:多项式根的有限性}
设\(f(x)\)是数域\(\mathbb{F}\)上的\(n\)次多项式,则\(f(x)\)在\(\mathbb{F}\)中最多只有\(n\)个根.
\end{theorem}
\begin{note}
由\hyperref[proposition:多项式根的有限性]{命题\ref{proposition:多项式根的有限性}}可知,若一个$n$次多项式的根超过$n$个,则这个多项式一定恒为零.
\end{note}
\begin{proof}
将 $f(x)$ 作标准因式分解,则由\hyperref[lemma:次数不为1得到不可约多项式没有根]{次数不为1得到不可约多项式没有根}知 $f(x)$ 在 $\mathbb{K}$ 中根的个数等于该分解式中一次因式的个数,它不会超过 $n$。
\end{proof}

\begin{corollary}[两个多项式相等的判定准则]\label{corollary:两个多项式相等的判定准则}
设 $f(x)$ 与 $g(x)$ 是 $\mathbb{K}$ 上的次数不超过 $n$ 的两个多项式,若存在 $\mathbb{K}$ 上 $n + 1$ 个不同的数 $b_1, b_2, \ldots, b_{n+1}$,使
\begin{align*}
f(b_i) = g(b_i), \quad i = 1, 2, \ldots, n + 1,
\end{align*}
则 $f(x) = g(x)$.
\end{corollary}
\begin{proof}
作 $h(x) = f(x) - g(x)$,显然 $h(x)$ 次数不超过 $n$。但它有 $n + 1$ 个不同的根,因此只可能 $h(x) = 0$,即 $f(x) = g(x)$。
\end{proof}

\begin{example}
求证:$f(x) = \sin x$ 在实数域内不能表示为 $x$ 的多项式。
\end{example}
\begin{proof}
注意到 $f(x) = \sin x$ 在实数域内有无穷多个根,而任一非零多项式只能有有限个根,因此 $f(x) = \sin x$ 在实数域内不能表示为 $x$ 的多项式。
\end{proof}

\begin{example}
设 $f(x)$ 是数域 $\mathbb{F}$ 上的多项式,若对 $\mathbb{F}$ 中某个非零常数 $a$,有 $f(x + a) = f(x)$,求证:$f(x)$ 必是常数多项式。
\end{example}
\begin{proof}
假设 $f(x)$ 不是常数多项式,则 $f(x) - f(a)$ 也不是常数多项式,但由 $f(x + a) = f(x)$ 可知,$ka \ (k \in \mathbb{Z})$ 是 $f(x) - f(a)$ 的无穷多个根,矛盾。
\end{proof}

\begin{example}
设 $f(x)$ 是非常数多项式且 $f(x)$ 可以整除 $f(x^m) \ (m \in \mathbb{N}_+)$,求证:$f(x)$ 的根只能是 $0$ 或 $1$ 的某个方根。
\end{example}
\begin{proof}
将 $f(x)$ 看成复数域上的多项式,则 $f(x^m) = f(x)g(x)$。假设 $c$ 是 $f(x)$ 的一个复根,即 $f(c) = 0$,则 $f(c^m) = 0$,即 $c^m$ 也是 $f(x)$ 的根。由此可知 $c^m, c^{m^2}, c^{m^3}, \ldots$ 也都是 $f(x)$ 的根。由于 $f(x)$ 只有有限个不同的复根,故存在正整数 $k>t$,使得 $c^{m^k} = c^{m^t}$。因此若 $c \neq 0$,取$n=m^k-m^t\in \mathbb{N}_+$,则有 $c^n = 1$。
\end{proof}


\begin{theorem}[余数定理]\label{theorem:余数定理}
设\(f(x)\in\mathbb{F}[x], b\in\mathbb{F}\),则存在\(\mathbb{F}\)上的多项式\(g(x)\),使得
\[
f(x)=(x - b)g(x)+f(b).
\]
特别地,\(b\)是\(f(x)\)的根的充要条件是\((x - b)\mid f(x)\).
\end{theorem}
\begin{note}
利用余数定理可以实现求根与判断整除性之间的相互转换.
\end{note}
\begin{proof}
由带余除法知
\begin{align*}
f(x) = (x - b)g(x) + r(x),
\end{align*}
其中 $\deg r(x) < 1$,因此 $r(x)$ 为常数多项式。在 上式中用 $b$ 代替 $x$,即得 $r(x) = f(b)$。
\end{proof}

\begin{example}
设 $n$ 是奇数,求证:$(x + y)(y + z)(x + z)$ 可整除 $(x + y + z)^n - x^n - y^n - z^n$。
\end{example}
\begin{proof}
将多项式 $(x + y + z)^n - x^n - y^n - z^n$ 看成是未定元 $x$ 的多项式。当 $x = -y$ 时,$(x + y + z)^n - x^n - y^n - z^n = 0$,因此由\hyperref[theorem:余数定理]{余数定理}可知$x + y$ 是 $(x + y + z)^n - x^n - y^n - z^n$ 的因式。同理 $x + z, y + z$ 也是因式。又这 3 个因式互素,故$(x + y)(y + z)(x + z)$ 可整除 $(x + y + z)^n - x^n - y^n - z^n$。
\end{proof}

\begin{example}
设 $f(x)$ 是一个 $n$ 次多项式,若当 $k = 0, 1, \ldots, n$ 时有 $f(k) = \frac{k}{k + 1}$,求 $f(n + 1)$。
\end{example}
\begin{proof}
解 今 $g(x) = (x + 1)f(x) - x$,则 $0, 1, \ldots, n$ 是 $g(x)$ 的根,因此
\begin{align*}
g(x) = c x(x - 1)(x - 2) \cdots (x - n),
\end{align*}
即
\begin{align*}
(x + 1)f(x) - x = c x(x - 1)(x - 2) \cdots (x - n),
\end{align*}
其中 $c$ 是一个常数。令 $x = -1$,可求出 $c = \frac{(-1)^{n+1}}{(n + 1)!}$。从而
\begin{align*}
f(x) = \frac{1}{x + 1} \left( \frac{(-1)^{n+1} x(x - 1) \cdots (x - n)}{(n + 1)!} + x \right),
\end{align*}
故
\begin{align*}
f(n + 1) = \frac{1}{n + 2} \left( \frac{(-1)^{n+1} (n + 1)!}{(n + 1)!} + n + 1 \right).
\end{align*}
当 $n$ 是奇数时,$f(n + 1) = 1$;当 $n$ 是偶数时,$f(n + 1) = \frac{n}{n + 2}$。
\end{proof}

\begin{example}
设 $(x^4 + x^3 + x^2 + x + 1) \mid (x^3 f_1(x^5) + x^2 f_2(x^5) + x f_3(x^5) + f_4(x^5))$,这里 $f_i(x) \ (1 \leqslant  i \leqslant  4)$ 都是实系数多项式,求证:$f_i(1) = 0 \ (1 \leqslant  i \leqslant  4)$。
\end{example}
\begin{proof}
设 $\varepsilon_i \ (1 \leqslant  i \leqslant  4)$ 是 1 的五次虚根,则 $\varepsilon_i \ (1 \leqslant  i \leqslant  4)$ 都适合 $x^5 - 1$,从而由余数定理可知
\begin{align*}
x^5 - 1 = (x - 1)(x - \varepsilon_1)(x - \varepsilon_2)(x - \varepsilon_3)(x - \varepsilon_4) = (x - 1)(x^4 + x^3 + x^2 + x + 1)。
\end{align*}
故
\begin{align*}
x^4 + x^3 + x^2 + x + 1 = (x - \varepsilon_1)(x - \varepsilon_2)(x - \varepsilon_3)(x - \varepsilon_4)。
\end{align*}
因此 $\varepsilon_i \ (1 \leqslant  i \leqslant  4)$ 都是 $x^4 + x^3 + x^2 + x + 1$ 的根.
由条件可得
\begin{align*}
\varepsilon_i^3 f_1(1) + \varepsilon_i^2 f_2(1) + \varepsilon_i f_3(1) + f_4(1) = 0 \quad (1 \leqslant  i \leqslant  4)。
\end{align*}
这是一个由 4 个未知数、4 个方程式组成的线性方程组(将 $f_i(1)$ 看成是未知数),其系数行列式是一个 Vandermonde 行列式,显然其值不等于零,因此 $f_i(1) = 0$。
\end{proof}


\end{document}
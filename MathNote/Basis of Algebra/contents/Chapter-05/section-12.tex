\documentclass[../../main.tex]{subfiles}
\graphicspath{{\subfix{../../image/}}} % 指定图片目录,后续可以直接使用图片文件名。

% 例如:
% \begin{figure}[H]
% \centering
% \includegraphics[scale=0.4]{图.png}
% \caption{}
% \label{figure:图}
% \end{figure}
% 注意:上述\label{}一定要放在\caption{}之后,否则引用图片序号会只会显示??.

\begin{document}

\section{其他}

\begin{theorem}[Gauss-Lucas定理]\label{theorem:Gauss-Lucas定理}
设 \( f \in \mathbb{C}[x] \) 且 \( \deg f \geqslant 1 \),证明 \( f' \) 所有零点位于 \( f \) 的零点的凸包内.
\end{theorem}
\begin{note}

\end{note}
\begin{proof}
设 \( f(z) = c \prod_{i=1}^m (z - z_i)^{m_i}, m_i \in \mathbb{N}, i = 1,2,\cdots,m, c \in \mathbb{C} \setminus \{0\} \),则
\[
\frac{f'(z)}{f(z)} = \sum_{i=1}^m \frac{m_i}{z - z_i}.
\]
设 \( f'(z_0) = 0, f(z_0) \neq 0 \),则
\[
0 = \sum_{i=1}^m \frac{m_i}{z_0 - z_i} = \sum_{i=1}^m \frac{m_i (\overline{z_0} - \overline{z_i})}{|z_0 - z_i|^2} \implies \sum_{i=1}^m \frac{m_i z_0}{|z_0 - z_i|^2} = \sum_{i=1}^m \frac{m_i z_i}{|z_0 - z_i|^2},
\]
即
\[
z_0 = \sum_{i=1}^m \lambda_i z_i, \lambda_i = \frac{m_i}{|z_0 - z_i|^2 \sum\limits_{j=1}^m \frac{m_j}{|z_0 - z_j|^2}} \in [0,1], j = 1,2,\cdots,m.
\]
当 \( f'(z_0) = 0 = f(z_0) \),此时当然定理更成立. 我们完成了证明.

\end{proof}

\begin{example}
如果 \( f \in \mathbb{R}[x] \) 是非负的, 证明: 存在 \( \varphi, \psi \in \mathbb{R}[x] \) 使得
\[
f = \varphi^2 + \psi^2.
\]
\end{example}
\begin{proof}
当 \( f \) 为常数多项式则显然, 下面假设 \( \deg f \geqslant 1 \).

设
\[
f(x) = a_0 (x - \alpha_1)^{k_1} (x - \alpha_2)^{k_2} \cdots (x - \alpha_r)^{k_r} g(x),
\]
这里 \( a_0 > 0 \) 且 \( k_i, i = 1,2,\cdots,r \) 都是正偶数且 \( g \) 没有实根.

设
\[
g(x) = (x - \beta_1)(x - \overline{\beta_1})(x - \beta_2)(x - \overline{\beta_2}) \cdots (x - \beta_s)(x - \overline{\beta_s}),
\]
我们有
\[
(x - \beta_1)(x - \beta_2) \cdots (x - \beta_s) = \mu(x) + i\nu(x), \mu, \nu \in \mathbb{R}[x].
\]
注意到
\[
(x - \overline{\beta_1})(x - \overline{\beta_2}) \cdots (x - \overline{\beta_s}) = \mu(x) - i\nu(x),
\]
于是
\[
g(x) = \mu^2(x) + \nu^2(x).
\]

考虑
\[
\varphi(x) \triangleq \sqrt{a_0} \prod_{i=1}^r (x - \alpha_i)^{\frac{k_i}{2}} \mu(x), \psi(x) \triangleq \sqrt{a_0} \prod_{i=1}^r (x - \alpha_i)^{\frac{k_i}{2}} \nu(x),
\]
就有
\[
f = \varphi^2 + \psi^2.
\]

\end{proof}

\begin{example}
设 \( f \) 是 \( \mathbb{C} \) 上无重根的非常数多项式且满足
\[
f(f(x)) = f^n(x) + a_{n-1}f^{n-1}(x) + \cdots + a_1f(x) + a_0,
\]
这里 \( a_0, a_1, \cdots, a_{n-1} \in \mathbb{Z} \). 证明 \( f \in \mathbb{Z}[x] \) 且若 \( a_0, a_1, \cdots, a_{n-1} \) 为奇数, 则 \( f \) 无偶整数根.
\end{example}
\begin{proof}
因为 \( f \) 值域包含无穷多个点, 并且由条件知
\[
f(x), x^n + a_{n-1}x^{n-1} + \cdots + a_1x + a_0,
\]
在无穷多个点上相等, 所以
\[
f(x) = x^n + a_{n-1}x^{n-1} + \cdots + a_1x + a_0 \in \mathbb{Z}[x].
\]
若 \( a_0, a_1, \cdots, a_{n-1} \) 为奇数, 由有理根的性质, 我们知道 \( f \) 的整数根 \( q \) 必须满足 \( q|a_0 \), 从而 \( q \) 为奇数, 这就完成了证明.

\end{proof}

\begin{proposition}
设 \( f,g \in \mathbb{C}[x] \) 是次数大于等于 1 的互素多项式, 证明必然存在唯一的 \( u,v \in \mathbb{C}[x] \) 使得
\[
f(x)u(x) + g(x)v(x) = 1,
\]
且 \( \deg u < \deg g, \deg v < \deg f \).
\end{proposition}
\begin{proof}
存在性:
因为 \( (f,g) = 1 \), 故必然存在 \( k,h \in \mathbb{C}[x] \) 使得 \( fh + gk = 1 \). 若 \( \deg h > \deg g \), 则做带余除法
\[
h = gq + u, \deg u < \deg g.
\]
于是
\[
f(gq + u) + gk = 1 \implies fu + g(fq + k) = 1.
\]
令 \( v = fq + k \), 则由 \( fu + g(fq + k) = 1 \) 和比较两边次数知 \( \deg v < \deg f \). 现在我们证明了存在性.

唯一性:
若还有 \( u_1, v_1 \in \mathbb{C}[x] \) 满足条件, 则
\[
f(u - u_1) = g(v_1 - v).
\]
又 \( f,g \) 互素, 故$
g|(u - u_1), f|(v_1 - v)$,又$\deg (u-u_1)<\deg g,\,\deg(v-v_1)<\deg f$,故$u = u_1, v = v_1$.
这就证明了唯一性.

\end{proof}

\begin{example}
设 \( \mathbb{F} \) 是一个数域, \( f,g \in \mathbb{F}[x], \deg g \geqslant 1 \), 证明: 存在唯一的 \( (f_0,f_1,\cdots,f_r,0,0,\cdots) \in (\mathbb{C}[x])^\infty \),使得
\[
\deg f_i < \deg g \text{或者} f_i = 0, 0 \leqslant i \leqslant r
\]
且
\[
f(x) = f_0(x) + f_1(x)g(x) + f_2(x)g^2(x) + \cdots + f_r(x)g^r(x).
\]
\end{example}
\begin{remark}
\( (\mathbb{C}[x])^\infty \) 表示 \( \mathbb{C}[x] \) 的可数笛卡尔积. 为书写方便, 我们可以记 \( f_i = 0, i \geqslant r + 1 \).
\end{remark}
\begin{proof}
为方便, 对 \( p \in \mathbb{C}[x] \), 我们约定 \( \deg p = -\infty \Leftrightarrow p = 0 \).

存在性:
由带余除法, 存在 \( r \in \mathbb{N}_0 \) 使得 \( \deg q_r < \deg g \)(否则,$g$就能做无穷次带余除法,但$g$的次数有限) 且
\[
f = q_1g + f_0, \deg f_0 < \deg g, q_1 = q_2g + f_1, \deg f_1 < \deg g
\]
\[
q_2 = q_3g + f_2, \deg f_2 < \deg g, q_3 = q_4g + f_3, \deg f_3 < \deg g
\]
\[
\vdots
\]
\[
q_{r-1} = q_r g + f_{r-1}, \deg f_{r-1} < \deg g,
\]
取 \( f_r = q_r \), 我们得
\[
f = f_0 + (q_2g + f_1)g = f_0 + f_1g + (q_3g + f_2)g^2 = \cdots
\]
\[
= f_0 + f_1g + \cdots + f_{r-1}g^{r-1} + q_r g^r = f_0 + f_1g + \cdots + f_r g^r.
\]

唯一性:
若有两种不同的表示
\[
f = \sum_{i=0}^\infty f_i g^i = \sum_{i=0}^\infty h_i g^i, \deg h_i, \deg f_i < \deg g, i \in \mathbb{N}_0,
\]
我们有
\[
\sum_{i=0}^\infty (f_i - h_i) g^i = 0.
\]
设上式使得 \( f_i - h_i \neq 0 \) 的最大的 \( i \) 为 \( k \), 则 \( \deg [(f_k - h_k) g^k] \geqslant k \deg g \) 以及
\[
\deg [(f_i - h_i) g^i] = \deg (f_i - h_i) + i \deg g < k \deg g, 0 \leqslant i \leqslant k,
\]
故 \( \sum_{i=0}^\infty (f_i - h_i) g^i \) 不可能是 0 多项式(最高次项消不掉), 矛盾! 这就证明了唯一性.

\end{proof}

\begin{example}
设 \( p,q \in \mathbb{R}[x] \) 满足
\[
p(x^2 + x + 1) = q(x^2 - x + 1),
\]
证明 \( p = q \) 为常数.
\end{example}
\begin{note}
利用等式关系反复迭代来得到无穷个根.
\end{note}
\begin{proof}
事实上
\[
(2x + 1)p'(x^2 + x + 1) = (2x - 1)q'(x^2 - x + 1), \forall x \in \mathbb{R},
\]
那么 \( p'\left( \left( \frac{1}{2} \right)^2 + \frac{1}{2} + 1 \right) = 0 \), 因此由(对称轴是$x=-\frac{1}{2}$)Vieta定理知
\[
\left( \frac{1}{2} \right)^2 + \frac{1}{2} + 1 = \left( -\frac{3}{2} \right)^2 - \frac{3}{2} + 1.
\]
从而 \( p'\left( \left( -\frac{3}{2} \right)^2 - \frac{3}{2} + 1 \right) = 0 \), 类似的从而 \( q'\left( \left( -\frac{3}{2} \right)^2 + \frac{3}{2} + 1 \right) = 0 \), (对称轴是$x=\frac{1}{2}$)从而 \( q'\left( \left( \frac{5}{2} \right)^2 - \frac{5}{2} + 1 \right) = 0 \), 从而 $p'$ $\left( \left( \frac{5}{2} \right)^2 + \frac{5}{2} + 1 \right) = 0$, 依次下去我们得到 \( p',q' \) 有无穷多个根, 因此都为 0, 这样结合 \( p(1) = q(1) \) 我们就证明了 \( p = q \) 为常数.

\end{proof}

\begin{example}
计算全部 \( f \in \mathbb{C}[x] \) 使得
\begin{align}
f(x^2) = f(x)f(x+1), \forall x \in \mathbb{C}. \label{23.173}
\end{align}
\end{example}
\begin{proof}
当 \( f = c \in \mathbb{C} \), 我们有 \( c = c^2 \), 故 \( f = 0 \) 或者 \( f = 1 \) 为所求.

当 \( f \) 不为常数, 设 \( f(x_1) = 0, x_1 \in \mathbb{C} \), 我们由\(\eqref{23.173}\)知
\[
f(x_1^2) = 0.
\]
反复运用\(\eqref{23.173}\)得 \( x_1^{2^n}, n \in \mathbb{N}_0 \) 都是 \( f \) 零点. 又非0多项式零点有限, 故设 \( x_1^{2^n} = x_1^{2^m}, n > m \geqslant 0 \), 则 \( x_1^{2^n - 2^m} = 1 \)或 \( x_1 = 0 \). 当前者发生, 我们有 \( |x_1| = 1 \). 从而 \( f \) 的所有根都在单位圆周上或者是 0.

设 \( f(x_1) = 0, x_1 \in \mathbb{C} \), 由\(\eqref{23.173}\)知 
\begin{align*}
f((x-1)^2)=f(x-1)f(x).
\end{align*}
故\( f\left( (x_1 - 1)^2 \right) = 0 \). 同理可得\( |x_1 - 1| = 1 \)或$x_1-1=0$.因此由$|x_1|=1,|x_1-1|=1,x_1=0\text{或}1$可知$f$只可能有$x_1=0,1,e^{\pm \frac{\pi}{3}i} $这四个零点.但是若$x_1=e^{\pm \frac{\pi}{3}i}$是$f$的零点,则由\eqref{23.173}知$x_1^2=e^{\pm \frac{2\pi}{3}i}$也是$f$的零点,矛盾!
从而 \( f \) 的零点只有 0 或者 1. 设 \( f(x) = cx^n(x - 1)^m \) 代入原方程得
\begin{align*}
f(x^2) &= cx^{2n}(x^2 - 1)^m = cx^{2n}(x - 1)^m(x + 1)^m \\
&= f(x)f(x+1) = c^2x^{n+m}(x - 1)^m(x + 1)^n \\
&\implies f(x) = x^n(x - 1)^n, n \in \mathbb{N}.
\end{align*}
这就完成了证明.

\end{proof}

\begin{example}
设$ p \in \mathbb{Z}[x] $使得$ p $在7个不同整数点取值为7, 证明$ p $没有整数根.
\end{example}
\begin{proof}
由条件可设
\begin{align*}
p\left( x \right) -7=\left( x-x_1 \right) \left( x-x_2 \right) \cdots \left( x-x_7 \right) q\left( x \right),
\end{align*}
其中$x_1,x_2,\cdots,x_7$为互不相同的整数,$q\in \mathbb{Z}[x]$.假设$a$为$p\left( x \right)$的整数根,则
\begin{align*}
-7=\left( a-x_1 \right) \left( a-x_2 \right) \cdots \left( a-x_7 \right) q\left( a \right),
\end{align*}
其中$a-x_i$为互不相同的整数.故$\left( a-x_1 \right),\left( a-x_2 \right),\cdots,\left( a-x_7 \right)$是$-7$的$7$个不同因子.这与$-7$只有因子$1,-1,7,-7$这四个不同因子矛盾!

\end{proof}

\begin{example}

\end{example}
\begin{proof}


\end{proof}







\end{document}
\documentclass[../../main.tex]{subfiles}
\graphicspath{{\subfix{../../image/}}} % 指定图片目录,后续可以直接使用图片文件名。

% 例如:
% \begin{figure}[h]
% \centering
% \includegraphics{image-01.01}
% \label{fig:image-01.01}
% \caption{图片标题}
% \end{figure}

\begin{document}

\section{复系数多项式}

\begin{theorem}[代数基本定理]\label{theorem:代数基本定理}
次数大于零的复数域上的一元多项式至少有一个复数根。
\end{theorem}
\begin{proof}
设复数域上的 $n$ 次多项式为
\begin{align}
f(z) = a_n z^n + a_{n-1} z^{n-1} + \cdots + a_1 z + a_0。 \label{theorem5.14-5.6.1}
\end{align}
首先证明,必存在一个复数 $z_0$,使对一切复数 $z$,有
\begin{align*}
|f(z)| \geq |f(z_0)|。
\end{align*}
令 $z = x + iy$,其中 $x, y$ 是实变量。展开 $f(x + iy)$ 并分开实部和虚部,则
\begin{align*}
f(z) = u(x, y) + iv(x, y),
\end{align*}
其中 $u(x, y)$ 及 $v(x, y)$ 为实系数二元多项式函数。又
\begin{align*}
|f(z)| = \sqrt{u(x, y)^2 + v(x, y)^2}
\end{align*}
是一个二元连续函数,但
\begin{align*}
|f(z)| &= |a_n z^n + a_{n-1} z^{n-1} + \cdots + a_0| \\
&\geq |a_n z^n| - |a_{n-1} z^{n-1} + \cdots + a_0| \\
&\geq |z|^n \left( |a_n| - \left| \frac{a_{n-1}}{z} \right| - \left| \frac{a_{n-2}}{z^2} \right| + \cdots + \left| \frac{a_0}{z^n} \right| \right),
\end{align*}
因此当 $|z| \to \infty$ 时,$|f(z)| \to \infty$。于是必存在一个实常数 $R$,当 $|z| \geq R$ 时,$|f(z)|$ 充分大,因此 $|f(z)|$ 的最小值必含于圆圈 $|z| \leq R$ 中。但这是平面上的一个闭区域,因此必存在 $z_0$ 使 $|f(z_0)|$ 为最小。

接下来要证明 $f(z_0) = 0$。用反证法,即若 $f(z_0) \neq 0$,则必可找到 $z_1$,使
$|f(z_1)| < |f(z_0)|$,这样就与 $|f(z_0)|$ 是最小值相矛盾。
将 $z = z_0 + h$ 代入 \eqref{theorem5.14-5.6.1}式便可得到一个关于 $h$ 的 $n$ 次多项式:
\begin{align}
f(z_0 + h) = b_n h^n + b_{n-1} h^{n-1} + \cdots + b_1 h + b_0. \label{theorem5.14-5.6.2}
\end{align}
当 $h = 0$ 时,$f(z_0) = b_0$,由假设 $f(z_0) \neq 0$,故
\begin{align*}
\frac{f(z_0 + h)}{f(z_0)} = \frac{b_n}{f(z_0)} h^n + \frac{b_{n-1}}{f(z_0)} h^{n-1} + \cdots + \frac{b_1}{f(z_0)} h + 1。
\end{align*}
$b_1, b_2, \ldots, b_n$ 中有些可能为零,但绝不全为零。设 $b_k$ 是第一个不为零的复数,则
\begin{align}
\frac{f(z_0 + h)}{f(z_0)} = 1 + c_k h^k + c_{k+1} h^{k+1} + \cdots + c_n h^n, \label{theorem5.14-5.6.3}
\end{align}
其中 $c_j = \frac{b_j}{f(z_0)}$。令 $d = \sqrt[k]{\frac{1}{|c_k|}}$,$h = ed$ 代入\eqref{theorem5.14-5.6.3}式得
\begin{align*}
\frac{f(z_0 + h)}{f(z_0)} = 1 - e^k + e^{k+1} (c_{k+1} d^{k+1} + c_{k+2} d^{k+2} e + \cdots)。
\end{align*}
取充分小的正实数 $e$(至少小于 1),使
\begin{align*}
e (|c_{k+1} d^{k+1}| + |c_{k+2} d^{k+2}| + \cdots) < \frac{1}{2},
\end{align*}
于是
\begin{align*}
\left| \frac{f(z_0 + h)}{f(z_0)} \right| &\leq |1 - e^k| + |e^{k+1} (c_{k+1} d^{k+1} + c_{k+2} d^{k+2} e + \cdots)| \\
&\leq 1 - e^k + e^{k+1} (|c_{k+1} d^{k+1}| + |c_{k+2} d^{k+2}| + \cdots) \\
&< 1 - e^k + \frac{1}{2} e^k \\
&= 1 - \frac{1}{2} e^k < 1。
\end{align*}
将这样的 $e$ 代入 $h = ed$,得
\begin{align*}
|f(z_0 + ed)| < |f(z_0)|。
\end{align*}
这就推出了矛盾。
\end{proof}

\begin{corollary}
\begin{enumerate}
\item 复数域上的一元 $n$ 次多项式恰有 $n$ 个复根(包括重根)。

\item 复数域上的不可约多项式都是一次多项式。

\item 复数域上的一元 $n$ 次多项式必可分解为一次因式的乘积。
\end{enumerate}
\end{corollary}

\begin{theorem}[Vieta定理]\label{theorem:Vieta定理}
若数域\(\mathbb{F}\)上的多项式\(f(x)=a_0x^n + a_1x^{n - 1}+\cdots + a_{n - 1}x + a_n\)在\(\mathbb{F}\)中有\(n\)个根\(x_1,x_2,\cdots,x_n\),则
\begin{align*}
&\,\,\,\, \sum_{i=1}^n{x_i}=x_1+x_2+\cdots +x_n=-\frac{a_1}{a_0},
\\
&\sum_{1\le i<j\le n}{x_ix_j}=x_1x_2+\cdots +x_1x_n+x_2x_3+\cdots +x_{n-1}x_n=\frac{a_2}{a_0},
\\
&\,\,\,\, \cdots \cdots \cdots \cdots 
\\
&\,\,\,\, x_1x_2\cdots x_n=(-1)^n\frac{a_n}{a_0}.
\end{align*}
\end{theorem}
\begin{proof}
$f(x) = a_0 (x - x_1)(x - x_2) \cdots (x - x_n)$,将这个式子的右边展开与 $f(x)$ 比较系数即得结论。
\end{proof}

\begin{example}
\begin{enumerate}[(1)]
\item 设三次方程 $x^3 + px^2 + qx + r = 0$ 的 3 个根成等差数列,求证:
\begin{align*}
2p^3 - 9pq + 27r = 0。
\end{align*}

\item 设三次方程 $x^3 + px^2 + qx + r = 0 \ (r \neq 0)$ 的 3 个根成等比数列,求证:
\begin{align*}
rp^3 = q^3。
\end{align*}

\item 设多项式 $x^3 + 3x^2 + mx + n$ 的 3 个根成等差数列,多项式 $x^3 - (m - 2)x^2 + (n - 3)x + 8$ 的 3 个根成等比数列,求 $m$ 和 $n$。
\end{enumerate}
\end{example}
\begin{proof}
\begin{enumerate}[(1)]
\item 设方程的 3 个根为 $c - d, c, c + d$,则由 Vieta 定理可得
\begin{align*}
\begin{cases}
3c = -p, \\
3c^2 - d^2 = q, \\
c(c^2 - d^2) = -r。
\end{cases}
\end{align*}
由此可得 $2p^3 - 9pq + 27r = 0。$

\item 设方程的 3 个根为 $\frac{c}{d}, c, cd$,则由 Vieta 定理可得
\begin{align*}
\begin{cases}
\frac{c}{d} + c + cd = -p, \\
\frac{c^2}{d} + c^2 + c^2 d = q, \\
\frac{c^3}{d} = -r。
\end{cases}
\end{align*}
由此可得 $rp^3 = q^3。$

\item 由(1)(2)可知,$m, n$ 应满足如下关系:
\begin{align*}
\begin{cases}
m = n + 2, \\
-8(m - 2)^3 = (n - 3)^3。
\end{cases}
\end{align*}
若 $n - 3 = -2(m - 2)$,则可联立求得 $m = 3, n = 1$。

若 $n - 3 = -2\omega(m - 2)$,其中 $\omega = -\frac{1}{2} + \frac{\sqrt{3}}{2}i$,则可联立求得 $m = 2 - \sqrt{3}i, n = -\sqrt{3}i$。

若 $n - 3 = -2\omega^2(m - 2)$,则可联立求得 $m = 2 + \sqrt{3}i, n = \sqrt{3}i。$
\end{enumerate}
\end{proof}

\begin{example}
设 $x_1, x_2, x_3$ 是三次方程 $x^3 + px^2 + qx + r = 0 \ (r \neq 0)$ 的 3 个根,求这 3 个根倒数的平方和。
\end{example}
\begin{proof}
由 Vieta 定理可得
\begin{align*}
\frac{1}{x_1^2} + \frac{1}{x_2^2} + \frac{1}{x_3^2} = \frac{(x_1 x_2 + x_1 x_3 + x_2 x_3)^2 - 2x_1 x_2 x_3 (x_1 + x_2 + x_3)}{x_1^2 x_2^2 x_3^2}= \frac{q^2 - 2pr}{r^2}。
\end{align*}
\end{proof}

\begin{example}
已知方程 $x^3 + px^2 + qx + r = 0$ 的 3 个根为 $x_1, x_2, x_3$,求一个三次方程使其根为 $x_1^3, x_2^3, x_3^3$。
\end{example}
\begin{note}
利用代数恒等式:$a^3+b^3+c^3=\left( a+b+c \right) ^3-3\left( a+b+c \right) \left( ab+bc+ac \right) +3abc$得到
\begin{align*}
x_{1}^{3}+x_{2}^{3}+x_{3}^{3}=\left( x_1+x_2+x_3 \right) ^3-3\left( x_1+x_2+x_3 \right) \left( x_1x_2+x_2x_3+x_1x_3 \right) +3x_1x_2x_3.
\\
x_{1}^{3}x_{2}^{3}+x_{1}^{3}x_{3}^{3}+x_{2}^{3}x_{3}^{3}=\left( x_1x_2+x_2x_3+x_1x_3 \right) ^3-3x_1x_2x_3\left( x_1+x_2+x_3 \right) \left( x_1x_2+x_2x_3+x_1x_3 \right) +3x_{1}^{3}x_{3}^{3}x_{2}^{3}x_{3}^{3}.
\end{align*}
即可由Vieta 定理得到结果.
\end{note}
\begin{proof}
由 Vieta 定理经计算可得
\begin{align*}
\begin{cases}
x_1^3 + x_2^3 + x_3^3 = -p^3 + 3pq - 3r, \\
x_1^3 x_2^3 + x_1^3 x_3^3 + x_2^3 x_3^3 = q^3 - 3pqr + 3r^2, \\
x_1^3 x_2^3 x_3^3 = -r^3。
\end{cases}
\end{align*}
因此,以 $x_1^3, x_2^3, x_3^3$ 为根的三次方程为
\begin{align*}
x^3 + (p^3 - 3pq + 3r)x^2 + (q^3 - 3pqr + 3r^2)x + r^3 = 0。□
\end{align*}
\end{proof}

\begin{example}
设多项式 $x^3 + px^2 + qx + r$ 的 3 个根都是实数,求证:$p^2 \geq 3q$。
\end{example}
\begin{proof}
设多项式的 3 个根为 $x_1, x_2, x_3$,由已知条件可知:
\begin{align*}
(x_1 - x_2)^2 + (x_2 - x_3)^2 + (x_1 - x_3)^2 \geq 0。
\end{align*}
用 Vieta 定理可计算出
\begin{align*}
&(x_1-x_2)^2+(x_2-x_3)^2+(x_1-x_3)^2=2\left( x_{1}^{2}+x_{2}^{2}+x_{3}^{2} \right) -2\left( x_1x_2+x_2x_3+x_1x_3 \right) 
\\
&=2\left( x_1+x_2+x_3 \right) ^2-6\left( x_1x_2+x_2x_3+x_1x_3 \right) =2(p^2-3q).
\end{align*}
因此结论为真。
\end{proof}

\begin{example}
设 $f(x) = a_n x^n + a_{n-1} x^{n-1} + \cdots + a_1 x + a_0$ 的 $n$ 个根 $x_1, x_2, \cdots, x_n$ 皆不等于零,求以 $\frac{1}{x_1}, \frac{1}{x_2}, \cdots, \frac{1}{x_n}$ 为根的多项式。
\end{example}
\begin{proof}
令
\begin{align*}
g(x) = a_0 x^n + a_1 x^{n-1} + \cdots + a_{n-1} x + a_n,
\end{align*}
则
\begin{align*}
x^n g\left(\frac{1}{x_i}\right) = a_0 + a_1 x_i + \cdots + a_{n-1} x_i^{n-1} + a_n x_i^n = f(x_i) = 0。
\end{align*}
因为 $x_i \neq 0$,故 $g\left(\frac{1}{x_i}\right) = 0$,即 $g(x)$ 的根为 $f(x)$ 根之倒数。
\end{proof}

\begin{example}
设 $f(x) = a_n x^n + a_{n-1} x^{n-1} + \cdots + a_1 x + a_0 \ (a_n \neq 0)$ 是数域 $\mathbb{F}$ 上的可约多项式,求证:多项式 $g(x) = a_0 x^n + a_1 x^{n-1} + \cdots + a_{n-1} x + a_n$ 在 $\mathbb{F}$ 上也可约。
\end{example}
\begin{proof}
设 $f(x) = p(x) q(x)$,其中 $\deg p(x) = m, \deg q(x) = n - m, 0 < m < n$,则
\begin{align*}
g(x) = x^n f\left(\frac{1}{x}\right) = x^n p\left(\frac{1}{x}\right) q\left(\frac{1}{x}\right) = \left(x^m p\left(\frac{1}{x}\right)\right) \left(x^{n-m} q\left(\frac{1}{x}\right)\right),
\end{align*}
因此 $g(x)$ 也可约.
\end{proof}


\end{document}
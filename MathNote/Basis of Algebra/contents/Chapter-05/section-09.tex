\documentclass[../../main.tex]{subfiles}
\graphicspath{{\subfix{../../image/}}} % 指定图片目录,后续可以直接使用图片文件名。

% 例如:
% \begin{figure}[H]
% \centering
% \includegraphics[scale=0.3]{image-01.01}
% \caption{图片标题}
% \label{figure:image-01.01}
% \end{figure}
% 注意:上述\label{}一定要放在\caption{}之后,否则引用图片序号会只会显示??.

\begin{document}

\section{多元多项式}

因式分解定理对多元多项式仍成立.(证明见抽象代数内容)

\begin{lemma}\label{lemma:多元多项式乘积的首项}
若 $f(x_1, x_2, \cdots, x_n)$ 及 $g(x_1, x_2, \cdots, x_n)$ 都是 $K$ 上非零的 $n$ 元多项式,则按字典排列法排列后乘积的首项等于 $f$ 的首项与 $g$ 的首项之积。
\end{lemma}
\begin{proof}
设 $ax_1^{i_1} x_2^{i_2} \cdots x_n^{i_n}$ 和 $bx_1^{j_1} x_2^{j_2} \cdots x_n^{j_n}$ 分别是 $f$ 和 $g$ 的首项(按字典排列法),它们的乘积为 $abx_1^{i_1 + j_1} x_2^{i_2 + j_2} \cdots x_n^{i_n + j_n}$。其他任意两个单项式 $cx_1^{k_1} x_2^{k_2} \cdots x_n^{k_n}$ 和 $\mathrm{d}x_1^{r_1} x_2^{r_2} \cdots x_n^{r_n}$ 之积为 $c\mathrm{d}x_1^{k_1 + r_1} x_2^{k_2 + r_2} \cdots x_n^{k_n + r_n}$。设 $i_1 = k_1, \cdots, i_{t-1} = k_{t-1}, i_t > k_t; j_1 = r_1, \cdots, j_{s-1} = r_{s-1}, j_s > r_s$。不妨设 $t \leq s$,显然

\begin{align*}
i_1 + j_1 = k_1 + r_1, \cdots, i_{t-1} + j_{t-1} = k_{t-1} + r_{t-1}, i_t + j_t > k_t + r_t.
\end{align*}

因此 $abx_1^{i_1 + j_1} x_2^{i_2 + j_2} \cdots x_n^{i_n + j_n}$ 先于 $c\mathrm{d}x_1^{k_1 + r_1} x_2^{k_2 + r_2} \cdots x_n^{k_n + r_n}$。

同理可证明:$abx_1^{i_1 + j_1} x_2^{i_2 + j_2} \cdots x_n^{i_n + j_n}$ 先于 $a\mathrm{d}x_1^{i_1 + r_1} x_2^{i_2 + r_2} \cdots x_n^{i_n + r_n}$ 和 $cbx_1^{k_1 + j_1} x_2^{k_2 + j_2} \cdots x_n^{k_n + j_n}$。因此它确是 $fg$ 的首项。
\end{proof}

\begin{proposition}[多元多项式的整性]\label{proposition:多元多项式的整性}
若 $f(x_1, x_2, \cdots, x_n) \neq 0, g(x_1, x_2, \cdots, x_n) \neq 0$,则
\begin{align*}
f(x_1, x_2, \cdots, x_n)g(x_1, x_2, \cdots, x_n) \neq 0.
\end{align*}
\end{proposition}
\begin{proof}
由 $f$ 和 $g$ 的首项不为零及\hyperref[lemma:多元多项式乘积的首项]{引理\ref{lemma:多元多项式乘积的首项}}可知 $fg$ 的首项不为零,于是 $fg \neq 0$。
\end{proof}

\begin{corollary}
若 $h(x_1, x_2, \cdots, x_n) \neq 0$,且
\begin{align*}
f(x_1, x_2, \cdots, x_n)h(x_1, x_2, \cdots, x_n) &= g(x_1, x_2, \cdots, x_n)h(x_1, x_2, \cdots, x_n),
\end{align*}
则
\begin{align*}
f(x_1, x_2, \cdots, x_n) &= g(x_1, x_2, \cdots, x_n).
\end{align*}
\end{corollary}
\begin{proof}
由条件可得
\begin{align*}
\left[ f(x_1,x_2,\cdots ,x_n)-g(x_1,x_2,\cdots ,x_n) \right] h(x_1,x_2,\cdots ,x_n)=0.
\end{align*}
又因为$h(x_1,x_2,\cdots ,x_n)\ne 0,所以若 f(x_1,x_2,\cdots ,x_n)-g(x_1,x_2,\cdots ,x_n)\ne 0$,则由命题 5.25 可知
\begin{align*}
\left[ f(x_1,x_2,\cdots ,x_n)-g(x_1,x_2,\cdots ,x_n) \right] h(x_1,x_2,\cdots ,x_n)\ne 0
\end{align*}
矛盾!故$f(x_1,x_2,\cdots ,x_n)-g(x_1,x_2,\cdots ,x_n)=0$,即\[f(x_1,x_2,\cdots ,x_n)=g(x_1,x_2,\cdots ,x_n).\]
\end{proof}



\begin{definition}[齐次多项式]
若一个多项式 \(f(x_1,x_2,\cdots ,x_n)\) 的每个单项式都是 \(k\) 次式,则\textbf{称之为 \(\boldsymbol{k}\) 次齐次多项式或 \(\boldsymbol{k}\) 次型}.
\end{definition}

\begin{proposition}[齐次多项式的基本性质]
\begin{enumerate}[(1)]
\item 两个次数相同的齐次多项式之和若不为零,则必仍是同次齐次多项式。任意两个齐次多项式之积仍为齐次多项式。

\item 任一 \(n\) 元多项式均可表示为若干个齐次多项式之和,
\end{enumerate}
\end{proposition}
\begin{proof}
\begin{enumerate}[(1)]
\item 显然.

\item 这只需要将各次数相等的项放在一起即可。
\end{enumerate}
\end{proof}

\begin{lemma}[多元多项式的非零性]\label{lemma:多元多项式的非零性}
设 \(f(x_1,x_2,\cdots ,x_n)\) 是 \(K\) 上非零的 \(n\) 元多项式,则必存在 \(K\) 中的数 \(a_1,a_2,\cdots ,a_n\),使 \(f(a_1,a_2,\cdots ,a_n) \ne 0\)。
\end{lemma}
\begin{proof}
对未定元个数 \(n\) 用数学归纳法。当 \(n=1\) 时,多项式 \(f(x)\) 只有有限个零点,故总有 \(a \in K\) 使 \(f(a) \ne 0\)。现设对有 \(n-1\) 个未定元的多项式结论成立,将 \(f(x_1,x_2,\cdots ,x_n)\) 写成未定元 \(x_n\) 的多项式:
\begin{align*}
f(x_1,x_2,\cdots ,x_n) = b_0 + b_1 x_n + \cdots + b_m x_n^m,
\end{align*}
其中 \(b_i = b_i(x_1,x_2,\cdots ,x_{n-1})\) 是 \(n-1\) 元多项式。因为 \(f(x_1,x_2,\cdots ,x_n) \ne 0\),故可设 \(b_m \ne 0\)。由归纳假设,存在 \(a_1,\cdots ,a_{n-1} \in K\),使
\begin{align*}
b_m(a_1,\cdots ,a_{n-1}) \ne 0.
\end{align*}
因而
\begin{align*}
f(a_1,\cdots ,a_{n-1},x_n) = b_0(a_1,\cdots ,a_{n-1}) + b_1(a_1,\cdots ,a_{n-1}) x_n + \cdots + b_m(a_1,\cdots ,a_{n-1}) x_n^m
\end{align*}
是一个非零的以 \(x_n\) 为未定元的一元多项式,故存在 \(a_n \in K\),使
\begin{align*}
f(a_1,a_2,\cdots ,a_n) \ne 0.
\end{align*}
\end{proof}

\begin{proposition}[多元多项式相等的充要条件]\label{proposition:多元多项式相等的充要条件}
数域 \(K\) 上的两个 \(n\) 元多项式 \(f(x_1,x_2,\cdots ,x_n)\) 与 \(g(x_1,x_2,\cdots ,x_n)\) 相等的充分必要条件是对一切 \(a_1,a_2,\cdots ,a_n \in K\),均有
\begin{align*}
f(a_1,a_2,\cdots ,a_n) = g(a_1,a_2,\cdots ,a_n).
\end{align*}
\end{proposition}
\begin{proof}
只需证明充分性。作
\begin{align*}
h(x_1,x_2,\cdots ,x_n) = f(x_1,x_2,\cdots ,x_n) - g(x_1,x_2,\cdots ,x_n).
\end{align*}
若 \(h(x_1,x_2,\cdots ,x_n) \ne 0\),则由\hyperref[lemma:多元多项式的非零性]{多元多项式的非零性非零}可知必有 \(a_1,a_2,\cdots ,a_n \in K\),使
\begin{align*}
h(a_1,a_2,\cdots ,a_n) \ne 0,
\end{align*}
这与假设矛盾.
\end{proof}

\begin{example}
设 \(f(x_1,\cdots ,x_n), g(x_1,\cdots ,x_n) \ne 0\) 是 \(K\) 上的多元多项式。假设对一切使 \(g(a_1,\cdots ,a_n) \ne 0\) 的 \(a_1,\cdots ,a_n \in K\),均有 \(f(a_1,\cdots ,a_n) = 0\),求证:
\begin{align*}
f(x_1,\cdots ,x_n) = 0.
\end{align*}
\end{example}
\begin{proof}
用反证法,假设 \(f(x_1,\cdots ,x_n) \ne 0\),则由\hyperref[proposition:多元多项式的整性]{多元多项式的整性}可知
\begin{align*}
h(x_1,\cdots ,x_n) = f(x_1,\cdots ,x_n) g(x_1,\cdots ,x_n) \ne 0,
\end{align*}
于是由\hyperref[lemma:多元多项式的非零性]{元多项式的非零性}存在 \(a_1,\cdots ,a_n \in K\),使得 \(h(a_1,\cdots ,a_n) \ne 0\),从而\(f(a_1,\cdots ,a_n) \ne 0\) 并且 \(g(a_1,\cdots ,a_n) \ne 0\),这与条件矛盾.
\end{proof}

\begin{proposition}
设 $A(x_1, x_2, \cdots, x_m) = (a_{ij})$ 为 $n$ 阶方阵,其元素 $a_{ij} = a_{ij}(x_1, x_2, \cdots, x_m)$ 都是 $\mathbb{K}$ 上的多元多项式。设 $g(x_1, x_2, \cdots, x_m), h_i(x_1, x_2, \cdots, x_m) \neq 0 \ (1 \leq i \leq k)$都是 $\mathbb{K}$ 上的多元多项式,
\[
U = \left\{ (a_1, a_2, \cdots, a_m) \in \mathbb{K}^m \mid h_i(a_1, a_2, \cdots, a_m) \neq 0 \ (1 \leq i \leq k) \right\}.
\]
若对所有的 $(a_1, a_2, \cdots, a_m) \in U$,都成立
\[
|A(a_1, a_2, \cdots, a_m)| = g(a_1, a_2, \cdots, a_m),
\]
证明:$|A(x_1, x_2, \cdots, x_m)| = g(x_1, x_2, \cdots, x_m)$。
\end{proposition}
\begin{note}
这个命题告诉我们:在元素为多元多项式的文字行列式的求值过程中,在假设某些非零条件下成立的情形下得到的结果,其实就是所求行列式的值。因此,在求行列式的过程中,\textbf{可以暂不考虑未定元取特殊值的情形,而把主要精力放在一般的情形进行计算即可}.
\end{note}
\begin{proof}
用反证法,设 $(|A| - g)(x_1, x_2, \cdots, x_m) \neq 0$,则由\hyperref[lemma:多元多项式的整性]{多元多项式的整性}可知
\[
(|A| - g)h_1 \cdots h_k (x_1, x_2, \cdots, x_m) \neq 0,
\]
于是存在 $a_1, a_2, \cdots, a_m \in \mathbb{K}$,使得 $(|A| - g)h_1 \cdots h_k (a_1, a_2, \cdots, a_m) \neq 0$,从而 $h_i(a_1, a_2, \cdots, a_m) \neq 0 \ (1 \leq i \leq k)$,即 $(a_1, a_2, \cdots, a_m) \in U$,并且 $(|A| - g)(a_1, a_2, \cdots, a_m) \neq 0$,即 $|A(a_1, a_2, \cdots, a_m)| \neq g(a_1, a_2, \cdots, a_m)$,这与条件矛盾。
\end{proof}

\begin{theorem}[行列式的求根法]\label{theorem:行列式的求根法}
设 $A(x_1, x_2, \cdots, x_m) = (a_{ij})$ 为 $n$ 阶方阵,其元素 $a_{ij} = a_{ij}(x_1, \cdots, x_m)$ 都是 $\mathbb{K}$ 上的多元多项式,于是 $|A|$ 也是 $\mathbb{K}$ 上的多元多项式。若把 $x_1$ 看成未定元,则可将 $|A|$ 整理成关于 $x_1$ 的一元多项式:
\begin{align}
|A| &= c_0(x_2, \cdots, x_m)x_1^d + c_1(x_2, \cdots, x_m)x_1^{d-1} + \cdots + c_d(x_2, \cdots, x_m),\label{equation5.1-theorem546}
\end{align}
其中 $c_0(x_2, \cdots, x_m) \neq 0, d \geq 1$ 为次数。假设存在互异的多项式 $g_1(x_2, \cdots, x_m), \cdots, g_d(x_2, \cdots, x_m)$,使得当 $x_1 = g_i(x_2, \cdots, x_m) (1 \leq i \leq d)$ 时 $|A| = 0$,证明:
\begin{align*}
|A| &= c_0(x_2, \cdots, x_m)(x_1 - g_1(x_2, \cdots, x_m)) \cdots (x_1 - g_d(x_2, \cdots, x_m))。
\end{align*}
\end{theorem}
\begin{proof}
由假设 $0 = c_0(x_2, \cdots, x_m)g_1^d + \cdots + c_{d-1}(x_2, \cdots, x_m)g_1 + c_d(x_2, \cdots, x_m)$,故有
\begin{align*}
|A| &= c_0(x_2, \cdots, x_m)(x_1^d - g_1^d) + \cdots + c_{d-1}(x_2, \cdots, x_m)(x_1 - g_1)R_1(x_1, x_2, \cdots, x_m)。
\end{align*}
在上述式中令 $x_1 = g_2(x_2, \cdots, x_m)$,则有 $0 = (g_2 - g_1)R_1(g_2, x_2, \cdots, x_m)$。注意到 $g_2 - g_1 \neq 0$,故由多元多项式的整性可得 $R_1(g_2, x_2, \cdots, x_m) = 0$。再由相同的讨论可知,$x_1 - g_2$ 是 $R_1(x_1, x_2, \cdots, x_m)$ 的因式,从而
\begin{align*}
|A| &= (x_1 - g_1)(x_1 - g_2)R_2(x_1, x_2, \cdots, x_m)。
\end{align*}
不断地这样做下去,可得
\begin{align*}
|A| &= (x_1 - g_1) \cdots (x_1 - g_d)R_d(x_1, x_2, \cdots, x_m)。
\end{align*}
最后与\eqref{equation5.1-theorem546}式比较 $x_1$ 的首项系数可得 $R_d(x_1, x_2, \cdots, x_m) = c_0(x_2, \cdots, x_m)$.
\end{proof}


\end{document}
\documentclass[../../main.tex]{subfiles}
\graphicspath{{\subfix{../../image/}}} % 指定图片目录,后续可以直接使用图片文件名。

% 例如:
% \begin{figure}[h]
% \centering
% \includegraphics{image-01.01}
% \caption{图片标题}
% \label{fig:image-01.01}
% \end{figure}
% 注意:上述\label{}一定要放在\caption{}之后,否则引用图片序号会只会显示??.

\begin{document}

\section{不可约多项式与因式分解}

\begin{definition}[不可约多项式的定义]\label{definition:不可约多项式的定义}
设\(f(x)\)是数域\(\mathbb{F}\)上的多项式,若\(f(x)\)可以分解为两个次数小于\(f(x)\)的\(\mathbb{F}\)上多项式之积,则称\(f(x)\)是\(\mathbb{F}\)上的可约多项式,否则称\(f(x)\)为\(\mathbb{F}\)上的不可约多项式.
\end{definition}

\begin{proposition}[不可约多项式的基本性质]\label{proposition:不可约多项式的基本性质}
\begin{enumerate}[(1)]
\item 设\(p(x)\)是数域\(\mathbb{K}\)上的不可约多项式, 则对\(\mathbb{K}\)上任一多项式\(f(x)\), 或者\(p(x)\mid f(x)\), 或者\((p(x),f(x)) = 1\).

\item 设\(p(x)\)是数域\(\mathbb{F}\)上的不可约多项式, \(f(x)\)是\(\mathbb{F}\)上的多项式. 证明: 若\(p(x)\)的某个复根\(a\)也是\(f(x)\)的根, 则\(p(x)\mid f(x)\). 特别地, \(p(x)\)的任一复根都是\(f(x)\)的根.
\end{enumerate}
\end{proposition}
\begin{note}
不可约多项式的基本性质(2)表明:不可约多项式也满足\hyperref[proposition:极小多项式的基本性质]{极小多项式的基本性质}.
\end{note}
\begin{proof}
\begin{enumerate}[(1)]
\item 设\(d(x)=(p(x),f(x))\). 因为\(p(x)\)不可约, 故\(f(x)\)的因式只能是非零常数多项式或\(cp(x)(c\neq 0)\), 从而或者\(d(x)=1\)或者\(d(x)=cp(x)\) (首一多项式), 故得结论.

\item 若\((p(x),f(x)) = 1\), 则存在\(\mathbb{F}\)上的多项式\(u(x),v(x)\), 使得\(p(x)u(x)+f(x)v(x)=1\). 令\(x = a\)可得\(1 = p(a)u(a)+f(a)v(a)=0\), 矛盾. 因此\(p(x)\)与\(f(x)\)不互素, 从而只能是\(p(x)\mid f(x)\), 结论得证. 
\end{enumerate}

\end{proof}

\begin{theorem}[不可约多项式的“素性”]\label{theorem:不可约多项式的“素性”}
设\(p(x)\)是数域\(\mathbb{F}\)上的非常数多项式,则\(p(x)\)为\(\mathbb{F}\)上不可约多项式的充要条件是对\(\mathbb{F}\)上任意适合\(p(x)\mid f(x)g(x)\)的多项式\(f(x)\)与\(g(x)\), 或者\(p(x)\mid f(x)\), 或者\(p(x)\mid g(x)\).
\end{theorem}
\begin{proof}
必要性:设\(p(x)\)是\(\mathbb{F}[x]\)中的不可约多项式,且\(p(x)\mid f(x)g(x)\)。若\(p(x)\mid f(x)\),则结论成立。
若\(p(x)\nmid f(x)\),则由定理可知\((p(x),f(x)) = 1\),从而由互素多项式与最大公因式的基本性质可知\(p(x)\mid g(x)\).

充分性:(反证法)假设\(p(x)\)可约,则必存在次数小于\(\text{deg}(p(x))\)的多项式\(f(x)\),\(g(x)\),使得\(p(x) = f(x)g(x)\).从而\(p(x) \mid f(x)g(x)\),于是由条件可知\(p(x) \mid f(x)\)或\(p(x) \mid g(x)\).因此\(\text{deg}(p(x)) \leq \text{deg}(f(x))\)或\(\text{deg}(g(x))\).这与\(\text{deg}(p(x)) > \text{deg}(f(x))\),\(\text{deg}(g(x))\)矛盾.
\end{proof}

\begin{corollary}\label{corollary:不可约多项式“素性”的推论}
设\(p(x)\)为不可约多项式且
\[
p(x)\mid f_1(x)f_2(x)\cdots f_m(x),
\]
则\(p(x)\)必可整除其中某个\(f_i(x)\).
\end{corollary}
\begin{proof}
由\hyperref[theorem:不可约多项式的“素性”]{不可约多项式的“素性”}归纳可得.
\end{proof}

\begin{proposition}\label{proposition:多项式可以写成不可约多项式的幂的充要条件}
设\(f(x)\)是数域\(\mathbb{F}\)上的非常数多项式, 求证: \(f(x)\)等于某个不可约多项式的幂的充要条件是对任意的非常数多项式\(g(x)\), 或者\(f(x)\)和\(g(x)\)互素, 或者\(f(x)\)可以整除\(g(x)\)的某个幂.
\end{proposition}
\begin{proof}
设\(f(x)=p(x)^k\), \(p(x)\)在\(\mathbb{F}\)上不可约,且\(f(x)\)和\(g(x)\)不互素, 则\(p(x)\)是\(f(x)\)和\(g(x)\)的公因式, 故\(f(x)\)可以整除\(g(x)^k\).

反之, 由\hyperref[theorem:因式分解定理]{因式分解定理},可设\(f(x)=p(x)^mh(x)\), \(p(x)\)在\(\mathbb{F}\)上不可约, \(\mathrm{deg }\,h(x)>0\), 且\(p(x)\)不能整除\(h(x)\), 则\(f(x)\mid h(x)\),故$f(x)$不和$h(x)$互素.由于$\mathrm{deg}\,h(x)<\mathrm{deg }\,f(x)$,因此$f(x)$也不能整除\(h(x)\),矛盾!
\end{proof}

\begin{definition}[代数数]\label{definition:代数数}
设\(u\)是复数域中某个数, 若\(u\)适合某个非零有理系数多项式 (或整系数多项式)\(f(x)=a_nx^n + a_{n - 1}x^{n - 1}+\cdots+a_1x + a_0\), 则称\(u\)是一个\textbf{代数数}.
\end{definition}

\begin{definition}[极小多项式(最小多项式)]\label{definition:极小多项式(最小多项式)}
对任一代数数\(u\), 存在唯一一个\(u\)适合的首一有理系数多项式\(g(x)\), 使得\(g(x)\)是\(u\)适合的所有非零有理系数多项式中次数最小者. 这样的\(g(x)\)称为\(u\)的\textbf{极小多项式}或\textbf{最小多项式}.
\end{definition}
\begin{proof}
现在证明这个定义是良定义的,只须证明对任一代数数所对应的极小多项式的存在性和唯一性.

先证存在性.在\(u\)适合的所有非零有理系数多项式构成的集合中 (由假设这个集合非空,否则$u$就不是一个代数数),由良序公理可知,存在一个次数最小的多项式,然后将其首一化, 即可得到\(u\)的极小多项式\(g(x)\). 

再证唯一性.为了证明极小多项式的唯一性, 我们先证明极小多项式的一个基本性质, 即极小多项式可以整除\(u\)适合的任一多项式\(f(x)\). 假设
\[
f(x)=g(x)q(x)+r(x), \mathrm{deg }r(x)<\mathrm{deg }g(x),
\]
则由\(f(u)=g(u)=0\)可知\(r(u)=0\). 若\(r(x)\neq 0\), 则\(u\)适合一个比\(g(x)\)的次数更小的多项式\(r(x)\), 这和\(g(x)\)是极小多项式矛盾. 因此\(r(x)=0\), 即\(g(x)\mid f(x)\). 设\(h(x)\)也是\(u\)的极小多项式, 则由上述性质可得\(g(x)\mid h(x), h(x)\mid g(x)\), 从而\(g(x)\)和\(h(x)\)只差一个非零常数, 又它们都是首一的, 故只能相等, 唯一性得证.
\end{proof}

\begin{proposition}[极小多项式的基本性质]\label{proposition:极小多项式的基本性质}
\begin{enumerate}[(1)]
\item 设$g(x)$为$u$的极小多项式,则$g(x)$一定整除\(u\)适合的任一多项式\(f(x)\).
\end{enumerate}
\end{proposition}
\begin{proof}
\begin{enumerate}[(1)]
\item 假设
\[
f(x)=g(x)q(x)+r(x), \mathrm{deg }r(x)<\mathrm{deg }g(x),
\]
则由\(f(u)=g(u)=0\)可知\(r(u)=0\). 若\(r(x)\neq 0\), 则\(u\)适合一个比\(g(x)\)的次数更小的多项式\(r(x)\), 这和\(g(x)\)是极小多项式矛盾. 因此\(r(x)=0\), 即\(g(x)\mid f(x)\). 
\end{enumerate}
\end{proof}

\begin{proposition}[极小多项式式的充要条件]\label{proposition:极小多项式式的充要条件}
设\(g(x)\)是一个\(u\)适合的首一有理系数多项式, 则\(g(x)\)是\(u\)的极小多项式的充要条件是\(g(x)\)是有理数域上的不可约多项式.
\end{proposition}
\begin{proof}
先证必要性. 若极小多项式\(g(x)\)在有理数域上可约, 则\(g(x)=g_1(x)g_2(x)\)可分解为两个比\(g(x)\)的次数更小的多项式的乘积. 由\(0 = g(u)=g_1(u)g_2(u)\)可知\(g_1(u)\)和\(g_2(u)\)中至少有一个等于零. 不妨设\(g_1(u)=0\), 则\(u\)适合一个比\(g(x)\)的次数更小的多项式\(g_1(x)\), 这和\(g(x)\)是极小多项式矛盾. 

再证充分性. 设\(g(x)\)是\(u\)适合的有理数域上的首一不可约多项式, \(h(x)\)是\(u\)的极小多项式. 由\hyperref[proposition:极小多项式的基本性质]{极小多项式的基本性质(1)}可知\(h(x)\mid g(x)\).因为$g(u)=h(u)=0$,所以$g(x)$和$h(x)$有公共根,从而$x-u$一定是$g(x),h(x)$的公因式,于是$g(x)$和$h(x)$不互素.又\(g(x)\)是不可约多项式,因此$g(x)\mid h(x)$.于是$g(x)\sim h(x)$,即\(g(x)\)和\(h(x)\)只差一个非零常数,而它们又都是首一的,故只能相等.因此\(g(x)\)就是\(u\)的极小多项式.
\end{proof}

\subsection{多项式的标准分解}

多项式的标准分解是证明某些问题的有力工具.

\begin{theorem}[因式分解定理]\label{theorem:因式分解定理}
设\(f(x)\)是数域\(\mathbb{K}\)上的多项式且\(\mathrm{deg }f(x)\geq 1\), 则

(1) \(f(x)\)可分解为有限个\(\mathbb{K}\)上的不可约多项式之积;

(2) 若
\begin{align}
f(x)=p_1(x)p_2(x)\cdots p_s(x)=q_1(x)q_2(x)\cdots q_t(x).\label{theorem5.12-5.4.1}
\end{align}
是\(f(x)\)的两个不可约分解, 即\(p_i(x),q_j(x)\)都是\(\mathbb{K}\)上的次数大于零的不可约多项式, 则\(s = t\), 且经过适当调换因式的次序以后, 有
\[
q_i(x)\sim p_i(x),\ i = 1,2,\cdots,s.
\]
\end{theorem}
\begin{note}
\begin{enumerate}
\item 这个定理表明, 任一多项式可唯一地分解为若干个不可约多项式之积. 这里唯一是在相伴意义下的唯一, 即相应的多项式可以差一个常数因子. 如果把分解式中相同或仅差一个常数的因式合并在一起, 就得到了一个 \textbf{标准分解式}:
\begin{align}
f(x)=cp_1(x)^{e_1}p_2(x)^{e_2}\cdots p_m(x)^{e_m}, \label{theorem5.12-5.4.2}
\end{align}
其中\(c\neq 0, p_i(x)\)是互异的首一不可约多项式, \(e_i\geq 1(i = 1,2,\cdots,m)\).

若\(e_i>1\ (e_i = 1)\), 我们称\eqref{theorem5.12-5.4.2}式中的因式\(p_i(x)\)为\(f(x)\)的\textbf{\(e_i\)重因式 (单因式)}. 显然这时\(p_i(x)^{e_i}\mid f(x)\), 但\(p_i(x)^{e_i + 1}\)不能整除\(f(x)\).

\item 设\(f(x),g(x)\)是\(\mathbb{K}\)上的两个多项式, 在它们的标准分解式中适当添加零次项,就能得到公共的标准分解. 故对\(\mathbb{K}\)上任意的两个多项式$f(x),g(X)$,都可以不妨设它们有如下的\textbf{公共的标准分解式}:
\[
f(x)=c_1p_1(x)^{e_1}p_2(x)^{e_2}\cdots p_n(x)^{e_n};
\]
\[
g(x)=c_2p_1(x)^{f_1}p_2(x)^{f_2}\cdots p_n(x)^{f_n},
\]
其中\(e_i\geq 0, f_i\geq 0(i = 1,2,\cdots,n)\).
\end{enumerate}
\end{note}
\begin{proof}
(1) 对多项式\(f(x)\)的次数用数学归纳法. 若\(\mathrm{deg }f(x)=1\), 结论显然成立. 设次数小于\(n\)的多项式都可以分解为\(\mathbb{K}\)上的不可约多项式之积而\(\mathrm{deg }f(x)=n\). 若\(f(x)\)不可约, 结论自然成立. 若\(f(x)\)可约, 则
\[
f(x)=f_1(x)f_2(x),
\]
其中\(f_1(x),f_2(x)\)的次数小于\(n\), 由归纳假设它们可以分解为有限个\(\mathbb{K}\)上的不可约多项式之积. 所有这些多项式之积就是\(f(x)\).

(2) 对\eqref{theorem5.12-5.4.1}式中的\(s\)用数学归纳法. 若\(s = 1\), 则\(f(x)=p_1(x)\), 因此\(f(x)\)是不可约多项式, 于是\(t = 1, q_1(x)=p_1(x)\). 现假设对不可约因式个数小于\(s\)的多项式结论正确. 由\eqref{theorem5.12-5.4.1}式, 有
\[
p_1(x)\mid q_1(x)q_2(x)\cdots q_t(x),
\]
由\hyperref[corollary:不可约多项式“素性”的推论]{推论\ref{corollary:不可约多项式“素性”的推论}}可知, 必存在某个\(i\), 不妨设\(i = 1\), 使
\[
p_1(x)\mid q_1(x).
\]
但是\(p_1(x),q_1(x)\)都是不可约多项式, 因此存在\(0\neq c_1\in\mathbb{K}\), 使
\[
q_1(x)=c_1p_1(x),
\]
此即\(p_1(x)\sim q_1(x)\). 将上式代入\eqref{theorem5.12-5.4.1}式并消去\(p_1(x)\), 得到
\[
p_2(x)\cdots p_s(x)=c_1q_2(x)\cdots q_t(x).
\]
这时左边为\(s - 1\)个不可约多项式之积, 由归纳假设, \(s - 1=t - 1\), 即\(s = t\). 另一方面, 存在\(0\neq c_i\in\mathbb{K}\), 使\(q_i(x)=c_ip_i(x)\).
\end{proof}

\begin{corollary}\label{corollary:标准分解式具有相同因子的多项式的最大公因式与最小公倍式}
设\(f(x),g(x)\)是\(\mathbb{K}\)上的两个多项式,不妨设它们有如下的公共的标准分解式:
\[
f(x)=c_1p_1(x)^{e_1}p_2(x)^{e_2}\cdots p_n(x)^{e_n};
\]
\[
g(x)=c_2p_1(x)^{f_1}p_2(x)^{f_2}\cdots p_n(x)^{f_n},
\]
其中\(e_i\geq 0, f_i\geq 0(i = 1,2,\cdots,n)\), 则\(f(x),g(x)\)的最大公因式
\[
(f(x),g(x))=p_1(x)^{k_1}p_2(x)^{k_2}\cdots p_n(x)^{k_n},
\]
其中\(k_i = \min\{e_i,f_i\}(i = 1,2,\cdots,n)\).

类似地, \(f(x),g(x)\)的最小公倍式
\[
[f(x),g(x)]=p_1(x)^{h_1}p_2(x)^{h_2}\cdots p_n(x)^{h_n},
\]
其中\(h_i = \max\{e_i,f_i\}(i = 1,2,\cdots,n)\).
\end{corollary}
\begin{proof}
利用最大公因式和最小公倍式的定义容易证明.
\end{proof}

\begin{proposition}[整除关系在平方下不变]\label{proposition:整除关系在平方下不变}
证明:\(g(x)^2 \mid f(x)^2\) 的充要条件是 \(g(x) \mid f(x)\).
\end{proposition}
\begin{proof}
充分性是显然的,只需证明必要性。设 \(f(x), g(x)\) 的公共标准分解为
\[
f(x) = cp_1(x)^{e_1}p_2(x)^{e_2} \cdots p_k(x)^{e_k}, \quad g(x) = dp_1(x)^{f_1}p_2(x)^{f_2} \cdots p_k(x)^{f_k},
\]
其中 \(p_i(x)\) 为互不相同的首一不可约多项式,\(c, d\) 是非零常数,则
\[
f(x)^2 = c^2 p_1(x)^{2e_1}p_2(x)^{2e_2} \cdots p_k(x)^{2e_k}, \quad g(x)^2 = d^2 p_1(x)^{2f_1}p_2(x)^{2f_2} \cdots p_k(x)^{2f_k}.
\]
若 \(g(x)^2 \mid f(x)^2\),则 \(2f_i \leq 2e_i\),从而 \(f_i \leq e_i (1 \leq i \leq k)\)。因此 \(g(x) \mid f(x)\).
\end{proof}


\end{document}
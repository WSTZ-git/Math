\documentclass[../../main.tex]{subfiles}
\graphicspath{{\subfix{./image/}}} % 指定图片目录,后续可以直接使用图片文件名
% 注意这里的文件路径不能用 ../../image/ ,否则用latexmk编译子文件会报错

% 例如:
% \begin{figure}[H]
% \centering
% \includegraphics[scale=0.4]{图.png}
% \caption{}
% \label{figure:图}
% \end{figure}
% 注意:上述\label{}一定要放在\caption{}之后,否则引用图片序号会只会显示??.

\begin{document}

\section{中国剩余定理}

\begin{theorem}[中国剩余定理]\label{theorem:中国剩余定理(多项式版)}
设$\mathbb{F}$是一个域,$m_1,m_2,\cdots,m_n\in\mathbb{F}[x]$且两两互素, 则对任何$a_1,a_2,\cdots,a_n\in\mathbb{F}[x]$, 存在$f\in\mathbb{F}[x]$使得
$$
f(x)\equiv a_i(x)\pmod{m_i(x)},i=1,2,\cdots,n,
$$
即存在$k_i\in\mathbb{F}[x]$使得
$$
f(x)=k_i(x)m_i(x)+a_i(x),i=1,2,\cdots,n.
$$
\end{theorem}
\begin{note}
本定理是抽象代数或者初等数论中的经典定理, 可以无脑直接使用, 和整数版本的证明也完全一致.
\end{note}
\begin{remark}
若
$$
f_1(x),f_2(x)\equiv a_i(x)\pmod{m_i(x)},i=1,2,\cdots,n,
$$
我们有
$$
f_1(x)-f_2(x)\equiv0\pmod{m_i(x)},i=1,2,\cdots,n,
$$
这等价于
$$
f_1-f_2\equiv0\pmod{\left(\prod_{i=1}^{n}m_i\right)}.
$$
即全部解为
$$
\left\{f_1(x)+k(x)\cdot\prod_{i=1}^{n}m_i(x):k\in\mathbb{F}[x]\right\}.
$$
其中$f_1(x)$是任意一个特定解.
\end{remark}
\begin{proof}
设
$$
\varphi_i(x)=m_1(x)m_2(x)\cdots m_{i-1}(x)m_{i+1}(x)\cdots m_n(x),i=1,2,\cdots,n.
$$
对每一个$i\in\{1,2,\cdots,n\}$, 我们有$\varphi_i$和$m_i$互素, 因此由裴蜀等式知存在$u_i,v_i\in\mathbb{F}[x]$使得
$$
\varphi_i(x)u_i(x)+m_i(x)v_i(x)=1.
$$
考虑
$$
f(x)=\sum_{j=1}^{n}a_j(x)\varphi_j(x)u_j(x)\in\mathbb{F}[x],
$$
我们有
$$
\begin{aligned}
f(x)-a_i(x)&=a_i(x)\left[\varphi_i(x)u_i(x)-1\right]+\sum_{\substack{1\leqslant j\leqslant n,\\j\neq i}}a_j(x)\varphi_j(x)u_j(x)\\
&=-a_i(x)v_i(x)m_i(x)+\sum_{\substack{1\leqslant j\leqslant n,\\j\neq i}}a_j(x)\varphi_j(x)u_j(x).
\end{aligned}
$$
现在$m_i$是上式右边每一项的因子, 从而
$$
m_i|(f-a_i),i=1,2,\cdots,n,
$$
即
$$
f(x)\equiv a_i(x)\pmod{m_i(x)},i=1,2,\cdots,n.
$$

\end{proof}

\begin{example}
求次数最小的 \( f \) 使得
\[
\begin{cases}
f(x) \equiv 6 \pmod{x + 1} \\
f(x) \equiv 3x \pmod{x^2 + x + 1} \\
f(x) \equiv (x - 1)^2 \pmod{2x^3 + 1}
\end{cases}
\]
\end{example}
\begin{remark}
直接用\hyperref[theorem:中国剩余定理(多项式版)]{中国剩余定理}是难算的.我们应该具体问题具体分析.
\end{remark}
\begin{remark}
由\hyperref[theorem:中国剩余定理(多项式版)]{中国剩余定理}的注可知,原方程的全部解为
\begin{align*}
f_1\left( x \right) +k\left( x \right) \left( x+1 \right) \left( x^2+x+1 \right) \left( 2x^3+1 \right) ,
\end{align*}
其中$f_1(x)$为任一特解,$k(x)\in \mathbb{C}[x]$.因为$\deg\left[(x + 1)(x^2 + x + 1)(2x^3 + 1)\right] = 6,$所以当$\deg f_1(x)\geqslant  6$时,可以取合适的$k(x)$,使得$f_1\left( x \right) +k\left( x \right) \left( x+1 \right) \left( x^2+x+1 \right) \left( 2x^3+1 \right)$中高于6次项全部消去.故原方程组的最小次解必定小于等于5次.
\end{remark}
\begin{proof}
因为
\[
\deg\left[(x + 1)(x^2 + x + 1)(2x^3 + 1)\right] = 6,
\]
我们设
\[
f(x) = a_0 + a_1x + a_2x^2 + a_3x^3 + a_4x^4 + a_5x^5.
\]
不妨在 \( \mathbb{Q} \) 上考虑,因为如果还有一个 \( g \in \mathbb{C}[x] \) 使得 \( \deg g \leqslant  5 \) 为解,则 \( g - f \) 有一个六次因子,即只能有 \( f = g \).

条件第一个同余式等价于
\[
f(-1) = 6 \Leftrightarrow a_0 - a_1 + a_2 - a_3 + a_4 - a_5 = 6.
\]
对第二个同余式,等价于若 \( x^2 + x + 1 = 0 \) 则 \( f(x) - 3x = 0 \).注意到若 \( x^2 + x + 1 = 0 \) 则
\[
x^3 - 1 = (x - 1)(x^2 + x + 1) = 0 \Rightarrow x^3 = 1,
\]
我们有
\begin{align*}
f(x)-3x&=a_0+a_1x+a_2x^2+a_3x^3+a_4x^4+a_5x^5-3x
\\
&=a_0+a_3+(a_1+a_4-3)x+(a_2+a_5)x^2
\\
&=a_0+a_3+(a_1+a_4-3)x+(a_2+a_5)\left( -x-1 \right) 
\\
&=a_0+a_3-a_2-a_5+(a_1+a_4-a_2-a_5-3)x,
\end{align*}
即容易验证等价于
\[
a_0 + a_3 - a_2 - a_5 = 0,\ a_1 + a_4 - a_2 - a_5 = 3.
\]
对第三个同余式,等价于若 \( 2x^3 + 1 = 0 \) 则 \( f(x) - x^2 + 2x - 1 = 0 \).现在若 \( 2x^3 + 1 = 0 \) 则
\[
f(x) - x^2 + 2x - 1 = \left(a_2 - \frac{1}{2}a_5 - 1\right)x^2 + \left(a_1 - \frac{1}{2}a_4 + 2\right)x + \left(a_0 - \frac{1}{2}a_3 - 1\right),
\]
即容易验证等价于
\[
a_2 - \frac{1}{2}a_5 = 1,\ a_1 - \frac{1}{2}a_4 = -2,\ a_0 - \frac{1}{2}a_3 = 1.
\]
解线性方程组得
\[
f(x) = 4x^4 + x^2 + 1,
\]
这就完成了计算.

\end{proof}























\end{document}
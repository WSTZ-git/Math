\documentclass[../../main.tex]{subfiles}
\graphicspath{{\subfix{../../image/}}} % 指定图片目录,后续可以直接使用图片文件名。

% 例如:
% \begin{figure}[H]
% \centering
% \includegraphics{image-01.01}
% \caption{图片标题}
% \label{figure:image-01.01}
% \end{figure}
% 注意:上述\label{}一定要放在\caption{}之后,否则引用图片序号会只会显示??.

\begin{document}

\section{幂等变换}

\begin{definition}[幂等变换]\label{definition:幂等变换}
线性变换\(\varphi\)若满足\(\varphi^2 = \varphi\),则称为\textbf{幂等变换}. 
\end{definition}

\begin{definition}[投影变换]\label{definition:投影变换}
设\(V = V_1\oplus V_2\oplus\cdots\oplus V_m\)为线性空间\(V\)关于子空间\(V_i(1\leq i\leq m)\)的直和分解,则\(V\)中任一向量\(\boldsymbol{v}\)可唯一地分解为\(\boldsymbol{v}=\boldsymbol{v}_1+\boldsymbol{v}_2+\cdots+\boldsymbol{v}_m\),其中\(\boldsymbol{v}_i\in V_i\). 定义\(\varphi_i:V\to V\),\(\varphi_i(\boldsymbol{v})=\boldsymbol{v}_i(1\leq i\leq m)\),容易验证\(\varphi_i\)是\(V\)上的线性变换,称为\(V\)到\(V_i\)上的\textbf{投影变换}. 
\end{definition}

\begin{proposition}[投影变换的性质]\label{proposition:投影变换的性质}
设\(V = V_1\oplus V_2\oplus\cdots\oplus V_m\)为线性空间\(V\)关于子空间\(V_i(1\leq i\leq m)\)的直和分解,\(\varphi_i\)为\(V\)到\(V_i\)上的投影变换.投影变换\(\varphi_i\)满足如下性质:
\begin{enumerate}[(1)]
\item \(\varphi_i^2 = \varphi_i\),\(\varphi_i\varphi_j = \boldsymbol{0}(i\neq j)\),\(\boldsymbol{I}_V = \varphi_1+\varphi_2+\cdots+\varphi_m\);

\item \(\text{Im}\varphi_i = V_i\),\(\text{Ker}\varphi_i = \bigoplus_{j\neq i}V_j\),\(V = \text{Im}\varphi_i\oplus\text{Ker}\varphi_i\).

\item 投影变换\(\varphi_i\)都是幂等变换;

\item 若取\(V_i\)的一组基拼成\(V\)的一组基,则\(\varphi_i\)在这组基下的表示矩阵为\(\text{diag}\{0,\cdots,0,1,\cdots,1,0,\cdots,0\}\),其中有\(\dim V_i\)个\(1\);

\item \(V = \text{Im}\varphi_1\oplus\text{Im}\varphi_2\oplus\cdots\oplus\text{Im}\varphi_m\);

\item \(\text{Ker}\varphi_1\cap\text{Ker}\varphi_2\cap\cdots\cap\text{Ker}\varphi_m = 0\).
\end{enumerate}
\end{proposition}
\begin{note}
提示:两个集合相等等价于这两个集合相互包含.
\end{note}
\begin{proof}
\begin{enumerate}[(1)]
\item 证明是显然的.

\item 证明是显然的.

\item 由(1)易得.

\item 由(2)易得.

\item 由\((2)\)易知\(V = \text{Im}\varphi_1\oplus\text{Im}\varphi_2\oplus\cdots\oplus\text{Im}\varphi_m\).

\item 任取\(\alpha \in \mathrm{Ker}\varphi_1\cap \mathrm{Ker}\varphi_2\cap \cdots \cap \mathrm{Ker}\varphi_m\).由投影变换性质\((2)\),可知\(\mathrm{Ker}\varphi_i = \bigoplus_{j\neq i}V_j\).于是对任意整数\(i, j\in [1, m]\)且\(i\neq j\),有\(\alpha \in \mathrm{Ker}\varphi_i\cap \mathrm{Ker}\varphi_j = \bigoplus_{k\neq i}V_k\cap \bigoplus_{k\neq j}V_k\).从而
\[
\alpha = v_1 + \cdots + v_{i - 1} + v_{i + 1} + \cdots + v_m = u_1 + \cdots + u_{j - 1} + u_{j + 1} + \cdots + u_m,
\]
其中\(v_k, u_k\in V_k\).上式经整理可得\(v_j - u_i = \sum_{k\neq i, j}(u_k - v_k)\in \bigoplus_{k\neq i, j}V_k\).又\(v_j - u_i\in V_i\oplus V_j\).故\(v_j - u_i\in (V_i\oplus V_j)\cap \bigoplus_{k\neq i, j}V_k\).而由于\(V = \bigoplus_{k = 1}^m V_k\),因此\((V_i\oplus V_j)\cap \bigoplus_{k\neq i, j}V_k = \mathbf{0}\).故\(v_j - u_i = \mathbf{0}\),从而\(v_j = u_i\in V_i\cap V_j\).又因为\(V_i\oplus V_j\),所以\(V_i\cap V_j = \mathbf{0}\).故\(v_j = u_i = \mathbf{0}\).再由\(i, j\)的任意性可知,\(v_i = \mathbf{0}, i = 1, 2, \cdots, m\).因此\(\alpha = \sum_{k\neq i}v_k = \mathbf{0}\).故\(\mathrm{Ker}\varphi_1\cap \mathrm{Ker}\varphi_2\cap \cdots \cap \mathrm{Ker}\varphi_m = \mathbf{0}\).
\end{enumerate}
\end{proof}

\begin{proposition}\label{proposition:幂等变换就是投影变换}
设\(\varphi\)是\(n\)维线性空间\(V\)上的幂等变换,证明:\(V = U\oplus W\),其中\(U = \text{Im}\varphi=\text{Ker}(\boldsymbol{I}_V - \varphi)\),\(W = \text{Im}(\boldsymbol{I}_V - \varphi)=\text{Ker}\varphi\),且\(\varphi\)就是\(V\)到\(U\)上的投影变换.
\end{proposition}
\begin{note}
由上述命题可知\textbf{\(n\)维线性空间\(V\)上的幂等变换\(\varphi\)也是\(V\)到\(\text{Im}\varphi\)上的投影变换}.于是由\hyperref[proposition:幂等变换就是投影变换]{命题\ref{proposition:幂等变换就是投影变换}}和\hyperref[proposition:投影变换的性质]{命题\ref{proposition:投影变换的性质}}可知,投影变换可以看作幂等变换,幂等变换也可以看作投影变换.(即\hypertarget{幂等变换和投影变换等价}{幂等变换和投影变换等价})
\end{note}
\begin{proof}
因为\(\varphi^2 = \varphi\),故\(\text{Im}\varphi\subseteq\text{Ker}(\boldsymbol{I} - \varphi)\),\(\text{Im}(\boldsymbol{I} - \varphi)\subseteq\text{Ker}\varphi\). 对任意的\(\boldsymbol{\alpha}\in V\),\(\varphi(\boldsymbol{\alpha})\in\text{Ker}(\boldsymbol{I} - \varphi)\),\((\boldsymbol{I} - \varphi)(\boldsymbol{\alpha})\in\text{Ker}\varphi\),于是\(\boldsymbol{\alpha}=(\boldsymbol{I} - \varphi)(\boldsymbol{\alpha})+\varphi(\boldsymbol{\alpha})\in\text{Ker}\varphi+\text{Ker}(\boldsymbol{I} - \varphi)\),从而\(V = \text{Ker}\varphi+\text{Ker}(\boldsymbol{I} - \varphi)\). 任取\(\boldsymbol{\beta}\in\text{Ker}\varphi\cap\text{Ker}(\boldsymbol{I} - \varphi)\),则\(\boldsymbol{\beta}=(\boldsymbol{I}-\varphi)(\boldsymbol{\beta})+\varphi(\boldsymbol{\beta}) = \boldsymbol{0}\),即\(\text{Ker}\varphi\cap\text{Ker}(\boldsymbol{I}-\varphi)= 0\). 因此,\(V = \text{Ker}\varphi\oplus\text{Ker}(\boldsymbol{I}-\varphi)\). 特别地,由维数公式可得\(\dim\text{Im}\varphi=\dim\text{Ker}(\boldsymbol{I} - \varphi)\),\(\dim\text{Im}(\boldsymbol{I} - \varphi)=\dim\text{Ker}\varphi\),从而\(\text{Im}\varphi=\text{Ker}(\boldsymbol{I} - \varphi)\),\(\text{Im}(\boldsymbol{I} - \varphi)=\text{Ker}\varphi\).

令\(U = \text{Im}\varphi=\text{Ker}(\boldsymbol{I} - \varphi)\),\(W = \text{Im}(\boldsymbol{I} - \varphi)=\text{Ker}\varphi\),则\(V = U\oplus W\). 注意到对任意的\(\boldsymbol{\alpha}\in V\),\(\boldsymbol{\alpha}=\varphi(\boldsymbol{\alpha})+(\boldsymbol{I} - \varphi)(\boldsymbol{\alpha})\),其中\(\varphi(\boldsymbol{\alpha})\in U\),\((\boldsymbol{I} - \varphi)(\boldsymbol{\alpha})\in W\),故\(\varphi\)就是\(V\)到\(U\)上的投影变换.
\end{proof}

\begin{corollary}\label{corollary:幂等变换在某组基下的表示矩阵是标准型}
对线性空间\(V\)上的幂等变换\(\varphi\),总存在\(V\)的一组基(它由\(\mathrm{Im}\varphi\)的基和\(\mathrm{Ker}\varphi \)的基拼成),使得\(\varphi\)在这组基下的表示矩阵为下列对角矩阵:
\[
\begin{pmatrix}
\boldsymbol{I}_r&\boldsymbol{O}\\
\boldsymbol{O}&\boldsymbol{O}
\end{pmatrix},
\]
其中\(\boldsymbol{I}_r\)为\(r\)阶单位矩阵,\(r\)等于\(\dim \mathrm{Im}\varphi\),即\(\varphi\)的像空间的维数.
\end{corollary}
\begin{proof}
由\hyperref[proposition:幂等变换就是投影变换]{这个命题\ref{proposition:幂等变换就是投影变换}}和\hyperref[proposition:投影变换的性质]{投影变换的性质}容易证明.
\end{proof}

\begin{proposition}\label{proposition:幂等矩阵的性质1}
设\(\boldsymbol{A}\)是数域\(\mathbb{F}\)上的\(n\)阶幂等矩阵,求证:

(1) 存在\(n\)阶非异阵\(\boldsymbol{P}\),使得\(\boldsymbol{P}^{-1}\boldsymbol{A}\boldsymbol{P}=\begin{pmatrix}\boldsymbol{I}_r&\boldsymbol{O}\\ \boldsymbol{O}&\boldsymbol{O}\end{pmatrix}\),其中\(r = \text{r}(\boldsymbol{A})\);

(2) \(\text{r}(\boldsymbol{A})=\text{tr}(\boldsymbol{A})\).
\end{proposition}
\begin{proof}
将\(\boldsymbol{A}\)看成是\(n\)维列向量空间\(\mathbb{F}^n\)上(由矩阵$\boldsymbol{A}$乘法诱导)的线性变换,则它是幂等变换,因此由\hyperref[corollary:幂等变换在某组基下的表示矩阵是标准型]{推论\ref{corollary:幂等变换在某组基下的表示矩阵是标准型}}即得(1). 

注意到\(\text{tr}(\boldsymbol{A})=\text{tr}(\boldsymbol{P}^{-1}\boldsymbol{A}\boldsymbol{P})=\text{tr}\begin{pmatrix}\boldsymbol{I}_r&\boldsymbol{O}\\ \boldsymbol{O}&\boldsymbol{O}\end{pmatrix}=r = \text{r}(\boldsymbol{A})\),故(2)也成立.
\end{proof}

\begin{example}
设\(\boldsymbol{A},\boldsymbol{B}\)是数域\(\mathbb{F}\)上的\(n\)阶幂等矩阵,且\(\boldsymbol{A}\)和\(\boldsymbol{B}\)的秩相同,求证:必存在\(\mathbb{F}\)上的\(n\)阶可逆矩阵\(\boldsymbol{C}\),使得\(\boldsymbol{C}\boldsymbol{B}=\boldsymbol{A}\boldsymbol{C}\).
\end{example}
\begin{proof}
由\hyperref[proposition:幂等矩阵的性质1]{命题\ref{proposition:幂等矩阵的性质1}}可知,\(\boldsymbol{A}\)和\(\boldsymbol{B}\)均相似于矩阵\(\begin{pmatrix}\boldsymbol{I}_r&\boldsymbol{O}\\ \boldsymbol{O}&\boldsymbol{O}\end{pmatrix}\),于是\(\boldsymbol{A}\)和\(\boldsymbol{B}\)相似,即存在可逆矩阵\(\boldsymbol{C}\),使得\(\boldsymbol{B}=\boldsymbol{C}^{-1}\boldsymbol{A}\boldsymbol{C}\),即\(\boldsymbol{C}\boldsymbol{B}=\boldsymbol{A}\boldsymbol{C}\).
\end{proof}

\begin{proposition}\label{proposition:幂等变换值域与核空间相等的充要条件1}
设\(\varphi,\psi\)是\(n\)维线性空间\(V\)上的幂等线性变换,求证:

(1) \(\text{Im}\varphi=\text{Im}\psi\)的充要条件是\(\varphi\psi=\psi\),\(\psi\varphi=\varphi\);

(2) \(\text{Ker}\varphi=\text{Ker}\psi\)的充要条件是\(\varphi\psi=\varphi\),\(\psi\varphi=\psi\).
\end{proposition}
\begin{note}
也可由幂等变换等价于投影变换来给出直观的几何证明.
\end{note}
\begin{proof}
(1) 由\(\psi=\varphi\psi\)可得\(\text{Im}\psi\subseteq\text{Im}\varphi\).同理由\(\varphi=\psi\varphi\)可得\(\text{Im}\varphi\subseteq\text{Im}\psi\).因此\(\text{Im}\varphi=\text{Im}\psi\).

反之,若\(\text{Im}\varphi=\text{Im}\psi\),则对任意的\(\boldsymbol{\alpha}\in V\),\(\psi(\boldsymbol{\alpha})\in\text{Im}\psi=\text{Im}\varphi\),故存在\(\boldsymbol{\beta}\in V\),使得\(\psi(\boldsymbol{\alpha})=\varphi(\boldsymbol{\beta})\).注意到\(\varphi^2=\varphi\),故\(\varphi\psi(\boldsymbol{\alpha})=\varphi^2(\boldsymbol{\beta})=\varphi(\boldsymbol{\beta})=\psi(\boldsymbol{\alpha})\),于是\(\varphi\psi=\psi\).同理可证\(\psi\varphi=\varphi\).

(2) 设\(\varphi\psi=\varphi\),\(\psi\varphi=\psi\).对任意的\(\boldsymbol{\alpha}\in\text{Ker}\varphi\),即\(\varphi(\boldsymbol{\alpha}) = \boldsymbol{0}\),有\(\psi(\boldsymbol{\alpha})=\psi\varphi(\boldsymbol{\alpha}) = \boldsymbol{0}\),即\(\boldsymbol{\alpha}\in\text{Ker}\psi\),于是\(\text{Ker}\varphi\subseteq\text{Ker}\psi\).同理可证\(\text{Ker}\psi\subseteq\text{Ker}\varphi\),因此\(\text{Ker}\varphi=\text{Ker}\psi\).

反之,设\(\text{Ker}\varphi=\text{Ker}\psi\).对任意的\(\boldsymbol{\alpha}\in V\),有\(\psi(\boldsymbol{\alpha}-\psi(\boldsymbol{\alpha}))=\psi(\boldsymbol{\alpha})-\psi^2(\boldsymbol{\alpha}) = \boldsymbol{0}\),因此\(\boldsymbol{\alpha}-\psi(\boldsymbol{\alpha})\in\text{Ker}\psi=\text{Ker}\varphi\),从而\(\varphi(\boldsymbol{\alpha}-\psi(\boldsymbol{\alpha})) = \boldsymbol{0}\),即\(\varphi(\boldsymbol{\alpha})=\varphi\psi(\boldsymbol{\alpha})\),于是\(\varphi=\varphi\psi\).同理可证\(\psi\varphi=\psi\).
\end{proof}

\begin{proposition}\label{proposition:幂等变换的和与差仍是幂等变换的充要条件1}
设\(\varphi,\psi\)是\(n\)维线性空间\(V\)上的幂等线性变换,求证:

(1) \(\varphi+\psi\)是幂等变换的充要条件是\(\varphi\psi=\psi\varphi = 0\);

(2) \(\varphi-\psi\)是幂等变换的充要条件是\(\varphi\psi=\psi\varphi=\psi\).
\end{proposition}
\begin{note}
也可由幂等变换等价于投影变换来给出直观的几何证明.
\end{note}
\begin{proof}
充分性容易验证,下面证明必要性.

(1) 若\((\varphi+\psi)^2=\varphi+\psi\),则\(\varphi\psi+\psi\varphi = 0\),即\(\varphi\psi=-\psi\varphi\).将上式两边分别左乘及右乘\(\varphi\),可得\(\varphi\psi\varphi=-\varphi\psi=-\psi\varphi\).因此\(\varphi\psi=\psi\varphi = 0\).

(2) 若\((\varphi - \psi)^2=\varphi - \psi\),则\(\varphi\psi+\psi\varphi = 2\psi\).将上式两边分别左乘及右乘\(\varphi\),可得\(\varphi\psi\varphi=\varphi\psi=\psi\varphi\).因此\(\varphi\psi=\psi\varphi=\psi\).
\end{proof}

\begin{proposition}\label{proposition:由投影变换性质反推直和分解}
设\(\varphi_1,\cdots,\varphi_m\)是\(n\)维线性空间\(V\)上的线性变换,且适合条件:
\[
\varphi_i^2 = \varphi_i,\ \varphi_i\varphi_j = 0\ (i\neq j),\ \text{Ker}\varphi_1\cap\cdots\cap\text{Ker}\varphi_m = 0.
\]
求证:\(V = \text{Im}\varphi_1\oplus\text{Im}\varphi_2\oplus\cdots\oplus\text{Im}\varphi_m\).
\end{proposition}
\begin{proof}
任取\(\boldsymbol{\alpha}\in\text{Im}\varphi_i\cap(\sum_{j\neq i}\text{Im}\varphi_j)\),设\(\boldsymbol{\alpha}=\varphi_i(\boldsymbol{\beta})\),其中\(\boldsymbol{\beta}\in V\),则\(\varphi_i(\boldsymbol{\alpha})=\varphi_i^2(\boldsymbol{\beta})=\varphi_i(\boldsymbol{\beta})=\boldsymbol{\alpha}\).又可设
\[
\boldsymbol{\alpha}=\varphi_1(\boldsymbol{\alpha}_1)+\cdots+\varphi_{i - 1}(\boldsymbol{\alpha}_{i - 1})+\varphi_{i + 1}(\boldsymbol{\alpha}_{i + 1})+\cdots+\varphi_m(\boldsymbol{\alpha}_m),
\]
于是
\[
\boldsymbol{\alpha}=\varphi_i(\boldsymbol{\alpha})=\varphi_i(\varphi_1(\boldsymbol{\alpha}_1)+\cdots+\varphi_{i - 1}(\boldsymbol{\alpha}_{i - 1})+\varphi_{i + 1}(\boldsymbol{\alpha}_{i + 1})+\cdots+\varphi_m(\boldsymbol{\alpha}_m)) = 0.
\]
因此\(\text{Im}\varphi_i\cap(\sum_{j\neq i}\text{Im}\varphi_j)=0\).

对\(V\)中任一向量\(\boldsymbol{\alpha}\)以及任意的\(i\),有
\[
\varphi_i(\boldsymbol{\alpha}-(\varphi_1(\boldsymbol{\alpha})+\cdots+\varphi_m(\boldsymbol{\alpha})))=\varphi_i(\boldsymbol{\alpha})-\varphi_i^2(\boldsymbol{\alpha}) = 0,
\]
因此
\[
\boldsymbol{\alpha}-(\varphi_1(\boldsymbol{\alpha})+\cdots+\varphi_m(\boldsymbol{\alpha}))\in\text{Ker}\varphi_1\cap\cdots\cap\text{Ker}\varphi_m = 0,
\]
从而\(\boldsymbol{\alpha}-(\varphi_1(\boldsymbol{\alpha})+\cdots+\varphi_m(\boldsymbol{\alpha})) = 0\),即\(\boldsymbol{\alpha}=\varphi_1(\boldsymbol{\alpha})+\cdots+\varphi_m(\boldsymbol{\alpha})\),于是\(V=\text{Im}\varphi_1+\cdots+\text{Im}\varphi_m\).这就证明了\(V\)是\(\text{Im}\varphi_1,\cdots,\text{Im}\varphi_m\)的直和.
\end{proof}

\begin{proposition}\label{proposition:投影变换的性质x}
设\(\varphi,\varphi_1,\cdots,\varphi_m\)是\(n\)维线性空间\(V\)上的线性变换,满足:\(\varphi^2 = \varphi\)且\(\varphi=\varphi_1+\varphi_2+\cdots+\varphi_m\).求证:\(\text{r}(\varphi)=\text{r}(\varphi_1)+\text{r}(\varphi_2)+\cdots+\text{r}(\varphi_m)\)成立的充要条件是\(\varphi_i^2 = \varphi_i\),\(\varphi_i\varphi_j = 0\ (i\neq j)\).
\end{proposition}
\begin{note}
$\mathrm{r(}\varphi )=\mathrm{r(}\varphi _1)+\mathrm{r(}\varphi _2)+\cdots +\mathrm{r(}\varphi _m)$等价于$\mathrm{dim}\,\mathrm{Im}\varphi =\mathrm{dim}\,\mathrm{Im}\varphi _1+\mathrm{dim}\,\mathrm{Im}\varphi _2+\cdots +\mathrm{dim}\,\mathrm{Im}\varphi _m$.
\end{note}
\begin{proof}
{\color{blue}证法一(几何方法):} 令\(V_0 = \text{Im}\varphi\),\(V_i = \text{Im}\varphi_i\),则由\(\varphi=\varphi_1+\varphi_2+\cdots+\varphi_m\)可得\(V_0\subseteq V_1 + V_2+\cdots+V_m\).

先证充分性. 由\(\varphi_i^2 = \varphi_i\),\(\varphi_i\varphi_j = 0\ (i\neq j)\)可得\(\varphi_i=(\varphi_1+\varphi_2+\cdots+\varphi_m)\varphi_i=\varphi\varphi_i\),故\(V_i\subseteq V_0\),从而\(V_0 = V_1 + V_2+\cdots+V_m\).要证上述和为直和,只要证明零向量表示唯一即可.设
\[
\boldsymbol{0}=\boldsymbol{\alpha}_1+\boldsymbol{\alpha}_2+\cdots+\boldsymbol{\alpha}_m,\boldsymbol{\alpha}_i=\varphi_i(\boldsymbol{v}_i)\in V_i(1\leq i\leq m),
\]
则\(\boldsymbol{0}=\varphi_i(\varphi_1(\boldsymbol{v}_1))+\varphi_i(\varphi_2(\boldsymbol{v}_2))+\cdots+\varphi_i(\varphi_m(\boldsymbol{v}_m))=\varphi_i^2(\boldsymbol{v}_i)=\varphi_i(\boldsymbol{v}_i)=\boldsymbol{\alpha}_i\).因此\(V_0 = V_1\oplus V_2\oplus\cdots\oplus V_m\).两边同取维数即得\(\text{r}(\varphi)=\text{r}(\varphi_1)+\text{r}(\varphi_2)+\cdots+\text{r}(\varphi_m)\).

再证必要性.由于\(V_0\subseteq V_1 + V_2+\cdots+V_m\),于是
\[
\dim V_0\leq\dim(V_1 + V_2+\cdots+V_m)\leq\dim V_1+\dim V_2+\cdots+\dim V_m,
\]
故由\(\text{r}(\varphi)=\text{r}(\varphi_1)+\text{r}(\varphi_2)+\cdots+\text{r}(\varphi_m)\)可得\(\dim V_0=\dim V_1+\dim V_2+\cdots+\dim V_m\),从而上式中的不等号只能取等号.由\hyperref[proposition:与全空间维数相同的子空间等于全空间]{命题\ref{proposition:与全空间维数相同的子空间等于全空间}}及直和的充要条件可知,\(V_1 + V_2+\cdots+V_m\)是直和,并且
\[
V_0 = V_1\oplus V_2\oplus\cdots\oplus V_m.
\]
因为\(\text{Im}\varphi_i = V_i\subseteq V_0 = \text{Im}\varphi\),故对\(V\)中任一向量\(\boldsymbol{\alpha}\),存在\(\boldsymbol{\beta}\in V\),使得\(\varphi_i(\boldsymbol{\alpha})=\varphi(\boldsymbol{\beta})\),从而
\begin{align*}
\varphi_i(\boldsymbol{\alpha})&=\varphi(\boldsymbol{\beta})=\varphi^2(\boldsymbol{\beta})=(\varphi_1+\varphi_2+\cdots+\varphi_m)\varphi(\boldsymbol{\beta})\\
&=(\varphi_1+\varphi_2+\cdots+\varphi_m)\varphi_i(\boldsymbol{\alpha})\\
&=\varphi_1\varphi_i(\boldsymbol{\alpha})+\varphi_2\varphi_i(\boldsymbol{\alpha})+\cdots+\varphi_m\varphi_i(\boldsymbol{\alpha}).
\end{align*}
由直和表示的唯一性可知
\[
\varphi_i^2(\boldsymbol{\alpha})=\varphi_i(\boldsymbol{\alpha}),\varphi_j\varphi_i(\boldsymbol{\alpha}) = 0\ (j\neq i),
\]
于是\(\varphi_i^2 = \varphi_i\),\(\varphi_i\varphi_j = 0\ (i\neq j)\).

{\color{blue}证法二(代数方法):}把问题转换成代数的语言:设\(\boldsymbol{A},\boldsymbol{A}_1,\boldsymbol{A}_2,\cdots,\boldsymbol{A}_m\)是\(n\)阶矩阵,满足\(\boldsymbol{A}^2 = \boldsymbol{A}\)且\(\boldsymbol{A}=\boldsymbol{A}_1+\boldsymbol{A}_2+\cdots+\boldsymbol{A}_m\),求证:\(\text{r}(\boldsymbol{A})=\text{r}(\boldsymbol{A}_1)+\text{r}(\boldsymbol{A}_2)+\cdots+\text{r}(\boldsymbol{A}_m)\)成立的充要条件是\(\boldsymbol{A}_i^2 = \boldsymbol{A}_i\),\(\boldsymbol{A}_i\boldsymbol{A}_j = \boldsymbol{O}\ (i\neq j)\).

先证充分性. 若\(\boldsymbol{A}_i^2 = \boldsymbol{A}_i\),则由\hyperref[proposition:幂等矩阵的性质1]{命题\ref{proposition:幂等矩阵的性质1}}可知\(\text{r}(\boldsymbol{A}_i)=\text{tr}(\boldsymbol{A}_i)\),从而
\begin{align*}
\text{r}(\boldsymbol{A})&=\text{tr}(\boldsymbol{A})=\text{tr}(\boldsymbol{A}_1+\boldsymbol{A}_2+\cdots+\boldsymbol{A}_m)\\
&=\text{tr}(\boldsymbol{A}_1)+\text{tr}(\boldsymbol{A}_2)+\cdots+\text{tr}(\boldsymbol{A}_m)=\text{r}(\boldsymbol{A}_1)+\text{r}(\boldsymbol{A}_2)+\cdots+\text{r}(\boldsymbol{A}_m).
\end{align*}

再证必要性. 因为\(\boldsymbol{A}\)是幂等矩阵,故由\hyperref[proposition:幂等矩阵关于秩的判定准则]{命题\ref{proposition:幂等矩阵关于秩的判定准则}}可得\(n=\text{r}(\boldsymbol{I}_n - \boldsymbol{A})+\text{r}(\boldsymbol{A})\),从而\(n=\text{r}(\boldsymbol{I}_n - \boldsymbol{A})+\text{r}(\boldsymbol{A}_1)+\text{r}(\boldsymbol{A}_2)+\cdots+\text{r}(\boldsymbol{A}_m)\).构造如下分块对角矩阵并对其实施分块初等变换,可得
\[
\begin{pmatrix}
\boldsymbol{I}_n - \boldsymbol{A}&&&&\\
&\boldsymbol{A}_1&&&\\
&&\boldsymbol{A}_2&&\\
&&&\ddots&\\
&&&&\boldsymbol{A}_m
\end{pmatrix}\to
\left( \begin{matrix}
\boldsymbol{I}_n&		&		&		&		\\
\boldsymbol{A}_1&		\boldsymbol{A}_1&		&		&		\\
\boldsymbol{A}_2&		&		\boldsymbol{A}_2&		&		\\
\vdots&		&		&		\ddots&		\\
\boldsymbol{A}_m&		&		&		&		\boldsymbol{A}_m\\
\end{matrix} \right) \to
\]
\[
\begin{pmatrix}
\boldsymbol{I}_n&\boldsymbol{A}_1&\boldsymbol{A}_2&\cdots&\boldsymbol{A}_m\\
\boldsymbol{A}_1&\boldsymbol{A}_1&&&\\
\boldsymbol{A}_2&&\boldsymbol{A}_2&&\\
\vdots&&&\ddots&\\
\boldsymbol{A}_m&&&&\boldsymbol{A}_m
\end{pmatrix}\to
\begin{pmatrix}
\boldsymbol{I}_n&\boldsymbol{O}&\boldsymbol{O}&\cdots&\boldsymbol{O}\\
\boldsymbol{O}&\boldsymbol{A}_1 - \boldsymbol{A}_1^2&-\boldsymbol{A}_1\boldsymbol{A}_2&\cdots&-\boldsymbol{A}_1\boldsymbol{A}_m\\
\boldsymbol{O}&-\boldsymbol{A}_2\boldsymbol{A}_1&\boldsymbol{A}_2 - \boldsymbol{A}_2^2&\cdots&-\boldsymbol{A}_2\boldsymbol{A}_m\\
\vdots&\vdots&\vdots&&\vdots\\
\boldsymbol{O}&-\boldsymbol{A}_m\boldsymbol{A}_1&-\boldsymbol{A}_m\boldsymbol{A}_2&\cdots&\boldsymbol{A}_m - \boldsymbol{A}_m^2
\end{pmatrix}.
\]
由\(n=\text{r}(\boldsymbol{I}_n - \boldsymbol{A})+\text{r}(\boldsymbol{A}_1)+\text{r}(\boldsymbol{A}_2)+\cdots+\text{r}(\boldsymbol{A}_m)\)可得最后一个矩阵的右下角部分必为零矩阵,从而\(\boldsymbol{A}_i^2 = \boldsymbol{A}_i\),\(\boldsymbol{A}_i\boldsymbol{A}_j = \boldsymbol{O}\ (i\neq j)\).
\end{proof}

\begin{corollary}\label{corollary:恒等变换的幂等分解}
若取\(\boldsymbol{I}_V\)为$n$维线性空间\(V\)上的恒等变换,并且此时线性变换\(\varphi_i\)满足\(\varphi_1+\varphi_2+\cdots+\varphi_m = \boldsymbol{I}_n\).如果下列条件之一成立:

(1) \(\dim V=\dim\text{Im}\varphi_1+\dim\text{Im}\varphi_2+\cdots+\dim\text{Im}\varphi_m\);

(2) \(\varphi_i^2 = \varphi_i\),\(\varphi_i\varphi_j = 0\ (i\neq j)\),

则\(V = \text{Im}\varphi_1\oplus\text{Im}\varphi_2\oplus\cdots\oplus\text{Im}\varphi_m\),并且\(\varphi_i\)就是\(V\)到\(\text{Im}\varphi_i\)上的投影变换.
\end{corollary}
\begin{proof}
由\hyperref[proposition:投影变换的性质x]{命题\ref{proposition:投影变换的性质x}}可知条件(1)(2)等价,并且由\hyperref[proposition:投影变换的性质x]{命题\ref{proposition:投影变换的性质x}证法一的必要性证明过程}可直接由条件(1)推出\(V = \text{Im}\varphi_1\oplus\text{Im}\varphi_2\oplus\cdots\oplus\text{Im}\varphi_m\)(直和的证明也可由条件(2)及\hyperref[proposition:投影变换的性质]{投影变换的性质}直接得到).又因为\hyperlink{幂等变换和投影变换等价}{幂等变换和投影变换等价},故由条件(2)可直接得到\(\varphi_i\)就是\(V\)到\(\text{Im}\varphi_i\)上的投影变换.因此结论得证.
\end{proof}






\end{document}
\documentclass[../../main.tex]{subfiles}
\graphicspath{{\subfix{../../image/}}} % 指定图片目录,后续可以直接使用图片文件名。

% 例如:
% \begin{figure}[H]
% \centering
% \includegraphics[scale=0.4]{图.png}
% \caption{}
% \label{figure:图}
% \end{figure}
% 注意:上述\label{}一定要放在\caption{}之后,否则引用图片序号会只会显示??.

\begin{document}

\section{基本定理和命题}

\begin{proposition}[满射的复合仍是满射]\label{proposition:满射的复合仍是满射}
若\(f,g\)均为满射,则\(f\circ g\)也是满射.
\end{proposition}
\begin{remark}
单射的复合不一定是单射.
\end{remark}
\begin{proof}
设\(f:V_1\rightarrow V_2\),\(g:V_2\rightarrow V_3\),则对\(\forall \alpha \in V_3\),由\(g\)为满射可知,存在\(\beta \in V_2\),使得\(\alpha = g(\beta)\).又由\(f\)为满射可知,存在\(\gamma \in V_1\),使得\(\beta = f(\gamma)\).从而\((g\circ f)(\gamma)=g(f(\gamma)) = g(\beta)=\alpha\).故\(g\circ f\)也是满射.
\end{proof}

\begin{definition}[线性映射的表示矩阵]\label{definition:线性映射的表示矩阵}
设\(\varphi\)是\(V\to U\)的线性映射,分别取\(V\)和\(U\)的基如下:
\[
V:\boldsymbol{e}_1,\boldsymbol{e}_2,\cdots,\boldsymbol{e}_n; \ U:\boldsymbol{f}_1,\boldsymbol{f}_2,\cdots,\boldsymbol{f}_m.
\]
假设有
\[
\begin{cases}
\varphi(\boldsymbol{e}_1)=a_{11}\boldsymbol{f}_1 + a_{21}\boldsymbol{f}_2+\cdots + a_{m1}\boldsymbol{f}_m,\\
\varphi(\boldsymbol{e}_2)=a_{12}\boldsymbol{f}_1 + a_{22}\boldsymbol{f}_2+\cdots + a_{m2}\boldsymbol{f}_m,\\
\cdots\cdots\cdots\\
\varphi(\boldsymbol{e}_n)=a_{1n}\boldsymbol{f}_1 + a_{2n}\boldsymbol{f}_2+\cdots + a_{mn}\boldsymbol{f}_m,
\end{cases}
\]
则矩阵
\[
\begin{pmatrix}
a_{11}&a_{12}&\cdots&a_{1n}\\
a_{21}&a_{22}&\cdots&a_{2n}\\
\vdots&\vdots&&\vdots\\
a_{m1}&a_{m2}&\cdots&a_{mn}
\end{pmatrix}
\]
称为线性映射\(\varphi\)在基$\left\{ \boldsymbol{e}_1,\boldsymbol{e}_2,\cdots ,\boldsymbol{e}_n \right\}$和$\left\{ \boldsymbol{f}_1,\boldsymbol{f}_2,\cdots ,\boldsymbol{f}_m \right\}$下的表示矩阵.
\end{definition}
\begin{remark}
若\(\varphi\)是向量空间\(V\)上的线性变换,则取\(V\)的一组基,而不取两组基.
\end{remark}

\begin{corollary}
设\(\varphi\)是\(V\rightarrow U\)的线性映射,分别取\(V\)和\(U\)的基如下:
\[
V: \boldsymbol{e}_1, \boldsymbol{e}_2, \cdots, \boldsymbol{e}_n; \quad U: \boldsymbol{f}_1, \boldsymbol{f}_2, \cdots, \boldsymbol{f}_m.
\]
并且设\(\varphi\)在基\(\{\boldsymbol{e}_1, \boldsymbol{e}_2, \cdots, \boldsymbol{e}_n\}\)和\(\{\boldsymbol{f}_1, \boldsymbol{f}_2, \cdots, \boldsymbol{f}_m\}\)下的表示矩阵为\(\boldsymbol{A}\).则有
\[
(\varphi(\boldsymbol{e}_1), \varphi(\boldsymbol{e}_2), \cdots, \varphi(\boldsymbol{e}_n)) = (\boldsymbol{f}_1, \boldsymbol{f}_2, \cdots, \boldsymbol{f}_m)\boldsymbol{A}.
\]
\(\forall \boldsymbol{\alpha} \in V\),设\(\boldsymbol{\alpha}\)在基\(\{\boldsymbol{e}_1, \boldsymbol{e}_2, \cdots, \boldsymbol{e}_n\}\)下的坐标向量为\((x_1, x_2, \cdots, x_n)'\),则
\[
\boldsymbol{\alpha} = x_1\boldsymbol{e}_1 + x_2\boldsymbol{e}_2 + \cdots + x_n\boldsymbol{e}_n = (\boldsymbol{e}_1, \boldsymbol{e}_2, \cdots, \boldsymbol{e}_n)\begin{pmatrix}
x_1 \\
x_2 \\
\vdots \\
x_n
\end{pmatrix},
\]
\begin{align*}
\varphi(\boldsymbol{\alpha}) &= \varphi(x_1\boldsymbol{e}_1 + x_2\boldsymbol{e}_2 + \cdots + x_n\boldsymbol{e}_n)
= (\varphi(\boldsymbol{e}_1), \varphi(\boldsymbol{e}_2), \cdots, \varphi(\boldsymbol{e}_n))\begin{pmatrix}
x_1 \\
x_2 \\
\vdots \\
x_n
\end{pmatrix} 
= (\boldsymbol{f}_1, \boldsymbol{f}_2, \cdots, \boldsymbol{f}_m)\boldsymbol{A}\begin{pmatrix}
x_1 \\
x_2 \\
\vdots \\
x_n
\end{pmatrix}.
\end{align*}
即\(\varphi(\boldsymbol{\alpha})\)在基\(\{\boldsymbol{f}_1, \boldsymbol{f}_2, \cdots, \boldsymbol{f}_m\}\)下的坐标向量为\(\boldsymbol{A}(x_1, x_2, \cdots, x_n)'\).
\end{corollary}

\begin{theorem}\label{theorem:向量空间上同一个线性变换在不同基下的表示矩阵必相似}
设\(V\)是数域\(\mathbb{F}\)上的\(n\)维向量空间,\(\varphi\)是\(V\)上的线性变换,\(\{\boldsymbol{e}_1,\boldsymbol{e}_2,\cdots,\boldsymbol{e}_n\}\)和\(\{\boldsymbol{f}_1,\boldsymbol{f}_2,\cdots,\boldsymbol{f}_n\}\)是\(V\)的两组基,从第一组基到第二组基的过渡矩阵为\(\boldsymbol{P}\). 假设\(\varphi\)在第一组基下的表示矩阵为\(\boldsymbol{A}\),在第二组基下的表示矩阵为\(\boldsymbol{B}\),则\(\boldsymbol{B}=\boldsymbol{P}^{-1}\boldsymbol{A}\boldsymbol{P}\),即向量空间上同一个线性变换在不同基下的表示矩阵必相似.
\end{theorem}


\end{document}
\documentclass[../../main.tex]{subfiles}
\graphicspath{{\subfix{../../image/}}} % 指定图片目录,后续可以直接使用图片文件名。

% 例如:
% \begin{figure}[h]
% \centering
% \includegraphics{image-01.01}
% \label{fig:image-01.01}
% \caption{图片标题}
% \end{figure}

\begin{document}

\section{线性映射与矩阵}

线性映射与矩阵的关系是这一章的核心. 线性映射是一个几何概念,矩阵是一个代数概念,它们之间的关系需要掌握以下几点:
\begin{enumerate}[(1)]
\item\label{线性映射与矩阵基本结论1} 记数域\(\mathbb{F}\)上\(n\)维向量空间\(V\)到\(m\)维向量空间\(U\)的线性映射全体为\(\mathcal{L}(V,U)\),\(\mathbb{F}\)上\(m\times n\)矩阵全体为\(M_{m\times n}(\mathbb{F})\). 各自取定\(V\)和\(U\)的一组基,设\(\varphi\in\mathcal{L}(V,U)\)在给定基下的表示矩阵为\(\boldsymbol{A}\),则\(\varphi\mapsto\boldsymbol{A}\)定义了从\(\mathcal{L}(V,U)\)到\(M_{m\times n}(\mathbb{F})\)的一一对应,这个对应还是一个线性同构. 若\(m = n\),则在这个对应下,线性同构(可逆线性映射)对应于可逆矩阵. 特别地,若\(V = U\),上述对应还定义了一个代数同构,即除了保持加法与数乘外,还保持乘法. 因此,两个向量空间之间线性映射的运算完全可以归结为矩阵的运算.

\item\label{线性映射与矩阵基本结论2} 设线性映射\(\varphi\)在给定基下的表示矩阵为\(\boldsymbol{A}\),则\(\text{Ker}\varphi\)和齐次线性方程组\(\boldsymbol{A}\boldsymbol{x}=0\)的解空间同构,\(\text{Im}\varphi\)和\(\boldsymbol{A}\)的全体列向量张成的向量空间同构.因此$\mathrm{dim}\,\mathrm{Im}\varphi =\mathrm{r}\left( \varphi \right) =\mathrm{r}\left( \boldsymbol{A} \right) ,\mathrm{dim}\,\mathrm{Ker}\varphi =n-\mathrm{r}\left( \varphi \right) =n-\mathrm{r}\left( \boldsymbol{A} \right)$.
\end{enumerate}
这两点由\hyperref[theorem:线性映射与矩阵基本定理]{定理\ref{theorem:线性映射与矩阵基本定理}}的结论即得.

\begin{theorem}\label{theorem:线性映射与矩阵基本定理}
设\(\varphi\)是数域\(\mathbb{F}\)上\(n\)维线性空间\(V\)到\(m\)维线性空间\(U\)的线性映射. 令\(\mathbb{F}^n\)和\(\mathbb{F}^m\)分别是\(\mathbb{F}\)上\(n\)维和\(m\)维列向量空间. 又设\(\boldsymbol{e}_1,\boldsymbol{e}_2,\cdots,\boldsymbol{e}_n\)和
\(\boldsymbol{f}_1,\boldsymbol{f}_2,\cdots,\boldsymbol{f}_m\)分别是\(V\)和\(U\)的基,\(\varphi\)在给定基下的表示矩阵为\(\boldsymbol{A}\). 记\(\eta_1:V\to\mathbb{F}^n\)为\(V\)中向量映射到它在基\(\boldsymbol{e}_1,\boldsymbol{e}_2,\cdots,\boldsymbol{e}_n\)下的坐标向量的线性同构,\(\eta_2:U\to\mathbb{F}^m\)为\(U\)中向量映射到它在基\(\boldsymbol{f}_1,\boldsymbol{f}_2,\cdots,\boldsymbol{f}_m\)下的坐标向量的线性同构,\(\boldsymbol{A}:\mathbb{F}^n\to\mathbb{F}^m\)为矩阵乘法诱导的线性映射,即\(\boldsymbol{A}(\boldsymbol{\alpha})=\boldsymbol{A}\boldsymbol{\alpha}\). 求证:\(\eta_2\varphi=\boldsymbol{A}\eta_1\),即下列图交换,并且\(\eta_1:\text{Ker}\varphi\to\text{Ker}\boldsymbol{A},\eta_2:\text{Im}\varphi\to\text{Im}\boldsymbol{A}\)都是线性同构.
\[\begin{tikzcd}
V \arrow[r, "\varphi"] \arrow[d, "\eta_1"'] & U \arrow[d, "\eta_2"] \\
\mathbb{F}^n \arrow[r, "\boldsymbol{A}"]    & \mathbb{F}^m         
\end{tikzcd}
\]
\end{theorem}
\begin{proof}

\end{proof}


\begin{proposition}\label{proposition:线性映射与过渡矩阵}
设\(\varphi\)是线性空间\(V\)到\(U\)的线性映射,\(\{\boldsymbol{e}_1,\boldsymbol{e}_2,\cdots,\boldsymbol{e}_n\}\)和\(\{\boldsymbol{f}_1,\boldsymbol{f}_2,\cdots,\boldsymbol{f}_n\}\)是\(V\)的两组基,\(\{\boldsymbol{e}_1,\boldsymbol{e}_2,\cdots,\boldsymbol{e}_n\}\)到\(\{\boldsymbol{f}_1,\boldsymbol{f}_2,\cdots,\boldsymbol{f}_n\}\)的过渡矩阵为\(\boldsymbol{P}\). \(\{\boldsymbol{g}_1,\boldsymbol{g}_2,\cdots,\boldsymbol{g}_m\}\)和\(\{\boldsymbol{h}_1,\boldsymbol{h}_2,\cdots,\boldsymbol{h}_m\}\)是\(U\)的两组基,\(\{\boldsymbol{g}_1,\boldsymbol{g}_2,\cdots,\boldsymbol{g}_m\}\)到\(\{\boldsymbol{h}_1,\boldsymbol{h}_2,\cdots,\boldsymbol{h}_m\}\)的过渡矩阵为\(\boldsymbol{Q}\). 又设\(\varphi\)在基\(\{\boldsymbol{e}_1,\boldsymbol{e}_2,\cdots,\boldsymbol{e}_n\}\)和基\(\{\boldsymbol{g}_1,\boldsymbol{g}_2,\cdots,\boldsymbol{g}_m\}\)下的表示矩阵为\(\boldsymbol{A}\),在基\(\{\boldsymbol{f}_1,\boldsymbol{f}_2,\cdots,\boldsymbol{f}_n\}\)和基\(\{\boldsymbol{h}_1,\boldsymbol{h}_2,\cdots,\boldsymbol{h}_m\}\)下的表示矩阵为\(\boldsymbol{B}\). 求证:\(\boldsymbol{B}=\boldsymbol{Q}^{-1}\boldsymbol{A}\boldsymbol{P}\).

特别地,若$\varphi$是线性空间$V$上的线性变换,\(\{\boldsymbol{e}_1,\boldsymbol{e}_2,\cdots,\boldsymbol{e}_n\}\)和\(\{\boldsymbol{f}_1,\boldsymbol{f}_2,\cdots,\boldsymbol{f}_n\}\)是\(V\)的两组基,\(\{\boldsymbol{e}_1,\boldsymbol{e}_2,\cdots,\boldsymbol{e}_n\}\)到\(\{\boldsymbol{f}_1,\boldsymbol{f}_2,\cdots,\boldsymbol{f}_n\}\)的过渡矩阵为\(\boldsymbol{P}\),又设\(\varphi\)在基\(\{\boldsymbol{e}_1,\boldsymbol{e}_2,\cdots,\boldsymbol{e}_n\}\)下的表示矩阵为\(\boldsymbol{A}\),则$\varphi$在基\(\{\boldsymbol{f}_1,\boldsymbol{f}_2,\cdots,\boldsymbol{f}_n\}\)下的表示矩阵$B=P^{-1}AP$.
\end{proposition}
\begin{proof}
任取\(\boldsymbol{v}\in V\),设它在基\(\{\boldsymbol{e}_1,\boldsymbol{e}_2,\cdots,\boldsymbol{e}_n\}\)下的坐标向量为\((x_1,x_2,\cdots,x_n)'\),则它在基\(\{\boldsymbol{f}_1,\boldsymbol{f}_2,\cdots,\boldsymbol{f}_n\}\)下的坐标向量为\(\boldsymbol{P}^{-1}(x_1,x_2,\cdots,x_n)'\). \(\varphi(\boldsymbol{v})\)在基\(\{\boldsymbol{g}_1,\boldsymbol{g}_2,\cdots,\boldsymbol{g}_m\}\)下的坐标向量为\(\boldsymbol{A}(x_1,x_2,\cdots,x_n)'\),在基\(\{\boldsymbol{h}_1,\boldsymbol{h}_2,\cdots,\boldsymbol{h}_m\}\)下的坐标向量为\(\boldsymbol{B}\boldsymbol{P}^{-1}(x_1,x_2,\cdots,x_n)'\). 由于从\(\{\boldsymbol{g}_1,\boldsymbol{g}_2,\cdots,\boldsymbol{g}_m\}\)到\(\{\boldsymbol{h}_1,\boldsymbol{h}_2,\cdots,\boldsymbol{h}_m\}\)的过渡矩阵
为\(\boldsymbol{Q}\),故
\begin{align*}
\boldsymbol{A}(x_1,x_2,\cdots,x_n)'=\boldsymbol{Q}\boldsymbol{B}\boldsymbol{P}^{-1}(x_1,x_2,\cdots,x_n)'.  
\end{align*}
因为\((x_1,x_2,\cdots,x_n)'\)是任意的,
故\(\boldsymbol{A}=\boldsymbol{Q}\boldsymbol{B}\boldsymbol{P}^{-1}\),即\(\boldsymbol{B}=\boldsymbol{Q}^{-1}\boldsymbol{A}\boldsymbol{P}\).
\end{proof}


\begin{proposition}\label{proposition:线性映射在基下的矩阵为标准型}
设\(\varphi\)是有限维线性空间\(V\)到\(U\)的线性映射,求证:必存在\(V\)和\(U\)的两组基,使线性映射\(\varphi\)在两组基下的表示矩阵为\(\begin{pmatrix}\boldsymbol{I}_r&\boldsymbol{O}\\\boldsymbol{O}&\boldsymbol{O}\end{pmatrix}\).
\end{proposition}
\begin{proof}
设\(\{\boldsymbol{e}_1,\boldsymbol{e}_2,\cdots,\boldsymbol{e}_n\}\)是\(V\)的一组基,\(\{\boldsymbol{g}_1,\boldsymbol{g}_2,\cdots,\boldsymbol{g}_m\}\)是\(U\)的一组基,\(\varphi\)在这两组基下的表示矩阵为\(\boldsymbol{A}\). 由相抵标准型理论可知,存在\(m\)阶非异阵\(\boldsymbol{Q}\),\(n\)阶非异阵\(\boldsymbol{P}\),使得\(\boldsymbol{Q}^{-1}\boldsymbol{A}\boldsymbol{P}=\begin{pmatrix}\boldsymbol{I}_r&\boldsymbol{O}\\\boldsymbol{O}&\boldsymbol{O}\end{pmatrix}\). 设\(\{\boldsymbol{f}_1,\boldsymbol{f}_2,\cdots,\boldsymbol{f}_n\}\)是\(V\)的一组新基,使得从\(\{\boldsymbol{e}_1,\boldsymbol{e}_2,\cdots,\boldsymbol{e}_n\}\)到\(\{\boldsymbol{f}_1,\boldsymbol{f}_2,\cdots,\boldsymbol{f}_n\}\)的过渡矩阵为\(\boldsymbol{P}\);设\(\{\boldsymbol{h}_1,\boldsymbol{h}_2,\cdots,\boldsymbol{h}_m\}\)是\(U\)的一组新基,使得从\(\{\boldsymbol{g}_1,\boldsymbol{g}_2,\cdots,\boldsymbol{g}_m\}\)到\(\{\boldsymbol{h}_1,\boldsymbol{h}_2,\cdots,\boldsymbol{h}_m\}\)的过渡矩阵为\(\boldsymbol{Q}\),则由\hyperref[proposition:线性映射与过渡矩阵]{命题\ref{proposition:线性映射与过渡矩阵}}可知,\(\varphi\)在两组新基下的表示矩阵为\(\boldsymbol{Q}^{-1}\boldsymbol{A}\boldsymbol{P}=\begin{pmatrix}\boldsymbol{I}_r&\boldsymbol{O}\\\boldsymbol{O}&\boldsymbol{O}\end{pmatrix}\). 
\end{proof}
\begin{remark}
利用这个命题可以得到\(\text{Ker}\varphi = L(\boldsymbol{f}_{r + 1},\cdots,\boldsymbol{f}_n),\text{Im}\varphi = L(\boldsymbol{h}_1,\cdots,\boldsymbol{h}_r)\),由此即得线性映射的维数公式.
\end{remark}

\begin{proposition}[线性映射维数公式]\label{proposition:值域和核空间维数之和等于原像空间维数}
设\(V,U\)是数域\(\mathbb{K}\)上的有限维线性空间,\(\varphi:V\to U\)是线性映射,\(\varphi:V\to U\)为线性映射,求证:\[
\dim\text{Ker}\varphi+\dim\text{Im}\varphi=\dim V.
\]
\end{proposition}
\begin{proof}
{\color{blue}证法一:}
设\(\dim V = n,\dim\text{Ker}\varphi = k\),我们只要证明\(\dim\text{Im}\varphi=n - k\)即可. 取\(\text{Ker}\varphi\)的一组基\(\boldsymbol{e}_1,\cdots,\boldsymbol{e}_k\),并将其扩张为\(V\)的一组基\(\boldsymbol{e}_1,\cdots,\boldsymbol{e}_k,\boldsymbol{e}_{k + 1},\cdots,\boldsymbol{e}_n\). 任取\(\boldsymbol{\alpha}\in V\),设\(\boldsymbol{\alpha}=c_1\boldsymbol{e}_1+\cdots + c_k\boldsymbol{e}_k + c_{k + 1}\boldsymbol{e}_{k + 1}+\cdots + c_n\boldsymbol{e}_n\),则\(\varphi(\boldsymbol{\alpha})=c_{k + 1}\varphi(\boldsymbol{e}_{k + 1})+\cdots + c_n\varphi(\boldsymbol{e}_n)\),即\(\text{Im}\varphi\)中任一向量都是\(\varphi(\boldsymbol{e}_{k + 1}),\cdots,\varphi(\boldsymbol{e}_n)\)的线性组合. 下证\(\varphi(\boldsymbol{e}_{k + 1}),\cdots,\varphi(\boldsymbol{e}_n)\)线性无关. 设\(\lambda_{k + 1}\varphi(\boldsymbol{e}_{k + 1})+\cdots + \lambda_n\varphi(\boldsymbol{e}_n)=\boldsymbol{0}\),则\(\varphi(\lambda_{k + 1}\boldsymbol{e}_{k + 1}+\cdots + \lambda_n\boldsymbol{e}_n)=\boldsymbol{0}\),即\(\lambda_{k + 1}\boldsymbol{e}_{k + 1}+\cdots + \lambda_n\boldsymbol{e}_n\in\text{Ker}\varphi\),故可设\(\lambda_{k + 1}\boldsymbol{e}_{k + 1}+\cdots + \lambda_n\boldsymbol{e}_n=\lambda_1\boldsymbol{e}_1+\cdots + \lambda_k\boldsymbol{e}_k\),再由\(\boldsymbol{e}_1,\cdots,\boldsymbol{e}_k,\boldsymbol{e}_{k + 1},\cdots,\boldsymbol{e}_n\)线性无关可知\(\lambda_1=\cdots=\lambda_k=\lambda_{k + 1}=\cdots=\lambda_n = 0\). 因此\(\varphi(\boldsymbol{e}_{k + 1}),\cdots,\varphi(\boldsymbol{e}_n)\)是\(\text{Im}\varphi\)的一组基,从而\(\dim\text{Im}\varphi=n - k\),结论得证. 

{\color{blue}证法二(从商空间的角度):}
设由\(\varphi\)诱导的线性映射\(\overline{\varphi}:V/\text{Ker}\varphi\to\text{Im}\varphi\),\(\overline{\varphi}(\boldsymbol{v}+\text{Ker}\varphi)=\varphi(\boldsymbol{v})\). 先证是$\overline{\varphi}$线性同构的.

首先,\(\overline{\varphi}\)的定义不依赖于\(\text{Ker}\varphi -\)陪集代表元的选取. 事实上,若\(\boldsymbol{v}_1+\text{Ker}\varphi=\boldsymbol{v}_2+\text{Ker}\varphi\),即\(\boldsymbol{v}_1 - \boldsymbol{v}_2\in\text{Ker}\varphi\),则\(0=\varphi(\boldsymbol{v}_1 - \boldsymbol{v}_2)=\varphi(\boldsymbol{v}_1)-\varphi(\boldsymbol{v}_2)\),即\(\varphi(\boldsymbol{v}_1)=\varphi(\boldsymbol{v}_2)\). 其次,容易验证\(\overline{\varphi}\)是一个线性映射. 再次,由\(\overline{\varphi}\)的定义不难看出它是满射. 最后,由\(\overline{\varphi}\)的定义可知\(\text{Ker}\overline{\varphi}=\{\boldsymbol{0}+\text{Ker}\varphi\}\)是商空间\(V/\text{Ker}\varphi\)的零子空间,故为单射,从而\(\overline{\varphi}:V/\text{Ker}\varphi\to\text{Im}\varphi\)是线性同构. 由\hyperref[proposition:商空间的维数公式和商空间与补空间同构]{商空间的维数公式}可得
\[
\dim\text{Im}\varphi=\dim(V/\text{Ker}\varphi)=\dim V-\dim\text{Ker}\varphi,
\]
由此即得线性映射的维数公式.
\end{proof}

\begin{corollary}\label{corollary:由核的基导出值域的基}
设\(\varphi:V\to U\)为线性映射,$\mathrm{Ker}\varphi $的一组基为$\boldsymbol{e}_{r+1}\cdots ,\boldsymbol{e}_n,$并将其扩张为$V$的一组基$\boldsymbol{e}_1,\boldsymbol{e}_2,\cdots ,\boldsymbol{e}_n.$
则$\varphi \left( \boldsymbol{e}_1 \right) ,\cdots ,\varphi \left( \boldsymbol{e}_r \right)$ 一定是Im$\varphi$ 的一组基.
\end{corollary}
\begin{proof}
由\hyperref[proposition:值域和核空间维数之和等于原像空间维数]{命题\ref{proposition:值域和核空间维数之和等于原像空间维数}}的证法一立得.

\end{proof}
\begin{proposition}\label{proposition:行/列满秩矩阵对应满/单射}
设\(\varphi\)是\(n\)维线性空间\(V\)到\(m\)维线性空间\(U\)的线性映射,\(\varphi\)在给定基下的表示矩阵为\(\boldsymbol{A}_{m\times n}\). 求证:\(\varphi\)是满映射的充要条件是\(\text{r}(\boldsymbol{A}) = m\)($\boldsymbol{A}$行满秩),\(\varphi\)是单映射的充要条件是\(\text{r}(\boldsymbol{A}) = n\)($\boldsymbol{A}$列满秩).
\end{proposition}
\begin{note}
\(\dim\text{Im}\varphi=\text{r}(\boldsymbol{A})\)和\(\dim\text{Ker}\varphi=n - \text{r}(\boldsymbol{A})\)的原因见\hyperref[线性映射与矩阵基本结论2]{线性映射与矩阵基本结论\ref{线性映射与矩阵基本结论2}}.
\end{note}
\begin{proof}
注意到\(\dim\text{Im}\varphi=\text{r}(\boldsymbol{A})\),并且\(\varphi\)是满映射的充要条件是\(\text{Im}\varphi = U\),这也等价于\(\dim\text{Im}\varphi=\dim U = m\),故第一个结论成立.

注意到\(\dim\text{Ker}\varphi=n - \text{r}(\boldsymbol{A})\),并且\(\varphi\)是单映射的充要条件是\(\text{Ker}\varphi = 0\),这也等价于\(\dim\text{Ker}\varphi = 0\),故第二个结论成立.
\end{proof}

\begin{proposition}\label{proposition:线性映射的秩1分解}
设\(\varphi:V\to U\)为线性映射且\(\varphi\)的秩为\(r\),证明:存在\(r\)个秩为\(1\)的线性映射\(\varphi_i:V\to U(1\leq i\leq r)\),使得\(\varphi=\varphi_1+\cdots+\varphi_r\).
\end{proposition}
\begin{proof}
取定\(V\)和\(U\)的两组基,设\(\varphi\)在这两组基下的表示矩阵为\(\boldsymbol{A}\),则\(\text{r}(\boldsymbol{A})=\text{r}(\varphi)=r\). 由\hyperref[proposition:矩阵的秩1分解]{矩阵的秩1分解}可知,存在\(r\)个秩为\(1\)的矩阵\(\boldsymbol{A}_i(1\leq i\leq r)\),使得\(\boldsymbol{A}=\boldsymbol{A}_1+\cdots+\boldsymbol{A}_r\). 由于线性映射和表示矩阵之间一一对应,故存在线性映射\(\varphi_i:V\to U(1\leq i\leq r)\),使得\(\varphi=\varphi_1+\cdots+\varphi_r\),且\(\text{r}(\varphi_i)=\text{r}(\boldsymbol{A}_i)=1\).
\end{proof}

\begin{proposition}\label{proposition:任一组基下的表示矩阵都相同的线性变换是纯量变换}
设\(\varphi\)是线性空间\(V\)上的线性变换,若它在\(V\)的任一组基下的表示矩阵都相同,求证:\(\varphi\)是纯量变换,即存在常数\(k\),使得\(\varphi(\boldsymbol{\alpha}) = k\boldsymbol{\alpha}\)对一切\(\boldsymbol{\alpha}\in V\)都成立.
\end{proposition}
\begin{proof}
取定\(V\)的一组基,设\(\varphi\)在这组基下的表示矩阵是\(\boldsymbol{A}\). 由已知条件可知,对任意一个同阶可逆矩阵\(\boldsymbol{P}\),\(\boldsymbol{A}=\boldsymbol{P}^{-1}\boldsymbol{A}\boldsymbol{P}\),即\(\boldsymbol{P}\boldsymbol{A}=\boldsymbol{A}\boldsymbol{P}\). 因此矩阵\(\boldsymbol{A}\)和任意一个可逆矩阵乘法可交换,于是由\hyperref[proposition:纯量阵的刻画]{命题\ref{proposition:纯量阵的刻画}}可知\(\boldsymbol{A}=k\boldsymbol{I}_n\),由此即知\(\varphi\)是纯量变换.
\end{proof}

\subsection{将矩阵问题转化为线性映射问题}
我们将线性映射的问题转化为矩阵问题来处理. 反之,我们也可将矩阵问题转化为线性映射(线性变换)问题来处理. 一般的处理方式如下:

设\(\boldsymbol{A}\)是数域\(\mathbb{F}\)上的\(m\times n\)矩阵,定义列向量空间\(\mathbb{F}^n\)到\(\mathbb{F}^m\)的线性映射:\(\varphi(\boldsymbol{\alpha})=\boldsymbol{A}\boldsymbol{\alpha}\),容易验证在\(\mathbb{F}^n\)和\(\mathbb{F}^m\)的标准单位列向量构成的基下,\(\varphi\)的表示矩阵就是\(\boldsymbol{A}\). 同理,若\(\boldsymbol{A}\)是\(\mathbb{F}\)上的\(n\)阶矩阵,定义\(\mathbb{F}^n\)上的线性变换:\(\varphi(\boldsymbol{\alpha})=\boldsymbol{A}\boldsymbol{\alpha}\),容易验证在\(\mathbb{F}^n\)的标准单位列向量构成的基下,\(\varphi\)的表示矩阵就是\(\boldsymbol{A}\). 

因此,我们有时就把这个线性映射(线性变换)写为\(\boldsymbol{A}\). 上述把代数问题转化成几何问题的语言表述,在后面的章节中一直会用到. 某些矩阵问题采用这种方式转化为线性映射(线性变换)问题后,往往变得比较容易解决或者可以充分利用几何直观去得到解题思路.
\begin{example}
设\(\boldsymbol{A},\boldsymbol{B}\)都是数域\(\mathbb{F}\)上的\(m\times n\)矩阵,求证:方程组\(\boldsymbol{A}\boldsymbol{x}=0,\boldsymbol{B}\boldsymbol{x}=0\)同解的充要条件是存在可逆矩阵\(\boldsymbol{P}\),使得\(\boldsymbol{B}=\boldsymbol{P}\boldsymbol{A}\).
\end{example}
\begin{proof}
因为\(\boldsymbol{P}\)是可逆矩阵,充分性是显然的. 现通过两种方法来证明必要性.

{\color{blue}证法一(代数方法):} 由条件可得方程组\(\boldsymbol{A}\boldsymbol{x}=0,\boldsymbol{B}\boldsymbol{x}=0,\begin{pmatrix}\boldsymbol{A}\\\boldsymbol{B}\end{pmatrix}\boldsymbol{x}=0\)都同解,从而有
\[
\text{r}(\boldsymbol{A})=\text{r}(\boldsymbol{B})=\text{r}\begin{pmatrix}\boldsymbol{A}\\\boldsymbol{B}\end{pmatrix}.
\]
注意到结论\(\boldsymbol{B}=\boldsymbol{P}\boldsymbol{A}\)就是说\(\boldsymbol{A},\boldsymbol{B}\)可以通过初等行变换相互转化,因此在证明的过程中,对\(\boldsymbol{A}\)或\(\boldsymbol{B}\)实施初等行变换不影响结论的证明. 设
\[
\boldsymbol{A}=\begin{pmatrix}\boldsymbol{\alpha}_1\\\boldsymbol{\alpha}_2\\\vdots\\\boldsymbol{\alpha}_m\end{pmatrix},\boldsymbol{B}=\begin{pmatrix}\boldsymbol{\beta}_1\\\boldsymbol{\beta}_2\\\vdots\\\boldsymbol{\beta}_m\end{pmatrix}
\]
分别为\(\boldsymbol{A},\boldsymbol{B}\)的行分块. 不妨对\(\boldsymbol{A},\boldsymbol{B}\)都进行行对换,故可设\(\boldsymbol{\alpha}_1,\cdots,\boldsymbol{\alpha}_r\)是\(\boldsymbol{A}\)的行向量的极大无关组,\(\boldsymbol{\beta}_1,\cdots,\boldsymbol{\beta}_r\)是\(\boldsymbol{B}\)的行向量的极大无关组. 由于\(\text{r}\begin{pmatrix}\boldsymbol{A}\\\boldsymbol{B}\end{pmatrix}=r\),故由\hyperref[proposition:表出向量组的秩不超过原向量组的秩]{命题\ref{proposition:表出向量组的秩不超过原向量组的秩}}可知,\(\boldsymbol{\alpha}_1,\cdots,\boldsymbol{\alpha}_r\)和\(\boldsymbol{\beta}_1,\cdots,\boldsymbol{\beta}_r\)是向量组\(\boldsymbol{\alpha}_1,\boldsymbol{\alpha}_2,\cdots,\boldsymbol{\alpha}_m,\)\(\boldsymbol{\beta}_1,\boldsymbol{\beta}_2,\cdots,\boldsymbol{\beta}_m\)的两组极大无关组. 设\(\boldsymbol{\beta}_i=\sum_{j = 1}^{r}c_{ij}\boldsymbol{\alpha}_j(1\leq i\leq r)\),则容易验证\(r\)阶方阵\(\boldsymbol{C}=(c_{ij})\)是非异阵. 设\(\boldsymbol{\beta}_i-\boldsymbol{\alpha}_i=\sum_{j = 1}^{r}d_{ij}\boldsymbol{\alpha}_j(r + 1\leq i\leq m)\),\(\boldsymbol{D}=(d_{ij})\)是\((m - r)\times r\)矩阵,则容易验证\(\boldsymbol{P}=\begin{pmatrix}\boldsymbol{C}&\boldsymbol{O}\\\boldsymbol{D}&\boldsymbol{I}_{m - r}\end{pmatrix}\)是\(m\)阶非异阵,并且满足\(\boldsymbol{B}=\boldsymbol{P}\boldsymbol{A}\).

{\color{blue}证法二(几何方法):}  将问题转化成几何的语言即为:设\(V\)是\(\mathbb{F}\)上的\(n\)维线性空间,\(U\)是\(\mathbb{F}\)上的\(m\)维线性空间,\(\varphi,\psi:V\to U\)是两个线性映射. 求证:若\(\text{Ker}\varphi=\text{Ker}\psi\),则存在\(U\)上的自同构\(\sigma\),使得\(\psi=\sigma\varphi\).

设\(\text{r}(\varphi)=r\),则\(\dim\text{Ker}\varphi=\dim\text{Ker}\psi=n - r\). 取\(\text{Ker}\varphi=\text{Ker}\psi\)的一组基\(\boldsymbol{e}_{r + 1},\cdots,\boldsymbol{e}_n\),并将其扩张为\(V\)的一组基\(\boldsymbol{e}_1,\cdots,\boldsymbol{e}_r,\boldsymbol{e}_{r + 1},\cdots,\boldsymbol{e}_n\). 根据\hyperref[corollary:由核的基导出值域的基]{推论\ref{corollary:由核的基导出值域的基}}可知,\(\varphi(\boldsymbol{e}_1),\cdots,\varphi(\boldsymbol{e}_r)\)是\(\text{Im}\varphi\)的一组基,故可将其扩张为\(U\)的一组基\(\varphi(\boldsymbol{e}_1),\cdots,\varphi(\boldsymbol{e}_r),\boldsymbol{f}_{r + 1},\cdots,\boldsymbol{f}_m\). 同理可知,\(\psi(\boldsymbol{e}_1),\cdots,\psi(\boldsymbol{e}_r)\)是\(\text{Im}\psi\)的一组基,故可将其扩张为\(U\)的一组基\(\psi(\boldsymbol{e}_1),\cdots,\psi(\boldsymbol{e}_r),\boldsymbol{g}_{r + 1},\cdots,\boldsymbol{g}_m\). 定义\(U\)上的线性变换\(\sigma\)如下:
\[
\sigma(\varphi(\boldsymbol{e}_i))=\psi(\boldsymbol{e}_i),1\leq i\leq r;\ \sigma(\boldsymbol{f}_j)=\boldsymbol{g}_j,r + 1\leq j\leq m.
\]
因为\(\sigma\)把\(U\)的一组基映射为\(U\)的另一组基,故\(\sigma\)是\(U\)的自同构. 又对\(r + 1\leq j\leq n\),\(\sigma(\varphi(\boldsymbol{e}_j))=0=\psi(\boldsymbol{e}_j)\),故\(\sigma\varphi=\psi\)成立. 
\end{proof}

\begin{proposition}\label{proposition:相似矩阵可看作一个线性变换在不同基下的表示矩阵}
若数域\(\mathbb{F}\)上的\(n\)阶方阵\(\boldsymbol{A}\)和\(\boldsymbol{B}\)相似,求证:它们可以看成是某个线性空间上同一个线性变换在不同基下的表示矩阵.
\end{proposition}
\begin{note}
由下面的证明可知这个线性变换\(\varphi\)就是由矩阵\(\boldsymbol{A}\)的乘法诱导的线性变换.两组不同的基就是标准基与可逆矩阵\(\boldsymbol{P}\)的列向量.
\end{note}
\begin{proof}
令\(V = \mathbb{F}^n\)是\(n\)维列向量空间,\(\{\boldsymbol{e}_1,\boldsymbol{e}_2,\cdots,\boldsymbol{e}_n\}\)是由\(n\)维标准单位列向量构成的基,\(\varphi\)是由矩阵\(\boldsymbol{A}\)的乘法诱导的线性变换,容易验证\(\varphi\)在基\(\{\boldsymbol{e}_1,\boldsymbol{e}_2,\cdots,\boldsymbol{e}_n\}\)下的表示矩阵就是\(\boldsymbol{A}\). 已知\(\boldsymbol{A}\)和\(\boldsymbol{B}\)相似,即存在可逆矩阵\(\boldsymbol{P}\),使得\(\boldsymbol{B}=\boldsymbol{P}^{-1}\boldsymbol{A}\boldsymbol{P}\).
令\(\boldsymbol{P}=(\boldsymbol{f}_1,\boldsymbol{f}_2,\cdots,\boldsymbol{f}_n)\)为其列分块,由于\(\boldsymbol{P}\)可逆,故\(\boldsymbol{f}_1,\boldsymbol{f}_2,\cdots,\boldsymbol{f}_n\)线性无关,从而是\(V\)的一组基. 注意到从基\(\{\boldsymbol{e}_1,\boldsymbol{e}_2,\cdots,\boldsymbol{e}_n\}\)到基\(\{\boldsymbol{f}_1,\boldsymbol{f}_2,\cdots,\boldsymbol{f}_n\}\)的过渡矩阵就是\(\boldsymbol{P}\),因此线性变换\(\varphi\)在基\(\{\boldsymbol{f}_1,\boldsymbol{f}_2,\cdots,\boldsymbol{f}_n\}\)下的表示矩阵为\(\boldsymbol{P}^{-1}\boldsymbol{A}\boldsymbol{P}=\boldsymbol{B}\). 
\end{proof}

\begin{example}
设\(V\)是数域\(\mathbb{F}\)上\(n\)阶矩阵全体构成的线性空间,\(\varphi\)是\(V\)上的线性变换:\(\varphi(\boldsymbol{A})=\boldsymbol{A}'\). 证明:存在\(V\)的一组基,使得\(\varphi\)在这组基下的表示矩阵是一个对角矩阵且主对角元素全是\(1\)或\(-1\),并求出\(1\)和\(-1\)的个数.
\end{example}
\begin{proof}
设\(V_1\)是由\(n\)阶对称矩阵组成的子空间,\(V_2\)是由反对称矩阵组成的子空间,则由\hyperref[proposition:矩阵空间可以分解为对称和反称矩阵空间的直和]{命题\ref{proposition:矩阵空间可以分解为对称和反称矩阵空间的直和}}可得
\[
V = V_1\oplus V_2.
\]
取\(V_1\)的一组基和\(V_2\)的一组基拼成\(V\)的一组基,则\(\varphi\)在这组基下的表示矩阵是对角矩阵且主对角元素或为\(1\)或为\(-1\). 因为\(\dim V_1=\frac{1}{2}n(n + 1)\),\(\dim V_2=\frac{1}{2}n(n - 1)\),故\(1\)的个数为\(\frac{1}{2}n(n + 1)\),\(-1\)的个数为\(\frac{1}{2}n(n - 1)\).
\end{proof}

\begin{example}
设\(V\)是数域\(\mathbb{K}\)上的\(n\)维线性空间,\(\varphi,\psi\)是\(V\)上的线性变换且\(\varphi^2 = 0\),\(\psi^2 = 0\),\(\varphi\psi+\psi\varphi=\boldsymbol{I}\),\(\boldsymbol{I}\)是\(V\)上的恒等变换. 求证:
\begin{enumerate}[(1)]
\item \(V=\text{Ker}\varphi\oplus\text{Ker}\psi\);

\item 若\(V\)是二维空间,则存在\(V\)的基\(\boldsymbol{e}_1,\boldsymbol{e}_2\),使得\(\varphi,\psi\)在这组基下的表示矩阵分别为
\[
\boldsymbol{A}=\begin{pmatrix}0&0\\1&0\end{pmatrix},\boldsymbol{B}=\begin{pmatrix}0&1\\0&0\end{pmatrix};
\]

\item \(V\)必是偶数维空间且若\(V\)是\(2k\)维空间,则存在\(V\)的一组基,使得\(\varphi,\psi\)在这组基下的表示矩阵分别为下列分块对角矩阵:
\[
\begin{pmatrix}
\boldsymbol{A}&\boldsymbol{O}&\cdots&\boldsymbol{O}\\
\boldsymbol{O}&\boldsymbol{A}&\cdots&\boldsymbol{O}\\
\vdots&\vdots&&\vdots\\
\boldsymbol{O}&\boldsymbol{O}&\cdots&\boldsymbol{A}
\end{pmatrix},\begin{pmatrix}
\boldsymbol{B}&\boldsymbol{O}&\cdots&\boldsymbol{O}\\
\boldsymbol{O}&\boldsymbol{B}&\cdots&\boldsymbol{O}\\
\vdots&\vdots&&\vdots\\
\boldsymbol{O}&\boldsymbol{O}&\cdots&\boldsymbol{B}
\end{pmatrix},
\]
其中主对角线上分别有\(k\)个\(\boldsymbol{A}\)和\(k\)个\(\boldsymbol{B}\).
\end{enumerate}
\end{example}
\begin{proof}
\begin{enumerate}[(1)]
\item 任取\(\boldsymbol{\alpha}\in V\),则由\(\boldsymbol{I}=\varphi\psi+\psi\varphi\)得到\(\boldsymbol{\alpha}=\varphi\psi(\boldsymbol{\alpha})+\psi\varphi(\boldsymbol{\alpha})\). 注意到\(\varphi\psi(\boldsymbol{\alpha})\in\text{Ker}\varphi\),\(\psi\varphi(\boldsymbol{\alpha})\in\text{Ker}\psi\),因此\(V=\text{Ker}\varphi+\text{Ker}\psi\). 又若\(\boldsymbol{\beta}\in\text{Ker}\varphi\cap\text{Ker}\psi\),则\(\boldsymbol{\beta}=\varphi\psi(\boldsymbol{\beta})+\psi\varphi(\boldsymbol{\beta}) = 0\),即\(\text{Ker}\varphi\cap\text{Ker}\psi = 0\). 于是\(V=\text{Ker}\varphi\oplus\text{Ker}\psi\).

\item 取\(\boldsymbol{0}\neq\boldsymbol{e}_1\in\text{Ker}\psi\),\(\boldsymbol{e}_2 = \varphi(\boldsymbol{e}_1)\),则\(\varphi(\boldsymbol{e}_2)=\varphi^2(\boldsymbol{e}_1)=0\),即\(\boldsymbol{e}_2\in\text{Ker}\varphi\). 又若\(\boldsymbol{e}_2 = \boldsymbol{0}\),则\(\boldsymbol{e}_1\in\text{Ker}\varphi\cap\text{Ker}\psi = 0\),和假设矛盾,于是\(\boldsymbol{e}_2\neq\boldsymbol{0}\). 因此\(\boldsymbol{e}_1,\boldsymbol{e}_2\)组成\(V\)的一组基,不难验证在这组基下,\(\varphi,\psi\)的表示矩阵符合要求($\psi \left( \boldsymbol{e}_2 \right) =\psi \left( \varphi \left( \boldsymbol{e}_1 \right) \right) =\psi \varphi \left( \boldsymbol{e}_1 \right) =\boldsymbol{I}\left( \boldsymbol{e}_1 \right) -\varphi \psi \left( \boldsymbol{e}_1 \right) =\boldsymbol{e}_1$).

\item 设\(\dim\text{Ker}\psi=k\),并取\(\text{Ker}\psi\)的一组基\(\boldsymbol{e}_1,\boldsymbol{e}_2,\cdots,\boldsymbol{e}_k\). 令\(\boldsymbol{e}_{k + 1}=\varphi(\boldsymbol{e}_1)\),\(\boldsymbol{e}_{k + 2}=\varphi(\boldsymbol{e}_2),\cdots,\boldsymbol{e}_{2k}=\varphi(\boldsymbol{e}_k)\),则由\(\varphi^2 = 0\)可得\(\boldsymbol{e}_{k + 1},\boldsymbol{e}_{k + 2},\cdots,\boldsymbol{e}_{2k}\)都属于\(\text{Ker}\varphi\). 我们先证明向量组\(\boldsymbol{e}_{k + 1},\boldsymbol{e}_{k + 2},\cdots,\boldsymbol{e}_{2k}\)是线性无关的. 设有
\[
c_1\boldsymbol{e}_{k + 1}+c_2\boldsymbol{e}_{k + 2}+\cdots + c_k\boldsymbol{e}_{2k}=\boldsymbol{0},
\]
两边作用\(\psi\),可得
\[
c_1\psi(\boldsymbol{e}_{k + 1})+c_2\psi(\boldsymbol{e}_{k + 2})+\cdots + c_k\psi(\boldsymbol{e}_{2k})=\boldsymbol{0}.
\]
注意到\(\boldsymbol{e}_1=\varphi\psi(\boldsymbol{e}_1)+\psi\varphi(\boldsymbol{e}_1)=\psi(\boldsymbol{e}_{k + 1})\),同理\(\boldsymbol{e}_2=\psi(\boldsymbol{e}_{k + 2}),\cdots,\boldsymbol{e}_k=\psi(\boldsymbol{e}_{2k})\). 因此上式就是
\[
c_1\boldsymbol{e}_1+c_2\boldsymbol{e}_2+\cdots + c_k\boldsymbol{e}_k=\boldsymbol{0}.
\]
而\(\boldsymbol{e}_1,\boldsymbol{e}_2,\cdots,\boldsymbol{e}_k\)线性无关,故\(c_1 = c_2=\cdots = c_k = 0\),即向量组\(\boldsymbol{e}_{k + 1},\boldsymbol{e}_{k + 2},\cdots,\boldsymbol{e}_{2k}\)线性无关. 特别地,我们有\(\dim\text{Ker}\varphi\geq k=\dim\text{Ker}\psi\). 由于\(\varphi,\psi\)的地位是对称的,故同理可证\(\dim\text{Ker}\psi\geq\dim\text{Ker}\varphi\),从而\(\dim\text{Ker}\varphi=\dim\text{Ker}\psi = k\),并且\(\boldsymbol{e}_{k + 1},\boldsymbol{e}_{k + 2},\cdots,\boldsymbol{e}_{2k}\)是\(\text{Ker}\varphi\)的一组基. 因为\(V=\text{Ker}\varphi\oplus\text{Ker}\psi\),故\(\boldsymbol{e}_1,\cdots,\boldsymbol{e}_k,\boldsymbol{e}_{k + 1},\cdots,\boldsymbol{e}_{2k}\)组成\(V\)的一组基. 现将基向量排列如下:
\[
\boldsymbol{e}_1,\boldsymbol{e}_{k + 1},\boldsymbol{e}_2,\boldsymbol{e}_{k + 2},\cdots,\boldsymbol{e}_k,\boldsymbol{e}_{2k}.
\]
不难验证,在这组基下\(\varphi,\psi\)的表示矩阵即为所求.
\end{enumerate}
\end{proof}


\end{document}
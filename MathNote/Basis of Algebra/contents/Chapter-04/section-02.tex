\documentclass[../../main.tex]{subfiles}
\graphicspath{{\subfix{../../image/}}} % 指定图片目录,后续可以直接使用图片文件名。

% 例如:
% \begin{figure}[H]
% \centering
% \includegraphics[scale=0.3]{image-01.01}
% \caption{图片标题}
% \label{figure:image-01.01}
% \end{figure}
% 注意:上述\label{}一定要放在\caption{}之后,否则引用图片序号会只会显示??.

\begin{document}

\section{线性映射及其运算}

在许多问题中,常常需要定义向量空间之间的线性映射(或某一向量空间上的线性变换). 一般来说,无须对向量空间中的每个元素进行定义,我们可采用下列两种方法来简化定义:第一,只要对向量空间的基向量进行定义即可;第二,若向量空间可分解为两个(或多个)子空间的直和,则只要对每个子空间进行定义即可.这两点可由下面两个定理(\hyperref[theorem:线性扩张定理]{定理\ref{theorem:线性扩张定理}}和定理\hyperref[theorem:利用直和构造线性映射]{定理\ref
{theorem:利用直和构造线性映射}})得到.

\begin{theorem}[线性扩张定理]\label{theorem:线性扩张定理}
设\(V\)和\(U\)是数域\(\mathbb{F}\)上的向量空间,\(\boldsymbol{e}_1,\boldsymbol{e}_2,\cdots,\boldsymbol{e}_n\)是\(V\)的一组基,\(\boldsymbol{u}_1,\boldsymbol{u}_2,\cdots,\boldsymbol{u}_n\)是\(U\)中\(n\)个向量,求证:存在唯一的\(V\)到\(U\)的线性映射\(\varphi\),使得\(\varphi(\boldsymbol{e}_i)=\boldsymbol{u}_i\).
\end{theorem}
\begin{remark}
这个线性扩张定理表明只要选定\(V\)的一组基和\(U\)中\(n\)个向量,则有且仅有一个线性映射将基向量映到对应的向量. 后面我们将经常采用线性扩张定理来构造线性映射以及判定两个线性映射是否相等.
\end{remark}
\begin{note}
这个线性扩张定理表明:

1.在向量空间上定义线性映射只要对向量空间的基向量进行定义即可.

2.判定两个线性映射是否相等只要判断这个两个线性映射原像空间的基向量的像是否相等即可.
\end{note}
\begin{proof}
先证存在性. 对任意的\(\boldsymbol{\alpha}\in V\),设\(\boldsymbol{\alpha}=a_1\boldsymbol{e}_1 + a_2\boldsymbol{e}_2+\cdots + a_n\boldsymbol{e}_n\),则\(a_1,a_2,\cdots,a_n\)被\(\boldsymbol{\alpha}\)唯一确定. 令
\[
\varphi(\boldsymbol{\alpha})=a_1\boldsymbol{u}_1 + a_2\boldsymbol{u}_2+\cdots + a_n\boldsymbol{u}_n,
\]
则\(\varphi\)是\(V\)到\(U\)的映射. 若另有\(\boldsymbol{\beta}=b_1\boldsymbol{e}_1 + b_2\boldsymbol{e}_2+\cdots + b_n\boldsymbol{e}_n\),则
\[
\varphi(\boldsymbol{\alpha}+\boldsymbol{\beta})=(a_1 + b_1)\boldsymbol{u}_1+(a_2 + b_2)\boldsymbol{u}_2+\cdots+(a_n + b_n)\boldsymbol{u}_n=\varphi(\boldsymbol{\alpha})+\varphi(\boldsymbol{\beta}).
\]
又对\(\mathbb{F}\)中的任意元素\(k\),有
\[
\varphi(k\boldsymbol{\alpha})=ka_1\boldsymbol{u}_1 + ka_2\boldsymbol{u}_2+\cdots + ka_n\boldsymbol{u}_n=k\varphi(\boldsymbol{\alpha}).
\]
因此\(\varphi\)是线性映射,显然它满足\(\varphi(\boldsymbol{e}_i)=\boldsymbol{u}_i\).

设另有\(V\)到\(U\)的线性映射\(\psi\)满足\(\psi(\boldsymbol{e}_i)=\boldsymbol{u}_i\),则对任意的\(\boldsymbol{\alpha}\in V\),有
\begin{align*}
\psi(\boldsymbol{\alpha})&=\psi(a_1\boldsymbol{e}_1 + a_2\boldsymbol{e}_2+\cdots + a_n\boldsymbol{e}_n)\\
&=a_1\psi(\boldsymbol{e}_1)+a_2\psi(\boldsymbol{e}_2)+\cdots + a_n\psi(\boldsymbol{e}_n)\\
&=a_1\boldsymbol{u}_1 + a_2\boldsymbol{u}_2+\cdots + a_n\boldsymbol{u}_n=\varphi(\boldsymbol{\alpha}).
\end{align*}
因此\(\psi = \varphi\),这就证明了唯一性. 
\end{proof}

\begin{theorem}\label{theorem:利用直和构造线性映射}
设线性空间\(V = V_1\oplus V_2\),并且\(\varphi_1\)及\(\varphi_2\)分别是\(V_1,V_2\)到\(U\)的线性映射,求证:存在唯一的从\(V\)到\(U\)的线性映射\(\varphi\),当\(\varphi\)限制在\(V_i\)上时等于\(\varphi_i\).
\end{theorem}
\begin{remark}
\hyperref[theorem:利用直和构造线性映射]{定理\ref{theorem:利用直和构造线性映射}}可以推广到多个子空间的情形:设\(V = V_1\oplus\cdots\oplus V_m\),给定线性映射\(\varphi_i:V_i\to U(1\leq i\leq m)\),则存在唯一的线性映射\(\varphi:V\to U\),使得\(\left.\varphi\right|_{V_i}=\varphi_i(1\leq i\leq m)\). 我们可以把这样的线性映射\(\varphi\)简记为\(\varphi_1\oplus\cdots\oplus\varphi_m\).
\end{remark}
\begin{note}
这个定理表明:在向量空间上定义线性映射,若这个向量空间可分解为两个(或多个)子空间的直和,则只要对每个子空间进行定义即可.
\end{note}
\begin{proof}
因为\(V = V_1\oplus V_2\),故对任意的\(\boldsymbol{\alpha}\in V\),\(\boldsymbol{\alpha}\)可唯一地写为\(\boldsymbol{\alpha}=\boldsymbol{\alpha}_1+\boldsymbol{\alpha}_2\),其中\(\boldsymbol{\alpha}_1\in V_1,\boldsymbol{\alpha}_2\in V_2\). 令\(\varphi(\boldsymbol{\alpha})=\varphi_1(\boldsymbol{\alpha}_1)+\varphi_2(\boldsymbol{\alpha}_2)\),则\(\varphi\)是\(V\)到\(U\)的映射. 不难验证\(\varphi\)保持加法和数乘,因此\(\varphi\)是线性映射. 若另有线性映射\(\psi\),它在\(V_i\)上的限制等于\(\varphi_i\),则
\[
\psi(\boldsymbol{\alpha})=\psi(\boldsymbol{\alpha}_1)+\psi(\boldsymbol{\alpha}_2)=\varphi_1(\boldsymbol{\alpha}_1)+\varphi_2(\boldsymbol{\alpha}_2)=\varphi(\boldsymbol{\alpha}).
\]
因此\(\psi = \varphi\),唯一性得证.
\end{proof}

\begin{example}
设\(\varphi\)是有限维线性空间\(V\)到\(U\)的线性映射,求证:必存在\(U\)到\(V\)的线性映射\(\psi\),使得\(\varphi\psi\varphi=\varphi\).
\end{example}
\begin{note}
\hypertarget{可以定义线性映射的原因}{可以直接定义\(\psi\)是\(U\)到\(V\)的线性映射的原因}:由\hyperref[theorem:线性扩张定理]{线性扩张定理}可知存在唯一的线性映射\(\psi\),使得它在基上的作用为
\[
\psi(\boldsymbol{f}_i)=\boldsymbol{e}_i,1\leq i\leq r;\psi(\boldsymbol{f}_j)=\boldsymbol{0},r + 1\leq j\leq m.
\]
以后这种利用\hyperref[theorem:线性扩张定理]{线性扩张定理}得到线性映射的存在性不再额外说明,而是直接定义.
\end{note}
\begin{proof}
{\color{blue}证法一:}
设\(V\)和\(U\)的维数分别是\(n\)和\(m\). 由\hyperref[proposition:线性映射在基下的矩阵为标准型]{命题\ref{proposition:线性映射在基下的矩阵为标准型}}可知,存在\(V\)和\(U\)的基\(\{\boldsymbol{e}_1,\boldsymbol{e}_2,\cdots,\boldsymbol{e}_n\},\{\boldsymbol{f}_1,\boldsymbol{f}_2,\cdots,\boldsymbol{f}_m\}\),使得\(\varphi\)在这两组基下的表示矩阵为
\[
\begin{pmatrix}
\boldsymbol{I}_r&\boldsymbol{O}\\
\boldsymbol{O}&\boldsymbol{O}
\end{pmatrix}.
\]
这就是\(\varphi(\boldsymbol{e}_i)=\boldsymbol{f}_i,1\leq i\leq r;\varphi(\boldsymbol{e}_j)=\boldsymbol{0},r + 1\leq j\leq n\).\hyperlink{可以定义线性映射的原因}{定义\(\psi\)是\(U\)到\(V\)的线性映射},它在基上的作用为
\[
\psi(\boldsymbol{f}_i)=\boldsymbol{e}_i,1\leq i\leq r;\psi(\boldsymbol{f}_j)=\boldsymbol{0},r + 1\leq j\leq m,
\]
则在\(V\)的基上,有
\begin{align*}
\varphi\psi\varphi(\boldsymbol{e}_i)&=\varphi\psi(\boldsymbol{f}_i)=\varphi(\boldsymbol{e}_i),1\leq i\leq r;\\
\varphi\psi\varphi(\boldsymbol{e}_j)&=\varphi\psi(\boldsymbol{0})=\boldsymbol{0}=\varphi(\boldsymbol{e}_j),r + 1\leq j\leq n.
\end{align*}
于是\(\varphi\psi\varphi=\varphi\).

{\color{blue}证法二(代数方法):}取定\(V\)和\(U\)的两组基,设\(\varphi\)在这两组基下的表示矩阵为\(m\times n\)矩阵\(\boldsymbol{A}\),则由\hyperref[example:3.26111]{例题\ref{example:3.26111}}可知,存在\(n\times m\)矩阵\(\boldsymbol{B}\),使得\(\boldsymbol{A}\boldsymbol{B}\boldsymbol{A}=\boldsymbol{A}\). 由矩阵\(\boldsymbol{B}\)可定义从\(U\)到\(V\)的线性映射\(\psi\),它适合\(\varphi\psi\varphi=\varphi\).
\end{proof}

\begin{proposition}
设有数域\(\mathbb{F}\)上的有限维线性空间\(V,V'\),又\(U\)是\(V\)的子空间,\(\varphi\)是\(U\)到\(V'\)的线性映射. 求证:必存在\(V\)到\(V'\)的线性映射\(\psi\),它在\(U\)上的限制就是\(\varphi\).
\end{proposition}
\begin{note}
\hypertarget{直接定义\(\psi\)为\(V\)到\(V'\)的线性映射的原因}{可以直接定义\(\psi\)为\(V\)到\(V'\)的线性映射的原因}:由\hyperref[theorem:利用直和构造线性映射]{定理\ref{theorem:利用直和构造线性映射}}可知,存在唯一的从\(V\)到\(V'\)的线性映射\(\psi\),使得它在\(U\)上的限制是\(\varphi\),它在\(W\)上的限制是零线性映射.

以后这种利用\hyperref[theorem:利用直和构造线性映射]{定理\ref{theorem:利用直和构造线性映射}}得到线性映射的存在性不再额外说明,而是直接定义.
\end{note}
\begin{proof}
令\(W\)是子空间\(U\)在\(V\)中的补空间,即\(V = U\oplus W\). \hypertarget{直接定义\(\psi\)为\(V\)到\(V'\)的线性映射的原因}{定义\(\psi\)为\(V\)到\(V'\)的线性映射},它在\(U\)上的限制是\(\varphi\),它在\(W\)上的限制是零线性映射,这样的\(\psi\)即为所求.     
\end{proof}

\begin{proposition}\label{proposition:线性映射是单射或满射的充要条件1}
设\(V,U\)是\(\mathbb{F}\)上的有限维线性空间,\(\varphi\)是\(V\)到\(U\)的线性映射,求证:
\begin{enumerate}[(1)]
\item  \(\varphi\)是单映射的充要条件是存在\(U\)到\(V\)的线性映射\(\psi\),使\(\psi\varphi = \text{Id}_V\),这里\(\text{Id}_V\)表示\(V\)上的恒等映射;

\item \(\varphi\)是满映射的充要条件是存在\(U\)到\(V\)的线性映射\(\eta\),使\(\varphi\eta = \text{Id}_U\),这里\(\text{Id}_U\)表示\(U\)上的恒等映射.
\end{enumerate}
\end{proposition}
\begin{proof}
{\color{blue}证法一:}
\begin{enumerate}[(1)]
\item 若\(\psi\varphi = \text{Id}_V\),则对任意的\(\boldsymbol{v}\in\text{Ker}\varphi\),\(\boldsymbol{v}=\psi(\varphi(\boldsymbol{v})) = \boldsymbol{0}\),即\(\text{Ker}\varphi = 0\),从而\(\varphi\)是单映射. 

反之,若\(\varphi\)是单映射,则定义映射\(\varphi_1:V\to\text{Im}\varphi\),它与\(\varphi\)有相同的映射法则,但值域变为\(\text{Im}\varphi\). 
因为\(\varphi_1\)与\(\varphi\)有相同的映射法则,所以对\(\forall \alpha \in \mathrm{Ker}\varphi_1\),有\(\varphi(\alpha)=\varphi_1(\alpha)=0\).于是\(\alpha \in \mathrm{Ker}\varphi\),故\(\mathrm{Ker}\varphi_1\subset \mathrm{Ker}\varphi\).又由\(\varphi\)是单射可知,\(\mathrm{Ker}\varphi = 0\).因此\(\mathrm{Ker}\varphi_1 = 0\),即\(\varphi_1\)也是单射.
因为\(\varphi_1\)与\(\varphi\)有相同的映射法则,所以对\(\forall \beta \in \mathrm{Im}\varphi\),存在\(b \in V\),使得\(\beta = \varphi(b)=\varphi_1(b) \in \mathrm{Im}\varphi_1\).故\(\mathrm{Im}\varphi\subset \mathrm{Im}\varphi_1\).又根据\(\varphi_1\)的定义可知,\(\mathrm{Im}\varphi_1\subset \mathrm{Im}\varphi\).因此\(\mathrm{Im}\varphi = \mathrm{Im}\varphi_1\),即\(\varphi_1\)是满射.
故\(\varphi_1\)是双射.由于\(\varphi_1\)与\(\varphi\)有相同的映射法则,因此容易验证\(\varphi_1\)是线性映射.综上,\(\varphi_1\)是线性同构.

设\(U_0\)是\(\text{Im}\varphi\)在\(U\)中的补空间,即\(U = \text{Im}\varphi\oplus U_0\). 定义\(\psi\)为\(U\)到\(V\)的线性映射,它在\(\text{Im}\varphi\)上的限制为\(\varphi_1^{-1}\),它在\(U_0\)上的限制是零线性映射,则容易验证\(\psi\varphi = \text{Id}_V\)成立.

\item 若\(\varphi\eta = \text{Id}_U\),则对任意的\(\boldsymbol{u}\in U\),\(\boldsymbol{u}=\varphi(\eta(\boldsymbol{u}))\),从而\(\varphi\)是满映射. 

反之,若\(\varphi\)是满映射,则可取\(U\)的一组基\(\boldsymbol{f}_1,\boldsymbol{f}_2,\cdots,\boldsymbol{f}_m\),一定存在\(V\)中的向量\(\boldsymbol{v}_1,\boldsymbol{v}_2,\cdots,\boldsymbol{v}_m\),使得\(\varphi(\boldsymbol{v}_i)=\boldsymbol{f}_i(1\leq i\leq m)\). 定义\(\eta\)为\(U\)到\(V\)的线性映射,它在基上的作用为\(\eta(\boldsymbol{f}_i)=\boldsymbol{v}_i(1\leq i\leq m)\),则容易验证\(\varphi\eta = \text{Id}_U\)成立.
\end{enumerate}
{\color{blue}证法二(代数方法):}

充分性同{\color{blue}证法一},现只证必要性. 取定\(V\)和\(U\)的两组基,设\(\varphi\)在这两组基下的表示矩阵为\(m\times n\)矩阵\(\boldsymbol{A}\).
\begin{enumerate}[(1)]
\item 若\(\varphi\)是单映射,则由\hyperref[proposition:行/列满秩矩阵对应满/单射]{命题\ref{proposition:行/列满秩矩阵对应满/单射}}可知\(\boldsymbol{A}\)是列满秩矩阵. 再由\hyperref[proposition:行/列满秩矩阵性质]{命题\ref{proposition:行/列满秩矩阵性质}(1)}可知,存在\(n\times m\)矩阵\(\boldsymbol{B}\),使得\(\boldsymbol{B}\boldsymbol{A}=\boldsymbol{I}_n\). 由矩阵\(\boldsymbol{B}\)可定义从\(U\)到\(V\)的线性映射\(\psi\),它适合\(\psi\varphi=\text{Id}_V\).

\item 若\(\varphi\)是满映射,则由\hyperref[proposition:行/列满秩矩阵对应满/单射]{命题\ref{proposition:行/列满秩矩阵对应满/单射}}可知\(\boldsymbol{A}\)是行满秩矩阵. 再由\hyperref[proposition:行/列满秩矩阵性质]{命题\ref{proposition:行/列满秩矩阵性质}(2)}可知,存在\(n\times m\)矩阵\(\boldsymbol{C}\),使得\(\boldsymbol{A}\boldsymbol{C}=\boldsymbol{I}_m\). 由矩阵\(\boldsymbol{C}\)可定义从\(U\)到\(V\)的线性映射\(\eta\),它适合\(\varphi\eta=\text{Id}_U\).
\end{enumerate}
\end{proof}

\begin{proposition}\label{proposition:核空间和值域与线性映射}
\begin{enumerate}[(1)]
\item 设 \(V,U\) 是数域 \(\mathbb{K}\) 上的有限维线性空间,\(\varphi,\psi:V\to U\) 是两个线性映射,证明:存在 \(U\) 上的线性变换 \(\xi\),使得 \(\psi = \xi\varphi\) 成立的充要条件是 \(\text{Ker}\varphi\subseteq\text{Ker}\psi\).

\item 设 \(V,U\) 是数域 \(\mathbb{K}\) 上的有限维线性空间,\(\varphi,\psi:V\to U\) 是两个线性映射,证明:存在 \(V\) 上的线性变换 \(\xi\),使得 \(\psi = \varphi\xi\) 成立的充要条件是 \(\text{Im}\psi\subseteq\text{Im}\varphi\).
\end{enumerate}
\end{proposition}
\begin{proof}
\begin{enumerate}[(1)]
\item 先证必要性:任取 \(\boldsymbol{v}\in\text{Ker}\varphi\),则 \(\psi(\boldsymbol{v}) = \xi\varphi(\boldsymbol{v}) = \boldsymbol{0}\),即有 \(\boldsymbol{v}\in\text{Ker}\psi\),从而 \(\text{Ker}\varphi\subseteq\text{Ker}\psi\). 

再证充分性:设 \(\dim V = n,\dim U = m,\dim\text{Ker}\varphi = n - r\). 取 \(\text{Ker}\varphi\) 的一组基 \(\boldsymbol{e}_{r + 1},\cdots,\boldsymbol{e}_n\),扩张为 \(V\) 的一组基 \(\boldsymbol{e}_1,\cdots,\boldsymbol{e}_r,\boldsymbol{e}_{r + 1},\cdots,\boldsymbol{e}_n\). 由\hyperref[corollary:由核的基导出值域的基]{推论\ref{corollary:由核的基导出值域的基}}可知,\(\varphi(\boldsymbol{e}_1),\cdots,\varphi(\boldsymbol{e}_r)\) 是 \(\text{Im}\varphi\) 的一组基,将其扩张为 \(U\) 的一组基 \(\varphi(\boldsymbol{e}_1),\cdots,\varphi(\boldsymbol{e}_r),\boldsymbol{g}_{r + 1},\cdots,\boldsymbol{g}_m\). 定义 \(\xi\) 为 \(U\) 上的线性变换,它在基上的作用为:\(\xi(\varphi(\boldsymbol{e}_i)) = \psi(\boldsymbol{e}_i)(1\leq i\leq r),\xi(\boldsymbol{g}_j) = \boldsymbol{0}(r + 1\leq j\leq m)\). 由于 \(\text{Ker}\varphi\subseteq\text{Ker}\psi\),故容易验证 \(\psi(\boldsymbol{e}_i) = \xi\varphi(\boldsymbol{e}_i)(1\leq i\leq n)\) 成立,从而 \(\psi = \xi\varphi\).

\item 先证必要性:任取 \(\boldsymbol{v}\in V\),则 \(\psi(\boldsymbol{v}) = \varphi(\xi(\boldsymbol{v}))\in\text{Im}\varphi\),从而 \(\text{Im}\psi\subseteq\text{Im}\varphi\). 

再证充分性:取 \(V\) 的一组基 \(\boldsymbol{e}_1,\boldsymbol{e}_2,\cdots,\boldsymbol{e}_n\),则 \(\psi(\boldsymbol{e}_i)\in\text{Im}\psi\subseteq\text{Im}\varphi\),从而存在 \(\boldsymbol{f}_i\in V\),使得 \(\varphi(\boldsymbol{f}_i) = \psi(\boldsymbol{e}_i)(1\leq i\leq n)\). 定义 \(\xi\) 为 \(V\) 上的线性变换,它在基上的作用为:\(\xi(\boldsymbol{e}_i) = \boldsymbol{f}_i(1\leq i\leq n)\). 容易验证 \(\psi(\boldsymbol{e}_i) = \varphi\xi(\boldsymbol{e}_i)(1\leq i\leq n)\) 成立,从而 \(\psi = \varphi\xi\).
\end{enumerate}
\end{proof}

\begin{proposition}\label{proposition:两个线性变换的值域相等的充要条件}
设\(V\)是有限维线性空间,\(T_2,T_1\)是\(V\)上的线性变换,证明\(\text{Im }T_2 = \text{Im }T_1\)的充分必要条件是存在\(V\)上的可逆变换\(S\)使得\(T_2 = T_1S\)。 
\end{proposition}
\begin{proof}
{\color{blue}证法一:}
任取$V$中的一组基$\{ e_1,e_2,\cdots ,e_n \}$,记$T_1,T_2$在这组基下的矩阵分别为$T_{1}',T_{2}'$.再记$T_{1}'$的分块列向量为$(\alpha _1,\alpha _2,\cdots ,\alpha _n)$,$T_{2}'$的分块列向量为$(\beta _1,\beta _2,\cdots ,\beta _n)$.

充分性:记可逆变换$S$在基$\{ e_1,e_2,\cdots ,e_n \}$下的矩阵为$S'$,则此时就有
\begin{align*}
T_{2}'=T_{1}'S',\quad T_{1}'=T_{2}'(S')^{-1}.
\end{align*}
因此$T_{1}'$和$T_{2}'$的列向量组$(\alpha _1,\alpha _2,\cdots ,\alpha _n)$和$(\beta _1,\beta _2,\cdots ,\beta _n)$可以相互线性表出,即$(\alpha _1,\alpha _2,\cdots ,\alpha _n)$和$(\beta _1,\beta _2,\cdots ,\beta _n)$等价.故由\refpro{proposition:等价向量组张成的子空间相同}可知
\begin{align*}
L(\alpha _1,\alpha _2,\cdots ,\alpha _n) =L(\beta _1,\beta _2,\cdots ,\beta _n).
\end{align*}
又由\refcor{corollary:线性映射与矩阵基本结论}可知
\begin{align*}
L(\alpha _1,\alpha _2,\cdots ,\alpha _n) \cong \mathrm{Im}T_1,\quad L(\beta _1,\beta _2,\cdots ,\beta _n) =\mathrm{Im}T_2.
\end{align*}
故$\mathrm{Im}T_1=\mathrm{Im}T_2$.

必要性:若$\mathrm{Im}T_1=\mathrm{Im}T_2$,又由\refcor{corollary:线性映射与矩阵基本结论}可知
\begin{align*}
L(\alpha _1,\alpha _2,\cdots ,\alpha _n) \cong \mathrm{Im}T_1,\quad L(\beta _1,\beta _2,\cdots ,\beta _n) =\mathrm{Im}T_2.
\end{align*}
故$L(\alpha _1,\alpha _2,\cdots ,\alpha _n) =L(\beta _1,\beta _2,\cdots ,\beta _n)$.再由\refpro{proposition:等价向量组张成的子空间相同}可知,$(\alpha _1,\alpha _2,\cdots ,\alpha _n)$和$(\beta _1,\beta _2,\cdots ,\beta _n)$等价.故$(\beta _1,\beta _2,\cdots ,\beta _n)$可以由$(\alpha _1,\alpha _2,\cdots ,\alpha _n)$线性表出,即
\begin{align*}
(\beta _1,\beta _2,\cdots ,\beta _n) = \left( \sum_{i=1}^n{a_{i1}\alpha _i},\sum_{i=1}^n{a_{i2}\alpha _i},\cdots ,\sum_{i=1}^n{a_{in}\alpha _i} \right),
\end{align*}
其中对$\forall j\in \{ 1,2,\cdots ,n \}$,都有$a_{1j},a_{2j},\cdots ,a_{nj}$不全为$0$.于是由上式可知$(\beta _1,\beta _2,\cdots ,\beta _n)$可由$(\alpha _1,\alpha _2,\cdots ,\alpha _n)$做一系列初等列变换得到.因此存在可逆矩阵$S'$,使得
\begin{align*}
(\beta _1,\beta _2,\cdots ,\beta _n) = (\alpha _1,\alpha _2,\cdots ,\alpha _n)S'.
\end{align*}
即$T_{2}'=T_{1}'S'$.记由$S'$乘法诱导的线性变换为$S$,则由$S'$可逆可得$S$也可逆,并且$T_2=T_1S$.

{\color{blue}证法二:}
先证充分性。设\(T_1(\alpha)\in \mathrm{Im}T_1\),则
\begin{align*}
T_1(\alpha) = T_2S^{-1}(\alpha)=T_2(S^{-1}(\alpha))\in \mathrm{Im}T_2.
\end{align*}
故\(\mathrm{Im}T_1\subset \mathrm{Im}T_2\)。再设\(T_2(\alpha)\in \mathrm{Im}T_2\),则
\begin{align*}
T_2(\alpha) = T_1S(\alpha)=T_1(S(\alpha))\in \mathrm{Im}T_1.
\end{align*}
故\(\mathrm{Im}T_1\supset \mathrm{Im}T_2\)。因此\(\mathrm{Im}T_1=\mathrm{Im}T_2\)。

再证必要性。取\(\mathrm{Im}T_1=\mathrm{Im}T_2\)的一组基\(\{e_1,e_2,\cdots,e_r\}\),将其扩充成\(V\)的一组基\(\{e_1,\cdots,e_r,e_{r + 1},\cdots,e_n\}\)。
任取\(\alpha\in V\),则
\begin{align*}
T_1(\alpha)\in \mathrm{Im}T_1 = L(e_1,e_2,\cdots,e_r),\\
T_2(\alpha)\in \mathrm{Im}T_2 = L(e_1,e_2,\cdots,e_r).
\end{align*}
于是存在两组不全为零的数\(a_1,a_2,\cdots,a_r\)和\(b_1,b_2,\cdots,b_n\),使得
\begin{align*}
T_1(\alpha)=(e_1,e_2,\cdots,e_n)\left( \begin{array}{c}
a_1\\
a_2\\
\vdots\\
a_r\\
0\\
\vdots\\
0\\
\end{array} \right),\quad 
T_2(\alpha)=(e_1,e_2,\cdots,e_n)\left( \begin{array}{c}
b_1\\
b_2\\
\vdots\\
b_r\\
0\\
\vdots\\
0\\
\end{array} \right).
\end{align*}
不妨设\(a_1,b_1\ne 0\),则取\(n\)阶矩阵\(S_{1}^{\prime}=\left( \begin{matrix}
S&O\\
O&I_{n - r}\\
\end{matrix} \right)\),其中\(S=\left( \begin{matrix}
\frac{a_1}{b_1}& & & \\
\frac{a_2}{b_1}&1& & \\
\vdots& &\ddots& \\
\frac{a_r}{b_1}& & &1\\
\end{matrix} \right)\)。
显然矩阵\(S_{1}^{\prime}\)非异。记由矩阵\(S_{1}^{\prime}\)乘法诱导的\(V\)上的线性变换为\(S_1\),则由矩阵\(S_{1}^{\prime}\)非异可知,线性变换\(S_1\)可逆。此时
\begin{align*}
S_1T_2(\alpha)=(e_1,e_2,\cdots,e_n)S_{1}^{\prime}\left( \begin{array}{c}
b_1\\
b_2\\
\vdots\\
b_r\\
0\\
\vdots\\
0\\
\end{array} \right)=(e_1,e_2,\cdots,e_n)\left( \begin{matrix}
S&O\\
O&I_{n - r}\\
\end{matrix} \right)\left( \begin{array}{c}
b_1\\
b_2\\
\vdots\\
b_r\\
0\\
\vdots\\
0\\
\end{array} \right)=(e_1,e_2,\cdots,e_n)\left( \begin{array}{c}
a_1\\
a_2\\
\vdots\\
a_r\\
0\\
\vdots\\
0\\
\end{array} \right)=T_1(\alpha).
\end{align*}
故再由\(\alpha\)的任意性可知,\(S_1T_2 = T_1\)。又\(S_1\)可逆,故\(T_2 = S_{1}^{-1}T_1\),因此令线性变换\(S = S_{1}^{-1}\)即可。
\end{proof}


\begin{proposition}\label{proposition:幂零线性变换的一组基}
设\(\varphi\)是\(n\)维线性空间\(V\)上的线性变换,\(\boldsymbol{\alpha}\in V\). 若\(\varphi^{m - 1}(\boldsymbol{\alpha})\neq\boldsymbol{0}\),而\(\varphi^{m}(\boldsymbol{\alpha})=\boldsymbol{0}\),求证:\(\boldsymbol{\alpha},\varphi(\boldsymbol{\alpha}),\varphi^{2}(\boldsymbol{\alpha}),\cdots,\varphi^{m - 1}(\boldsymbol{\alpha})\)线性无关.
\end{proposition}
\begin{proof}
设有\(m\)个数\(a_0,a_1,\cdots,a_{m - 1}\),使得
\[
a_0\boldsymbol{\alpha}+a_1\varphi(\boldsymbol{\alpha})+\cdots+a_{m - 1}\varphi^{m - 1}(\boldsymbol{\alpha})=\boldsymbol{0}.
\]
上式两边同时作用\(\varphi^{m - 1}\),则有\(a_0\varphi^{m - 1}(\boldsymbol{\alpha})=\boldsymbol{0}\),由于\(\varphi^{m - 1}(\boldsymbol{\alpha})\neq\boldsymbol{0}\),故\(a_0 = 0\). 上式两边同时作用\(\varphi^{m - 2}\),则有\(a_1\varphi^{m - 1}(\boldsymbol{\alpha})=\boldsymbol{0}\),由于\(\varphi^{m - 1}(\boldsymbol{\alpha})\neq\boldsymbol{0}\),故\(a_1 = 0\). 不断这样做下去,最后可得\(a_0 = a_1=\cdots=a_{m - 1}=0\),于是\(\boldsymbol{\alpha},\varphi(\boldsymbol{\alpha}),\varphi^{2}(\boldsymbol{\alpha}),\cdots,\varphi^{m - 1}(\boldsymbol{\alpha})\)线性无关.
\end{proof}

\begin{corollary}\label{corollary:幂零线性变换基的表示矩阵}
设\(V\)是数域\(\mathbb{K}\)上的\(n\)维线性空间,\(\varphi\)是\(V\)上的幂零线性变换,满足\(\mathrm{r}(\varphi)=n - 1\). 求证:存在\(V\)的一组基,使得\(\varphi\)在这组基下的表示矩阵为
\[
\boldsymbol{A}=\begin{pmatrix}
0&0&\cdots&0&0\\
1&0&\cdots&0&0\\
0&1&\cdots&0&0\\
\vdots&\vdots&&\vdots&\vdots\\
0&0&\cdots&1&0
\end{pmatrix}.
\]
\end{corollary}
\begin{proof}
由假设存在正整数\(m\),使得\(\varphi^{m}=0,\varphi^{m - 1}\neq 0\),从而存在\(\boldsymbol{\alpha}\in V\),使得\(\varphi^{m}(\boldsymbol{\alpha}) = 0,\varphi^{m - 1}(\boldsymbol{\alpha})\neq 0\). 由\hyperref[proposition:幂零线性变换的一组基]{命题\ref{proposition:幂零线性变换的一组基}}可知,\(\boldsymbol{\alpha},\varphi(\boldsymbol{\alpha}),\cdots,\varphi^{m - 1}(\boldsymbol{\alpha})\)线性无关,于是\(m\leq\dim V=n\). 另一方面,由\hyperref[proposition:Sylvester不等式]{Sylvester不等式}以及\(\mathrm{r}(\varphi)=n - 1\)可知,\(\mathrm{r}(\varphi^{2})\geq 2\mathrm{r}(\varphi)-n=n - 2\). 不断这样讨论下去,最终可得\(0=\mathrm{r}(\varphi^{m})\geq n - m\),即有\(m\geq n\),从而\(m = n\). 于是\(\boldsymbol{\alpha},\varphi(\boldsymbol{\alpha}),\cdots,\varphi^{n - 1}(\boldsymbol{\alpha})\)是\(V\)的一组基,\(\varphi\)在这组基下的表示矩阵为\(\boldsymbol{A}\).
\end{proof}


\end{document}
\documentclass[../../main.tex]{subfiles}
\graphicspath{{\subfix{../../image/}}} % 指定图片目录,后续可以直接使用图片文件名。

% 例如:
% \begin{figure}[H]
% \centering
% \includegraphics[scale=0.3]{image-01.01}
% \caption{图片标题}
% \label{figure:image-01.01}
% \end{figure}
% 注意:上述\label{}一定要放在\caption{}之后,否则引用图片序号会只会显示??.

\begin{document}

\section{线性同构}
线性同构刻画了不同线性空间之间的相同本质,即同构的线性空间具有相同的线性结构(或从线性结构的观点来看没有任何区别).要证明线性映射\(\varphi:V\to U\)是线性同构,通常一方面需要验证\(\varphi\)是单映射(或等价地验证\(\text{Ker}\varphi = 0\)),另一方面需要验证\(\varphi\)是满映射(或等价地验证\(\text{Im}\varphi = U\)). 但若已知前后两个线性空间的维数相等,则由线性映射的维数公式容易证明,\(\varphi\)是线性同构当且仅当\(\varphi\)是单映射,也当且仅当\(\varphi\)是满映射,从而只需验证\(\varphi\)是单映射或满映射即可得到\(\varphi\)是线性同构.

\begin{proposition}\label{proposition:像和原像空间维数相同时线性同构的充要条件}
设$V,U$为两个线性空间,若\(\dim V = \dim U\),则线性映射\(\varphi:V\rightarrow U\)是线性同构当且仅当\(\varphi\)是单映射,也当且仅当\(\varphi\)是满映射.
\end{proposition}
\begin{proof}
由\hyperref[proposition:值域和核空间维数之和等于原像空间维数]{线性映射的维数公式}容易证明.
\end{proof}

\begin{corollary}\label{corollary:线性变换自同构的充要条件}
设$V$为线性空间,则线性变换\(\varphi:V\rightarrow V\)是自同构当且仅当\(\varphi\)是单映射,也当且仅当\(\varphi\)是满映射.
\end{corollary}
\begin{proof}
由\hyperref[proposition:像和原像空间维数相同时线性同构的充要条件]{命题\ref{proposition:像和原像空间维数相同时线性同构的充要条件}}立得.
\end{proof}

\begin{lemma}\label{lemma:证明Lagrange插值公式}
设\(a_0,a_1,\cdots,a_n\)是数域\(\mathbb{F}\)中\(n + 1\)个不同的数,\(V\)是\(\mathbb{F}\)上次数不超过\(n\)的多项式全体组成的线性空间. 设\(\varphi\)是\(V\)到\(n + 1\)维行向量空间\(U\)的映射:
\[
\varphi(f)=(f(a_0),f(a_1),\cdots,f(a_n)),
\]
求证:\(\varphi\)是线性同构.
\end{lemma}
\begin{proof}
不难验证\(\varphi\)是一个线性映射. 若\(f(x)\in\text{Ker}\varphi\),则\(f(a_i)=0(0\leq i\leq n)\). 因为\(f(x)\)的次数不超过\(n\),故由\hyperref[proposition:多项式根的有限性]{多项式根的有限性}可知\(f(x)=0\),即\(\text{Ker}\varphi = 0\),这证明了映射\(\varphi\)是单映射. 注意到线性空间\(V\)和\(U\)的维数都等于\(n + 1\),因此由\hyperref[proposition:像和原像空间维数相同时线性同构的充要条件]{命题\ref{proposition:像和原像空间维数相同时线性同构的充要条件}}可知\(\varphi\)是线性同构.
\end{proof}

\begin{theorem}[Lagrange插值公式]\label{theorem:Lagrange插值公式}
设\(a_0,a_1,\cdots,a_n\)是数域\(\mathbb{F}\)中\(n + 1\)个不同的数,\(b_0,b_1,\cdots,b_n\)是\(\mathbb{F}\)中任意\(n + 1\)个数,求证:必存在\(\mathbb{F}\)上次数不超过\(n\)的多项式\(f(x)\),使得\(f(a_i)=b_i(0\leq i\leq n)\),并将\(f(x)\)构造出来.
\end{theorem}
\begin{proof}
由\hyperref[lemma:证明Lagrange插值公式]{引理\ref{lemma:证明Lagrange插值公式}}可知映射\(\varphi\)是映上的(满射),因此存在性已经证明. 现来构造\(f(x)\). 设\(\boldsymbol{e}_i=(0,\cdots,1,\cdots,0)(1\leq i\leq n + 1)\)是\(\mathbb{F}\)上的\(n + 1\)维标准单位行向量. 对任意的\(0\leq i\leq n\),令
\[
f_i(x)=\frac{(x - a_0)\cdots(x - a_{i - 1})(x - a_{i + 1})\cdots(x - a_n)}{(a_i - a_0)\cdots(a_i - a_{i - 1})(a_i - a_{i + 1})\cdots(a_i - a_n)},
\]
则\(f_i(a_i)=1,f_i(a_j)=0(j\neq i)\),于是\(\varphi(f_i)=\boldsymbol{e}_{i + 1}(0\leq i\leq n)\). 再令
\[
f(x)=b_0f_0(x)+b_1f_1(x)+\cdots + b_nf_n(x),
\]
则容易验证\(\varphi(f)=(b_0,b_1,\cdots,b_n)\),即\(f(a_i)=b_i(0\leq i\leq n)\)成立. 
\end{proof}

\subsection{证明线性变换可逆的方法}

要证明某个有限维线性空间\(V\)上的线性变换\(\varphi\)是自同构(可逆线性变换),通常有\(3\)种方法. 一是可尝试直接构造出\(\varphi\)的逆变换. 二是证明\(\varphi\)是单映射或者\(\varphi\)是满映射(两者只需其一)(\hyperref[proposition:像和原像空间维数相同时线性同构的充要条件]{命题\ref{proposition:像和原像空间维数相同时线性同构的充要条件}}). 三是用矩阵方法,即选取\(V\)的一组基,设\(\varphi\)在这组基下的表示矩阵为\(\boldsymbol{A}\),设法证明\(\boldsymbol{A}\)是可逆矩阵. 

对于无限维线性空间之间的线性映射,我们并没有定义表示矩阵这一概念,也没有维数公式等结论,因此研究线性映射或线性变换,无限维线性空间的情形远比有限维线性空间的情形难得多,也常出现对有限维线性空间成立的结论在无限维线性空间却不成立的情况. 例如,要证明无限维线性空间上的线性变换是自同构,只能按照定义证明它既是单映射又是满映射,而不能像有限维线性空间上的线性变换那样,只验证它是单映射或满映射即可.
\begin{example}
设\(\varphi\)是数域\(\mathbb{F}\)上线性空间\(V\)上的线性变换,若存在正整数\(n\)以及\(a_1,a_2,\cdots,a_n\in\mathbb{F}\),使得
\[
\varphi^{n}+a_1\varphi^{n - 1}+\cdots+a_{n - 1}\varphi+a_nI_V = 0,
\]
其中\(I_V\)表示恒等变换并且\(a_n\neq 0\),求证:\(\varphi\)是\(V\)上的自同构.
\end{example}
\begin{proof}
由条件可得
\[
\varphi^{n}+a_1\varphi^{n - 1}+\cdots+a_{n - 1}\varphi=-a_nI_V,
\]
从而
\[
\varphi\left(-\frac{1}{a_n}(\varphi^{n - 1}+\cdots+a_{n - 1}I_V)\right)=I_V,
\]
于是
\[
\varphi^{-1}=-\frac{1}{a_n}(\varphi^{n - 1}+\cdots+a_{n - 1}I_V). 
\]
\end{proof}

\begin{proposition}\label{proposition:线性变换是可逆变换的充要条件1}
设\(\varphi\)是\(n\)维线性空间\(V\)上的线性变换,证明:\(\varphi\)是可逆变换的充要条件是\(\varphi\)将\(V\)的基变为基.
\end{proposition}
\begin{proof}
若\(\varphi\)是可逆变换,则显然\(\varphi\)将\(V\)的基变为基.

反之,{\color{blue}证法一:}
若\(\boldsymbol{e}_1,\boldsymbol{e}_2,\cdots,\boldsymbol{e}_n\)和\(\boldsymbol{f}_1,\boldsymbol{f}_2,\cdots,\boldsymbol{f}_n\)是\(V\)的两组基,使得\(\varphi(\boldsymbol{e}_i)=\boldsymbol{f}_i(1\leq i\leq n)\),则对任意\(\boldsymbol{\alpha}\in V\),\(\boldsymbol{\alpha}=\lambda_1\boldsymbol{f}_1+\lambda_2\boldsymbol{f}_2+\cdots+\lambda_n\boldsymbol{f}_n\),有\(\varphi(\lambda_1\boldsymbol{e}_1+\lambda_2\boldsymbol{e}_2+\cdots+\lambda_n\boldsymbol{e}_n)=\boldsymbol{\alpha}\),即\(\varphi\)是满映射,从而是自同构.(我们也可以证明\(\varphi\)是单映射,从而是自同构.) 

{\color{blue}证法二:}若\(\boldsymbol{e}_1,\boldsymbol{e}_2,\cdots,\boldsymbol{e}_n\)和\(\boldsymbol{f}_1,\boldsymbol{f}_2,\cdots,\boldsymbol{f}_n\)是\(V\)的两组基,使得\(\varphi(\boldsymbol{e}_i)=\boldsymbol{f}_i(1\leq i\leq n)\).设从基\(\boldsymbol{e}_1,\boldsymbol{e}_2,\cdots,\boldsymbol{e}_n\)到基\(\boldsymbol{f}_1,\boldsymbol{f}_2,\cdots,\boldsymbol{f}_n\)的过渡矩阵为\(\boldsymbol{P}\),则\(\varphi\)在基\(\boldsymbol{e}_1,\boldsymbol{e}_2,\cdots,\boldsymbol{e}_n\)下的表示矩阵就是\(\boldsymbol{P}\),这是一个可逆矩阵,从而\(\varphi\)是可逆变换. 
\end{proof}

\begin{proposition}\label{proposition:两个维数相同的线性空间一定可以通过一个可逆线性变换联系起来}
设\(U_1,U_2\)是\(n\)维线性空间\(V\)的子空间,假设它们维数相同. 求证:存在\(V\)上的可逆线性变换\(\varphi\),使得\(U_2 = \varphi(U_1)\).
\end{proposition}
\begin{proof}
取\(U_1\)的一组基\(\boldsymbol{e}_1,\cdots,\boldsymbol{e}_m\),并扩张为\(V\)的一组基\(\boldsymbol{e}_1,\cdots,\boldsymbol{e}_m,\boldsymbol{e}_{m + 1},\cdots,\boldsymbol{e}_n\);取\(U_2\)的一组基\(\boldsymbol{f}_1,\cdots,\boldsymbol{f}_m\),并扩张为\(V\)的一组基\(\boldsymbol{f}_1,\cdots,\boldsymbol{f}_m,\boldsymbol{f}_{m + 1},\cdots,\boldsymbol{f}_n\). 定义\(\varphi\)为\(V\)上的线性变换,它在基上的作用为:\(\varphi(\boldsymbol{e}_i)=\boldsymbol{f}_i(1\leq i\leq n)\),则由\hyperref[proposition:线性变换是可逆变换的充要条件1]{命题\ref{proposition:线性变换是可逆变换的充要条件1}}可知,\(\varphi\)是可逆线性变换,再由定义容易验证\(\varphi(U_1)=U_2\)成立. 
\end{proof}

\begin{example}
设\(\varphi\)是\(n\)维线性空间\(V\)上的线性变换,若对\(V\)中任一向量\(\boldsymbol{\alpha}\),总存在正整数\(m\)(\(m\)可能和\(\boldsymbol{\alpha}\)有关),使得\(\varphi^{m}(\boldsymbol{\alpha}) = 0\). 求证:\(I_V-\varphi\)是自同构.
\end{example}
\begin{proof}
{\color{blue}证法一:}
首先证明线性变换\(\varphi\)是幂零的. 设\(\boldsymbol{e}_1,\boldsymbol{e}_2,\cdots,\boldsymbol{e}_n\)是线性空间\(V\)的一组基. 对每个\(\boldsymbol{e}_i\),都有\(m_i\),使得\(\varphi^{m_i}(\boldsymbol{e}_i)=0\),令\(m\)为诸\(m_i\)中最大者. 对\(V\)中任一向量\(\boldsymbol{v}\),设\(\boldsymbol{v}=a_1\boldsymbol{e}_1 + a_2\boldsymbol{e}_2+\cdots + a_n\boldsymbol{e}_n\),则有
\[
\varphi^{m}(\boldsymbol{v})=a_1\varphi^{m}(\boldsymbol{e}_1)+a_2\varphi^{m}(\boldsymbol{e}_2)+\cdots + a_n\varphi^{m}(\boldsymbol{e}_n)=\boldsymbol{0}.
\]
因此\(\varphi^{m}=0\).

注意到下列等式:
\[
(I_V - \varphi)(I_V+\varphi+\varphi^{2}+\cdots+\varphi^{m - 1})=I_V-\varphi^{m}=I_V.
\]
由此即知\(I_V - \varphi\)是自同构.

{\color{blue}证法二:} 只要证明\(I_V - \varphi\)是单映射即可. 任取\(\boldsymbol{\alpha}\in\text{Ker}(I_V - \varphi)\),即\((I_V - \varphi)(\boldsymbol{\alpha}) = 0\),则\(\varphi(\boldsymbol{\alpha})=\boldsymbol{\alpha}\). 设\(m\)为正整数,使得\(\varphi^{m}(\boldsymbol{\alpha}) = 0\),则\(0=\varphi^{m}(\boldsymbol{\alpha})=\varphi^{m - 1}(\boldsymbol{\alpha})=\cdots=\varphi(\boldsymbol{\alpha})=\boldsymbol{\alpha}\),故\(\text{Ker}(I_V - \varphi)=0\),即\(I_V - \varphi\)是单映射. 
\end{proof}

\begin{example}
设\(V = M_n(\mathbb{F})\)是\(\mathbb{F}\)上\(n\)阶矩阵全体组成的线性空间,\(\boldsymbol{A},\boldsymbol{B}\)是两个\(n\)阶矩阵,定义\(V\)上的变换:\(\varphi(\boldsymbol{X})=\boldsymbol{A}\boldsymbol{X}\boldsymbol{B}\). 求证:\(\varphi\)是\(V\)上的线性变换,\(\varphi\)是可逆变换的充要条件是\(\boldsymbol{A}\)和\(\boldsymbol{B}\)都是可逆矩阵.
\end{example}
\begin{remark}
用\hyperref[proposition:无限维线性空间的可逆线性变换充要条件1]{命题\ref{proposition:无限维线性空间的可逆线性变换充要条件1}}的结论来看这个例题,就能发现\(D\)之所以不是可逆变换,是因为它的右逆变换除了\(S\)之外,还有无穷多个.
\end{remark}
\begin{proof}
容易验证\(\varphi\)是线性变换. 若\(\boldsymbol{A},\boldsymbol{B}\)都是可逆矩阵,则\(\psi(\boldsymbol{X})=\boldsymbol{A}^{-1}\boldsymbol{X}\boldsymbol{B}^{-1}\)是\(\varphi\)的逆线性变换. 下面用两种方法来证明必要性.

{\color{blue}证法一:} 若\(\boldsymbol{A}\)是不可逆矩阵,则我们可证明\(\varphi\)不是单映射,即存在\(\boldsymbol{X}\neq\boldsymbol{O}\),使得\(\varphi(\boldsymbol{X})=\boldsymbol{A}\boldsymbol{X}\boldsymbol{B}=\boldsymbol{O}\),从而\(\varphi\)不是可逆变换. 事实上,若\(\boldsymbol{A}\)的秩等于\(r < n\),则存在可逆矩阵\(\boldsymbol{P}\)和\(\boldsymbol{Q}\),使得\(\boldsymbol{P}\boldsymbol{A}\boldsymbol{Q}=\begin{pmatrix}\boldsymbol{I}_r&\boldsymbol{O}\\\boldsymbol{O}&\boldsymbol{O}\end{pmatrix}\). 令\(\boldsymbol{C}=\begin{pmatrix}\boldsymbol{O}&\boldsymbol{O}\\\boldsymbol{O}&\boldsymbol{I}_{n - r}\end{pmatrix}\),则\(\boldsymbol{P}\boldsymbol{A}\boldsymbol{Q}\boldsymbol{C}=\boldsymbol{O}\),而\(\boldsymbol{P}\)是可逆矩阵,故\(\boldsymbol{A}\boldsymbol{Q}\boldsymbol{C}=\boldsymbol{O}\),再令\(\boldsymbol{X}=\boldsymbol{Q}\boldsymbol{C}\)即可. 同理,若\(\boldsymbol{B}\)的秩小于\(n\),也可以证明\(\varphi\)不是可逆变换.

{\color{blue}证法二:} 若\(\boldsymbol{A}\)是不可逆矩阵,则对任意的\(n\)阶矩阵\(\boldsymbol{X}\),\(\varphi(\boldsymbol{X})=\boldsymbol{A}\boldsymbol{X}\boldsymbol{B}\)总是不可逆矩阵(行列式一定为零). 因此\(\varphi\)不可能是映上的(可逆矩阵不在值域里但是在像空间中). 同理,若\(\boldsymbol{B}\)是不可逆矩阵,\(\varphi\)也不是映上的.
\end{proof}

\begin{example}
设\(V\)是实系数多项式全体构成的实线性空间,定义\(V\)上的变换\(D,S\)如下:
\[
D(f(x))=\frac{\mathrm{d}}{\mathrm{d}x}f(x),\ S(f(x))=\int_{0}^{x}f(t)\mathrm{d}t.
\]
证明:\(D,S\)均为\(V\)上的线性变换且\(DS = I_V\),但\(SD\neq I_V\).
\end{example}
\begin{proof}
简单验证即得结论. 由\(DS = I_V\)可知,\(S\)是单线性映射,\(D\)是满线性映射. 又容易看出\(S\)不是满映射(值域不包含常数),\(D\)不是单映射,从而它们都不是自同构.
\end{proof}

\begin{proposition}\label{proposition:无限维线性空间的可逆线性变换充要条件1}
设\(V\)是\(\mathbb{K}\)上的无限维线性空间,\(\varphi,\psi\)是\(V\)上的线性变换.
\begin{enumerate}[(1)]
\item 证明:\(\varphi\)和\(\psi\)都是可逆变换的充要条件是\(\varphi\psi\)和\(\psi\varphi\)都是可逆变换;

\item 若\(\psi\varphi = I_V\),则称\(\psi\)是\(\varphi\)的左逆变换,\(\varphi\)是\(\psi\)的右逆变换. 证明:\(\varphi\)是可逆变换的充要条件是\(\varphi\)有且仅有一个左逆变换(右逆变换).
\end{enumerate}
\end{proposition}
\begin{note}
这个命题对有限维空间仍成立.
\end{note}
\begin{proof}
\begin{enumerate}[(1)]
\item 若\(\varphi\)和\(\psi\)都是可逆变换,则\((\psi^{-1}\varphi^{-1})(\varphi\psi)=(\varphi\psi)(\psi^{-1}\varphi^{-1}) = I_V\),\((\varphi^{-1}\psi^{-1})(\psi\varphi)=(\psi\varphi)(\varphi^{-1}\psi^{-1}) = I_V\),因此\(\varphi\psi\)和\(\psi\varphi\)都是可逆变换. 反之,若\(\varphi\psi\)和\(\psi\varphi\)都是可逆变换,则存在\(V\)上的线性变换\(\xi,\eta\),使得\(\varphi\psi\xi=\xi\varphi\psi = I_V\),\(\psi\varphi\eta=\eta\psi\varphi = I_V\). 由\(\varphi\psi\xi = I_V\)及\hyperref[proposition:值域和核空间维数之和等于原像空间维数]{命题\ref{proposition:值域和核空间维数之和等于原像空间维数}(2)}可得\(\varphi\)是满映射,由\(\eta\psi\varphi = I_V\)及\hyperref[proposition:值域和核空间维数之和等于原像空间维数]{命题\ref{proposition:值域和核空间维数之和等于原像空间维数}(1)}可得\(\varphi\)是单映射,从而\(\varphi\)是可逆变换. 同理可证\(\psi\)也是可逆变换.

\item 若\(\varphi\)是可逆变换,任取\(\varphi\)的一个左逆变换\(\psi\),则
\[
\psi=\psi I_V=\psi\varphi\varphi^{-1}=I_V\varphi^{-1}=\varphi^{-1},
\]
即\(\varphi\)的任一左逆变换都是逆变换\(\varphi^{-1}\). 由逆变换的唯一性可知,\(\varphi\)有且仅有一个左逆变换. 反之,若\(\varphi\)有且仅有一个左逆变换\(\psi\),则\(\psi\varphi = I_V\),且有
\[
(\psi+\varphi\psi - I_V)\varphi=\psi\varphi+\varphi\psi\varphi-\varphi=I_V+\varphi-\varphi=I_V,
\]
即\(\psi+\varphi\psi - I_V\)也是\(\varphi\)的左逆变换,从而\(\psi+\varphi\psi - I_V=\psi\),即\(\varphi\psi = I_V\). 因此\(\psi\)也是\(\varphi\)的右逆变换,从而\(\varphi\)是可逆变换. 同理可证关于右逆变换的结论.
\end{enumerate}
\end{proof}

\begin{example}
试构造无限维线性空间\(V\)以及\(V\)上的线性变换\(\varphi,\psi\),使得\(\varphi\psi-\psi\varphi = I_V\).
\end{example}
\begin{solution}
设\(V\)是实系数多项式全体构成的实线性空间,线性变换\(\varphi,\psi\)定义为:对任一\(f(x)\in V\),\(\varphi(f(x)) = f^\prime(x)\),\(\psi(f(x)) = xf(x)\). 容易验证\(\varphi\psi-\psi\varphi = I_V\)成立.
\end{solution}
\begin{remark}
事实上,满足上述性质的线性变换\(\varphi,\psi\)绝不可能存在于有限维线性空间\(V\)上. 若存在,取\(V\)的一组基并设\(\varphi,\psi\)的表示矩阵为\(\boldsymbol{A},\boldsymbol{B}\),则有\(\boldsymbol{A}\boldsymbol{B}-\boldsymbol{B}\boldsymbol{A}=\boldsymbol{I}\)成立. 上式两边同时取迹,可得
\[
0=\text{tr}(\boldsymbol{A}\boldsymbol{B}-\boldsymbol{B}\boldsymbol{A})=\text{tr}(\boldsymbol{I})=\dim V,
\]
导出矛盾.
\end{remark}


\end{document}
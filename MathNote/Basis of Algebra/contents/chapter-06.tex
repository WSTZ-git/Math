\documentclass[../main.tex]{subfiles} % 使用 subfiles 文档类,指定主文档
\graphicspath{{\subfix{../image/}}} % 指定图片目录,后续可以直接使用图片文件名。

% 例如:
% \begin{figure}[h]
% \centering
% \includegraphics{image-01.01}
% \caption{图片标题}
% \label{fig:image-01.01}
% \end{figure}
% 注意:上述\label{}一定要放在\caption{}之后,否则引用图片序号会只会显示??.

\begin{document}

\chapter{特征值}

\begin{remark}
代数基本定理保证了任一 $n(n \geq 1)$ 阶复矩阵 $A$ 或 $n$ 维复线性空间 $V$ 上的线性变换 $\varphi$ 至少有一个复特征值 $\lambda_0$, 线性方程组的求解理论保证了 $\lambda_0$ 至少有一个复特征向量。如果是在数域 $\mathbb{F}$ 上, 则需要 $A$ 或 $\varphi$ 的特征值 $\lambda_0$ 属于 $\mathbb{F}$, 然后线性方程组的求解理论才能保证 $\lambda_0$ 在 $\mathbb{F}^n$ 或 $V$ 中有对应的特征向量。因此, \textbf{后面如无特殊说明, 总是假设在复数域 $\mathbb{C}$ 上考虑问题}.
\end{remark}

% \subfile{Chapter-06/section-01.tex}

\subfile{Chapter-06/section-03.tex}

\subfile{Chapter-06/section-04.tex}

\subfile{Chapter-06/section-05.tex}

\subfile{Chapter-06/section-06.tex}

\subfile{Chapter-06/section-02.tex}

% \subfile{Chapter-06/section-07.tex}

% \subfile{Chapter-06/section-08.tex}

% \subfile{Chapter-06/section-09.tex}

% \subfile{Chapter-06/section-10.tex}

\end{document}
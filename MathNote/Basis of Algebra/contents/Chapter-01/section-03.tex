\documentclass[../../main.tex]{subfiles}
\graphicspath{{\subfix{../../image/}}} % 指定图片目录,后续可以直接使用图片文件名。

% 例如:
% \begin{figure}[H]
% \centering
% \includegraphics{image-01.01}
% \caption{图片标题}
% \label{figure:image-01.01}
% \end{figure}
% 注意:上述\label{}一定要放在\caption{}之后,否则引用图片序号会只会显示??.

\begin{document}

\section{行列式综合问题}

\begin{example}\label{example--经典行列式1}
(1)设 \(|\boldsymbol{A}|\) 是 \(n\) 阶行列式, \(|\boldsymbol{A}|\) 的第 \((i,j)\) 元素 \(a_{ij}=\max\{i,j\}\), 试求 \(|\boldsymbol{A}|\) 的值.
 
(2)设 \(|\boldsymbol{A}|\) 是 \(n\) 阶行列式, \(|\boldsymbol{A}|\) 的第 \((i,j)\) 元素 \(a_{ij}=|i - j|\), 试求 \(|\boldsymbol{A}|\) 的值.
\end{example}
\begin{solution}
(1)写出行列式为
\begin{align*}
\begin{vmatrix}
1 & 2 & 3 & \cdots & n \\
2 & 2 & 3 & \cdots & n \\
3 & 3 & 3 & \cdots & n \\
\vdots & \vdots & \vdots & & \vdots \\
n & n & n & \cdots & n
\end{vmatrix}
\end{align*}
依次将第 \(i\) 行乘以 \(-1\) 加到第 \(i - 1\) 行上去 \((i = 2,\cdots,n)\), 就可以得到一个下三角行列式, 求得值为 \((-1)^{n - 1}n\).此即
\begin{align*}
\left| \begin{matrix}
1&		2&		3&		\cdots&		n\\
2&		2&		3&		\cdots&		n\\
3&		3&		3&		\cdots&		n\\
\vdots&		\vdots&		\vdots&		&		\vdots\\
n&		n&		n&		\cdots&		n\\
\end{matrix} \right|=\left| \begin{matrix}
-1&		0&		0&		\cdots&		0\\
-1&		-1&		0&		\cdots&		0\\
-1&		-1&		-1&		\cdots&		0\\
\vdots&		\vdots&		\vdots&		&		\vdots\\
n&		n&		n&		\cdots&		n\\
\end{matrix} \right|=\left( -1 \right) ^{n-1}n.
\end{align*}

(2)写出行列式为
\begin{align*}
\begin{vmatrix}
0 & 1 & 2 & \cdots & n - 1 \\
1 & 0 & 1 & \cdots & n - 2 \\
2 & 1 & 0 & \cdots & n - 3 \\
\vdots & \vdots & \vdots & & \vdots \\
n - 1 & n - 2 & n - 3 & \cdots & 0
\end{vmatrix}
\end{align*}
从最后一列起每一列减去前一列, 再将得到的行列式的最后一行加到前面的每一行上去, 就可以得到一个下三角行列式, 求得值为 \((-1)^{n - 1}(n - 1)2^{n - 2}\). 此即
\begin{align*}
\left| \begin{matrix}
0&		1&		2&		\cdots&		n-1\\
1&		0&		1&		\cdots&		n-2\\
2&		1&		0&		\cdots&		n-3\\
\vdots&		\vdots&		\vdots&		&		\vdots\\
n-1&		n-2&		n-3&		\cdots&		0\\
\end{matrix} \right|=\left| \begin{matrix}
0&		1&		1&		\cdots&		1\\
1&		-1&		1&		\cdots&		1\\
2&		-1&		-1&		\cdots&		1\\
\vdots&		\vdots&		\vdots&		&		\vdots\\
n-1&		-1&		-1&		\cdots&		-1\\
\end{matrix} \right|=\left| \begin{matrix}
n-1&		0&		0&		\cdots&		0\\
n+2&		-2&		0&		\cdots&		0\\
n+1&		-2&		-2&		\cdots&		0\\
\vdots&		\vdots&		\vdots&		&		\vdots\\
n-1&		-1&		-1&		\cdots&		-1\\
\end{matrix} \right|=\left( -1 \right) ^{n-1}\left( n-1 \right) 2^{n-2}.
\end{align*}
\end{solution}

\begin{example}
求下列$n$阶行列式的值:
\[
\left| A \right|=\begin{vmatrix}
(x - a_1)^2 & a_2^2 & \cdots & a_n^2 \\
a_1^2 & (x - a_2)^2 & \cdots & a_n^2 \\
\vdots & \vdots & \ddots & \vdots \\
a_1^2 & a_2^2 & \cdots & (x - a_n)^2
\end{vmatrix}.
\]
\end{example}
\begin{note}
注意到这个行列式每行元素除了主对角元素外,其余位置元素都相同.因此这个行列式是\hyperref["爪"型行列式的推广]{推广的"爪"型行列式}.
\end{note}
\begin{solution}
\begin{align*}
&\left| A \right|=\left| \begin{matrix}
(x-a_1)^2&		a_{2}^{2}&		\cdots&		a_{n}^{2}\\
a_{1}^{2}&		(x-a_2)^2&		\cdots&		a_{n}^{2}\\
\vdots&		\vdots&		\ddots&		\vdots\\
a_{1}^{2}&		a_{2}^{2}&		\cdots&		(x-a_n)^2\\
\end{matrix} \right|=\left| \begin{matrix}
(x-a_1)^2&		a_{2}^{2}&		\cdots&		a_{n}^{2}\\
2a_1x-x^2&		x^2-2a_2x&		\cdots&		0\\
\vdots&		\vdots&		\ddots&		\vdots\\
2a_1x-x^2&		0&		\cdots&		x^2-2a_nx\\
\end{matrix} \right|
\\
&\xlongequal{\hyperref["爪"型行列式]{\text{"爪"型行列式}}}(x-a_1)^2\prod_{i=2}^n{\left( x^2-2a_ix \right)}-\sum_{i=2}^n{a_{i}^{2}\left( 2a_1x-x^2 \right) \left( x^2-2a_2x \right) \cdots \widehat{\left( x^2-2a_ix \right) }\cdots}\left( x^2-2a_nx \right) 
\\
&=(x-a_1)^2\prod_{i=2}^n{\left( x^2-2a_ix \right)}+\sum_{i=2}^n{a_{i}^{2}\left( x^2-2a_1x \right) \left( x^2-2a_2x \right) \cdots \widehat{\left( x^2-2a_ix \right) }\cdots}\left( x^2-2a_nx \right) 
\\
&=(x-a_1)^2\prod_{i=2}^n{\left( x^2-2a_ix \right)}+\sum_{i=2}^n{\left( x^2-2a_1x \right) \cdots \left( x^2-2a_{i-1}x \right) a_{i}^{2}\left( x^2-2a_{i+1}x \right) \cdots}\left( x^2-2a_nx \right) 
\\
&=\left[ \left( x^2-2a_1x \right) +a_{1}^{2} \right] \prod_{i=2}^n{\left( x^2-2a_ix \right)}+\sum_{i=2}^n{\left( x^2-2a_1x \right) \cdots \left( x^2-2a_{i-1}x \right) a_{i}^{2}\left( x^2-2a_{i+1}x \right) \cdots}\left( x^2-2a_nx \right) 
\\
&=\prod_{i=1}^n{\left( x^2-2a_ix \right)}+\sum_{i=1}^n{\left( x^2-2a_1x \right) \cdots \left( x^2-2a_{i-1}x \right) a_{i}^{2}\left( x^2-2a_{i+1}x \right) \cdots}\left( x^2-2a_nx \right).
\end{align*}
\end{solution}

\begin{example}
求下列行列式式的值:
\[
\left| \boldsymbol{A} \right|=\begin{vmatrix}
(a + b)^2 & c^2 & c^2 \\
a^2 & (b + c)^2 & a^2 \\
b^2 & b^2 & (c + a)^2
\end{vmatrix}.
\]
\end{example}
\begin{solution}
{\color{blue}解法一:}
\begin{align*}
&\left| \boldsymbol{A} \right|=\left| \begin{matrix}
(a+b)^2&		c^2&		c^2\\
a^2&		(b+c)^2&		a^2\\
b^2&		b^2&		(c+a)^2\\
\end{matrix} \right|\xlongequal[i=1,2]{-j_1+j_i}\left| \begin{matrix}
(a+b)^2-c^2&		c^2&		0\\
a^2-(b+c)^2&		(b+c)^2&		a^2-(b+c)^2\\
0&		b^2&		(c+a)^2-b^2\\
\end{matrix} \right|
\\
&=(a+b+c)^2\left| \begin{matrix}
a+b-c&		c^2&		0\\
a-b-c&		(b+c)^2&		a-b-c\\
0&		b^2&		a+c-b\\
\end{matrix} \right|\xlongequal[i=1,2]{-r_i+r_2}(a+b+c)^2\left| \begin{matrix}
a+b-c&		c^2&		0\\
-2b&		2bc&		-2c\\
0&		b^2&		a+c-b\\
\end{matrix} \right|
\\
&\xlongequal[\frac{b}{2}j_3+j_2]{\frac{c}{2}j_1+j_2}(a+b+c)^2\left| \begin{matrix}
a+b-c&		\frac{c}{2}\left( a+b+c \right)&		0\\
-2b&		0&		-2c\\
0&		\frac{b}{2}\left( a+b+c \right)&		a+c-b\\
\end{matrix} \right|=(a+b+c)^3\left| \begin{matrix}
a+b-c&		\frac{c}{2}&		0\\
-2b&		0&		-2c\\
0&		\frac{b}{2}&		a+c-b\\
\end{matrix} \right|
\\
&=2abc(a+b+c)^3.
\end{align*}

{\color{blue}解法二(求根法):}
\end{solution}

\begin{example}
证明:若一个\(n(n>1)\)阶行列式中元素或为\(1\)或为\(-1\),则其值必为偶数.
\end{example}
\begin{proof}
将该行列式的任意一行加到另一行上去得到的行列式有一行元素全是偶数(注意:零也是偶数),由行列式的基本性质知道,可将因子2提出,剩下的行列式的元素都是整数,其值也是整数,乘以2后必是偶数.
\end{proof}

\begin{example}
\(n\) 阶行列式\(\vert \boldsymbol{A}\vert\)的值为\(c\),若从第二列开始每一列加上它前面的一列,同时对第一列加上\(\vert \boldsymbol{A}\vert\)的第\(n\)列,求得到的新行列式\(\vert \boldsymbol{B}\vert\)的值.
\end{example}
\begin{solution}
\begin{align*}
\left| \boldsymbol{B} \right|&=\left| \boldsymbol{\alpha }_1+\boldsymbol{\alpha }_n,\boldsymbol{\alpha }_2+\boldsymbol{\alpha }_1,\cdots ,\boldsymbol{\alpha }_n+\boldsymbol{\alpha }_{n-1} \right|
\\
&=\left| \boldsymbol{\alpha }_1,\boldsymbol{\alpha }_2,\cdots ,\boldsymbol{\alpha }_n \right|+\left| \boldsymbol{\alpha }_n,\boldsymbol{\alpha }_1,\cdots ,\boldsymbol{\alpha }_{n-1} \right|
+\sum_{1\leqslant k\leqslant n-2}{\sum_{2\leq j_1\leq j_2\leq \cdots\leq j_k\leq n}{\begin{array}{c}
\begin{array}{c@{}c@{}c@{}c@{}c@{}c@{}c@{}c@{}c@{}c@{}c@{}}
& 1 & \cdots & j_1 &\cdots &j_2 &\cdots &j_k &\cdots &n \\
\left.\right|
&\boldsymbol{\alpha }_n,&\cdots ,&\boldsymbol{\alpha }_{j_1+1},&\cdots ,&\boldsymbol{\alpha }_{j_2+1},&\cdots ,&\boldsymbol{\alpha }_{j_k+1},&\cdots ,&\boldsymbol{\alpha }_{n-1}& \left|\right.
\end{array}\\
\\
\end{array}}}
\\
&\quad +\sum_{1\leqslant k\leqslant n-2}{\sum_{2\leq j_1\leq j_2\leq \cdots\leq j_k\leq n}{\begin{array}{c}
\begin{array}{c@{}c@{}c@{}c@{}c@{}c@{}c@{}c@{}c@{}c@{}c@{}}
& 1 & \cdots & j_1 &\cdots &j_2 &\cdots &j_k &\cdots &n \\
\left.\right|
&\boldsymbol{\alpha }_1,&\cdots ,&\boldsymbol{\alpha }_{j_1+1},&\cdots ,&\boldsymbol{\alpha }_{j_2+1},&\cdots ,&\boldsymbol{\alpha }_{j_k+1},&\cdots ,&\boldsymbol{\alpha }_{n-1}& \left|\right.
\end{array}\\
\\
\end{array}}}.
\\
&=\left| \boldsymbol{\alpha }_1,\boldsymbol{\alpha }_2,\cdots ,\boldsymbol{\alpha }_n \right|+\left| \boldsymbol{\alpha }_n,\boldsymbol{\alpha }_1,\cdots ,\boldsymbol{\alpha }_{n-1} \right|
=c+\left( -1 \right) ^{n-1}\left| \boldsymbol{\alpha }_1,\boldsymbol{\alpha }_2,\cdots ,\boldsymbol{\alpha }_n \right|
\\
&=c+\left( -1 \right) ^{n-1}c
=\begin{cases}
0 \,\,,n\text{为偶数}\\
2c,n\text{为奇数}\\
\end{cases}
\end{align*}
\end{solution}

\begin{example}
令
\[
\left( a_{1} a_{2} \cdots a_{n} \right) = 
\begin{vmatrix}
a_{1} & 1 &   &   &   \\
-1 & a_{2} & 1 &   &   \\
& -1 & a_{3} & \ddots &   \\
&   & \ddots & \ddots & 1 \\
&   &   & -1 & a_{n}
\end{vmatrix},
\]
证明关于连分数的如下等式成立:
\[
a_{1} + \frac{1}{a_{2} + \frac{1}{a_{3} + \cdots + \frac{1}{a_{n - 1} + \frac{1}{a_{n}}}}} = \frac{\left( a_{1} a_{2} \cdots a_{n} \right)}{\left( a_{2} a_{3} \cdots a_{n} \right)}.
\]
\end{example}
\begin{solution}
假设等式对$\forall n\leq k-1,k\in \mathbb{N}_+$都成立.则当$n=k$时,将行列式$(a_1a_2,\cdots,a_k)$按第一列展开得
\begin{align*}
\left( a_1a_2\cdots a_k \right) &=\left| \begin{matrix}
a_1&		1&		&		&		\\
-1&		a_2&		1&		&		\\
&		-1&		a_3&		\ddots&		\\
&		&		\ddots&		\ddots&		1\\
&		&		&		-1&		a_k\\
\end{matrix} \right|=a_1\left| \begin{matrix}
a_2&		1&		&		\\
-1&		a_3&		\ddots&		\\
&		\ddots&		\ddots&		1\\
&		&		-1&		a_k\\
\end{matrix} \right|+\left| \begin{matrix}
a_3&		1&		&		\\
-1&		a_4&		\ddots&		\\
&		\ddots&		\ddots&		1\\
&		&		-1&		a_k\\
\end{matrix} \right|
\\
&=a_1\left( a_2a_3\cdots a_k \right) +\left( a_3a_4\cdots a_k \right).
\end{align*}
从而
\begin{align*}
\frac{\left( a_1a_2\cdots a_k \right)}{\left( a_2a_3\cdots a_k \right)}=a_1+\frac{\left( a_3a_4\cdots a_k \right)}{\left( a_2a_3\cdots a_k \right)}=a_1+\frac{1}{\frac{\left( a_2a_3\cdots a_k \right)}{\left( a_3a_4\cdots a_k \right)}}.
\end{align*}
于是由归纳假设可知
\begin{align*}
\frac{\left( a_1a_2\cdots a_k \right)}{\left( a_2a_3\cdots a_k \right)}=a_1+\frac{1}{\frac{\left( a_2a_3\cdots a_k \right)}{\left( a_3a_4\cdots a_k \right)}}=a_1+\frac{1}{a_2+\frac{1}{a_3+\cdots +\frac{1}{a_{n-1}+\frac{1}{a_n}}}}.
\end{align*}
故由数学归纳法可知结论成立.
\end{solution}

\begin{example}
设\(\vert A\vert\)是\(n\)阶行列式,\(\vert A\vert\)的第\((i,j)\)元素\(a_{ij}=\max\{i,j\}\),试求\(\vert A\vert\)的值.
\end{example}
\begin{solution}
\begin{align*}
\left| \boldsymbol{A} \right|=\left| \begin{matrix}
1&		2&		3&		\cdots&		n\\
2&		2&		3&		\cdots&		n\\
3&		3&		3&		\cdots&		n\\
\vdots&		\vdots&		\vdots&		&		\vdots\\
n&		n&		n&		\cdots&		n\\
\end{matrix} \right|\xlongequal[i=n,n-1,\cdots ,2]{-r_i+r_{i-1}}\left| \begin{matrix}
-1&		0&		0&		\cdots&		0\\
2&		-1&		0&		\cdots&		0\\
3&		3&		-1&		\cdots&		0\\
\vdots&		\vdots&		\vdots&		&		\vdots\\
n&		n&		n&		\cdots&		n\\
\end{matrix} \right|=\left( -1 \right) ^{n-1}n.
\end{align*}
\end{solution}

\begin{example}
设\(\vert A\vert\)是\(n\)阶行列式,\(\vert A\vert\)的第\((i,j)\)元素\(a_{ij}=\vert i - j\vert\),试求\(\vert A\vert\)的值.
\end{example}
\begin{note}
注意:这只是一个\textbf{对称行列式},不是循环行列式.
类似这种每行、每列元素有一定的等差递进关系的行列式,都可以先尝试用每一列减去前面一列.
\end{note}
\begin{solution}
\begin{align*}
\left| \boldsymbol{A} \right|&=\left| \begin{matrix}
0&		1&		2&		\cdots&		n-2&		n-1\\
1&		0&		1&		\cdots&		n-3&		n-2\\
2&		1&		0&		\cdots&		n-4&		n-3\\
\vdots&		\vdots&		\vdots&		&		\vdots&		\vdots\\
n-1&		n-2&		n-3&		\cdots&		1&		0\\
\end{matrix} \right|\xlongequal[i=n,n-1,\cdots ,2]{-j_{i-1}+j_i}\left| \begin{matrix}
0&		1&		1&		\cdots&		1&		1\\
1&		-1&		1&		\cdots&		1&		1\\
2&		-1&		-1&		\cdots&		1&		1\\
\vdots&		\vdots&		\vdots&		&		\vdots&		\vdots\\
n-1&		-1&		-1&		\cdots&		-1&		-1\\
\end{matrix} \right|
\\
&\xlongequal[i=n-1,n-2,\cdots ,1]{r_n+r_i}\left| \begin{matrix}
n-1&		0&		0&		\cdots&		0&		0\\
n&		-2&		0&		\cdots&		0&		0\\
n+1&		-2&		-2&		\cdots&		0&		0\\
\vdots&		\vdots&		\vdots&		&		\vdots&		\vdots\\
n-1&		-1&		-1&		\cdots&		-1&		-1\\
\end{matrix} \right|=\left( -2 \right) ^{n-2}\left( n-1 \right) .
\end{align*}
\end{solution}

\begin{example}
求下列\(n\)阶行列式的值:
\[
\left| \boldsymbol{A} \right| = 
\begin{vmatrix}
1 & x_1(x_1 - a) & x_1^2(x_1 - a) & \cdots & x_1^{n - 1}(x_1 - a)\\
1 & x_2(x_2 - a) & x_2^2(x_2 - a) & \cdots & x_2^{n - 1}(x_2 - a)\\
\vdots & \vdots & \vdots & \ddots & \vdots\\
1 & x_n(x_n - a) & x_n^2(x_n - a) & \cdots & x_n^{n - 1}(x_n - a)
\end{vmatrix}.
\]
\end{example}
\begin{note}
当行列式的行或列有一定的规律性时,但是由于缺少一行或一列导致这个行列式行或列的规律性并不完整.此时我们可以尝试\hyperlink{行列式计算:升阶法}{升阶法}补全这个行列式行或列的规律,再对行列式进行化简.

本题若直接使用\hyperref[大拆分法]{大拆分法}会得到比较多的行列式,而且每个行列式并不是完整的$Vandermode$行列式.后续求解很繁琐,因此不采取\hyperref[大拆分法]{大拆分法}.
\end{note}
\begin{solution}
(\hyperlink{行列式计算:升阶法}{升阶法})考虑$n+1$阶行列式\(|\boldsymbol{B}|=\left|\begin{matrix}
1 & x_1 - a & x_1(x_1 - a) & x_{1}^{2}(x_1 - a) & \cdots & x_{1}^{n - 1}(x_1 - a)\\
1 & x_2 - a & x_2(x_2 - a) & x_{2}^{2}(x_2 - a) & \cdots & x_{2}^{n - 1}(x_2 - a)\\
\vdots & \vdots & \vdots & \vdots &  & \vdots\\
1 & x_n - a & x_n(x_n - a) & x_{n}^{2}(x_n - a) & \cdots & x_{n}^{n - 1}(x_n - a)\\
1 & y - a & y(y - a) & y^2(y - a) & \cdots & y^{n - 1}(y - a)
\end{matrix}\right|\),则
\begin{align*}
|\boldsymbol{B}|=\left|\begin{matrix}
1 & x_1 & x_{1}^{2} & x_{1}^{3} & \cdots & x_{1}^{n}\\
1 & x_2 & x_{2}^{2} & x_{2}^{3} & \cdots & x_{2}^{n}\\
\vdots & \vdots & \vdots & \vdots &  & \vdots\\
1 & x_n & x_{n}^{2} & x_{n}^{3} & \cdots & x_{n}^{n}\\
1 & y & y^2 & y^3 & \cdots & y^n
\end{matrix}\right|=\prod_{k = 1}^{n}(y - x_k)\prod_{1\leqslant i < j\leqslant n}(x_j - x_i).
\end{align*}
由上式可知,\(|\boldsymbol{B}|\)可以看作一个关于\(y\)的\(n\)次多项式.
将\(|\boldsymbol{B}|\)按最后一行展开得到
\begin{align*}
|\boldsymbol{B}|=\sum_{i = 1}^{n + 1}(-1)^{n + i}B_{n + 1,i}y^{i - 1},\text{其中}B_{ni}\text{是}|\boldsymbol{B}|\text{的第}(n + 1,i)\text{元的余子式},i = 1,2,\cdots,n + 1.
\end{align*}
从而
\begin{align}\label{eq:两多项式相等1.1}
|\boldsymbol{B}|=(-1)^{n + 2}B_{n + 1,1}+\sum_{i = 2}^{n + 1}(-1)^{n + i + 1}B_{n + 1,i}y^{i - 2}(y - a)=\prod_{k = 1}^{n}(y - x_k)\prod_{1\leqslant i < j\leqslant n}(x_j - x_i).
\end{align}
又易知\(B_{n + 1,2}=|\boldsymbol{A}|\),而当\(a = 0\)时,由等式\eqref{eq:两多项式相等1.1}可知,\(|\boldsymbol{B}|\)中\(y\)前面的系数只有\(B_{n + 1,2}\).比较等式\eqref{eq:两多项式相等1.1}两边\(y\)的系数可得
\begin{align*}
(-1)^{n + 3}|\boldsymbol{A}|=(-1)^{n + 3}B_{n + 1,2}=\prod_{1\leqslant i < j\leqslant n}(x_j - x_i)\left(\sum_{i = 1}^{n}(-x_1)\cdots (-x_{i - 1})(-x_{i + 1})\cdots (-x_n)\right).
\end{align*}
于是\(|\boldsymbol{A}|=(-1)^{n + 3}(-1)^{n - 1}\prod_{1\leqslant i < j\leqslant n}(x_j - x_i)\left(\sum_{i = 1}^{n}x_1\cdots x_{i - 1}x_{i + 1}\cdots x_n\right)=\prod_{1\leqslant i < j\leqslant n}(x_j - x_i)\left(\sum_{i = 1}^{n}x_1\cdots x_{i - 1}x_{i + 1}\cdots x_n\right)\).

当\(a\neq 0\)时,由等式\eqref{eq:两多项式相等1.1}可知,\(|\boldsymbol{B}|\)中\(y\)前面的系数不只有\(B_{n + 1,2}\),但是,我们比较等式\eqref{eq:两多项式相等1.1}两边的常数项可得
\begin{align}\label{eq:等式1.2}
(-1)^{n + 2}B_{n + 1,1}-a(-1)^{n + 3}B_{n + 1,2}=\prod_{1\leqslant i < j\leqslant n}(x_j - x_i)\prod_{k = 1}^{n}(-x_k).
\end{align}
又因为
\begin{align*}
B_{n + 1,1}&=\left|\begin{matrix}
x_1 - a & x_1(x_1 - a) & x_{1}^{2}(x_1 - a) & \cdots & x_{1}^{n - 1}(x_1 - a)\\
x_2 - a & x_2(x_2 - a) & x_{2}^{2}(x_2 - a) & \cdots & x_{2}^{n - 1}(x_2 - a)\\
\vdots & \vdots & \vdots &  & \vdots\\
x_n - a & x_n(x_n - a) & x_{n}^{2}(x_n - a) & \cdots & x_{n}^{n - 1}(x_n - a)
\end{matrix}\right|
\\
&=\prod_{i = 1}^{n}(x_i - a)\left|\begin{matrix}
1 & x_1 & x_{1}^{2} & x_{1}^{3} & \cdots & x_{1}^{n - 1}\\
1 & x_2 & x_{2}^{2} & x_{2}^{3} & \cdots & x_{2}^{n - 1}\\
\vdots & \vdots & \vdots & \vdots &  & \vdots\\
1 & x_n & x_{n}^{2} & x_{n}^{3} & \cdots & x_{n}^{n - 1}
\end{matrix}\right|=\prod_{i = 1}^{n}(x_i - a)\prod_{1\leqslant i < j\leqslant n}(x_j - x_i).
\end{align*}
所以再结合等式\eqref{eq:等式1.2}可得
\begin{align*}
-a(-1)^{n + 3}|\boldsymbol{A}|&=-a(-1)^{n + 3}B_{n + 1,2}=\prod_{1\leqslant i < j\leqslant n}(x_j - x_i)\prod_{k = 1}^{n}(-x_k)-(-1)^{n + 2}B_{n + 1,1}
\\
&=(-1)^n\prod_{k = 1}^{n}x_k\prod_{1\leqslant i < j\leqslant n}(x_j - x_i)+(-1)^{n + 1}\prod_{i = 1}^{n}(x_i - a)\prod_{1\leqslant i < j\leqslant n}(x_j - x_i)
\\
&=(-1)^n\prod_{1\leqslant i < j\leqslant n}(x_j - x_i)\left[\prod_{k = 1}^{n}x_k-\prod_{i = 1}^{n}(x_i - a)\right].
\end{align*}
故此时\(|\boldsymbol{A}|=\prod_{1\leqslant i < j\leqslant n}(x_j - x_i)\left(\prod_{k = 1}^{n}x_k-\prod_{i = 1}^{n}(x_i - a)\right)\).
\end{solution}

\begin{example}
求下列行列式式的值($n$为偶数)
\begin{align*}
I=\left| \begin{matrix}
1&		1&		\cdots&		1&		1\\
2&		2^2&		\cdots&		2^n&		2^{n+1}\\
\vdots&		\vdots&		\ddots&		\vdots&		\vdots\\
n&		n^2&		\cdots&		n^n&		n^{n+1}\\
\frac{n}{2}&		\frac{n^2}{3}&		\cdots&		\frac{n^n}{n+1}&		\frac{n^{n+1}}{n+2}\\
\end{matrix} \right|.
\end{align*}
\end{example}
\begin{note}
应用\hyperref[proposition:行列式的求导运算]{行列式函数求导求行列式}的值.
\end{note}
\begin{solution}
令\(G(x)=\left|\begin{matrix}
1 & 1 & \cdots & 1 & 1\\
2 & 2^2 & \cdots & 2^n & 2^{n + 1}\\
\vdots & \vdots & \ddots & \vdots & \vdots\\
n & n^2 & \cdots & n^n & n^{n + 1}\\
\frac{x^2}{2} & \frac{x^3}{3} & \cdots & \frac{x^{n + 1}}{n + 1} & \frac{x^{n + 2}}{n + 2}
\end{matrix}\right|\),则\(I = \frac{G(n)}{n}\)且\(G(0) = 0\).      
利用行列式求导公式,可得
\begin{align*}
G'(x)&=\left|\begin{matrix}
1 & 1 & \cdots & 1 & 1\\
2 & 2^2 & \cdots & 2^n & 2^{n + 1}\\
\vdots & \vdots & \ddots & \vdots & \vdots\\
n & n^2 & \cdots & n^n & n^{n + 1}\\
x & x^2 & \cdots & x^n & x^{n + 1}
\end{matrix}\right|
= n!x\left|\begin{matrix}
1 & 1 & \cdots & 1 & 1\\
1 & 2 & \cdots & 2^{n - 1} & 2^n\\
\vdots & \vdots & \ddots & \vdots & \vdots\\
1 & n & \cdots & n^{n - 1} & n^n\\
1 & x & \cdots & x^{n - 1} & x^n
\end{matrix}\right|
= n!\prod_{1\leqslant i < j\leqslant n}(j - i)\prod_{k = 0}^{n}(x - k).
\end{align*}
因此
\begin{align*}
I &= \frac{G(n)}{n}=\frac{\int_{0}^{n}G'(x)dx}{n}=(n - 1)!\prod_{1\leqslant i < j\leqslant n}(j - i)\int_{0}^{n}\prod_{k = 0}^{n}(x - k)dx
\\
&\stackrel{\text{区间再现}}{=}(n - 1)!\prod_{1\leqslant i < j\leqslant n}(j - i)\int_{0}^{n}\prod_{k = 0}^{n}(n - k - x)dx
\\
&= (-1)^{n + 1}(n - 1)!\prod_{1\leqslant i < j\leqslant n}(j - i)\int_{0}^{n}\prod_{k = 0}^{n}(x - k)dx
\\
&= (-1)^{n + 1}I.
\end{align*}      
由于\(n\)为偶数,所以\((-1)^{n + 1} = -1\).于是\(I = -I\).故\(I = 0\). 
\end{solution}









\end{document}
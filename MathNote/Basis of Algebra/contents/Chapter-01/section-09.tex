\documentclass[../../main.tex]{subfiles}
\graphicspath{{\subfix{../../image/}}} % 指定图片目录,后续可以直接使用图片文件名。

% 例如:
% \begin{figure}[H]
% \centering
% \includegraphics[scale=0.4]{图.png}
% \caption{}
% \label{figure:图}
% \end{figure}
% 注意:上述\label{}一定要放在\caption{}之后,否则引用图片序号会只会显示??.

\begin{document}

\section{升阶法}

\begin{example}\label{升阶法的应用(1)例题}
计算$n$阶行列式:
\begin{align*}
|\boldsymbol{A}|=\left| \begin{matrix}
1+x_1&		1+x_2&		\cdots&		1+x_{1}^{n}\\1+x_{1}
&		1+x_{2}^{2}&		\cdots&		1+x_{2}^{n}\\
\vdots&		\vdots&		&		\vdots\\
1+x_n&		1+x_{n}^{2}&		\cdots&		1+x_{n}^{n}\\
\end{matrix} \right|.
\end{align*}
\end{example}
\begin{note}
本题也可以使用\hyperref[大拆分法]{大拆分法}进行求解.但我们以本题为例介绍利用\textbf{升阶法}计算行列式.
\end{note}
\begin{solution}
{\color{blue}解法一(\hyperref[行列式计算:升阶法]{升阶法}):}
\begin{align*}
|\boldsymbol{A}|&=\left| \begin{matrix}
1&		0&		0&		\cdots&		0\\
1&		1+x_1&		1+x_{1}^{2}&		\cdots&		1+x_{1}^{n}\\
1&		1+x_2&		1+x_{2}^{2}&		\cdots&		1+x_{2}^{n}\\
\vdots&		\vdots&		\vdots&		&		\vdots\\
1&		1+x_n&		1+x_{n}^{2}&		\cdots&		1+x_{n}^{n}\\
\end{matrix} \right|=\left| \begin{matrix}
1&		-1&		-1&		\cdots&		-1\\
1&		x_1&		x_{1}^{2}&		\cdots&		x_{1}^{n}\\
1&		x_2&		x_{2}^{2}&		\cdots&		x_{2}^{n}\\
\vdots&		\vdots&		\vdots&		&		\vdots\\
1&		x_n&		x_{n}^{2}&		\cdots&		x_{n}^{n}\\
\end{matrix} \right|
\\
&\xlongequal{\hyperlink{小拆分法}{\text{小拆分法}}}\left| \begin{matrix}
2&		0&		0&		\cdots&		0\\
1&		x_1&		x_{1}^{2}&		\cdots&		x_{1}^{n}\\
1&		x_2&		x_{2}^{2}&		\cdots&		x_{2}^{n}\\
\vdots&		\vdots&		\vdots&		&		\vdots\\
1&		x_n&		x_{n}^{2}&		\cdots&		x_{n}^{n}\\
\end{matrix} \right|+\left| \begin{matrix}
-1&		-1&		-1&		\cdots&		-1\\
1&		x_1&		x_{1}^{2}&		\cdots&		x_{1}^{n}\\
1&		x_2&		x_{2}^{2}&		\cdots&		x_{2}^{n}\\
\vdots&		\vdots&		\vdots&		&		\vdots\\
1&		x_n&		x_{n}^{2}&		\cdots&		x_{n}^{n}\\
\end{matrix} \right|
\\
&=2\left| \begin{matrix}
x_1&		x_{1}^{2}&		\cdots&		x_{1}^{n}\\
x_2&		x_{2}^{2}&		\cdots&		x_{2}^{n}\\
\vdots&		\vdots&		&		\vdots\\
x_n&		x_{n}^{2}&		\cdots&		x_{n}^{n}\\
\end{matrix} \right|-\left| \begin{matrix}
1&		1&		1&		\cdots&		1\\
1&		x_1&		x_{1}^{2}&		\cdots&		x_{1}^{n}\\
1&		x_2&		x_{2}^{2}&		\cdots&		x_{2}^{n}\\
\vdots&		\vdots&		\vdots&		&		\vdots\\
1&		x_n&		x_{n}^{2}&		\cdots&		x_{n}^{n}\\
\end{matrix} \right|
\\
&=2x_1x_2\cdots x_n\left| \begin{matrix}
1&		x_1&		\cdots&		x_{1}^{n-1}\\
1&		x_2&		\cdots&		x_{2}^{n-1}\\
\vdots&		\vdots&		&		\vdots\\
1&		x_n&		\cdots&		x_{n}^{n-1}\\
\end{matrix} \right|-\left( x_1-1 \right) \left( x_2-1 \right) \cdots \left( x_n-1 \right) \prod_{1\le i<j\le n}{\left( x_j-x_i \right)}
\\
&=2x_1x_2\cdots x_n\prod_{1\le i<j\le n}{\left( x_j-x_i \right)}-\left( x_1-1 \right) \left( x_2-1 \right) \cdots \left( x_n-1 \right) \prod_{1\le i<j\le n}{\left( x_j-x_i \right)}
\\
&=\left[ 2x_1x_2\cdots x_n-\left( x_1-1 \right) \left( x_2-1 \right) \cdots \left( x_n-1 \right) \right] \prod_{1\le i<j\le n}{\left( x_j-x_i \right)}.
\end{align*}

{\color{blue}解法二(\hyperref[大拆分法]{大拆分法}):}
设\(\vert\boldsymbol{B}(t)\vert=\left|\begin{matrix}
x_1 + t & x_{1}^{2} + t & \cdots & x_{1}^{n} + t\\
x_2 + t & x_{2}^{2} + t & \cdots & x_{2}^{n} + t\\
\vdots & \vdots &  & \vdots\\
x_n + t & x_{n}^{2} + t & \cdots & x_{n}^{n} + t
\end{matrix}\right|\),且\(B_{ij}\)是\(\vert\boldsymbol{B}(0)\vert\)的第\((i,j)\)元素的代数余子式.

根据行列式的性质将\(\vert\boldsymbol{A}\vert\)每一列都拆分成两列,然后按\(t\)所在的列展开得到
\begin{gather*}
\vert\boldsymbol{A}\vert=\vert\boldsymbol{B}(1)\vert=\vert\boldsymbol{B}(0)\vert+\sum_{i,j = 1}^{n}B_{ij},
\\
\vert\boldsymbol{B}(-1)\vert=\vert\boldsymbol{B}(0)\vert-\sum_{i,j = 1}^{n}B_{ij}.
\end{gather*}
于是\(\vert\boldsymbol{A}\vert = 2\vert\boldsymbol{B}(0)\vert - \vert\boldsymbol{B}(-1)\vert\).注意到
\begin{align*}
\vert\boldsymbol{B}(0)\vert=\left|\begin{matrix}
x_1 & x_{1}^{2} & \cdots & x_{1}^{n}\\
x_2 & x_{2}^{2} & \cdots & x_{2}^{n}\\
\vdots & \vdots &  & \vdots\\
x_n & x_{n}^{2} & \cdots & x_{n}^{n}
\end{matrix}\right|=x_1x_2\cdots x_n\left|\begin{matrix}
1 & x_1 & \cdots & x_{1}^{n}\\
1 & x_2 & \cdots & x_{2}^{n}\\
\vdots & \vdots &  & \vdots\\
1 & x_n & \cdots & x_{n}^{n}
\end{matrix}\right|=x_1x_2\cdots x_n\prod_{1\leqslant i < j\leqslant n}(x_j - x_i). 
\end{align*}
又由\hyperref[corollary:多项式行列式与Vandermode行列式相关]{推论\ref{corollary:多项式行列式与Vandermode行列式相关}}或\hyperref[proposition:将矩阵拆分成Vandermode矩阵的形式]{命题\ref{proposition:将矩阵拆分成Vandermode矩阵的形式}}可得
\begin{align*}
\vert\boldsymbol{B}(-1)\vert&=\left|\begin{matrix}
x_1 - 1 & x_{1}^{2} - 1 & \cdots & x_{1}^{n} - 1\\
x_2 - 1 & x_{2}^{2} - 1 & \cdots & x_{2}^{n} - 1\\
\vdots & \vdots &  & \vdots\\
x_n - 1 & x_{n}^{2} - 1 & \cdots & x_{n}^{n} - 1
\end{matrix}\right|
=(x_1 - 1)(x_2 - 1)\cdots (x_n - 1)\left|\begin{matrix}
1 & x_1 + 1 & \cdots & x_{1}^{n - 1} + x_{1}^{n - 2}\cdots + x_1 + 1\\
1 & x_2 + 1 & \cdots & x_{2}^{n - 1} + x_{2}^{n - 2}\cdots + x_2 + 1\\
\vdots & \vdots &  & \vdots\\
1 & x_n + 1 & \cdots & x_{n}^{n - 1} + x_{n}^{n - 2}\cdots + x_n + 1
\end{matrix}\right|
\\
&=(x_1 - 1)(x_2 - 1)\cdots (x_n - 1)\left|\begin{matrix}
1 & x_1 & \cdots & x_{1}^{n - 1}\\
1 & x_2 & \cdots & x_{2}^{n - 1}\\
\vdots & \vdots &  & \vdots\\
1 & x_n & \cdots & x_{n}^{n - 1}
\end{matrix}\right|
=(x_1 - 1)(x_2 - 1)\cdots (x_n - 1)\prod_{1\leqslant i < j\leqslant n}(x_j - x_i).
\end{align*}
故可得
\begin{align*}
\vert\boldsymbol{A}\vert &= 2\vert\boldsymbol{B}(0)\vert - \vert\boldsymbol{B}(-1)\vert
=2x_1x_2\cdots x_n\prod_{1\leqslant i < j\leqslant n}(x_j - x_i)-(x_1 - 1)(x_2 - 1)\cdots (x_n - 1)\prod_{1\leqslant i < j\leqslant n}(x_j - x_i)
\\
&=\left[2x_1x_2\cdots x_n-(x_1 - 1)(x_2 - 1)\cdots (x_n - 1)\right]\prod_{1\leqslant i < j\leqslant n}(x_j - x_i).
\end{align*}
\end{solution}
\begin{conclusion}\label{行列式计算:升阶法}
\hypertarget{行列式计算:升阶法}{\textbf{升阶法:}}
将原行列式加上一行和一列使得到到新行列式的阶数比原行列式要高一阶.

\textbf{升阶法的应用:}

(1)当原行列式每一行具有相同的结构时,我们可以在原行列式的基础上加上一行和一列,新加上的一列和一行需要满足:新的一列除了与新的一行交叉位置的元素为1外其余全为0(这样才能保证新的行列式按新的一行或一列展开后与原行列式相同),并且新加上的一行除1以外其他位置的元素就取原行列式中每一行所具有的相同结构(这样可以利用行列式的性质将每一行中的相同的结构减去,进而达到简化原行列式的目的).具体例子见练习\ref{升阶法的应用(1)例题}.

(2)当原行列式是我们由熟悉的行列式去掉某一行、或某一列、或某一行和一列得到的,我们可以在原行列式的基础上补充上缺少的那一行和一列,再进行计算得到新行列式的式子.再将新行列式按照新添加的一行或一列展开得到的对应元素乘与其对应的代数余子式,而新添加的一行和一列交叉位置的元素对应的余子式就是原行列式,最后两边式子比较系数一般就能得到原行列式的值.
具体例子见练习\ref{升阶法的应用(2)例题}.
\end{conclusion}

\begin{example}\label{升阶法的应用(2)例题}
求下列$n$阶行列式的值($1\le i\le n-1$):
\begin{align*}
|\boldsymbol{A}|=\left| \begin{matrix}
1&		x_1&		\cdots&		x_{1}^{i-1}&		x_{1}^{i+1}&		\cdots&		x_{1}^{n}\\
1&		x_2&		\cdots&		x_{2}^{i-1}&		x_{2}^{i+1}&		\cdots&		x_{2}^{n}\\
\vdots&		\vdots&		&		\vdots&		\vdots&		&		\vdots\\
1&		x_n&		\cdots&		x_{n}^{i-1}&		x_{n}^{i+1}&		\cdots&		x_{n}^{n}\\
\end{matrix} \right|.
\end{align*}
\end{example}
\begin{solution}
令
\begin{align*}
|\boldsymbol{B}|=\left|\begin{matrix}
1 & x_1 & \cdots & x_{1}^{i - 1} & x_{1}^{i} & x_{1}^{i + 1} & \cdots & x_{1}^{n}\\
1 & x_2 & \cdots & x_{2}^{i - 1} & x_{2}^{i} & x_{2}^{i + 1} & \cdots & x_{2}^{n}\\
\vdots & \vdots &  & \vdots & \vdots & \vdots &  & \vdots\\
1 & x_n & \cdots & x_{n}^{i - 1} & x_{n}^{i} & x_{n}^{i + 1} & \cdots & x_{n}^{n}\\
1 & y & \cdots & y^{i - 1} & y^i & y^{i + 1} & \cdots & y^n
\end{matrix}\right|=(y - x_1)(y - x_2)\cdots (y - x_n)\prod_{1\leqslant i < j\leqslant n}(x_j - x_i).
\end{align*}
而上式右边是关于\(y\)的\(n\)次多项式,并且其\(y^i\)前的系数是
\begin{align}\label{eq:1.4(Vandermode行列式升阶法)}
\sum_{1\leqslant k_1 < k_2 < \cdots < k_{n - i}\leqslant n}(-1)^{n - i}x_{k_1}x_{k_2}\cdots x_{k_{n - i}}\prod_{1\leqslant i < j\leqslant n}(x_j - x_i).
\end{align}
将\(|\boldsymbol{B}|\)按最后一行展开,得
\begin{align*}
|\boldsymbol{B}|=A_{n1} + A_{n2}y + \cdots + A_{ni}y^i + \cdots + A_{nn}y^n, 
\end{align*}
其中\(A_{nk}\)为\(|\boldsymbol{B}|\)的\((n,k)\)位置元素的代数余子式,\(k = 1,2,\cdots,n\).

注意到\(A_{nk}\)均与\(y\)无关.因此\(|\boldsymbol{B}|\)作为关于\(y\)的\(n\)次多项式,其\(y^i\)前的系数是
\begin{align*}
A_{ni}=(-1)^{n + 1 + i + 1}|\boldsymbol{A}|=(-1)^{n + i}|\boldsymbol{A}|.
\end{align*}
再结合\eqref{eq:1.4(Vandermode行列式升阶法)}式,可知
\begin{align*}
(-1)^{n + i}|\boldsymbol{A}|=\sum_{1\leqslant k_1 < k_2 < \cdots < k_{n - i}\leqslant n}(-1)^{n - i}x_{k_1}x_{k_2}\cdots x_{k_{n - i}}\prod_{1\leqslant i < j\leqslant n}(x_j - x_i). 
\end{align*}
故\(|\boldsymbol{A}|=\sum_{1\leqslant k_1<k_2<\cdots <k_{n-i}\leqslant n}{x_{k_1}x_{k_2}\cdots x_{k_{n-i}}\prod_{1\leqslant i<j\leqslant n}{(x_j}-x_i)}\).
\end{solution}




\end{document}
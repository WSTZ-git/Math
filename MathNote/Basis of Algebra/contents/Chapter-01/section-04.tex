\documentclass[../../main.tex]{subfiles}
\graphicspath{{\subfix{../../image/}}} % 指定图片目录,后续可以直接使用图片文件名。

% 例如:
% \begin{figure}[h]
% \centering
% \includegraphics{image-01.01}
% \caption{图片标题}
% \label{fig:image-01.01}
% \end{figure}
% 注意:上述\label{}一定要放在\caption{}之后,否则引用图片序号会只会显示??.

\begin{document}

\section{Vandermode行列式}

本节我们用\(V_n(x_1,x_2,\cdots,x_n)\)表示\(n\)阶Vandermonde行列式.
\begin{definition}
对\(1\leqslant i\leqslant n,V_n^{(i)}(x_1,x_2,\cdots,x_n)\)表示删除\(V_n(x_1,x_2,\cdots,x_n)\)的第\(i\)行\((x_1^{i - 1},x_2^{i - 1},\cdots,x_n^{i - 1})\)之后新添第\(n\)行\((x_1^{n},x_2^{n},\cdots,x_n^{n})\)所得\(n\)阶行列式.
\end{definition}

\begin{definition}
\(\Delta_n(x_1,x_2,\cdots,x_n)\)表示将\(V_n(x_1,x_2,\cdots,x_n)\)的第\(n\)行换成\((x_1^{n + 1},x_2^{n + 1},\cdots,x_n^{n + 1})\)所得\(n\)阶行列式.
\end{definition}


\begin{example}
设初等对称多项式
\begin{align}\label{23.137}
\sigma_j=\sum_{1\leqslant k_1<k_2<\cdots<k_j\leqslant n}x_{k_1}x_{k_2}\cdots x_{k_j},j = 1,2,\cdots,n,
\end{align}
我们有
\begin{align}
V_n^{(i)}(x_1,x_2,\cdots,x_n)=\sigma_{n - i + 1}V_n(x_1,x_2,\cdots,x_n),i = 1,2,\cdots,n.\label{23.138}
\end{align}
\end{example}
\begin{proof}
(加边法)不妨设\(x_i,1\leqslant i\leqslant n\)互不相同. 设
\begin{align*}
D_n(x)\triangleq\begin{vmatrix}
1&1&1&\cdots&1\\
-x&x_1&x_2&\cdots&x_n\\
(-x)^2&x_1^2&x_2^2&\cdots&x_n^2\\
\vdots&\vdots&\vdots&\ddots&\vdots\\
(-x)^n&x_1^n&x_2^n&\cdots&x_n^n
\end{vmatrix}.
\end{align*}
由行列式性质我们知道\(D_n\)是\(n\)次多项式且有\(n\)个根\(-x_1,-x_2,\cdots,-x_n\). 于是我们有
\begin{align}\label{eqeqeq1}
D_n(x)=c(x + x_1)(x + x_2)\cdots(x + x_n).
\end{align}
把\(D_n(x)\)按第一列展开得
\begin{align}\label{eqeqeq2}
D_n(x)=\sum_{i = 1}^{n}V_n^{(i)}(x_1,x_2,\cdots,x_n)x^{i - 1}+V_n(x_1,x_2,\cdots,x_n)x^n.
\end{align}
于是比较\eqref{eqeqeq1}式和\eqref{eqeqeq2}式最高次项系数,我们有\(c = V_n(x_1,x_2,\cdots,x_n)\). 定义\(\sigma_0 = 1\),利用根和系数的关系(Vieta定理),结合\eqref{eqeqeq1}式和\eqref{eqeqeq2}式得
\begin{align*}
D_n(x)=\sum_{i = 1}^{n + 1}\sigma_{n - i + 1}V_n(x_1,x_2,\cdots,x_n)x^{i - 1}=\sum_{i = 1}^{n}V_n^{(i)}(x_1,x_2,\cdots,x_n)x^{i - 1}+V_n(x_1,x_2,\cdots,x_n)x^n,
\end{align*}
比较上式等号两边$x^i(1\leq i\leq n)$的系数就能得到\eqref{23.138}. 
\end{proof}

\begin{example}
证明:
\begin{align}
\Delta_n(x_1,x_2,\cdots,x_n)=\left(\sum_{k = 1}^n x_k^2+\sum_{1\leq i<j\leq n}x_ix_j\right)V_n(x_1,x_2,\cdots,x_n) \label{23.139}
\end{align} 
\end{example}
\begin{proof}
不妨设\(x_i,1\leq i\leq n\)互不相同。设\(n + 1\)次多项式
\begin{align*}
P_{n + 1}(x)\triangleq\begin{vmatrix}
1&1&1&\cdots&1\\
-x&x_1&x_2&\cdots&x_n\\
(-x)^2&x_1^2&x_2^2&\cdots&x_n^2\\
\vdots&\vdots&\vdots&\ddots&\vdots\\
(-x)^{n - 1}&x_1^{n - 1}&x_2^{n - 1}&\cdots&x_n^{n - 1}\\
(-x)^{n + 1}&x_1^{n + 1}&x_2^{n + 1}&\cdots&x_n^{n + 1}
\end{vmatrix}
\end{align*}
注意到有\(n\)个根\(-x_1,-x_2,\cdots,-x_n\)。我们用\(-x_{n + 1}\)表示\(P_{n + 1}\)第\(n + 1\)个根。于是我们有
\begin{align}
P_{n + 1}(x)=c(x + x_1)(x + x_2)\cdots(x + x_n)(x + x_{n + 1}).\label{23.111}
\end{align}
将\(P_{n + 1}(x)\)按第一列展开得
\begin{align}\label{23.140}
P_{n + 1}(x)=-V_n(x_1,x_2,\cdots,x_n)x^{n + 1}+\Delta_n(x_1,x_2,\cdots,x_n)x^{n - 1}+a_{n - 2}x^{n - 2}+\cdots+a_0
\end{align}
其中\(a_{n - 2},\cdots,a_0\)是某些与\(x_j\)有关的\(n\)阶行列式。
比较\eqref{23.111}和\eqref{23.140}式的系数可知\(c = -V_n(x_1,x_2,\cdots,x_n)\).于是结合\eqref{23.111}式,并利用Vieta定理得
\begin{align}\label{23.141}
P_{n + 1}(x)=-V_n(x_1,x_2,\cdots,x_n)(x^{n + 1}+\delta_1x^n+\delta_2x^{n - 1}+\cdots+\delta_{n - 1})
\end{align}
这里\(\delta_j\)类似\eqref{23.137}式定义是\(x_1,x_2,\cdots,x_n,x_{n + 1}\)的初等对称多项式。
比较\eqref{23.140}\eqref{23.141}式的\(x^{n - 1}\)系数可得\(\Delta_n(x_1,x_2,\cdots,x_n)=-\delta_2V_n(x_1,x_2,\cdots,x_n)\)。
因为\(P_{n + 1}(x)\)没有\(x^n\)的项,所以
\begin{align*}
\delta _1=x_1+x_2+\cdots +x_{n+1}=0\Rightarrow x_{n+1}=-\left( x_1+x_2+\cdots +x_n \right) .
\end{align*}
从而
\begin{align*}
\delta _2&=\sum_{1\le i<j\le n+1}{x_ix_j}=\sum_{1\le i<j\le n}{x_ix_j}+x_{n+1}\sum_{i=1}^n{x_i}
\\
&=\sum_{1\le i<j\le n}{x_ix_j}-\left( x_1+x_2+\cdots +x_n \right) \sum_{i=1}^n{x_i}
\\
&=\sum_{1\le i<j\le n}{x_ix_j}-\left( \sum_{i=1}^n{x_i} \right) ^2=-\sum_{i=1}^n{x_{i}^{2}}-\sum_{1\le i<j\le n}{x_ix_j}.
\end{align*}
现在就有\eqref{23.139}成立。 
\end{proof}

\begin{proposition}\label{proposition:将矩阵拆分成Vandermode矩阵的形式}
设\(A = (a_{ij})_{n\times n},f_i(x)=a_{i1}+a_{i2}x+\cdots+a_{in}x^{n - 1}(i = 1,2,\cdots,n)\),证明:对任何复数\(x_1,x_2,\cdots,x_n\),都有
\begin{align*}
\begin{vmatrix}
f_1(x_1)&f_1(x_2)&\cdots&f_1(x_n)\\
f_2(x_1)&f_2(x_2)&\cdots&f_2(x_n)\\
\vdots&\vdots&\ddots&\vdots\\
f_n(x_1)&f_n(x_2)&\cdots&f_n(x_n)
\end{vmatrix}=|A|\cdot V_n(x_1,x_2,\cdots,x_n)
\end{align*}
这里\(V_n(x_1,x_2,\cdots,x_n)\)表示\(x_1,x_2,\cdots,x_n\)的Vandermonde行列式。
\end{proposition}
\begin{note}
关键是利用\hyperref[proposition:一些能写成两个向量乘积的矩阵]{命题\ref{proposition:一些能写成两个向量乘积的矩阵}}.
\end{note}
\begin{proof}
直接由矩阵乘法观察知显然。 
\end{proof}

\begin{corollary}\label{corollary:多项式行列式与Vandermode行列式相关}
设$f_k\left( x \right) =x^k+a_{k1}x^{k-1}+a_{k2}x^{k-2}+\cdots +a_{kk}
$,求下列行列式的值:
\begin{align*}
\left| \begin{matrix}
1&		f_1(x_1)&		f_2(x_1)&		\cdots&		f_{n-1}(x_1)\\
1&		f_1(x_2)&		f_2(x_2)&		\cdots&		f_{n-1}(x_2)\\
\vdots&		\vdots&		\vdots&		&		\vdots\\
1&		f_1(x_n)&		f_2(x_n)&		\cdots&		f_{n-1}(x_n)\\
\end{matrix} \right|.
\end{align*}
\end{corollary}
\begin{note}
知道这类行列式化简的操作即可.以后这种行列式化简操作不再作额外说明.
\end{note}
\begin{remark}
也可以由\hyperref[proposition:将矩阵拆分成Vandermode矩阵的形式]{命题\ref{proposition:将矩阵拆分成Vandermode矩阵的形式}}直接得到.
\end{remark}
\begin{solution}
{\color{blue}解法一:}
利用行列式的性质可得
\begin{align*}
&\left| \begin{matrix}
1&		f_1(x_1)&		f_2(x_1)&		\cdots&		f_{n-1}(x_1)\\
1&		f_1(x_2)&		f_2(x_2)&		\cdots&		f_{n-1}(x_2)\\
\vdots&		\vdots&		\vdots&		&		\vdots\\
1&		f_1(x_n)&		f_2(x_n)&		\cdots&		f_{n-1}(x_n)\\
\end{matrix} \right|
=\left| \begin{matrix}
1&		x_1+a_{11}&		x_{1}^{2}+a_{21}x_1+a_{22}&		\cdots&		x_{1}^{n-1}+a_{n-1,1}x_{1}^{n-2}+\cdots +a_{n-1,n-2}x_1+a_{n-1,n-1}\\
1&		x_2+a_{11}&		x_{2}^{2}+a_{21}x_2+a_{22}&		\cdots&		x_{2}^{n-1}+a_{n-1,1}x_{2}^{n-2}+\cdots +a_{n-1,n-2}x_2+a_{n-1,n-1}\\
\vdots&		\vdots&		\vdots&		&		\vdots\\
1&		x_n+a_{11}&		x_{n}^{2}+a_{21}x_n+a_{22}&		\cdots&		x_{n}^{n-1}+a_{n-1,1}x_{n}^{n-2}+\cdots +a_{n-1,n-2}x_n+a_{n-1,n-1}\\
\end{matrix} \right|
\\
&\xlongequal[\begin{array}{c}
\cdots\\
-a_{i,i-\left( n-3 \right)}j_{n-2}+j_{i+1},i=n-2,n-1\\
-a_{n-1,1}j_{n-1}+j_n\\
\end{array}]{\begin{array}{c}
-a_{ii}j_1+j_{i+1},i=1,2,\cdots n-1\\
-a_{i,i-1}j_2+j_{i+1},i=2,3,\cdots ,n-1\\
\end{array}}\left| \begin{matrix}
1&		x_1&		x_{1}^{2}&		\cdots&		x_{1}^{n-1}\\
1&		x_2&		x_{2}^{2}&		\cdots&		x_{2}^{n-1}\\
\vdots&		\vdots&		\vdots&		&		\vdots\\
1&		x_n&		x_{n}^{2}&		\cdots&		x_{n}^{n-1}\\
\end{matrix} \right|=\prod_{1\le i<j\le n}{\left( x_j-x_i \right) }.
\end{align*}

{\color{blue}解法二:}由\hyperref[proposition:将矩阵拆分成Vandermode矩阵的形式]{命题\ref{proposition:将矩阵拆分成Vandermode矩阵的形式}}可得
\begin{align*}
\left| \begin{matrix}
1&		f_1(x_1)&		f_2(x_1)&		\cdots&		f_{n-1}(x_1)\\
1&		f_1(x_2)&		f_2(x_2)&		\cdots&		f_{n-1}(x_2)\\
\vdots&		\vdots&		\vdots&		&		\vdots\\
1&		f_1(x_n)&		f_2(x_n)&		\cdots&		f_{n-1}(x_n)\\
\end{matrix} \right|&=\left| \begin{matrix}
1&		1&		\cdots&		1\\
f_1(x_1)&		f_1(x_2)&		\cdots&		f_1(x_n)\\
f_2(x_1)&		f_2(x_2)&		\cdots&		f_2(x_n)\\
\vdots&		\vdots&		&		\vdots\\
f_{n-1}(x_1)&		f_{n-1}(x_2)&		\cdots&		f_{n-1}(x_n)\\
\end{matrix} \right|=\left| \begin{matrix}
1&		1&		\cdots&		1\\
x_1+a_{11}&		x_2+a_{11}&		\cdots&		x_n+a_{11}\\
x_{1}^{2}+a_{21}x_1+a_{22}&		x_{2}^{2}+a_{21}x_2+a_{22}&		\cdots&		x_{n}^{2}+a_{21}x_n+a_{22}\\
\vdots&		\vdots&		&		\vdots\\
x_{1}^{n-1}+\cdots +a_{n-1,n-1}&		x_{2}^{n-1}+\cdots +a_{n-1,n-1}&		\cdots&		x_{n}^{n-1}+\cdots +a_{n-1,n-1}\\
\end{matrix} \right|
\\
&=V_n\left( x_1,x_2,\cdots ,x_n \right) \left| \begin{matrix}
1&		0&		0&		\cdots&		0\\
1&		1&		0&		\cdots&		0\\
1&		1&		1&		\cdots&		0\\
\vdots&		\vdots&		\vdots&		&		\vdots\\
1&		1&		1&		\cdots&		1\\
\end{matrix} \right|=V_n\left( x_1,x_2,\cdots ,x_n \right) =\prod_{1\le i<j\le n}{\left( x_j-x_i \right) .}
\end{align*}
\end{solution}

\begin{example}
计算
\begin{align*}
\begin{vmatrix}
1&1&1&\cdots&1\\
x_1 + 1&x_2 + 1&x_3 + 1&\cdots&x_n + 1\\
x_1^2 + x_1&x_2^2 + x_2&x_3^2 + x_3&\cdots&x_n^2 + x_n\\
\vdots&\vdots&\vdots&\ddots&\vdots\\
x_1^{n - 1} + x_1^{n - 2}&x_2^{n - 1} + x_2^{n - 2}&x_3^{n - 1} + x_3^{n - 2}&\cdots&x_n^{n - 1} + x_n^{n - 2}
\end{vmatrix}
\end{align*}
\end{example}
\begin{proof}
由\hyperref[proposition:将矩阵拆分成Vandermode矩阵的形式]{命题\ref{proposition:将矩阵拆分成Vandermode矩阵的形式}}我们知道
\begin{align*}
&\left| \begin{matrix}
1&		1&		1&		\cdots&		1\\
x_1+1&		x_2+1&		x_3+1&		\cdots&		x_n+1\\
x_{1}^{2}+x_1&		x_{2}^{2}+x_2&		x_{3}^{2}+x_3&		\cdots&		x_{n}^{2}+x_n\\
\vdots&		\vdots&		\vdots&		\ddots&		\vdots\\
x_{1}^{n-1}+x_{1}^{n-2}&		x_{2}^{n-1}+x_{2}^{n-2}&		x_{3}^{n-1}+x_{3}^{n-2}&		\cdots&		x_{n}^{n-1}+x_{n}^{n-2}\\
\end{matrix} \right|=V_n\left( x_1,x_2,\cdots ,x_n \right) \cdot \left| \begin{matrix}
1&		0&		0&		\cdots&		0&		0&		0\\
1&		1&		0&		\cdots&		0&		0&		0\\
0&		1&		1&		\cdots&		0&		0&		0\\
\vdots&		\vdots&		\vdots&		\ddots&		\vdots&		\vdots&		\vdots\\
0&		0&		0&		\cdots&		1&		1&		0\\
0&		0&		0&		\cdots&		0&		1&		1\\
\end{matrix} \right|
\\
&=\prod_{1\le i<j\le n}{(x_j}-x_i)\cdot \left| \begin{matrix}
1&		0&		0&		\cdots&		0&		0&		0\\
1&		1&		0&		\cdots&		0&		0&		0\\
0&		1&		1&		\cdots&		0&		0&		0\\
\vdots&		\vdots&		\vdots&		\ddots&		\vdots&		\vdots&		\vdots\\
0&		0&		0&		\cdots&		1&		1&		0\\
0&		0&		0&		\cdots&		0&		1&		1\\
\end{matrix} \right|=\prod_{1\le i<j\le n}{\left( x_j-x_i \right)}.
\end{align*} 
\end{proof}


\begin{proposition}\label{proposition:Vandermode行列式的"卷积"形式}
计算下列行列式的值:

\[
|\boldsymbol{A}|=\left| \begin{matrix}
a_{1}^{n-1}&		a_{1}^{n-2}b_1&		\cdots&		a_1b_{1}^{n-2}&		b_{1}^{n-1}\\
a_{2}^{n-1}&		a_{2}^{n-2}b_2&		\cdots&		a_2b_{2}^{n-2}&		b_{2}^{n-1}\\
\vdots&		\vdots&		&		\vdots&		\vdots\\
a_{n}^{n-1}&		a_{n}^{n-2}b_n&		\cdots&		a_nb_{n}^{n-2}&		b_{n}^{n-1}\\
\end{matrix} \right|.
\]
\end{proposition}
\begin{solution}
若所有的$a_i(i=1,2,\cdots,n)$都不为0,则有
\begin{align*}
|\boldsymbol{A}|&=\left| \begin{matrix}
a_{1}^{n-1}&		a_{1}^{n-2}b_1&		\cdots&		a_1b_{1}^{n-2}&		b_{1}^{n-1}\\
a_{2}^{n-1}&		a_{2}^{n-2}b_2&		\cdots&		a_2b_{2}^{n-2}&		b_{2}^{n-1}\\
\vdots&		\vdots&		&		\vdots&		\vdots\\
a_{n}^{n-1}&		a_{n}^{n-2}b_n&		\cdots&		a_nb_{n}^{n-2}&		b_{n}^{n-1}\\
\end{matrix} \right|=\prod_{i=1}^n{a_{i}^{n-1}}\left| \begin{matrix}
1&		\frac{b_1}{a_1}&		\cdots&		\frac{b_{1}^{n-2}}{a_{1}^{n-2}}&		\frac{b_{1}^{n-1}}{a_{1}^{n-1}}\\
1&		\frac{b_2}{a_2}&		\cdots&		\frac{b_{2}^{n-2}}{a_{2}^{n-2}}&		\frac{b_{2}^{n-1}}{a_{2}^{n-1}}\\
\vdots&		\vdots&		&		\vdots&		\vdots\\
1&		\frac{b_n}{a_n}&		\cdots&		\frac{b_{n}^{n-2}}{a_{n}^{n-2}}&		\frac{b_{n}^{n-2}}{a_{n}^{n-2}}\\
\end{matrix} \right|
\\
&=\prod_{i=1}^n{a_{i}^{n-1}}\prod_{1\le i<j\le n}{\left( \frac{b_j}{a_j}-\frac{b_i}{a_i} \right)}=\prod_{i=1}^n{a_{i}^{n-1}}\prod_{1\le i<j\le n}{\frac{a_ib_j-a_jb_i}{a_ja_i}}\hyperlink{连乘号计算小结论(1)}{=}\prod_{1\le i<j\le n}{(a_ib_j-a_jb_i)}.
\end{align*}
若只有一个$a_i$为0,则将原行列式按第$i$行展开得到具有相同类型的$n-1$阶行列式
\begin{align*}
|\boldsymbol{A}|&=\left| \begin{matrix}
a_{1}^{n-1}&		a_{1}^{n-2}b_1&		\cdots&		a_1b_{1}^{n-2}&		b_{1}^{n-1}\\
a_{2}^{n-1}&		a_{2}^{n-2}b_2&		\cdots&		a_2b_{2}^{n-2}&		b_{2}^{n-1}\\
\vdots&		\vdots&		&		\vdots&		\vdots\\
a_{i}^{n-1}&		a_{i}^{n-2}b_i&		\cdots&		a_ib_{i}^{n-2}&		b_{i}^{n-1}\\
\vdots&		\vdots&		&		\vdots&		\vdots\\
a_{n}^{n-1}&		a_{n}^{n-2}b_n&		\cdots&		a_nb_{n}^{n-2}&		b_{n}^{n-1}\\
\end{matrix} \right|=\left| \begin{matrix}
a_{1}^{n-1}&		a_{1}^{n-2}b_1&		\cdots&		a_1b_{1}^{n-2}&		b_{1}^{n-1}\\
a_{2}^{n-1}&		a_{2}^{n-2}b_2&		\cdots&		a_2b_{2}^{n-2}&		b_{2}^{n-1}\\
\vdots&		\vdots&		&		\vdots&		\vdots\\
0&		0&		\cdots&		0&		b_{i}^{n-1}\\
\vdots&		\vdots&		&		\vdots&		\vdots\\
a_{n}^{n-1}&		a_{n}^{n-2}b_n&		\cdots&		a_nb_{n}^{n-2}&		b_{n}^{n-1}\\
\end{matrix} \right|
\\
&\xlongequal{\text{按第}i\text{行展开}}\left( -1 \right) ^{n+i}b_{i}^{n-1}\left| \begin{matrix}
a_{1}^{n-1}&		a_{1}^{n-2}b_1&		\cdots&		a_1b_{1}^{n-2}\\
a_{2}^{n-1}&		a_{2}^{n-2}b_2&		\cdots&		a_2b_{2}^{n-2}\\
\vdots&		\vdots&		&		\vdots\\
a_{i-1}^{n-1}&		a_{i-1}^{n-2}b_{i-1}&		\cdots&		a_{i-1}b_{i-1}^{n-2}\\
a_{i+!}^{n-1}&		a_{i+1}^{n-2}b_{i+1}&		\cdots&		a_{i+1}b_{i+1}^{n-2}\\
\vdots&		\vdots&		&		\vdots\\
a_{n}^{n-1}&		a_{n}^{n-2}b_n&		\cdots&		a_nb_{n}^{n-2}\\
\end{matrix} \right|.
\end{align*}
此时同理可得
\begin{align*}
&|\boldsymbol{A}|=\left( -1 \right) ^{n+i}b_{i}^{n-1}\left| \begin{matrix}
a_{1}^{n-1}&		a_{1}^{n-2}b_1&		\cdots&		a_1b_{1}^{n-2}\\
a_{2}^{n-1}&		a_{2}^{n-2}b_2&		\cdots&		a_2b_{2}^{n-2}\\
\vdots&		\vdots&		&		\vdots\\
a_{i-1}^{n-1}&		a_{i-1}^{n-2}b_{i-1}&		\cdots&		a_{i-1}b_{i-1}^{n-2}\\
a_{i+1}^{n-1}&		a_{i+1}^{n-2}b_{i+1}&		\cdots&		a_{i+1}b_{i+1}^{n-2}\\
\vdots&		\vdots&		&		\vdots\\
a_{n}^{n-1}&		a_{n}^{n-2}b_n&		\cdots&		a_nb_{n}^{n-2}\\
\end{matrix} \right|=\left( -1 \right) ^{n+i}b_{i}^{n-1}\prod_{\substack{1\le k\le n\\
k\ne i\\}}{a_{k}^{n-1}}\left| \begin{matrix}
1&		\frac{b_1}{a_1}&		\cdots&		\frac{b_{1}^{n-2}}{a_{1}^{n-2}}\\
1&		\frac{b_2}{a_2}&		\cdots&		\frac{b_{2}^{n-2}}{a_{2}^{n-2}}\\
\vdots&		\vdots&		&		\vdots\\
1&		\frac{b_{i-1}}{a_{i-1}}&		\cdots&		\frac{b_{i-1}^{n-2}}{a_{i-1}^{n-2}}\\
1&		\frac{b_{i+1}}{a_{i+1}}&		\cdots&		\frac{b_{i+1}^{n-2}}{a_{i+1}^{n-2}}\\
\vdots&		\vdots&		&		\vdots\\
1&		\frac{b_n}{a_n}&		\cdots&		\frac{b_{n}^{n-2}}{a_{n}^{n-2}}\\
\end{matrix} \right|
\\
&=\left( -1 \right) ^{n+i}b_{i}^{n-1}\prod_{\substack{1\le k\le n\\
k\ne i\\}}{a_{k}^{n-1}}\prod_{\substack{1\le k<l\le n\\
k,l\ne i\\}}{\left( \frac{b_l}{a_l}-\frac{b_k}{a_k} \right)}=\left( -1 \right) ^{n+i}b_{i}^{n-1}\prod_{\substack{
1\le k\le n\\
k\ne i\\
}}{a_{k}^{n-1}}\prod_{\substack{
1\le k<l\le n\\
k,l\ne i\\
}}{\frac{a_kb_l-a_lb_k}{a_ka_l}}
\\
&\hyperlink{连乘号计算小结论(2)}{=}\left( -1 \right) ^{n+i}b_{i}^{n-1}\prod_{\substack{
1\le k\le n\\
k\ne i\\
}}{a_k}\cdot \prod_{\substack{
1\le k<l\le n\\
k,l\ne i\\
}}{\left( a_kb_l-a_lb_k \right)}=\left( -1 \right) ^{n-i}b_{i}^{n-1}\prod_{\substack{
1\le k\le n\\
k\ne i\\
}}{a_k}\cdot \prod_{\substack{
1\le k<l\le n\\
k,l\ne i\\
}}{\left( a_kb_l-a_lb_k \right)}
\\
&=\prod_{1\le k<i}{a_kb_i}\prod_{i<l\le n}{\left( -a_lb_i \right)}\cdot \prod_{\substack{
1\le k<l\le n\\
k,l\ne i\\
}}{\left( a_kb_l-a_lb_k \right)}
\\
&=\prod_{1\le k<l\le n}{\left( a_kb_l-a_lb_k \right)}.\left( a_i=0 \right).
\end{align*}
若至少有两个$a_i=a_j=0$,则第$i$行与第$j$行成比例,因此行列式的值等于0.经过计算发现,后面两种情形的答案都可以统一到第一种情形的答案.

综上所述,$|\boldsymbol{A}|=\prod_{1\le i<j\le n}{(a_ib_j-a_jb_i)}.$

\end{solution}
\begin{conclusion}\label{连乘号计算小技巧1}
\textbf{连乘号计算小结论:}

\hypertarget{连乘号计算小结论(1)}{(1)}$\prod_{1\le i<j\le n}{a_ia_j}=\prod_{i=1}^n{a_{i}^{n-1}}.$
\begin{align*}
\text{证明:}&\prod_{1\le i<j\le n}{a_ia_j}=\underset{n-1\text{组}}{\underbrace{a_2a_1\cdot a_3a_2a_3a_1\cdot a_4a_3a_4a_2a_4a_1\cdots \cdots \overset{k-1\text{对}}{\overbrace{a_ka_{k-1}a_ka_{k-2}\cdots a_ka_1}}\cdots \cdots \overset{n-1\text{对}}{\overbrace{a_na_{n-1}a_na_{n-2}\cdots a_na_1}}}}
\\
&\xlongequal{\text{从左往右按组计数}}a_{1}^{n-1}a_{2}^{1+n-2}a_{3}^{2+n-3}a_{4}^{3+n-4}\cdots a_{k}^{k-1+n-k}\cdots a_{n}^{n-1}=\prod_{i=1}^n{a_{i}^{n-1}}.
\end{align*}
\hypertarget{连乘号计算小结论(2)}{(2)}$\prod_{\substack{1\le i<j\le n\\i,j\ne k}}{a_ia_j}=\prod_{\substack{
1\le i\le n\\
i\ne k\\}}{a_{i}^{n-2}}$,其中$k\in [1,n]\cap \mathbb{N_+}$.

\begin{align*}
\text{证明:}&\prod_{\substack{
1\le i<j\le n\\
i,j\ne k\\
}}{a_ia_j}=\underset{n-2\text{组}}{\underbrace{a_2a_1\cdot a_3a_2a_3a_1\cdots \cdots \overset{k-2\text{对}}{\overbrace{a_{k-1}a_{k-2}\cdots a_{k-1}a_1}}\cdot \overset{k-1\text{对}}{\overbrace{a_{k+1}a_{k-1}\cdots a_{k+1}a_1}}\cdots \cdots \overset{n-2\text{对}}{\overbrace{a_na_{n-1}\cdots a_na_{k+1}a_na_{k-1}\cdots a_na_1}}}}
\\
&\xlongequal{\text{从左往右按组计数}}a_{1}^{n-2}a_{2}^{1+n-3}a_{3}^{2+n-4}a_{4}^{3+n-4}\cdots a_{k-1}^{k-2+n-k}a_{k+1}^{k-1+n-k-1}\cdots a_{n}^{n-2}=\prod_{\substack{
1\le i\le n\\
i\ne k\\}}{a_{i}^{n-2}}.
\end{align*}
注意:从第$k-1$组开始,后面每组都比原来少一对(后面每组均缺少原本含$a_k$的那一对).
\end{conclusion}

\begin{example}
计算$D_{n+1}=\left| \begin{matrix}
\left( a_0+b_0 \right) ^n&		\left( a_0+b_1 \right) ^n&		\cdots&		\left( a_0+b_n \right) ^n\\
\left( a_1+b_0 \right) ^n&		\left( a_1+b_1 \right) ^n&		\cdots&		\left( a_1+b_n \right) ^n\\
\vdots&		\vdots&		\vdots&		\\
\left( a_n+b_0 \right) ^n&		\left( a_n+b_1 \right) ^n&		\cdots&		\left( a_n+b_n \right) ^n\\
\end{matrix} \right|$.
\end{example}
\begin{solution}
由二项式定理可知
\begin{align*}
\left( a_i+b_j \right) ^n={a_i}^n+\mathrm{C}_{n}^{1}{a_i}^{n-1}b_j+\cdots +\mathrm{C}_{n}^{n-1}a_i{b_j}^{n-1}+{b_j}^n,\text{其中}i,j=0,1,\cdots ,n.
\nonumber
\end{align*}
从而
\begin{align*}
D_{n+1}&=\left| \begin{matrix}
{a_0}^n+\mathrm{C}_{n}^{1}{a_0}^{n-1}b_0+\cdots +\mathrm{C}_{n}^{n-1}a_0{b_0}^{n-1}+{b_0}^n&		\cdots&		{a_0}^n+\mathrm{C}_{n}^{1}{a_0}^{n-1}b_n+\cdots +\mathrm{C}_{n}^{n-1}a_0{b_n}^{n-1}+{b_n}^n\\
{a_1}^n+\mathrm{C}_{n}^{1}{a_1}^{n-1}b_0+\cdots +\mathrm{C}_{n}^{n-1}a_1{b_0}^{n-1}+{b_0}^n&		\cdots&		{a_1}^n+\mathrm{C}_{n}^{1}{a_1}^{n-1}b_n+\cdots +\mathrm{C}_{n}^{n-1}a_1{b_n}^{n-1}+{b_n}^n\\
\vdots&		&		\vdots\\
{a_{n-1}}^n+\mathrm{C}_{n}^{1}{a_{n\-1}}^{n-1}b_0+\cdots +\mathrm{C}_{n}^{n-1}a_{n-1}{b_0}^{n-1}+{b_0}^n&		\cdots&		{a_{n-1}}^n+\mathrm{C}_{n}^{1}{a_{n-1}}^{n-1}b_n+\cdots +\mathrm{C}_{n}^{n-1}a_{n-1}{b_n}^{n-1}+{b_n}^n\\
{a_n}^n+\mathrm{C}_{n}^{1}{a_n}^{n-1}b_0+\cdots +\mathrm{C}_{n}^{n-1}a_n{b_0}^{n-1}+{b_0}^n&		\cdots&		{a_n}^n+\mathrm{C}_{n}^{1}{a_n}^{n-1}b_n+\cdots +\mathrm{C}_{n}^{n-1}a_n{b_n}^{n-1}+{b_n}^n\\
\end{matrix} \right|
\\
&=\left| \begin{matrix}
{a_0}^n&		{a_0}^{n-1}&		\cdots&		a_0&		1\\
{a_1}^n&		{a_1}^{n-1}&		\cdots&		a_1&		1\\
\vdots&		\vdots&		&		\vdots&		\vdots\\
{a_{n-1}}^n&		{a_{n-1}}^{n-1}&		\cdots&		a_{n-1}&		1\\
{a_n}^n&		{a_n}^{n-1}&		\cdots&		a_n&		1\\
\end{matrix} \right|\cdot \left| \begin{matrix}
1&		1&		\cdots&		1&		1\\
\mathrm{C}_{n}^{1}b_0&		\mathrm{C}_{n}^{1}b_1&		\cdots&		\mathrm{C}_{n}^{1}b_{n-1}&		\mathrm{C}_{n}^{1}b_n\\
\vdots&		\vdots&		&		\vdots&		\vdots\\
\mathrm{C}_{n}^{n-1}{b_0}^{n-1}&		\mathrm{C}_{n}^{n-1}{b_1}^{n-1}&		\cdots&		\mathrm{C}_{n}^{n-1}{b_{n-1}}^{n-1}&		\mathrm{C}_{n}^{n-1}{b_n}^{n-1}\\
{b_0}^n&		{b_1}^n&		\cdots&		{b_{n-1}}^n&		{b_n}^n\\
\end{matrix} \right|
\\
&\xlongequal{\hyperlink{行列式计算常识}{\text{列倒排}}}\left( -1 \right) ^{\frac{n\left( n+1 \right)}{2}}\left| \begin{matrix}
1&		a_0&		\cdots&		{a_0}^{n-1}&		{a_0}^n\\
1&		a_1&		\cdots&		{a_1}^{n-1}&		{a_1}^n\\
\vdots&		\vdots&		&		\vdots&		\vdots\\
1&		a_n&		\cdots&		{a_n}^{n-1}&		{a_n}^n\\
\end{matrix} \right|\cdot \prod_{i=1}^{n-1}{\mathrm{C}_{n}^{i}\left| \begin{matrix}
1&		1&		\cdots&		1&		1\\
b_0&		b_1&		\cdots&		b_{n-1}&		b_n\\
\vdots&		\vdots&		&		\vdots&		\vdots\\
{b_0}^{n-1}&		{b_1}^{n-1}&		\cdots&		{b_{n-1}}^{n-1}&		{b_n}^{n-1}\\
{b_0}^n&		{b_1}^n&		\cdots&		{b_{n-1}}^n&		{b_n}^n\\
\end{matrix} \right|}
\\
&=\left( -1 \right) ^{\frac{n\left( n+1 \right)}{2}}\prod_{0\le j<i\le n}{\left( a_i-a_j \right)}\prod_{i=1}^{n-1}{\mathrm{C}_{n}^{i}\prod_{0\le j<i\le n}{\left( b_i-b_j \right)}}
= =\prod_{i=1}^{n-1}{\mathrm{C}_{n}^{i}\prod_{0\le j<i\le n}{\left( a_j-a_i \right) \left( b_i-b_j \right)}}.
\end{align*}
\end{solution}

\begin{example}\label{Vandermode行列式三角函数例题}
求下列行列式的值:
\begin{align*}
|\boldsymbol{A}|=\left| \begin{matrix}
1&		\cos \theta _1&		\cos 2\theta _1&		\cdots&		\cos\mathrm{(}n-1)\theta _1\\
1&		\cos \theta _2&		\cos 2\theta _2&		\cdots&		\cos\mathrm{(}n-1)\theta _2\\
\vdots&		\vdots&		\vdots&		&		\vdots\\
1&		\cos \theta _n&		\cos 2\theta _n&		\cdots&		\cos\mathrm{(}n-1)\theta _n\\
\end{matrix} \right|.
\end{align*}
\end{example}
\begin{solution}
由De Moivre公式及二项式定理,可得
\begin{align*}
&\cos k\theta +\mathrm{i}\sin k\theta =(\cos \theta +\mathrm{i}\sin \theta )^k
\\
&=\cos ^k\theta +\mathrm{iC}_{k}^{1}\cos ^{k-1}\theta \sin \theta -\mathrm{C}_{k}^{2}\cos ^{k-2}\theta \sin ^2\theta +\mathrm{iC}_{k}^{3}\cos ^{k-3}\theta \sin ^3\theta -\cdots 
\\
&=\cos ^k\theta +\mathrm{iC}_{k}^{1}\cos ^{k-1}\theta \sin \theta -\mathrm{C}_{k}^{2}\cos ^{k-2}\theta \left( 1-\cos ^2\theta \right) +\mathrm{iC}_{k}^{3}\cos ^{k-3}\theta \sin ^3\theta -\cdots 
\end{align*}
比较实部可得
\begin{align*}
\cos k\theta& =\cos ^k\theta \left( 1+\mathrm{C}_{k}^{2}+\mathrm{C}_{k}^{4}+\cdots \right) -\mathrm{C}_{k}^{2}\cos ^{k-2}+\mathrm{C}_{k}^{4}\cos ^{k-4}-\cdots 
\\
&\hyperlink{组合式计算常用公式}{=}2^{k-1}\cos ^k\theta -\mathrm{C}_{k}^{2}\cos ^{k-2}+\mathrm{C}_{k}^{4}\cos ^{k-4}-\cdots 
\end{align*}

利用这个事实,依次将原行列式各列表示成\(\cos \theta _j\)(\(j = 2,3,\cdots,n\))的多项式.

再利用行列式的性质,可依次将第\(3,4,\cdots,n\)列消去除最高次项外的其他项,从而得到
\begin{align*}
|\boldsymbol{A}|&=\left| \begin{matrix}
1&		\cos \theta _1&		2\cos ^2\theta _1&		\cdots&		2^{n-2}\cos ^{n-1}\theta _1\\
1&		\cos \theta _2&		2\cos ^2\theta _2&		\cdots&		2^{n-2}\cos ^{n-1}\theta _2\\
\vdots&		\vdots&		\vdots&		&		\vdots\\
1&		\cos \theta _n&		2\cos ^2\theta _n&		\cdots&		2^{n-2}\cos ^{n-1}\theta _n\\
\end{matrix} \right|=2^{\frac{1}{2}(n-1)(n-2)}\left| \begin{matrix}
1&		\cos \theta _1&		\cos ^2\theta _1&		\cdots&		\cos ^{n-1}\theta _1\\
1&		\cos \theta _2&		\cos ^2\theta _2&		\cdots&		\cos ^{n-1}\theta _2\\
\vdots&		\vdots&		\vdots&		&		\vdots\\
1&		\cos \theta _n&		\cos ^2\theta _n&		\cdots&		\cos ^{n-1}\theta _n\\
\end{matrix} \right|
\\
&=2^{\frac{1}{2}(n-1)(n-2)}\prod_{1\le i<j\le n}{\left( \cos \theta _j-\cos \theta _i \right)}.
\end{align*}
\end{solution}
\begin{conclusion}
\hypertarget{组合式计算常用公式}{组合式计算常用公式:}

(1)$\mathrm{C}_{n}^{m}=\mathrm{C}_{n-1}^{m}+\mathrm{C}_{n-1}^{m-1}$

(2)$\mathrm{C}_{n}^{0}+\mathrm{C}_{n}^{2}+\cdots =\mathrm{C}_{n}^{1}+\mathrm{C}_{n}^{3}+\cdots =2^{n-1}$

证明:(1)\begin{align*}
\mathrm{C}_{n}^{m}&=\frac{n!}{m!\left( n-m \right) !}=\frac{\left( n-1 \right) !\left( n-m+m \right)}{m!\left( n-m \right) !}=\frac{\left( n-1 \right) !\left( n-m \right)}{m!\left( n-m \right) !}+\frac{\left( n-1 \right) !m}{m!\left( n-m \right) !}
\\
&=\frac{\left( n-1 \right) !}{m!\left( n-m-1 \right) !}+\frac{\left( n-1 \right) !}{\left( m-1 \right) !\left( n-m \right) !}=\mathrm{C}_{n-1}^{m}+\mathrm{C}_{n-1}^{m-1}
\end{align*}
(2)(i)当\(n\)为奇数时,由\(C_{n}^{m}=C_{n - 1}^{m - 1} + C_{n - 1}^{m}\),可得
\begin{align*}
&C_{n}^{0} + C_{n}^{2} + C_{n}^{4} \cdots + C_{n}^{n - 1} = C_{n - 1}^{0} + C_{n - 1}^{1} + C_{n - 1}^{2} + C_{n - 1}^{3} + C_{n - 1}^{4} \cdots + C_{n - 1}^{n - 2} + C_{n - 1}^{n - 1}
\\
&C_{n}^{1} + C_{n}^{3} + C_{n}^{5} \cdots + C_{n}^{n} = C_{n - 1}^{0} + C_{n - 1}^{1} + C_{n - 1}^{2} + C_{n - 1}^{3} + C_{n - 1}^{4} + C_{n - 1}^{5} + \cdots + C_{n - 1}^{n - 1} + C_{n - 1}^{n}
\end{align*}
由于\(C_{n - 1}^{n} = 0\),再对比上面两式每一项可知,上面两式相等.

而上面两式相加,得$
C_{n}^{0} + C_{n}^{1} + C_{n}^{2} \cdots + C_{n}^{n - 1} + C_{n}^{n} = (1 + 1)^n = 2^n.$

故\(C_{n}^{0} + C_{n}^{2} + C_{n}^{4} \cdots + C_{n}^{n - 1} = C_{n}^{1} + C_{n}^{3} + C_{n}^{5} \cdots + C_{n}^{n} = 2^{n - 1}\).

(ii)当\(n\)为偶数时,由\(C_{n}^{m} = C_{n - 1}^{m - 1} + C_{n - 1}^{m}\),可得
\begin{align*}
&C_{n}^{0} + C_{n}^{2} + C_{n}^{4} \cdots + C_{n}^{n} = C_{n - 1}^{0} + C_{n - 1}^{1} + C_{n - 1}^{2} + C_{n - 1}^{3} + C_{n - 1}^{4} \cdots + C_{n - 1}^{n - 1} + C_{n - 1}^{n} 
\\
&C_{n}^{1} + C_{n}^{3} + C_{n}^{5} \cdots + C_{n}^{n - 1} = C_{n - 1}^{0} + C_{n - 1}^{1} + C_{n - 1}^{2} + C_{n - 1}^{3} + C_{n - 1}^{4} + C_{n - 1}^{5} + \cdots + C_{n - 1}^{n - 2} + C_{n - 1}^{n - 1}
\end{align*}
由于\(C_{n - 1}^{n} = 0\),再对比上面两式每一项可知,上面两式相等.

而上面两式相加,得
$C_{n}^{0} + C_{n}^{1} + C_{n}^{2} \cdots + C_{n}^{n - 1} + C_{n}^{n} = (1 + 1)^n = 2^n.$

故\(C_{n}^{0} + C_{n}^{2} + C_{n}^{4} \cdots + C_{n}^{n - 1} = C_{n}^{1} + C_{n}^{3} + C_{n}^{5} \cdots + C_{n}^{n} = 2^{n - 1}\).

综上所述,\(C_{n}^{0} + C_{n}^{2} + \cdots = C_{n}^{1} + C_{n}^{3} + \cdots = 2^{n - 1}\). 
\end{conclusion}

\begin{example}
求下列行列式式的值:
\begin{align*}
|\boldsymbol{A}|=\left| \begin{matrix}
\sin \theta _1&		\sin 2\theta _1&		\cdots&		\sin n\theta _1\\
\sin \theta _2&		\sin 2\theta _2&		\cdots&		\sin n\theta _2\\
\vdots&		\vdots&		&		\vdots\\
\sin \theta _n&		\sin 2\theta _n&		\cdots&		\sin n\theta _n\\
\end{matrix} \right|.
\end{align*}
\end{example}
\begin{note}
可以利用\hyperref[Vandermode行列式三角函数例题]{上一题}类似的方法求解.但我们给出另外一种解法,目的是直接利用\hyperref[Vandermode行列式三角函数例题]{上一题}的结论.
\end{note}
\begin{solution}
根据和差化积公式,可得
\begin{align*}
\sin k\theta -\sin \left( k-2 \right) \theta =2\sin \theta \cos \left( k-1 \right) \theta ,k=2,3,\cdots ,n.
\end{align*}
再结合上一题结论,可得
\begin{align*}
|\boldsymbol{A}|&=\left| \begin{matrix}
\sin \theta _1&		\sin 2\theta _1&		\cdots&		\sin n\theta _1\\
\sin \theta _2&		\sin 2\theta _2&		\cdots&		\sin n\theta _2\\
\vdots&		\vdots&		&		\vdots\\
\sin \theta _n&		\sin 2\theta _n&		\cdots&		\sin n\theta _n\\
\end{matrix} \right|=\left| \begin{matrix}
\sin \theta _1&		2\sin \theta _1\cos \theta _1&		\cdots&		2\sin \theta _1\cos \left( n-1 \right) \theta _1\\
\sin \theta _2&		2\sin \theta _2\cos \theta _2&		\cdots&		2\sin \theta _2\cos \left( n-1 \right) \theta _2\\
\vdots&		\vdots&		&		\vdots\\
\sin \theta _n&		2\sin \theta _n\cos \theta _n&		\cdots&		2\sin \theta _n\cos \left( n-1 \right) \theta _n\\
\end{matrix} \right|
\\
&=2^{n-1}\prod_{i=1}^n{\sin \theta _i}\left| \begin{matrix}
\cos \theta _1&		\cos 2\theta _1&		\cdots&		\cos (n-1)\theta _1\\
\cos \theta _2&		\cos 2\theta _2&		\cdots&		\cos (n-1)\theta _2\\
\vdots&		\vdots&		&		\vdots\\
\cos \theta _n&		\cos 2\theta _n&		\cdots&		\cos (n-1)\theta _n\\
\end{matrix} \right|=2^{\frac{1}{2}\left( n-2 \right) \left( n-1 \right) +n-1}\prod_{i=1}^n{\sin \theta _i}\prod_{1\le i<j\le n}{\left( \cos \theta _j-\cos \theta _i \right)}
\\
&=2^{\frac{1}{2}n\left( n-1 \right)}\prod_{i=1}^n{\sin \theta _i}\prod_{1\le i<j\le n}{\left( \cos \theta _j-\cos \theta _i \right)}.
\end{align*}
\end{solution}


\begin{proposition}[\hypertarget{多项式根的有限性}{多项式根的有限性}]\label{proposition:多项式根的有限性}
设多项式
\[
f(x)=a_nx^n + a_{n - 1}x^{n - 1}+\cdots+a_1x + a_0
\]
若\(f(x)\)有\(n + 1\)个不同的根\(b_1,b_2,\cdots,b_{n+1}\),即\(f(b_1)=f(b_2)=\cdots=f(b_{n+1})=0\),
求证:\(f(x)\)是零多项式,即\(a_n=a_{n - 1}=\cdots=a_1=a_0 = 0\).
\end{proposition}
\begin{proof}
由\(f(b_1)=f(b_2)=\cdots=f(b_{n+1})=0\),可知$x_0=a_0,x_1=a_1,\cdots ,x_{n-1}=a_{n-1},x_n=a_n$是下列线性方程组的解:
\begin{align*}
\left\{ \begin{aligned}
&x_0+b_1x_1+\cdots +b_{1}^{n-1}x_{n-1}+b_{1}^{n}x_n=0,\\
&x_0+b_2x_1+\cdots +b_{2}^{n-1}x_{n-1}+b_{2}^{n}x_n=0,\\
&\qquad \qquad \qquad \cdots \cdots \cdots \cdots\\
&x_0+b_{n+1}x_1+\cdots +b_{n+1}^{n-1}x_{n-1}+b_{n+1}^{n}x_n=0.\\
\end{aligned} \right. 
\end{align*}
上述线性方程组的系数行列式是一个Vandermode行列式,由于$b_1,b_2,\cdots,b_{n+1}$互不相同,所以系数行列式不等于零.由Crammer法则可知上述方程组只有零解.即有$a_n=a_{n - 1}=\cdots=a_1=a_0 = 0$.
\end{proof}






\end{document}
\documentclass[../../main.tex]{subfiles}
\graphicspath{{\subfix{../../image/}}} % 指定图片目录,后续可以直接使用图片文件名。

% 例如:
% \begin{figure}[H]
% \centering
% \includegraphics[scale=0.3]{image-01.01}
% \caption{图片标题}
% \label{figure:image-01.01}
% \end{figure}
% 注意:上述\label{}一定要放在\caption{}之后,否则引用图片序号会只会显示??.

\begin{document}

\section{Laplace定理}

\begin{example}
利用行列式的Laplace定理证明恒等式:
\[
(ab' - a'b)(cd' - c'd)-(ac' - a'c)(bd' - b'd)+(ad' - a'd)(bc' - b'c)=0.
\]
\end{example}
\begin{solution}
显然下列行列式的值为零:
\begin{align*}
\left| \begin{matrix}
a&		a^{\prime}&		a&		a^{\prime}\\
b&		b^{\prime}&		b&		b^{\prime}\\
c&		c^{\prime}&		c&		c^{\prime}\\
d&		d^{\prime}&		d&		d^{\prime}\\
\end{matrix} \right|.
\end{align*}
利用$Laplace$定理按第一、二列展开得
\begin{align*}
\left| \begin{matrix}
a&		a^{\prime}&		a&		a^{\prime}\\
b&		b^{\prime}&		b&		b^{\prime}\\
c&		c^{\prime}&		c&		c^{\prime}\\
d&		d^{\prime}&		d&		d^{\prime}\\
\end{matrix} \right|&=\left( -1 \right) ^{1+2+1+2}\left| \begin{matrix}
a&		a^{\prime}\\
b&		b^{\prime}\\
\end{matrix} \right|\left| \begin{matrix}
c&		c^{\prime}\\
d&		d^{\prime}\\
\end{matrix} \right|+\left( -1 \right) ^{1+2+1+3}\left| \begin{matrix}
a&		a^{\prime}\\
c&		c^{\prime}\\
\end{matrix} \right|\left| \begin{matrix}
b&		b^{\prime}\\
d&		d^{\prime}\\
\end{matrix} \right|+\left( -1 \right) ^{1+2+1+4}\left| \begin{matrix}
a&		a^{\prime}\\
d&		d^{\prime}\\
\end{matrix} \right|\left| \begin{matrix}
b&		b^{\prime}\\
c&		c^{\prime}\\
\end{matrix} \right|
\\
&\quad+\left( -1 \right) ^{1+2+2+3}\left| \begin{matrix}
b&		b^{\prime}\\
c&		c^{\prime}\\
\end{matrix} \right|\left| \begin{matrix}
a&		a^{\prime}\\
d&		d^{\prime}\\
\end{matrix} \right|+\left( -1 \right) ^{1+2+2+4}\left| \begin{matrix}
b&		b^{\prime}\\
d&		d^{\prime}\\
\end{matrix} \right|\left| \begin{matrix}
a&		a^{\prime}\\
c&		c^{\prime}\\
\end{matrix} \right|
\\
&\quad+\left( -1 \right) ^{1+2+3+4}\left| \begin{matrix}
c&		c^{\prime}\\
d&		d^{\prime}\\
\end{matrix} \right|\left| \begin{matrix}
a&		a^{\prime}\\
b&		b^{\prime}\\
\end{matrix} \right|
\\
&=2\left| \begin{matrix}
a&		a^{\prime}\\
b&		b^{\prime}\\
\end{matrix} \right|\left| \begin{matrix}
c&		c^{\prime}\\
d&		d^{\prime}\\
\end{matrix} \right|-2\left| \begin{matrix}
a&		a^{\prime}\\
c&		c^{\prime}\\
\end{matrix} \right|\left| \begin{matrix}
b&		b^{\prime}\\
d&		d^{\prime}\\
\end{matrix} \right|+2\left| \begin{matrix}
a&		a^{\prime}\\
d&		d^{\prime}\\
\end{matrix} \right|\left| \begin{matrix}
b&		b^{\prime}\\
c&		c^{\prime}\\
\end{matrix} \right|=0.
\end{align*}
上式等价于
\begin{align*}
\left| \begin{matrix}
a&		a^{\prime}\\
b&		b^{\prime}\\
\end{matrix} \right|\left| \begin{matrix}
c&		c^{\prime}\\
d&		d^{\prime}\\
\end{matrix} \right|-\left| \begin{matrix}
a&		a^{\prime}\\
c&		c^{\prime}\\
\end{matrix} \right|\left| \begin{matrix}
b&		b^{\prime}\\
d&		d^{\prime}\\
\end{matrix} \right|+\left| \begin{matrix}
a&		a^{\prime}\\
d&		d^{\prime}\\
\end{matrix} \right|\left| \begin{matrix}
b&		b^{\prime}\\
c&		c^{\prime}\\
\end{matrix} \right|=0.
\end{align*}
整理可得
\begin{align*}
(ab' - a'b)(cd' - c'd)-(ac' - a'c)(bd' - b'd)+(ad' - a'd)(bc' - b'c)=0.
\end{align*}
\end{solution}

\begin{example}
求\(2n\)阶行列式的值(空缺处都是零):
\begin{align*}
\left| \begin{matrix}
a&		&		&		&		&		b\\
&		\ddots&		&		&		\begin{turn}{80}$\ddots$\end{turn}&		\\
&		&		a&		b&		&		\\
&		&		b&		a&		&		\\
&		\begin{turn}{80}$\ddots$\end{turn}&		&		&		\ddots&		\\
b&		&		&		&		&		a\\
\end{matrix} \right|.
\end{align*}
\end{example}
\begin{solution}
设原行列式为$D_{2n}$,其中$2n$为行列式的阶数.
不断用$Laplace$定理按第一行及最后一行展开,可得
\begin{align*}
D_{2n}=\left| \begin{matrix}
a&		&		&		&		&		b\\
&		\ddots&		&		&		\begin{turn}{80}$\ddots$\end{turn}&		\\
&		&		a&		b&		&		\\
&		&		b&		a&		&		\\
&		\begin{turn}{80}$\ddots$\end{turn}&		&		&		\ddots&		\\
b&		&		&		&		&		a\\
\end{matrix} \right|\xlongequal[]{\text{按第一行及最后一行展开}}\left| \begin{matrix}
a&		b\\
b&		a\\
\end{matrix} \right|D_{2n-2}=\left( a^2-b^2 \right) D_{2\left( n-1 \right)}.
\end{align*}
进而,由上述递推式可得
\begin{align*}
D_{2n}&=\left( a^2-b^2 \right) D_{2\left( n-1 \right)}=\left( a^2-b^2 \right) ^2D_{2\left( n-2 \right)}=\cdots =\left( a^2-b^2 \right) ^{n-1}D_2
\\
&=\left( a^2-b^2 \right) ^{n-1}\left| \begin{matrix}
a&		b\\
b&		a\\
\end{matrix} \right|=\left( a^2-b^2 \right) ^n.
\end{align*}
\end{solution}


\end{document}
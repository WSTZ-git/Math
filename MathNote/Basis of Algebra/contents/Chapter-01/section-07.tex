\documentclass[../../main.tex]{subfiles}
\graphicspath{{\subfix{../../image/}}} % 指定图片目录,后续可以直接使用图片文件名。

% 例如:
% \begin{figure}[h]
% \centering
% \includegraphics{image-01.01}
% \caption{图片标题}
% \label{fig:image-01.01}
% \end{figure}
% 注意:上述\label{}一定要放在\caption{}之后,否则引用图片序号会只会显示??.

\begin{document}

\section{递推法与数学归纳法}

\begin{proposition}[\hypertarget{三对角行列式}{三对角行列式}]\label{三对角行列式}
求下列行列式的递推关系式(空白处均为0):
\begin{equation}
\begin{split}
D_n=\left| \begin{matrix}
a_1&		b_1&		&		&		&		\\
c_1&		a_2&		b_2&		&		&		\\
&		c_2&		a_3&		\ddots&		&		\\
&		&		\ddots&		\ddots&		\ddots&		\\
&		&		&		\ddots&		a_{n-1}&		b_{n-1}\\
&		&		&		&		c_{n-1}&		a_n\\
\end{matrix} \right|.
\end{split}
\nonumber
\end{equation}
\end{proposition}
\begin{note}
记忆三对角行列式的计算方法和结果:
$\boldsymbol{D}_{\boldsymbol{n}}=\boldsymbol{a}_{\boldsymbol{n}}\boldsymbol{D}_{\boldsymbol{n}-\boldsymbol{1}}-\boldsymbol{b}_{\boldsymbol{n}-\boldsymbol{1}}\boldsymbol{c}_{\boldsymbol{n}-\boldsymbol{1}}\boldsymbol{D}_{\boldsymbol{n}-\boldsymbol{2}}\boldsymbol{(n}\ge \boldsymbol{2)}$,

即按最后一列(或行)展开得到递推公式.
\end{note}
\begin{solution}
显然$D_0=1,D_1=a_1$.当$n\ge2$时,我们有
\begin{align*}
D_n&=\left| \begin{matrix}
a_1&		b_1&		&		&		&		\\
c_1&		a_2&		b_2&		&		&		\\
&		c_2&		a_3&		\ddots&		&		\\
&		&		\ddots&		\ddots&		\ddots&		\\
&		&		&		\ddots&		a_{n-1}&		b_{n-1}\\
&		&		&		&		c_{n-1}&		a_n\\
\end{matrix} \right|=\left| \begin{matrix}
a_1&		b_1&		&		&		&		&		\\
c_1&		a_2&		b_2&		&		&		&		\\
&		c_2&		a_3&		\ddots&		&		&		\\
&		&		\ddots&		\ddots&		\ddots&		&		\\
&		&		&		\ddots&		a_{n-2}&		b_{n-2}&		\\
&		&		&		&		c_{n-2}&		a_{n-1}&		b_{n-1}\\
&		&		&		&		&		c_{n-1}&		a_n\\
\end{matrix} \right|
\\
&\xlongequal[]{\text{按最后一列展开}}a_n\left| \begin{matrix}
a_1&		b_1&		&		&		&		\\
c_1&		a_2&		b_2&		&		&		\\
&		c_2&		a_3&		\ddots&		&		\\
&		&		\ddots&		\ddots&		\ddots&		\\
&		&		&		\ddots&		a_{n-2}&		b_{n-2}\\
&		&		&		&		c_{n-2}&		a_{n-1}\\
\end{matrix} \right|-b_{n-1}\left| \begin{matrix}
a_1&		b_1&		&		&		&		&		\\
c_1&		a_2&		b_2&		&		&		&		\\
&		c_2&		a_3&		\ddots&		&		&		\\
&		&		\ddots&		\ddots&		\ddots&		&		\\
&		&		&		\ddots&		a_{n-3}&		b_{n-3}&		\\
&		&		&		&		c_{n-3}&		a_{n-2}&		b_{n-2}\\
&		&		&		&		&		0&		c_{n-1}\\
\end{matrix} \right|
\\
&\xlongequal[]{\text{第二项按最后}\mathbf{一行}\text{展开}}a_n\left| \begin{matrix}
a_1&		b_1&		&		&		&		\\
c_1&		a_2&		b_2&		&		&		\\
&		c_2&		a_3&		\ddots&		&		\\
&		&		\ddots&		\ddots&		\ddots&		\\
&		&		&		\ddots&		a_{n-2}&		b_{n-2}\\
&		&		&		&		c_{n-2}&		a_{n-1}\\
\end{matrix} \right|-b_{n-1}c_{n-1}\left| \begin{matrix}
a_1&		b_1&		&		&		&		\\
c_1&		a_2&		b_2&		&		&		\\
&		c_2&		a_3&		\ddots&		&		\\
&		&		\ddots&		\ddots&		\ddots&		\\
&		&		&		\ddots&		a_{n-3}&		b_{n-3}\\
&		&		&		&		c_{n-3}&		a_{n-2}\\
\end{matrix} \right|
\\
&=a_nD_{n-1}-b_{n-1}c_{n-1}D_{n-2}.
\nonumber
\end{align*}
\end{solution}

\begin{corollary}\label{corollary:对角线元素相同的三对角行列式}
计算$n$阶行列式$(bc\ne0)$:
\begin{equation}
\begin{split}
D_n=\left| \begin{matrix}
a&		b&		&		&		&		\\
c&		a&		b&		&		&		\\
&		c&		a&		b&		&		\\
&		&		\ddots&		\ddots&		\ddots&		\\
&		&		&		c&		a&		b\\
&		&		&		&		c&		a\\
\end{matrix} \right|.
\end{split}
\nonumber
\end{equation}
\end{corollary}
\begin{note}
解递推式:$D_n=aD_{n-1}-bcD_{n-2}(n\ge2)$对应的特征方程:$x^2-ax+bc=0$得到两根$\alpha =\frac{a+\sqrt{a^2-4bc}}{2},\beta =\frac{a-\sqrt{a^2-4bc}}{2}$,由Vieta定理可知$a=\alpha+\beta,bc=\alpha\beta$.

若$a,b,c$均为复数,则上述特征方程
\end{note}
\begin{solution}
由\hyperref[三对角行列式]{命题\ref{三对角行列式}}可知,递推式为$D_n=aD_{n-1}-bcD_{n-2}(n\ge2)$.
又易知$D_0=1,D_1=a$.令$\alpha =\frac{a+\sqrt{a^2-4bc}}{2},\beta =\frac{a-\sqrt{a^2-4bc}}{2}$,
则$a=\alpha+\beta,bc=\alpha\beta$,于是$D_n=\left( \alpha +\beta \right) D_{n-1}-\alpha \beta D_{n-2}(n\ge2)$.
从而
\begin{gather}
D_n-\alpha D_{n-1}=\beta \left( D_{n-1}-\alpha D_{n-2} \right) ,D_n-\beta D_{n-1}=\alpha \left( D_{n-1}-\beta D_{n-2} \right).
\nonumber
\end{gather}
于是
\begin{gather}
D_n-\alpha D_{n-1}=\beta ^{n-1}\left( D_1-\alpha D_0 \right) =\beta ^{n-1}\left( a-\alpha \right) =\beta ^n,
\nonumber\\
D_n-\beta D_{n-1}=\alpha ^{n-1}\left( D_1-\beta D_0 \right) =\alpha ^{n-1}\left( a-\beta \right) =\alpha ^n.
\nonumber
\end{gather}
因此,若$a^2\ne4bc(\text{即}\alpha\ne\beta)$,则联立上面两式,解得
\begin{equation}
D_n=\frac{\alpha ^{n+1}-\beta ^{n+1}}{\alpha -\beta}; 
\nonumber
\end{equation}
若$a^2=4bc(\text{即}\alpha=\beta)$,则由$a=\alpha+\beta$可知,$\alpha=\beta=\frac{a}{2}$.
又由$D_n-\alpha D_{n-1}=\beta ^n$可得
\begin{gather}
D_n=\left( \frac{a}{2} \right) ^n+\frac{a}{2}D_{n-1}=\left( \frac{a}{2} \right) ^n+\frac{a}{2}\left( \left( \frac{a}{2} \right) ^{n-1}+\frac{a}{2}D_{n-2} \right) =2\left( \frac{a}{2} \right) ^n+\left( \frac{a}{2} \right) ^2D_{n-2}=\cdots =n\left( \frac{a}{2} \right) ^n+\left( \frac{a}{2} \right) ^nD_0=\left( n+1 \right) \left( \frac{a}{2} \right) ^n.
\nonumber
\end{gather}
综上,我们有
\begin{align*}
D_n=\begin{cases}
\frac{\alpha ^{n+1}-\beta ^{n+1}}{\alpha -\beta},a^2\ne 4bc,\\
\left( n+1 \right) \left( \frac{\alpha}{2} \right) ^n,a^2=4bc.\\
\end{cases}
\end{align*}
\end{solution}

\begin{exercise}\label{三对角行列式例题1}
求证:$n$阶行列式
\begin{equation}
|\boldsymbol{A}|=\left| \begin{matrix}
\cos x&		1&		0&		0&		\cdots&		0&		0&		0\\
1&		2\cos x&		1&		0&		\cdots&		0&		0&		0\\
0&		1&		2\cos x&		1&		\cdots&		0&		0&		0\\
\vdots&		\vdots&		\vdots&		\vdots&		&		\vdots&		\vdots&		\vdots\\
0&		0&		0&		0&		\cdots&		1&		2\cos x&		1\\
0&		0&		0&		0&		\cdots&		0&		1&		2\cos x\\
\end{matrix} \right|=\cos nx.
\nonumber
\end{equation}
\end{exercise}
\begin{solution}
{\color{blue} \text{解法一:}}

设$|\boldsymbol{A}|=D_n$,其中$n$表示$|\boldsymbol{A}|$的阶数$(n\ge0)$.易知$D_0=1,D_1=\cos x$.

从而$|\boldsymbol{A}|=D_n\xlongequal[\text{命题}\ref{三对角行列式}]{\text{按最后一列展开}}2\cos xD_{n-1}-D_{n-2}\left( n\ge 2 \right)$.

其对应的特征方程为$\lambda ^2=2\cos x\lambda -1$,解得$\lambda _1=\cos x+i\sin x,\lambda _2=\cos x-i\sin x$.

于是当$n\ge2$时,我们有$D_n=\left( \lambda _1+\lambda _2 \right) D_{n-1}+\lambda _1\lambda _2D_{n-2}$.

进而
\begin{equation}
\label{eq:递推式1.1}
\begin{split}
&D_n-\lambda _1D_{n-1}=\lambda _2\left( D_n-\lambda _1D_{n-1} \right),
\\
&D_n-\lambda _2D_{n-1}=\lambda _1\left( D_n-\lambda _2D_{n-1} \right).
\end{split}
\end{equation}
由此可得
\begin{gather}
D_n-\lambda _1D_{n-1}={\lambda _2}^{n-1}\left( D_1-\lambda _1D_0 \right) =-i\sin x\cdot {\lambda _2}^{n-1},
\nonumber\\
D_n-\lambda _2D_{n-1}={\lambda _1}^{n-1}\left( D_1-\lambda _2D_0 \right) =i\sin x\cdot {\lambda _1}^{n-1}.
\nonumber
\end{gather}
若$x\ne k\pi(k\in\mathbb{Z})$,则联立上面两式,解得
\begin{equation}
\begin{split}
D_n&=\frac{i\sin x\cdot {\lambda _1}^n+i\sin x\cdot {\lambda _2}^n}{\lambda _1-\lambda _2}=\frac{i\sin x\cdot \left( \cos x+i\sin x \right) ^n+i\sin x\cdot \left( \cos x-i\sin x \right) ^n}{2i\sin x}
\\
&\xlongequal[e^{ix}=\cos x+i\sin x,e^{-ix}=\cos x-i\sin x.]{Euler\text{公式}}\frac{i\sin x\cdot e^{nxi}+i\sin x\cdot e^{-nxi}}{2i\sin x}=\frac{i\sin x\cdot \left( \cos nx+i\sin nx \right) +i\sin x\cdot \left( \cos nx-i\sin nx \right)}{2i\sin x}
\\
&=\frac{2i\sin x\cdot \cos nx}{2i\sin x}=\cos nx.
\end{split}
\nonumber
\end{equation}
若$x=k\pi(k\in\mathbb{Z})$,则$\lambda _1=\lambda _2=\cos k\pi$.
从而由\eqref{eq:递推式1.1}式可得,$D_n-\cos k\pi D_{n-1}=-i\sin x\cdot \left( \cos k\pi \right) =0.$

于是
\begin{align*}
D_n=\cos k\pi D_{n-1}=\left( \cos k\pi \right) ^2D_{n-2}=\cdots =\left( \cos k\pi \right) ^nD_0=\left( \cos k\pi \right) ^n=\left( -1 \right) ^{kn}=\cos \left( nk\pi \right) =\cos nx.
\nonumber
\end{align*}
{\color{blue} \text{解法二:}}仿照练习\ref{使用数学归纳法计算行列式例题1}中的数学归纳法证明.
\end{solution}

\begin{exercise}\label{三对角行列式例题2}
求下列$n$阶行列式的值:
\begin{gather}
D_{n}=\begin{vmatrix}1-a_{1}&a_{2}&0&0&\cdots&0&0\\ -1&1-a_{2}&a_{3}&0&\cdots&0&0\\ 0&-1&1-a_{3}&a_{4}&\cdots&0&0\\ \vdots&\vdots&\vdots&\vdots&&\vdots&\vdots\\ 0&0&0&0&\cdots&-1&1-a_{n}\end{vmatrix}.
\nonumber
\end{gather}
\end{exercise}
\begin{note}
观察原行列式我们可以得到,$D_n$的每列和有一定的规律,即除了第一列和最后一列,中间每列和均为0.并且$D_n$是三对角行列式.
因此,我们既可以直接应用三对角行列式的结论(即命题\ref{三对角行列式}),又可以使用求和法进行求解.
如果我们直接应用三对角行列式的结论(即命题\ref{三对角行列式}),按照对一般的三对角行列式展开的方法能得到相应递推式,但是这样得到的递推式并不是相邻两项之间的递推,后续求解通项并不简便.
又因为使用求和法计算行列式后续计算一般比较简便所以我们先采用求和法进行尝试.
\end{note}
\begin{solution}
{\color{blue} \text{解法一:}}
当$n\ge1$时,我们有
\begin{align}
D_n&=\left| \begin{matrix}
1-a_1&		a_2&		0&		0&		\cdots&		0&		0\\
-1&		1-a_2&		a_3&		0&		\cdots&		0&		0\\
0&		-1&		1-a_3&		a_4&		\cdots&		0&		0\\
\vdots&		\vdots&		\vdots&		\vdots&		&		\vdots&		\vdots\\
0&		0&		0&		0&		\cdots&		-1&		1-a_n\\
\end{matrix} \right|\xlongequal[i=2,\cdots ,n]{r_i+r_1}\left| \begin{matrix}
-a_1&		0&		0&		0&		\cdots&		0&		1\\
-1&		1-a_2&		a_3&		0&		\cdots&		0&		0\\
0&		-1&		1-a_3&		a_4&		\cdots&		0&		0\\
\vdots&		\vdots&		\vdots&		\vdots&		&		\vdots&		\vdots\\
0&		0&		0&		0&		\cdots&		-1&		1-a_n\\
\end{matrix} \right|
\nonumber\\\nonumber
&\xlongequal[]{\text{按第一行展开}}-a_1D_{n-1}+\left( -1 \right) ^{n+1}\left| \begin{matrix}
-1&		1-a_2&		a_3&		0&		\cdots&		0\\
0&		-1&		1-a_3&		a_4&		\cdots&		0\\
\vdots&		\vdots&		\vdots&		\vdots&		&		\vdots\\
0&		0&		0&		0&		\cdots&		-1\\
\end{matrix} \right|
\\\nonumber
&=-a_1D_{n-1}+\left( -1 \right) ^{n+1}\left( -1 \right) ^{n-1}
\\\nonumber
&=1-a_1D_{n-1}.
\end{align}
其中$D_{n-i}$表示$D_{n-i+1}$去掉第一行和第一列得到的$n-i$阶行列式,$i=1,2,\cdots,n-1$.
(或者称$D_{n-i}$表示以$a_{i+1},\cdots,a_n$为未定元的$n-i$阶行列式,$i=1,2,\cdots,n-1$)

由递推不难得到
\begin{align*}
D_n=1-a_1\left( 1-a_2D_{n-2} \right) =1-a_1+a_1a_2D_{n-2}=\cdots =1-a_1+a_1a_2-a_1a_2a_3+\cdots +\left( -1 \right) ^na_1a_2\cdots a_n.
\nonumber
\end{align*}
{\color{blue} \text{解法二:}}仿照练习\ref{使用数学归纳法计算行列式例题1}中的数学归纳法证明.
\end{solution}

\begin{proposition}\label{proposition:小拆分法经典例题}
计算$n$阶行列式:
\begin{align*}
D_n=\left| \begin{matrix}
x_1&		y&		y&		\cdots&		y&		y\\
z&		x_2&		y&		\cdots&		y&		y\\
z&		z&		x_3&		\cdots&		y&		y\\
\vdots&		\vdots&		\vdots&		&		\vdots&		\vdots\\
z&		z&		z&		\cdots&		x_{n-1}&		y\\
z&		z&		z&		\cdots&		z&		x_n\\
\end{matrix} \right|.
\end{align*}
\end{proposition}
\begin{note}
{\color{blue}解法二:}
$f(x) \triangleq \left| \begin{matrix}
x_1+x & y+x & \cdots & y+x \\
z+x & x_2+x & \cdots & y+x \\ 
\vdots & \vdots & & \vdots \\
z+x & z+x & \cdots & x_n+x \\
\end{matrix} \right|=\left| \begin{matrix}
x_1+x & y+x & \cdots & y+x \\
z-x_1 & x_2-y & \cdots & 0 \\
\vdots & \vdots & & \vdots \\
z-x_1 & z-y & \cdots & x_n-y \\
\end{matrix} \right|$,再按第一行展开可得$f(x)$一定为关于$x$的线性函数.
\end{note}
\begin{solution}
{\color{blue}解法一\hyperref[小拆分法]{(小拆分法)}:}
对第$n$列进行拆分即可得到递推式:
(对第1或n行(或列)拆分都可以得到相同结果)
\begin{align}
&D_n=\left| \begin{matrix}
x_1&		y&		y&		\cdots&		y&		y+0\\
z&		x_2&		y&		\cdots&		y&		y+0\\
z&		z&		x_3&		\cdots&		y&		y+0\\
\vdots&		\vdots&		\vdots&		&		\vdots&		\vdots\\
z&		z&		z&		\cdots&		x_{n-1}&		y+0\\
z&		z&		z&		\cdots&		z&		y+x_n-y\\
\end{matrix} \right|=\left| \begin{matrix}
x_1&		y&		y&		\cdots&		y&		y\\
z&		x_2&		y&		\cdots&		y&		y\\
z&		z&		x_3&		\cdots&		y&		y\\
\vdots&		\vdots&		\vdots&		&		\vdots&		\vdots\\
z&		z&		z&		\cdots&		x_{n-1}&		y\\
z&		z&		z&		\cdots&		z&		y\\
\end{matrix} \right|+\left| \begin{matrix}
x_1&		y&		y&		\cdots&		y&		0\\
z&		x_2&		y&		\cdots&		y&		0\\
z&		z&		x_3&		\cdots&		y&		0\\
\vdots&		\vdots&		\vdots&		&		\vdots&		\vdots\\
z&		z&		z&		\cdots&		x_{n-1}&		0\\
z&		z&		z&		\cdots&		z&		x_n-y\\
\end{matrix} \right|
\nonumber\\
&=\left| \begin{matrix}
x_1-z&		0&		0&		\cdots&		0&		0\\
0&		x_2-z&		0&		\cdots&		0&		0\\
0&		0&		x_3-z&		\cdots&		0&		0\\
\vdots&		\vdots&		\vdots&		&		\vdots&		\vdots\\
0&		0&		0&		\cdots&		x_{n-1}-z&		0\\
z&		z&		z&		\cdots&		z&		y\\
\end{matrix} \right|+\left( x_n-y \right) D_{n-1}=y\prod\limits_{i=1}^{n-1}{\left( x_i-z \right)}+\left( x_n-y \right) D_{n-1}.
\label{eq:递推式1.2}
\end{align}
将原行列式转置后,同理可得
\begin{align}
&D_n=D_{n}^{T}=\left| \begin{matrix}
x_1&		z&		z&		\cdots&		z&		z+0\\
y&		x_2&		z&		\cdots&		z&		z+0\\
y&		y&		x_3&		\cdots&		z&		z+0\\
\vdots&		\vdots&		\vdots&		&		\vdots&		\vdots\\
y&		y&		y&		\cdots&		x_{n-1}&		z+0\\
y&		y&		y&		\cdots&		y&		z+x_n-z\\
\end{matrix} \right|=\left| \begin{matrix}
x_1&		z&		z&		\cdots&		z&		z\\
y&		x_2&		z&		\cdots&		z&		z\\
y&		y&		x_3&		\cdots&		z&		z\\
\vdots&		\vdots&		\vdots&		&		\vdots&		\vdots\\
y&		y&		y&		\cdots&		x_{n-1}&		z\\
y&		y&		y&		\cdots&		y&		z\\
\end{matrix} \right|+\left| \begin{matrix}
x_1&		z&		z&		\cdots&		z&		0\\
y&		x_2&		z&		\cdots&		z&		0\\
y&		y&		x_3&		\cdots&		z&		0\\
\vdots&		\vdots&		\vdots&		&		\vdots&		\vdots\\
y&		y&		y&		\cdots&		x_{n-1}&		0\\
y&		y&		y&		\cdots&		y&		x_n-z\\
\end{matrix} \right|
\nonumber\\
&=\left| \begin{matrix}
x_1-y&		0&		0&		\cdots&		0&		0\\
0&		x_2-y&		0&		\cdots&		0&		0\\
0&		0&		x_3-y&		\cdots&		0&		0\\
\vdots&		\vdots&		\vdots&		&		\vdots&		\vdots\\
0&		0&		0&		\cdots&		x_{n-1}-y&		0\\
y&		y&		y&		\cdots&		y&		z\\
\end{matrix} \right|+\left( x_n-z \right) D_{n-1}^{T}=z\prod\limits_{i=1}^{n-1}{\left( x_i-y \right)}+\left( x_n-z \right) D_{n-1}.
\label{eq:递推式1.3}
\end{align}
若$z\ne y$,则联立\eqref{eq:递推式1.2}\eqref{eq:递推式1.3}式,解得
\begin{equation}
D_n=\frac{1}{z-y}\biggl[ z\prod\limits_{i=1}^n{(x_i}-y)-y\prod\limits_{i=1}^n{(x_i}-z) \biggr];
\nonumber
\end{equation}
若$z= y$,则由\eqref{eq:递推式1.2}式递推可得
\begin{equation}
\begin{split}
D_n&=y\prod\limits_{i=1}^{n-1}{\left( x_i-y \right)}+\left( x_n-y \right) D_{n-1}
\\
&=y\prod\limits_{i=1}^{n-1}{\left( x_i-y \right)}+\left( x_n-y \right) \left( y\prod\limits_{i=1}^{n-2}{\left( x_i-y \right)}+\left( x_{n-1}-y \right) D_{n-2} \right) 
\\
&=y\prod\limits_{j\ne n}^{}{\left( x_i-y \right)}+y\prod\limits_{j\ne n-1}^{}{\left( x_i-y \right)}+\left( x_n-y \right) \left( x_{n-1}-y \right) D_{n-2}
\\
&=\cdots =y\sum_{i=1}^n{\prod\limits_{j\ne i}{(x_j}}-y)+\prod\limits_{i=1}^n{(x_i}-y)D_0
\\
&=y\sum_{i=1}^n{\prod\limits_{j\ne i}{(x_j}}-y)+\prod\limits_{i=1}^n{(x_i}-y).
\end{split}
\nonumber
\end{equation}

{\color{blue}解法二\hyperref[大拆分法]{(大拆分法)}:}令$f(x) \triangleq \left| \begin{matrix}
x_1+x & y+x & \cdots & y+x \\
z+x & x_2+x & \cdots & y+x \\
\vdots & \vdots & & \vdots \\
z+x & z+x & \cdots & x_n+x \\
\end{matrix} \right|$,则$f(x)$一定是线性函数,从而设$f(x) =ax+b$.注意到
\begin{align*}
f(-z) =\left| \begin{matrix}
x_1-z & y-z & \cdots & y-z \\
0 & x_2-z & \cdots & y-z \\
\vdots & \vdots & & \vdots \\
0 & 0 & \cdots & x_n-z \\
\end{matrix} \right|=\prod_{i=1}^n{(x_i-z)},\quad f(-y) =\left| \begin{matrix}
x_1-y & 0 & \cdots & 0 \\
z-y & x_2-y & \cdots & 0 \\
\vdots & \vdots & & \vdots \\
z-y & z-y & \cdots & x_n-y \\
\end{matrix} \right|=\prod_{i=1}^n{(x_i-y)}.
\end{align*}
当$y\ne z$时,将上式代入$f(x) =ax+b$(即线性函数$f(x)$过两点$(-y,f(-y))$,$(-z,f(-z))$,再利用两点式)解得
\begin{align*}
f(x) =\ddfrac{f(-z) -f(-y)}{-z-(-y)}(x+y) +f(-y) =\ddfrac{\prod_{i=1}^n{(x_i-z)}-\prod_{i=1}^n{(x_i-y)}}{y-z}(x+y) +\prod_{i=1}^n{(x_i-y)}.
\end{align*}
从而此时就有
\begin{align}
D_n=f(0) =\ddfrac{y\prod_{i=1}^n{(x_i}-z)-z\prod_{i=1}^n{(x_i}-y)}{y-z}.\label{equation24334-1.2}
\end{align}
当$y=z$时,将$D_n$看作关于$y$的连续函数,记为$g(y) =D_n$,则此时由$g$的连续性及\eqref{equation24334-1.2}式和L'Hospital法则可得
\begin{align*}
D_n&=g\left( z \right) =\underset{y\rightarrow z}{\lim}g\left( y \right) =\underset{y\rightarrow z}{\lim}\ddfrac{y\prod_{i=1}^n{(x_i}-z)-z\prod_{i=1}^n{(x_i}-y)}{y-z}
\\
&=\underset{y\rightarrow z}{\lim}\ddfrac{\prod_{i=1}^n{(x_i}-z)+y\sum_{i=1}^n{\prod_{j\ne i}{(x_j}}-y)}{1}=\prod_{i=1}^n{(x_i}-z)+z\sum_{i=1}^n{\prod_{j\ne i}{(x_j}}-z).
\end{align*}
\end{solution}

\begin{example}
\begin{enumerate}[(1)]
\item 计算
\begin{align*}
|B| = 
\begin{vmatrix}
2a_1 & a_1 + a_2 & \cdots & a_1 + a_n \\
a_2 + a_1 & 2a_2 & \cdots & a_2 + a_n \\
\vdots & \vdots & \ddots & \vdots \\
a_n + a_1 & a_n + a_2 & \cdots & 2a_n
\end{vmatrix}.
\end{align*}

\item 求下列$n$阶行列式的值:
\begin{align*}
|\boldsymbol{A}|=\left| \begin{matrix}
0&		a_1+a_2&		\cdots&		a_1+a_{n-1}&		a_1+a_n\\
a_2+a_1&		0&		\cdots&		a_2+a_{n-1}&		a_2+a_n\\
\vdots&		\vdots&		&		\vdots&		\vdots\\
a_{n-1}+a_1&		a_{n-1}+a_2&		\cdots&		0&		a_{n-1}+a_n\\
a_n+a_1&		a_n+a_2&		\cdots&		a_n+a_{n-1}&		0\\
\end{matrix} \right|.
\end{align*}
\end{enumerate}
\end{example}
\begin{note}
第(2)问{\color{blue}解法一}中不仅使用了\hyperlink{行列式计算:升阶法}{升阶法}还使用了\hyperref[proposition:分块"爪"型行列式]{分块"爪"型行列式的计算方法}.观察到各行各列有不同的公共项,因此可以利用升阶法将各行各列的公共项消去.
\end{note}
\begin{remark}
因为第(2)问中,当$a_i\ne 0(i=1,2,\cdots,n)$时,最后的结果不含$a_i$的分式结构,所以当存在$a_i=0$,其中$i\in{1,2,\cdot,ns}$时,根据行列式(可以看作多元多项式函数)的连续性可知,此时最后的结果就是将$a_i$中相应为零的值代入当$a_i\ne 0(i=1,2,\cdots,n)$时的结果中.因此我吗们可以直接不妨设$a_i\ne 0(i=1,2,\cdots,n)$,只需考虑这一种情况即可.
\end{remark}
\begin{solution}
\begin{enumerate}[(1)]
\item 注意到$B=\left( \begin{matrix}
2a_1 & a_1+a_2 & \cdots & a_1+a_n \\
a_2+a_1 & 2a_2 & \cdots & a_2+a_n \\
\vdots & \vdots & \ddots & \vdots \\
a_n+a_1 & a_n+a_2 & \cdots & 2a_n \\
\end{matrix} \right) =\left( \begin{matrix}
1 & a_1 \\
1 & a_2 \\
\vdots & \vdots \\
1 & a_n \\
\end{matrix} \right) \left( \begin{matrix}
1 & 1 & \cdots & 1 \\
a_1 & a_2 & \cdots & a_n \\
\end{matrix} \right)$.由Cauchy-Binet公式可知,$\left| B \right|=\begin{cases}
0, & n\ge 3, \\
-(a_1-a_2)^2, & n=2, \\
2a_1, & n=1. \\
\end{cases}$.

\item (i)当$a_i\ne 0(1\le i\le n)$时,
{\color{blue}解法一(\hyperlink{行列式计算:升阶法}{升阶法}):}
\begin{align*}
&|\boldsymbol{A}|\xlongequal[]{\text{升阶}}\left| \begin{matrix}
1&		-a_1&		-a_2&		\cdots&		-a_{n-1}&		-a_n\\
0&		0&		a_1+a_2&		\cdots&		a_1+a_{n-1}&		a_1+a_n\\
0&		a_2+a_1&		0&		\cdots&		a_2+a_{n-1}&		a_2+a_n\\
\vdots&		\vdots&		\vdots&		&		\vdots&		\vdots\\
0&		a_{n-1}+a_1&		a_{n-1}+a_2&		\cdots&		0&		a_{n-1}+a_n\\
0&		a_n+a_1&		a_n+a_2&		\cdots&		a_n+a_{n-1}&		0\\
\end{matrix} \right|
\\
&\xlongequal[i=1,2,\cdots ,n+1]{r_1+r_i}\left| \begin{matrix}
1&		-a_1&		-a_2&		\cdots&		-a_{n-1}&		-a_n\\
1&		-a_1&		a_1&		\cdots&		a_1&		a_1\\
1&		a_2&		-a_2&		\cdots&		a_2&		a_2\\
\vdots&		\vdots&		\vdots&		&		\vdots&		\vdots\\
1&		a_{n-1}&		a_{n-1}&		\cdots&		-a_{n-1}&		a_{n-1}\\
1&		a_n&		a_n&		\cdots&		a_n&		-a_n\\
\end{matrix} \right|\xlongequal[]{\text{升阶}}\left| \begin{matrix}
1&		0&		0&		0&		\cdots&		0&		0\\
0&		1&		-a_1&		-a_2&		\cdots&		-a_{n-1}&		-a_n\\
-a_1&		1&		-a_1&		a_1&		\cdots&		a_1&		a_1\\
-a_2&		1&		a_2&		-a_2&		\cdots&		a_2&		a_2\\
\vdots&		\vdots&		\vdots&		\vdots&		&		\vdots&		\vdots\\
-a_{n-1}&		1&		a_{n-1}&		a_{n-1}&		\cdots&		-a_{n-1}&		a_{n-1}\\
-a_n&		1&		a_n&		a_n&		\cdots&		a_n&		-a_n\\
\end{matrix} \right|
\\
&\xlongequal[i=1,3,4\cdots ,n+2]{j_1+j_i}\left| \begin{matrix}
1&		0&		1&		1&		\cdots&		1&		1\\
0&		1&		-a_1&		-a_2&		\cdots&		-a_{n-1}&		-a_n\\
-a_1&		1&		-2a_1&		0&		\cdots&		0&		0\\
-a_2&		1&		0&		-2a_2&		\cdots&		0&		0_2\\
\vdots&		\vdots&		\vdots&		\vdots&		&		\vdots&		\vdots\\
-a_{n-1}&		1&		0&		0&		\cdots&		-2a_{n-1}&		0\\
-a_n&		1&		0&		0&		\cdots&		0&		-2a_n\\
\end{matrix} \right|
\\
&\xlongequal[i=3,4\cdots ,n+2]{\begin{array}{c}
-\frac{1}{2}j_i+j_1\\
\frac{1}{2a_{i-2}}j_i+j_2\\
\end{array}}\left| \begin{matrix}
1-\frac{n}{2}&		\frac{S}{2}&		1&		1&		\cdots&		1&		1\\
\frac{T}{2}&		1-\frac{n}{2}&		-a_1&		-a_2&		\cdots&		-a_{n-1}&		-a_n\\
0&		0&		-2a_1&		0&		\cdots&		0&		0\\
0&		0&		0&		-2a_2&		\cdots&		0&		0_2\\
\vdots&		\vdots&		\vdots&		\vdots&		&		\vdots&		\vdots\\
0&		0&		0&		0&		\cdots&		-2a_{n-1}&		0\\
0&		0&		0&		0&		\cdots&		0&		-2a_n\\
\end{matrix} \right|. 
\end{align*}
其中\(S = a_1 + a_2 + \cdots + a_n\),\(T = \frac{1}{a_1} + \frac{1}{a_2} + \cdots + \frac{1}{a_n}\).注意到上述行列式是分块上三角行列式,从而可得
\begin{align*}
\left| \boldsymbol{A} \right|&=(-2)^n\prod_{i=1}^n{a_i}\cdot \frac{(n-2)^2-ST}{4}=(-2)^{n-2}\prod_{i=1}^n{a_i[(n}-2)^2-(\sum_{i=1}^n{a_i)(\sum_{i=1}^n{\frac{1}{a_i})]}}
\\
&=(-2)^{n-2}\prod_{i=1}^n{a_i}\left( n-2 \right) ^2-(-2)^{n-2}\sum_{i=1}^n{a_i}\sum_{j=1}^n{\prod_{k\ne j}{a_k}}.
\end{align*}

{\color{blue}解法二(\hyperref[proposition:直接计算两个矩阵和的行列式]{直接计算两个矩阵和的行列式})(不推荐使用!):}

设\(\boldsymbol{B}=\left(\begin{matrix}
2a_1 & a_1 + a_2 & \cdots & a_1 + a_n\\
a_2 + a_1 & 2a_2 & \cdots & a_2 + a_n\\
\vdots & \vdots &  & \vdots\\
a_n + a_1 & a_n + a_2 & \cdots & 2a_n
\end{matrix}\right)\),\(\boldsymbol{C}=\left(\begin{matrix}
-2a_1 &  &  & \\
& -2a_2 &  & \\
&  & \ddots & \\
&  &  & -2a_n
\end{matrix}\right)\),则\(|\boldsymbol{A}| = |\boldsymbol{B} + \boldsymbol{C}|\).

从而利用\hyperref[proposition:直接计算两个矩阵和的行列式]{直接计算两个矩阵和的行列式}的结论得到
\begin{align}\label{eq(行列式):1.5式}
&|\boldsymbol{A}| = |\boldsymbol{B}| + |\boldsymbol{C}| + \sum_{1\leqslant k\leqslant n - 1}\left(\sum_{\begin{array}{c}
1\leqslant i_1 < i_2 < \cdots < i_k\leqslant n\\
1\leqslant j_1 < j_2 < \cdots < j_k\leqslant n
\end{array}}\boldsymbol{B}\left(\begin{matrix}
i_1 & i_2 & \cdots & i_k\\
j_1 & j_2 & \cdots & j_k
\end{matrix}\right)\widehat{\boldsymbol{C}}\left(\begin{matrix}
i_1 & i_2 & \cdots & i_k\\
j_1 & j_2 & \cdots & j_k
\end{matrix}\right)\right)
\end{align}
其中\(\widehat{\boldsymbol{C}}\left(\begin{matrix}
i_1 & i_2 & \cdots & i_k\\
j_1 & j_2 & \cdots & j_k
\end{matrix}\right)\)是\(\boldsymbol{C}\left(\begin{matrix}
i_1 & i_2 & \cdots & i_k\\
j_1 & j_2 & \cdots & j_k
\end{matrix}\right)\)的代数余子式.

我们先来计算\(\boldsymbol{B}\left(\begin{matrix}
i_1 & i_2 & \cdots & i_k\\
j_1 & j_2 & \cdots & j_k
\end{matrix}\right)\),\(k = 1,2,\cdots,n\).拆分\(\boldsymbol{B}\left(\begin{matrix}
i_1 & i_2 & \cdots & i_k\\
j_1 & j_2 & \cdots & j_k
\end{matrix}\right)\)的第一列得到
\begin{align*}
&\boldsymbol{B}\left(\begin{matrix}
i_1 & i_2 & \cdots & i_k\\
j_1 & j_2 & \cdots & j_k
\end{matrix}\right) = \left|\begin{matrix}
a_{i_1} + a_{j_1} & a_{i_1} + a_{j_2} & \cdots & a_{i_1} + a_{j_k}\\
a_{i_2} + a_{j_1} & a_{i_2} + a_{j_2} & \cdots & a_{i_2} + a_{j_k}\\
\vdots & \vdots &  & \vdots\\
a_{i_k} + a_{j_1} & a_{i_k} + a_{j_2} & \cdots & a_{i_k} + a_{j_k}
\end{matrix}\right|
\\
&=\left|\begin{matrix}
a_{i_1} & a_{i_1} + a_{j_2} & \cdots & a_{i_1} + a_{j_k}\\
a_{i_2} & a_{i_2} + a_{j_2} & \cdots & a_{i_2} + a_{j_k}\\
\vdots & \vdots &  & \vdots\\
a_{i_k} & a_{i_k} + a_{j_2} & \cdots & a_{i_k} + a_{j_k}
\end{matrix}\right| + \left|\begin{matrix}
a_{j_1} & a_{i_1} + a_{j_2} & \cdots & a_{i_1} + a_{j_k}\\
a_{j_1} & a_{i_2} + a_{j_2} & \cdots & a_{i_2} + a_{j_k}\\
\vdots & \vdots &  & \vdots\\
a_{j_1} & a_{i_k} + a_{j_2} & \cdots & a_{i_k} + a_{j_k}
\end{matrix}\right|
\\
&=\left|\begin{matrix}
a_{i_1} & a_{j_2} & \cdots & a_{j_k}\\
a_{i_2} & a_{j_2} & \cdots & a_{j_k}\\
\vdots & \vdots &  & \vdots\\
a_{i_k} & a_{j_2} & \cdots & a_{j_k}
\end{matrix}\right| + \left|\begin{matrix}
a_{j_1} & a_{i_1} & \cdots & a_{i_1}\\
a_{j_1} & a_{i_2} & \cdots & a_{i_2}\\
\vdots & \vdots &  & \vdots\\
a_{j_1} & a_{i_k} & \cdots & a_{i_k}
\end{matrix}\right|
\end{align*}
因此当\(k\geqslant 3\)时,\(\boldsymbol{B}\left(\begin{matrix}
i_1 & i_2 & \cdots & i_k\\
j_1 & j_2 & \cdots & j_k
\end{matrix}\right) = 0\);
当\(k = 2\)时,\(\boldsymbol{B}\left(\begin{matrix}
i_1 & i_2 & \cdots & i_k\\
j_1 & j_2 & \cdots & j_k
\end{matrix}\right) = \boldsymbol{B}\left(\begin{matrix}
i_1 & i_2\\
j_1 & j_2
\end{matrix}\right) = \left|\begin{matrix}
a_{i_1} & a_{j_2}\\
a_{i_2} & a_{j_2}
\end{matrix}\right| + \left|\begin{matrix}
a_{j_1} & a_{i_1}\\
a_{j_1} & a_{i_2}
\end{matrix}\right| = (a_{i_1}a_{j_2} - a_{i_2}a_{j_2})(a_{i_2}a_{j_1} - a_{i_1}a_{j_1})\);
当\(k = 1\)时,\(\boldsymbol{B}\left(\begin{matrix}
i_1 & i_2 & \cdots & i_k\\
j_1 & j_2 & \cdots & j_k
\end{matrix}\right) = \boldsymbol{B}\left(\begin{array}{c}
i_1\\
j_1
\end{array}\right) = a_{i_1} + a_{j_1}\).

又注意到\(|\boldsymbol{C}|\)只有主子式非零,而其主子式\(\boldsymbol{C}\left(\begin{matrix}
i_1 & i_2 & \cdots & i_k\\
i_1 & i_2 & \cdots & i_k
\end{matrix}\right) = (-2)^ka_{i_1}a_{i_2}\cdots a_{i_k}\).
于是当\(\exists m\in \{1,2,\cdots,k\}\),使得\(i_m\neq j_m\)时,\(\widehat{\boldsymbol{C}}\left(\begin{matrix}
i_1 & i_2 & \cdots & i_k\\
j_1 & j_2 & \cdots & j_k
\end{matrix}\right) = 0\);
当\(i_m\neq j_m\),\(m = 1,2,\cdots,k\)时,\(\widehat{\boldsymbol{C}}\left(\begin{matrix}
i_1 & i_2 & \cdots & i_k\\
j_1 & j_2 & \cdots & j_k
\end{matrix}\right) = \widehat{\boldsymbol{C}}\left(\begin{matrix}
i_1 & i_2 & \cdots & i_k\\
i_1 & i_2 & \cdots & i_k
\end{matrix}\right) = (-2)^{n - k}a_1\cdots \hat{a}_{i_1}\cdots \hat{a}_{i_2}\cdots \hat{a}_{i_k}\cdots a_n\).

故当\(n\geqslant 3\)时,\eqref{eq(行列式):1.5式}式可化为
\begin{align*}
&|\boldsymbol{A}| = |\boldsymbol{B}| + |\boldsymbol{C}| + \sum_{1\leqslant k\leqslant n - 1}\left(\sum_{\substack{
1\leqslant i_1 < i_2 < \cdots < i_k\leqslant n\\
1\leqslant j_1 < j_2 < \cdots < j_k\leqslant n
}}\boldsymbol{B}\left(\begin{matrix}
i_1 & i_2 & \cdots & i_k\\
j_1 & j_2 & \cdots & j_k
\end{matrix}\right)\widehat{\boldsymbol{C}}\left(\begin{matrix}
i_1 & i_2 & \cdots & i_k\\
j_1 & j_2 & \cdots & j_k
\end{matrix}\right)\right) 
\\
&= |\boldsymbol{C}| + \sum_{\substack{
1\leqslant i_1\leqslant n\\
1\leqslant j_1\leqslant n
}}\boldsymbol{B}\left(\substack{
i_1\\
j_1
}\right)\widehat{\boldsymbol{C}}\left(\begin{array}{c}
i_1\\
j_1
\end{array}\right) + \sum_{\substack{
1\leqslant i_1 < i_2\leqslant n\\
1\leqslant j_1 < j_2\leqslant n
}}\boldsymbol{B}\left(\begin{matrix}
i_1 & i_2\\
j_1 & j_2
\end{matrix}\right)\widehat{\boldsymbol{C}}\left(\begin{matrix}
i_1 & i_2\\
j_1 & j_2
\end{matrix}\right)
\\
&= |\boldsymbol{C}| + \sum_{1\leqslant i_1\leqslant n}\boldsymbol{B}\left(\begin{array}{c}
i_1\\
i_1
\end{array}\right)\widehat{\boldsymbol{C}}\left(\begin{array}{c}
i_1\\
i_1
\end{array}\right) + \sum_{1\leqslant i_1 < i_2\leqslant n}\boldsymbol{B}\left(\begin{matrix}
i_1 & i_2\\
i_1 & i_2
\end{matrix}\right)\widehat{\boldsymbol{C}}\left(\begin{matrix}
i_1 & i_2\\
i_1 & i_2
\end{matrix}\right)
= |\boldsymbol{C}| + \sum_{1\leqslant i\leqslant n}\boldsymbol{B}\left(\begin{matrix}
i\\
i
\end{matrix}
\right)\widehat{\boldsymbol{C}}\left(\begin{matrix}
i\\
i
\end{matrix}
\right) + \sum_{1\leqslant i < j\leqslant n}\boldsymbol{B}\left(\begin{matrix}
i & j\\
i & j
\end{matrix}\right)\widehat{\boldsymbol{C}}\left(\begin{matrix}
i & j\\
i & j
\end{matrix}\right)
\\
&= (-2)^na_1a_2\cdots a_n + \sum_{1\leqslant i\leqslant n}2a_i(-2)^{n - 1}a_1\cdots \hat{a}_i\cdots a_n
+\sum_{1\leqslant i < j\leqslant n}[(a_ia_j - a_{j}^{2})(a_ia_j - a_{i}^{2})(-2)^{n - 2}a_1\cdots \hat{a}_i\cdots \hat{a}_j\cdots a_n]
\\
&= (-2)^na_1a_2\cdots a_n - (-2)^n\sum_{1\leqslant i\leqslant n}a_1a_2\cdots \cdots a_n
+ (-2)^{n - 2}\sum_{1\leqslant i < j\leqslant n}[-(a_i - a_j)^2a_1\cdots \hat{a}_i\cdots \hat{a}_j\cdots a_n]
\\
&= (-2)^na_1a_2\cdots a_n - (-2)^nna_1a_2\cdots \cdots a_n
- (-2)^{n - 2}\sum_{1\leqslant i < j\leqslant n}[(a_i - a_j)^2a_1\cdots \hat{a}_i\cdots \hat{a}_j\cdots a_n]
\\
&= (-2)^n\prod_{i = 1}^n{a_i}(1 - n) - (-2)^{n - 2}\prod_{i = 1}^n{a_i}\sum_{1\leqslant i < j\leqslant n}\frac{(a_i - a_j)^2}{a_{i}a_{j}}
\\
& = (-2)^{n - 2}\prod_{i = 1}^n{a_i}[(n - 2)^2 - (\sum_{i = 1}^n{a_i})(\sum_{i = 1}^n{\frac{1}{a_i}})]
\\
&=(-2)^n\prod_{i=1}^n{a_i(1}-n)-(-2)^{n-2}\prod_{i=1}^n{a_i\sum_{1\leqslant i<j\leqslant n}{\frac{(a_i-a_j)^2}{a_ia_j}}}
\\
&=\left( -2 \right) ^{n-2}\prod_{i=1}^n{a_i}\left[ 4-4n-\sum_{1\leqslant i<j\leqslant n}{\frac{(a_i-a_j)^2}{a_ia_j}} \right] 
\\
&=\left( -2 \right) ^{n-2}\prod_{i=1}^n{a_i}\left[ 4-4n-\sum_{1\leqslant i<j\leqslant n}{\left( \frac{a_j}{a_i}+\frac{a_i}{a_j}-2 \right)} \right] 
\\
&=\left( -2 \right) ^{n-2}\prod_{i=1}^n{a_i}\left[ 4-4n-\sum_{\substack{
1\leqslant i,j\leqslant n\\
i\ne j\\
}}{\frac{a_i}{a_j}}+\sum_{1\leqslant i<j\leqslant n}{2} \right] 
\\
&=\left( -2 \right) ^{n-2}\prod_{i=1}^n{a_i}\left[ 4-4n-\left( \sum_{1\leqslant i,j\leqslant n}{\frac{a_i}{a_j}}-\sum_{i=1}^n{\frac{a_i}{a_i}} \right) +\sum_{i=1}^{n-1}{\sum_{j=i+1}^n{2}} \right] 
\\
&=\left( -2 \right) ^{n-2}\prod_{i=1}^n{a_i}\left[ 4-4n-\left( \sum_{1\leqslant i,j\leqslant n}{\frac{a_i}{a_j}}-n \right) +2\sum_{i=1}^{n-1}{\left( n-i \right)} \right] 
\\
&=\left( -2 \right) ^{n-2}\prod_{i=1}^n{a_i}\left[ 4-4n+n+n\left( n-1 \right) -\sum_{i=1}^n{\sum_{j=1}^n{\frac{a_i}{a_j}}} \right] 
\\
&=\left( -2 \right) ^{n-2}\prod_{i=1}^n{a_i}\left[ n^2-4n+4-\sum_{i=1}^n{a_i\sum_{j=1}^n{\frac{1}{a_j}}} \right] 
\\
&=\left( -2 \right) ^{n-2}\prod_{i=1}^n{a_i[(n}-2)^2-(\sum_{i=1}^n{a_i)(\sum_{i=1}^n{\frac{1}{a_i})]}}
\\
&=(-2)^{n-2}\prod_{i=1}^n{a_i}\left( n-2 \right) ^2-(-2)^{n-2}\sum_{i=1}^n{a_i}\sum_{j=1}^n{\prod_{k\ne j}{a_k}}.
\end{align*}
{\color{blue}解法三(降价公式)(推荐使用!):}
令$\varLambda=\left( \begin{matrix}
a_1&		1\\
a_2&		1\\
\vdots&		\vdots\\
a_n&		1\\
\end{matrix} \right) ,B=\left( \begin{matrix}
-2a_1&		&		&		\\
&		-2a_2&		&		\\
&		&		\ddots&		\\
&		&		&		-2a_n\\
\end{matrix} \right) $,则
\begin{align*}
A=\left( \begin{matrix}
-2a_1&		&		&		\\
&		-2a_2&		&		\\
&		&		\ddots&		\\
&		&		&		-2a_n\\
\end{matrix} \right) +\left( \begin{matrix}
a_1&		1\\
a_2&		1\\
\vdots&		\vdots\\
a_n&		1\\
\end{matrix} \right) I_{2}^{-1}\left( \begin{matrix}
1&		1&		\cdots&		1\\
a_1&		a_2&		\cdots&		a_n\\
\end{matrix} \right) =B+\varLambda I_{2}^{-1}\varLambda '.
\end{align*}
于是由降价公式(打洞原理)我们有
\begin{align*}
|A|&=|I|\left|B + \Lambda I_{2}^{-1}\Lambda '\right|=\left|\begin{matrix}
I_2 & \Lambda '\\
\Lambda & B
\end{matrix}\right|=|B|\left|I_2 - \Lambda 'B^{-1}\Lambda\right|\\
&=\left|\begin{matrix}
-2a_1 & & & \\
& -2a_2 & & \\
& & \ddots & \\
& & & -2a_n
\end{matrix}\right|\cdot\left|I_2 - \left(\begin{matrix}
1 & 1 & \cdots & 1\\
a_1 & a_2 & \cdots & a_n
\end{matrix}\right)\left(\begin{matrix}
-\frac{1}{2a_1} & & & \\
& -\frac{1}{2a_2} & & \\
& & \ddots & \\
& & & -\frac{1}{2a_n}
\end{matrix}\right)\left(\begin{matrix}
a_1 & 1\\
a_2 & 1\\
\vdots & \vdots\\
a_n & 1
\end{matrix}\right)\right|\\
&=(-2)^n\prod_{i = 1}^n a_i\left|I_2 - \left(\begin{matrix}
-\frac{1}{2a_1} & -\frac{1}{2a_2} & \cdots & -\frac{1}{2a_n}\\
-\frac{1}{2} & -\frac{1}{2} & \cdots & -\frac{1}{2}
\end{matrix}\right)\left(\begin{matrix}
a_1 & 1\\
a_2 & 1\\
\vdots & \vdots\\
a_n & 1
\end{matrix}\right)\right|\\
&=(-2)^n\prod_{i = 1}^n a_i\left|I_2 - \left(\begin{matrix}
-\frac{n}{2} & -\frac{1}{2}\sum_{i = 1}^n\frac{1}{a_i}\\
-\frac{1}{2}\sum_{i = 1}^n a_i & -\frac{n}{2}
\end{matrix}\right)\right|=(-2)^n\prod_{i = 1}^n a_i\left|\begin{matrix}
\frac{n + 2}{2} & \frac{1}{2}\sum_{i = 1}^n\frac{1}{a_i}\\
\frac{1}{2}\sum_{i = 1}^n a_i & \frac{n + 2}{2}
\end{matrix}\right|\\
&=(-2)^{n - 2}\prod_{i = 1}^n a_i\left[(n + 2)^2 - \left(\sum_{i = 1}^n a_i\right)\left(\sum_{i = 1}^n\frac{1}{a_i}\right)\right]
\\
&=(-2)^{n-2}\prod_{i=1}^n{a_i}\left( n-2 \right) ^2-(-2)^{n-2}\sum_{i=1}^n{a_i}\sum_{j=1}^n{\prod_{k\ne j}{a_k}}.
\end{align*}

(ii)当存在$a_i=0$,其中$i\in{1,2,\cdot,ns}$时,不妨设只有$a_{i_1},a_{i_2},\cdots,a_{i_m}=0,i_1,i_2,\cdots,i_m\in1,2,\cdots,n$,则可将$|A|$看作关于$a_{i_1},a_{i_2},\cdots,a_{i_m}$连续的多元多项式函数$g(a_{i_1},a_{i_2},\cdots,a_{i_m})$,于是由$g$的连续性可得
\begin{align*}
&g(0,0,\cdots ,0)=\underset{\left( a_{i_1},a_{i_2},\cdots ,a_{i_m} \right) \rightarrow \left( 0,0,\cdots ,0 \right)}{\lim}g(a_{i_1},a_{i_2},\cdots ,a_{i_m})
\\
&=\underset{\left( a_{i_1},a_{i_2},\cdots ,a_{i_m} \right) \rightarrow \left( 0,0,\cdots ,0 \right)}{\lim}\left[ (-2)^{n-2}\prod_{i=1}^n{a_i\left( n-2 \right) ^2}-(-2)^{n-2}\sum_{i=1}^n{a_i\sum_{j=1}^n{\prod_{k\ne j}{a_k}}} \right]=0.
\end{align*}
即由行列式的连续性可知
\begin{align*}
|A|=(-2)^{n-2}\prod_{i=1}^n{a_i}\left( n-2 \right) ^2-(-2)^{n-2}\sum_{i=1}^n{a_i}\sum_{j=1}^n{\prod_{k\ne j}{a_k}}.
\end{align*}
对某些$a_i$为0时也成立.
\end{enumerate}
\end{solution}
\begin{conclusion}\label{对角矩阵行列式的子式和余子式}
\hypertarget{对角矩阵行列式的子式和余子式}{\textbf{对角矩阵行列式的子式和余子式:}}

设\(|\boldsymbol{A}|=\left|\begin{matrix}
a_1 & 0 & \cdots & 0\\
0 & a_2 & \cdots & 0\\
\vdots & \vdots & \ddots & \vdots\\
0 & 0 & \cdots & a_n
\end{matrix}\right|\),则其\(k\)阶子式\(\boldsymbol{A}\left(\begin{matrix}
i_1 & i_2 & \cdots & i_k\\
j_1 & j_2 & \cdots & j_k
\end{matrix}\right)\)除\(k\)阶主子式\(\boldsymbol{A}\left(\begin{matrix}
i_1 & i_2 & \cdots & i_k\\
i_1 & i_2 & \cdots & i_k
\end{matrix}\right)\)外都为零,其中\(k = 1,2,\cdots,n\).

记\(\widehat{\boldsymbol{A}}\left(\begin{matrix}
i_1 & i_2 & \cdots & i_k\\
j_1 & j_2 & \cdots & j_k
\end{matrix}\right)\)为\(\boldsymbol{A}\left(\begin{matrix}
i_1 & i_2 & \cdots & i_k\\
j_1 & j_2 & \cdots & j_k
\end{matrix}\right)\)的代数余子式(\(n - k\)阶).于是\(\widehat{\boldsymbol{A}}\left(\begin{matrix}
i_1 & i_2 & \cdots & i_k\\
j_1 & j_2 & \cdots & j_k
\end{matrix}\right)\)除\(\widehat{\boldsymbol{A}}\left(\begin{matrix}
i_1 & i_2 & \cdots & i_k\\
i_1 & i_2 & \cdots & i_k
\end{matrix}\right)\)外也都为零,其中\(k = 1,2,\cdots,n\).

并且
\begin{align*}
&\boldsymbol{A}\left(\begin{matrix}
i_1 & i_2 & \cdots & i_k\\
i_1 & i_2 & \cdots & i_k
\end{matrix}\right) = a_{i_1}a_{i_2}\cdots a_{i_k},
\\
&\widehat{\boldsymbol{A}}\left(\begin{matrix}
i_1 & i_2 & \cdots & i_k\\
i_1 & i_2 & \cdots & i_k
\end{matrix}\right) = a_1\cdots \hat{a}_{i_1}\cdots \hat{a}_{i_2}\cdots \hat{a}_{i_k}\cdots a_n\,
\end{align*}
其中\(k = 1,2,\cdots,n\).
\end{conclusion}

\begin{proposition}[\hypertarget{Cauchy行列式}{Cauchy行列式}]\label{Cauchy行列式}
证明:
\begin{gather}
|\boldsymbol{A}|=\left| \begin{matrix}
(a_1+b_1)^{-1}&		(a_1+b_2)^{-1}&		\cdots&		(a_1+b_n)^{-1}\\
(a_2+b_1)^{-1}&		(a_2+b_2)^{-1}&		\cdots&		(a_2+b_n)^{-1}\\
\vdots&		\vdots&		&		\vdots\\
(a_n+b_1)^{-1}&		(a_n+b_2)^{-1}&		\cdots&		(a_n+b_n)^{-1}\\
\end{matrix} \right|=\frac{\prod\limits_{1\le i<j\le n}{(a_j}-a_i)(b_j-b_i)}{\prod\limits_{1\leqslant i<j\leqslant m}{\left( a_i+b_j \right)}}.
\nonumber
\end{gather}
\end{proposition}
\begin{note}
需要记忆Cauchy行列式的计算方法.

1.分式分母有公共部分可以作差,得到的分子会变得相对简便.

2.行列式内行列做加减一般都是加减同一行(或列).但是在\hyperlink{循环行列式}{循环行列式}中,我们一般采取相邻两行(或列)相加减的方法.
\end{note}
\begin{proof}
\begin{align*}
&|\boldsymbol{A}|=\left| \begin{matrix}
\frac{1}{a_1+b_1}&		\frac{1}{a_1+b_2}&		\cdots&		\frac{1}{a_1+b_n}\\
\frac{1}{a_2+b_1}&		\frac{1}{a_2+b_2}&		\cdots&		\frac{1}{a_2+b_n}\\
\vdots&		\vdots&		&		\vdots\\
\frac{1}{a_n+b_1}&		\frac{1}{a_n+b_2}&		\cdots&		\frac{1}{a_n+b_n}\\
\end{matrix} \right|
\\
&\xlongequal[i=n-1,\cdots ,1]{-j_n+j_i}\left| \begin{matrix}
\frac{b_n-b_1}{\left( a_1+b_1 \right) \left( a_1+b_n \right)}&		\frac{b_n-b_2}{\left( a_1+b_2 \right) \left( a_1+b_n \right)}&		\cdots&		\frac{b_n-b_{n-1}}{\left( a_1+b_{n-1} \right) \left( a_1+b_n \right)}&		\frac{1}{a_1+b_n}\\
\frac{b_n-b_1}{\left( a_2+b_1 \right) \left( a_2+b_n \right)}&		\frac{b_n-b_2}{\left( a_2+b_2 \right) \left( a_2+b_n \right)}&		\cdots&		\frac{b_n-b_{n-1}}{\left( a_1+b_{n-1} \right) \left( a_2+b_n \right)}&		\frac{1}{a_2+b_n}\\
\vdots&		\vdots&		&		\vdots&		\vdots\\
\frac{b_n-b_1}{\left( a_n+b_1 \right) \left( a_n+b_n \right)}&		\frac{b_n-b_2}{\left( a_n+b_2 \right) \left( a_n+b_n \right)}&		\cdots&		\frac{b_n-b_{n-1}}{\left( a_1+b_{n-1} \right) \left( a_n+b_n \right)}&		\frac{1}{a_n+b_n}\\
\end{matrix} \right|
\\
&=\frac{\prod\limits_{i=1}^{n-1}{\left( b_n-b_i \right)}}{\prod\limits_{j=1}^n{\left( a_j+b_n \right)}}\left| \begin{matrix}
\frac{1}{a_1+b_1}&		\frac{1}{a_1+b_2}&		\cdots&		\frac{1}{a_1+b_{n-1}}&		1\\
\frac{1}{a_2+b_1}&		\frac{1}{a_2+b_2}&		\cdots&		\frac{1}{a_2+b_{n-1}}&		1\\
\vdots&		\vdots&		&		\vdots&		\vdots\\
\frac{1}{a_n+b_1}&		\frac{1}{a_n+b_2}&		\cdots&		\frac{1}{a_n+b_{n-1}}&		1\\
\end{matrix} \right|
\\
&\xlongequal[i=n-1,\cdots ,1]{-r_n+r_i}\frac{\prod\limits_{i=1}^{n-1}{\left( b_n-b_i \right)}}{\prod\limits_{j=1}^n{\left( a_j+b_n \right)}}\left| \begin{matrix}
\frac{a_n-a_1}{\left( a_1+b_1 \right) \left( a_n+b_1 \right)}&		\frac{a_n-a_1}{\left( a_1+b_2 \right) \left( a_n+b_2 \right)}&		\cdots&		\frac{a_n-a_1}{\left( a_1+b_{n-1} \right) \left( a_n+b_{n-1} \right)}&		0\\
\frac{a_n-a_2}{\left( a_2+b_1 \right) \left( a_n+b_1 \right)}&		\frac{a_n-a_2}{\left( a_1+b_2 \right) \left( a_n+b_2 \right)}&		\cdots&		\frac{a_n-a_2}{\left( a_1+b_{n-1} \right) \left( a_n+b_{n-1} \right)}&		0\\
\vdots&		\vdots&		&		\vdots&		\vdots\\
\frac{a_n-a_{n-1}}{\left( a_{n-1}+b_1 \right) \left( a_n+b_1 \right)}&		\frac{a_n-a_{n-1}}{\left( a_{n-1}+b_2 \right) \left( a_n+b_2 \right)}&		\cdots&		\frac{a_n-a_{n-1}}{\left( a_{n-1}+b_{n-1} \right) \left( a_n+b_{n-1} \right)}&		0\\
\frac{1}{a_n+b_1}&		\frac{1}{a_n+b_2}&		\cdots&		\frac{1}{a_n+b_{n-1}}&		1\\
\end{matrix} \right|
\\
&=\frac{\prod\limits_{i=1}^{n-1}{\left( b_n-b_i \right)}}{\prod\limits_{j=1}^n{\left( a_j+b_n \right)}}\cdot \frac{\prod\limits_{i=1}^{n-1}{\left( a_n-a_i \right)}}{\prod\limits_{k=1}^{n-1}{\left( a_n+b_k \right)}}\left| \begin{matrix}
\frac{1}{a_1+b_1}&		\frac{1}{a_1+b_2}&		\cdots&		\frac{1}{a_1+b_{n-1}}&		0\\
\frac{1}{a_2+b_1}&		\frac{1}{a_2+b_2}&		\cdots&		\frac{1}{a_2+b_{n-1}}&		0\\
\vdots&		\vdots&		&		\vdots&		\vdots\\
\frac{1}{a_{n-1}+b_1}&		\frac{1}{a_{n-1}+b_2}&		\cdots&		\frac{1}{a_{n-1}+b_{n-1}}&		0\\
1&		1&		\cdots&		1&		1\\
\end{matrix} \right|
\\
&\xlongequal[]{\text{按最后一列展开}}\frac{\prod\limits_{i=1}^{n-1}{\left( b_n-b_i \right) \left( a_n-a_i \right)}}{\prod\limits_{j=1}^n{\left( a_j+b_n \right) \prod\limits_{k=1}^{n-1}{\left( a_n+b_k \right)}}}\left| \begin{matrix}
\frac{1}{a_1+b_1}&		\frac{1}{a_1+b_2}&		\cdots&		\frac{1}{a_1+b_{n-1}}\\
\frac{1}{a_2+b_1}&		\frac{1}{a_2+b_2}&		\cdots&		\frac{1}{a_2+b_{n-1}}\\
\vdots&		\vdots&		&		\vdots\\
\frac{1}{a_{n-1}+b_1}&		\frac{1}{a_{n-1}+b_2}&		\cdots&		\frac{1}{a_{n-1}+b_{n-1}}\\
\end{matrix} \right|
\\
&=\frac{\prod\limits_{i=1}^{n-1}{\left( b_n-b_i \right) \left( a_n-a_i \right)}}{\prod\limits_{j=1}^n{\left( a_j+b_n \right) \prod\limits_{k=1}^{n-1}{\left( a_n+b_k \right)}}}\cdot D_{n-1}.
\nonumber
\end{align*}
不断递推下去即得
\begin{align*}
&D_n=\frac{\prod\limits_{i=1}^{n-1}{\left( b_n-b_i \right) \left( a_n-a_i \right)}}{\prod\limits_{j=1}^n{\left( a_j+b_n \right) \prod\limits_{k=1}^{n-1}{\left( a_n+b_k \right)}}}\cdot D_{n-1}=\frac{\prod\limits_{i=1}^{n-1}{\left( b_n-b_i \right) \left( a_n-a_i \right)}}{\prod\limits_{j=1}^n{\left( a_j+b_n \right) \prod\limits_{k=1}^{n-1}{\left( a_n+b_k \right)}}}\cdot \frac{\prod\limits_{i=1}^{n-2}{\left( b_{n-1}-b_i \right) \left( a_{n-1}-a_i \right)}}{\prod\limits_{j=1}^{n-1}{\left( a_j+b_{n-1} \right) \prod\limits_{k=1}^{n-2}{\left( a_{n-1}+b_k \right)}}}\cdot D_{n-2}
\\
&=\cdots =\frac{\prod\limits_{i=1}^{n-1}{\left( b_n-b_i \right) \left( a_n-a_i \right)}}{\prod\limits_{j=1}^n{\left( a_j+b_n \right) \prod\limits_{k=1}^{n-1}{\left( a_n+b_k \right)}}}\cdot \frac{\prod\limits_{i=1}^{n-2}{\left( b_{n-1}-b_i \right) \left( a_{n-1}-a_i \right)}}{\prod\limits_{j=1}^{n-1}{\left( a_j+b_{n-1} \right) \prod\limits_{k=1}^{n-2}{\left( a_{n-1}+b_k \right)}}}\cdots \cdots \frac{\prod\limits_{i=1}^2{\left( b_3-b_i \right) \left( a_3-a_i \right)}}{\prod\limits_{j=1}^3{\left( a_j+b_3 \right) \prod\limits_{k=1}^2{\left( a_3+b_k \right)}}}\cdot D_2
\\
&=\frac{\prod\limits_{i=1}^{n-1}{\left( b_n-b_i \right) \left( a_n-a_i \right)}}{\prod\limits_{j=1}^n{\left( a_j+b_n \right) \prod\limits_{k=1}^{n-1}{\left( a_n+b_k \right)}}}\cdot \frac{\prod\limits_{i=1}^{n-2}{\left( b_{n-1}-b_i \right) \left( a_{n-1}-a_i \right)}}{\prod\limits_{j=1}^{n-1}{\left( a_j+b_{n-1} \right) \prod\limits_{k=1}^{n-2}{\left( a_{n-1}+b_k \right)}}}\cdots 
\frac{\prod\limits_{i=1}^2{\left( b_3-b_i \right) \left( a_3-a_i \right)}}{\prod\limits_{j=1}^3{\left( a_j+b_3 \right) \prod\limits_{k=1}^2{\left( a_3+b_k \right)}}}\cdot \frac{\left( b_2-b_1 \right) \left( a_2-a_1 \right)}{\prod\limits_{j=1}^2{\left( a_j+b_2 \right) \left( a_2+b_1 \right)}}\cdot D_1
\\
&=\frac{\prod\limits_{i=1}^{n-1}{\left( b_n-b_i \right) \left( a_n-a_i \right)}}{\prod\limits_{j=1}^n{\left( a_j+b_n \right) \prod\limits_{k=1}^{n-1}{\left( a_n+b_k \right)}}}\cdot \frac{\prod\limits_{i=1}^{n-2}{\left( b_{n-1}-b_i \right) \left( a_{n-1}-a_i \right)}}{\prod\limits_{j=1}^{n-1}{\left( a_j+b_{n-1} \right) \prod\limits_{k=1}^{n-2}{\left( a_{n-1}+b_k \right)}}}\cdots 
\frac{\prod\limits_{i=1}^2{\left( b_3-b_i \right) \left( a_3-a_i \right)}}{\prod\limits_{j=1}^3{\left( a_j+b_3 \right) \prod\limits_{k=1}^2{\left( a_3+b_k \right)}}}\cdot \frac{\left( b_2-b_1 \right) \left( a_2-a_1 \right)}{\prod\limits_{j=1}^2{\left( a_j+b_2 \right) \left( a_2+b_1 \right)}}\cdot \frac{1}{a_1+b_1}
\\
&=\frac{\prod\limits_{1\le i<j\le n}{(a_j}-a_i)(b_j-b_i)}{\prod\limits_{1\le i\le j\le n}{(a_i}+b_j)\prod\limits_{1\le j<i\le n}{(a_i}+b_j)}=\frac{\prod\limits_{1\le i<j\le n}{(a_j}-a_i)(b_j-b_i)}{\prod\limits_{1\leqslant i<j\leqslant m}{\left( a_i+b_j \right)}}.
\nonumber
\end{align*}
\end{proof}

\begin{example}\label{使用数学归纳法计算行列式例题1}
    设$n$阶行列式
    \begin{align}
    A_n=\left| \begin{matrix}
    a_0+a_1&		a_1&		0&		0&		\cdots&		0&		0\\
    a_1&		a_1+a_2&		a_2&		0&		\cdots&		0&		0\\
    0&		a_2&		a_2+a_3&		a_3&		\cdots&		0&		0\\
    \vdots&		\vdots&		\vdots&		\vdots&		&		\vdots&		\vdots\\
    0&		0&		0&		0&		\cdots&		a_{n-1}&		a_{n-1}+a_n\\
    \end{matrix} \right|,
    \nonumber
    \end{align}
    求证:
    \begin{align}
    A_n=a_0a_1\cdots a_n\left( \frac{1}{a_0}+\frac{1}{a_1}+\cdots +\frac{1}{a_n} \right) .
    \nonumber
    \end{align}
    \end{example}
    \begin{note}
    用\hypertarget{用数学归纳法与行列式有关的结论}{\textbf{数学归纳法}}证明与行列式有关的结论.
    
    练习\ref{三对角行列式例题1}和练习\ref{三对角行列式例题2}都可同理使用用数学归纳法证明(对阶数$n$进行归纳即可).
    \end{note}
    \begin{proof}
    (数学归纳法)对阶数$n$进行归纳.当$n=1,2$时,结论显然成立.假设阶数小于$n$结论成立.
    
    现证明$n$阶的情形.注意到
    \begin{align*}
    A_n=\left| \begin{matrix}
    a_0+a_1&		a_1&		0&		0&		\cdots&		0	&0\\
    a_1&		a_1+a_2&		a_2&		0&		\cdots&		0&		0\\
    0&		a_2&		a_2+a_3&		a_3&		\cdots&		0&		0\\
    \vdots&		\vdots&		\vdots&		\vdots&		&		\vdots&		\vdots\\
    0&		0&		0&		0&		\cdots&		a_{n-1}&		a_{n-1}+a_n\\
    \end{matrix} \right|=\left( a_{n-1}+a_n \right) A_{n-1}-a_{n-1}^{2}A_{n-2}.
    \nonumber
    \end{align*}
    将归纳假设代入上面的式子中得
    \begin{align*}
    A_n&=\left( a_{n-1}+a_n \right) A_{n-1}-a_{n-1}^{2}A_{n-2}
    \\
    &=\left( a_{n-1}+a_n \right) a_0a_1\cdots a_{n-1}\left( \frac{1}{a_0}+\frac{1}{a_1}+\cdots +\frac{1}{a_{n-1}} \right) -a_{n-1}^{2}a_0a_1\cdots a_{n-2}\left( \frac{1}{a_0}+\frac{1}{a_1}+\cdots +\frac{1}{a_{n-2}} \right) 
    \\
    &=a_0a_1\cdots a_n\left( \frac{1}{a_0}+\frac{1}{a_1}+\cdots +\frac{1}{a_{n-1}} \right) +a_0a_1\cdots a_{n-2}a_{n-1}^{2}\frac{1}{a_{n-1}}
    \\
    &=a_0a_1\cdots a_{n-1}\left[ a_n\left( \frac{1}{a_0}+\frac{1}{a_1}+\cdots +\frac{1}{a_{n-1}} \right) +1 \right] 
    \\
    &=a_0a_1\cdots a_{n-1}a_n\left( \frac{1}{a_0}+\frac{1}{a_1}+\cdots +\frac{1}{a_{n-1}}+\frac{1}{a_n} \right) .
    \nonumber
    \end{align*}
    故由数学归纳法可知,结论对任意正整数$n$都成立.
    \end{proof}

\begin{example}
设\(n(n > 2)\)阶行列式\(\vert A \vert\)的所有元素或为\(1\)或为\(-1\),求证:\(\vert A \vert\)的绝对值小于等于\(\frac{2}{3}n!\).
\end{example}
\begin{solution}
对阶数$n$进行归纳.当$n=3$时,将$\left| A \right|$的第一列元素为-1的行都乘以-1,再将$\left| A \right|$的第一行元素为1的列都乘以-1,$\left| A \right|$的绝对值不改变.

因此不妨设$\left| A \right|=\left| \begin{matrix}
1&		-1&		-1\\
1&		a_0&		b_0\\
1&		c_0&		d_0\\
\end{matrix} \right|,\text{其中}a_0,b_0,c_0,d_0=1\text{或}-1.$

从而
\begin{align*}
\left| A \right|=\left| \begin{matrix}
1&		-1&		-1\\
1&		a_0&		b_0\\
1&		c_0&		d_0\\
\end{matrix} \right|\xlongequal[i=2,3]{j_1+j_i}\left| \begin{matrix}
1&		0&		0\\
1&		a&		b\\
1&		c&		d\\
\end{matrix} \right|,\text{其中}a,b,c,d=0\text{或}2.
\nonumber
\end{align*}
于是
\begin{align*}
abs \left( \left| A \right| \right) =abs \left( \left| \begin{matrix}
1&		0&		0\\
1&		a&		b\\
1&		c&		d\\
\end{matrix} \right| \right) =abs \left( ad-bc \right) \leqslant 4=\frac{2}{3}\cdot 3!
\nonumber
\end{align*}
假设n-1阶时结论成立,现证$n$阶的情形.将$\left| A \right|$按第一行展开得
\begin{align*}
\left| A \right|=a_{11}A_{11}+a_{12}A_{12}+\cdots +a_{1n}A_{1n},\text{其中}a_{1i}=1\text{或}-1\left( i=1,2\cdots ,n \right) .
\nonumber
\end{align*}
从而由归纳假设可得
\begin{align*}
abs \left( \left| A \right| \right) &=abs \left( a_{11}A_{11}+a_{12}A_{12}+\cdots +a_{1n}A_{1n} \right) \leqslant abs \left( A_{11} \right) +abs \left( A_{12} \right) +\cdots +abs \left( A_{1n} \right) 
\\
&\leqslant \frac{2}{3}\left( n-1 \right) !+\frac{2}{3}\left( n-1 \right) !+\cdots +\frac{2}{3}\left( n-1 \right) !
\\
&=n\cdot \frac{2}{3}\left( n-1 \right) !=\frac{2}{3}n!.
\nonumber
\end{align*}
故由数学归纳法可知结论对任意正整数都成立.
\end{solution}

\begin{proposition}[行列式的求导运算]\label{proposition:行列式的求导运算}
设\(f_{ij}(t)\)是可微函数,
\begin{align*}
F(t) = 
\left| \begin{matrix}
f_{11}(t) & f_{12}(t) & \cdots & f_{1n}(t) \\
f_{21}(t) & f_{22}(t) & \cdots & f_{2n}(t) \\
\vdots & \vdots &  & \vdots \\
f_{n1}(t) & f_{n2}(t) & \cdots & f_{nn}(t)
\end{matrix} \right| 
\nonumber
\end{align*}
求证:$\frac{d}{dt}F\left( t \right) =\sum_{j=1}^n{F_j\left( t \right)}$,其中
\begin{align*}
F_{j}(t) = 
\left| \begin{matrix}
f_{11}(t)&		f_{12}(t)&		\cdots&		\frac{d}{dt}f_{1j}(t)&		\cdots&		f_{1n}(t)\\
f_{21}(t)&		f_{22}(t)&		\cdots&		\frac{d}{dt}f_{2j}(t)&		\cdots&		f_{2n}(t)\\
\vdots&		\vdots&		&		\vdots&		&		\vdots\\
f_{n1}(t)&		f_{n2}(t)&		\cdots&		\frac{d}{dt}f_{nj}(t)&		\cdots&		f_{nn}(t)\\
\end{matrix} \right| 
\nonumber
\end{align*}
\end{proposition}
\begin{proof}
{\color{blue}证法一(数学归纳法):}对阶数$n$进行归纳.当$n=1$时结论显然成立.假设$n-1$阶时结论成立,现证$n$阶的情形.

将$F(t)$按第一列展开得
\begin{align*}
F\left( t \right) =f_{11}\left( t \right) A_{11}\left( t \right) +f_{21}\left( t \right) A_{21}\left( t \right) +\cdots +f_{n1}\left( t \right) A_{n1}\left( t \right) .
\nonumber
\end{align*}
其中$A_{i1}(t)$是元素$f_{i1}(t)$的代数余子式.($i=1,2,\cdots,n$)

从而由归纳假设可得
\begin{gather*}
A_{i1}^{\prime}\left( t \right) =\frac{d}{dt}A_{i1}\left( t \right)=\sum_{k=2}^{n}{A_{i1}^{k}(t),i=1,2,\cdots ,n}. 
\\
\text{其中}A_{i1}^{k}(t)=\left| \begin{matrix}
f_{12}\left( t \right)&		\cdots&		\frac{d}{dt}f_{1k}\left( t \right)&		\cdots&		f_{1n}\left( t \right)\\
\vdots&		&		\vdots&		&		\vdots\\
f_{i-1,2}(t)&		\cdots&		\frac{d}{dt}f_{i-1,k}\left( t \right)&		\cdots&		f_{i-1,n}\left( t \right)\\
f_{i+1,2}\left( t \right)&		\cdots&		\frac{d}{dt}f_{i+1,k}(t)&		\cdots&		f_{i+1,n}\left( t \right)\\
\vdots&		&		\vdots&		&		\vdots\\
f_{n2}\left( t \right)&		\cdots&		\frac{d}{dt}f_{nk}\left( t \right)&		\cdots&		f_{nn}\left( t \right)\\
\end{matrix} \right|,k=2,3,\cdots ,n.
\nonumber
\end{gather*}
于是,我们就有
\begin{align*}
\frac{d}{dt}F\left( t \right) &=\frac{d}{dt}\left[ f_{11}\left( t \right) A_{11}\left( t \right) +f_{21}\left( t \right) A_{21}\left( t \right) +\cdots +f_{n1}\left( t \right) A_{n1}\left( t \right) \right] 
\\
&=f_{11}^{\prime}\left( t \right) A_{11}\left( t \right) +f_{21}^{\prime}\left( t \right) A_{21}\left( t \right) +\cdots +f_{n1}^{\prime}\left( t \right) A_{n1}\left( t \right) +f_{11}\left( t \right) A_{11}^{\prime}\left( t \right) +f_{21}\left( t \right) A_{21}^{\prime}\left( t \right) +\cdots +f_{n1}\left( t \right) A_{n1}^{\prime}\left( t \right) 
\\
&=\sum_{i=1}^n{f_{i1}^{\prime}\left( t \right) A_{i1}\left( t \right)}+f_{11}\left( t \right) \sum_{k=2}^{n}{A_{11}^{k}(t)}+f_{21}\left( t \right) \sum_{k=2}^{n}{A_{21}^{k}(t)}+\cdots +f_{n1}\left( t \right) \sum_{k=2}^{n}{A_{n1}^{k}(t)}
\\
&=\sum_{i=1}^n{f_{i1}^{\prime}\left( t \right) A_{i1}\left( t \right)}+\sum_{i=1}^n{\left( f_{i1}\left( t \right) \sum_{k=2}^n{A_{i1}^{k}\left( t \right)} \right)}
\\
&=\sum_{i=1}^n{f_{i1}^{\prime}\left( t \right) A_{i1}\left( t \right)}+\sum_{i=1}^n{f_{i1}\left( t \right) \left( A_{i1}^{2}+A_{i1}^{3}+\cdots +A_{i1}^{n} \right)}
\\
&=\sum_{i=1}^n{f_{i1}^{\prime}\left( t \right) A_{i1}\left( t \right)}+\sum_{i=1}^n{f_{i1}\left( t \right) A_{i1}^{2}}+\sum_{i=1}^n{f_{i1}\left( t \right) A_{i1}^{3}}+\cdots +\sum_{i=1}^n{f_{i1}\left( t \right) A_{i1}^{n}}
\\
&=F_1\left( t \right) +F_2\left( t \right) +F_3\left( t \right) +\cdots +F_n\left( t \right) 
\\
&=\sum_{j=1}^n{F_j\left( t \right)}.
\end{align*}
故由数学归纳法可知结论对任意正整数都成立.

{\color{blue}证法二(行列式的组合定义):}由行列式的组合定义可得
\begin{align*}
F(t)=\sum_{1\le k_1,k_2,\cdots ,k_n\le n}{(}-1)^{\tau (k_1k_2\cdots k_n)}f_{k_11}(t)f_{k_22}(t)\cdots f_{k_nn}(t).
\end{align*}
因此
\begin{align*}
\frac{d}{dt}F(t)&=\sum_{1\le k_1,k_2,\cdots ,k_n\le n}{(}-1)^{\tau (k_1k_2\cdots k_n)}f_{k_{11}}(t)f_{k_{22}}(t)\cdots f_{k_{nn}}(t)
\\
&\quad+\sum_{1\le k_1,k_2,\cdots ,k_n\le n}{(}-1)^{\tau (k_1k_2\cdots k_n)}f_{k_{11}}(t)f\prime_{k_{22}}(t)\cdots f_{k_{nn}}(t)
\\
&\quad+\cdots +\sum_{1\le k_1,k_2,\cdots ,k_n\le n}{(}-1)^{\tau (k_1k_2\cdots k_n)}f_{k_{11}}(t)f_{k_{22}}(t)\cdots f\prime_{k_{nn}}(t)
\\
&=F_1(t)+F_2(t)+\cdots +F_n(t).
\end{align*}
\end{proof}








\end{document}
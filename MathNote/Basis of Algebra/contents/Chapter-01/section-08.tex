\documentclass[../../main.tex]{subfiles}
\graphicspath{{\subfix{../../image/}}} % 指定图片目录,后续可以直接使用图片文件名。

% 例如:
% \begin{figure}[H]
% \centering
% \includegraphics[scale=0.4]{图.png}
% \caption{}
% \label{figure:图}
% \end{figure}
% 注意:上述\label{}一定要放在\caption{}之后,否则引用图片序号会只会显示??.

\begin{document}

\section{拆分法}

\begin{proposition}[\hypertarget{大拆分法}{大拆分法}]\label{大拆分法}
设\(t\)是一个参数,
\begin{align*}
|A(t)| = 
\begin{vmatrix}
a_{11}+t & a_{12}+t & \cdots & a_{1n}+t \\
a_{21}+t & a_{22}+t & \cdots & a_{2n}+t \\
\vdots & \vdots & \ddots & \vdots \\
a_{n1}+t & a_{n2}+t & \cdots & a_{nn}+t
\end{vmatrix}
\nonumber
\end{align*}

求证:
\begin{align*}
|A(t)| = |A(0)| + t \sum_{i,j = 1}^{n} A_{ij},
\nonumber
\end{align*}
其中\(A_{ij}\)是\(a_{ij}\)在\(|A(0)|\)中的代数余子式.
\end{proposition}
\begin{note}
大拆分法的想法:
\textbf{将行列式的每一行/列拆分成两行/列},得到
\begin{align*}
|\boldsymbol{A}(t)|=|\boldsymbol{A}(0)|+t\sum_{j=1}^n{|A_j|}.
\text{其中}A_j=\bordermatrix{%
&1 &	\cdots	&	i&	\cdots	&n		\cr
& a_{11}&		\cdots&		t&		\cdots&		a_{1n}\cr
&a_{21}&		\cdots&		t&		\cdots&		a_{2n}\cr
&\vdots&		&		\vdots&		&		\vdots\cr
&a_{n1}&		\cdots&		t&		\cdots&		a_{nn}
},j=1,2,\cdots ,n.
\end{align*}
大拆分法的关键是\textbf{拆分},根据行列式的性质将原行列式拆分成$2^n$个行列式.(不一定需要公共的$t$).不仅要熟悉大拆分法的想法还要记住大拆分法的这个命题.
\end{note}
\begin{remark}
大拆分法后续计算不一定要按行/列展开,拆分的方式一般比较多,只要拆分的方式方便后续计算即可.
\end{remark}
\begin{proof}
将行列式第一列拆成两列再展开得到
\begin{align*}
|\boldsymbol{A}(t)|=\left| \begin{matrix}
a_{11}&		a_{12}+t&		\cdots&		a_{1n}+t\\
a_{21}&		a_{22}+t&		\cdots&		a_{2n}+t\\
\vdots&		\vdots&		&		\vdots\\
a_{n1}&		a_{n2}+t&		\cdots&		a_{nn}+t\\
\end{matrix} \right|+\left| \begin{matrix}
t&		a_{12}+t&		\cdots&		a_{1n}+t\\
t&		a_{22}+t&		\cdots&		a_{2n}+t\\
\vdots&		\vdots&		&		\vdots\\
t&		a_{n2}+t&		\cdots&		a_{nn}+t\\
\end{matrix} \right|.
\nonumber
\end{align*}
将上式右边第二个行列式的第一列乘-1加到后面每一列上,得到
\begin{align*}
\left| \boldsymbol{A} \right|=\left| \begin{matrix}
a_{11}&		a_{12}+t&		\cdots&		a_{1n}+t\\
a_{21}&		a_{22}+t&		\cdots&		a_{2n}+t\\
\vdots&		\vdots&		&		\vdots\\
a_{n1}&		a_{n2}+t&		\cdots&		a_{nn}+t\\
\end{matrix} \right|+\left| \begin{matrix}
t&		a_{12}&		\cdots&		a_{1n}\\
t&		a_{22}&		\cdots&		a_{2n}\\
\vdots&		\vdots&		&		\vdots\\
t&		a_{n2}&		\cdots&		a_{nn}\\
\end{matrix} \right| .
\nonumber
\end{align*}
再对上式右边第一个行列式的第二列拆成两列展开,不断这样做下去就可得到
\begin{gather*}
|\boldsymbol{A}(t)|=\left| \begin{matrix}
a_{11}&		a_{12}&		\cdots&		a_{1n}\\
a_{21}&		a_{22}&		\cdots&		a_{2n}\\
\vdots&		\vdots&		&		\vdots\\
a_{n1}&		a_{n2}&		\cdots&		a_{nn}\\
\end{matrix} \right|+\left| \begin{matrix}
t&		a_{12}&		\cdots&		a_{1n}\\
t&		a_{22}&		\cdots&		a_{2n}\\
\vdots&		\vdots&		&		\vdots\\
t&		a_{n2}&		\cdots&		a_{nn}\\
\end{matrix} \right|+\cdots +\left| \begin{matrix}
a_{11}&		a_{1n}&		\cdots&		t\\
a_{21}&		a_{2n}&		\cdots&		t\\
\vdots&		\vdots&		&		\vdots\\
a_{n1}&		a_{nn}&		\cdots&		t\\
\end{matrix} \right|=|\boldsymbol{A}(0)|+\sum_{j=1}^n{|A_j|}.
\end{gather*}
其中$A_j=\bordermatrix{%
&1 &	\cdots	&	i&	\cdots	&n		\cr
& a_{11}&		\cdots&		t&		\cdots&		a_{1n}\cr
&a_{21}&		\cdots&		t&		\cdots&		a_{2n}\cr
&\vdots&		&		\vdots&		&		\vdots\cr
&a_{n1}&		\cdots&		t&		\cdots&		a_{nn}
}$,$j=1,2,\cdots ,n.$
将$A_j$按第$j$列展开可得
\begin{align*}
A_j=\left| \begin{matrix}
a_{11}&		\cdots&		t&		\cdots&		a_{1n}\\
a_{21}&		\cdots&		t&		\cdots&		a_{2n}\\
\vdots&		&		\vdots&		&		\vdots\\
a_{n1}&		\cdots&		t&		\cdots&		a_{nn}\\
\end{matrix} \right|=t\left( A_{1j}+A_{2j}+\cdots +A_{nj} \right) =t\sum_{i=1}^n{A_{ij}}.
\nonumber
\end{align*}
从而
\begin{align*}
|\boldsymbol{A}(t)|=|\boldsymbol{A}(0)|+\sum_{i=1}^n{A_i}=|\boldsymbol{A}(0)|+t\sum_{i=1}^n{\sum_{i=1}^n{A_{ij}}}=|\boldsymbol{A}(0)|+t\sum_{i,j=1}^n{A_{ij}}.
\end{align*}
\end{proof}

\begin{corollary}[\hypertarget{大拆分法的推广}{推广的大拆分法}]\label{大拆分法的推广}
设
\begin{align*}
|A| = 
\begin{vmatrix}
a_{11} & a_{12} & \cdots & a_{1n} \\
a_{21} & a_{22} & \cdots & a_{2n} \\
\vdots & \vdots & \ddots & \vdots \\
a_{n1} & a_{n2} & \cdots & a_{nn}
\end{vmatrix},
\nonumber
\end{align*}
则
\begin{align*}
|A(t_1,t_2,\cdots,t_n)| = 
\begin{vmatrix}
a_{11}+t_1 & a_{12}+t_2 & \cdots & a_{1n}+t_n \\
a_{21}+t_1 & a_{22}+t_2 & \cdots & a_{2n}+t_n \\
\vdots & \vdots & \ddots & \vdots \\
a_{n1}+t_1 & a_{n2}+t_2 & \cdots & a_{nn}+t_n
\end{vmatrix}
= |A| + \sum_{j = 1}^{n} \left( t_j \sum_{i = 1}^{n} A_{ij} \right).
\nonumber
\end{align*}
\end{corollary}
\begin{note}
记忆这种推广的大拆分法的想法(即\textbf{将行列式的每一行/列拆分成两行/列}).

这里推广的大拆分法的关键也是\textbf{要找到合适的$t_1,t_2,\cdots,t_n$}进行拆分将原行列式拆分成更好处理的形式.
\end{note}
\begin{remark}
大拆分法后续计算不一定要按行/列展开,拆分的方式一般比较多,只要拆分的方式方便后续计算即可.
\end{remark}
\begin{proof}
运用\hyperlink{大拆分法}{大拆分法}的证明方法不难得到.
\end{proof}

\begin{proposition}[\hypertarget{小拆分法}{小拆分法}]\label{小拆分法}
设
\begin{align*}
|A| = 
\begin{vmatrix}
a_{11} & a_{12} & \cdots & a_{1n} \\
a_{21} & a_{22} & \cdots & a_{2n} \\
\vdots & \vdots & \ddots & \vdots \\
a_{n1} & a_{n2} & \cdots & a_{nn}
\end{vmatrix},
\nonumber
\end{align*}
并且$a_{in}$可以拆分成$b_{in}+c_{in}$,$\,\,i=1,2,\cdots,n.$

则
\begin{align*}
\left| \boldsymbol{A} \right|=\left| \begin{matrix}
a_{11}&		a_{12}&		\cdots&		a_{1n}\\
a_{21}&		a_{22}&		\cdots&		a_{2n}\\
\vdots&		\vdots&		&		\vdots\\
a_{n1}&		a_{n2}&		\cdots&		a_{nn}\\
\end{matrix} \right|=\left| \begin{matrix}
a_{11}&		a_{12}&		\cdots&		b_{1n}+c_{1n}\\
a_{21}&		a_{22}&		\cdots&		b_{2n}+c_{2n}\\
\vdots&		\vdots&		&		\vdots\\
a_{n1}&		a_{n2}&		\cdots&		b_{nn}+c_{nn}\\
\end{matrix} \right|=\left| \begin{matrix}
a_{11}&		a_{12}&		\cdots&		b_{1n}\\
a_{21}&		a_{22}&		\cdots&		b_{2n}\\
\vdots&		\vdots&		&		\vdots\\
a_{n1}&		a_{n2}&		\cdots&		b_{nn}\\
\end{matrix} \right|+\left| \begin{matrix}
a_{11}&		a_{12}&		\cdots&		c_{1n}\\
a_{21}&		a_{22}&		\cdots&		c_{2n}\\
\vdots&		\vdots&		&		\vdots\\
a_{n1}&		a_{n2}&		\cdots&		c_{nn}\\
\end{matrix} \right|.
\end{align*}
\end{proposition}
\begin{note}
记忆小拆分法的想法(即\textbf{拆边列/行,再展开得到递推式}).
\end{note}
\begin{remark}
若已知的拆分不是最后一列而是其他的某一行或某一列,则可以通过\hyperref[pro:行列式计算常识]{倒排、旋转、翻转、两行或两列对换}的方法将这一行或一列变成最后一列,再按照上述方法进行拆分即可.

小拆分法后续计算也不一定要按行/列展开,拆分的方式一般比较多,只要拆分的方式方便后续计算即可. 
\end{remark}
\begin{proof}
由行列式的性质可直接得到结论.
\end{proof}

\begin{proposition}\label{大/小拆分法例题1}
计算$n$阶行列式:
\begin{align*}
|\boldsymbol{A}|=\left| \begin{matrix}
a&		b&		\cdots&		b\\
b&		a&		\cdots&		b\\
\vdots&		\vdots&		&		\vdots\\
b&		b&		\cdots&		a\\
\end{matrix} \right|.
\end{align*}
\end{proposition}
\begin{solution}
{\color{blue}\text{解法一(\hyperlink{大拆分法}{大拆分法}):}}
注意到
\begin{align*}
|\boldsymbol{A}|&=\left| \begin{matrix}
a&		b&		\cdots&		b\\
b&		a&		\cdots&		b\\
\vdots&		\vdots&		&		\vdots\\
b&		b&		\cdots&		a\\
\end{matrix} \right|=\left| \begin{matrix}
b+\left( a-b \right)&		b+0&		\cdots&		b+0\\
b+0&		b+\left( a-b \right)&		\cdots&		b+0\\
\vdots&		\vdots&		&		\vdots\\
b+0&		b+0&		\cdots&		b+\left( a-b \right)\\
\end{matrix} \right|
\\
&=\left| \begin{matrix}
a-b&		0&		\cdots&		0\\
0&		a-b&		\cdots&		0\\
\vdots&		\vdots&		&		\vdots\\
0&		0&		\cdots&		a-b\\
\end{matrix} \right|+\sum_{i=1}^n{A_i}=\left( a-b \right) ^n+\sum_{i=1}^n{A_i}.
\end{align*}
其中$A_i$是第$i$行元素全为$b$,主对角元素除了$( i,i )$元外都为$a-b$,其他元素都为0的$n$阶行列式.

又因为
\begin{align*}
A_i=\begin{array}{l}
1\\
\vdots\\
i\\
\vdots\\
n\\
\end{array}\left| \begin{matrix}
a-b&		&		&		&		\\
&		\ddots&		&		&		\\
b&		\cdots&		b&		\cdots&		b\\
&		&		&		\ddots&		\\
&		&		&		&		a-b\\
\end{matrix} \right|=b\left( a-b \right) ^{n-1},i=1,2,\cdots ,n. 
\end{align*}
所以
\begin{align*}
|\boldsymbol{A}|=\left( a-b \right) ^n+\sum_{i=1}^n{A_i}=\left( a-b \right) ^n+nb\left( a-b \right) ^{n-1}=\left[ a+\left( n-1 \right) b \right] \left( a-b \right) ^{n-1}.
\end{align*}
{\color{blue}\text{解法二(\hyperlink{小拆分法}{小拆分法}):}}
记原行列式为$D_n$,其中$n$为原行列式的阶数.则将原行列式按第一列拆开为两个行列式得
\begin{align*}
D_n&=\left| \begin{matrix}
a&		b&		\cdots&		b\\
b&		a&		\cdots&		b\\
\vdots&		\vdots&		&		\vdots\\
b&		b&		\cdots&		a\\
\end{matrix} \right|=\left| \begin{matrix}
b+\left( a-b \right)&		b&		\cdots&		b\\
b+0&		a&		\cdots&		b\\
\vdots&		\vdots&		&		\vdots\\
b+0&		b&		\cdots&		a\\
\end{matrix} \right|=\left| \begin{matrix}
b&		b&		\cdots&		b\\
b&		a&		\cdots&		b\\
\vdots&		\vdots&		&		\vdots\\
b&		b&		\cdots&		a\\
\end{matrix} \right|+\left| \begin{matrix}
a-b&		b&		\cdots&		b\\
0&		a&		\cdots&		b\\
\vdots&		\vdots&		&		\vdots\\
0&		b&		\cdots&		a\\
\end{matrix} \right|
\\
&=\left| \begin{matrix}
b&		b&		\cdots&		b\\
0&		a-b&		\cdots&		0\\
\vdots&		\vdots&		&		\vdots\\
0&		0&		\cdots&		a-b\\
\end{matrix} \right|+\left( a-b \right) D_{n-1}=b\left( a-b \right) ^{n-1}+\left( a-b \right) D_{n-1}.
(n\ge2)
\end{align*}
从而由上式递推可得
\begin{align*}
D_n&=b\left( a-b \right) ^{n-1}+\left( a-b \right) D_{n-1}
\\
&=b\left( a-b \right) ^{n-1}+\left( a-b \right) \left[ b\left( a-b \right) ^{n-2}+\left( a-b \right) D_{n-2} \right] 
=2b\left( a-b \right) ^{n-1}+\left( a-b \right) ^2D_{n-2}
\\
&=\cdots =\left( n-1 \right) b\left( a-b \right) ^{n-1}+\left( a-b \right) ^{n-1}D_1
\\
&=\left( n-1 \right) b\left( a-b \right) ^{n-1}+\left( a-b \right) ^{n-1}a
\\
&=\left[ a+\left( n-1 \right) b \right] \left( a-b \right) ^{n-1}.
\end{align*}
{\color{blue}\text{解法三(\hyperlink{行列式计算:求和法}{求和法}):}}
\begin{align*}
|\boldsymbol{A}|&=\left| \begin{matrix}
a&		b&		\cdots&		b\\
b&		a&		\cdots&		b\\
\vdots&		\vdots&		&		\vdots\\
b&		b&		\cdots&		a\\
\end{matrix} \right|\xlongequal[i=2,3,\cdots ,n]{j_i+j_1}\left| \begin{matrix}
a+\left( n-1 \right) b&		b&		\cdots&		b\\
a+\left( n-1 \right) b&		a&		\cdots&		b\\
\vdots&		\vdots&		&		\vdots\\
a+\left( n-1 \right) b&		b&		\cdots&		a\\
\end{matrix} \right|=\left[ a+\left( n-1 \right) b \right] \left| \begin{matrix}
1&		b&		\cdots&		b\\
1&		a&		\cdots&		b\\
\vdots&		\vdots&		&		\vdots\\
1&		b&		\cdots&		a\\
\end{matrix} \right|
\\
&\xlongequal[i=2,3,\cdots ,n]{-r_1+r_i}\left[ a+\left( n-1 \right) b \right] \left| \begin{matrix}
1&		b&		\cdots&		b\\
0&		a-b&		\cdots&		0\\
\vdots&		\vdots&		&		\vdots\\
0&		0&		\cdots&		a-b\\
\end{matrix} \right|=\left[ a+\left( n-1 \right) b \right] \left( a-b \right) ^{n-1}.
\end{align*}
{\color{blue}\text{解法四(\hyperlink{"爪"型行列式的推广}{"爪"型行列式的推广}):}}
\begin{align*}
|\boldsymbol{A}|&=\left| \begin{matrix}
a&		b&		\cdots&		b\\
b&		a&		\cdots&		b\\
\vdots&		\vdots&		&		\vdots\\
b&		b&		\cdots&		a\\
\end{matrix} \right|\xlongequal[i=2,3,\cdots ,n]{-r_1+r_i}\left| \begin{matrix}
a&		b&		\cdots&		b\\
b-a&		a-b&		\cdots&		0\\
\vdots&		\vdots&		&		\vdots\\
b-a&		0&		\cdots&		a-b\\
\end{matrix} \right|
\\
&\xlongequal[i=2,3,\cdots ,n]{-j_i+j_1}\left| \begin{matrix}
a-\left( n-1 \right) b&		b&		\cdots&		b\\
0&		a-b&		\cdots&		0\\
\vdots&		\vdots&		&		\vdots\\
0&		0&		\cdots&		a-b\\
\end{matrix} \right|=\left[ a-\left( n-1 \right) b \right] \left( a-b \right) ^{n-1}.
\end{align*}
\end{solution}

\begin{proposition}\label{proposition:常见行列式1}
计算$n$阶行列式:
\begin{align*}
|\boldsymbol{A}|=\left| \begin{matrix}
a&		b&		\cdots&		b\\
c&		a&		\cdots&		b\\
\vdots&		\vdots&		&		\vdots\\
c&		c&		\cdots&		a\\
\end{matrix} \right|.
\end{align*}
\end{proposition}
\begin{solution}
{\color{blue}\text{解法一(\hyperlink{大拆分法}{大拆分法}):}}
令
\begin{align*}
|\boldsymbol{A}(t)|=\left| \begin{matrix}
a+t&		b+t&		\cdots&		b+t\\
c+t&		a+t&		\cdots&		b+t\\
\vdots&		\vdots&		&		\vdots\\
c+t&		c+t&		\cdots&		a+t\\
\end{matrix} \right|=|\boldsymbol{A}|+tu,  u=\sum_{i,j=1}^n{A_{ij}.}
\nonumber
\end{align*}
当$t=-b$时,可得
\begin{align*}
|\boldsymbol{A}(-b)|=\left| \begin{matrix}
a-b&		0&		\cdots&		0\\
c-b&		a-b&		\cdots&		0\\
\vdots&		\vdots&		&		\vdots\\
c-b&		c-b&		\cdots&		a-b\\
\end{matrix} \right|=|\boldsymbol{A}|-bu=(a-b)^n.
\end{align*}
当$t=-c$时,可得
\begin{align*}
|\boldsymbol{A}(-c)|=\left| \begin{matrix}
a-c&		b-c&		\cdots&		b-c\\
0&		a-c&		\cdots&		b-c\\
\vdots&		\vdots&		&		\vdots\\
0&		0&		\cdots&		a-c\\
\end{matrix} \right|=|\boldsymbol{A}|-cu=(a-c)^n.
\end{align*}
若$b\ne c$,则联立上面两式可得
\begin{align*}
\left| \boldsymbol{A} \right|=\frac{b\left( a-c \right) ^n-c\left( a-b \right) ^n}{b-c}.
\nonumber
\end{align*}
若$b=c$,则由\refpro{大/小拆分法例题1}可知
\begin{align*}
|\boldsymbol{A}|=\left[ a+\left( n-1 \right) b \right] \left( a-b \right) ^{n-1}.
\nonumber
\end{align*}
{\color{blue}\text{解法二(\hyperlink{小拆分法}{小拆分法}):}}
记原行列式为$D_n$,其中$n$为原行列式的阶数.则将原行列式分别按第一行、第一列拆开为两个行列式得
\begin{align*}
D_n&=\left| \begin{matrix}
a&		b&		\cdots&		b\\
c&		a&		\cdots&		b\\
\vdots&		\vdots&		&		\vdots\\
c&		c&		\cdots&		a\\
\end{matrix} \right|=\left| \begin{matrix}
b+\left( a-b \right)&		b+0&		\cdots&		b+0\\
c&		a&		\cdots&		b\\
\vdots&		\vdots&		&		\vdots\\
c&		c&		\cdots&		a\\
\end{matrix} \right|=\left| \begin{matrix}
b&		b&		\cdots&		b\\
c&		a&		\cdots&		b\\
\vdots&		\vdots&		&		\vdots\\
c&		c&		\cdots&		a\\
\end{matrix} \right|+\left| \begin{matrix}
a-b&		0&		\cdots&		0\\
c&		a&		\cdots&		b\\
\vdots&		\vdots&		&		\vdots\\
c&		c&		\cdots&		a\\
\end{matrix} \right|
\\
&=b\left| \begin{matrix}
1&		1&		\cdots&		1\\
c&		a&		\cdots&		b\\
\vdots&		\vdots&		&		\vdots\\
c&		c&		\cdots&		a\\
\end{matrix} \right|+\left( a-b \right) D_{n-1}=b\left| \begin{matrix}
1&		1&		\cdots&		1\\
0&		a-c&		\cdots&		b-c\\
\vdots&		\vdots&		&		\vdots\\
0&		0&		\cdots&		a-c\\
\end{matrix} \right|+\left( a-b \right) D_{n-1}
\\
&=b\left( a-c \right) ^{n-1}++\left( a-b \right) D_{n-1}.\left( n\ge 2 \right) 
\end{align*}
\begin{align*}
D_n&=\left| \begin{matrix}
a&		b&		\cdots&		b\\
c&		a&		\cdots&		b\\
\vdots&		\vdots&		&		\vdots\\
c&		c&		\cdots&		a\\
\end{matrix} \right|=\left| \begin{matrix}
c+\left( a-c \right)&		b&		\cdots&		b\\
c+0&		a&		\cdots&		b\\
\vdots&		\vdots&		&		\vdots\\
c+0&		c&		\cdots&		a\\
\end{matrix} \right|=\left| \begin{matrix}
c&		b&		\cdots&		b\\
c&		a&		\cdots&		b\\
\vdots&		\vdots&		&		\vdots\\
c&		c&		\cdots&		a\\
\end{matrix} \right|+\left| \begin{matrix}
a-c&		b&		\cdots&		b\\
0&		a&		\cdots&		b\\
\vdots&		\vdots&		&		\vdots\\
0&		c&		\cdots&		a\\
\end{matrix} \right|
\\
&=c\left| \begin{matrix}
1&		b&		\cdots&		b\\
1&		a&		\cdots&		b\\
\vdots&		\vdots&		&		\vdots\\
1&		c&		\cdots&		a\\
\end{matrix} \right|+\left( a-c \right) D_{n-1}=c\left| \begin{matrix}
1&		0&		\cdots&		0\\
1&		a-b&		\cdots&		0\\
\vdots&		\vdots&		&		\vdots\\
1&		c-b&		\cdots&		a-b\\
\end{matrix} \right|+\left( a-c \right) D_{n-1}
\\
&=c\left( a-b \right) ^{n-1}++\left( a-c \right) D_{n-1}.\left( n\ge 2 \right) 
\end{align*}
若$b\ne c$,则联立上面两式可得
\begin{align*}
\left| \boldsymbol{A} \right|=D_n=\frac{b\left( a-c \right) ^n-c\left( a-b \right) ^n}{b-c}.
\nonumber
\end{align*}
若$b=c$,则由上面式子递推可得
\begin{align*}
\left| \boldsymbol{A} \right|&=D_n=b\left( a-b \right) ^{n-1}+\left( a-b \right) D_{n-1}
\\
&=b\left( a-b \right) ^{n-1}+\left( a-b \right) \left[ b\left( a-b \right) ^{n-2}+\left( a-b \right) D_{n-2} \right] 
=2b\left( a-b \right) ^{n-1}+\left( a-b \right) ^2D_{n-2}
\\
&=\cdots =\left( n-1 \right) b\left( a-b \right) ^{n-1}+\left( a-b \right) ^{n-1}D_1
\\
&=\left( n-1 \right) b\left( a-b \right) ^{n-1}+\left( a-b \right) ^{n-1}a
\\
&=\left[ a+\left( n-1 \right) b \right] \left( a-b \right) ^{n-1}.
\end{align*}

当$b=c$时,也可以由\refpro{大/小拆分法例题1}可知
\begin{align*}
|\boldsymbol{A}|=\left[ a+\left( n-1 \right) b \right] \left( a-b \right) ^{n-1}.
\nonumber
\end{align*}
\end{solution}

\begin{example}
设\(f_1(x), f_2(x), \cdots, f_n(x)\)是次数不超过\(n - 2\)的多项式,求证:对任意\(n\)个数\(a_1, a_2, \cdots, a_n\),均有
\begin{align*}
\begin{vmatrix}
f_1(a_1) & f_2(a_1) & \cdots & f_n(a_1) \\
f_1(a_2) & f_2(a_2) & \cdots & f_n(a_2) \\
\vdots & \vdots & \ddots & \vdots \\
f_1(a_n) & f_2(a_n) & \cdots & f_n(a_n)
\end{vmatrix} = 0.
\end{align*}
\end{example}
\begin{proof}
{\color{blue}\text{证法一(\hyperlink{大拆分法}{大拆分法}):}}
因为\(f_k(x)(1 \leqslant  k \leqslant  n)\)的次数不超过\(n - 2\),所以它们都是单项式\(1,x,\cdots,x^{n - 2}\)的线性组合.将原行列式中每一列的多项式都按这\(n - 1\)个单项式进行拆分,最后得到至多$(n-1)!$个简单行列式之和,这些行列式中每一列的多项式只是单项式.由于每个简单行列式都有\(n\)列,根据抽屉原理,每个简单行列式中至少有两列是共用同一个单项式(可能相差一个常系数),于是这两列成比例,从而所有这样的简单行列式都等于零,因此原行列式也等于零.

{\color{blue}\text{证法二(\hyperlink{多项式根的有限性}{多项式根的有限性}):}}
令$f\left( x \right) =\left| \begin{matrix}
f_1(x)&		f_2(a_1)&		\cdots&		f_n(a_1)\\
f_1(x)&		f_2(a_2)&		\cdots&		f_n(a_2)\\
\vdots&		\vdots&		\ddots&		\vdots\\
f_1(x)&		f_2(a_n)&		\cdots&		f_n(a_n)\\
\end{matrix} \right|$,则将$f(x)$按第一列展开得到
\begin{align*}
f\left( x \right) =k_1f_1\left( x \right) +k_2f_2\left( x \right) +\cdots +k_nf_n\left( x \right) .
\end{align*}
其中$k_i$为行列式$f\left( x \right)$的第$\left( i,1 \right)$元素的代数余子式,$i=1,2,\cdots ,n$.

注意$k_i$与$x$无关,均为常数.若$f(x)$不恒为0,则又因为\(f_k(x)(1 \leqslant  k \leqslant  n)\)的次数不超过\(n - 2\),所以$degf(x)\le n-2$.
但是,注意到$f(a_2)=f(a_3)=\cdots=f(a_n)=0$,即$f(x)$有$n-1$个根.于是由\hyperlink{余数定理}{余数定理}可知,$\left( x-a_2 \right) \cdots \left( x-a_n \right) |f\left( x \right)$.从而$n-1=deg\left( x-a_2 \right) \cdots \left( x-a_n \right) \ge degf\left( x \right)$.这与$degf(x)\le n-2$矛盾.故$f(x)\equiv 0$,当然也有$f(a_1)=0$.

{\color{blue}证法三:}

设多项式
\[
f_k(x)=c_{k,n - 2}x^{n - 2}+\cdots+c_{k1}x + c_{k0},1\leqslant  k\leqslant  n.
\]
则有如下的矩阵分解:
\[
\begin{pmatrix}
f_1(a_1) & f_2(a_1) & \cdots & f_n(a_1)\\
f_1(a_2) & f_2(a_2) & \cdots & f_n(a_2)\\
\vdots & \vdots & & \vdots\\
f_1(a_n) & f_2(a_n) & \cdots & f_n(a_n)
\end{pmatrix}
=
\begin{pmatrix}
1 & a_1 & \cdots & a_1^{n - 2}\\
1 & a_2 & \cdots & a_2^{n - 2}\\
\vdots & \vdots & & \vdots\\
1 & a_n & \cdots & a_n^{n - 2}
\end{pmatrix}
\begin{pmatrix}
c_{10} & c_{20} & \cdots & c_{n0}\\
c_{11} & c_{21} & \cdots & c_{n1}\\
\vdots & \vdots & & \vdots\\
c_{1,n - 2} & c_{2,n - 2} & \cdots & c_{n,n - 2}
\end{pmatrix}.
\]
注意到上式右边的两个矩阵分别是\(n\times(n - 1)\)和\((n - 1)\times n\)矩阵,故由\hyperref[theorem:Cauchy-Binet公式]{Cauchy - Binet公式}马上得到左边矩阵的行列式值等于零.
\end{proof}

\begin{example}
求下列$n$阶行列式的值:
\begin{align*}
D_n = 
\begin{vmatrix}
1 + a_1^2 & a_1 a_2 & \cdots & a_1 a_n \\
a_2 a_1 & 1 + a_2^2 & \cdots & a_2 a_n \\
\vdots & \vdots & \ddots & \vdots \\
a_n a_1 & a_n a_2 & \cdots & 1 + a_n^2
\end{vmatrix}
\end{align*}
\end{example}
\begin{note}
本题行列式每行或每列求和后得到的结果不具备明显的规律性,故不适合使用\hyperref[行列式计算:求和法]{求和法}.

本题行列式难以找到合适的$t$对其进行\hyperref[大拆分法]{大拆分},故也不适合使用大拆分法.(并且因为难以找到合适的$t_i$,所以\hyperref[大拆分法的推广]{推广的大拆分}也不行)
\end{note}
\begin{solution}
(\hyperlink{小拆分法}{小拆分法})
将$D_n$最后一列拆成两列得
\begin{align*}
D_n&=\left| \begin{matrix}
1+a_{1}^{2}&		a_1a_2&		\cdots&		a_1a_n\\
a_2a_1&		1+a_{2}^{2}&		\cdots&		a_2a_n\\
\vdots&		\vdots&		\ddots&		\vdots\\
a_na_1&		a_na_2&		\cdots&		1+a_{n}^{2}\\
\end{matrix} \right|=\left| \begin{matrix}
1+a_{1}^{2}&		a_1a_2&		\cdots&		a_1a_n\\
a_2a_1&		1+a_{2}^{2}&		\cdots&		a_2a_n\\
\vdots&		\vdots&		\ddots&		\vdots\\
a_na_1&		a_na_2&		\cdots&		a_{n}^{2}\\
\end{matrix} \right|+\left| \begin{matrix}
1+a_{1}^{2}&		a_1a_2&		\cdots&		0\\
a_2a_1&		1+a_{2}^{2}&		\cdots&		0\\
\vdots&		\vdots&		\ddots&		\vdots\\
a_na_1&		a_na_2&		\cdots&		1\\
\end{matrix} \right|
\\
&=\left| \begin{matrix}
1+a_{1}^{2}&		a_1a_2&		\cdots&		a_1a_n\\
a_2a_1&		1+a_{2}^{2}&		\cdots&		a_2a_n\\
\vdots&		\vdots&		\ddots&		\vdots\\
a_na_1&		a_na_2&		\cdots&		a_{n}^{2}\\
\end{matrix} \right|+D_{n-1}.
\end{align*}
若$a_n\ne0$,则由上式可得
\begin{align*}
D_n=a_n\left| \begin{matrix}
1+a_{1}^{2}&		a_1a_2&		\cdots&		a_1\\
a_2a_1&		1+a_{2}^{2}&		\cdots&		a_2\\
\vdots&		\vdots&		\ddots&		\vdots\\
a_na_1&		a_na_2&		\cdots&		a_n\\
\end{matrix} \right|+D_{n-1}\xlongequal[i=1,2,\cdots ,n]{\text{对第一个行列式}:-a_ij_n+j_i}a_n\left| \begin{matrix}
1&		0&		\cdots&		a_1\\
0&		1&		\cdots&		a_2\\
\vdots&		\vdots&		\ddots&		\vdots\\
0&		0&		\cdots&		a_n\\
\end{matrix} \right|+D_{n-1}=a_{n}^{2}+D_{n-1}.\left( n\ge 2 \right) 
\end{align*}
若$a_n=0$,则上面第一个行列式等于0,进而$D_n=D_{n-1}(n\ge0)$.仍然满足上述递推式.

从而由上式递推可得
\begin{align*}
D_n=a_{n}^{2}+D_{n-1}=a_{n}^{2}+\left( a_{n-1}^{2}+D_{n-2} \right) =\cdots =\sum_{i=2}^n{a_{i}^{2}}+D_1=1+\sum_{i=1}^n{a_{i}^{2}}.
\end{align*}
\end{solution}








\end{document}
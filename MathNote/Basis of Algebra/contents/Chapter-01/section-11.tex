\documentclass[../../main.tex]{subfiles}
\graphicspath{{\subfix{../../image/}}} % 指定图片目录,后续可以直接使用图片文件名。

% 例如:
% \begin{figure}[H]
% \centering
% \includegraphics[scale=0.4]{图.png}
% \caption{}
% \label{figure:图}
% \end{figure}
% 注意:上述\label{}一定要放在\caption{}之后,否则引用图片序号会只会显示??.

\begin{document}

\section{行列式的组合定义}

\begin{example}
若\(n\)阶行列式\(\vert A\vert\)中零元素的个数超过\(n^2 - n\)个,证明:\(\vert A\vert = 0\).
\end{example}
\begin{proof}
由行列式的组合定义可得
\[
|A|=\sum_{1\leqslant  k_1k_2\cdots k_n\leqslant  n}(-1)^{\tau (k_1,k_2,\cdots,k_n)}a_{k_{11}}a_{k_{22}}\cdots a_{k_{nn}}
\]
由于\(|A|\)中零元素的个数超过\(n^2 - n\)个,故\(a_{k_{11}},a_{k_{22}},\cdots,a_{k_{nn}}\)中至少有一个为零,从而\(a_{k_{11}}a_{k_{22}}\cdots a_{k_{nn}} = 0\),因此\(|A| = 0\).如直接利用行列式的性质,也可以这样来证明:因为\(|A|\)中零元素的个数超过\(n^2 - n\)个,由抽屉原理可知,\(|A|\)至少有一列其零元素的个数大于等于\(\left\lfloor\frac{n^2 - n}{n}\right\rfloor+ 1=n\),即\(|A|\)至少有一列其元素全为零,因此\(|A| = 0\).

\end{proof}

\begin{example}
设\(A=(a_{ij})\)是\(n(n\geqslant 2)\)阶非异整数方阵,满足对任意的\(i,j\),\(\vert A\vert\)均可整除\(a_{ij}\),证明:\(\vert A\vert=\pm1\).
\end{example}
\begin{solution}
\(\vert A\vert\)可整除每个元素\(a_{i,j}\),故由行列式的组合定义
\[
\sum_{1\le k_1,k_2,\cdots ,k_n\le n}{\left( -1 \right) ^{\tau \left( k_1k_2\cdots k_n \right)}a_{k_{11}}a_{k_{22}}\cdots a_{k_{nn}}}
\]
可知\(\vert A\vert^n\)可整除\(\vert A\vert\)中每个单项\(a_{k_{11}}a_{k_{22}}\cdots a_{k_{nn}}\),从而\(\vert A\vert^n\)可整除\(\vert A\vert\),即有\(\vert A\vert^{n - 1}\)可整除\(1\),于是\(\vert A\vert^{n - 1}=\pm1\).又由行列式的组合定义可知\(\vert A\vert\)是整数,从而只能是\(\vert A\vert=\pm1\).

\end{solution}

\begin{proposition}[奇数阶反对称行列式的值等于零]\label{proposition:奇数阶反对称行列式的值等于零}
如果\(n\)阶行列式\(\vert A\vert\)的元素满足\(a_{ij}=-a_{ji}(1\leqslant  i,j\leqslant  n)\),则称为反对称行列式.求证:奇数阶反对称行列式的值等于零.
\end{proposition}
\begin{note}
{\color{blue}证法二}的想法是将行列式按组合的定义写成(n-1)!个单项的和.然后将其两两分组再求和(因为一共有(n-1)!个单项,即和式中共有偶数个单项,所以只要使用合适的分组方式就一定能够将其两两分组再求和),最后发现每组的和均为0.

构造的这个映射$\varphi$的目的是为了更加准确、严谨地说明分组的方式.证明这个映射$\varphi$是一个双射是为了保证原来的和式中的每一个单项都能与和式中另一个单项一一对应.\CJKunderline*{然后利用反证法证明了这两个一一对应的单项一定互不相同}\textbf{(注:我认为这步有些多余.这里应该只需要说明这两个一一对应的单项是原和式中不同的单项即可,即这两个单项的角标不完全相同就行,其实,这个在我们定义映射$\varphi$的时候就已经满足了.满足这个条件就足以说明原和式可以按照这种方式进行分组.并且利用反对称行列式的性质也能够证明这两个单项不仅互不相同,还能进一步得到这两个单项互为相反数)}.于是我们就可以将原和式中的每一个单项与其在双射$\varphi$作用下的像看成一组,按照这种方式就可以将原和式进行分组再求和.
\end{note}
\begin{proof}
{\color{blue}证法一(行列式的性质):}
由反对称行列式的定义可知,\(\vert A\vert\)的转置\(\vert A^{\prime}\vert\)与\(\vert A\vert\)的每个元素都相差一个符号,将\(\vert A^{\prime}\vert\)的每一行都提出公因子\(-1\)可得\(\vert A\vert=\vert A^{\prime}\vert=(-1)^{n}\vert A\vert=-\vert A\vert\),从而\(\vert A\vert = 0\).

{\color{blue}证法二(行列式的组合定义):}
由于\(\vert A\vert\)的主对角元全为0,故由组合定义,只需考虑下列单项:
\[
T = \{a_{k_11}a_{k_22}\cdots a_{k_{nn}} \mid k_i\neq i(1\leqslant  i\leqslant  n)\}
\]
定义映射\(\varphi:T\to T\),\(a_{k_11}a_{k_22}\cdots a_{k_{nn}}\mapsto a_{1k_1}a_{2k_2}\cdots a_{nk_n}\).显然\(\varphi^2 = \text{Id}_T\),于是\(\varphi\)是一个双射.我们断言:\(a_{k_11}a_{k_22}\cdots a_{k_{nn}}\)和\(a_{1k_1}a_{2k_2}\cdots a_{nk_n}\)作为\(\vert A\vert\)的单项不相同,否则\(\{1,2,\cdots,n\}\)必可分成若干对\((i_1,j_1),\cdots,(i_t,j_t)\),使得$a_{k_11}$$a_{k_22}$$\cdots $$a_{k_{nn}}$ $=$ $a_{i_1j_1}$$a_{j_1i_1}$$\cdots$$ a_{i_tj_t}$$a_{j_ti_t}$,这与\(n\)为奇数矛盾.将上述两个单项看成一组,则它们在\(\vert A\vert\)中符号均为\((-1)^{\tau(k_1k_2\cdots k_n)}\).由于\(\vert A\vert\)反对称,故
\[
a_{1k_1}a_{2k_2}\cdots a_{nk_n}=(-1)^n a_{k_11}a_{k_22}\cdots a_{k_{nn}}=-a_{k_11}a_{k_22}\cdots a_{k_{nn}}
\]
从而每组和为0,于是\(\vert A\vert = 0\).

\end{proof}

\begin{proposition}[直接计算两个矩阵和的行列式]\label{proposition:直接计算两个矩阵和的行列式}
设\(A,B\)都是\(n\)阶矩阵,求证:
\begin{align*}
|\boldsymbol{A}+\boldsymbol{B}|=|\boldsymbol{A}|+|\boldsymbol{B}|+\sum_{1\le k\le n-1}{\left( \sum_{\substack{1\le i_1<i_2<\cdots <i_k\le n\\1\le j_1<j_2<\cdots <j_k\le n}}{\boldsymbol{A}\left( \begin{matrix}
i_1&		i_2&		\cdots&		i_k\\
j_1&		j_2&		\cdots&		j_k\\
\end{matrix} \right) \widehat{\boldsymbol{B}}\left( \begin{matrix}
i_1&		i_2&		\cdots&		i_k\\
j_1&		j_2&		\cdots&		j_k\\
\end{matrix} \right)} \right)}.
\end{align*}
其中$\widehat{\boldsymbol{B}}\left( \begin{matrix}
i_1&		i_2&		\cdots&		i_k\\
j_1&		j_2&		\cdots&		j_k\\
\end{matrix} \right)$是$|\boldsymbol{B}|$的$k$阶子式$\boldsymbol{B}\left( \begin{matrix}
i_1&		i_2&		\cdots&		i_k\\
j_1&		j_2&		\cdots&		j_k\\
\end{matrix} \right)$的代数余子式.
\end{proposition}
\begin{note}
当\(\boldsymbol{A}\),\(\boldsymbol{B}\)之一是比较简单的矩阵(例如对角矩阵或秩较小的矩阵)时,可利用这个命题计算$|\boldsymbol{A}+\boldsymbol{B}|$.
\end{note}
\begin{proof}
设\(|\boldsymbol{A}| = |\alpha_1,\alpha_2,\cdots,\alpha_n|\),\(|\boldsymbol{B}| = |\beta_1,\beta_2,\cdots,\beta_n|\),其中\(\alpha_j,\beta_j\)(\(j = 1,2,\cdots,n\))分别是\(\boldsymbol{A}\)和\(\boldsymbol{B}\)的列向量.注意到
\begin{align*}
|\boldsymbol{A} + \boldsymbol{B}| = |\alpha_1 + \beta_1,\alpha_2 + \beta_2,\cdots,\alpha_n + \beta_n|.
\end{align*}
对\(|\boldsymbol{A} + \boldsymbol{B}|\),按列用行列式的性质展开,使每个行列式的每一列或者只含有\(\alpha_j\),或者只含有\(\beta_j\)(即利用大拆分法按列向量将行列式完全拆分开),
则\(|\boldsymbol{A} + \boldsymbol{B}|\)可以表示为\(2^n\)个这样的行列式之和.即(并且单独把\(k = 0,n\)的项分离出来,即将\(|\boldsymbol{A}|\)、\(|\boldsymbol{B}|\)分离出来)
\begin{align*}
&|\boldsymbol{A} + \boldsymbol{B}| = |\alpha_1 + \beta_1,\alpha_2 + \beta_2,\cdots,\alpha_n + \beta_n| 
\\
&=|\boldsymbol{A}|+|\boldsymbol{B}|+\sum_{1\leqslant k\leqslant n-1}{\sum_{1\leqslant  j_1\leqslant  j_2\leqslant  \cdots\leqslant  j_k\leqslant  n}{\begin{array}{c}
\begin{array}{c@{}c@{}c@{}c@{}c@{}c@{}c@{}c@{}c@{}c@{}c@{}}
& 1 & \cdots & j_1 &\cdots &j_2 &\cdots &j_k &\cdots &n \\
\left.\right|
&\beta _1,&\cdots ,&\alpha _{j_1},&\cdots ,&\alpha_{j_2},&\cdots ,&\alpha_{j_k},&\cdots ,&\beta_n& \left|\right.
\end{array}\\
\\
\end{array}}}.
\end{align*}
再对上式右边除\(|\boldsymbol{A}|\)、\(|\boldsymbol{B}|\)外的每个行列式用Laplace定理按含有\(\boldsymbol{A}\)的列向量的那些列展开得到
\begin{align*}
&|\boldsymbol{A} + \boldsymbol{B}| =|\boldsymbol{A}|+|\boldsymbol{B}|+\sum_{1\leqslant k\leqslant n-1}{\sum_{1\leqslant  j_1\leqslant  j_2\leqslant  \cdots\leqslant  j_k\leqslant  n}{\begin{array}{c}
\begin{array}{c@{}c@{}c@{}c@{}c@{}c@{}c@{}c@{}c@{}c@{}c@{}}
& 1 & \cdots & j_1 &\cdots &j_2 &\cdots &j_k &\cdots &n \\
\left.\right|
&\beta _1,&\cdots ,&\alpha _{j_1},&\cdots ,&\alpha_{j_2},&\cdots ,&\alpha_{j_k},&\cdots ,&\beta_n& \left|\right.
\end{array}\\
\\
\end{array}}}
\\
&= |\boldsymbol{A}| + |\boldsymbol{B}| + \sum_{1\leqslant k\leqslant n - 1}\sum_{1\leqslant j_1,j_2,\cdots,j_k\leqslant n}\sum_{1\leqslant i_1,i_2,\cdots,i_k\leqslant n}\boldsymbol{A}\left(\begin{matrix}
i_1 & i_2 & \cdots & i_k\\
j_1 & j_2 & \cdots & j_k
\end{matrix}\right)\widehat{\boldsymbol{B}}\left(\begin{matrix}
i_1 & i_2 & \cdots & i_k\\
j_1 & j_2 & \cdots & j_k
\end{matrix}\right)
\\
&=|\boldsymbol{A}|+|\boldsymbol{B}|+\sum_{1\le k\le n-1}{\left( \sum_{\substack{1\le i_1<i_2<\cdots <i_k\le n\\1\le j_1<j_2<\cdots <j_k\le n}}{\boldsymbol{A}\left( \begin{matrix}
i_1&		i_2&		\cdots&		i_k\\
j_1&		j_2&		\cdots&		j_k\\
\end{matrix} \right) \widehat{\boldsymbol{B}}\left( \begin{matrix}
i_1&		i_2&		\cdots&		i_k\\
j_1&		j_2&		\cdots&		j_k\\
\end{matrix} \right)} \right)}.
\end{align*}

\end{proof}

\begin{example}\label{example:特征行列式写成多项式形式的系数}
设
\[
f(x)=\left| \begin{matrix}
x-a_{11}&		-a_{12}&		\cdots&		-a_{1n}\\
-a_{21}&		x-a_{22}&		\cdots&		-a_{2n}\\
\vdots&		\vdots&		\ddots&		\vdots\\
-a_{n1}&		-a_{n2}&		\cdots&		x-a_{nn}\\
\end{matrix} \right|,
\]
其中\(x\)是未定元,\(a_{ij}\)是常数.证明:\(f(x)\)是一个最高次项系数为\(1\)的\(n\)次多项式,且其\(n - 1\)次项的系数等于\(-(a_{11}+a_{22}+\cdots + a_{nn})\).
\end{example}
\begin{note}
注意$f(x)$的每行每列除主对角元素外,其他元素均不相同.因此$f(x)$并不是\hyperref["爪"型行列式的推广]{推广的"爪"型行列式}.
\end{note}
\begin{solution}
由行列式的组合定义可知,\(f(x)\)的最高次项出现在组合定义展开式中的单项\((x - a_{11})(x - a_{22})\cdots(x - a_{nn})\)中,且展开式中的其他单项作为\(x\)的多项式其次数小于等于\(n - 2\).因此\(f(x)\)是一个最高次项系数为\(1\)的\(n\)次多项式,且其\(n - 1\)次项的系数等于\(-(a_{11}+a_{22}+\cdots + a_{nn})\).

\end{solution}
\begin{remark}
将这个例题进行推广再结合\hyperref[proposition:直接计算两个矩阵和的行列式]{直接计算两个矩阵和的行列式的结论}可以得到下述推论.
\end{remark}

\begin{corollary}\label{corollary:特征多项式系数与矩阵子式的关系}
设\(A=(a_{ij})\)为\(n\)阶方阵,\(\,x\)为未定元,
\[
f(x)=\vert xI_n - A\vert = 
\begin{vmatrix}
x - a_{11} & -a_{12} & \cdots & -a_{1n} \\
-a_{21} & x - a_{22} & \cdots & -a_{2n} \\
\vdots & \vdots & \ddots & \vdots \\
-a_{n1} & -a_{n2} & \cdots & x - a_{nn}
\end{vmatrix}
\]

证明:\(f(x)=x^n + a_1x^{n - 1}+ \cdots + a_{n - 1}x + a_n\),其中
\[
a_k=(-1)^k \sum_{1\leqslant  i_1 < i_2<\cdots <i_k\leqslant  n} A
\begin{pmatrix}
i_1 & i_2 & \cdots & i_k \\
i_1 & i_2 & \cdots & i_k
\end{pmatrix}, 1\leqslant  k\leqslant  n.
\]
\end{corollary}
\begin{note}
需要注意上述推论中$a_1=-(a_{11}+a_{22}+\cdots+a_{nn}),a_n=\left( -1 \right) ^n\left| \boldsymbol{A} \right|.$
\end{note}
\begin{proof}
注意到 \(xI_{n}\) 非零的 \(n - k\) 阶子式只有 \(n - k\) 阶主子式,并且其值为 \(x^{n - k}\),其余$n-k$阶子式均为零.
记$\widehat{x\boldsymbol{I}_n}\left( \begin{matrix}
i_1&		i_2&		\cdots&		i_k\\
j_1&		j_2&		\cdots&		j_k\\
\end{matrix} \right)$是$x\boldsymbol{I}_n\left( \begin{matrix}
i_1&		i_2&		\cdots&		i_k\\
j_1&		j_2&		\cdots&		j_k\\
\end{matrix} \right)$的代数余子式,则$\widehat{x\boldsymbol{I}_n}\left( \begin{matrix}
i_1&		i_2&		\cdots&		i_k\\
j_1&		j_2&		\cdots&		j_k\\
\end{matrix} \right)$是\(xI_{n}\)非零的 \(n - k\) 阶子式.于是我们有
\begin{align*}
\widehat{x\boldsymbol{I}_n}\left( \begin{matrix}
i_1&		i_2&		\cdots&		i_k\\
j_1&		j_2&		\cdots&		j_k\\
\end{matrix} \right) =x^{n-k}.
\end{align*}
再利用\hyperref[proposition:直接计算两个矩阵和的行列式]{直接计算两个矩阵和的行列式的结论}就可以得到
\begin{align*}
&f(x)=|x\boldsymbol{I}_n-\boldsymbol{A}|=\left| \begin{matrix}
x-a_{11}&		-a_{12}&		\cdots&		-a_{1n}\\
-a_{21}&		x-a_{22}&		\cdots&		-a_{2n}\\
\vdots&		\vdots&		&		\vdots\\
-a_{n1}&		-a_{n2}&		\cdots&		x-a_{nn}\\
\end{matrix} \right|=\left| \left( \begin{matrix}
-a_{11}&		-a_{12}&		\cdots&		-a_{1n}\\
-a_{21}&		-a_{22}&		\cdots&		-a_{2n}\\
\vdots&		\vdots&		&		\vdots\\
-a_{n1}&		-a_{n2}&		\cdots&		-a_{nn}\\
\end{matrix} \right) +\left( \begin{matrix}
x&		0&		\cdots&		0\\
0&		x&		\cdots&		0\\
\vdots&		\vdots&		&		\vdots\\
0&		0&		\cdots&		x\\
\end{matrix} \right) \right|
\\
&=\left| \begin{matrix}
-a_{11}&		-a_{12}&		\cdots&		-a_{1n}\\
-a_{21}&		-a_{22}&		\cdots&		-a_{2n}\\
\vdots&		\vdots&		&		\vdots\\
-a_{n1}&		-a_{n2}&		\cdots&		-a_{nn}\\
\end{matrix} \right|+\left| \begin{matrix}
x&		0&		\cdots&		0\\
0&		x&		\cdots&		0\\
\vdots&		\vdots&		&		\vdots\\
0&		0&		\cdots&		x\\
\end{matrix} \right|+\sum_{1\le k\le n-1}{\sum_{\substack{1\le i_1,i_2,\cdots ,i_k\le n\\
1\le j_1,j_2,\cdots ,j_k\le n\\}
}{\left( -\boldsymbol{A} \right) \left( \begin{matrix}
i_1&		i_2&		\cdots&		i_k\\
j_1&		j_2&		\cdots&		j_k\\
\end{matrix} \right) \widehat{x\boldsymbol{I}_n}\left( \begin{matrix}
i_1&		i_2&		\cdots&		i_k\\
j_1&		j_2&		\cdots&		j_k\\
\end{matrix} \right)}}
\\
&=\left( -1 \right) ^n\left| \boldsymbol{A} \right|+x^n+\sum_{1\le k\le n-1}{\sum_{1\le i_1,i_2,\cdots ,i_k\le n}{\left( -1 \right) ^k\boldsymbol{A}\left( \begin{matrix}
i_1&		i_2&		\cdots&		i_k\\
i_1&		i_2&		\cdots&		i_k\\
\end{matrix} \right) \widehat{x\boldsymbol{I}_n}\left( \begin{matrix}
i_1&		i_2&		\cdots&		i_k\\
i_1&		i_2&		\cdots&		i_k\\
\end{matrix} \right)}}
\\
&=x^n+\sum_{1\le k\le n-1}{\left( -1 \right) ^k\sum_{1\le i_1,i_2,\cdots ,i_k\le n}{\boldsymbol{A}\left( \begin{matrix}
i_1&		i_2&		\cdots&		i_k\\
i_1&		i_2&		\cdots&		i_k\\
\end{matrix} \right) \cdot x^{n-k}}}+\left( -1 \right) ^n\left| \boldsymbol{A} \right|
\\
&=x^n+\sum_{1\le k\le n-1}{x^{n-k}\left( -1 \right) ^k\sum_{1\le i_1,i_2,\cdots ,i_k\le n}{\boldsymbol{A}\left( \begin{matrix}
i_1&		i_2&		\cdots&		i_k\\
i_1&		i_2&		\cdots&		i_k\\
\end{matrix} \right)}}+\left( -1 \right) ^n\left| \boldsymbol{A} \right|.
\end{align*}
因此
\(f(x)=x^n + a_1x^{n - 1}+ \cdots + a_{n - 1}x + a_n\),其中
\[
a_k=(-1)^k \sum_{1\leqslant  i_1 < i_2<\cdots <i_k\leqslant  n} A
\begin{pmatrix}
i_1 & i_2 & \cdots & i_k \\
i_1 & i_2 & \cdots & i_k
\end{pmatrix}, 1\leqslant  k\leqslant  n.
\]

\end{proof}


\end{document}
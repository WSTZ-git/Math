\documentclass[../../main.tex]{subfiles}
\graphicspath{{\subfix{../../image/}}} % 指定图片目录,后续可以直接使用图片文件名。

% 例如:
% \begin{figure}[H]
% \centering
% \includegraphics[scale=0.4]{图.png}
% \caption{}
% \label{figure:图}
% \end{figure}
% 注意:上述\label{}一定要放在\caption{}之后,否则引用图片序号会只会显示??.

\begin{document}

\section{其他}

\begin{theorem}\label{theorem:中国剩余定理推广(模不互质的情况)}
设$\mathbb{F}$是一个域,$m_i,a_i\in \mathbb{F}[x],i=1,2,\cdots,n$,且$\mathrm{gcd}\left( m_i,m_j \right) |\left( a_i-a_j \right) ,i,j\in \left\{ 1,2,\cdots ,n \right\}$,则存在$f\in\mathbb{F}[x]$使得
$$
f(x)\equiv a_i(x)\pmod{m_i(x)},i=1,2,\cdots,n,
$$
即存在$k_i\in\mathbb{F}[x]$使得
$$
f(x)=k_i(x)m_i(x)+a_i(x),i=1,2,\cdots,n.
$$
\end{theorem}
\begin{proof}

\end{proof}

\begin{example}
设 \( \boldsymbol{A}_1, \boldsymbol{A}_2, \cdots, \boldsymbol{A}_m \in M_n(\mathbb{K}) \),\( g(x) \in \mathbb{K}[x] \),使得 \( g(\boldsymbol{A}_1), g(\boldsymbol{A}_2), \cdots, g(\boldsymbol{A}_m) \) 都是非异阵. 试用两种方法证明: 存在 \( h(x) \in \mathbb{K}[x] \),使得 \( g(\boldsymbol{A}_i)^{-1} = h(\boldsymbol{A}_i) \) 对所有的 \( 1 \leq i \leq m \) 都成立.
\end{example}
\begin{proof}
{\color{blue}证法一:}由\refpro{proposition:矩阵的逆可以用其多项式表示}可知,对\(\forall i\in \{1,2,\cdots ,m\}\),存在\(h_i\in \mathbb{K}[x]\),使得
\begin{align}
g^{-1}(A_i) = h_i(A_i). \label{eq:107.105}
\end{align}
记\(A_i\)的极小多项式为\(n_i(x)\),\(i=1,2,\cdots,m\)。考虑\(\gcd(n_i,n_j)\)(\(i,j\in \{1,2,\cdots ,m\}\)),设\(x_0\in \mathbb{C}\)是\(\gcd(n_i,n_j)\)的根,则\((x-x_0)|n_i,n_j\),即\(x_0\)是\(A_i\)和\(A_j\)的公共特征值。
由\refpro{proposition:矩阵多项式的特征值就是原特征值代入多项式得到的数}和\refpro{proposition:逆矩阵的特征值}可知,\(h_i(x_0)\)是\(h_i(A_i)\)的特征值,\(\frac{1}{g(x_0)}\)是\(g^{-1}(A_i)\)的特征值。再由\eqref{eq:107.105}式可知
\begin{align*}
g^{-1}(A_i) = h_i(A_i) \Longrightarrow \frac{1}{g(x_0)} = h_i(x_0),\quad i=1,2,\cdots,m.
\end{align*}
于是
\begin{align*}
h_i(x_0) - h_j(x_0) = \frac{1}{g(x_0)} - \frac{1}{g(x_0)} = 0.
\end{align*}
因此\((x-x_0)|(h_i - h_j)\)。故在\(\mathbb{C}\)上就有\(\gcd(n_i,n_j)|(h_i - h_j)\)。又因为整除不随数域扩张而改变,所以在\(\mathbb{K}\)上也有\(\gcd(n_i,n_j)|(h_i(x) - h_j(x))\)。于是由\hyperref[theorem:中国剩余定理推广(模不互质的情况)]{中国剩余定理的推广}可知,方程
\begin{align*}
h(x) \equiv h_i(x) \pmod{n_i(x)},\quad i=1,2,\cdots,m
\end{align*}
在\(\mathbb{K}\)上有解。故存在\(h\in \mathbb{K}[x]\),使得
\begin{align*}
h(A_i) = h_i(A_i) = g^{-1}(A_i),\quad i=1,2,\cdots,m.
\end{align*}

{\color{blue}证法二:}记\(A_i\)的极小多项式为\(n_i(x)\),\(i=1,2,\cdots,m\),由\refpro{proposition:g(A)可逆与A的特征多项式与极小多项式的关系}可知
\begin{align*}
(n_i,g)=1,i=1,2,\cdots,m.
\end{align*}
从而$\left( n_1n_2\cdots n_m,g \right) =1$.因此存在$h,k\in \mathbb{F}[x]$,使得
\begin{align*}
h\left( x \right) g\left( x \right) +n_1\left( x \right) n_2\left( x \right) \cdots n_m\left( x \right) k\left( x \right) =1.
\end{align*}
从而
\begin{align*}
h\left( A_i \right) =g^{-1}\left( A_i \right) ,i=1,2,\cdots ,m.
\end{align*}
\end{proof}

\begin{proposition}\label{proposition:与主对角元都为0的上三角阵相似的矩阵一定是幂零阵}
设$n\in \mathbb{N}$且$A\in \mathbb{C}^{n\times n}$,若$A\sim \widetilde{A}$,其中$\widetilde{A}$是主对角元都为0的上三角阵,则$A$是幂零矩阵.
\end{proposition}
\begin{proof}
由条件可知存在可逆阵$P$,使得$A=P^{-1}\widetilde{A}P$.从而根据矩阵乘法易得
\begin{align*}
A^n=P^{-1}\widetilde{A}^nP=O.
\end{align*}
故$A$是幂零矩阵.
\end{proof}

\begin{example}
设 $n \in \mathbb{N}$ 且 $A, B \in \mathbb{C}^{n \times n}$ 满足
\[
AB + A = BA + B.
\]
证明:
\[
(A - B)^n = 0.
\]
\end{example}
\begin{proof}
注意到
\[
AB - BA = B - A.
\]
由\refpro{proposition:两个矩阵可同时上三角化的条件}知 $A, B$ 可同时相似上三角化. 不妨设 $A, B$ 都是上三角矩阵, 由\hyperref[proposition:上三角阵性质]{上三角阵的性质}可知$AB-BA$也是上三角阵且主对角元都为0.再由上式可知$A - B$是对角线全为0的上三角阵,故由\refpro{proposition:与主对角元都为0的上三角阵相似的矩阵一定是幂零阵}知$A-B$是幂零矩阵. 现在我们知道
\[
(A - B)^n = 0.
\]
\end{proof}

\begin{example}
设 \( A \in \mathbb{R}^{n \times n} \) 是对角矩阵. 考虑
\[
S(A) \triangleq \{ P^{-1}AP : P \in \mathbb{R}^{n \times n} \text{是可逆矩阵} \}.
\]
证明: \( S(A) \) 在 \( \mathbb{R}^{n \times n} \) 中是闭集. 反过来, 如果 \( S(A) \) 是闭集, 证明 \( A \) 在 \( \mathbb{C} \) 上一定可对角化.
\end{example}
\begin{proof}
设 \( A \) 的极小多项式为 \( m \), 特征多项式为 \( p \),则由知$m$无重根.设 \( A_k \in S(A), k = 1,2,\cdots \) 满足
\[
\lim_{k \to \infty} A_k = \tilde{A}.
\]
由\refthe{theorem:复矩阵极限的性质}知
\begin{gather*}
m(\tilde{A})=\lim_{k\rightarrow \infty} m(A_k)=0,
\\
\left| \lambda I-\tilde{A} \right|=\left| \lambda I-\lim_{k\rightarrow \infty} A_k \right|=\lim_{k\rightarrow \infty} \left| \lambda I-A_k \right|=\lim_{k\rightarrow \infty} \left| \lambda I-A \right|=\left| \lambda I-A \right|=p\left( \lambda \right) .
\end{gather*}
因此\( \tilde{A} \) 的特征多项式也是 \( p \),\( \tilde{A} \) 极小多项式整除$m$,从而\( \tilde{A} \) 极小多项式也无重根.  因此 \( \tilde{A} \) 和 \( A \) 有相同的特征值且可对角化, 故 \( \tilde{A} \in S(A) \).

反之, 如果 \( S(A) \) 是闭的,
由\hyperref[theorem:实数域上的广义Jordan标准型]{实数域上的广义Jordan标准型}知,$A$ 在实数域$\mathbb{R}$上相似于下列分块对角矩阵:
\begin{align*}
\widetilde{J} = \mathrm{diag}\{J_{r_1}(\lambda_1),\cdots,J_{r_k}(\lambda_k),\widetilde{J}_{s_1}(a_1,b_1),\cdots,\widetilde{J}_{s_l}(a_l,b_l)\},
\end{align*}
其中 $\lambda_1,\cdots,\lambda_k,a_1,b_1,\cdots,a_l,b_l$ 都是实数,$b_1,\cdots,b_l$ 都非零,且$\lambda_j$都是$A$的实特征值,$a_j\pm \mathrm{i}b_j$都是$A$的复特征值,$J_{r_i}(\lambda_i)$ 表示以 $\lambda_i$ 为特征值的通常意义下的 Jordan 块,并且
\begin{gather*}
c_j=-2a_j,d_j=a_j^2+b_j^2,
\quad
R_j = \left( \begin{matrix}
0&		-d_j\\
1&		-c_j\\
\end{matrix} \right),C_2 = \begin{pmatrix}0 & 0 \\ 1 & 0\end{pmatrix},
\\
\widetilde{J}_{s_j}(a_j,b_j) = 
\begin{pmatrix}
R_j & C_2 & & \\
& R_j & C_2 & \\
& & \ddots & \ddots \\
& & & R_j & C_2 \\
& & & & R_j
\end{pmatrix}.
\end{gather*}
对于$J_{r_j}(\lambda_j)$,在实数域上,我们有
\begin{align*}
J_{r_j}\left( \lambda _j \right) \sim \left( \begin{matrix}
\lambda _j&		\frac{1}{n}&		&		\\
&		\lambda _j&		\ddots&		\\
&		&		\ddots&		\frac{1}{n}\\
&		&		&		\lambda _j\\
\end{matrix} \right)\triangleq J_{r_j}^{(n)}(\lambda_j),\forall n\in \mathbb{N} .
\end{align*}
对于$\widetilde{J}_{s_j}(a_j,b_j)$,在实数域上,我们有
\begin{align*}
J_{s_j}\left( a_j,b_j \right) \sim {\scriptsize \left( \begin{matrix}
0&		-d_j&		&		&		&		&		&		&		\\
1&		-c_j&		\frac{1}{n}&		&		&		&		&		&		\\
&		&		0&		-d_j&		&		&		&		&		\\
&		&		1&		-c_j&		\frac{1}{n}&		&		&		&		\\
&		&		&		&		\ddots&		&		&		&		\\
&		&		&		&		&		0&		-d_j&		&		\\
&		&		&		&		&		1&		-c_j&		\frac{1}{n}&		\\
&		&		&		&		&		&		&		0&		-d_j\\
&		&		&		&		&		&		&		1&		-c_j\\
\end{matrix} \right)}\triangleq J_{s_j}^{(n)}(a_j,b_j) ,\forall n\in \mathbb{N} .
\end{align*}
于是在实数域上,就有
\begin{align*}
A\sim \widetilde{J}\sim \mathrm{diag}\{J_{r_1}^{\left( n \right)}(\lambda _1),\cdots ,J_{r_k}^{\left( n \right)}(\lambda _k),\widetilde{J}_{s_1}^{\left( n \right)}(a_1,b_1),\cdots ,\widetilde{J}_{s_l}^{\left( n \right)}(a_l,b_l)\}\triangleq \widetilde{J}^{\left( n \right)},\forall n\in \mathbb{N}.
\end{align*}
故$ \widetilde{J}^{\left( n \right)}\in S(A)$.因为$S(A)$是闭集,所以$\underset{n\rightarrow +\infty}{\lim}\widetilde{J}^{\left( n \right)}\in S\left( A \right) .$
不难发现
\begin{gather*}
\underset{n\rightarrow +\infty}{\lim}J_{r_j}^{(n)}\left( a_j,b_j \right) =\left( \begin{matrix}
\lambda _j&		&		&		\\
&		\lambda _j&		&		\\
&		&		\ddots&		\\
&		&		&		\lambda _j\\
\end{matrix} \right) ,
\\
\underset{n\rightarrow +\infty}{\lim}J_{s_j}^{(n)}\left( a_j,b_j \right) ={\scriptsize \left( \begin{matrix}
0&		-d_j&		&		&		&		&		\\
1&		-c_j&		&		&		&		&		\\
&		&		0&		-d_j&		&		&		\\
&		&		1&		-c_j&		&		&		\\
&		&		&		&		\ddots&		&		\\
&		&		&		&		&		0&		-d_j\\
&		&		&		&		&		1&		-c_j\\
\end{matrix} \right)} =\left( \begin{matrix}
R_j&		&		&		\\
&		R_j&		&		\\
&		&		\ddots&		\\
&		&		&		R_j\\
\end{matrix} \right) ,
\end{gather*}
注意到$R_j$的极小多项式等于
\begin{align*}
x^2+c_jx+d_j=(x-a_j)^2+b_j^2=\left( x-a_j-\mathrm{i}b_j \right) \left( x-a_j+\mathrm{i}b_j \right)
\end{align*}
在复数域$\mathbb{C}$上无重根,故$R_j$在复数域$\mathbb{C}$上可对角化,从而$\underset{n\rightarrow +\infty}{\lim}J_{s_j}^{(n)}\left( a_j,b_j \right)$在复数域$\mathbb{C}$上也可对角化.因此
\begin{align*}
\underset{n\rightarrow +\infty}{\lim}\widetilde{J}^{\left( n \right)}=\mathrm{diag}\{\underset{n\rightarrow +\infty}{\lim}J_{r_1}^{\left( n \right)}(\lambda _1),\cdots ,\underset{n\rightarrow +\infty}{\lim}J_{r_k}^{\left( n \right)}(\lambda _k),\underset{n\rightarrow +\infty}{\lim}\widetilde{J}_{s_1}^{\left( n \right)}(a_1,b_1),\cdots ,\underset{n\rightarrow +\infty}{\lim}\widetilde{J}_{s_l}^{\left( n \right)}(a_l,b_l)\}
\end{align*}
在复数域$\mathbb{C}$上可对角化.又$\underset{n\rightarrow +\infty}{\lim}\widetilde{J}^{\left( n \right)}\in S\left( A \right),$故$A$在复数域$\mathbb{C}$上相似于$\underset{n\rightarrow +\infty}{\lim}\widetilde{J}^{\left( n \right)}$.因此$A$在复数域$\mathbb{C}$上可对角化.
\end{proof}

\begin{example}
设 \( n \geq 2, A \in \mathbb{R}^{n \times n} \) 是实对称矩阵, \( v \in \mathbb{R}^{n \times n} \setminus \{0\} \). 证明:
\[
\text{tr}(A^TA) \geq \frac{2n - 1}{2n - 2} \cdot \frac{\|Av\|^2}{\|v\|^2} - \frac{1}{n - 1}[\text{tr}(A)]^2.
\]
\end{example}
\begin{proof}
不妨设 \( A \) 为实对角矩阵, 即
\[
A = \text{diag}\{\lambda_1, \lambda_2, \cdots, \lambda_n\}, \lambda_i \in \mathbb{R}, i = 1,2,\cdots, n.
\]
现在再设 \( v = \begin{pmatrix} v_1 \\ v_2 \\ \vdots \\ v_n \end{pmatrix} \), 我们有原不等式等价于
\begin{align*}
\sum_{i=1}^n \lambda_i^2 \geq \frac{2n - 1}{2n - 2} \cdot \frac{\sum\limits_{i=1}^n \lambda_i^2 v_i^2}{\sum\limits_{i=1}^n v_i^2} - \frac{1}{n - 1} \left( \sum_{i=1}^n \lambda_i \right)^2. 
\end{align*}
不妨设 \( \lambda_1 \) 是 \( \lambda_i, i = 1,2,\cdots, n \) 的最大值, 则
\[
\sum_{i=1}^n \lambda_i^2 - \frac{2n - 1}{2n - 2} \cdot \frac{\sum\limits_{i=1}^n \lambda_i^2 v_i^2}{\sum\limits_{i=1}^n v_i^2} + \frac{1}{n - 1} \left( \sum\limits_{i=1}^n \lambda_i \right)^2 \geqslant \sum_{i=1}^n \lambda_i^2 - \frac{2n - 1}{2n - 2} \cdot \lambda_1^2 + \frac{1}{n - 1} \left( \sum_{i=1}^n \lambda_i \right)^2.
\]
于是打开上述右边括号知原不等式等价于
\begin{align}
&\quad \quad \sum_{i=1}^n{\lambda _{i}^{2}}-\frac{2n-1}{2n-2}\cdot \lambda _{1}^{2}+\frac{1}{n-1}\left( \sum_{i=1}^n{\lambda _i} \right) ^2\geqslant 0
\nonumber \\
&\Longleftrightarrow \sum_{i=1}^n{\lambda _{i}^{2}}-\frac{2n-1}{2n-2}\cdot \lambda _{1}^{2}+\frac{1}{n-1}\left( \sum_{i=1}^n{\lambda _{i}^{2}}+2\sum_{1\le i<j\le n}{\lambda _i\lambda _j} \right) \geqslant 0
\nonumber \\
&\Longleftrightarrow \frac{1}{2n-2}\lambda _{1}^{2}+\frac{n}{n-1}\sum_{i=2}^n{\lambda _{i}^{2}}+\frac{2}{n-1}\sum_{1\le i<j\le n}{\lambda _i\lambda _j}\geqslant 0
\nonumber \\
&\Longleftrightarrow \lambda _{1}^{2}+2n\sum_{i=2}^n{\lambda _{i}^{2}}+4\sum_{1\le i<j\le n}{\lambda _i\lambda _j}\geqslant 0.\label{eq:::-----1568165435
3615-11}
\end{align}
上述关于 \( \lambda_i \) 的二次型矩阵为
\[
\begin{pmatrix} 1 & 2 & \cdots & 2 \\ 2 & 2n & \cdots & 2 \\ \vdots & \vdots & \ddots & \vdots \\ 2 & 2 & \cdots & 2n \end{pmatrix}.
\]
直接计算行列式得
\begin{align*}
&\left| \begin{matrix}
\lambda -1&		-2&		\cdots&		-2\\
-2&		\lambda -2n&		\cdots&		-2\\
\vdots&		\vdots&		\ddots&		\vdots\\
-2&		-2&		\cdots&		\lambda -2n\\
\end{matrix} \right|\xlongequal[i=2,3,\cdots ,n]{\left( -1 \right) r_1+r_i}\left| \begin{matrix}
\lambda -1&		-2&		\cdots&		-2\\
-\lambda -1&		\lambda -2n+2&		\cdots&		0\\
\vdots&		\vdots&		\ddots&		\vdots\\
-\lambda -1&		0&		\cdots&		\lambda -2n+2\\
\end{matrix} \right|
\\
&\xlongequal{\text{\hyperref["爪"型行列式]{“爪”型行列式}}}\left( \lambda -1 \right) \left( \lambda -2n+2 \right) ^{n-1}-2\left( n-1 \right) \left( \lambda +1 \right) \left( \lambda -2n+2 \right) ^{n-2}
\\
&=\left( \lambda -2n+2 \right) ^{n-2}\left[ \left( \lambda -1 \right) \left( \lambda -2n+2 \right) -2\left( n-1 \right) \left( \lambda +1 \right) \right] 
\\
&=\left( \lambda -2n+2 \right) ^{n-2}\lambda \left( \lambda -4n+3 \right) .
\end{align*}
现在矩阵特征值是
\[
0, 4n - 3, \underbrace{2n - 2, 2n - 2, \cdots, 2n - 2}_{n - 2 \text{个}}.\quad (n\geqslant 2)
\]
故矩阵的特征值全都大于等于0.于是矩阵半正定, 从而这个矩阵对应的二次型大于等于0.这就得到了不等式 \eqref{eq:::-----1568165435
3615-11}.
\end{proof}

\begin{example}

\end{example}
\begin{proof}

\end{proof}

\begin{example}

\end{example}
\begin{proof}

\end{proof}

\begin{example}

\end{example}
\begin{proof}

\end{proof}

\begin{example}

\end{example}
\begin{proof}

\end{proof}

\begin{example}

\end{example}
\begin{proof}

\end{proof}



























\end{document}
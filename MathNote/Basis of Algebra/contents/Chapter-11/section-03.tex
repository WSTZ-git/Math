\documentclass[../../main.tex]{subfiles}
\graphicspath{{\subfix{../../image/}}} % 指定图片目录,后续可以直接使用图片文件名。

% 例如:
% \begin{figure}[H]
% \centering
% \includegraphics[scale=0.4]{图.png}
% \caption{}
% \label{figure:图}
% \end{figure}
% 注意:上述\label{}一定要放在\caption{}之后,否则引用图片序号会只会显示??.

\begin{document}

\section{其他}

\begin{theorem}\label{theorem:中国剩余定理推广(模不互质的情况)}
设$\mathbb{F}$是一个域,$m_i,a_i\in \mathbb{F}[x],i=1,2,\cdots,n$,且$\mathrm{gcd}\left( m_i,m_j \right) |\left( a_i-a_j \right) ,i,j\in \left\{ 1,2,\cdots ,n \right\}$,则存在$f\in\mathbb{F}[x]$使得
$$
f(x)\equiv a_i(x)\pmod{m_i(x)},i=1,2,\cdots,n,
$$
即存在$k_i\in\mathbb{F}[x]$使得
$$
f(x)=k_i(x)m_i(x)+a_i(x),i=1,2,\cdots,n.
$$
\end{theorem}
\begin{proof}

\end{proof}

\begin{example}
设 \( \boldsymbol{A}_1, \boldsymbol{A}_2, \cdots, \boldsymbol{A}_m \in M_n(\mathbb{K}) \),\( g(x) \in \mathbb{K}[x] \),使得 \( g(\boldsymbol{A}_1), g(\boldsymbol{A}_2), \cdots, g(\boldsymbol{A}_m) \) 都是非异阵. 试用两种方法证明: 存在 \( h(x) \in \mathbb{K}[x] \),使得 \( g(\boldsymbol{A}_i)^{-1} = h(\boldsymbol{A}_i) \) 对所有的 \( 1 \leq i \leq m \) 都成立.
\end{example}
\begin{proof}
{\color{blue}证法一:}由\refpro{proposition:矩阵的逆可以用其多项式表示}可知,对\(\forall i\in \{1,2,\cdots ,m\}\),存在\(h_i\in \mathbb{K}[x]\),使得
\begin{align}
g^{-1}(A_i) = h_i(A_i). \label{eq:107.105}
\end{align}
记\(A_i\)的极小多项式为\(n_i(x)\),\(i=1,2,\cdots,m\)。考虑\(\gcd(n_i,n_j)\)(\(i,j\in \{1,2,\cdots ,m\}\)),设\(x_0\in \mathbb{C}\)是\(\gcd(n_i,n_j)\)的根,则\((x-x_0)|n_i,n_j\),即\(x_0\)是\(A_i\)和\(A_j\)的公共特征值。
由\refpro{proposition:矩阵多项式的特征值就是原特征值代入多项式得到的数}和\refpro{proposition:逆矩阵的特征值}可知,\(h_i(x_0)\)是\(h_i(A_i)\)的特征值,\(\frac{1}{g(x_0)}\)是\(g^{-1}(A_i)\)的特征值。再由\eqref{eq:107.105}式可知
\begin{align*}
g^{-1}(A_i) = h_i(A_i) \Longrightarrow \frac{1}{g(x_0)} = h_i(x_0),\quad i=1,2,\cdots,m.
\end{align*}
于是
\begin{align*}
h_i(x_0) - h_j(x_0) = \frac{1}{g(x_0)} - \frac{1}{g(x_0)} = 0.
\end{align*}
因此\((x-x_0)|(h_i - h_j)\)。故在\(\mathbb{C}\)上就有\(\gcd(n_i,n_j)|(h_i - h_j)\)。又因为整除不随数域扩张而改变,所以在\(\mathbb{K}\)上也有\(\gcd(n_i,n_j)|(h_i(x) - h_j(x))\)。于是由\hyperref[theorem:中国剩余定理推广(模不互质的情况)]{中国剩余定理的推广}可知,方程
\begin{align*}
h(x) \equiv h_i(x) \pmod{n_i(x)},\quad i=1,2,\cdots,m
\end{align*}
在\(\mathbb{K}\)上有解。故存在\(h\in \mathbb{K}[x]\),使得
\begin{align*}
h(A_i) = h_i(A_i) = g^{-1}(A_i),\quad i=1,2,\cdots,m.
\end{align*}

{\color{blue}证法二:}记\(A_i\)的极小多项式为\(n_i(x)\),\(i=1,2,\cdots,m\),由\refpro{proposition:g(A)可逆与A的特征多项式与极小多项式的关系}可知
\begin{align*}
(n_i,g)=1,i=1,2,\cdots,m.
\end{align*}
从而$\left( n_1n_2\cdots n_m,g \right) =1$.因此存在$h,k\in \mathbb{F}[x]$,使得
\begin{align*}
h\left( x \right) g\left( x \right) +n_1\left( x \right) n_2\left( x \right) \cdots n_m\left( x \right) k\left( x \right) =1.
\end{align*}
从而
\begin{align*}
h\left( A_i \right) =g^{-1}\left( A_i \right) ,i=1,2,\cdots ,m.
\end{align*}
\end{proof}

\begin{example}

\end{example}
\begin{proof}

\end{proof}


























\end{document}
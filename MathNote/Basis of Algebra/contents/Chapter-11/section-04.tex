\documentclass[../../main.tex]{subfiles}
\graphicspath{{\subfix{../../image/}}} % 指定图片目录,后续可以直接使用图片文件名。

% 例如:
% \begin{figure}[H]
% \centering
% \includegraphics[scale=0.4]{图.png}
% \caption{}
% \label{figure:图}
% \end{figure}
% 注意:上述\label{}一定要放在\caption{}之后,否则引用图片序号会只会显示??.

\begin{document}

\section{组合数论类问题}

\begin{example}
设$A$为$n \times n$实矩阵,其中元素$A(i,j) \in \{-1,1\}$,且$A$的行向量两两正交。若对任意的$i \in \{1,2,\ldots,k\}$,$j \in \{1,2,\ldots,l\}$,都有$A(i,j) = 1$,证明:$kl \leqslant n$。
\end{example}
\begin{proof}
设$\mathbf{a}_1, \mathbf{a}_2, \ldots, \mathbf{a}_k$是矩阵$A$的前$k$个行向量。根据题意,这些向量两两正交。即对于任意$i \neq j$且$1 \leqslant i, j \leqslant k$,我们有:
\[
\mathbf{a}_i \cdot \mathbf{a}_j = \sum_{p=1}^n A(i, p)A(j, p) = 0.
\]
我们可以将这个求和式分解为两部分:前$l$个分量和后$n - l$个分量。
\[
\sum_{p=1}^l A(i, p)A(j, p) + \sum_{p=l+1}^n A(i, p)A(j, p) = 0 
\]
根据题设,对于$1 \leqslant i \leqslant k$和$1 \leqslant p \leqslant l$,都有$A(i, p) = 1$。所以,对于任意$1 \leqslant i, j \leqslant k$且$i \neq j$,上式的第一部分为:
\[
\sum_{p=1}^l A(i, p)A(j, p) = \sum_{p=1}^l (1)(1) = l 
\]
将此结果代入,我们得到:
\[
l + \sum_{p=l+1}^n A(i, p)A(j, p) = 0 \implies \sum_{p=l+1}^n A(i, p)A(j, p) = -l 
\]
现在,我们定义$k$个新的向量$\mathbf{c}_1, \mathbf{c}_2, \ldots, \mathbf{c}_k \in \mathbb{R}^{n - l}$,其中每个向量$\mathbf{c}_i$由对应行向量$\mathbf{a}_i$的后$n - l$个分量构成:
\[
\mathbf{c}_i = (A(i, l + 1), A(i, l + 2), \ldots, A(i, n)).
\]
对于这些新向量,我们有如下的点积关系:

(i)对于$i \neq j$,$\mathbf{c}_i \cdot \mathbf{c}_j = \sum_{p=l+1}^n A(i, p)A(j, p) = -l$。

(ii)对于$i = j$,$\mathbf{c}_i \cdot \mathbf{c}_i = \sum_{p=l+1}^n (A(i, p))^2 = \sum_{p=l+1}^n 1^2 = n - l$,因为$A(i, p) \in \{-1, 1\}$。

考虑这些向量的和向量$\mathbf{v} = \sum_{i=1}^k \mathbf{c}_i$。我们来计算其模的平方$\|\mathbf{v}\|^2$。
\begin{align*}
\|\mathbf{v}\|^2 &= \mathbf{v} \cdot \mathbf{v} = \left( \sum_{i=1}^k \mathbf{c}_i \right) \cdot \left( \sum_{j=1}^k \mathbf{c}_j \right) \\
&= \sum_{i=1}^k \sum_{j=1}^k (\mathbf{c}_i \cdot \mathbf{c}_j)= \sum_{i=1}^k (\mathbf{c}_i \cdot \mathbf{c}_i) + \sum_{1 \leqslant i \neq j \leqslant k} (\mathbf{c}_i \cdot \mathbf{c}_j).
\end{align*}
在上式中,对角线上的项($i = j$)有$k$个,非对角线上的项($i \neq j$)有$k(k - 1)$个。代入我们之前计算的点积值:
\begin{align*}
\|\mathbf{v}\|^2 &= k \cdot (n - l) + k(k - 1) \cdot (-l) = kn - kl - k^2 l + kl \\
&= kn - k^2 l = k(n - kl)
\end{align*}
因为向量模的平方必须是非负的,所以$\|\mathbf{v}\|^2 \geqslant 0$。
\[
k(n - kl) \geqslant 0
\]
根据题意,$k$是行数,所以$k \geqslant 1$(如果$k = 0$或$l = 0$,则$kl = 0 \leqslant n$显然成立).故
\[
n - kl \geqslant 0.
\]
这直接导出了我们要证明的结论:
\[
kl \leqslant n \label{eq:12}
\]
证明完毕。

\end{proof}

\begin{example}
设$n \geqslant 4$。设$n$阶实方阵$A = (a_{ij})$满足$a_{ij} = \pm 1$且它的行向量两两正交。证明:
\begin{enumerate}[(1)]
\item $A$的列向量两两正交。

\item $n$是偶数,并且$n$是4的倍数。
\end{enumerate}
\end{example}
\begin{proof}
\begin{enumerate}[(1)]
\item 设$A$的列向量分块为$(\alpha_1,\alpha_2,\cdots,\alpha_n)$,则由$A$的行向量两两正交可知
\begin{align*}
AA^T=nI_n\Longrightarrow A^TA=nI_n\Longrightarrow \begin{pmatrix}
(\alpha_1,\alpha_1) & (\alpha_1,\alpha_2) & \cdots & (\alpha_1,\alpha_n) \\
(\alpha_2,\alpha_1) & (\alpha_2,\alpha_2) & \cdots & (\alpha_2,\alpha_n) \\
\vdots & \vdots & \ddots & \vdots \\
(\alpha_n,\alpha_1) & (\alpha_n,\alpha_2) & \cdots & (\alpha_n,\alpha_n)
\end{pmatrix} = \begin{pmatrix}
n & 0 & \cdots & 0 \\
0 & n & \cdots & 0 \\
\vdots & \vdots & \ddots & \vdots \\
0 & 0 & \cdots & n
\end{pmatrix}.
\end{align*}
故$(\alpha_i,\alpha_j)=0,i\ne j$。因此$A$的列向量也两两正交。

\item 不妨设$A$的第一行全为$1$,否则对第一行中$-1$所在列乘$-1$。由$A$的第一行与第二行正交知
\begin{align*}
\sum_{j=1}^n a_{1j}a_{2j} = \sum_{j=1}^n a_{2j} = 0.
\end{align*}
从而$A$的第二行元素中$1$和$-1$的个数相同,因此$2|n$。不妨设
\begin{gather*}
a_{2j}=1,\quad j=1,\cdots,\frac{n}{2}; \\
a_{2j}=-1,\quad j=\frac{n}{2}+1,\cdots,n.
\end{gather*}
再由$A$的第三行与第一行正交以及第三行与第二行正交可得
\begin{align*}
\sum_{j=1}^{\frac{n}{2}} a_{1j}a_{3j} + \sum_{j=\frac{n}{2}}^n a_{1j}a_{3j} = \sum_{j=1}^{\frac{n}{2}} a_{3j} + \sum_{j=\frac{n}{2}}^n a_{3j} = 0, \\
\sum_{j=1}^{\frac{n}{2}} a_{2j}a_{3j} + \sum_{j=\frac{n}{2}}^n a_{2j}a_{3j} = \sum_{j=1}^{\frac{n}{2}} a_{3j} - \sum_{j=\frac{n}{2}}^n a_{3j} = 0.
\end{align*}
由此可得
\begin{align*}
\sum_{j=1}^{\frac{n}{2}} a_{3j} = \sum_{j=\frac{n}{2}}^n a_{3j} = 0.
\end{align*}
因此$a_{31},a_{32},\cdots,a_{3,\frac{n}{2}}$中$-1$和$1$的个数相同,故$2|\frac{n}{2}$,即$4|n$。
\end{enumerate}

\end{proof}

\begin{example}
\begin{enumerate}[(1)]
\item 证明: 不存在$A,B,C \in \mathbb{Z}^{2 \times 2}$使得$\det A = \det B = \det C = 1$, $A^4 + B^4 = C^4$.

\item 是否存在$A,B,C \in \mathbb{Z}^{2 \times 2}$使得$\det A = \det B = \det C = 1$, $A^2 + B^2 = C^2$?
\end{enumerate}
\end{example}
\begin{proof}
\begin{enumerate}[(1)]
\item 反证,假设存在$A,B,C\in \mathbb{Z}^{2 \times 2}$,使得结论成立.则由$\det A = \det B = \det C = 1$知,$A,B,C$的特征多项式可设为
\begin{align*}
|xI - A| = x^2 - ax + 1, \quad |xI - B| = x^2 - bx + 1, \quad |xI - C| = x^2 - cx + 1,
\end{align*}
其中$\mathrm{tr}(A) = a$,$\mathrm{tr}(B) = b$,$\mathrm{tr}(C) = c$,且$a,b,c\in \mathbb{Z}$.由\hyperref[theorem:Cayley-Hamilton定理]{Cayley-Hamilton定理}知
\begin{align*}
A^2 - aA + I = 0, \quad B^2 - bB + I = 0, \quad C^2 - cC + I = 0.
\end{align*}
即
\begin{align*}
A^2 = aA - I, \quad B^2 = bB - I, \quad C^2 = cC - I.
\end{align*}
进而
\begin{align*}
\mathrm{tr}(A^2) = a^2 - 2, \quad \mathrm{tr}(B^2) = b^2 - 2, \quad \mathrm{tr}(C^2) = c^2 - 2.
\end{align*}
从而
\begin{align*}
A^4 = (A^2)^2 = (aA - I)^2 = a^2A^2 - 2aA + I,
\end{align*}
\begin{align*}
B^4 = (B^2)^2 = (bB - I)^2 = b^2B^2 - 2bB + I,
\end{align*}
\begin{align*}
C^4 = (C^2)^2 = (cC - I)^2 = c^2C^2 - 2cC + I.
\end{align*}
于是
\begin{align*}
\mathrm{tr}(A^4) = \mathrm{tr}(a^2A^2 - 2aA + I) = a^4 - 4a^2 + 2,
\end{align*}
\begin{align*}
\mathrm{tr}(B^4) = \mathrm{tr}(b^2B^2 - 2bB + I) = b^4 - 4b^2 + 2,
\end{align*}
\begin{align*}
\mathrm{tr}(C^4) = \mathrm{tr}(c^2C^2 - 2cC + I) = c^4 - 4c^2 + 2.
\end{align*}
由$A^4 + B^4 = C^4$知
\begin{align*}
\mathrm{tr}(A^4 + B^4) = \mathrm{tr}(C^4) \Longleftrightarrow a^4 + b^4 - 4(a^2 + b^2) + 4 = c^4 - 4c^2 + 2
\end{align*}
\begin{align}
\Longleftrightarrow a^4 + b^4 - c^4 = 4(a^2 + b^2 - c^2) - 2. \label{eq:21124232424ggg4334k2903jfm23234}
\end{align}
注意到任意的完全平方数除$4$的余数只有$0$或$1$,故
\begin{align*}
a^4, b^4, c^4 \equiv 0 \text{或} 1 \pmod{4}.
\end{align*}
又$4(a^2 + b^2 - c^2) - 2 \equiv 2 \pmod{4}$,故
\begin{align*}
a^4 + b^4 - c^4 \equiv 2 \pmod{4}.
\end{align*}
因此$a^4, b^4 \equiv 1 \pmod{4}$,$c^4 \equiv 0 \pmod{4}$.故$a,b$都是奇数,$c$是偶数.设$a = 2k_1 + 1$,$b = 2k_2 + 1$,$c = 2k_3$,则
\begin{align*}
a^4 = (2k_1 + 1)^4 = (4k_1^2 + 4k_1 + 1)^2 = 16k_1^4 + 32k_1^3 + 24k_1^2 + 8k_1 + 1,
\end{align*}
\begin{align*}
b^4 = (2k_2 + 1)^4 = (4k_2^2 + 4k_2 + 1)^2 = 16k_2^4 + 32k_2^3 + 24k_2^2 + 8k_2 + 1,
\end{align*}
\begin{align*}
c^4 = (2k_3)^4 = 16k_3^4.
\end{align*}
从而$a^4, b^4 \equiv 1 \pmod{8}$,$c^4 \equiv 0 \pmod{8}$.于是$a^4 + b^4 - c^4 \equiv 2 \pmod{8}$.因此再由\eqref{eq:21124232424ggg4334k2903jfm23234}式可得
\begin{align*}
4(a^2 + b^2 - c^2) - 2 \equiv 2 \pmod{8} \Longrightarrow 4(a^2 + b^2 - c^2) - 4 \equiv 0 \pmod{8}.
\end{align*}
故存在$k\in \mathbb{Z}$,使得
\begin{align*}
4(a^2 + b^2 - c^2) - 4 = 8k \Longrightarrow (a^2 + b^2 - c^2) - 1 = 2k \Longrightarrow c^2 = a^2 + b^2 - (2k + 1).
\end{align*}
又因为$a,b$都是奇数,所以$c^2 = a^2 + b^2 - (2k + 1) \equiv 1 \pmod{2}$,即$c^2$是奇数.这与$c$是偶数矛盾!

\item 存在.只需取$A\in \mathbb{Z}^{2 \times 2}$满足$\det A = 1$且$\mathrm{tr}(A) = 1$,$B = I - A$,再取$C\in \mathbb{Z}^{2 \times 2}$满足$\det C = 1$且$C^2 = -I$,$\mathrm{tr}(C) = 0$.此时由\hyperref[theorem:Cayley-Hamilton定理]{Cayley-Hamilton定理}定理知$A^2 - A + I = 0$,$C^2 + I = 0$从而
\begin{align*}
\det B = \det(I - A) = \det(A^2) = 1,
\end{align*}
\begin{align*}
A^2 + B^2 = A^2 + (I - A)^2 = 2A^2 - 2A + I = -I = C^2.
\end{align*}
因此我们可取$A = \begin{pmatrix} 1 & 1 \\ -1 & 0 \end{pmatrix}$,$B = \begin{pmatrix} 0 & -1 \\ 1 & 1 \end{pmatrix}$,$C = \begin{pmatrix} 0 & 1 \\ -1 & 0 \end{pmatrix}$.
\end{enumerate}

\end{proof}


















\end{document}
\documentclass[../../main.tex]{subfiles}
\graphicspath{{\subfix{../../image/}}} % 指定图片目录,后续可以直接使用图片文件名。

% 例如:
% \begin{figure}[H]
% \centering
% \includegraphics[scale=0.4]{图.png}
% \caption{}
% \label{figure:图}
% \end{figure}
% 注意:上述\label{}一定要放在\caption{}之后,否则引用图片序号会只会显示??.

\begin{document}

\section{CMC红宝书高代习题}

\begin{example}
设 \(n \geqslant 3\), \(n\) 阶矩阵
\[
\boldsymbol{A} = \begin{pmatrix} 1 & a & \cdots & a \\ a & 1 & \cdots & a \\ \vdots & \vdots & & \vdots \\ a & a & \cdots & 1 \end{pmatrix},
\]
计算 \(\operatorname{rank}(\boldsymbol{A})\).
\end{example}
\begin{solution}
\[
\begin{aligned}
|\boldsymbol{A}| &= \begin{vmatrix} (n - 1)a + 1 & a & \cdots & a \\ (n - 1)a + 1 & 1 & \cdots & a \\ \vdots & \vdots & & \vdots \\ (n - 1)a + 1 & a & \cdots & 1 \end{vmatrix} = [(n - 1)a + 1] \begin{vmatrix} 1 & a & \cdots & a \\ 0 & 1 - a & \cdots & 0 \\ \vdots & \vdots & & \vdots \\ 0 & 0 & \cdots & 1 - a \end{vmatrix} \\
&= [(n - 1)a + 1](1 - a)^{n - 1}.
\end{aligned}
\]
若 \(a \neq 1\) 且 \(a \neq \dfrac{1}{1 - n}\), 则 \(|\boldsymbol{A}| \neq 0\), \(\operatorname{rank}(\boldsymbol{A}) = n\). 若 \(a = 1\), 易知 \(\operatorname{rank}(\boldsymbol{A}) = 1\). 若 \(a = \dfrac{1}{1 - n}\), 容易看出 \(\boldsymbol{A}\) 的前 \(n - 1\) 行、\(n - 1\) 列构成的子矩阵为对角占优矩阵, 行列式不为 0. 所以 \(\operatorname{rank}(\boldsymbol{A}) = n - 1\).
\end{solution}

\begin{example}
设 \(\boldsymbol{A}\) 为 2023 阶非零实矩阵. \(A_{ij}\) 为 \(\boldsymbol{A}\) 中元素 \(a_{ij}\) 的代数余子式且 \(A_{ij} = a_{ij}\). 求 \(\boldsymbol{A}\) 的秩与行列式.
\end{example}
\begin{solution}
由于 \(\boldsymbol{A} \neq \boldsymbol{O}\), 不妨设 \(a_{kl} \neq 0\). \(\boldsymbol{A}\) 按第 \(k\) 行展开有
\[
|\boldsymbol{A}| = a_{k1}^2 + a_{k2}^2 + \cdots + a_{kn}^2 \neq 0.
\]
故 \(\operatorname{rank}(\boldsymbol{A}) = 2023\).

由已知有 \(\boldsymbol{A}\) 的伴随矩阵 \(\boldsymbol{A}^* = \boldsymbol{A}^\mathrm{T}\). 所以由 \(\boldsymbol{A}\boldsymbol{A}^* = |\boldsymbol{A}|\boldsymbol{E}\) 有 \(\boldsymbol{A}\boldsymbol{A}^\mathrm{T} = |\boldsymbol{A}|\boldsymbol{E}\). 两边取行列式有
\[
|\boldsymbol{A}|^2 = |\boldsymbol{A}|^{2023}.
\]
注意到 \(|\boldsymbol{A}| \neq 0\), 所以 \(|\boldsymbol{A}|^{2021} = 1\). 又 \(|\boldsymbol{A}|\) 为实数, 所以 \(|\boldsymbol{A}| = 1\).
\end{solution}

\begin{example}
设 1013 阶实方阵 \(\boldsymbol{A}\) 可对角化且满足
\[
\boldsymbol{A}^2 - 1013\boldsymbol{A} + 2022\boldsymbol{E} = \boldsymbol{O}
\]
和
\[
\operatorname{rank}(\boldsymbol{A} - 2\boldsymbol{E}) = 3.
\]
求 \(\boldsymbol{A}\) 的特征值.
\end{example}
\begin{solution}
由于 \(\boldsymbol{A}\) 可对角化, 故 \(\boldsymbol{A}\) 的每个特征值的代数重数等于其几何重数. 令 \(\lambda\) 为 \(\boldsymbol{A}\) 的特征值, 则有
\[
\lambda^2 - 1013\lambda + 2022 = 0,
\]
即
\[
(\lambda - 2)(\lambda - 1011) = 0,
\]
\[
\lambda_1 = 2, \quad \lambda_2 = 1011.
\]
故 \(\boldsymbol{A}\) 的特征值只可能是 \(\lambda_1 = 2, \lambda_2 = 1011\). 注意到 \(\operatorname{rank}(\boldsymbol{A} - 2\boldsymbol{E}) = 3 < 1013\). 所以$|\boldsymbol{A} - 2\boldsymbol{E}|=0$,故 \(2\) 是 \(\boldsymbol{A}\) 的一个特征值.

由于可对角化, 特征值 \(2\) 的代数重数等于其几何重数. 由于 \((\boldsymbol{A} - 2\boldsymbol{E})\boldsymbol{x} = \boldsymbol{0}\) 的解空间维数为 \(1013 - \operatorname{rank}(\boldsymbol{A} - 2\boldsymbol{E}) = 1010\), 因此 \(2\) 作为 \(\boldsymbol{A}\) 的特征值, 其代数重数为 \(1010\). 故 \(\boldsymbol{A}\) 还有其他特征值, 且一定为 \(1011\). 其代数重数为 \(1013 - 1010 = 3\). 至此 \(\boldsymbol{A}\) 的全部特征值为 \(\lambda_1 = 2 \, (1010 \text{ 重}), \lambda_2 = 1011 \, (3 \text{ 重})\).
\end{solution}

\begin{example}

\end{example}
\begin{solution}

\end{solution}

\begin{example}

\end{example}
\begin{solution}

\end{solution}

\begin{example}

\end{example}
\begin{solution}

\end{solution}

\begin{example}

\end{example}
\begin{solution}

\end{solution}

\begin{example}

\end{example}
\begin{solution}

\end{solution}

\begin{example}

\end{example}
\begin{solution}

\end{solution}

\begin{example}

\end{example}
\begin{solution}

\end{solution}

\begin{example}

\end{example}
\begin{solution}

\end{solution}

\begin{example}

\end{example}
\begin{solution}

\end{solution}

\begin{example}

\end{example}
\begin{solution}

\end{solution}

\begin{example}

\end{example}
\begin{solution}

\end{solution}

\begin{example}

\end{example}
\begin{solution}

\end{solution}

\begin{example}

\end{example}
\begin{solution}

\end{solution}

\begin{example}

\end{example}
\begin{solution}

\end{solution}

\begin{example}

\end{example}
\begin{solution}

\end{solution}

\begin{example}

\end{example}
\begin{solution}

\end{solution}

\begin{example}

\end{example}
\begin{solution}

\end{solution}

\begin{example}

\end{example}
\begin{solution}

\end{solution}

\begin{example}

\end{example}
\begin{solution}

\end{solution}

\begin{example}

\end{example}
\begin{solution}

\end{solution}

\begin{example}

\end{example}
\begin{solution}

\end{solution}

\begin{example}

\end{example}
\begin{solution}

\end{solution}

\begin{example}

\end{example}
\begin{solution}

\end{solution}

\begin{example}

\end{example}
\begin{solution}

\end{solution}

\begin{example}

\end{example}
\begin{solution}

\end{solution}

\begin{example}

\end{example}
\begin{solution}

\end{solution}

\begin{example}

\end{example}
\begin{solution}

\end{solution}

\begin{example}

\end{example}
\begin{solution}

\end{solution}

\begin{example}

\end{example}
\begin{solution}

\end{solution}
















\end{document}
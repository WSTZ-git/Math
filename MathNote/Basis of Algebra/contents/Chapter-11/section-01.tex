\documentclass[../../main.tex]{subfiles}
\graphicspath{{\subfix{../../image/}}} % 指定图片目录,后续可以直接使用图片文件名。

% 例如:
% \begin{figure}[H]
% \centering
% \includegraphics[scale=0.4]{图.png}
% \caption{}
% \label{figure:图}
% \end{figure}
% 注意:上述\label{}一定要放在\caption{}之后,否则引用图片序号会只会显示??.

\begin{document}

\section{CMC真题和红宝书高代习题}

\begin{example}
设 \(n \geqslant 3\), \(n\) 阶矩阵
\[
\boldsymbol{A} = \begin{pmatrix} 1 & a & \cdots & a \\ a & 1 & \cdots & a \\ \vdots & \vdots & & \vdots \\ a & a & \cdots & 1 \end{pmatrix},
\]
计算 \(\operatorname{rank}(\boldsymbol{A})\).
\end{example}
\begin{solution}
\[
\begin{aligned}
|\boldsymbol{A}| &= \begin{vmatrix} (n - 1)a + 1 & a & \cdots & a \\ (n - 1)a + 1 & 1 & \cdots & a \\ \vdots & \vdots & & \vdots \\ (n - 1)a + 1 & a & \cdots & 1 \end{vmatrix} = [(n - 1)a + 1] \begin{vmatrix} 1 & a & \cdots & a \\ 0 & 1 - a & \cdots & 0 \\ \vdots & \vdots & & \vdots \\ 0 & 0 & \cdots & 1 - a \end{vmatrix} \\
&= [(n - 1)a + 1](1 - a)^{n - 1}.
\end{aligned}
\]
若 \(a \neq 1\) 且 \(a \neq \dfrac{1}{1 - n}\), 则 \(|\boldsymbol{A}| \neq 0\), \(\operatorname{rank}(\boldsymbol{A}) = n\). 若 \(a = 1\), 易知 \(\operatorname{rank}(\boldsymbol{A}) = 1\). 若 \(a = \dfrac{1}{1 - n}\), 容易看出 \(\boldsymbol{A}\) 的前 \(n - 1\) 行、\(n - 1\) 列构成的子矩阵为严格对角占优矩阵, 由\refpro{proposition:严格对角占优阵必是非异阵}知其行列式不为 0. 所以 \(\operatorname{rank}(\boldsymbol{A}) = n - 1\).
\end{solution}

\begin{example}
设 \(\boldsymbol{A}\) 为 2023 阶非零实矩阵. \(A_{ij}\) 为 \(\boldsymbol{A}\) 中元素 \(a_{ij}\) 的代数余子式且 \(A_{ij} = a_{ij}\). 求 \(\boldsymbol{A}\) 的秩与行列式.
\end{example}
\begin{solution}
由于 \(\boldsymbol{A} \neq \boldsymbol{O}\), 不妨设 \(a_{kl} \neq 0\). \(\boldsymbol{A}\) 按第 \(k\) 行展开有
\[
|\boldsymbol{A}| = a_{k1}^2 + a_{k2}^2 + \cdots + a_{kn}^2 \neq 0.
\]
故 \(\operatorname{rank}(\boldsymbol{A}) = 2023\).

由已知有 \(\boldsymbol{A}\) 的伴随矩阵 \(\boldsymbol{A}^* = \boldsymbol{A}^\mathrm{T}\). 所以由 \(\boldsymbol{A}\boldsymbol{A}^* = |\boldsymbol{A}|\boldsymbol{E}\) 有 \(\boldsymbol{A}\boldsymbol{A}^\mathrm{T} = |\boldsymbol{A}|\boldsymbol{E}\). 两边取行列式有
\[
|\boldsymbol{A}|^2 = |\boldsymbol{A}|^{2023}.
\]
注意到 \(|\boldsymbol{A}| \neq 0\), 所以 \(|\boldsymbol{A}|^{2021} = 1\). 又 \(|\boldsymbol{A}|\) 为实数, 所以 \(|\boldsymbol{A}| = 1\).
\end{solution}

\begin{example}
设 1013 阶实方阵 \(\boldsymbol{A}\) 可对角化且满足
\[
\boldsymbol{A}^2 - 1013\boldsymbol{A} + 2022\boldsymbol{E} = \boldsymbol{O}
\]
和
\[
\operatorname{rank}(\boldsymbol{A} - 2\boldsymbol{E}) = 3.
\]
求 \(\boldsymbol{A}\) 的特征值.
\end{example}
\begin{solution}
由于 \(\boldsymbol{A}\) 可对角化, 故 \(\boldsymbol{A}\) 的每个特征值的代数重数等于其几何重数. 令 \(\lambda\) 为 \(\boldsymbol{A}\) 的特征值, 则有
\[
\lambda^2 - 1013\lambda + 2022 = 0,
\]
即
\[
(\lambda - 2)(\lambda - 1011) = 0,
\]
\[
\lambda_1 = 2, \quad \lambda_2 = 1011.
\]
故 \(\boldsymbol{A}\) 的特征值只可能是 \(\lambda_1 = 2, \lambda_2 = 1011\). 注意到 \(\operatorname{rank}(\boldsymbol{A} - 2\boldsymbol{E}) = 3 < 1013\). 所以$|\boldsymbol{A} - 2\boldsymbol{E}|=0$,故 \(2\) 是 \(\boldsymbol{A}\) 的一个特征值.

由于可对角化, 特征值 \(2\) 的代数重数等于其几何重数. 由于 \((\boldsymbol{A} - 2\boldsymbol{E})\boldsymbol{x} = \boldsymbol{0}\) 的解空间维数为 \(1013 - \operatorname{rank}(\boldsymbol{A} - 2\boldsymbol{E}) = 1010\), 因此 \(2\) 作为 \(\boldsymbol{A}\) 的特征值, 其代数重数为 \(1010\). 故 \(\boldsymbol{A}\) 还有其他特征值, 且一定为 \(1011\). 其代数重数为 \(1013 - 1010 = 3\). 至此 \(\boldsymbol{A}\) 的全部特征值为 \(\lambda_1 = 2 \, (1010 \text{ 重}), \lambda_2 = 1011 \, (3 \text{ 重})\).
\end{solution}

\begin{example}
设 $\boldsymbol{A}$ 为 3 阶实对称矩阵, $\operatorname{rank}(\boldsymbol{A}) = 2$ 且
$$
\boldsymbol{A}\begin{pmatrix} 1 & 1 \\ 0 & 0 \\ -1 & 1 \end{pmatrix} = \begin{pmatrix} -1 & 1 \\ 0 & 0 \\ 1 & 1 \end{pmatrix}.
$$

求: (1) $\boldsymbol{A}$ 的所有特征值.
$\quad \quad$
(2) 矩阵 $\boldsymbol{A}$.
\end{example}
\begin{solution}
(1) 由 $\operatorname{rank}(\boldsymbol{A}) = 2 < 3$, 知 $0$ 是 $\boldsymbol{A}$ 的特征值. 由
$$
\boldsymbol{A}\begin{pmatrix} 1 \\ 0 \\ -1 \end{pmatrix} = \begin{pmatrix} -1 \\ 0 \\ 1 \end{pmatrix} = -\begin{pmatrix} 1 \\ 0 \\ -1 \end{pmatrix}, \quad \boldsymbol{A}\begin{pmatrix} 1 \\ 0 \\ 1 \end{pmatrix} = \begin{pmatrix} 1 \\ 0 \\ 1 \end{pmatrix}
$$
得到 $-1, 1$ 也是 $\boldsymbol{A}$ 的特征值. 因此 $\boldsymbol{A}$ 的全部特征值为 $0, 1, -1$.

(2) 先求 $\boldsymbol{A}$ 的属于特征值 $0$ 的特征向量.

由于 $\boldsymbol{A}$ 为实对称矩阵, 属于不同特征值的特征向量正交. 设 $\boldsymbol{v} = \begin{pmatrix} x_1 \\ x_2 \\ x_3 \end{pmatrix}$ 是 $\boldsymbol{A}$ 的属于特征值 $0$ 的特征向量, 则有
$$
\begin{cases}
x_1 + x_3 = 0, \\
x_1 - x_3 = 0,
\end{cases}
$$
得到 $x_1 = x_3 = 0$, $x_2$ 为自由未知量. 从而 $\boldsymbol{v} = \begin{pmatrix} 0 \\ c \\ 0 \end{pmatrix}$, 其中 $c \neq 0$.

取 $c = 1$, 则有
$$
\begin{pmatrix} 0 & 1 & 1 \\ 1 & 0 & 0 \\ 0 & 1 & -1 \end{pmatrix}^{-1} \boldsymbol{A} \begin{pmatrix} 0 & 1 & 1 \\ 1 & 0 & 0 \\ 0 & 1 & -1 \end{pmatrix} = \begin{pmatrix} 0 & & \\ & 1 & \\ & & -1 \end{pmatrix}.
$$
故
$$
\begin{aligned}
\boldsymbol{A} &= \begin{pmatrix} 0 & 1 & 1 \\ 1 & 0 & 0 \\ 0 & 1 & -1 \end{pmatrix} \begin{pmatrix} 0 & & \\ & 1 & \\ & & -1 \end{pmatrix} \begin{pmatrix} 0 & 1 & 1 \\ 1 & 0 & 0 \\ 0 & 1 & -1 \end{pmatrix}^{-1} \\
&= \begin{pmatrix} 0 & 1 & -1 \\ 0 & 0 & 0 \\ 0 & 1 & 1 \end{pmatrix} \begin{pmatrix} 0 & 1 & 0 \\ \frac{1}{2} & 0 & \frac{1}{2} \\ \frac{1}{2} & 0 & -\frac{1}{2} \end{pmatrix} \\
&= \begin{pmatrix} 0 & 0 & 1 \\ 0 & 0 & 0 \\ 1 & 0 & 0 \end{pmatrix}.
\end{aligned}
$$
\end{solution}

\begin{example}
设 $n \geqslant 3$ 且 $\begin{pmatrix} a_1 \\ a_2 \\ \vdots \\ a_n \end{pmatrix}$ 与 $\begin{pmatrix} b_1 \\ b_2 \\ \vdots \\ b_n \end{pmatrix}$ 为正交的非零实向量. 求矩阵
$$
\boldsymbol{A} = \begin{pmatrix} a_1 + b_1 & a_1 + b_2 & \cdots & a_1 + b_n \\ a_2 + b_1 & a_2 + b_2 & \cdots & a_2 + b_n \\ a_3 + b_1 & a_3 + b_2 & \cdots & a_3 + b_n \\ \vdots & \vdots & & \vdots \\ a_n + b_1 & a_n + b_2 & \cdots & a_n + b_n \end{pmatrix}
$$
的全部特征值.
\end{example}
\begin{solution}
{\color{blue}解法一:}首先计算 $|\boldsymbol{A}|$ 以便确定 $0$ 是否为 $\boldsymbol{A}$ 的特征值. 易知
$$
|\boldsymbol{A}| = \begin{vmatrix} a_1 + b_1 & a_1 + b_2 & \cdots & a_1 + b_n \\ a_2 - a_1 & a_2 - a_1 & \cdots & a_2 - a_1 \\ a_3 - a_1 & a_3 - a_1 & \cdots & a_3 - a_1 \\ \vdots & \vdots & & \vdots \\ a_n - a_1 & a_n - a_1 & \cdots & a_n - a_1 \end{vmatrix} = 0.
$$
故 $0$ 是 $\boldsymbol{A}$ 的特征值. 下面确定 $0$ 的代数重数.

由于 $\operatorname{rank}(\boldsymbol{A} - 0\boldsymbol{E}) = \operatorname{rank}(\boldsymbol{A}) \leqslant 2$, 因此 $0$ 的代数重数 $\geqslant 0$ 的几何重数 $\geqslant n - 2$, 即 $\boldsymbol{A}$ 至少有 $n - 2$ 个特征值为 $0$.

由于 $a_1b_1 + a_2b_2 + \cdots + a_nb_n = 0$, 因此
$$
\boldsymbol{A} \begin{pmatrix} a_1 \\ a_2 \\ \vdots \\ a_n \end{pmatrix} = \begin{pmatrix} a_1^2 + a_1a_2 + \cdots + a_1a_n \\ a_1a_2 + a_2^2 + \cdots + a_2a_n \\ \vdots \\ a_1a_n + a_2a_n + \cdots + a_n^2 \end{pmatrix} = (a_1 + a_2 + \cdots + a_n) \begin{pmatrix} a_1 \\ a_2 \\ \vdots \\ a_n \end{pmatrix}.
$$
故 $a_1 + a_2 + \cdots + a_n$ 也是 $\boldsymbol{A}$ 的特征值, 而
$$
\boldsymbol{A}^{\mathrm{T}} \begin{pmatrix} b_1 \\ b_2 \\ \vdots \\ b_n \end{pmatrix} = (b_1 + b_2 + \cdots + b_n) \begin{pmatrix} b_1 \\ b_2 \\ \vdots \\ b_n \end{pmatrix}.
$$
故 $b_1 + b_2 + \cdots + b_n$ 是 $\boldsymbol{A}^{\mathrm{T}}$ 的特征值. 从而也是 $\boldsymbol{A}$ 的特征值.

现在设 $\boldsymbol{A}$ 的 $n$ 个特征值为 $\underbrace{0, \cdots, 0}_{n - 2}, \lambda_{n - 1}, \lambda_n$. 于是
$$
\operatorname{tr}(\boldsymbol{A}) = \lambda_{n - 1} + \lambda_n = (a_1 + a_2 + \cdots + a_n) + (b_1 + b_2 + \cdots + b_n).
$$

若 $a_1 + a_2 + \cdots + a_n \neq 0$, 则 $a_1 + a_2 + \cdots + a_n$ 为 $\boldsymbol{A}$ 的一个非零特征值, 从而 $\lambda_{n - 1}$ 与 $\lambda_n$ 中一个为 $a_1 + a_2 + \cdots + a_n$, 另一个为 $b_1 + b_2 + \cdots + b_n$.

同理, 当 $b_1 + b_2 + \cdots + b_n \neq 0$ 时, $\lambda_{n - 1}$ 与 $\lambda_n$ 也是一个为 $a_1 + a_2 + \cdots + a_n$, 另一个为 $b_1 + b_2 + \cdots + b_n$.

最后考察 $a_1 + a_2 + \cdots + a_n = b_1 + b_2 + \cdots + b_n = 0$ 的情形.

注意到 $\boldsymbol{A}$ 的特征多项式 $\lambda^{n - 2}$ 的系数是 $\boldsymbol{A}$ 的所有二阶主子式之和. 从而
$$
\begin{aligned}
\lambda_{n - 1}\lambda_n &= \boldsymbol{A} \text{的所有阶主子式之和} \\
&= \sum_{1 \leqslant j < i \leqslant n} \begin{vmatrix} a_j + b_j & a_j + b_i \\ a_i + b_j & a_i + b_i \end{vmatrix} \\
&= \sum_{1 \leqslant j < i \leqslant n} (a_i - a_j)(b_j - b_i) \\
&= \sum_{1 \leqslant j < i \leqslant n} (a_ib_j + a_jb_i) - \sum_{1 \leqslant j < i \leqslant n} (a_ib_i + a_jb_j),
\end{aligned}
$$
$$
\begin{aligned}
\sum_{1 \leqslant j < i \leqslant n} (a_ib_j + a_jb_i) &= \sum_{1 \leqslant j < i \leqslant n} a_ib_j + \sum_{1 \leqslant j < i \leqslant n} a_jb_i = \sum_{\substack{i, j = 1 \\ i \neq j}}^n a_ib_j \\
&= \sum_{i = 1}^n a_i \sum_{j = 1}^n b_j - (a_1b_1 + a_2b_2 + \cdots + a_nb_n) = 0,
\end{aligned}
$$
$$
\sum_{1 \leqslant j < i \leqslant n} (a_ib_i + a_jb_j) = (n - 1)(a_1b_1 + a_2b_2 + \cdots + a_nb_n) = 0.
$$

因此 $\lambda_{n - 1}\lambda_n = 0$, 又 $\lambda_{n - 1} + \lambda_n = 0$, 从而 $\lambda_{n - 1} = \lambda_n = 0$. 这时 $\boldsymbol{A}$ 的全部特征值均为 $0$ ($n$ 重).

{\color{blue}解法二:}根据题设可知, 矩阵 $\boldsymbol{A}$ 可表示为
$$
\boldsymbol{A} = \begin{pmatrix} a_1 & 1 \\ a_2 & 1 \\ \vdots & \vdots \\ a_n & 1 \end{pmatrix} \begin{pmatrix} 1 & 1 & \cdots & 1 \\ b_1 & b_2 & \cdots & b_n \end{pmatrix} = \boldsymbol{B}\boldsymbol{C},
$$
其中
$$
\boldsymbol{B} = \begin{pmatrix} a_1 & 1 \\ a_2 & 1 \\ \vdots & \vdots \\ a_n & 1 \end{pmatrix}, \quad \boldsymbol{C} = \begin{pmatrix} 1 & 1 & \cdots & 1 \\ b_1 & b_2 & \cdots & b_n \end{pmatrix}.
$$
利用行列式降阶公式, 矩阵 $\boldsymbol{A}$ 的特征多项式可化为
$$
|\lambda\boldsymbol{E} - \boldsymbol{A}| = |\lambda\boldsymbol{E} - \boldsymbol{B}\boldsymbol{C}| = \lambda^{n - 2}|\lambda\boldsymbol{E} - \boldsymbol{C}\boldsymbol{B}|.
$$

因为 $\sum_{k = 1}^n a_kb_k = 0$, 所以
$$
\boldsymbol{C}\boldsymbol{B} = \begin{pmatrix} 1 & 1 & \cdots & 1 \\ b_1 & b_2 & \cdots & b_n \end{pmatrix} \begin{pmatrix} a_1 & 1 \\ a_2 & 1 \\ \vdots & \vdots \\ a_n & 1 \end{pmatrix}
$$
$$
= \begin{pmatrix} \sum_{k = 1}^n a_k & n \\ \sum_{k = 1}^n a_kb_k & \sum_{k = 1}^n b_k \end{pmatrix} = \begin{pmatrix} \sum_{k = 1}^n a_k & n \\ 0 & \sum_{k = 1}^n b_k \end{pmatrix},
$$

从而有
$$
|\lambda\boldsymbol{E} - \boldsymbol{A}| = \lambda^{n - 2} \begin{vmatrix} \lambda - \sum_{k = 1}^n a_k & -n \\ 0 & \lambda - \sum_{k = 1}^n b_k \end{vmatrix} = \lambda^{n - 2} \left( \lambda - \sum_{k = 1}^n a_k \right) \left( \lambda - \sum_{k = 1}^n b_k \right).
$$

因此 $\boldsymbol{A}$ 的特征值为 $\lambda_1 = 0$ ($n - 2$ 重), $\lambda_2 = \sum_{k = 1}^n a_k$, $\lambda_3 = \sum_{k = 1}^n b_k$.
\end{solution}

\begin{example}
计算 $\begin{pmatrix} \cos\theta & -\sin\theta \\ \sin\theta & \cos\theta \end{pmatrix}^n$ 的迹.
\end{example}
\begin{remark}
注意到 $\begin{pmatrix} \cos\theta & -\sin\theta \\ \sin\theta & \cos\theta \end{pmatrix}$ 是第一类正交矩阵, 其几何意义表示平面上的旋转变换, 所以也可直接得到
$$
\begin{pmatrix} \cos\theta & -\sin\theta \\ \sin\theta & \cos\theta \end{pmatrix}^n = \begin{pmatrix} \cos n\theta & -\sin n\theta \\ \sin n\theta & \cos n\theta \end{pmatrix}.
$$
\end{remark}
\begin{solution}
容易计算
$$
\begin{pmatrix} \cos\theta & -\sin\theta \\ \sin\theta & \cos\theta \end{pmatrix}^2 = \begin{pmatrix} \cos2\theta & -\sin2\theta \\ \sin2\theta & \cos2\theta \end{pmatrix}.
$$

利用数学归纳法容易证明
$$
\begin{pmatrix} \cos\theta & -\sin\theta \\ \sin\theta & \cos\theta \end{pmatrix}^n = \begin{pmatrix} \cos n\theta & -\sin n\theta \\ \sin n\theta & \cos n\theta \end{pmatrix}.
$$

从而
$$
\operatorname{tr}\begin{pmatrix} \cos\theta & -\sin\theta \\ \sin\theta & \cos\theta \end{pmatrix}^n = 2\cos n\theta.
$$
\end{solution}

\begin{example}
设 $\boldsymbol{A}, \boldsymbol{B}$ 为 $n$ 阶实方阵, 且 $\boldsymbol{AB} - \boldsymbol{BA} = \boldsymbol{A}^{\mathrm{T}} + \boldsymbol{B}^{\mathrm{T}}$. 求 $\boldsymbol{A} + \boldsymbol{B}$.
\end{example}
\begin{solution}
由 $\boldsymbol{AB} - \boldsymbol{BA} = \boldsymbol{A}^{\mathrm{T}} + \boldsymbol{B}^{\mathrm{T}}$ 得到
$$
(\boldsymbol{AB} - \boldsymbol{BA})(\boldsymbol{A} + \boldsymbol{B}) = (\boldsymbol{A} + \boldsymbol{B})^{\mathrm{T}}(\boldsymbol{A} + \boldsymbol{B}).
$$

由于
$$
\begin{aligned}
\operatorname{tr}\left[(\boldsymbol{AB} - \boldsymbol{BA})(\boldsymbol{A} + \boldsymbol{B})\right] &= \operatorname{tr}\left(\boldsymbol{ABA} - \boldsymbol{BA}^2 + \boldsymbol{AB}^2 - \boldsymbol{BAB}\right) \\
&= \left[\operatorname{tr}\left(\boldsymbol{ABA}\right) - \operatorname{tr}\left(\boldsymbol{BA}^2\right)\right] + \left[\operatorname{tr}\left(\boldsymbol{AB}^2\right) - \operatorname{tr}\left(\boldsymbol{BAB}\right)\right] \\
&= 0 + 0 = 0,
\end{aligned}
$$
因此 $\operatorname{tr}\left[(\boldsymbol{A} + \boldsymbol{B})^{\mathrm{T}}(\boldsymbol{A} + \boldsymbol{B})\right] = 0$. 从而 $\boldsymbol{A} + \boldsymbol{B}$ 的所有元素的平方和为 $0$. 又 $\boldsymbol{A} + \boldsymbol{B}$ 为实矩阵, 所以 $\boldsymbol{A} + \boldsymbol{B} = \boldsymbol{O}$.
\end{solution}

\begin{example}
设 \( A = (a_{ij}) \) 为 \( n \) 阶复矩阵,\( \lambda \) 为 \( A \) 的模最小的一个特征值. 证明
\[
|\lambda| \leqslant \sqrt{n} \max_{1 \leqslant i,j \leqslant n} |a_{ij}|.
\]
\end{example}
\begin{proof}
由\hyperref[theorem:Schur(舒尔)定理]{Schur(舒尔)分解}知,存在西矩阵 \( U \) 使得
\[
T = UAU^* = \left( \begin{matrix}
\lambda _1&		&		&		*\\
&		\lambda _2&		&		\\
&		&		\ddots&		\\
0&		&		&		\lambda _n\\
\end{matrix} \right) .
\]
这里 \( U^* = \overline{U}^\mathrm{T} =U^{-1}\)(共轭转置).

进而
\begin{align*}
\mathrm{tr}(TT^*)= \mathrm{tr}(UAU^*UA^*U^*)= \mathrm{tr}(UAA^*U^*) = \mathrm{tr}(AA^*).
\end{align*}
故有
\[
\sum_{i=1}^n |\lambda_i|^2 = \sum_{i,j=1}^n |a_{ij}|^2 \leqslant n^2 \max_{1 \leqslant i,j \leqslant n} |a_{ij}|^2.
\]
因此
\[
n|\lambda|^2 \leqslant \sum_{i=1}^n |\lambda_i|^2 \leqslant n^2 \max_{1 \leqslant i,j \leqslant n} |a_{ij}|^2,
\]
即
\[
|\lambda|^2 \leqslant n \max_{1 \leqslant i,j \leqslant n} |a_{ij}|^2.
\]
故
\[
|\lambda| \leqslant \sqrt{n} \max_{1 \leqslant i,j \leqslant n} |a_{ij}|.
\]
\end{proof}

\begin{example}
设 $\boldsymbol{A},\boldsymbol{B}$ 为 $n$ 阶方阵. 齐次线性方程组 $\boldsymbol{A}\boldsymbol{x} = \boldsymbol{0}$ 与 $\boldsymbol{B}\boldsymbol{x} = \boldsymbol{0}$ 同解, 且它们的基础解系中含 $m$ 个线性无关的解向量. 证明 $\mathrm{rank}(\boldsymbol{B} - \boldsymbol{A}) \leqslant n - m$.
\end{example}
\begin{remark}
$N(B)$表示$\boldsymbol{B}\boldsymbol{x} = \boldsymbol{0}$的解空间.
\end{remark}
\begin{proof}
若 $\boldsymbol{B}\boldsymbol{x} = \boldsymbol{0}$, 则 $\boldsymbol{A}\boldsymbol{x} = \boldsymbol{0}$, 因而 $(\boldsymbol{B} - \boldsymbol{A})\boldsymbol{x} = \boldsymbol{0}$. 故 $N(\boldsymbol{B}) \subseteq N(\boldsymbol{B} - \boldsymbol{A})$. 因此
\[
n - \mathrm{rank}(\boldsymbol{B} - \boldsymbol{A}) = \dim N(\boldsymbol{B} - \boldsymbol{A}) \geqslant \dim N(\boldsymbol{B}) = m,
\]
从而
\[
\mathrm{rank}(\boldsymbol{B} - \boldsymbol{A}) \leqslant n - m.
\]
\end{proof}

\begin{example}
设 $\boldsymbol{A},\boldsymbol{B}$ 为 $n$ 阶矩阵, $\boldsymbol{AB} = \boldsymbol{BA}$. 证明
\[
\mathrm{rank}(\boldsymbol{A} + \boldsymbol{B}) \leqslant \mathrm{rank}(\boldsymbol{A}) + \mathrm{rank}(\boldsymbol{B}) - \mathrm{rank}(\boldsymbol{AB}).
\]
\end{example}
\begin{proof}
由 $\boldsymbol{A}\boldsymbol{X} = \boldsymbol{0}$ 和 $\boldsymbol{B}\boldsymbol{X} = \boldsymbol{0}$ 有 $(\boldsymbol{A} + \boldsymbol{B})\boldsymbol{X} = \boldsymbol{0}$. 从而
\[
N(\boldsymbol{A}) \cap N(\boldsymbol{B}) \subseteq N(\boldsymbol{A} + \boldsymbol{B}).
\]
显然
\[
N(\boldsymbol{A}) \subseteq N(\boldsymbol{BA}),
\]
\[
N(\boldsymbol{B}) \subseteq N(\boldsymbol{AB}).
\]
又 $\boldsymbol{AB} = \boldsymbol{BA}$, 因此 $N(\boldsymbol{BA}) = N(\boldsymbol{AB})$. 进而 $N(\boldsymbol{A}) + N(\boldsymbol{B}) \subseteq N(\boldsymbol{AB})$. 所以
\begin{align*}
n - \mathrm{rank}(\boldsymbol{AB}) &= \dim (N(\boldsymbol{AB})) \geqslant \dim (N(\boldsymbol{A}) + N(\boldsymbol{B})) \\
&= \dim (N(\boldsymbol{A})) + \dim (N(\boldsymbol{B})) - \dim (N(\boldsymbol{A}) \cap N(\boldsymbol{B})) \\
&\geqslant (\dim (N(\boldsymbol{A})) + \dim (N(\boldsymbol{B}))) - \dim (N(\boldsymbol{A} + \boldsymbol{B})) \\
&= (n - \mathrm{rank}(\boldsymbol{A})) + (n - \mathrm{rank}(\boldsymbol{B})) - (n - \mathrm{rank}(\boldsymbol{A} + \boldsymbol{B})) \\
&= n - (\mathrm{rank}(\boldsymbol{A}) + \mathrm{rank}(\boldsymbol{B})) + \mathrm{rank}(\boldsymbol{A} + \boldsymbol{B}),
\end{align*}
故 $\mathrm{rank}(\boldsymbol{A} + \boldsymbol{B}) \leqslant \mathrm{rank}(\boldsymbol{A}) + \mathrm{rank}(\boldsymbol{B}) - \mathrm{rank}(\boldsymbol{AB})$.
\end{proof}

\begin{example}
设 $\boldsymbol{A}$ 为 $n$ 阶反对称实矩阵. 证明关于 $\boldsymbol{X}$ 的矩阵方程 $\boldsymbol{AX} = \boldsymbol{X}$ 只有零解.
\end{example}
\begin{proof}
{\color{blue}证法一:}注意到
\begin{align*}
&\quad \quad \boldsymbol{AX}=\boldsymbol{X}\text{只有零解}
\\
&\Longleftrightarrow \left( \boldsymbol{A}-\boldsymbol{I}_n \right) \boldsymbol{X}=\boldsymbol{O}\text{只有零解}
\\
&\Longleftrightarrow 1\text{不是是}\boldsymbol{A}\text{的特征值}.
\end{align*}
设1是$\boldsymbol{A}$的特征值,$\boldsymbol{\alpha }$为对应特征向量,则$\boldsymbol{A\alpha }=\boldsymbol{\alpha }$.
又由\refpro{proposition:反对称阵的刻画}知$\boldsymbol{\alpha }^{\boldsymbol{T}}\boldsymbol{A\alpha }=0.$故$\boldsymbol{\alpha }^{\boldsymbol{T}}\boldsymbol{A\alpha }=\boldsymbol{\alpha }^{\boldsymbol{T}}\boldsymbol{\alpha }=0,$因此$\boldsymbol{\alpha }=0.$这与$\boldsymbol{\alpha }$为特征向量矛盾!

{\color{blue}证法二:}
由 $\boldsymbol{AX} = \boldsymbol{X}$ 得 $\boldsymbol{X}^\mathrm{T}\boldsymbol{A}\boldsymbol{X} = \boldsymbol{X}^\mathrm{T}\boldsymbol{X}$, 于是
\begin{align*}
\mathrm{tr}\left( \boldsymbol{X}^\mathrm{T}\boldsymbol{X} \right) &= \mathrm{tr}\left( \boldsymbol{X}^\mathrm{T}\boldsymbol{A}\boldsymbol{X} \right) = \mathrm{tr}\left( \left( \boldsymbol{X}^\mathrm{T}\boldsymbol{A}\boldsymbol{X} \right)^\mathrm{T} \right) \\
&= \mathrm{tr}\left( \boldsymbol{X}^\mathrm{T}\boldsymbol{A}^\mathrm{T}\boldsymbol{X} \right) = -\mathrm{tr}\left( \boldsymbol{X}^\mathrm{T}\boldsymbol{A}\boldsymbol{X} \right) = -\mathrm{tr}\left( \boldsymbol{X}^\mathrm{T}\boldsymbol{X} \right),
\end{align*}
从而 $\mathrm{tr}\left( \boldsymbol{X}^\mathrm{T}\boldsymbol{X} \right) = 0$, 所以由\refpro{proposition:矩阵与其转置乘积的迹}知 $\boldsymbol{X} = \boldsymbol{O}$.
\end{proof}

\begin{example}
设 $\boldsymbol{A}$ 为 $n$ 阶实矩阵, 且 $\boldsymbol{E} - \boldsymbol{A}$ 的特征值的模均小于 $1$. 证明 $0 < |\boldsymbol{A}| < 2^n$.
\end{example}
\begin{proof}
设 $\boldsymbol{A}$ 的复特征值为 $\lambda_1, \lambda_2, \cdots, \lambda_n$, 则 $\boldsymbol{E} - \boldsymbol{A}$ 的特征值为 $1 - \lambda_1, 1 - \lambda_2, \cdots, 1 - \lambda_n$. 故对任意 $i$,
\[
|1 - \lambda_i| < 1.
\]
若 $\lambda_i$ 为实数, 由 $|1 - \lambda_i| < 1$ 得 $-1 < 1 - \lambda_i < 1$, 即 $0 < \lambda_i < 2$.
若 $\lambda_i$ 为非实数, 则 $\overline{\lambda_i}$ 也是 $\boldsymbol{A}$ 的特征值. 此时
\[
|\lambda_i| = |1 - (1 - \lambda_i)| \leqslant 1 + |1 - \lambda_i| < 2.
\]
这时
\[
\lambda_i \overline{\lambda_i} = |\lambda_i|^2 < 2^2.
\]
由 $|\boldsymbol{A}| = \lambda_1 \lambda_2 \cdots \lambda_n$, 可得 $0 < |\boldsymbol{A}| < 2^n$.
\end{proof}

\begin{example}
设 $\boldsymbol{A},\boldsymbol{B}$ 分别为 $m$ 阶和 $n$ 阶方阵, $\boldsymbol{C}$ 为 $m \times n$ 矩阵, 且满足
\[
\boldsymbol{AC} = \boldsymbol{CB}, \quad \mathrm{rank}(\boldsymbol{C}) = r.
\]
证明 $\boldsymbol{A}$ 与 $\boldsymbol{B}$ 至少有 $r$ 个相同的特征值 (包括重数).
\end{example}
\begin{proof}
由于 $\mathrm{rank}(\boldsymbol{C}) = r$, 故存在 $m$ 阶可逆矩阵 $\boldsymbol{P}$, $n$ 阶可逆矩阵 $\boldsymbol{Q}$, 使得
\[
\boldsymbol{PCQ} = \begin{pmatrix} 
\boldsymbol{E}_r & \boldsymbol{O} \\
\boldsymbol{O} & \boldsymbol{O} 
\end{pmatrix}.
\]
由 $\boldsymbol{AC} = \boldsymbol{CB}$ 得到
\[
\boldsymbol{PAP}^{-1}\boldsymbol{PCQQ}^{-1} = \boldsymbol{PAC} = \boldsymbol{PCB} = \boldsymbol{PCQQ}^{-1}\boldsymbol{B},
\]
即
\[
\boldsymbol{PAP}^{-1} \begin{pmatrix} 
\boldsymbol{E}_r & \boldsymbol{O} \\
\boldsymbol{O} & \boldsymbol{O} 
\end{pmatrix} \boldsymbol{Q}^{-1} = \begin{pmatrix} 
\boldsymbol{E}_r & \boldsymbol{O} \\
\boldsymbol{O} & \boldsymbol{O} 
\end{pmatrix} \boldsymbol{Q}^{-1}\boldsymbol{B},
\]
或
\[
\boldsymbol{PAP}^{-1} \begin{pmatrix} 
\boldsymbol{E}_r & \boldsymbol{O} \\
\boldsymbol{O} & \boldsymbol{O} 
\end{pmatrix} = \begin{pmatrix} 
\boldsymbol{E}_r & \boldsymbol{O} \\
\boldsymbol{O} & \boldsymbol{O} 
\end{pmatrix} \boldsymbol{Q}^{-1}\boldsymbol{BQ}.
\]
令
\[
\boldsymbol{H} = \bordermatrix{%
& r & m-r \cr
r & \boldsymbol{H}_{11}& \boldsymbol{H}_{12}  \cr
m-r & \boldsymbol{H}_{21}& \boldsymbol{H}_{22}  \cr
}, \quad \boldsymbol{T} = \boldsymbol{Q}^{-1}\boldsymbol{BQ} = \bordermatrix{%
& r & n-r \cr
r & \boldsymbol{T}_{11}& \boldsymbol{T}_{12}  \cr
n-r & \boldsymbol{T}_{21}& \boldsymbol{T}_{22}  \cr
}
\]
则有
\[
\begin{pmatrix} 
\boldsymbol{H}_{11} & \boldsymbol{H}_{12} \\
\boldsymbol{H}_{21} & \boldsymbol{H}_{22} 
\end{pmatrix} \begin{pmatrix} 
\boldsymbol{E}_r & \boldsymbol{O} \\
\boldsymbol{O} & \boldsymbol{O} 
\end{pmatrix} = \begin{pmatrix} 
\boldsymbol{E}_r & \boldsymbol{O} \\
\boldsymbol{O} & \boldsymbol{O} 
\end{pmatrix} \begin{pmatrix} 
\boldsymbol{T}_{11} & \boldsymbol{T}_{12} \\
\boldsymbol{T}_{21} & \boldsymbol{T}_{22} 
\end{pmatrix},
\]
即
\[
\begin{pmatrix} 
\boldsymbol{H}_{11} & \boldsymbol{O} \\
\boldsymbol{H}_{21} & \boldsymbol{O} 
\end{pmatrix} = \begin{pmatrix} 
\boldsymbol{T}_{11} & \boldsymbol{T}_{12} \\
\boldsymbol{O} & \boldsymbol{O} 
\end{pmatrix},
\]
故有 $\boldsymbol{H}_{11} = \boldsymbol{T}_{11}, \boldsymbol{T}_{12} = \boldsymbol{O}, \boldsymbol{H}_{21} = \boldsymbol{O}$. 结果
\[
\boldsymbol{H} = \begin{pmatrix} 
\boldsymbol{H}_{11} & \boldsymbol{H}_{12} \\
\boldsymbol{O} & \boldsymbol{H}_{22} 
\end{pmatrix}, \quad \boldsymbol{T} = \begin{pmatrix} 
\boldsymbol{H}_{11} & \boldsymbol{O} \\
\boldsymbol{T}_{21} & \boldsymbol{T}_{22} 
\end{pmatrix}.
\]
这里用到 $\boldsymbol{H}_{11}$ 有 $r$ 个特征值,这对任意数域不一定成立. 但显然复数域上的矩阵 $\boldsymbol{H}_{11}$ 有$r$ 个特征值,且这$r$个特征值是 $\boldsymbol{H}$ 与 $\boldsymbol{T}$ 的公共特征值. 注意到 $\boldsymbol{A}$ 与 $\boldsymbol{H}$ 相似, $\boldsymbol{B}$ 与 $\boldsymbol{T}$ 相似. 故 $\boldsymbol{A}$ 与 $\boldsymbol{H}$ 有相同的特征值, $\boldsymbol{B}$ 与 $\boldsymbol{T}$ 有相同的特征值. 因此 $\boldsymbol{A}$ 与 $\boldsymbol{B}$ 至少有 $r$ 个相同的特征值.
\end{proof}

\begin{example}
计算
\[
\begin{pmatrix} 
1 & 0 & 0 \\
0 & 1 & 1 \\
0 & 0 & 1 
\end{pmatrix}^{2023} 
\begin{pmatrix} 
1 & 2 & 3 \\
4 & 5 & 6 \\
7 & 8 & 9 
\end{pmatrix} 
\begin{pmatrix} 
0 & 0 & 1 \\
0 & 1 & 0 \\
1 & 0 & 0 
\end{pmatrix}^{2000}.
\]
\end{example}
\begin{solution}
注意到
\[
\begin{pmatrix} 
1 & 0 & 0 \\
0 & 1 & 1 \\
0 & 0 & 1 
\end{pmatrix}
\quad \text{和} \quad
\begin{pmatrix} 
0 & 0 & 1 \\
0 & 1 & 0 \\
1 & 0 & 0 
\end{pmatrix}
\]
都是初等矩阵. 前一个在
\[
\begin{pmatrix} 
1 & 2 & 3 \\
4 & 5 & 6 \\
7 & 8 & 9 
\end{pmatrix}
\]
左边相乘, 即对
\[
\begin{pmatrix} 
1 & 2 & 3 \\
4 & 5 & 6 \\
7 & 8 & 9 
\end{pmatrix}
\]
作初等行变换: 把第 3 行加到第 2 行.

后一个在
\[
\begin{pmatrix} 
1 & 2 & 3 \\
4 & 5 & 6 \\
7 & 8 & 9 
\end{pmatrix}
\]
右边相乘, 即对矩阵
\[
\begin{pmatrix} 
1 & 2 & 3 \\
4 & 5 & 6 \\
7 & 8 & 9 
\end{pmatrix}
\]
作初等列变换: 交换第 1,3 两列.

所以最后结果是把矩阵
\[
\begin{pmatrix} 
1 & 2 & 3 \\
4 & 5 & 6 \\
7 & 8 & 9 
\end{pmatrix}
\]
第 3 行加到第 2 行 2023 次, 再把得到的矩阵第 1,3 列互换 2000 次. 故
\[
\text{原式} = \begin{pmatrix} 
1 & 2 & 3 \\
4 + 2023 \times 7 & 5 + 2023 \times 8 & 6 + 2023 \times 9 \\
7 & 8 & 9 
\end{pmatrix} = \begin{pmatrix} 
1 & 2 & 3 \\
14165 & 16189 & 18213 \\
7 & 8 & 9 
\end{pmatrix}.
\]
\end{solution}

\begin{example}
【例 1.2.38】 设
\[
\boldsymbol{A} = \begin{pmatrix} 
1 & 0 & 1 \\
4 & 1 & 3 \\
-3 & -1 & -2 
\end{pmatrix}, \quad \boldsymbol{B} = \begin{pmatrix} 
1 & 0 & 1 \\
0 & 1 & -1 \\
0 & 0 & 0 
\end{pmatrix}.
\]

问: (1) 是否存在可逆矩阵 $\boldsymbol{P}$ 使得 $\boldsymbol{PA} = \boldsymbol{B}$. 若存在, 则求出一个这样的 $\boldsymbol{P}$.

(2) 求满足 $\boldsymbol{XA} = \boldsymbol{B}$ 的所有三阶矩阵 $\boldsymbol{X}$.
\end{example}
\begin{solution}
解 (1) $\boldsymbol{PA} = \boldsymbol{B}$ 等价于对 $\boldsymbol{A}$ 施行一系列初等行变换得到 $\boldsymbol{B}$. 因此
\[
\boldsymbol{A} = \begin{pmatrix} 
1 & 0 & 1 \\
4 & 1 & 3 \\
-3 & -1 & -2 
\end{pmatrix} \xrightarrow{\begin{array}{c}
r_2-4r_1\\
r_3+3r_1\\
\end{array}}\begin{pmatrix} 
1 & 0 & 1 \\
0 & 1 & -1 \\
0 & -1 & 1 
\end{pmatrix} \\
\xrightarrow{r_3 + r_2} \begin{pmatrix} 
1 & 0 & 1 \\
0 & 1 & -1 \\
0 & 0 & 0 
\end{pmatrix} = \boldsymbol{B},
\]
所以存在可逆矩阵 $\boldsymbol{P}$ 使得 $\boldsymbol{PA} = \boldsymbol{B}$.

此时可求出一个 $\boldsymbol{P}$ 为
\[
\boldsymbol{P} = \begin{pmatrix} 
1 & 0 & 0 \\
0 & 1 & 0 \\
0 & 1 & 1 
\end{pmatrix} \begin{pmatrix} 
1 & 0 & 0 \\
0 & 1 & 0 \\
3 & 0 & 1 
\end{pmatrix} \begin{pmatrix} 
1 & 0 & 0 \\
-4 & 1 & 0 \\
0 & 0 & 1 
\end{pmatrix} = \begin{pmatrix} 
1 & 0 & 0 \\
-4 & 1 & 0 \\
-1 & 1 & 1 
\end{pmatrix}.
\]

(2) 将 $\boldsymbol{XA} = \boldsymbol{B}$ 变形为 $\boldsymbol{A}^\mathrm{T}\boldsymbol{X}^\mathrm{T} = \boldsymbol{B}^\mathrm{T}$.
\begin{align*}
\left( \boldsymbol{A}^{\mathrm{T}}\mid \boldsymbol{B}^{\mathrm{T}} \right) &=\left( \begin{matrix}
1&		4&		-3&		1&		0&		0\\
0&		1&		-1&		0&		1&		0\\
1&		3&		-2&		1&		-1&		0\\
\end{matrix} \right) \rightarrow \left( \begin{matrix}
1&		4&		-3&		1&		0&		0\\
0&		1&		-1&		0&		1&		0\\
0&		-1&		1&		0&		-1&		0\\
\end{matrix} \right) 
\\
&\rightarrow \left( \begin{matrix}
1&		4&		-3&		1&		0&		0\\
0&		1&		-1&		0&		1&		0\\
0&		0&		0&		0&		0&		0\\
\end{matrix} \right) \rightarrow \left( \begin{matrix}
1&		0&		1&		1&		-4&		0\\
0&		1&		-1&		0&		1&		0\\
0&		0&		0&		0&		0&		0\\
\end{matrix} \right) .
\end{align*}
$\boldsymbol{A}^\mathrm{T}\boldsymbol{X}^\mathrm{T} = \boldsymbol{O}$ 的基础解系为 $\begin{cases} x_1 = -x_3, \\ x_2 = x_3, \end{cases}$ 即 $\begin{pmatrix} -k \\ k \\ k \end{pmatrix}$.

令 $\boldsymbol{X}^\mathrm{T} = \left( \boldsymbol{\beta}_1, \boldsymbol{\beta}_2, \boldsymbol{\beta}_3 \right)$, 则
\[
\boldsymbol{\beta}_1 = \begin{pmatrix} 1 - k_1 \\ k_1 \\ k_1 \end{pmatrix}, \quad \boldsymbol{\beta}_2 = \begin{pmatrix} -4 - k_2 \\ 1 + k_2 \\ k_2 \end{pmatrix}, \quad \boldsymbol{\beta}_3 = \begin{pmatrix} -k_3 \\ k_3 \\ k_3 \end{pmatrix}.
\]
从而
\[
\boldsymbol{X} = \begin{pmatrix} 
\boldsymbol{\beta}_1^\mathrm{T} \\
\boldsymbol{\beta}_2^\mathrm{T} \\
\boldsymbol{\beta}_3^\mathrm{T} 
\end{pmatrix} = \begin{pmatrix} 
1 - k_1 & k_1 & k_1 \\
-4 - k_2 & 1 + k_2 & k_2 \\
-k_3 & k_3 & k_3 
\end{pmatrix},
\]
其中 $k_1, k_2, k_3$ 为任意数.
取 $k_1 = k_2 = 0, k_3 = 1$ 即得 (1) 中的 $\boldsymbol{P}$.
\end{solution}

\begin{example}
设 $a_1, a_2, \cdots, a_n$ 为数域 $\mathbb{F}$ 上的 $n$ 个数,
\[
\boldsymbol{A} = \begin{pmatrix} 
0 & & & -a_n \\
1 & \ddots & & \vdots \\
& \ddots & \ddots & -a_2 \\
& & 1 & -a_1 
\end{pmatrix}.
\]
试求 $\boldsymbol{A}$ 的不变因子及Smith(史密斯)标准形.
\end{example}
\begin{solution}
$$\lambda\boldsymbol{E}-\boldsymbol{A}=\begin{pmatrix}
\lambda& & & a_n\\
-1& \ddots& & \vdots\\
& \ddots& \ddots& a_2\\
& & -1& \lambda +a_1\\
\end{pmatrix}\xrightarrow[\substack{\cdots \cdots\\\lambda ^{n-1}r_n+r_1}]{\substack{\lambda r_2+r_1\\\lambda ^2r_3+r_1}}\begin{pmatrix}
0& & & f\left( \lambda \right)\\
-1& \lambda& & \vdots\\
& \ddots& \ddots& a_2\\
& & -1& \lambda +a_1\\
\end{pmatrix}$$
或
$$\lambda\boldsymbol{E}-\boldsymbol{A}=\begin{pmatrix}
\lambda& & & a_n\\
-1& \ddots& & \vdots\\
& \ddots& \ddots& a_2\\
& & -1& \lambda +a_1\\
\end{pmatrix}\xrightarrow[\substack{\cdots \cdots\\\lambda r_2+r_1}]{\substack{\lambda r_n+r_1\\\lambda r_{n-1}+r_1}}\begin{pmatrix}
0& & & f\left( \lambda \right)\\
-1& 0& & \vdots\\
& \ddots& \ddots& \lambda ^2+a_1\lambda +a_2\\
& & -1& \lambda +a_1\\
\end{pmatrix},$$
其中 $f(\lambda) = \lambda^n + a_1\lambda^{n-1} + \cdots + a_n$. 于是 $\det(\lambda\boldsymbol{E} - \boldsymbol{A}) = f(\lambda)$, 从而行列式因子为
\[
D_n(\lambda) = f(\lambda), \quad D_{n-1}(\lambda) = 1, \quad \cdots, \quad D_0(\lambda) = 1.
\]
因此
\[
d_n(\lambda) = \frac{D_n(\lambda)}{D_{n-1}(\lambda)} = f(\lambda), \quad d_{n-1}(\lambda) = \cdots = d_1(\lambda) = 1.
\]
故 $\boldsymbol{A}$ 的 Smith 标准形为
\[
\begin{pmatrix} 
1 & & & \\
& \ddots & & \\
& & 1 & \\
& & & f(\lambda) 
\end{pmatrix},
\]
不变因子为
\[
1, 1, \cdots, f(\lambda).
\]
\end{solution}

\begin{example}
设 $\boldsymbol{B}$ 为 $p$ 阶复方阵,
\[
\boldsymbol{B} = \begin{pmatrix} 
a_0 & a_1 & \cdots & \cdots & a_{p-1} \\
& a_0 & a_1 & \cdots & a_{p-2} \\
& & \ddots & \ddots & \vdots \\
& & & a_0 & a_1 \\
& & & & a_0 
\end{pmatrix}.
\]
试求 $\boldsymbol{B}$ 的初等因子.
\end{example}
\begin{solution}
显然, $\boldsymbol{B}$ 的特征多项式为 $(\lambda - a_0)^p$, $\boldsymbol{B}$ 的特征值都等于 $a_0$.

(1) 当 $a_1 \neq 0$ 时, $\mathrm{rank}(\boldsymbol{B} - a_0\boldsymbol{E}) = p - 1$, 特征值 $a_0$ 的几何重数为 1. 故 $\boldsymbol{B}$ 的 Jordan 标准形中只能有一个 Jordan 块, 即 $\boldsymbol{B}$ 只有一个初等因子 $(\lambda - a_0)^p$.

(2) 当 $a_1 = a_2 = \cdots = a_{k-1} = 0, a_k \neq 0$ ,$k=2,3,\cdots,p-1$时, 为找出 $\boldsymbol{B}$ 的所有初等因子, 我们先来确定 $\boldsymbol{B}$ 的 Jordan 标准形中 Jordan 块的形状和个数.

注意到
\[
\boldsymbol{B} = a_0\boldsymbol{E} + a_1\boldsymbol{H}^1 + \cdots + a_{p-1}\boldsymbol{H}^{p-1}, \quad \boldsymbol{H} = \begin{pmatrix} 
0 & 1 & & \\
& 0 & 1 & \\
& & \ddots & \ddots \\
& & & 0 & 1 \\
& & & & 0 
\end{pmatrix},
\]
\[
(\boldsymbol{B} - a_0\boldsymbol{E})^j = a_k^j\boldsymbol{H}^{kj} + \text{若干 } \boldsymbol{H} \text{ 的更高幂次项}.
\]
所以
\begin{align}
\mathrm{rank}(\boldsymbol{B} - a_0\boldsymbol{E})^j = \begin{cases} 
0, & kj > p, \\
p - kj, & kj \leqslant p.
\end{cases}\label{eq::::::--000002}
\end{align}

{\color{blue}解法一:}记$N_j$为矩阵$\boldsymbol{B}$的Jordan标准型$J$中$j(1 \leqslant j \leqslant p)$阶Jordan块的个数,也即$\boldsymbol{B}$的初等因子中$(\lambda - \lambda_0)^j$的个数,则由\refthe{theorem:Jordan块的个数与秩的关系}可知
$$N_j = \mathrm{rank}(\boldsymbol{B} - a_0\boldsymbol{E})^{j+1} + \mathrm{rank}(\boldsymbol{B} - a_0\boldsymbol{E})^{j-1} - 2\mathrm{rank}(\boldsymbol{B} - a_0\boldsymbol{E})^j, j = 1,2,\cdots,p.$$
\one 当$p \leqslant kj - k$或$p \geqslant kj + k$时,由$(1)$式,此时有
$$N_j = 0, j \in \{j = 1,2,\cdots,p | p \leqslant kj - k \text{ 或 } p \geqslant kj + k\}.$$
\two 当$kj - k < p < kj + k$时,由式,此时有
\begin{align}
N_j = p - kj + k - 2\mathrm{rank}(\boldsymbol{B} - a_0\boldsymbol{E})^j, j \in \{j = 1,2,\cdots,p | kj - k < p < kj + k\}.\label{107.50}
\end{align}
由带余除法可知,存在$q,h \in \mathbb{N}$,且$0 \leqslant h < k$,使得
\begin{align}
p = qk + h.\label{107.51}
\end{align}
$\left( \mathrm{i} \right)$若$p < kj$,则此时有
\begin{align*}
&\quad \quad \,\, kj-k<p<kj,j\in \mathbb{N} 
\\
&\Longleftrightarrow q\leqslant q+\frac{h}{k}<j<q+1+\frac{h}{k}<q+2,j\in \mathbb{N} 
\\
&\Longleftrightarrow j=q+1.
\end{align*}
于是由\eqref{eq::::::--000002}\eqref{107.50}\eqref{107.51}式可得
$$N_{q+1} = p - k(q + 1) + k = h.$$
$\left( \mathrm{ii} \right)$若$p \geqslant kj$,则此时有
\begin{align*}
&\quad \quad \,\, kj\leqslant p<kj+k,j\in \mathbb{N} 
\\
&\Longleftrightarrow q-1\leqslant q-1+\frac{h}{k}<j\leqslant q+\frac{h}{k}<q+1,j\in \mathbb{N} 
\\
&\Longleftrightarrow j=q.
\end{align*}
于是由\eqref{eq::::::--000002}\eqref{107.50}\eqref{107.51}式可得
$$N_q = p - kq + k - 2(p - kq - k) = k - h.$$
综上可知,当 $a_1 = a_2 = \cdots = a_{k-1} = 0, a_k \neq 0$,$k=2,3,\cdots,p-1$ 时 $\boldsymbol{B}$ 的初等因子有: $k - h$ 个 $(\lambda - a_0)^q$, $h$ 个 $(\lambda - a_0)^{q+1}$.

{\color{blue}解法二:}
由\eqref{eq::::::--000002}式可知
\[
d_j = p - \mathrm{rank}(\boldsymbol{B} - a_0\boldsymbol{E})^j = \min\{p, kj\}.
\]
记 $p = qk + h(0 \leqslant h < k)$ (带余除法), 则
\[
d_1 = k, d_2 = 2k, \cdots, d_q = qk, d_{q+1} = d_{q+2} = \cdots = p.
\]
现假设 $\boldsymbol{B}$ 的初等因子有 $g_1$ 个 $(\lambda - a_0), g_2$ 个 $(\lambda - a_0)^2, \cdots, g_m$ 个 $(\lambda - a_0)^m(g_m \geqslant 1)$. 对此 $\boldsymbol{B}$ 有相应的 Jordan 标准形. 从 $\boldsymbol{B}$ 的这个 Jordan 标准形又可定义出各个 $d_j = p - \mathrm{rank}(\boldsymbol{B} - a_0\boldsymbol{E})^j$ 的值:
\[
\begin{cases} 
d_1 = g_1 + g_2 + g_3 + \cdots + g_m, \\
d_2 = g_1 + 2g_2 + 2g_3 + \cdots + 2g_m, \\
d_3 = g_1 + 2g_2 + 3g_3 + \cdots + 3g_m, \\
\cdots \cdots \\
d_m = g_1 + 2g_2 + 3g_3 + \cdots + mg_m, \\
d_{m+1} = d_{m+2} = \cdots = d_p.
\end{cases}
\]
将这组 $d_j$ 的值与前面已经得到的值进行比较, 可得
\begin{gather*}
m=q+1,
\\
g_1=2d_1-d_2=2k-2k=0,
\\
g_2=2d_2-d_1-d_3=2(2k)-k-3k=0,
\\
\cdots \cdots \cdots \cdots 
\\
g_{q-1}=2d_{q-1}-d_{q-2}-d_q=2(q-1)k-(q-2)k-qk=0,
\\
g_q=2d_q-d_{q-1}-d_{q+1}=2qk-(q-1)k-p=k-h,
\\
g_{q+1}=2d_{q+1}-d_q-d_{q+2}=2p-qk-p=h,
\\
g_{q+2}=g_{q+3}=\cdots =0.
\end{gather*}
所以当 $a_1 = a_2 = \cdots = a_{k-1} = 0, a_k \neq 0$,$k=2,3,\cdots,p-1$ 时 $\boldsymbol{B}$ 的初等因子有: $k - h$ 个 $(\lambda - a_0)^q$, $h$ 个 $(\lambda - a_0)^{q+1}$.

(3) 最后, 显然当 $a_1 = a_2 = \cdots = a_{p-1} = 0$ 时,$\boldsymbol{B}$是纯量阵,此时 $\boldsymbol{B}$ 有 $p$ 个初等因子, 它们都是 $\lambda - a_0$.
\end{solution}

\begin{example}
设 \( n \) 阶实方阵 \( \boldsymbol{A},\boldsymbol{B} \) 满足 \( \mathrm{rank}(\boldsymbol{ABA}) = \mathrm{rank}(\boldsymbol{B}) \)。证明: \( \boldsymbol{AB} \) 与 \( \boldsymbol{BA} \) 相似.
\end{example}
\begin{proof}
设 \( \mathrm{rank}(\boldsymbol{A}) = r \),且设
\[
\boldsymbol{A} = \boldsymbol{P} \begin{pmatrix} \boldsymbol{E}_r & \boldsymbol{O} \\ \boldsymbol{O} & \boldsymbol{O} \end{pmatrix} \boldsymbol{Q}, \quad \boldsymbol{B} = \boldsymbol{Q}^{-1} \begin{pmatrix} \boldsymbol{B}_1 & \boldsymbol{B}_2 \\ \boldsymbol{B}_3 & \boldsymbol{B}_4 \end{pmatrix} \boldsymbol{P}^{-1},
\]
其中 \( \boldsymbol{P},\boldsymbol{Q} \) 皆为可逆方阵,\( \boldsymbol{B}_1 \) 为 \( r \) 阶方阵. 则有
\[
\boldsymbol{AB} = \boldsymbol{P} \begin{pmatrix} \boldsymbol{B}_1 & \boldsymbol{B}_2 \\ \boldsymbol{O} & \boldsymbol{O} \end{pmatrix} \boldsymbol{P}^{-1},
\]
\[
\boldsymbol{BA} = \boldsymbol{Q}^{-1} \begin{pmatrix} \boldsymbol{B}_1 & \boldsymbol{O} \\ \boldsymbol{B}_3 & \boldsymbol{O} \end{pmatrix} \boldsymbol{Q}, \quad \boldsymbol{ABA} = \boldsymbol{P} \begin{pmatrix} \boldsymbol{B}_1 & \boldsymbol{O} \\ \boldsymbol{O} & \boldsymbol{O} \end{pmatrix} \boldsymbol{Q}.
\]

由题设知
\[
\mathrm{rank}(\boldsymbol{B}) = \mathrm{rank}(\boldsymbol{ABA}) = \mathrm{rank}(\boldsymbol{B}_1),
\]
故由\refcor{corollary:分块矩阵的秩与单个块的秩相同则矩阵方阵有解}知,存在矩阵 \( \boldsymbol{X}, \boldsymbol{Y} \) 使得
\[
\boldsymbol{B}_2 = \boldsymbol{B}_1\boldsymbol{X}, \quad \boldsymbol{B}_3 = \boldsymbol{Y}\boldsymbol{B}_1,
\]
从而有
\[
\boldsymbol{AB} = \boldsymbol{P} \begin{pmatrix} \boldsymbol{E} & -\boldsymbol{X} \\ \boldsymbol{O} & \boldsymbol{E} \end{pmatrix} \begin{pmatrix} \boldsymbol{B}_1 & \boldsymbol{O} \\ \boldsymbol{O} & \boldsymbol{O} \end{pmatrix} \begin{pmatrix} \boldsymbol{E} & \boldsymbol{X} \\ \boldsymbol{O} & \boldsymbol{E} \end{pmatrix} \boldsymbol{P}^{-1},
\]
\[
\boldsymbol{BA} = \boldsymbol{Q}^{-1} \begin{pmatrix} \boldsymbol{E} & \boldsymbol{O} \\ \boldsymbol{Y} & \boldsymbol{E} \end{pmatrix} \begin{pmatrix} \boldsymbol{B}_1 & \boldsymbol{O} \\ \boldsymbol{O} & \boldsymbol{O} \end{pmatrix} \begin{pmatrix} \boldsymbol{E} & \boldsymbol{O} \\ -\boldsymbol{Y} & \boldsymbol{O} \end{pmatrix} \boldsymbol{Q},
\]
\[
\boldsymbol{AB} \sim \begin{pmatrix} \boldsymbol{B}_1 & \boldsymbol{O} \\ \boldsymbol{O} & \boldsymbol{O} \end{pmatrix} \sim \boldsymbol{BA}.
\]
\end{proof}

\begin{example}
设
\[
\boldsymbol{B} = \begin{pmatrix} 0 & 2022 & 2023 \\ 0 & 0 & 2024 \\ 0 & 0 & 0 \end{pmatrix}.
\]
证明矩阵方程 \( \boldsymbol{X}^2 = \boldsymbol{B} \) 无解, 这里 \( \boldsymbol{X} \) 为三阶未知复方阵.
\end{example}
\begin{proof}
反证法. 设矩阵方程有解, 即存在复矩阵 \( \boldsymbol{A} \) 使得 \( \boldsymbol{A}^2 = \boldsymbol{B} \). 注意到 \( \boldsymbol{B} \) 的特征值全为 \( 0 \), 所以 \( \boldsymbol{A} \) 的特征值全为 \( 0 \), 因此 \( \boldsymbol{A} \) 的 Jordan 标准形 \( \boldsymbol{J} \) 只能是下列 \( 4 \) 种情形之一:
\[
\boldsymbol{J} = \begin{pmatrix} 0 & 0 & 0 \\ 0 & 0 & 0 \\ 0 & 0 & 0 \end{pmatrix}, \quad \boldsymbol{J} = \begin{pmatrix} 0 & 1 & 0 \\ 0 & 0 & 0 \\ 0 & 0 & 0 \end{pmatrix},
\]
\[
\boldsymbol{J} = \begin{pmatrix} 0 & 0 & 0 \\ 0 & 0 & 1 \\ 0 & 0 & 0 \end{pmatrix}, \quad \boldsymbol{J} = \begin{pmatrix} 0 & 1 & 0 \\ 0 & 0 & 1 \\ 0 & 0 & 0 \end{pmatrix}.
\]
因此, \( \mathrm{rank}(\boldsymbol{B}) = \mathrm{rank}(\boldsymbol{A}^2) = \mathrm{rank}(\boldsymbol{J}^2) \leqslant 1 \), 但题设矩阵 \( \boldsymbol{B} \) 的秩显然等于 \( 2 \), 矛盾.
\end{proof}

\begin{example}
设实对称矩阵
\[
\boldsymbol{A} = \begin{pmatrix} 2 & 0 & 0 \\ 0 & 0 & 1 \\ 0 & 1 & x \end{pmatrix}
\quad \text{与} \quad 
\boldsymbol{B} = \begin{pmatrix} 2 &  &  \\  & y &  \\  &  & -1 \end{pmatrix}
\]
相似, 求 \( x,y \) 及满足 \( \boldsymbol{Q}^{-1}\boldsymbol{AQ} = \boldsymbol{B} \) 的正交矩阵 \( \boldsymbol{Q} \).
\end{example}
\begin{solution}
\( \boldsymbol{A} \) 与 \( \boldsymbol{B} \) 相似 \( \Rightarrow \boldsymbol{A} \) 与 \( \boldsymbol{B} \) 有相同的迹和行列式. 进而得
\[
\begin{cases}
2 + x = 2 + y + (-1), \\
-2 = |\boldsymbol{A}| = -2y,
\end{cases}
\]
解得 \( \begin{cases} x = 0, \\ y = 1. \end{cases} \) 直接验证可知, 当 \( x = 0, y = 1 \) 时, \( \boldsymbol{A} \) 与 \( \boldsymbol{B} \) 相似. 接下来, 求出 \( \boldsymbol{Q} \).

显然, \( \boldsymbol{A} \) 的三个特征值为 \( 2, 1, -1 \).

相应于 \( \lambda_1 = 2 \), \( \boldsymbol{A} \) 的一个特征向量为 \( \boldsymbol{p}_1 = \begin{pmatrix} 1 \\ 0 \\ 0 \end{pmatrix} \), \( V_{\lambda_1} \) 中的单位向量为
\[
\begin{pmatrix} 1 \\ 0 \\ 0 \end{pmatrix}, \quad \begin{pmatrix} -1 \\ 0 \\ 0 \end{pmatrix}.
\]

相应于 \( \lambda_2 = 1 \), \( \boldsymbol{A} \) 的一个特征向量为 \( \boldsymbol{p}_2 = \begin{pmatrix} 0 \\ 1 \\ 1 \end{pmatrix} \), \( V_{\lambda_2} \) 中的单位向量为
\[
\begin{pmatrix} 0 \\ \dfrac{1}{\sqrt{2}} \\ \dfrac{1}{\sqrt{2}} \end{pmatrix}, \quad \begin{pmatrix} 0 \\ -\dfrac{1}{\sqrt{2}} \\ -\dfrac{1}{\sqrt{2}} \end{pmatrix}.
\]

相应于 \( \lambda_3 = -1 \), \( \boldsymbol{A} \) 的一个特征向量为 \( \boldsymbol{p}_3 = \begin{pmatrix} 0 \\ 1 \\ -1 \end{pmatrix} \), \( V_{\lambda_3} \) 中的单位向量为
\[
\begin{pmatrix} 0 \\ \dfrac{1}{\sqrt{2}} \\ -\dfrac{1}{\sqrt{2}} \end{pmatrix}, \quad \begin{pmatrix} 0 \\ -\dfrac{1}{\sqrt{2}} \\ \dfrac{1}{\sqrt{2}} \end{pmatrix}.
\]
因为$\boldsymbol{A}$是实对称阵,所以由\refcor{corollary:Hermite矩阵和实对称矩阵关于特征值的相关性质}知$\boldsymbol{A}$的特征值全是实数且属于不同特征值的特征向量互相正交.
因此, 所求的 \( \boldsymbol{Q} \) 共有 \( 8 \) 个, 为
$$
\begin{scriptsize}
\left\{ (\boldsymbol{q}_1, \boldsymbol{q}_2, \boldsymbol{q}_3) \,\big|\, \boldsymbol{q}_1 \in \left\{ \begin{pmatrix} 1 \\ 0 \\ 0 \end{pmatrix}, \begin{pmatrix} -1 \\ 0 \\ 0 \end{pmatrix} \right\}, \boldsymbol{q}_2 \in \left\{ \begin{pmatrix} 0 \\ \dfrac{1}{\sqrt{2}} \\ \dfrac{1}{\sqrt{2}} \end{pmatrix}, \begin{pmatrix} 0 \\ -\dfrac{1}{\sqrt{2}} \\ -\dfrac{1}{\sqrt{2}} \end{pmatrix} \right\}, \right.,
\left. \boldsymbol{q}_3 \in \left\{ \begin{pmatrix} 0 \\ \dfrac{1}{\sqrt{2}} \\ -\dfrac{1}{\sqrt{2}} \end{pmatrix}, \begin{pmatrix} 0 \\ -\dfrac{1}{\sqrt{2}} \\ \dfrac{1}{\sqrt{2}} \end{pmatrix} \right\} \right\}.
\end{scriptsize}
$$
\end{solution}

\begin{example}
设$\boldsymbol{A}$为复数域上的4阶幂等阵(即$\boldsymbol{A}^2 = \boldsymbol{A}$). 证明: 存在$c_0,c_1,c_2,c_3 \in \mathbb{C}$使得$\boldsymbol{A}$相似于
$$
\begin{pmatrix}
c_0 & c_1 & c_2 & c_3 \\
c_3 & c_0 & c_1 & c_2 \\
c_2 & c_3 & c_0 & c_1 \\
c_1 & c_2 & c_3 & c_0
\end{pmatrix}.
$$
\end{example}
\begin{note}
显然上述矩阵是一个循环矩阵,由\refpro{proposition:循环矩阵的特征值}可知其特征值,由此不难得到证明思路.
\end{note}
\begin{proof}
(1) 由\refpro{proposition:幂等矩阵的性质1}知,幂等阵可对角化.由$\boldsymbol{A}(\boldsymbol{A}-\boldsymbol{E}_4)=\boldsymbol{O}$知其特征值$\lambda_1,\lambda_2,\lambda_3,\lambda_4$要么为1, 要么为0, 因此有
$$
\boldsymbol{A} \sim \begin{pmatrix}
\lambda_1 & & & \\
& \lambda_2 & & \\
& & \lambda_3 & \\
& & & \lambda_4
\end{pmatrix}.
$$

(2) 令
$$
\boldsymbol{B} = \begin{pmatrix}
0 & 1 & 0 & 0 \\
0 & 0 & 1 & 0 \\
0 & 0 & 0 & 1 \\
1 & 0 & 0 & 0
\end{pmatrix},
$$
则有
$$
\boldsymbol{B}^2 = \begin{pmatrix}
0 & 0 & 1 & 0 \\
0 & 0 & 0 & 1 \\
1 & 0 & 0 & 0 \\
0 & 1 & 0 & 0
\end{pmatrix}, \quad \boldsymbol{B}^3 = \begin{pmatrix}
0 & 0 & 0 & 1 \\
1 & 0 & 0 & 0 \\
0 & 1 & 0 & 0 \\
0 & 0 & 1 & 0
\end{pmatrix}, \quad \boldsymbol{B}^4 = \boldsymbol{E}_4.
$$
由于$\boldsymbol{B}$的特征多项式$|\lambda \boldsymbol{E} - \boldsymbol{B}| = \lambda^4 - 1$, 故$\boldsymbol{B}$有4个各不相同的特征值: $\xi_1,\xi_2,\xi_3,\xi_4$, 进而
$$
\boldsymbol{B} \sim \begin{pmatrix}
\xi_1 & & & \\
& \xi_2 & & \\
& & \xi_3 & \\
& & & \xi_4
\end{pmatrix}.
$$
考察关于$a_0,a_1,a_2,a_3$的线性方程组:
$$
\begin{cases}
a_0 + a_1 \xi_1 + a_2 \xi_1^2 + a_3 \xi_1^3 = \lambda_1, \\
a_0 + a_1 \xi_2 + a_2 \xi_2^2 + a_3 \xi_2^3 = \lambda_2, \\
a_0 + a_1 \xi_3 + a_2 \xi_3^2 + a_3 \xi_3^3 = \lambda_3, \\
a_0 + a_1 \xi_4 + a_2 \xi_4^2 + a_3 \xi_4^3 = \lambda_4,
\end{cases}
$$
其系数行列式是Vandermonde行列式, 它显然不为零, 故有唯一解: $c_0,c_1,c_2,c_3$. 令
$$
f(x) = c_0 + c_1 x + c_2 x^2 + c_3 x^3,
$$
则有$f(\xi_i) = \lambda_i, \, i = 1,2,3,4$. 所以
$$
f(\boldsymbol{B}) \sim \begin{pmatrix}
f(\xi_1) & & & \\
& f(\xi_2) & & \\
& & f(\xi_3) & \\
& & & f(\xi_4)
\end{pmatrix} = \begin{pmatrix}
\lambda_1 & & & \\
& \lambda_2 & & \\
& & \lambda_3 & \\
& & & \lambda_4
\end{pmatrix} \sim \boldsymbol{A}.
$$
而
$$
f(\boldsymbol{B}) = c_0 \boldsymbol{E} + c_1 \boldsymbol{B} + c_2 \boldsymbol{B}^2 + c_3 \boldsymbol{B}^3 = \begin{pmatrix}
c_0 & c_1 & c_2 & c_3 \\
c_3 & c_0 & c_1 & c_2 \\
c_2 & c_3 & c_0 & c_1 \\
c_1 & c_2 & c_3 & c_0
\end{pmatrix},
$$
因此本题获证.
\end{proof}

\begin{example}
设 $\mathbb{F}$ 为数域, $\boldsymbol{A} \in M_n(\mathbb{F})$. 证明: $M_n(\mathbb{F})$ 中存在可逆的对称矩阵 $\boldsymbol{Q}$ 使得 $\boldsymbol{A}\boldsymbol{Q}$ 仍为对称矩阵.
\end{example}
\begin{proof}
首先注意到存在可逆矩阵 $\boldsymbol{P} \in M_n(\mathbb{F})$ 使得 $\boldsymbol{A} = \boldsymbol{P}\boldsymbol{B}_{\mathbb{F}}\boldsymbol{P}^{-1}$, 其中
$$
\boldsymbol{B}_{\mathbb{F}} = \begin{pmatrix}
\boldsymbol{B}_1 & & \\
& \ddots & \\
& & \boldsymbol{B}_s
\end{pmatrix}
$$
为 $\boldsymbol{A}$ 在 $\mathbb{F}$ 上的有理标准形. 现在分别对 $\boldsymbol{B}_1, \boldsymbol{B}_2, \dots, \boldsymbol{B}_s$ 进行分析. 设 $\boldsymbol{B}_t$ 的阶数为 $r$, 则
$$
\boldsymbol{B}_t = \begin{pmatrix}
0 & & & -b_r \\
1 & \ddots & & \vdots \\
& \ddots & \ddots & -b_2 \\
& & 1 & -b_1
\end{pmatrix}.
$$
令
$$
\boldsymbol{W}_t = \begin{pmatrix}
b_{r-1} & b_{r-2} & b_{r-3} & \cdots & b_1 & 1 \\
b_{r-2} & b_{r-3} & b_{r-4} & \cdots & 1 & \\
b_{r-3} & b_{r-4} & b_{r-5} & \begin{turn}{80}$\ddots$\end{turn} & & \\
\vdots & \vdots & \begin{turn}{80}$\ddots$\end{turn}& & & \\
b_1 & 1 & & & & \\
1 & & & & &
\end{pmatrix},
$$
则 $\boldsymbol{W}_t$ 为对称可逆矩阵, 且通过直接计算可知
\begin{align*}
\boldsymbol{B}_t\boldsymbol{W}_t=\left( \begin{matrix}
-b_r&		0&		0&		\cdots&		0&		0\\
0&		b_{r-2}&		b_{r-3}&		\cdots&		b_1&		1\\
0&		b_{r-3}&		b_{r-4}&		\cdots&		1&		\\
\vdots&		\vdots&		&		\begin{turn}{80}$\ddots$\end{turn}&		&		\\
0&		b_1&		\begin{turn}{80}$\ddots$\end{turn}&		&		&		\\
0&		1&		&		&		&		\\
\end{matrix} \right) .
\end{align*}
为对称矩阵. 若令
$$
\boldsymbol{W} = \begin{pmatrix}
\boldsymbol{W}_1 & & \\
& \ddots & \\
& & \boldsymbol{W}_s
\end{pmatrix},
$$
则
$$
\begin{aligned}
\boldsymbol{A} &= \boldsymbol{P}\boldsymbol{B}_{\mathbb{F}}\boldsymbol{P}^{-1} = \boldsymbol{P}\boldsymbol{B}_{\mathbb{F}}\boldsymbol{W}\boldsymbol{W}^{-1}\boldsymbol{P}^{-1} \\
&= \boldsymbol{P}\begin{pmatrix}
\boldsymbol{B}_1 & & \\
& \ddots & \\
& & \boldsymbol{B}_s
\end{pmatrix}\begin{pmatrix}
\boldsymbol{W}_1 & & \\
& \ddots & \\
& & \boldsymbol{W}_s
\end{pmatrix}\begin{pmatrix}
\boldsymbol{W}_1^{-1} & & \\
& \ddots & \\
& & \boldsymbol{W}_s^{-1}
\end{pmatrix}\boldsymbol{P}^{-1} \\
&= \boldsymbol{P}\begin{pmatrix}
\boldsymbol{B}_1\boldsymbol{W}_1 & & \\
& \ddots & \\
& & \boldsymbol{B}_s\boldsymbol{W}_s
\end{pmatrix}\boldsymbol{P}^{\mathrm{T}}(\boldsymbol{P}^{-1})^{\mathrm{T}}\begin{pmatrix}
\boldsymbol{W}_1^{-1} & & \\
& \ddots & \\
& & \boldsymbol{W}_s^{-1}
\end{pmatrix}\boldsymbol{P}^{-1}.
\end{aligned}
$$

再令
$$
\boldsymbol{S} = \boldsymbol{P}\begin{pmatrix}
\boldsymbol{B}_1\boldsymbol{W}_1 & & \\
& \ddots & \\
& & \boldsymbol{B}_s\boldsymbol{W}_s
\end{pmatrix}\boldsymbol{P}^{\mathrm{T}},
\quad
\boldsymbol{T} = (\boldsymbol{P}^{-1})^{\mathrm{T}}\begin{pmatrix}
\boldsymbol{W}_1^{-1} & & \\
& \ddots & \\
& & \boldsymbol{W}_s^{-1}
\end{pmatrix}\boldsymbol{P}^{-1},
$$
则 $\boldsymbol{S}, \boldsymbol{T}$ 皆为对称矩阵, $\boldsymbol{A} = \boldsymbol{S}\boldsymbol{T}$, $\boldsymbol{T}$ 可逆. 若记 $\boldsymbol{Q} = \boldsymbol{T}^{-1}$, 则 $\boldsymbol{A}\boldsymbol{Q} = \boldsymbol{S}$ 为对称矩阵, 且 $\boldsymbol{Q}$ 对称可逆.
\end{proof}

\begin{example}
设 $\boldsymbol{\alpha}_i = (1, t_i, t_i^2, \cdots, t_i^{n-1})^{\mathrm{T}}, i = 1, 2, \cdots, m$. 试讨论 $\boldsymbol{\alpha}_1, \boldsymbol{\alpha}_2, \cdots, \boldsymbol{\alpha}_m$ 的线性相关性.
\end{example}
\begin{solution}
第一种情形: 当 $n = m$, 且 $t_1, t_2, \cdots, t_n$ 互异时, 由 Vandermonde 行列式知, $\boldsymbol{\alpha}_1, \boldsymbol{\alpha}_2, \cdots, \boldsymbol{\alpha}_m$ 线性无关.

第二种情形: 当 $m > n$ 时, 此时向量组中的向量个数大于这些向量所在的向量空间的维数, 因此必线性相关.

第三种情形: 当 $m < n$ 时, 考察矩阵
$$
\left( \boldsymbol{\alpha}_1, \boldsymbol{\alpha}_2, \cdots, \boldsymbol{\alpha}_m \right) = \begin{pmatrix}
1 & \cdots & 1 \\
t_1 & \cdots & t_m \\
\vdots & & \vdots \\
t_1^{n-1} & \cdots & t_m^{n-1}
\end{pmatrix}.
$$
若 $t_1, t_2, \cdots, t_m$ 各不相同, 则 $\left( \boldsymbol{\alpha}_1, \boldsymbol{\alpha}_2, \cdots, \boldsymbol{\alpha}_m \right)$ 有 $m$ 阶子式不为 $0$, $\boldsymbol{\alpha}_1, \boldsymbol{\alpha}_2, \cdots, \boldsymbol{\alpha}_m$ 线性无关. 若 $t_1, t_2, \cdots, t_m$ 有两个相同, 则 $\left( \boldsymbol{\alpha}_1, \boldsymbol{\alpha}_2, \cdots, \boldsymbol{\alpha}_m \right)$ 至少有两列相同, 从而 $\boldsymbol{\alpha}_1, \boldsymbol{\alpha}_2, \cdots, \boldsymbol{\alpha}_m$ 线性相关.
\end{solution}

\begin{example}

\end{example}
\begin{proof}

\end{proof}

\begin{example}

\end{example}
\begin{solution}

\end{solution}

\begin{example}

\end{example}
\begin{proof}

\end{proof}

\begin{example}

\end{example}
\begin{solution}

\end{solution}

\begin{example}

\end{example}
\begin{proof}

\end{proof}

\begin{example}

\end{example}
\begin{solution}

\end{solution}

\begin{example}

\end{example}
\begin{proof}

\end{proof}







\end{document}
% contents/chapter-02/section-08.tex 第二章第三节
\documentclass[../../main.tex]{subfiles}
\graphicspath{{\subfix{../../image/}}} % 指定图片目录,后续可以直接使用图片文件名。

% 例如:
% \begin{figure}[h]
% \centering
% \includegraphics{image-01.01}
% \caption{图片标题}
% \label{fig:image-01.01}
% \end{figure}
% 注意:上述\label{}一定要放在\caption{}之后,否则引用图片序号会只会显示??.

\begin{document}

\section{分块矩阵的初等变换与降价公式(打洞原理)}

\begin{proposition}[打洞原理]\label{proposition:打洞原理}
(1)设
\[
\boldsymbol{M} = 
\begin{pmatrix}
\boldsymbol{A} & \boldsymbol{B} \\
\boldsymbol{C} & \boldsymbol{D}
\end{pmatrix}_{(n + m) \times (n + m)}
\]
是一个方阵,并且\(\boldsymbol{A}\)为\(n\)阶可逆子方阵,那么
\[
|\boldsymbol{M}| = |\boldsymbol{A}| \cdot |\boldsymbol{D} - \boldsymbol{C}\boldsymbol{A}^{-1}\boldsymbol{B}|.
\]

(2)设
\[
\boldsymbol{M} = 
\begin{pmatrix}
\boldsymbol{A} & \boldsymbol{B} \\
\boldsymbol{C} & \boldsymbol{D}
\end{pmatrix}_{(n + m) \times (n + m)}
\]
是一个方阵,并且\(\boldsymbol{D}\)为\(n\)阶可逆子方阵,那么
\[
|\boldsymbol{M}| = |\boldsymbol{D}| \cdot |\boldsymbol{A} - \boldsymbol{B}\boldsymbol{D}^{-1}\boldsymbol{C}|.
\]
\end{proposition}
\begin{note}
\hyperref[proposition:打洞原理]{打洞原理}是一个重要结论,必须要熟练掌握.但是在实际解题中我们一般不会直接套用\hyperref[proposition:打洞原理]{打洞原理}的结论,而是利用分块矩阵的初等变换书写过程.

记忆打洞原理公式的小技巧:先记住一个模板$\left| \Box \right|=\left| \Box \right|\left| \Box -\Box \Box ^{-1}\Box \right|$,然后从左往右填入子矩阵(每个子矩阵只能填一次),第一个$\Box$填相应的可逆子矩阵,再从主对角线上另外一个子矩阵开始,按顺时针顺序将子矩阵填入$\Box$即可.
\end{note}
\begin{proof}
(核心想法:利用分块矩阵的初等变换消去$\boldsymbol{B}$或$\boldsymbol{C}$)

(1)根据分块矩阵的初等变换,对$\boldsymbol{M}$的第一行左乘$(-\boldsymbol{CA}^{-1})$再加到第二行得到
\begin{align*}
\left( \begin{matrix}
\boldsymbol{I}_n&		\boldsymbol{O}\\
-\boldsymbol{CA}^{-1}&		\boldsymbol{I}_m\\
\end{matrix} \right) \left( \begin{matrix}
\boldsymbol{A}&		\boldsymbol{B}\\
\boldsymbol{C}&		\boldsymbol{D}\\
\end{matrix} \right) =\left( \begin{matrix}
\boldsymbol{A}&		\boldsymbol{B}\\
\boldsymbol{O}&		\boldsymbol{D}-\boldsymbol{CA}^{-1}\boldsymbol{B}\\
\end{matrix} \right) .
\end{align*}
然后两边同时取行列式就得到
\begin{align*}
\left| \boldsymbol{M} \right|=\left| \begin{matrix}
\boldsymbol{I}_n&		\boldsymbol{O}\\
-\boldsymbol{CA}^{-1}&		\boldsymbol{I}_m\\
\end{matrix} \right|\left| \begin{matrix}
\boldsymbol{A}&		\boldsymbol{B}\\
\boldsymbol{C}&		\boldsymbol{D}\\
\end{matrix} \right|=\left| \left( \begin{matrix}
\boldsymbol{I}_n&		\boldsymbol{O}\\
-\boldsymbol{CA}^{-1}&		\boldsymbol{I}_m\\
\end{matrix} \right) \left( \begin{matrix}
\boldsymbol{A}&		\boldsymbol{B}\\
\boldsymbol{C}&		\boldsymbol{D}\\
\end{matrix} \right) \right|=\left| \begin{matrix}
\boldsymbol{A}&		\boldsymbol{B}\\
\boldsymbol{O}&		\boldsymbol{D}-\boldsymbol{CA}^{-1}\boldsymbol{B}\\
\end{matrix} \right|=|\boldsymbol{A}|\cdot |\boldsymbol{D}-\boldsymbol{CA}^{-1}\boldsymbol{B}|.
\end{align*}

(2)根据分块矩阵的初等变换,对$\boldsymbol{M}$的第二行左乘$(-\boldsymbol{BD}^{-1})$再加到第一行得到
\begin{align*}
\left( \begin{matrix}
\boldsymbol{I}_n&		-\boldsymbol{BD}^{-1}\\
\boldsymbol{O}&		\boldsymbol{I}_m\\
\end{matrix} \right) \left( \begin{matrix}
\boldsymbol{A}&		\boldsymbol{B}\\
\boldsymbol{C}&		\boldsymbol{D}\\
\end{matrix} \right) =\left( \begin{matrix}
\boldsymbol{A}-\boldsymbol{BD}^{-1}\boldsymbol{C}&		\boldsymbol{O}\\
\boldsymbol{C}&		\boldsymbol{D}\\
\end{matrix} \right) .
\end{align*}
然后两边同时取行列式就得到
\begin{align*}
\left| \boldsymbol{M} \right|=\left| \begin{matrix}
\boldsymbol{I}_n&		-\boldsymbol{BD}^{-1}\\
\boldsymbol{O}&		\boldsymbol{I}_m\\
\end{matrix} \right|\left| \begin{matrix}
\boldsymbol{A}&		\boldsymbol{B}\\
\boldsymbol{C}&		\boldsymbol{D}\\
\end{matrix} \right|=\left| \left( \begin{matrix}
\boldsymbol{I}_n&		-\boldsymbol{BD}^{-1}\\
\boldsymbol{O}&		\boldsymbol{I}_m\\
\end{matrix} \right) \left( \begin{matrix}
\boldsymbol{A}&		\boldsymbol{B}\\
\boldsymbol{C}&		\boldsymbol{D}\\
\end{matrix} \right) \right|=\left| \begin{matrix}
\boldsymbol{A}-\boldsymbol{BD}^{-1}\boldsymbol{C}&		\boldsymbol{O}\\
\boldsymbol{C}&		\boldsymbol{D}\\
\end{matrix} \right|=|\boldsymbol{D}|\cdot |\boldsymbol{A}-\boldsymbol{BD}^{-1}\boldsymbol{C}|.
\end{align*}

\end{proof}

\begin{corollary}[打洞原理推论]\label{corollary:打洞原理推论}
设\(\boldsymbol{A}\)是\(n\times m\)矩阵,\(\boldsymbol{B}\)是\(m\times n\)矩阵,则
\[
\lambda^{m}|\lambda\boldsymbol{I}_{n}-\boldsymbol{AB}|=\lambda^{n}|\lambda\boldsymbol{I}_{m}-\boldsymbol{BA}|.
\]
\end{corollary}
\begin{note}
这个推论能将原本复杂的矩阵$AB$通过交换顺序变成相对简单的矩阵$BA$.例如:\hyperref[example:1895]{例题\ref{example:1895}}.
\end{note}
\begin{remark}
这是由\hyperref[proposition:打洞原理]{打洞原理}得到的一个重要结论,也需要熟练掌握.同样地,在实际解题中如果不能直接套用\hyperref[corollary:打洞原理推论]{打洞原理推论}的结论,就需要利用分块矩阵的初等变换书写过程.
\end{remark}
\begin{proof}
当$\lambda=0$时,结论显然成立.

当$\lambda\ne 0$时,根据分块矩阵的初等变换可知
\begin{gather*}
\left( \begin{matrix}
\boldsymbol{I}_n&		-\boldsymbol{A}\\
\boldsymbol{O}&		\boldsymbol{I}_m\\
\end{matrix} \right) \left( \begin{matrix}
\lambda \boldsymbol{I}_n&		\boldsymbol{A}\\
\boldsymbol{B}&		\boldsymbol{I}_m\\
\end{matrix} \right) =\left( \begin{matrix}
\lambda \boldsymbol{I}_n-\boldsymbol{AB}&		\boldsymbol{O}\\
\boldsymbol{B}&		\boldsymbol{I}_m\\
\end{matrix} \right) ,
\\
\left( \begin{matrix}
\boldsymbol{I}_n&		\boldsymbol{O}\\
-\frac{1}{\lambda}\boldsymbol{B}&		\boldsymbol{I}_m\\
\end{matrix} \right) \left( \begin{matrix}
\lambda \boldsymbol{I}_n&		\boldsymbol{A}\\
\boldsymbol{B}&		\boldsymbol{I}_m\\
\end{matrix} \right) =\left( \begin{matrix}
\lambda \boldsymbol{I}_n&		\boldsymbol{A}\\
\boldsymbol{O}&		\boldsymbol{I}_m-\frac{1}{\lambda}\boldsymbol{BA}\\
\end{matrix} \right) .
\end{gather*}
再对上式两边分别取行列式得到
\begin{gather*}
\left| \begin{matrix}
\lambda \boldsymbol{I}_n&		\boldsymbol{A}\\
\boldsymbol{B}&		\boldsymbol{I}_m\\
\end{matrix} \right|=\left| \begin{matrix}
\boldsymbol{I}_n&		-\boldsymbol{A}\\
\boldsymbol{O}&		\boldsymbol{I}_m\\
\end{matrix} \right|\left| \begin{matrix}
\lambda \boldsymbol{I}_n&		\boldsymbol{A}\\
\boldsymbol{B}&		\boldsymbol{I}_m\\
\end{matrix} \right|=\left| \left( \begin{matrix}
\boldsymbol{I}_n&		-\boldsymbol{A}\\
\boldsymbol{O}&		\boldsymbol{I}_m\\
\end{matrix} \right) \left( \begin{matrix}
\lambda \boldsymbol{I}_n&		\boldsymbol{A}\\
\boldsymbol{B}&		\boldsymbol{I}_m\\
\end{matrix} \right) \right|=\left| \begin{matrix}
\lambda \boldsymbol{I}_n-\boldsymbol{AB}&		\boldsymbol{O}\\
\boldsymbol{B}&		\boldsymbol{I}_m\\
\end{matrix} \right|=\left| \lambda \boldsymbol{I}_n-\boldsymbol{AB} \right|.
\\
\left| \begin{matrix}
\lambda \boldsymbol{I}_n&		\boldsymbol{A}\\
\boldsymbol{B}&		\boldsymbol{I}_m\\
\end{matrix} \right|=\left| \begin{matrix}
\boldsymbol{I}_n&		\boldsymbol{O}\\
-\frac{1}{\lambda}\boldsymbol{B}&		\boldsymbol{I}_m\\
\end{matrix} \right|\left| \begin{matrix}
\lambda \boldsymbol{I}_n&		\boldsymbol{A}\\
\boldsymbol{B}&		\boldsymbol{I}_m\\
\end{matrix} \right|=\left| \left( \begin{matrix}
\boldsymbol{I}_n&		\boldsymbol{O}\\
-\frac{1}{\lambda}\boldsymbol{B}&		\boldsymbol{I}_m\\
\end{matrix} \right) \left( \begin{matrix}
\lambda \boldsymbol{I}_n&		\boldsymbol{A}\\
\boldsymbol{B}&		\boldsymbol{I}_m\\
\end{matrix} \right) \right|=\left| \begin{matrix}
\lambda \boldsymbol{I}_n&		\boldsymbol{A}\\
\boldsymbol{O}&		\boldsymbol{I}_m-\frac{1}{\lambda}\boldsymbol{BA}\\
\end{matrix} \right|=\lambda ^n\left| \boldsymbol{I}_m-\frac{1}{\lambda}\boldsymbol{BA} \right|=\lambda ^{n-m}\left| \lambda \boldsymbol{I}_m-\boldsymbol{BA} \right|.    
\end{gather*}
于是$\left| \begin{matrix}
\lambda \boldsymbol{I}_n&		\boldsymbol{A}\\
\boldsymbol{B}&		\boldsymbol{I}_m\\
\end{matrix} \right|=\left| \lambda \boldsymbol{I}_n-\boldsymbol{AB} \right|=\lambda ^{n-m}\left| \lambda \boldsymbol{I}_m-\boldsymbol{BA} \right|.
$.故$\lambda ^m\left| \lambda \boldsymbol{I}_n-\boldsymbol{AB} \right|=\lambda ^n\left| \lambda \boldsymbol{I}_m-\boldsymbol{BA} \right|$.
\end{proof}

\begin{example}
\begin{enumerate}
\item 设\(A\)是\(n\)阶矩阵,\(D\)是\(m\)阶矩阵。\(|A|\neq0\),\(|D - CA^{-1}B|\neq0\),计算逆矩阵\(\begin{pmatrix}A & B \\ C & D\end{pmatrix}^{-1}\)。

\item 设\(A\)是\(n\)阶矩阵,\(C\)是\(m\)阶矩阵。计算伴随矩阵\(\begin{pmatrix}A & B \\ 0 & C\end{pmatrix}^*\)。 
\end{enumerate}
\end{example}
\begin{proof}
\begin{enumerate}
\item 由条件可知\(A,D-CA^{-1}B\)都非异,于是由分块矩阵初等变换可得
\begin{align*}
&\left( \begin{matrix}
I_n&O\\
-CA^{-1}&I_m\\
\end{matrix} \right) 
\left( \begin{matrix}
A&B\\
C&D\\
\end{matrix} \right) 
\left( \begin{matrix}
I_n&-A^{-1}B\\
O&I_m\\
\end{matrix} \right) 
\left( \begin{matrix}
A^{-1}&O\\
O&\left( D-CA^{-1}B \right)^{-1}\\
\end{matrix} \right) 
\\
&=
\left( \begin{matrix}
A&O\\
O&D-CA^{-1}B\\
\end{matrix} \right) 
\left( \begin{matrix}
A^{-1}&O\\
O&\left( D-CA^{-1}B \right)^{-1}\\
\end{matrix} \right) 
=
\left( \begin{matrix}
I_n&O\\
O&I_m\\
\end{matrix} \right).
\end{align*}
于是对上式两边同时取逆可得
\begin{align*}
\left( \begin{matrix}
A^{-1}&O\\
O&\left( D-CA^{-1}B \right)^{-1}\\
\end{matrix} \right)^{-1}
\left( \begin{matrix}
I_n&-A^{-1}B\\
O&I_m\\
\end{matrix} \right)^{-1}
\left( \begin{matrix}
A&B\\
C&D\\
\end{matrix} \right)^{-1}
\left( \begin{matrix}
I_n&O\\
-CA^{-1}&I_m\\
\end{matrix} \right)^{-1}
=
\left( \begin{matrix}
I_n&O\\
O&I_m\\
\end{matrix} \right).
\end{align*}
故
\begin{align*}
\left( \begin{matrix}
A&B\\
C&D\\
\end{matrix} \right)^{-1}
&=
\left( \begin{matrix}
I_n&-A^{-1}B\\
O&I_m\\
\end{matrix} \right) 
\left( \begin{matrix}
A^{-1}&O\\
O&\left( D-CA^{-1}B \right)^{-1}\\
\end{matrix} \right) 
\left( \begin{matrix}
I_n&O\\
-CA^{-1}&I_m\\
\end{matrix} \right) 
\\
&=
\left( \begin{matrix}
A^{-1}+A^{-1}B\left( D-CA^{-1}B \right)^{-1}CA^{-1}&-A^{-1}B\left( D-CA^{-1}B \right)^{-1}\\
-\left( D-CA^{-1}B \right)^{-1}CA^{-1}&\left( D-CA^{-1}B \right)^{-1}\\
\end{matrix} \right).
\end{align*}

\item 若\(A,C\)非异,则\(\left| \begin{matrix} A&B\\ O&C\\ \end{matrix} \right|=\left| A \right|\left| C \right|\ne 0\),进而\(\left( \begin{matrix} A&B\\ O&C\\ \end{matrix} \right)\)非异。利用分块矩阵初等变换可得
\begin{align*}
\left( \begin{matrix} A&B\\ O&C\\ \end{matrix} \right) 
\left( \begin{matrix} I_n&-A^{-1}B\\ O&I_m\\ \end{matrix} \right) 
\left( \begin{matrix} A^{-1}&O\\ O&C^{-1}\\ \end{matrix} \right) 
=
\left( \begin{matrix} I_n&O\\ O&I_m\\ \end{matrix} \right).
\end{align*}
故此时就有
\begin{align*}
\left( \begin{matrix} A&B\\ O&C\\ \end{matrix} \right)^{-1}
=
\left( \begin{matrix} I_n&-A^{-1}B\\ O&I_m\\ \end{matrix} \right) 
\left( \begin{matrix} A^{-1}&O\\ O&C^{-1}\\ \end{matrix} \right) 
=
\left( \begin{matrix} A^{-1}&-A^{-1}BC^{-1}\\ O&C^{-1}\\ \end{matrix} \right).
\end{align*}
于是
\begin{align*}
\left( \begin{matrix} A&B\\ O&C\\ \end{matrix} \right)^*
&=
\left| \begin{matrix} A&B\\ O&C\\ \end{matrix} \right|
\left( \begin{matrix} A&B\\ O&C\\ \end{matrix} \right)^{-1}
\\
&=
\left( \begin{matrix} 
\left| A \right|\left| C \right|A^{-1} & -\left| A \right|\left| C \right|A^{-1}BC^{-1} \\
O & \left| A \right|\left| C \right|C^{-1} \\
\end{matrix} \right).
\end{align*}

对于一般的方阵\(A,C\),存在一列有理数\(t_k\rightarrow 0\),使得\(t_kI_n+A,t_kI_m+C\)都是非异阵。于是由上述非异的情形可得
\begin{align*}
\left( \begin{matrix} t_kI_n+A&B\\ O&t_kI_n+C\\ \end{matrix} \right)^*
=
\left( \begin{matrix} 
\left| t_kI_n+A \right|\left| t_kI_n+C \right|\left( t_kI_n+A \right)^{-1} & -\left| t_kI_n+A \right|\left| t_kI_n+C \right|\left( t_kI_n+A \right)^{-1}B\left( t_kI_n+C \right)^{-1} \\
O & \left| t_kI_n+A \right|\left| t_kI_n+C \right|\left( t_kI_n+C \right)^{-1} \\
\end{matrix} \right).
\end{align*}
上式两边的矩阵中的元素都是\(t_k\)的多项式,从而都关于\(t_k\)连续。于是令\(k\rightarrow \infty\),得
\begin{align*}
\left( \begin{matrix} A&B\\ O&C\\ \end{matrix} \right)^*
=
\left( \begin{matrix} 
\left| A \right|\left| C \right|A^{-1} & -\left| A \right|\left| C \right|A^{-1}BC^{-1} \\
O & \left| A \right|\left| C \right|C^{-1} \\
\end{matrix} \right).
\end{align*}
\end{enumerate}
\end{proof}

\begin{example}\label{example:1895}
求下列矩阵的行列式的值:
\[
A = 
\begin{pmatrix}
a_1^2 & a_1a_2 + 1 & \cdots & a_1a_n + 1\\
a_2a_1 + 1 & a_2^2 & \cdots & a_2a_n + 1\\
\vdots & \vdots & & \vdots\\
a_na_1 + 1 & a_na_2 + 1 & \cdots & a_n^2
\end{pmatrix}.
\]
\end{example}
\begin{solution}
令$\boldsymbol{\varLambda }=\left( \begin{matrix}
a_1&		1\\
a_2&		1\\
\vdots&		\vdots\\
a_n&		1\\
\end{matrix} \right)$,则由降价公式(打洞原理)我们有
\begin{align*}
&\boldsymbol{A}=-\boldsymbol{I}_n+\left( \begin{matrix}
a_1&		1\\
a_2&		1\\
\vdots&		\vdots\\
a_n&		1\\
\end{matrix} \right) \boldsymbol{I}_{2}^{-1}\left( \begin{matrix}
a_1&		a_2&		\cdots&		a_n\\
1&		1&		\cdots&		1\\
\end{matrix} \right) =\left( -1 \right) ^n\left| \boldsymbol{I}_2 \right|\left| \boldsymbol{I}_n-\left( \begin{matrix}
a_1&		1\\
a_2&		1\\
\vdots&		\vdots\\
a_n&		1\\
\end{matrix} \right) \boldsymbol{I}_{2}^{-1}\left( \begin{matrix}
a_1&		a_2&		\cdots&		a_n\\
1&		1&		\cdots&		1\\
\end{matrix} \right) \right|
\\
&=\left( -1 \right) ^n\left| \begin{matrix}
\boldsymbol{I}_2&		\boldsymbol{\varLambda }'\\
\boldsymbol{\varLambda }&		\boldsymbol{I}_n\\
\end{matrix} \right|=\left( -1 \right) ^n\left| \boldsymbol{I}_n \right|\left| \boldsymbol{I}_2-\left( \begin{matrix}
a_1&		a_2&		\cdots&		a_n\\
1&		1&		\cdots&		1\\
\end{matrix} \right) \boldsymbol{I}_{n}^{-1}\left( \begin{matrix}
a_1&		1\\
a_2&		1\\
\vdots&		\vdots\\
a_n&		1\\
\end{matrix} \right) \right|
\\
&=\left( -1 \right) ^n\left| \boldsymbol{I}_2-\left( \begin{matrix}
\sum_{i=1}^n{a_{i}^{2}}&		\sum_{i=1}^n{a_i}\\
\sum_{i=1}^n{a_i}&		n\\
\end{matrix} \right) \right|=\left( -1 \right) ^n\left[ \left( 1-n \right) \left( 1-\sum_{i=1}^n{a_{i}^{2}} \right) -\left( \sum_{i=1}^n{a_i} \right) ^2 \right] .
\end{align*}
\end{solution}

\begin{example}
计算矩阵\(A\)的行列式的值:
\[
A = 
\begin{pmatrix}
1 + a_1^2 & a_1a_2 & \cdots & a_1a_n\\
a_2a_1 & 1 + a_2^2 & \cdots & a_2a_n\\
\vdots & \vdots & & \vdots\\
a_na_1 & a_na_2 & \cdots & 1 + a_n^2
\end{pmatrix}.
\]
\end{example}
\begin{solution}
注意到\begin{align*}
A-I_n=\left( \begin{array}{c}
a_1\\
a_2\\
\vdots\\
a_n\\
\end{array} \right) \left( \begin{matrix}
a_1&		a_2&		\cdots&		a_n\\
\end{matrix} \right) .
\end{align*}
从而由降价公式可得\begin{align*}
\left| A \right|=\left| I_n+\left( \begin{array}{c}
a_1\\
a_2\\
\vdots\\
a_n\\
\end{array} \right) \left( \begin{matrix}
a_1&		a_2&		\cdots&		a_n\\
\end{matrix} \right) \right|=\left| I_n \right|\left| 1+\left( \begin{matrix}
a_1&		a_2&		\cdots&		a_n\\
\end{matrix} \right) I_{n}^{-1}\left( \begin{array}{c}
a_1\\
a_2\\
\vdots\\
a_n\\
\end{array} \right) \right|=1+\sum_{i=1}^n{a_{i}^{2}}.
\end{align*}
\end{solution}

\begin{example}
计算矩阵\(A\)的行列式的值:
\[
A = 
\begin{pmatrix}
a_1 - b_1 & a_1 - b_2 & \cdots & a_1 - b_n\\
a_2 - b_1 & a_2 - b_2 & \cdots & a_2 - b_n\\
\vdots & \vdots & & \vdots\\
a_n - b_1 & a_n - b_2 & \cdots & a_n - b_n
\end{pmatrix}.
\]
\end{example}
\begin{solution}
注意到\begin{align*}
A=\left( \begin{matrix}
a_1-b_1&		a_1-b_2&		\cdots&		a_1-b_n\\
a_2-b_1&		a_2-b_2&		\cdots&		a_2-b_n\\
\vdots&		\vdots&		&		\vdots\\
a_n-b_1&		a_n-b_2&		\cdots&		a_n-b_n\\
\end{matrix} \right) =\left( \begin{matrix}
a_1&		-1\\
a_2&		-1\\
\vdots&		\vdots\\
a_n&		-1\\
\end{matrix} \right) \left( \begin{matrix}
1&		1&		\cdots&		1\\
b_1&		b_2&		\cdots&		b_n\\
\end{matrix} \right) .
\end{align*}
当$n>2$时,由Cauchy-Binet公式可知$\left| A \right|=0$.当$n=2$时,$\left| A \right|=a_1b_1+a_2b_2-a_1b_2-b_1a_2$.当$n=1$时,$\left| A \right|=a_1-b_1$.
\end{solution}

\begin{example}
求下列矩阵的行列式的值:
\[
A = 
\begin{pmatrix}
0 & 2 & 3 & \cdots & n\\
1 & 0 & 3 & \cdots & n\\
1 & 2 & 0 & \cdots & n\\
\vdots & \vdots & \vdots & & \vdots\\
1 & 2 & 3 & \cdots & 0
\end{pmatrix}.
\]
\end{example}
\begin{solution}
将\(A\)化为
\[
A = 
\begin{pmatrix}
-1 & 0 & \cdots & 0\\
0 & -2 & \cdots & 0\\
\vdots & \vdots & & \vdots\\
0 & 0 & \cdots & -n
\end{pmatrix}
+
\begin{pmatrix}
1\\
1\\
\vdots\\
1
\end{pmatrix}
(1,2,\cdots,n),
\]
利用降阶公式容易求得\(|A| = (-1)^nn!(1 - n)\).
\end{solution}

\begin{proposition}\label{proposition:对角相同分块矩阵行列式计算}
设\(A,B\)是\(n\)阶矩阵,求证:
\[
\begin{vmatrix}
A & B\\
B & A
\end{vmatrix}=|A + B||A - B|.
\]
\end{proposition}
\begin{proof}
将分块矩阵的第二行加到第一行上,再将第二列减去第一列,可得
\[
\begin{pmatrix}
A & B\\
B & A
\end{pmatrix}\to\begin{pmatrix}
A + B & A + B\\
B & A
\end{pmatrix}\to\begin{pmatrix}
A + B & O\\
B & A - B
\end{pmatrix}.
\]
第三类分块初等变换不改变行列式的值,因此可得
\[
\begin{vmatrix}
A & B\\
B & A
\end{vmatrix}=\begin{vmatrix}
A + B & O\\
B & A - B
\end{vmatrix}=|A + B||A - B|.
\]
\end{proof}

\begin{example}
计算:
\[
|A| = 
\begin{vmatrix}
x & y & z & w\\
y & x & w & z\\
z & w & x & y\\
w & z & y & x
\end{vmatrix}.
\]
\end{example}
\begin{solution}
{\color{blue}解法一:}
令
\[
B = 
\begin{pmatrix}
x & y\\
y & x
\end{pmatrix},
C = 
\begin{pmatrix}
z & w\\
w & z
\end{pmatrix},
\]
则\(|A| = 
\begin{vmatrix}
B & C\\
C & B
\end{vmatrix}\).由\hyperref[proposition:对角相同分块矩阵行列式计算]{命题\ref{proposition:对角相同分块矩阵行列式计算}}可得
\begin{align*}
|A|&=|B + C||B - C|
=\begin{vmatrix}
x + z & y + w\\
y + w & x + z
\end{vmatrix}\begin{vmatrix}
x - z & y - w\\
y - w & x - z
\end{vmatrix}\\
&=(x + y + z + w)(x + z - y - w)(x + y - z - w)(x + w - y - z).
\end{align*}
{\color{blue}解法二(求根法):}
\end{solution}

\begin{proposition}\label{proposition:对角相同的复分块矩阵行列式计算}
设\(A,B\)是\(n\)阶复矩阵,求证:
\[
\begin{vmatrix}
A & -B\\
B & A
\end{vmatrix}=|A + \mathrm{i}B||A - \mathrm{i}B|.
\]
\end{proposition}
\begin{proof}
将分块矩阵的第二行乘以\(\mathrm{i}\)加到第一行上,再将第一列乘以\(-\mathrm{i}\)加到第二列上,可得
\[
\begin{pmatrix}
A & -B\\
B & A
\end{pmatrix}\to\begin{pmatrix}
A + \mathrm{i}B & \mathrm{i}A - B\\
B & A
\end{pmatrix}\to\begin{pmatrix}
A + \mathrm{i}B & O\\
B & A - \mathrm{i}B
\end{pmatrix}.
\]
第三类分块初等变换不改变行列式的值,因此可得
\[
\begin{vmatrix}
A & -B\\
B & A
\end{vmatrix}=\begin{vmatrix}
A + \mathrm{i}B & O\\
B & A - \mathrm{i}B
\end{vmatrix}=|A + \mathrm{i}B||A - \mathrm{i}B|.
\]
\end{proof}

\begin{example}
设\(A,B,C,D\)都是\(n\)阶矩阵,求证:
\[
|M| = 
\begin{vmatrix}
A & B & C & D\\
B & A & D & C\\
C & D & A & B\\
D & C & B & A
\end{vmatrix}
= |A + B + C + D||A + B - C - D||A - B + C - D||A - B - C + D|.
\]
\end{example}
\begin{solution}
反复利用\hyperref[proposition:对角相同分块矩阵行列式计算]{命题\ref{proposition:对角相同分块矩阵行列式计算}}的结论可得
\begin{align*}
|M|&=\left|\begin{pmatrix}
A & B\\
B & A
\end{pmatrix}+\begin{pmatrix}
C & D\\
D & C
\end{pmatrix}\right|\cdot\left|\begin{pmatrix}
A & B\\
B & A
\end{pmatrix}-\begin{pmatrix}
C & D\\
D & C
\end{pmatrix}\right|
=\begin{vmatrix}
A + C & B + D\\
B + D & A + C
\end{vmatrix}\cdot\begin{vmatrix}
A - C & B - D\\
B - D & A - C
\end{vmatrix}\\
&=|A + B + C + D||A - B + C - D||A + B - C - D||A - B - C + D|.
\end{align*}
\end{solution}


\begin{example}\label{example:2.28}
设\(A,B\)是\(n\)阶矩阵且\(AB = BA\),求证:
\[
\begin{vmatrix}
A & -B\\
B & A
\end{vmatrix}=|A^2 + B^2|.
\]
\end{example}
\begin{proof}
由\hyperref[proposition:对角相同的复分块矩阵行列式计算]{命题\ref{proposition:对角相同的复分块矩阵行列式计算}}的结论可得
\begin{align*}
\begin{vmatrix}
A & -B\\
B & A
\end{vmatrix}&=|A + \mathrm{i}B|\cdot|A - \mathrm{i}B|
=|(A + \mathrm{i}B)(A - \mathrm{i}B)|
=|A^2 + B^2 - \mathrm{i}(AB - BA)|
=|A^2 + B^2|.
\end{align*}
\end{proof}

\begin{example}
设\(A,B\)是\(n\)阶实矩阵,求证:
\[
\begin{vmatrix}
A & -B\\
B & A
\end{vmatrix}\geq0.
\]
\end{example}
\begin{proof}
注意到\(A,B\)都是实矩阵,故\(\overline{|A + \mathrm{i}B|}=|\overline{A + \mathrm{i}B}|=|A - \mathrm{i}B|\),再由\hyperref[proposition:对角相同的复分块矩阵行列式计算]{命题\ref{proposition:对角相同的复分块矩阵行列式计算}}的结论可得
\begin{align*}
\begin{vmatrix}
A & -B\\
B & A
\end{vmatrix}&=|A + \mathrm{i}B|\cdot|A - \mathrm{i}B|=|A + \mathrm{i}B|\cdot\overline{|A + \mathrm{i}B|}\geq0.
\end{align*}
\end{proof}



\end{document}
% contents/chapter-02/section-02.tex 第二章第二节
\documentclass[../../main.tex]{subfiles}
\graphicspath{{\subfix{../../image/}}} % 指定图片目录,后续可以直接使用图片文件名。

% 例如:
% \begin{figure}[H]
% \centering
% \includegraphics{image-01.01}
% \caption{图片标题}
% \label{figure:image-01.01}
% \end{figure}
% 注意:上述\label{}一定要放在\caption{}之后,否则引用图片序号会只会显示??.

\begin{document}

\section{矩阵的初等变换}

\subsection{相抵标准型}


\begin{definition}[矩阵相抵的定义]\label{definition:矩阵相抵的定义}
设矩阵$A,B$,若$A$经有限次初等变换后变成$B$,则称$A$与$B$\textbf{相抵},记作$A\sim B$.
\end{definition}
\begin{note}
容易验证相抵是\(M_{s\times n}(K)\)上的一个等价关系.在相抵关系下,矩阵\(A\)的等价类称为\(A\)的相抵类.
\end{note}

\begin{proposition}[矩阵相抵的等价命题]\label{proposition:矩阵相抵的等价命题}
数域\(K\)上\(s\times n\)矩阵\(A\)和\(B\)相抵等价于:
\begin{enumerate}
\item $A$\text{可以经过初等行变换和初等列变换变成}\(B\).

\item \text{存在}$K$\text{上}$s$\text{级初等矩阵}$P_1,P_2,\cdots,P_t$\text{与}$n$\text{级初等矩阵}$Q_1,Q_2,\cdots,Q_m$,\text{使得}
$P_t\cdots P_2P_1AQ_1Q_2\cdots Q_m = B$.

\item \text{存在}\(K\)\text{上}\(s\)\text{级可逆矩阵}\(P\)\text{与}\(n\)\text{级可逆矩阵}\(Q\),\text{使得}:
\(PAQ = B\).
\end{enumerate}
\end{proposition}

\begin{theorem}[相抵标准型]\label{theorem:相抵标准型}
设数域\(K\)上\(s\times n\)矩阵\(A\)的秩为\(r\).如果\(r > 0\),那么\(A\)相抵于下述形式的矩阵:
\begin{align}\label{equation:相抵标准型1}
\begin{pmatrix}
I_r & 0 \\
0 & 0
\end{pmatrix}   
\end{align}
称矩阵\eqref{equation:相抵标准型1}为\(A\)的相抵标准形;如果\(r = 0\),那么\(A\)相抵于零矩阵,此时称\(A\)的相抵标准形是零矩阵.
\end{theorem}

\begin{corollary}
\begin{enumerate}
\item 数域\(K\)上\(s\times n\)矩阵\(A\)和\(B\)相抵当且仅当它们的秩相等.

\item 设数域\(K\)上\(s\times n\)矩阵\(A\)的秩为\(r(r > 0)\),则存在\(K\)上的\(s\)级、\(n\)级可逆矩阵\(P\)、\(Q\),使得
\(A = P\begin{pmatrix}
I_r & 0 \\
0 & 0
\end{pmatrix}Q\).
\end{enumerate}
\end{corollary}

\begin{proposition}[奇异阵的充要条件]\label{proposition:奇异阵的充要条件}
数域$K$上的$n$阶矩阵$A$是奇异阵的充要条件有:
\begin{enumerate}
\item\label{proposition:奇异阵的充要条件1} 存在数域$K$上不为零的同阶方阵$B$,使得$AB=O$.
\item\label{proposition:奇异阵的充要条件2} 存在数域$K$上的$n$维非零列向量$x$,使得$Ax=0$.
\end{enumerate}
\end{proposition}
\begin{proof}
\begin{enumerate}
\item 充分性($\Leftarrow$):显然若\(A\)可逆,则从\(AB = O\)可得到\(B = O\),因此充分性成立.

必要性($\Rightarrow$):反之,若\(A\)是奇异阵,则存在数域$K$上的可逆阵\(P,Q\),使得\(PAQ = \begin{pmatrix}I_r & O \\ O & O\end{pmatrix}\),其中\(r < n\).令\(C = \begin{pmatrix}O & O \\ O & I_{n - r}\end{pmatrix}\),则\(PAQC = O\).又因为\(P\)可逆,故\(AQC = O\).只要令\(B = QC\in K\)就得到了结论.

\item 充分性($\Leftarrow$):显然若\(A\)可逆,则从\(Ax = 0\)可得到\(x = 0\),因此充分性成立.

必要性($\Rightarrow$):反之,若\(A\)是奇异阵,则存在数域$K$上的可逆阵\(P,Q\),使\(PAQ = \begin{pmatrix}I_r & O \\ O & O\end{pmatrix}\),其中\(r < n\).
令\(y = (0,\cdots,0,1)'\)为\(n\)维列向量,则\(PAQy = 0\).又因为\(P\)可逆,故\(AQy = 0\).
只要令\(x = Qy\in K\)就得到了结论.
\end{enumerate}
\end{proof}

\begin{example}
设\(A \in \mathbb{F}^{n\times n}\),\(\mathbb{F}\)是域。证明:若\(\det A = 0\),则只用初等行变换可以把\(A\)的某一行变成\(0\)。
\end{example}
\begin{proof}
设\(\alpha_i\)为矩阵\(A\)的第\(i\)行行向量,则\(A=\left( \begin{array}{c} \alpha_1\\ \alpha_2\\ \vdots\\ \alpha_n\\ \end{array} \right)\)。由于\(\det A=0\),故\(\text{r}(A) < n\),从而\(\alpha_1,\alpha_2,\cdots,\alpha_n\)线性相关。
于是存在一组不全为零的数\(k_1,k_2,\cdots,k_n\in \mathbb{F}\),使得
\begin{align*}
k_1\alpha_1 + k_2\alpha_2 + \cdots + k_n\alpha_n = 0.
\end{align*}
不妨设\(k_1\ne 0\),则
\begin{align*}
\alpha_1 = -\frac{k_2}{k_1}\alpha_2 - \cdots - \frac{k_n}{k_1}\alpha_n.
\end{align*}
于是利用初等行变换可得
\begin{align*}
A = \left( \begin{array}{c} \alpha_1\\ \alpha_2\\ \vdots\\ \alpha_n\\ \end{array} \right) \xrightarrow{r_1 + \frac{k_2}{k_1}r_2 + \cdots + \frac{k_n}{k_1}r_n} \left( \begin{array}{c} 0\\ \alpha_2\\ \vdots\\ \alpha_n\\ \end{array} \right).
\end{align*}
\end{proof}

\begin{example}
证明:
\begin{align*}
\begin{pmatrix}
1 & 1 & \cdots & 1 \\
1 & 1 & \cdots & 1 \\
\vdots & \vdots & \ddots & \vdots \\
1 & 1 & \cdots & 1
\end{pmatrix}
\sim
\begin{pmatrix}
0 & 0 & \cdots & 1 \\
0 & 0 & \cdots & 2 \\
\vdots & \vdots & \ddots & \vdots \\
0 & 0 & \cdots & n
\end{pmatrix}.
\end{align*} 
\end{example}
\begin{proof}
利用矩阵的初等变换可得
\begin{align*}
\left( \begin{matrix}
1&		1&		\cdots&		1\\
1&		1&		\cdots&		1\\
\vdots&		\vdots&		\ddots&		\vdots\\
1&		1&		\cdots&		1\\
\end{matrix} \right) \xrightarrow[\begin{array}{c}
\cdots\\
r_n+\left( n-1 \right) r_1\\
\end{array}]{\begin{array}{c}
r_2+r_1\\
r_3+2r_1\\
\end{array}}\left( \begin{matrix}
1&		1&		\cdots&		1\\
2&		2&		\cdots&		2\\
\vdots&		\vdots&		\ddots&		\vdots\\
n&		n&		\cdots&		n\\
\end{matrix} \right) \xrightarrow[\begin{array}{c}
\cdots\\
j_{n-1}-j_n\\
\end{array}]{\begin{array}{c}
j_1-j_n\\
j_2-j_n\\
\end{array}}\left( \begin{matrix}
0&		0&		\cdots&		1\\
0&		0&		\cdots&		2\\
\vdots&		\vdots&		\ddots&		\vdots\\
0&		0&		\cdots&		n\\
\end{matrix} \right) .
\end{align*}
故
\begin{align*}
\left( \begin{matrix}
1&1&\cdots&1\\
1&1&\cdots&1\\
\vdots&\vdots&\ddots&\vdots\\
1&1&\cdots&1
\end{matrix} \right) 
\sim
\left( \begin{matrix}
0&0&\cdots&1\\
0&0&\cdots&2\\
\vdots&\vdots&\ddots&\vdots\\
0&0&\cdots&n
\end{matrix} \right).
\end{align*} 
\end{proof}




\subsection{练习}

\begin{exercise}
设\(A\)为\(n\)阶实反对称阵,证明:\(I_n - A\)是非异阵.
\end{exercise}
\begin{proof}
(反证法)假设是$I_n-A$是奇异阵,则由\hyperref[proposition:奇异阵的充要条件2]{命题\ref{proposition:奇异阵的充要条件}的2},可知存在\(n\)维非零实列向量\(x\),使得\((I_n - A)x = 0\),即\(Ax = x\).设\(x = (a_1,a_2,\cdots,a_n)'\),其中\(a_i\)都是实数,则由\(A\)的反对称性以及\hyperref[proposition:反对称阵的刻画]{命题\ref{proposition:反对称阵的刻画}},可知
\begin{align*}
0 = x'Ax = x'x = a_1^2 + a_2^2 + \cdots + a_n^2.
\end{align*}
从而\(a_1 = a_2 = \cdots = a_n = 0\),即\(x = 0\),这与已知矛盾.
\end{proof}


\begin{exercise}\label{可逆阵都能只用第三类初等变换化为对角阵}
设\(A\)为\(n\)阶可逆阵,求证:只用第三类初等变换就可以将\(A\)化为如下形状:
\[
\mathrm{diag}\{1,\cdots,1,|A|\}.
\]
\end{exercise}
\begin{proof}
假设\(A\)的第\((1,1)\)元素等于零,因为\(A\)可逆,故第一行必有元素不为零.用第三初等变换将非零元素所在的列加到第一列,则到的矩阵中第\((1,1)\)元素不为零.因此不设\(A\)的第\((1,1)\)元素非零,于是可用三类初等变换将\(A\)的第一行及第一列其余素都消为零.这就是说,\(A\)经过第三类初变换可化为如下形状:
\[
\begin{pmatrix}
a & O \\
O & A_1
\end{pmatrix}.
\]
再对\(A_1\)同样处理,不断做下去,可将\(A\)化为对角阵,并且对角元素均非零.因此我们只要对对角阵证明结论即可.为简化讨论,我们先考虑二阶对角阵:
\[
\begin{pmatrix}
a & 0 \\
0 & b
\end{pmatrix}.
\]
将其第一行乘以\((1 - a)a^{-1}\)加到第行上,再将第二行加到第一行上得到:
\[
\begin{pmatrix}
a & 0 \\
0 & b
\end{pmatrix} \to
\begin{pmatrix}
a & 0 \\
1 - a & b
\end{pmatrix} \to
\begin{pmatrix}
1 & b \\
1 - a & b
\end{pmatrix}.
\]
将其第一列乘以\(-b\)加到第二列上,再将第行乘以\(a - 1\)加到第二行上得到:
\[
\begin{pmatrix}
1 & b \\
1 - a & b
\end{pmatrix} \to
\begin{pmatrix}
1 & 0 \\
1 - a & ab
\end{pmatrix} \to
\begin{pmatrix}
1 & 0 \\
0 & ab
\end{pmatrix}.
\]
从而原结论对二阶对角阵成立.对于$n$阶对角阵$B=diag\{a_1,a_2,\cdots,a_n\}$而言,按照上述方法对$B\left( \begin{matrix}
1&		2\\
1&		2\\
\end{matrix} \right) $所对应的子矩阵进行第三类初等变换得到
\begin{align*}
\left( \begin{matrix}
a_1&		&		&		\\
&		a_2&		&		\\
&		&		\ddots&		\\
&		&		&		a_n\\
\end{matrix} \right) \longrightarrow \left( \begin{matrix}
1&		&		&		\\
&		a_1a_2&		&		\\
&		&		\ddots&		\\
&		&		&		a_n\\
\end{matrix} \right) .
\end{align*}
按照上述方法对再对$B\left( \begin{matrix}
2&		3\\
2&		3\\
\end{matrix} \right) $所对应的子矩阵进行第三类初等变换得到
\begin{align*}
\left( \begin{matrix}
1&		&		&		&		\\
&		a_1a_2&		&		&		\\
&		&		a_3&		&		\\
&		&		&		\ddots&		\\
&		&		&		&		a_n\\
\end{matrix} \right) \longrightarrow \left( \begin{matrix}
1&		&		&		&		\\
&		1&		&		&		\\
&		&		a_1a_2a_3&		&		\\
&		&		&		\ddots&		\\
&		&		&		&		a_n\\
\end{matrix} \right) .
\end{align*}
同理依次对$B\left( \begin{matrix}
k&		k+1\\
k&		k+1\\
\end{matrix} \right),k=1,2\cdots,n-1$所对应的子矩阵按照上述方法进行第三类初等变换,最后得到
\begin{align*}
B=\left( \begin{matrix}
a_1&		&		&		\\
&		a_2&		&		\\
&		&		\ddots&		\\
&		&		&		a_n\\
\end{matrix} \right) \longrightarrow \left( \begin{matrix}
1&		&		&		\\
&		1&		&		\\
&		&		\ddots&		\\
&		&		&		a_1a_2\cdots a_n\\
\end{matrix} \right) .
\end{align*}
于是原结论对对角阵也成立.而我们所用的初等变换始终是第三类初等变换.这就得到了结论.
\end{proof}

\begin{exercise}
求证:任一\(n\)阶矩阵均可表示为形如\(I_n + a_{ij}E_{ij}\)这样的矩阵之积,其中\(E_{ij}\)是\(n\)阶基础矩阵.
\end{exercise}
\begin{proof}
由\hyperref[theorem:相抵标准型]{命题\ref{theorem:相抵标准型}}可知任意一个\(n\)阶矩阵都可表示为有限个初等阵和具有下列形状的对角阵\(D\)之积:
\[
D = \mathrm{diag}\{1,\cdots,1,0,\cdots,0\},
\]
故只要对初等阵和\(D\)证明结论即可.对\(D\),假设\(D\)有\(r\)个\(1\),则
\[
D = (I_n - E_{r + 1,r + 1})\cdots(I_n - E_{nn}).
\]
第三类初等阵已经是这种形状了,即$P_{ij}\left( c \right) =I_n+cE_{ij}$.对第二类初等阵\(P_i(c)\),显然我们有\(P_i(c) = I_n + (c - 1)E_{ii}\).对第一类初等阵\(P_{ij}\),由\hyperref[可逆阵都能只用第三类初等变换化为对角阵]{练习\ref{可逆阵都能只用第三类初等变换化为对角阵}}可知,只用第三类初等变换就可以将\(P_{ij}\)化为\(P_n(-1) = \mathrm{diag}\{1,\cdots,1,-1\}\),因此对第一类初等阵结论也成立.具体地,我们有
\begin{align*}
P_{ij}\cdot P_{ij}\left( -1 \right) P_j\left( -1 \right) P_{ji}\left( -1 \right) P_{ij}\left( 1 \right) =I_n. 
\end{align*}
由此可得
\begin{align*}
&P_{ij}=\left[ P_{ij}\left( -1 \right) P_j\left( -1 \right) P_{ji}\left( -1 \right) P_{ij}\left( 1 \right) \right] ^{-1}=P_{ij}^{-1}\left( 1 \right) P_{ji}^{-1}\left( -1 \right) P_{j}^{-1}\left( -1 \right) P_{ij}^{-1}\left( -1 \right) 
\\
&=P_{ij}\left( -1 \right) P_{ji}\left( 1 \right) P_j\left( -1 \right) P_{ij}\left( 1 \right) =\left( I_n-E_{ij} \right) \left( I_n+E_{ji} \right) \left( I_n-2E_{jj} \right) \left( I_n+E_{ij} \right) .
\end{align*}
\end{proof}


\end{document}
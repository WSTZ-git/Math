% contents/chapter-02/section-03.tex 第二章第三节
\documentclass[../../main.tex]{subfiles}
\graphicspath{{\subfix{../../image/}}} % 指定图片目录,后续可以直接使用图片文件名。

% 例如:
% \begin{figure}[H]
% \centering
% \includegraphics[scale=0.4]{图.png}
% \caption{}
% \label{figure:图}
% \end{figure}
% 注意:上述\label{}一定要放在\caption{}之后,否则引用图片序号会只会显示??.

\begin{document}

\section{伴随矩阵}

\begin{definition}[伴随矩阵定义]\label{definition:伴随矩阵定义}
设$A=(a_{ij})_{n\times n}$,若
\[
A^* = 
\begin{pmatrix}
A_{11} & A_{21} & \cdots & A_{n - 1,1} & A_{n1} \\
A_{12} & A_{22} & \cdots & A_{n - 1,2} & A_{n2} \\
\vdots & \vdots & & \vdots & \vdots \\
A_{1,n - 1} & A_{2,n - 1} & \cdots & A_{n - 1,n - 1} & A_{n - 1,n} \\
A_{1n} & A_{2n} & \cdots & A_{n,n - 1} & A_{nn}
\end{pmatrix}
\]
其中\(A_{ij}\)是\(a_{ij}\)的代数余子式.则称\(A^*\)为\(A\)的\textbf{伴随矩阵}.
\end{definition}

\begin{theorem}\label{theorem:伴随矩阵的基本性质}
设$A$为$n$阶矩阵,$n\geqslant  2$,则
\begin{enumerate}[(i)]
\item $AA^*=A^*A=\left| A \right|I_n$.
\item\label{伴随矩阵基本性质2} 当$A$可逆时,有$A^{-1}=\frac{1}{\left| A \right|}A^*$,$A^*=|A|A^{-1}$.
\end{enumerate}
\end{theorem}
\begin{proof}
由伴随矩阵的定义不难证明.
\end{proof}

\begin{proposition}
设\(A\)为\(n\)阶矩阵,满足\(A^m = I_n\),则\((A^*)^m = I_n\).
\end{proposition}
\begin{proof}
由\(A^m = I_n\)得\(|A|^m = 1\ne 0\),于是矩阵$A$可逆.又\(A^* = |A|A^{-1}\),故$(A^*)^m = |A|^m(A^{-1})^m = (A^m)^{-1} = I_n$.
\end{proof}

\begin{theorem}[矩阵乘积的伴随]\label{theorem:矩阵乘积的伴随}
设\(A,B\)为\(n\)阶矩阵,\(n\geqslant  2\),则\((AB)^* = B^*A^*\).
\end{theorem}
\begin{proof}
{\color{blue}证法一(\hyperref[corollary:Cauchy-Binet公式推论]{Cauchy-Binet公式推论}):}设\(C = AB\).记\(M_{ij}, N_{ij}, P_{ij}\)分别是\(A, B, C\)中第\((i, j)\)元素的余子式,\(A_{ij}, B_{ij}, C_{ij}\)分别是\(A, B, C\)中第\((i, j)\)元素的代数余子式.注意到
\[
A^* = 
\begin{pmatrix}
A_{11} & A_{21} & \cdots & A_{n1} \\
A_{12} & A_{22} & \cdots & A_{n2} \\
\vdots & \vdots & & \vdots \\
A_{1n} & A_{2n} & \cdots & A_{nn}
\end{pmatrix},
\quad
B^* = 
\begin{pmatrix}
B_{11} & B_{21} & \cdots & B_{n1} \\
B_{12} & B_{22} & \cdots & B_{n2} \\
\vdots & \vdots & & \vdots \\
B_{1n} & B_{2n} & \cdots & B_{nn}
\end{pmatrix},
\]
\(B^*A^*\)的第\((i, j)\)元素为\(\sum_{k = 1}^{n} B_{ki}A_{jk}\).而\(C^*\)的第\((i, j)\)元素就是\(C_{ji} = (-1)^{j + i}P_{ji}\).

由\hyperref[corollary:Cauchy-Binet公式推论]{Cauchy-Binet公式推论}可得
\begin{align*}
C_{ji} &= (-1)^{j + i}P_{ji} = (-1)^{j + i} \sum_{k = 1}^{n} M_{jk}N_{ki}\\
&= \sum_{k = 1}^{n} (-1)^{j + k}M_{jk}(-1)^{i + k}N_{ki} = \sum_{k =1}^{n} A_{jk}B_{ki}
\end{align*}
故结论成立.

{\color{blue}证法二(\hyperref[example:摄动法]{摄动法}):}若\(A,B\)均为非异阵,则\(A^* = |A|A^{-1},B^* = |B|B^{-1}\),从而
\[
(AB)^* = |AB|(AB)^{-1} = |A||B|(B^{-1}A^{-1})=(|B|B^{-1})(|A|A^{-1}) = B^*A^*.
\]

由\hyperref[proposition:摄动法基本命题]{命题\ref{proposition:摄动法基本命题}},可知对于一般的方阵\(A,B\),可取到一列有理数\(t_k\rightarrow0\),使得\(t_kI_n + A\)与\(t_kI_n + B\)均为非异阵. 由非异阵情形的证明可得
\[
((t_kI_n + A)(t_kI_n + B))^*=(t_kI_n + B)^*(t_kI_n + A)^*.
\]
注意到上式两边均为\(n\)阶方阵,其元素都是\(t_k\)的多项式,从而关于\(t_k\)连续. 上式两边同时取极限,令\(t_k\rightarrow0\),即有\((AB)^* = B^*A^*\)成立.
\end{proof}

\begin{theorem}[伴随矩阵的秩]\label{theorem:伴随矩阵的秩}
设\(A\)为\(n\)阶矩阵,\(n\geqslant  2\),则
\[
\mathrm{rank}A^* = 
\begin{cases}
n, & \mathrm{rank}A = n, \\
1, & \mathrm{rank}A = n - 1, \\
0, & \mathrm{rank}A < n - 1.
\end{cases}
\]
\end{theorem}
\begin{proof}
当\(\mathrm{rank}A = n\)时,则\(\vert A\vert\neq 0\),\(A\)可逆,又$AA^*=A^*A=\left| A \right|I_n$,两边同时取行列式,可得\(\left| A^* \right|\cdot \left| A \right|=\left| A^*A \right|=\left| \left| A \right|I_n \right|=\left| A \right|^n
\),于是$\left| A^* \right|=\left| A \right|^{n-1}\ne 0$.所以\(\mathrm{rank}A^* = n\).

当\(\mathrm{rank}A = n - 1\)时,\(A\)至少存在一个\(n - 1\)阶子式不等于\(0\),故\(A^*\neq 0\),即\(\mathrm{rank}A^*\geqslant  1\);由\(\mathrm{rank}A < n\)知\(\vert A\vert = 0\),从而\(AA^* = \vert A\vert E = 0\),故由\hyperref[corollary:矩阵的秩不等式1]{定理\ref{corollary:矩阵的秩不等式1}}可知\(\mathrm{rank}A^* \leqslant  n - \mathrm{rank}A = 1\),于是\(\mathrm{rank}A^* = 1\).
(另证:若\(\boldsymbol{A}\)的秩等于\(n - 1\),则由\hyperref[proposition:奇异系数矩阵Ax=0的解空间]{命题\ref{proposition:奇异系数矩阵Ax=0的解空间}}可知\(\boldsymbol{A}^*\)的\(n\)个列向量都成比例且至少有一列不为零,故\(\boldsymbol{A}^*\)的秩等于\(1\).)

当\(\mathrm{rank}A < n - 1\)时,\(A\)的所有\(n - 1\)阶子式均等于\(0\),即\(A^* = 0\),故\(\mathrm{rank}A^* = 0\).
\end{proof}


\begin{proposition}[伴随矩阵的性质]\label{proposition:伴随矩阵的性质}
设\(A\)为\(n\)阶矩阵,\(n\geqslant  2\),则
\begin{enumerate}
\item\label{伴随矩阵的性质1}  \((A^{\mathrm{T}})^* = (A^*)^{\mathrm{T}}\).

\item  \((kA)^* = k^{n - 1}A^*\),\(k\)为常数.

\item\label{伴随矩阵的性质3}  若$A$为可逆阵,则$A^*$也可逆,并且\((A^{-1})^* = (A^*)^{-1}\).

\item  \((A^{m})^* = (A^*)^{m}\),\(m\)为正整数.

\item\label{伴随矩阵的性质5} 若$A$可逆,则\(|A^*| = |A|^{n - 1}\).

\item  \((A^*)^* = |A|^{n - 2}A\).
\end{enumerate}
\end{proposition}
\begin{proof}
\begin{enumerate}
\item 由伴随矩阵的定义及行列式的性质即得.

\item 由伴随矩阵的定义及行列式的性质即得.

\item 由\hyperref[theorem:矩阵乘积的伴随]{定理\ref{theorem:矩阵乘积的伴随}}可知
$A^*\left( A^{-1} \right) ^*=\left( A^{-1}A \right) ^*=I_{n}^{*}=I_n$.从而$(A^{-1})^* = (A^*)^{-1}$.

\item 多次利用\hyperref[theorem:矩阵乘积的伴随]{定理\ref{theorem:矩阵乘积的伴随}}即得.

\item {\color{blue}证法一:}当\(A\)可逆时,有\(A^* = |A|A^{-1}\),从而\(|A^*| = |A|^{n - 1}\);当\(A\)不可逆时,有$\mathrm{rank}A<n$,由\hyperref[theorem:伴随矩阵的秩]{定理\ref{theorem:伴随矩阵的秩}}知$\mathrm{rank}A^*<n$.于是\(|A^*| = |A| = 0\),故\(|A^*| = |A|^{n - 1}\).

{\color{blue}证法二:}若\(A\)是非异阵,有\(A^* = |A|A^{-1}\),从而\(|A^*| = |A|^{n - 1}\). 对于一般的方阵\(A\),由\hyperref[proposition:摄动法基本命题]{命题\ref{proposition:摄动法基本命题}}可知,可取到一列有理数\(t_k\rightarrow0\),使得\(t_kI_n + A\)为非异阵. 由非异阵情形的证明可得
\[
|(t_kI_n + A)^*| = |t_kI_n + A|^{n - 1}.
\]
注意到上式两边均为行列式的幂次,其值都是\(t_k\)的多项式,从而关于\(t_k\)连续. 上式两边同时取极限(上式两边都是关于$t_k$的多项式函数),令\(t_k\rightarrow0\),即有\(|A^*| = |A|^{n - 1}\)成立.

\item {\color{blue}证法一:}当\(A\)可逆时,\(A^*\)也可逆,且\((A^*)^{-1} = \frac{1}{|A|}A\),从而由 \hyperref[伴随矩阵的性质5]{伴随矩阵的性质\ref{伴随矩阵的性质5}}得
\[
(A^*)^* = |A^*|(A^*)^{-1} = |A|^{n - 1}\frac{1}{|A|}A = |A|^{n - 2}A.
\]

当\(A\)不可逆时,则\(|A| = 0\),且由\hyperref[theorem:伴随矩阵的秩]{定理\ref{theorem:伴随矩阵的秩}}及$n\geqslant  2$知\(\mathrm{rank}A^* \leqslant  1<n-1\),从而\(\mathrm{rank}(A^*)^* = 0\),即\((A^*)^* = 0\),因此\((A^*)^* = |A|^{n - 2}A\).

{\color{blue}证法二:}若\(A\)是非异阵,\(A^*\)也可逆,且\((A^*)^{-1} = \frac{1}{|A|}A\),从而由 \hyperref[伴随矩阵的性质5]{伴随矩阵的性质\ref{伴随矩阵的性质5}}得
\[
(A^*)^* = |A^*|(A^*)^{-1} = |A|^{n - 1}\frac{1}{|A|}A = |A|^{n - 2}A.
\]
对于一般的方阵\(A\),由\hyperref[proposition:摄动法基本命题]{命题\ref{proposition:摄动法基本命题}}可知,可取到一列有理数\(t_k\rightarrow0\),使得\(t_kI_n + A\)为非异阵. 由非异阵情形的证明可得
\[
((t_kI_n + A)^*)^* = |t_kI_n + A|^{n - 2}(t_kI_n + A).
\]
注意到上式两边均为\(n\)阶方阵,其元素都是\(t_k\)的多项式(上式两边的矩阵每个元素都是关于$t_k$的多项式函数),从而关于\(t_k\)连续. 上式两边同时取极限,令\(t_k\rightarrow0\),即有\((A^*)^* = |A|^{n - 2}A\)成立. 
\end{enumerate}
\end{proof}

\begin{proposition}[伴随矩阵的继承性]\label{proposition:伴随矩阵的继承性}
\begin{enumerate}
\item\label{伴随矩阵的继承性1} 对角矩阵的伴随矩阵是对角矩阵;

\item 对称矩阵的伴随矩阵是对称矩阵;

\item 上(下)三角矩阵的伴随矩阵是上(下)三角矩阵;

\item 可逆矩阵的伴随矩阵是可逆;

\item 正交矩阵的伴随矩阵是正交矩阵;

\item 半正定(正定)矩阵的伴随矩阵是半正定(正定)矩阵;

\item 可对角化矩阵的伴随矩阵是可对角化矩阵.
\end{enumerate}
\end{proposition}
\begin{proof}
\begin{enumerate}
\item 设\(n\)阶矩阵\(A = (a_{ij}),\ n \geqslant  2\).

\item 若\(A\)为对角矩阵,则\(a_{ij} = 0 (i \neq j)\),从而\(i \neq j\)时,\(M_{ij}\)是对角行列式,且主对角元必有零,即\(M_{ij} = 0\),故\(A_{ij} = 0\),于是\(A^*\)为对角矩阵.

\item 若\(A\)为对称矩阵,则\(a_{ij} = a_{ji} (i, j = 1, 2, \cdots, n)\),因此\(i, j = 1, 2, \cdots, n\)时,\(M_{ij}\)是对称行列式,从而\(A_{ij} = A_{ji}\),即\(A^*\)为对称矩阵.

\item 若\(A\)为上三角矩阵,则\(1 \leqslant  j < i \leqslant  n\)时,\(a_{ij} = 0\),所以\(1 \leqslant  i < j \leqslant  n\)时,\(M_{ij}\)是上三角行列式,且主对角元必有零,即\(M_{ij} = 0\),从而\(A_{ij} = 0\),所以\(A^*\)为上三角矩阵.同理可证:下三角矩阵的伴随矩阵是下三角矩阵.

\item 由\(\vert A \vert \neq 0\)和\(A^* = \vert A \vert A^{-1}\)即知.

\item 因为\(A\)为正交矩阵等价于\(A^{-1} = A^{\mathrm{T}}\),所以\(\vert A \vert^{-1} = \vert A \vert\).从而由\hyperref[伴随矩阵基本性质2]{定理\ref{theorem:伴随矩阵的基本性质}\ref{伴随矩阵基本性质2}},有
\[
(A^*)^{-1} = \vert A \vert^{-1} A = (\vert A \vert A^{\mathrm{T}})^{\mathrm{T}} = (A^*)^{\mathrm{T}},
\]
故\(A^*\)为正交矩阵.

\item 由于\(A\)为半正定矩阵等价于存在实矩阵\(C\),使得\(A = C^{\mathrm{T}}C\),因此由\hyperref[theorem:矩阵乘积的伴随]{定理\ref{theorem:矩阵乘积的伴随}}和\hyperref[伴随矩阵的性质1]{伴随矩阵的性质\ref{伴随矩阵的性质1}},有
\[
A^* = (C^{\mathrm{T}}C)^* = C^*(C^{\mathrm{T}})^* = C^*(C^*)^{\mathrm{T}},
\]
于是\(A^*\)为半正定矩阵.当\(A\)为正定矩阵时,同理可证\(A^*\)为正定矩阵.

\item 若\(A\)可对角化,则存在可逆矩阵\(P\),使得\(A = P\Lambda P^{\mathrm{T}}\),其中\(\Lambda\)为对角矩阵,从而由\hyperref[theorem:矩阵乘积的伴随]{定理\ref{theorem:矩阵乘积的伴随}}和\hyperref[伴随矩阵的性质1]{伴随矩阵的性质\ref{伴随矩阵的性质1}},有
\[
A^* = (P^{\mathrm{T}})^*\Lambda^*P^* = (P^*)^{\mathrm{T}}\Lambda^*P^*,
\]
再根据\hyperref[伴随矩阵的继承性1]{伴随矩阵的继承性\ref{伴随矩阵的继承性1}}和\hyperref[伴随矩阵的性质3]{伴随矩阵的性质\ref{伴随矩阵的性质3}},知\(\Lambda^*\)为对角矩阵,\(P^*\)为可逆矩阵,故\(A^*\)可对角化.
\end{enumerate}
\end{proof}

\begin{proposition}[分块矩阵的伴随矩阵]\label{proposition:分块矩阵的伴随矩阵}
设\(A\)为\(m\)阶矩阵,\(B\)为\(n\)阶矩阵,分块对角阵\(C\)为
\[
C = 
\begin{pmatrix}
A & O \\
O & B
\end{pmatrix}.
\]
则分块对角阵$C$的伴随矩阵为:
\[
C^* = 
\begin{pmatrix}
|B|A^* & O \\
O & |A|B^*
\end{pmatrix}.
\]
\end{proposition}
\begin{proof}
{\color{blue}证法一:}
设\(A = (a_{ij})_{m\times m}\),元素\(a_{ij}\)的余子式和代数余子式分别记为\(M_{ij}\)和\(A_{ij}\);\(B = (b_{ij})_{n\times n}\),元素\(b_{ij}\)的余子式和代数余子式分别记为\(N_{ij}\)和\(B_{ij}\).利用Laplace定理可以容易地计算出:当\(1 \leqslant  i, j \leqslant  m\)时,\(C\)的第\((i, j)\)元素的代数余子式为\((-1)^{i + j}M_{ij}|B| = |B|A_{ij}\);当\(m + 1 \leqslant  i, j \leqslant  m + n\)时,由Laplace定理,可知\(C\)的第\((i, j)\)元素的代数余子式为\((-1)^{i + j}N_{i - m, j - m}|A| = |A|B_{i - m, j - m}\);当\(i, j\)属于其他范围时,由Laplace定理,当$1\leqslant  i\leqslant  m,m\leqslant  j\leqslant  m+n$时,将其按前$m$列展开,当$m\leqslant  i\leqslant  m+n,1\leqslant  j\leqslant  m$时,将其按前$m$行展开,可得\(C\)的第\((i, j)\)元素的代数余子式等于零.因此我们有
\[
C^* = 
\begin{pmatrix}
|B|A^* & O \\
O & |A|B^*
\end{pmatrix}.
\]
{\color{blue}证法二:}若\(A,B\)均为非异阵,则
\begin{align*}
C^*=\left| C \right|C^{-1}=\left| A \right|\left| B \right|\left( \begin{matrix}
A^{-1}&		O\\
O&		B^{-1}\\
\end{matrix} \right) =\left( \begin{matrix}
\left| B \right|\left| A \right|A^{-1}&		O\\
O&		\left| A \right|\left| B \right|B^{-1}\\
\end{matrix} \right) =\left( \begin{matrix}
|B|A^*&		O\\
O&		|A|B^*\\
\end{matrix} \right) .
\end{align*}

对于一般的方阵\(A,B\),由\hyperref[proposition:摄动法基本命题]{命题\ref{proposition:摄动法基本命题}}可知,可取到一列有理数\(t_k\rightarrow0\),使得\(t_kI_m + A\)与\(t_kI_n + B\)均为非异阵. 由非异阵情形的证明可得
\[
\begin{pmatrix}
t_kI_m + A & O\\
O & t_kI_n + B
\end{pmatrix}^*=\begin{pmatrix}
|t_kI_n + B|(t_kI_m + A)^* & O\\
O & |t_kI_m + A|(t_kI_n + B)^*
\end{pmatrix}.
\]
注意到上式两边均为\(m + n\)阶方阵,其元素都是\(t_k\)的多项式,从而关于\(t_k\)连续. 上式两边同时取极限,令\(t_k\rightarrow0\),即有\(\begin{pmatrix}
A & O\\
O & B
\end{pmatrix}^*=\begin{pmatrix}
|B|A^* & O\\
O & |A|B^*
\end{pmatrix}\)成立.
\end{proof}

\begin{example}
设\(A,B\)为\(n\)阶方阵,满足\(AB = BA\),证明:\(AB^* = B^*A\).
\end{example}
\begin{proof}
若\(B\)为非异阵,则由\(AB = BA\)可得\(AB^{-1}=B^{-1}A\). 又\(B^* = |B|B^{-1}\),于是\(AB^* = B^*A\)成立. 对于一般的方阵\(B\),可取到一列有理数\(t_k\rightarrow0\),使得\(t_kI_n + B\)为非异阵,此时\(A(t_kI_n + B)=(t_kI_n + B)A\)仍然成立. 由非异阵情形的证明可得
\[
A(t_kI_n + B)^*=(t_kI_n + B)^*A.
\]
注意到上式两边均为\(n\)阶方阵,其元素都是\(t_k\)的多项式,从而关于\(t_k\)连续. 上式两边同时取极限,令\(t_k\rightarrow0\),即有\(AB^* = B^*A\)成立.
\end{proof}

\begin{example}
设\(n\)阶矩阵
\[
A = 
\begin{pmatrix}
2 & 2 & 2 & \cdots & 2 \\
0 & 1 & 1 & \cdots & 1 \\
0 & 0 & 1 & \cdots & 1 \\
\vdots & \vdots & \vdots & & \vdots \\
0 & 0 & 0 & \cdots & 1
\end{pmatrix},
\]
求\(\sum_{i,j = 1}^{n} A_{ij}\).
\end{example}
\begin{solution}
{\color{blue}解法一:}显然\(\vert A\vert = 2\),用初等变换不难求出
\[
A^{-1} = 
\begin{pmatrix}
\frac{1}{2} & -1 & 0 & \cdots & 0 & 0 \\
0 & 1 & -1 & \cdots & 0 & 0 \\
\vdots & \vdots & \vdots & & \vdots & \vdots \\
0 & 0 & 0 & \cdots & 1 & -1 \\
0 & 0 & 0 & \cdots & 0 & 1
\end{pmatrix},
\]
故
\[
A^* = 2A^{-1} = 
\begin{pmatrix}
1 & -2 & 0 & \cdots & 0 & 0 \\
0 & 2 & -2 & \cdots & 0 & 0 \\
\vdots & \vdots & \vdots & & \vdots & \vdots \\
0 & 0 & 0 & \cdots & 2 & -2 \\
0 & 0 & 0 & \cdots & 0 & 2
\end{pmatrix}.
\]
将\(A^*\)的所有元素加起来,可得\(\sum_{i,j = 1}^{n} A_{ij} = 1\).

{\color{blue}解法二:}由\hyperref[根据行列式代数余子式构造行列式]{命题\ref{根据行列式代数余子式构造行列式}}可得
\begin{align*}
-\sum_{i,j=1}^n{A_{ij}}=\left| \begin{matrix}
2&		2&		2&		\cdots&		2&		1\\
0&		1&		1&		\cdots&		1&		1\\
0&		0&		1&		\cdots&		1&		1\\
\vdots&		\vdots&		\vdots&		&		\vdots&		\vdots\\
0&		0&		0&		\cdots&		1&		1\\
1&		1&		1&		\cdots&		1&		0\\
\end{matrix} \right|=\left| \begin{matrix}
2&		2&		2&		\cdots&		2&		1\\
0&		1&		1&		\cdots&		1&		1\\
0&		0&		1&		\cdots&		1&		1\\
\vdots&		\vdots&		\vdots&		&		\vdots&		\vdots\\
0&		0&		0&		\cdots&		1&		1\\
0&		0&		0&		\cdots&		0&		-\frac{1}{2}\\
\end{matrix} \right|=-1.
\end{align*}
于是$\sum_{i,j=1}^n{A_{ij}}=1$.

{\color{blue}解法三:}由\hyperref[大拆分法]{大拆分法}可得$\left| A\left( -1 \right) \right|=\left| A \right|-\sum_{i,j=1}^n{A_{ij}}$,且
\begin{align*}
\left| A\left( -1 \right) \right|=\left| \begin{matrix}
1&		1&		\cdots&		1&		1\\
-1&		0&		\cdots&		0&		0\\
-1&		-1&		\cdots&		0&		0\\
\vdots&		\vdots&		&		\vdots&		\vdots\\
-1&		-1&		\cdots&		-1&		0\\
\end{matrix} \right|=\left( -1 \right) ^{n+1}\left| \begin{matrix}
-1&		0&		\cdots&		0\\
-1&		-1&		\cdots&		0\\
\vdots&		\vdots&		&		\vdots\\
-1&		-1&		\cdots&		-1\\
\end{matrix} \right|=1.
\end{align*}
故$\sum_{i,j=1}^n{A_{ij}}=\left| A\left( -1 \right) \right|-\left| A \right|$.

{\color{blue}解法四:}由\hyperref[example:求矩阵代数余子式和的方法1]{例题\ref{example:求矩阵代数余子式和的方法1}}可得
\begin{align*}
\sum_{i,j=1}^n{A_{ij}}=\left| \begin{matrix}
0&		0&		\cdots&		0&		1\\
-1&		0&		\cdots&		0&		1\\
0&		-1&		\cdots&		0&		1\\
\vdots&		\vdots&		&		\vdots&		\vdots\\
0&		0&		\cdots&		-1&		1\\
\end{matrix} \right|=(-1)^{n+1}\left| \begin{matrix}
-1&		0&		\cdots&		0\\
0&		-1&		\cdots&		0\\
\vdots&		\vdots&		&		\vdots\\
0&		0&		\cdots&		-1\\
\end{matrix} \right|=1.
\end{align*}
\end{solution}


\end{document}
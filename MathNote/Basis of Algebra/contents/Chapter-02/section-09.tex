% contents/chapter-02/section-09.tex 第二章第三节
\documentclass[../../main.tex]{subfiles}
\graphicspath{{\subfix{../../image/}}} % 指定图片目录,后续可以直接使用图片文件名。

% 例如:
% \begin{figure}[H]
% \centering
% \includegraphics[scale=0.4]{图.png}
% \caption{}
% \label{figure:图}
% \end{figure}
% 注意:上述\label{}一定要放在\caption{}之后,否则引用图片序号会只会显示??.

\begin{document}


\section{摄动法}
\textbf{摄动法的原理}

(1) 证明矩阵问题对非异阵成立.

(2) 对任意的 \(n\) 阶矩阵 \(A\), 由上例可知, 存在一列有理数 \(t_k\rightarrow0\), 使得 \(t_kI_n + A\) 都是非异阵. 验证 \(t_kI_n + A\) 仍满足矩阵问题的条件, 从而该问题对 \(t_kI_n + A\) 成立.

(3) 若矩阵问题关于 \(t_k\) 连续, 则可取极限令 \(t_k\rightarrow0\), 从而得到该问题对一般的矩阵 \(A\) 也成立.

\begin{remark}
\begin{enumerate}
\item 矩阵问题对非异阵成立以及矩阵问题关于 \(t_k\) 连续, 这两个要求缺一不可, 否则将不能使用摄动法进行证明.
\item 验证摄动矩阵仍然满足矩阵问题的条件是必要的. 例如, 若矩阵问题中有 \(AB=-BA\) 这一条, 但 \((t_kI_n + A)B\neq -B(t_kI_n + A)\), 因此便不能使用摄动法.
\item 根据实际问题的需要, 也可以使用其他非异阵来替代 \(I_n\) 对 \(A\) 进行摄动.
\end{enumerate}
\end{remark}
\begin{note}
关于伴随矩阵的问题中经常会使用摄动法.
\end{note}


\begin{proposition}\label{proposition:摄动法基本命题}
设\(A\)是一个\(n\)阶方阵,求证:存在一个正数\(a\),使得对任意的\(0 < t < a\),矩阵\(tI_n+A\)都是非异阵. 
\end{proposition}
\begin{note}
这个命题告诉我们对任意的 \(n\) 阶矩阵 \(A\),经过微小的一维摄动之后,\(tI_n + A\) 总能成为一个非异阵.
\end{note}
\begin{proof}
通过简单的计算可得
\[
|tI_n + A| = t^n + a_1t^{n - 1}+\cdots+a_{n - 1}t + a_n,
\]
这是一个关于未定元\(t\)的\(n\)次多项式. 由\hyperref[proposition:多项式根的有限性]{多项式根的有限性}可知上述多项式至多只有\(n\)个不同的根. 若上述多项式的根都是零,则不妨取\(a = 1\);若上述多项式有非零根, 则令\(a\)为\(|tI_n + A|\)所有非零根的模长的最小值. 因此对任意的\(0 < t_0 < a\), \(t_0\)都不是\(|tI_n + A|\)的根, 即\(|t_0I_n + A|\neq0\), 从而\(t_0I_n + A\)是非异阵.

\end{proof}

\begin{corollary}\label{corollary:矩阵可逆摄动的推论}
设\(A\)是一个\(n\)阶方阵,$P$是一个$n$阶可逆方阵,求证:存在一个正数\(a\),使得对任意的\(0 < t < a\),矩阵\(tP+A\)都是非异阵. 
\end{corollary}
\begin{proof}
由\refpro{proposition:摄动法基本命题}知,存在一个正数$a$,使得对$\forall t\in (0,a)$,都有$tI_n+P^{-1}A$是非异阵.于是
\begin{align*}
|tP+A|=|P||tI_n+P^{-1}A|\ne 0.
\end{align*}
故对$\forall t\in (0,a)$,$|tP+A|$都是非异阵.

\end{proof}

\begin{example}
设\(A,B,C,D\)是\(n\)阶矩阵且\(AC = CA\),求证:
\[
\begin{vmatrix}
A & B\\
C & D
\end{vmatrix}=|AD - CB|.
\]
\end{example}
\begin{note}
本题也给出了\hyperref[example:2.28]{例题\ref{example:2.28}}的摄动法证明.
\end{note}
\begin{proof}
若\(A\)为非异阵,则由降阶公式,再结合条件\(AC = CA\)可得
\[
\left|\begin{matrix}
A & B\\
C & D
\end{matrix}\right| = |A|\left|D - CA^{-1}B\right| = \left|AD - ACA^{-1}B\right| = |AD - CB|.
\]
对于一般的方阵\(A\),由\refpro{proposition:摄动法基本命题}可知,存在一列有理数\(t_k\rightarrow 0\),使得\(t_kI_n + A\)是非异阵,并且条件\((t_kI_n + A)C = C(t_kI_n + A)\)仍然成立.于是
\[
\left|\begin{matrix}
t_kI_n + A & B\\
C & D
\end{matrix}\right| = \left|(t_kI_n + A)D - CB\right|.
\]
上式两边都是行列式,其值都是\(t_k\)的多项式,从而都关于\(t_k\)连续.上式两边同时令\(t_k\rightarrow 0\),即有\(\left|\begin{matrix}
A & B\\
C & D
\end{matrix}\right| = |AD - CB|\)成立.

\end{proof}

\begin{example}
设 \( A,B,C,D \) 为 \( n \) 阶矩阵,且 \( CD^T = DC^T \),证明:
$$\det \begin{pmatrix} A & B \\ C & D \end{pmatrix} = \det(AD^T - BC^T).$$
\end{example}
\begin{proof}
(i) 若$D$可逆,则由$CD^T=DC^T$可知
$$D^{-1}CD^T=C^T.$$
于是
\begin{align*}
\left| \begin{matrix}
A&		B\\
C&		D\\
\end{matrix} \right|&=\left| \begin{matrix}
A&		B\\
O&		D-CA^{-1}B\\
\end{matrix} \right|=\left| D \right|\left| A-BD^{-1}C \right|
\\
&=\left| A-BD^{-1}C \right|\left| D^T \right|=\left| AD^T-BD^{-1}CD^T \right|
\\
&=\left| AD^T-BC^T \right|.
\end{align*}

(ii) 若$D$不可逆,则设$\text{r}(C)=r$,则存在可逆阵$P,Q$,使得
$$PCQ=\begin{pmatrix} E_r & O \\ O & O \end{pmatrix}.$$
由\refcor{corollary:矩阵可逆摄动的推论}知,存在$a>0$,使得对$\forall t\in (0,a)$,都有$tP^{-1}Q^T+D$是可逆阵.
注意到
\begin{gather*}
C\left( tP^{-1}Q^T+D \right) ^T=tCQ\left( P^{-1} \right) ^T+CD^T=tP^{-1}\left( \begin{matrix}
E_r&		O\\
O&		O\\
\end{matrix} \right) \left( P^{-1} \right) ^T+CD^T;
\\
\left( tP^{-1}Q^T+D \right) C^T=tP^{-1}Q^TC^T+DC^T=tP^{-1}\left( \begin{matrix}
E_r&		O\\
O&		O\\
\end{matrix} \right) ^T\left( P^{-1} \right) ^T+CD^T.
\end{gather*}
故$C\left( tP^{-1}Q^T+D \right)^T=\left( tP^{-1}Q^T+D \right) C^T$.因此由(i)同理可得
$$\begin{vmatrix} A & B \\ C & tP^{-1}Q^T+D \end{vmatrix}=\left| A\left( tP^{-1}Q^T+D \right)^T-BC^T \right|,\quad \forall t\in (0,a).$$
令$t\rightarrow 0^+$得
$$\begin{vmatrix} A & B \\ C & D \end{vmatrix}=\left| AD^T-BC^T \right|.$$

\end{proof}




\end{document}
% contents/chapter-02/section-10.tex 第二章第三节
\documentclass[../../main.tex]{subfiles}
\graphicspath{{\subfix{../../image/}}} % 指定图片目录,后续可以直接使用图片文件名。

% 例如:
% \begin{figure}[H]
% \centering
% \includegraphics[scale=0.4]{图.png}
% \caption{}
% \label{figure:图}
% \end{figure}
% 注意:上述\label{}一定要放在\caption{}之后,否则引用图片序号会只会显示??.

\begin{document}

\section{其他}

\begin{exercise}
设\(A=(a_{ij})\)为\(n\)阶方阵,定义函数\(f(A)=\sum_{i,j = 1}^{n}a_{ij}^2\). 设\(P\)为\(n\)阶可逆矩阵,使得对任意的\(n\)阶方阵\(A\)成立:\(f(PAP^{-1}) = f(A)\). 证明:存在非零常数\(c\),使得\(P'P = cI_n\).
\end{exercise}
\begin{proof}
{\color{blue}证法一:}
由假设知\(f(A)=\text{tr}(AA')\),因此
\[
f(PAP^{-1})=\text{tr}(PAP^{-1}(P')^{-1}A'P')=\text{tr}((P'P)A(P'P)^{-1}A')=\text{tr}(AA').
\]
以下设\(P'P=(c_{ij})\),\((P'P)^{-1}=(d_{ij})\). 注意\(P'P\)是对称矩阵,后面要用到. 令\(A = E_{ij}\),其中$1\leqslant  i,j\leqslant  n$.并将其代入$(P'P)A(P'P)^{-1}A'$可得
\begin{align*}
&(P'P)A(P'P)^{-1}A'=(P'P)E_{ij}(P'P)^{-1}E_{ji}
\\
&=\bordermatrix{%
&    &       &             j&     &
\cr
&    &		&		c_{1i}&		&		\cr
&    &		&		c_{2i}&		&		\cr
&    &		&		\vdots&		&		\cr
&    &		&		c_{ii}&		&		\cr
&    &		&		\vdots&		&		\cr
&    &		&		c_{ni}&		&		\cr
}\bordermatrix{%
&    &       &             i&     &
\cr
&    &		&		d_{1j}&		&		\cr
&    &		&		d_{2j}&		&		\cr
&    &		&		\vdots&		&		\cr
&    &		&		d_{jj}&		&		\cr
&    &		&		\vdots&		&		\cr
&    &		&		d_{nj}&		&		\cr
}=\bordermatrix{%
&    &       &             i&     &
\cr
&    &		&		c_{1i}d_{jj}&		&		\cr
&    &		&		c_{2i}d_{jj}&		&		\cr
&    &		&		\vdots&		&		\cr
&    &		&		c_{ii}d_{jj}&		&		\cr
&    &		&		\vdots&		&		\cr
&    &		&		c_{ni}d_{jj}&		&		\cr
}
\end{align*}
于是$\mathrm{tr}\left( \left( P'P \right) A\left( P'P \right) ^{-1}A' \right) =c_{ii}d_{jj}$.
而$\mathrm{tr}\left( AA' \right) =\mathrm{tr}\left( E_{ij}E_{ji} \right) =\mathrm{tr}\left( E_{ii} \right) =1$.
则由$\text{tr}((P'P)A(P'P)^{-1}A')=\text{tr}(AA')$可知
\begin{align}\label{equation:eq542}
c_{ii}d_{jj}=1. 
\end{align}
再令\(A = E_{ij}+E_{kl}\),其中$1\leqslant  i,j,k,l\leqslant  n$.不妨设$k\geqslant  i,l\geqslant  j$,将其代入$(P'P)A(P'P)^{-1}A'$可得
\begin{align*}
&(P'P)A(P'P)^{-1}A'=(P'P)(E_{ij}+E_{kl})(P'P)^{-1}(E_{ji}+E_{lk})
\\
&=\left[\bordermatrix{%
&    &       &             j&     &
\cr
&    &		&		c_{1i}&		&		\cr
&    &		&		c_{2i}&		&		\cr
&    &		&		\vdots&		&		\cr
&    &		&		c_{ii}&		&		\cr
&    &		&		\vdots&		&		\cr
&    &		&		c_{ki}&		&		\cr
&    &		&		\vdots&		&		\cr
&    &		&		c_{ni}&		&		\cr
}+\bordermatrix{%
&    &       &             l&     &
\cr
&    &		&		c_{1k}&		&		\cr
&    &		&		c_{2k}&		&		\cr
&    &		&		\vdots&		&		\cr
&    &		&		c_{ik}&		&		\cr
&    &		&		\vdots&		&		\cr
&    &		&		c_{kk}&		&		\cr
&    &		&		\vdots&		&		\cr
&    &		&		c_{nk}&		&		\cr
}\right]\left[\bordermatrix{%
&    &       &             i&     &
\cr
&    &		&		d_{1j}&		&		\cr
&    &		&		d_{2j}&		&		\cr
&    &		&		\vdots&		&		\cr
&    &		&		d_{jj}&		&		\cr
&    &		&		\vdots&		&		\cr
&    &		&		d_{lj}&		&		\cr
&    &		&		\vdots&		&		\cr
&    &		&		d_{nj}&		&		\cr
}+\bordermatrix{%
&    &       &             k&     &
\cr
&    &		&		d_{1l}&		&		\cr
&    &		&		d_{2l}&		&		\cr
&    &		&		\vdots&		&		\cr
&    &		&		d_{jl}&		&		\cr
&    &		&		\vdots&		&		\cr
&    &		&		d_{ll}&		&		\cr
&    &		&		\vdots&		&		\cr
&    &		&		d_{nl}&		&		\cr
}\right]\\
&=\bordermatrix{%
&    &       &   j&  &          l&     &
\cr
&    &		&	c_{1i}& \cdots&c_{1k}&		&		\cr
&    &		&	c_{2i}&	\cdots&c_{2k}&		&		\cr
&    &		&	\vdots&	&\vdots&		&		\cr
&    &		&	c_{ii}&	\cdots&c_{ik}&		&		\cr
&    &		&	\vdots&	&\vdots&		&		\cr
&    &		&	c_{ki}&	\cdots&c_{kk}&		&		\cr
&    &		&	\vdots&	&\vdots&		&		\cr
&    &		&	c_{ni}&	\cdots&c_{nk}&		&		\cr
}\bordermatrix{%
&    &       &   i&  &          k&     &
\cr
&    &		&	d_{1j}& \cdots&	d_{1l}&		&		\cr
&    &		&	d_{2j}&	\cdots&d_{2l}&		&		\cr
&    &		&	\vdots&	&\vdots&		&		\cr
&    &		&	d_{jj}&	\cdots&d_{jl}&		&		\cr
&    &		&	\vdots&	&\vdots&		&		\cr
&    &		&	d_{lj}&	\cdots&d_{ll}&		&		\cr
&    &		&	\vdots&	&\vdots&		&		\cr
&    &		&	d_{nj}&	\cdots&d_{nl}&		&		\cr
}=\bordermatrix{%
&    &       &   i&  &          k&     &
\cr
&    &		&	c_{1i}d_{jj}+c_{1k}d_{lj}& \cdots&	c_{1i}d_{jl}+c_{1k}d_{ll}&		&		\cr
&    &		&	c_{2i}d_{jj}+c_{2k}d_{lj}& \cdots&	c_{2i}d_{jl}+c_{2k}d_{ll}&		&		\cr
&    &		&	\vdots&	&\vdots&		&		\cr
&    &		&	c_{ii}d_{jj}+c_{ik}d_{lj}& \cdots&	c_{ii}d_{jl}+c_{ik}d_{ll}&		&		\cr
&    &		&	\vdots&	&\vdots&		&		\cr
&    &		&	c_{ki}d_{jj}+c_{kk}d_{lj}& \cdots&	c_{ki}d_{jl}+c_{kk}d_{ll}&		&		\cr
&    &		&	\vdots&	&\vdots&		&		\cr
&    &		&	c_{ni}d_{jj}+c_{nk}d_{lj}& \cdots&	c_{ni}d_{jl}+c_{nk}d_{ll}&		&		\cr
}
\end{align*}
从而$\mathrm{tr}\left( \left( P'P \right) A\left( P'P \right) ^{-1}A' \right) =c_{ii}d_{jj}+c_{kk}d_{ll}+c_{ki}d_{jl}+c_{ik}d_{lj}$.而
\begin{align*}
\mathrm{tr}\left( AA' \right) =\mathrm{tr}\left( \left( E_{ij}+E_{kl} \right) \left( E_{ji}+E_{lk} \right) \right) =\mathrm{tr}\left( E_{ij}E_{ji}+E_{ij}E_{lk}+E_{kl}E_{ji}+E_{kl}E_{lk} \right)=2 + 2\delta_{ik}\delta_{jl}.    
\end{align*}
于是由$\text{tr}((P'P)A(P'P)^{-1}A')=\text{tr}(AA')$可知
\begin{align}\label{equation:eq745}
c_{ii}d_{jj}+c_{kk}d_{ll}+c_{ki}d_{jl}+c_{ik}d_{lj}=2 + 2\delta_{ik}\delta_{jl},  
\end{align}
其中\(\delta_{ik}\)是Kronecker符号. 由上述\eqref{equation:eq542}\eqref{equation:eq745}两式可得
\[
c_{ki}d_{jl}+c_{ik}d_{lj}=2\delta_{ik}\delta_{jl}.
\]
在上式中令\(j = l\),\(i\neq k\),注意到\(d_{jj}\neq0\),故有\(c_{ik}+c_{ki}=0\),又因为\(P'P\)是对称矩阵,所以\(c_{ik}=c_{ki}\).故\(c_{ik}=0\),\(\forall i\neq k\). 于是\(P'P\)是一个对角矩阵,从而由\eqref{equation:eq542}式可得\(d_{jj}=c_{jj}^{-1}\),由此可得\(c_{ii}=c_{jj}\),\(\forall i,j\). 因此\(P'P = cI_n\),其中\(c = c_{11}\neq0\).

{\color{blue}证法二:}
我们把数域限定在实数域上,并取 \(V = M_n(\mathbb{R})\) 上的 \hyperlink{Frobenius 内积}{Frobenius 内积},则 \(f(A)=\sum_{i,j = 1}^{n}a_{ij}^2=\|A\|^2\)。设 \(\varphi(A)=PAP^{-1}\) 为 \(V\) 上的线性变换,则题目条件可改写为 \(\|\varphi(A)\|=\|A\|\) 对任意的 \(A\in V\) 成立,于是由\nrefthe{theorem:正交变换和酉变换的基本性质}{-2}可知 \(\varphi\) 是正交算子,从而由\nrefpro{proposition:例9.29}{(2)}即得结论。
\end{proof}


\subsection{矩阵方幂的计算}

计算方阵的方幂一般有三种方法:

(1) 基于相似变换的计算法, 更常见的情形是利用相似标准形 (比如 Jordan 标准形) 来进行计算.

(2) 利用递推公式法或者说数学归纳法.

(3) 基于方阵分解. 然后利用二项式定理来进行计算.

\begin{example}
设 $\boldsymbol{A},\boldsymbol{B}$ 均是 $n$ 阶方阵且满足 $\boldsymbol{A}^2 = \boldsymbol{A}$, $\boldsymbol{B}^2 = \boldsymbol{B}$, $(\boldsymbol{A} + \boldsymbol{B})^2 = \boldsymbol{A} + \boldsymbol{B}$, 求 $\boldsymbol{AB}$.
\end{example}
\begin{solution}
由 $(\boldsymbol{A} + \boldsymbol{B})^2 = \boldsymbol{A} + \boldsymbol{B}$ 得到
\[
\boldsymbol{A}^2 + \boldsymbol{AB} + \boldsymbol{BA} + \boldsymbol{B}^2 = \boldsymbol{A} + \boldsymbol{B},
\]
又 $\boldsymbol{A}^2 = \boldsymbol{A}$, $\boldsymbol{B}^2 = \boldsymbol{B}$, 所以
\[
\boldsymbol{AB} = -\boldsymbol{BA},
\]
从而
\[
\boldsymbol{A} \cdot \boldsymbol{AB} = -\boldsymbol{A} \cdot \boldsymbol{BA},
\]
\[
\boldsymbol{AB} \cdot \boldsymbol{A} = -\boldsymbol{BA} \cdot \boldsymbol{A},
\]
即
\begin{gather*}
\boldsymbol{AB}=-\boldsymbol{ABA};
\\
-\boldsymbol{BA}=\boldsymbol{ABA},
\end{gather*}
因此
\[
\boldsymbol{AB} = \boldsymbol{BA}.
\]
由此得到 $2\boldsymbol{AB} = \boldsymbol{O}$, 故有 $\boldsymbol{AB} = \boldsymbol{O}$.
\end{solution}

\begin{example}
设
\[
\boldsymbol{A} = \begin{pmatrix} 0 & -1 & 0 \\ 1 & 0 & 0 \\ 0 & 0 & -1 \end{pmatrix},
\]
\(\boldsymbol{B} = \boldsymbol{Q}^{-1}\boldsymbol{A}\boldsymbol{Q}\),其中 \(\boldsymbol{Q}\) 为任意 3 阶可逆矩阵 (或者说 \(\boldsymbol{B}\) 与 \(\boldsymbol{A}\) 相似). 求 \(\boldsymbol{B}^{2024} - 2\boldsymbol{A}^2\).
\end{example}
\begin{solution}
计算得到
\[
\boldsymbol{A}^2 = \begin{pmatrix} -1 & 0 & 0 \\ 0 & -1 & 0 \\ 0 & 0 & 1 \end{pmatrix} = \begin{pmatrix} -\boldsymbol{E}_2 & \boldsymbol{0} \\ \boldsymbol{0} & 1 \end{pmatrix},
\]
所以 \(\boldsymbol{A}^4 = \boldsymbol{E}\). 故 \(\boldsymbol{B}^{2024} = \boldsymbol{Q}^{-1}\boldsymbol{A}^{2024}\boldsymbol{Q} = \boldsymbol{E}\). 所以
\[
\boldsymbol{B}^{2024} - 2\boldsymbol{A}^2 = \boldsymbol{E} - 2\begin{pmatrix} -\boldsymbol{E}_2 & \boldsymbol{0} \\ \boldsymbol{0} & 1 \end{pmatrix} = \begin{pmatrix} 3 & 0 & 0 \\ 0 & 3 & 0 \\ 0 & 0 & -1 \end{pmatrix}.
\]
\end{solution}

\begin{example}
设
\[
\boldsymbol{A} = \begin{pmatrix} 1 & -1 & -1 & -1 \\ -1 & 1 & -1 & -1 \\ -1 & -1 & 1 & -1 \\ -1 & -1 & -1 & 1 \end{pmatrix},
\]
\(k\) 为正整数, 求 \(\boldsymbol{A}^k\).
\end{example}
\begin{remark}
不难发现$\boldsymbol{A}$是实对称矩阵,且是正交矩阵,因此$\boldsymbol{A}$一定可以对角化.
\end{remark}
\begin{solution}
注意到 \(\boldsymbol{A}^2 = 4\boldsymbol{E} = 2^2\boldsymbol{E}\), 所以
\[
\boldsymbol{A}^3 = \boldsymbol{A}^2 \cdot \boldsymbol{A} = 4\boldsymbol{A} = 2^2\boldsymbol{A}, \quad \boldsymbol{A}^4 = 2^4\boldsymbol{E}.
\]
故由归纳法可知
\[
\boldsymbol{A}^k = \begin{cases} 2^{k - 1}\boldsymbol{A}, & k \text{ 为奇数}, \\ 2^k\boldsymbol{E}, & k \text{ 为偶数}. \end{cases}.
\]
\end{solution}

\begin{example}
设
\[
\boldsymbol{A} = \begin{pmatrix} 2 & 0 & 0 & 0 & 0 & 0 \\ 3 & 2 & 0 & 0 & 0 & 0 \\ 0 & 3 & 2 & 0 & 0 & 0 \\ 0 & 0 & 0 & 1 & 2 & 0 \\ 0 & 0 & 0 & 0 & 1 & 2 \\ 0 & 0 & 0 & 0 & 0 & 1 \end{pmatrix},
\]
求 \(\boldsymbol{A}^{100}\).
\end{example}
\begin{solution}
设
\[
\boldsymbol{A}_1 = \begin{pmatrix} 2 & 0 & 0 \\ 3 & 2 & 0 \\ 0 & 3 & 2 \end{pmatrix}, \quad \boldsymbol{A}_2 = \begin{pmatrix} 1 & 2 & 0 \\ 0 & 1 & 2 \\ 0 & 0 & 1 \end{pmatrix}.
\]
则 \(\boldsymbol{A}\) 可写为如下分块形式:
\[
\boldsymbol{A} = \begin{pmatrix} \boldsymbol{A}_1 &  \\  & \boldsymbol{A}_2 \end{pmatrix},
\]
于是
\[
\boldsymbol{A}^{100} = \begin{pmatrix} \boldsymbol{A}_1^{100} &  \\  & \boldsymbol{A}_2^{100} \end{pmatrix}.
\]
将 \(\boldsymbol{A}_1, \boldsymbol{A}_2\) 分解为
\[
\boldsymbol{A}_1 = 2\boldsymbol{E} + 3\boldsymbol{S}, \quad \text{其中} \quad \boldsymbol{S} = \begin{pmatrix} 0 & 0 & 0 \\ 1 & 0 & 0 \\ 0 & 1 & 0 \end{pmatrix};
\]
\[
\boldsymbol{A}_2 = \boldsymbol{E} + 2\boldsymbol{H}, \quad \text{其中} \quad \boldsymbol{H} = \begin{pmatrix} 0 & 1 & 0 \\ 0 & 0 & 1 \\ 0 & 0 & 0 \end{pmatrix}.
\]
注意到 \(\boldsymbol{S}\) 与 \(\boldsymbol{E}, \boldsymbol{H}\) 与 \(\boldsymbol{E}\) 均可交换且
\[
\boldsymbol{S}^2 = \begin{pmatrix} 0 & 0 & 0 \\ 0 & 0 & 0 \\ 1 & 0 & 0 \end{pmatrix}, \quad \boldsymbol{S}^i = \boldsymbol{O} \ (i \geqslant 3), \quad \boldsymbol{H}^2 = \begin{pmatrix} 0 & 0 & 1 \\ 0 & 0 & 0 \\ 0 & 0 & 0 \end{pmatrix}, \quad \boldsymbol{H}^i = \boldsymbol{O} \ (i \geqslant 3).
\]
由二项式定理
\[
\begin{aligned}
\boldsymbol{A}_1^{100} &= (2\boldsymbol{E} + 3\boldsymbol{S})^{100} = \sum_{i=0}^{100} \binom{100}{i} (2\boldsymbol{E})^{100 - i} (3\boldsymbol{S})^i \\
&= \binom{100}{0} (2\boldsymbol{E})^{100} + \binom{100}{1} (2\boldsymbol{E})^{99} \cdot 3\boldsymbol{S} + \binom{100}{2} (2\boldsymbol{E})^{98} (3\boldsymbol{S})^2 \\
&= 2^{100}\boldsymbol{E} + 300\boldsymbol{S} \cdot 2^{99}\boldsymbol{E} + \frac{100 \times 99}{2} \cdot 9\boldsymbol{S}^2 \cdot 2^{98}\boldsymbol{E} \\
&= 2^{100}\boldsymbol{E} + 300 \cdot 2^{99} \cdot \boldsymbol{S} + 50 \cdot 99 \cdot 9 \cdot 2^{98} \cdot \boldsymbol{S}^2
\end{aligned}
\]
\[
= \begin{pmatrix} 2^{100} & 0 & 0 \\ 3 \cdot 5^2 \cdot 2^{101} & 2^{100} & 0 \\ 11 \cdot 5^2 \cdot 3^4 \cdot 2^{99} & 3 \cdot 5^2 \cdot 2^{101} & 2^{100} \end{pmatrix},
\]
\[
\begin{aligned}
\boldsymbol{A}_2^{100} &= (\boldsymbol{E} + 2\boldsymbol{H})^{100} = \boldsymbol{E}^{100} + 100 \cdot \boldsymbol{E}^{99} \cdot (2\boldsymbol{H}) + \frac{100 \times 99}{2} \cdot \boldsymbol{E}^{98} \cdot (2\boldsymbol{H})^2 \\
&= \boldsymbol{E} + 200\boldsymbol{H} + 19800\boldsymbol{H}^2 \\
&= \begin{pmatrix} 1 & 200 & 19800 \\ 0 & 1 & 200 \\ 0 & 0 & 1 \end{pmatrix}.
\end{aligned}
\]
因此
\[
\boldsymbol{A}^{100} = \begin{pmatrix} 2^{100} & 0 & 0 & 0 & 0 & 0 \\ 3 \cdot 5^2 \cdot 2^{101} & 2^{100} & 0 & 0 & 0 & 0 \\ 11 \cdot 5^2 \cdot 3^4 \cdot 2^{99} & 3 \cdot 5^2 \cdot 2^{101} & 2^{100} & 0 & 0 & 0 \\ 0 & 0 & 0 & 1 & 200 & 19800 \\ 0 & 0 & 0 & 0 & 1 & 200 \\ 0 & 0 & 0 & 0 & 0 & 1 \end{pmatrix}.
\]
\end{solution}


\subsection{矩阵可逆的判定及计算}

求逆矩阵的方法通常有下面三种.

(1) 公式法, 利用公式 \(\boldsymbol{A} \cdot \boldsymbol{A}^* = |\boldsymbol{A}|\boldsymbol{E}\), 即 \(\boldsymbol{A} \cdot \dfrac{1}{|\boldsymbol{A}|}\boldsymbol{A}^* = \boldsymbol{E}\), 所以
\[
\boldsymbol{A}^{-1} = \dfrac{1}{|\boldsymbol{A}|}\boldsymbol{A}^*.
\]

(2) 初等行变换法, 即
\[
(\boldsymbol{A} | \boldsymbol{E}) \longrightarrow (\boldsymbol{E} | \boldsymbol{P}),
\]
则 \(\boldsymbol{A}^{-1} = \boldsymbol{P}\).

(3) 定义法, 由 \(\boldsymbol{A}\boldsymbol{X} = \boldsymbol{E}\) 求出 \(\boldsymbol{X}\).

若矩阵 \(\boldsymbol{A}\) 没有具体给出, 是抽象的, 则首先想到的是用定义来求 \(\boldsymbol{A}^{-1}\).

\begin{proposition}
设 \(\boldsymbol{A}\) 为 \(n\) 阶矩阵. 则有
\[
\begin{aligned}
\boldsymbol{A} \text{ 可逆} &\Leftrightarrow \exists \boldsymbol{B} \text{ 使得 } \boldsymbol{BA} = \boldsymbol{E} \\
&\Leftrightarrow \exists \boldsymbol{C} \text{ 使得 } \boldsymbol{AC} = \boldsymbol{E} \\
&\Leftrightarrow \operatorname{rank}(\boldsymbol{A}) = n \\
&\Leftrightarrow \boldsymbol{A} \sim \boldsymbol{E} \\
&\Leftrightarrow \boldsymbol{A} \text{ 的最简行阶梯形为 } \boldsymbol{E} \\
&\Leftrightarrow |\boldsymbol{A}| \neq 0 \\
&\Leftrightarrow \text{齐次线性方程组 } \boldsymbol{Ax} = \boldsymbol{0} \text{ 只有零解} \\
&\Leftrightarrow \text{线性方程组 } \boldsymbol{Ax} = \boldsymbol{b} \text{ 有唯一解} \\
&\Leftrightarrow \boldsymbol{A} \text{ 的特征值均不为 } 0 \\
&\Leftrightarrow \boldsymbol{A} \text{ 的伴随矩阵 } \boldsymbol{A}^* \text{ 可逆} \\
&\Leftrightarrow \boldsymbol{A} \text{ 是一些初等矩阵之积}.
\end{aligned}
\]
\end{proposition}

\begin{example}
设 \(\boldsymbol{A}_1\) 为 \(m\) 阶可逆矩阵, \(\boldsymbol{A}_4\) 为 \(n\) 阶可逆矩阵, \(\boldsymbol{A}_2\) 为 \(m \times n\) 矩阵, \(\boldsymbol{A}_3\) 为 \(n \times m\) 矩阵. 设
\[
\boldsymbol{A} = \begin{pmatrix} \boldsymbol{A}_1 & \boldsymbol{A}_2 \\ \boldsymbol{A}_3 & \boldsymbol{A}_4 \end{pmatrix}.
\]
判断何时矩阵 \(\boldsymbol{A}\) 可逆. 并在 \(\boldsymbol{A}\) 可逆时求 \(\boldsymbol{A}^{-1}\).
\end{example}
\begin{solution}
{\color{blue}解法一:}
若 \(\boldsymbol{A}\) 可逆, 则存在矩阵 \(\boldsymbol{X}\) 使得 \(\boldsymbol{A}\boldsymbol{X} = \boldsymbol{E}_{m + n}\). 令
\[
\boldsymbol{X} = \begin{pmatrix} \boldsymbol{X}_1 & \boldsymbol{X}_2 \\ \boldsymbol{X}_3 & \boldsymbol{X}_4 \end{pmatrix},
\]
则有
\[
\begin{aligned}
\boldsymbol{A}\boldsymbol{X} = \begin{pmatrix} \boldsymbol{A}_1 & \boldsymbol{A}_2 \\ \boldsymbol{A}_3 & \boldsymbol{A}_4 \end{pmatrix} \begin{pmatrix} \boldsymbol{X}_1 & \boldsymbol{X}_2 \\ \boldsymbol{X}_3 & \boldsymbol{X}_4 \end{pmatrix} = \begin{pmatrix} \boldsymbol{A}_1\boldsymbol{X}_1 + \boldsymbol{A}_2\boldsymbol{X}_3 & \boldsymbol{A}_1\boldsymbol{X}_2 + \boldsymbol{A}_2\boldsymbol{X}_4 \\ \boldsymbol{A}_3\boldsymbol{X}_1 + \boldsymbol{A}_4\boldsymbol{X}_3 & \boldsymbol{A}_3\boldsymbol{X}_2 + \boldsymbol{A}_4\boldsymbol{X}_4 \end{pmatrix} = \begin{pmatrix} \boldsymbol{E}_m & \boldsymbol{O} \\ \boldsymbol{O} & \boldsymbol{E}_n \end{pmatrix},
\end{aligned}
\]
即
\[
\begin{cases}
\boldsymbol{A}_1\boldsymbol{X}_1 + \boldsymbol{A}_2\boldsymbol{X}_3 = \boldsymbol{E}_m, \\
\boldsymbol{A}_1\boldsymbol{X}_2 + \boldsymbol{A}_2\boldsymbol{X}_4 = \boldsymbol{O}, \\
\boldsymbol{A}_3\boldsymbol{X}_1 + \boldsymbol{A}_4\boldsymbol{X}_3 = \boldsymbol{O}, \\
\boldsymbol{A}_3\boldsymbol{X}_2 + \boldsymbol{A}_4\boldsymbol{X}_4 = \boldsymbol{E}_n,
\end{cases}
\]
由于 \(\boldsymbol{A}_4\) 可逆, 所以 \(\boldsymbol{X}_3 = -\boldsymbol{A}_4^{-1}\boldsymbol{A}_3\boldsymbol{X}_1\). 进而
\[
\boldsymbol{A}_1\boldsymbol{X}_1 + \boldsymbol{A}_2(-\boldsymbol{A}_4^{-1}\boldsymbol{A}_3)\boldsymbol{X}_1 = \boldsymbol{E}_m,
\]
即
\[
(\boldsymbol{A}_1 - \boldsymbol{A}_2\boldsymbol{A}_4^{-1}\boldsymbol{A}_3)\boldsymbol{X}_1 = \boldsymbol{E}_m.
\]
所以 \(\boldsymbol{A}_1 - \boldsymbol{A}_2\boldsymbol{A}_4^{-1}\boldsymbol{A}_3\) 可逆, 且 \(\boldsymbol{X}_1 = (\boldsymbol{A}_1 - \boldsymbol{A}_2\boldsymbol{A}_4^{-1}\boldsymbol{A}_3)^{-1}\). 由此得到
\[
\boldsymbol{X}_3 = -\boldsymbol{A}_4^{-1}\boldsymbol{A}_3(\boldsymbol{A}_1 - \boldsymbol{A}_2\boldsymbol{A}_4^{-1}\boldsymbol{A}_3)^{-1}.
\]
同理, 由 \(\boldsymbol{A}_1\) 可逆得到 \(\boldsymbol{X}_2 = -\boldsymbol{A}_1^{-1}\boldsymbol{A}_2\boldsymbol{X}_4\), 从而
\[
(\boldsymbol{A}_4 - \boldsymbol{A}_3\boldsymbol{A}_1^{-1}\boldsymbol{A}_2)\boldsymbol{X}_4 = \boldsymbol{E}_n.
\]
故 \(\boldsymbol{A}_4 - \boldsymbol{A}_3\boldsymbol{A}_1^{-1}\boldsymbol{A}_2\) 可逆且 \(\boldsymbol{X}_4 = (\boldsymbol{A}_4 - \boldsymbol{A}_3\boldsymbol{A}_1^{-1}\boldsymbol{A}_2)^{-1}\). 由此可得
\[
\boldsymbol{X}_2 = -\boldsymbol{A}_1^{-1}\boldsymbol{A}_2(\boldsymbol{A}_4 - \boldsymbol{A}_3\boldsymbol{A}_1^{-1}\boldsymbol{A}_2)^{-1}.
\]
这样, 若 \(\boldsymbol{A}\) 可逆, 则 \(\boldsymbol{A}_1 - \boldsymbol{A}_2\boldsymbol{A}_4^{-1}\boldsymbol{A}_3\) 和 \(\boldsymbol{A}_4 - \boldsymbol{A}_3\boldsymbol{A}_1^{-1}\boldsymbol{A}_2\) 均可逆.

反之, 若 \(\boldsymbol{A}_1 - \boldsymbol{A}_2\boldsymbol{A}_4^{-1}\boldsymbol{A}_3\) 和 \(\boldsymbol{A}_4 - \boldsymbol{A}_3\boldsymbol{A}_1^{-1}\boldsymbol{A}_2\) 均可逆, 令
\[
\boldsymbol{X} = \begin{pmatrix} (\boldsymbol{A}_1 - \boldsymbol{A}_2\boldsymbol{A}_4^{-1}\boldsymbol{A}_3)^{-1} & -\boldsymbol{A}_1^{-1}\boldsymbol{A}_2(\boldsymbol{A}_4 - \boldsymbol{A}_3\boldsymbol{A}_1^{-1}\boldsymbol{A}_2)^{-1} \\ -\boldsymbol{A}_4^{-1}\boldsymbol{A}_3(\boldsymbol{A}_1 - \boldsymbol{A}_2\boldsymbol{A}_4^{-1}\boldsymbol{A}_3)^{-1} & (\boldsymbol{A}_4 - \boldsymbol{A}_3\boldsymbol{A}_1^{-1}\boldsymbol{A}_2)^{-1} \end{pmatrix},
\]
则有 \(\boldsymbol{A}\boldsymbol{X} = \boldsymbol{E}_{m + n}\). 这便得到 \(\boldsymbol{A}\) 可逆当且仅当 \(\boldsymbol{A}_1 - \boldsymbol{A}_2\boldsymbol{A}_4^{-1}\boldsymbol{A}_3\) 和 \(\boldsymbol{A}_4 - \boldsymbol{A}_3\boldsymbol{A}_1^{-1}\boldsymbol{A}_2\) 均可逆且
\[
\boldsymbol{A}^{-1} = \boldsymbol{X} = \begin{pmatrix} (\boldsymbol{A}_1 - \boldsymbol{A}_2\boldsymbol{A}_4^{-1}\boldsymbol{A}_3)^{-1} & -\boldsymbol{A}_1^{-1}\boldsymbol{A}_2(\boldsymbol{A}_4 - \boldsymbol{A}_3\boldsymbol{A}_1^{-1}\boldsymbol{A}_2)^{-1} \\ -\boldsymbol{A}_4^{-1}\boldsymbol{A}_3(\boldsymbol{A}_1 - \boldsymbol{A}_2\boldsymbol{A}_4^{-1}\boldsymbol{A}_3)^{-1} & (\boldsymbol{A}_4 - \boldsymbol{A}_3\boldsymbol{A}_1^{-1}\boldsymbol{A}_2)^{-1} \end{pmatrix}.
\]

{\color{blue}解法二:}由\hyperref[proposition:打洞原理]{打洞原理}可知
\begin{align*}
|\boldsymbol{A}|=\left| \boldsymbol{A}_1 \right|| \boldsymbol{A}_4-\boldsymbol{A}_3\boldsymbol{A}_{1}^{-1}\boldsymbol{A}_2 |=\left| \boldsymbol{A}_4 \right|| \boldsymbol{A}_1-\boldsymbol{A}_2\boldsymbol{A}_{4}^{-1}\boldsymbol{A}_3 |.
\end{align*}
故当且仅当\(\boldsymbol{A}_1 - \boldsymbol{A}_2\boldsymbol{A}_4^{-1}\boldsymbol{A}_3\) 和 \(\boldsymbol{A}_4 - \boldsymbol{A}_3\boldsymbol{A}_1^{-1}\boldsymbol{A}_2\) 均可逆时$\boldsymbol{A}$可逆.注意到
\begin{align*}
\left( \begin{matrix}
\boldsymbol{E}&		\boldsymbol{O}\\
-\boldsymbol{A}_3\boldsymbol{A}_{1}^{-1}&		\boldsymbol{E}\\
\end{matrix} \right) \left( \begin{matrix}
\boldsymbol{A}_1&		\boldsymbol{A}_2\\
\boldsymbol{A}_3&		\boldsymbol{A}_4\\
\end{matrix} \right) \left( \begin{matrix}
\boldsymbol{E}&		-\boldsymbol{A}_{1}^{-1}\boldsymbol{A}_2\\
\boldsymbol{O}&		\boldsymbol{E}\\
\end{matrix} \right) \left( \begin{matrix}
\boldsymbol{A}_{1}^{-1}&		\boldsymbol{O}\\
\boldsymbol{O}&		\left( \boldsymbol{A}_4-\boldsymbol{A}_3\boldsymbol{A}_{1}^{-1}\boldsymbol{A}_2 \right) ^{-1}\\
\end{matrix} \right) =\boldsymbol{E}.
\end{align*}
故
\begin{align*}
\boldsymbol{A}^{-1}&=\left( \begin{matrix}
\boldsymbol{A}_1&		\boldsymbol{A}_2\\
\boldsymbol{A}_3&		\boldsymbol{A}_4\\
\end{matrix} \right) ^{-1}=\left[ \left( \begin{matrix}
\boldsymbol{E}&		\boldsymbol{O}\\
-\boldsymbol{A}_3\boldsymbol{A}_{1}^{-1}&		\boldsymbol{E}\\
\end{matrix} \right) ^{-1}\left( \begin{matrix}
\boldsymbol{A}_{1}^{-1}&		\boldsymbol{O}\\
\boldsymbol{O}&		\left( \boldsymbol{A}_4-\boldsymbol{A}_3\boldsymbol{A}_{1}^{-1}\boldsymbol{A}_2 \right) ^{-1}\\
\end{matrix} \right) ^{-1}\left( \begin{matrix}
\boldsymbol{E}&		-\boldsymbol{A}_{1}^{-1}\boldsymbol{A}_2\\
\boldsymbol{O}&		\boldsymbol{E}\\
\end{matrix} \right) ^{-1} \right] ^{-1}
\\
&=\left( \begin{matrix}
\boldsymbol{E}&		-\boldsymbol{A}_{1}^{-1}\boldsymbol{A}_2\\
\boldsymbol{O}&		\boldsymbol{E}\\
\end{matrix} \right) \left( \begin{matrix}
\boldsymbol{A}_{1}^{-1}&		\boldsymbol{O}\\
\boldsymbol{O}&		\left( \boldsymbol{A}_4-\boldsymbol{A}_3\boldsymbol{A}_{1}^{-1}\boldsymbol{A}_2 \right) ^{-1}\\
\end{matrix} \right) \left( \begin{matrix}
\boldsymbol{E}&		\boldsymbol{O}\\
-\boldsymbol{A}_3\boldsymbol{A}_{1}^{-1}&		\boldsymbol{E}\\
\end{matrix} \right) 
\\
&=\left( \begin{matrix}
\boldsymbol{A}_{1}^{-1}&		-\boldsymbol{A}_{1}^{-1}\boldsymbol{A}_2\left( \boldsymbol{A}_4-\boldsymbol{A}_3\boldsymbol{A}_{1}^{-1}\boldsymbol{A}_2 \right) ^{-1}\\
\boldsymbol{O}&		\left( \boldsymbol{A}_4-\boldsymbol{A}_3\boldsymbol{A}_{1}^{-1}\boldsymbol{A}_2 \right) ^{-1}\\
\end{matrix} \right) \left( \begin{matrix}
\boldsymbol{E}&		\boldsymbol{O}\\
-\boldsymbol{A}_3\boldsymbol{A}_{1}^{-1}&		\boldsymbol{E}\\
\end{matrix} \right) 
\\
&=\left( \begin{matrix}
\boldsymbol{A}_{1}^{-1}+\boldsymbol{A}_{1}^{-1}\boldsymbol{A}_2\left( \boldsymbol{A}_4-\boldsymbol{A}_3\boldsymbol{A}_{1}^{-1}\boldsymbol{A}_2 \right) ^{-1}\boldsymbol{A}_3\boldsymbol{A}_{1}^{-1}&		-\boldsymbol{A}_{1}^{-1}\boldsymbol{A}_2\left( \boldsymbol{A}_4-\boldsymbol{A}_3\boldsymbol{A}_{1}^{-1}\boldsymbol{A}_2 \right) ^{-1}\\
-\left( \boldsymbol{A}_4-\boldsymbol{A}_3\boldsymbol{A}_{1}^{-1}\boldsymbol{A}_2 \right) ^{-1}\boldsymbol{A}_3\boldsymbol{A}_{1}^{-1}&		\left( \boldsymbol{A}_4-\boldsymbol{A}_3\boldsymbol{A}_{1}^{-1}\boldsymbol{A}_2 \right) ^{-1}\\
\end{matrix} \right) .
\end{align*}
实际上,到这一步已经完成了这题.接下来的证明是为了与解法一相互验证(也可以类似\hyperref[proposition:矩阵可逆的重要结论1]{命题\ref{proposition:矩阵可逆的重要结论1}的解法二}进行验证).注意到
\begin{align*}
&\left( \boldsymbol{A}_1-\boldsymbol{A}_2\boldsymbol{A}_{4}^{-1}\boldsymbol{A}_3 \right) \left( \boldsymbol{A}_{1}^{-1}+\boldsymbol{A}_{1}^{-1}\boldsymbol{A}_2\left( \boldsymbol{A}_4-\boldsymbol{A}_3\boldsymbol{A}_{1}^{-1}\boldsymbol{A}_2 \right) ^{-1}\boldsymbol{A}_3\boldsymbol{A}_{1}^{-1} \right) 
\\
&=\boldsymbol{E}+\boldsymbol{A}_2\left( \boldsymbol{A}_4-\boldsymbol{A}_3\boldsymbol{A}_{1}^{-1}\boldsymbol{A}_2 \right) ^{-1}\boldsymbol{A}_3\boldsymbol{A}_{1}^{-1}-\boldsymbol{A}_2\boldsymbol{A}_{4}^{-1}\boldsymbol{A}_3\boldsymbol{A}_{1}^{-1}-\boldsymbol{A}_2\boldsymbol{A}_{4}^{-1}\boldsymbol{A}_3\boldsymbol{A}_{1}^{-1}\boldsymbol{A}_2\left( \boldsymbol{A}_4-\boldsymbol{A}_3\boldsymbol{A}_{1}^{-1}\boldsymbol{A}_2 \right) ^{-1}\boldsymbol{A}_3\boldsymbol{A}_{1}^{-1}
\\
&=\boldsymbol{E}+\boldsymbol{A}_2\left[ \left( \boldsymbol{A}_4-\boldsymbol{A}_3\boldsymbol{A}_{1}^{-1}\boldsymbol{A}_2 \right) ^{-1}-\boldsymbol{A}_{4}^{-1}-\boldsymbol{A}_{4}^{-1}\boldsymbol{A}_3\boldsymbol{A}_{1}^{-1}\boldsymbol{A}_2\left( \boldsymbol{A}_4-\boldsymbol{A}_3\boldsymbol{A}_{1}^{-1}\boldsymbol{A}_2 \right) ^{-1} \right] \boldsymbol{A}_3\boldsymbol{A}_{1}^{-1}
\\
&=\boldsymbol{E}+\boldsymbol{A}_2\left( \boldsymbol{E}-\boldsymbol{A}_{4}^{-1}\left( \boldsymbol{A}_4-\boldsymbol{A}_3\boldsymbol{A}_{1}^{-1}\boldsymbol{A}_2 \right) -\boldsymbol{A}_{4}^{-1}\boldsymbol{A}_3\boldsymbol{A}_{1}^{-1}\boldsymbol{A}_2 \right) \left( \boldsymbol{A}_4-\boldsymbol{A}_3\boldsymbol{A}_{1}^{-1}\boldsymbol{A}_2 \right) ^{-1}\boldsymbol{A}_3\boldsymbol{A}_{1}^{-1}
\\
&=\boldsymbol{E}+\boldsymbol{A}_2\left( \boldsymbol{E}-\boldsymbol{E}+\boldsymbol{A}_{4}^{-1}\boldsymbol{A}_3\boldsymbol{A}_{1}^{-1}\boldsymbol{A}_2-\boldsymbol{A}_{4}^{-1}\boldsymbol{A}_3\boldsymbol{A}_{1}^{-1}\boldsymbol{A}_2 \right) \left( \boldsymbol{A}_4-\boldsymbol{A}_3\boldsymbol{A}_{1}^{-1}\boldsymbol{A}_2 \right) ^{-1}\boldsymbol{A}_3\boldsymbol{A}_{1}^{-1}
\\
&=\boldsymbol{E},
\end{align*}
\begin{align*}
&\boldsymbol{A}_{4}^{-1}\boldsymbol{A}_3(\boldsymbol{A}_1-\boldsymbol{A}_2\boldsymbol{A}_{4}^{-1}\boldsymbol{A}_3)^{-1}=\boldsymbol{A}_{4}^{-1}\boldsymbol{A}_3\left( \boldsymbol{A}_{1}^{-1}+\boldsymbol{A}_{1}^{-1}\boldsymbol{A}_2\left( \boldsymbol{A}_4-\boldsymbol{A}_3\boldsymbol{A}_{1}^{-1}\boldsymbol{A}_2 \right) ^{-1}\boldsymbol{A}_3\boldsymbol{A}_{1}^{-1} \right) 
\\
&=\boldsymbol{A}_{4}^{-1}\boldsymbol{A}_3\boldsymbol{A}_{1}^{-1}+\boldsymbol{A}_{4}^{-1}\boldsymbol{A}_3\boldsymbol{A}_{1}^{-1}\boldsymbol{A}_2\left( \boldsymbol{A}_4-\boldsymbol{A}_3\boldsymbol{A}_{1}^{-1}\boldsymbol{A}_2 \right) ^{-1}\boldsymbol{A}_3\boldsymbol{A}_{1}^{-1}
\\
&=\left( \boldsymbol{A}_{4}^{-1}\left( \boldsymbol{A}_4-\boldsymbol{A}_3\boldsymbol{A}_{1}^{-1}\boldsymbol{A}_2 \right) +\boldsymbol{A}_{4}^{-1}\boldsymbol{A}_3\boldsymbol{A}_{1}^{-1}\boldsymbol{A}_2 \right) \left( \boldsymbol{A}_4-\boldsymbol{A}_3\boldsymbol{A}_{1}^{-1}\boldsymbol{A}_2 \right) ^{-1}\boldsymbol{A}_3\boldsymbol{A}_{1}^{-1}
\\
&=\left( \boldsymbol{A}_4-\boldsymbol{A}_3\boldsymbol{A}_{1}^{-1}\boldsymbol{A}_2 \right) ^{-1}\boldsymbol{A}_3\boldsymbol{A}_{1}^{-1}.
\end{align*}
故
\begin{align*}
\left( \boldsymbol{A}_1-\boldsymbol{A}_2\boldsymbol{A}_{4}^{-1}\boldsymbol{A}_3 \right) ^{-1}=\boldsymbol{A}_{1}^{-1}+\boldsymbol{A}_{1}^{-1}\boldsymbol{A}_2\left( \boldsymbol{A}_4-\boldsymbol{A}_3\boldsymbol{A}_{1}^{-1}\boldsymbol{A}_2 \right) ^{-1}\boldsymbol{A}_3\boldsymbol{A}_{1}^{-1}.
\end{align*}
\begin{align*}
\boldsymbol{A}_{4}^{-1}\boldsymbol{A}_3(\boldsymbol{A}_1-\boldsymbol{A}_2\boldsymbol{A}_{4}^{-1}\boldsymbol{A}_3)^{-1}=\left( \boldsymbol{A}_4-\boldsymbol{A}_3\boldsymbol{A}_{1}^{-1}\boldsymbol{A}_2 \right) ^{-1}\boldsymbol{A}_3\boldsymbol{A}_{1}^{-1}.
\end{align*}
因此
\begin{align*}
\boldsymbol{A}^{-1}&=\left( \begin{matrix}
\boldsymbol{A}_{1}^{-1}+\boldsymbol{A}_{1}^{-1}\boldsymbol{A}_2\left( \boldsymbol{A}_4-\boldsymbol{A}_3\boldsymbol{A}_{1}^{-1}\boldsymbol{A}_2 \right) ^{-1}\boldsymbol{A}_3\boldsymbol{A}_{1}^{-1}&		-\boldsymbol{A}_{1}^{-1}\boldsymbol{A}_2\left( \boldsymbol{A}_4-\boldsymbol{A}_3\boldsymbol{A}_{1}^{-1}\boldsymbol{A}_2 \right) ^{-1}\\
-\left( \boldsymbol{A}_4-\boldsymbol{A}_3\boldsymbol{A}_{1}^{-1}\boldsymbol{A}_2 \right) ^{-1}\boldsymbol{A}_3\boldsymbol{A}_{1}^{-1}&		\left( \boldsymbol{A}_4-\boldsymbol{A}_3\boldsymbol{A}_{1}^{-1}\boldsymbol{A}_2 \right) ^{-1}\\
\end{matrix} \right) 
\\
&=\left( \begin{matrix}
(\boldsymbol{A}_1-\boldsymbol{A}_2\boldsymbol{A}_{4}^{-1}\boldsymbol{A}_3)^{-1}&		-\boldsymbol{A}_{1}^{-1}\boldsymbol{A}_2(\boldsymbol{A}_4-\boldsymbol{A}_3\boldsymbol{A}_{1}^{-1}\boldsymbol{A}_2)^{-1}\\
-\boldsymbol{A}_{4}^{-1}\boldsymbol{A}_3(\boldsymbol{A}_1-\boldsymbol{A}_2\boldsymbol{A}_{4}^{-1}\boldsymbol{A}_3)^{-1}&		(\boldsymbol{A}_4-\boldsymbol{A}_3\boldsymbol{A}_{1}^{-1}\boldsymbol{A}_2)^{-1}\\
\end{matrix} \right) .
\end{align*}
\end{solution}



































\end{document}
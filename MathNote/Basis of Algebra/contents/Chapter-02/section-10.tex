% contents/chapter-02/section-10.tex 第二章第三节
\documentclass[../../main.tex]{subfiles}
\graphicspath{{\subfix{../../image/}}} % 指定图片目录,后续可以直接使用图片文件名。

% 例如:
% \begin{figure}[H]
% \centering
% \includegraphics[scale=0.4]{image-01.01}
% \caption{图片标题}
% \label{figure:image-01.01}
% \end{figure}
% 注意:上述\label{}一定要放在\caption{}之后,否则引用图片序号会只会显示??.

\begin{document}

\section{练习}

\begin{exercise}
设\(A=(a_{ij})\)为\(n\)阶方阵,定义函数\(f(A)=\sum_{i,j = 1}^{n}a_{ij}^2\). 设\(P\)为\(n\)阶可逆矩阵,使得对任意的\(n\)阶方阵\(A\)成立:\(f(PAP^{-1}) = f(A)\). 证明:存在非零常数\(c\),使得\(P'P = cI_n\).
\end{exercise}
\begin{proof}
{\color{blue}证法一:}
由假设知\(f(A)=\text{tr}(AA')\),因此
\[
f(PAP^{-1})=\text{tr}(PAP^{-1}(P')^{-1}A'P')=\text{tr}((P'P)A(P'P)^{-1}A')=\text{tr}(AA').
\]
以下设\(P'P=(c_{ij})\),\((P'P)^{-1}=(d_{ij})\). 注意\(P'P\)是对称矩阵,后面要用到. 令\(A = E_{ij}\),其中$1\leq i,j\leq n$.并将其代入$(P'P)A(P'P)^{-1}A'$可得
\begin{align*}
&(P'P)A(P'P)^{-1}A'=(P'P)E_{ij}(P'P)^{-1}E_{ji}
\\
&=\bordermatrix{%
&    &       &             j&     &
\cr
&    &		&		c_{1i}&		&		\cr
&    &		&		c_{2i}&		&		\cr
&    &		&		\vdots&		&		\cr
&    &		&		c_{ii}&		&		\cr
&    &		&		\vdots&		&		\cr
&    &		&		c_{ni}&		&		\cr
}\bordermatrix{%
&    &       &             i&     &
\cr
&    &		&		d_{1j}&		&		\cr
&    &		&		d_{2j}&		&		\cr
&    &		&		\vdots&		&		\cr
&    &		&		d_{jj}&		&		\cr
&    &		&		\vdots&		&		\cr
&    &		&		d_{nj}&		&		\cr
}=\bordermatrix{%
&    &       &             i&     &
\cr
&    &		&		c_{1i}d_{jj}&		&		\cr
&    &		&		c_{2i}d_{jj}&		&		\cr
&    &		&		\vdots&		&		\cr
&    &		&		c_{ii}d_{jj}&		&		\cr
&    &		&		\vdots&		&		\cr
&    &		&		c_{ni}d_{jj}&		&		\cr
}
\end{align*}
于是$\mathrm{tr}\left( \left( P'P \right) A\left( P'P \right) ^{-1}A' \right) =c_{ii}d_{jj}$.
而$\mathrm{tr}\left( AA' \right) =\mathrm{tr}\left( E_{ij}E_{ji} \right) =\mathrm{tr}\left( E_{ii} \right) =1$.
则由$\text{tr}((P'P)A(P'P)^{-1}A')=\text{tr}(AA')$可知
\begin{align}\label{equation:eq542}
c_{ii}d_{jj}=1. 
\end{align}
再令\(A = E_{ij}+E_{kl}\),其中$1\leq i,j,k,l\leq n$.不妨设$k\geq i,l\geq j$,将其代入$(P'P)A(P'P)^{-1}A'$可得
\begin{align*}
&(P'P)A(P'P)^{-1}A'=(P'P)(E_{ij}+E_{kl})(P'P)^{-1}(E_{ji}+E_{lk})
\\
&=\left[\bordermatrix{%
&    &       &             j&     &
\cr
&    &		&		c_{1i}&		&		\cr
&    &		&		c_{2i}&		&		\cr
&    &		&		\vdots&		&		\cr
&    &		&		c_{ii}&		&		\cr
&    &		&		\vdots&		&		\cr
&    &		&		c_{ki}&		&		\cr
&    &		&		\vdots&		&		\cr
&    &		&		c_{ni}&		&		\cr
}+\bordermatrix{%
&    &       &             l&     &
\cr
&    &		&		c_{1k}&		&		\cr
&    &		&		c_{2k}&		&		\cr
&    &		&		\vdots&		&		\cr
&    &		&		c_{ik}&		&		\cr
&    &		&		\vdots&		&		\cr
&    &		&		c_{kk}&		&		\cr
&    &		&		\vdots&		&		\cr
&    &		&		c_{nk}&		&		\cr
}\right]\left[\bordermatrix{%
&    &       &             i&     &
\cr
&    &		&		d_{1j}&		&		\cr
&    &		&		d_{2j}&		&		\cr
&    &		&		\vdots&		&		\cr
&    &		&		d_{jj}&		&		\cr
&    &		&		\vdots&		&		\cr
&    &		&		d_{lj}&		&		\cr
&    &		&		\vdots&		&		\cr
&    &		&		d_{nj}&		&		\cr
}+\bordermatrix{%
&    &       &             k&     &
\cr
&    &		&		d_{1l}&		&		\cr
&    &		&		d_{2l}&		&		\cr
&    &		&		\vdots&		&		\cr
&    &		&		d_{jl}&		&		\cr
&    &		&		\vdots&		&		\cr
&    &		&		d_{ll}&		&		\cr
&    &		&		\vdots&		&		\cr
&    &		&		d_{nl}&		&		\cr
}\right]\\
&=\bordermatrix{%
&    &       &   j&  &          l&     &
\cr
&    &		&	c_{1i}& \cdots&c_{1k}&		&		\cr
&    &		&	c_{2i}&	\cdots&c_{2k}&		&		\cr
&    &		&	\vdots&	&\vdots&		&		\cr
&    &		&	c_{ii}&	\cdots&c_{ik}&		&		\cr
&    &		&	\vdots&	&\vdots&		&		\cr
&    &		&	c_{ki}&	\cdots&c_{kk}&		&		\cr
&    &		&	\vdots&	&\vdots&		&		\cr
&    &		&	c_{ni}&	\cdots&c_{nk}&		&		\cr
}\bordermatrix{%
&    &       &   i&  &          k&     &
\cr
&    &		&	d_{1j}& \cdots&	d_{1l}&		&		\cr
&    &		&	d_{2j}&	\cdots&d_{2l}&		&		\cr
&    &		&	\vdots&	&\vdots&		&		\cr
&    &		&	d_{jj}&	\cdots&d_{jl}&		&		\cr
&    &		&	\vdots&	&\vdots&		&		\cr
&    &		&	d_{lj}&	\cdots&d_{ll}&		&		\cr
&    &		&	\vdots&	&\vdots&		&		\cr
&    &		&	d_{nj}&	\cdots&d_{nl}&		&		\cr
}=\bordermatrix{%
&    &       &   i&  &          k&     &
\cr
&    &		&	c_{1i}d_{jj}+c_{1k}d_{lj}& \cdots&	c_{1i}d_{jl}+c_{1k}d_{ll}&		&		\cr
&    &		&	c_{2i}d_{jj}+c_{2k}d_{lj}& \cdots&	c_{2i}d_{jl}+c_{2k}d_{ll}&		&		\cr
&    &		&	\vdots&	&\vdots&		&		\cr
&    &		&	c_{ii}d_{jj}+c_{ik}d_{lj}& \cdots&	c_{ii}d_{jl}+c_{ik}d_{ll}&		&		\cr
&    &		&	\vdots&	&\vdots&		&		\cr
&    &		&	c_{ki}d_{jj}+c_{kk}d_{lj}& \cdots&	c_{ki}d_{jl}+c_{kk}d_{ll}&		&		\cr
&    &		&	\vdots&	&\vdots&		&		\cr
&    &		&	c_{ni}d_{jj}+c_{nk}d_{lj}& \cdots&	c_{ni}d_{jl}+c_{nk}d_{ll}&		&		\cr
}
\end{align*}
从而$\mathrm{tr}\left( \left( P'P \right) A\left( P'P \right) ^{-1}A' \right) =c_{ii}d_{jj}+c_{kk}d_{ll}+c_{ki}d_{jl}+c_{ik}d_{lj}$.而
\begin{align*}
\mathrm{tr}\left( AA' \right) =\mathrm{tr}\left( \left( E_{ij}+E_{kl} \right) \left( E_{ji}+E_{lk} \right) \right) =\mathrm{tr}\left( E_{ij}E_{ji}+E_{ij}E_{lk}+E_{kl}E_{ji}+E_{kl}E_{lk} \right)=2 + 2\delta_{ik}\delta_{jl}.    
\end{align*}
于是由$\text{tr}((P'P)A(P'P)^{-1}A')=\text{tr}(AA')$可知
\begin{align}\label{equation:eq745}
c_{ii}d_{jj}+c_{kk}d_{ll}+c_{ki}d_{jl}+c_{ik}d_{lj}=2 + 2\delta_{ik}\delta_{jl},  
\end{align}
其中\(\delta_{ik}\)是Kronecker符号. 由上述\eqref{equation:eq542}\eqref{equation:eq745}两式可得
\[
c_{ki}d_{jl}+c_{ik}d_{lj}=2\delta_{ik}\delta_{jl}.
\]
在上式中令\(j = l\),\(i\neq k\),注意到\(d_{jj}\neq0\),故有\(c_{ik}+c_{ki}=0\),又因为\(P'P\)是对称矩阵,所以\(c_{ik}=c_{ki}\).故\(c_{ik}=0\),\(\forall i\neq k\). 于是\(P'P\)是一个对角矩阵,从而由\eqref{equation:eq542}式可得\(d_{jj}=c_{jj}^{-1}\),由此可得\(c_{ii}=c_{jj}\),\(\forall i,j\). 因此\(P'P = cI_n\),其中\(c = c_{11}\neq0\).

{\color{blue}证法二:}
我们把数域限定在实数域上,并取 \(V = M_n(\mathbb{R})\) 上的 \hyperlink{Frobenius 内积}{Frobenius 内积},则 \(f(A)=\sum_{i,j = 1}^{n}a_{ij}^2=\|A\|^2\)。设 \(\varphi(A)=PAP^{-1}\) 为 \(V\) 上的线性变换,则题目条件可改写为 \(\|\varphi(A)\|=\|A\|\) 对任意的 \(A\in V\) 成立,于是由\nrefthe{theorem:正交变换和酉变换的基本性质}{-2}可知 \(\varphi\) 是正交算子,从而由\nrefpro{proposition:例9.29}{(2)}即得结论。
\end{proof}













\end{document}
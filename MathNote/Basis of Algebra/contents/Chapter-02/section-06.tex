% contents/chapter-02/section-06.tex 第二章第三节
\documentclass[../../main.tex]{subfiles}
\graphicspath{{\subfix{../../image/}}} % 指定图片目录,后续可以直接使用图片文件名。

% 例如:
% \begin{figure}[H]
% \centering
% \includegraphics{image-01.01}
% \caption{图片标题}
% \label{figure:image-01.01}
% \end{figure}
% 注意:上述\label{}一定要放在\caption{}之后,否则引用图片序号会只会显示??.

\begin{document}

\section{Cauchy-Binet公式}

\begin{theorem}[Cauchy-Binet公式]\label{theorem:Cauchy-Binet公式}
设\(A=(a_{ij})\)是\(m\times n\)矩阵,\(B=(b_{ij})\)是\(n\times m\)矩阵.\(A\left(\begin{matrix}
i_1 & \cdots & i_s\\
j_1 & \cdots & j_s
\end{matrix}\right)\)表示\(A\)的一个\(s\)阶子式,它是由\(A\)的第\(i_1,\cdots,i_s\)行与第\(j_1,\cdots,j_s\)列交点上的元素按原次序排列组成的行列式.同理定义\(B\)的\(s\)阶子式.

(1) 若\(m > n\),则有\(\vert AB\vert=0\);

(2) 若\(m\leq n\),则有
\[\vert AB\vert=\sum_{1\leq j_1<j_2<\cdots<j_m\leq n}A\left(\begin{matrix}
1 & 2 & \cdots & m\\
j_1 & j_2 & \cdots & j_m
\end{matrix}\right)B\left(\begin{matrix}
j_1 & j_2 & \cdots & j_m\\
1 & 2 & \cdots & m
\end{matrix}\right).\]
\end{theorem}
\begin{proof}
\begin{enumerate}[(1)]
\item 若 \(m > n\), 则 \(r(AB) \leq \min\{r(A), r(B)\} \leq \min\{m, n\} = n < m\), 故 \(|-AB| = 0\).

\item 
\end{enumerate}
\end{proof}

\begin{corollary}[Cauchy-Binet公式推论]\label{corollary:Cauchy-Binet公式推论}
设\(A=(a_{ij})\)是\(m\times n\)矩阵,\(B=(b_{ij})\)是\(n\times m\)矩阵,\(r\)是一个正整数且\(r\leq m\).

(1) 若\(r > n\),则\(AB\)的任意\(r\)阶子式都等于零;

(2) 若\(r\leq n\),则\(AB\)的\(r\)阶子式
\[AB\left(\begin{matrix}
i_1 & i_2 & \cdots & i_r\\
j_1 & j_2 & \cdots & j_r
\end{matrix}\right)=\sum_{1\leq k_1<k_2<\cdots<k_r\leq n}A\left(\begin{matrix}
i_1 & i_2 & \cdots & i_r\\
k_1 & k_2 & \cdots & k_r
\end{matrix}\right)B\left(\begin{matrix}
k_1 & k_2 & \cdots & k_r\\
j_1 & j_2 & \cdots & j_r
\end{matrix}\right).\]
\end{corollary}

\begin{example}
设\(n\geq3\),证明下列矩阵是奇异阵:
\[
A = 
\begin{pmatrix}
\cos(\alpha_1 - \beta_1) & \cos(\alpha_1 - \beta_2) & \cdots & \cos(\alpha_1 - \beta_n)\\
\cos(\alpha_2 - \beta_1) & \cos(\alpha_2 - \beta_2) & \cdots & \cos(\alpha_2 - \beta_n)\\
\vdots & \vdots & & \vdots\\
\cos(\alpha_n - \beta_1) & \cos(\alpha_n - \beta_2) & \cdots & \cos(\alpha_n - \beta_n)
\end{pmatrix}.
\]    
\end{example}
\begin{solution}
注意到
\begin{align*}
A=\left( \begin{matrix}
\cos\mathrm{(}\alpha _1-\beta _1)&		\cos\mathrm{(}\alpha _1-\beta _2)&		\cdots&		\cos\mathrm{(}\alpha _1-\beta _n)\\
\cos\mathrm{(}\alpha _2-\beta_1)&		\cos\mathrm{(}\alpha_2-\beta _2)&		\cdots&	\cos\mathrm{(}\alpha _2-\beta _n)\\
\vdots&		\vdots&		&	\vdots\\
\cos\mathrm{(}\alpha _n-\beta_1)&		\cos\mathrm{(}\alpha_n-\beta _2)&		\cdots&	\cos\mathrm{(}\alpha _n-\beta _n)\\
\end{matrix} \right) 
=\left( \begin{matrix}
\cos \alpha _1&		\sin \alpha_1\\
\cos \alpha _2&		\sin \alpha_2\\
\vdots&		\vdots\\
\cos \alpha _n&		\sin \alpha_n\\
\end{matrix} \right) \left( \begin{matrix}
\cos \beta _1&		\cos \beta _2&		\cdots&		\cos \beta _n\\
\sin \beta _1&		\sin \beta _2&		\cdots&		\sin \beta _n\\
\end{matrix} \right) .
\end{align*}
由\hyperref[theorem:Cauchy-Binet公式]{Cauchy-Binet公式},可知$\left| A \right|=0$.
\end{solution}

\begin{example}
设\(A\)是\(m\times n\)实矩阵,求证:矩阵\(AA'\)的任一主子式都非负.
\end{example}
\begin{proof}
若\(r\leq n\),则由\hyperref[corollary:Cauchy-Binet公式推论]{Cauchy-Binet公式推论}可得
\[
AA'\begin{pmatrix}
i_1 & i_2 & \cdots & i_r\\
i_1 & i_2 & \cdots & i_r
\end{pmatrix}=\sum_{1\leq j_1<j_2<\cdots<j_r\leq n}\left(A\begin{pmatrix}
i_1 & i_2 & \cdots & i_r\\
j_1 & j_2 & \cdots & j_r
\end{pmatrix}\right)^2\geq0;
\]
若\(r > n\),则\(AA'\)的任一\(r\)阶主子式都等于零,结论也成立.
\end{proof}

\begin{example}
设\(A\)是\(n\)阶实方阵且\(AA' = I_n\).求证:若\(1\leq i_1 < i_2 < \cdots < i_r\leq n\),则
\[
\sum_{1\leq j_1<j_2<\cdots<j_r\leq n}\left(A\begin{pmatrix}
i_1 & i_2 & \cdots & i_r\\
j_1 & j_2 & \cdots & j_r
\end{pmatrix}\right)^2 = 1.
\]
\end{example}
\begin{proof}
对等式$AA' = I_n$两边同时求r阶子式,因为$r\leq n$,所以由\hyperref[theorem:Cauchy-Binet公式]{Cauchy-Binet公式}即得结论成立.
\end{proof}

\begin{example}
设\(A,B\)分别是\(m\times n\),\(n\times m\)矩阵,求证:\(AB\)和\(BA\)的\(r\)阶主子式之和相等,其中\(1\leq r\leq\min\{m,n\}\).
\end{example}
\begin{proof}
由\hyperref[theorem:Cauchy-Binet公式]{Cauchy-Binet公式}可得
\begin{align*}
&\sum_{1\leq i_1 < i_2 < \cdots < i_r\leq m}AB\begin{pmatrix}
i_1 & i_2 & \cdots & i_r\\
i_1 & i_2 & \cdots & i_r
\end{pmatrix}\\
=&\sum_{1\leq i_1 < i_2 < \cdots < i_r\leq m}\sum_{1\leq j_1 < j_2 < \cdots < j_r\leq n}A\begin{pmatrix}
i_1 & i_2 & \cdots & i_r\\
j_1 & j_2 & \cdots & j_r
\end{pmatrix}B\begin{pmatrix}
j_1 & j_2 & \cdots & j_r\\
i_1 & i_2 & \cdots & i_r
\end{pmatrix}\\
=&\sum_{1\leq j_1 < j_2 < \cdots < j_r\leq n}\sum_{1\leq i_1 < i_2 < \cdots < i_r\leq m}B\begin{pmatrix}
j_1 & j_2 & \cdots & j_r\\
i_1 & i_2 & \cdots & i_r
\end{pmatrix}A\begin{pmatrix}
i_1 & i_2 & \cdots & i_r\\
j_1 & j_2 & \cdots & j_r
\end{pmatrix}\\
=&\sum_{1\leq j_1 < j_2 < \cdots < j_r\leq n}BA\begin{pmatrix}
j_1 & j_2 & \cdots & j_r\\
j_1 & j_2 & \cdots & j_r
\end{pmatrix}.
\end{align*}
\end{proof}

\begin{lemma}[Lagrange恒等式]\label{lemma:Lagrange恒等式}
证明Lagrange恒等式\((n\geq2)\):
\[
\left(\sum_{i = 1}^{n}a_i^2\right)\left(\sum_{i = 1}^{n}b_i^2\right)-\left(\sum_{i = 1}^{n}a_ib_i\right)^2=\sum_{1\leq i<j\leq n}(a_ib_j - a_jb_i)^2.
\]
\end{lemma}
\begin{proof}
左边的式子等于
\[
\begin{vmatrix}
\sum_{i = 1}^{n}a_i^2 & \sum_{i = 1}^{n}a_ib_i\\
\sum_{i = 1}^{n}a_ib_i & \sum_{i = 1}^{n}b_i^2
\end{vmatrix},
\]
这个行列式对应的矩阵可化为
\[
\begin{pmatrix}
a_1 & a_2 & \cdots & a_n\\
b_1 & b_2 & \cdots & b_n
\end{pmatrix}
\begin{pmatrix}
a_1 & b_1\\
a_2 & b_2\\
\vdots & \vdots\\
a_n & b_n
\end{pmatrix}.
\]
由Cauchy - Binet公式可得
\[
\begin{vmatrix}
\sum\limits_{i = 1}^{n}a_i^2 & \sum\limits_{i = 1}^{n}a_ib_i\\
\sum\limits_{i = 1}^{n}a_ib_i & \sum\limits_{i = 1}^{n}b_i^2
\end{vmatrix}
=\sum_{1\leq i<j\leq n}
\begin{vmatrix}
a_i & a_j\\
b_i & b_j
\end{vmatrix}
\begin{vmatrix}
a_i & b_i\\
a_j & b_j
\end{vmatrix}
=\sum_{1\leq i<j\leq n}(a_ib_j - a_jb_i)^2.
\]  
\end{proof}

\begin{theorem}[Cauchy - Schwarz不等式]\label{theorem:Cauchy - Schwarz不等式}
设\(a_i,b_i\)都是实数,证明Cauchy - Schwarz不等式:
\[
\left(\sum_{i = 1}^{n}a_i^2\right)\left(\sum_{i = 1}^{n}b_i^2\right)\geq\left(\sum_{i = 1}^{n}a_ib_i\right)^2.
\]
\end{theorem}
\begin{proof}
由\hyperref[lemma:Lagrange恒等式]{Lagrange恒等式},恒等式右边总非负,即得结论.
\end{proof}

\begin{example}
设\(A,B\)都是\(m\times n\)实矩阵,求证:
\[
|AA'||BB'|\geq|AB'|^2.
\]
\end{example}
\begin{proof}
若\(m > n\),则\(|AA'| = |BB'| = |AB'| = 0\),结论显然成立.
若\(m\leq n\),则由Cauchy - Binet公式可得
\[
|AA'|=\sum_{1\leq j_1<j_2<\cdots<j_m\leq n}\left(A\begin{pmatrix}
1 & 2 & \cdots & m\\
j_1 & j_2 & \cdots & j_m
\end{pmatrix}\right)^2;
\]
\[
|BB'|=\sum_{1\leq j_1<j_2<\cdots<j_m\leq n}\left(B\begin{pmatrix}
1 & 2 & \cdots & m\\
j_1 & j_2 & \cdots & j_m
\end{pmatrix}\right)^2;
\]
\[
|AB'|=\sum_{1\leq j_1<j_2<\cdots<j_m\leq n}A\begin{pmatrix}
1 & 2 & \cdots & m\\
j_1 & j_2 & \cdots & j_m
\end{pmatrix}B\begin{pmatrix}
1 & 2 & \cdots & m\\
j_1 & j_2 & \cdots & j_m
\end{pmatrix},
\]
再由\hyperref[theorem:Cauchy - Schwarz不等式]{Cauchy - Schwarz不等式}即得结论.
\end{proof}




\end{document}
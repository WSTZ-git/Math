\documentclass[../../main.tex]{subfiles}
\graphicspath{{\subfix{../../image/}}} % 指定图片目录,后续可以直接使用图片文件名。

% 例如:
% \begin{figure}[H]
% \centering
% \includegraphics{image-01.01}
% \caption{图片标题}
% \label{figure:image-01.01}
% \end{figure}
% 注意:上述\label{}一定要放在\caption{}之后,否则引用图片序号会只会显示??.

\begin{document}

\section{伴随相关应用}

\begin{proposition}\label{proposition:例9.29}
设 \(V\) 是由 \(n\) 阶实矩阵全体构成的欧氏空间(取 \hyperlink{Frobenius 内积}{Frobenius内积}),\(V\) 上的线性变换 \(\varphi\) 定义为 \(\varphi(A)=PAQ\),其中 \(P,Q\in V\)。
\begin{enumerate}[(1)]
\item 求 \(\varphi\) 的伴随 \(\varphi^*\)($\varphi^*(A)=P'AQ'$);
\item 若 \(P,Q\) 都是可逆矩阵,求证:\(\varphi\) 是正交算子的充要条件是 \(P'P = cI_n\),\(QQ' = c^{-1}I_n\),其中 \(c\) 是正实数;
\item 若 \(P,Q\) 都是可逆矩阵,求证:\(\varphi\) 是自伴随算子的充要条件是 \(P' = \pm P\),\(Q' = \pm Q\);
\item 若 \(P,Q\) 都是可逆矩阵,求证:\(\varphi\) 是正规算子的充要条件是 \(P,Q\) 都是正规矩阵。
\end{enumerate}
\end{proposition}
\begin{solution}
\begin{enumerate}[(1)]
\item 对任意的 \(A,B\in V\),由迹的交换性可得
\begin{align*}
(\varphi(A),B)&=\mathrm{tr}(PAQB')=\mathrm{tr}(AQB'P)=\mathrm{tr}(A(P'BQ')')=(A,P'BQ').
\end{align*}
定义 \(V\) 上的线性变换 \(\psi\) 为 \(\psi(B)=P'BQ'\),则上式即为 \((\varphi(A),B)=(A,\psi(B))\)。由伴随的唯一性即得 \(\varphi^* = \psi\)。
\item 若 \(\varphi\) 是正交算子,即 \(\varphi^*\varphi = I_V\),则由 (1) 可知,\(P'PAQQ' = A\) 对任意的 \(A\in V\) 成立。由 \(Q\) 的非异性可得 \(P'PA = A(QQ')^{-1}\) 对任意的 \(A\in V\) 成立。令 \(A = I_n\) 可得 \(P'P=(QQ')^{-1}\),因此上式即言 \(P'P\) 与任意的 \(A\) 均乘法可交换,于是存在实数 \(c\),使得 \(P'P = cI_n\)。又 \(P\) 可逆,故 \(P'P\) 正定,从而 \(c > 0\),由此即得必要性。充分性显然成立。
\item 若 \(\varphi\) 是自伴随算子,即 \(\varphi^* = \varphi\),则由 (1) 可知,\(P'AQ' = PAQ\) 对任意的 \(A\in V\) 成立。由 \(P,Q\) 的非异性可得 \(P^{-1}P'A = AQ(Q')^{-1}\) 对任意的 \(A\in V\) 成立。令 \(A = I_n\) 可得 \(P^{-1}P' = Q(Q')^{-1}\),因此上式即言 \(P^{-1}P'\) 与任意的 \(A\) 均乘法可交换,于是存在实数 \(c\),使得 \(P^{-1}P' = cI_n\),即 \(P' = cP\)。此式转置后可得 \(P = cP' = c^2P\),又 \(P\) 可逆,故 \(c^2 = 1\),从而 \(c = \pm1\),由此即得必要性。充分性显然成立。 
\item 若 \(\varphi\) 是正规算子,即 \(\varphi^*\varphi = \varphi\varphi^*\),则由 (1) 可知,\(P'PAQQ' = PP'AQ'Q\) 对任意的 \(A\in V\) 成立。由 \(P,Q\) 的非异性可得 \((PP')^{-1}P'PA = AQ'Q(QQ')^{-1}\) 对任意的 \(A\in V\) 成立。令 \(A = I_n\) 可得 \((PP')^{-1}P'P = Q'Q(QQ')^{-1}\),因此上式即言 \((PP')^{-1}P'P\) 与任意的 \(A\) 均乘法可交换,于是存在实数 \(c\),使得 \((PP')^{-1}P'P = cI_n\),即 \(P'P = cPP'\)。上式两边同时取迹,由于 \(P\) 可逆,故由\refpro{proposition:正定和半正定阵关于迹的判定准则}可知 \(\mathrm{tr}(P'P)=\mathrm{tr}(PP') > 0\),从而 \(c = 1\),由此即得必要性。充分性显然成立。 
\end{enumerate} 
\end{solution}

\begin{proposition}\label{proposition:例9.30}
设 \(V\) 是 \(n\) 阶实对称矩阵构成的欧氏空间(取 \hyperlink{Frobenius 内积}{Frobenius内积})。
\begin{enumerate}[(1)]
\item 求出 \(V\) 的一组标准正交基;
\item 设 \(T\) 是一个 \(n\) 阶实矩阵,\(V\) 上的线性变换 \(\varphi\) 定义为 \(\varphi(A)=T'AT\),求证:\(\varphi\) 是自伴随算子的充要条件是 \(T\) 为对称矩阵或反对称矩阵。
\end{enumerate}
\end{proposition}
\begin{proof}
\begin{enumerate}[(1)]
\item 记 \(E_{ij}\) 为 \(n\) 阶基础矩阵,则容易验证下列矩阵构成了 \(V\) 的一组标准正交基:
\begin{align}\label{equation:::---97}
E_{ii}\ (1\leq i\leq n);\ \frac{1}{\sqrt{2}}(E_{ij}+E_{ji})\ (1\leq i < j\leq n).
\end{align}
显然\eqref{equation:::---97}是$V$的一组基,对$\forall i,j,k\in {1,2,\cdots,n}$且$i<j,k\ne i,j$,我们有
\begin{gather*}
\mathrm{tr}\left( E_{ii}'E_{jj} \right) =\mathrm{tr}\left( E_{ii}E_{jj} \right) =\mathrm{tr}\left( O \right) =0;
\\
\mathrm{tr}\left( E_{kk}'\frac{1}{\sqrt{2}}\left( E_{ij}+E_{ji} \right) \right) =\frac{1}{\sqrt{2}}\mathrm{tr}\left( E_{kk}\left( E_{ij}+E_{ji} \right) \right) =\frac{1}{\sqrt{2}}\mathsf{t}\mathrm{r}\left( O \right) =0;
\\
\mathrm{tr}\left( \frac{1}{\sqrt{2}}\left( E_{ik}+E_{ki} \right) ' \frac{1}{\sqrt{2}}\left( E_{ij}+E_{ji} \right) \right) =\frac{1}{2}\mathrm{tr}\left( \left( E_{ik}+E_{ki} \right) \left( E_{ij}+E_{ji} \right) \right) =\frac{1}{2}\mathsf{t}\mathrm{r}\left( E_{kj} \right) =0;
\\
\mathrm{tr}\left( \frac{1}{\sqrt{2}}\left( E_{kj}+E_{jk} \right) ' \frac{1}{\sqrt{2}}\left( E_{ij}+E_{ji} \right) \right) =\frac{1}{2}\mathrm{tr}\left( \left( E_{kj}+E_{jk} \right) \left( E_{ij}+E_{ji} \right) \right) =\frac{1}{2}\mathsf{t}\mathrm{r}\left( E_{ki} \right) =0;
\\
\mathrm{tr}\left( E_{ii} \right) =1,\,\,\mathrm{tr}\left( \frac{1}{\sqrt{2}}\left( E_{ij}+E_{ji} \right) ' \frac{1}{\sqrt{2}}\left( E_{ij}+E_{ji} \right) \right) =\frac{1}{2}\mathrm{tr}\left( E_{ii}+E_{jj} \right) =1.
\end{gather*}
故\eqref{equation:::---97}是$V$的一组标准正交基.

\item 先证充分性. 若 \(T\) 为对称矩阵或反对称矩阵,则由\refpro{proposition:例9.29}可知,\(\varphi^*(A)=(T')'AT'=TAT'=T'AT=\varphi(A)\) 对任一 \(A\in V\) 成立,故 \(\varphi = \varphi^*\) 是自伴随算子.
再证必要性. 若 \(\varphi\) 是自伴随算子,则同上理由可得 \(TAT' = T'AT\) 对任一 \(A\in V\) 成立. 设 \(T=(t_{ij})\),令 \(A = E_{ij}+E_{ji}\) 代入上述等式可得
\begin{align}
TAT' &= \begin{pmatrix}
t_{11} & t_{12} & \cdots & t_{1n} \\
t_{21} & t_{22} & \cdots & t_{2n} \\
\vdots & \vdots & \ddots & \vdots \\
t_{n1} & t_{n2} & \cdots & t_{nn} \\
\end{pmatrix}
(E_{ij}+E_{ji})
\begin{pmatrix}
t_{11} & t_{21} & \cdots & t_{n1} \\
t_{12} & t_{22} & \cdots & t_{n2} \\
\vdots & \vdots & \ddots & \vdots \\
t_{1n} & t_{2n} & \cdots & t_{nn} \\
\end{pmatrix}
\nonumber
\\
&= \bordermatrix{%
&    &    i&       &   j&  \cr
& \cdots & t_{1j} & \cdots & t_{1i} & \cdots \cr
&\cdots & t_{2j} & \cdots & t_{2i} & \cdots \cr
&  & \vdots&  &\vdots &\cr
&\cdots & t_{nj} & \cdots & t_{ni} & \cdots \cr
}
\begin{pmatrix}
t_{11} & t_{21} & \cdots & t_{n1} \\
t_{12} & t_{22} & \cdots & t_{n2} \\
\vdots & \vdots & \ddots & \vdots \\
t_{1n} & t_{2n} & \cdots & t_{nn} \\
\end{pmatrix}
\nonumber
\\
&= \begin{pmatrix}
t_{1j}t_{1i} + t_{1i}t_{1j} & t_{1j}t_{2i} + t_{1i}t_{2j} & \cdots & t_{1j}t_{ni} + t_{1i}t_{nj} \\
t_{2j}t_{1i} + t_{2i}t_{1j} & t_{2j}t_{2i} + t_{2i}t_{2j} & \cdots & t_{2j}t_{ni} + t_{2i}t_{nj} \\
\vdots & \vdots & \ddots & \vdots \\
t_{nj}t_{1i} + t_{ni}t_{1j} & t_{nj}t_{2i} + t_{ni}t_{2j} & \cdots & t_{nj}t_{ni} + t_{ni}t_{nj} \\
\end{pmatrix},\label{equation:matrix--1}
\end{align}
\begin{align}
T'AT &= \begin{pmatrix}
t_{11} & t_{21} & \cdots & t_{n1} \\
t_{12} & t_{22} & \cdots & t_{n2} \\
\vdots & \vdots & \ddots & \vdots \\
t_{1n} & t_{2n} & \cdots & t_{nn} \\
\end{pmatrix}
(E_{ij}+E_{ji})
\begin{pmatrix}
t_{11} & t_{12} & \cdots & t_{1n} \\
t_{21} & t_{22} & \cdots & t_{2n} \\
\vdots & \vdots & \ddots & \vdots \\
t_{n1} & t_{n2} & \cdots & t_{nn} \\
\end{pmatrix}
\nonumber
\\
&= \bordermatrix{%
&    &    i&       &   j&  \cr
& \cdots & t_{j1} & \cdots & t_{i1} & \cdots \cr
&\cdots & t_{j2} & \cdots & t_{i2} & \cdots \cr
&  & \vdots&  &\vdots &\cr
&\cdots & t_{jn} & \cdots & t_{in} & \cdots \cr
}
\begin{pmatrix}
t_{11} & t_{12} & \cdots & t_{1n} \\
t_{21} & t_{22} & \cdots & t_{2n} \\
\vdots & \vdots & \ddots & \vdots \\
t_{n1} & t_{n2} & \cdots & t_{nn} \\
\end{pmatrix}
\nonumber
\\
&= \begin{pmatrix}
t_{j1}t_{i1} + t_{i1}t_{j1} & t_{j1}t_{i2} + t_{i1}t_{j2} & \cdots & t_{j1}t_{in} + t_{i1}t_{jn} \\
t_{j2}t_{i1} + t_{i2}t_{j1} & t_{j2}t_{i2} + t_{i2}t_{j2} & \cdots & t_{j2}t_{in} + t_{i2}t_{jn} \\
\vdots & \vdots & \ddots & \vdots \\
t_{jn}t_{i1} + t_{in}t_{j1} & t_{jn}t_{i2} + t_{in}t_{j2} & \cdots & t_{jn}t_{in} + t_{in}t_{jn} \\
\end{pmatrix}.\label{equation:matrix--2}
\end{align}
比较\eqref{equation:matrix--1}和\eqref{equation:matrix--2}可得
\begin{align}
t_{ik}t_{jl}+t_{il}t_{jk}&=t_{ki}t_{lj}+t_{li}t_{kj}\label{eq:9.7}
\end{align}
对一切 \(i,j,k,l\) 都成立. 令 \(k = l\),则可得
\begin{align}\label{equation:---7156864}
t_{ik}t_{jk}=t_{ki}t_{kj}
\end{align}
对一切 \(i,j,k\) 都成立. 进一步令 \(i = j\),则可得 \(t_{ik}^2 = t_{ki}^2\) 对一切 \(i,k\) 都成立,因此 \(t_{ik}=t_{ki}\) 或 \(t_{ik}=-t_{ki}\). 假设有某个 \(i\neq k\),\(t_{ik}=t_{ki}\neq0\);又有某个 \(t_{uv}=-t_{vu}\neq0\),则利用\eqref{equation:---7156864}式可得 \(t_{ik}t_{uk}=t_{ki}t_{ku}\) 可推出 \(t_{uk}=t_{ku}\). 这时若 \(t_{uk}\neq0\),则再利用\eqref{equation:---7156864}式可得 \(t_{uk}t_{uv}=t_{ku}t_{vu}\) 可推出 \(t_{uv}=t_{vu}\),矛盾. 若 \(t_{uk}=0\),则在 \eqref{eq:9.7} 式中令 \(j = u\),\(l = v\),可得$t_{ik}t_{uv}=t_{ki}t_{vu}$,而\(t_{ik}=t_{ki}\neq0\),故仍可推出 \(t_{uv}=t_{vu}\),依然矛盾. 于是或者 \(t_{ik}=t_{ki}\) 对一切 \(i,k\) 成立,或者 \(t_{ik}=-t_{ki}\) 对一切 \(i,k\) 成立,即 \(T\) 或者是对称矩阵,或者是反对称矩阵.
\end{enumerate} 
\end{proof}

\begin{proposition}\label{proposition:例9.31}
设 \(U = \mathbb{R}[x]\),取\refexa{example:常见内积和内积空间-例9.1}{(6)}中的内积. 任取 \(f(x),g(x)\in U\),若设某些系数为零,则可将它们都写成统一的形式:\(f(x)=a_0 + a_1x+\cdots + a_nx^n\),\(g(x)=b_0 + b_1x+\cdots + b_nx^n\)。
\begin{enumerate}[(1)]
\item 线性变换 \(\varphi\) 定义为 \(\varphi(f(x))=a_1 + a_2x+\cdots + a_nx^{n - 1}\),试求 \(\varphi\) 的伴随;
\item 线性变换 \(\varphi\) 定义为 \(\varphi(f(x))=a_0 + a_1(1 + x)+a_2(1 + x + x^2)+\cdots + a_n(\sum_{i = 0}^{n}x^i)\),求证:\(\varphi\) 的伴随不存在。
\end{enumerate}
\end{proposition}
\begin{proof}
\begin{enumerate}[(1)]
\item 经简单的计算可知,\(\varphi^*(g(x))=b_0x + b_1x^2+\cdots + b_{n - 1}x^n + b_nx^{n + 1}\)。
\item 注意到 \((f(x),x^i)=a_i\),也就是说 \(f(x)\) 和 \(x^i\) 的内积就是 \(f(x)\) 的 \(x^i\) 项系数. 用反证法来证明,设 \(\varphi\) 的伴随算子 \(\varphi^*\) 存在,我们来推出矛盾. 对任意的 \(n\geq m\),我们有 \((\varphi(x^n),x^m)=(1 + x+\cdots + x^n,x^m)=1\),故 \((x^n,\varphi^*(x^m)) = 1\) 对任意给定的 \(m\) 以及所有的 \(n\geq m\) 都成立,这说明 \(\varphi^*(x^m)\) 有无穷多个单项的系数不为零,这与 \(\varphi^*(x^m)\) 是多项式相矛盾. 因此 \(\varphi\) 的伴随不存在.
\end{enumerate} 
\end{proof}
















































\end{document}
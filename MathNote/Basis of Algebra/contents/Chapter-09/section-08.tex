\documentclass[../../main.tex]{subfiles}
\graphicspath{{\subfix{../../image/}}} % 指定图片目录,后续可以直接使用图片文件名。

% 例如:
% \begin{figure}[H]
% \centering
% \includegraphics[scale=0.4]{图.png}
% \caption{}
% \label{figure:图}
% \end{figure}
% 注意:上述\label{}一定要放在\caption{}之后,否则引用图片序号会只会显示??.

\begin{document}

\section{Gram-Schmidt正交化方法和正交补空间}

设 $V$ 为 $n$ 维内积空间,则由\refpro{proposition:例9.4}可知,任一 $n$ 阶正定实对称矩阵(正定 Hermite 矩阵)$H$ 都能成为 $V$ 的某组基的 Gram 矩阵。特别地,取 $H = I_n$,则存在 $V$ 的一组基 $\{f_1,f_2,\cdots,f_n\}$,使得它的 Gram 矩阵就是单位矩阵 $I_n$,即 $\{f_1,f_2,\cdots,f_n\}$ 是 $V$ 的一组标准正交基。由\refpro{proposition:例9.3}我们也可以具体地构造出一组标准正交基,以下不妨设 $V$ 是欧氏空间。首先,任取 $V$ 的一组基 $\{e_1,e_2,\cdots,e_n\}$,设其 Gram 矩阵为 $G$,则 $G$ 是正定实对称矩阵。其次,通过对称初等变换法可将 $G$ 化为单位矩阵 $I_n$,即存在 $n$ 阶非异实矩阵 $C = (c_{ij})$,使得 $C'GC = I_n$。最后,令
\[
(f_1,f_2,\cdots,f_n)=(e_1,e_2,\cdots,e_n)C,
\]
即 $f_j = \sum_{i = 1}^{n}c_{ij}e_i$,则 $\{f_1,f_2,\cdots,f_n\}$ 是 $V$ 的一组基,并且它的 Gram 矩阵就是 $C'GC = I_n$。从上述过程不难看出,因为当 $n\geqslant 2$ 时,过渡矩阵 $C$ 有无穷多种选法,所以可构造出 $V$ 的无穷多组标准正交基。

从几何的层面上看,上述构造标准正交基的代数方法虽然简单,但缺乏几何直观和意义。然而,Gram - Schmidt 方法却是一个从几何直观入手的向量组的正交化方法,具有重要的几何意义。Gram - Schmidt 方法粗略地说就是,如果前 $k - 1$ 个向量 $v_1,\cdots,v_{k - 1}$ 已经两两正交,那么只要将第 $k$ 个向量 $u_k$ 减去其在 $v_1,\cdots,v_{k - 1}$ 张成子空间上的正交投影,即可得到与 $v_1,\cdots,v_{k - 1}$ 都正交的向量 $v_k$。特别地,若 $\{u_1,u_2,\cdots,u_n\}$ 是欧氏空间 $V$ 的一组基,则通过 Gram - Schmidt 方法可得到一组正交基 $\{v_1,v_2,\cdots,v_n\}$,再将每个基向量标准化,即可得到 $V$ 的一组标准正交基 $\{w_1,w_2,\cdots,w_n\}$。这 3 组基之间的关系为
\[
(u_1,u_2,\cdots,u_n)=(v_1,v_2,\cdots,v_n)B=(w_1,w_2,\cdots,w_n)C,
\]
其中 $B$ 是主对角元全为 1 的上三角矩阵,$C$ 是主对角元全为正实数的上三角矩阵。设 $A = G(u_1,u_2,\cdots,u_n)$,$D = G(v_1,v_2,\cdots,v_n)$ 分别是对应的 Gram 矩阵,则 $A$ 是正定实对称矩阵,$D$ 是正定对角矩阵,由\refpro{proposition:例9.3} 可得 $A$ 的如下分解:
\[
A = B'DB = C'C,
\]
这就是\refpro{proposition:正定阵的3个充要条件} 中关于正定实对称矩阵 $A$ 的两种分解,再由\refpro{proposition:正定阵的3个充要条件} 后面的注可知上述两种分解的唯一性。因此,基 $\{u_1,u_2,\cdots,u_n\}$ 的 Gram 矩阵的分解 $A = B'DB$ 一一对应于通过 Gram - Schmidt 方法得到的正交基 $\{v_1,v_2,\cdots,v_n\}$,而 Gram 矩阵的 Cholesky 分解 $A = C'C$ 则一一对应于通过 Gram - Schmidt 正交化和标准化得到的标准正交基 $\{w_1,w_2,\cdots,w_n\}$。

除了求标准正交基之外,Gram - Schmidt 方法还有许多其他的应用。设 $V$ 是内积空间,$u$ 是 $V$ 中的向量,$\{w_1,\cdots,w_k\}$ 是子空间 $W$ 的一组标准正交基,则由 Gram - Schmidt 方法可知 $v = u - \sum_{i = 1}^{k}(u,w_i)w_i$ 与 $w_1,\cdots,w_k$ 正交。令 $w = \sum_{i = 1}^{k}(u,w_i)w_i$,则 $u = v + w$ 且 $(v,w) = 0$,于是 $\|u\|^2 = \|v\|^2 + \|w\|^2$。由此可得

(1) Bessel 不等式: $\|u\|^2\geqslant \|w\|^2 = \sum_{i = 1}^{k}|(u,w_i)|^2$;

(2) 向量 $u$ 到子空间 $W$ 的距离为 $\|v\|$,即 $\min_{x\in W}\|u - x\| = \|v\|$. 

\begin{example}\label{example:例9.11}
设 $V = \mathbb{R}[x]_n$ 为次数小于等于 $n$ 的实系数多项式构成的欧氏空间,对任意的 $f(x),g(x)$,其内积定义为 $(f(x),g(x)) = \int_{-1}^{1}f(x)g(x)\mathrm{d}x$(参考\nrefexa{example:常见内积和内积空间-例9.1}{(5)})。设 $u_0(x)=1$,$u_k(x)=\frac{\mathrm{d}^k}{\mathrm{d}x^k}[(x^2 - 1)^k]$($k\geqslant 1$),$m_k = \sqrt{\frac{2^{k + 1}k!(2k)!}{(2k + 1)!!}}$($k\geqslant 0$)。求证:从基 $\{1,x,\cdots,x^n\}$ 出发,由 Gram - Schmidt 正交化方法得到的标准正交基为 $\left\{\frac{u_k(x)}{m_k},0\leqslant  k\leqslant  n\right\}$,称之为 Legendre 多项式。
\end{example}
\begin{proof}
{\color{blue}证法一:}
由 Gram - Schmidt 正交化方法,从 $1,x,x^2,x^3$ 可得标准正交基中前 4 个基向量分别为 $w_0(x)=\frac{1}{\sqrt{2}}$,$w_1(x)=\sqrt{\frac{3}{2}}x$,$w_2(x)=\sqrt{\frac{5}{8}}(3x^2 - 1)$,$w_3(x)=\sqrt{\frac{7}{8}}(5x^3 - 3x)$,读者不难验证这就是 Legendre 多项式的前 4 个多项式。不过这样的计算很难推广到一般的情形,但我们可以通过验证 $\{u_k(x)\}$ 是一组正交基以及 Cholesky 分解与 Gram - Schmidt 正交化和标准化之间的一一对应来证明结论。

首先注意到,对任意的 $j < k$,有 $\left.\frac{\mathrm{d}^j}{\mathrm{d}x^j}[(x^2 - 1)^k]\right|_{x = \pm1}=0$,故由分部积分可得
\[
(u_k(x),x^j)=\int_{-1}^{1}\frac{\mathrm{d}^k}{\mathrm{d}x^k}[(x^2 - 1)^k]x^j\mathrm{d}x=-j\int_{-1}^{1}\frac{\mathrm{d}^{k - 1}}{\mathrm{d}x^{k - 1}}[(x^2 - 1)^k]x^{j - 1}\mathrm{d}x.
\]
不断做下去可知,当 $j < k$ 时,$(u_k(x),x^j)=0$;$(u_k(x),x^k)=(-1)^kk!\int_{-1}^{1}(x^2 - 1)^k\mathrm{d}x$。注意到 $u_k(x)$ 是一个 $k$ 次多项式且首项系数为 $2k(2k - 1)\cdots(k + 1)$,由上述结果并且经过进一步的计算可知,
\[
\|u_k(x)\|^2=\frac{2^{k + 1}k!(2k)!}{(2k + 1)!!},\quad (u_k(x),u_l(x))=0\ (k > l),
\]
因此 $\left\{\frac{u_k(x)}{m_k},0\leqslant  k\leqslant  n\right\}$ 是 $V$ 的一组标准正交基。设从基 $\{1,x,\cdots,x^n\}$ 到基 $\left\{\frac{u_k(x)}{m_k},0\leqslant  k\leqslant  n\right\}$ 的过渡矩阵为 $P$,基 $\{1,x,\cdots,x^n\}$ 的 Gram 矩阵为 $A$,则 $P$ 是一个主对角元全大于零的上三角矩阵,且由\refpro{proposition:例9.3}可得 $I_{n + 1}=P'AP$,从而 $A=(P^{-1})'P^{-1}$ 是Cholesky分解。由 Cholesky 分解的唯一性以及它与 Gram - Schmidt 正交化和标准化之间的一一对应可知,$\left\{\frac{u_k(x)}{m_k},0\leqslant  k\leqslant  n\right\}$ 就是从基 $\{1,x,\cdots,x^n\}$ 出发由 Gram - Schmidt 正交化方法得到的标准正交基.

{\color{blue}证法二:}设 $V_k$ 是由次数小于等于 $k$ 的实系数多项式构成的子空间,$w_k(x) = \frac{u_k(x)}{m_k} (0 \leqslant  k \leqslant  n)$,同证法 1 的计算可知这是一组两两正交的单位向量。

下面用归纳法来证明结论。当 $k = 0$ 时结论显然成立,假设从 $1, x, \cdots, x^k$ 出发,经过 Gram - Schmidt 正交化方法得到 $V_k$ 的一组标准正交基为 $w_0(x), w_1(x), \cdots, w_k(x)$。

现设 $x^{k + 1}$ 经过 Gram - Schmidt 正交化方法得到的单位向量为 $\widetilde{w}_{k + 1}(x)$,满足 $(w_i(x), \widetilde{w}_{k + 1}(x)) = 0 (0 \leqslant  i \leqslant  k)$,于是 $V_{k + 1} = V_k \perp L(w_{k + 1}(x)) = V_k \perp L(\widetilde{w}_{k + 1}(x))$。因此 $L(w_{k + 1}(x)) = L(\widetilde{w}_{k + 1}(x))$ 是 $V_k$ 在 $V_{k + 1}$ 中的正交补空间,注意到 $w_{k + 1}(x)$ 和 $\widetilde{w}_{k + 1}(x)$ 都是范数为 1 且首项系数为正数的 $k + 1$ 次多项式,故 $\widetilde{w}_{k + 1}(x) = w_{k + 1}(x)$,结论得证。
\end{proof}

\begin{example}
设 $V = \mathbb{R}[x]_3$ 为次数小于等于 3 的实系数多项式构成的欧氏空间,其内积定义同\refexa{example:例9.11},试求 $\min_{f(x)\in V}\int_{-1}^{1}(e^x - f(x))^2\mathrm{d}x$。
\end{example}
\begin{solution}
本题即求 $\min_{f(x)\in V}\|e^x - f(x)\|^2$。由\refexa{example:例9.11} 可知,$V$ 的一组标准正交基为
$w_0(x)=\frac{1}{\sqrt{2}}$,$w_1(x)=\sqrt{\frac{3}{2}}x$,$w_2(x)=\sqrt{\frac{5}{8}}(3x^2 - 1)$,$w_3(x)=\sqrt{\frac{7}{8}}(5x^3 - 3x)$,经计算可得 $(e^x,w_0(x))=\frac{\sqrt{2}}{2}(e - e^{-1})$,$(e^x,w_1(x))=\sqrt{6}e^{-1}$,$(e^x,w_2(x))=\frac{\sqrt{10}}{2}(e - 7e^{-1})$,$(e^x,w_3(x))=\frac{\sqrt{14}}{2}(37e^{-1} - 5e)$。因此,由 Gram - Schmidt 方法的几何意义可得
\begin{align*}
\min_{f(x)\in V}\|e^x - f(x)\|^2&=\|e^x - \sum_{i = 0}^{3}(e^x,w_i(x))w_i(x)\|^2\\
&=\|e^x - \frac{1}{2}(e - e^{-1}) - 3e^{-1}x - \frac{5}{4}(e - 7e^{-1})(3x^2 - 1) - \frac{7}{4}(37e^{-1} - 5e)(5x^3 - 3x)\|^2\\
&\approx 0.00002228887.
\end{align*}
\end{solution}

\begin{proposition}\label{proposition:例9.14}
设 $V$ 是 $n$ 维欧氏空间,$A$ 是 $m$ 阶半正定实对称矩阵且 $\mathrm{r}(A)=r\leqslant  n$,求证:必存在 $V$ 上的向量组 $\{\alpha_1,\alpha_2,\cdots,\alpha_m\}$,使得其 Gram 矩阵就是 $A$。
\end{proposition}
\begin{proof}
因为 $A$ 是秩为 $r$ 的 $m$ 阶半正定阵,故由\refpro{proposition:半正定阵的行满秩分解}可知,存在 $r\times m$ 实矩阵 $T$,使得 $A = T'T$。取 $V$ 的一组标准正交基 $\{e_1,e_2,\cdots,e_n\}$,令
\[
(\alpha_1,\alpha_2,\cdots,\alpha_m)=(e_1,e_2,\cdots,e_r)T,
\]
则由\refcor{corollary:例9.3的推论}即得
\[
G(\alpha_1,\alpha_2,\cdots,\alpha_m)=T'G(e_1,e_2,\cdots,e_r)T=T'T = A.
\]
\end{proof}

\begin{proposition}\label{proposition:例9.15}
对任意内积空间$V$,
证明: 若用 Gram-Schmidt 方法将线性无关的向量组 $\boldsymbol{u}_1,\boldsymbol{u}_2,\cdots,\boldsymbol{u}_m$ 变成正交向量组 $\boldsymbol{v}_1,\boldsymbol{v}_2,\cdots,\boldsymbol{v}_m$, 则这两组向量的 Gram 矩阵的行列式值不变, 即
\[
|\boldsymbol{G}(\boldsymbol{u}_1,\boldsymbol{u}_2,\cdots,\boldsymbol{u}_m)| = |\boldsymbol{G}(\boldsymbol{v}_1,\boldsymbol{v}_2,\cdots,\boldsymbol{v}_m)| = \|\boldsymbol{v}_1\|^2\|\boldsymbol{v}_2\|^2\cdots\|\boldsymbol{v}_m\|^2.
\]
\end{proposition}
\begin{proof}
由 \hyperref[theorem:Gram-Schmidt正交化方法]{Gram-Schmidt 正交化过程}可得
\begin{align*}
(\boldsymbol{u}_1,\boldsymbol{u}_2,\cdots,\boldsymbol{u}_m) &= (\boldsymbol{v}_1,\boldsymbol{v}_2,\cdots,\boldsymbol{v}_m)\boldsymbol{B},
\end{align*}
其中 $\boldsymbol{B}$ 是一个主对角元全为 1 的上三角矩阵, 再由\refpro{proposition:例9.14}的证明过程可得
\begin{align*}
\boldsymbol{G}(\boldsymbol{u}_1,\boldsymbol{u}_2,\cdots,\boldsymbol{u}_m) &= \boldsymbol{B}'\boldsymbol{G}(\boldsymbol{v}_1,\boldsymbol{v}_2,\cdots,\boldsymbol{v}_m)\boldsymbol{B}.
\end{align*}
注意到 $\boldsymbol{G}(\boldsymbol{v}_1,\boldsymbol{v}_2,\cdots,\boldsymbol{v}_m)$ 是主对角元分别为 $\|\boldsymbol{v}_1\|^2,\|\boldsymbol{v}_2\|^2,\cdots,\|\boldsymbol{v}_m\|^2$ 的对角矩阵, 故上式两边同取行列式即得结论. 
\end{proof}

\begin{proposition}\label{proposition:例9.16}
对任意内积空间$V$,
证明下列不等式:
\[
0 \leqslant  |\boldsymbol{G}(\boldsymbol{u}_1,\boldsymbol{u}_2,\cdots,\boldsymbol{u}_m)| \leqslant  \|\boldsymbol{u}_1\|^2\|\boldsymbol{u}_2\|^2\cdots\|\boldsymbol{u}_m\|^2,
\]
后一个等号成立的充要条件是 $\boldsymbol{u}_i$ 两两正交或者某个 $\boldsymbol{u}_i = \boldsymbol{0}$.
\end{proposition}
\begin{proof}
(i)对实内积空间:

{\color{blue}证法一:}
由\refpro{proposition:欧氏空间中Gram阵的性质-例9.5}可知 $\boldsymbol{G}(\boldsymbol{u}_1,\boldsymbol{u}_2,\cdots,\boldsymbol{u}_m)$ 是一个半正定实对称矩阵, 故由\nrefpro{proposition:正定和半正定阵的判定准则}{(2)}可知 $|\boldsymbol{G}(\boldsymbol{u}_1,\boldsymbol{u}_2,\cdots,\boldsymbol{u}_m)| \geqslant  0$. 对第二个不等式, 我们分情况讨论. 若 $\boldsymbol{G}(\boldsymbol{u}_1,\boldsymbol{u}_2,\cdots,\boldsymbol{u}_m)$ 是非正定的半正定阵, 则 $0 = |\boldsymbol{G}(\boldsymbol{u}_1,\boldsymbol{u}_2,\cdots,\boldsymbol{u}_m)| \leqslant $   $\|\boldsymbol{u}_1\|^2$  $\|\boldsymbol{u}_2\|^2$ $\cdots$ $\|\boldsymbol{u}_m\|^2$, 并且等号成立的充要条件是某个 $\boldsymbol{u}_i = \boldsymbol{0}$. 若 $\boldsymbol{G}(\boldsymbol{u}_1,\boldsymbol{u}_2,\cdots,\boldsymbol{u}_m)$ 是正定阵, 则由\refpro{proposition:欧氏空间中Gram阵的性质-例9.5}可知 $\boldsymbol{u}_1,\boldsymbol{u}_2,\cdots,\boldsymbol{u}_m$ 线性无关. 由\hyperref[theorem:Gram-Schmidt正交化方法]{Gram-Schmidt 正交化过程}可得
\begin{align*}
\boldsymbol{v}_i &= \boldsymbol{u}_i - \sum_{j = 1}^{i - 1} \frac{(\boldsymbol{u}_i,\boldsymbol{v}_j)}{\|\boldsymbol{v}_j\|^2} \boldsymbol{v}_j.
\end{align*}
再由\hyperref[corollary:范数性质的相关推广-勾股定理]{勾股定理}可得 $\|\boldsymbol{u}_i\|^2 = \|\boldsymbol{v}_i\|^2 + \sum_{j = 1}^{i - 1} \frac{(\boldsymbol{u}_i,\boldsymbol{v}_j)^2}{\|\boldsymbol{v}_j\|^2} \geqslant  \|\boldsymbol{v}_i\|^2 > 0$. 最后由\refpro{proposition:例9.15}可得
\begin{align*}
|\boldsymbol{G}(\boldsymbol{u}_1,\boldsymbol{u}_2,\cdots,\boldsymbol{u}_m)| &= \|\boldsymbol{v}_1\|^2\|\boldsymbol{v}_2\|^2\cdots\|\boldsymbol{v}_m\|^2 \leqslant  \|\boldsymbol{u}_1\|^2\|\boldsymbol{u}_2\|^2\cdots\|\boldsymbol{u}_m\|^2,
\end{align*}
等号成立当且仅当 $\|\boldsymbol{v}_i\|^2 = \|\boldsymbol{u}_i\|^2 (1 \leqslant  i \leqslant  m)$, 这也当且仅当 $\boldsymbol{v}_i = \boldsymbol{u}_i (1 \leqslant  i \leqslant  m)$, 从而当且仅当 $\boldsymbol{u}_i$ 两两正交.

{\color{blue}证法二:}
由\refpro{proposition:欧氏空间中Gram阵的性质-例9.5}可知 $\boldsymbol{G}(\boldsymbol{u}_1,\boldsymbol{u}_2,\cdots,\boldsymbol{u}_m)$ 是一个半正定实对称矩阵, 故再由\refpro{proposition:半正定阵行列式与主对角乘积的不等式}立得.

(ii)对复内积空间:由\refpro{proposition:欧氏空间中Gram阵的性质-例9.5}可知 $\boldsymbol{G}(\boldsymbol{u}_1,\boldsymbol{u}_2,\cdots,\boldsymbol{u}_m)$ 是一个半正定Hermite阵,故由(i)同理可证.
\end{proof}

\begin{proposition}\label{proposition:矩阵的Hadamard不等式-例9.17}
(1)设 $\boldsymbol{A} = (a_{ij})$ 是 $n$ 阶实矩阵, 证明下列 Hadamard 不等式:
\[
|\boldsymbol{A}|^2 \leqslant  \prod_{j = 1}^{n} \sum_{i = 1}^{n} a_{ij}^2.
\]

(2)设$\boldsymbol{A} = (a_{ij})$是$n$ 阶复矩阵, 证明下列Hadamard不等式:
\begin{align*}
|\det \boldsymbol{A}|^2 &\leqslant  \prod_{j = 1}^{n} \sum_{i = 1}^{n} |a_{ij}|^2.
\end{align*} 
\end{proposition}
\begin{proof}
(1){\color{blue}证法一:}设 $\boldsymbol{u}_1,\boldsymbol{u}_2,\cdots,\boldsymbol{u}_n$ 是 $\boldsymbol{A}$ 的 $n$ 个列向量, 则 $\boldsymbol{G} = \boldsymbol{A}'\boldsymbol{A}$ 可以看成是 $\boldsymbol{u}_1,\boldsymbol{u}_2,\cdots,\boldsymbol{u}_n$ 在 $\mathbb{R}^n$ 的标准内积下的 Gram 矩阵. 由\refpro{proposition:例9.16}可得
\begin{align*}
|\boldsymbol{A}|^2 &= |\boldsymbol{A}'\boldsymbol{A}| = |\boldsymbol{G}| \leqslant  \prod_{j = 1}^{n} \|\boldsymbol{u}_j\|^2 = \prod_{j = 1}^{n} \sum_{i = 1}^{n} a_{ij}^2.
\end{align*} 

{\color{blue}证法二:}由\nrefpro{proposition:半正定阵的判定准则123}{(2)}可知$\boldsymbol{A}'\boldsymbol{A}$是半正定阵,并且主对角元为$\sum_{i = 1}^{n} a_{ij}^2(1\leqslant  j\leqslant  n)$.故再由\refpro{proposition:半正定阵行列式与主对角乘积的不等式}立得.


(2)设 $\boldsymbol{u}_1,\boldsymbol{u}_2,\cdots,\boldsymbol{u}_n$ 是 $\boldsymbol{A}$ 的 $n$ 个列向量, 则 $\boldsymbol{G} = \boldsymbol{A}'\overline{\boldsymbol{A}}$ 可以看成是 $\boldsymbol{u}_1,\boldsymbol{u}_2,\cdots,\boldsymbol{u}_n$ 在 $\mathbb{C}^n$ 的标准内积下的 Gram 矩阵. 由\refpro{proposition:例9.16}及\refpro{pro:行列式计算常识}可得
\begin{align*}
|\det \boldsymbol{A}|^2 &= \left| \boldsymbol{A} \right|\overline{\left| \boldsymbol{A} \right|}=\left| \boldsymbol{A}' \right|\left| \overline{\boldsymbol{A}} \right|= |\boldsymbol{A}'\overline{\boldsymbol{A}}| = |\boldsymbol{G}| \leqslant  \prod_{j = 1}^{n} \|\boldsymbol{u}_j\|^2 = \prod_{j = 1}^{n} \sum_{i = 1}^{n} a_{ij}^2.
\end{align*} 
\end{proof}

\begin{corollary}\label{corollary:矩阵的Hadamard不等式的推论}
若 $n$ 阶实矩阵 $\boldsymbol{A} = (a_{ij})$ 满足 $|a_{ij}| \leqslant  M (1 \leqslant  i, j \leqslant  n)$, 则 $|\boldsymbol{A}| \leqslant  M^n \cdot n^{\frac{n}{2}}$. 
\end{corollary}
\begin{proof}
由\refpro{proposition:矩阵的Hadamard不等式-例9.17}可得
\begin{align*}
|\boldsymbol{A}|^2\leqslant   \prod_{j = 1}^{n} \sum_{i = 1}^{n} a_{ij}^2 \leqslant  \prod_{j = 1}^{n}M^2n \leqslant   M^{2n}n^n.
\end{align*}
故$|\boldsymbol{A}| \leqslant  M^n \cdot n^{\frac{n}{2}}$. 
\end{proof}

\begin{proposition}\label{proposition:例9.18}
设 $U_1,U_2,U$ 是 $n$ 维内积空间 $V$ 的子空间, 求证:
\begin{enumerate}[(1)]
\item $(U^\perp)^\perp = U$;
\item $(U_1 + U_2)^\perp = U_1^\perp \cap U_2^\perp$;
\item $(U_1 \cap U_2)^\perp = U_1^\perp + U_2^\perp$;
\item $V^\perp = \boldsymbol{0}$, $\boldsymbol{0}^\perp = V$.
\end{enumerate}
\end{proposition}
\begin{proof}
\begin{enumerate}[(1)]
\item 因为 $V = U^\perp \oplus (U^\perp)^\perp$, 故 $\dim (U^\perp)^\perp = n - \dim U^\perp = \dim U$. 另一方面, 显然有 $U \subseteq (U^\perp)^\perp$, 因此 $(U^\perp)^\perp = U$.
\item 任取$\alpha\in (U_1+U_2)^\perp$,则$(\alpha,U_1+0)=(\alpha,0+U_2)=0 $,故$(U_1 + U_2)^\perp \subseteq U_1^\perp$, $(U_1 + U_2)^\perp \subseteq U_2^\perp$, 于是 $(U_1 + U_2)^\perp \subseteq U_1^\perp \cap U_2^\perp$. 反之, 对任一 $\boldsymbol{\alpha} \in U_1^\perp \cap U_2^\perp$, $\boldsymbol{\beta} \in U_1 + U_2$, 记 $\boldsymbol{\beta} = \boldsymbol{\beta}_1 + \boldsymbol{\beta}_2$, 其中 $\boldsymbol{\beta}_1 \in U_1$, $\boldsymbol{\beta}_2 \in U_2$, 则
\begin{align*}
(\boldsymbol{\alpha},\boldsymbol{\beta}) &= (\boldsymbol{\alpha},\boldsymbol{\beta}_1 + \boldsymbol{\beta}_2) = (\boldsymbol{\alpha},\boldsymbol{\beta}_1) + (\boldsymbol{\alpha},\boldsymbol{\beta}_2) = 0,
\end{align*}
故 $\boldsymbol{\alpha} \in (U_1 + U_2)^\perp$, 于是 $U_1^\perp \cap U_2^\perp \subseteq (U_1 + U_2)^\perp$. 因此 $(U_1 + U_2)^\perp = U_1^\perp \cap U_2^\perp$.
\item 由 (1) 及 (2), 有 $(U_1^\perp + U_2^\perp)^\perp = (U_1^\perp)^\perp \cap (U_2^\perp)^\perp = U_1 \cap U_2$.
\item 显然成立. 
\end{enumerate} 
\end{proof}

\begin{proposition}\label{proposition:例9.19}
设 $S$ 是 $n$ 维内积空间 $V$ 的子集, 证明:
\begin{enumerate}[(1)]
\item $S^\perp = \{ \boldsymbol{\alpha} \in V \mid (\boldsymbol{\alpha}, S) = 0\}$ 是 $V$ 的子空间;
\item 设由 $S$ 生成的子空间(由$S$中所有向量张成的子空间)为$U$,则$S^\perp=U^\perp$ .进而,$(S^\perp)^\perp=U$.
\end{enumerate}
\end{proposition}
\begin{remark}
注意这里的$S$只是一个子集,而不是子空间.对$S$取两次正交补实际上将$S$扩充成$V$的一个子空间.
\end{remark}
\begin{proof}
\begin{enumerate}[(1)]
\item 显然成立, 下证明 (2). 设 $S$ 生成的子空间为 $U$, 显然$U\subseteq S$,从而一方面有 $U^\perp \subseteq S^\perp$. 另一方面, 对任一 $\boldsymbol{v} \in S^\perp$, $\boldsymbol{u} \in U$, 将 $\boldsymbol{u}$ 表示为 $S$ 中向量的线性组合, $\boldsymbol{u} = a_1\boldsymbol{x}_1 + \cdots + a_k\boldsymbol{x}_k$, 其中 $\boldsymbol{x}_i \in S$. 由 $(\boldsymbol{x}_i, \boldsymbol{v}) = 0$ 可得 $(\boldsymbol{u}, \boldsymbol{v}) = 0$, 于是 $\boldsymbol{v} \in U^\perp$, 从而 $S^\perp \subseteq U^\perp$, 因此 $S^\perp = U^\perp$. 最后由\nrefpro{proposition:例9.18}{(1)} 可知 $(S^\perp)^\perp = (U^\perp)^\perp = U$. 
\end{enumerate} 
\end{proof}

\begin{proposition}\label{proposition:例9.20}
设 $A$ 为 $m \times n$ 实矩阵,齐次线性方程组 $A\boldsymbol{x} = \boldsymbol{0}$ 的解空间为 $U$,求 $U^{\perp}$ 适合的线性方程组。
\end{proposition}
\begin{solution}
设 $A$ 的秩为 $r$,则解空间 $U$ 是 $\mathbb{R}^n$ (取标准内积) 的 $n - r$ 维子空间。取 $U$ 的一组基 $\boldsymbol{\eta}_1, \cdots, \boldsymbol{\eta}_{n - r}$,令 $B = (\boldsymbol{\eta}_1, \cdots, \boldsymbol{\eta}_{n - r})$ 为 $n \times (n - r)$ 实矩阵,则由\nrefpro{proposition:例9.19}{(2)}的证明可得
\begin{align*}
U^{\bot}=\left( L\left( \eta _1,\cdots ,\eta _{n-r} \right) \right) ^{\bot}=\left\{ \eta _1,\cdots ,\eta _{n-r} \right\} ^{\bot}=\left\{ \boldsymbol{x}\in \mathbb{R} ^n|\left( \eta _i,\boldsymbol{x} \right) =\eta _i\boldsymbol{x}=0,1\le i\le n-r \right\} =\left\{ \boldsymbol{x}\in \mathbb{R} ^n|\boldsymbol{B}' \boldsymbol{x}=0 \right\} 
\end{align*}
故$U^{\perp}$就是线性方程组$B'\boldsymbol{x} = \boldsymbol{0}$的解空间,即 $U^{\perp}$ 适合的线性方程组为 $B'\boldsymbol{x} = \boldsymbol{0}$。
\end{solution}

\begin{corollary}\label{corollary:齐次方程的解空间是其系数矩阵的正交补空间}
设$A$为$m\times n$实矩阵,$A=(\alpha_1,\alpha_2,\cdots,\alpha_n)$为$A$的列分块,齐次线性方程组$Ax=O$的解空间为$U$,任取$U$的一组基$\{\eta_1,\eta_2,\cdots,\eta_{n-r}\}$,令$B=(\eta_1,\eta_2,\cdots,\eta_{n-r})$,则在$\mathbb{R}^n$(取标准内积)中,就有
\begin{align}\label{equation:::---56486441}
\begin{aligned}
\left( L\left( \alpha _1,\cdots ,\alpha _n \right) \right) ^{\bot}&=\left\{ \alpha _1,\cdots ,\alpha _n \right\} ^{\bot}=\left\{ \boldsymbol{x}\in \mathbb{R} ^n|\left( \alpha _i,\boldsymbol{x} \right) =\alpha _i\boldsymbol{x}=0,1\le i\le n \right\} 
\\
&=\left\{ \boldsymbol{x}\in \mathbb{R} ^n|\boldsymbol{Ax}=0 \right\} =L\left( \eta _1,\cdots ,\eta _{n-r} \right) =U
\end{aligned}
\end{align}
\begin{align}\label{equation:::---5648644}
\begin{aligned}
U^{\bot}&=\left( L\left( \eta _1,\cdots ,\eta _{n-r} \right) \right) ^{\bot}=\left\{ \eta _1,\cdots ,\eta _{n-r} \right\} ^{\bot}
\\
&=\left\{ \boldsymbol{x}\in \mathbb{R} ^n|\left( \eta _i,\boldsymbol{x} \right) =\eta _i\boldsymbol{x}=0,1\le i\le n-r \right\} =\left\{ \boldsymbol{x}\in \mathbb{R} ^n|\boldsymbol{B}'\boldsymbol{x}=0 \right\} 
\end{aligned}
\end{align}
因此$U^{\perp}$就是线性方程组$B'\boldsymbol{x} = \boldsymbol{0}$的解空间,即 $U^{\perp}$ 适合的线性方程组为 $B'\boldsymbol{x} = \boldsymbol{0}$。
\end{corollary}
\begin{proof}
\eqref{equation:::---56486441}式由\refpro{proposition:例9.20}同理可证,\eqref{equation:::---5648644}可由\refpro{proposition:例9.20}直接得到.
\end{proof}

\begin{proposition}\label{proposition:例9.21}
设 $A$ 为 $m \times n$ 实矩阵,求证:非齐次线性方程组 $A\boldsymbol{x} = \boldsymbol{\beta}$ 有解的充要条件是向量 $\boldsymbol{\beta}$ 属于齐次线性方程组 $A'\boldsymbol{y} = \boldsymbol{0}$ 解空间的正交补空间。
\end{proposition}
\begin{proof}
设 $A = (\boldsymbol{\alpha}_1, \boldsymbol{\alpha}_2, \cdots, \boldsymbol{\alpha}_n)$ 为列分块,$U = L(\boldsymbol{\alpha}_1, \boldsymbol{\alpha}_2, \cdots, \boldsymbol{\alpha}_n)$ 为 $\mathbb{R}^m$ (取标准内积) 的子空间,则 $A\boldsymbol{x} = \boldsymbol{\beta}$ 有解当且仅当 $\boldsymbol{\beta} \in U$。另一方面,由\refcor{corollary:齐次方程的解空间是其系数矩阵的正交补空间}可知$A'\boldsymbol{y} = \boldsymbol{0}$ 的解空间即为 $\{\boldsymbol{y} \in \mathbb{R}^m \mid (\boldsymbol{\alpha}_i, \boldsymbol{y}) = 0, 1 \leqslant  i \leqslant  n\} = U^{\perp}$,注意到 $U = (U^{\perp})^{\perp}$,故结论得证。
\end{proof}

\begin{proposition}\label{proposition:例9.22}
设 $V$ 为 $n$ 阶实矩阵全体构成的欧氏空间(取 \hyperlink{Frobenius 内积}{Frobenius 内积}),$V_1$,$V_2$ 分别为 $n$ 阶实对称矩阵全体和 $n$ 阶实反对称矩阵全体构成的子空间,求证:
\[ V = V_1 \perp V_2 \]
\end{proposition}
\begin{proof}
一方面,由\refpro{proposition:矩阵空间可以分解为对称和反称矩阵空间的直和}可知 $V = V_1 \oplus V_2$。另一方面,对任意的 $\boldsymbol{A} \in V_1$,$\boldsymbol{B} \in V_2$,由迹的交换性可得
\begin{align*}
(\boldsymbol{A}, \boldsymbol{B}) &= \mathrm{tr}(\boldsymbol{A}\boldsymbol{B}') = -\mathrm{tr}(\boldsymbol{A}\boldsymbol{B}) = -\mathrm{tr}(\boldsymbol{B}\boldsymbol{A}) = -\mathrm{tr}(\boldsymbol{B}\boldsymbol{A}') = -(\boldsymbol{B}, \boldsymbol{A}) = -(\boldsymbol{A}, \boldsymbol{B})
\end{align*}
于是 $(\boldsymbol{A}, \boldsymbol{B}) = 0$,从而 $V_1 \perp V_2$,因此 $V = V_1 \perp V_2$。
\end{proof}








\end{document}
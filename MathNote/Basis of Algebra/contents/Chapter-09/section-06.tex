\documentclass[../../main.tex]{subfiles}
\graphicspath{{\subfix{../../image/}}} % 指定图片目录,后续可以直接使用图片文件名。

% 例如:
% \begin{figure}[h]
% \centering
% \includegraphics{image-01.01}
% \caption{图片标题}
% \label{fig:image-01.01}
% \end{figure}
% 注意:上述\label{}一定要放在\caption{}之后,否则引用图片序号会只会显示??.

\begin{document}

\section{复正规算子}

\begin{definition}[正规算子和正规矩阵]
设 $\varphi$ 是内积空间 $V$ 上的线性变换,$\varphi^*$ 是其伴随,若 $\varphi\varphi^* = \varphi^*\varphi$,则称 $\varphi$ 是 $V$ 上的\textbf{正规算子}. 

为了不引起混淆,我们也称酉空间 (欧氏空间) $V$ 上的正规算子 $\varphi$ 为\textbf{复正规算子} (\textbf{实正规算子}). 

复矩阵 $A$ 若适合 $\overline{A}'A = A\overline{A}'$,则称其为\textbf{复正规矩阵}. 

实矩阵 $A$ 若适合 $A'A = AA'$,则称其为\textbf{实正规矩阵}.
\end{definition}

\begin{proposition}\label{proposition:常见的正规矩阵}
\begin{enumerate}
\item 酉算子 (酉矩阵) 和 Hermite 算子 (Hermite 矩阵) 都是复正规算子 (矩阵).

\item 正交变换 (正交矩阵) 和对称变换 (实对称矩阵) 都是实正规算子 (矩阵).
\end{enumerate}
\end{proposition}
\begin{proof}
证明都是显然的.
\end{proof}

\begin{theorem}
酉空间 (欧氏空间) $V$ 上的线性变换 $\varphi$ 是复 (实) 正规算子的充分必要条件是 $\varphi$ 在 $V$ 的某一组或任一组标准正交基下的表示矩阵都是复 (实) 正规矩阵. 因此,复 (实) 矩阵的正规性在酉 (正交) 相似下是不变的. 
\end{theorem}
\begin{proof}
证明都是显然的.
\end{proof}

\begin{lemma}\label{lemma:正规算子与伴随的像的范数相同}
设 $\varphi$ 是内积空间 $V$ 上的正规算子,则对任意的 $\alpha \in V$,成立
\[
\|\varphi(\alpha)\| = \|\varphi^*(\alpha)\|.
\]
\end{lemma}
\begin{proof}
由 $\varphi$ 的正规性,有
\begin{align*}
\|\varphi(\alpha)\|^2 &= (\varphi(\alpha),\varphi(\alpha)) = (\alpha,\varphi^*\varphi(\alpha))\\
&= (\alpha,\varphi\varphi^*(\alpha)) = (\varphi^*(\alpha),\varphi^*(\alpha))\\
&= \|\varphi^*(\alpha)\|^2. 
\end{align*}
\end{proof}

\begin{proposition}\label{proposition:正规算子的特征值与特征向量的相关性质}
设 $V$ 是 $n$ 维酉空间,$\varphi$ 是 $V$ 上的正规算子.
\begin{enumerate}[(1)]
\item 向量 $u$ 是 $\varphi$ 属于特征值 $\lambda$ 的特征向量的充分必要条件为 $u$ 是 $\varphi^*$ 属于特征值 $\overline{\lambda}$ 的特征向量;

\item 属于 $\varphi$ 不同特征值的特征向量必正交.
\end{enumerate}
\end{proposition}
\begin{proof}
\begin{enumerate}[(1)]
\item 若 $\lambda$ 是任一数,则 $(\lambda I - \varphi)^* = \overline{\lambda}I - \varphi^*$,且
\[
(\lambda I - \varphi)(\overline{\lambda}I - \varphi^*) = (\overline{\lambda}I - \varphi^*)(\lambda I - \varphi),
\]
即 $\lambda I - \varphi$ 也是正规算子. 于是由\reflemma{lemma:正规算子与伴随的像的范数相同},
\[
\|(\lambda I - \varphi)(\alpha)\| = \|(\overline{\lambda}I - \varphi^*)(\alpha)\|
\]
对一切 $\alpha \in V$ 成立,故 $(\lambda I - \varphi)(u) = 0$ 当且仅当 $(\overline{\lambda}I - \varphi^*)(u) = 0$ 成立.

\item 设 $\varphi(u) = \lambda u, \varphi(v) = \mu v$ 且 $\lambda \neq \mu$,则由 (1) 知 $\varphi^*(v) = \overline{\mu}v$,于是
\[
\lambda(u,v) = (\lambda u,v) = (\varphi(u),v) = (u,\varphi^*(v)) = (u,\overline{\mu}v) = \mu(u,v).
\]
因为 $\lambda \neq \mu$,故 $(u,v) = 0$. 
\end{enumerate}
\end{proof}

\begin{lemma}\label{lemma:表示矩阵是上三角阵的正规算子表示矩阵必是对角阵}
设 $V$ 是 $n$ 维酉空间,$\varphi$ 是 $V$ 上的线性变换,又 $\{e_1,e_2,\cdots,e_n\}$ 是 $V$ 的一组标准正交基. 设 $\varphi$ 在这组基下的表示矩阵 $A$ 是一个上三角阵,则 $\varphi$ 是正规算子的充分必要条件是 $A$ 为对角阵.
\end{lemma}
\begin{proof}
若 $A$ 是对角阵,则 $A\overline{A}' = \overline{A}'A$,故 $\varphi\varphi^* = \varphi^*\varphi$,即 $\varphi$ 是正规算子. 反之,设 $\varphi$ 是正规算子. 由于 $A$ 是上三角阵,可记 $A = (a_{ij}), a_{ij} = 0 (i > j)$. 于是 $\varphi(e_1) = a_{11}e_1$,再由上面的命题可知 $\varphi^*(e_1) = \overline{a}_{11}e_1$. 另一方面,有
\[
\varphi^*(e_1) = \overline{a}_{11}e_1 + \overline{a}_{12}e_2 + \cdots + \overline{a}_{1n}e_n.
\]
因此 $a_{1j} = 0$ 对一切 $j > 1$ 成立. 又因为 $A$ 是上三角阵,所以
\[
\varphi(e_2) = a_{22}e_2,
\]
故又有 $\varphi^*(e_2) = \overline{a}_{22}e_2$ 及 $a_{2j} = 0 (j > 2)$. 不断这样做下去即得 $A$ 是对角阵.
\end{proof}

\begin{theorem}[Schur(舒尔)定理]\label{theorem:Schur(舒尔)定理}
设 $V$ 是 $n$ 维酉空间,$\varphi$ 是 $V$ 上的线性算子,则存在 $V$ 的一组标准正交基,使 $\varphi$ 在这组基下的表示矩阵为上三角阵.
\end{theorem}
\begin{proof}
对 $V$ 的维数 $n$ 用数学归纳法. 当 $n = 1$ 时结论显然成立. 设对 $n - 1$ 维酉空间结论成立,现证 $n$ 维酉空间的情形. 由于 $V$ 是复线性空间,故 $\varphi^*$ 总存在特征值与特征向量,即有
\[
\varphi^*(e) = \lambda e.
\]
设 $W$ 是由 $e$ 张成的一维子空间的正交补空间,由\refproposition{proposition:线性算子的正交补空间就是伴随算子的不变子空间}知 $W$ 是 $(\varphi^*)^* = \varphi$ 的不变子空间,将 $\varphi$ 限制在 $W$ 上得到 $W$ 上的一个线性变换. 注意到 $\dim W = n - 1$,故由归纳假设,存在 $W$ 的一组标准正交基 $\{e_1,e_2,\cdots,e_{n - 1}\}$,使 $\varphi|_W$ 在这组基下的表示矩阵为上三角阵. 令 $e_n = \frac{e}{\|e\|}$,则 $\{e_1,e_2,\cdots,e_n\}$ 成为 $V$ 的一组标准正交基,使 $\varphi$ 在这组基下的表示矩阵为上三角阵.
\end{proof}

\begin{corollary}[Schur定理]\label{corollary:Schur(舒尔)定理}
任一 $n$ 阶复矩阵均酉相似于一个上三角阵.
\end{corollary}

\begin{theorem}
设 $V$ 是 $n$ 维酉空间,$\varphi$ 是 $V$ 上的线性算子,则 $\varphi$ 为正规算子的充分必要条件是存在 $V$ 的一组标准正交基,使 $\varphi$ 在这组基下的表示矩阵是对角阵. 特别,这组基恰为 $\varphi$ 的 $n$ 个线性无关的特征向量.
\end{theorem}
\begin{proof}
利用\reflemma{lemma:表示矩阵是上三角阵的正规算子表示矩阵必是对角阵}和\hyperref[corollary:Schur(舒尔)定理]{Schur定理},我们立即得到证明.
\end{proof}

\begin{theorem}\label{theorem:复正规矩阵等价于酉相似于对角阵的矩阵}
复矩阵 $A$ 为复正规矩阵的充分必要条件是 $A$ 酉相似于对角阵.
\end{theorem}
\begin{proof}
利用\reflemma{lemma:表示矩阵是上三角阵的正规算子表示矩阵必是对角阵}和\hyperref[theorem:Schur(舒尔)定理]{Schur定理},我们立即得到证明.
\end{proof}

\begin{theorem}\label{theorem:复正规矩阵的特征值就是复正规矩阵在酉相似关系下的全系不变量}
复正规矩阵的特征值就是复正规矩阵在酉相似关系下的全系不变量,即两个复正规矩阵酉相似的充分必要条件是它们具有相同的特征值.
\end{theorem}
\begin{proof}

\end{proof}

\begin{proposition}
设 $\varphi$ 是 $n$ 维酉空间 $V$ 上的线性算子,$\lambda_1,\lambda_2,\cdots,\lambda_k$ 是 $\varphi$ 的全体不同特征值,$V_1,V_2,\cdots,V_k$ 是对应的特征子空间,则 $\varphi$ 是正规算子的充分必要条件是
\begin{align}\label{equation---------9.6.1}
V = V_1 \perp V_2 \perp \cdots \perp V_k. 
\end{align}
\end{proposition}
\begin{proof}
设 $\varphi$ 是正规算子,则它是一个可对角化线性变换,因此
\[
V = V_1 \oplus V_2 \oplus \cdots \oplus V_k.
\]
又从\refproposition{proposition:正规算子的特征值与特征向量的相关性质}知道,若 $i \neq j$,则 $V_i \perp V_j$,所以 \eqref{equation---------9.6.1}式成立.

反之,若\eqref{equation---------9.6.1}式成立,则在每个 $V_i$ 中取一组标准正交基,将这些基向量组成 $V$ 的一组标准正交基. 因为每个 $V_i$ 都是 $\varphi$ 的特征子空间,即 $\varphi(\alpha) = \lambda_i\alpha$ 对一切 $\alpha \in V_i$ 成立,故 $\varphi$ 在这组基下的表示矩阵是对角阵,因此 $\varphi$ 是正规算子. 
\end{proof}

\begin{theorem}
任一 $n$ 阶酉矩阵必酉相似于下列对角阵:
\[
\mathrm{diag}\{c_1,c_2,\cdots,c_n\},
\]
其中 $c_i$ 为模长等于 $1$ 的复数.
\end{theorem}
\begin{proof}
由\refproposition{proposition:常见的正规矩阵}及\reftheorem{theorem:复正规矩阵等价于酉相似于对角阵的矩阵}知酉矩阵酉相似于 $\mathrm{diag}\{c_1,c_2,\cdots,c_n\}$. 由于与酉矩阵酉相似的矩阵仍是酉矩阵,故 $\mathrm{diag}$\{$c_1$,$c_2$,$\cdots$,$c_n$ $\}$ 是酉矩阵,因此 $|c_i| = 1$.
\end{proof}
























\end{document}
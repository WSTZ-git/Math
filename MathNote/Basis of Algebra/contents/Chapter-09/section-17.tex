\documentclass[../../main.tex]{subfiles}
\graphicspath{{\subfix{../../image/}}} % 指定图片目录,后续可以直接使用图片文件名。

% 例如:
% \begin{figure}[H]
% \centering
% \includegraphics[scale=0.4]{图.png}
% \caption{}
% \label{figure:图}
% \end{figure}
% 注意:上述\label{}一定要放在\caption{}之后,否则引用图片序号会只会显示??.

\begin{document}

\section{实正规矩阵的正交相似标准型}

\begin{lemma}\label{lemma:正规矩阵与其转置有相同特征向量}
设 $A \in \mathbb{R}^{n \times n}$ 是正规矩阵. 若 $\lambda \in \mathbb{C}$ 是 $A$ 的特征值, 则 $\overline{\lambda}$ 是 $A^T$ 的特征值, 且 $A$ 和 $A^T$ 有分别属于 $\lambda,\overline{\lambda}$ 的相同的特征向量.
\end{lemma}
\begin{proof}
设 $\alpha \in \mathbb{C}^n$, 注意到
\begin{align*}
&A\alpha = \lambda\alpha \iff (A - \lambda E)\alpha = 0 \stackrel{\text{注意内积的性质}}{\iff} \alpha^*(A^T - \overline{\lambda}E)(A - \lambda E)\alpha = 0 \\
&\stackrel{\text{矩阵乘法之后利用正规定义}}{\iff} \alpha^*(A - \lambda E)(A^T - \overline{\lambda}E)\alpha = 0 \iff (A^T - \overline{\lambda}E)\alpha = 0 
\\
&\iff A^T\alpha = \overline{\lambda}\alpha,
\end{align*}
这就完成了证明.
\end{proof}

\begin{theorem}[实正规矩阵的正交相似标准型]\label{theorem:实正规矩阵的正交相似标准型}
设 $A \in \mathbb{R}^{n \times n}$ 是实正规矩阵. 设 $A$ 的全部特征值是
$$\lambda_1,\lambda_2,\cdots,\lambda_s,a_1 \pm ib_1,a_2 \pm ib_2,\cdots,a_t \pm ib_t,$$
这里
$$\lambda_i,a_j \in \mathbb{R},b_j \in \mathbb{R} \setminus \{0\},1 \leqslant i \leqslant s,1 \leqslant j \leqslant t,$$
且允许相同.记
\begin{gather*}
c_j=-2a_j,d_j=a_j^2+b_j^2,
\\
R_j=\left( \begin{matrix}
a_j&		-b_j\\
b_j&		a_j\\
\end{matrix} \right) \text{或}\left( \begin{matrix}
a_j&		b_j\\
-b_j&		a_j\\
\end{matrix} \right) \text{或}\left( \begin{matrix}
0&		-d_j\\
1&		-c_j\\
\end{matrix} \right) \text{或}\left( \begin{matrix}
0&		1\\
-d_j&		-c_j\\
\end{matrix} \right) 
,
\end{gather*}
则存在实正交矩阵 $T$,使得
\begin{align}
T^TAT = 
\left( \begin{matrix}
\lambda _1&		&		&		&		&		\\
&		\ddots&		&		&		&		\\
&		&		\lambda _s&		&		&		\\
&		&		&		R_1&		&		\\
&		&		&		&		\ddots&		\\
&		&		&		&		&		R_t\\
\end{matrix} \right) .
\label{eq:23.22234820uu0358023r2312r}
\end{align}
也存在实正交矩阵$P$,使得
\begin{align*}
P^T A P= \left( \begin{matrix}
I_{r_1}&		&		&		&		&		\\
&		-I_{r_2}&		&		&		&		\\
&		&		O&		&		&		\\
&		&		&		R_1&		&		\\
&		&		&		&		\ddots&		\\
&		&		&		&		&		R_t\\
\end{matrix} \right)  ,
\end{align*}
其中1的个数与$A$的正实数特征值的个数相同,-1的个数与$A$的负实数特征值的个数相同,0的个数与$A$的零特征值个数相同.
\end{theorem}
\begin{note}
读者应该仔细计算\eqref{eq:23.22234820uu0358023r2312r}中每个块对应的特征值. 本结果如果不是被直接考察,可以直接使用.
\end{note}
\begin{proof}
$\mathbf{Step}\mathbf{1}$ 当 $n=1$ 时命题是显然的. 对 $n=2$ 时证明蕴含在下面证明中. 假定命题对于小于等于 $n-1$ 的所有实正规矩阵成立,现在考虑 $n$ 阶正规矩阵. 证明的想法就是降阶之后用归纳假设.

$\mathbf{Step}\mathbf{2}$  若 $A$ 有实特征值 $\lambda$, 取 $\alpha \in \mathbb{R}^n$ 是 $A$ 的单位特征向量, 并将其扩充为 $\mathbb{R}^n$ 的标准正交基, 则在这组基下有
\begin{align*}
A = \begin{pmatrix} \lambda & C \\ 0 & H \end{pmatrix}.
\end{align*}
现在由 $A^TA = AA^T$ 知
\begin{align*}
\begin{pmatrix} \lambda^2 & \lambda C \\ \lambda C^T & C^TC + H^TH \end{pmatrix} = \begin{pmatrix} \lambda^2 + CC^T & CH^T \\ HC^T & HH^T \end{pmatrix}.
\end{align*}
于是
$$CC^T = 0 \stackrel{\text{\refpro{proposition:零矩阵的充要条件}}}{\implies} C = 0 \implies H^TH = HH^T,$$
故 $H$ 是 $n-1$ 阶实正规矩阵,此时用归纳假设即得存在 $n-1$ 阶实正交矩阵 $T_1$,使得 $T_1^THT_1$ 形如\eqref{eq:23.22234820uu0358023r2312r}右边矩阵的形状. 于是可取 $T = \text{diag}\{1,T_1\}$,即得\eqref{eq:23.22234820uu0358023r2312r}.

$\mathbf{Step}\mathbf{3}$ 若 $A$ 有特征值 $a + bi,a \in \mathbb{R},b \in \mathbb{R} \setminus \{0\}$. 则存在不同时为 0 的 $\beta,\eta \in \mathbb{R}^n$ 使得
\begin{align}
A(\beta + i\eta) = (a + bi)(\beta + i\eta) \iff 
\begin{cases} 
A\beta = a\beta - b\eta \\
A\eta = a\eta + b\beta 
\end{cases}
\iff A(\beta \ \eta) = (\beta \ \eta) \begin{pmatrix} a & -b \\ b & a \end{pmatrix}.\label{eq:2325y5h345634t.22234820uu0358023r2312r}
\end{align}
于是式\eqref{eq:2325y5h345634t.22234820uu0358023r2312r}暗示我们想到 $\beta,\eta$ 是标准正交的.

$\mathbf{Step}\mathbf{4}$  由\reflem{lemma:正规矩阵与其转置有相同特征向量},我们知道 $A^T(\beta + i\eta) = (a - bi)(\beta + i\eta)$,于是类似\eqref{eq:2325y5h345634t.22234820uu0358023r2312r}可得
\begin{align*}
\begin{cases} 
A\beta = a\beta - b\eta \\
A\eta = a\eta + b\beta 
\end{cases},
\begin{cases} 
A^T\beta = a\beta + b\eta \\
A^T\eta = a\eta - b\beta 
\end{cases}.
\end{align*}
由此可得
\begin{align*}
\begin{cases} 
\beta^TA\beta = a\beta^T\beta - b\beta^T\eta \\
\beta^TA\beta = a\beta^T\beta + b\eta^T\beta 
\end{cases}
\implies \beta^T\eta + \eta^T\beta = 0 \stackrel{\text{都是数字}}{\implies} \beta^T\eta = \eta^T\beta = 0,
\end{align*}
以及
\begin{align*}
\begin{cases} 
\eta^TA^T\beta = a\eta^T\beta + b\eta^T\eta \\
\beta^TA\eta = a\beta^T\eta + b\beta^T\beta 
\end{cases}
\implies \eta^T\eta = \beta^T\beta.
\end{align*}
因此 $\beta,\eta$ 是想要的正交的. 不妨设为单位向量并将其扩充到全空间构成标准正交基. 则在这组基下, $A$ 形如
\begin{align*}
\begin{pmatrix} a & -b & C_1 \\ b & a & C_2 \\ 0 & 0 & H \end{pmatrix}.
\end{align*}
类似实特征值的情况,我们直接矩阵乘法可得 $C_1 = C_2 = 0, H$ 正规. 于是类似由归纳假设即可得\eqref{eq:23.22234820uu0358023r2312r}. 我们完成了证明.

\end{proof}

\begin{proposition}[正定阵与实反称阵可同时合同对角化]\label{proposition:正定阵与实反称阵可同时合同对角化-例9.114}
设 $A$ 为 $n$ 阶正定实对称矩阵, $S$ 是同阶实反对称矩阵, 求证: 存在可逆矩阵 $C$, 使得
\begin{align*}
C'AC &= I_n,\quad C'SC = \mathrm{diag}\left\{\begin{pmatrix}0 & b_1\\ -b_1 & 0\end{pmatrix},\dots,\begin{pmatrix}0 & b_r\\ -b_r & 0\end{pmatrix},0,\dots,0\right\},
\end{align*}
其中 $b_1,\dots,b_r$ 是非零实数.
\end{proposition}
\begin{proof}
因为 $A$ 是正定阵, 故存在可逆矩阵 $P$, 使得 $P'AP = I_n$. 又矩阵 $P'SP$ 还是实反对称矩阵, 故由\hyperref[theorem:反对称矩阵的合同标准型]{反对称矩阵的合同标准型}知,存在正交矩阵 $Q$, 使得
\begin{align*}
Q'(P'SP)Q &= \mathrm{diag}\left\{\begin{pmatrix}0 & b_1\\ -b_1 & 0\end{pmatrix},\dots,\begin{pmatrix}0 & b_r\\ -b_r & 0\end{pmatrix},0,\dots,0\right\},
\end{align*}
其中 $b_1,\dots,b_r$ 是非零实数. 此时 $Q'(P'AP)Q = I_n$, 只需令 $C = PQ$ 即得结论.

\end{proof}

\begin{proposition}\label{proposition:例9.115}
设\( n \)阶实矩阵\( A \)满足\( A+A' \)正定(即\( A \)是亚正定阵). 求证:
\begin{align*}
|A+A'| \leqslant 2^n|A|
\end{align*}
且等号成立的充要条件是\( A \)为对称矩阵.
\end{proposition}
\begin{proof}
{\color{blue}证法一:}注意到矩阵\( A \)的如下分解:
\begin{align*}
A = \frac{1}{2}(A+A') + \frac{1}{2}(A-A')
\end{align*}
其中\( \frac{1}{2}(A+A') \)是正定阵, \( \frac{1}{2}(A-A') \)是实反对称矩阵, 故由\refpro{proposition:A+S的行列式的相关结论}可得\( |A| \geqslant \frac{1}{2^n}|A+A'| \), 等号成立的充要条件是\( \frac{1}{2}(A-A') = O \), 即\( A \)为对称矩阵.

{\color{blue}证法二:}设可逆矩阵 \( C \in \mathbb{R}^{n \times n} \) 使得 \( C^T (A + A^T) C = E_n \),要证不等式等价于两边乘以 \( (\det C)^2 > 0 \) 还成立,因此可以不妨设 \( A + A^T = E_n \)。

现在 \( (A - \frac{1}{2}E_n)^T = A^T - \frac{1}{2}E_n = \frac{1}{2}E_n - A \),即 \( A - \frac{1}{2}E_n \) 是实反对称矩阵。由\hyperref[proposition:实对称(反称)阵的特征值]{实反对称矩阵特征值为0或者纯虚数}我们有 \( A \) 的特征值形如 \( \frac{1}{2}, \frac{1}{2} \pm ai, a \neq 0 \),注意到复特征值成对出现且有
\begin{align*}
\left( \frac{1}{2} + ai \right) \left( \frac{1}{2} - ai \right) = \frac{1}{4} + a^2 \geqslant \left( \frac{1}{2} \right)^2,
\end{align*}
我们由行列式为特征值的积得
\begin{align*}
\det (2A) = 2^n \det A \geqslant 2^n \cdot \left( \frac{1}{2} \right)^n = 1.
\end{align*}
若等号成立,则 \( A - \frac{1}{2}E_n \) 特征值全为0. 考虑\hyperref[theorem:实正规矩阵的正交相似标准型]{实正规矩阵的正交相似标准型}得 \( A = \frac{1}{2}E_n = A^T \) 而矛盾!我们完成了证明.

\end{proof}

\begin{example}
设\( A,B \)为\( n \)阶实矩阵, 其中\( A \)的\( n \)个特征值都是正实数, 并且满足\( AB + BA' = 2AA' \). 证明:

(1) \( B \)必为对称矩阵;

(2) \( A \)为对称矩阵当且仅当\( A = B \), 也当且仅当\( \operatorname{tr}(B^2) = \operatorname{tr}(AA') \);

(3) \( |B| \geqslant |A| \), 且等号成立的充要条件是\( A = B \).
\end{example}
\begin{proof}
(1) 考虑矩阵方程
\begin{align}\label{eq:9.18w3523rfw325}
AX - X(-A') = 2AA'
\end{align}
由于\( A \)的特征值都是正实数, 故\( -A' \)的特征值都是负实数, 从而它们没有公共的特征值. 由\refpro{proposition:AX-XB相关命题1}可知, 矩阵方程\(\eqref{eq:9.18w3523rfw325}\)存在唯一解\( X = B \in M_n(\mathbb{R}) \). 将等式\( AB + BA' = 2AA' \)两边同时转置, 可得\( AB' + B'A' = 2AA' \), 即\( X = B' \)也是矩阵方程\(\eqref{eq:9.18w3523rfw325}\)的解, 由解的唯一性可得\( B = B' \), 即\( B \)为对称矩阵.

(2) 若\( A \)为对称矩阵, 则\( X = A \)也是矩阵方程\(\eqref{eq:9.18w3523rfw325}\)的解, 由解的唯一性可得\( B = A \), 于是\( B^2 = AA' \), 从而\( \operatorname{tr}(B^2) = \operatorname{tr}(AA') \). 反之, 若\( \operatorname{tr}(B^2) = \operatorname{tr}(AA') \), 则
\begin{align*}
\operatorname{tr}\left((A - B)(A - B)'\right) = \operatorname{tr}\left((A - B)(A' - B)\right) = \operatorname{tr}\left(AA' + B^2 - (AB + BA')\right) = \operatorname{tr}(B^2) - \operatorname{tr}(AA') = 0,
\end{align*}
由迹的正定性可得\( A - B = O \), 即\( A = B \)是对称矩阵.

(3) 注意到\( AB + (AB)' = 2AA' \)为正定阵且\( |A| > 0 \), 故由\refpro{proposition:例9.115}可得
\begin{align*}
|2AA' |=\left| AB+(AB)' \right|\leqslant 2^n|AB|, 
\end{align*}
由此可得\( |B| \geqslant |A| \), 等号成立当且仅当\( AB \)为对称矩阵, 即当且仅当\( AB = AA' \), 这也当且仅当\( A = B \). 

\end{proof}

\begin{example}
设\( A \)为\( n \)阶实正规矩阵, 求证: 存在特征值为1或\(-1\)的正交矩阵\( P \), 使得\( P'AP = A' \).
\end{example}
\begin{proof}
由\hyperref[theorem:实正规矩阵的正交相似标准型]{实正规矩阵的正交相似标准型}可知存在正交矩阵$Q$, 使得
\begin{align*}
Q'AQ = \operatorname{diag}\left\{ \begin{pmatrix} a_1 & b_1 \\ -b_1 & a_1 \end{pmatrix}, \cdots, \begin{pmatrix} a_r & b_r \\ -b_r & a_r \end{pmatrix}, c_{2r+1}, \cdots, c_n \right\}
\end{align*}
为正交相似标准型, 上式两边转置后有
\begin{align*}
Q'A'Q = \operatorname{diag}\left\{ \begin{pmatrix} a_1 & -b_1 \\ b_1 & a_1 \end{pmatrix}, \cdots, \begin{pmatrix} a_r & -b_r \\ b_r & a_r \end{pmatrix}, c_{2r+1}, \cdots, c_n \right\}.
\end{align*}
设正交矩阵\( R = \operatorname{diag}\left\{ \begin{pmatrix} 0 & 1 \\ 1 & 0 \end{pmatrix}, \cdots, \begin{pmatrix} 0 & 1 \\ 1 & 0 \end{pmatrix}, 1, \cdots, 1 \right\} \), 其中有\( r \)个二阶分块, 则容易验证\( R'(Q'AQ)R = Q'A'Q \), 即有\( (QRQ')'A(QRQ') = A' \). 令\( P = QRQ' \), 则\( P \)为正交矩阵且\( P'AP = A' \). 又\( P \)正交相似于\( R \), 故其特征值为1或\(-1\). 

\end{proof}


\begin{proposition}\label{proposition:两正交矩阵的和的秩与n的差为奇数则行列式和为0}
设\(\boldsymbol{A},\boldsymbol{B}\)为\(n\)阶正交矩阵, 求证: \(\vert\boldsymbol{A}\vert+\vert\boldsymbol{B}\vert = 0\)当且仅当\(n-\mathrm{r}(\boldsymbol{A}+\boldsymbol{B})\)为奇数.
\end{proposition}
\begin{remark}
这个命题的直接推论是: 若正交矩阵\(\boldsymbol{A},\boldsymbol{B}\)满足\(\vert\boldsymbol{A}\vert+\vert\boldsymbol{B}\vert = 0\), 则\(\vert\boldsymbol{A}+\boldsymbol{B}\vert = 0\). 这一结论也可由第2章矩阵的技巧 (类似于例2.19的讨论) 来得到. 又因为正交矩阵行列式的值等于\(1\)或\(-1\), 故例9.119的等价命题为: 设\(\boldsymbol{A},\boldsymbol{B}\)为\(n\)阶正交矩阵, 则\(\vert\boldsymbol{A}\vert = \vert\boldsymbol{B}\vert\)当且仅当\(n-\mathrm{r}(\boldsymbol{A}+\boldsymbol{B})\)为偶数.
\end{remark}
\begin{proof}
因为正交矩阵的逆矩阵以及正交矩阵的乘积都是正交矩阵, 故\(\boldsymbol{AB}^{-1}\)还是正交矩阵. \(\vert\boldsymbol{A}\vert+\vert\boldsymbol{B}\vert = 0\)等价于\(\vert\boldsymbol{AB}^{-1}\vert = -1\), 又\(\mathrm{r}(\boldsymbol{A}+\boldsymbol{B})=\mathrm{r}(\boldsymbol{AB}^{-1}+\boldsymbol{I}_{n})\), 故只要证明: 若\(\boldsymbol{A}\)是\(n\)阶正交矩阵, 则\(\vert\boldsymbol{A}\vert = -1\)当且仅当\(n-\mathrm{r}(\boldsymbol{A}+\boldsymbol{I}_{n})\)为奇数即可. 下面给出两种证法.

{\color{blue}证法一:}设\(\boldsymbol{P}\)是正交矩阵, 使得
\begin{align*}
\boldsymbol{P}^{\prime}\boldsymbol{AP}=\mathrm{diag}\left\{\begin{pmatrix}\cos\theta_{1}&-\sin\theta_{1}\\\sin\theta_{1}&\cos\theta_{1}\end{pmatrix},\cdots,\begin{pmatrix}\cos\theta_{r}&-\sin\theta_{r}\\\sin\theta_{r}&\cos\theta_{r}\end{pmatrix},1,\cdots,1, -1,\cdots,-1\right\},
\end{align*}
其中\(\sin\theta_{i}\neq 0(1\leqslant  i\leqslant  r)\), 且有\(s\)个\(1\), \(t\)个\(-1\). 于是\(\vert\boldsymbol{A}\vert = (-1)^{t}\), 并且
\begin{align*}
\boldsymbol{P}^{\prime}(\boldsymbol{A}+\boldsymbol{I}_{n})\boldsymbol{P}&=\mathrm{diag}\left\{\begin{pmatrix}1 + \cos\theta_{1}&-\sin\theta_{1}\\\sin\theta_{1}&1 + \cos\theta_{1}\end{pmatrix},\cdots,\begin{pmatrix}1 + \cos\theta_{r}&-\sin\theta_{r}\\\sin\theta_{r}&1 + \cos\theta_{r}\end{pmatrix},2,\cdots,2,0,\cdots,0\right\},
\end{align*}
从而\(\mathrm{r}(\boldsymbol{A}+\boldsymbol{I}_{n}) = n - t\). 因此\(\vert\boldsymbol{A}\vert = -1\)当且仅当\(t\)为奇数, 即当且仅当\(n-\mathrm{r}(\boldsymbol{A}+\boldsymbol{I}_{n})\)为奇数.

{\color{blue}证法二:}由于正交矩阵\(\boldsymbol{A}\)也是复正规矩阵, 从而酉相似于对角矩阵, 特别地, \(\boldsymbol{A}\)可复对角化. 注意到\(\boldsymbol{A}\)的特征值是模长等于\(1\)的复数, 故或者是模长等于\(1\)的共轭虚特征值, 或者是\(\pm1\). 设\(\boldsymbol{A}\)的特征值\(-1\)的几何重数\(n-\mathrm{r}(\boldsymbol{A}+\boldsymbol{I}_{n}) = t\), 则其代数重数也为\(t\), 于是\(\vert\boldsymbol{A}\vert = (-1)^{t} = -1\)当且仅当\(n-\mathrm{r}(\boldsymbol{A}+\boldsymbol{I}_{n}) = t\)为奇数.


\end{proof}




\end{document}
\documentclass[../../main.tex]{subfiles}
\graphicspath{{\subfix{../../image/}}} % 指定图片目录,后续可以直接使用图片文件名。

% 例如:
% \begin{figure}[H]
% \centering
% \includegraphics{image-01.01}
% \caption{图片标题}
% \label{figure:image-01.01}
% \end{figure}
% 注意:上述\label{}一定要放在\caption{}之后,否则引用图片序号会只会显示??.

\begin{document}

\section{内积空间与Gram阵}

\begin{remark}
因为实数的共轭等于自身,所以实内积空间的定义相容于复内积空间的定义。因此在后面很多例题的叙述和解答的过程中,除非题目已标明是哪一类内积空间,否则我们一般都按照复内积空间的情形来处理。
\end{remark}

\begin{proposition}\label{proposition:例9.2}
设 $V$ 为内积空间,求证:

(1) 若 $(\alpha,\beta)=0$ 对任意的 $\beta\in V$ 都成立,则 $\alpha = 0$;若 $(\alpha,\beta)=0$ 对任意的 $\alpha\in V$ 都成立,则 $\beta = 0$;

(2) 设 $\{e_1,e_2,\cdots,e_n\}$ 是 $V$ 的一组基,若 $(\alpha,e_i)=(\beta,e_i)$ 对任意的 $i$ 都成立,则 $\alpha = \beta$。
\end{proposition}
\begin{proof}
(1) 若 $(\alpha,\beta)=0$ 对任意的 $\beta\in V$ 都成立,令 $\beta = \alpha$,可得 $(\alpha,\alpha)=0$,由内积的正定性即得 $\alpha = 0$。同理可证另一情形。

(2) 若 $(\alpha,e_i)=(\beta,e_i)$ 对任意的 $i$ 都成立,则 $(\alpha - \beta,e_i)=0$ 对任意的 $i$ 都成立。设 $\alpha - \beta = \sum_{i = 1}^{n}c_ie_i$,则由内积的第二变量的共轭线性可得
\begin{align*}
(\alpha - \beta,\alpha - \beta)=(\alpha - \beta,\sum_{i = 1}^{n}c_ie_i)=\sum_{i = 1}^{n}\overline{c_i}(\alpha - \beta,e_i)=0
\end{align*}
再由内积的正定性即得 $\alpha = \beta$。
\end{proof}

\begin{proposition}\label{proposition:例9.3}
设 $V$ 为 $n$ 维内积空间,$\{e_1,e_2,\cdots,e_n\}$ 和 $\{f_1,f_2,\cdots,f_n\}$ 分别是 $V$ 的两组基。设基 $\{e_1,e_2,\cdots,e_n\}$ 的 Gram 矩阵为 $G$,基 $\{f_1,f_2,\cdots,f_n\}$ 的 Gram 矩阵为 $H$,从基 $\{e_1,e_2,\cdots,e_n\}$ 到基 $\{f_1,f_2,\cdots,f_n\}$ 的过渡矩阵为 $C$。求证:若 $V$ 为欧氏空间,则 $H = C'GC$;若 $V$ 为酉空间,则 $H = C' \overline{G}C$。
\end{proposition}
\begin{proof}
设 $V$ 为酉空间,$G = (g_{ij})$,$H = (h_{ij})$,$C = (c_{ij})$,则 $f_k = \sum_{i = 1}^{n}c_{ik}e_i$,于是
\begin{align*}
h_{kl}&=(f_k,f_l)=(\sum_{i = 1}^{n}c_{ik}e_i,\sum_{j = 1}^{n}c_{jl}e_j)=\sum_{i,j = 1}^{n}c_{ik}\overline{c_{jl}}(e_i,e_j)=\sum_{i,j = 1}^{n}c_{ik}g_{ij}\overline{c_{jl}}.
\end{align*}
上式左边是 $H$ 的第 $(k, l)$ 元素,右边是 $C' \overline{G}C$ 的第 $(k, l)$ 元素,从而结论得证。
\end{proof}

\begin{corollary}\label{corollary:例9.3的推论}
若向量组 $\{\alpha_1,\alpha_2,\cdots,\alpha_m\}$ 与 $\{\beta_1,\beta_2,\cdots,\beta_k\}$ 满足 $\alpha_j = \sum_{i = 1}^{k}c_{ij}\beta_i$($1\leq j\leq m$),即 $(\alpha_1,\alpha_2,\cdots,\alpha_m)=(\beta_1,\beta_2,\cdots,\beta_k)C$,其中 $C = (c_{ij})_{k\times m}$,则有
\[
G(\alpha_1,\alpha_2,\cdots,\alpha_m)=C'G(\beta_1,\beta_2,\cdots,\beta_k)C.
\]
\end{corollary}
\begin{proof}
采用与\refpro{proposition:例9.3}类似的讨论即可证明.
\end{proof}

\begin{proposition}\label{proposition:例9.4}
设 $V$ 是 $n$ 维实(复)内积空间,$H$ 是一个 $n$ 阶正定实对称矩阵(正定 Hermite 矩阵),求证:必存在 $V$ 上的一组基 $\{f_1,f_2,\cdots,f_n\}$,使得它的 Gram 矩阵就是 $H$。
\end{proposition}
\begin{remark}
这个\refpro{proposition:例9.4}告诉我们,若给定一个 $n$ 维实(复)内积空间 $V$,则从 $V$ 所有的基构成的集合到所有 $n$ 阶正定实对称矩阵($n$ 阶正定 Hermite 矩阵)构成的集合有一个满射,它将 $V$ 的一组基映为这组基的 Gram 矩阵。这个映射当然不会是单射,请读者自行思考其中的原因。 
\end{remark}
\begin{proof}
任取 $V$ 的一组基 $\{e_1,e_2,\cdots,e_n\}$,设其 Gram 矩阵为 $G$,这也是一个 $n$ 阶正定实对称矩阵(正定 Hermite 矩阵),于是 $G$ 与 $H$ 合同(复相合),即存在 $n$ 阶非异阵 $C = (c_{ij})$,使得 $H = C'GC$($H = C' \overline{G}C$)。令 $f_j = \sum_{i = 1}^{n}c_{ij}e_i$($1 \leq j \leq n$),则由 $C$ 非异可知 $\{f_1,f_2,\cdots,f_n\}$ 是 $V$ 的一组基,并且从基 $\{e_1,e_2,\cdots,e_n\}$ 到基 $\{f_1,f_2,\cdots,f_n\}$ 的过渡矩阵恰为 $C$,再由\refpro{proposition:例9.3}可知,基 $\{f_1,f_2,\cdots,f_n\}$ 的 Gram 矩阵就是 $C'GC = H$($C' \overline{G}C = H$).
\end{proof}

\begin{proposition}\label{proposition:例9.6}
证明:在 $n$ 维欧氏空间 $V$ 中,两两夹角大于直角的向量个数至多是 $n + 1$ 个。
\end{proposition}
\begin{proof}
用反证法证明。假设存在 $n + 2$ 个两两夹角大于直角的向量 $\alpha_1,\alpha_2,\cdots,\alpha_{n + 1},\alpha_{n + 2}\in V$,则由 $\dim V = n$ 可知,$\alpha_1,\alpha_2,\cdots,\alpha_{n + 1}$ 必线性相关,即存在不全为零的实数 $c_1,c_2,\cdots,c_{n + 1}$,使得 $c_1\alpha_1 + c_2\alpha_2 + \cdots + c_{n + 1}\alpha_{n + 1} = \mathbf{0}$。将此式按照系数正负整理为如下形式:
\begin{align}
\sum_{c_i > 0}c_i\alpha_i = \sum_{c_j < 0}(-c_j)\alpha_j.\label{eq:9.4}
\end{align}
由 $c_1,c_2,\cdots,c_{n + 1}$ 不全为零不妨设存在某个 $c_i > 0$。若 \eqref{eq:9.4} 式两边都等于零,则有
\begin{align*}
0&=(\sum_{c_i > 0}c_i\alpha_i,\alpha_{n + 2})=\sum_{c_i > 0}c_i(\alpha_i,\alpha_{n + 2})<0,
\end{align*}
矛盾。因此 \eqref{eq:9.4} 式两边都非零,从而也存在某个 $c_j < 0$,于是
\begin{align*}
0&<(\sum_{c_i > 0}c_i\alpha_i,\sum_{c_i > 0}c_i\alpha_i)=(\sum_{c_i > 0}c_i\alpha_i,\sum_{c_j < 0}(-c_j)\alpha_j)=\sum_{c_i > 0}\sum_{c_j < 0}c_i(-c_j)(\alpha_i,\alpha_j)<0,
\end{align*}
矛盾。例如,不妨设 $V = \mathbb{R}^n$(取标准内积),则向量 $\alpha_1 = (n,-1,\cdots,-1)'$,$\alpha_2 = (-1,n,\cdots,-1)'$,$\alpha_n = (-1,-1,\cdots,n)'$,$\alpha_{n + 1} = (-1,-1,\cdots,-1)'$ 就满足两两夹角大于直角。因此,$n + 1$ 就是两两夹角大于直角的向量个数的最佳上界,结论得证。 
\end{proof}

\begin{corollary}\label{corollary:例9.6}
设 $\alpha_1,\alpha_2,\cdots,\alpha_{n + 1}$ 是 $n$ 维欧氏空间 $V$ 中两两夹角大于直角的 $n + 1$ 个向量,则

(1) $\alpha_1,\alpha_2,\cdots,\alpha_{n + 1}$ 中任意 $n$ 个向量必线性无关;

(2) $\alpha_1,\alpha_2,\cdots,\alpha_{n + 1}$ 中任一向量必为其余向量的负系数线性组合。
\end{corollary}
\begin{proof}
利用与\refpro{proposition:例9.6}的证明完全类似的讨论就能得到证明.
\end{proof}

\begin{proposition}\label{proposition:例9.7}
设 $V$ 是 $n$ 维欧氏空间,$\{e_1,e_2,\cdots,e_n\}$ 是 $V$ 的一组基,$c_1,c_2,\cdots,c_n$ 是 $n$ 个实数,求证:存在唯一的向量 $\alpha\in V$,使得对任意的 $i$,$(\alpha,e_i)=c_i$.
\end{proposition}
\begin{proof}
设 $\alpha = x_1e_1 + x_2e_2 + \cdots + x_ne_n$,则 $(\alpha,e_i)=c_i$($1\leq i\leq n$)等价于如下线性方程组:
\[
\begin{cases}
(e_1,e_1)x_1 + (e_1,e_2)x_2 + \cdots + (e_1,e_n)x_n = c_1,\\
(e_2,e_1)x_1 + (e_2,e_2)x_2 + \cdots + (e_2,e_n)x_n = c_2,\\
\cdots\cdots\cdots\cdots\\
(e_n,e_1)x_1 + (e_n,e_2)x_2 + \cdots + (e_n,e_n)x_n = c_n.
\end{cases}
\]
注意到上述方程组的系数矩阵是基 $\{e_1,e_2,\cdots,e_n\}$ 的 Gram 矩阵,即系数矩阵是度量矩阵,从而系数矩阵是正定矩阵,故其行列式非零,从而上述方程组有唯一解,于是满足条件的 $\alpha$ 存在且唯一。
\end{proof}

\begin{proposition}
设 $V$ 是实系数多项式全体构成的实线性空间,任取
\[
f(x)=a_0 + a_1x + \cdots + a_nx^n,\quad g(x)=b_0 + b_1x + \cdots + b_mx^m,
\]
证明:如下定义的二元运算是 $V$ 上的内积:
\[
(f,g)=\sum_{i,j}\frac{a_ib_j}{i + j + 1}.
\]
\end{proposition}
\begin{proof}
容易验证 $(f(x),g(x)) = \int_{0}^{1}f(x)g(x)\mathrm{d}x$,故由\hyperref[example:常见内积和内积空间-例9.1]{例题\ref{example:常见内积和内积空间-例9.1}(5)}即得结论。因为 $1,x,\cdots,x^{n - 1}$ 是 $V$ 中一组线性无关的向量,所以由\refpro{theorem:度量矩阵的性质} 知其 Gram 矩阵 $A = \left(\frac{1}{i + j - 1}\right)_{n\times n}$ 是一个正定阵,这也给出了\nrefexa{example:例8.48}{(2)}的几何证明.
\end{proof}

\begin{proposition}\label{proposition:例9.9}
设 $A$ 是 $n$ 阶半正定实对称矩阵,求证:对任意的 $n$ 维实列向量 $x, y$,有
\[
(x'Ay)^2 \leq (x'Ax)(y'Ay).
\]
\end{proposition}
\begin{proof}
{\color{blue}证法一:}
由\refpro{proposition:半正定阵关于摄动的充要条件}可知,对任意正实数 $t$,$A + tI_n$ 都是正定阵,这决定了 $n$ 维列向量空间 $\mathbb{R}^n$ 上的一个内积,故由 \hyperref[theorem:范数的基本性质]{Cauchy - Schwarz 不等式}可得
\[
(x'(A + tI_n)y)^2 \leq (x'(A + tI_n)x)(y'(A + tI_n)y).
\]
注意到上式两边都是关于 $t$ 的连续函数,同时取极限,令 $t \to 0^+$,即得结论。

{\color{blue}证法二:}
由于 $A$ 半正定,故存在实矩阵 $C$,使得 $A = C'C$。考虑 $n$ 维列向量空间 $\mathbb{R}^n$ 上的标准内积,由 \hyperref[theorem:范数的基本性质]{Cauchy - Schwarz 不等式}可得
\[
(x'Ay)^2 = ((Cx)'(Cy))^2 = (Cx, Cy)^2 \leq \|Cx\|^2\|Cy\|^2 = (Cx,Cx)(Cy,Cy) = (Cx)'(Cx)(Cy)'(Cy) = (x'Ax)(y'Ay).
\]

{\color{blue}证法三:}
因为 $A$ 是半正定阵,故对任意的实数 $t$,有
\[
(x'Ax)t^2 + 2(x'Ay)t + (y'Ay) = (tx + y)'A(tx + y) \geq 0.
\]
若 $x'Ax = 0$,则由\refpro{proposition:半正/负定阵关于线性方程的充要条件}可知 $Ax = 0$,从而 $x'Ay = (Ax)'y = 0$,又$A$是半正定阵,于是结论成立。
若 $x'Ax \neq 0$,则上述关于 $t$ 的二次方程恒大于等于零当且仅当其判别式小于等于零,由此即得要证的结论。
\end{proof}

\begin{lemma}\label{lemma:复内积空间的恒等式}
在$\mathbb{C}^n$(取标准内积)中,对任意$\alpha,\beta\in \mathbb{C}^n$,都有
\begin{align*}
\left( \varphi \left( \alpha \right) ,\beta \right) =\frac{1}{4}\left( \varphi \left( \alpha +\beta \right) ,\alpha +\beta \right) -\frac{1}{4}\left( \varphi \left( \alpha -\beta \right) ,\alpha -\beta \right) +\frac{\mathrm{i}}{4}\left( \varphi \left( \alpha +\mathrm{i}\beta \right) ,\alpha +\mathrm{i}\beta \right) -\frac{\mathrm{i}}{4}\left( \varphi \left( \alpha -\mathrm{i}\beta \right) ,\alpha -\mathrm{i}\beta \right) .
\end{align*}
\end{lemma}
\begin{proof}
\begin{align*}
&\frac{1}{4}\left( \varphi \left( \alpha +\beta \right) ,\alpha +\beta \right) -\frac{1}{4}\left( \varphi \left( \alpha -\beta \right) ,\alpha -\beta \right) +\frac{\mathrm{i}}{4}\left( \varphi \left( \alpha +\mathrm{i}\beta \right) ,\alpha +\mathrm{i}\beta \right) -\frac{\mathrm{i}}{4}\left( \varphi \left( \alpha -\mathrm{i}\beta \right) ,\alpha -\mathrm{i}\beta \right) 
\\
&=\frac{1}{4}\left[ \left( \varphi \left( \alpha \right) ,\alpha \right) +\left( \varphi \left( \alpha \right) ,\beta \right) +\left( \varphi \left( \beta \right) ,\alpha \right) +\left( \varphi \left( \beta \right) ,\beta \right) \right] -\frac{1}{4}\left[ \left( \varphi \left( \alpha \right) ,\alpha \right) -\left( \varphi \left( \alpha \right) ,\beta \right) -\left( \varphi \left( \beta \right) ,\alpha \right) +\left( \varphi \left( \beta \right) ,\beta \right) \right] 
\\
&\quad +\frac{\mathrm{i}}{4}\left[ \left( \varphi \left( \alpha \right) ,\alpha \right) +\left( \varphi \left( \alpha \right) ,\mathrm{i}\beta \right) +\left( \varphi \left( \mathrm{i}\beta \right) ,\alpha \right) +\left( \varphi \left( \mathrm{i}\beta \right) ,\mathrm{i}\beta \right) \right] -\frac{\mathrm{i}}{4}\left[ \left( \varphi \left( \alpha \right) ,\alpha \right) -\left( \varphi \left( \alpha \right) ,\mathrm{i}\beta \right) -\left( \varphi \left( \mathrm{i}\beta \right) ,\alpha \right) +\left( \varphi \left( \mathrm{i}\beta \right) ,\mathrm{i}\beta \right) \right] 
\\
&=\frac{1}{2}\left[ \left( \varphi \left( \alpha \right) ,\beta \right) +\left( \varphi \left( \beta \right) ,\alpha \right) \right] +\frac{\mathrm{i}}{2}\left[ \left( \varphi \left( \alpha \right) ,\mathrm{i}\beta \right) +\left( \varphi \left( \mathrm{i}\beta \right) ,\alpha \right) \right] 
\\
&=\frac{1}{2}\left[ \left( \varphi \left( \alpha \right) ,\beta \right) +\left( \varphi \left( \beta \right) ,\alpha \right) \right] +\frac{\mathrm{i}}{2}\left[ -\mathrm{i}\left( \varphi \left( \alpha \right) ,\beta \right) +\mathrm{i}\left( \varphi \left( \beta \right) ,\alpha \right) \right] 
\\
&=\frac{1}{2}\left[ \left( \varphi \left( \alpha \right) ,\beta \right) +\left( \varphi \left( \beta \right) ,\alpha \right) \right] -\frac{1}{2}\left[ -\left( \varphi \left( \alpha \right) ,\beta \right) +\left( \varphi \left( \beta \right) ,\alpha \right) \right] 
\\
&=\left( \varphi \left( \alpha \right) ,\beta \right) .
\end{align*}
\end{proof}

\begin{proposition}\label{proposition:例9.10}
证明:在 $\mathbb{R}^n$(取标准内积)中存在一个非零线性变换 $\varphi$,使 $\varphi(\alpha)\perp\alpha$ 对任意的 $\alpha\in\mathbb{R}^n$ 成立,但是在 $\mathbb{C}^n$(取标准内积)中这样的非零线性变换不存在。
\end{proposition}
\begin{proof}
任取一个 $n$ 阶非零实反对称矩阵 $A$,对任意的 $\alpha\in\mathbb{R}^n$,定义 $\varphi(\alpha)=A\alpha$,则由\refpro{proposition:反对称阵的刻画}可得 $(\alpha,\varphi(\alpha))=\alpha'A\alpha = 0$。下面给出 $\mathbb{C}^n$ 情形的 3 种证法。用反证法来证明,设在 $\mathbb{C}^n$(取标准内积)中存在满足条件的非零线性变换 $\varphi$。

{\color{blue}证法一:}
设 $\{e_1,e_2,\cdots,e_n\}$ 是 $\mathbb{C}^n$ 的标准单位列向量,$\varphi$ 在这组基下的表示矩阵为 $A=(a_{ij})$,则对任意的 $\alpha\in\mathbb{C}^n$,$\varphi(\alpha)=A\alpha$。由假设可知,对任意的 $\alpha\in\mathbb{C}^n$,有 $(\varphi(\alpha),\alpha)=\alpha'A'\overline{\alpha}=0$。取 $\alpha = e_i$,代入条件可得 $a_{ii}=0$($1\leq i\leq n$)。取 $\alpha = e_i + e_j$,代入条件可得 $a_{ij}+a_{ji}=0$($1\leq i<j\leq n$)。取 $\alpha = e_i + \mathrm{i}e_j$,代入条件可得 $a_{ij}-a_{ji}=0$($1\leq i<j\leq n$)。于是 $a_{ij}=a_{ji}=0$($1\leq i<j\leq n$),从而 $A = O$,这与 $\varphi\neq0$ 矛盾!

{\color{blue}证法二:}
首先,我们证明 $\varphi$ 的特征值全部为零。设 $\lambda_0$ 是 $\varphi$ 的特征值,$\alpha$ 是对应的特征向量,则 $0 = (\varphi(\alpha),\alpha)=(\lambda_0\alpha,\alpha)=\lambda_0(\alpha,\alpha)$,由于 $(\alpha,\alpha)\neq0$,故只能是 $\lambda_0 = 0$。其次,由 Jordan 标准型理论可知,存在 $\mathbb{C}^n$ 的一组基 $\{e_1,e_2,\cdots,e_n\}$,使得 $\varphi$ 在这组基下的表示矩阵为 $\mathrm{diag}\{J_{r_1}(0),J_{r_2}(0),\cdots,J_{r_k}(0)\}$。若 $\varphi$ 不可对角化,则必存在某个 $r_i>1$,不妨设 $r_1>1$,于是 $\varphi(e_1)=0$,$\varphi(e_2)=e_1$。由 $(\varphi(e_2),e_2)=0$ 可得 $(e_1,e_2)=0$,再由 $(\varphi(e_1 + e_2),e_1 + e_2)=0$ 可得 $(e_1,e_1)=0$,从而 $e_1 = 0$,这与假设矛盾,于是 $\varphi$ 可对角化。最后,由 $\varphi$ 的 Jordan 标准型是零矩阵可知 $\varphi = 0$,这与假设矛盾。

{\color{blue}证法三:}
由\reflem{lemma:复内积空间的恒等式}可知,对任意的 $\alpha,\beta\in\mathbb{C}^n$,有
\begin{align*}
(\varphi(\alpha),\beta)&=\frac{1}{4}(\varphi(\alpha + \beta),\alpha + \beta)-\frac{1}{4}(\varphi(\alpha - \beta),\alpha - \beta)\\
&+\frac{\mathrm{i}}{4}(\varphi(\alpha + \mathrm{i}\beta),\alpha + \mathrm{i}\beta)-\frac{\mathrm{i}}{4}(\varphi(\alpha - \mathrm{i}\beta),\alpha - \mathrm{i}\beta)=0.
\end{align*}
令 $\beta = \varphi(\alpha)$,由内积的正定性可得 $\varphi(\alpha)=0$ 对任意的 $\alpha\in\mathbb{C}^n$ 成立,即 $\varphi = 0$,这与假设矛盾。因此在 $\mathbb{C}^n$ 中满足条件的非零线性变换不存在。
\end{proof}

























\end{document}
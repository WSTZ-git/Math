\documentclass[../../main.tex]{subfiles}
\graphicspath{{\subfix{../../image/}}} % 指定图片目录,后续可以直接使用图片文件名。

% 例如:
% \begin{figure}[H]
% \centering
% \includegraphics[scale=0.4]{图.png}
% \caption{}
% \label{figure:图}
% \end{figure}
% 注意:上述\label{}一定要放在\caption{}之后,否则引用图片序号会只会显示??.

\begin{document}

\section{同时合同对角化}

\begin{proposition}[同时合同对角化]\label{proposition:例9.75}
设\(A\)是\(n\)阶正定实对称矩阵,\(B\)是同阶实对称矩阵,求证:必存在可逆矩阵\(C\),使得
\begin{align}
C'AC = I_n,\quad C'BC = \mathrm{diag}\{\lambda_1,\lambda_2,\cdots,\lambda_n\},\label{eq:9.11}
\end{align}
其中\(\lambda_1,\lambda_2,\cdots,\lambda_n\)是\(A^{-1}B\)的特征值.
\end{proposition}
\begin{proof}
因为\(A\)正定,故存在可逆矩阵\(P\),使得\(P'AP = I_n\). 由于\(P'BP\)仍为实对称矩阵,故存在正交矩阵\(Q\),使得
\begin{align*}
Q'(P'BP)Q = \mathrm{diag}\{\lambda_1,\lambda_2,\cdots,\lambda_n\}.
\end{align*}
令\(C = PQ\),则\(C\)满足\eqref{eq:9.11}式的要求. 注意到
\begin{align*}
C'(\lambda A - B)C=\lambda I_n - C'BC = \mathrm{diag}\{\lambda - \lambda_1,\lambda - \lambda_2,\cdots,\lambda - \lambda_n\},
\end{align*}
故\(\lambda_i\)是多项式\(\vert\lambda A - B\vert\)的根,又\(A\)可逆,所以也是\(\vert\lambda I_n - A^{-1}B\vert\)的根,即为\(A^{-1}B\)的特征值. 

\end{proof}

\begin{proposition}\label{proposition:例9.76}
设\(A\)是\(n\)阶半正定实对称矩阵,\(B\)是\(n\)阶半正定实对称矩阵. 求证:
\begin{align*}
\vert A + B\vert\geqslant \vert A\vert+\vert B\vert,
\end{align*}
等号成立的充要条件是\(n = 1\)或当\(n\geqslant 2\)时,\(B = O\).
\end{proposition}
\begin{remark}
这个\refpro{proposition:例9.76}也可通过与\refpro{proposition:反称矩阵A的I_n+A的行列式相关结论}和\refpro{proposition:A+S的行列式的相关结论}完全类似的讨论来得到,具体的细节留给读者完成. 
\end{remark}
\begin{proof}
不妨设$A$是正定矩阵,否则用$A+tI_n$摄动即可.由\refpro{proposition:例9.75}可知,存在可逆矩阵\(C\),使得
\begin{align*}
C'AC = I_n,\quad C'BC = \mathrm{diag}\{\lambda_1,\lambda_2,\cdots,\lambda_n\}.
\end{align*}
因为\(B\)半正定,故\(C'BC\)也半正定,从而\(\lambda_i\geqslant 0\). 注意到
\begin{align*}
\vert C'\vert\vert A + B\vert\vert C\vert&=\vert C'AC + C'BC\vert=(1 + \lambda_1)(1 + \lambda_2)\cdots(1 + \lambda_n)\\
&\geqslant 1 + \lambda_1\lambda_2\cdots\lambda_n=\vert C'AC\vert+\vert C'BC\vert=\vert C'\vert(\vert A\vert+\vert B\vert)\vert C\vert,
\end{align*}
故有\(\vert A + B\vert\geqslant \vert A\vert+\vert B\vert\),等号成立当且仅当\(n = 1\)或当\(n\geqslant 2\)时,所有的\(\lambda_i = 0\),这也当且仅当\(n = 1\)或当\(n\geqslant 2\)时,\(B = O\).

\end{proof}

\begin{proposition}\label{proposition:例9.77}
设\(A\),\(B\)都是\(n\)阶半正定实对称矩阵,求证:
\begin{align*}
\vert A + B\vert\geqslant 2^n\vert A\vert^{\frac{1}{2}}\vert B\vert^{\frac{1}{2}},
\end{align*}
等号成立的充要条件是\(A = B\).
\end{proposition}
\begin{proof}
不妨设$A$是正定矩阵,否则用$A+tI_n$摄动即可.由\refpro{proposition:例9.75}可知,存在可逆矩阵\(C\),使得
\begin{align*}
C'AC = I_n,\quad C'BC = \mathrm{diag}\{\lambda_1,\lambda_2,\cdots,\lambda_n\}.
\end{align*}
因为\(B\)正定,故\(C'BC\)也正定,从而\(\lambda_i>0\). 注意到
\begin{align*}
\vert C'\vert\vert A + B\vert\vert C\vert&=\vert C'AC + C'BC\vert=(1 + \lambda_1)(1 + \lambda_2)\cdots(1 + \lambda_n)\geqslant 2\sqrt{\lambda _1}\cdot 2\sqrt{\lambda _2}\cdots 2\sqrt{\lambda _n}\\
&\geqslant 2^n\sqrt{\lambda_1\lambda_2\cdots\lambda_n}=2^n\vert C'AC\vert^{\frac{1}{2}}\vert C'BC\vert^{\frac{1}{2}}=\vert C'\vert(2^n\vert A\vert^{\frac{1}{2}}\vert B\vert^{\frac{1}{2}})\vert C\vert,
\end{align*}
故\(\vert A + B\vert\geqslant 2^n\vert A\vert^{\frac{1}{2}}\vert B\vert^{\frac{1}{2}}\),等号成立当且仅当所有的\(\lambda_i = 1\),也当且仅当\(A = B\).

\end{proof}

\begin{example}\label{example:例9.78}
设\(A\),\(B\)都是\(n\)阶正定实对称矩阵,满足\(A\geqslant  B\),求证:\(B^{-1}\geqslant  A^{-1}\).
\end{example}
\begin{proof}
由\refpro{proposition:例9.75}可知,存在可逆矩阵\(C\),使得
\begin{align*}
C'AC = I_n,\quad C'BC = \mathrm{diag}\{\lambda_1,\lambda_2,\cdots,\lambda_n\}.
\end{align*}
因为\(B\)正定,故\(C'BC\)也正定,从而\(\lambda_i>0\). 一方面,我们有
\begin{align*}
C'(A - B)C = \mathrm{diag}\{1 - \lambda_1,1 - \lambda_2,\cdots,1 - \lambda_n\},
\end{align*}
因为\(A - B\)半正定,故\(\lambda_i\leqslant 1\),从而\(\lambda_i^{-1}\geqslant 1\). 另一方面,我们有
\begin{align*}
C^{-1}A^{-1}(C')^{-1} = I_n,\quad C^{-1}B^{-1}(C')^{-1} = \mathrm{diag}\{\lambda_1^{-1},\lambda_2^{-1},\cdots,\lambda_n^{-1}\},
\end{align*}
于是
\begin{align*}
C^{-1}(B^{-1} - A^{-1})(C^{-1})' = \mathrm{diag}\{\lambda_1^{-1} - 1,\lambda_2^{-1} - 1,\cdots,\lambda_n^{-1} - 1\}
\end{align*}
为半正定阵,因此\(B^{-1} - A^{-1}\)也是半正定阵.

\end{proof}

\begin{example}\label{example:例9.79}
设\(A\),\(B\)都是\(n\)阶正定实对称矩阵,满足\(A\geqslant  B\),求证:\(A^{\frac{1}{2}}\geqslant  B^{\frac{1}{2}}\).
\end{example}
\begin{proof}
由\refpro{proposition:例9.75}可知,存在可逆矩阵\(C\),使得
\begin{align*}
(C^{-1})'A^{\frac{1}{2}}C^{-1} = I_n,\quad (C^{-1})'B^{\frac{1}{2}}C^{-1} = \Lambda = \mathrm{diag}\{\lambda_1,\lambda_2,\cdots,\lambda_n\}.
\end{align*}
因为\(B^{\frac{1}{2}}\)正定,故\((C^{-1})'B^{\frac{1}{2}}C^{-1}\)也正定,从而\(\lambda_i>0\). 设正定阵\(CC' = D=(d_{ij})\),则\(d_{ii}>0\). 注意到\(A^{\frac{1}{2}} = C'C\),\(B^{\frac{1}{2}} = C'\Lambda C\),故有
\begin{align*}
A - B=(C'C)^2-(C'\Lambda C)^2 = C'(D - \Lambda D\Lambda)C\geqslant  O,
\end{align*}
于是\(D - \Lambda D\Lambda\)是半正定阵,从而其\((i,i)\)元素\(d_{ii}(1 - \lambda_i^2)\geqslant 0\),故\(0<\lambda_i\leqslant 1\). 因此
\begin{align*}
A^{\frac{1}{2}} - B^{\frac{1}{2}} = C'(I_n - \Lambda)C = C'\mathrm{diag}\{1 - \lambda_1,1 - \lambda_2,\cdots,1 - \lambda_n\}C\geqslant  O,
\end{align*}
从而结论得证.

\end{proof}

\begin{example}\label{example:例9.80}
设\(A\),\(B\)是\(n\)阶实对称矩阵,其中\(A\)正定且\(B\)与\(A - B\)均半正定,求证:\(\vert\lambda A - B\vert = 0\)的所有根全落在\([0,1]\)中,并且\(\vert A\vert\geqslant \vert B\vert\).
\end{example}
\begin{remark}
这一不等式也可由\refpro{proposition:例9.76}的半正定版本得到.
\end{remark}
\begin{proof}
由\refpro{proposition:例9.75}可知,存在可逆矩阵\(C\),使得
\begin{align*}
C'AC = I_n,\quad C'BC = \mathrm{diag}\{\lambda_1,\lambda_2,\cdots,\lambda_n\},
\end{align*}
其中\(\lambda_i\)是矩阵\(A^{-1}B\)的特征值,即是\(\vert\lambda A - B\vert = 0\)的根. 因为\(B\)半正定,故\(C'BC\)也半正定,从而\(\lambda_i\geqslant 0\). 因为\(A - B\)半正定,故\(C'(A - B)C = \mathrm{diag}\{1 - \lambda_1,1 - \lambda_2,\cdots,1 - \lambda_n\}\)也半正定,从而\(\lambda_i\leqslant 1\),因此\(\vert\lambda A - B\vert = 0\)的所有根\(\lambda_i\)全落在\([0,1]\)中. 由\(\vert A^{-1}B\vert=\lambda_1\lambda_2\cdots\lambda_n\leqslant 1\)可得\(\vert A\vert\geqslant \vert B\vert\). 

\end{proof}

\begin{example}\label{example:例9.81}
设\(A\)是\(m\times n\)实矩阵,\(B\)是\(s\times n\)实矩阵,又假设它们都是行满秩的. 令\(M = AB'(BB')^{-1}BA'\),求证:\(M\)和\(AA' - M\)都是半正定阵,并且\(\vert M\vert\leqslant \vert AA'\vert\).
\end{example}
\begin{proof}
设\(C=\begin{pmatrix}A\\B\end{pmatrix}\),则\(CC'=\begin{pmatrix}A\\B\end{pmatrix}(A',B')=\begin{pmatrix}AA'&AB'\\BA'&BB'\end{pmatrix}\)是半正定阵. 因为\(A\),\(B\)都是行满秩阵,故由\refpro{proposition:第8章解答题6}可得\(AA'\),\(BB'\)都是正定阵,从而\((BB')^{-1}\)也是正定阵,于是\(M = AB'(BB')^{-1}BA'\)是半正定阵. 对矩阵\(CC'\)实施对称分块初等变换可得
\begin{align*}
\begin{pmatrix}AA'&AB'\\BA'&BB'\end{pmatrix}\to\begin{pmatrix}AA' - AB'(BB')^{-1}BA'&O\\BA'&BB'\end{pmatrix}\to\begin{pmatrix}AA' - M&O\\O&BB'\end{pmatrix},
\end{align*}
由此即得\(AA' - M\)是半正定阵. 再由\refpro{proposition:例9.76}或\refexa{example:例9.80}即得
\begin{align*}
\left| M \right|\leqslant \left| AA' -M \right|+\left| M \right|\leqslant \left| AA'-M+M \right|=\left| AA\right|.
'\end{align*}

\end{proof}

\begin{proposition}[两个半正定阵可同时合同对角化]\label{proposition:两个半正定阵可同时合同对角化2}
设\(A\),\(B\)都是\(n\)阶半正定实对称矩阵,求证:存在可逆矩阵\(C\),使得
\begin{align*}
C'AC = \mathrm{diag}\{1,\cdots,1,0,\cdots,0\},\,\,C'BC = \mathrm{diag}\{\mu_1,\cdots,\mu_r,\mu_{r + 1},\cdots,\mu_n\}.
\end{align*}
\end{proposition}
\begin{remark}
\refpro{proposition:例9.75}的结论一般并不能推广到一个是半正定阵,另一个是实对称矩阵的情形. 例如,\(A=\begin{pmatrix}1&0\\0&0\end{pmatrix}\),\(B=\begin{pmatrix}0&1\\1&0\end{pmatrix}\),经过简单的计算可知\(A\),\(B\)不能同时合同对角化. 不过这个\refpro{proposition:两个半正定阵可同时合同对角化2}告诉我们,若\(A\),\(B\)都是半正定阵,则它们可以同时合同对角化.
\end{remark}
\begin{proof}
因为\(A\)是半正定阵,故存在可逆矩阵\(P\),使得\(P'AP=\begin{pmatrix}I_r&O\\O&O\end{pmatrix}\). 此时\(P'BP=\begin{pmatrix}B_{11}&B_{12}\\B_{21}&B_{22}\end{pmatrix}\)仍是半正定阵. 由\refpro{proposition:例8.75}可知\(\mathrm{r}(B_{21},B_{22})=\mathrm{r}(B_{22})\),故存在实矩阵\(M\),使得\(B_{21}=B_{22}M\). 考虑两个矩阵如下的同时合同变换:
\begin{align*}
\begin{pmatrix}I_r&-M'\\O&I_{n - r}\end{pmatrix}\begin{pmatrix}B_{11}&B_{12}\\B_{21}&B_{22}\end{pmatrix}\begin{pmatrix}I_r&O\\-M&I_{n - r}\end{pmatrix}=\begin{pmatrix}B_{11}-M'B_{22}M&O\\O&B_{22}\end{pmatrix},
\end{align*}
\begin{align*}
\begin{pmatrix}I_r&-M'\\O&I_{n - r}\end{pmatrix}\begin{pmatrix}I_r&O\\O&O\end{pmatrix}\begin{pmatrix}I_r&O\\-M&I_{n - r}\end{pmatrix}=\begin{pmatrix}I_r&O\\O&O\end{pmatrix}.
\end{align*}
由于\(B_{11}-M'B_{22}M\)和\(B_{22}\)都是半正定阵,故存在正交矩阵\(Q_1\),\(Q_2\),使得
\[
Q_1'(B_{11}-M'B_{22}M)Q_1 = \mathrm{diag}\{\mu_1,\cdots,\mu_r\},
Q_2'B_{22}Q_2 = \mathrm{diag}\{\mu_{r + 1},\cdots,\mu_n\}.
\]
令\(C = P\begin{pmatrix}I_r&O\\-M&I_{n - r}\end{pmatrix}\begin{pmatrix}Q_1&O\\O&Q_2\end{pmatrix}\),则\(C\)是可逆矩阵,使得
\[
C'AC = \mathrm{diag}\{1,\cdots,1,0,\cdots,0\},
C'BC = \mathrm{diag}\{\mu_1,\cdots,\mu_r,\mu_{r + 1},\cdots,\mu_n\}. 
\]

\end{proof}

\begin{proposition}\label{proposition:两个半正定阵可同时合同对角化3}
设\(A\),\(B\)都是\(n\)阶半正定实对称矩阵,求证:

(1) \(A + B\)是正定阵的充要条件是存在\(n\)个线性无关的实列向量\(\alpha_1,\alpha_2,\cdots,\alpha_n\),以及指标集\(I\subseteq\{1,2,\cdots,n\}\),使得
\(\alpha_i'A\alpha_j=\alpha_i'B\alpha_j = 0\ (\forall i\neq j)\),\(\alpha_i'A\alpha_i>0\ (\forall i\in I)\),\(\alpha_j'B\alpha_j>0\ (\forall j\notin I)\);

(2) \(\mathrm{r}(A\,\,|\,\, B)=\mathrm{r}(A + B)\).
\end{proposition}
\begin{proof}
(1) 在\refpro{proposition:两个半正定阵可同时合同对角化2}中,令\(C = (\alpha_1,\alpha_2,\cdots,\alpha_n)\)为其列分块,由此即得结论.

(2) 证明\(\mathrm{r}(A\,\,|\,\, B)=\mathrm{r}(A + B)\)有3种方法. 

第一种是利用线性方程组的求解理论,其讨论过程类似于\refpro{proposition:例8.76}的证法一. 

第二种方法是直接利用\refpro{proposition:例8.76}的结论,请参考\refpro{proposition:例8.77}的证明. 

第三种方法是直接利用\refpro{proposition:两个半正定阵可同时合同对角化2}的结论,有\(\mathrm{r}(A\,\,|\,\, B)=\mathrm{r}(C'AC\,\,|\,\, C'BC)\),此时\(C'AC\)和\(C'BC\)都是半正定对角矩阵. 若\(C'AC\)和\(C'BC\)同一行的主对角元全为零,则\((C'AC\,\,|\,\, C'BC)\)和\(C'(A + B)C\)的这一行都是零向量,对求秩不起作用;若\(C'AC\)和\(C'BC\)同一行的主对角元至少有一个大于零,则\((C'AC\,\,|\,\, C'BC)\)和\(C'(A + B)C\)的这一行对求秩都起了加1的作用,因此\(\mathrm{r}(A\,\,|\,\, B)=\mathrm{r}(C'AC\,\,|\,\, C'BC)=\mathrm{r}(C'(A + B)C)=\mathrm{r}(A + B)\).

\end{proof}

\begin{proposition}\label{proposition:两个半正定阵可同时合同对角化4}
设\(A\),\(B\),\(C\)都是\(n\)阶半正定实对称矩阵,使得\(ABC\)是对称矩阵,即满足\(ABC = CBA\),求证:\(ABC\)是半正定阵.
\end{proposition}
\begin{proof}
由\refpro{proposition:两个半正定阵可同时合同对角化2}可知,存在可逆矩阵\(P\),使得
\(P'AP = \mathrm{diag}\{1,\cdots,1,0,\cdots,0\}\),\(P'CP = \mathrm{diag}\{\mu_1,\cdots,\mu_r,\mu_{r + 1},\cdots,\mu_n\}\).
注意到问题的条件和结论在合同变换\(A\mapsto P'AP\),\(B\mapsto P^{-1}B(P^{-1})'\),\(C\mapsto P'CP\)下不改变,故不妨从一开始就假设
\begin{align*}
A=\begin{pmatrix}I_r&O\\O&O\end{pmatrix},\,\,C=\begin{pmatrix}\Lambda_1&O\\O&\Lambda_2\end{pmatrix},
\end{align*}
其中\(r = \mathrm{r}(A)\),\(\Lambda_1 = \mathrm{diag}\{\mu_1,\cdots,\mu_r\}\),\(\Lambda_2 = \mathrm{diag}\{\mu_{r + 1},\cdots,\mu_n\}\)都是半正定对角矩阵. 设\(B=\begin{pmatrix}B_{11}&B_{12}\\B_{21}&B_{22}\end{pmatrix}\)为对应的分块,则由\(ABC=\begin{pmatrix}B_{11}\Lambda_1&B_{12}\Lambda_2\\O&O\end{pmatrix}\)是对称矩阵可知,\(B_{11}\Lambda_1\)是对称矩阵且\(B_{12}\Lambda_2 = O\). 由\(B\)半正定可得\(B_{11}\)半正定,再由\refpro{proposition:例9.67}的半正定版本可知\(B_{11}\Lambda_1\)是半正定阵,因此\(ABC = \mathrm{diag}\{B_{11}\Lambda_1,O\}\)也是半正定阵. 

\end{proof}






\end{document}
\documentclass[../../main.tex]{subfiles}
\graphicspath{{\subfix{../../image/}}} % 指定图片目录,后续可以直接使用图片文件名。

% 例如:
% \begin{figure}[h]
% \centering
% \includegraphics{image-01.01}
% \caption{图片标题}
% \label{fig:image-01.01}
% \end{figure}
% 注意:上述\label{}一定要放在\caption{}之后,否则引用图片序号会只会显示??.

\begin{document}

\section{内积空间的基本概念}

\begin{definition}[Euclid空间]\label{definition:Euclid空间}
设 \(V\) 是实数域上的线性空间,若存在某种规则,使对 \(V\) 中任意一组有序向量 \(\{\alpha, \beta\}\),都唯一地对应一个实数,记为 \((\alpha, \beta)\),且适合如下规则:
\begin{flalign*}
& (1)~(\beta, \alpha)=(\alpha, \beta);
\\
& (2)~(\alpha + \beta, \gamma)=(\alpha, \gamma)+(\beta, \gamma);
\\
& (3)~(c\alpha, \beta)=c(\alpha, \beta),c\text{ 为任一实数};
\\
& (4)~(\alpha, \alpha)\geq0\text{ 且等号成立当且仅当 }\alpha = 0, &
\end{flalign*}
则称在 \(V\) 上定义了一个内积。实数 \((\alpha, \beta)\) 称为 \(\alpha\) 与 \(\beta\) 的内积。线性空间 \(V\) 称为\textbf{实内积空间}。有限维实内积空间称为\textbf{ Euclid 空间},简称为\textbf{欧氏空间}。
\end{definition}

\begin{definition}[酉空间]\label{definition:酉空间}
设 \(V\) 是复数域上的线性空间,若存在某种规则,使对 \(V\) 中任意一组有序向量 \(\{\alpha, \beta\}\),都唯一地对应一个复数,记为 \((\alpha, \beta)\),且适合如下规则:
\begin{flalign*}
&(1)~(\beta, \alpha)=\overline{(\alpha, \beta)};\\
&(2)~(\alpha + \beta, \gamma)=(\alpha, \gamma)+(\beta, \gamma);\\
&(3)~(c\alpha, \beta)=c(\alpha, \beta),c\text{ 为任一复数};\\
&(4)~(\alpha, \alpha)\geq0\text{ 且等号成立当且仅当 }\alpha = 0,&
\end{flalign*}
则称在 \(V\) 上定义了一个内积。复数 \((\alpha, \beta)\) 称为 \(\alpha\) 与 \(\beta\) 的内积。线性空间 \(V\) 称为\textbf{复内积空间}。有限维复内积空间称为\textbf{酉空间}。 
\end{definition}

\begin{remark}
实内积空间的定义与复内积空间的定义是相容的。事实上,对一个实数 \(a\),\(\overline{a}=a\),故\refdef{definition:Euclid空间} 中的 (1) 与\refdef{definition:酉空间}中的 (1) 是一致的。因此,我们经常将这两种空间统称为内积空间,在某些定理的叙述及证明中也不区分它们,而统一作为复内积空间来处理。但是,需要注意的是对复内积空间,\refdef{definition:酉空间}中的 (1), (3) 意味着:
\begin{align*}
(\alpha, c\beta)=\overline{c}(\alpha, \beta).
\end{align*} 
\end{remark}

\begin{definition}[标准内积]
\begin{enumerate}
\item 设 \(\mathbb{R}^n\) 是 \(n\) 维实列向量空间,\(\alpha=(x_1,x_2,\cdots,x_n)'\),\(\beta=(y_1,y_2,\cdots,y_n)'\),定义
\begin{align*}
(\alpha,\beta)=x_1y_1 + x_2y_2+\cdots + x_ny_n,
\end{align*}
则在此定义下 \(\mathbb{R}^n\) 成为一个欧氏空间,上述内积称为 \(\mathbb{R}^n\) 的标准内积。

\item 设 \(\mathbb{C}^n\) 是 \(n\) 维复列向量空间,\(\alpha=(x_1,x_2,\cdots,x_n)'\),\(\beta=(y_1,y_2,\cdots,y_n)'\),定义
\begin{align*}
(\alpha,\beta)=x_1\overline{y_1}+x_2\overline{y_2}+\cdots + x_n\overline{y_n},
\end{align*}
则在此定义下 \(\mathbb{C}^n\) 成为一个酉空间,上述内积称为 \(\mathbb{C}^n\) 的标准内积。
\end{enumerate}
\end{definition}
\begin{remark}
对 \(n\) 维实或复行向量空间,我们也可同样定义标准内积。 
\end{remark}

\begin{example}\label{example:一些常见的内积及内积空间}
\begin{enumerate}
\item 设 \(V\) 是由 \([a,b]\) 区间上连续函数全体构成的实线性空间,设 \(f(t),g(t)\in V\),定义
\begin{align*}
(f,g)=\int_{a}^{b}f(t)g(t)\mathrm{d}t,
\end{align*}
则不难验证这是一个内积,于是 \(V\) 成为内积空间。这是一个无限维实内积空间。

\item (1) 设 \(V\) 是 \(n\) 维实列向量空间,\(G\) 是 \(n\) 阶正定实对称阵,对 \(\alpha,\beta\in V\),定义
\begin{align*}
(\alpha,\beta)=\alpha'G\beta,
\end{align*}
则这是一个内积,并且 \(V\) 在上式的定义下成为欧氏空间。

(2)设 \(U\)是 \(n\) 维复列向量空间,若有正定 Hermite 矩阵 \(H\),对$\alpha ,\beta \in U$,定义 :
\begin{align*}
(\alpha,\beta)=\alpha'H\overline{\beta}.
\end{align*}
则这个\(U\) 上的一个内积,并且$U$在上式的定义下成为酉空间.
\end{enumerate}
\end{example}
\begin{proof}
\begin{enumerate}
\item 由内积空间的定义不难验证.

\item (1)\refdef{definition:Euclid空间}中的 (2), (3) 显然成立。对 (1),注意到 \(\alpha'G\beta\) 是实数,其转置仍是它自己,而 \(G\) 是对称阵,故
\begin{align*}
(\alpha,\beta)=\alpha'G\beta=(\alpha'G\beta)'=\beta'G'\alpha=\beta'G\alpha=(\beta,\alpha).
\end{align*}
又从 \(G\) 是正定阵即可知道 (4) 成立。 

(2)根据内积和酉空间的定义不难验证.
\end{enumerate}
\end{proof}
\begin{remark}
当 \(G = I_n\) 为单位阵时,\(V\) 上内积就是标准内积。对实列向量空间,标准内积可用矩阵乘法表示为
\begin{align*}
(\alpha,\beta)=\alpha'\beta.
\end{align*}
对实行向量空间,标准内积也可表示为
\begin{align*}
(\alpha,\beta)=\alpha\beta'.
\end{align*}

当 \(H = I_n\) 为单位阵时,\(U\) 上内积就是标准内积。对复列向量空间,标准内积可用矩阵乘法表示为
\begin{align*}
(\alpha,\beta)=\alpha'\overline{\beta}.
\end{align*}
对复行向量空间,标准内积也可表示为
\begin{align*}
(\alpha,\beta)=\alpha\overline{\beta}'.
\end{align*} 
\end{remark}

\begin{example}[常见内积和内积空间]\label{example:常见内积和内积空间-例9.1}
    
证明下列线性空间在给定的二元运算下成为内积空间:
(1) 设 $V = \mathbb{R}^n$ 为 $n$ 维实列向量空间,$G$ 为 $n$ 阶正定实对称矩阵,对任意的 $\alpha, \beta \in V$,定义 $(\alpha, \beta) = \alpha'G\beta$;

(2) 设 $V = \mathbb{R}_n$ 为 $n$ 维实行向量空间,$G$ 为 $n$ 阶正定实对称矩阵,对任意的 $\alpha, \beta \in V$,定义 $(\alpha, \beta) = \alpha G\beta'$;

(3) 设 $V = \mathbb{C}^n$ 为 $n$ 维复列向量空间,$H$ 为 $n$ 阶正定 Hermite 矩阵,对任意的 $\alpha, \beta \in V$,定义 $(\alpha, \beta) = \alpha' H\overline{\beta}$;

(4) 设 $V = \mathbb{C}^n$ 为 $n$ 维复行向量空间,$H$ 为 $n$ 阶正定 Hermite 矩阵,对任意的 $\alpha, \beta \in V$,定义 $(\alpha, \beta) = \alpha H\overline{\beta}'$;

(5) 设 $V = C[a, b]$ 为闭区间 $[a, b]$ 上的连续函数全体构成的实线性空间,对任意的 $f(t), g(t) \in V$,定义 $(f(t), g(t)) = \int_{a}^{b} f(t)g(t) \mathrm{d}t$;

(6) 设 $V = \mathbb{R}[x]$ 为实系数多项式全体构成的实线性空间,对任意的 $f(x) = a_0 + a_1x + \cdots + a_nx^n$, $g(x) = b_0 + b_1x + \cdots + b_mx^m$,定义 $(f(x), g(x)) = a_0b_0 + a_1b_1 + \cdots + a_kb_k$,其中 $k = \min\{n, m\}$;

(7) 设 $V = M_n(\mathbb{R})$ 为 $n$ 阶实矩阵全体构成的实线性空间,对任意的 $A = (a_{ij})$, $B = (b_{ij}) \in V$,定义 $(A, B) = \mathrm{tr}(AB') = \sum_{i, j = 1}^{n} a_{ij}b_{ij}$;

(8) 设 $V = M_n(\mathbb{C})$ 为 $n$ 阶复矩阵全体构成的复线性空间,对任意的 $A = (a_{ij})$, $B = (b_{ij}) \in V$,定义 $(A, B) = \mathrm{tr}(A\overline{B}') = \sum_{i, j = 1}^{n} a_{ij}\overline{b_{ij}}$。
\end{example}
\begin{proof}
(1) 首先注意到 $\alpha'G\beta$ 是一个数,$G$ 是实对称矩阵,故它们都等于自身的转置,从而 $(\alpha, \beta) = \alpha'G\beta = (\alpha'G\beta)' = \beta'G'\alpha = \beta G\alpha' = (\beta, \alpha)$,即得对称性;其次由矩阵乘法的性质可得第一变量的线性;最后由 $G$ 的正定性可知,$(\alpha, \alpha) = \alpha'G\alpha \geq 0$,且等号成立当且仅当 $\alpha = 0$,即得正定性。因此上述二元运算是 $\mathbb{R}^n$ 上的内积,称为由正定实对称矩阵 $G$ 定义的内积。当 $G = I_n$ 时,上述内积称为 $\mathbb{R}^n$ 上的标准内积。

(2) 类似于 (1) 的证明可得。当 $G = I_n$ 时,上述内积称为 $\mathbb{R}_n$ 上的标准内积。

(3) 首先注意到 $\overline{H}' = H$,故 $\overline{(\alpha, \beta)} = \overline{\alpha' H\overline{\beta}} = (\overline{\alpha' H\overline{\beta}})' = \beta'\overline{H}'\overline{\alpha} = \beta' H\overline{\alpha} = (\beta, \alpha)$,即得共轭对称性;其次由矩阵乘法的性质可得第一变量的线性;最后由 $H$ 的正定性可知,$(\alpha, \alpha) = \alpha' H\overline{\alpha} \geq 0$,且等号成立当且仅当 $\alpha = 0$,即得正定性。因此上述二元运算是 $\mathbb{C}^n$ 上的内积,称为由正定 Hermite 矩阵 $H$ 定义的内积。当 $H = I_n$ 时,上述内积称为 $\mathbb{C}^n$ 上的标准内积。

(4) 类似于 (3) 的证明可得。当 $H = I_n$ 时,上述内积称为 $\mathbb{C}_n$ 上的标准内积。

(5) 对称性显然成立;由积分运算的线性可得第一变量的线性;由连续函数的性质可得正定性,因此上述二元运算是 $C[a, b]$ 上的内积。

(6) 容易验证对称性、第一变量的线性和正定性都成立。

(7)由求迹运算的对称性、线性和正定性即得上述二元运算的对称性、线性和正定性,因此它是 $M_n(\mathbb{R})$ 上的内积。

(8) 证明是类似的。这两种由矩阵的迹定义的内积称为矩阵空间上的 Frobenius 内积。
\end{proof}

\begin{definition}[范数]\label{definition:向量的长度或范数}
设 \(V\) 是实或复的内积空间,\(\alpha\) 是 \(V\) 中的向量,定义 \(\alpha\) 的\textbf{长度}(或\textbf{范数})为
\begin{align*}
\|\alpha\|=(\alpha,\alpha)^{\frac{1}{2}},
\end{align*}
即实数 \((\alpha,\alpha)\) 的算术平方根。
\end{definition}
\begin{remark}
注意由\refdef{definition:Euclid空间}和\refdef{definition:酉空间}中的规则 (4) 可知,\((\alpha,\alpha)\) 总是非负实数。从长度的定义知,\(\|\alpha\| = 0\) 当且仅当 \(\alpha = 0\)。当 \(V=\mathbb{R}^n\) 且内积为标准内积时,若 \(\alpha=(x_1,x_2,\cdots,x_n)\),则
\begin{align*}
\|\alpha\|=\sqrt{x_1^2 + x_2^2+\cdots + x_n^2}.
\end{align*}
\end{remark}

\begin{definition}[两个向量的距离]\label{definition:两个向量的距离}
定义内积空间中两个向量的距离。设 \(\alpha,\beta\in V\),定义 \(\alpha\) 与 \(\beta\) 的\textbf{距离}为
\begin{align*}
d(\alpha,\beta)=\|\alpha - \beta\|.
\end{align*}
显然 \(d(\alpha,\beta)=d(\beta,\alpha)\)。
\end{definition}

\begin{theorem}[范数的基本性质]\label{theorem:范数的基本性质}
设 \(V\) 是实或复的内积空间,\(\alpha,\beta\in V\),\(c\) 是任一常数(实数或复数),则
\begin{flalign*}
&(1)~\|c\alpha\| = |c|\|\alpha\|;\\
&(2)(\mathrm{Cauchy}-\mathrm{Schwarz}\text{不等式})~|(\alpha,\beta)| \leq \|\alpha\|\cdot\|\beta\|;\\
&(3)(\text{三角不等式})~\|\alpha + \beta\| \leq \|\alpha\| + \|\beta\|.&
\end{flalign*}
\end{theorem}
\begin{proof}
\begin{enumerate}[(1)]
\item \(\|c\alpha\|^2=(c\alpha,c\alpha)=c\overline{c}(\alpha,\alpha)=|c|^2\|\alpha\|^2\),故 \(\|c\alpha\| = |c|\|\alpha\|\)。

\item 若 \(\alpha = 0\),则 \((0,\beta)=(0 + 0,\beta)=2(0,\beta)\),故 \((0,\beta)=0\),因此 (2) 成立。若 \(\alpha\neq0\),令
\begin{align*}
v = \beta - \frac{(\beta,\alpha)}{\|\alpha\|^2}\alpha,
\end{align*}
则 \((v,\alpha)=0\),且
\begin{align*}
0\leq\|v\|^2&=\left(\beta - \frac{(\beta,\alpha)}{\|\alpha\|^2}\alpha,\beta - \frac{(\beta,\alpha)}{\|\alpha\|^2}\alpha\right)\\
&=(\beta,\beta)-\frac{(\beta,\alpha)}{\|\alpha\|^2}(\alpha,\beta)\\
&=\|\beta\|^2 - \frac{|(\alpha,\beta)|^2}{\|\alpha\|^2},
\end{align*}
由此即可得 (2)。

\item 我们有
\begin{align*}
\|\alpha + \beta\|^2&=(\alpha + \beta,\alpha + \beta)\\
&=\|\alpha\|^2 + (\alpha,\beta)+(\beta,\alpha)+\|\beta\|^2\\
&=\|\alpha\|^2+\|\beta\|^2+(\alpha,\beta)+\overline{(\alpha,\beta)}.
\end{align*}
由 (2) 得
\(|(\alpha,\beta)|\leq\|\alpha\|\|\beta\|,\overline{|(\alpha,\beta)|}\leq\|\alpha\|\|\beta\|\),
故
\(\|\alpha + \beta\|^2\leq\|\alpha\|^2+\|\beta\|^2 + 2\|\alpha\|\|\beta\|=(\|\alpha\|+\|\beta\|)^2\).
\end{enumerate}
\end{proof}

\begin{definition}[向量的夹角]
当 \(V\) 是实内积空间时,定义非零向量 \(\alpha,\beta\) 的夹角 \(\theta\) 之余弦为
\begin{align}
\cos\theta=\frac{(\alpha,\beta)}{\|\alpha\|\|\beta\|}.\label{definition:equatioin:-9.1.3}
\end{align}
当 \(V\) 是复内积空间时,定义非零向量 \(\alpha,\beta\) 的夹角 \(\theta\) 之余弦为
\begin{align*}
\cos\theta=\frac{|(\alpha,\beta)|}{\|\alpha\|\|\beta\|}.
\end{align*}
内积空间中两个向量 \(\alpha,\beta\) 若适合 \((\alpha,\beta)=0\),则称 \(\alpha\) 与 \(\beta\) \textbf{垂直}或\textbf{正交},我们用记号 \(\alpha\perp\beta\) 来表示。
显然,我们有以下结论:
\begin{enumerate}
\item 零向量和任何向量都正交;

\item 若 \(\alpha\) 与 \(\beta\) 正交,则 \(\beta\) 也与 \(\alpha\) 正交;

\item 两个非零向量 \(\alpha,\beta\) 正交时夹角为 \(90^{\circ}\)。
\end{enumerate}
\end{definition}
\begin{remark}
\eqref{definition:equatioin:-9.1.3}式中要使 \(\theta\) 有意义,必须保证 \(|\cos\theta|\leq1\),而这就是\hyperref[theorem:范数的基本性质]{范数的基本性质}中的 (2)。因此上述定义的$\theta$都是良定义的.
\end{remark}

\begin{corollary}\label{corollary:范数性质的相关推广}
\begin{enumerate}
\item\label{corollary:范数性质的相关推广-勾股定理} (勾股定理)在\hyperref[theorem:范数的基本性质]{范数的基本性质(3)}的证明中我们可看出:若 \(\alpha\) 与 \(\beta\) 正交,则 \((\alpha,\beta)=(\beta,\alpha)=0\),因此
\begin{align*}
\|\alpha + \beta\|^2=\|\alpha\|^2+\|\beta\|^2.
\end{align*}
上式通常称为\text{勾股定理},它是平面几何中勾股定理的推广。

\item\label{corollary:范数性质的相关推广-Cauchy不等式} (Cauchy不等式)设 \(V\) 是 \(n\) 维实行向量空间,内积取标准内积,从\hyperref[theorem:范数的基本性质]{范数的基本性质(2)}立即可得到下列 Cauchy 不等式:
\begin{align*}
(x_1y_1 + x_2y_2+\cdots + x_ny_n)^2\leq(x_1^2 + x_2^2+\cdots + x_n^2)(y_1^2 + y_2^2+\cdots + y_n^2).
\end{align*}

\item\label{corollary:范数性质的相关推广-Schwarz不等式} (Schwarz不等式)设 \(V\) 是由 \([a,b]\) 区间上连续函数全体构成的实线性空间,内积如\hyperref[example:一些常见的内积及内积空间]{例题\ref{example:一些常见的内积及内积空间}(1)},则从\hyperref[theorem:范数的基本性质]{范数的基本性质(2)}可得下列 Schwarz 不等式:
\begin{align*}
\left(\int_{a}^{b}f(t)g(t)\mathrm{d}t\right)^2\leq\int_{a}^{b}f(t)^2\mathrm{d}t\int_{a}^{b}g(t)^2\mathrm{d}t.
\end{align*} 
\end{enumerate}
\end{corollary}









\end{document}
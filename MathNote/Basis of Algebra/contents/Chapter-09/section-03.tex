\documentclass[../../main.tex]{subfiles}
\graphicspath{{\subfix{../../image/}}} % 指定图片目录,后续可以直接使用图片文件名。

% 例如:
% \begin{figure}[h]
% \centering
% \includegraphics{image-01.01}
% \caption{图片标题}
% \label{fig:image-01.01}
% \end{figure}
% 注意:上述\label{}一定要放在\caption{}之后,否则引用图片序号会只会显示??.

\begin{document}

\section{伴随}

\begin{definition}[伴随]
设 $\varphi$ 是内积空间 $V$ 上的线性算子,若存在 $V$ 上的线性算子 $\varphi^*$,使等式
\begin{align*}
(\varphi(\alpha),\beta) = (\alpha,\varphi^*(\beta))
\end{align*}
对一切 $\alpha,\beta \in V$ 成立,则称 $\varphi^*$ 是 $\varphi$ 的伴随算子,简称为 $\varphi$ 的\textbf{伴随}。
\end{definition}

\begin{theorem}
设 $V$ 是 $n$ 维内积空间,$\varphi$ 是 $V$ 上的线性变换,则存在 $V$ 上唯一的线性变换 $\varphi^*$,使对一切 $\alpha,\beta \in V$,成立
\begin{align*}
(\varphi(\alpha),\beta) = (\alpha,\varphi^*(\beta)).
\end{align*}
\end{theorem}
\begin{note}
这个定理表明:对有限维内积空间$V$上的任一线性算子,它的伴随必存在且唯一.
\end{note}
\begin{proof}
只需证明唯一性. 若 $\varphi^{\sharp}$ 是 $V$ 上的线性变换且
\[
(\varphi(\alpha),\beta) = (\alpha,\varphi^{\sharp}(\beta))
\]
对一切 $\alpha,\beta \in V$ 成立,则 $(\alpha,\varphi^{\sharp}(\beta)) = (\alpha,\varphi^*(\beta))$ 对一切 $\alpha \in V$ 成立,即 $(\alpha,\varphi^{\sharp}(\beta) - \varphi^*(\beta)) = 0$ 对一切 $\alpha \in V$ 成立,特别,对 $\alpha = \varphi^{\sharp}(\beta) - \varphi^*(\beta)$ 也成立. 由内积定义即知 $\varphi^{\sharp}(\beta) - \varphi^*(\beta) = 0$,即 $\varphi^{\sharp}(\beta) = \varphi^*(\beta)$. 而 $\beta$ 是任意的,故有 $\varphi^{\sharp} = \varphi^*$.
\end{proof}

\begin{theorem}\label{theorem:伴随算子的矩阵}
设 $V$ 是 $n$ 维内积空间,$\{e_1,e_2,\cdots,e_n\}$ 是 $V$ 的一组标准正交基. 若 $V$ 上的线性算子 $\varphi$ 在这组基下的表示矩阵为 $A$,则

(1)当 $V$ 是酉空间时,$\varphi^*$ 在同一组基下的表示矩阵为 $\overline{A}'$,即 $A$ 的共轭转置;

(2)当 $V$ 是欧氏空间时,$\varphi^*$ 的表示矩阵为 $A'$,即 $A$ 的转置.
\end{theorem}
\begin{proof}
由伴随的唯一性知道本节一开始由 $\overline{A}'$ 定义的线性变换 $\psi$ 就是 $\varphi$ 的伴随,而 $\psi$ 的表示矩阵就是 $\overline{A}'$. 
\end{proof}

\begin{theorem}[伴随算子的性质]\label{theorem:伴随算子的性质}
设 $V$ 是有限维内积空间,若 $\varphi$ 及 $\psi$ 是 $V$ 上的线性变换,$c$ 为常数,则
\begin{enumerate}[(1)]
\item $(\varphi + \psi)^* = \varphi^* + \psi^*$;

\item $(c\varphi)^* = \overline{c}\varphi^*$;

\item $(\varphi\psi)^* = \psi^*\varphi^*$;

\item $(\varphi^*)^* = \varphi$.
\end{enumerate} 
\end{theorem}
\begin{proof}
由矩阵和线性变换的一一对应关系及矩阵共轭转置的性质即得.
\end{proof}

\begin{proposition}\label{proposition:线性算子的正交补空间就是伴随算子的不变子空间}
设 $V$ 是 $n$ 维内积空间,$\varphi$ 是 $V$ 上的线性算子.
\begin{enumerate}[(1)]
\item 若 $U$ 是 $\varphi$ 的不变子空间,则 $U^{\perp}$ 是 $\varphi^*$ 的不变子空间;

\item 若 $\varphi$ 的全体特征值为 $\lambda_1,\lambda_2,\cdots,\lambda_n$,则 $\varphi^*$ 的全体特征值为 $\overline{\lambda}_1,\overline{\lambda}_2,\cdots,\overline{\lambda}_n$.
\end{enumerate}
\end{proposition}
\begin{proof}
\begin{enumerate}[(1)]
\item 任取 $\alpha \in U, \beta \in U^{\perp}$,因为
\[
(\alpha,\varphi^*(\beta)) = (\varphi(\alpha),\beta) = 0,
\]
所以 $U^{\perp}$ 是 $\varphi^*$ 的不变子空间.

\item 取 $V$ 的一组标准正交基,设 $\varphi$ 在这组基下的表示矩阵为 $A$,则无论 $V$ 是酉空间还是欧氏空间,$\varphi^*$ 的表示矩阵总可写为 $\overline{A}'$. 由假设
\[
|\lambda I_n - A| = (\lambda - \lambda_1)(\lambda - \lambda_2)\cdots(\lambda - \lambda_n),
\]
则容易验证
\[
|\lambda I_n - \overline{A}'| = (\lambda - \overline{\lambda}_1)(\lambda - \overline{\lambda}_2)\cdots(\lambda - \overline{\lambda}_n),
\]
故结论成立.
\end{enumerate}
\end{proof}







\end{document}
\documentclass[../../main.tex]{subfiles}
\graphicspath{{\subfix{../../image/}}} % 指定图片目录,后续可以直接使用图片文件名。

% 例如:
% \begin{figure}[h]
% \centering
% \includegraphics{image-01.01}
% \caption{图片标题}
% \label{fig:image-01.01}
% \end{figure}
% 注意:上述\label{}一定要放在\caption{}之后,否则引用图片序号会只会显示??.

\begin{document}

\section{自伴随算子}

\begin{lemma}\label{lemma:正交基之间的过渡矩阵是正交(酉)矩阵}
欧氏空间中两组标准正交基之间的过渡矩阵是正交矩阵;

酉空间中两组标准正交基之间的过渡矩阵是酉矩阵.
\end{lemma}
\begin{proof}
设 $V$ 是 $n$ 维欧氏空间,$\{e_1,e_2,\cdots,e_n\}$ 和 $\{f_1,f_2,\cdots,f_n\}$ 是 $V$ 的两组标准正交基且
\[
\begin{cases}
f_1 = a_{11}e_1 + a_{21}e_2 + \cdots + a_{n1}e_n, \\
f_2 = a_{12}e_1 + a_{22}e_2 + \cdots + a_{n2}e_n, \\
\cdots\cdots\cdots\cdots \\
f_n = a_{1n}e_1 + a_{2n}e_2 + \cdots + a_{nn}e_n.
\end{cases}
\]
因为 $(f_i,f_i) = 1$,故
\begin{align}
a_{1i}^2 + a_{2i}^2 + \cdots + a_{ni}^2 = 1.\label{equation--------9.5.1}
\end{align}
又若 $i \neq j$,则 $(f_i,f_j) = 0$,故
\begin{align}
a_{1i}a_{1j} + a_{2i}a_{2j} + \cdots + a_{ni}a_{nj} = 0.\label{equation--------9.5.2}
\end{align}
\eqref{equation--------9.5.1}式、\eqref{equation--------9.5.2}式和\reftheorem{theorem:正交矩阵的基本性质1}表明过渡矩阵
\[
P = \begin{pmatrix}
a_{11} & a_{12} & \cdots & a_{1n} \\
a_{21} & a_{22} & \cdots & a_{2n} \\
\vdots & \vdots & & \vdots \\
a_{n1} & a_{n2} & \cdots & a_{nn}
\end{pmatrix}
\]
是正交矩阵. 同理可证明酉空间中两组标准正交基之间的过渡矩阵是酉矩阵.
\end{proof}

\begin{definition}[正交相似和酉相似]
\begin{enumerate}
\item 设 $A,B$ 是 $n$ 阶实矩阵,若存在正交矩阵 $P$,使 
\begin{align*}
B = P'AP=P^{-1}AP.
\end{align*}
则称 $B$ 和 $A$ \textbf{正交相似}. 

\item 设 $A,B$ 是 $n$ 阶复矩阵,若存在酉矩阵 $P$,使 
\begin{align*}
B = \overline{P}'AP=P^{-1}AP.
\end{align*}
则称 $B$ 和 $A$ \textbf{酉相似}.
\end{enumerate}
\end{definition}
\begin{remark}
和矩阵的相似关系一样,我们不难证明正交 (酉) 相似关系是等价关系,即:
\begin{enumerate}
\item $n$ 阶矩阵 $A$ 和自己正交 (酉) 相似;

\item 若 $B$ 和 $A$ 正交 (酉) 相似,则 $A$ 和 $B$ 也正交 (酉) 相似;

\item 若 $B$ 和 $A$ 正交 (酉) 相似,$C$ 和 $B$ 正交 (酉) 相似,则 $C$ 和 $A$ 也正交 (酉) 相似. 
\end{enumerate}
\end{remark}

\begin{definition}[自伴随算子]
设 $\varphi$ 是内积空间 $V$ 上的线性变换,$\varphi^*$ 是 $\varphi$ 的伴随,若 $\varphi^* = \varphi$,则称 $\varphi$ 是\textbf{自伴随算子}. 

当 $V$ 是欧氏空间时,$\varphi$ 也称为\textbf{对称算子}或\textbf{对称变换};

当 $V$ 是酉空间时,$\varphi$ 也称为 \textbf{Hermite 算子}或 \textbf{Hermite 变换}.
\end{definition}

\begin{proposition}
设 $V$ 是 $n$ 维内积空间,$V_0$ 是 $V$ 的子空间,$V = V_0 \oplus V_0^\perp$. 令 $E$ 是 $V$ 到 $V_0$ 上的正交投影,则 $E$ 是自伴随算子. 
\end{proposition}
\begin{proof}
由\refproposition{proposition:正交投影的性质}立得.
\end{proof}

\begin{theorem}\label{theorem:内积空间的自伴随算子关于表示矩阵的充要条件}
\begin{enumerate}
\item 设 $V$ 是 $n$ 维欧氏空间,$\varphi$是$V$上的线性变换,则$\varphi$是自伴随算子的充要条件是其在任一组标准正交基下的表示矩阵都是实对称阵.

\item 设 $V$ 是 $n$ 维酉空间,$\varphi$是$V$上的线性变换,则$\varphi$是自伴随算子的充要条件是其在任一组标准正交基下的表示矩阵都是实对称阵. 
\end{enumerate}
\end{theorem}
\begin{proof}
\begin{enumerate}
\item 设$\{e_1,e_2,\cdots,e_n\}$ 是$V$ 中的一组标准正交基,若 $\varphi$ 在这组基下的表示矩阵为 $A$,则 $\varphi^*$ 在同一组基下的表示矩阵为 $A'$. 若 $\varphi^* = \varphi$,则 $A' = A$. 也就是说欧氏空间上的自伴随算子在任一组标准正交基下的表示矩阵都是实对称阵.

反之亦容易看出,若欧氏空间上的线性变换 $\varphi$ 在某一组标准正交基下的表示矩阵是实对称阵,则 $\varphi^* = \varphi$.

\item 同理可证明对酉空间也有类似结论. 
\end{enumerate}
\end{proof}

\begin{theorem}\label{theorem:Hermite算子关于特征值的相关性质}
设 $V$ 是 $n$ 维酉空间,$\varphi$ 是 $V$ 上的自伴随算子,则 $\varphi$ 的特征值全是实数且属于不同特征值的特征向量互相正交.
\end{theorem}
\begin{proof}
设 $\lambda$ 是 $\varphi$ 的特征值,$x$ 是属于 $\lambda$ 的特征向量,则
\begin{align*}
\lambda(x,x) &= (\lambda x,x) = (\varphi(x),x) = (x,\varphi^*(x))\\
&= (x,\varphi(x)) = (x,\lambda x) = \overline{\lambda}(x,x).
\end{align*}
因为 $(x,x) \neq 0$,故 $\overline{\lambda} = \lambda$,即 $\lambda$ 是实数. 又若设 $\mu$ 是 $\varphi$ 的另一个特征值,$y$ 是属于 $\mu$ 的特征向量,注意到 $\lambda,\mu$ 都是实数,故有
\begin{align*}
\lambda(x,y) &= (\lambda x,y) = (\varphi(x),y) = (x,\varphi^*(y))\\
&= (x,\varphi(y)) = (x,\mu y) = \mu(x,y).
\end{align*}
由于 $\lambda \neq \mu$,故 $(x,y) = 0$,即 $x \perp y$.
\end{proof}

\begin{corollary}\label{corollary:Hermite矩阵和实对称矩阵关于特征值的相关性质}
Hermite 矩阵的特征值全是实数,实对称阵的特征值也全是实数. 这两种矩阵属于不同特征值的特征向量互相正交.
\end{corollary}
\begin{proof}
Hermite 矩阵的结论是定理的显然推论,而实对称阵也是 Hermite 矩阵,因此结论成立.
\end{proof}

\begin{theorem}\label{theorem:关于自伴随算子正交对角化}
设 $V$ 是 $n$ 维内积空间,$\varphi$ 是 $V$ 上的自伴随算子,则存在 $V$ 的一组标准正交基,使 $\varphi$ 在这组基下的表示矩阵为实对角阵,且这组基恰为 $\varphi$ 的 $n$ 个线性无关的特征向量.
\end{theorem}
\begin{proof}
首先需要说明的是,若 $V$ 是欧氏空间,则由于自伴随算子 $\varphi$ 的特征值都是实数,故有实的特征向量. 不妨设 $u$ 是 $\varphi$ 的特征向量,令 $v_1 = \frac{u}{\|u\|}$,则 $v_1$ 是 $\varphi$ 的长度等于 $1$ 的特征向量. 我们对维数 $n$ 用归纳法.

若 $\dim V = 1$,结论已成立. 设对小于 $n$ 维的内积空间结论成立. 令 $W$ 为由 $v_1$ 张成的子空间,$W^\perp$ 为 $W$ 的正交补空间,则 $W$ 是 $\varphi$ 的不变子空间且
\[
V = W \oplus W^\perp, \quad \dim W^\perp = n - 1.
\]
由命题 9.3.1 可知 $W^\perp$ 是 $\varphi^* = \varphi$ 的不变子空间. 将 $\varphi$ 限制在 $W^\perp$ 上仍是自伴随算子. 由归纳假设,存在 $W^\perp$ 的一组标准正交基 $\{v_2,\cdots,v_n\}$,使 $\varphi$ 在这组基下的表示矩阵为实对角阵,且 $\{v_2,\cdots,v_n\}$ 是其特征向量. 因此,$\{v_1,v_2,\cdots,v_n\}$ 构成了 $V$ 的一组标准正交基,$\varphi$ 在这组基下的表示矩阵为实对角阵,且 $\{v_1,v_2,\cdots,v_n\}$ 为 $\varphi$ 的 $n$ 个线性无关的特征向量.
\end{proof}

\begin{theorem}\label{theorem:实对称和Hermite矩阵的正交对角化}
\begin{enumerate}
\item 设 $A$ 是 $n$ 阶实对称阵,则存在正交矩阵 $P$,使 $P'AP$ 为对角阵,且 $P$ 的 $n$ 个列向量恰为 $A$ 的 $n$ 个两两正交的单位特征向量.

\item 设 $A$ 是 $n$ 阶 Hermite 矩阵,则存在酉矩阵 $P$,使 $\overline{P}'AP$ 为实对角阵,且 $P$ 的 $n$ 个列向量恰为 $A$ 的 $n$ 个两两正交的单位特征向量.
\end{enumerate}
\end{theorem}
\begin{proof}
由\reftheorem{theorem:关于自伴随算子正交对角化}即知实对称阵正交相似于对角阵,Hermite 矩阵酉相似于实对角阵. 从对角化相关理论知道 $P$ 的列向量都是 $A$ 的特征向量,又 $P$ 是正交 (酉) 矩阵,故这些列向量两两正交且长度等于 $1$. 
\end{proof}

\begin{corollary}\label{corollary:实对称和Hermite矩阵的全系不变量}
实对称阵的全体特征值是实对称阵在正交相似关系下的全系不变量,Hermite 矩阵的全体特征值是 Hermite 矩阵在酉相似关系下的全系不变量.
\end{corollary}
\begin{proof}
只证明实的情形. 显然正交相似的矩阵有相同的特征值. 另一方面,由\reftheorem{theorem:实对称和Hermite矩阵的正交对角化}知道只需对对角阵证明,若它们的特征值相同,则必正交相似即可. 设
\[
B = \mathrm{diag}\{\lambda_1,\lambda_2,\cdots,\lambda_n\}, \quad D = \mathrm{diag}\{\lambda_{i_1},\lambda_{i_2},\cdots,\lambda_{i_n}\},
\]
其中 $\{\lambda_{i_1},\lambda_{i_2},\cdots,\lambda_{i_n}\}$ 是 $\{\lambda_1,\lambda_2,\cdots,\lambda_n\}$ 的一个排列. 由于任一排列可通过若干次对换来实现,因此只要证明对 $B$ 的第 $(i,i)$ 元素和第 $(j,j)$ 元素对换后得到的矩阵与 $B$ 正交相似即可. 设 $P_{ij}$ 是第一类初等矩阵,则 $P_{ij}$ 是正交矩阵且 $P_{ij}' = P_{ij}$,因此 $P_{ij}'BP_{ij}$ 和 $B$ 正交相似. 这就是我们要证明的结论.
\end{proof}

\begin{theorem}\label{theorem:二次型经正交变换后可化为标准型}
设 $f(x) = x'Ax$ 是 $n$ 元实二次型,系数矩阵 $A$ 的特征值为 $\lambda_1,\lambda_2,\cdots,\lambda_n$,则 $f$ 经过正交变换 $x = Py$ 可以化为下列标准型:
\[
\lambda_1y_1^2 + \lambda_2y_2^2 + \cdots + \lambda_ny_n^2.
\]
因此,$f$ 的正惯性指数等于 $A$ 的正特征值的个数,负惯性指数等于 $A$ 的负特征值的个数,$f$ 的秩等于 $A$ 的非零特征值的个数.
\end{theorem}
\begin{proof}
注意到正交相似既是相似又是合同,故由\reftheorem{theorem:实对称和Hermite矩阵的正交对角化}即得结论. 
\end{proof}

\begin{corollary}\label{corollary:二次型式(半)正定型(负定型)的充要条件}
设 $f(x) = x'Ax$ 是 $n$ 元实二次型,则 
\begin{enumerate}
\item $f$ 是正定型当且仅当系数矩阵 $A$ 的特征值全是正数,

\item $f$ 是负定型当且仅当 $A$ 的特征值全是负数,

\item $f$ 是半正定型当且仅当 $A$ 的特征值全非负,

\item $f$ 是半负定型当且仅当 $A$ 的特征值全非正.
\end{enumerate}
\end{corollary}

\vspace{0.5cm}

下面我们通过具体例子来说明对实对称阵 $A$,如何求正交矩阵 $P$,使 $P'AP$ 是对角阵. 我们的方法和 $\S 6.2$ 类似,只是增加了特征向量的标准正交化过程.
\begin{example}
求正交矩阵 $P$,使 $P'AP$ 为对角阵,其中
\[
A = \begin{pmatrix}
4 & 2 & 2 \\
2 & 4 & 2 \\
2 & 2 & 4
\end{pmatrix}.
\]
\end{example}
\begin{solution}
先求特征值
\[
|\lambda I - A| = 
\begin{vmatrix}
\lambda - 4 & -2 & -2 \\
-2 & \lambda - 4 & -2 \\
-2 & -2 & \lambda - 4
\end{vmatrix}
= (\lambda - 8)(\lambda - 2)^2.
\]
因此,$A$ 的特征值为 $\lambda_1 = 8$,$\lambda_2 = \lambda_3 = 2$. 当 $\lambda = 8$ 时,求解齐次线性方程组 $(\lambda I - A)x = 0$,得到基础解系 (只有一个向量):
\[
\eta_1 = (1,1,1)'.
\]
当 $\lambda = 2$ 时,求解齐次线性方程组 $(\lambda I - A)x = 0$,得到基础解系 (有两个向量):
\[
\eta_2 = (-1,1,0)', \quad \eta_3 = (-1,0,1)'.
\]
由于实对称阵属于不同特征值的特征向量必正交,因此只要对上面两个向量正交化,用 Gram - Schmidt 方法将 $\eta_2,\eta_3$ 正交化得到
\[
\xi_2 = (-1,1,0)', \quad \xi_3 = (-\frac{1}{2},-\frac{1}{2},1)'.
\]
再将 $\eta_1,\xi_2,\xi_3$ 化为单位向量得到
\[
v_1 = (\frac{1}{\sqrt{3}},\frac{1}{\sqrt{3}},\frac{1}{\sqrt{3}})', \quad 
v_2 = (-\frac{1}{\sqrt{2}},\frac{1}{\sqrt{2}},0)', \quad 
v_3 = (-\frac{1}{\sqrt{6}},-\frac{1}{\sqrt{6}},\frac{2}{\sqrt{6}})'.
\]
令
\[
P = (v_1,v_2,v_3) = 
\begin{pmatrix}
\frac{1}{\sqrt{3}} & -\frac{1}{\sqrt{2}} & -\frac{1}{\sqrt{6}} \\
\frac{1}{\sqrt{3}} & \frac{1}{\sqrt{2}} & -\frac{1}{\sqrt{6}} \\
\frac{1}{\sqrt{3}} & 0 & \frac{2}{\sqrt{6}}
\end{pmatrix},
\]
于是
\[
P'AP = 
\begin{pmatrix}
8 & 0 & 0 \\
0 & 2 & 0 \\
0 & 0 & 2
\end{pmatrix}.
\]
\end{solution}
\begin{remark}
上述方法也适用于 Hermite 矩阵,即对 Hermite 矩阵 $A$,求酉矩阵 $P$,使 $\overline{P}'AP$ 为对角阵.
\end{remark}

\begin{example}
设 $A$ 是三阶实对称阵,$A$ 的特征值为 $0,3,3$. 已知属于特征值 $0$ 的特征向量为 $v_1 = (1,1,1)'$,又向量 $v_2 = (-1,1,0)'$ 是属于特征值 $3$ 的特征向量,求矩阵 $A$.
\end{example}
\begin{solution}
设 $A$ 属于特征值 $3$ 的另一特征向量 $v_3 = (x_1,x_2,x_3)'$ 和 $v_1,v_2$ 都正交,则
\[
\begin{cases}
x_1 + x_2 + x_3 = 0, \\
-x_1 + x_2 = 0,
\end{cases}
\]
求出一个非零解 $v_3 = (1,1,-2)'$. 因为 $v_1,v_2,v_3$ 已经两两正交,故只需将 $v_1,v_2,v_3$ 标准化,得到
\[
\xi_1 = 
\begin{pmatrix}
\frac{1}{\sqrt{3}} \\
\frac{1}{\sqrt{3}} \\
\frac{1}{\sqrt{3}}
\end{pmatrix}, \quad 
\xi_2 = 
\begin{pmatrix}
-\frac{1}{\sqrt{2}} \\
\frac{1}{\sqrt{2}} \\
0
\end{pmatrix}, \quad 
\xi_3 = 
\begin{pmatrix}
\frac{1}{\sqrt{6}} \\
\frac{1}{\sqrt{6}} \\
-\frac{2}{\sqrt{6}}
\end{pmatrix}.
\]
令
\[
P = (\xi_1,\xi_2,\xi_3) = 
\begin{pmatrix}
\frac{1}{\sqrt{3}} & -\frac{1}{\sqrt{2}} & \frac{1}{\sqrt{6}} \\
\frac{1}{\sqrt{3}} & \frac{1}{\sqrt{2}} & \frac{1}{\sqrt{6}} \\
\frac{1}{\sqrt{3}} & 0 & -\frac{2}{\sqrt{6}}
\end{pmatrix}, \quad 
B = 
\begin{pmatrix}
0 & 0 & 0 \\
0 & 3 & 0 \\
0 & 0 & 3
\end{pmatrix},
\]
则
\[
A = PBP' = 
\begin{pmatrix}
2 & -1 & -1 \\
-1 & 2 & -1 \\
-1 & -1 & 2
\end{pmatrix}.
\]
\end{solution}

\begin{example}
设 $A$ 是 $n$ 阶实对称阵且 $A^3 = I_n$,证明: $A = I_n$.
\end{example}
\begin{solution}
设 $A$ 的特征值为 $\lambda$,则 $\lambda^3 = 1$. 因为 $\lambda$ 是实数,故 $\lambda = 1$,这就是说 $A$ 的特征值全是 $1$. 由\reftheorem{theorem:实对称和Hermite矩阵的正交对角化},存在正交矩阵 $P$,使 $P'AP = I_n$,于是 $A = PI_nP' = I_n$.
\end{solution}












\end{document}
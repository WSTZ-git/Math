\documentclass[../../main.tex]{subfiles}
\graphicspath{{\subfix{../../image/}}} % 指定图片目录,后续可以直接使用图片文件名。

% 例如:
% \begin{figure}[H]
% \centering
% \includegraphics[scale=0.4]{图.png}
% \caption{}
% \label{figure:图}
% \end{figure}
% 注意:上述\label{}一定要放在\caption{}之后,否则引用图片序号会只会显示??.

\begin{document}

\section{Schur定理}

回顾\hyperref[theorem:Schur(舒尔)定理]{Schur定理}.

\begin{proposition}\label{proposition:两个半正定阵可同时合同对角化6}
设\(A\)是\(n\)阶实矩阵,虚数\(a + b\mathrm{i}\)是\(A\)的一个特征值,\(\boldsymbol{u}+ \boldsymbol{v}\mathrm{i}\)是对应的特征向量,其中\(\boldsymbol{u}\),\(\boldsymbol{v}\)是实列向量. 求证:\(\boldsymbol{u}\),\(\boldsymbol{v}\)必线性无关. 若\(A\)是正规矩阵,则\(\boldsymbol{u}\),\(\boldsymbol{v}\)相互正交且长度相同(取实列向量空间的标准内积).
\end{proposition}
\begin{proof}
由假设
\begin{align}
A(\boldsymbol{u}+ \boldsymbol{v}\mathrm{i})=(a + b\mathrm{i})(\boldsymbol{u}+ \boldsymbol{v}\mathrm{i})=(a\boldsymbol{u}-b\boldsymbol{v})+(a\boldsymbol{v}+b\boldsymbol{u})\mathrm{i}.\label{eq:9.12}
\end{align}
假设\(\boldsymbol{u}\),\(\boldsymbol{v}\)线性相关,不妨设\(\boldsymbol{u}\neq\boldsymbol{0}\),\(\boldsymbol{v}=k\boldsymbol{u}\),则\((1 + k\mathrm{i})A\boldsymbol{u}=(1 + k\mathrm{i})(a + b\mathrm{i})\boldsymbol{u}\),于是\(A\boldsymbol{u}=(a + b\mathrm{i})\boldsymbol{u}\),由此可得\(A\boldsymbol{u}=a\boldsymbol{u}\),\(b\boldsymbol{u}=\boldsymbol{0}\),这与\(b\neq0\)且\(\boldsymbol{u}\neq\boldsymbol{0}\)相矛盾.

若\(A\)是正规矩阵,在\eqref{eq:9.12}式中比较实部和虚部得到
\begin{align*}
A\boldsymbol{u}=a\boldsymbol{u}-b\boldsymbol{v},\quad A\boldsymbol{v}=a\boldsymbol{v}+b\boldsymbol{u}.
\end{align*}
因为\(A\)正规,故由\refpro{proposition:正规算子的特征值与特征向量的相关性质-例9.28}可知,\(\boldsymbol{u}+ \boldsymbol{v}\mathrm{i}\)也是\(A'\)的属于特征值\(a - b\mathrm{i}\)的特征向量,即
\begin{align*}
A'(\boldsymbol{u}+ \boldsymbol{v}\mathrm{i})=(a - b\mathrm{i})(\boldsymbol{u}+ \boldsymbol{v}\mathrm{i})=(a\boldsymbol{u}+b\boldsymbol{v})+(a\boldsymbol{v}-b\boldsymbol{u})\mathrm{i}.
\end{align*}
比较实部和虚部得到
\begin{align*}
A'\boldsymbol{u}=a\boldsymbol{u}+b\boldsymbol{v},\quad A'\boldsymbol{v}=a\boldsymbol{v}-b\boldsymbol{u}.
\end{align*}
又\((A\boldsymbol{u},\boldsymbol{u})=(\boldsymbol{u},A'\boldsymbol{u})\),\((A\boldsymbol{u},\boldsymbol{v})=(\boldsymbol{u},A'\boldsymbol{v})\),将\(A\boldsymbol{u}\),\(A'\boldsymbol{u}\)及\(A'\boldsymbol{v}\)代入得到
\begin{align*}
(a\boldsymbol{u}-b\boldsymbol{v},\boldsymbol{u})=(\boldsymbol{u},a\boldsymbol{u}+b\boldsymbol{v}),\quad (a\boldsymbol{u}-b\boldsymbol{v},\boldsymbol{v})=(\boldsymbol{u},a\boldsymbol{v}-b\boldsymbol{u}).
\end{align*}
由此可得\((\boldsymbol{u},\boldsymbol{v}) = 0\),\((\boldsymbol{u},\boldsymbol{u})=(\boldsymbol{v},\boldsymbol{v})\).

\end{proof}

\begin{proposition}\label{proposition:两个半正定阵可同时合同对角化7}
证明:\(n\)阶实方阵\(A\)必正交相似于下列分块上三角矩阵:
\[
C = 
\begin{pmatrix}
A_1 & & & * \\
& \ddots & & \\
& & A_r & \\
& & & c_1 & \\
& & & & \ddots & \\
& & & & & c_k
\end{pmatrix}
\]
其中\(A_i(1\leqslant  i\leqslant  r)\)是二阶实矩阵且\(A_i\)的特征值具有\(a_i\pm b_i\mathrm{i}(b_i\neq0)\)的形状,\(c_j(1\leqslant  j\leqslant  k)\)是实数.
\end{proposition}
\begin{proof}
对阶数\(n\)进行归纳. 当\(n = 0\)时表示归纳过程已结束,当\(n = 1\)时结论显然成立. 现设对阶小于\(n\)的矩阵结论成立,下分两种情况对\(n\)阶矩阵\(A\)进行讨论.

首先,假设\(A\)有实特征值\(\lambda\). 因为\(A\)和\(A'\)有相同的特征值,故\(\lambda\)也是\(A'\)的特征值. 将\(A\)看成是\(n\)维实列向量空间\(\mathbb{R}^n\)(取标准内积)上的线性变换,显然\(A'\)是\(A\)的伴随. 设\(\boldsymbol{e}_n\)是\(A'\)的属于特征值\(\lambda\)的单位特征向量,则\(L(\boldsymbol{e}_n)^\perp\)是\(A\)的不变子空间. 将\(A\)限制在\(L(\boldsymbol{e}_n)^\perp\)上,由归纳假设,存在\(L(\boldsymbol{e}_n)^\perp\)的标准正交基\(\boldsymbol{e}_1,\cdots,\boldsymbol{e}_{n - 1}\),使得线性变换\(A\)在这组基下的表示矩阵为分块上三角矩阵. 于是在标准正交基\(\boldsymbol{e}_1,\boldsymbol{e}_2,\cdots,\boldsymbol{e}_n\)下,线性变换\(A\)的表示矩阵就是要求的矩阵\(C\). 因为线性变换\(A'\)在同一组标准正交基下的表示矩阵为\(C'\),故由\(A'\boldsymbol{e}_n=\lambda\boldsymbol{e}_n\)可知\(\lambda = c_k\).

其次,假设\(A\)没有实特征值,并设\(a + b\mathrm{i}\)是\(A\)的虚特征值. 因为\(A\)和\(A'\)有相同的特征值,故\(a + b\mathrm{i}\)也是\(A'\)的特征值. 假设\(A'\)的属于特征值\(a + b\mathrm{i}\)的特征向量为\(\boldsymbol{\alpha}+ \boldsymbol{\beta}\mathrm{i}\),其中\(\boldsymbol{\alpha}\),\(\boldsymbol{\beta}\)是实列向量,则有
\begin{align*}
A'(\boldsymbol{\alpha}+ \boldsymbol{\beta}\mathrm{i})=(a + b\mathrm{i})(\boldsymbol{\alpha}+ \boldsymbol{\beta}\mathrm{i}).
\end{align*}
比较实部和虚部得到
\begin{align*}
A'\boldsymbol{\alpha}=a\boldsymbol{\alpha}-b\boldsymbol{\beta},\quad A'\boldsymbol{\beta}=b\boldsymbol{\alpha}+a\boldsymbol{\beta}.
\end{align*}
由例9.86可知,\(\boldsymbol{\alpha}\),\(\boldsymbol{\beta}\)必线性无关. 设\(U = L(\boldsymbol{\alpha},\boldsymbol{\beta})\)为\(\mathbb{R}^n\)的子空间,则上式表明\(U\)是线性变换\(A'\)的不变子空间,于是\(U^\perp\)是\(A'\)的伴随\(A\)的不变子空间. 注意到\(\dim U^\perp=n - 2\),故由归纳假设,存在\(U^\perp\)的标准正交基\(\boldsymbol{e}_1,\cdots,\boldsymbol{e}_{n - 2}\),使得线性变换\(A\)在这组基下的表示矩阵为分块上三角矩阵:
\[
\begin{pmatrix}
A_1 & & * \\
& \ddots & \\
& & A_{r - 1}
\end{pmatrix}.
\]
在\(U\)中选取一组标准正交基\(\boldsymbol{e}_{n - 1},\boldsymbol{e}_n\),则在标准正交基\(\boldsymbol{e}_1,\boldsymbol{e}_2,\cdots,\boldsymbol{e}_n\)下,线性变换\(A\)的表示矩阵为:
\[
D = 
\begin{pmatrix}
A_1 & & & * \\
& \ddots & & \\
& & A_{r - 1} & \\
& & & A_r
\end{pmatrix}.
\]
由于线性变换\(A'\)在同一组标准正交基下的表示矩阵为\(D'\),故\(A_r\)是\(A'\)在\(U\)的标准正交基\(\boldsymbol{e}_{n - 1},\boldsymbol{e}_n\)下的表示矩阵. 又\(\begin{pmatrix}a&b\\-b&a\end{pmatrix}\)是\(A'\)在\(U\)的基\(\boldsymbol{\alpha},\boldsymbol{\beta}\)下的表示矩阵,于是\(A_r\)相似于\(\begin{pmatrix}a&b\\-b&a\end{pmatrix}\),从而它的特征值也为\(a\pm b\mathrm{i}\). 

\end{proof}

\begin{proposition}\label{proposition:两个半正定阵可同时合同对角化8}
设\(n\)阶实矩阵\(A\)的特征值全是实数,求证:\(A\)正交相似于上三角矩阵.
\end{proposition}
\begin{proof}
这是\refpro{proposition:两个半正定阵可同时合同对角化7}的直接推论. 另外,也可由\refpro{proposition:特征值全在同一数域的矩阵可上三角化}和\hyperref[theorem:施密特正交化对应的矩阵分解,QR分解]{矩阵QR分解}的实版本进行证明.

\end{proof}

\begin{proposition}\label{proposition:两个半正定阵可同时合同对角化9}
设\(A\),\(B\)是实方阵且分块矩阵\(\begin{pmatrix}A&C\\O&B\end{pmatrix}\)是实正规矩阵,求证:\(C = O\)且\(A\),\(B\)也是正规矩阵.
\end{proposition}
\begin{proof}
由已知
\begin{align*}
\begin{pmatrix}A&C\\O&B\end{pmatrix}\begin{pmatrix}A'&O\\C'&B'\end{pmatrix}=\begin{pmatrix}A'&O\\C'&B'\end{pmatrix}\begin{pmatrix}A&C\\O&B\end{pmatrix},
\end{align*}
从而\(AA'+CC' = A'A\). 由于\(\mathrm{tr}(AA'+CC')=\mathrm{tr}(A'A)=\mathrm{tr}(AA')\),故可得\(\mathrm{tr}(CC') = 0\),再由\(C\)是实矩阵可推出\(C = O\),于是\(AA' = A'A\),\(BB' = B'B\).

\end{proof}

\begin{proposition}\label{proposition:例9.90}
设\(A\)是\(n\)阶实正规矩阵,求证:存在正交矩阵\(P\),使得
\[
P'AP = \mathrm{diag}\{A_1,\cdots,A_r,c_{2r + 1},\cdots,c_n\},
\]
其中\(A_i=\begin{pmatrix}a_i&b_i\\-b_i&a_i\end{pmatrix}(1\leqslant  i\leqslant  r)\)是二阶实矩阵,\(c_j(2r + 1\leqslant  j\leqslant  n)\)是实数.
\end{proposition}
\begin{proof}
由\refpro{proposition:两个半正定阵可同时合同对角化7},\(A\)正交相似于\refpro{proposition:两个半正定阵可同时合同对角化7}中的分块上三角矩阵,再反复用\refpro{proposition:两个半正定阵可同时合同对角化9}的结论可知这是个分块对角矩阵. 又因为每一块都是正规矩阵,故或是二阶正规矩阵\(A_i\),或是实数\(c_j\)(一阶矩阵). 对于二阶正规矩阵的情形,由\refpro{proposition:两个半正定阵可同时合同对角化6}的证明过程可知,若设\(A_i\)的特征值为\(a_i + b_i\mathrm{i}\),对应的特征向量为\(\boldsymbol{u}+ \boldsymbol{v}\mathrm{i}\),令\(P_i = (\frac{\boldsymbol{u}}{\|\boldsymbol{u}\|},\frac{\boldsymbol{v}}{\|\boldsymbol{v}\|})\),则\(P_i\)为二阶正交矩阵,且\(P_i'A_iP_i=\begin{pmatrix}a_i&b_i\\-b_i&a_i\end{pmatrix}\). 

\end{proof}














\end{document}
% contents/chapter-03/section-03.tex 第三章第三节
\documentclass[../../main.tex]{subfiles}
\graphicspath{{\subfix{../../image/}}} % 指定图片目录,后续可以直接使用图片文件名。

% 例如:
% \begin{figure}[H]
% \centering
% \includegraphics[scale=0.4]{图.png}
% \caption{}
% \label{figure:图}
% \end{figure}
% 注意:上述\label{}一定要放在\caption{}之后,否则引用图片序号会只会显示??.

\begin{document}

\section{线性同构和几何问题代数化}

\hypertarget{线性同构1}{我们有一类特别重要的线性同构}. 设\(V\)是数域\(\mathbb{K}\)上的\(n\)维线性空间,\(\{\boldsymbol{e}_1,\boldsymbol{e}_2,\cdots,\boldsymbol{e}_n\}\)是\(V\)的一组基并固定次序. 对任一向量\(\boldsymbol{\alpha}\in V\),设\(\boldsymbol{\alpha}=\lambda_1\boldsymbol{e}_1+\lambda_2\boldsymbol{e}_2+\cdots+\lambda_n\boldsymbol{e}_n\),则映射\(\eta:V\to\mathbb{K}^n\)定义为:\(\eta(\boldsymbol{\alpha})=(\lambda_1,\lambda_2,\cdots,\lambda_n)'\),即\(\eta\)将\(V\)中的向量映射到它在给定基下的坐标向量. 容易验证\(\eta:V\to\mathbb{K}^n\)是一个线性同构. 因此,通过这个线性同构,我们可将抽象的线性空间\(V\)和具体的列向量空间\(\mathbb{K}^n\)等同起来.

\begin{theorem}\label{theorem:线性空间的同构}
\begin{enumerate}[(1)]
\item 同构关系是一种等价关系;
\item 线性同构不仅将线性相关的向量组映射为线性相关的向量组,而且将线性无关的向量组映射为线性无关的向量组;
\item 同一个数域\(\mathbb{F}\)上的线性空间同构的充要条件是它们具有相同的维数.
\end{enumerate}
\end{theorem}
\begin{proof}

\end{proof}

\begin{theorem}\label{theorem:线性空间的同构11}
定理 假设和记号同上,设\(\boldsymbol{\alpha}_1,\boldsymbol{\alpha}_2,\cdots,\boldsymbol{\alpha}_m,\boldsymbol{\beta}\)是\(V\)中向量,它们在给定基下的坐标向量记为\(\widetilde{\boldsymbol{\alpha}}_1,\widetilde{\boldsymbol{\alpha}}_2,\cdots,\widetilde{\boldsymbol{\alpha}}_m,\widetilde{\boldsymbol{\beta}}\),则
\begin{enumerate}[(1)]
\item \(\boldsymbol{\alpha}_1,\boldsymbol{\alpha}_2,\cdots,\boldsymbol{\alpha}_m\)线性无关的充要条件是\(\widetilde{\boldsymbol{\alpha}}_1,\widetilde{\boldsymbol{\alpha}}_2,\cdots,\widetilde{\boldsymbol{\alpha}}_m\)线性无关.

\item \(\boldsymbol{\beta}\)可以用\(\boldsymbol{\alpha}_1,\boldsymbol{\alpha}_2,\cdots,\boldsymbol{\alpha}_m\)线性表示的充要条件是\(\widetilde{\boldsymbol{\beta}}\)可以用\(\widetilde{\boldsymbol{\alpha}}_1,\widetilde{\boldsymbol{\alpha}}_2,\cdots,\widetilde{\boldsymbol{\alpha}}_m\)线性表示,
并且线性表示的系数不变.即\begin{align*}
\boldsymbol{\beta }=c_1\boldsymbol{\alpha }_1+c_2\boldsymbol{\alpha }_2+\cdots +c_m\boldsymbol{\alpha }_m\Leftrightarrow \widetilde{\boldsymbol{\beta }}=c_1\widetilde{\boldsymbol{\alpha }}_1+c_2\widetilde{\boldsymbol{\alpha }}_2+\cdots +c_m\widetilde{\boldsymbol{\alpha }}_m.
\end{align*}

\item \(\boldsymbol{\alpha}_{i_1},\boldsymbol{\alpha}_{i_2},\cdots,\boldsymbol{\alpha}_{i_r}\)是向量组\(\boldsymbol{\alpha}_1,\boldsymbol{\alpha}_2,\cdots,\boldsymbol{\alpha}_m\)的极大无关组的充要条件是\(\widetilde{\boldsymbol{\alpha}}_{i_1},\widetilde{\boldsymbol{\alpha}}_{i_2},\cdots,\widetilde{\boldsymbol{\alpha}}_{i_r}\)是向量组\(\widetilde{\boldsymbol{\alpha}}_1,\widetilde{\boldsymbol{\alpha}}_2,\cdots,\widetilde{\boldsymbol{\alpha}}_m\)的极大无关组. 特别地,我们有
\[
\mathrm{r}(\boldsymbol{\alpha}_1,\boldsymbol{\alpha}_2,\cdots,\boldsymbol{\alpha}_m)=\mathrm{r}(\widetilde{\boldsymbol{\alpha}}_1,\widetilde{\boldsymbol{\alpha}}_2,\cdots,\widetilde{\boldsymbol{\alpha}}_m).
\]
\end{enumerate}
\end{theorem}
\begin{note}
由上述定理,我们可以将抽象线性空间\(V\)中向量组线性关系的判定和秩的计算,转化为具体列向量空间\(\mathbb{K}^n\)中由它们的坐标向量构成的列向量组线性关系的判定和秩的计算. 由于后者通常可以通过矩阵的方法来处理,故上述过程被称为“几何问题代数化”.
\end{note}
\begin{proof}
将\hyperref[theorem:线性空间的同构]{定理\ref{theorem:线性空间的同构}}运用到\hyperlink{线性同构1}{线性同构$\eta$}上就能得到证明.
\end{proof}

\begin{proposition}\label{proposition:系数矩阵与向量组的秩}
设\(\boldsymbol{\alpha}_1,\boldsymbol{\alpha}_2,\cdots,\boldsymbol{\alpha}_k;\boldsymbol{\beta}_1,\boldsymbol{\beta}_2,\cdots,\boldsymbol{\beta}_m\)是向量空间\(V\)中的向量,且满足:
\[
\begin{cases}
\boldsymbol{\beta}_1 = c_{11}\boldsymbol{\alpha}_1 + c_{12}\boldsymbol{\alpha}_2+\cdots + c_{1k}\boldsymbol{\alpha}_k,\\
\boldsymbol{\beta}_2 = c_{21}\boldsymbol{\alpha}_1 + c_{22}\boldsymbol{\alpha}_2+\cdots + c_{2k}\boldsymbol{\alpha}_k,\\
\cdots\cdots\cdots\cdots\\
\boldsymbol{\beta}_m = c_{m1}\boldsymbol{\alpha}_1 + c_{m2}\boldsymbol{\alpha}_2+\cdots + c_{mk}\boldsymbol{\alpha}_k.
\end{cases}
\]
记上述表示式中的系数矩阵为\(\boldsymbol{C}=(c_{ij})_{m\times k}\),则
\begin{enumerate}[(1)]
\item 若\(\mathrm{r}(\boldsymbol{C}) = k\),则这两组向量等价.
\item 若\(\mathrm{r}(\boldsymbol{C}) = r\),则向量组\(\boldsymbol{\beta}_1,\boldsymbol{\beta}_2,\cdots,\boldsymbol{\beta}_m\)的秩不超过\(r\).
\end{enumerate}
\end{proposition}
\begin{proof}
\begin{enumerate}[(1)]
\item 在\(V\)中取定一组基\(\boldsymbol{e}_1,\boldsymbol{e}_2,\cdots,\boldsymbol{e}_n\),假设在这组基下\(\boldsymbol{\alpha}_i\)的坐标向量是\(\widetilde{\boldsymbol{\alpha}}_i(1\leqslant  i\leqslant  k)\),\(\boldsymbol{\beta}_j\)的坐标向量是\(\widetilde{\boldsymbol{\beta}}_j(1\leqslant  j\leqslant  m)\),则
\[
\begin{cases}
\widetilde{\boldsymbol{\beta}}_1 = c_{11}\widetilde{\boldsymbol{\alpha}}_1 + c_{12}\widetilde{\boldsymbol{\alpha}}_2+\cdots + c_{1k}\widetilde{\boldsymbol{\alpha}}_k,\\
\widetilde{\boldsymbol{\beta}}_2 = c_{21}\widetilde{\boldsymbol{\alpha}}_1 + c_{22}\widetilde{\boldsymbol{\alpha}}_2+\cdots + c_{2k}\widetilde{\boldsymbol{\alpha}}_k,\\
\cdots\cdots\cdots\cdots\\
\widetilde{\boldsymbol{\beta}}_m = c_{m1}\widetilde{\boldsymbol{\alpha}}_1 + c_{m2}\widetilde{\boldsymbol{\alpha}}_2+\cdots + c_{mk}\widetilde{\boldsymbol{\alpha}}_k,
\end{cases}
\]
写成矩阵形式为
\[
(\widetilde{\boldsymbol{\beta}}_1,\widetilde{\boldsymbol{\beta}}_2,\cdots,\widetilde{\boldsymbol{\beta}}_m)=(\widetilde{\boldsymbol{\alpha}}_1,\widetilde{\boldsymbol{\alpha}}_2,\cdots,\widetilde{\boldsymbol{\alpha}}_k)\boldsymbol{C}'.
\]
因为\(\boldsymbol{C}'\)是一个行满秩\(k\times m\)矩阵,故由\hyperref[proposition:行/列满秩矩阵性质]{行满秩矩阵性质}可知,存在\(m\times k\)矩阵\(\boldsymbol{T}\),使得\(\boldsymbol{C}'\boldsymbol{T}=\boldsymbol{I}_k\),于是
\[
(\widetilde{\boldsymbol{\beta}}_1,\widetilde{\boldsymbol{\beta}}_2,\cdots,\widetilde{\boldsymbol{\beta}}_m)\boldsymbol{T}=(\widetilde{\boldsymbol{\alpha}}_1,\widetilde{\boldsymbol{\alpha}}_2,\cdots,\widetilde{\boldsymbol{\alpha}}_k).
\]
这表明\(\boldsymbol{\alpha}_1,\boldsymbol{\alpha}_2,\cdots,\boldsymbol{\alpha}_k\)可用\(\boldsymbol{\beta}_1,\boldsymbol{\beta}_2,\cdots,\boldsymbol{\beta}_m\)来线性表示,于是这两组向量等价.

\item 在\(V\)中取定一组基\(\boldsymbol{e}_1,\boldsymbol{e}_2,\cdots,\boldsymbol{e}_n\),假设在这组基下\(\boldsymbol{\alpha}_i\)的坐标向量是\(\widetilde{\boldsymbol{\alpha}}_i(1\leqslant  i\leqslant  k)\),\(\boldsymbol{\beta}_j\)的坐标向量是\(\widetilde{\boldsymbol{\beta}}_j(1\leqslant  j\leqslant  m)\),则
\[
\begin{cases}
\widetilde{\boldsymbol{\beta}}_1 = c_{11}\widetilde{\boldsymbol{\alpha}}_1 + c_{12}\widetilde{\boldsymbol{\alpha}}_2+\cdots + c_{1k}\widetilde{\boldsymbol{\alpha}}_k,\\
\widetilde{\boldsymbol{\beta}}_2 = c_{21}\widetilde{\boldsymbol{\alpha}}_1 + c_{22}\widetilde{\boldsymbol{\alpha}}_2+\cdots + c_{2k}\widetilde{\boldsymbol{\alpha}}_k,\\
\cdots\cdots\cdots\cdots\\
\widetilde{\boldsymbol{\beta}}_m = c_{m1}\widetilde{\boldsymbol{\alpha}}_1 + c_{m2}\widetilde{\boldsymbol{\alpha}}_2+\cdots + c_{mk}\widetilde{\boldsymbol{\alpha}}_k,
\end{cases}
\]
写成矩阵形式为
\[
(\widetilde{\boldsymbol{\beta}}_1,\widetilde{\boldsymbol{\beta}}_2,\cdots,\widetilde{\boldsymbol{\beta}}_m)=(\widetilde{\boldsymbol{\alpha}}_1,\widetilde{\boldsymbol{\alpha}}_2,\cdots,\widetilde{\boldsymbol{\alpha}}_k)\boldsymbol{C}'.
\]
由于两个矩阵乘积的秩不超过每个矩阵的秩,因此
\begin{align*}
\mathrm{r}\left( \boldsymbol{\beta }_1,\boldsymbol{\beta }_2,\cdots ,\boldsymbol{\beta }_m \right) =\mathrm{r}\left( \widetilde{\boldsymbol{\beta }}_1,\widetilde{\boldsymbol{\beta }}_2,\cdots ,\widetilde{\boldsymbol{\beta }}_m \right) =\mathrm{r}\left( \left( \widetilde{\boldsymbol{\alpha }}_1,\widetilde{\boldsymbol{\alpha }}_2,\cdots ,\widetilde{\boldsymbol{\alpha }}_k \right) \boldsymbol{C}' \right) \leqslant \mathrm{r}\left( \boldsymbol{C}'\right) =\mathrm{r}\left( \boldsymbol{C} \right) =r.
\end{align*}
\end{enumerate}
\end{proof}

\begin{proposition}\label{proposition:向量方程的解空间}
设\(\boldsymbol{\alpha}_1,\boldsymbol{\alpha}_2,\cdots,\boldsymbol{\alpha}_m\)是数域\(\mathbb{F}\)上\(n\)维线性空间\(V\)中的\(m\)个向量,且已知它们的秩等于\(r\). 求证: 全体满足\(x_1\boldsymbol{\alpha}_1 + x_2\boldsymbol{\alpha}_2+\cdots + x_m\boldsymbol{\alpha}_m = \boldsymbol{0}\)的列向量\((x_1,x_2,\cdots,x_m)'(x_i\in\mathbb{F})\)构成数域\(\mathbb{F}\)上\(m\)维列向量空间\(\mathbb{F}^m\)的\(m - r\)维子空间.
\end{proposition}
\begin{proof}
在\(V\)中引进基以后,记\(\widetilde{\boldsymbol{\alpha}}_i\)是\(\boldsymbol{\alpha}_i\)的坐标向量,则\(x_1\boldsymbol{\alpha}_1 + x_2\boldsymbol{\alpha}_2+\cdots + x_m\boldsymbol{\alpha}_m = \boldsymbol{0}\)等价于\(x_1\widetilde{\boldsymbol{\alpha}}_1 + x_2\widetilde{\boldsymbol{\alpha}}_2+\cdots + x_m\widetilde{\boldsymbol{\alpha}}_m = \boldsymbol{0}\). 而后者是一个齐次线性方程组,其系数矩阵的秩等于\(r\)(将\(x_i\)视为未知数),故其解构成\(\mathbb{F}^m\)的\(m - r\)维子空间.
\end{proof}


\begin{example}
设\(\{\boldsymbol{e}_1,\boldsymbol{e}_2,\cdots,\boldsymbol{e}_n\}\)是线性空间\(V\)的一组基,问:\(\{\boldsymbol{e}_1,\boldsymbol{e}_1+\boldsymbol{e}_2,\cdots,\boldsymbol{e}_1+\boldsymbol{e}_2+\cdots+\boldsymbol{e}_n\}\)是否也是\(V\)的基?
\end{example}
\begin{note}
利用\hyperref[theorem:线性空间的同构11]{定理\ref{theorem:线性空间的同构11}}即可.
\end{note}
\begin{solution}
将\(\{\boldsymbol{e}_1,\boldsymbol{e}_1+\boldsymbol{e}_2,\cdots,\boldsymbol{e}_1+\boldsymbol{e}_2+\cdots+\boldsymbol{e}_n\}\)对应的坐标向量拼成如下矩阵:
\[
\boldsymbol{A}=\begin{pmatrix}
1&1&\cdots&1\\
0&1&\cdots&1\\
\vdots&\vdots&&\vdots\\
0&0&\cdots&1
\end{pmatrix}.
\]
显然\(|\boldsymbol{A}| = 1\),从而\(\boldsymbol{A}\)是满秩阵,于是\(\{\boldsymbol{e}_1,\boldsymbol{e}_1+\boldsymbol{e}_2,\cdots,\boldsymbol{e}_1+\boldsymbol{e}_2+\cdots+\boldsymbol{e}_n\}\)也是\(V\)的一组基. 
\end{solution}

\begin{example}
已知向量组\(\{\boldsymbol{\alpha}_1,\boldsymbol{\alpha}_2,\cdots,\boldsymbol{\alpha}_s\}(s > 1)\)是线性空间\(V\)的一组基,设\(\boldsymbol{\beta}_1=\boldsymbol{\alpha}_1+\boldsymbol{\alpha}_2,\boldsymbol{\beta}_2=\boldsymbol{\alpha}_2+\boldsymbol{\alpha}_3,\cdots,\boldsymbol{\beta}_s=\boldsymbol{\alpha}_s+\boldsymbol{\alpha}_1\). 讨论向量\(\boldsymbol{\beta}_1,\boldsymbol{\beta}_2,\cdots,\boldsymbol{\beta}_s\)的线性相关性.
\end{example}
\begin{note}
利用\hyperref[theorem:线性空间的同构11]{定理\ref{theorem:线性空间的同构11}}即可.
\end{note}
\begin{solution}
将\(\boldsymbol{\beta}_1,\boldsymbol{\beta}_2,\cdots,\boldsymbol{\beta}_s\)对应的坐标向量拼成如下矩阵:
\[
\boldsymbol{A}=\begin{pmatrix}
1&0&\cdots&1\\
1&1&\cdots&0\\
0&1&\cdots&0\\
\vdots&\vdots&&\vdots\\
0&0&\cdots&1
\end{pmatrix}.
\]
经计算可得\(|\boldsymbol{A}| = 1+(-1)^{s + 1}\). 因此当\(s\)为偶数时,\(|\boldsymbol{A}| = 0\),从而向量\(\boldsymbol{\beta}_1,\boldsymbol{\beta}_2,\cdots,\boldsymbol{\beta}_s\)线性相关;当\(s\)为奇数时,\(|\boldsymbol{A}| = 2\),从而向量\(\boldsymbol{\beta}_1,\boldsymbol{\beta}_2,\cdots,\boldsymbol{\beta}_s\)线性无关. 
\end{solution}

\begin{example}
设\(\{\boldsymbol{e}_1,\boldsymbol{e}_2,\boldsymbol{e}_3,\boldsymbol{e}_4\}\)是线性空间\(V\)的一组基,已知
\[
\begin{cases}
\boldsymbol{\alpha}_1=\boldsymbol{e}_1+\boldsymbol{e}_2+\boldsymbol{e}_3 + 3\boldsymbol{e}_4,\\
\boldsymbol{\alpha}_2=-\boldsymbol{e}_1-3\boldsymbol{e}_2 + 5\boldsymbol{e}_3+\boldsymbol{e}_4,\\
\boldsymbol{\alpha}_3=3\boldsymbol{e}_1+2\boldsymbol{e}_2-\boldsymbol{e}_3 + 4\boldsymbol{e}_4,\\
\boldsymbol{\alpha}_4=-2\boldsymbol{e}_1-6\boldsymbol{e}_2 + 10\boldsymbol{e}_3+2\boldsymbol{e}_4,
\end{cases}
\]
求\(\boldsymbol{\alpha}_1,\boldsymbol{\alpha}_2,\boldsymbol{\alpha}_3,\boldsymbol{\alpha}_4\)的一个极大无关组.
\end{example}
\begin{note}
利用\hyperref[theorem:线性空间的同构11]{定理\ref{theorem:线性空间的同构11}}即可.
\end{note}
\begin{solution}
将\(\boldsymbol{\alpha}_1,\boldsymbol{\alpha}_2,\boldsymbol{\alpha}_3,\boldsymbol{\alpha}_4\)对应的坐标向量拼成如下矩阵,并用初等行变换将其化为阶梯形矩阵:
\[
\boldsymbol{A}=\begin{pmatrix}
1&-1&3&-2\\
1&-3&2&-6\\
1&5&-1&10\\
3&1&4&2
\end{pmatrix}\to\begin{pmatrix}
1&-1&3&-2\\
0&-2&-1&-4\\
0&0&-7&0\\
0&0&0&0
\end{pmatrix}.
\]
因此,矩阵\(\boldsymbol{A}\)的第一列、第二列和第三列是坐标向量组的极大无关组,从而\(\boldsymbol{\alpha}_1,\boldsymbol{\alpha}_2,\boldsymbol{\alpha}_3\)是\(\boldsymbol{\alpha}_1,\boldsymbol{\alpha}_2,\boldsymbol{\alpha}_3,\boldsymbol{\alpha}_4\)的一个极大无关组. 
\end{solution}

\begin{example}\label{example:3.110.1}
设\(a_1,a_2,\cdots,a_n\)是\(n\)个不同的数,\(\{\boldsymbol{e}_1,\boldsymbol{e}_2,\cdots,\boldsymbol{e}_n\}\)是线性空间\(V\)的一组基,已知
\[
\begin{cases}
\boldsymbol{\alpha}_1=\boldsymbol{e}_1 + a_1\boldsymbol{e}_2+\cdots + a_1^{n - 1}\boldsymbol{e}_n,\\
\boldsymbol{\alpha}_2=\boldsymbol{e}_1 + a_2\boldsymbol{e}_2+\cdots + a_2^{n - 1}\boldsymbol{e}_n,\\
\cdots\cdots\cdots\cdots\\
\boldsymbol{\alpha}_n=\boldsymbol{e}_1 + a_n\boldsymbol{e}_2+\cdots + a_n^{n - 1}\boldsymbol{e}_n,
\end{cases}
\]
求证:\(\{\boldsymbol{\alpha}_1,\boldsymbol{\alpha}_2,\cdots,\boldsymbol{\alpha}_n\}\)也是\(V\)的一组基.
\end{example}
\begin{note}
利用\hyperref[theorem:线性空间的同构11]{定理\ref{theorem:线性空间的同构11}}即可.
\end{note}
\begin{proof}
将\(\boldsymbol{\alpha}_1,\boldsymbol{\alpha}_2,\cdots,\boldsymbol{\alpha}_n\)对应的坐标向量拼成如下矩阵:
\[
\boldsymbol{A}=\begin{pmatrix}
1&1&\cdots&1\\
a_1&a_2&\cdots&a_n\\
\vdots&\vdots&&\vdots\\
a_1^{n - 1}&a_2^{n - 1}&\cdots&a_n^{n - 1}
\end{pmatrix}.
\]
显然,\(|\boldsymbol{A}|=\prod_{1\leqslant  i<j\leqslant  n}(a_j - a_i)\neq0\),故\(\boldsymbol{A}\)是满秩阵,从而\(\{\boldsymbol{\alpha}_1,\boldsymbol{\alpha}_2,\cdots,\boldsymbol{\alpha}_n\}\)也是\(V\)的一组基. 
\end{proof}


\end{document}
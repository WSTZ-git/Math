% contents/chapter-03/section-05.tex 第三章第三节
\documentclass[../../main.tex]{subfiles}
\graphicspath{{\subfix{../../image/}}} % 指定图片目录,后续可以直接使用图片文件名。

% 例如:
% \begin{figure}[H]
% \centering
% \includegraphics[scale=0.4]{图.png}
% \caption{}
% \label{figure:图}
% \end{figure}
% 注意:上述\label{}一定要放在\caption{}之后,否则引用图片序号会只会显示??.

\begin{document}

\section{子空间、直和与商空间}

\begin{theorem}[基扩张定理]\label{theorem:基扩张定理}
设 $V$ 是 $n$ 维线性空间,$v_1, v_2, \cdots, v_m$ 是 $V$ 中 $m (m < n)$ 个线性无关的向量,又假设 $\{e_1, e_2, \cdots, e_n\}$ 是 $V$ 的一组基,则必可在 $\{e_1, e_2, \cdots, e_n\}$ 中选出 $n - m$ 个向量,使之和 $v_1, v_2, \cdots, v_m$ 一起组成 $V$ 的一组基。

基扩张定理还有几种等价形式:

(1)$n$ 维线性空间 $V$ 中任意 $m (m < n)$ 个线性无关的向量均可扩张为 $V$ 的一组基.

(2)$n$ 维线性空间 $V$的任意一个子空间的基均可扩张为 $V$ 的一组基。
\end{theorem}
\begin{proof}
将 $e_i (i = 1, \cdots, n)$ 依次放入 $\{v_1, v_2, \cdots, v_m\}$,则必有一个 $e_{i'}$,使 $v_1, v_2, \cdots, v_m, e_{i'}$ 线性无关。这是因为若任一 $e_i$ 加入 $v_1, v_2, \cdots, v_m$ 后线性相关,则每个 $e_i$ 可用 $v_1, v_2, \cdots, v_m$ 线性表示,将和\hyperref[theorem:向量的线性关系定理2]{定理\ref{theorem:向量的线性关系定理2}(1)}的结论矛盾。现不妨设 $i' = m + 1$。若 $m + 1 < n$,又可从 $e_1, e_2, \cdots, e_n$ 中找到一个向量,加入 $\{v_1, v_2, \cdots, v_m, e_{m + 1}\}$ 后仍线性无关。不断这样做下去,便可将 $v_1, v_2, \cdots, v_m$ 扩张成为 $V$ 的一组基。
\end{proof}

\begin{definition}[直和]\label{definition:直和}
设\(V_1,V_2,\cdots,V_k\)是线性空间\(V\)的子空间,若对任意的\(i(1\leqslant  i\leqslant  k)\),均有
\[
V_i\cap(V_1+\cdots+V_{i - 1}+V_{i + 1}+\cdots+V_k)=0,
\]
则称和\(V_1 + V_2+\cdots+V_k\)是直接和,简称直和,记为\(V_1\oplus V_2\oplus\cdots\oplus V_k\).
\end{definition}

\begin{theorem}[直和的等价条件]\label{theorem:直和的等价条件}
设\(V_1,V_2,\cdots,V_k\)是线性空间\(V_0\)的子空间,\(V_0 = V_1 + V_2+\cdots+V_k\),则下列命题等价:
\begin{enumerate}[(1)]
\item \label{theorem:直和的等价条件1}\(V_0 = V_1\oplus V_2\oplus\cdots\oplus V_k\);
\item \label{theorem:直和的等价条件2}对任意的\(2\leqslant  i\leqslant  k\),有\(V_i\cap(V_1 + V_2+\cdots+V_{i - 1}) = 0\);
\item \label{theorem:直和的等价条件3}\(\dim V_0=\dim V_1+\dim V_2+\cdots+\dim V_k\);
\item \label{theorem:直和的等价条件4}\(V_1,V_2,\cdots,V_k\)的一组基可以拼成\(V_0\)的一组基;
\item \label{theorem:直和的等价条件5}\(V_0\)中的向量表示为\(V_1,V_2,\cdots,V_k\)中的向量之和时其表示唯一.
\item \label{theorem:直和的等价条件6}在$ V_1 + V_2+\cdots+V_k$中零向量的表示唯一,即如果
\begin{align*}
v_1+v_2+\cdots+v_k=0,v_i\in V_i,i=1,2,\cdots,k.
\end{align*}
则$v_i=0,i=1,2,\cdots,k.$
\end{enumerate}
\end{theorem}
\begin{proof}

\end{proof}

\begin{theorem}[交和空间维数公式]\label{theorem:交和空间维数公式}
设\(V_1,V_2\)是线性空间\(V\)的两个子空间,则
\[
\dim(V_1 + V_2)=\dim V_1+\dim V_2-\dim(V_1\cap V_2).
\]
\end{theorem}
\begin{proof}
设 $\dim V_1 = n_1$, $\dim V_2 = n_2$, $\dim(V_1 \cap V_2) = m$. 取 $V_1 \cap V_2$ 的一组基 $\{\boldsymbol{\alpha}_1, \cdots, \boldsymbol{\alpha}_m\}$, 由于 $V_1 \cap V_2$ 是 $V_1$ 的子空间, 故可添上 $V_1$ 中的向量 $\boldsymbol{\alpha}_{m + 1}, \cdots, \boldsymbol{\alpha}_{n_1}$, 使 $\{\boldsymbol{\alpha}_1, \cdots, \boldsymbol{\alpha}_m, \boldsymbol{\alpha}_{m + 1}, \cdots, \boldsymbol{\alpha}_{n_1}\}$ 是 $V_1$ 的一组基. 同样道理, 可添上 $\boldsymbol{\beta}_{m + 1}, \cdots, \boldsymbol{\beta}_{n_2}$, 使 $\{\boldsymbol{\alpha}_1, \cdots, \boldsymbol{\alpha}_m, \boldsymbol{\beta}_{m + 1}, \cdots, \boldsymbol{\beta}_{n_2}\}$ 成为 $V_2$ 的一组基. 显然, $V_1 + V_2$ 中的向量均可由向量组
$$
\boldsymbol{\alpha}_1, \cdots, \boldsymbol{\alpha}_m, \boldsymbol{\alpha}_{m + 1}, \cdots, \boldsymbol{\alpha}_{n_1}, \boldsymbol{\beta}_{m + 1}, \cdots, \boldsymbol{\beta}_{n_2}
$$
的线性组合给出. 如能证明上式中的向量线性无关, 则它们构成 $V_1 + V_2$ 的一组基, 由此即可推出所要的结论. 现假设
$$
\lambda_1\boldsymbol{\alpha}_1 + \cdots + \lambda_m\boldsymbol{\alpha}_m + \lambda_{m + 1}\boldsymbol{\alpha}_{m + 1} + \cdots + \lambda_{n_1}\boldsymbol{\alpha}_{n_1} + \mu_{m + 1}\boldsymbol{\beta}_{m + 1} + \cdots + \mu_{n_2}\boldsymbol{\beta}_{n_2} = \boldsymbol{0},
$$
则
$$
\lambda_1\boldsymbol{\alpha}_1 + \cdots + \lambda_m\boldsymbol{\alpha}_m + \lambda_{m + 1}\boldsymbol{\alpha}_{m + 1} + \cdots + \lambda_{n_1}\boldsymbol{\alpha}_{n_1} = -(\mu_{m + 1}\boldsymbol{\beta}_{m + 1} + \cdots + \mu_{n_2}\boldsymbol{\beta}_{n_2}).
$$
上式左端属于 $V_1$, 右端属于 $V_2$, 故
$$
\mu_{m + 1}\boldsymbol{\beta}_{m + 1} + \cdots + \mu_{n_2}\boldsymbol{\beta}_{n_2} \in V_1 \cap V_2,
$$
即存在 $\xi_1, \cdots, \xi_m \in \mathbb{K}$, 使
$$
\mu_{m + 1}\boldsymbol{\beta}_{m + 1} + \cdots + \mu_{n_2}\boldsymbol{\beta}_{n_2} = \xi_1\boldsymbol{\alpha}_1 + \cdots + \xi_m\boldsymbol{\alpha}_m.
$$
但 $\boldsymbol{\alpha}_1, \cdots, \boldsymbol{\alpha}_m, \boldsymbol{\beta}_{m + 1}, \cdots, \boldsymbol{\beta}_{n_2}$ 是 $V_2$ 的基, 因此 $\mu_{m + 1} = \cdots = \mu_{n_2} = \xi_1 = \cdots = \xi_m = 0$. 再由 $\boldsymbol{\alpha}_1, \cdots, \boldsymbol{\alpha}_m, \boldsymbol{\alpha}_{m + 1}, \cdots, \boldsymbol{\alpha}_{n_1}$ 线性无关得 $\lambda_1 = \cdots = \lambda_m = \lambda_{m + 1} = \cdots = \lambda_{n_1} = 0$. 
\end{proof}

\begin{corollary}\label{corollary:空间分解关于基的结论}
设\(V_1,V_2\)是线性空间\(V\)的两个子空间,$\dim V_1 = n_1$, $\dim V_2 = n_2$, $\dim(V_1 \cap V_2) = m$,取$V_1 \cap V_2$ 的一组基$$\{\boldsymbol{\alpha}_1, \cdots, \boldsymbol{\alpha}_m\},$$将其扩张为$V_1$ 的一组基$$\{\boldsymbol{\alpha}_1, \cdots, \boldsymbol{\alpha}_m, \boldsymbol{\alpha}_{m + 1}, \cdots, \boldsymbol{\alpha}_{n_1}\},$$
再将其扩张为$V_2$ 的一组基$$\{\boldsymbol{\alpha}_1, \cdots, \boldsymbol{\alpha}_m, \boldsymbol{\beta}_{m + 1}, \cdots, \boldsymbol{\beta}_{n_2}\}.$$
则
$$
\{\boldsymbol{\alpha}_1, \cdots, \boldsymbol{\alpha}_m, \boldsymbol{\alpha}_{m + 1}, \cdots, \boldsymbol{\alpha}_{n_1}, \boldsymbol{\beta}_{m + 1}, \cdots, \boldsymbol{\beta}_{n_2}\}
$$
就是$V_1+V_2$的一组基.
\end{corollary}
\begin{proof}
显然, $V_1 + V_2$ 中的向量均可由向量组
$$
\boldsymbol{\alpha}_1, \cdots, \boldsymbol{\alpha}_m, \boldsymbol{\alpha}_{m + 1}, \cdots, \boldsymbol{\alpha}_{n_1}, \boldsymbol{\beta}_{m + 1}, \cdots, \boldsymbol{\beta}_{n_2}
$$
的线性组合给出.现假设
$$
\lambda_1\boldsymbol{\alpha}_1 + \cdots + \lambda_m\boldsymbol{\alpha}_m + \lambda_{m + 1}\boldsymbol{\alpha}_{m + 1} + \cdots + \lambda_{n_1}\boldsymbol{\alpha}_{n_1} + \mu_{m + 1}\boldsymbol{\beta}_{m + 1} + \cdots + \mu_{n_2}\boldsymbol{\beta}_{n_2} = \boldsymbol{0},
$$
则
$$
\lambda_1\boldsymbol{\alpha}_1 + \cdots + \lambda_m\boldsymbol{\alpha}_m + \lambda_{m + 1}\boldsymbol{\alpha}_{m + 1} + \cdots + \lambda_{n_1}\boldsymbol{\alpha}_{n_1} = -(\mu_{m + 1}\boldsymbol{\beta}_{m + 1} + \cdots + \mu_{n_2}\boldsymbol{\beta}_{n_2}).
$$
上式左端属于 $V_1$, 右端属于 $V_2$, 故
$$
\mu_{m + 1}\boldsymbol{\beta}_{m + 1} + \cdots + \mu_{n_2}\boldsymbol{\beta}_{n_2} \in V_1 \cap V_2,
$$
即存在 $\xi_1, \cdots, \xi_m \in \mathbb{K}$, 使
$$
\mu_{m + 1}\boldsymbol{\beta}_{m + 1} + \cdots + \mu_{n_2}\boldsymbol{\beta}_{n_2} = \xi_1\boldsymbol{\alpha}_1 + \cdots + \xi_m\boldsymbol{\alpha}_m.
$$
但 $\boldsymbol{\alpha}_1, \cdots, \boldsymbol{\alpha}_m, \boldsymbol{\beta}_{m + 1}, \cdots, \boldsymbol{\beta}_{n_2}$ 是 $V_2$ 的基, 因此 $\mu_{m + 1} = \cdots = \mu_{n_2} = \xi_1 = \cdots = \xi_m = 0$. 再由 $\boldsymbol{\alpha}_1, \cdots, \boldsymbol{\alpha}_m, \boldsymbol{\alpha}_{m + 1}, \cdots, \boldsymbol{\alpha}_{n_1}$ 线性无关得 $\lambda_1 = \cdots = \lambda_m = \lambda_{m + 1} = \cdots = \lambda_{n_1} = 0$. 
\end{proof}

\subsection{证明直和的方法}
证明直和的方法大致有两种:

第一种:先证和,再证直和. 

第二种:对于给定的\(V,V_1,V_2\),求证\(V = V_1\oplus V_2\)的题目,如果“和”不好证明的话,可以记\(W = V_1 + V_2\),先证\(W = V_1\oplus V_2\),再证\(V = W\)(证明\(V = W\)通常会利用\hyperref[proposition:与全空间维数相同的子空间等于全空间]{命题\ref{proposition:与全空间维数相同的子空间等于全空间}}).具体例子见\hyperref[example:561.16]{例题\ref{example:561.16}}

\begin{proposition}\label{proposition:矩阵空间可以分解为对称和反称矩阵空间的直和}
设\(V\)是数域\(\mathbb{F}\)上\(n\)阶矩阵组成的向量空间,\(V_1\)和\(V_2\)分别是\(\mathbb{F}\)上对称矩阵和反对称矩阵组成的子集. 求证:\(V_1\)和\(V_2\)都是\(V\)的子空间且\(V = V_1\oplus V_2\).
\end{proposition}
\begin{note}
要证明向量空间\(V\)是其子空间\(V_1,V_2\)的直和,只需证明两件事:一是证明\(V\)中任一向量均可表示为\(V_1\)与\(V_2\)中向量之和,即\(V = V_1 + V_2\);二是证明\(V_1\)与\(V_2\)的交等于零.
\end{note}
\begin{proof}
由于对称矩阵之和仍是对称矩阵,一个数乘以对称矩阵仍是对称矩阵,因此\(V_1\)是\(V\)的子空间. 同理\(V_2\)也是\(V\)的子空间. 又由\hyperref[proposition:任一阶方阵可表示为对称阵与反对称阵之和]{命题\ref{proposition:任一阶方阵可表示为对称阵与反对称阵之和}}可知,任一\(n\)阶矩阵都可以表示为一个对称矩阵和一个反对称矩阵之和,故\(V = V_1 + V_2\). 若一个矩阵既是对称矩阵又是反对称矩阵,则它一定是零矩阵. 这就是说\(V_1\cap V_2 =\mathbf{0}\). 于是\(V = V_1\oplus V_2\). 
\end{proof}

\begin{proposition}\label{proposition:直和分解的每个子空间的和还是直和}
设$V_1,V_2,\cdots,V_n$是数域$\mathbb{F}$上的$n$个线性空间,且$V_1\oplus V_2\oplus \cdots \oplus V_n$.若
\begin{align*}
W_1\subseteq V_1,W_2\subseteq V_2,\cdots ,W_n\subseteq V_n,
\end{align*}
则
\begin{align*}
W_1\oplus W_2\oplus \cdots \oplus W_n.
\end{align*}
\end{proposition}
\begin{proof}
因为$V_1\oplus V_2\oplus \cdots \oplus V_n$,所以由\hyperref[theorem:直和的等价条件]{直和的等价条件(6)}知,如果
\begin{align*}
v_1+v_2+\cdots +v_n=0,\ v_i\in V_i,\ i=1,2,\cdots,n,
\end{align*}
则$v_i=0$,$i=1,2,\cdots,n$.设
\begin{align*}
w_1+w_2+\cdots +w_n=0,\ w_i\in W_i,\ i=1,2,\cdots,n.
\end{align*}
则$w_i\in W_i\subseteq V_i$,$i=1,2,\cdots,n$.从而$w_i=0$,$i=1,2,\cdots,n$.故$W_1+W_2+\cdots +W_n$中零向量的表示唯一,因此由\hyperref[theorem:直和的等价条件]{直和的等价条件(6)}知
\begin{align*}
W_1\oplus W_2\oplus \cdots \oplus W_n.
\end{align*}
\end{proof}

\begin{example}
设\(V_1,V_2\)分别是数域\(\mathbb{F}\)上的齐次线性方程组\(x_1 = x_2=\cdots = x_n\)与\(x_1 + x_2+\cdots + x_n = 0\)的解空间,求证:\(\mathbb{F}^n = V_1\oplus V_2\).
\end{example}
\begin{note}
要证明向量空间\(V\)是其子空间\(V_1,V_2\)的直和,只需证明两件事:一是证明\(V\)中任一向量均可表示为\(V_1\)与\(V_2\)中向量之和,即\(V = V_1 + V_2\);二是证明\(V_1\)与\(V_2\)的交等于零.
\end{note}
\begin{proof}
由线性方程组解的定理知,\(V_1\)的维数是\(1\),\(V_2\)的维数是\(n - 1\). 若列向量\(\boldsymbol{\alpha}\in V_1\cap V_2\),则\(\boldsymbol{\alpha}\)既是第一个线性方程组的解,也是第二个线性方程组的解,不难看出\(\boldsymbol{\alpha}\)只能等于零向量,因此\(V_1\cap V_2 = 0\). 又因为
\[
\dim(V_1\oplus V_2)=\dim V_1+\dim V_2=1+(n - 1)=n=\dim\mathbb{F}^n,
\]
故\(\mathbb{F}^n = V_1\oplus V_2\).
\end{proof}

\begin{example}
设\(U,V\)是数域\(\mathbb{K}\)上的两个线性空间,\(W = U\times V\)是\(U\)和\(V\)的积集合,即\(W=\{(\boldsymbol{u},\boldsymbol{v})|\boldsymbol{u}\in U,\boldsymbol{v}\in V\}\). 现在\(W\)上定义加法和数乘:
\[
(\boldsymbol{u}_1,\boldsymbol{v}_1)+(\boldsymbol{u}_2,\boldsymbol{v}_2)=(\boldsymbol{u}_1+\boldsymbol{u}_2,\boldsymbol{v}_1+\boldsymbol{v}_2),k(\boldsymbol{u},\boldsymbol{v})=(k\boldsymbol{u},k\boldsymbol{v}).
\]
验证:\(W\)是\(\mathbb{K}\)上的线性空间(这个线性空间称为\(U\)和\(V\)的外直和).

又若设\(U'=\{(\boldsymbol{u},\boldsymbol{0})|\boldsymbol{u}\in U\},V'=\{(\boldsymbol{0},\boldsymbol{v})|\boldsymbol{v}\in V\}\),求证:\(U',V'\)是\(W\)的子空间,\(U'\)和\(U\)同构,\(V'\)和\(V\)同构,并且\(W = U'\oplus V'\).
\end{example}
\begin{proof}
易验证\(W\)在上述加法和数乘下满足线性空间的8条公理,从而是\(\mathbb{K}\)上的线性空间. 任取\((\boldsymbol{u}_1,\boldsymbol{0}),(\boldsymbol{u}_2,\boldsymbol{0})\in U',k\in\mathbb{K}\),则\((\boldsymbol{u}_1,\boldsymbol{0})+(\boldsymbol{u}_2,\boldsymbol{0})=(\boldsymbol{u}_1+\boldsymbol{u}_2,\boldsymbol{0})\in U',k(\boldsymbol{u}_1,\boldsymbol{0})=(k\boldsymbol{u}_1,\boldsymbol{0})\in U'\),因此\(U'\)是\(W\)的子空间. 同理可证\(V'\)是\(W\)的子空间. 构造映射\(\varphi:U\to U',\varphi(\boldsymbol{u})=(\boldsymbol{u},\boldsymbol{0})\),容易验证\(\varphi\)是一一对应并且保持加法和数乘运算,所以\(\varphi:U\to U'\)是一个线性同构. 构造映射\(\psi:V\to V',\psi(\boldsymbol{v})=(\boldsymbol{0},\boldsymbol{v})\),同理可证\(\psi:V\to V'\)是一个线性同构. 显然\(U'\cap V' = 0\),又对\(W\)中任一向量\((\boldsymbol{u},\boldsymbol{v})\),有\((\boldsymbol{u},\boldsymbol{v})=(\boldsymbol{u},\boldsymbol{0})+(\boldsymbol{0},\boldsymbol{v})\in U'+V'\),因此\(W = U'\oplus V'\). 
\end{proof}

\begin{example}\label{example:561.16}
给定数域\(P\),设\(\boldsymbol{A}\)是数域\(P\)上的一个\(n\)级可逆方阵,\(\boldsymbol{A}\)的前\(r\)个行向量组成的矩阵为\(\boldsymbol{B}\),后\(n - r\)个行向量组成的矩阵为\(\boldsymbol{C}\),\(n\)元线性方程组\(\boldsymbol{B}\boldsymbol{X}=0\)与\(\boldsymbol{C}\boldsymbol{X}=0\)的解空间分别为\(V_1,V_2\),证\(P^n = V_1\oplus V_2\).
\end{example}
\begin{proof}
先记\(W = V_1 + V_2\). 若\(\boldsymbol{\alpha}\in V_1\cap V_2\),则\(\boldsymbol{B}\boldsymbol{\alpha}=\boldsymbol{C}\boldsymbol{\alpha}=0\),所以
\[
\boldsymbol{A}\boldsymbol{\alpha}=\begin{pmatrix}
\boldsymbol{B}\\
\boldsymbol{C}
\end{pmatrix}\boldsymbol{\alpha}=0.
\]
由于\(\boldsymbol{A}\)可逆,知\(\boldsymbol{\alpha}=0\),所以\(V_1\cap V_1 = \{0\}\),即\(W = V_1\oplus V_2\).

最后说\(W = P^n\):显然\(r(\boldsymbol{B}) = r\),\(r(\boldsymbol{C}) = n - r\),则\(\dim V_1 = n - r\),\(\dim V_2 = n-(n - r)=r\). 所以
\[
\dim W=\dim V_1+\dim V_2=n=\dim P^n.
\]
又\(W = V_1\oplus V_2\subseteq P^n\),从而\(W = P^n\),即
\[
P^n = V_1\oplus V_2.
\]  
\end{proof}


\begin{proposition}[任意子空间一定存在相应的补空间]\label{proposition:补空间}
设\(U\)是\(V\)的子空间,则一定存在\(V\)的子空间\(W\),使得\(V = U\oplus W\). 这样的子空间\(W\)称为子空间\(U\)在\(V\)中的\textbf{补空间}.
\end{proposition}
\begin{remark}
在这个命题中\(U\cap W = \{ \boldsymbol{0}\}\),而不是\(U\cap W=\varnothing\);同时\(V = U + W\)是子空间的和,而不是\(V = U\cup W\). 因此,补空间绝不是补集,请读者务必注意!一般来说,补空间并不唯一. 例如下面证明中,取$U$中不同的基,再将基扩张得到的补空间也不相同.
还例如,若\(\dim V-\dim U\geqslant 1\)且\(\dim U\geqslant 1\),则\(U\)有无限个补空间.
\end{remark}
\begin{proof}
取子空间\(U\)的一组基\(\{\boldsymbol{e}_1,\cdots,\boldsymbol{e}_m\}\),由\hyperref[theorem:基扩充定理]{基扩张定理}可将其扩张为\(V\)的一组基\(\{\boldsymbol{e}_1,\cdots,\boldsymbol{e}_m,\boldsymbol{e}_{m + 1},\cdots,\boldsymbol{e}_n\}\). 令\(W = L(\boldsymbol{e}_{m + 1},\cdots,\boldsymbol{e}_n)\),则\(V = U+W\). 由于\(\{\boldsymbol{e}_{m + 1},\cdots,\boldsymbol{e}_n\}\)是\(W\)的一组基,故\(\dim V=\dim U+\dim W\),从而\(V = U\oplus W\). 
\end{proof}

\begin{proposition}\label{proposition:直和的传递性}
若\(V = U\oplus W\)且\(U = U_1\oplus U_2\),求证:\(V = U_1\oplus U_2\oplus W\). 
\end{proposition}
\begin{proof}
由\(U = U_1\oplus U_2\)可得\(U_1\cap U_2 = 0\);由\(V = U\oplus W\)可得\((U_1 + U_2)\cap W=U\cap W = 0\),因此由\hyperref[theorem:直和的等价条件2]{定理\ref{theorem:直和的等价条件}\ref{theorem:直和的等价条件2}}可得\(U_1 + U_2+W\)是直和,从而\(V = U_1 + U_2+W = U_1\oplus U_2\oplus W\).
\end{proof}

\begin{proposition}\label{proposition:n维线性空间的一维直和分解}
每一个\(n\)维线性空间均可表示为\(n\)个一维子空间的直和.
\end{proposition}
\begin{proof}
设\(V\)是\(n\)维线性空间,取其一组基为\(\{\boldsymbol{e}_1,\boldsymbol{e}_2,\cdots,\boldsymbol{e}_n\}\). 设\(V_i = L(\boldsymbol{e}_i)(1\leqslant  i\leqslant  n)\),则\(V_i\)是\(V\)的一维子空间.任取$\alpha\in V$,存在唯一一组常数$k_1,k_2,\cdots,k_n$,使得$\alpha =k_1\boldsymbol{e}_1+k_2\boldsymbol{e}_2+\cdots +k_n\boldsymbol{e}_n$,而$k_i\boldsymbol{e}_i\in V_i,i=1,2,\cdots ,n.$因此\(V = V_1 + V_2+\cdots+V_n\). 注意到\(\dim V = n=\dim V_1+\dim V_2+\cdots+\dim V_n\),故由\hyperref[theorem:直和的等价条件3]{定理\ref{theorem:直和的等价条件}\ref{theorem:直和的等价条件3}}可知,\(V = V_1\oplus V_2\oplus\cdots\oplus V_n\). 

(注意到\(V_i\)的基是\(\{\boldsymbol{e}_i\}\),因此\(V_i(1\leqslant  i\leqslant  n)\)的基能拼成\(V\)的基,故由\hyperref[theorem:直和的等价条件4]{定理\ref{theorem:直和的等价条件}\ref{theorem:直和的等价条件4}}也可得到结论. 再注意到\(V\)中任一向量写成基向量\(\{\boldsymbol{e}_1,\boldsymbol{e}_2,\cdots,\boldsymbol{e}_n\}\)的线性组合时,其表示是唯一的. 这就是说,\(V\)中任一向量写成\(V_i\)中的向量之和时,其表示是唯一的,故由\hyperref[theorem:直和的等价条件5]{定理\ref{theorem:直和的等价条件}\ref{theorem:直和的等价条件5}}同样可得结论. )
\end{proof}

\begin{proposition}\label{proposition:真子空间至多包含n-1个基向量}
设\(V_0\)是数域\(\mathbb{F}\)上$n$维向量空间\(V\)的真子空间,则\(V_0\)至多包含$n-1$个$V$中的基向量.
\end{proposition}
\begin{proof}
反证法,若$V_0$包含\(n\)个\(V\)中的基向量,则$V_i$就包含了$V$的一组基.不妨设$V_0$中的这组基向量为$\{e_1,e_2,\cdots e_n\}$,则$\forall \alpha\in V$,有$\alpha =k_1e_1+k_2e_2+\cdots +k_ne_n \in V_0$,其中$k_i\in \mathbb{F}$,$i=1,2,\cdots,n$.故$V_0\supset V$,又$V_0\subset V$,因此$V_0=V$.这与\(V_0\)是\(V\)的真子空间矛盾.
\end{proof}

\begin{proposition}\label{proposition:真子空间外仍有向量存在}
设\(V_1,V_2,\cdots,V_m\)是数域\(\mathbb{F}\)上向量空间\(V\)的\(m\)个真子空间,证明:在\(V\)中必存在一个向量\(\boldsymbol{\alpha}\),它不属于任何一个\(V_i\).
\end{proposition}
\begin{note}
这个命题表明:\textbf{有限个真子空间不能覆盖全空间}.
\end{note}
\begin{proof}
{\color{blue}证法一:}
对个数\(m\)进行归纳,当\(m = 1\)时结论显然成立. 设\(m = k\)时结论成立,现要证明\(m = k + 1\)时结论也成立. 由归纳假设,存在向量\(\boldsymbol{\alpha}\),它不属于任何一个\(V_i(1\leqslant  i\leqslant  k)\). 若\(\boldsymbol{\alpha}\)也不属于\(V_{k + 1}\),则结论已成立,因此可设\(\boldsymbol{\alpha}\in V_{k + 1}\). 在\(V_{k + 1}\)外选一个向量\(\boldsymbol{\beta}\),作集合
\begin{align*}
M = \{t\boldsymbol{\alpha}+\boldsymbol{\beta}|t\in\mathbb{F}\}.
\end{align*}
事实上,我们可将\(M\)看成是通过\(\boldsymbol{\beta}\)的终点且平行于\(\boldsymbol{\alpha}\)的一根“直线”,现要证明它和每个\(V_i\)最多只有一个交点. 首先,\(M\)和\(V_{k + 1}\)无交点,因为若\(t\boldsymbol{\alpha}+\boldsymbol{\beta}\in V_{k + 1}\),则从\(t\boldsymbol{\alpha}\in V_{k + 1}\)可推出\(\boldsymbol{\beta}\in V_{k + 1}\),与假设矛盾. 又若对某个\(V_i(i<k + 1)\),存在\(t_1\neq t_2\),使得\(t_1\boldsymbol{\alpha}+\boldsymbol{\beta}\in V_i\),\(t_2\boldsymbol{\alpha}+\boldsymbol{\beta}\in V_i\),则\((t_1 - t_2)\boldsymbol{\alpha}\in V_i\),从而导致\(\boldsymbol{\alpha}\in V_i\),与假设矛盾. 因此,\(M\)和每个\(V_i\)最多只有一个交点,从而\(M\)中只有有限个向量属于\(V_i\)的并集,而\(t\)有无穷多个选择,由此即得结论. 

{\color{blue}证法二:}任取\(V\)的一组基\(\{\boldsymbol{e}_1,\boldsymbol{e}_2,\cdots,\boldsymbol{e}_n\}\). 对任意的正整数\(k\),构造\(V\)中向量\(\boldsymbol{\alpha}_k=\boldsymbol{e}_1 + k\boldsymbol{e}_2+\cdots + k^{n - 1}\boldsymbol{e}_n\),设向量族\(S = \{\boldsymbol{\alpha}_k|k = 1,2,\cdots\}\). 由\hyperref[example:3.110.1]{例题\ref{example:3.110.1}}可知,\(S\)中任意\(n\)个不同的向量都构成\(V\)的一组基. 因为\(V_i\)都是\(V\)的真子空间,所以每个\(V_i\)至多包含\(S\)中\(n - 1\)个向量.因此$\bigcup_{i=1}^m{V_i}$至多包含$S$中$m(n-1)$个向量.
又由于\(S\)是无限集合,故存在某个向量\(\boldsymbol{\alpha}_k\),使得\(\boldsymbol{\alpha}_k\)不属于任何一个\(V_i\).
\end{proof}
\begin{remark}
上述证明要用到任意一个数域都有无穷个元素这一事实. 因此,对于有限域(读者以后可能会学到)上的向量空间,上例结论不一定成立.
\end{remark}

\begin{proposition}\label{proposition:真子空间外仍有一组基存在}
设\(V_1,V_2,\cdots,V_m\)是数域\(\mathbb{F}\)上向量空间\(V\)的\(m\)个真子空间,证明:\(V\)中必有一组基,使得每个基向量都不在诸\(V_i\)的并中.
\end{proposition}
\begin{proof}
{\color{blue}证法一:}
由\hyperref[proposition:真子空间外仍有向量存在]{命题\ref{proposition:真子空间外仍有向量存在}}可知,存在非零向量\(\boldsymbol{e}_1\in V\),使得\(\boldsymbol{e}_1\notin\bigcup_{i = 1}^{m}V_i\). 定义\(V_{m + 1}=L(\boldsymbol{e}_1)\),再由\hyperref[proposition:真子空间外仍有向量存在]{命题\ref{proposition:真子空间外仍有向量存在}}可知,存在向量\(\boldsymbol{e}_2\in V\),使得\(\boldsymbol{e}_2\notin\bigcup_{i = 1}^{m + 1}V_i\). 由\hyperref[corollary:线性无关向量组的命题1]{推论\ref{corollary:线性无关向量组的命题1}}可知,\(\boldsymbol{e}_2\notin L(\boldsymbol{e}_1)\)意味着\(\boldsymbol{e}_1,\boldsymbol{e}_2\)线性无关. 重新定义\(V_{m + 1}=L(\boldsymbol{e}_1,\boldsymbol{e}_2)\),再由\hyperref[proposition:真子空间外仍有向量存在]{命题\ref{proposition:真子空间外仍有向量存在}}可知,存在向量\(\boldsymbol{e}_3\in V\),使得\(\boldsymbol{e}_3\notin\bigcup_{i = 1}^{m + 1}V_i\). 再由\hyperref[corollary:线性无关向量组的命题1]{推论\ref{corollary:线性无关向量组的命题1}}可知,\(\boldsymbol{e}_3\notin L(\boldsymbol{e}_1,\boldsymbol{e}_2)\)意味着\(\boldsymbol{e}_1,\boldsymbol{e}_2,\boldsymbol{e}_3\)线性无关. 不断重复上述讨论,即添加线性无关的向量重新定义\(V_{m + 1}\),并反复利用\hyperref[proposition:真子空间外仍有向量存在]{命题\ref{proposition:真子空间外仍有向量存在}}和\hyperref[corollary:线性无关向量组的命题1]{推论\ref{corollary:线性无关向量组的命题1}}的结论,最后可以得到\(n\)个线性无关的向量\(\boldsymbol{e}_1,\boldsymbol{e}_2,\cdots,\boldsymbol{e}_n\),它们构成\(V\)的一组基,且满足\(\boldsymbol{e}_j\notin\bigcup_{i = 1}^{m}V_i(1\leqslant  j\leqslant  n)\). 

{\color{blue}证法二:}任取\(V\)的一组基\(\{\boldsymbol{e}_1,\boldsymbol{e}_2,\cdots,\boldsymbol{e}_n\}\). 对任意的正整数\(k\),构造\(V\)中向量\(\boldsymbol{\alpha}_k=\boldsymbol{e}_1 + k\boldsymbol{e}_2+\cdots + k^{n - 1}\boldsymbol{e}_n\),设向量族\(S = \{\boldsymbol{\alpha}_k|k = 1,2,\cdots\}\). 由\hyperref[example:3.110.1]{例题\ref{example:3.110.1}}可知,\(S\)中任意\(n\)个不同的向量都构成\(V\)的一组基. 因为\(V_i\)都是\(V\)的真子空间,所以每个\(V_i\)至多包含\(S\)中\(n - 1\)个向量.因此$\bigcup_{i=1}^m{V_i}$至多包含$S$中$m(n-1)$个向量.
又由于\(S\)是无限集合,故存在某个向量\(\boldsymbol{\alpha}_k\),使得\(\boldsymbol{\alpha}_k\)不属于任何一个\(V_i\).
进一步,在\(S\)中一定还存在\(n\)个不同的向量\(\boldsymbol{\alpha}_{k_1},\boldsymbol{\alpha}_{k_2},\cdots,\boldsymbol{\alpha}_{k_n}\),使得每个\(\boldsymbol{\alpha}_{k_j}\)都不属于任何一个\(V_i\),此时\(\{\boldsymbol{\alpha}_{k_1},\boldsymbol{\alpha}_{k_2},\cdots,\boldsymbol{\alpha}_{k_n}\}\)就构成了\(V\)的一组基.
\end{proof}

\begin{definition}[$U-$陪集与商空间]\label{definition:U-陪集与商空间}
设\(V\)是数域\(\mathbb{K}\)上的线性空间,\(U\)是\(V\)的子空间. 对任意的\(\boldsymbol{v}\in V\),集合\(\boldsymbol{v}+U:=\{\boldsymbol{v}+\boldsymbol{u}|\boldsymbol{u}\in U\}\)称为\(\boldsymbol{v}\)的\textbf{\(U -\)陪集}. 在所有\(U -\)陪集构成的集合\(S = \{\boldsymbol{v}+U|\boldsymbol{v}\in V\}\)中,定义加法和数乘如下,其中\(\boldsymbol{v}_1,\boldsymbol{v}_2\in V\),\(k\in\mathbb{K}\):
\[
(\boldsymbol{v}_1 + U)+(\boldsymbol{v}_2 + U):=(\boldsymbol{v}_1+\boldsymbol{v}_2)+U,\ k\cdot(\boldsymbol{v}_1 + U):=k\cdot\boldsymbol{v}_1+U.
\]

\(S\)在上述加法和数乘下成为数域\(\mathbb{K}\)上的线性空间,称为\(V\)关于子空间\(U\)的\textbf{商空间},记为\(V/U\).
\end{definition}
\begin{note}
容易验证\(S\)在上述加法和数乘下满足线性空间的8条公理,因此商空间是良定义的.故任意$V$的子空间$U$都存在相应的商空间.
\end{note}
\begin{remark}
商空间的向量是$U-$陪集.商空间的零向量就是$\boldsymbol{0} + U=U$.
\end{remark}

\begin{proposition}[$U-$陪集的性质]\label{proposition:U-陪集的性质}
\begin{enumerate}[(1)]
\item \(U -\)陪集之间的关系是:作为集合或者相等,或者不相交;
\item  \(\boldsymbol{v}_1 + U=\boldsymbol{v}_2 + U\)(作为集合相等)当且仅当\(\boldsymbol{v}_1-\boldsymbol{v}_2\in U\). 特别地,\(\boldsymbol{v}+U\)是\(V\)的子空间当且仅当\(\boldsymbol{v}\in U\);
\item  \(S\)中的加法以及\(\mathbb{K}\)关于\(S\)的数乘不依赖于代表元的选取,即若\(\boldsymbol{v}_1 + U=\boldsymbol{v}_1'+U\)以及\(\boldsymbol{v}_2 + U=\boldsymbol{v}_2'+U\),则\((\boldsymbol{v}_1 + U)+(\boldsymbol{v}_2 + U)=(\boldsymbol{v}_1'+U)+(\boldsymbol{v}_2'+U)\),以及\(k\cdot(\boldsymbol{v}_1 + U)=k\cdot(\boldsymbol{v}_1'+U)\);
\end{enumerate}
\end{proposition}
\begin{proof}
\begin{enumerate}[(1)]
\item 设\((\boldsymbol{v}_1 + U)\cap(\boldsymbol{v}_2 + U)\neq\varnothing\),即存在\(\boldsymbol{u}_1,\boldsymbol{u}_2\in U\),使得\(\boldsymbol{v}_1+\boldsymbol{u}_1=\boldsymbol{v}_2+\boldsymbol{u}_2\),从而\(\boldsymbol{v}_1-\boldsymbol{v}_2=\boldsymbol{u}_2-\boldsymbol{u}_1\in U\),于是
\[
\boldsymbol{v}_1 + U=\boldsymbol{v}_2+(\boldsymbol{v}_1 - \boldsymbol{v}_2)+U\subseteq\boldsymbol{v}_2 + U,\ \boldsymbol{v}_2 + U=\boldsymbol{v}_1+(\boldsymbol{v}_2 - \boldsymbol{v}_1)+U\subseteq\boldsymbol{v}_1 + U,
\]
因此\(\boldsymbol{v}_1 + U=\boldsymbol{v}_2 + U\).
\item 由(1)的证明过程即得. 特别地,\(\boldsymbol{v}+U\)是\(V\)的子空间\(\Rightarrow\mathbf{0}\in \boldsymbol{v}+U\Rightarrow \text{存在}\boldsymbol{u}\in U,\text{使得}\mathbf{0}=\boldsymbol{v}+\boldsymbol{u}\Rightarrow \boldsymbol{v}=-\boldsymbol{u}\in U \).

若$\boldsymbol{v}\in U$,则一方面,$\forall \boldsymbol{\alpha }\in \boldsymbol{v}+U$,存在$\boldsymbol{u}'\in U$,使得$\boldsymbol{\alpha }=\boldsymbol{v}+\boldsymbol{u}'$.又$\boldsymbol{v}\in U$,因此$\boldsymbol{\alpha }=\boldsymbol{v}+\boldsymbol{u}'\in U$.故$\boldsymbol{v}+U\subset U$.
另一方面,$\forall \boldsymbol{\beta }\in U$,有$\boldsymbol{\beta }=\boldsymbol{v}+\boldsymbol{\beta }-\boldsymbol{v}$.又由$\boldsymbol{v}\in U$可知$\boldsymbol{\beta }-\boldsymbol{v}\in U$,于是$\boldsymbol{\beta }=\boldsymbol{v}+\boldsymbol{\beta }-\boldsymbol{v}\in \boldsymbol{v}+U$.故$\boldsymbol{v}+U\supset U$.因此$\boldsymbol{v}+U = U$是\(V\)的子空间.

(实际上,若$\boldsymbol{v}\in U$,则因为$\boldsymbol{v}\in U\text{并且}\boldsymbol{v}\in \boldsymbol{v}+U$,所以$\boldsymbol{v}+U\cap U\ne \varnothing$.故由(1)可知$\boldsymbol{v}+U=U$是\(V\)的子空间.这样也能得到证明.)

\item 若\(\boldsymbol{v}_1 + U=\boldsymbol{v}_1'+U\)以及\(\boldsymbol{v}_2 + U=\boldsymbol{v}_2'+U\),则\hyperlink{remark1陪集}{存在\(\boldsymbol{u}_1,\boldsymbol{u}_2\in U\),使得\(\boldsymbol{v}_1-\boldsymbol{v}_1'=\boldsymbol{u}_1\),\(\boldsymbol{v}_2-\boldsymbol{v}_2'=\boldsymbol{u}_2\)},从而\((\boldsymbol{v}_1+\boldsymbol{v}_2)-(\boldsymbol{v}_1'+\boldsymbol{v}_2')=\boldsymbol{u}_1+\boldsymbol{u}_2\in U\),\(k\cdot\boldsymbol{v}_1 - k\cdot\boldsymbol{v}_1'=k\cdot\boldsymbol{u}_1\in U\),于是由(2)可得
\[
(\boldsymbol{v}_1 + U)+(\boldsymbol{v}_2 + U)=(\boldsymbol{v}_1+\boldsymbol{v}_2)+U=(\boldsymbol{v}_1'+\boldsymbol{v}_2')+U=(\boldsymbol{v}_1'+U)+(\boldsymbol{v}_2'+U),
\]
\[
k\cdot(\boldsymbol{v}_1 + U)=k\cdot\boldsymbol{v}_1+U=k\cdot\boldsymbol{v}_1'+U=k\cdot(\boldsymbol{v}_1'+U).
\]
\end{enumerate}
\end{proof}
\begin{remark}
\hypertarget{remark1陪集}{若}\(\boldsymbol{v}_1 + U=\boldsymbol{v}_1'+U\)以及\(\boldsymbol{v}_2 + U=\boldsymbol{v}_2'+U\),则$\forall \boldsymbol{u}_{1}^{\prime}\in U$,有$\boldsymbol{v}_1+\boldsymbol{u}_{1}^{\prime}\in \boldsymbol{v}_1 + U = \boldsymbol{v}_1' + U$.从而存在$\boldsymbol{u}_{1}''\in U$,使得$\boldsymbol{v}_1+\boldsymbol{u}_{1}^{\prime}=\boldsymbol{v}_1'+\boldsymbol{u}_{1}''$.于是$\boldsymbol{v}_1 - \boldsymbol{v}_1'=\boldsymbol{u}_{1}''-\boldsymbol{u}_{1}^{\prime}$.再令$\boldsymbol{u}_1=\boldsymbol{u}_{1}''-\boldsymbol{u}_{1}^{\prime}$,则$\boldsymbol{v}_1 - \boldsymbol{v}_1'=\boldsymbol{u}_1\in U$.同理可得,存在$u_2\in U$,使得$\boldsymbol{v}_2-\boldsymbol{v}_{2}^{\prime}=\boldsymbol{u}_2\in U$.
\end{remark}


\begin{proposition}[商空间的维数公式和商空间与补空间同构]\label{proposition:商空间的维数公式和商空间与补空间同构}
设\(V\)是数域\(\mathbb{K}\)上的\(n\)维线性空间,\(U\)是\(V\)的子空间,\(W\)是\(U\)的补空间,证明:\(\dim V/U=\dim V-\dim U\),并且存在线性同构\(\varphi:W\to V/U\).
\end{proposition}
\begin{proof}
取子空间\(U\)的一组基\(\{\boldsymbol{e}_1,\cdots,\boldsymbol{e}_m\}\),补空间\(W\)的一组基\(\{\boldsymbol{e}_{m + 1},\cdots,\boldsymbol{e}_n\}\),则\(\{\boldsymbol{e}_1,\cdots,\boldsymbol{e}_m,\boldsymbol{e}_{m + 1},\cdots,\boldsymbol{e}_n\}\)是\(V\)的一组基. 我们断言\(\{\boldsymbol{e}_{m + 1}+U,\cdots,\boldsymbol{e}_n+U\}\)是商空间\(V/U\)的一组基. 一方面,对任意的\(\boldsymbol{v}\in V\),设\(\boldsymbol{v}=\sum_{i = 1}^{n}a_i\boldsymbol{e}_i\),则
\[
\boldsymbol{v}+U=\left(\sum_{i = 1}^{n}a_i\boldsymbol{e}_i\right)+U=\left(\sum_{i = m + 1}^{n}a_i\boldsymbol{e}_i\right)+U=\sum_{i = m + 1}^{n}a_i(\boldsymbol{e}_i+U).
\]
另一方面,设\(a_{m + 1},\cdots,a_n\in\mathbb{K}\),使得\(\sum_{i = m + 1}^{n}a_i(\boldsymbol{e}_i+U)=\boldsymbol{0}+U\),即\(\left(\sum_{i = m + 1}^{n}a_i\boldsymbol{e}_i\right)+U = U\),从而\(\sum_{i = m + 1}^{n}a_i\boldsymbol{e}_i\in U\). 于是存在\(a_1,\cdots,a_m\in\mathbb{K}\),使得\(\sum_{i = m + 1}^{n}a_i\boldsymbol{e}_i=-\sum_{i = 1}^{m}a_i\boldsymbol{e}_i\),即\(\sum_{i = 1}^{n}a_i\boldsymbol{e}_i=\boldsymbol{0}\),从而\(a_i = 0(1\leqslant  i\leqslant  n)\).于是\(\{\boldsymbol{e}_{m + 1}+U,\cdots,\boldsymbol{e}_n+U\}\)线性无关. 因此,\(\dim V/U=n - m=\dim V-\dim U\).

对任意的\(\boldsymbol{w}\in W\),设\(\boldsymbol{w}=\sum_{i = m + 1}^{n}a_i\boldsymbol{e}_i\),定义映射\(\varphi:W\to V/U\)为
\[
\varphi(\boldsymbol{w})=\boldsymbol{w}+U=\sum_{i = m + 1}^{n}a_i(\boldsymbol{e}_i+U).
\]
容易验证\(\varphi\)保持加法和数乘,并且是一一对应($W$的基$e_i$映射过去得到$\varphi(e_i)$仍是$V/U$的基,$i=m+1,\cdots,n$.),从而是线性同构. 
\end{proof}



\subsection{练习}

\begin{exercise}\label{exercise:矩阵乘法可交换的子空间C(A)}
设\(V = M_n(\mathbb{K})\)是数域\(\mathbb{K}\)上的\(n\)阶矩阵全体组成的线性空间,\(\boldsymbol{A}\in V\),求证:与\(\boldsymbol{A}\)乘法可交换的矩阵全体\(C(\boldsymbol{A})\)组成\(V\)的子空间且其维数不为零. 又若\(T\)是\(V\)的非空子集,求证:与\(T\)中任一矩阵乘法可交换的矩阵全体\(C(T)\)也构成\(V\)的子空间且其维数不为零.
\end{exercise}
\begin{proof}
由于纯量阵\(c\boldsymbol{I}_n\)与任一\(n\)阶矩阵\(\boldsymbol{A}\)乘法可交换,故\(L(\boldsymbol{I}_n)\subseteq C(\boldsymbol{A})\). 任取\(\boldsymbol{B},\boldsymbol{C}\in C(\boldsymbol{A})\),\(k\in\mathbb{K}\),容易验证\(\boldsymbol{B}+\boldsymbol{C}\in C(\boldsymbol{A})\),\(k\boldsymbol{B}\in C(\boldsymbol{A})\),故\(C(\boldsymbol{A})\)是\(M_n(\mathbb{K})\)的子空间且其维数不为零. \(C(T)\)的结论同理可证.
\end{proof}

\begin{exercise}
设\(\boldsymbol{\alpha}_1=(1,0, - 1,0),\boldsymbol{\alpha}_2=(0,1,2,1),\boldsymbol{\alpha}_3=(2,1,0,1)\)是四维实行向量空间\(V\)中的向量,它们生成的子空间为\(V_1\),又向量\(\boldsymbol{\beta}_1=(-1,1,1,1),\boldsymbol{\beta}_2=(1,-1,-3,-1),\boldsymbol{\beta}_3=(-1,1,-1,1)\)生成的子空间为\(V_2\),求子空间\(V_1 + V_2\)和\(V_1\cap V_2\)的基.
\end{exercise}
\begin{solution}
{\color{blue}解法一:}\(V_1 + V_2\)是由\(\boldsymbol{\alpha}_i\)和\(\boldsymbol{\beta}_i\)生成的,因此只要求出这6个向量的极大无关组即可. 将这6个向量按列分块方式拼成矩阵,并用初等行变换将其化为阶梯形矩阵:
\[
\begin{pmatrix}
1&0&2&-1&1&-1\\
0&1&1&1&-1&1\\
-1&2&0&1&-3&-1\\
0&1&1&1&-1&1
\end{pmatrix}\to
\begin{pmatrix}
1&0&2&-1&1&-1\\
0&1&1&1&-1&1\\
0&2&2&0&-2&-2\\
0&0&0&0&0&0
\end{pmatrix}\to
\begin{pmatrix}
1&0&2&-1&1&-1\\
0&1&1&1&-1&1\\
0&0&0&-2&0&-4\\
0&0&0&0&0&0
\end{pmatrix},
\]
故可取\(\boldsymbol{\alpha}_1,\boldsymbol{\alpha}_2,\boldsymbol{\beta}_1\)为\(V_1 + V_2\)的基(不唯一).

再来求\(V_1\cap V_2\)的基. 首先注意到\(\boldsymbol{\alpha}_1,\boldsymbol{\alpha}_2\)是\(V_1\)的基(从上面的矩阵即可看出),又不难验证\(\boldsymbol{\beta}_1,\boldsymbol{\beta}_2\)是\(V_2\)的基,\(V_2\)中的向量可以表示为\(\boldsymbol{\beta}_1,\boldsymbol{\beta}_2\)的线性组合. 假设\(t_1\boldsymbol{\beta}_1 + t_2\boldsymbol{\beta}_2\)属于\(V_1\),则向量组\(\boldsymbol{\alpha}_1,\boldsymbol{\alpha}_2,t_1\boldsymbol{\beta}_1 + t_2\boldsymbol{\beta}_2\)和向量组\(\boldsymbol{\alpha}_1,\boldsymbol{\alpha}_2\)的秩相等(因为\(\boldsymbol{\alpha}_1,\boldsymbol{\alpha}_2\)是\(V_1\)的基). 因此,我们可以用矩阵方法来求出参数\(t_1,t_2\). 注意到
\[
\begin{pmatrix}
1&0&-t_1 + t_2\\
0&1&t_1 - t_2\\
-1&2&t_1 - 3t_2\\
0&1&t_1 - t_2
\end{pmatrix}\to
\begin{pmatrix}
1&0&-t_1 + t_2\\
0&1&t_1 - t_2\\
0&2&-2t_2\\
0&0&0
\end{pmatrix}\to
\begin{pmatrix}
1&0&-t_1 + t_2\\
0&1&t_1 - t_2\\
0&0&-2t_1\\
0&0&0
\end{pmatrix},
\]
故可得\(t_1 = 0\),所以\(V_1\cap V_2\)的基可取为\(\boldsymbol{\beta}_2\).

{\color{blue}解法二:} 求\(V_1 + V_2\)的基同解法1,现用解线性方程组的方法来求\(V_1\cap V_2\)的基. 因为\(\boldsymbol{\alpha}_1,\boldsymbol{\alpha}_2\)是\(V_1\)的基,\(\boldsymbol{\beta}_1,\boldsymbol{\beta}_2\)是\(V_2\)的基,故对任一向量\(\boldsymbol{\gamma}\in V_1\cap V_2\),\(\boldsymbol{\gamma}=x_1\boldsymbol{\alpha}_1 + x_2\boldsymbol{\alpha}_2=(-x_3)\boldsymbol{\beta}_1 + (-x_4)\boldsymbol{\beta}_2\). 因此,求向量\(\boldsymbol{\gamma}\)等价于求解线性方程组
\[
x_1\boldsymbol{\alpha}_1 + x_2\boldsymbol{\alpha}_2 + x_3\boldsymbol{\beta}_1 + x_4\boldsymbol{\beta}_2=\boldsymbol{0}.
\]
通过初等行变换将其系数矩阵\((\boldsymbol{\alpha}_1,\boldsymbol{\alpha}_2,\boldsymbol{\beta}_1,\boldsymbol{\beta}_2)\)进行化简:
\[
\begin{pmatrix}
1&0&-1&1\\
0&1&1&-1\\
0&0&-2&0\\
0&0&0&0
\end{pmatrix}\to
\begin{pmatrix}
1&0&0&1\\
0&1&0&-1\\
0&0&1&0\\
0&0&0&0
\end{pmatrix},
\]
故上述线性方程组的通解为\((x_1,x_2,x_3,x_4)=k(-1,1,0,1)\),从而\(\boldsymbol{\gamma}=-k(\boldsymbol{\alpha}_1 - \boldsymbol{\alpha}_2)=-k\boldsymbol{\beta}_2(k\in\mathbb{R})\),于是\(\boldsymbol{\beta}_2\)是\(V_1\cap V_2\)的基. 
\end{solution}


\end{document}
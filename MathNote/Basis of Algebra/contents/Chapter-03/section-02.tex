% contents/chapter-03/section-02.tex 第三章第二节
\documentclass[../../main.tex]{subfiles}
\graphicspath{{\subfix{../../image/}}} % 指定图片目录,后续可以直接使用图片文件名。

% 例如:
% \begin{figure}[H]
% \centering
% \includegraphics[scale=0.4]{image-01.01}
% \caption{图片标题}
% \label{figure:image-01.01}
% \end{figure}
% 注意:上述\label{}一定要放在\caption{}之后,否则引用图片序号会只会显示??.

\begin{document}

\section{线性空间}

\begin{example}[$\,\,$常见的线性空间]\label{example:常见的线性空间}
\begin{enumerate}[(1)]
\item 数域\(\mathbb{K}\)上\(n\)维行(列)向量集合\(\mathbb{K}_n(\mathbb{K}^n)\),在行(列)向量的加法和数乘下成为\(\mathbb{K}\)上的线性空间,称为数域\(\mathbb{K}\)上的\(n\)维行(列)向量空间.
\item 数域\(\mathbb{K}\)上的一元多项式全体\(\mathbb{K}[x]\),在多项式的加法和数乘下成为\(\mathbb{K}\)上的线性空间. 在\(\mathbb{K}[x]\)中,取次数小于等于\(n\)的多项式全体,记这个集合为\(\mathbb{K}_n[x]\),则\(\mathbb{K}_n[x]\)也是\(\mathbb{K}\)上的线性空间.
\item 数域\(\mathbb{K}\)上\(m\times n\)矩阵全体\(M_{m\times n}(\mathbb{K})\),在矩阵的加法和数乘下成为\(\mathbb{K}\)上的线性空间.
\item \label{example:常见的线性空间(4)}实数域\(\mathbb{R}\)上的连续函数全体记为\(C(\mathbb{R})\),函数的加法及数乘分别定义为\((f + g)(x)=f(x)+g(x)\),\((kf)(x)=kf(x)\),则\(C(\mathbb{R})\)是\(\mathbb{R}\)上的线性空间.
\end{enumerate}
\end{example}

\begin{proposition}\label{proposition:数域上的线性空间}
若两个数域\(\mathbb{K}_1\subseteq\mathbb{K}_2\),则\(\mathbb{K}_2\)可以看成是\(\mathbb{K}_1\)上的线性空间. 向量就是\(\mathbb{K}_2\)中的数,向量的加法就是数的加法,数乘就是\(\mathbb{K}_1\)中的数乘以\(\mathbb{K}_2\)中的数. 特别地,数域\(\mathbb{K}\)也可以看成是\(\mathbb{K}\)自身上的线性空间.
\end{proposition}

\begin{example}\label{example:3.3}
判断下列集合是否构成实数域\(\mathbb{R}\)上的线性空间:
\begin{enumerate}[(1)]
\item \(V\)为次数等于\(n(n\geq1)\)的实系数多项式全体,加法和数乘就是多项式的加法和数乘.
\item \(V = M_n(\mathbb{R})\),数乘就是矩阵的数乘,加法\(\oplus\)定义为\(\boldsymbol{A}\oplus\boldsymbol{B}=\boldsymbol{A}\boldsymbol{B}-\boldsymbol{B}\boldsymbol{A}\),其中等式右边是矩阵的乘法和减法.
\item \(V = M_n(\mathbb{R})\),数乘就是矩阵的数乘,加法\(\oplus\)定义为\(\boldsymbol{A}\oplus\boldsymbol{B}=\boldsymbol{A}\boldsymbol{B}+\boldsymbol{B}\boldsymbol{A}\),其中等式右边是矩阵的乘法和加法.
\item \label{example:3.3(4)}\(V\)是以\(0\)为极限的实数数列全体,即\(V = \left\{\{a_n\}\mid\lim_{n\rightarrow\infty}a_n = 0\right\}\),定义两个数列的加法\(\oplus\)及数乘\(\circ\)为:\(\{a_n\}\oplus\{b_n\}=\{a_n + b_n\}\),\(k\circ\{a_n\}=\{ka_n\}\),其中等式右边分别是数的加法和乘法.
\item \label{example:3.3(5)}\(V\)是正实数全体\(\mathbb{R}^+\),加法\(\oplus\)定义为\(a\oplus b = ab\),数乘\(\circ\)定义为\(k\circ a = a^k\),其中等式右边分别是数的乘法和乘方.
\item \label{example:3.3(6)}\(V\)为实数对全体\(\{(a,b)\mid a,b\in\mathbb{R}\}\),加法\(\oplus\)定义为\((a_1,b_1)\oplus(a_2,b_2)=(a_1 + a_2,b_1 + b_2 + a_1a_2)\),数乘\(\circ\)定义为\(k\circ(a,b)=(ka,kb+\frac{k(k - 1)}{2}a^2)\),其中等式右边分别是数的加法和乘法.
\end{enumerate}
\end{example}
\begin{solution}
(1) \(V\)不是线性空间, 因为加法不封闭.

(2) \(V\)不是线性空间, 因为加法不满足交换律, 即 \(\boldsymbol{A}\oplus\boldsymbol{B}\neq\boldsymbol{B}\oplus\boldsymbol{A}\).

(3) \(V\)不是线性空间, 因为加法不满足结合律, 即 \((\boldsymbol{A}\oplus\boldsymbol{B})\oplus\boldsymbol{C}\neq\boldsymbol{A}\oplus(\boldsymbol{B}\oplus\boldsymbol{C})\).

(4)、(5)、(6) \(V\)都是线性空间, 特别是 (5) 和 (6), 其加法和数乘的定义都不是线性的, 但它们竟然都是线性空间! 请读者自己验证线性空间的 8 条公理的确成立, 在下一节我们会从线性同构的角度来说明它们成为线性空间的深层次理由.
\end{solution}

\begin{proposition}
设\(V\)是\(n\)维线性空间,\(\boldsymbol{e}_1,\boldsymbol{e}_2,\cdots,\boldsymbol{e}_n\)是\(V\)中\(n\)个向量. 若它们满足下列条件之一:

(1) \(\boldsymbol{e}_1,\boldsymbol{e}_2,\cdots,\boldsymbol{e}_n\)线性无关;

(2) \(V\)中任一向量均可由\(\boldsymbol{e}_1,\boldsymbol{e}_2,\cdots,\boldsymbol{e}_n\)线性表示,
求证:\(\boldsymbol{e}_1,\boldsymbol{e}_2,\cdots,\boldsymbol{e}_n\)是\(V\)的一组基.
\end{proposition}
\begin{proof}
证明完全类似\hyperref[proposition:极大无关组的判定条件]{命题\ref{proposition:极大无关组的判定条件}}.
\end{proof}

\begin{theorem}[基扩充定理]\label{theorem:基扩充定理}
设\(V\)是\(n\)维线性空间,\(\boldsymbol{v}_1,\boldsymbol{v}_2,\cdots,\boldsymbol{v}_m\)是一组线性无关的向量(\(V\)的子空间\(U\)的一组基),\(\boldsymbol{e}_1,\boldsymbol{e}_2,\cdots,\boldsymbol{e}_n\)是\(V\)的一组基. 则必可在\(\boldsymbol{e}_1,\boldsymbol{e}_2,\cdots,\boldsymbol{e}_n\)中选出\(n - m\)个向量,使之和\(\boldsymbol{v}_1,\boldsymbol{v}_2,\cdots,\boldsymbol{v}_m\)一起组成\(V\)的一组基.
\end{theorem}
\begin{proof}
若\(m< n\),将\(\boldsymbol{e}_i(1\leq i\leq n)\)依次加入向量组\(\boldsymbol{v}_1,\boldsymbol{v}_2,\cdots,\boldsymbol{v}_m\),则必有一个\(\boldsymbol{e}_i\),使得\(\boldsymbol{v}_1,\boldsymbol{v}_2,\cdots,\boldsymbol{v}_m,\boldsymbol{e}_i\)线性无关. 这是因为若任意一个\(\boldsymbol{e}_i\)加入\(\boldsymbol{v}_1,\boldsymbol{v}_2,\cdots,\boldsymbol{v}_m\)后均线性相关,则由\hyperref[proposition:线性无关向量组的命题1]{命题\ref{proposition:线性无关向量组的命题1}}可知,每个\(\boldsymbol{e}_i\)都可用\(\boldsymbol{v}_1,\boldsymbol{v}_2,\cdots,\boldsymbol{v}_m\)线性表示,由\hyperref[theorem:向量的线性关系定理2]{定理\ref{theorem:向量的线性关系定理2}}可得\(n\leq m\),矛盾. 将新加入的向量$e_i$记作$v_{m+1}$,则原线性无关向量组扩张为$v_1,v_2,\cdots,v_{m+1}$,并且仍线性无关.若\(m + 1< n\),则同理又可从\(\boldsymbol{e}_1,\boldsymbol{e}_2,\cdots,\boldsymbol{e}_n\)中找到一个向量,加入$v_1,v_2,\cdots,v_{m+1}$之后仍线性无关.将新加入的向量记作$v_{m+2}$,则原线性无关向量组扩张为$v_1,v_2,\cdots,v_{m+2}$,并且仍线性无关.不断这样做下去,直到$m+n-m-1=n-1<n$时,同理可从\(\boldsymbol{e}_1,\boldsymbol{e}_2,\cdots,\boldsymbol{e}_n\)中找到一个向量,加入$v_1,v_2,\cdots,v_{n-1}$之后仍线性无关.将新加入的向量记作$v_{n}$,则可将\(\boldsymbol{v}_1,\boldsymbol{v}_2,\cdots,\boldsymbol{v}_m\)扩张成为$v_1,v_2,\cdots,v_{n}$,并且仍线性无关.此时$v_1,v_2,\cdots,v_{n}$就是\(V\)的一组基.
\end{proof}

\begin{proposition}\label{proposition:包含所有向量的空间也包含这些向量张成的空间}
若$\boldsymbol{\alpha }_1,\boldsymbol{\alpha }_2,\cdots ,\boldsymbol{\alpha }_n\subset V$,则$L\left( \boldsymbol{\alpha }_1,\boldsymbol{\alpha }_2,\cdots ,\boldsymbol{\alpha }_n \right) \subset V$.
\end{proposition}
\begin{proof}
证明是显然的.
\end{proof}

\begin{proposition}\label{proposition:与全空间维数相同的子空间等于全空间}
设\(V_1,V_2\)均为线性空间,若\(V_1\subset V_2\),并且\(\dim V_1 = \dim V_2\),则\(V_1 = V_2\).
\end{proposition}
\begin{proof}
取$V_1$的一组基即可得到证明.
\end{proof}

\begin{proposition}\label{proposition:和空间包含于空间的和}
证明:$L\left( \boldsymbol{\alpha }_1+\boldsymbol{\beta }_1,\boldsymbol{\alpha }_2+\boldsymbol{\beta }_2,\cdots ,\boldsymbol{\alpha }_n+\boldsymbol{\beta }_n \right) \subset L\left( \boldsymbol{\alpha }_1,\boldsymbol{\alpha }_2,\cdots ,\boldsymbol{\alpha }_n \right) +L\left( \boldsymbol{\beta }_1,\boldsymbol{\beta }_2,\cdots ,\boldsymbol{\beta }_n \right)$.
\end{proposition}
\begin{proof}
$\forall \boldsymbol{\alpha }\in L(\boldsymbol{\alpha }_1+\boldsymbol{\beta }_1,\boldsymbol{\alpha }_2+\boldsymbol{\beta }_2,\cdots,\boldsymbol{\alpha }_n+\boldsymbol{\beta }_n)$,则
\[
\boldsymbol{\alpha } = k_1(\boldsymbol{\alpha }_1+\boldsymbol{\beta }_1)+k_2(\boldsymbol{\alpha }_2+\boldsymbol{\beta }_2)+\cdots +k_n(\boldsymbol{\alpha }_n+\boldsymbol{\beta }_n) = k_1\boldsymbol{\alpha }_1 + k_2\boldsymbol{\alpha }_2+\cdots +k_n\boldsymbol{\alpha }_n + k_1\boldsymbol{\beta }_1 + k_2\boldsymbol{\beta }_2+\cdots +k_n\boldsymbol{\beta }_n.
\]
其中$k_1,k_2,\cdots,k_n\in \mathbb{R}$.

令$\boldsymbol{\beta } = k_1\boldsymbol{\alpha }_1 + k_2\boldsymbol{\alpha }_2+\cdots +k_n\boldsymbol{\alpha }_n$,$\boldsymbol{\gamma } = k_1\boldsymbol{\beta }_1 + k_2\boldsymbol{\beta }_2+\cdots +k_n\boldsymbol{\beta }_n$,则$\boldsymbol{\beta }\in L(\boldsymbol{\alpha }_1,\boldsymbol{\alpha }_2,\cdots,\boldsymbol{\alpha }_n)$,$\boldsymbol{\gamma }\in L(\boldsymbol{\beta }_1,\boldsymbol{\beta }_2,\cdots,\boldsymbol{\beta }_n)$.从而
\[
\boldsymbol{\alpha }=\boldsymbol{\beta }+\boldsymbol{\gamma }\in L(\boldsymbol{\alpha }_1,\boldsymbol{\alpha }_2,\cdots,\boldsymbol{\alpha }_n)+L(\boldsymbol{\beta }_1,\boldsymbol{\beta }_2,\cdots,\boldsymbol{\beta }_n).
\]
故$L\left( \boldsymbol{\alpha }_1+\boldsymbol{\beta }_1,\boldsymbol{\alpha }_2+\boldsymbol{\beta }_2,\cdots ,\boldsymbol{\alpha }_n+\boldsymbol{\beta }_n \right) \subset L\left( \boldsymbol{\alpha }_1,\boldsymbol{\alpha }_2,\cdots ,\boldsymbol{\alpha }_n \right) +L\left( \boldsymbol{\beta }_1,\boldsymbol{\beta }_2,\cdots ,\boldsymbol{\beta }_n \right)$.
\end{proof}

\begin{proposition}\label{proposition:等价向量组张成的子空间相同}
向量组$(\alpha_1,\alpha_2,\cdots,\alpha_n)$和$(\beta_1,\beta_2,\cdots,\beta_n)$等价的充要条件是$L(\alpha_1,\alpha_2,\cdots,\alpha_n) = L(\beta_1,\beta_2,\cdots,\beta_n)$.
\end{proposition}
\begin{proof}
必要性:若$(\alpha _1,\alpha _2,\cdots ,\alpha _n)$和$(\beta _1,\beta _2,\cdots ,\beta _n)$等价,则对$\forall i\in \{ 1,2,\cdots ,n \}$,都有
\begin{align*}
\alpha _i=\sum_{j=1}^n{k_j\beta _j},
\end{align*}
其中$k_1,k_2,\cdots ,k_n$不全为$0$.于是任取$\alpha \in L(\alpha _1,\alpha _2,\cdots ,\alpha _n)$,就有
\begin{align*}
\alpha =\sum_{i=1}^n{a_i\alpha _i}=\sum_{i=1}^n{\left( a_i\sum_{j=1}^n{k_j\beta _j} \right)}=\left( \sum_{i=1}^n{a_i} \right) \cdot \left( \sum_{j=1}^n{k_j\beta _j} \right) .
\end{align*}
因此$\alpha \in L(\beta _1,\beta _2,\cdots ,\beta _n)$.于是$L(\alpha _1,\alpha _2,\cdots ,\alpha _n) \subset L(\beta _1,\beta _2,\cdots ,\beta _n)$.同理可证$L(\alpha _1,\alpha _2,\cdots ,\alpha _n) \supset L(\beta _1,\beta _2,\cdots ,\beta _n)$.故$L(\alpha _1,\alpha _2,\cdots ,\alpha _n) =L(\beta _1,\beta _2,\cdots ,\beta _n)$.

充分性:若$L(\alpha _1,\alpha _2,\cdots ,\alpha _n) =L(\beta _1,\beta _2,\cdots ,\beta _n)$,则$(\alpha _1,\alpha _2,\cdots ,\alpha _n)$和$(\beta _1,\beta _2,\cdots ,\beta _n)$显然可以相互线性表出,故$(\alpha _1,\alpha _2,\cdots ,\alpha _n)$和$(\beta _1,\beta _2,\cdots ,\beta _n)$等价.
\end{proof}

\begin{example}[$\,\,$一些常见线性空间的基]\label{example:一些常见线性空间的基}
\begin{enumerate}[(1)]
\item \label{example:一些常见线性空间的基(1)}设\(V\)是数域\(\mathbb{K}\)上次数不超过\(n\)的多项式全体构成的线性空间,求证:\(\{1,x,x^2,\cdots,x^n\}\)是\(V\)的一组基,并且\(\{1,x + 1,(x + 1)^2,\cdots,(x + 1)^n\}\)也是\(V\)的一组基.
\item  \label{example:一些常见线性空间的基(2)}设\(V\)是数域\(\mathbb{K}\)上次数小于\(n\)的多项式全体构成的线性空间,\(a_1,a_2,\cdots,a_n\)是\(\mathbb{K}\)中互不相同的\(n\)个数,\(f(x)=(x - a_1)(x - a_2)\cdots(x - a_n)\),\(f_i(x)=f(x)/(x - a_i)\),求证:\(\{f_1(x),f_2(x),\cdots,f_n(x)\}\)组成\(V\)的一组基.
\item \label{example:一些常见线性空间的基(3)}设\(V\)是数域\(\mathbb{K}\)上\(m\times n\)矩阵全体组成的线性空间,令\(\boldsymbol{E}_{ij}(1\leq i\leq m,1\leq j\leq n)\)是第\((i,j)\)元素为\(1\)、其余元素为\(0\)的\(m\times n\)矩阵,求证:全体\(\boldsymbol{E}_{ij}\)组成了\(V\)的一组基,从而\(V\)是\(mn\)维线性空间.
\item  \label{example:一些常见线性空间的基(4)}\(V\)是数域\(\mathbb{K}\)上\(n\)阶上三角矩阵全体组成的线性空间.容易验证\(\{\boldsymbol{E}_{ij}(1\leq i\leq j\leq n)\}\)是\(V\)的一组基,因此\(\dim V=\frac{n(n + 1)}{2}\).
\item  \label{example:一些常见线性空间的基(5)}\(V\)是数域\(\mathbb{K}\)上\(n\)阶对称矩阵全体组成的线性空间.容易验证\(\{\boldsymbol{E}_{ii}(1\leq i\leq n);\boldsymbol{E}_{ij}+\boldsymbol{E}_{ji}(1\leq i< j\leq n)\}\)是\(V\)的一组基,因此\(\dim V=\frac{n(n + 1)}{2}\).
\item  \label{example:一些常见线性空间的基(6)}\(V\)是数域\(\mathbb{K}\)上\(n\)阶反对称矩阵全体组成的线性空间.容易验证\(\{\boldsymbol{E}_{ij}-\boldsymbol{E}_{ji}(1\leq i< j\leq n)\}\)是\(V\)的一组基,因此\(\dim V=\frac{n(n - 1)}{2}\).
\end{enumerate}
\end{example}
\begin{proof}
\begin{enumerate}[(1)]
\item 根据多项式的定义容易验证\(\{1,x,x^2,\cdots,x^n\}\)是\(V\)的一组基,特别地,\(\dim V=n + 1\). 对任意的\(f(x)\in V\),设\(y=x + 1\),则
\[
f(x)=f(y - 1)=b_ny^n+\cdots+b_1y + b_0=b_n(x + 1)^n+\cdots+b_1(x + 1)+b_0,
\]
其中\(b_n,\cdots,b_1,b_0\)是\(\mathbb{K}\)中的数. 因此,\(V\)中任一多项式\(f(x)\)均可由\(1,x + 1,(x + 1)^2,\cdots,(x + 1)^n\)线性表示. 由例3.24可知,\(\{1,x + 1,(x + 1)^2,\cdots,(x + 1)^n\}\)是\(V\)的一组基.
\item 因为\(V\)是\(n\)维线性空间,故由例3.24只需证明\(n\)个向量\(f_1(x),f_2(x),\cdots,f_n(x)\)线性无关即可. 设
\[
k_1f_1(x)+k_2f_2(x)+\cdots+k_nf_n(x)=0,
\]
依次令\(x = a_1,a_2,\cdots,a_n\),即可求出\(k_1 = k_2=\cdots=k_n = 0\). 
\item 一方面,对任意的\(\boldsymbol{A}=(a_{ij})\in V\),容易验证\(\boldsymbol{A}=\sum_{i = 1}^{m}\sum_{j = 1}^{n}a_{ij}\boldsymbol{E}_{ij}\). 另一方面,设\(mn\)个数\(c_{ij}(1\leq i\leq m,1\leq j\leq n)\)满足\(\sum_{i = 1}^{m}\sum_{j = 1}^{n}c_{ij}\boldsymbol{E}_{ij}=\boldsymbol{O}\),则由矩阵相等的定义可得所有的\(c_{ij}=0\). 因此,全体\(\boldsymbol{E}_{ij}\)组成了\(V\)的一组基,从而\(\dim V = mn\).
\item 
\item 
\item 
\end{enumerate}
\end{proof}

\begin{example}[$\,\,$\(n\)阶(斜)Hermite矩阵全体构成的线性空间]\label{example:n阶(斜)Hermite矩阵全体构成的线性空间}

设\(V_1 = \{ \boldsymbol{A} \in M_n(\mathbb{C})|\overline{\boldsymbol{A}}'=\boldsymbol{A}\}\)为\(n\)阶Hermite矩阵全体,\(V_2 = \{ \boldsymbol{A} \in M_n(\mathbb{C})|\overline{\boldsymbol{A}}'=-\boldsymbol{A}\}\)为\(n\)阶斜Hermite矩阵全体,求证:在矩阵加法和实数关于矩阵的数乘下,\(V_1,V_2\)成为实数域\(\mathbb{R}\)上的线性空间,并且具有相同的维数.
\end{example}
\begin{proof}
首先,容易证明对任意的\(\boldsymbol{A},\boldsymbol{B} \in V_i,c \in \mathbb{R}\),我们有\(\boldsymbol{A}+\boldsymbol{B} \in V_i,c\boldsymbol{A} \in V_i\),这就验证了上述加法和数乘是定义好的运算. 其次,容易验证线性空间的8条公理成立,因此\(V_1,V_2\)是实线性空间(注意虽然向量都是复矩阵,但它们绝不是复线性空间). 最后,容易验证\(\{\boldsymbol{E}_{ii}(1\leq i\leq n);\boldsymbol{E}_{ij}+\boldsymbol{E}_{ji}(1\leq i< j\leq n);\mathrm{i}\boldsymbol{E}_{ij}-\mathrm{i}\boldsymbol{E}_{ji}(1\leq i< j\leq n)\}\)是\(V_1\)的一组基,\(\{\mathrm{i}\boldsymbol{E}_{ii}(1\leq i\leq n);\boldsymbol{E}_{ij}-\boldsymbol{E}_{ji}(1\leq i< j\leq n);\mathrm{i}\boldsymbol{E}_{ij}+\mathrm{i}\boldsymbol{E}_{ji}(1\leq i< j\leq n)\}\)是\(V_2\)的一组基,因此\(\dim_{\mathbb{R}}V_1=\dim_{\mathbb{R}}V_2=n^2\).
\end{proof}

\begin{example}
设\(\mathbb{Q}(\sqrt[3]{2})=\{a + b\sqrt[3]{2}+c\sqrt[3]{4}\}\),其中\(a,b,c\)均是有理数,证明:\(\mathbb{Q}(\sqrt[3]{2})\)是有理数域上的线性空间并求其维数.
\end{example}
\begin{proof}
事实上,我们可以证明\(\mathbb{Q}(\sqrt[3]{2})\)是一个数域. 加法、减法和乘法的封闭性都是显然的,我们只要证明除法封闭,或等价地证明非零数的倒数封闭即可. 为此首先需要找出一个数非零的充要条件. 我们断言以下3个结论等价:
\begin{gather*}
(1) a + b\sqrt[3]{2}+c\sqrt[3]{4}=0;\quad(2) a^3 + 2b^3+4c^3-6abc = 0;\quad (3) a = b = c = 0.
\end{gather*}
由公式\((x + y + z)(x^2 + y^2+z^2-xy - yz - zx)=x^3 + y^3+z^3-3xyz\)很容易从(1)推出(2). 假设(2)对不全为零的有理数\(a,b,c\)成立,将(2)式两边同时乘以\(a,b,c\)公分母的立方,可将\(a,b,c\)化为整数; 又可将整数\(a,b,c\)的最大公因数从(2)式提出,因此不妨假设满足(2)式的\(a,b,c\)是互素的整数. 由(2)式可得\(a\)是偶数,可设\(a = 2a_1\),代入(2)式可得\((2')4a_1^3 + b^3+2c^3-6a_1bc = 0\); 由\((2')\)式可得\(b\)是偶数,可设\(b = 2b_1\),代入\((2')\)式可得\((2'')2a_1^3 + 4b_1^3+c^3-6a_1b_1c = 0\); 由\((2'')\)式可得\(c\)是偶数,可设\(c = 2c_1\),这样\(a,b,c\)就有了公因子\(2\),这与它们互素矛盾. 因此,从(2)可以推出(3). 从(3)推出(1)是显然的.

任取\(\mathbb{Q}(\sqrt[3]{2})\)中的非零数\(a + b\sqrt[3]{2}+c\sqrt[3]{4}\),由上述充要条件以及公式可得
\[
(a + b\sqrt[3]{2}+c\sqrt[3]{4})((a^2 - 2bc)+(c^2 - ab)\sqrt[3]{2}+(b^2 - ac)\sqrt[3]{4})=a^3 + 2b^3+4c^3-6abc\neq0,
\]
从而\((a^2 - 2bc)+(c^2 - ab)\sqrt[3]{2}+(b^2 - ac)\sqrt[3]{4}\neq0\). 将倒数\(\frac{1}{a + b\sqrt[3]{2}+c\sqrt[3]{4}}\)的分子分母同时乘以非零数\((a^2 - 2bc)+(c^2 - ab)\sqrt[3]{2}+(b^2 - ac)\sqrt[3]{4}\)进行化简,可得
\[
\frac{1}{a + b\sqrt[3]{2}+c\sqrt[3]{4}}=\frac{(a^2 - 2bc)+(c^2 - ab)\sqrt[3]{2}+(b^2 - ac)\sqrt[3]{4}}{a^3 + 2b^3+4c^3-6abc}\in\mathbb{Q}(\sqrt[3]{2}).
\]
这就证明了\(\mathbb{Q}(\sqrt[3]{2})\)是一个数域. 因为\(\mathbb{Q}\subseteq\mathbb{Q}(\sqrt[3]{2})\),故由\hyperref[proposition:数域上的线性空间]{命题\ref{proposition:数域上的线性空间}}可知,\(\mathbb{Q}(\sqrt[3]{2})\)是有理数域上的线性空间.

由\(\mathbb{Q}(\sqrt[3]{2})\)的定义可知,\(\mathbb{Q}(\sqrt[3]{2})\)中每个数都是\(1,\sqrt[3]{2},\sqrt[3]{4}\)的\(\mathbb{Q}\)-线性组合; 又由上述充要条件可知,\(1,\sqrt[3]{2},\sqrt[3]{4}\)是\(\mathbb{Q}\)-线性无关的. 因此,\(\{1,\sqrt[3]{2},\sqrt[3]{4}\}\)是\(\mathbb{Q}(\sqrt[3]{2})\)的一组基. 特别地,\(\dim_{\mathbb{Q}}\mathbb{Q}(\sqrt[3]{2}) = 3\).
\end{proof}

\begin{proposition}\label{proposition:数域上的线性空间的维数的传递性}
设\(\mathbb{K}_1,\mathbb{K}_2,\mathbb{K}_3\)是数域且\(\mathbb{K}_1\subseteq\mathbb{K}_2\subseteq\mathbb{K}_3\).若将\(\mathbb{K}_2\)看成是\(\mathbb{K}_1\)上的线性空间,其维数为\(m\),又将\(\mathbb{K}_3\)看成是\(\mathbb{K}_2\)上的线性空间,其维数为\(n\).则如将\(\mathbb{K}_3\)看成是\(\mathbb{K}_1\)上的线性空间,则其维数为\(mn\).
\end{proposition}
\begin{proof}
\(\mathbb{K}_2\)作为\(\mathbb{K}_1\)上的线性空间,取其一组基为\(\{\alpha_1,\alpha_2,\cdots,\alpha_m\}\);\(\mathbb{K}_3\)作为\(\mathbb{K}_2\)上的线性空间,取其一组基为\(\{\beta_1,\beta_2,\cdots,\beta_n\}\).注意到\(\alpha_i,\beta_j\)都是数,现在我们断言:\(\mathbb{K}_3\)作为\(\mathbb{K}_1\)上的线性空间,\(\{\alpha_i\beta_j(1\leq i\leq m,1\leq j\leq n)\}\)恰为其一组基.

一方面,对\(\mathbb{K}_3\)中任一数\(a\),存在\(\mathbb{K}_2\)中的数\(b_1,b_2,\cdots,b_n\),使得
\[
a = b_1\beta_1 + b_2\beta_2+\cdots + b_n\beta_n.
\]
又对\(b_j\in\mathbb{K}_2\),存在\(\mathbb{K}_1\)中的数\(c_{1j},c_{2j},\cdots,c_{mj}\),使得
\[
b_j = c_{1j}\alpha_1 + c_{2j}\alpha_2+\cdots + c_{mj}\alpha_m,1\leq j\leq n.
\]
将上述两式进行整理,可得
\[
a=\sum_{j = 1}^{n}b_j\beta_j=\sum_{j = 1}^{n}\left(\sum_{i = 1}^{m}c_{ij}\alpha_i\right)\beta_j=\sum_{j = 1}^{n}\sum_{i = 1}^{m}c_{ij}\alpha_i\beta_j,
\]
即\(\mathbb{K}_3\)中任一数均可由\(\{\alpha_i\beta_j(1\leq i\leq m,1\leq j\leq n)\}\)线性表示.

另一方面,设有\(\mathbb{K}_1\)中的数\(k_{ij}(1\leq i\leq m,1\leq j\leq n)\),使得
\[
\sum_{j = 1}^{n}\sum_{i = 1}^{m}k_{ij}\alpha_i\beta_j = 0,
\]
则经过变形可得
\[
\sum_{j = 1}^{n}\left(\sum_{i = 1}^{m}k_{ij}\alpha_i\right)\beta_j = 0.
\]
注意到\(\sum_{i = 1}^{m}k_{ij}\alpha_i\in\mathbb{K}_2\)且\(\beta_1,\beta_2,\cdots,\beta_n\)是\(\mathbb{K}_3/\mathbb{K}_2\)的一组基,故有\(\sum_{i = 1}^{m}k_{ij}\alpha_i = 0(1\leq j\leq n)\).又因为\(\{\alpha_1,\alpha_2,\cdots,\alpha_m\}\)是\(\mathbb{K}_2/\mathbb{K}_1\)的一组基,故有\(k_{ij}=0(1\leq i\leq m,1\leq j\leq n)\),即\(\{\alpha_i\beta_j(1\leq i\leq m,1\leq j\leq n)\}\)是\(\mathbb{K}_1\) - 线性无关的.

综上所述,\(\{\alpha_i\beta_j(1\leq i\leq m,1\leq j\leq n)\}\)是\(\mathbb{K}_3/\mathbb{K}_1\)的一组基,特别地,\(\dim_{\mathbb{K}_1}\mathbb{K}_3=mn=\dim_{\mathbb{K}_1}\mathbb{K}_2\cdot\dim_{\mathbb{K}_2}\mathbb{K}_3\).
\end{proof}

\begin{proposition}\label{proposition:数域上的线性空间的基的传递性}
设\(\mathbb{F}\subseteq\mathbb{K}\)为数域,\(\mathbb{K}\)作为\(\mathbb{F}\)上的线性空间,一组基为\(\{\alpha_1,\alpha_2,\cdots,\alpha_m\}\);设\(V\)为\(\mathbb{K}\)上的\(n\)维线性空间,一组基为\(\{\boldsymbol{e}_1,\boldsymbol{e}_2,\cdots,\boldsymbol{e}_n\}\).则\(V\)是\(\mathbb{F}\)上的\(mn\)维线性空间,一组基可选择为\(\{\alpha_i\boldsymbol{e}_j(1\leq i\leq m,1\leq j\leq n)\}\).
\end{proposition}
\begin{proof}
证明与\hyperref[proposition:数域上的线性空间的维数的传递性]{命题\ref{proposition:数域上的线性空间的维数的传递性}}完全类似.
\end{proof}

\begin{example}
证明下列线性空间是实数域上的无限维线性空间:

(1) 实数域\(\mathbb{R}\)上的连续函数全体构成的线性空间\(C(\mathbb{R})\)(见\hyperref[example:常见的线性空间(4)]{例题\ref{example:常见的线性空间}\ref{example:常见的线性空间(4)}});

(2) 以\(0\)为极限的实数数列全体构成的线性空间\(V = \left\{\{a_n\}|\lim_{n\rightarrow\infty}a_n = 0\right\}\)(见\hyperref[example:3.3(4)]{例题\ref{example:3.3}\ref{example:3.3(4)}}).
\end{example}
\begin{proof}
我们用反证法来证明.

(1) 若\(C(\mathbb{R})\)是有限维线性空间,则可取到正整数\(k>\dim C(\mathbb{R})\). 然而由\hyperref[example:3.1.1.1]{例题\ref{example:3.1.1.1}}可知\(\sin x,\sin 2x,\cdots,\sin kx\)是\(\mathbb{R}\) - 线性无关的,矛盾.

(2) 若\(V\)是有限维线性空间,则可取到正整数\(k>\dim V\). 构造\(V\)中\(k\)个数列:
\[
\left\{a_n^{(1)}=\frac{1}{n}\right\},\left\{a_n^{(2)}=\frac{1}{n^2}\right\},\cdots,\left\{a_n^{(k)}=\frac{1}{n^k}\right\}.
\]
设有实数\(c_1,c_2,\cdots,c_k\),使得
\[
c_1\left\{a_n^{(1)}\right\}+c_2\left\{a_n^{(2)}\right\}+\cdots + c_k\left\{a_n^{(k)}\right\}=\{0\},
\]
则对于任意的正整数\(n\),成立
\[
\frac{c_1}{n}+\frac{c_2}{n^2}+\cdots+\frac{c_k}{n^k}=0.
\]
任取\(k\)个不同的正整数代入上式,并利用Vandermonde行列式即得\(c_1 = c_2=\cdots = c_k = 0\),从而上述\(k\)个数列线性无关,矛盾. 
\end{proof}

\begin{definition}[无限维空间基的定义]\label{definition:无限维空间基的定义}
设\(B = \{ \boldsymbol{e}_i\}_{i\in I}\)为线性空间\(V\)中的向量族,若\(B\)中任意有限个向量都线性无关,则称向量族\(B\)线性无关;若向量\(\boldsymbol{\alpha}\)可表示为\(B\)中有限个向量的线性组合,则称\(\boldsymbol{\alpha}\)可被向量族\(B\)线性表示. 若线性空间\(V\)中存在线性无关的向量族\(B\),使得\(V = L(B)\),即\(V\)中任一向量都可被\(B\)线性表示,则称向量族\(B\)是\(V\)的一组基.
\end{definition}
\begin{note}
再利用选择公理或Zorn引理就可以证明任意线性空间中基的存在性了.因此这个定义是良定义.
\end{note}


\end{document}
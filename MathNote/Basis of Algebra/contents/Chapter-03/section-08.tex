% contents/chapter-03/section-08.tex 第三章第三节
\documentclass[../../main.tex]{subfiles}
\graphicspath{{\subfix{../../image/}}} % 指定图片目录,后续可以直接使用图片文件名。

% 例如:
% \begin{figure}[H]
% \centering
% \includegraphics{image-01.01}
% \caption{图片标题}
% \label{figure:image-01.01}
% \end{figure}
% 注意:上述\label{}一定要放在\caption{}之后,否则引用图片序号会只会显示??.

\begin{document}

\section{线性方程组的解及其应用}

\subsection{线性方程组的解的讨论}

\begin{proposition}\label{proposition:线性方程组同解系数矩阵秩相同}
线性方程组$\boldsymbol{Ax}=\mathbf{0}$与$\boldsymbol{Bx}=\mathbf{0}$同解当且仅当$r\left( \boldsymbol{A} \right) =\mathrm{r}\left( \boldsymbol{B} \right)$ .
\end{proposition}
\begin{proof}

\end{proof}

\begin{example}
设\(\boldsymbol{A}\)是一个\(m\times n\)矩阵,记\(\boldsymbol{\alpha}_i\)是\(\boldsymbol{A}\)的第\(i\)个行向量,\(\boldsymbol{\beta}=(b_1,b_2,\cdots,b_n)\). 求证:若齐次线性方程组\(\boldsymbol{A}\boldsymbol{x}=\boldsymbol{0}\)的解全是方程\(b_1x_1 + b_2x_2+\cdots + b_nx_n = 0\)的解,则\(\boldsymbol{\beta}\)是\(\boldsymbol{\alpha}_1,\boldsymbol{\alpha}_2,\cdots,\boldsymbol{\alpha}_m\)的线性组合.
\end{example}
\begin{proof}
令\(\boldsymbol{B}=\begin{pmatrix}
\boldsymbol{A}\\
\boldsymbol{\beta}
\end{pmatrix}\),由已知,方程组\(\boldsymbol{A}\boldsymbol{x}=\boldsymbol{0}\)和方程组\(\boldsymbol{B}\boldsymbol{x}=\boldsymbol{0}\)同解,故\(\mathrm{r}(\boldsymbol{A})=\mathrm{r}(\boldsymbol{B})\),从而\(\boldsymbol{A}\)的行向量的极大无关组也是\(\boldsymbol{B}\)的行向量的极大无关组. 因此,\(\boldsymbol{\beta}\)可表示为\(\boldsymbol{\alpha}_1,\boldsymbol{\alpha}_2,\cdots,\boldsymbol{\alpha}_m\)的线性组合. 
\end{proof}

\begin{example}
设\(\boldsymbol{A}\boldsymbol{x}=\boldsymbol{\beta}\)是\(m\)个方程式\(n\)个未知数的线性方程组,求证:它有解的充要条件是方程组\(\boldsymbol{A}'\boldsymbol{y}=\boldsymbol{0}\)的任一解\(\boldsymbol{\alpha}\)均适合等式\(\boldsymbol{\alpha}'\boldsymbol{\beta}=0\).
\end{example}
\begin{proof}
方程组\(\boldsymbol{A}\boldsymbol{x}=\boldsymbol{\beta}\)有解当且仅当\(\mathrm{r}\left( \boldsymbol{A}\,|\,\boldsymbol{\beta } \right) =\mathrm{r}(\boldsymbol{A})\),当且仅当\(\mathrm{r}\begin{pmatrix}
\boldsymbol{A}'\\
\boldsymbol{\beta}'
\end{pmatrix}=\mathrm{r}(\boldsymbol{A}')\),当且仅当方程组\(\begin{pmatrix}
\boldsymbol{A}'\\
\boldsymbol{\beta}'
\end{pmatrix}\boldsymbol{y}=\boldsymbol{0}\)与\(\boldsymbol{A}'\boldsymbol{y}=\boldsymbol{0}\)同解,而这当且仅当\(\boldsymbol{A}'\boldsymbol{y}=\boldsymbol{0}\)的任一解\(\boldsymbol{\alpha}\)均适合等式\(\boldsymbol{\beta}'\boldsymbol{\alpha}=0\),即\(\boldsymbol{\alpha}'\boldsymbol{\beta}=0\).
\end{proof}

\begin{example}
设有两个线性方程组:
\begin{align}\label{equation:3.301.1}
\begin{cases}
a_{11}x_1 + a_{12}x_2+\cdots + a_{1n}x_n = b_1,\\
a_{21}x_1 + a_{22}x_2+\cdots + a_{2n}x_n = b_2,\\
\cdots\cdots\cdots\cdots\\
a_{m1}x_1 + a_{m2}x_2+\cdots + a_{mn}x_n = b_m;
\end{cases} 
\end{align}
\begin{align}\label{equation:3.301.2}
\begin{cases}
a_{11}x_1 + a_{21}x_2+\cdots + a_{m1}x_m = 0,\\
a_{12}x_1 + a_{22}x_2+\cdots + a_{m2}x_m = 0,\\
\cdots\cdots\cdots\cdots\\
a_{1n}x_1 + a_{2n}x_2+\cdots + a_{mn}x_m = 0,\\
b_1x_1 + b_2x_2+\cdots + b_mx_m = 1.
\end{cases}
\end{align}
求证: 方程组\eqref{equation:3.301.1}有解的充要条件是方程组\eqref{equation:3.301.2}无解.
\end{example}
\begin{proof}
设第一个线性方程组的系数矩阵为\(\boldsymbol{A}\), 常数向量为\(\boldsymbol{\beta}\), 则第二个线性方程组的系数矩阵和增广矩阵分别为
\[
\boldsymbol{B}=\begin{pmatrix}
\boldsymbol{A}'\\
\boldsymbol{\beta}'
\end{pmatrix},\widetilde{\boldsymbol{B}}=\begin{pmatrix}
\boldsymbol{A}'&\boldsymbol{O}\\
\boldsymbol{\beta}'&1
\end{pmatrix}.
\]
显然,由矩阵初等变换可知,我们有\(\mathrm{r}(\widetilde{\boldsymbol{B}})=\mathrm{r}(\boldsymbol{A}') + 1=\mathrm{r}(\boldsymbol{A})+1\).

若方程组\eqref{equation:3.301.1}有解, 则\(\mathrm{r}\left( \boldsymbol{A}\,|\,\boldsymbol{\beta } \right) =\mathrm{r}(\boldsymbol{A})\), 故\(\mathrm{r}(\boldsymbol{B})=\mathrm{r}(\boldsymbol{B}')=\mathrm{r}\left( \boldsymbol{A}\,|\,\boldsymbol{\beta } \right) =\mathrm{r}(\boldsymbol{A})\neq\mathrm{r}(\widetilde{\boldsymbol{B}})\). 因此, 方程组\eqref{equation:3.301.2}无解.

反之, 若方程组\eqref{equation:3.301.1}无解, 则\(\mathrm{r}\left( \boldsymbol{A}\,|\,\boldsymbol{\beta } \right) =\mathrm{r}(\boldsymbol{A}) + 1\), 故\(\mathrm{r}(\boldsymbol{B})=\mathrm{r}(\boldsymbol{B}')=\mathrm{r}\left( \boldsymbol{A}\,|\,\boldsymbol{\beta } \right) =\mathrm{r}(\boldsymbol{A})+1=\mathrm{r}(\widetilde{\boldsymbol{B}})\). 因此, 方程组\eqref{equation:3.301.2}有解.
\end{proof}

\begin{proposition}\label{proposition:线性方程组的解结论1}
设\(\boldsymbol{A}\)是秩为\(r\)的\(m\times n\)矩阵,求证:必存在秩为\(n - r\)的\(n\times(n - r)\)矩阵\(\boldsymbol{B}\),使得\(\boldsymbol{A}\boldsymbol{B}=\boldsymbol{O}\).
\end{proposition}
\begin{proof}
考虑线性方程组\(\boldsymbol{A}\boldsymbol{x}=\boldsymbol{0}\),它有\(n - r\)个基础解系,不妨设为\(\boldsymbol{\beta}_1,\cdots,\boldsymbol{\beta}_{n - r}\). 令\(\boldsymbol{B}=(\boldsymbol{\beta}_1,\cdots,\boldsymbol{\beta}_{n - r})\),则\(\boldsymbol{A}\boldsymbol{B}=(\boldsymbol{A}\boldsymbol{\beta}_1,\cdots,\boldsymbol{A}\boldsymbol{\beta}_{n - r})=\boldsymbol{O}\),结论得证. 
\end{proof}

\begin{example}
设
\[
\boldsymbol{A}=\begin{pmatrix}
a_{11}&a_{12}&\cdots&a_{1n}\\
a_{21}&a_{22}&\cdots&a_{2n}\\
\vdots&\vdots&&\vdots\\
a_{m1}&a_{m2}&\cdots&a_{mn}
\end{pmatrix}(m < n),
\]
已知\(\boldsymbol{A}\boldsymbol{x}=\boldsymbol{0}\)的基础解系为\(\boldsymbol{\beta}_i=(b_{i1},b_{i2},\cdots,b_{in})'(1\leq i\leq n - m)\),试求齐次线性方程组
\[
\sum_{j = 1}^{n}b_{ij}y_j = 0(i = 1,2,\cdots,n - m)
\]
的基础解系.
\end{example}
\begin{solution}
令\(\boldsymbol{B}=(\boldsymbol{\beta}_1,\boldsymbol{\beta}_2,\cdots,\boldsymbol{\beta}_{n - m})\),则\(\boldsymbol{A}\boldsymbol{B}=\boldsymbol{O},\boldsymbol{B}'\boldsymbol{A}'=\boldsymbol{O}\). 因为\(\boldsymbol{A}\boldsymbol{x}=\boldsymbol{0}\)有$n-m$个基础解系,所以\(\boldsymbol{A}\)的秩为\(m\).又由于$\mathrm{r}\left( \boldsymbol{B} \right) =\mathrm{r}\left( \boldsymbol{B}‘ \right) =n-m$,因此\(\boldsymbol{B}'\boldsymbol{y}=\boldsymbol{0}\)的基础解系有$m$个.故\(\boldsymbol{B}'\boldsymbol{y}=\boldsymbol{0}\)的基础解系为\(\boldsymbol{A}'\)的全部列向量,即\(\boldsymbol{A}\)的所有行向量.
\end{solution}

\begin{proposition}\label{proposition:一个线性子空间对应一个线性方程组的系数矩阵}
设\(V_0\)是数域\(\mathbb{K}\)上\(n\)维列向量空间的真子空间,求证:必存在矩阵\(\boldsymbol{A}\),使得\(V_0\)是\(n\)元齐次线性方程组\(\boldsymbol{A}\boldsymbol{x}=\boldsymbol{0}\)的解空间.
\end{proposition}
\begin{proof}
设\(\boldsymbol{\beta}_1,\cdots,\boldsymbol{\beta}_r\)是子空间\(V_0\)的一组基. 令\(\boldsymbol{B}=(\boldsymbol{\beta}_1,\cdots,\boldsymbol{\beta}_r)\),这是一个\(n\times r\)矩阵. 考虑齐次线性方程组\(\boldsymbol{B}'\boldsymbol{x}=\boldsymbol{0}\),因为\(\boldsymbol{B}\)的秩等于\(r\),故其基础解系含\(n - r\)个向量,记为\(\boldsymbol{\alpha}_1,\cdots,\boldsymbol{\alpha}_{n - r}\). 令\(\boldsymbol{A}=(\boldsymbol{\alpha}_1,\cdots,\boldsymbol{\alpha}_{n - r})'\),这是个\((n - r)\times n\)矩阵且秩为\(n - r\). 由\(\boldsymbol{B}'\boldsymbol{A}'=\boldsymbol{O}\)可得\(\boldsymbol{A}\boldsymbol{B}=\boldsymbol{O}\),因此齐次线性方程组\(\boldsymbol{A}\boldsymbol{x}=\boldsymbol{0}\)的基础解系是\(\boldsymbol{\beta}_1,\cdots,\boldsymbol{\beta}_r\),其解空间就是\(V_0\).
\end{proof}
\begin{remark}
设\(\boldsymbol{\beta}_1,\cdots,\boldsymbol{\beta}_r\)是子空间\(V_0\)的一组基. 令\(\boldsymbol{B}=(\boldsymbol{\beta}_1,\cdots,\boldsymbol{\beta}_r)\),这是一个\(n\times r\)矩阵.也可以由\hyperref[proposition:线性方程组的解结论1]{命题\ref{proposition:线性方程组的解结论1}}直接得到存在矩阵$\boldsymbol{A}$,使得\(\boldsymbol{A}\boldsymbol{B}=\boldsymbol{O}\),因此齐次线性方程组\(\boldsymbol{A}\boldsymbol{x}=\boldsymbol{0}\)的基础解系是\(\boldsymbol{\beta}_1,\cdots,\boldsymbol{\beta}_r\),其解空间就是\(V_0\)..
\end{remark}

\begin{example}
设\(\boldsymbol{A}\)是秩为\(r\)的\(m\times n\)矩阵,\(\boldsymbol{\alpha}_1,\cdots,\boldsymbol{\alpha}_{n - r}\)与\(\boldsymbol{\beta}_1,\cdots,\boldsymbol{\beta}_{n - r}\)是齐次线性方程组\(\boldsymbol{A}\boldsymbol{x}=\boldsymbol{0}\)的两个基础解系. 求证:必存在\(n - r\)阶可逆矩阵\(\boldsymbol{P}\),使得
\[
(\boldsymbol{\beta}_1,\cdots,\boldsymbol{\beta}_{n - r})=(\boldsymbol{\alpha}_1,\cdots,\boldsymbol{\alpha}_{n - r})\boldsymbol{P}.
\]
\end{example}
\begin{proof}
设\(U\)是齐次线性方程组\(\boldsymbol{A}\boldsymbol{x}=\boldsymbol{0}\)的解空间,则向量组\(\boldsymbol{\alpha}_1,\cdots,\boldsymbol{\alpha}_{n - r}\)与\(\boldsymbol{\beta}_1,\cdots,\boldsymbol{\beta}_{n - r}\)是\(U\)的两组基. 令\(\boldsymbol{P}\)是这两组基之间的过渡矩阵,则
\[
(\boldsymbol{\beta}_1,\cdots,\boldsymbol{\beta}_{n - r})=(\boldsymbol{\alpha}_1,\cdots,\boldsymbol{\alpha}_{n - r})\boldsymbol{P}. 
\]  
\end{proof}

\begin{theorem}\label{theorem:矩阵方程有解的充要条件}
\begin{enumerate}
\item 设\(\boldsymbol{A},\boldsymbol{B}\)为\(m\times n\)和\(m\times p\)矩阵,\(\boldsymbol{X}\)为\(n\times p\)未知矩阵,证明:矩阵方程\(\boldsymbol{A}\boldsymbol{X}=\boldsymbol{B}\)有解的充要条件是\(\mathrm{r}\left( \boldsymbol{A}\,\,\boldsymbol{B} \right)=\mathrm{r}(\boldsymbol{A})\).

\item 设\(\boldsymbol{A},\boldsymbol{B}\)为\(m\times n\)和\(p\times n\)矩阵,\(\boldsymbol{X}\)为\(p\times m\)未知矩阵,证明:矩阵方程\(\boldsymbol{X}\boldsymbol{A}=\boldsymbol{B}\)有解的充要条件是\(\mathrm{r}\left( \begin{array}{c}
\boldsymbol{A}\\
\boldsymbol{B}\\
\end{array} \right) =\mathrm{r}\left( \boldsymbol{A} \right) \).
\end{enumerate}
\end{theorem}
\begin{proof}
\begin{enumerate}
\item 设\(\boldsymbol{A}=(\boldsymbol{\alpha}_1,\cdots,\boldsymbol{\alpha}_n),\boldsymbol{B}=(\boldsymbol{\beta}_1,\cdots,\boldsymbol{\beta}_p),\boldsymbol{X}=(\boldsymbol{x}_1,\cdots,\boldsymbol{x}_p)\)为对应的列分块. 设\(\mathrm{r}(\boldsymbol{A}) = r\)且\(\boldsymbol{\alpha}_{i_1},\cdots,\boldsymbol{\alpha}_{i_r}\)是\(\boldsymbol{A}\)的列向量的极大无关组. 注意到矩阵方程\(\boldsymbol{A}\boldsymbol{X}=\boldsymbol{B}\)有解当且仅当\(p\)个线性方程组\(\boldsymbol{A}\boldsymbol{x}_i=\boldsymbol{\beta}_i(1\leq i\leq p)\)都有解. 

因此,若\(\boldsymbol{A}\boldsymbol{X}=\boldsymbol{B}\)有解,则每个\(\boldsymbol{\beta}_i\)都是\(\boldsymbol{A}\)的列向量的线性组合,从而是\(\boldsymbol{\alpha}_{i_1},\cdots,\boldsymbol{\alpha}_{i_r}\)的线性组合,于是\(\boldsymbol{\alpha}_{i_1},\cdots,\boldsymbol{\alpha}_{i_r}\)是\(\left( \boldsymbol{A}\,|\,\boldsymbol{B} \right)\)的列向量的极大无关组,故\(\mathrm{r}\left( \boldsymbol{A}\,|\,\boldsymbol{B} \right) = r\). 

反之,若\(\mathrm{r}\left( \boldsymbol{A}\,|\,\boldsymbol{B} \right) = r\),则由\hyperref[proposition:极大无关组的判定条件]{命题\ref{proposition:极大无关组的判定条件}(1)}可知,\(\boldsymbol{\alpha}_{i_1},\cdots,\boldsymbol{\alpha}_{i_r}\)是\(\left( \boldsymbol{A}\,|\,\boldsymbol{B} \right)\)的列向量的极大无关组,于是每个\(\boldsymbol{\beta}_i\)都是\(\boldsymbol{A}\)的列向量的线性组合,从而\(\boldsymbol{A}\boldsymbol{X}=\boldsymbol{B}\)有解.

\item 线性方程组\(\boldsymbol{XA}=\boldsymbol{B}\)两边取转置可得\(\boldsymbol{A}^{\prime}\boldsymbol{X}^{\prime}=\boldsymbol{B}^{\prime}\), 从而
\[
\text{线性方程组}\boldsymbol{XA}=\boldsymbol{B}\text{有解}\Leftrightarrow\text{线性方程组}\boldsymbol{A}^{\prime}\boldsymbol{Y}=\boldsymbol{B}^{\prime}\text{有解}.
\]
又由第一问可知
\begin{align*}
\text{线性方程组}\boldsymbol{A}^{\prime}\boldsymbol{Y}=\boldsymbol{B}^{\prime}\text{有解}\Leftrightarrow \mathrm{r}\begin{pmatrix}\boldsymbol{A}^{\prime}&\boldsymbol{B}^{\prime}\end{pmatrix}=\mathrm{r}(\boldsymbol{A}^{\prime}).
\end{align*}
而\(\mathrm{r}(\boldsymbol{A}^{\prime})=\mathrm{r}(\boldsymbol{A})\),\(\mathrm{r}\begin{pmatrix}\boldsymbol{A}^{\prime}&\boldsymbol{B}^{\prime}\end{pmatrix}=\mathrm{r}\left(\begin{pmatrix}\boldsymbol{A}^{\prime}&\boldsymbol{B}^{\prime}\end{pmatrix}^{\prime}\right)=\mathrm{r}\begin{pmatrix}\boldsymbol{A}\\\boldsymbol{B}\end{pmatrix}\), 故
\begin{align*}
\text{线性方程组}\boldsymbol{XA}=\boldsymbol{B}\text{有解}\Leftrightarrow \mathrm{r}\begin{pmatrix}\boldsymbol{A}\\\boldsymbol{B}\end{pmatrix}=\mathrm{r}(\boldsymbol{A}).
\end{align*} 
\end{enumerate}
\end{proof}

\begin{proposition}
矩阵方程\(\boldsymbol{A}\boldsymbol{X}=\boldsymbol{B}\)有解当且仅当\(p\)个线性方程组\(\boldsymbol{A}\boldsymbol{x}_i=\boldsymbol{\beta}_i(1\leq i\leq p)\)都有解.从而每个\(\boldsymbol{\beta}_i\)都是\(\boldsymbol{A}\)的列向量的线性组合.
\end{proposition}
\begin{proof}
证明是显然的.
\end{proof}

\begin{proposition}
设\(\boldsymbol{A},\boldsymbol{B}\)为\(m\times n\)和\(n\times p\)矩阵,证明:存在\(p\times n\)矩阵\(\boldsymbol{C}\),使得\(\boldsymbol{A}\boldsymbol{B}\boldsymbol{C}=\boldsymbol{A}\)的充要条件是\(\mathrm{r}(\boldsymbol{A})=\mathrm{r}(\boldsymbol{A}\boldsymbol{B})\).
\end{proposition}
\begin{proof}
必要性由秩的不等式\(\mathrm{r}(\boldsymbol{A})\geq\mathrm{r}(\boldsymbol{A}\boldsymbol{B})\geq\mathrm{r}(\boldsymbol{A}\boldsymbol{B}\boldsymbol{C})=\mathrm{r}(\boldsymbol{A})\)即得. 

充分性由秩的不等式可知\(\mathrm{r}\left( \boldsymbol{A} \right) =\mathrm{r}\left( \boldsymbol{AB} \right) \le \mathrm{r}\left( \boldsymbol{AB}\,|\,\boldsymbol{B} \right) =\mathrm{r}\left( \boldsymbol{A}\left( \boldsymbol{B}\,|\,\boldsymbol{I}_n \right) \right) \le \mathrm{r}\left( \boldsymbol{A} \right)\).故$\mathrm{r}\left( \boldsymbol{A} \right) =\mathrm{r}\left( \boldsymbol{AB}\,|\,\boldsymbol{B} \right)$.于是由
\hyperref[theorem:矩阵方程有解的充要条件]{定理\ref{theorem:矩阵方程有解的充要条件}}可知,矩阵方程$\boldsymbol{ABX}=\boldsymbol{A}$有解.即存在\(p\times n\)矩阵\(\boldsymbol{C}\),使得\(\boldsymbol{A}\boldsymbol{B}\boldsymbol{C}=\boldsymbol{A}\)的充要条件是\(\mathrm{r}(\boldsymbol{A})=\mathrm{r}(\boldsymbol{A}\boldsymbol{B})\).
\end{proof}

\subsection{线性方程组的公共解}

对两个非齐次线性方程组,若只已知它们的通解,而不知道方程组本身,要求它们的公共解,我们可以这样来做:
设\(\boldsymbol{A}\boldsymbol{x}=\boldsymbol{\beta}_1,\boldsymbol{B}\boldsymbol{x}=\boldsymbol{\beta}_2\)是两个含\(n\)个未知数的非齐次线性方程组. 方程组\(\boldsymbol{A}\boldsymbol{x}=\boldsymbol{\beta}_1\)有特解\(\boldsymbol{\gamma}\)且\(\boldsymbol{A}\boldsymbol{x}=\boldsymbol{0}\)的基础解系为\(\boldsymbol{\eta}_1,\cdots,\boldsymbol{\eta}_{n - r}\). 方程组\(\boldsymbol{B}\boldsymbol{x}=\boldsymbol{\beta}_2\)有特解\(\boldsymbol{\delta}\)且\(\boldsymbol{B}\boldsymbol{x}=\boldsymbol{0}\)的基础解系为\(\boldsymbol{\xi}_1,\cdots,\boldsymbol{\xi}_{n - s}\).

{\color{blue}方法一:} 假设它们的公共解为\(\boldsymbol{\gamma}+t_1\boldsymbol{\eta}_1+\cdots+t_{n - r}\boldsymbol{\eta}_{n - r}\),则\(\boldsymbol{\gamma}+t_1\boldsymbol{\eta}_1+\cdots+t_{n - r}\boldsymbol{\eta}_{n - r}-\boldsymbol{\delta}\)是\(\boldsymbol{B}\boldsymbol{x}=\boldsymbol{0}\)的解,因此可以表示为\(\boldsymbol{\xi}_1,\cdots,\boldsymbol{\xi}_{n - s}\)的线性组合. 于是矩阵\((\boldsymbol{\xi}_1,\cdots,\boldsymbol{\xi}_{n - s},\boldsymbol{\gamma}+t_1\boldsymbol{\eta}_1+\cdots+t_{n - r}\boldsymbol{\eta}_{n - r}-\boldsymbol{\delta})\)的秩等于\(n - s\). 由此可以求出\(t_1,\cdots,t_{n - r}\),从而求出公共解.

{\color{blue}方法二:}假设它们的公共解为\(\boldsymbol{\zeta}\),则
\[
\boldsymbol{\zeta}=\boldsymbol{\gamma}+t_1\boldsymbol{\eta}_1+\cdots+t_{n - r}\boldsymbol{\eta}_{n - r}=\boldsymbol{\delta}+(-u_1)\boldsymbol{\xi}_1+\cdots+(-u_{n - s})\boldsymbol{\xi}_{n - s}.
\]
要求公共解\(\boldsymbol{\zeta}\)等价于求解下列关于未定元\(t_1,\cdots,t_{n - r};u_1,\cdots,u_{n - s}\)的线性方程组:
\[
t_1\boldsymbol{\eta}_1+\cdots+t_{n - r}\boldsymbol{\eta}_{n - r}+u_1\boldsymbol{\xi}_1+\cdots+u_{n - s}\boldsymbol{\xi}_{n - s}=\boldsymbol{\delta}-\boldsymbol{\gamma}.
\]

\begin{example}
设有两个非齐次线性方程组(I), (II),它们的通解分别为
\[
\boldsymbol{\gamma}+t_1\boldsymbol{\eta}_1 + t_2\boldsymbol{\eta}_2;\boldsymbol{\delta}+k_1\boldsymbol{\xi}_1 + k_2\boldsymbol{\xi}_2,
\]
其中\(\boldsymbol{\gamma}=(5,-3,0,0)',\boldsymbol{\eta}_1=(-6,5,1,0)',\boldsymbol{\eta}_2=(-5,4,0,1)';\boldsymbol{\delta}=(-11,3,0,0)',\boldsymbol{\xi}_1=(8,-1,1,0)',\boldsymbol{\xi}_2=(10,-2,0,1)'\). 求这两个方程组的公共解.
\end{example}
\begin{proof}
{\color{blue}解法一:} 设公共解为
\[
\boldsymbol{\gamma}+t_1\boldsymbol{\eta}_1 + t_2\boldsymbol{\eta}_2=\begin{pmatrix}
5 - 6t_1 - 5t_2\\
-3 + 5t_1 + 4t_2\\
t_1\\
t_2
\end{pmatrix}.
\]
注意矩阵\((\boldsymbol{\xi}_1,\boldsymbol{\xi}_2,\boldsymbol{\gamma}-\boldsymbol{\delta}+t_1\boldsymbol{\eta}_1 + t_2\boldsymbol{\eta}_2)\)的秩等于2,对此矩阵作初等行变换:
\[
\begin{pmatrix}
8&10&16 - 6t_1 - 5t_2\\
-1&-2&-6 + 5t_1 + 4t_2\\
1&0&t_1\\
0&1&t_2
\end{pmatrix}\to\begin{pmatrix}
1&0&t_1\\
0&1&t_2\\
8&10&16 - 6t_1 - 5t_2\\
-1&-2&-6 + 5t_1 + 4t_2
\end{pmatrix}\to
\begin{pmatrix}
1&0&t_1\\
0&1&t_2\\
0&0&16 - 14t_1 - 15t_2\\
0&0&-6 + 6t_1 + 6t_2
\end{pmatrix}
\]
可得关于\(t_1,t_2\)的方程组
\[
\begin{cases}
14t_1 + 15t_2 = 16,\\
6t_1 + 6t_2 = 6.
\end{cases}
\]
解得\(t_1=-1,t_2 = 2\),所以公共解为 (只有一个向量)\(\boldsymbol{\gamma}-\boldsymbol{\eta}_1 + 2\boldsymbol{\eta}_2=(1,0,-1,2)'\).

{\color{blue}解法二:} 求公共解等价于求解下列线性方程组:
\[
t_1\boldsymbol{\eta}_1 + t_2\boldsymbol{\eta}_2 + u_1\boldsymbol{\xi}_1 + u_2\boldsymbol{\xi}_2=\boldsymbol{\delta}-\boldsymbol{\gamma}.
\]
对其增广矩阵实施初等行变换,可得
\[
\begin{pmatrix}
-6&-5&8&10&-16\\
5&4&-1&-2&6\\
1&0&1&0&0\\
0&1&0&1&0
\end{pmatrix}\to\begin{pmatrix}
1&0&0&0&-1\\
0&1&0&0&2\\
0&0&1&0&1\\
0&0&0&1&-2
\end{pmatrix},
\]
故\((t_1,t_2,u_1,u_2)\)只有唯一解\((-1,2,1,-2)\). 因此,公共解为\(\boldsymbol{\gamma}-\boldsymbol{\eta}_1 + 2\boldsymbol{\eta}_2=\boldsymbol{\delta}-\boldsymbol{\xi}_1 + 2\boldsymbol{\xi}_2=(1,0,-1,2)'\).
\end{proof}

\begin{example}
设有非齐次线性方程组 (I):
\[
\begin{cases}
7x_1 - 6x_2 + 3x_3 = b,\\
8x_1 - 9x_2 + ax_4 = 7.
\end{cases}
\]
又已知方程组 (II) 的通解为
\[
(1,1,1,0)' + t_1(1,0,-1,0)' + t_2(2,3,0,1)'.
\]
若这两个方程组有无穷多组公共解,求出 \(a,b\) 的值并求出公共解.
\end{example}
\begin{proof}
将 (II) 的通解写为 \((1 + t_1 + 2t_2,1 + 3t_2,-t_1,t_2)'\),代入方程组 (I) 化简得到
\[
\begin{cases}
4t_1 - 4t_2 = b - 1,\\
8t_1 + (a - 11)t_2 = 8.
\end{cases}
\]
要使这两个方程组有无穷多组公共解,\(t_1,t_2\) 必须有无穷多组解,于是上面方程组的系数矩阵和增广矩阵的秩都应该等于 1,从而 \(a = 3,b = 5\). 解出方程组得到 \(t_1 = t_2 + 1\),因此方程组 (I), (II) 的公共解为
\[
(1 + t_1 + 2t_2,1 + 3t_2,-t_1,t_2)' = (2,1,-1,0)' + t_2(3,3,-1,1)',
\]
其中 \(t_2\) 为任意数.
\end{proof}

\subsection{在解析几何上的应用}

\begin{proposition}\label{proposition:平面上点位于同一条直线上的充要条件}
求平面上 \(n\) 个点 \((x_1,y_1),(x_2,y_2),\cdots,(x_n,y_n)\) 位于同一条直线上的充要条件.
\end{proposition}
\begin{proof}
充要条件为第一个点和其余点代表的向量之差属于一个一维子空间,即 \((x_i - x_1,y_i - y_1)\) 都成比例. 写成矩阵形式为
\[
\mathrm{r}\begin{pmatrix}
x_2 - x_1&x_3 - x_1&\cdots&x_n - x_1\\
y_2 - y_1&y_3 - y_1&\cdots&y_n - y_1
\end{pmatrix}\leq1,
\]
或
\[
\mathrm{r}\begin{pmatrix}
x_1&x_2&x_3&\cdots&x_n\\
y_1&y_2&y_3&\cdots&y_n\\
1&1&1&\cdots&1
\end{pmatrix}\leq2. 
\]
\end{proof}

\begin{proposition}\label{proposition:三维实空间中4点共面}
求三维实空间中 4 点 \((x_i,y_i,z_i)(1\leq i\leq 4)\) 共面的充要条件.
\end{proposition}
\begin{proof}
设 4 点的向量为 \(\boldsymbol{\alpha}_1,\boldsymbol{\alpha}_2,\boldsymbol{\alpha}_3,\boldsymbol{\alpha}_4\),则 4 点共面的充要条件是:向量组 \(\boldsymbol{\alpha}_2 - \boldsymbol{\alpha}_1,\boldsymbol{\alpha}_3 - \boldsymbol{\alpha}_1,\boldsymbol{\alpha}_4 - \boldsymbol{\alpha}_1\) 的秩不超过 2. 不难将此写成矩阵形式:
\[
\mathrm{r}\begin{pmatrix}
x_2 - x_1&x_3 - x_1&x_4 - x_1\\
y_2 - y_1&y_3 - y_1&y_4 - y_1\\
z_2 - z_1&z_3 - z_1&z_4 - z_1
\end{pmatrix}\leq2,
\]
或
\[
\mathrm{r}\begin{pmatrix}
x_1&x_2&x_3&x_4\\
y_1&y_2&y_3&y_4\\
z_1&z_2&z_3&z_4\\
1&1&1&1
\end{pmatrix}\leq3. 
\]
\end{proof}

\begin{example}\label{example:3.35115}
证明: 通过平面内不在一条直线上的 3 点 \((x_1,y_1),(x_2,y_2),(x_3,y_3)\) 的圆方程为
\[
\begin{vmatrix}
x^2 + y^2&x&y&1\\
x_1^2 + y_1^2&x_1&y_1&1\\
x_2^2 + y_2^2&x_2&y_2&1\\
x_3^2 + y_3^2&x_3&y_3&1
\end{vmatrix}=0.
\]
\end{example}
\begin{proof}
圆方程可设为
\[
u_1(x^2 + y^2)+u_2x + u_3y + u_4 = 0,
\]
于是得到未知数 \(u_1,u_2,u_3,u_4\) 的方程组为
\[
\begin{cases}
(x_1^2 + y_1^2)u_1 + x_1u_2 + y_1u_3 + u_4 = 0,\\
(x_2^2 + y_2^2)u_1 + x_2u_2 + y_2u_3 + u_4 = 0,\\
(x_3^2 + y_3^2)u_1 + x_3u_2 + y_3u_3 + u_4 = 0.
\end{cases}
\]
上述方程组加上原方程组成一个含 4 个未知数、4 个方程式的齐次线性方程组,它有非零解的充要条件是系数行列式等于零,即
\[
\begin{vmatrix}
x^2 + y^2&x&y&1\\
x_1^2 + y_1^2&x_1&y_1&1\\
x_2^2 + y_2^2&x_2&y_2&1\\
x_3^2 + y_3^2&x_3&y_3&1
\end{vmatrix}=0.
\]
由\hyperref[proposition:平面上点位于同一条直线上的充要条件]{命题\ref{proposition:平面上点位于同一条直线上的充要条件}}可知 3 点不在一条直线上意味着
\[
\begin{vmatrix}
x_1&y_1&1\\
x_2&y_2&1\\
x_3&y_3&1
\end{vmatrix}\neq0,
\]
故圆方程不退化.  
\end{proof}

\begin{proposition}\label{proposition:平面上4点共圆的充要条件}
求平面上不在一条直线上的 4 个点 \((x_1,y_1),(x_2,y_2),(x_3,y_3),(x_4,y_4)\) 位于同一个圆上的充要条件.
\end{proposition}
\begin{proof}
由\hyperref[example:3.35115]{例题\ref{example:3.35115}}可得充要条件为
\[
\begin{vmatrix}
x_1^2 + y_1^2&x_1&y_1&1\\
x_2^2 + y_2^2&x_2&y_2&1\\
x_3^2 + y_3^2&x_3&y_3&1\\
x_4^2 + y_4^2&x_4&y_4&1
\end{vmatrix}=0.
\]
\end{proof}

\begin{example}
已知平面上两条不同的二次曲线 \(a_ix^2 + b_ixy + c_iy^2 + d_ix + e_iy + f_i = 0(i = 1,2)\) 交于 4 个不同的点 \((x_i,y_i)(1\leq i\leq 4)\). 求证: 过这 4 个点的二次曲线均可写为如下形状:
\[
\lambda_1(a_1x^2 + b_1xy + c_1y^2 + d_1x + e_1y + f_1)+\lambda_2(a_2x^2 + b_2xy + c_2y^2 + d_2x + e_2y + f_2)=0.
\]
\end{example}
\begin{proof}
显然上述曲线过这 4 个交点. 现设 \(ax^2 + bxy + cy^2 + dx + ey + f = 0\) 是过这 4 个交点的二次曲线,则有
\begin{align}\label{equation:3.363.3}
\begin{cases}
ax_1^2 + bx_1y_1 + cy_1^2 + dx_1 + ey_1 + f = 0,\\
ax_2^2 + bx_2y_2 + cy_2^2 + dx_2 + ey_2 + f = 0,\\
ax_3^2 + bx_3y_3 + cy_3^2 + dx_3 + ey_3 + f = 0,\\
ax_4^2 + bx_4y_4 + cy_4^2 + dx_4 + ey_4 + f = 0.
\end{cases}
\end{align}
视 \(a,b,c,d,e,f\) 为未知数,则线性方程组 \eqref{equation:3.363.3} 有线性无关的解 \((a_1,b_1,c_1,d_1,e_1,f_1)',(a_2,b_2,c_2,d_2,e_2,f_2)'\). 如果能证明方程组 \eqref{equation:3.363.3} 的系数矩阵的秩等于 4, 则这两个解就构成了基础解系,从而即得结论.

容易验证 4 个交点中的任意 3 个点都不共线,而且经过坐标轴适当的旋转,可以假设这 4 个交点的横坐标 \(x_1,x_2,x_3,x_4\) 互不相同. 用反证法证明结论,设方程组 \eqref{equation:3.363.3} 系数矩阵 \(\boldsymbol{A}\) 的秩小于 4. 由任意 3 个交点不共线以及\hyperref[proposition:平面上点位于同一条直线上的充要条件]{命题\ref{proposition:平面上点位于同一条直线上的充要条件}}知,\((x_1,x_2,x_3,x_4)',(y_1,y_2,y_3,y_4)',(1,1,1,1)'\) 线性无关,从而它们是 \(\boldsymbol{A}\) 的列向量的极大无关组,于是 \((x_1^2,x_2^2,x_3^2,x_4^2)'\) 是它们的线性组合,故可设 \(x_i^2 = rx_i + sy_i + t(1\leq i\leq 4)\),其中 \(r,s,t\) 是实数. 由于 \(x_1,x_2,x_3,x_4\) 互不相同,故 \(s\neq 0\),于是 \(y_i = \frac{1}{s}x_i^2 - \frac{r}{s}x_i - \frac{t}{s}(1\leq i\leq 4)\). 考虑 \(\boldsymbol{A}\) 的第一列、第二列、第四列和第六列构成的四阶行列式 \(|\boldsymbol{B}|\),利用 Vander - monde 行列式容易算出 \(|\boldsymbol{B}|=-\frac{1}{s}\prod_{1\leq i < j\leq 4}(x_i - x_j)\neq 0\),于是 \(\boldsymbol{A}\) 的秩等于 4, 这与假设矛盾. 因此方程组 \eqref{equation:3.363.3} 的系数矩阵的秩只能等于 4.
\end{proof}


\end{document}
% contents/chapter-03/section-01.tex 第三章第一节
\documentclass[../../main.tex]{subfiles}
\graphicspath{{\subfix{../../image/}}} % 指定图片目录,后续可以直接使用图片文件名。

% 例如:
% \begin{figure}[H]
% \centering
% \includegraphics[scale=0.4]{image-01.01}
% \caption{图片标题}
% \label{figure:image-01.01}
% \end{figure}
% 注意:上述\label{}一定要放在\caption{}之后,否则引用图片序号会只会显示??.

\begin{document}

\section{向量的线性关系}

\begin{theorem}\label{theorem:向量的线性关系定理1}
设\(\boldsymbol{\alpha}_1,\boldsymbol{\alpha}_2,\cdots,\boldsymbol{\alpha}_m,\boldsymbol{\beta}\)是线性空间\(V\)中的向量.

(1) 若\(\boldsymbol{\alpha}_1,\boldsymbol{\alpha}_2,\cdots,\boldsymbol{\alpha}_m\)线性相关,则任意一组包含这组向量的向量组必线性相关;若\(\boldsymbol{\alpha}_1,\boldsymbol{\alpha}_2,\cdots,\boldsymbol{\alpha}_m\)线性无关,则从这组向量中任意取出一组向量必线性无关.

(2) 向量组\(\boldsymbol{\alpha}_1,\boldsymbol{\alpha}_2,\cdots,\boldsymbol{\alpha}_m\)线性相关的充要条件是其中至少有一个向量可以表示为其余向量的线性组合.

(3) 若\(\boldsymbol{\beta}\)可表示为\(\boldsymbol{\alpha}_1,\boldsymbol{\alpha}_2,\cdots,\boldsymbol{\alpha}_m\)的线性组合,即
\[
\boldsymbol{\beta}=k_1\boldsymbol{\alpha}_1 + k_2\boldsymbol{\alpha}_2+\cdots + k_m\boldsymbol{\alpha}_m,
\]
则表示唯一的充要条件是向量\(\boldsymbol{\alpha}_1,\boldsymbol{\alpha}_2,\cdots,\boldsymbol{\alpha}_m\)线性无关.
\end{theorem}
\begin{proof}

\end{proof}

\begin{theorem}\label{theorem:向量的线性关系定理2}
(1)设\(A,B\)是两组向量,\(A\)含有\(r\)个向量,\(B\)含有\(s\)个向量,且\(A\)中每个向量均可用\(B\)中向量线性表示. 若\(A\)中向量线性无关,则\(r\leq s\).

(2)设\(A,B\)是两组向量,\(A\)含有\(r\)个向量,\(B\)含有\(s\)个向量,且\(A\)中每个向量均可用\(B\)中向量线性表示. 若\(r > s\),则\(A\)中向量线性相关.
\end{theorem}
\begin{proof}
(2)是(1)的逆否命题,因此我们只证明(1).
\end{proof}



\begin{proposition}\label{proposition:阶梯形矩阵的非零行对应其列向量的极大无关组}
设\(A\)是\(m\times n\)阶梯形矩阵,证明:\(A\)的秩等于其非零行的个数,且阶梯点所在的列向量是\(A\)的列向量的极大无关组.
\end{proposition}
\begin{proof}
设
\[
A=\left( \begin{matrix}
0&		\cdots&		a_{1k_1}&		\cdots&		\cdots&		\cdots&		\cdots&		\cdots\\
0&		\cdots&		0&		\cdots&		a_{2k_2}&		\cdots&		\cdots&		\cdots\\
\vdots&		&		\vdots&		&		\vdots&		&		\vdots&		\vdots\\
0&		\cdots&		0&		\cdots&		0&		\cdots&		a_{rk_r}&		\cdots\\
0&		\cdots&		0&		\cdots&		0&		\cdots&		0&		\cdots\\
\vdots&		&		\vdots&		&		\vdots&		&		\vdots&		\\
0&		\cdots&		0&		\cdots&		0&		\cdots&		0&		\cdots\\
\end{matrix} \right) .
\]
其中\(a_{1k_1},a_{2k_2},\cdots,a_{rk_r}\)是\(A\)的阶梯点. 设\(\boldsymbol{\alpha}_1,\boldsymbol{\alpha}_2,\cdots,\boldsymbol{\alpha}_r\)是\(A\)的前\(r\)行,我们先证明它们线性无关. 设
\[
c_1\boldsymbol{\alpha}_1 + c_2\boldsymbol{\alpha}_2+\cdots + c_r\boldsymbol{\alpha}_r=\boldsymbol{0}
\]
其中\(c_1,c_2,\cdots,c_r\)是常数. 上式是关于\(n\)维行向量的等式,先考察行向量的第\(k_1\)分量,可得\(c_1a_{1k_1}=0\). 因为\(a_{1k_1}\neq0\),故\(c_1 = 0\);再依次考察行向量的第\(k_2,\cdots,k_r\)分量,最后可得\(c_1 = c_2=\cdots = c_r = 0\). 因此\(\boldsymbol{\alpha}_1,\boldsymbol{\alpha}_2,\cdots,\boldsymbol{\alpha}_r\)线性无关,从而\(A\)的秩等于\(r\),即其非零行的个数.

再将\(r\)个阶梯点所在的列向量取出,拼成一个新的矩阵:
\[
B=\left( \begin{matrix}
a_{1k_1}&		\cdots&		\cdots&		\cdots\\
0&		a_{2k_2}&		\cdots&		\cdots\\
\vdots&		\vdots&		&		\vdots\\
0&		0&		\cdots&		a_{rk_r}\\
0&		0&		\cdots&		0\\
\vdots&		\vdots&		&		\vdots\\
0&		0&		\cdots&		0\\
\end{matrix} \right) .
\]
采用相同的方法可证明矩阵\(B\)的前\(r\)行线性无关,因此\(\text{r}(B)=r\),从而阶梯点所在的列向量组的秩也等于\(r\). 又因为\(\text{r}(A)=r\),故它们是\(A\)的列向量的极大无关组.
\end{proof}

\begin{proposition}\label{proposition:乘可逆矩阵不改变向量组的极大无关组的位置}
设\(A = (\boldsymbol{\alpha}_1,\boldsymbol{\alpha}_2,\cdots,\boldsymbol{\alpha}_n)\)是一个\(m\times n\)矩阵,\(\boldsymbol{\alpha}_1,\boldsymbol{\alpha}_2,\cdots,\boldsymbol{\alpha}_n\)是列向量. \(P\)是一个\(m\)阶可逆矩阵,\(B = PA=(\boldsymbol{\beta}_1,\boldsymbol{\beta}_2,\cdots,\boldsymbol{\beta}_n)\),其中\(\boldsymbol{\beta}_j = P\boldsymbol{\alpha}_j(1\leq j\leq n)\). 证明:若\(\boldsymbol{\alpha}_{i_1},\boldsymbol{\alpha}_{i_2},\cdots,\boldsymbol{\alpha}_{i_r}\)是\(A\)的列向量的极大无关组,则\(\boldsymbol{\beta}_{i_1},\boldsymbol{\beta}_{i_2},\cdots,\boldsymbol{\beta}_{i_r}\)是\(B\)的列向量的极大无关组.
\end{proposition}
\begin{proof}
先证明向量组\(\boldsymbol{\beta}_{i_1},\boldsymbol{\beta}_{i_2},\cdots,\boldsymbol{\beta}_{i_r}\)线性无关. 设
\[
c_1\boldsymbol{\beta}_{i_1}+c_2\boldsymbol{\beta}_{i_2}+\cdots + c_r\boldsymbol{\beta}_{i_r}=\boldsymbol{0},
\]
即
\[
c_1P\boldsymbol{\alpha}_{i_1}+c_2P\boldsymbol{\alpha}_{i_2}+\cdots + c_rP\boldsymbol{\alpha}_{i_r}=\boldsymbol{0}.
\]
已知\(P\)是可逆矩阵,因此
\[
c_1\boldsymbol{\alpha}_{i_1}+c_2\boldsymbol{\alpha}_{i_2}+\cdots + c_r\boldsymbol{\alpha}_{i_r}=\boldsymbol{0}.
\]
而向量组\(\boldsymbol{\alpha}_{i_1},\boldsymbol{\alpha}_{i_2},\cdots,\boldsymbol{\alpha}_{i_r}\)线性无关,故\(c_1 = c_2=\cdots = c_r = 0\),这证明了向量组\(\boldsymbol{\beta}_{i_1},\boldsymbol{\beta}_{i_2},\cdots,\boldsymbol{\beta}_{i_r}\)线性无关. 要证这是\(B\)的列向量的极大无关组,只需证明\(B\)的任意一个列向量都是这些向量的线性组合即可. 设\(\boldsymbol{\beta}_j\)是\(B\)的任意一个列向量,则\(\boldsymbol{\beta}_j = P\boldsymbol{\alpha}_j\). 因为\(\boldsymbol{\alpha}_{i_1},\boldsymbol{\alpha}_{i_2},\cdots,\boldsymbol{\alpha}_{i_r}\)是\(A\)的列向量的极大无关组,故\(\boldsymbol{\alpha}_j\)可用\(\boldsymbol{\alpha}_{i_1},\boldsymbol{\alpha}_{i_2},\cdots,\boldsymbol{\alpha}_{i_r}\)线性表示. 不妨设
\[
\boldsymbol{\alpha}_j = b_1\boldsymbol{\alpha}_{i_1}+b_2\boldsymbol{\alpha}_{i_2}+\cdots + b_r\boldsymbol{\alpha}_{i_r},
\]
则
\[
P\boldsymbol{\alpha}_j = b_1P\boldsymbol{\alpha}_{i_1}+b_2P\boldsymbol{\alpha}_{i_2}+\cdots + b_rP\boldsymbol{\alpha}_{i_r},
\]
即
\[
\boldsymbol{\beta}_j = b_1\boldsymbol{\beta}_{i_1}+b_2\boldsymbol{\beta}_{i_2}+\cdots + b_r\boldsymbol{\beta}_{i_r}.
\]
\end{proof}

\begin{corollary}\label{corollary:线性无关向量组乘可逆矩阵仍然线性无关}
设\(n\)维向量\(\boldsymbol{\alpha}_1,\boldsymbol{\alpha}_2,\cdots,\boldsymbol{\alpha}_m\)线性无关,\(A\)为\(n\)阶可逆矩阵,求证:\(A\boldsymbol{\alpha}_1,A\boldsymbol{\alpha}_2,\cdots,A\boldsymbol{\alpha}_m\)线性无关.
\end{corollary}
\begin{proof}
由\hyperref[proposition:乘可逆矩阵不改变向量组的极大无关组的位置]{命题\ref{proposition:乘可逆矩阵不改变向量组的极大无关组的位置}}即得.
\end{proof}

\begin{proposition}\label{proposition:线性无关向量组的命题1}
设\(\boldsymbol{\alpha}_1,\boldsymbol{\alpha}_2,\cdots,\boldsymbol{\alpha}_m\)是线性空间\(V\)中一组线性无关的向量,\(\boldsymbol{\beta}\)是\(V\)中的向量. 求证:或者\(\boldsymbol{\alpha}_1,\boldsymbol{\alpha}_2,\cdots,\boldsymbol{\alpha}_m,\boldsymbol{\beta}\)线性无关,或者\(\boldsymbol{\beta}\)是\(\boldsymbol{\alpha}_1,\boldsymbol{\alpha}_2,\cdots,\boldsymbol{\alpha}_m\)的线性组合.
\end{proposition}
\begin{proof}
若\(\boldsymbol{\alpha}_1,\boldsymbol{\alpha}_2,\cdots,\boldsymbol{\alpha}_m,\boldsymbol{\beta}\)线性无关,则结论得证. 若\(\boldsymbol{\alpha}_1,\boldsymbol{\alpha}_2,\cdots,\boldsymbol{\alpha}_m,\boldsymbol{\beta}\)线性相关,则存在不全为零的数\(c_1,c_2,\cdots,c_m,d\),使得
\[
c_1\boldsymbol{\alpha}_1 + c_2\boldsymbol{\alpha}_2+\cdots + c_m\boldsymbol{\alpha}_m + d\boldsymbol{\beta}=\boldsymbol{0}.
\]
若\(d = 0\),则\(c_1,c_2,\cdots,c_m\)不全为零且\(c_1\boldsymbol{\alpha}_1 + c_2\boldsymbol{\alpha}_2+\cdots + c_m\boldsymbol{\alpha}_m=\boldsymbol{0}\),这与\(\boldsymbol{\alpha}_1,\boldsymbol{\alpha}_2,\cdots,\boldsymbol{\alpha}_m\)线性无关矛盾. 因此\(d\neq0\),从而
\[
\boldsymbol{\beta}=-\frac{c_1}{d}\boldsymbol{\alpha}_1-\frac{c_2}{d}\boldsymbol{\alpha}_2-\cdots-\frac{c_m}{d}\boldsymbol{\alpha}_m,
\]
即\(\boldsymbol{\beta}\)是\(\boldsymbol{\alpha}_1,\boldsymbol{\alpha}_2,\cdots,\boldsymbol{\alpha}_m\)的线性组合.
\end{proof}

\begin{corollary}\label{corollary:线性无关向量组的命题1}
若\(\boldsymbol{\alpha}_1,\boldsymbol{\alpha}_2,\cdots,\boldsymbol{\alpha}_m\)线性无关且\(\boldsymbol{\beta}\notin L(\boldsymbol{\alpha}_1,\boldsymbol{\alpha}_2,\cdots,\boldsymbol{\alpha}_m)\),则\(\boldsymbol{\alpha}_1,\boldsymbol{\alpha}_2,\cdots,\boldsymbol{\alpha}_m,\boldsymbol{\beta}\)线性无关. 
\end{corollary}
\begin{note}
这个推论与\hyperref[proposition:线性无关向量组的命题1]{上一个命题\ref{proposition:线性无关向量组的命题1}}等价.虽然这个等价命题很简单,但后面经常会用到.
\end{note}

\begin{proposition}\label{proposition:线性无关向量组的命题2}
设向量\(\boldsymbol{\beta}\)可由向量\(\boldsymbol{\alpha}_1,\boldsymbol{\alpha}_2,\cdots,\boldsymbol{\alpha}_m\)线性表示,但不能由其中任何个数少于\(m\)的部分向量线性表示,则这\(m\)个向量线性无关.
\end{proposition}
\begin{proof}
用反证法,设\(\boldsymbol{\alpha}_1,\boldsymbol{\alpha}_2,\cdots,\boldsymbol{\alpha}_m\)线性相关,则至少有一个向量是其余向量的线性组合. 不妨设\(\boldsymbol{\alpha}_m\)是\(\boldsymbol{\alpha}_1,\boldsymbol{\alpha}_2,\cdots,\boldsymbol{\alpha}_{m - 1}\)的线性组合,则由线性组合的传递性可知,\(\boldsymbol{\beta}\)也是\(\boldsymbol{\alpha}_1,\boldsymbol{\alpha}_2,\cdots,\boldsymbol{\alpha}_{m - 1}\)的线性组合,这与假设矛盾. 
\end{proof}

\begin{proposition}\label{proposition:线性无关向量组的命题3}
设线性空间\(V\)中向量\(\boldsymbol{\alpha}_1,\boldsymbol{\alpha}_2,\cdots,\boldsymbol{\alpha}_r\)线性无关,已知有序向量组\(\{\boldsymbol{\beta},\boldsymbol{\alpha}_1,\boldsymbol{\alpha}_2,\cdots,\boldsymbol{\alpha}_r\}\)线性相关,求证:最多只有一个\(\boldsymbol{\alpha}_i\)可以表示为前面向量的线性组合.
\end{proposition}
\begin{proof}
用反证法,设存在\(1\leq i<j\leq r\),使得
\begin{align*}
\boldsymbol{\alpha}_i&=b\boldsymbol{\beta}+a_1\boldsymbol{\alpha}_1 + a_2\boldsymbol{\alpha}_2+\cdots + a_{i - 1}\boldsymbol{\alpha}_{i - 1},\\
\boldsymbol{\alpha}_j&=d\boldsymbol{\beta}+c_1\boldsymbol{\alpha}_1 + c_2\boldsymbol{\alpha}_2+\cdots + c_{j - 1}\boldsymbol{\alpha}_{j - 1}.
\end{align*}
由于\(\boldsymbol{\alpha}_1,\boldsymbol{\alpha}_2,\cdots,\boldsymbol{\alpha}_r\)线性无关,故\(b\neq0\).若$b=0$,则\(\boldsymbol{\alpha}_i\)是\(\boldsymbol{\alpha}_1,\boldsymbol{\alpha}_2,\cdots,\boldsymbol{\alpha}_{i - 1}\)的线性组合,这与\(\boldsymbol{\alpha}_1,\boldsymbol{\alpha}_2,\cdots,\boldsymbol{\alpha}_r\)线性无关矛盾.将第一个等式乘以\(-\frac{d}{b}\)加到第二个等式上,可得\(\boldsymbol{\alpha}_j\)是\(\boldsymbol{\alpha}_1,\boldsymbol{\alpha}_2,\cdots,\boldsymbol{\alpha}_{j - 1}\)的线性组合,这与\(\boldsymbol{\alpha}_1,\boldsymbol{\alpha}_2,\cdots,\boldsymbol{\alpha}_r\)线性无关矛盾.
\end{proof}

\begin{proposition}\label{proposition:线性无关向量组的命题4}
设\(A\)是\(n\times m\)矩阵,\(B\)是\(m\times n\)矩阵,满足\(AB = I_n\),求证:\(B\)的\(n\)个列向量线性无关.
\end{proposition}
\begin{note}
实际上,由\(AB = I_n\)可知$A,B$互为逆矩阵,从而$A,B$满秩,结论得证.
\end{note}
\begin{proof}
设\(B = (\boldsymbol{\beta}_1,\boldsymbol{\beta}_2,\cdots,\boldsymbol{\beta}_n)\)为列分块,则\(AB=(A\boldsymbol{\beta}_1,A\boldsymbol{\beta}_2,\cdots,A\boldsymbol{\beta}_n)\). 由\(AB = I_n\)可得\(A\boldsymbol{\beta}_i=\boldsymbol{e}_i(1\leq i\leq n)\),其中\(\boldsymbol{e}_i\)是\(n\)维标准单位列向量. 设
\[
c_1\boldsymbol{\beta}_1 + c_2\boldsymbol{\beta}_2+\cdots + c_n\boldsymbol{\beta}_n=\boldsymbol{0},
\]
上式两边同时左乘\(A\),可得
\[
\boldsymbol{0}=c_1A\boldsymbol{\beta}_1 + c_2A\boldsymbol{\beta}_2+\cdots + c_nA\boldsymbol{\beta}_n=c_1\boldsymbol{e}_1 + c_2\boldsymbol{e}_2+\cdots + c_n\boldsymbol{e}_n=(c_1,c_2,\cdots,c_n)',
\]
因此\(c_1 = c_2=\cdots = c_n = 0\),即\(B\)的\(n\)个列向量\(\boldsymbol{\beta}_1,\boldsymbol{\beta}_2,\cdots,\boldsymbol{\beta}_n\)线性无关.
\end{proof}

\begin{proposition}[缩短向量与延伸向量]\label{proposition:线性相关向量组的缩短组也线性相关}
1.设\(\{\boldsymbol{\alpha}_i=(a_{i1},a_{i2},\cdots,a_{in}),1\leq i\leq m\}\)是一组\(n\)维行向量,\(1\leq j_1<j_2<\cdots<j_t\leq n\)是给定的\(t(t < n)\)个指标. 定义\(\widetilde{\boldsymbol{\alpha}}_i=(a_{ij_1},a_{ij_2},\cdots,a_{ij_t})\),称\(\widetilde{\boldsymbol{\alpha}}_i\)为\(\boldsymbol{\alpha}_i\)的\(t\)维\textbf{缩短向量}. 则

(1) 若\(\boldsymbol{\alpha}_1,\boldsymbol{\alpha}_2,\cdots,\boldsymbol{\alpha}_m\)线性相关,则\(\widetilde{\boldsymbol{\alpha}}_1,\widetilde{\boldsymbol{\alpha}}_2,\cdots,\widetilde{\boldsymbol{\alpha}}_m\)也线性相关;

(2) 设\(n\)维行向量\(\boldsymbol{\alpha}=(a_1,a_2,\cdots,a_n)\)是\(\boldsymbol{\alpha}_1,\boldsymbol{\alpha}_2,\cdots,\boldsymbol{\alpha}_m\)的线性组合,则\(\widetilde{\boldsymbol{\alpha}}\)也是\(\widetilde{\boldsymbol{\alpha}}_1,\widetilde{\boldsymbol{\alpha}}_2,\cdots,\widetilde{\boldsymbol{\alpha}}_m\)的线性组合.

2.设\(\{\boldsymbol{\alpha}_i=(a_{i1},a_{i2},\cdots,a_{in}),1\leq i\leq m\}\)是一组\(n\)维行向量,\(j_1,j_2,\cdots,j_t\geq 1\)是给定的\(t(t > n)\)个指标. 定义\(\overline{\boldsymbol{\alpha}}_i=(a_{ij_1},a_{ij_2},\cdots,a_{ij_t})\),称\(\overline{\boldsymbol{\alpha}}_i\)为\(\boldsymbol{\alpha}_i\)的\(t\)维\textbf{延伸向量}.则

(1)若\(\boldsymbol{\alpha}_1,\boldsymbol{\alpha}_2,\cdots,\boldsymbol{\alpha}_m\)线性无关,则\(\overline{\boldsymbol{\alpha}}_1,\overline{\boldsymbol{\alpha}}_2,\cdots,\overline{\boldsymbol{\alpha}}_m\)也线性无关.
\end{proposition}
\begin{note}
这个命题告诉我们:\textbf{线性相关向量组的任意缩短组也是线性相关的,线性无关向量组的任意延伸组也是线性无关的.}
\end{note}
\begin{proof}
1. (1) 由假设存在不全为零的数\(c_1,c_2,\cdots,c_m\),使得
\[
\boldsymbol{0}=c_1\boldsymbol{\alpha}_1 + c_2\boldsymbol{\alpha}_2+\cdots + c_m\boldsymbol{\alpha}_m=\left(\sum_{i = 1}^{m}c_ia_{i1},\sum_{i = 1}^{m}c_ia_{i2},\cdots,\sum_{i = 1}^{m}c_ia_{in}\right).
\]
在等式两边同时取\(t\)维缩短向量,可得
\[
\boldsymbol{0}=\left(\sum_{i = 1}^{m}c_ia_{ij_1},\sum_{i = 1}^{m}c_ia_{ij_2},\cdots,\sum_{i = 1}^{m}c_ia_{ij_t}\right)=c_1\widetilde{\boldsymbol{\alpha}}_1 + c_2\widetilde{\boldsymbol{\alpha}}_2+\cdots + c_m\widetilde{\boldsymbol{\alpha}}_m,
\]
从而结论成立.

(2) 设\(\boldsymbol{\alpha}=c_1\boldsymbol{\alpha}_1 + c_2\boldsymbol{\alpha}_2+\cdots + c_m\boldsymbol{\alpha}_m\),则
\[
\boldsymbol{\alpha}=c_1\boldsymbol{\alpha}_1 + c_2\boldsymbol{\alpha}_2+\cdots + c_m\boldsymbol{\alpha}_m=\left(\sum_{i = 1}^{m}c_ia_{i1},\sum_{i = 1}^{m}c_ia_{i2},\cdots,\sum_{i = 1}^{m}c_ia_{in}\right).
\]
在等式两边同时取\(t\)维缩短向量,可得
\[
\widetilde{\boldsymbol{\alpha}}=\left(\sum_{i = 1}^{m}c_ia_{ij_1},\sum_{i = 1}^{m}c_ia_{ij_2},\cdots,\sum_{i = 1}^{m}c_ia_{ij_t}\right)=c_1\widetilde{\boldsymbol{\alpha}}_1 + c_2\widetilde{\boldsymbol{\alpha}}_2+\cdots + c_m\widetilde{\boldsymbol{\alpha}}_m,
\]
从而结论成立. 

2.这个命题就是1(1)的逆否命题,从而结论成立.
\end{proof}

\begin{example}\label{example:3.1.1.1}
设\(V\)是实数域上连续函数全体构成的实线性空间,求证下列函数线性无关:

(1) \(\sin x,\sin 2x,\cdots,\sin nx\);

(2) \(1,\cos x,\cos 2x,\cdots,\cos nx\);

(3) \(1,\sin x,\cos x,\sin 2x,\cos 2x,\cdots,\sin nx,\cos nx\).
\end{example}
\begin{proof}
{\color{blue}证法一:}根据向量线性无关的基本性质,我们只要证明(3)即可. 对\(n\)进行归纳,当\(n = 0\)时,显然\(1\)作为一个函数线性无关. 假设命题对小于\(n\)的自然数成立,现证明等于\(n\)的情形. 设
\[
a + b_1\sin x + c_1\cos x + b_2\sin 2x + c_2\cos 2x+\cdots + b_n\sin nx + c_n\cos nx = 0,
\]
其中\(a,b_i,c_i\)都是实数. 对上式两次求导,可得
\[
-b_1\sin x - c_1\cos x - 4b_2\sin 2x - 4c_2\cos 2x-\cdots - n^2b_n\sin nx - n^2c_n\cos nx = 0,
\]
再将第一个式子乘以\(n^2\)加到第二个式子上,可得
\[
an^2+\sum_{i = 1}^{n - 1}b_i(n^2 - i^2)\sin ix+\sum_{i = 1}^{n - 1}c_i(n^2 - i^2)\cos ix = 0,
\]
由归纳假设即得\(a = b_1 = c_1=\cdots = b_{n - 1} = c_{n - 1} = 0\). 将此结论代入第一个式子可得\(b_n\sin nx + c_n\cos nx = 0\). 若\(b_n\neq0\) (\(c_n\neq0\)),则\(\tan nx=-c_n/b_n\) (\(\cot nx=-b_n/c_n\))为常数($\tan nx,\cot nx$都不是常函数),矛盾. 因此,\(b_n = c_n = 0\).

{\color{blue}证法二:} 设
\[
f(x)=a + b_1\sin x + c_1\cos x + b_2\sin 2x + c_2\cos 2x+\cdots + b_n\sin nx + c_n\cos nx = 0,
\]
其中\(a,b_i,c_i\)都是实数. 依次设\(g(x)=1,\sin x,\cos x,\sin 2x,\cos 2x,\cdots,\sin nx,\cos nx\),并分别计算定积分\(\int_{0}^{2\pi}f(x)g(x)\mathrm{d}x\),可得\(a = b_1 = c_1=\cdots = b_n = c_n = 0\). 
\end{proof}



\begin{proposition}\label{proposition:线性无关向量组的命题5}
设向量组\(\boldsymbol{\alpha}_1,\boldsymbol{\alpha}_2,\cdots,\boldsymbol{\alpha}_r\)线性无关,又
\[
\begin{cases}
\boldsymbol{\beta}_1 = a_{11}\boldsymbol{\alpha}_1 + a_{12}\boldsymbol{\alpha}_2+\cdots + a_{1r}\boldsymbol{\alpha}_r,\\
\boldsymbol{\beta}_2 = a_{21}\boldsymbol{\alpha}_1 + a_{22}\boldsymbol{\alpha}_2+\cdots + a_{2r}\boldsymbol{\alpha}_r,\\
\cdots\cdots\cdots\cdots\\
\boldsymbol{\beta}_r = a_{r1}\boldsymbol{\alpha}_1 + a_{r2}\boldsymbol{\alpha}_2+\cdots + a_{rr}\boldsymbol{\alpha}_r.
\end{cases}
\]
则:\(\boldsymbol{\beta}_1,\boldsymbol{\beta}_2,\cdots,\boldsymbol{\beta}_r\)线性相关的充要条件是系数矩阵\(A=(a_{ij})_{r\times r}\)的行列式为零.
\end{proposition}
\begin{proof}
记\(A\)的行向量为\(\boldsymbol{\gamma}_1,\boldsymbol{\gamma}_2,\cdots,\boldsymbol{\gamma}_r\). 若\(|A| = 0\),则\(A\)的行向量线性相关,即存在不全为零的\(r\)个数\(c_1,c_2,\cdots,c_r\),使得
\[
c_1\boldsymbol{\gamma}_1 + c_2\boldsymbol{\gamma}_2+\cdots + c_r\boldsymbol{\gamma}_r=\boldsymbol{0}.
\]
设$\boldsymbol{\alpha }=\left( \boldsymbol{\alpha }_1,\boldsymbol{\alpha }_2,\cdots ,\boldsymbol{\alpha }_m \right)'$,则经简单计算可得
\[
c_1\boldsymbol{\beta }_1+c_2\boldsymbol{\beta }_2+\cdots +c_r\boldsymbol{\beta }_r=c_1\boldsymbol{\gamma }_1\boldsymbol{\alpha }+c_1\boldsymbol{\gamma }_2\boldsymbol{\alpha }+\cdots +c_1\boldsymbol{\gamma }_r\boldsymbol{\alpha }=\left( c_1\boldsymbol{\gamma }_1+c_2\boldsymbol{\gamma }_2+\cdots +c_r\boldsymbol{\gamma }_r \right) \boldsymbol{\alpha }=\mathbf{0}\boldsymbol{\alpha }=\mathbf{0},
\]
从而\(\boldsymbol{\beta}_1,\boldsymbol{\beta}_2,\cdots,\boldsymbol{\beta}_r\)线性相关.

反之,若\(A\)可逆,如有\(k_1,k_2,\cdots,k_r\),使得
\[
k_1\boldsymbol{\beta}_1 + k_2\boldsymbol{\beta}_2+\cdots + k_r\boldsymbol{\beta}_r=\boldsymbol{0},
\]
将\(\boldsymbol{\beta}_i\)代入,并利用\(\boldsymbol{\alpha}_1,\boldsymbol{\alpha}_2,\cdots,\boldsymbol{\alpha}_r\)的线性无关性,可得以\(k_i\)为未知数的线性方程组:
\[
\begin{cases}
a_{11}k_1 + a_{21}k_2+\cdots + a_{r1}k_r = 0,\\
a_{12}k_1 + a_{22}k_2+\cdots + a_{r2}k_r = 0,\\
\cdots\cdots\cdots\cdots\\
a_{1r}k_1 + a_{2r}k_2+\cdots + a_{rr}k_r = 0.
\end{cases}
\]
因为\(A\)可逆,所以该方程组只有零解,从而\(\boldsymbol{\beta}_1,\boldsymbol{\beta}_2,\cdots,\boldsymbol{\beta}_r\)线性无关. 
\end{proof}

\begin{proposition}\label{proposition:线性无关向量组的命题6}
设\(\boldsymbol{\alpha}_1,\boldsymbol{\alpha}_2,\cdots,\boldsymbol{\alpha}_m\)是一组线性无关的向量,向量组\(\boldsymbol{\beta}_1,\boldsymbol{\beta}_2,\cdots,\boldsymbol{\beta}_k\)可用\(\boldsymbol{\alpha}_1,\boldsymbol{\alpha}_2,\cdots,\boldsymbol{\alpha}_m\)线性表示如下:
\[
\begin{cases}
\boldsymbol{\beta}_1 = a_{11}\boldsymbol{\alpha}_1 + a_{12}\boldsymbol{\alpha}_2+\cdots + a_{1m}\boldsymbol{\alpha}_m,\\
\boldsymbol{\beta}_2 = a_{21}\boldsymbol{\alpha}_1 + a_{22}\boldsymbol{\alpha}_2+\cdots + a_{2m}\boldsymbol{\alpha}_m,\\
\cdots\cdots\cdots\cdots\\
\boldsymbol{\beta}_k = a_{k1}\boldsymbol{\alpha}_1 + a_{k2}\boldsymbol{\alpha}_2+\cdots + a_{km}\boldsymbol{\alpha}_m.
\end{cases}
\]
记表示矩阵\(A=(a_{ij})_{k\times m}\),求证:向量组\(\boldsymbol{\beta}_1,\boldsymbol{\beta}_2,\cdots,\boldsymbol{\beta}_k\)的秩等于\(\text{r}(A)\).
\end{proposition}
\begin{proof}
{\color{blue}证法一:}
设\(\text{r}(A)=r\),记\(A\)的\(k\)个行向量为\(\boldsymbol{\gamma}_1,\boldsymbol{\gamma}_2,\cdots,\boldsymbol{\gamma}_k\). 不失一般性,可假设\(A\)的前\(r\)个行向量线性无关,其余向量均可用前\(r\)个行向量线性表示. 若
\[
\boldsymbol{\gamma}_i = c_1\boldsymbol{\gamma}_1 + c_2\boldsymbol{\gamma}_2+\cdots + c_r\boldsymbol{\gamma}_r,
\]
设$\boldsymbol{\alpha }=\left( \boldsymbol{\alpha }_1,\boldsymbol{\alpha }_2,\cdots ,\boldsymbol{\alpha }_m \right)'$,则经过简单计算可得
\begin{align*}
\boldsymbol{\beta }_i&=\boldsymbol{\gamma }_i\boldsymbol{\alpha }=\left( c_1\boldsymbol{\gamma }_1+c_2\boldsymbol{\gamma }_2+\cdots +c_r\boldsymbol{\gamma }_r \right) \boldsymbol{\alpha }
\\
&=c_1\boldsymbol{\gamma }_1\boldsymbol{\alpha }+c_1\boldsymbol{\gamma }_2\boldsymbol{\alpha }+\cdots +c_1\boldsymbol{\gamma }_r\boldsymbol{\alpha }
\\
&=c_1\boldsymbol{\beta }_1+c_2\boldsymbol{\beta }_2+\cdots +c_r\boldsymbol{\beta }_r. 
\end{align*}
另一方面,若
\[
c_1\boldsymbol{\beta}_1 + c_2\boldsymbol{\beta}_2+\cdots + c_r\boldsymbol{\beta}_r=\boldsymbol{0},
\]
则
\[
c_1(a_{11}\boldsymbol{\alpha}_1+\cdots + a_{1m}\boldsymbol{\alpha}_m)+\cdots + c_r(a_{r1}\boldsymbol{\alpha}_1+\cdots + a_{rm}\boldsymbol{\alpha}_m)=\boldsymbol{0},
\]
即
\[
(c_1a_{11}+\cdots + c_ra_{r1})\boldsymbol{\alpha}_1+\cdots + (c_1a_{1m}+\cdots + c_ra_{rm})\boldsymbol{\alpha}_m=\boldsymbol{0}.
\]
因为\(\boldsymbol{\alpha}_1,\cdots,\boldsymbol{\alpha}_m\)线性无关,故可得
\[
\begin{cases}
a_{11}c_1 + a_{21}c_2+\cdots + a_{r1}c_r = 0,\\
a_{12}c_1 + a_{22}c_2+\cdots + a_{r2}c_r = 0,\\
\cdots\cdots\cdots\cdots\\
a_{1m}c_1 + a_{2m}c_2+\cdots + a_{rm}c_r = 0.
\end{cases}
\]
将上述方程组看成是未知数\(c_i\)的齐次线性方程组,其系数矩阵的秩为\(r\),未知数个数也是\(r\),因此只有唯一一组解,即零解. 这表明\(\boldsymbol{\beta}_1,\boldsymbol{\beta}_2,\cdots,\boldsymbol{\beta}_r\)是向量组\(\boldsymbol{\beta}_1,\boldsymbol{\beta}_2,\cdots,\boldsymbol{\beta}_k\)的极大无关组,因此向量组\(\boldsymbol{\beta}_1,\boldsymbol{\beta}_2,\cdots,\boldsymbol{\beta}_k\)的秩等于\(r\).

{\color{blue}证法二:}令\(V\)是由\(\boldsymbol{\alpha}_1,\boldsymbol{\alpha}_2,\cdots,\boldsymbol{\alpha}_m\)生成的向量空间. 因为\(\boldsymbol{\alpha}_1,\boldsymbol{\alpha}_2,\cdots,\boldsymbol{\alpha}_m\)线性无关,故它们组成\(V\)的一组基,\(V\)的维数等于\(m\). 注意到\(\boldsymbol{\beta}_i\)在这组基下的坐标向量为\((a_{i1},a_{i2},\cdots,a_{im})'\),故由这些列向量组成的矩阵就是\(\boldsymbol{A}'\),从而
\begin{align*}
\mathrm{r}\left( \boldsymbol{\beta }_1,\boldsymbol{\beta }_2,\cdots ,\boldsymbol{\beta }_k \right) =\mathrm{r}\left( \boldsymbol{A}' \right) =\mathrm{r}\left( \boldsymbol{A} \right) .
\end{align*}
\end{proof}

\begin{proposition}\label{proposition:表出向量组的秩不超过原向量组的秩}
设\(\boldsymbol{\alpha}_1,\boldsymbol{\alpha}_2,\cdots,\boldsymbol{\alpha}_m\)是向量空间\(V\)中一组向量,向量组\(\boldsymbol{\beta}_1,\boldsymbol{\beta}_2,\cdots,\boldsymbol{\beta}_k\)可用\(\boldsymbol{\alpha}_1,\boldsymbol{\alpha}_2,\cdots,\boldsymbol{\alpha}_m\)线性表出,求证:向量组\(\boldsymbol{\beta}_1,\boldsymbol{\beta}_2,\cdots,\boldsymbol{\beta}_k\)的秩小于等于向量组\(\boldsymbol{\alpha}_1,\boldsymbol{\alpha}_2,\cdots,\boldsymbol{\alpha}_m\)的秩.
\end{proposition}
\begin{remark}
如果将向量组\(\boldsymbol{\alpha}_1,\boldsymbol{\alpha}_2,\cdots,\boldsymbol{\alpha}_m\)称为原向量组,将向量组\(\boldsymbol{\beta}_1,\boldsymbol{\beta}_2,\cdots,\boldsymbol{\beta}_k\)称为表出向量组,则这个命题可简述为:“\textbf{表出向量组的秩不超过原向量组的秩}.”从几何上看,这是一个自然的结果. 因为每个\(\boldsymbol{\beta}_i\)都属于由\(\boldsymbol{\alpha}_1,\boldsymbol{\alpha}_2,\cdots,\boldsymbol{\alpha}_m\)生成的子空间,故它们的秩不会超过该子空间的维数.
\end{remark}
\begin{proof}
不失一般性,可设\(\boldsymbol{\alpha}_1,\boldsymbol{\alpha}_2,\cdots,\boldsymbol{\alpha}_r\)是向量组\(\boldsymbol{\alpha}_1,\boldsymbol{\alpha}_2,\cdots,\boldsymbol{\alpha}_m\)的极大无关组,\(\boldsymbol{\beta}_1,\boldsymbol{\beta}_2,\cdots,\boldsymbol{\beta}_s\)是向量组\(\boldsymbol{\beta}_1,\boldsymbol{\beta}_2,\cdots,\boldsymbol{\beta}_k\)的极大无关组. 因为\(\boldsymbol{\alpha}_1,\boldsymbol{\alpha}_2,\cdots,\boldsymbol{\alpha}_m\)可用\(\boldsymbol{\alpha}_1,\boldsymbol{\alpha}_2,\cdots,\boldsymbol{\alpha}_r\)线性表出,所以\(\boldsymbol{\beta}_1,\boldsymbol{\beta}_2,\cdots,\boldsymbol{\beta}_s\)也可用\(\boldsymbol{\alpha}_1,\boldsymbol{\alpha}_2,\cdots,\boldsymbol{\alpha}_r\)线性表出,从而由\hyperref[theorem:向量的线性关系定理2]{定理\ref{theorem:向量的线性关系定理2}(1)}可知\(s\leq r\),结论成立.
\end{proof}

\begin{proposition}\label{proposition:极大无关组的判定条件}
设\(\boldsymbol{\alpha}_1,\boldsymbol{\alpha}_2,\cdots,\boldsymbol{\alpha}_m\)是向量空间\(V\)中一组向量且其秩等于\(r\),\(\boldsymbol{\alpha}_{i_1},\boldsymbol{\alpha}_{i_2},\cdots,\boldsymbol{\alpha}_{i_r}\)是其中\(r\)个向量. 若下列条件之一成立:

(1) \(\boldsymbol{\alpha}_{i_1},\boldsymbol{\alpha}_{i_2},\cdots,\boldsymbol{\alpha}_{i_r}\)线性无关;

(2) 任一\(\boldsymbol{\alpha}_i\)均可由\(\boldsymbol{\alpha}_{i_1},\boldsymbol{\alpha}_{i_2},\cdots,\boldsymbol{\alpha}_{i_r}\)线性表示.

则\(\boldsymbol{\alpha}_{i_1},\boldsymbol{\alpha}_{i_2},\cdots,\boldsymbol{\alpha}_{i_r}\)是向量组的极大无关组.
\end{proposition}
\begin{proof}
(1) 设\(\boldsymbol{\alpha}_{i_1},\boldsymbol{\alpha}_{i_2},\cdots,\boldsymbol{\alpha}_{i_r}\)线性无关,又设\(\boldsymbol{\alpha}_{j_1},\boldsymbol{\alpha}_{j_2},\cdots,\boldsymbol{\alpha}_{j_r}\)是向量组的极大无关组. 对任意的\(1\leq i\leq m\),\(\boldsymbol{\alpha}_{i_1},\boldsymbol{\alpha}_{i_2},\cdots,\boldsymbol{\alpha}_{i_r},\boldsymbol{\alpha}_i\)均可由\(\boldsymbol{\alpha}_{j_1},\boldsymbol{\alpha}_{j_2},\cdots,\boldsymbol{\alpha}_{j_r}\)线性表示,由\hyperref[theorem:向量的线性关系定理2]{定理\ref{theorem:向量的线性关系定理2}(2)}可知\(\boldsymbol{\alpha}_{i_1},\boldsymbol{\alpha}_{i_2},\cdots,\boldsymbol{\alpha}_{i_r},\boldsymbol{\alpha}_i\)必线性相关. 再由\hyperref[proposition:线性无关向量组的命题1]{命题\ref{proposition:线性无关向量组的命题1}}可知\(\boldsymbol{\alpha}_i\)可由\(\boldsymbol{\alpha}_{i_1},\boldsymbol{\alpha}_{i_2},\cdots,\boldsymbol{\alpha}_{i_r}\)线性表示,从而\(\boldsymbol{\alpha}_{i_1},\boldsymbol{\alpha}_{i_2},\cdots,\boldsymbol{\alpha}_{i_r}\)也是向量组的极大无关组.

(2) 设任一\(\boldsymbol{\alpha}_i\)均可由\(\boldsymbol{\alpha}_{i_1},\boldsymbol{\alpha}_{i_2},\cdots,\boldsymbol{\alpha}_{i_r}\)线性表示. 不失一般性,可设\(\boldsymbol{\alpha}_{i_1},\boldsymbol{\alpha}_{i_2},\cdots,\boldsymbol{\alpha}_{i_s}\)是向量组\(\boldsymbol{\alpha}_{i_1},\boldsymbol{\alpha}_{i_2},\cdots,\boldsymbol{\alpha}_{i_r}\)的极大无关组. 因此,\(\boldsymbol{\alpha}_{i_1},\boldsymbol{\alpha}_{i_2},\cdots,\boldsymbol{\alpha}_{i_s}\)线性无关. 再由线性组合的传递性可知,任一\(\boldsymbol{\alpha}_i\)均可由\(\boldsymbol{\alpha}_{i_1},\boldsymbol{\alpha}_{i_2},\cdots,\boldsymbol{\alpha}_{i_s}\)线性表示,故\(\boldsymbol{\alpha}_{i_1},\boldsymbol{\alpha}_{i_2},\cdots,\boldsymbol{\alpha}_{i_s}\)是原向量组的极大无关组,从而\(s = r\),即\(\boldsymbol{\alpha}_{i_1},\boldsymbol{\alpha}_{i_2},\cdots,\boldsymbol{\alpha}_{i_r}\)是原向量组的极大无关组.
\end{proof}

\begin{proposition}\label{proposition:对称矩阵或反称矩阵的极大无关组}
若\(\boldsymbol{A}\)是对称矩阵或反称矩阵,并且\(\boldsymbol{A}\)的第\(i_1,\cdots,i_r\)行是\(\boldsymbol{A}\)的行向量的极大无关组,则它的第\(i_1,\cdots,i_r\)列也是\(\boldsymbol{A}\)的列向量的极大无关组.
\end{proposition}
\begin{proof}
设\(\boldsymbol{A}=\begin{pmatrix}
\boldsymbol{\alpha}_1 \\
\boldsymbol{\alpha}_2 \\
\vdots \\
\boldsymbol{\alpha}_m
\end{pmatrix}=(\boldsymbol{\beta}_1,\boldsymbol{\beta}_2,\cdots,\boldsymbol{\beta}_n)\)为矩阵\(\boldsymbol{A}\)的行分块和列分块,则由条件可知,\(\boldsymbol{\alpha}_{i_1},\boldsymbol{\alpha}_{i_2},\cdots,\boldsymbol{\alpha}_{i_r}\)线性无关.从而存在一组不全为零的数\(k_1,k_2,\cdots,k_r\),使得
\[k_1\boldsymbol{\alpha}_{i_1}+k_2\boldsymbol{\alpha}_{i_2}+\cdots +k_r\boldsymbol{\alpha}_{i_r}=0.\]
又因为\(\boldsymbol{A}\)为对称或反称矩阵,所以\(\boldsymbol{\alpha}_i = \pm\boldsymbol{\beta}_{i}^{\prime}, i = 1,2,\cdots,n\).代入上式可得
\[k_1\boldsymbol{\beta}_{i_1}^{\prime}+k_2\boldsymbol{\beta}_{i_2}^{\prime}+\cdots +k_r\boldsymbol{\beta}_{i_r}^{\prime}=0.\]
再对上式两边同时取转置可得
\[k_1\boldsymbol{\beta}_{i_1}+k_2\boldsymbol{\beta}_{i_2}+\cdots +k_r\boldsymbol{\beta}_{i_r}=0.\]
故\(\boldsymbol{\beta}_{i_1},\boldsymbol{\beta}_{i_2},\cdots,\boldsymbol{\beta}_{i_r}\)线性无关.
\end{proof}

\begin{proposition}[向量组等价的充要条件]\label{proposition:向量组等价的充要条件}
设有两个向量组\(A = \{\boldsymbol{\alpha}_1,\boldsymbol{\alpha}_2,\cdots,\boldsymbol{\alpha}_m\}\)和\(B = \{\boldsymbol{\beta}_1,\boldsymbol{\beta}_2,\cdots,\boldsymbol{\beta}_n\}\). 求证:它们等价的充要条件是它们的秩相等且其中一组向量可以用另外一组向量线性表示.
\end{proposition}
\begin{note}
遇到向量组相关的问题,一般都会先设出各个向量组的极大无关组.
\end{note}
\begin{proof}
必要性由向量组等价的定义和\hyperref[proposition:表出向量组的秩不超过原向量组的秩]{命题\ref{proposition:表出向量组的秩不超过原向量组的秩}}即得,下证充分性. 假设向量组\(A\)可用向量组\(B\)线性表示,且它们的秩都等于\(r\). 不失一般性,设\(\boldsymbol{\alpha}_1,\boldsymbol{\alpha}_2,\cdots,\boldsymbol{\alpha}_r\)是向量组\(A\)的极大无关组,\(\boldsymbol{\beta}_1,\boldsymbol{\beta}_2,\cdots,\boldsymbol{\beta}_r\)是向量组\(B\)的极大无关组. 考虑向量组\(C = \{\boldsymbol{\alpha}_1,\boldsymbol{\alpha}_2,\cdots,\boldsymbol{\alpha}_r,\boldsymbol{\beta}_1,\boldsymbol{\beta}_2,\cdots,\boldsymbol{\beta}_r\}\). 因为\(\boldsymbol{\alpha}_1,\boldsymbol{\alpha}_2,\cdots,\boldsymbol{\alpha}_r\)可用\(\boldsymbol{\beta}_1,\boldsymbol{\beta}_2,\cdots,\boldsymbol{\beta}_r\)线性表示,故\(\boldsymbol{\beta}_1,\boldsymbol{\beta}_2,\cdots,\boldsymbol{\beta}_r\)是向量组\(C\)的极大无关组,从而向量组\(C\)的秩等于\(r\). 又因为\(\boldsymbol{\alpha}_1,\boldsymbol{\alpha}_2,\cdots,\boldsymbol{\alpha}_r\)线性无关,故由\hyperref[proposition:表出向量组的秩不超过原向量组的秩]{命题\ref{proposition:表出向量组的秩不超过原向量组的秩}(1)}可知,\(\boldsymbol{\alpha}_1,\boldsymbol{\alpha}_2,\cdots,\boldsymbol{\alpha}_r\)也是向量组\(C\)的极大无关组,从而\(\boldsymbol{\beta}_1,\boldsymbol{\beta}_2,\cdots,\boldsymbol{\beta}_r\)可用\(\boldsymbol{\alpha}_1,\boldsymbol{\alpha}_2,\cdots,\boldsymbol{\alpha}_r\)线性表示,于是向量组\(B\)也可用向量组\(A\)线性表示. 因此,向量组\(A\)与向量组\(B\)等价. 
\end{proof}


\begin{proposition}\label{proposition:线性无关的向量组与另一个转置的乘积积生成的矩阵也线性无关}
设\(\{\boldsymbol{\alpha}_{1},\boldsymbol{\alpha}_{2},\cdots,\boldsymbol{\alpha}_{p}\}\subseteq\mathbb{K}^{m}\)是\(p\)个线性无关的\(m\)维列向量, \(\{\boldsymbol{\beta}_{1},\boldsymbol{\beta}_{2},\cdots,\boldsymbol{\beta}_{q}\}\subseteq\mathbb{K}^{n}\)是\(q\)个线性无关的\(n\)维列向量. 求证: \(\{\boldsymbol{\alpha}_{i}\cdot\boldsymbol{\beta}_{j}^{\prime}\mid1\leq i\leq p,1\leq j\leq q\}\)是\(pq\)个线性无关的\(m\times n\)矩阵.
\end{proposition}
\begin{proof}
设\(c_{ij}\in\mathbb{K}\), 使得\(\sum_{i = 1}^{p}\sum_{j = 1}^{q}c_{ij}\boldsymbol{\alpha}_{i}\cdot\boldsymbol{\beta}_{j}^{\prime}=\boldsymbol{O}\), 则有
\begin{align}
\sum_{i = 1}^{p}\boldsymbol{\alpha}_{i}\cdot\left(\sum_{j = 1}^{q}c_{ij}\boldsymbol{\beta}_{j}^{\prime}\right)=\boldsymbol{O}.\label{3.4}
\end{align}
设\(\sum_{j = 1}^{q}c_{ij}\boldsymbol{\beta}_{j}^{\prime}=(a_{i1},a_{i2},\cdots,a_{in})(1\leq i\leq p)\),则\eqref{3.4}式可化为
\begin{align}\label{3.5}
\sum_{i=1}^p{\boldsymbol{\alpha }_i\left( a_{i1},a_{i2},\cdots ,a_{in} \right)}=\sum_{i=1}^p{\left( a_{i1}\boldsymbol{\alpha }_i,a_{i2}\boldsymbol{\alpha }_i,\cdots ,a_{in}\boldsymbol{\alpha }_i \right)}=\left( \sum_{i=1}^p{a_{i1}\boldsymbol{\alpha }_i},\sum_{i=1}^p{a_{i2}\boldsymbol{\alpha }_i},\cdots ,\sum_{i=1}^p{a_{in}\boldsymbol{\alpha }_i} \right) =\boldsymbol{O}.
\end{align}
从而比较\eqref{3.5}式两边矩阵的第\(k\)列有\(\sum_{i = 1}^{p}a_{ik}\boldsymbol{\alpha}_{i}=\boldsymbol{0}\). 由\(\boldsymbol{\alpha}_{1},\boldsymbol{\alpha}_{2},\cdots,\boldsymbol{\alpha}_{p}\)线性无关可得\(a_{ik}=0(1\leq i\leq p,1\leq k\leq n)\), 于是\(\sum_{j = 1}^{q}c_{ij}\boldsymbol{\beta}_{j}^{\prime}=\boldsymbol{0}(1\leq i\leq p)\). 再由\(\boldsymbol{\beta}_{1},\boldsymbol{\beta}_{2},\cdots,\boldsymbol{\beta}_{q}\)线性无关可得\(c_{ij}=0(1\leq i\leq p,1\leq j\leq q)\), 因此\(\{\boldsymbol{\alpha}_{i}\cdot\boldsymbol{\beta}_{j}^{\prime}\mid1\leq i\leq p,1\leq j\leq q\}\)线性无关. 
\end{proof}



\end{document}
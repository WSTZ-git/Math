% contents/chapter-03/section-06.tex 第三章第三节
\documentclass[../../main.tex]{subfiles}
\graphicspath{{\subfix{../../image/}}} % 指定图片目录,后续可以直接使用图片文件名。

% 例如:
% \begin{figure}[H]
% \centering
% \includegraphics[scale=0.4]{图.png}
% \caption{}
% \label{figure:图}
% \end{figure}
% 注意:上述\label{}一定要放在\caption{}之后,否则引用图片序号会只会显示??.

\begin{document}

\everymath{\displaystyle} % 让全文的行内公式都显示行间公式效果

\section{矩阵的秩}

\subsection{初等变换法}

矩阵的秩在初等变换或分块初等变换下不变.

\textbf{想法:遇到关于秩不等式的问题,可以考虑构造分块矩阵,对其做适当的初等变换,再利用秩的基本公式.}

\begin{theorem}\label{theorem:矩阵的秩与子式}
矩阵\(\boldsymbol{A}\)的秩等于\(r\)的充要条件是\(\boldsymbol{A}\)有一个\(r\)阶子式不等于零,而\(\boldsymbol{A}\)的所有\(r + 1\)阶子式都等于零.
\end{theorem}
\begin{proof}


\end{proof}

\begin{proposition}[矩阵秩的基本公式]\label{proposition:矩阵秩的基本公式}
\begin{enumerate}[(1)]
\item \label{矩阵秩的基本公式1}若\(k\neq0\),\(\mathrm{r}(k\boldsymbol{A})=\mathrm{r}(\boldsymbol{A})\);

\item \label{矩阵秩的基本公式2}\(\mathrm{r}(\boldsymbol{A}\boldsymbol{B})\leqslant \min\{\mathrm{r}(\boldsymbol{A}),\mathrm{r}(\boldsymbol{B})\}\);

\item \label{矩阵秩的基本公式3}\(\mathrm{r}\begin{pmatrix}\boldsymbol{A}&\boldsymbol{O}\\\boldsymbol{O}&\boldsymbol{B}\end{pmatrix}=\mathrm{r}(\boldsymbol{A})+\mathrm{r}(\boldsymbol{B})\),进而$\mathrm{r}\left( \begin{matrix}
\boldsymbol{A}_1&		&		&		\\
&		\boldsymbol{A}_2&		&		\\
&		&		\ddots&		\\
&		&		&		\boldsymbol{A}_n\\
\end{matrix} \right) =\mathrm{r}\left( \boldsymbol{A}_1 \right) +\mathrm{r}\left( \boldsymbol{A}_2 \right) +\cdots +\mathrm{r}\left( \boldsymbol{A}_n \right) .$;

\item \label{矩阵秩的基本公式4}\(\mathrm{r}\begin{pmatrix}\boldsymbol{A}&\boldsymbol{C}\\\boldsymbol{O}&\boldsymbol{B}\end{pmatrix}\geqslant \mathrm{r}(\boldsymbol{A})+\mathrm{r}(\boldsymbol{B})\),\(\mathrm{r}\begin{pmatrix}\boldsymbol{A}&\boldsymbol{O}\\\boldsymbol{D}&\boldsymbol{B}\end{pmatrix}\geqslant \mathrm{r}(\boldsymbol{A})+\mathrm{r}(\boldsymbol{B})\);

\item \label{矩阵秩的基本公式5}\(\mathrm{r}\left( \boldsymbol{A}\pm \boldsymbol{B} \right) \leqslant \mathrm{r}\left( \begin{array}{c}
\boldsymbol{A}\\
\boldsymbol{B}\\
\end{array} \right) =\mathrm{r}\left( \begin{matrix}
\boldsymbol{A}&		\boldsymbol{B}\\
\end{matrix} \right) \leqslant \mathrm{r}\left( \boldsymbol{A} \right) +\mathrm{r}\left( \boldsymbol{B} \right) \);

\item \label{矩阵秩的基本公式6}\(\mathrm{r}(\boldsymbol{A}-\boldsymbol{B})\geqslant |\mathrm{r}(\boldsymbol{A})-\mathrm{r}(\boldsymbol{B})|\).

\item \label{矩阵秩的基本公式7} $\mathrm{r}\left( \begin{array}{c}
\boldsymbol{A}\\
\boldsymbol{B}\\
\end{array} \right) =\mathrm{r}\left( \begin{matrix}
\boldsymbol{A}&		\boldsymbol{B}\\
\end{matrix} \right) \geqslant \max \left\{ \mathrm{r}\left( \boldsymbol{A} \right) ,\mathrm{r}\left( \boldsymbol{B} \right) \right\}$,进而
\begin{gather*}
\mathrm{r}\left( \begin{matrix}
\boldsymbol{A}&		\boldsymbol{B}\\
\boldsymbol{C}&		\boldsymbol{D}\\
\end{matrix} \right) \geqslant \mathrm{r}\left( \begin{matrix}
\boldsymbol{A}&		\boldsymbol{B}\\
\end{matrix} \right) \geqslant \mathrm{r}\left( \boldsymbol{A} \right) ,\mathrm{r}\left( \boldsymbol{B} \right) ,
\\
\mathrm{r}\left( \begin{matrix}
\boldsymbol{A}&		\boldsymbol{B}\\
\boldsymbol{C}&		\boldsymbol{D}\\
\end{matrix} \right) \geqslant \mathrm{r}\left( \begin{matrix}
\boldsymbol{C}&		\boldsymbol{D}\\
\end{matrix} \right) \geqslant \mathrm{r}\left( \boldsymbol{C} \right) ,\mathrm{r}\left( \boldsymbol{D} \right) ,
\\
\mathrm{r}\left( \begin{matrix}
\boldsymbol{A}&		\boldsymbol{B}\\
\boldsymbol{C}&		\boldsymbol{D}\\
\end{matrix} \right) \geqslant \mathrm{r}\left( \begin{array}{c}
\boldsymbol{A}\\
\boldsymbol{C}\\
\end{array} \right) \geqslant \mathrm{r}\left( \boldsymbol{A} \right) ,\mathrm{r}\left( \boldsymbol{C} \right) ,
\\
\mathrm{r}\left( \begin{matrix}
\boldsymbol{A}&		\boldsymbol{B}\\
\boldsymbol{C}&		\boldsymbol{D}\\
\end{matrix} \right) \geqslant \mathrm{r}\left( \begin{array}{c}
\boldsymbol{B}\\
\boldsymbol{D}\\
\end{array} \right) \geqslant \mathrm{r}\left( \boldsymbol{B} \right) ,\mathrm{r}\left( \boldsymbol{D} \right) .
\end{gather*}
\end{enumerate}
\end{proposition}
\begin{proof}
\begin{enumerate}[(1)]
\item 由于\(k\boldsymbol{A}=\boldsymbol{P}_1(k)\boldsymbol{P}_2(k)\cdots\boldsymbol{P}_m(k)\boldsymbol{A}\),故\(\mathrm{r}(k\boldsymbol{A})=\mathrm{r}(\boldsymbol{A})\).

\item  {\color{blue}证法一:}设\(\boldsymbol{A}\)是\(m\times n\)矩阵,\(\boldsymbol{B}\)是\(n\times s\)矩阵. 将矩阵\(\boldsymbol{B}\)按列分块,\(\boldsymbol{B}=(\boldsymbol{\beta}_1,\boldsymbol{\beta}_2,\cdots,\boldsymbol{\beta}_s)\),则\(\boldsymbol{A}\boldsymbol{B}=(\boldsymbol{A}\boldsymbol{\beta}_1,\boldsymbol{A}\boldsymbol{\beta}_2,\cdots,\boldsymbol{A}\boldsymbol{\beta}_s)\). 若\(\boldsymbol{B}\)列向量的极大无关组为\(\{\boldsymbol{\beta}_{j_1},\boldsymbol{\beta}_{j_2},\cdots,\boldsymbol{\beta}_{j_r}\}\),则\(\boldsymbol{B}\)的任一列向量\(\boldsymbol{\beta}_j\)均可用\(\{\boldsymbol{\beta}_{j_1},\boldsymbol{\beta}_{j_2},\cdots,\boldsymbol{\beta}_{j_r}\}\)线性表示. 于是任一\(\boldsymbol{A}\boldsymbol{\beta}_j\)也可用\(\{\boldsymbol{A}\boldsymbol{\beta}_{j_1},\boldsymbol{A}\boldsymbol{\beta}_{j_2},\cdots,\boldsymbol{A}\boldsymbol{\beta}_{j_r}\}\)来线性表示. 因此,向量组\(\{\boldsymbol{A}\boldsymbol{\beta}_1,\boldsymbol{A}\boldsymbol{\beta}_2,\cdots,\boldsymbol{A}\boldsymbol{\beta}_s\}\)的秩不超过\(r\),即\(\mathrm{r}(\boldsymbol{A}\boldsymbol{B})\leqslant \mathrm{r}(\boldsymbol{B})\). 同理,对矩阵\(\boldsymbol{A}\)用行分块的方法可以证明\(\mathrm{r}(\boldsymbol{A}\boldsymbol{B})\leqslant \mathrm{r}(\boldsymbol{A})\).

{\color{blue}证法二:}见\hyperref[example:4.243456]{例题\ref{example:4.243456}}.

\item  设\(\boldsymbol{A},\boldsymbol{B}\)的秩分别为\(r_1,r_2\),则存在非异阵\(\boldsymbol{P}_1,\boldsymbol{Q}_1\)和非异阵\(\boldsymbol{P}_2,\boldsymbol{Q}_2\),使得
\[
\boldsymbol{P}_1\boldsymbol{A}\boldsymbol{Q}_1=\begin{pmatrix}
\boldsymbol{I}_{r_1}&\boldsymbol{O}\\
\boldsymbol{O}&\boldsymbol{O}
\end{pmatrix}, \boldsymbol{P}_2\boldsymbol{B}\boldsymbol{Q}_2=\begin{pmatrix}
\boldsymbol{I}_{r_2}&\boldsymbol{O}\\
\boldsymbol{O}&\boldsymbol{O}
\end{pmatrix}.
\]
于是
\[
\begin{pmatrix}
\boldsymbol{P}_1&\boldsymbol{O}\\
\boldsymbol{O}&\boldsymbol{P}_2
\end{pmatrix}
\begin{pmatrix}
\boldsymbol{A}&\boldsymbol{O}\\
\boldsymbol{O}&\boldsymbol{B}
\end{pmatrix}
\begin{pmatrix}
\boldsymbol{Q}_1&\boldsymbol{O}\\
\boldsymbol{O}&\boldsymbol{Q}_2
\end{pmatrix}=
\begin{pmatrix}
\boldsymbol{P}_1\boldsymbol{A}\boldsymbol{Q}_1&\boldsymbol{O}\\
\boldsymbol{O}&\boldsymbol{P}_2\boldsymbol{B}\boldsymbol{Q}_2
\end{pmatrix}=
\begin{pmatrix}
\boldsymbol{I}_{r_1}&\boldsymbol{O}&\boldsymbol{O}&\boldsymbol{O}\\
\boldsymbol{O}&\boldsymbol{O}&\boldsymbol{O}&\boldsymbol{O}\\
\boldsymbol{O}&\boldsymbol{O}&\boldsymbol{I}_{r_2}&\boldsymbol{O}\\
\boldsymbol{O}&\boldsymbol{O}&\boldsymbol{O}&\boldsymbol{O}
\end{pmatrix}.
\]
因此,\(\mathrm{r}\begin{pmatrix}
\boldsymbol{A}&\boldsymbol{O}\\
\boldsymbol{O}&\boldsymbol{B}
\end{pmatrix}=r_1 + r_2=\mathrm{r}(\boldsymbol{A})+\mathrm{r}(\boldsymbol{B})\).

\item  {\color{blue}证法一:} 我们只证明第一个不等式,第二个不等式同理可证. 设\(\boldsymbol{A},\boldsymbol{B}\)的秩分别为\(r_1,r_2\),则存在非异阵\(\boldsymbol{P}_1,\boldsymbol{Q}_1\)和非异阵\(\boldsymbol{P}_2,\boldsymbol{Q}_2\),使得
\[
\boldsymbol{P}_1\boldsymbol{A}\boldsymbol{Q}_1=\begin{pmatrix}
\boldsymbol{I}_{r_1}&\boldsymbol{O}\\
\boldsymbol{O}&\boldsymbol{O}
\end{pmatrix}, \boldsymbol{P}_2\boldsymbol{B}\boldsymbol{Q}_2=\begin{pmatrix}
\boldsymbol{I}_{r_2}&\boldsymbol{O}\\
\boldsymbol{O}&\boldsymbol{O}
\end{pmatrix}.
\]
于是
\[
\begin{pmatrix}
\boldsymbol{P}_1&\boldsymbol{O}\\
\boldsymbol{O}&\boldsymbol{P}_2
\end{pmatrix}
\begin{pmatrix}
\boldsymbol{A}&\boldsymbol{C}\\
\boldsymbol{O}&\boldsymbol{B}
\end{pmatrix}
\begin{pmatrix}
\boldsymbol{Q}_1&\boldsymbol{O}\\
\boldsymbol{O}&\boldsymbol{Q}_2
\end{pmatrix}=
\begin{pmatrix}
\boldsymbol{P}_1\boldsymbol{A}\boldsymbol{Q}_1&\boldsymbol{P}_1\boldsymbol{C}\boldsymbol{Q}_2\\
\boldsymbol{O}&\boldsymbol{P}_2\boldsymbol{B}\boldsymbol{Q}_2
\end{pmatrix}=
\begin{pmatrix}
\boldsymbol{I}_{r_1}&\boldsymbol{O}&\boldsymbol{C}_{11}&\boldsymbol{C}_{12}\\
\boldsymbol{O}&\boldsymbol{O}&\boldsymbol{C}_{21}&\boldsymbol{C}_{22}\\
\boldsymbol{O}&\boldsymbol{O}&\boldsymbol{I}_{r_2}&\boldsymbol{O}\\
\boldsymbol{O}&\boldsymbol{O}&\boldsymbol{O}&\boldsymbol{O}
\end{pmatrix}.
\]
在上面的分块矩阵中实施第三类分块初等变换,用\(\boldsymbol{I}_{r_1}\)消去同行的矩阵;用\(\boldsymbol{I}_{r_2}\)消去
同列的矩阵,再将\(\boldsymbol{C}_{22}\)对换到第\((2,2)\)位置:
\[
\begin{pmatrix}
\boldsymbol{I}_{r_1}&\boldsymbol{O}&\boldsymbol{C}_{11}&\boldsymbol{C}_{12}\\
\boldsymbol{O}&\boldsymbol{O}&\boldsymbol{C}_{21}&\boldsymbol{C}_{22}\\
\boldsymbol{O}&\boldsymbol{O}&\boldsymbol{I}_{r_2}&\boldsymbol{O}\\
\boldsymbol{O}&\boldsymbol{O}&\boldsymbol{O}&\boldsymbol{O}
\end{pmatrix}
\rightarrow
\begin{pmatrix}
\boldsymbol{I}_{r_1}&\boldsymbol{O}&\boldsymbol{O}&\boldsymbol{O}\\
\boldsymbol{O}&\boldsymbol{O}&\boldsymbol{O}&\boldsymbol{C}_{22}\\
\boldsymbol{O}&\boldsymbol{O}&\boldsymbol{I}_{r_2}&\boldsymbol{O}\\
\boldsymbol{O}&\boldsymbol{O}&\boldsymbol{O}&\boldsymbol{O}
\end{pmatrix}
\rightarrow
\begin{pmatrix}
\boldsymbol{I}_{r_1}&\boldsymbol{O}&\boldsymbol{O}&\boldsymbol{O}\\
\boldsymbol{O}&\boldsymbol{C}_{22}&\boldsymbol{O}&\boldsymbol{O}\\
\boldsymbol{O}&\boldsymbol{O}&\boldsymbol{I}_{r_2}&\boldsymbol{O}\\
\boldsymbol{O}&\boldsymbol{O}&\boldsymbol{O}&\boldsymbol{O}
\end{pmatrix},
\]
最后由(3)的结论可得
\[
\mathrm{r}\begin{pmatrix}
\boldsymbol{A}&\boldsymbol{C}\\
\boldsymbol{O}&\boldsymbol{B}
\end{pmatrix}=\mathrm{r}(\boldsymbol{I}_{r_1})+\mathrm{r}(\boldsymbol{C}_{22})+\mathrm{r}(\boldsymbol{I}_{r_2})\geqslant  r_1 + r_2=\mathrm{r}(\boldsymbol{A})+\mathrm{r}(\boldsymbol{B}).
\]
{\color{blue}证法二:} 我们也可用子式法来证明. 设\(\mathrm{r}\begin{pmatrix}
\boldsymbol{A}&\boldsymbol{O}\\
\boldsymbol{O}&\boldsymbol{B}
\end{pmatrix}=r\),则由\hyperref[theorem:矩阵的秩与子式]{定理\ref{theorem:矩阵的秩与子式}}可知,\(\begin{pmatrix}
\boldsymbol{A}&\boldsymbol{O}\\
\boldsymbol{O}&\boldsymbol{B}
\end{pmatrix}\)有一个\(r\)阶子式不为零,不妨设为\(\begin{vmatrix}
\boldsymbol{A}_1&\boldsymbol{O}\\
\boldsymbol{O}&\boldsymbol{B}_1
\end{vmatrix}\),其中\(\boldsymbol{A}_1,\boldsymbol{B}_1\)分别是\(\boldsymbol{A},\boldsymbol{B}\)的子阵. 注意\(\boldsymbol{A}_1\)或\(\boldsymbol{B}_1\)允许是零阶矩阵,这对应于该子式完全包含在\(\boldsymbol{B}\)或\(\boldsymbol{A}\)中,但若\(\boldsymbol{A}_1,\boldsymbol{B}_1\)的阶数都大于零,则通过该子式非零,再结合由Laplace定理容易验证\(\boldsymbol{A}_1,\boldsymbol{B}_1\)都是方阵. 设在矩阵\(\begin{pmatrix}
\boldsymbol{A}&\boldsymbol{C}\\
\boldsymbol{O}&\boldsymbol{B}
\end{pmatrix}\)中对应的\(r\)阶子式是\(\begin{vmatrix}
\boldsymbol{A}_1&\boldsymbol{C}_1\\
\boldsymbol{O}&\boldsymbol{B}_1
\end{vmatrix}\),则由Laplace定理可得\(\begin{vmatrix}
\boldsymbol{A}_1&\boldsymbol{C}_1\\
\boldsymbol{O}&\boldsymbol{B}_1
\end{vmatrix}=|\boldsymbol{A}_1||\boldsymbol{B}_1|=\begin{vmatrix}
\boldsymbol{A}_1&\boldsymbol{O}\\
\boldsymbol{O}&\boldsymbol{B}_1
\end{vmatrix}\neq0\),再次由\hyperref[theorem:矩阵的秩与子式]{定理\ref{theorem:矩阵的秩与子式}}可得
\[
\mathrm{r}\begin{pmatrix}
\boldsymbol{A}&\boldsymbol{C}\\
\boldsymbol{O}&\boldsymbol{B}
\end{pmatrix}\geqslant  r=\mathrm{r}\begin{pmatrix}
\boldsymbol{A}&\boldsymbol{O}\\
\boldsymbol{O}&\boldsymbol{B}
\end{pmatrix}=\mathrm{r}(\boldsymbol{A})+\mathrm{r}(\boldsymbol{B}). 
\]
{\color{blue}证法三:}设\(\boldsymbol{A}=(\boldsymbol{\alpha}_1,\boldsymbol{\alpha}_2,\cdots,\boldsymbol{\alpha}_n)\)是\(\boldsymbol{A}\)的列分块,\(\boldsymbol{\alpha}_{i_1},\boldsymbol{\alpha}_{i_2},\cdots,\boldsymbol{\alpha}_{i_r}\)是\(\boldsymbol{A}\)的列向量的极大无关组;设\(\boldsymbol{B}=(\boldsymbol{\beta}_1,\boldsymbol{\beta}_2,\cdots,\boldsymbol{\beta}_l)\),\(\boldsymbol{C}=(\boldsymbol{\gamma}_1,\boldsymbol{\gamma}_2,\cdots,\boldsymbol{\gamma}_l)\)是\(\boldsymbol{B},\boldsymbol{C}\)的列分块,\(\boldsymbol{\beta}_{j_1},\boldsymbol{\beta}_{j_2},\cdots,\boldsymbol{\beta}_{j_s}\)是\(\boldsymbol{B}\)的列向量的极大无关组,则\(\mathrm{r}(\boldsymbol{A}) = r\)且\(\mathrm{r}(\boldsymbol{B}) = s\). 我们接下来证明:作为\(\begin{pmatrix}
\boldsymbol{A}&\boldsymbol{C}\\
\boldsymbol{O}&\boldsymbol{B}
\end{pmatrix}\)的列向量,\(\begin{pmatrix}
\boldsymbol{\alpha}_{i_1}\\
\boldsymbol{0}
\end{pmatrix},\cdots,\begin{pmatrix}
\boldsymbol{\alpha}_{i_r}\\
\boldsymbol{0}
\end{pmatrix},\begin{pmatrix}
\boldsymbol{\gamma}_{j_1}\\
\boldsymbol{\beta}_{j_1}
\end{pmatrix},\cdots,\begin{pmatrix}
\boldsymbol{\gamma}_{j_s}\\
\boldsymbol{\beta}_{j_s}
\end{pmatrix}\)线性无关. 设
\[
c_1\begin{pmatrix}
\boldsymbol{\alpha}_{i_1}\\
\boldsymbol{0}
\end{pmatrix}+\cdots + c_r\begin{pmatrix}
\boldsymbol{\alpha}_{i_r}\\
\boldsymbol{0}
\end{pmatrix}+d_1\begin{pmatrix}
\boldsymbol{\gamma}_{j_1}\\
\boldsymbol{\beta}_{j_1}
\end{pmatrix}+\cdots + d_s\begin{pmatrix}
\boldsymbol{\gamma}_{j_s}\\
\boldsymbol{\beta}_{j_s}
\end{pmatrix}=\boldsymbol{0},
\]
即
\[
c_1\boldsymbol{\alpha}_{i_1}+\cdots + c_r\boldsymbol{\alpha}_{i_r}+d_1\boldsymbol{\gamma}_{j_1}+\cdots + d_s\boldsymbol{\gamma}_{j_s}=\boldsymbol{0},d_1\boldsymbol{\beta}_{j_1}+\cdots + d_s\boldsymbol{\beta}_{j_s}=\boldsymbol{0}.
\]
由上面的假设即得\(c_1=\cdots = c_r = d_1=\cdots = d_s = 0\),于是上述结论得证. 因为\(\begin{pmatrix}
\boldsymbol{A}&\boldsymbol{C}\\
\boldsymbol{O}&\boldsymbol{B}
\end{pmatrix}\)的列向量中有\(r + s\)个线性无关,故\(\mathrm{r}\begin{pmatrix}
\boldsymbol{A}&\boldsymbol{C}\\
\boldsymbol{O}&\boldsymbol{B}
\end{pmatrix}\geqslant  r + s=\mathrm{r}(\boldsymbol{A})+\mathrm{r}(\boldsymbol{B})\).

\item  注意到
\[
\left( \begin{matrix}
\boldsymbol{I}&		\boldsymbol{I}\\
\end{matrix} \right) \left( \begin{matrix}
\boldsymbol{A}&		\boldsymbol{O}\\
\boldsymbol{O}&		\boldsymbol{B}\\
\end{matrix} \right) =\left( \begin{matrix}
\boldsymbol{A}&		\boldsymbol{B}\\
\end{matrix} \right) ,\left( \begin{matrix}
\boldsymbol{A}&		\boldsymbol{O}\\
\boldsymbol{O}&		\boldsymbol{B}\\
\end{matrix} \right) \left( \begin{array}{c}
\boldsymbol{I}\\
\boldsymbol{I}\\
\end{array} \right) =\left( \begin{array}{c}
\boldsymbol{A}\\
\boldsymbol{B}\\
\end{array} \right) .
\]
故由(2)和(3)可得
\begin{align*}
\mathrm{r}\left( \begin{matrix}
\boldsymbol{A}&		\boldsymbol{B}\\
\end{matrix} \right) =\mathrm{r}\left( \left( \begin{matrix}
\boldsymbol{I}&		\boldsymbol{I}\\
\end{matrix} \right) \left( \begin{matrix}
\boldsymbol{A}&		\boldsymbol{O}\\
\boldsymbol{O}&		\boldsymbol{B}\\
\end{matrix} \right) \right) \leqslant \mathrm{r}\left( \begin{matrix}
\boldsymbol{A}&		\boldsymbol{O}\\
\boldsymbol{O}&		\boldsymbol{B}\\
\end{matrix} \right) =\mathrm{r}\left( \boldsymbol{A} \right) +\mathrm{r}\left( \boldsymbol{B} \right) ,
\\
\mathrm{r}\left( \begin{array}{c}
\boldsymbol{A}\\
\boldsymbol{B}\\
\end{array} \right) =\mathrm{r}\left( \left( \begin{matrix}
\boldsymbol{A}&		\boldsymbol{O}\\
\boldsymbol{O}&		\boldsymbol{B}\\
\end{matrix} \right) \left( \begin{array}{c}
\boldsymbol{I}\\
\boldsymbol{I}\\
\end{array} \right) \right) \leqslant \mathrm{r}\left( \begin{matrix}
\boldsymbol{A}&		\boldsymbol{O}\\
\boldsymbol{O}&		\boldsymbol{B}\\
\end{matrix} \right) =\mathrm{r}\left( \boldsymbol{A} \right) +\mathrm{r}\left( \boldsymbol{B} \right) .
\end{align*}

注意到
\[
\left( \begin{matrix}
\boldsymbol{A}&		\boldsymbol{B}\\
\end{matrix} \right) \left( \begin{array}{c}
\boldsymbol{I}\\
\boldsymbol{I}\\
\end{array} \right) =\boldsymbol{A}+\boldsymbol{B},\left( \begin{matrix}
\boldsymbol{A}&		\boldsymbol{B}\\
\end{matrix} \right) \left( \begin{array}{c}
\boldsymbol{I}\\
-\boldsymbol{I}\\
\end{array} \right) =\boldsymbol{A}-\boldsymbol{B}.
\]
故由(2)可得
\begin{align*}
\mathrm{r}\left( \boldsymbol{A}+\boldsymbol{B} \right) =\mathrm{r}\left( \left( \begin{matrix}
\boldsymbol{A}&		\boldsymbol{B}\\
\end{matrix} \right) \left( \begin{array}{c}
\boldsymbol{I}\\
\boldsymbol{I}\\
\end{array} \right) \right) \leqslant \mathrm{r}\left( \begin{matrix}
\boldsymbol{A}&		\boldsymbol{B}\\
\end{matrix} \right) \leqslant \mathrm{r}\left( \boldsymbol{A} \right) +\mathrm{r}\left( \boldsymbol{B} \right) ,
\\
\mathrm{r}\left( \boldsymbol{A}-\boldsymbol{B} \right) =\mathrm{r}\left( \left( \begin{matrix}
\boldsymbol{A}&		\boldsymbol{B}\\
\end{matrix} \right) \left( \begin{array}{c}
\boldsymbol{I}\\
-\boldsymbol{I}\\
\end{array} \right) \right) \leqslant \mathrm{r}\left( \begin{matrix}
\boldsymbol{A}&		\boldsymbol{B}\\
\end{matrix} \right) \leqslant \mathrm{r}\left( \boldsymbol{A} \right) +\mathrm{r}\left( \boldsymbol{B} \right) .
\end{align*}

\item 由于\(\mathrm{r}(\boldsymbol{A}-\boldsymbol{B})=\mathrm{r}(\boldsymbol{B}-\boldsymbol{A})\),故不妨设\(\mathrm{r}(\boldsymbol{A})\geqslant \mathrm{r}(\boldsymbol{B})\),则由(6)可得\(\mathrm{r}(\boldsymbol{A}-\boldsymbol{B})+\mathrm{r}(\boldsymbol{B})\geqslant \mathrm{r}(\boldsymbol{A}-\boldsymbol{B}+\boldsymbol{B})=\mathrm{r}(\boldsymbol{A})\),即\(\mathrm{r}(\boldsymbol{A}-\boldsymbol{B})\geqslant \mathrm{r}(\boldsymbol{A})-\mathrm{r}(\boldsymbol{B})\).

\item 第一个不等式是显然的(考虑极大无关组即可).由第一个不等式可得
\begin{align*}
\mathrm{r}\left( \begin{matrix}
\boldsymbol{A}&		\boldsymbol{B}\\
\boldsymbol{C}&		\boldsymbol{D}\\
\end{matrix} \right) =\mathrm{r}\left( \begin{matrix}
\boldsymbol{A}&		\boldsymbol{B}\\
\end{matrix} \right) \geqslant \mathrm{r}\left( \boldsymbol{A} \right) ,\mathrm{r}\left( \boldsymbol{B} \right) .
\end{align*}
其他同理可证.
\end{enumerate}

\end{proof}

\begin{corollary}\label{corollary:列分块与行分块秩等于其交叉块的秩}
若分块矩阵\(\begin{pmatrix}
\boldsymbol{A}&\boldsymbol{B}\\
\boldsymbol{C}&\boldsymbol{D}
\end{pmatrix}\)满足\(\mathrm{r}\begin{pmatrix}
\boldsymbol{A}&\boldsymbol{B}\\
\boldsymbol{C}&\boldsymbol{D}
\end{pmatrix}=\mathrm{r}(\boldsymbol{A})\), 则\(\mathrm{r}\begin{pmatrix}
\boldsymbol{A}&\boldsymbol{B}
\end{pmatrix}=\mathrm{r}\begin{pmatrix}
\boldsymbol{A}\\
\boldsymbol{C}
\end{pmatrix}=\mathrm{r}(\boldsymbol{A})\).
\end{corollary}
\begin{proof}
由条件可得
\begin{align*}
\mathrm{r}\left( A \right) \leqslant \mathrm{r}\left( \begin{matrix}
A&		B\\
\end{matrix} \right) =\mathrm{r}\left( \begin{array}{c}
A\\
C\\
\end{array} \right) \leqslant \mathrm{r}\left( \begin{matrix}
A&		B\\
C&		D\\
\end{matrix} \right) =\mathrm{r}\left( A \right) .
\end{align*}
故\(\mathrm{r}\begin{pmatrix}
\boldsymbol{A}&\boldsymbol{B}
\end{pmatrix}=\mathrm{r}\begin{pmatrix}
\boldsymbol{A}\\
\boldsymbol{C}
\end{pmatrix}=\mathrm{r}(\boldsymbol{A})\).

\end{proof}

\begin{example}
设\(\boldsymbol{A}=(a_{ij}),\boldsymbol{B}=(b_{ij})\)是\(m\times n\)矩阵,且\(b_{ij}=(-1)^{i + j}a_{ij}\). 求证:\(\mathrm{r}(\boldsymbol{A})=\mathrm{r}(\boldsymbol{B})\).
\end{example}
\begin{proof}
将\(\boldsymbol{A}\)的第\(i\)行乘以\((-1)^i\),又将第\(j\)列乘以\((-1)^j\),即得矩阵\(\boldsymbol{B}\),因此\(\boldsymbol{A}\)和\(\boldsymbol{B}\)相抵,故结论成立. 

\end{proof}

\begin{proposition}[Sylvester(西尔维斯特)不等式]\label{proposition:Sylvester(西尔维斯特)不等式}
设$\boldsymbol{A}$是$m\times n$矩阵,$\boldsymbol{B}$是$n\times t$矩阵,求证:
\[
\mathrm{r}(\boldsymbol{A}\boldsymbol{B})\geqslant \mathrm{r}(\boldsymbol{A})+\mathrm{r}(\boldsymbol{B}) - n.
\]
\end{proposition}
\begin{proof}
{\color{blue}证法一:}
考虑下列矩阵的分块初等变换:
\[
\begin{pmatrix}
\boldsymbol{I}_n&\boldsymbol{O}\\
\boldsymbol{O}&\boldsymbol{A}\boldsymbol{B}
\end{pmatrix}\to
\begin{pmatrix}
\boldsymbol{I}_n&\boldsymbol{O}\\
\boldsymbol{A}&\boldsymbol{A}\boldsymbol{B}
\end{pmatrix}\to
\begin{pmatrix}
\boldsymbol{I}_n&-\boldsymbol{B}\\
\boldsymbol{A}&\boldsymbol{O}
\end{pmatrix}\to
\begin{pmatrix}
\boldsymbol{B}&\boldsymbol{I}_n\\
\boldsymbol{O}&\boldsymbol{A}
\end{pmatrix},
\]
由\hyperref[矩阵秩的基本公式3]{矩阵秩的基本公式(3)}和\hyperref[矩阵秩的基本公式4]{矩阵秩的基本公式(4)}可得
\[
\mathrm{r}(\boldsymbol{A}\boldsymbol{B}) + n=\mathrm{r}\begin{pmatrix}
\boldsymbol{I}_n&\boldsymbol{O}\\
\boldsymbol{O}&\boldsymbol{A}\boldsymbol{B}
\end{pmatrix}=\mathrm{r}\begin{pmatrix}
\boldsymbol{B}&\boldsymbol{I}_n\\
\boldsymbol{O}&\boldsymbol{A}
\end{pmatrix}\geqslant \mathrm{r}(\boldsymbol{A})+\mathrm{r}(\boldsymbol{B}),
\]
即\(\mathrm{r}(\boldsymbol{A}\boldsymbol{B})\geqslant \mathrm{r}(\boldsymbol{A})+\mathrm{r}(\boldsymbol{B}) - n\).

{\color{blue}证法二:}见\hyperref[example:4.243456]{例题\ref{example:4.243456}}.

\end{proof}

\begin{corollary}\label{corollary:矩阵的秩不等式1}
若\(\boldsymbol{A}\)是\(m\times n\)矩阵,\(\boldsymbol{B}\)是\(n\times t\)矩阵且\(\boldsymbol{A}\boldsymbol{B}=\boldsymbol{O}\),则\(\mathrm{r}(\boldsymbol{A})+\mathrm{r}(\boldsymbol{B})\leqslant  n\).
\end{corollary}

\begin{proposition}[Sylvester不等式的推广]\label{proposition:Sylvester(西尔维斯特)不等式的推广}
设\(\boldsymbol{A}_1,\boldsymbol{A}_2,\cdots,\boldsymbol{A}_m\)为\(n\)阶方阵,求证:
\[
\mathrm{r}(\boldsymbol{A}_1)+\mathrm{r}(\boldsymbol{A}_2)+\cdots+\mathrm{r}(\boldsymbol{A}_m)\leqslant (m - 1)n+\mathrm{r}(\boldsymbol{A}_1\boldsymbol{A}_2\cdots\boldsymbol{A}_m).
\]
特别地,若\(\boldsymbol{A}_1\boldsymbol{A}_2\cdots\boldsymbol{A}_m=\boldsymbol{O}\),则\(\mathrm{r}(\boldsymbol{A}_1)+\mathrm{r}(\boldsymbol{A}_2)+\cdots+\mathrm{r}(\boldsymbol{A}_m)\leqslant (m - 1)n\).
\end{proposition}
\begin{proof}
反复利用\hyperref[proposition:Sylvester(西尔维斯特)不等式]{Sylvester不等式}可得
\begin{align*}
&\mathrm{r}(\boldsymbol{A}_1)+\mathrm{r}(\boldsymbol{A}_2)+\mathrm{r}(\boldsymbol{A}_3)+\cdots+\mathrm{r}(\boldsymbol{A}_m)\\
\leqslant &n+\mathrm{r}(\boldsymbol{A}_1\boldsymbol{A}_2)+\mathrm{r}(\boldsymbol{A}_3)+\cdots+\mathrm{r}(\boldsymbol{A}_m)\\
\leqslant &2n+\mathrm{r}(\boldsymbol{A}_1\boldsymbol{A}_2\boldsymbol{A}_3)+\cdots+\mathrm{r}(\boldsymbol{A}_m)\\
\leqslant &\cdots\leqslant (m - 1)n+\mathrm{r}(\boldsymbol{A}_1\boldsymbol{A}_2\cdots\boldsymbol{A}_m).
\end{align*}

\end{proof}

\begin{example}
设\(\boldsymbol{A},\boldsymbol{B}\)为\(n\)阶方阵,满足\(\boldsymbol{A}\boldsymbol{B}=\boldsymbol{O}\). 证明:若\(n\)是奇数,则\(\boldsymbol{A}\boldsymbol{B}'+\boldsymbol{A}'\boldsymbol{B}\)必为奇异阵;若\(n\)为偶数,举例说明上述结论一般不成立.
\end{example}
\begin{proof}
由\hyperref[corollary:矩阵的秩不等式1]{推论\ref{corollary:矩阵的秩不等式1}}可知,\(\mathrm{r}(\boldsymbol{A})+\mathrm{r}(\boldsymbol{B})\leqslant  n\). 若\(n\)为奇数,则\(\mathrm{r}(\boldsymbol{A}),\mathrm{r}(\boldsymbol{B})\)中至少有一个小于等于\(\frac{n}{2}\),从而小于等于\(\frac{n - 1}{2}\). 不妨设\(\mathrm{r}(\boldsymbol{A})\leqslant \frac{n - 1}{2}\),于是
\[
\mathrm{r}(\boldsymbol{A}\boldsymbol{B}'+\boldsymbol{A}'\boldsymbol{B})\leqslant \mathrm{r}(\boldsymbol{A}\boldsymbol{B}')+\mathrm{r}(\boldsymbol{A}'\boldsymbol{B})\leqslant \mathrm{r}(\boldsymbol{A})+\mathrm{r}(\boldsymbol{A}') = 2\mathrm{r}(\boldsymbol{A})\leqslant  n - 1,
\]
从而\(\boldsymbol{A}\boldsymbol{B}'+\boldsymbol{A}'\boldsymbol{B}\)为奇异阵. 例如,当\(n = 2\)时,令\(\boldsymbol{A}=\boldsymbol{B}=\begin{pmatrix}
0&1\\
0&0
\end{pmatrix}\),则\(\boldsymbol{A}\boldsymbol{B}=\boldsymbol{O}\),但\(\boldsymbol{A}\boldsymbol{B}'+\boldsymbol{A}'\boldsymbol{B}=\boldsymbol{I}_2\)为非异阵.


\end{proof}

\begin{proposition}[Frobenius(弗罗贝尼乌斯)不等式]\label{proposition:Frobenius(弗罗贝尼乌斯)不等式}证明:\(\mathrm{r}(\boldsymbol{A}\boldsymbol{B}\boldsymbol{C})\geqslant \mathrm{r}(\boldsymbol{A}\boldsymbol{B})+\mathrm{r}(\boldsymbol{B}\boldsymbol{C})-\mathrm{r}(\boldsymbol{B})\).
\end{proposition}
\begin{remark}
取$\boldsymbol{B}$为单位矩阵即得\hyperref[proposition:Sylvester(西尔维斯特)不等式]{Sylvester(西尔维斯特)不等式}.
\end{remark}
\begin{proof}
{\color{blue}证法一:}
考虑下列分块初等变换:
\[
\begin{pmatrix}
\boldsymbol{A}\boldsymbol{B}\boldsymbol{C}&\boldsymbol{O}\\
\boldsymbol{O}&\boldsymbol{B}
\end{pmatrix}\to
\begin{pmatrix}
\boldsymbol{A}\boldsymbol{B}\boldsymbol{C}&\boldsymbol{A}\boldsymbol{B}\\
\boldsymbol{O}&\boldsymbol{B}
\end{pmatrix}\to
\begin{pmatrix}
\boldsymbol{O}&\boldsymbol{A}\boldsymbol{B}\\
-\boldsymbol{B}\boldsymbol{C}&\boldsymbol{B}
\end{pmatrix}\to
\begin{pmatrix}
\boldsymbol{A}\boldsymbol{B}&\boldsymbol{O}\\
\boldsymbol{B}&\boldsymbol{B}\boldsymbol{C}
\end{pmatrix}.
\]
由\hyperref[矩阵秩的基本公式3]{矩阵秩的基本公式(3)}和\hyperref[矩阵秩的基本公式4]{矩阵秩的基本公式(4)}可得
\[
\mathrm{r}(\boldsymbol{A}\boldsymbol{B}\boldsymbol{C})+\mathrm{r}(\boldsymbol{B})=\mathrm{r}\begin{pmatrix}
\boldsymbol{A}\boldsymbol{B}\boldsymbol{C}&\boldsymbol{O}\\
\boldsymbol{O}&\boldsymbol{B}
\end{pmatrix}=\mathrm{r}\begin{pmatrix}
\boldsymbol{A}\boldsymbol{B}&\boldsymbol{O}\\
\boldsymbol{B}&\boldsymbol{B}\boldsymbol{C}
\end{pmatrix}\geqslant \mathrm{r}(\boldsymbol{A}\boldsymbol{B})+\mathrm{r}(\boldsymbol{B}\boldsymbol{C}),
\]
由此即得结论. 

{\color{blue}证法二(几何方法):}将问题转化成几何的语言即为:设\(\varphi:V_1\to V_2,\psi:V_2\to V_3,\theta:V_3\to V_4\)是线性映射,证明:\(\text{r}(\theta\psi\varphi)\geqslant \text{r}(\theta\psi)+\text{r}(\psi\varphi)-\text{r}(\psi)\).

下面考虑通过定义域的限制得到的线性映射. 将\(\theta\)的定义域限制在\(\text{Im}\psi\varphi\)上可得线性映射\(\theta_1:\text{Im}\psi\varphi\to V_4\),它的像空间是\(\text{Im}\theta\psi\varphi\),核空间是\(\text{Ker}\theta\cap\text{Im}\psi\varphi\);将\(\theta\)的定义域限制在\(\text{Im}\psi\)上可得线性映射\(\theta_2:\text{Im}\psi\to V_4\),它的像空间是\(\text{Im}\theta\psi\),核空间是\(\text{Ker}\theta\cap\text{Im}\psi\),故由\hyperref[theorem:值域和核空间维数之和等于原像空间维数]{线性映射的维数公式}可得
\begin{align}
\dim(\text{Im}\psi\varphi)&=\dim(\text{Ker}\theta\cap\text{Im}\psi\varphi)+\dim(\text{Im}\theta\psi\varphi),\label{equation:3.334.1}\\
\dim(\text{Im}\psi)&=\dim(\text{Ker}\theta\cap\text{Im}\psi)+\dim(\text{Im}\theta\psi).\label{equation:3.334.2}
\end{align}
注意到\(\text{Im}\psi\varphi\subseteq\text{Im}\psi\),故\(\dim(\text{Ker}\theta\cap\text{Im}\psi\varphi)\leqslant \dim(\text{Ker}\theta\cap\text{Im}\psi)\),从而由\eqref{equation:3.334.1}式和\eqref{equation:3.334.2}式可得
\begin{align*}
&\mathrm{dim}\left( \mathrm{Im}\psi \varphi \right) -\mathrm{dim}\left( \mathrm{Im}\theta \psi \varphi \right) =\mathrm{dim}\left( \mathrm{Ker}\theta \cap \mathrm{Im}\psi \varphi \right) 
\\
&\leqslant \mathrm{dim}\left( \mathrm{Ker}\theta \cap \mathrm{Im}\psi \right) =\mathrm{dim}\left( \mathrm{Im}\psi \right) -\mathrm{dim}\left( \mathrm{Im}\theta \psi \right) .
\end{align*}
于是
\begin{gather*}
\mathrm{dim}\left( \mathrm{Im}\psi \varphi \right) -\mathrm{dim}\left( \mathrm{Im}\theta \psi \varphi \right) \leqslant \mathrm{dim}\left( \mathrm{Im}\psi \right) -\mathrm{dim}\left( \mathrm{Im}\theta \psi \right) 
\\
\Leftrightarrow \text{r}(\psi\varphi)-\text{r}(\theta\psi\varphi)\leqslant \text{r}(\psi)-\text{r}(\theta\psi),
\end{gather*}
结论得证. 

\end{proof}

\begin{proposition}[幂等矩阵关于秩的判定准则]\label{proposition:幂等矩阵关于秩的判定准则}
设数域 $\mathbb{F}$ 和 $A \in \mathbb{F}^{n \times n}$,则\(n\)阶矩阵\(\boldsymbol{A}\)是幂等矩阵(即\(\boldsymbol{A}^2 = \boldsymbol{A}\))的充要条件是:
\[
\mathrm{r}(\boldsymbol{A})+\mathrm{r}(\boldsymbol{I}_n - \boldsymbol{A}) = n.
\]
\end{proposition}
\begin{note}
实际上,由\hyperref[矩阵秩的基本公式5]{矩阵秩的基本公式\ref{矩阵秩的基本公式5}}和\refcor{corollary:矩阵的秩不等式1}可立得.
\end{note}
\begin{proof}
{\color{blue}证法一:}
在下列矩阵的分块初等变换中矩阵的秩保持不变:
\[
\begin{pmatrix}
\boldsymbol{A}&\boldsymbol{O}\\
\boldsymbol{O}&\boldsymbol{I}-\boldsymbol{A}
\end{pmatrix}\to
\begin{pmatrix}
\boldsymbol{A}&\boldsymbol{A}\\
\boldsymbol{O}&\boldsymbol{I}-\boldsymbol{A}
\end{pmatrix}\to
\begin{pmatrix}
\boldsymbol{A}&\boldsymbol{A}\\
\boldsymbol{A}&\boldsymbol{I}
\end{pmatrix}\to
\begin{pmatrix}
\boldsymbol{A}-\boldsymbol{A}^2&\boldsymbol{A}\\
\boldsymbol{O}&\boldsymbol{I}
\end{pmatrix}\to
\begin{pmatrix}
\boldsymbol{A}-\boldsymbol{A}^2&\boldsymbol{O}\\
\boldsymbol{O}&\boldsymbol{I}
\end{pmatrix}.
\]
因此
\[
\mathrm{r}\begin{pmatrix}
\boldsymbol{A}&\boldsymbol{O}\\
\boldsymbol{O}&\boldsymbol{I}-\boldsymbol{A}
\end{pmatrix}=\mathrm{r}\begin{pmatrix}
\boldsymbol{A}-\boldsymbol{A}^2&\boldsymbol{O}\\
\boldsymbol{O}&\boldsymbol{I}
\end{pmatrix},
\]
即\(\mathrm{r}(\boldsymbol{A})+\mathrm{r}(\boldsymbol{I}-\boldsymbol{A})=\mathrm{r}(\boldsymbol{A}-\boldsymbol{A}^2)+n\),由此即得结论.

{\color{blue}证法二:}将$A$看作矩阵$A$左乘诱导在$\mathbb{F}^n$上的线性变换,记$f(x)=x^2-x$,$f_1(x)=x$,$f_2(x)=x-1$,则$f=f_1f_2$,$(f_1,f_2)=1$.从而由\refthe{theorem:线性变换互素分解对应核子空间直和分解}知
\begin{align}
\ker f(A) = \ker f_1(A) \oplus \ker f_2(A) = \ker A \oplus \ker (A - I). \label{eq:103.112}
\end{align}
注意到
\begin{align*}
A^2 = A \Longleftrightarrow \ker f(A) = \mathbb{F}^n \Longleftrightarrow \dim \ker f(A) = n,
\end{align*}
故再由\eqref{eq:103.112}式和\hyperref[theorem:值域和核空间维数之和等于原像空间维数]{维数公式}可得
\begin{align*}
&\qquad \,\, \mathrm{dim}\,\mathrm{ker}A+\mathrm{dimker}\left( A-I \right) =n
\\
&\Longleftrightarrow n-\mathrm{dim}\,\mathrm{Im}A+n-\mathrm{dim}\,\mathrm{Im}\left( A-I \right) =n
\\
&\Longleftrightarrow \mathrm{dim}\,\mathrm{Im}A+\mathrm{dim}\,\mathrm{Im}\left( A-I \right) =n
\\
&\Longleftrightarrow r\left( A \right) +r\left( A-I \right) =n.
\end{align*}

\end{proof}

\begin{example}
设$n$阶方阵$A,B$满足:$(A+B)^2=A+B$,$r(A+B)=r(A)+r(B)$,证明:
\begin{align*}
A^2=A, \quad B^2=B, \quad AB=BA=O.
\end{align*}
\end{example}
\begin{proof}
{\color{blue}证法一:}由\refpro{proposition:幂等矩阵关于秩的判定准则}可得$n=r(A+B)+r(I_n - A - B)=r(A)+r(B)+r(I_n - A - B)$。构造如下分块对角阵,并对其实施分块初等变换,可得
\begin{align*}
&\begin{pmatrix}
A & O & O \\
O & B & O \\
O & O & I_n - A - B
\end{pmatrix} \to \begin{pmatrix}
A & O & O \\
O & B & O \\
A & B & I_n - A - B
\end{pmatrix} \to \begin{pmatrix}
A & O & A \\
O & B & B \\
A & B & I_n
\end{pmatrix} \\
&\to \begin{pmatrix}
A - A^2 & -AB & O \\
-BA & B - B^2 & O \\
A & B & I_n
\end{pmatrix} \to \begin{pmatrix}
A - A^2 & -AB & O \\
-BA & B - B^2 & O \\
O & O & I_n
\end{pmatrix}.
\end{align*}
注意到分块初等变换不改变矩阵的秩,故可得$r\begin{pmatrix}\begin{array}{cc}A - A^2 & -AB \\ -BA & B - B^2\end{array}\end{pmatrix}=0$,从而$A^2=A$,$B^2=B$,$AB=BA=O$。

{\color{blue}证法二:}由\refpro{proposition:矩阵和的秩等于秩的和可同时相抵标准化}知,存在可逆阵$P,Q$,使得
\begin{align*}
PAQ=\left( \begin{matrix}
I_r&		O&		O\\
O&		O&		O\\
O&		O&		O\\
\end{matrix} \right) ,\quad PBQ=\left( \begin{matrix}
O&		O&		O\\
O&		I_s&		O\\
O&		O&		O\\
\end{matrix} \right) ,
\end{align*}
其中$r=r(A),s=r(B).$记
\begin{align*}
D=Q^{-1}P^{-1}=\begin{pmatrix}
D_1&	D_2&	D_3\\
D_4&	D_5&	D_6\\
D_7&	D_8&	D_9\\
\end{pmatrix},
\end{align*}
则由$(A+B)^2=A+B$可得
\begin{align*}
&\qquad \,\, P\left( A+B \right) ^2Q=P\left( A+B \right) Q
\\
&\Longleftrightarrow P\left( A+B \right) Q\left( Q^{-1}P^{-1} \right) P\left( A+B \right) Q=P\left( A+B \right) Q
\\
&\Longleftrightarrow \left( \begin{matrix}
I_r&		O&		O\\
O&		I_s&		O\\
O&		O&		O\\
\end{matrix} \right) \left( \begin{matrix}
D_1&		D_2&		D_3\\
D_4&		D_5&		D_6\\
D_7&		D_8&		D_9\\
\end{matrix} \right) \left( \begin{matrix}
I_r&		O&		O\\
O&		I_s&		O\\
O&		O&		O\\
\end{matrix} \right) =\left( \begin{matrix}
I_r&		O&		O\\
O&		I_s&		O\\
O&		O&		O\\
\end{matrix} \right) 
\\
&\Longleftrightarrow \left( \begin{matrix}
D_1&		D_2&		O\\
D_4&		D_5&		O\\
O&		O&		O\\
\end{matrix} \right) =\left( \begin{matrix}
I_r&		O&		O\\
O&		I_s&		O\\
O&		O&		O\\
\end{matrix} \right) ,
\end{align*}
故$D_1=I_r,D_5=I_s,D_2=D_4=O$,因此
\begin{align*}
D=Q^{-1}P^{-1}=\begin{pmatrix}
I_r&	O&	D_3\\
O&	I_s&	D_6\\
D_7&	D_8&	D_9\\
\end{pmatrix}.
\end{align*}
从而
\begin{align*}
PA^2Q&=PAQ\left( Q^{-1}P^{-1} \right) PAQ\\
&=\begin{pmatrix}
I_r&	O&	O\\
O&	O&	O\\
O&	O&	O\\
\end{pmatrix} \begin{pmatrix}
I_r&	O&	D_3\\
O&	I_s&	D_6\\
D_7&	D_8&	D_9\\
\end{pmatrix} \begin{pmatrix}
I_r&	O&	O\\
O&	O&	O\\
O&	O&	O\\
\end{pmatrix}\\
&=\begin{pmatrix}
I_r&	O&	O\\
O&	O&	O\\
O&	O&	O\\
\end{pmatrix}=PAQ\Longrightarrow A^2=A,
\end{align*}
\begin{align*}
PB^2Q&=PBQ\left( Q^{-1}P^{-1} \right) PBQ\\
&=\begin{pmatrix}
O&	O&	O\\
O&	I_s&	O\\
O&	O&	O\\
\end{pmatrix} \begin{pmatrix}
I_r&	O&	D_3\\
O&	I_s&	D_6\\
D_7&	D_8&	D_9\\
\end{pmatrix} \begin{pmatrix}
O&	O&	O\\
O&	I_s&	O\\
O&	O&	O\\
\end{pmatrix}\\
&=\begin{pmatrix}
O&	O&	O\\
O&	I_s&	O\\
O&	O&	O\\
\end{pmatrix}=PBQ\Longrightarrow B^2=B,
\end{align*}
\begin{align*}
PABQ&=PAQ\left( Q^{-1}P^{-1} \right) PBQ\\
&=\begin{pmatrix}
I_r&	O&	O\\
O&	O&	O\\
O&	O&	O\\
\end{pmatrix} \begin{pmatrix}
I_r&	O&	D_3\\
O&	I_s&	D_6\\
D_7&	D_8&	D_9\\
\end{pmatrix} \begin{pmatrix}
O&	O&	O\\
O&	I_s&	O\\
O&	O&	O\\
\end{pmatrix}\\
&=O\Longrightarrow AB=O,
\end{align*}
\begin{align*}
PBAQ&=PBQ\left( Q^{-1}P^{-1} \right) PAQ\\
&=\begin{pmatrix}
O&	O&	O\\
O&	I_s&	O\\
O&	O&	O\\
\end{pmatrix} \begin{pmatrix}
I_r&	O&	D_3\\
O&	I_s&	D_6\\
D_7&	D_8&	D_9\\
\end{pmatrix} \begin{pmatrix}
I_r&	O&	O\\
O&	O&	O\\
O&	O&	O\\
\end{pmatrix}\\
&=O\Longrightarrow BA=O.
\end{align*}

\end{proof}

\begin{proposition}[对合矩阵关于秩的判定准则]\label{proposition:对合矩阵关于秩的判定准则}
求证:\(n\)阶矩阵\(\boldsymbol{A}\)是对合矩阵(即\(\boldsymbol{A}^2 = \boldsymbol{I}_n\))的充要条件是:
\[
\mathrm{r}(\boldsymbol{I}_n+\boldsymbol{A})+\mathrm{r}(\boldsymbol{I}_n - \boldsymbol{A}) = n.
\]
\end{proposition}
\begin{note}
实际上,由\hyperref[矩阵秩的基本公式5]{矩阵秩的基本公式\ref{矩阵秩的基本公式5}}和\refcor{corollary:矩阵的秩不等式1}可立得.
\end{note}
\begin{proof}
在下列矩阵的分块初等变换中,矩阵的秩保持不变:
\[
\begin{pmatrix}
\boldsymbol{I}_n+\boldsymbol{A}&\boldsymbol{O}\\
\boldsymbol{O}&\boldsymbol{I}_n - \boldsymbol{A}
\end{pmatrix}\to
\begin{pmatrix}
\boldsymbol{I}_n+\boldsymbol{A}&\boldsymbol{I}_n+\boldsymbol{A}\\
\boldsymbol{O}&\boldsymbol{I}_n - \boldsymbol{A}
\end{pmatrix}\to
\begin{pmatrix}
\boldsymbol{I}_n+\boldsymbol{A}&\boldsymbol{I}_n+\boldsymbol{A}\\
\boldsymbol{I}_n+\boldsymbol{A}&2\boldsymbol{I}_n
\end{pmatrix}\to
\begin{pmatrix}
\frac{1}{2}(\boldsymbol{I}_n - \boldsymbol{A}^2)&\boldsymbol{I}_n+\boldsymbol{A}\\
\boldsymbol{O}&2\boldsymbol{I}_n
\end{pmatrix}\to
\begin{pmatrix}
\frac{1}{2}(\boldsymbol{I}_n - \boldsymbol{A}^2)&\boldsymbol{O}\\
\boldsymbol{O}&2\boldsymbol{I}_n
\end{pmatrix}.
\]
因此
\[
\mathrm{r}\begin{pmatrix}
\boldsymbol{I}_n+\boldsymbol{A}&\boldsymbol{O}\\
\boldsymbol{O}&\boldsymbol{I}_n - \boldsymbol{A}
\end{pmatrix}=\mathrm{r}\begin{pmatrix}
\frac{1}{2}(\boldsymbol{I}_n - \boldsymbol{A}^2)&\boldsymbol{O}\\
\boldsymbol{O}&2\boldsymbol{I}_n
\end{pmatrix},
\]
即\(\mathrm{r}(\boldsymbol{I}_n+\boldsymbol{A})+\mathrm{r}(\boldsymbol{I}_n - \boldsymbol{A})=\mathrm{r}(\boldsymbol{I}_n - \boldsymbol{A}^2)+n\),由此即得结论. 

\end{proof}

\begin{example}
设\(\boldsymbol{A}\)是\(n\)阶矩阵,求证:\(\mathrm{r}(\boldsymbol{A})+\mathrm{r}(\boldsymbol{I}_n+\boldsymbol{A})\geqslant  n\).
\end{example}
\begin{proof}
{\color{blue}证法一:}
由下列分块初等变换即得结论
\[
\begin{pmatrix}
\boldsymbol{A}&\boldsymbol{O}\\
\boldsymbol{O}&\boldsymbol{I}+\boldsymbol{A}
\end{pmatrix}\to
\begin{pmatrix}
\boldsymbol{A}&\boldsymbol{A}\\
\boldsymbol{O}&\boldsymbol{I}+\boldsymbol{A}
\end{pmatrix}\to
\begin{pmatrix}
\boldsymbol{A}&\boldsymbol{A}\\
-\boldsymbol{A}&\boldsymbol{I}
\end{pmatrix}\to
\begin{pmatrix}
\boldsymbol{A}+\boldsymbol{A}^2&\boldsymbol{A}\\
\boldsymbol{O}&\boldsymbol{I}
\end{pmatrix}\to
\begin{pmatrix}
\boldsymbol{A}+\boldsymbol{A}^2&\boldsymbol{O}\\
\boldsymbol{O}&\boldsymbol{I}
\end{pmatrix}.
\]

{\color{blue}证法二:}
\(\mathrm{r}(\boldsymbol{A})+\mathrm{r}(\boldsymbol{I}+\boldsymbol{A})=\mathrm{r}(-\boldsymbol{A})+\mathrm{r}(\boldsymbol{I}+\boldsymbol{A})\geqslant \mathrm{r}(-\boldsymbol{A}+\boldsymbol{I}+\boldsymbol{A})=\mathrm{r}(\boldsymbol{I}) = n\).

\end{proof}

\begin{proposition}[秩的降阶公式]\label{proposition:秩的降阶公式}
设有分块矩阵\(M = \begin{pmatrix}
\boldsymbol{A}&\boldsymbol{B}\\
\boldsymbol{C}&\boldsymbol{D}
\end{pmatrix}\),证明:
\begin{enumerate}[(1)]
\item 若\(\boldsymbol{A}\)可逆,则\(\mathrm{r}(M)=\mathrm{r}(\boldsymbol{A})+\mathrm{r}(\boldsymbol{D}-\boldsymbol{C}\boldsymbol{A}^{-1}\boldsymbol{B})\);

\item 若\(\boldsymbol{D}\)可逆,则\(\mathrm{r}(M)=\mathrm{r}(\boldsymbol{D})+\mathrm{r}(\boldsymbol{A}-\boldsymbol{B}\boldsymbol{D}^{-1}\boldsymbol{C})\);

\item 若\(\boldsymbol{A},\boldsymbol{D}\)都可逆,则\(\mathrm{r}(\boldsymbol{A})+\mathrm{r}(\boldsymbol{D}-\boldsymbol{C}\boldsymbol{A}^{-1}\boldsymbol{B})=\mathrm{r}(\boldsymbol{D})+\mathrm{r}(\boldsymbol{A}-\boldsymbol{B}\boldsymbol{D}^{-1}\boldsymbol{C})\).
\end{enumerate}
\end{proposition}
\begin{proof}
\begin{enumerate}[(1)]
\item 由分块初等变换可得
\[
\begin{pmatrix}
\boldsymbol{A}&\boldsymbol{B}\\
\boldsymbol{C}&\boldsymbol{D}
\end{pmatrix}\to
\begin{pmatrix}
\boldsymbol{A}&\boldsymbol{B}\\
\boldsymbol{O}&\boldsymbol{D}-\boldsymbol{C}\boldsymbol{A}^{-1}\boldsymbol{B}
\end{pmatrix}\to
\begin{pmatrix}
\boldsymbol{A}&\boldsymbol{O}\\
\boldsymbol{O}&\boldsymbol{D}-\boldsymbol{C}\boldsymbol{A}^{-1}\boldsymbol{B}
\end{pmatrix},
\]
由此即得结论.

\item 同理可证明.

\item 由(1)和(2)即得. 
\end{enumerate}

\end{proof}

\begin{example}
设
\[
M = \begin{pmatrix}
a_1^2&a_1a_2 + 1&\cdots&a_1a_n + 1\\
a_2a_1 + 1&a_2^2&\cdots&a_2a_n + 1\\
\vdots&\vdots&&\vdots\\
a_na_1 + 1&a_na_2 + 1&\cdots&a_n^2
\end{pmatrix},
\]
证明:\(\mathrm{r}(M)\geqslant  n - 1\),等号成立当且仅当\(|M| = 0\).
\end{example}
\begin{proof}
若\(n = 1\),结论显然成立. 下设\(n\geqslant 2\). 取\(\boldsymbol{A}=\begin{pmatrix}
a_1&a_2&\cdots&a_n\\
1&1&\cdots&1
\end{pmatrix}\),则
\begin{align*}
\boldsymbol{M}=-\boldsymbol{I}_n+\left( \begin{matrix}
a_1&		1\\
a_2&		1\\
\vdots&		\vdots\\
a_n&		1\\
\end{matrix} \right) \boldsymbol{I}_{2}^{-1}\left( \begin{matrix}
a_1&		a_2&		\cdots&		a_n\\
1&		1&		\cdots&		1\\
\end{matrix} \right) =-\boldsymbol{I}_n+\boldsymbol{A}'\boldsymbol{I}_{2}^{-1}\boldsymbol{A}=-\left( \boldsymbol{I}_n-\boldsymbol{A}'\boldsymbol{I}_{2}^{-1}\boldsymbol{A} \right) .
\end{align*}
由\hyperref[proposition:秩的降阶公式]{秩的降阶公式}可得
\begin{align*}
2+\mathrm{r}\left( \boldsymbol{M} \right) =2+\mathrm{r}\left( -\boldsymbol{M} \right) =\mathrm{r}\left( \boldsymbol{I}_2 \right) +\mathrm{r}\left( \boldsymbol{I}_n-\boldsymbol{A'}\boldsymbol{I}_{2}^{-1}\boldsymbol{A} \right) =\mathrm{r}\left( \boldsymbol{I}_n \right) +\mathrm{r}\left( \boldsymbol{I}_2-\boldsymbol{A}\boldsymbol{I}_{n}^{-1}\boldsymbol{A}' \right) =n+\mathrm{r}\left( \begin{matrix}
\sum_{i=1}^n{a_{i}^{2}}-1&		\sum_{i=1}^n{a_i}\\
\sum_{i=1}^n{a_i}&		n-1\\
\end{matrix} \right) .
\end{align*}
而$\mathrm{r}\left( \begin{matrix}
\sum_{i=1}^n{a_{i}^{2}}-1&		\sum_{i=1}^n{a_i}\\
\sum_{i=1}^n{a_i}&		n-1\\
\end{matrix} \right) \geqslant 1$,
于是\(\mathrm{r}\left( \boldsymbol{M} \right) =n-2+\mathrm{r}\left( \begin{matrix}
\sum_{i=1}^n{a_{i}^{2}}-1&		\sum_{i=1}^n{a_i}\\
\sum_{i=1}^n{a_i}&		n-1\\
\end{matrix} \right) \geqslant n-1.\),等号成立当且仅当\(M\)不满秩,即\(|M| = 0\).

\end{proof}

\begin{proposition}\label{proposition:矩阵乘法可交换诱导的秩不等式}
设\(\boldsymbol{A},\boldsymbol{B}\)都是数域\(\mathbb{K}\)上的\(n\)阶矩阵且\(\boldsymbol{A}\boldsymbol{B}=\boldsymbol{B}\boldsymbol{A}\),证明:
\[
\mathrm{r(}\boldsymbol{A}+\boldsymbol{B})\leqslant \mathrm{r(}\boldsymbol{A})+\mathrm{r(}\boldsymbol{B})-\mathrm{r(}\boldsymbol{A}\,\,\,\boldsymbol{B}).
\]
\end{proposition}
\begin{proof}
将$\boldsymbol{A},\boldsymbol{B}$分别看作在$\mathbb{R} ^n$上左乘诱导的线性变换,则
\begin{align}
&\quad \quad \mathrm{r(}\boldsymbol{AB})+\mathrm{r(}\boldsymbol{A}\,,\,\boldsymbol{B})\leqslant \mathrm{r(}\boldsymbol{A})+\mathrm{r(}\boldsymbol{B})
\nonumber \\
&\Longleftrightarrow \mathrm{dim}\,\mathrm{Im}\boldsymbol{AB}+\mathrm{dim}\left( \,\mathrm{Im}\boldsymbol{A}+\,\mathrm{Im}\boldsymbol{B} \right) \leqslant \mathrm{dim}\,\mathrm{Im}\boldsymbol{A}+\mathrm{dim}\,\mathrm{Im}\boldsymbol{B}
\nonumber \\
&\Longleftrightarrow \mathrm{dim}\,\mathrm{Im}\boldsymbol{AB}\leqslant \mathrm{dim}\left( \,\mathrm{Im}\boldsymbol{A}\cap \,\mathrm{Im}\boldsymbol{B} \right) .\label{eq:::23924234252342423423423324057823802344223452322}
\end{align}
任取$\boldsymbol{ABx}\in \mathrm{Im}\boldsymbol{AB}$,则由$\boldsymbol{AB}=\boldsymbol{BA}$可知
\begin{align*}
\boldsymbol{A}\left( \boldsymbol{Bx} \right) \in \mathrm{Im}\boldsymbol{A},\quad \boldsymbol{ABx}=\boldsymbol{B}\left( \boldsymbol{Ax} \right) \in \mathrm{Im}\boldsymbol{B}.
\end{align*}
故$\mathrm{Im}\boldsymbol{AB}\subseteq \mathrm{Im}\boldsymbol{A}\cap \mathrm{Im}\boldsymbol{B}$.因此\eqref{eq:::23924234252342423423423324057823802344223452322}式成立,结论得证.

\end{proof}

\begin{proposition}\label{proposition:矩阵乘法可交换的秩不等式}
设\(\boldsymbol{A},\boldsymbol{B}\)都是数域\(\mathbb{K}\)上的\(n\)阶矩阵且\(\boldsymbol{A}\boldsymbol{B}=\boldsymbol{B}\boldsymbol{A}\),证明:
\[
\mathrm{r}(\boldsymbol{A}+\boldsymbol{B})\leqslant \mathrm{r}(\boldsymbol{A})+\mathrm{r}(\boldsymbol{B})-\mathrm{r}(\boldsymbol{A}\boldsymbol{B}).
\]
\end{proposition}
\begin{remark}
{\color{blue}证法一:}这里乘的不只是初等变换矩阵.记住这个分块矩阵乘法和构造.

{\color{blue}证法二:}和{\color{blue}证法三:}思路分析:将秩不等式转化为维数公式就能自然得到证明的想法.
\end{remark}
\begin{proof}
{\color{blue}证法一:}
考虑如下分块矩阵的乘法:
\[
\begin{pmatrix}
\boldsymbol{I}&\boldsymbol{I}\\
\boldsymbol{O}&\boldsymbol{I}
\end{pmatrix}
\begin{pmatrix}
\boldsymbol{A}&\boldsymbol{O}\\
\boldsymbol{O}&\boldsymbol{B}
\end{pmatrix}
\begin{pmatrix}
\boldsymbol{I}&-\boldsymbol{B}\\
\boldsymbol{I}&\boldsymbol{A}
\end{pmatrix}=
\begin{pmatrix}
\boldsymbol{A}+\boldsymbol{B}&-\boldsymbol{A}\boldsymbol{B}+\boldsymbol{B}\boldsymbol{A}\\
\boldsymbol{B}&\boldsymbol{B}\boldsymbol{A}
\end{pmatrix}=
\begin{pmatrix}
\boldsymbol{A}+\boldsymbol{B}&\boldsymbol{O}\\
\boldsymbol{B}&\boldsymbol{A}\boldsymbol{B}
\end{pmatrix}.
\]
由\hyperref[矩阵秩的基本公式2]{矩阵秩的基本公式(2)}和\hyperref[矩阵秩的基本公式4]{矩阵秩的基本公式(4)}可得
\[
\mathrm{r}(\boldsymbol{A})+\mathrm{r}(\boldsymbol{B})=\mathrm{r}\begin{pmatrix}
\boldsymbol{A}&\boldsymbol{O}\\
\boldsymbol{O}&\boldsymbol{B}
\end{pmatrix}\geqslant \mathrm{r}\begin{pmatrix}
\boldsymbol{A}+\boldsymbol{B}&\boldsymbol{O}\\
\boldsymbol{B}&\boldsymbol{B}\boldsymbol{A}
\end{pmatrix}\geqslant \mathrm{r}(\boldsymbol{A}+\boldsymbol{B})+\mathrm{r}(\boldsymbol{A}\boldsymbol{B}),
\]
由此即得结论.

{\color{blue}证法二:}设\(V_{\boldsymbol{A}}\)是方程组\(\boldsymbol{A}\boldsymbol{x}=\boldsymbol{0}\)的解空间,\(V_{\boldsymbol{B}},V_{\boldsymbol{A}\boldsymbol{B}},V_{\boldsymbol{A}+\boldsymbol{B}}\)的意义同理. 若列向量\(\boldsymbol{\alpha}\in V_{\boldsymbol{A}}\cap V_{\boldsymbol{B}}\),即\(\boldsymbol{\alpha}\)满足\(\boldsymbol{A}\boldsymbol{\alpha}=\boldsymbol{0}\)且\(\boldsymbol{B}\boldsymbol{\alpha}=\boldsymbol{0}\),于是\((\boldsymbol{A}+\boldsymbol{B})\boldsymbol{\alpha}=\boldsymbol{0}\),即\(\boldsymbol{\alpha}\in V_{\boldsymbol{A}+\boldsymbol{B}}\),从而\(V_{\boldsymbol{A}}\cap V_{\boldsymbol{B}}\subseteq V_{\boldsymbol{A}+\boldsymbol{B}}\). 同理可证\(V_{\boldsymbol{A}}\subseteq V_{\boldsymbol{B}\boldsymbol{A}}\),\(V_{\boldsymbol{B}}\subseteq V_{\boldsymbol{A}\boldsymbol{B}}\). 因为\(\boldsymbol{A}\boldsymbol{B}=\boldsymbol{B}\boldsymbol{A}\),所以\(V_{\boldsymbol{B}\boldsymbol{A}} = V_{\boldsymbol{A}\boldsymbol{B}}\),从而\(V_{\boldsymbol{A}}+V_{\boldsymbol{B}}\subseteq V_{\boldsymbol{A}\boldsymbol{B}}\). 因此,我们有
\[
\dim(V_{\boldsymbol{A}}\cap V_{\boldsymbol{B}})\leqslant \dim V_{\boldsymbol{A}+\boldsymbol{B}}=n - \mathrm{r}(\boldsymbol{A}+\boldsymbol{B}),\quad\dim(V_{\boldsymbol{A}}+V_{\boldsymbol{B}})\leqslant \dim V_{\boldsymbol{A}\boldsymbol{B}}=n - \mathrm{r}(\boldsymbol{A}\boldsymbol{B}).
\]
将上面两个不等式相加,再由交和空间维数公式可得
\begin{align*}
n - \mathrm{r}(\boldsymbol{A}+\boldsymbol{B})+n - \mathrm{r}(\boldsymbol{A}\boldsymbol{B})&\geqslant \dim(V_{\boldsymbol{A}}\cap V_{\boldsymbol{B}})+\dim(V_{\boldsymbol{A}}+V_{\boldsymbol{B}})\\
&=\dim V_{\boldsymbol{A}}+\dim V_{\boldsymbol{B}}=n - \mathrm{r}(\boldsymbol{A})+n - \mathrm{r}(\boldsymbol{B}),
\end{align*}
因此\(\mathrm{r}(\boldsymbol{A}+\boldsymbol{B})+\mathrm{r}(\boldsymbol{A}\boldsymbol{B})\leqslant \mathrm{r}(\boldsymbol{A})+\mathrm{r}(\boldsymbol{B})\),结论得证.

{\color{blue}证法三:}设\(\boldsymbol{A}=(\boldsymbol{\alpha}_1,\boldsymbol{\alpha}_2,\cdots,\boldsymbol{\alpha}_n)\)为\(\boldsymbol{A}\)的列分块,\(\boldsymbol{B}=(\boldsymbol{\beta}_1,\boldsymbol{\beta}_2,\cdots,\boldsymbol{\beta}_n)\)为\(\boldsymbol{B}\)的列分块. 记\(U_{\boldsymbol{A}} = L(\boldsymbol{\alpha}_1,\boldsymbol{\alpha}_2,\cdots,\boldsymbol{\alpha}_n)\)为\(\boldsymbol{A}\)的列向量生成的\(\mathbb{K}^n\)的子空间,\(U_{\boldsymbol{B}},U_{\boldsymbol{A}\boldsymbol{B}},U_{\boldsymbol{A}+\boldsymbol{B}}\)的意义同理. 因为向量组\(\boldsymbol{\alpha}_1,\boldsymbol{\alpha}_2,\cdots,\boldsymbol{\alpha}_n\)的极大无关组就是\(L(\boldsymbol{\alpha}_1,\boldsymbol{\alpha}_2,\cdots,\boldsymbol{\alpha}_n)\)的一组基,故\(\mathrm{r}(\boldsymbol{A})=\dim U_{\boldsymbol{A}}\),关于\(\boldsymbol{B},\boldsymbol{A}\boldsymbol{B},\boldsymbol{A}+\boldsymbol{B}\)的等式同理可得. 由\refpro{proposition:与全空间维数相同的子空间等于全空间},我们有\(U_{\boldsymbol{A}+\boldsymbol{B}}\subseteq U_{\boldsymbol{A}}+U_{\boldsymbol{B}}\). 注意到\(\boldsymbol{A}\boldsymbol{B}=(\boldsymbol{A}\boldsymbol{\beta}_1,\boldsymbol{A}\boldsymbol{\beta}_2,\cdots,\boldsymbol{A}\boldsymbol{\beta}_n)\),若设\(\boldsymbol{\beta}_j=(b_{1j},b_{2j},\cdots,b_{nj})'\),则\(\boldsymbol{A}\boldsymbol{B}\)的列向量\(\boldsymbol{A}\boldsymbol{\beta}_j=b_{1j}\boldsymbol{\alpha}_1 + b_{2j}\boldsymbol{\alpha}_2+\cdots + b_{nj}\boldsymbol{\alpha}_n\in U_{\boldsymbol{A}}\),从而\(U_{\boldsymbol{A}\boldsymbol{B}}\subseteq U_{\boldsymbol{A}}\). 同理可得\(U_{\boldsymbol{B}\boldsymbol{A}}\subseteq U_{\boldsymbol{B}}\).又因为\(\boldsymbol{A}\boldsymbol{B}=\boldsymbol{B}\boldsymbol{A}\),故\(U_{\boldsymbol{A}\boldsymbol{B}}\subseteq U_{\boldsymbol{A}}\cap U_{\boldsymbol{B}}\). 最后,由上述包含关系以及交和空间维数公式可得
\begin{align*}
\mathrm{r}(\boldsymbol{A}+\boldsymbol{B})+\mathrm{r}(\boldsymbol{A}\boldsymbol{B})&=\dim U_{\boldsymbol{A}+\boldsymbol{B}}+\dim U_{\boldsymbol{A}\boldsymbol{B}}\leqslant \dim(U_{\boldsymbol{A}}+U_{\boldsymbol{B}})+\dim(U_{\boldsymbol{A}}\cap U_{\boldsymbol{B}})\\
&=\dim U_{\boldsymbol{A}}+\dim U_{\boldsymbol{B}}=\mathrm{r}(\boldsymbol{A})+\mathrm{r}(\boldsymbol{B}).
\end{align*}

{\color{blue}证法四:}由\refpro{proposition:矩阵乘法可交换诱导的秩不等式}和\hyperref[proposition:矩阵秩的基本公式]{矩阵秩的基本公式}立得.

\end{proof}

\begin{example}
设$\boldsymbol{A},\boldsymbol{B}$为$n$阶方阵,且$\boldsymbol{AB} = \boldsymbol{A} + 2025\boldsymbol{B}$,证明:
$$\mathbf{r}\left( \boldsymbol{A} \right) +\mathbf{r}\left( \boldsymbol{B} \right) \geqslant \mathbf{r}\left( \boldsymbol{A}+\boldsymbol{B} \right) +\mathbf{r}\left( \boldsymbol{AB} \right) .$$
\end{example}
\begin{proof}
{\color{blue}证法一:}由条件可知
$$\left( 2025\boldsymbol{I}_n-\boldsymbol{A} \right) \left( \boldsymbol{I}_n-\boldsymbol{B} \right) =2025\boldsymbol{I}_n+\boldsymbol{AB}-\boldsymbol{A}-2025\boldsymbol{B}=2025\boldsymbol{I}_n.$$
故$2025\boldsymbol{I}_n-\boldsymbol{A}$和$\frac{1}{2025}\left( \boldsymbol{I}_n-\boldsymbol{B} \right)$互为其逆矩阵.于是
$$2025\boldsymbol{I}_n=\left( \boldsymbol{I}_n-\boldsymbol{B} \right) \left( 2025\boldsymbol{I}_n-\boldsymbol{A} \right) =2025\boldsymbol{I}_n+\boldsymbol{BA}-\boldsymbol{A}-2025\boldsymbol{B}.$$
从而
$$\boldsymbol{BA}=\boldsymbol{A}+2025\boldsymbol{B}=\boldsymbol{AB}.$$
故由\refpro{proposition:矩阵乘法可交换的秩不等式}知
$$\mathbf{r}\left( \boldsymbol{A} \right) +\mathbf{r}\left( \boldsymbol{B} \right) \geqslant \mathbf{r}\left( \boldsymbol{A}+\boldsymbol{B} \right) +\mathbf{r}\left( \boldsymbol{AB} \right) .$$

{\color{blue}证法二:}由条件可知
$$\boldsymbol{A}\left( \boldsymbol{B}-\boldsymbol{I} \right) =\boldsymbol{AB}-\boldsymbol{A}=2025\boldsymbol{B}.$$
考虑如下分块矩阵乘法
$$\left( \begin{matrix}
\boldsymbol{I}&		\boldsymbol{I}\\
\boldsymbol{O}&		\boldsymbol{I}\\
\end{matrix} \right) \left( \begin{matrix}
\boldsymbol{A}&		\boldsymbol{O}\\
\boldsymbol{O}&		\boldsymbol{B}\\
\end{matrix} \right) \left( \begin{matrix}
\boldsymbol{I}&		\boldsymbol{B}-\boldsymbol{I}\\
\boldsymbol{I}&		-2025\boldsymbol{I}\\
\end{matrix} \right) =\left( \begin{matrix}
\boldsymbol{I}&		\boldsymbol{I}\\
\boldsymbol{O}&		\boldsymbol{I}\\
\end{matrix} \right) \left( \begin{matrix}
\boldsymbol{A}&		2025\boldsymbol{B}\\
\boldsymbol{B}&		-2025\boldsymbol{B}\\
\end{matrix} \right) =\left( \begin{matrix}
\boldsymbol{A}+\boldsymbol{B}&		\boldsymbol{O}\\
\boldsymbol{B}&		-2025\boldsymbol{B}\\
\end{matrix} \right) .$$
由\hyperref[proposition:矩阵秩的基本公式]{矩阵秩的基本公式}知
\begin{align*}
&\mathbf{r}\left( \boldsymbol{A} \right) +\mathbf{r}\left( \boldsymbol{B} \right) =\mathbf{r}\left( \begin{matrix}
\boldsymbol{A}&		\boldsymbol{O}\\
\boldsymbol{O}&		\boldsymbol{B}\\
\end{matrix} \right) \geqslant \mathbf{r}\left( \begin{matrix}
\boldsymbol{A}+\boldsymbol{B}&		\boldsymbol{O}\\
\boldsymbol{B}&		-2025\boldsymbol{B}\\
\end{matrix} \right) 
\\
&\geqslant \mathbf{r}\left( \boldsymbol{A}+\boldsymbol{B} \right) +\mathrm{r}\left( -2025\boldsymbol{B} \right) =\mathbf{r}\left( \boldsymbol{A}+\boldsymbol{B} \right) +\mathrm{r}\left( \boldsymbol{B} \right) 
\\
&\geqslant \mathbf{r}\left( \boldsymbol{A}+\boldsymbol{B} \right) +\mathrm{r}\left( \boldsymbol{AB} \right) .
\end{align*}

\end{proof}

\begin{proposition}\label{proposition:矩阵与单位阵差相关不等式}
设 $\boldsymbol{A},\boldsymbol{B}$ 为 $n$ 阶矩阵. 证明
\[
\mathrm{rank}(\boldsymbol{AB} - \boldsymbol{I}_n) \leqslant \mathrm{rank}(\boldsymbol{A} - \boldsymbol{I}_n) + \mathrm{rank}(\boldsymbol{B} - \boldsymbol{I}_n).
\]
\end{proposition}
\begin{proof}
{\color{blue}证法一:}因为
\[
\begin{pmatrix} 
\boldsymbol{A} - \boldsymbol{I}_n & \boldsymbol{O} \\
\boldsymbol{O} & \boldsymbol{B} - \boldsymbol{I}_n 
\end{pmatrix} \to \begin{pmatrix} 
\boldsymbol{A} - \boldsymbol{I}_n & \boldsymbol{AB} - \boldsymbol{B} \\
\boldsymbol{O} & \boldsymbol{B} - \boldsymbol{I}_n 
\end{pmatrix} \to \begin{pmatrix} 
\boldsymbol{A} - \boldsymbol{I}_n & \boldsymbol{AB} - \boldsymbol{B} \\
\boldsymbol{A} - \boldsymbol{I}_n & \boldsymbol{AB} - \boldsymbol{I}_n 
\end{pmatrix},
\]
所以由\nrefpro{proposition:矩阵秩的基本公式}{(2)}和\nrefpro{proposition:矩阵秩的基本公式}{(7)}可得
\[
\mathrm{rank}(\boldsymbol{A} - \boldsymbol{I}_n) + \mathrm{rank}(\boldsymbol{B} - \boldsymbol{I}_n) = \mathrm{rank}\begin{pmatrix} 
\boldsymbol{A} - \boldsymbol{I}_n & \boldsymbol{AB} - \boldsymbol{B} \\
\boldsymbol{A} - \boldsymbol{I}_n & \boldsymbol{AB} - \boldsymbol{I}_n 
\end{pmatrix} \geqslant \mathrm{rank}(\boldsymbol{AB} - \boldsymbol{I}_n).
\]

{\color{blue}证法二:}直接利用关于矩阵秩的不等式, 得
\begin{align*}
\mathrm{rank}(\boldsymbol{AB} - \boldsymbol{I}_n) &= \mathrm{rank}((\boldsymbol{A} - \boldsymbol{I}_n)\boldsymbol{B} + (\boldsymbol{B} - \boldsymbol{I}_n)) \\
&\leqslant \mathrm{rank}((\boldsymbol{A} - \boldsymbol{I}_n)\boldsymbol{B}) + \mathrm{rank}(\boldsymbol{B} - \boldsymbol{I}_n) \\
&\leqslant \mathrm{rank}(\boldsymbol{A} - \boldsymbol{I}_n) + \mathrm{rank}(\boldsymbol{B} - \boldsymbol{I}_n).
\end{align*}

\end{proof}


\subsection{利用线性方程组的求解理论讨论矩阵的秩}

\begin{theorem}\label{theorem:系数矩阵的秩与解空间的维数}
设\(\boldsymbol{A}\)是数域\(\mathbb{K}\)上的\(m\times n\)矩阵,则齐次线性方程组\(\boldsymbol{A}\boldsymbol{x}=\boldsymbol{0}\)的解集\(V_{\boldsymbol{A}}\)是\(n\)维列向量空间\(\mathbb{K}^n\)的子空间. 根据线性方程组的求解理论,我们有
\[
\dim V_{\boldsymbol{A}}+\mathrm{r}(\boldsymbol{A}) = n,
\]
\end{theorem}
\begin{note}
即齐次线性方程组解空间的维数与系数矩阵的秩之和等于未知数的个数. 根据上述公式,由矩阵的秩可以讨论线性方程组解的性质;反过来,也可以由线性方程组解的性质讨论矩阵的秩. 
\end{note}

\begin{corollary}\label{corollary:线性方程组只有零解的充要条件}
线性方程组\(\boldsymbol{A}\boldsymbol{x}=\boldsymbol{0}\)只有零解的充要条件是\(\boldsymbol{A}\)为列满秩阵. 特别地,若\(\boldsymbol{A}\)是方阵,则线性方程组\(\boldsymbol{A}\boldsymbol{x}=\boldsymbol{0}\)只有零解的充要条件是\(\boldsymbol{A}\)为非异阵. 
\end{corollary}
\begin{proof}
由\hyperref[theorem:系数矩阵的秩与解空间的维数]{定理\ref{theorem:系数矩阵的秩与解空间的维数}中的公式}\(\dim V_{\boldsymbol{A}}+\mathrm{r}(\boldsymbol{A}) = n\)即可得到证明.

\end{proof}

\begin{proposition}\label{proposition:r(AA')=r(A)}
\begin{enumerate}[(1)]
\item 设\(\boldsymbol{A}\)是\(m\times n\)实矩阵,求证:\(\mathrm{r}(\boldsymbol{A}'\boldsymbol{A})=\mathrm{r}(\boldsymbol{A}\boldsymbol{A}')=\mathrm{r}(\boldsymbol{A})\).

\item 若\(\boldsymbol{A}\)是\(m\times n\)复矩阵,则\(\mathrm{r}\left(\overline{\boldsymbol{A}}'\boldsymbol{A}\right)=\mathrm{r}\left(\boldsymbol{A}\overline{\boldsymbol{A}}'\right)=\mathrm{r}(\boldsymbol{A})\).
\end{enumerate}
\end{proposition}
\begin{proof}
\begin{enumerate}[(1)]
\item 首先证明\(\mathrm{r}(\boldsymbol{A}'\boldsymbol{A})=\mathrm{r}(\boldsymbol{A})\),为此我们将证明齐次线性方程组\(\boldsymbol{A}\boldsymbol{x}=\boldsymbol{0}\)和\(\boldsymbol{A}'\boldsymbol{A}\boldsymbol{x}=\boldsymbol{0}\)同解. 显然\(\boldsymbol{A}\boldsymbol{x}=\boldsymbol{0}\)的解都是\(\boldsymbol{A}'\boldsymbol{A}\boldsymbol{x}=\boldsymbol{0}\)的解. 反之,任取方程组\(\boldsymbol{A}'\boldsymbol{A}\boldsymbol{x}=\boldsymbol{0}\)的解\(\boldsymbol{\alpha}\in\mathbb{R}^n\),则\(\boldsymbol{\alpha}'\boldsymbol{A}'\boldsymbol{A}\boldsymbol{\alpha}=0\),即\((\boldsymbol{A}\boldsymbol{\alpha})'(\boldsymbol{A}\boldsymbol{\alpha}) = 0\). 记\(\boldsymbol{A}\boldsymbol{\alpha}=(b_1,b_2,\cdots,b_m)'\in\mathbb{R}^m\),则
\[
b_1^2 + b_2^2+\cdots + b_m^2 = 0.
\]
因为\(b_i\)是实数,故每个\(b_i = 0\),即\(\boldsymbol{A}\boldsymbol{\alpha}=\boldsymbol{0}\),也即\(\boldsymbol{\alpha}\)是\(\boldsymbol{A}\boldsymbol{x}=0\)的解. 这就证明了方程组\(\boldsymbol{A}\boldsymbol{x}=0\)和\(\boldsymbol{A}'\boldsymbol{A}\boldsymbol{x}=0\)同解,即\(V_{\boldsymbol{A}}=V_{\boldsymbol{A}'\boldsymbol{A}}\),于是由\hyperref[theorem:系数矩阵的秩与解空间的维数]{定理\ref{theorem:系数矩阵的秩与解空间的维数}}可得\(\mathrm{r}(\boldsymbol{A}'\boldsymbol{A})=\mathrm{r}(\boldsymbol{A})\). 在上述等式中用\(\boldsymbol{A}'\)替代\(\boldsymbol{A}\)可得\(\mathrm{r}(\boldsymbol{A}\boldsymbol{A}')=\mathrm{r}(\boldsymbol{A}')\),又因为\(\mathrm{r}(\boldsymbol{A})=\mathrm{r}(\boldsymbol{A}')\),故结论得证.

\item 由(1)类似的方法可以证明.
\end{enumerate}

\end{proof}

\begin{example}
设$n$阶实方阵$A$满足$AA'=a^2I_n$,其中$a$为实数,证明:
\begin{align*}
r(aI_n - A)=r((aI_n - A)^2).
\end{align*}
\end{example}
\begin{proof}
$\left( \mathrm{i} \right) $当$a=0$时,有$AA'=O$,从而由\refpro{proposition:零矩阵的充要条件}知$A=O$,结论显然成立.

$\left( \mathrm{ii} \right)$ 当$a\ne 0$时,由条件知$A,A'$都可逆.于是
\begin{align*}
r\left( \left( aI_n - A \right)^2 \right) &= r\left( \left( aI_n - A \right)^2 \frac{1}{a} \left( -A' \right) \right) = r\left( \left( aI_n - A \right) \left( \frac{1}{a}AA' - A' \right) \right) \\
&= r\left( \left( aI_n - A \right) \left( aI_n - A' \right) \right) \xlongequal{\text{\refpro{proposition:r(AA')=r(A)}}} r\left( aI_n - A \right).
\end{align*}

\end{proof}

\begin{example}
设\(\boldsymbol{A}\)和\(\boldsymbol{B}\)是数域\(\mathbb{K}\)上的\(n\)阶矩阵,若线性方程组\(\boldsymbol{A}\boldsymbol{x}=\boldsymbol{0}\)和\(\boldsymbol{B}\boldsymbol{x}=\boldsymbol{0}\)同解,且每个方程组的基础解系含\(m\)个线性无关的向量,求证:\(\mathrm{r}(\boldsymbol{A}-\boldsymbol{B})\leqslant  n - m\).
\end{example}
\begin{proof}
由方程组\(\boldsymbol{A}\boldsymbol{x}=\boldsymbol{0}\)和\(\boldsymbol{B}\boldsymbol{x}=\boldsymbol{0}\)同解可知,\(\boldsymbol{A}\boldsymbol{x}=\boldsymbol{0}\)的解都是\((\boldsymbol{A}-\boldsymbol{B})\boldsymbol{x}=\boldsymbol{0}\)的解,即\(V_{\boldsymbol{A}}\subseteq V_{\boldsymbol{A}-\boldsymbol{B}}\),从而\(\dim V_{\boldsymbol{A}-\boldsymbol{B}}\geqslant \dim V_{\boldsymbol{A}} = m\),于是由\hyperref[theorem:系数矩阵的秩与解空间的维数]{定理\ref{theorem:系数矩阵的秩与解空间的维数}}可得$\mathrm{r(}\boldsymbol{A}-\boldsymbol{B})=n-\mathrm{dim}V_{\boldsymbol{A}-\boldsymbol{B}}\le n-m$.

\end{proof}

\begin{proposition}\label{proposition:ABx=O与Bx=O同解的充要条件}
设\(\boldsymbol{A}\)是\(m\times n\)矩阵,\(\boldsymbol{B}\)是\(n\times k\)矩阵,证明:方程组\(\boldsymbol{A}\boldsymbol{B}\boldsymbol{x}=\boldsymbol{0}\)和方程组\(\boldsymbol{B}\boldsymbol{x}=\boldsymbol{0}\)同解的充要条件是\(\mathrm{r}(\boldsymbol{A}\boldsymbol{B})=\mathrm{r}(\boldsymbol{B})\).
\end{proposition}
\begin{proof}
显然方程组\(\boldsymbol{B}\boldsymbol{x}=\boldsymbol{0}\)的解都是方程组\(\boldsymbol{A}\boldsymbol{B}\boldsymbol{x}=\boldsymbol{0}\)的解,即\(V_{\boldsymbol{B}}\subseteq V_{\boldsymbol{A}\boldsymbol{B}}\),于是两个线性方程组同解,即\(V_{\boldsymbol{B}} = V_{\boldsymbol{A}\boldsymbol{B}}\)的充要条件是\(\dim V_{\boldsymbol{B}}=\dim V_{\boldsymbol{A}\boldsymbol{B}}\). 又由\hyperref[theorem:系数矩阵的秩与解空间的维数]{定理\ref{theorem:系数矩阵的秩与解空间的维数}}可知\(\dim V_{\boldsymbol{B}}=k - \mathrm{r}(\boldsymbol{B})\),\(\dim V_{\boldsymbol{A}\boldsymbol{B}}=k - \mathrm{r}(\boldsymbol{A}\boldsymbol{B})\),因此上述两个方程组同解的充要条件是\(\mathrm{r}(\boldsymbol{A}\boldsymbol{B})=\mathrm{r}(\boldsymbol{B})\). 

\end{proof}

\begin{proposition}\label{proposition:r(AB)=r(B)可推出r(ABC)=r(BC)}
设\(\boldsymbol{A}\)是\(m\times n\)矩阵,\(\boldsymbol{B}\)是\(n\times k\)矩阵. 若\(\boldsymbol{A}\boldsymbol{B}\)和\(\boldsymbol{B}\)有相同的秩,求证:对任意的\(k\times l\)矩阵\(\boldsymbol{C}\),矩阵\(\boldsymbol{A}\boldsymbol{B}\boldsymbol{C}\)和矩阵\(\boldsymbol{B}\boldsymbol{C}\)也有相同的秩.
\end{proposition}
\begin{proof}
{\color{blue}证法一:}
由假设和\hyperref[proposition:ABx=O与Bx=O同解的充要条件]{命题\ref{proposition:ABx=O与Bx=O同解的充要条件}}可知,方程组\(\boldsymbol{A}\boldsymbol{B}\boldsymbol{x}=\boldsymbol{0}\)和方程组\(\boldsymbol{B}\boldsymbol{x}=\boldsymbol{0}\)同解. 要证明\(\mathrm{r}(\boldsymbol{A}\boldsymbol{B}\boldsymbol{C})=\mathrm{r}(\boldsymbol{B}\boldsymbol{C})\),我们只要证明方程组\(\boldsymbol{A}\boldsymbol{B}\boldsymbol{C}\boldsymbol{x}=\boldsymbol{0}\)和方程组\(\boldsymbol{B}\boldsymbol{C}\boldsymbol{x}=\boldsymbol{0}\)同解即可. 显然方程组\(\boldsymbol{B}\boldsymbol{C}\boldsymbol{x}=\boldsymbol{0}\)的解都是方程组\(\boldsymbol{A}\boldsymbol{B}\boldsymbol{C}\boldsymbol{x}=\boldsymbol{0}\)的解. 反之,若列向量\(\boldsymbol{\alpha}\)是方程组\(\boldsymbol{A}\boldsymbol{B}\boldsymbol{C}\boldsymbol{x}=\boldsymbol{0}\)的解,则\(\boldsymbol{C}\boldsymbol{\alpha}\)是方程组\(\boldsymbol{A}\boldsymbol{B}\boldsymbol{x}=\boldsymbol{0}\)的解,因此\(\boldsymbol{C}\boldsymbol{\alpha}\)也是方程组\(\boldsymbol{B}\boldsymbol{x}=\boldsymbol{0}\)的解,即\(\boldsymbol{B}\boldsymbol{C}\boldsymbol{\alpha}=\boldsymbol{0}\),于是\(\boldsymbol{\alpha}\)也是方程组\(\boldsymbol{B}\boldsymbol{C}\boldsymbol{x}=\boldsymbol{0}\)的解. 这就证明了方程组\(\boldsymbol{A}\boldsymbol{B}\boldsymbol{C}\boldsymbol{x}=\boldsymbol{0}\)和方程组\(\boldsymbol{B}\boldsymbol{C}\boldsymbol{x}=\boldsymbol{0}\)同解,从而结论得证.

{\color{blue}证法二:}
由\hyperref[proposition:Frobenius(弗罗贝尼乌斯)不等式]{Frobenius不等式}可得
\[
\mathrm{r}(\boldsymbol{A}\boldsymbol{B}\boldsymbol{C})\geqslant \mathrm{r}(\boldsymbol{A}\boldsymbol{B})+\mathrm{r}(\boldsymbol{B}\boldsymbol{C})-\mathrm{r}(\boldsymbol{B})=\mathrm{r}(\boldsymbol{B}\boldsymbol{C}),
\]
又由\hyperref[矩阵秩的基本公式2]{矩阵秩的基本公式(2)}可知\(\mathrm{r}(\boldsymbol{A}\boldsymbol{B}\boldsymbol{C})\leqslant \mathrm{r}(\boldsymbol{B}\boldsymbol{C})\),故结论得证.

\end{proof}

\begin{proposition}\label{proposition:奇异系数矩阵Ax=0的解空间}
设数域\(\mathbb{K}\)上的\(n\)阶矩阵\(\boldsymbol{A}=(a_{ij})\)满足:\(|\boldsymbol{A}| = 0\)且某个元素\(a_{ij}\)的代数余子式\(A_{ij}\neq0\). 求证:齐次线性方程组\(\boldsymbol{A}\boldsymbol{x}=\boldsymbol{0}\)的所有解都可写为下列形式:
\[
k\begin{pmatrix}
A_{i1}\\
A_{i2}\\
\vdots\\
A_{in}
\end{pmatrix},k\in\mathbb{K}.
\]
\end{proposition}
\begin{proof}
由条件和\hyperref[theorem:矩阵的秩与子式]{定理\ref{theorem:矩阵的秩与子式}}可知\(\boldsymbol{A}\)的秩等于\(n - 1\),因此线性方程组\(\boldsymbol{A}\boldsymbol{x}=\boldsymbol{0}\)的基础解系只含一个向量. 注意到\(|\boldsymbol{A}| = 0\),故\(\boldsymbol{A}\boldsymbol{A}^*=|\boldsymbol{A}|\boldsymbol{I}_n=\boldsymbol{O}\),于是伴随矩阵\(\boldsymbol{A}^*\)的任一列向量都是\(\boldsymbol{A}\boldsymbol{x}=\boldsymbol{0}\)的解. 又已知\(A_{ij}\neq0\),因此\(\boldsymbol{A}^*\)的第\(i\)个列向量\((A_{i1},A_{i2},\cdots,A_{in})'\)(不是零向量)是\(\boldsymbol{A}\boldsymbol{x}=\boldsymbol{0}\)的基础解系.   

\end{proof}

\begin{definition}[对角占优阵]\label{definition:对角占优阵}
如果\(n\)阶实方阵\(\boldsymbol{A}=(a_{ij})\)适合条件:
\[
|a_{ii}|\geqslant \sum_{j = 1,j\neq i}^{n}|a_{ij}|,1\leqslant  i\leqslant  n,
\]
则称\(\boldsymbol{A}\)是\textbf{(弱)对角占优阵}.

如果\(n\)阶实方阵\(\boldsymbol{A}=(a_{ij})\)适合条件:
\[
|a_{ii}|>\sum_{j = 1,j\neq i}^{n}|a_{ij}|,1\leqslant  i\leqslant  n,
\]
则称\(\boldsymbol{A}\)是\textbf{严格对角占优阵}.
\end{definition}

\begin{proposition}[严格对角占优阵必是非异阵]\label{proposition:严格对角占优阵必是非异阵}
如果\(n\)阶实方阵\(\boldsymbol{A}=(a_{ij})\)是严格对角占优阵,则\(\boldsymbol{A}\)必是非异阵.
\end{proposition}
\begin{proof}
{\color{blue}证法一:}
只需证明线性方程组\(\boldsymbol{A}\boldsymbol{x}=\boldsymbol{0}\)只有零解. 若有非零解,设为\((c_1,c_2,\cdots,c_n)\),假设\(c_k\)是其中绝对值最大者. 将解代入该方程组的第\(k\)个方程式,得
\[
a_{k1}c_1+\cdots + a_{kk}c_k+\cdots + a_{kn}c_n = 0,
\]
即有
\[
-a_{kk}c_k=a_{k1}c_1+\cdots + a_{k,k - 1}c_{k - 1}+a_{k,k + 1}c_{k + 1}+\cdots + a_{kn}c_n.
\]
上式两边取绝对值,由三角不等式以及\(c_k\)是绝对值最大的假设可得
\begin{align*}
|a_{kk}||c_k|&\leqslant |a_{k1}||c_1|+\cdots + |a_{k,k - 1}||c_{k - 1}|+|a_{k,k + 1}||c_{k + 1}|+\cdots + |a_{kn}||c_n|\leqslant \left(\sum_{j = 1,j\neq k}^{n}|a_{kj}|\right)|c_k|,
\end{align*}
从而有
\[
|a_{kk}|\leqslant \sum_{j = 1,j\neq k}^{n}|a_{kj}|,
\]
得到矛盾. 因此,方程组\(\boldsymbol{A}\boldsymbol{x}=\boldsymbol{0}\)只有零解.

{\color{blue}证法二:}由\hyperref[theorem:第一圆盘定理]{第一圆盘定理}, \(A\)的特征值落在下列戈氏圆盘中:
\begin{align*}
|z - a_{ii}| \leqslant  R_{i} = \sum_{j = 1,j\neq i}^{n}|a_{ij}|,\ 1 \leqslant  i \leqslant  n.
\end{align*}
\(A\)的严格对角占优条件保证了复平面的原点不落在这些戈氏圆盘中, 因此\(A\)的特征值全不为零, 从而\(A\)是非异阵.


\end{proof}

\begin{proposition}\label{proposition:更严格对角占优阵行列式必大于零}
若\(n\)阶实方阵\(\boldsymbol{A}=(a_{ij})\)是严格对角占优阵,即满足
\[
a_{ii}>\sum_{j = 1,j\neq i}^{n}|a_{ij}|,1\leqslant  i\leqslant  n,
\]
求证: \(|\boldsymbol{A}|>0\).
\end{proposition}
\begin{proof}
{\color{blue}证法一:}
考虑矩阵\(t\boldsymbol{I}_n+\boldsymbol{A}\),当\(t\geqslant 0\)时,这是一个严格对角占优阵,因此由\hyperref[proposition:严格对角占优阵必是非异阵]{上一个命题}可知其行列式\(f(t)=|t\boldsymbol{I}_n+\boldsymbol{A}|\)不为零. 又\(f(t)\)是关于\(t\)的多项式且首项系数为\(1\),所以当\(t\)充分大时,\(f(t)>0\). 注意到\(f(t)\)是\([0,+\infty)\)上处处不为零的连续函数,并且当\(t\)充分大时取值为正,因此\(f(t)\)在\([0,+\infty)\)上取值恒为正(原因见:\hyperref[Basis of Analytics-proposition:连续函数无零点则一定恒大于零或恒小于零]{命题\ref{Basis of Analytics-proposition:连续函数无零点则一定恒大于零或恒小于零}}). 特别地,\(f(0)=|\boldsymbol{A}|>0\).

{\color{blue}证法二:}由\hyperref[theorem:第一圆盘定理]{第一圆盘定理}, \(A\)的特征值落在下列戈氏圆盘中:
\begin{align*}
|z - a_{ii}| \leqslant  R_{i} = \sum_{j = 1,j\neq i}^{n}|a_{ij}|,\ 1 \leqslant  i \leqslant  n.
\end{align*}
由条件可知\(a_{ii} > R_{i}(1 \leqslant  i \leqslant  n)\),从而这些戈氏圆盘全部位于虚轴的右侧, 因此\(A\)的特征值\(\lambda_{i}\)或者是正实数, 或者是实部为正的共轭虚数, 从而\(|A|=\lambda_{1}\lambda_{2}\cdots\lambda_{n}>0\). 

{\color{blue}证法三:}设$\lambda$为$A$的特征值,$\alpha=(x_1,x_2,\cdots,x_n)^T$为与之对应的特征向量,则
\begin{align*}
A\alpha = \lambda\alpha \Longleftrightarrow \begin{pmatrix}
\sum_{j=1}^n a_{1j}x_j \\
\sum_{j=1}^n a_{2j}x_j \\
\vdots \\
\sum_{j=1}^n a_{nj}x_j
\end{pmatrix} = \begin{pmatrix}
\lambda x_1 \\
\lambda x_2 \\
\vdots \\
\lambda x_n
\end{pmatrix} \Longleftrightarrow \sum_{j=1}^n a_{ij}x_j = \lambda x_i,\quad i=1,2,\cdots,n.
\end{align*}
记$|x_k| = \max\limits_{1\leqslant i\leqslant n}\{|x_i|\}$,其中$k\in[1,n]\cap\mathbb{N}$。由$\alpha$为非零向量知$|x_k|>0$,则
\begin{align*}
\lambda = \frac{\sum\limits_{j=1}^n a_{kj}x_j}{x_k} = a_{kk} + \frac{\sum\limits_{j=1,j\ne k}^n a_{kj}x_j}{x_k} \geqslant |a_{kk}| - \frac{\left|\sum\limits_{j=1,j\ne k}^n a_{kj}x_j\right|}{|x_k|} \\
\geqslant a_{kk} - \frac{\sum\limits_{j=1,j\ne k}^n |a_{kj}||x_j|}{|x_k|} \geqslant a_{kk} - \sum\limits_{j=1,j\ne k}^n |a_{kj}| > 0.
\end{align*}
故$A$的特征值全都大于$0$,因此$|A|>0$.

\end{proof}

\begin{proposition}\label{proposition:实对称与实反称矩阵性质}
\begin{enumerate}[(1)]
\item 设\(\boldsymbol{A}\)是\(n\)阶实对称阵,则:\(\boldsymbol{I}_n + \mathrm{i}\boldsymbol{A}\)和\(\boldsymbol{I}_n - \mathrm{i}\boldsymbol{A}\)都是非异阵.

\item 设\(\boldsymbol{A}\)是\(n\)阶实反对称阵,\(\,\boldsymbol{D}=\mathrm{diag}\{d_1,d_2,\cdots,d_n\}\)是同阶对角阵且主对角元素全大于零,求证:\(|\boldsymbol{A}+\boldsymbol{D}|>0\). 特别地,\(\,|\boldsymbol{I}_n\pm\boldsymbol{A}|>0\),从而\(\boldsymbol{I}_n\pm\boldsymbol{A}\)都是非异阵.
\end{enumerate}
\end{proposition}
\begin{note}
(2)的证明思路:利用行列式构造连续的多项式函数,再利用函数连续的性质证明.
\end{note}
\begin{proof}
\begin{enumerate}[(1)]
\item {\color{blue}证法一:}只需证明\((\boldsymbol{I}_n + \mathrm{i}\boldsymbol{A})\boldsymbol{x}=\boldsymbol{0}\)只有零解. 由\(\overline{\boldsymbol{x}}'(\boldsymbol{I}_n + \mathrm{i}\boldsymbol{A})\boldsymbol{x}=0\)共轭转置可得\(\overline{\boldsymbol{x}}'(\boldsymbol{I}_n - \mathrm{i}\boldsymbol{A})\boldsymbol{x}=0\). 上述两式相加,可得\(\overline{\boldsymbol{x}}'\boldsymbol{I}_n\boldsymbol{x}=0\),因此\(\boldsymbol{x}=\boldsymbol{0}\).

{\color{blue}证法二:}设 $\lambda_0 \in \mathbb{C}$ 是 $A$ 的任一特征值, $\alpha = (a_1, a_2, \dots, a_n)' \in \mathbb{C}^n$ 是对应的特征向量, 即 $A\alpha = \lambda_0 \alpha$. 此式两边同时左乘 $\overline{\alpha }' $, 则有
\begin{align*}
\overline{\alpha }'  A \alpha = \lambda_0 \overline{\alpha }'  \alpha.
\end{align*}
注意到 $\alpha$ 是非零向量, 故 $\overline{\alpha }' \alpha = \sum_{i=1}^n |a_i|^2 > 0$. 注意到 $A$ 为实对称矩阵, 故
\begin{align*}
\overline{(\overline{\alpha }'  A\alpha )}' = \overline{\alpha }' A \alpha,
\end{align*}
即 $\overline{\alpha }'  A \alpha$ 是一个实数, 从而 $\lambda_0 = \overline{\alpha }'  A \alpha / \overline{\alpha }'  \alpha$ 也是实数. 于是$I_n\pm \mathrm{i}A$的任一特征值为$1\pm \mathrm{i}\lambda_0\ne 0$.因此由\hyperref[proposition:特征值与特征多项式系数的关系]{特征值与特征多项式系数的关系}可知$I_n\pm \mathrm{i}A$是非异阵.

\item {\color{blue}证法一:}先证明\(|\boldsymbol{A}+\boldsymbol{D}|\neq0\),只需证明\((\boldsymbol{A}+\boldsymbol{D})\boldsymbol{x}=\boldsymbol{0}\)只有零解. 因为\(\boldsymbol{x}'(\boldsymbol{A}+\boldsymbol{D})\boldsymbol{x}=0\),转置可得\(\boldsymbol{x}'(-\boldsymbol{A}+\boldsymbol{D})\boldsymbol{x}=0\),上述两式相加即得\(\boldsymbol{x}'\boldsymbol{D}\boldsymbol{x}=0\). 若设\(\boldsymbol{x}=(x_1,x_2,\cdots,x_n)'\),则有\(d_1x_1^2 + d_2x_2^2+\cdots + d_nx_n^2 = 0\). 由于\(d_i\)都大于零并且\(x_i\)都是实数,故只能是\(x_1 = x_2=\cdots = x_n = 0\),即有\(\boldsymbol{x}=\boldsymbol{0}\).

再证明(2)的结论. 设\(f(t)=|t\boldsymbol{A}+\boldsymbol{D}|\),则\(f(t)\)是关于\(t\)的多项式,从而是关于\(t\)的连续函数. 注意到对任意的实数\(t\),\(t\boldsymbol{A}\)仍是实反对称阵,故由上面的讨论可得\(f(t)=|t\boldsymbol{A}+\boldsymbol{D}|\neq0\),即\(f(t)\)是\(\mathbb{R}\)上处处不为零的连续函数. 注意到当\(t = 0\)时,\(f(0)=|\boldsymbol{D}|>0\),因此\(f(t)\)只能是\(\mathbb{R}\)上取值恒为正数的连续函数(原因见:\hyperref[Basis of Analytics-proposition:连续函数无零点则一定恒大于零或恒小于零]{命题\ref{Basis of Analytics-proposition:连续函数无零点则一定恒大于零或恒小于零}}).特别地,\(f(1)=|\boldsymbol{A}+\boldsymbol{D}|>0\). 

{\color{blue}证法二:}设 $\lambda_0 \in \mathbb{C}$ 是 $A$ 的任一特征值, $\alpha = (a_1, a_2, \dots, a_n)' \in \mathbb{C}^n$ 是对应的特征向量, 即 $A\alpha = \lambda_0 \alpha$. 此式两边同时左乘 $\overline{\alpha }' $, 则有
\begin{align*}
\overline{\alpha }'  A \alpha = \lambda_0 \overline{\alpha }'  \alpha.
\end{align*}
注意到 $\alpha$ 是非零向量, 故 $\overline{\alpha }' \alpha = \sum_{i=1}^n |a_i|^2 > 0$. 注意到 $A$ 为实反称矩阵, 故
\begin{align*}
\overline{(\overline{\alpha }'  A\alpha )}' = -\overline{\alpha }' A \alpha,
\end{align*}
即 $\overline{\alpha }'  A \alpha$ 是零或纯虚数, 从而 $\lambda_0 = \overline{\alpha }'  A \alpha / \overline{\alpha }'  \alpha$ 也是零或纯虚数.设$\lambda=c\mathrm{i}$,其中$c\in \mathbb{C}$.
于是 $I_n \pm A$ 的任一特征值为 $1 \pm \lambda_0 = 1 \pm c\mathrm{i} \neq 0\ne 0$,由\hyperref[proposition:特征值与特征多项式系数的关系]{特征值与特征多项式系数的关系}可知$I_n\pm A$是非异阵.
\end{enumerate}

\end{proof}



\subsection{利用线性空间理论讨论矩阵的秩}
按照最初的定义,矩阵的秩就是矩阵的行(列)向量组的秩,因此通过线性空间理论去讨论矩阵的秩是十分自然的事情.

\begin{proposition}\label{proposition:矩阵的秩与子式及其加边子式的关系}
求证: 矩阵\(\boldsymbol{A}\)的秩等于\(r\)的充要条件是\(\boldsymbol{A}\)存在一个\(r\)阶子式\(|\boldsymbol{D}|\)不等于零,而\(|\boldsymbol{D}|\)的所有\(r + 1\)阶加边子式全等于零.
\end{proposition}
\begin{proof}
必要性由\hyperref[theorem:矩阵的秩与子式]{定理\ref{theorem:矩阵的秩与子式}}可直接得到,只需证明充分性. 不失一般性,我们可设\(|\boldsymbol{D}|\)是由\(\boldsymbol{A}\)的前\(r\)行和前\(r\)列构成的\(r\)阶子式. 设
\[
\boldsymbol{A}=\begin{pmatrix}
\boldsymbol{\alpha}_1\\
\boldsymbol{\alpha}_2\\
\vdots\\
\boldsymbol{\alpha}_m
\end{pmatrix}=(\boldsymbol{\beta}_1,\boldsymbol{\beta}_2,\cdots,\boldsymbol{\beta}_n)
\]
为矩阵\(\boldsymbol{A}\)的行分块和列分块,记\(\tau_{\leqslant  r}\boldsymbol{\alpha}_i\)为行向量\(\boldsymbol{\alpha}_i\)关于前\(r\)列的缩短向量,\(\tau_{\leqslant  r}\boldsymbol{\beta}_j\)为列向量\(\boldsymbol{\beta}_j\)关于前\(r\)行的缩短向量. 由\(|\boldsymbol{D}|\neq0\)可得\(\tau_{\leqslant  r}\boldsymbol{\alpha}_1,\cdots,\tau_{\leqslant  r}\boldsymbol{\alpha}_r\)线性无关,由\hyperref[proposition:线性相关向量组的缩短组也线性相关]{命题\ref{proposition:线性相关向量组的缩短组也线性相关}}可知\(\boldsymbol{\alpha}_1,\cdots,\boldsymbol{\alpha}_r\)线性无关. 

我们只要证明\(\boldsymbol{\alpha}_1,\cdots,\boldsymbol{\alpha}_r\)是\(\boldsymbol{A}\)的行向量的极大无关组即可得到\(\mathrm{r}(\boldsymbol{A}) = r\). 用反证法证明,若它们不是极大无关组,则可以添加一个行向量,不妨设为\(\boldsymbol{\alpha}_{r + 1}\),使得\(\boldsymbol{\alpha}_1,\cdots,\boldsymbol{\alpha}_r,\boldsymbol{\alpha}_{r + 1}\)线性无关. 设\(\boldsymbol{A}_1\)是\(\boldsymbol{A}\)的前\(r + 1\)行构成的矩阵,则\(\boldsymbol{A}_1 = (\tau_{\leqslant  r + 1}\boldsymbol{\beta}_1,\tau_{\leqslant  r + 1}\boldsymbol{\beta}_2,\cdots,\tau_{\leqslant  r + 1}\boldsymbol{\beta}_n)\)且\(\mathrm{r}(\boldsymbol{A}_1)=r + 1\). 由\(|\boldsymbol{D}|\neq0\)可得\(\tau_{\leqslant  r}\boldsymbol{\beta}_1,\cdots,\tau_{\leqslant  r}\boldsymbol{\beta}_r\)线性无关,由\hyperref[proposition:线性相关向量组的缩短组也线性相关]{命题\ref{proposition:线性相关向量组的缩短组也线性相关}}可知\(\tau_{\leqslant  r + 1}\boldsymbol{\beta}_1,\cdots,\tau_{\leqslant  r + 1}\boldsymbol{\beta}_r\)线性无关. 因为\(\mathrm{r}(\boldsymbol{A}_1)=r + 1\),故存在\(\boldsymbol{A}_1\)的一个列向量,不妨设为\(\tau_{\leqslant  r + 1}\boldsymbol{\beta}_{r + 1}\),使得\(\tau_{\leqslant  r + 1}\boldsymbol{\beta}_1,\cdots,\tau_{\leqslant  r + 1}\boldsymbol{\beta}_r,\tau_{\leqslant  r + 1}\boldsymbol{\beta}_{r + 1}\)
线性无关. 设\(\boldsymbol{A}_2 = (\tau_{\leqslant  r + 1}\boldsymbol{\beta}_1,\cdots,\tau_{\leqslant  r + 1}\boldsymbol{\beta}_r,\tau_{\leqslant  r + 1}\boldsymbol{\beta}_{r + 1})\),即\(\boldsymbol{A}_2\)是\(\boldsymbol{A}\)的前\(r + 1\)行和前\(r + 1\)列构成的方阵,则\(\mathrm{r}(\boldsymbol{A}_2)=r + 1\). 因此,\(|\boldsymbol{A}_2|\neq0\)是包含\(|\boldsymbol{D}|\)的\(r + 1\)阶加边子式,这与假设矛盾.

\end{proof}

\begin{proposition}\label{proposition:极大无关组对应的子式不为零}
设\(m\times n\)矩阵\(\boldsymbol{A}\)的\(m\)个行向量为\(\boldsymbol{\alpha}_1,\boldsymbol{\alpha}_2,\cdots,\boldsymbol{\alpha}_m\),且\(\boldsymbol{\alpha}_{i_1},\boldsymbol{\alpha}_{i_2},\cdots,\boldsymbol{\alpha}_{i_r}\)是其极大无关组,又设\(\boldsymbol{A}\)的\(n\)个列向量为\(\boldsymbol{\beta}_1,\boldsymbol{\beta}_2,\cdots,\boldsymbol{\beta}_n\),且\(\boldsymbol{\beta}_{j_1},\boldsymbol{\beta}_{j_2},\cdots,\boldsymbol{\beta}_{j_r}\)是其极大无关组. 证明:\(\boldsymbol{\alpha}_{i_1},\boldsymbol{\alpha}_{i_2},\cdots,\boldsymbol{\alpha}_{i_r}\)和\(\boldsymbol{\beta}_{j_1},\boldsymbol{\beta}_{j_2},\cdots,\boldsymbol{\beta}_{j_r}\)交叉点上的元素组成的子矩阵\(\boldsymbol{D}\)的行列式\(|\boldsymbol{D}|\neq0\).
\end{proposition}
\begin{proof}
因为\(\boldsymbol{\alpha}_{i_1},\boldsymbol{\alpha}_{i_2},\cdots,\boldsymbol{\alpha}_{i_r}\)是极大无关组,故\(\boldsymbol{A}\)的任一行向量\(\boldsymbol{\alpha}_s\)均可表示为\(\boldsymbol{\alpha}_{i_1},\boldsymbol{\alpha}_{i_2},\cdots,\boldsymbol{\alpha}_{i_r}\)的线性组合. 记\(\widetilde{\boldsymbol{\alpha}}_{i_1},\widetilde{\boldsymbol{\alpha}}_{i_2},\cdots,\widetilde{\boldsymbol{\alpha}}_{i_r},\widetilde{\boldsymbol{\alpha}}_s\)分别是\(\boldsymbol{\alpha}_{i_1},\boldsymbol{\alpha}_{i_2},\cdots,\boldsymbol{\alpha}_{i_r},\boldsymbol{\alpha}_s\)在\(j_1,j_2,\cdots,j_r\)列处的缩短向量,由\hyperref[proposition:线性相关向量组的缩短组也线性相关]{命题\ref{proposition:线性相关向量组的缩短组也线性相关}}可知,\(\widetilde{\boldsymbol{\alpha}}_s\)均可表示为\(\widetilde{\boldsymbol{\alpha}}_{i_1},\widetilde{\boldsymbol{\alpha}}_{i_2},\cdots,\widetilde{\boldsymbol{\alpha}}_{i_r}\)的线性组合.

考虑由列向量\(\boldsymbol{\beta}_{j_1},\boldsymbol{\beta}_{j_2},\cdots,\boldsymbol{\beta}_{j_r}\)组成的矩阵\(\boldsymbol{B}=(\boldsymbol{\beta}_{j_1},\boldsymbol{\beta}_{j_2},\cdots,\boldsymbol{\beta}_{j_r})\),这是一个\(m\times r\)矩阵且秩等于\(r\). 由于矩阵\(\boldsymbol{B}\)的任一行向量\(\widetilde{\boldsymbol{\alpha}}_s\)均可用\(\widetilde{\boldsymbol{\alpha}}_{i_1},\widetilde{\boldsymbol{\alpha}}_{i_2},\cdots,\widetilde{\boldsymbol{\alpha}}_{i_r}\)线性表示,并且\(\boldsymbol{B}\)的行秩等于\(r\),故由\hyperref[proposition:极大无关组的判定条件]{命题\ref{proposition:极大无关组的判定条件}}可知,\(\widetilde{\boldsymbol{\alpha}}_{i_1},\widetilde{\boldsymbol{\alpha}}_{i_2},\cdots,\widetilde{\boldsymbol{\alpha}}_{i_r}\)是\(\boldsymbol{B}\)的行向量的极大无关组,从而它们线性无关. 注意到\(r\)阶方阵\(\boldsymbol{D}\)的行向量恰好是\(\widetilde{\boldsymbol{\alpha}}_{i_1},\widetilde{\boldsymbol{\alpha}}_{i_2},\cdots,\widetilde{\boldsymbol{\alpha}}_{i_r}\),因此\(\boldsymbol{D}\)是满秩阵,从而\(|\boldsymbol{D}|\neq0\).

\end{proof}

\begin{definition}[主子式]
设\(\boldsymbol{A}\)是一个\(n\)阶方阵,\(\boldsymbol{A}\)的第\(i_1,\cdots,i_r\)行和第\(i_1,\cdots,i_r\)列交叉点上的元素组成的子式称为\(\boldsymbol{A}\)的\textbf{主子式}. 
\end{definition}

\begin{proposition}\label{proposition:对称阵或反对称阵必有非零主子式}
若\(\boldsymbol{A}\)是对称阵或反对称阵且秩等于\(r\),求证:\(\boldsymbol{A}\)必有一个\(r\)阶主子式不等于零.
\end{proposition}
\begin{proof}
{\color{blue}证法一:}
由对称性或反对称性,设\(\boldsymbol{A}\)的第\(i_1,\cdots,i_r\)行是\(\boldsymbol{A}\)的行向量的极大无关组,则由\hyperref[proposition:对称矩阵或反称矩阵的极大无关组]{命题\ref{proposition:对称矩阵或反称矩阵的极大无关组}}它的第\(i_1,\cdots,i_r\)列也是\(\boldsymbol{A}\)的列向量的极大无关组,因此由\hyperref[proposition:极大无关组对应的子式不为零]{命题\ref{proposition:极大无关组对应的子式不为零}}可知,它们交叉点上的元素组成的\(r\)阶主子式不等于零. 

{\color{blue}证法二:}设\(A\)的行向量分别为\(\boldsymbol{\alpha}_1,\boldsymbol{\alpha}_2,\cdots,\boldsymbol{\alpha}_n\),列向量分别为\(\boldsymbol{\beta}_1,\boldsymbol{\beta}_2,\cdots,\boldsymbol{\beta}_n\)。设\(\boldsymbol{\alpha}_{i_1},\boldsymbol{\alpha}_{i_2},\cdots,\boldsymbol{\alpha}_{i_r}\)是\(A\)行向量的极大无关组,用行对换可将这些行向量换到前\(r\)行,再用对称的列对换可将列向量\(\boldsymbol{\beta}_{i_1},\boldsymbol{\beta}_{i_2},\cdots,\boldsymbol{\beta}_{i_r}\)换到前\(r\)列,得到的矩阵记为\(B\),则\(B\)仍是对称矩阵(或反对称矩阵),且\(A\)的第\(i_1,i_2,\cdots,i_r\)行和列交点上的元素组成的主子式变成矩阵\(B\)的第\(r\)个顺序主子式\(\vert D\vert\)。只要证明\(\vert D\vert\neq 0\)即可。由于\(B\)的后\(n - r\)个行向量都是前\(r\)个行向量的线性组合,故可用第三类初等行变换将它们消去。接着进行对称的第三类初等列变换,得到的矩阵记为\(C\),则\(C\)仍是对称矩阵(或反对称矩阵)。由对称性(或反对称性)可知\(C\)具有下列形式:
\begin{align*}
C = \begin{pmatrix}
D & O \\
O & O
\end{pmatrix},
\end{align*}
因为\(C\)的秩等于\(A\)的秩,故\(D\)的秩等于\(r\),从而\(\vert D\vert\neq 0\)。 

\end{proof}

\begin{proposition}[反对称阵的秩必为偶数]\label{proposition:反对称阵的秩必为偶数}
证明:反对称阵的秩必为偶数.
\end{proposition}
\begin{proof}
用反证法,设反对称阵\(\boldsymbol{A}\)的秩等于\(2r + 1\),则由\hyperref[proposition:对称阵或反对称阵必有非零主子式]{命题\ref{proposition:对称阵或反对称阵必有非零主子式}}可知,\(\boldsymbol{A}\)有一个\(2r + 1\)阶主子式\(|\boldsymbol{D}|\)不等于零. 注意到反对称阵的主子式是反对称行列式,而由\hyperref[proposition:奇数阶反对称行列式的值等于零]{命题\ref{proposition:奇数阶反对称行列式的值等于零}}奇数阶反对称行列式的值等于零,从而\(|\boldsymbol{D}| = 0\),矛盾.

\end{proof}




\end{document}
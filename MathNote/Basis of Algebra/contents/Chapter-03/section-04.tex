% contents/chapter-03/section-04.tex 第三章第三节
\documentclass[../../main.tex]{subfiles}
\graphicspath{{\subfix{../../image/}}} % 指定图片目录,后续可以直接使用图片文件名。

% 例如:
% \begin{figure}[H]
% \centering
% \includegraphics[scale=0.4]{图.png}
% \caption{}
% \label{figure:图}
% \end{figure}
% 注意:上述\label{}一定要放在\caption{}之后,否则引用图片序号会只会显示??.

\begin{document}

\section{基变换与过渡矩阵}

\begin{definition}[过渡矩阵]\label{definition:过渡矩阵}
设\(\{\boldsymbol{e}_1,\boldsymbol{e}_2,\cdots,\boldsymbol{e}_n\}\)和\(\{\boldsymbol{f}_1,\boldsymbol{f}_2,\cdots,\boldsymbol{f}_n\}\)是\(n\)维线性空间\(V\)的两组基,若
\[
\begin{cases}
\boldsymbol{f}_1 = a_{11}\boldsymbol{e}_1 + a_{21}\boldsymbol{e}_2+\cdots + a_{n1}\boldsymbol{e}_n,\\
\boldsymbol{f}_2 = a_{12}\boldsymbol{e}_1 + a_{22}\boldsymbol{e}_2+\cdots + a_{n2}\boldsymbol{e}_n,\\
\cdots\cdots\cdots\cdots\\
\boldsymbol{f}_n = a_{1n}\boldsymbol{e}_1 + a_{2n}\boldsymbol{e}_2+\cdots + a_{nn}\boldsymbol{e}_n,
\end{cases}
\]
则矩阵
\[
\boldsymbol{A}=\begin{pmatrix}
a_{11}&a_{12}&\cdots&a_{1n}\\
a_{21}&a_{22}&\cdots&a_{2n}\\
\vdots&\vdots&&\vdots\\
a_{n1}&a_{n2}&\cdots&a_{nn}
\end{pmatrix}
\]
称为从基\(\{\boldsymbol{e}_1,\boldsymbol{e}_2,\cdots,\boldsymbol{e}_n\}\)到基\(\{\boldsymbol{f}_1,\boldsymbol{f}_2,\cdots,\boldsymbol{f}_n\}\)的过渡矩阵.并且$\left( \boldsymbol{f}_1,\boldsymbol{f}_2,\cdots ,\boldsymbol{f}_n \right) =\left( \boldsymbol{e}_1,\boldsymbol{e}_2,\cdots ,\boldsymbol{e}_n \right) \boldsymbol{A}$.
\end{definition}

\begin{theorem}[同一向量在不同基下坐标向量的关系]\label{theorem:同一向量在不同基下坐标向量的关系}
设\(V\)是数域\(\mathbb{F}\)上\(n\)维线性空间,从基\(\{\boldsymbol{e}_1,\boldsymbol{e}_2,\cdots,\boldsymbol{e}_n\}\)到基\(\{\boldsymbol{f}_1,\boldsymbol{f}_2,\cdots,\boldsymbol{f}_n\}\)的过渡矩阵为\(\boldsymbol{A}=(a_{ij})\). 若\(V\)中向量\(\boldsymbol{\alpha}\)在基\(\{\boldsymbol{e}_1,\boldsymbol{e}_2,\cdots,\boldsymbol{e}_n\}\)下的坐标向量是\((x_1,x_2,\cdots,x_n)'\),在基\(\{\boldsymbol{f}_1,\boldsymbol{f}_2,\cdots,\boldsymbol{f}_n\}\)下的坐标向量是\((y_1,y_2,\cdots,y_n)'\),则
\begin{gather*}
\begin{pmatrix}
x_1\\
x_2\\
\vdots\\
x_n
\end{pmatrix}=\begin{pmatrix}
a_{11}&a_{12}&\cdots&a_{1n}\\
a_{21}&a_{22}&\cdots&a_{2n}\\
\vdots&\vdots&&\vdots\\
a_{n1}&a_{n2}&\cdots&a_{nn}
\end{pmatrix}\begin{pmatrix}
y_1\\
y_2\\
\vdots\\
y_n
\end{pmatrix}.
\end{gather*}
\end{theorem}
\begin{proof}
由过渡矩阵定义可得
\begin{align*}
\boldsymbol{\alpha }=\left( \boldsymbol{e}_1,\boldsymbol{e}_2,\cdots ,\boldsymbol{e}_n \right) \left( \begin{array}{c}
x_1\\
x_2\\
\vdots\\
x_n\\
\end{array} \right) =\left( \boldsymbol{f}_1,\boldsymbol{f}_2,\cdots ,\boldsymbol{f}_n \right) \left( \begin{array}{c}
y_1\\
y_2\\
\vdots\\
y_n\\
\end{array} \right) =\left( \boldsymbol{e}_1,\boldsymbol{e}_2,\cdots ,\boldsymbol{e}_n \right) \boldsymbol{A}\left( \begin{array}{c}
y_1\\
y_2\\
\vdots\\
y_n\\
\end{array} \right) .
\end{align*}
又因为$\left( \boldsymbol{e}_1,\boldsymbol{e}_2,\cdots ,\boldsymbol{e}_n \right) $可逆,所以
\begin{gather*}
\begin{pmatrix}
x_1\\
x_2\\
\vdots\\
x_n
\end{pmatrix}=\begin{pmatrix}
a_{11}&a_{12}&\cdots&a_{1n}\\
a_{21}&a_{22}&\cdots&a_{2n}\\
\vdots&\vdots&&\vdots\\
a_{n1}&a_{n2}&\cdots&a_{nn}
\end{pmatrix}\begin{pmatrix}
y_1\\
y_2\\
\vdots\\
y_n
\end{pmatrix}.
\end{gather*}
\end{proof}

\begin{theorem}\label{theorem:两次基变换后的过渡矩阵}
矩阵\(\boldsymbol{A}\)是\(n\)维线性空间\(V\)的基\(\{\boldsymbol{e}_1,\boldsymbol{e}_2,\cdots,\boldsymbol{e}_n\}\)到基\(\{\boldsymbol{f}_1,\boldsymbol{f}_2,\cdots,\boldsymbol{f}_n\}\)的过渡矩阵,则\(\boldsymbol{A}\)是可逆矩阵且从基\(\{\boldsymbol{f}_1,\boldsymbol{f}_2,\cdots,\boldsymbol{f}_n\}\)到基\(\{\boldsymbol{e}_1,\boldsymbol{e}_2,\cdots,\boldsymbol{e}_n\}\)的过渡矩阵为\(\boldsymbol{A}^{-1}\). 又若\(\boldsymbol{B}\)是从基\(\{\boldsymbol{f}_1,\boldsymbol{f}_2,\cdots,\boldsymbol{f}_n\}\)到基\(\{\boldsymbol{g}_1,\boldsymbol{g}_2,\cdots,\boldsymbol{g}_n\}\)的过渡矩阵,则从基\(\{\boldsymbol{e}_1,\boldsymbol{e}_2,\cdots,\boldsymbol{e}_n\}\)到基\(\{\boldsymbol{g}_1,\boldsymbol{g}_2,\cdots,\boldsymbol{g}_n\}\)的过渡矩阵为\(\boldsymbol{A}\boldsymbol{B}\).
\end{theorem}
\begin{proof}

\end{proof}

\begin{example}
设\(\{\boldsymbol{u}_1,\boldsymbol{u}_2,\cdots,\boldsymbol{u}_n\},\{\boldsymbol{e}_1,\boldsymbol{e}_2,\cdots,\boldsymbol{e}_n\},\{\boldsymbol{f}_1,\boldsymbol{f}_2,\cdots,\boldsymbol{f}_n\}\)是向量空间\(V\)的3组基. 若从\(\boldsymbol{u}_1,\boldsymbol{u}_2,\cdots,\boldsymbol{u}_n\)到\(\boldsymbol{e}_1,\boldsymbol{e}_2,\cdots,\boldsymbol{e}_n\)的过渡矩阵是\(\boldsymbol{A}\),从\(\boldsymbol{u}_1,\boldsymbol{u}_2,\cdots,\boldsymbol{u}_n\)到\(\boldsymbol{f}_1,\boldsymbol{f}_2,\cdots,\boldsymbol{f}_n\)的过渡矩阵是\(\boldsymbol{B}\),求从\(\boldsymbol{e}_1,\boldsymbol{e}_2,\cdots,\boldsymbol{e}_n\)到\(\boldsymbol{f}_1,\boldsymbol{f}_2,\cdots,\boldsymbol{f}_n\)的过渡矩阵.
\end{example}
\begin{solution}
从\(\boldsymbol{e}_1,\boldsymbol{e}_2,\cdots,\boldsymbol{e}_n\)到\(\boldsymbol{u}_1,\boldsymbol{u}_2,\cdots,\boldsymbol{u}_n\)的过渡矩阵为\(\boldsymbol{A}^{-1}\),故从\(\boldsymbol{e}_1,\boldsymbol{e}_2,\cdots,\boldsymbol{e}_n\)到\(\boldsymbol{f}_1,\boldsymbol{f}_2,\cdots,\boldsymbol{f}_n\)的过渡矩阵为\(\boldsymbol{A}^{-1}\boldsymbol{B}\).

\end{solution}

\begin{example}
在四维行向量空间中求从基\(\boldsymbol{e}_1,\boldsymbol{e}_2,\cdots,\boldsymbol{e}_n\)到\(\boldsymbol{f}_1,\boldsymbol{f}_2,\cdots,\boldsymbol{f}_n\)的过渡矩阵,其中
\begin{align*}
\boldsymbol{e}_1&=(1,1,0,1),\boldsymbol{e}_2=(2,1,2,0),\boldsymbol{e}_3=(1,1,0,0),\boldsymbol{e}_4=(0,1,-1,-1),\\
\boldsymbol{f}_1&=(1,0,0,1),\boldsymbol{f}_2=(0,0,1,-1),\boldsymbol{f}_3=(2,1,0,3),\boldsymbol{f}_4=(-1,0,1,2).
\end{align*}
\end{example}
\begin{note}
这类题如用求解线性方程组的方法比较繁,可采用下列方法.
\end{note}
\begin{solution}
设该向量空间的标准基为
\[
\boldsymbol{u}_1=(1,0,0,0),\boldsymbol{u}_2=(0,1,0,0),\boldsymbol{u}_3=(0,0,1,0),\boldsymbol{u}_4=(0,0,0,1),
\]
则由条件可知从\(\boldsymbol{u}_1,\boldsymbol{u}_2,\boldsymbol{u}_3,\boldsymbol{u}_4\)到\(\boldsymbol{e}_1,\boldsymbol{e}_2,\boldsymbol{e}_3,\boldsymbol{e}_4\)的过渡矩阵为
\[
\boldsymbol{A}=\begin{pmatrix}
1&2&1&0\\
1&1&1&1\\
0&2&0&-1\\
1&0&0&-1
\end{pmatrix},
\]
于是就有
\begin{align}\label{equation:3.131.1}
\left( \boldsymbol{e}_1,\boldsymbol{e}_2,\boldsymbol{e}_3,\boldsymbol{e}_4 \right) =\left( \boldsymbol{u}_1,\boldsymbol{u}_2,\boldsymbol{u}_3,\boldsymbol{u}_4 \right) \boldsymbol{A}\Rightarrow \left( \boldsymbol{u}_1,\boldsymbol{u}_2,\boldsymbol{u}_3,\boldsymbol{u}_4 \right) =\left( \boldsymbol{e}_1,\boldsymbol{e}_2,\boldsymbol{e}_3,\boldsymbol{e}_4 \right) \boldsymbol{A}^{-1}.
\end{align}
又由条件可知从\(\boldsymbol{u}_1,\boldsymbol{u}_2,\boldsymbol{u}_3,\boldsymbol{u}_4\)到\(\boldsymbol{f}_1,\boldsymbol{f}_2,\boldsymbol{f}_3,\boldsymbol{f}_4\)的过渡矩阵为
\[
\boldsymbol{B}=\begin{pmatrix}
1&0&2&-1\\
0&0&1&0\\
0&1&0&1\\
1&-1&3&2
\end{pmatrix}.
\]
于是就有
\begin{align}\label{equation:3.131.2}
\left( \boldsymbol{f}_1,\boldsymbol{f}_2,\boldsymbol{f}_3,\boldsymbol{f}_4 \right) =\left( \boldsymbol{u}_1,\boldsymbol{u}_2,\boldsymbol{u}_3,\boldsymbol{u}_4 \right) \boldsymbol{B}.
\end{align}
从而由\eqref{equation:3.131.1}\eqref{equation:3.131.2}式可得
\begin{align*}
\left( \boldsymbol{f}_1,\boldsymbol{f}_2,\boldsymbol{f}_3,\boldsymbol{f}_4 \right) =\left( \boldsymbol{u}_1,\boldsymbol{u}_2,\boldsymbol{u}_3,\boldsymbol{u}_4 \right) \boldsymbol{B}=\left( \boldsymbol{e}_1,\boldsymbol{e}_2,\boldsymbol{e}_3,\boldsymbol{e}_4 \right) \boldsymbol{A}^{-1}\boldsymbol{B}.
\end{align*}
故从基\(\boldsymbol{e}_1,\boldsymbol{e}_2,\boldsymbol{e}_3,\boldsymbol{e}_4\)到\(\boldsymbol{f}_1,\boldsymbol{f}_2,\boldsymbol{f}_3,\boldsymbol{f}_4\)的过渡矩阵为\(\boldsymbol{A}^{-1}\boldsymbol{B}\). 它可以用初等变换和求逆矩阵类似的方法直接求得(对矩阵\((\boldsymbol{A}|\boldsymbol{B})\)进行初等行变换,将\(\boldsymbol{A}\)化为单位矩阵,则右边一块就化为了\(\boldsymbol{A}^{-1}\boldsymbol{B}\))
因此,所求之过渡矩阵为
\[
\begin{pmatrix}
-1&0&1&5\\
-1&1&-1&2\\
4&-2&3&-10\\
-2&1&-2&3
\end{pmatrix}.
\] 
\end{solution}

\begin{example}
设\(a\)为常数,求向量\(\boldsymbol{\alpha}=(a_1,a_2,\cdots,a_n)\)在基\(\{\boldsymbol{f}_1=(a^{n - 1},a^{n - 2},\cdots,a,1),\boldsymbol{f}_2=(a^{n - 2},a^{n - 3},\cdots,1,0),\cdots,\boldsymbol{f}_n=(1,0,\cdots,0,0)\}\)下的坐标.
\end{example}
\begin{proof}
设\(\boldsymbol{e}_1,\boldsymbol{e}_2,\cdots,\boldsymbol{e}_n\)是标准单位行向量,则\(\boldsymbol{\alpha}\)在\(\{\boldsymbol{e}_1,\boldsymbol{e}_2,\cdots,\boldsymbol{e}_n\}\)下的坐标向量就是$\boldsymbol{\alpha}'$,并且从\(\{\boldsymbol{e}_1,\boldsymbol{e}_2,\cdots,\boldsymbol{e}_n\}\)到\(\{\boldsymbol{f}_1,\boldsymbol{f}_2,\cdots,\boldsymbol{f}_n\}\)的过渡矩阵是
\[
\boldsymbol{A}=\begin{pmatrix}
a^{n - 1}&a^{n - 2}&\cdots&1\\
a^{n - 2}&a^{n - 3}&\cdots&0\\
\vdots&\vdots&&\vdots\\
a&1&\cdots&0\\
1&0&\cdots&0
\end{pmatrix}.
\]
设\(\boldsymbol{\alpha}\)在\(\{\boldsymbol{f}_1,\boldsymbol{f}_2,\cdots,\boldsymbol{f}_n\}\)下的坐标向量为\(\boldsymbol{x}=(x_1,x_2,\cdots,x_n)\).则由\hyperref[theorem:同一向量在不同基下坐标向量的关系]{同一向量在不同基下坐标向量的关系}有\(\boldsymbol{A}\boldsymbol{x}'=\boldsymbol{\alpha}'\). 这是一个非齐次线性方程组,可由初等行变换求出方程组的解:
\begin{gather*}
\left( \begin{matrix}
a^{n-1}&		a^{n-2}&		\cdots&		1&		a_1\\
a^{n-2}&		a^{n-3}&		\cdots&		0&		a_2\\
\vdots&		\vdots&		&		\vdots&		\vdots\\
a&		1&		\cdots&		0&		a_{n-1}\\
1&		0&		\cdots&		0&		a_n\\
\end{matrix} \right) \rightarrow \left( \begin{matrix}
0&		a^{n-2}&		\cdots&		1&		a_1-a^{n-1}a_n\\
0&		a^{n-3}&		\cdots&		0&		a_2-a^{n-2}a_n\\
\vdots&		\vdots&		&		\vdots&		\vdots\\
0&		1&		\cdots&		0&		a_{n-1}-aa_n\\
1&		0&		\cdots&		0&		a_n\\
\end{matrix} \right) 
\\
\rightarrow \left( \begin{matrix}
0&		0&		\cdots&		1&		a_1-aa_2\\
0&		0&		\cdots&		0&		a_2-aa_3\\
\vdots&		\vdots&		&		\vdots&		\vdots\\
0&		1&		\cdots&		0&		a_{n-1}-aa_n\\
1&		0&		\cdots&		0&		a_n\\
\end{matrix} \right) \rightarrow \left( \begin{matrix}
1&		0&		\cdots&		0&		a_n\\
0&		1&		\cdots&		0&		a_{n-1}-aa_n\\
\vdots&		\vdots&		&		\vdots&		\vdots\\
0&		0&		\cdots&		0&		a_2-aa_3\\
0&		0&		\cdots&		1&		a_1-aa_2\\
\end{matrix} \right),
\end{gather*}
因此\(\boldsymbol{x}=(a_n,a_{n - 1} - a a_n,\cdots,a_2 - a a_3,a_1 - a a_2)\).
\end{proof}

\begin{example}
设\(V\)是次数不超过\(n\)的实系数多项式全体组成的线性空间,求从基\(\{1,x,x^2,\cdots,x^n\}\)到基\(\{1,x - a,(x - a)^2,\cdots,(x - a)^n\}\)的过渡矩阵,并以此证明多项式的Taylor公式:
\[
f(x)=f(a)+\frac{f'(a)}{1!}(x - a)+\frac{f''(a)}{2!}(x - a)^2+\cdots+\frac{f^{(n)}(a)}{n!}(x - a)^n,
\]
其中\(f^{(n)}(x)\)表示\(f(x)\)的\(n\)次导数.
\end{example}
\begin{solution}
从基\(\{1,x,x^2,\cdots,x^n\}\)到基\(\{1,x - a,(x - a)^2,\cdots,(x - a)^n\}\)的过渡矩阵\((n + 1\)阶\()\)利用二项式定理容易求出为
\[
\boldsymbol{P}=\begin{pmatrix}
1&-a&a^2&\cdots&(-1)^na^n\\
0&1&-2a&\cdots&(-1)^{n - 1}na^{n - 1}\\
0&0&1&\cdots&(-1)^{n - 2}\frac{n(n - 1)}{2!}a^{n - 2}\\
\vdots&\vdots&\vdots&&\vdots\\
0&0&0&\cdots&1
\end{pmatrix}.
\]
注意\(\boldsymbol{P}\)的逆矩阵实际上就是从基\(\{1,x - a,(x - a)^2,\cdots,(x - a)^n\}\)到基\(\{1,x,x^2,\cdots,x^n\}\)的过渡矩阵,结合$x^n=[(x-a)+a]^n$,再利用二项式定理可以马上得到(不必用初等变换法求逆矩阵):
\[
\boldsymbol{P}^{-1}=\begin{pmatrix}
1&a&a^2&\cdots&a^n\\
0&1&2a&\cdots&na^{n - 1}\\
0&0&1&\cdots&\frac{n(n - 1)}{2!}a^{n - 2}\\
\vdots&\vdots&\vdots&&\vdots\\
0&0&0&\cdots&1
\end{pmatrix}.
\]
设\(f(x)=a_0 + a_1x + a_2x^2+\cdots + a_nx^n\),则\(f(x)\)在基\(\{1,x ,x^2,\cdots,x^n\}\)下的坐标向量为$(a_0,a_1,\cdots,a_n)'$.设\(f(x)\)在基\(\{1,x - a,(x - a)^2,\cdots,(x - a)^n\}\)下的坐标向量为$(y_0,y_1,\cdots,y_n)'$.则由\hyperref[theorem:同一向量在不同基下坐标向量的关系]{同一向量在不同基下坐标向量的关系}可知
\begin{align*}
\left( \begin{array}{c}
a_0\\
a_1\\
a_2\\
\vdots\\
a_n\\
\end{array} \right) =\boldsymbol{P}\left( \begin{array}{c}
y_0\\
y_1\\
y_2\\
\vdots\\
y_n\\
\end{array} \right) .
\end{align*}
于是\begin{align*}
\left( \begin{array}{c}
y_0\\
y_1\\
y_2\\
\vdots\\
y_n\\
\end{array} \right) =\boldsymbol{P}^{-1}\left( \begin{array}{c}
a_0\\
a_1\\
a_2\\
\vdots\\
a_n\\
\end{array} \right) =\left( \begin{matrix}
1&		a&		a^2&		\cdots&		a^n\\
0&		1&		2a&		\cdots&		na^{n-1}\\
0&		0&		1&		\cdots&		\frac{n(n-1)}{2!}a^{n-2}\\
\vdots&		\vdots&		\vdots&		&		\vdots\\
0&		0&		0&		\cdots&		1\\
\end{matrix} \right) \left( \begin{array}{c}
a_0\\
a_1\\
a_2\\
\vdots\\
a_n\\
\end{array} \right) =\left( \begin{array}{c}
f(a)\\
\frac{f'(a)}{1!}\\
\vdots\\
\frac{f^{(n)}(a)}{n!}\\
\end{array} \right) .
\end{align*}
由此可得\begin{align*}
f(x)=\left( 1,x-a,(x-a)^2,\cdots ,(x-a)^n \right) \left( \begin{array}{c}
f(a)\\
\frac{f'(a)}{1!}\\
\vdots\\
\frac{f^{(n)}(a)}{n!}\\
\end{array} \right) =f(a)+\frac{f'(a)}{1!}(x-a)+\frac{f''(a)}{2!}(x-a)^2+\cdots +\frac{f^{(n)}(a)}{n!}(x-a)^n.        
\end{align*}
\end{solution}

\subsection{练习}

\begin{exercise}
验证下列映射是线性同构:
\begin{enumerate}[(1)]
\item 一维实行向量空间\(\mathbb{R}\),\hyperref[example:3.3(5)]{例题}\ref{example:3.3}\ref{example:3.3(5)}中的实线性空间\(\mathbb{R}^+\),映射\(\varphi:\mathbb{R}\to\mathbb{R}^+\)定义为\(\varphi(x)=\mathrm{e}^x\);
\item 二维实行向量空间\(\mathbb{R}_2\),\hyperref[example:3.3(6)]{例题}\ref{example:3.3}\ref{example:3.3(6)}中的实线性空间\(V\),映射\(\varphi:\mathbb{R}_2\to V\)定义为\(\varphi(a,b)=(a,b+\frac{1}{2}a^2)\).
\end{enumerate}
\end{exercise}
\begin{solution}
\begin{enumerate}[(1)]
\item  \(\varphi\)的逆映射是\(\psi:\mathbb{R}^+\to\mathbb{R},\psi(y)=\ln y\),故\(\varphi\)是一一对应的. 根据加法和数乘的定义可得
\[
\varphi(x + y)=\mathrm{e}^{x + y}=\mathrm{e}^x\mathrm{e}^y=\varphi(x)\oplus\varphi(y),\varphi(kx)=\mathrm{e}^{kx}=(\mathrm{e}^x)^k=k\circ\varphi(x),
\]
因此\(\varphi:\mathbb{R}\to\mathbb{R}^+\)是线性同构.
\item \(\varphi\)的逆映射是\(\psi:V\to\mathbb{R}_2,\psi(x,y)=(x,y-\frac{x^2}{2})\),故\(\varphi\)是一一对应的. 根据具体的计算可得
\[
\varphi(a_1 + a_2,b_1 + b_2)=\varphi(a_1,b_1)\oplus\varphi(a_2,b_2),\varphi(ka,kb)=k\circ\varphi(a,b),
\]
因此\(\varphi:\mathbb{R}_2\to V\)是线性同构.
\end{enumerate}
\end{solution}

\begin{exercise}
构造下列线性空间之间的线性同构:
\begin{enumerate}[(1)]
\item \(V\)是数域\(\mathbb{K}\)上的\(n\)阶上三角矩阵构成的线性空间,\(U\)是数域\(\mathbb{K}\)上的\(n\)阶对称矩阵构成的线性空间(\hyperref[example:一些常见线性空间的基(4)]{例题\ref{example:一些常见线性空间的基}\ref{example:一些常见线性空间的基(4)}});
\item \(V\)是数域\(\mathbb{K}\)上主对角元全为零的\(n\)阶上三角矩阵构成的线性空间,\(U\)是数域\(\mathbb{K}\)上的\(n\)阶反对称矩阵构成的线性空间(\hyperref[example:一些常见线性空间的基(6)]{例题\ref{example:一些常见线性空间的基}\ref{example:一些常见线性空间的基(6)}});
\item \(V\)是\(n\)阶Hermite矩阵构成的实线性空间,\(U\)是\(n\)阶斜Hermite矩阵构成的实线性空间(\hyperref[example:n阶(斜)Hermite矩阵全体构成的线性空间]{例题\ref{example:n阶(斜)Hermite矩阵全体构成的线性空间}}).
\end{enumerate}
\end{exercise}
\begin{solution}
\begin{enumerate}[(1)]
\item  \(\varphi:V\to U\)定义为:对任意的\(\boldsymbol{A}=(a_{ij})\in V\),当\(i\leqslant  j\)时,矩阵\(\varphi(\boldsymbol{A})\)的第\((i,j)\)元素为\(a_{ij}\);当\(i > j\)时,矩阵\(\varphi(\boldsymbol{A})\)的第\((i,j)\)元素为\(a_{ji}\). 容易验证\(\varphi:V\to U\)是定义好的映射,并且是数域\(\mathbb{K}\)上的线性同构.
\item \(\varphi:V\to U\)定义为:对任意的\(\boldsymbol{A}=(a_{ij})\in V\),\(\varphi(\boldsymbol{A})=\boldsymbol{A}-\boldsymbol{A}'\). 容易验证\(\varphi:V\to U\)是定义好的映射,并且是数域\(\mathbb{K}\)上的线性同构.
\item \(\varphi:V\to U\)定义为:对任意的\(\boldsymbol{A}=(a_{ij})\in V\),\(\varphi(\boldsymbol{A})=\mathrm{i}\boldsymbol{A}\). 容易验证\(\varphi:V\to U\)是定义好的映射,并且是实数域上的线性同构. 注意到\(\varphi\)的逆映射\(\psi:U\to V\)为:\(\psi(\boldsymbol{B})=-\mathrm{i}\boldsymbol{B}\). 
\end{enumerate}
\end{solution}



\end{document}
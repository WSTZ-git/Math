\documentclass[../../main.tex]{subfiles}
\graphicspath{{\subfix{../../image/}}} % 指定图片目录,后续可以直接使用图片文件名。

% 例如:
% \begin{figure}[H]
% \centering
% \includegraphics[scale=0.4]{图.png}
% \caption{}
% \label{figure:图}
% \end{figure}
% 注意:上述\label{}一定要放在\caption{}之后,否则引用图片序号会只会显示??.

\begin{document}

\section{环}

\begin{definition}[环]
我们称$(R, +, \cdot)$ 是一个\textbf{环},当 $(R, +)$ 是个阿贝尔群,$(R, \cdot)$ 是个幺半群,且乘法对加法有左右分配律,即
\begin{align*}
&\forall a,b,c\in R, a(b + c)=ab + ac ,\\
&\forall a,b,c\in R, (a + b)c=ac + bc .
\end{align*} 
\end{definition}
\begin{remark}
我们把环$(R, +, \cdot)$ 中的加法单位元记作0,乘法单位元记作1.对任意的$a\in R$,我们将$a$的加法逆元记作$-a$,乘法逆元记作$a^{-1}$.
\end{remark}
\begin{note}
最常见的环是整数环$(\mathbb{Z},+,\cdot)$.
\end{note}

\begin{definition}[交换环]
设$(R, +, \cdot)$ 是一个环,我们称 $R$ 是一个\textbf{交换环},当 $R$ 对乘法有交换律,即
\begin{align*}
\forall a,b\in R, ab = ba.
\end{align*}
也即$(R,\cdot)$是一个交换幺半群.
\end{definition}

\begin{example}
\begin{enumerate}
\item $(\mathbb{Z}_n,+,\cdot)$是一个交换环.

\item $(M(n,\mathbb{R}),+,\cdot)$是一个环(不是交换环).
\end{enumerate}
\end{example}
\begin{proof}
\begin{enumerate}
\item 由\refpro{proposition:Z_n是一个Abel群}可知 $(\mathbb{Z}_n,+)$是一个Abel群.又由\refpro{proposition:Z_n是一个交换幺半群}可知$(\mathbb{Z}_n,\cdot)$是一个交换幺群.因此我们只须证明分配律即可.对$\forall \overline{a},\overline{b},\overline{c}\in \mathbb{Z} _n$,都有
\begin{gather*}
\overline{a}\left( \overline{b}+\overline{c} \right) =\overline{a}\left( \overline{b+c} \right) =\overline{a\left( b+c \right) }=\overline{ab+ac}=\overline{ab}+\overline{ac}=\overline{a}\overline{b}+\overline{a}\,\overline{c}.
\\
\left( \overline{a}+\overline{b} \right) \overline{c}=\left( \overline{a+b} \right) \overline{c}=\overline{\left( a+b \right) c}=\overline{ac+bc}=\overline{ac}+\overline{bc}=\overline{a}\,\overline{c}+\overline{b}\overline{c}.
\end{gather*}
综上,$(\mathbb{Z}_n,+,\cdot)$是一个交换环.

\item $(M(n,\mathbb{R}),+,\cdot)$是一个环的证明是显然的.
\end{enumerate}
\end{proof}

\begin{proposition}\label{proposition:环的加法单位元和乘法逆元的性质}
设$(R, +, \cdot)$ 是一个环,而 $a,b,c\in R$,则
\begin{enumerate}[(1)]
\item $a0 = 0a = 0,$

\item $a(-b)=(-a)b=-(ab),$

\item $(-a)(-b)=ab.$
\end{enumerate}
\end{proposition}
\begin{proof}
\begin{enumerate}[(1)]
\item 首先,利用分配律,
\begin{align*}
a0 = a(0 + 0)=a0 + a0.
\end{align*}
因此 $a0 = 0$。根据对称性,$0a = a$.

\item 根据对称性,我们只须证明 $a(-b)=-(ab)$.而这是因为
\begin{align*}
a(-b)+ab = a(-b + b)=a0 = 0.
\end{align*}

\item 利用两次(2)的结论和\refpro{proposition:群中元素取两次逆元还是其自身},我们就得到
\begin{align*}
(-a)(-b)=-(a(-b))=-(-(ab)) = ab.
\end{align*} 
\end{enumerate}
\end{proof}

\begin{definition}[零环]
有一个重要的环是零环,它是最平凡的环,即 $(0, +, \cdot)$,也记作$\{0\}$.它只有一个元素,既是加法单位元也是乘法单位元,定义为
\begin{gather*}
00 = 0 ,\\
0 + 0 = 0 .
\end{gather*} 
\end{definition}
\begin{note}
很容易检验这是一个环。
\end{note}

\begin{proposition}[零环的充要条件]\label{proposition:零环的充要条件}
设 $(R, +, \cdot)$ 是一个环,则 $R = \{0\}$ 当且仅当 $0 = 1$。
\end{proposition}
\begin{proof}
必要性($\Rightarrow$)是显然的。

我们来证明充分性($\Leftarrow$)。假设 $0 = 1$,我们只须证明对所有 $a \in R$,都有 $a = 0$。由\refpro{proposition:环的加法单位元和乘法逆元的性质}可知
\begin{align*}
a = a1 = a0 = 0
\end{align*}
这就证明了这个命题。 
\end{proof}

\begin{definition}[单位及所有单位构成的群]
设$(R, +, \cdot)$ 是一个环,则 $(R^\times, \cdot)$,是由 $R$ 中所有乘法可逆元素构成的群。$R$ 中的乘法可逆元素又被称为 $R$ 中的\textbf{单位}。 
\end{definition}
\begin{remark}
由\reflem{lemma:幺半群中所有可逆元构成了群}可知, $R$ 中所有乘法可逆元素构成了一个群.故上述$(R^\times, \cdot)$的定义是良定义的.
\end{remark}

\begin{proposition}\label{proposition:非零环中0不是单位,1是单位}
设$(R, +, \cdot)$ 是一个环,若$R\ne \{0\}$,则0一定不是单位,1一定是单位.
\end{proposition}
\begin{proof}
因为$R\ne \{0\}$,所以由\refpro{proposition:零环的充要条件}可知$0\ne 1$.于是对$\forall a\in R$,由\refpro{proposition:环的加法单位元和乘法逆元的性质}可知$a\cdot 0=0\ne 1.$故0一定没有逆元,即0不是单位.

由于$1\cdot 1=1$,因此1的逆元就是其自身,故1一定是单位.
\end{proof}

\begin{definition}[除环]
设 $(R, +, \cdot)$ 是一个环,我们称 $(R, +, \cdot)$ 是一个\textbf{除环},若
\begin{align*}
R\setminus\{0\} = R^\times
\end{align*}
也即,所有非零元素都是单位。 
\end{definition}

\begin{proposition}[除环的充要条件]\label{proposition:除环的充要条件}
$(R, +, \cdot)$ 是一个除环,当且仅当同时满足下面三个条件
\begin{enumerate}[(i)]
\item $(R, +)$  是一个Abel群,

\item $(R\setminus\{0\}, \cdot)$ 是一个群,

\item 乘法对加法有左右分配律.
\end{enumerate}
\end{proposition}
\begin{proof}
根据定义,这是显然的。
\end{proof}

\begin{definition}[交换的除环]
设$(R,+,\cdot)$是一个除环,我们称$(R,+,\cdot)$是一个\textbf{交换的除环},当 $R$ 对乘法有交换律,即
\begin{align*}
\forall a,b\in R, ab = ba.
\end{align*}
即$(R,\cdot)$是一个交换幺半群.
也即$(R\setminus\{0\}, \cdot)=(R^\times,\cdot)$是一个Abel群.
\end{definition}

\begin{definition}[域]
设$(R, +, \cdot)$ 是一个环,我们称 $(R, +, \cdot)$ 是一个\textbf{域},若它是一个交换的除环。
\end{definition}

\begin{proposition}[域的充要条件]\label{proposition:域的充要条件}
$(R, +, \cdot)$ 是一个域,当且仅当同时满足下面三个条件
\begin{enumerate}[(i)]
\item $(R, +)$  是一个Abel群,

\item $(R\setminus\{0\}, \cdot)$ 是一个Abel群,

\item 乘法对加法有左右分配律.
\end{enumerate}
\end{proposition}
\begin{proof}
根据定义,这是显然的。
\end{proof}

\begin{proposition}\label{proposition:域中加法单位元和乘法单位元一定不相等即0不等于1}
设$(R, +, \cdot)$ 是一个域,则$R\ne \{0\}$,进而$0\ne 1$.
\end{proposition}
\begin{proof}
反证,若$R=\{0\}$.则$R\setminus \{0\}=\varnothing.$而空集一定不是群,故$R\setminus \{0\}=\varnothing$一定不是Abel群,而由\refpro{proposition:域的充要条件}可知$R\setminus \{0\}=\varnothing$是Abel群,矛盾!进而,由\refpro{proposition:零环的充要条件}可知$0\ne 1.$
\end{proof}

\begin{definition}[子环]
设 $(R, +, \cdot)$ 是一个环,而 $S \subset R$。我们称 $S$ 是 $R$ 的\textbf{子环},记作 $S < R$,若同时满足下面三个条件
\begin{enumerate}[(i)]
\item $0, 1 \in S,$

\item $\forall a, b \in S, a + b, ab \in S,$

\item $\forall a \in S, -a \in S.$
\end{enumerate}
\end{definition}
\begin{note}
事实上,这就是说 $(S, +)$ 是 $(R, +)$ 的子群,$(S, \cdot)$ 是 $(R, \cdot)$ 的子幺半群。 又因为$(R,+)$是Abel群,所以$(S,+)$一定是$(R,+)$的正规子群.
\end{note}

\begin{lemma}[子环的充要条件]\label{lemma:子环的充要条件}
设 $(R, +, \cdot)$ 是一个环,而 $S \subset R$,则 $S < R$ 当且仅当
\begin{gather*}
1 \in S ,\\
\forall a, b \in S, \, a - b, ab \in S .
\end{gather*}
\end{lemma}
\begin{note}
例如 $\mathbb{Z}< \mathbb{Q}< \mathbb{R}< \mathbb{C}$ . 
\end{note}
\begin{proof}
假如满足了这两个条件,那么 $0 = 1 - 1 \in S$。而 $-a = 0 - a \in S$,$a + b = a - (-b) \in S$。这就证明了这是个子环。

另一个方向是显然的。假如 $S$ 是子环,那么 $a - b = a + (-b) \in S$。
\end{proof}

\begin{proposition}[子环仍是环]\label{proposition:子环仍是环}
设$(R,+,\cdot)$是一个环,$S$是其子环,则$(S,+,\cdot)$也是环.
\end{proposition}
\begin{proof}
由子环的定义可知$S$对加法和乘法满足封闭性,从而加法和乘法是$S$上代数运算.于是再结合$0,1\in S$且$S\subset R$,将$(R,+,\cdot)$的性质照搬过来即可.
\end{proof}

\begin{definition}[由子集生成的子环]
设$(R, +, \cdot)$ 是一个环,而 $A \subset R$,则 \textbf{$\boldsymbol{A}$ 生成的子环},记作 $\langle A\rangle$,定义为所有包含了 $A$ 的子环的交集,即
\begin{align*}
\langle A\rangle = \bigcap \{S \subset R : S \supset A, S < R\} .
\end{align*}
\end{definition}

\begin{proposition}[由子集生成的子环仍是子环]\label{proposition:生成的子环仍是子环}
设$(R, +, \cdot)$ 是一个环,而 $A \subset R$,则 $\langle A\rangle < R$。
\end{proposition}
\begin{proof}
首先这个集族是非空的,因为 $R$ 本身就是一个包含了 $A$ 的子环。

接下来,我们利用上面的引理。令 $S$ 是一个包含了 $A$ 的子环。因为 $1$ 在每一个这样的 $S$ 中,所以 $1 \in \langle A\rangle$。

令 $a, b \in \langle A\rangle$,则 $a - b, ab$ 在每一个这样的 $S$ 中,因为每一个 $S$ 都是子环。因此 $a - b, ab \in \langle A\rangle$。

综上所述,$\langle A\rangle < R$。
\end{proof}

\begin{definition}[环的直积]
设$((R_i, +_i, \cdot_i)_{i \in I}$ 是一族环。我们定义这一族环的\textbf{直积},为 $(\prod_{i \in I} R_i, +, \cdot)$。对于 $(x_i)_{i \in I}, (y_i)_{i \in I} \in \prod_{i \in I} R_i$,我们
\begin{align}
(x_i)_{i \in I} + (y_i)_{i \in I} &= (x_i +_i y_i)_{i \in I} \label{eq:2.23}\\
(x_i)_{i \in I} \cdot (y_i)_{i \in I} &= (x_i \cdot_i y_i)_{i \in I} \label{eq:2.24}
\end{align}
\end{definition}

\begin{proposition}[环的直积仍是环]\label{proposition:环的直积仍是环}
设$((R_i, +_i, \cdot_i)_{i \in I}$ 是一族环,则它们的直积 $(\prod_{i \in I} R_i, +, \cdot)$ 还是一个环。
\end{proposition}
\begin{proof}
由\refpro{proposition:一族幺半群的直积仍是幺半群}和\refpro{proposition:一族Abel群的直积仍是Abel群}可知,幺半群和Abel群对直积是保持的,从而我们立刻知道 $\prod_{i \in I} R_i$ 对加法构成Abel群,对乘法构成幺半群。因此只须检验乘法对加法的左右分配律。根据对称性,我们只证明左分配律。

由于$((R_i, +_i, \cdot_i)_{i \in I}$ 是一族环,因此$((R_i, +_i, \cdot_i)_{i \in I}$的乘法对加法有左分配律.故令 $(x_i)_{i \in I}, (y_i)_{i \in I}, (z_i)_{i \in I} \in \prod_{i \in I} R_i$,则
\begin{align*}
(x_i)_{i\in I}\cdot ((y_i)_{i\in I}+(z_i)_{i\in I})&=(x_i\cdot _i(y_i+_iz_i))_{i\in I}=(x_i\cdot _iy_i+_ix_i\cdot _iz_i)_{i\in I}\\
&=(x_i\cdot _iy_i)_{i\in I}+(x_i\cdot _iz_i)_{i\in I}=(x_i)_{i\in I}\cdot (y_i)_{i\in I}+(x_i)_{i\in I}\cdot (z_i)_{i\in I}.
\end{align*}
因此,$(\prod_{i \in I} R_i, +, \cdot)$ 是一个环。这就证明了这个命题。 
\end{proof}






\end{document}
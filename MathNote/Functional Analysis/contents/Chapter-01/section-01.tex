\documentclass[../../main.tex]{subfiles}
\graphicspath{{\subfix{../../image/}}} % 指定图片目录,后续可以直接使用图片文件名。

% 例如:
% \begin{figure}[H]
% \centering
% \includegraphics[scale=0.4]{图.png}
% \caption{}
% \label{figure:图}
% \end{figure}
% 注意:上述\label{}一定要放在\caption{}之后,否则引用图片序号会只会显示??.

\begin{document}

\section{压缩映象原理}

\begin{definition}
设$\mathscr{X}$是一个非空集。$\mathscr{X}$叫做\textbf{距离(度量)空间},是指在$\mathscr{X}$上定义了一个双变量的实值函数$\rho(x,y)$,满足下列三个条件:
\begin{enumerate}[(1)]
\item $\rho(x,y)\geqslant 0$,而且$\rho(x,y)=0$,当且仅当$x=y$;
\item $\rho(x,y)=\rho(y,x)$;
\item $\rho(x,z)\leqslant \rho(x,y)+\rho(y,z)$ $(\forall x,y,z\in \mathscr{X})$。
\end{enumerate}
这里$\rho$叫做$\mathscr{X}$上的一个\textbf{距离};以$\rho$为距离的距离空间$\mathscr{X}$记做$(\mathscr{X},\rho)$。
\end{definition}
\begin{remark}
距离概念是欧氏空间中两点间距离的抽象。事实上,如果对$\forall x=(x_1,x_2,\cdots,x_n)$,$y=(y_1,y_2,\cdots,y_n)\in \mathbb{R}^n$,令
\begin{align*}
\rho(x,y)=\left[(x_1 - y_1)^2 + \cdots + (x_n - y_n)^2\right]^{\frac{1}{2}}.
\end{align*}
容易看到(1),(2),(3)都满足。以后当说到欧氏空间时,我们始终用这个$\rho$规定其上的距离。
\end{remark}
\begin{note}
引进距离的目的是刻划“收敛”。
\end{note}

\begin{example}[空间C[a,b]]

区间$[a,b]$上的连续函数全体记为 $C[a,b]$,按距离
\begin{align}
\label{eq:1.1.2}
\rho(x,y) \triangleq \max_{a \leqslant t \leqslant b} |x(t) - y(t)|
\end{align}
形成距离空间$(C[a,b],\rho)$,以后简记作 $C[a,b]$。以后当说到\textbf{连续函数空间} $C[a,b]$时,我们始终用\eqref{eq:1.1.2}规定的 $\rho$ 作为其上的距离,除非另外说明.
\end{example}
\begin{proof}
证明是显然的.

\end{proof}

\begin{definition}
距离空间$(\mathscr{X},\rho)$上的点列$\{x_n\}$叫做收敛到$x_0$的是指:$\rho(x_n,x_0)\to 0\,(n\to \infty)$。这时记作$\lim\limits_{n \to \infty} x_n = x_0$。或简单地记作$x_n \to x_0$。
\end{definition}
\begin{remark}
在$C[a,b]$中点列$\{x_n\}$收敛到$x_0$是指:$\{x_n(t)\}$一致收敛到$x_0(t)$。
\end{remark}
\begin{note}
与实数集合一样,对于一般的度量空间可引进闭集和完备性等概念.
\end{note}

\begin{definition}
度量空间$(\mathscr{X},\rho)$中的一个子集$E$称为\textbf{闭集},是指:$\forall \{x_n\} \subset E$,若$x_n \to x_0$,则$x_0 \in E$。
\end{definition}

\begin{definition}
距离空间$(\mathscr{X},\rho)$上的点列$\{x_n\}$叫做\textbf{基本列},是指:$\rho(x_n,x_m) \to 0\,(n,m \to \infty)$。这也就是说:$\forall \varepsilon > 0$,$\exists N(\varepsilon)$,使得$m,n \geqslant N(\varepsilon) \Rightarrow \rho(x_n,x_m) < \varepsilon$。如果空间中所有基本列都是收敛列,那末就称该空间是\textbf{完备的}.
\end{definition}

\begin{example}
$(\mathbb{R}^n,\rho)$是完备的.
\end{example}
\begin{proof}


\end{proof}

\begin{example}
$(C[a,b],\rho)$是完备的。
\end{example}
\begin{proof}
设$\{x_n\}$是$(C[a,b],\rho)$中的一串基本列,那么$\forall \varepsilon>0$,$\exists N(\varepsilon)$,使得对$\forall m,n\geqslant N(\varepsilon)$有
$$\rho(x_m,x_n)=\max_{a\leqslant t\leqslant b}|x_m(t)-x_n(t)|<\varepsilon,$$
因此,对$\forall t\in [a,b]$,
\begin{align}
\label{eq:1.1.3}
|x_m(t)-x_n(t)|<\varepsilon \quad (\forall m,n\geqslant N(\varepsilon)).
\end{align}
固定$t\in [a,b]$,我们看到数列$\{x_n(t)\}$是基本的,由于$(\mathbb{R}^n,\rho)$是完备的,因此极限$\lim\limits_{n\to \infty} x_n(t)$存在。让我们用$x_0(t)$表示此极限,在\eqref{eq:1.1.3}中令$m\to \infty$得到$|x_0(t)-x_n(t)|\leqslant \varepsilon (\forall n\geqslant N(\varepsilon))$。由此可见$x_n(t)$一致收敛到$x_0(t)$,从而$x_0(t)$连续并在$C[a,b]$中$x_n$收敛到$x_0$。

\end{proof}

\begin{definition}
设$T: (\mathscr{X}, \rho) \to (\mathscr{Y}, r)$是一个映射,称它是\textbf{连续的},如果对于$\mathscr{X}$中的任意点列$\{x_n\}$和点$x_0$,
$$\rho(x_n, x_0) \to 0 \Rightarrow r(Tx_n, Tx_0) \to 0 \quad (n \to \infty).$$
\end{definition}

\begin{proposition}[连续映射充要条件]\label{proposition:距离空间的连续映射充要条件}
$T: (\mathscr{X}, \rho) \to (\mathscr{Y}, r)$是连续的的充要条件是对$\forall \varepsilon > 0$,$\forall x_0 \in \mathscr{X}$,$\exists \delta = \delta(x_0, \varepsilon) > 0$,使得
\begin{align}
\label{eq:1.1.4}
\rho(x, x_0) < \delta \Rightarrow r(Tx, Tx_0) < \varepsilon \quad (\forall x \in \mathscr{X}).
\end{align}
\end{proposition}
\begin{proof}
必要性。若\eqref{eq:1.1.4}不成立,必$\exists x_0 \in \mathscr{X}$,$\exists \varepsilon > 0$,使得$\forall n \in \mathbb{N}$,$\exists x_n$使得$\rho(x_n, x_0) < 1/n$,但$r(Tx_n, Tx_0) \geqslant \varepsilon$,即得$\lim\limits_{n \to \infty} \rho(x_n, x_0) = 0$,但$\lim\limits_{n \to \infty} r(Tx_n, Tx_0) \neq 0$,矛盾。

充分性。设\eqref{eq:1.1.4}成立,且$\lim\limits_{n \to \infty} \rho(x_n, x_0) = 0$,那么$\forall \varepsilon > 0$,$\exists N = N(\delta(x_0, \varepsilon))$,使得当$n > N$时,$\rho(x_n, x_0) < \delta$。从而$r(Tx_n, Tx_0) < \varepsilon$,即得$\lim\limits_{n \to \infty} r(Tx_n, Tx_0) = 0$。

\end{proof}

\begin{definition}
设$(\mathscr{X},\rho)$是一个距离空间,称映射$T:(\mathscr{X},\rho) \to (\mathscr{X},\rho)$满足$\mathbf{Lipschitz}$\textbf{条件},$L$是$\mathbf{Lipschitz}$\textbf{常数}.如果存在$L >0$,使得$$\rho(Tx,Ty) \leqslant L\rho(x,y)\,(\forall x,y \in \mathscr{X}).$$
\end{definition}

\begin{theorem}\label{theorem:满足Lipschitz条件的映射必连续}
设$(\mathscr{X},\rho)$是一个距离空间,映射$T:(\mathscr{X},\rho) \to (\mathscr{X},\rho)$满足Lipschitz条件,$L$是Lipschitz常数,则的映射$T$是连续的。
\end{theorem}
\begin{proof}
对$\forall \varepsilon > 0$,$\forall x_0 \in \mathscr{X}$,取$\delta=\frac{\varepsilon}{L}$,使得当$\rho(x, x_0) < \delta$时,有
\begin{align*}
\rho \left( Tx,Tx_0 \right) \leqslant L\rho \left( x,x_0 \right) <\varepsilon \quad \left( \forall x\in \mathscr{X} \right) .
\end{align*}
故由\refpro{proposition:距离空间的连续映射充要条件}知$T$是连续的.

\end{proof}

\begin{definition}
设$(\mathscr{X},\rho)$是一个距离空间,称$T:(\mathscr{X},\rho) \to (\mathscr{X},\rho)$是一个\textbf{压缩映射},如果存在$0 < \alpha < 1$,使得$$\rho(Tx,Ty) \leqslant \alpha\rho(x,y)\,(\forall x,y \in \mathscr{X}).$$
\end{definition}
\begin{note}
显然压缩映射满足Lipschitz条件.进而由\refthe{theorem:满足Lipschitz条件的映射必连续}可知\textbf{压缩映射一定是连续的}.
\end{note}

\begin{example}
设$\mathscr{X} = [0,1]$,$T(x)$是$[0,1]$上的一个可微函数,满足条件:
\begin{align}
\label{eq:1.1.8}
T(x) \in [0,1] \quad (\forall x \in [0,1]),
\end{align}
以及
\begin{align}
\label{eq:1.1.9}
|T'(x)| \leqslant \alpha < 1 \quad (\forall x \in [0,1]),
\end{align}
则映射$T:\mathscr{X} \to \mathscr{X}$是一个压缩映射。
\end{example}
\begin{proof}
由Lagrange中值定理可知
\begin{align*}
\rho(Tx,Ty) &= |T(x) - T(y)|= |T'(\theta x + (1 - \theta)y)(x - y)| \\
&\leqslant \alpha|x - y| = \alpha\rho(x,y) \quad (\forall x,y \in \mathscr{X}, 0 < \theta < 1).
\end{align*}

\end{proof}

\begin{theorem}[Banach不动点定理——压缩映象原理]\label{theorem:Banach不动点定理——压缩映象原理}
设$(\mathscr{X},\rho)$是一个完备的距离空间,$T$是$(\mathscr{X},\rho)$到其自身的一个压缩映射,则$T$在$\mathscr{X}$上存在唯一的不动点。
\end{theorem}
\begin{note}
压缩映射原理就是距离空间上的一个很简单而基本的不动点定理,也是泛函分析中的一个最常用、最简单的存在性定理.
\end{note}
\begin{proof}
任取初始点$x_0 \in \mathscr{X}$。考察迭代产生的序列
$$x_{n+1} = Tx_n \quad (n = 0,1,2,\cdots).$$
我们有
\begin{align*}
\rho(x_{n+1}, x_n) = \rho(Tx_n, Tx_{n-1}) \leqslant \alpha \rho(x_n, x_{n-1}) \leqslant \cdots \leqslant \alpha^n \rho(x_1, x_0).
\end{align*}
从而对$\forall p \in \mathbb{N}$,
\begin{align*}
\rho(x_{n+P}, x_n) \leqslant \sum_{i=1}^p \rho(x_{n+i}, x_{n+i-1}) \leqslant \frac{\alpha^n}{1 - \alpha} \rho(x_1, x_0) \to 0.
\end{align*}
(当$n \to \infty$,对$\forall P \in \mathbb{N}$一致)。由此可见$\{x_n\}$是一个基本列.因为$(\mathbb{R}^1,\rho)$是完备的,所以$\{x_n\}$收敛,设$\underset{n\rightarrow \infty}{\lim}x_n=x^*.$又$T$是连续的,故$\underset{n\rightarrow \infty}{\lim}Tx_n=Tx^*.$于是对$x_{n+1} = Tx_n$两边同时取极限得$x^*=Tx^*,$即$x^*$是$T$在$\mathscr{X}$上的一个不动点。若$x^*, x^{**}$都是不动点,则
\begin{align*}
\rho \left( x^*,x^{**} \right) =\rho \left( Tx^*,Tx^{**} \right) \leqslant \alpha \rho \left( x^*,x^{**} \right) \Longrightarrow \left( 1-\alpha \right) \rho \left( x^*,x^{**} \right) \leqslant 0.
\end{align*}
由此推出
$x^* = x^{**}.$

\end{proof}
\begin{remark}
我们可以把一些问题转化为不动点问题.比如下面的例子和\refthe{theorem:常微分方程初值问题解的局部存在唯一性}.

设$\varphi$是$\mathbb{R}^1$上定义的实函数,求方程
$$\varphi(x) = 0$$
的根的问题可以看成$\mathbb{R}^1 \to \mathbb{R}^1$的映射
$$f(x) = x - \varphi(x)$$
的不动点问题。即求$x \in \mathbb{R}^1$满足:
$$f(x) = x.$$
\end{remark}

\begin{proposition}\label{proposition:习题1.1.1}
证明:完备空间的闭子集是一个完备的子空间,而任一度量空间中的完备子空间必是闭子集。
\end{proposition}
\begin{proof}
(1) 设 \( X \) 是完备度量空间,\( M \subset X \) 是闭的. 要证 \( M \) 是一个完备的子空间.
\[
\forall\ x_m, x_n \in M ,\|x_m - x_n\| \to 0 (m, n \to \infty)\implies \forall\ x_m, x_n \in X , \|x_m - x_n\| \to 0 (m, n \to \infty).
\]
因为\(X \) 是完备度量空间,所以\( \exists\ x \in X \),使得 \( x_n \to x \).于是
\[
\begin{cases} 
x_n \in M,\ x_n \to x \\
M \subset X \text{ 是闭的}
\end{cases} \implies x \in M.
\]
\( \forall\ x_m, x_n \in M, \|x_m - x_n\| \to 0, (m, n \to \infty) \)
因为\(X \) 是完备度量空间,所以
\( \exists\ x \in M \),使得 \( x_n \to x \).故$M$是一个完备的子空间.

(2) 设 \( X \) 是一度量空间,\( M \) 是 \( X \) 的一个完备子空间.

要证 \( M \) 是闭子集. 即,若 \( x_n \in M, x_n \to x \),要证 \( x \in M \).

因为收敛列是基本列,所以
\( x_n \in M \),\( \|x_m - x_n\| \to 0 \),\( (m, n \to \infty) \),又 \( M \) 是完备度量空间,
所以 \( \exists\ x' \in M \),使得 \( x_n \to x' \).于是
\[
\begin{cases} 
x_n \to x \\
x_n \to x'
\end{cases} \implies x = x' \in M.
\]

\end{proof}

\begin{theorem}[常微分方程初值问题解的局部存在唯一性]\label{theorem:常微分方程初值问题解的局部存在唯一性}
设$\xi \in \mathbb{R}$,考虑常微分方程
\begin{align}
\label{eq:1.1.5}
\begin{cases}
\dfrac{\mathrm{d}x}{\mathrm{d}t} = F(t,x), \\
x(t_0) = \xi
\end{cases}
\end{align}
其中$F(t,x)$是在矩形域
\begin{align*}
D:\left| t-t_0 \right|\leqslant h_0,\left| x-\xi \right|\leqslant \delta 
\end{align*}
上的二元连续函数,并且对变元$x$关于$t$一致地满足局部Lipschitz条件:$\exists L > 0$,使得当$\left| t-t_0 \right|\leqslant h_0$,$|x_1(t) - \xi| \leqslant \delta$,$|x_2(t) - \xi| \leqslant \delta$时,有
\begin{align}
\label{eq:1.1.10}
|F(t, x_1) - F(t, x_2)| \leqslant L|x_1(t) - x_2(t)|,
\end{align}
则当$h \leqslant \min\{\frac{\delta}{M},h_0\},M=\underset{\left( t,x \right) \in D}{\max}\left| F\left( t,x \right) \right|$时,微分方程\eqref{eq:1.1.5}的初值问题在$[t_0-h,t_0+h]$上存在唯一连续解.
\end{theorem}
\begin{remark}

1.在这里我们把常数$\xi$看成是$[-h_0,h_0]$上恒等于$\xi$的常值函数.

2.本题中我们不能直接取$C[-h_0,h_0]$为\hyperref[theorem:Banach不动点定理——压缩映象原理]{Banach不动点定理——压缩映象原理}中的距离空间$\mathscr{X}$,因为当$Lh_0 < 1$时,$T$只是在$C[-h_0,h_0]$的子集$\bar{B}(\xi,\delta)$上才是压缩的。
\end{remark}
\begin{proof}
不妨设$t_0=0,$否则用$x(t+t_0)$代替$x(t)$即可.注意到
\eqref{eq:1.1.5}式或它的等价形式,即求连续函数$x(t)$满足下列积分方程的问题:
\begin{align}
\label{eq:1.1.6}
x(t) = \xi + \int_0^t F(\tau, x(\tau))\mathrm{d}\tau,
\end{align}
可以看成是一个不动点问题。为此,在以$t = 0$为中心的某区间$[-h_0,h_0]$上考察距离空间$C[-h_0,h_0]$,并引入映射
\begin{align}
\label{eq:1.1.7}
(Tx)(t) = \xi + \int_0^t F(\tau, x(\tau))\mathrm{d}\tau,
\end{align}
则\eqref{eq:1.1.6}等价于求$C[-h_0,h_0]$上的一个点$x$,使得$x = Tx$,即求$T$的不动点。

现在我们已经把这个问题化归为一个求不动点的问题了。先在$C[-h_0,h_0]$上考察由\eqref{eq:1.1.7}定义的映射$T$,注意到
\begin{align*}
\rho (Tx,Ty)&=\max_{|t|\leqslant h_0} \left| \int_0^t{F(\tau ,x(\tau ))\mathrm{d}\tau }-\int_0^t{F(\tau ,y(\tau ))\mathrm{d}\tau } \right|
\\
&\leqslant \int_0^{h_0}{\left| F(\tau ,x(\tau ))-F\left( \tau ,y\left( \tau \right) \right) \right|\mathrm{d}\tau }
\\
&\leqslant h_0\max_{|t|\leqslant h_0} |F(t,x(t))-F(t,y(t))|,
\end{align*}
再结合\eqref{eq:1.1.10}式就有
\begin{align}
\rho (Tx,Ty)\leqslant h_0\max_{|t|\leqslant h_0} |F(t,x(t))-F(t,y(t))|\leqslant Lh_0\max_{|t|\leqslant h_0} \left| x\left( t \right) -y\left( t \right) \right|=Lh_0\rho (x,y)\quad (\forall x,y\in \bar{B}(\xi ,\delta )),\label{eq::::--1.1.000005}
\end{align}
其中 $\bar{B}(\xi,\delta) \triangleq \{x(t) \in C[-h_0,h_0] \Big| \max_{|t| \leqslant h_0} |x(t) - \xi| \leqslant \delta\}$。
我们取$\mathscr{X} = \bar{B}(\xi,\delta)$,令$T_1=\begin{cases}
T&,x\in \overline{B}\left( \xi ,\delta \right)\\
0&,x\notin \overline{B}\left( \xi ,\delta \right)\\
\end{cases},$再设
$$M \triangleq \max \{ |F(t,x)| \Big| (t,x) \in [-h_0,h_0] \times [\xi - \delta, \xi + \delta] \},$$
则当$h \leqslant \min\{\frac{\delta}{M},h_0\}$时,对$\forall x \in \bar{B}(\xi,\delta)$有
$$\max |(Tx)(t) - \xi| = \max \left| \int_0^t F(t, x(\tau))\mathrm{d}\tau  \right| \leqslant Mh \leqslant \delta\Longrightarrow Tx \in \bar{B}(\xi,\delta),$$
故$T_1:\mathscr{X} \to \mathscr{X}.$再结合\eqref{eq::::--1.1.000005}式知$T_1$是$(\mathscr{X},\rho)$上的压缩映射.
由于$(C[-h,h],\rho)$是一个完备的距离空间,而$\mathscr{X}$又是它的一个闭子集,因此$(\mathscr{X},\rho)$还是一个完备的距离空间(\refpro{proposition:习题1.1.1}).于是由\hyperref[theorem:Banach不动点
定理——压缩映象原理]{压缩映象原理}可知,$T_1$在$(\mathscr{X},\rho)$存在唯一不动点,故结论得证.

\end{proof}

\begin{theorem}[隐函数存在定理]\label{theorem:隐函数存在定理}
设 \( f : \mathbb{R}^n \times \mathbb{R}^m \to \mathbb{R}^m \),\( U \times V \subset \mathbb{R}^n \times \mathbb{R}^m \) 是 \( (x_0, y_0) \in \mathbb{R}^n \times \mathbb{R}^m \) 的一个邻域。设 \( f \) 在 \( U \times V \) 内连续并且 \( \forall x \in U \),关于 \( y \in V \) 连续可微。又设
\[
f(x_0, y_0) = 0; \quad \left[ \det \left( \frac{\partial f}{\partial y} \right) \right] (x_0, y_0) \neq 0,
\]
则 \( \exists x_0 \) 的一个邻域 \( U_0 \subset U \) 以及唯一的连续函数 \( \varphi : U_0 \to \mathbb{R}^m \),满足
\[
\begin{cases} 
f(x, \varphi(x)) = 0 & (\text{当 } x \in U_0), \\
\varphi(x_0) = y_0.
\end{cases}
\]
\end{theorem}
\begin{remark}
$\frac{\partial f}{\partial y}$表示$f$关于$y$的Jacobian(雅可比)矩阵,$\det \left( \frac{\partial f}{\partial y} \right)$表示$f$关于$y$的Jacobian(雅可比)行列式.我们有
\begin{align*}
\left[ \det \left( \frac{\partial f}{\partial y} \right) \right] (x_0,y_0)=\left| \begin{matrix}
\frac{\partial f_1}{\partial y_1}\left( x_0,y_0 \right)&		\frac{\partial f_1}{\partial y_2}\left( x_0,y_0 \right)&		\cdots&		\frac{\partial f_1}{\partial y_m}\left( x_0,y_0 \right)\\
\frac{\partial f_2}{\partial y_1}\left( x_0,y_0 \right)&		\frac{\partial f_2}{\partial y_2}\left( x_0,y_0 \right)&		\cdots&		\frac{\partial f_2}{\partial y_m}\left( x_0,y_0 \right)\\
\vdots&		\vdots&		&		\vdots\\
\frac{\partial f_m}{\partial y_1}\left( x_0,y_0 \right)&		\frac{\partial f_m}{\partial y_2}\left( x_0,y_0 \right)&		\cdots&		\frac{\partial f_m}{\partial y_m}\left( x_0,y_0 \right)\\
\end{matrix} \right| .
\end{align*}
\end{remark}
\begin{proof}
考察映射 \( T : \varphi \mapsto T\varphi \),
\[
(T\varphi)(x) \triangleq \varphi(x) - \left( \frac{\partial f}{\partial y}(x_0, y_0) \right)^{-1} f(x, \varphi(x)),
\]
其中 \( \varphi \in C(\bar{B}(x_0, r), \mathbb{R}^m) \),这里 \( r > 0 \),\( C(\bar{B}(x_0, r), \mathbb{R}^m) \) 表示定义在闭球 \( \bar{B}(x_0, r) \) 上取值在 \( \mathbb{R}^m \) 上的向量值连续函数空间,其距离规定为
\[
\rho(\varphi, \psi) \triangleq \max_{\substack{x \in \bar{B}(x_0, r) \\ 1 \leq i \leq m}} |\varphi_i(x) - \psi_i(x)|,
\]
其中 \( \varphi = (\varphi_1, \varphi_2, \ldots, \varphi_m) \);\( \psi = (\psi_1, \psi_2, \ldots, \psi_m) \)。对 \( x \in \mathbb{R}^n \) 与 \( y_i \in \mathbb{R}^m (i = 1, 2, \ldots, m) \),记
\[
D_y f(x, y_1, \ldots, y_m) \triangleq \left( \begin{matrix}
\frac{\partial f_1}{\partial y_1}(x,y_1)&		\cdots&		\frac{\partial f_1}{\partial y_m}(x,y_1)\\
\vdots&		&		\vdots\\
\frac{\partial f_m}{\partial y_1}(x,y_m)&		\cdots&		\frac{\partial f_m}{\partial y_m}(x,y_m)\\
\end{matrix} \right) .
\]
因为\( \frac{\partial f}{\partial y} \) 在 \( U \times V \) 上连续,所以 \( \exists \delta > 0 \),使得
\begin{align}
\left| \delta_{ij} - \left[ \left( \frac{\partial f}{\partial y}(x_0, y_0) \right)^{-1} D_y f(x, y_1, \ldots, y_m) \right]_{ij} \right| < \frac{1}{2m}.\label{eq:::111616156}
\end{align}
\( (i, j = 1, 2, \ldots, m, x \in \bar{B}(x_0, \delta), y_1, \ldots, y_m \in \bar{B}(y_0, \delta)) \),其中 \( [\cdot]_{ij} \) 表示括号内的矩阵的第 \( i \) 行、第 \( j \) 列元素,而
\[
\delta_{ij} \triangleq \begin{cases} 
1 & (i = j), \\
0 & (i \neq j).
\end{cases}
\]
记 \( d_i(x) \triangleq \varphi_i(x) - \psi_i(x) (i = 1, 2, \ldots, m) \)。根据多元函数微分中值定理和\eqref{eq:::111616156}式就有
\begin{align}
&\rho(T\varphi, T\psi) = \max_{\substack{x \in \bar{B}(x_0, r) \\ 1 \leq i \leq m}} \left| \left[ (T\varphi)(x) - (T\psi)(x) \right]_i \right| \nonumber \\
&= \max_{\substack{x \in \bar{B}(x_0, r) \\ 1 \leq i \leq m}} \left| \varphi_i(x) - \psi_i(x) - \left[ \left( \frac{\partial f}{\partial y}(x_0, y_0) \right)^{-1} \left( f(x, \varphi(x)) - f(x, \psi(x)) \right) \right]_i \right| \nonumber \\
&= \max_{\substack{x \in \bar{B}(x_0, r) \\ 1 \leq i \leq m}} \left| d_i(x) - \left[ \left( \frac{\partial f}{\partial y}(x_0, y_0) \right)^{-1} D_y f(x, \hat{y}_1, \ldots, \hat{y}_m) \left( \varphi(x) - \psi(x) \right) \right]_i \right| \nonumber \\
&= \max_{\substack{x \in \bar{B}(x_0, r) \\ 1 \leq i \leq m}} \left| d_i(x) - \sum_{j=1}^m \left[ \left( \frac{\partial f}{\partial y}(x_0, y_0) \right)^{-1} D_y f(x, \hat{y}_1, \ldots, \hat{y}_m) \right]_{ij} d_j(x) \right| \nonumber \\
&< \frac{1}{2} \max_{\substack{x \in \bar{B}(x_0, r) \\ 1 \leq i \leq m}} |d_i(x)| = \frac{1}{2} \rho(\varphi, \psi),
\label{1.1.11}
\end{align}
其中 \( r < \delta \),使得 \( \varphi(x), \psi(x) \in \bar{B}(y_0, \delta) (\forall x \in \bar{B}(x_0, r)) \),\( 0 < \theta_i = \theta_i(x) < 1 \),\( \hat{y}_i(x) = \theta_i \varphi(x) + (1 - \theta_i) \psi(x) (i = 1, 2, \ldots, m) \)。今取
\[
\mathcal{X} \triangleq \{ \varphi \in C(\bar{B}(x_0, r), \mathbb{R}^m) | \varphi(x_0) = y_0, \varphi(x) \in \bar{B}(y_0, \delta) \},
\]
则 \( \mathcal{X} \) 在 \( C(\bar{B}(x_0, r), \mathbb{R}^m) \) 中是闭的,从而是一个完备的度量空间。\eqref{1.1.11}式表明 \( T \) 在 \( \mathcal{X} \)上是压缩的。剩下来只要再证 \( T : \mathcal{X} \to \mathcal{X} \) 就够了。因为根据\hyperref[theorem:Banach不动点定理——压缩映象原理]{压缩映象原理},\( T \) 存在唯一的不动点,这就是我们所要证的。注意到
\begin{align*}
\rho(T\varphi, y_0) &\leq \rho(T\varphi, T y_0) + \rho(T y_0, y_0) \\
&\leq \frac{1}{2} \rho(\varphi, y_0) + \max_{\substack{x \in \bar{B}(x_0, r) \\ 1 \leq i \leq m}} \left| \left[ \left( \frac{\partial f}{\partial y}(x_0, y_0) \right)^{-1} f(x, y_0) \right]_i \right|,
\end{align*}
又由 \( f \) 的连续性,
\begin{align*}
&\max_{\substack{x \in \bar{B}(x_0, r) \\ 1 \leq i \leq m}} \left| \left[ \left( \frac{\partial f}{\partial y}(x_0, y_0) \right)^{-1} f(x, y_0) \right]_i \right| \\
&= \max_{\substack{x \in \bar{B}(x_0, r) \\ 1 \leq i \leq m}} \left| \left[ \left( \frac{\partial f}{\partial y}(x_0, y_0) \right)^{-1} (f(x, y_0) - f(x_0, y_0)) \right]_i \right| \\
&< \frac{\delta}{2} \quad (\text{当 } r < \eta \text{ 足够小}).
\end{align*}
因此,当 \( 0 < r < \min(\eta, \delta) \) 时,\( \rho(T\varphi, y_0) < \delta \)。此外还有
\[
(T\varphi)(x_0) = y_0 + \left( \frac{\partial f}{\partial y}(x_0, y_0) \right)^{-1} f(x_0, \varphi(x_0)) = y_0,
\]
所以 \( T : \mathcal{X} \to \mathcal{X} \)。

\end{proof}








\end{document}
\documentclass[../../main.tex]{subfiles}
\graphicspath{{\subfix{../../image/}}} % 指定图片目录,后续可以直接使用图片文件名。

% 例如:
% \begin{figure}[H]
% \centering
% \includegraphics[scale=0.4]{图.png}
% \caption{}
% \label{figure:图}
% \end{figure}
% 注意:上述\label{}一定要放在\caption{}之后,否则引用图片序号会只会显示??.

\begin{document}

\section{利用留数定理计算定积分}

\subsection{$\int_{-\infty}^{\infty}{f\left( x \right) \mathrm{d}x}$型积分}

\begin{theorem}\label{theorem:定理5.5.1}
设 \( f \) 在上半平面 \( \{ z: \mathrm{Im}z > 0 \} \) 中除去 \( a_1, \cdots, a_n \) 外是全纯的,在 \( \{ z: \mathrm{Im}z \geqslant 0 \} \) 中除去 \( a_1, \cdots, a_n \) 外是连续的.如果 \( \lim_{z \to \infty} z f(z) = 0 \),那么
\begin{align}
\int_{-\infty}^{\infty} f(x) \mathrm{d}x = 2\pi \mathrm{i} \sum_{k = 1}^{n} \mathrm{Res}(f, a_k). \label{eq:::---8979807671283-1}
\end{align}
\end{theorem}
\begin{figure}[H]
\centering
\includegraphics[scale=0.4]{图5.2.png}
\caption{}
\label{figure:图5.2}
\end{figure}
\begin{proof}
\reffig{figure:图5.2}所示,取充分大的 \( R \),使得 \( a_1, \cdots, a_n \) 包含在半圆盘 \( \{ z: |z| < R, \mathrm{Im}z > 0 \} \) 中,记 \( \gamma_R = \{ z: z = R\mathrm{e}^{\mathrm{i}\theta}, 0 \leqslant \theta \leqslant \pi \} \),由\hyperref[theorem:留数定理(残数定理)-定理5.4.9]{留数定理}得
\begin{align}
\int_{-R}^{R} f(x) \mathrm{d}x + \int_{\gamma_R} f(z) \mathrm{d}z = 2\pi \mathrm{i} \sum_{k = 1}^{n} \mathrm{Res}(f, a_k).\label{eq:::---8979807671283-2}
\end{align}
记 \( M(R) = \max \{ |f(z)| : z \in \gamma_R \} \),由假定,\( \lim_{R \to \infty} R M(R) = 0 \),因而
\begin{align*}
\left| \int_{\gamma_R} f(z) \mathrm{d}z \right| = \left| \int_{0}^{\pi} f(R\mathrm{e}^{\mathrm{i}\theta}) R\mathrm{e}^{\mathrm{i}\theta} \mathrm{i} \mathrm{d}\theta \right| \leqslant \pi R M(R) \to 0 \, (R \to \infty).
\end{align*}
在\(\eqref{eq:::---8979807671283-2}\)式中令 \( R \to \infty \),即得公式\(\eqref{eq:::---8979807671283-1}\).
\end{proof}

\begin{corollary}\label{corollary:推论5.5.2}
设 \( P \) 和 \( Q \) 是两个既约多项式,\( Q \) 没有实的零点,且 \( \deg Q - \deg P \geqslant 2 \),那么
\[
\int_{-\infty}^{\infty} \frac{P(x)}{Q(x)} \mathrm{d}x = 2\pi \mathrm{i} \sum_{k = 1}^{n} \mathrm{Res}\left( \frac{P(z)}{Q(z)}, a_k \right),
\]
这里,\( a_k (k = 1, \cdots, n) \) 为 \( Q \) 在上半平面中的全部零点,\( \deg P, \deg Q \) 分别为 \( P \) 和 \( Q \) 的次数.
\end{corollary}
\begin{proof}
令 \( f(z) = \frac{P(z)}{Q(z)} \),则 \( f \) 满足\refthe{theorem:定理5.5.1}的条件,由\refthe{theorem:定理5.5.1}即得本推论.
\end{proof}

\begin{example}
计算积分
\[
\int_{-\infty}^{\infty} \frac{x^2 - x + 2}{x^4 + 10x^2 + 9} \mathrm{d}x.
\]
\end{example}
\begin{solution}
令 \( f(z) = \frac{z^2 - z + 2}{z^4 + 10z^2 + 9} \),它满足\refcor{corollary:推论5.5.2}的条件.容易看出,分母 \( Q(z) = z^4 + 10z^2 + 9 \) 有4个零点 \( \pm \mathrm{i} \) 和 \( \pm 3\mathrm{i} \),但在上半平面中的零点只有 \( a_1 = \mathrm{i} \) 和 \( a_2 = 3\mathrm{i} \) 两个.容易算得
\[
\mathrm{Res}(f, \mathrm{i}) = \frac{-1 - \mathrm{i}}{16},
\quad
\mathrm{Res}(f, 3\mathrm{i}) = \frac{3 - 7\mathrm{i}}{48},
\]
故得
\[
\int_{-\infty}^{\infty} \frac{x^2 - x + 2}{x^4 + 10x^2 + 9} \mathrm{d}x = \frac{5}{12}\pi.
\]
\end{solution}

\begin{example}
计算积分
\[
\int_{-\infty}^{\infty} \frac{\mathrm{d}x}{(1 + x^2)^{n + 1}}.
\]
\end{example}
\begin{solution}
令 \( f(z) = \frac{1}{(1 + z^2)^{n + 1}} \),它显然满足\refcor{corollary:推论5.5.2}的条件,且在上半平面中只有一个 \( n + 1 \) 阶极点 \( z = \mathrm{i} \).应\refpro{proposition:命题5.4.2},通过直接计算得
\[
\mathrm{Res}(f, \mathrm{i}) = \frac{1}{2\mathrm{i}} \frac{(2n)!}{2^{2n} (n!)^2},
\]
于是得
\[
\int_{-\infty}^{\infty} \frac{\mathrm{d}x}{(1 + x^2)^{n + 1}} = \frac{(2n)! \pi}{2^{2n} (n!)^2}.
\]
\end{solution}














\end{document}
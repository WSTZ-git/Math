\documentclass[../../main.tex]{subfiles}
\graphicspath{{\subfix{../../image/}}} % 指定图片目录,后续可以直接使用图片文件名。

% 例如:
% \begin{figure}[H]
% \centering
% \includegraphics[scale=0.4]{图.png}
% \caption{}
% \label{figure:图}
% \end{figure}
% 注意:上述\label{}一定要放在\caption{}之后,否则引用图片序号会只会显示??.

\begin{document}

\section{利用留数定理计算定积分}

\subsection{$\int_{-\infty}^{\infty}{f\left( x \right) \mathrm{d}x}$型积分}

\begin{theorem}\label{theorem:定理5.5.1}
设 \( f \) 在上半平面 \( \{ z: \mathrm{Im}z > 0 \} \) 中除去 \( a_1, \cdots, a_n \) 外是全纯的,在 \( \{ z: \mathrm{Im}z \geqslant 0 \} \) 中除去 \( a_1, \cdots, a_n \) 外是连续的.如果 \( \lim_{z \to \infty} z f(z) = 0 \),那么
\begin{align}
\int_{-\infty}^{\infty} f(x) \mathrm{d}x = 2\pi \mathrm{i} \sum_{k = 1}^{n} \mathrm{Res}(f, a_k). \label{eq:::---8979807671283-1}
\end{align}
\end{theorem}
\begin{figure}[H]
\centering
\includegraphics[scale=0.4]{图5.2.png}
\caption{}
\label{figure:图5.2}
\end{figure}
\begin{proof}
\reffig{figure:图5.2}所示,取充分大的 \( R \),使得 \( a_1, \cdots, a_n \) 包含在半圆盘 \( \{ z: |z| < R, \mathrm{Im}z > 0 \} \) 中,记 \( \gamma_R = \{ z: z = R\mathrm{e}^{\mathrm{i}\theta}, 0 \leqslant \theta \leqslant \pi \} \),由\hyperref[theorem:留数定理(残数定理)-定理5.4.9]{留数定理}得
\begin{align}
\int_{-R}^{R} f(x) \mathrm{d}x + \int_{\gamma_R} f(z) \mathrm{d}z = 2\pi \mathrm{i} \sum_{k = 1}^{n} \mathrm{Res}(f, a_k).\label{eq:::---8979807671283-2}
\end{align}
记 \( M(R) = \max \{ |f(z)| : z \in \gamma_R \} \),由假定,\( \lim_{R \to \infty} R M(R) = 0 \),因而
\begin{align*}
\left| \int_{\gamma_R} f(z) \mathrm{d}z \right| = \left| \int_{0}^{\pi} f(R\mathrm{e}^{\mathrm{i}\theta}) R\mathrm{e}^{\mathrm{i}\theta} \mathrm{i} \mathrm{d}\theta \right| \leqslant \pi R M(R) \to 0 \, (R \to \infty).
\end{align*}
在\(\eqref{eq:::---8979807671283-2}\)式中令 \( R \to \infty \),即得公式\(\eqref{eq:::---8979807671283-1}\).
\end{proof}

\begin{corollary}\label{corollary:推论5.5.2}
设 \( P \) 和 \( Q \) 是两个既约多项式,\( Q \) 没有实的零点,且 \( \deg Q - \deg P \geqslant 2 \),那么
\[
\int_{-\infty}^{\infty} \frac{P(x)}{Q(x)} \mathrm{d}x = 2\pi \mathrm{i} \sum_{k = 1}^{n} \mathrm{Res}\left( \frac{P(z)}{Q(z)}, a_k \right),
\]
这里,\( a_k (k = 1, \cdots, n) \) 为 \( Q \) 在上半平面中的全部零点,\( \deg P, \deg Q \) 分别为 \( P \) 和 \( Q \) 的次数.
\end{corollary}
\begin{proof}
令 \( f(z) = \frac{P(z)}{Q(z)} \),则 \( f \) 满足\refthe{theorem:定理5.5.1}的条件,由\refthe{theorem:定理5.5.1}即得本推论.
\end{proof}

\begin{example}
计算积分
\[
\int_{-\infty}^{\infty} \frac{x^2 - x + 2}{x^4 + 10x^2 + 9} \mathrm{d}x.
\]
\end{example}
\begin{solution}
令 \( f(z) = \frac{z^2 - z + 2}{z^4 + 10z^2 + 9} \),它满足\refcor{corollary:推论5.5.2}的条件.容易看出,分母 \( Q(z) = z^4 + 10z^2 + 9 \) 有4个零点 \( \pm \mathrm{i} \) 和 \( \pm 3\mathrm{i} \),但在上半平面中的零点只有 \( a_1 = \mathrm{i} \) 和 \( a_2 = 3\mathrm{i} \) 两个.容易算得
\[
\mathrm{Res}(f, \mathrm{i}) = \frac{-1 - \mathrm{i}}{16},
\quad
\mathrm{Res}(f, 3\mathrm{i}) = \frac{3 - 7\mathrm{i}}{48},
\]
故得
\[
\int_{-\infty}^{\infty} \frac{x^2 - x + 2}{x^4 + 10x^2 + 9} \mathrm{d}x = \frac{5}{12}\pi.
\]
\end{solution}

\begin{example}
计算积分
\[
\int_{-\infty}^{\infty} \frac{\mathrm{d}x}{(1 + x^2)^{n + 1}}.
\]
\end{example}
\begin{solution}
令 \( f(z) = \frac{1}{(1 + z^2)^{n + 1}} \),它显然满足\refcor{corollary:推论5.5.2}的条件,且在上半平面中只有一个 \( n + 1 \) 阶极点 \( z = \mathrm{i} \).应\refpro{proposition:命题5.4.2},通过直接计算得
\[
\mathrm{Res}(f, \mathrm{i}) = \frac{1}{2\mathrm{i}} \frac{(2n)!}{2^{2n} (n!)^2},
\]
于是得
\[
\int_{-\infty}^{\infty} \frac{\mathrm{d}x}{(1 + x^2)^{n + 1}} = \frac{(2n)! \pi}{2^{2n} (n!)^2}.
\]
\end{solution}

\begin{lemma}[Jordan引理]\label{lemma:Jordan引理}
设$f$在$\{ z: R_0 \leqslant |z| < \infty, \mathrm{Im} z \geqslant 0 \}$上连续,且$\lim_{\substack{z \to \infty \\ \mathrm{Im} z \geqslant 0}} f(z) = 0$,则对任意$\alpha > 0$,有
\[
\lim_{R \to \infty} \int_{\gamma_R} \mathrm{e}^{\mathrm{i}\alpha z} f(z) \mathrm{d}z = 0,
\]
这里,$\gamma_R = \{ z: z = R\mathrm{e}^{\mathrm{i}\theta}, 0 \leqslant \theta \leqslant \pi, R \geqslant R_0 \}$。
\end{lemma}
\begin{remark}
在计算$\int_{-\infty}^{\infty} \mathrm{e}^{\mathrm{i}\alpha x} f(x) \mathrm{d}x$这种类型的积分时,需要应用Jordan引理.
\end{remark}
\begin{proof}
记$M(R) = \max\{ |f(z)| : z \in \gamma_R \}$,则由假定,$M(R) \to 0$($R \to \infty$)。因为
\[
\int_{\gamma_R} \mathrm{e}^{\mathrm{i}\alpha z} f(z) \mathrm{d}z = \int_{0}^{\pi} \mathrm{e}^{\mathrm{i}\alpha R\cos\theta} \mathrm{e}^{-\alpha R\sin\theta} f(R\mathrm{e}^{\mathrm{i}\theta}) R\mathrm{i}\mathrm{e}^{\mathrm{i}\theta} \mathrm{d}\theta,
\]
所以
\begin{align*}
\left| \int_{\gamma_R} \mathrm{e}^{\mathrm{i}\alpha z} f(z) \mathrm{d}z \right| &\leqslant R M(R) \int_{0}^{\pi} \mathrm{e}^{-\alpha R\sin\theta} \mathrm{d}\theta
= 2 R M(R) \int_{0}^{\frac{\pi}{2}} \mathrm{e}^{-\alpha R\sin\theta} \mathrm{d}\theta
\\
&\leqslant 2 R M(R) \int_{0}^{\frac{\pi}{2}} \mathrm{e}^{-\frac{2}{\pi}\alpha R\theta} \mathrm{d}\theta
= \frac{\pi}{2} M(R) (1 - \mathrm{e}^{-\alpha R})
\to 0 \ (R \to \infty).
\end{align*}
这里,我们已经利用了不等式
\[
\sin\theta \geqslant \frac{2}{\pi}\theta \left( 0 \leqslant \theta \leqslant \frac{\pi}{2} \right).
\]
\end{proof}

\begin{theorem}\label{theorem:定理5.5.6}
设$f$在上半平面$\{ z: \mathrm{Im} z > 0 \}$中除去$a_1, \cdots, a_n$外是全纯的,在$\{ z: \mathrm{Im} z \geqslant 0 \}$中除去$a_1, \cdots, a_n$外是连续的。如果$\lim_{\substack{z \to \infty}} f(z) = 0$,那么对任意$\alpha > 0$,有
\begin{align}
\int_{-\infty}^{\infty} \mathrm{e}^{\mathrm{i}\alpha x} f(x) \mathrm{d}x = 2\pi \mathrm{i} \sum_{k=1}^{n} \mathrm{Res}(\mathrm{e}^{\mathrm{i}\alpha z} f(z), a_k). \label{thm5.5.6_eq1}
\end{align}
\end{theorem}
\begin{proof}
取充分大的$R$,使得$a_1, \cdots, a_n$都包含在半圆盘$\{ z: |z| < R, \mathrm{Im} z > 0 \}$中。对函数
\[
F(z) = \mathrm{e}^{\mathrm{i}\alpha z} f(z)
\]
用\hyperref[theorem:留数定理(残数定理)-定理5.4.9]{留数定理},得
\begin{align}
\int_{-R}^{R} \mathrm{e}^{\mathrm{i}\alpha x} f(x) \mathrm{d}x + \int_{\gamma_R} \mathrm{e}^{\mathrm{i}\alpha z} f(z) \mathrm{d}z = 2\pi \mathrm{i} \sum_{k=1}^{n} \mathrm{Res}(\mathrm{e}^{\mathrm{i}\alpha z} f(z), a_k). \label{thm5.5.6_eq2}
\end{align}
根据\hyperref[lemma:Jordan引理]{Jordan引理},有
\[
\lim_{R \to \infty} \int_{\gamma_R} \mathrm{e}^{\mathrm{i}\alpha z} f(z) \mathrm{d}z = 0.
\]
在\eqref{thm5.5.6_eq2}式的两端让$R \to \infty$,即得公式\eqref{thm5.5.6_eq1}。
\end{proof}

\begin{corollary}\label{corollary:推论5.5.7}
设$f$在上半平面$\{ z: \mathrm{Im} z > 0 \}$中除去$a_1, \cdots, a_n$外是全纯的,在$\{ z: \mathrm{Im} z \geqslant 0 \}$中除去$a_1, \cdots, a_n$外是连续的。如果$\lim_{\substack{z \to \infty}} f(z) = 0$,那么对任意$\alpha > 0$,有
\[
\int_{-\infty}^{\infty} f(x) \cos\alpha x \mathrm{d}x = \mathrm{Re} \left\{ 2\pi \mathrm{i} \sum_{k=1}^{n} \mathrm{Res}(\mathrm{e}^{\mathrm{i}\alpha z} f(z), a_k) \right\},
\]
\[
\int_{-\infty}^{\infty} f(x) \sin\alpha x \mathrm{d}x = \mathrm{Im} \left\{ 2\pi \mathrm{i} \sum_{k=1}^{n} \mathrm{Res}(\mathrm{e}^{\mathrm{i}\alpha z} f(z), a_k) \right\}.
\]
\end{corollary}
\begin{proof}
注意到
\[
\mathrm{e}^{\mathrm{i}\alpha x} = \cos\alpha x + \mathrm{i}\sin\alpha x,
\]
在公式\eqref{thm5.5.6_eq1}的两端分别取实部和虚部,即得.
\end{proof}

\begin{example}
计算积分
\[
\int_{-\infty}^{\infty} \frac{\cos ax}{b^2 + x^2} \mathrm{d}x \ (a > 0, \ b > 0).
\]
\end{example}
\begin{solution}
令\( f(z) = \frac{1}{b^2 + z^2} \),它满足\refthe{theorem:定理5.5.6}的条件。因为\( \frac{\mathrm{e}^{\mathrm{i}az}}{b^2 + z^2} \)在上半平面中只有一个1阶极点\( z = bi \),且
\[
\mathrm{Res}\left( \frac{\mathrm{e}^{\mathrm{i}az}}{b^2 + z^2}, bi \right) = \frac{\mathrm{e}^{-ab}}{2bi},
\]
根据\refcor{corollary:推论5.5.7},即得
\[
\int_{-\infty}^{\infty} \frac{\cos ax}{b^2 + x^2} \mathrm{d}x = \frac{\pi}{b} \mathrm{e}^{-ab}.
\]
\end{solution}

\begin{lemma}\label{lemma:引理5.5.9}
设$f$在扇状域
\[
G = \{ z = a + \rho \mathrm{e}^{\mathrm{i}\theta} : 0 < \rho \leqslant \rho_0, \ \theta_0 \leqslant \theta \leqslant \theta_0 + \alpha \}
\]
上连续,如果\(\lim_{\substack{z \to a}} (z - a)f(z) = A\),那么
\[
\lim_{\rho \to 0} \int_{\gamma_{\rho}} f(z) \mathrm{d}z = \mathrm{i}A\alpha, \label{lem5.5.9_eq}
\]
这里,\(\gamma_{\rho} = \{ z = a + \rho \mathrm{e}^{\mathrm{i}\theta} : \theta_0 \leqslant \theta \leqslant \theta_0 + \alpha \}\),它的方向是沿着辐角增加的方向。
\end{lemma}
\begin{remark}
遇到$f$在实轴上有奇点的情况时,常要使用这个引理.
\end{remark}
\begin{proof}
令\( g(z) = (z - a)f(z) - A \),则\(\lim_{\substack{z \to a}} g(z) = 0\)。若记\( M_{\rho} = \sup\{ |g(z)| : z = a + \rho \mathrm{e}^{\mathrm{i}\theta}, \theta_0 \leqslant \theta \leqslant \theta_0 + \alpha \} \),则\(\lim_{\rho \to 0} M_{\rho} = 0\)。于是
\[
\left| \int_{\gamma_{\rho}} \frac{g(z)}{z - a} \mathrm{d}z \right| = \left| \int_{\theta_0}^{\theta_0 + \alpha} \frac{g(a + \rho \mathrm{e}^{\mathrm{i}\theta})}{\rho \mathrm{e}^{\mathrm{i}\theta}} \rho \mathrm{i}\mathrm{e}^{\mathrm{i}\theta} \mathrm{d}\theta \right|
\leqslant M_{\rho}\alpha
\to 0 \ (\rho \to 0).
\]
由此即得
\begin{align*}
\int_{\gamma _{\rho}}{f(z)\mathrm{d}z}&=\int_{\gamma _{\rho}}{\frac{A}{z-a}\mathrm{d}z}+\int_{\gamma _{\rho}}{\frac{g(z)}{z-a}\mathrm{d}z}=\int_{\theta _0}^{\theta _0+\alpha}{\frac{A}{\rho e^{\mathrm{i}\theta}}\rho \mathrm{i}e^{\mathrm{i}\theta}\mathrm{d}\theta}+\int_{\gamma _{\rho}}{\frac{g(z)}{z-a}\mathrm{d}z}
\\
&=\mathrm{i}A\alpha +\int_{\gamma _{\rho}}{\frac{g(z)}{z-a}\mathrm{d}z}\rightarrow \mathrm{i}A\alpha \,\,(\rho \rightarrow 0).
\end{align*}
\end{proof}

\begin{example}
计算积分
\[
\int_{0}^{\infty} \frac{\sin x}{x} \mathrm{d}x.
\]
\end{example}
\begin{figure}[H]
\centering
\includegraphics[scale=0.4]{图5.3.png}
\caption{}
\label{figure:图5.3}
\end{figure}
\begin{solution}
取函数\( f(z) = \frac{\mathrm{e}^{\mathrm{i}z}}{z} \),取围道如\reffig{figure:图5.3}所示,它由线段\([-R, -\rho]\),\([\rho, R]\)和半圆周\(\gamma_{\rho}\),\(\gamma_R\)组成。在此围道围成的域中,\( f \)是全纯的,因而由\hyperref[theorem:Cauchy-Goursat定理(Cauchy积分定理)]{Cauchy积分定理}得
\begin{align}
\int_{-R}^{-\rho} \frac{\mathrm{e}^{\mathrm{i}x}}{x} \mathrm{d}x + \int_{\gamma_{\rho}^{-}} \frac{\mathrm{e}^{\mathrm{i}z}}{z} \mathrm{d}z + \int_{\rho}^{R} \frac{\mathrm{e}^{\mathrm{i}x}}{x} \mathrm{d}x + \int_{\gamma_R} \frac{\mathrm{e}^{\mathrm{i}z}}{z} \mathrm{d}z = 0. \label{ex5.5.10_eq1}
\end{align}
由\reflem{lemma:Jordan引理}知道
\[
\lim_{R \to \infty} \int_{\gamma_R} \frac{\mathrm{e}^{\mathrm{i}z}}{z} \mathrm{d}z = 0.
\]
由\reflem{lemma:引理5.5.9}得
\[
\lim_{\rho \to 0} \int_{\gamma_{\rho}^{-}} \frac{\mathrm{e}^{\mathrm{i}z}}{z} \mathrm{d}z = -\mathrm{i}\pi.
\]
在\(\eqref{ex5.5.10_eq1}\)式中令\(\rho \to 0\),\( R \to \infty \),于是得
\[
\int_{-\infty}^{0} \frac{\mathrm{e}^{\mathrm{i}x}}{x} \mathrm{d}x + \int_{0}^{\infty} \frac{\mathrm{e}^{\mathrm{i}x}}{x} \mathrm{d}x = \mathrm{i}\pi,
\]
即
\[
\int_{-\infty}^{\infty} \frac{\mathrm{e}^{\mathrm{i}x}}{x} \mathrm{d}x = \mathrm{i}\pi.
\]
两边取虚部,得
\[
\int_{-\infty}^{\infty} \frac{\sin x}{x} \mathrm{d}x = \pi,
\]
因而
\[
\int_{0}^{\infty} \frac{\sin x}{x} \mathrm{d}x = \frac{1}{2} \int_{-\infty}^{\infty} \frac{\sin x}{x} \mathrm{d}x
= \frac{\pi}{2}.
\]
\end{solution}

\subsection{$\int_0^{\infty}{f\left( x \right) \mathrm{d}x}$型积分}

用\hyperref[theorem:留数定理(残数定理)-定理5.4.9]{留数定理}计算\(\int_{0}^{\infty} f(x) \mathrm{d}x\)这种类型的积分,往往要借助于对数函数,不像计算\(\int_{-\infty}^{\infty} f(x) \mathrm{d}x\)型积分直接。

\begin{example}\label{example:例5.5.11}
计算积分
\[
\int_{0}^{\infty} \frac{x^{p - 1}}{(1 + x)^m} \mathrm{d}x,
\]
这里,\( m \)是正整数,\( p \)不是整数,\( 0 < p < m \)。
\end{example}
\begin{figure}[H]
\centering
\includegraphics[scale=0.4]{图5.4.png}
\caption{}
\label{figure:图5.4}
\end{figure}
\begin{solution}
取\( f(z) = \frac{z^{p - 1}}{(1 + z)^m} \),因为\( p \)不是整数,所以
\[
z^{p - 1} = \mathrm{e}^{(p - 1)\log z}
\]
是一个多值函数。在复平面上,取正实轴作割线得一域,\( z^{p - 1} \)在这个域中能分出单值的全纯分支。今取定在正实轴上沿取实值的那个全纯分支,即主支:\( z^{p - 1} = \mathrm{e}^{(p - 1)\log z} \)。让\( f(z) = \frac{\mathrm{e}^{(p - 1)\log z}}{(1 + z)^m} \)沿如下的闭曲线\( \Gamma \)积分:先沿正实轴的上沿从\( \rho \)到\( R \)(\( 0 < \rho < 1 < R < \infty \)),再按反时针方向,沿以原点为中心、\( R \)为半径的圆周\( \gamma_R \)回到出发处,再沿正实轴的下沿从\( R \)到\( \rho \),最后按顺时针方向沿以原点为中心、\( \rho \)为半径的圆周\( \gamma_{\rho} \)回到原来的出发处(\reffig{figure:图5.4})。在正实轴上沿,有
\[
f(z) = \frac{\mathrm{e}^{(p - 1)\log x}}{(1 + x)^m} = \frac{x^{p - 1}}{(1 + x)^m};
\]
在正实轴下沿,由于
\[
\log z = \log|z| + 2\pi \mathrm{i},
\]
所以
\[
\mathrm{e}^{(p - 1)\log z} = \mathrm{e}^{(p - 1)(\log x + 2\pi \mathrm{i})}
= x^{p - 1} \mathrm{e}^{(p - 1)2\pi \mathrm{i}}
= \mathrm{e}^{2p\pi \mathrm{i}} x^{p - 1},
\]
因而
\[
f(z) = \mathrm{e}^{2p\pi \mathrm{i}} \frac{x^{p - 1}}{(1 + x)^m}.
\]

显然,\( f \)在由\( \Gamma \)围成的域中只有一个\( m \)阶极点\( z = -1 \)。由\hyperref[theorem:留数定理(残数定理)-定理5.4.9]{留数定理},有
\begin{align}
\int_{\rho}^{R} \frac{x^{p - 1}}{(1 + x)^m} \mathrm{d}x + \int_{\gamma_R} \frac{z^{p - 1}}{(1 + z)^m} \mathrm{d}z + \mathrm{e}^{2p\pi \mathrm{i}} \int_{R}^{\rho} \frac{x^{p - 1}}{(1 + x)^m} \mathrm{d}x
+ \int_{\gamma_{\rho}} \frac{z^{p - 1}}{(1 + z)^m} \mathrm{d}z = 2\pi \mathrm{i} \mathrm{Res}\left( \frac{z^{p - 1}}{(1 + z)^m}, -1 \right). \label{ex5.5.11_eq1}
\end{align}
当\( z \in \gamma_R \)时,\( z = R\mathrm{e}^{\mathrm{i}\theta} \),\( \log z = \log R + \mathrm{i}\theta \),所以
\[
\frac{|z^{p - 1}|}{|1 + z|^m} = \frac{|\mathrm{e}^{(p - 1)\log z}|}{|1 + z|^m} \leqslant \frac{R^{p - 1}}{(R - 1)^m}.
\]
同样道理,当\( z \in \gamma_{\rho} \)时,有
\[
\frac{|z^{p - 1}|}{|1 + z|^m} \leqslant \frac{\rho^{p - 1}}{(1 - \rho)^m}.
\]
于是
\begin{gather*}
\left| \int_{\gamma_R} \frac{z^{p - 1}}{(1 + z)^m} \mathrm{d}z \right| \leqslant \frac{R^{p - 1}}{(R - 1)^m} 2\pi R
= 2\pi \frac{R^p}{(R - 1)^m}
\to 0 \ (R \to \infty),
\\
\left| \int_{\gamma_{\rho}} \frac{z^{p - 1}}{(1 + z)^m} \mathrm{d}z \right| \leqslant \frac{\rho^{p - 1}}{(1 - \rho)^m} 2\pi \rho
= 2\pi \frac{\rho^p}{(1 - \rho)^m}
\to 0 \ (\rho \to 0).
\end{gather*}
在\(\eqref{ex5.5.11_eq1}\)式中令\( \rho \to 0 \),\( R \to \infty \),即得
\[
(1 - \mathrm{e}^{2p\pi \mathrm{i}}) \int_{0}^{\infty} \frac{x^{p - 1}}{(1 + x)^m} \mathrm{d}x = 2\pi \mathrm{i} \mathrm{Res}\left( \frac{z^{p - 1}}{(1 + z)^m}, -1 \right).
\]
由\refpro{proposition:命题5.4.3},容易算出,当\( m = 1 \)时
\[
\mathrm{Res}\left( \frac{z^{p - 1}}{1 + z}, -1 \right) = \mathrm{e}^{(p - 1)\pi \mathrm{i}} = -\mathrm{e}^{p\pi \mathrm{i}};
\]
当\( m > 1 \)时
\[
\mathrm{Res}\left( \frac{z^{p - 1}}{(1 + z)^m}, -1 \right) = -\frac{1}{(m - 1)!}(1 - p)(2 - p)
\cdots (m - 1 - p) \mathrm{e}^{p\pi \mathrm{i}}.
\]
由此即得
\[
\int_{0}^{\infty} \frac{x^{p - 1}}{1 + x} \mathrm{d}x = \frac{\pi}{\sin p\pi} \ (0 < p < 1),
\]
\[
\int_{0}^{\infty} \frac{x^{p - 1}}{(1 + x)^m} \mathrm{d}x = \frac{\pi}{\sin p\pi} \frac{1}{(m - 1)!}(1 - p)(2 - p)
\cdots (m - 1 - p).
\]
\end{solution}
\begin{remark}
上面的方法可用来计算一般的积分
\[
\int_{0}^{\infty} f(x) x^{p - 1} \mathrm{d}x \ (0 < p < 1).
\]
\end{remark}

\begin{example}
计算积分
\[
\int_{0}^{\infty} \frac{\log x}{(1 + x^2)^2} \mathrm{d}x.
\]
\end{example}
\begin{solution}
{\color{blue}解法一:}取函数\( f(z) = \frac{\log^2 z}{(1 + z^2)^2} \),取围道如\reffig{figure:图5.4}所示。在正实轴的上沿,有
\[
f(z) = \frac{\log^2 x}{(1 + x^2)^2};
\]
在正实轴的下沿,由于\( \log z = \log x + 2\pi \mathrm{i} \),所以
\[
\log^2 z = (\log x + 2\pi \mathrm{i})^2
= \log^2 x + 4\pi \mathrm{i} \log x - 4\pi^2,
\]
因而
\[
f(z) = \frac{\log^2 x}{(1 + x^2)^2} + 4\pi \mathrm{i} \frac{\log x}{(1 + x^2)^2} - 4\pi^2 \frac{1}{(1 + x^2)^2}.
\]
\( f \)在\( \Gamma \)所围成的域中有两个2阶极点\( z = \pm \mathrm{i} \)。对\( f \)用\hyperref[theorem:留数定理(残数定理)-定理5.4.9]{留数定理},得
\begin{align}
&\int_{\rho}^{R} \frac{\log^2 x}{(1 + x^2)^2} \mathrm{d}x + \int_{\gamma_R} \frac{\log^2 z}{(1 + z^2)^2} \mathrm{d}z + \int_{R}^{\rho} \frac{\log^2 x}{(1 + x^2)^2} \mathrm{d}x
\nonumber\\
&+ 4\pi \mathrm{i} \int_{R}^{\rho} \frac{\log x}{(1 + x^2)^2} \mathrm{d}x - 4\pi^2 \int_{R}^{\rho} \frac{\mathrm{d}x}{(1 + x^2)^2} + \int_{\gamma_{\rho}} \frac{\log^2 z}{(1 + z^2)^2} \mathrm{d}z
\nonumber\\
&= 2\pi \mathrm{i} \left[ \mathrm{Res}\left( \frac{\log^2 z}{(1 + z^2)^2}, \mathrm{i} \right) + \mathrm{Res}\left( \frac{\log^2 z}{(1 + z^2)^2}, -\mathrm{i} \right) \right]. \label{ex5.5.12_eq1}
\end{align}
\(\eqref{ex5.5.12_eq1}\)式左端的第一个和第三个积分互相抵消了。\( \gamma_R \)和\( \gamma_{\rho} \)上两个积分的估计与\refexa{example:例5.5.11}一样:
\[
\left| \int_{\gamma_R} \frac{\log^2 z}{(1 + z^2)^2} \mathrm{d}z \right| = \left| \int_{0}^{2\pi} \frac{(\log R + \mathrm{i}\theta)^2}{(1 + R^2 \mathrm{e}^{2\mathrm{i}\theta})^2} R\mathrm{i}\mathrm{e}^{\mathrm{i}\theta} \mathrm{d}\theta \right|
\leqslant 2\pi R \frac{(\log R + 2\pi)^2}{(R^2 - 1)^2}
\to 0 \ (R \to \infty),
\]
\[
\left| \int_{\gamma_{\rho}} \frac{\log^2 z}{(1 + z^2)^2} \mathrm{d}z \right| = \left| \int_{0}^{2\pi} \frac{(\log \rho + \mathrm{i}\theta)^2}{(1 + \rho^2 \mathrm{e}^{2\mathrm{i}\theta})^2} \rho\mathrm{i}\mathrm{e}^{\mathrm{i}\theta} \mathrm{d}\theta \right|
\leqslant 2\pi \rho \frac{(\log \rho + 2\pi)^2}{(1 - \rho^2)^2}
\to 0 \ (\rho \to 0).
\]
直接计算留数,得
\[
\mathrm{Res}\left( \frac{\log^2 z}{(1 + z^2)^2}, \mathrm{i} \right) = \frac{-4\pi + \pi^2 \mathrm{i}}{16},
\quad
\mathrm{Res}\left( \frac{\log^2 z}{(1 + z^2)^2}, -\mathrm{i} \right) = \frac{12\pi - 9\pi^2 \mathrm{i}}{16}.
\]
在\(\eqref{ex5.5.12_eq1}\)式中令\( \rho \to 0 \),\( R \to \infty \),并取两端的虚部,即得
\[
\int_{0}^{\infty} \frac{\log x}{(1 + x^2)^2} \mathrm{d}x = -\frac{\pi}{4}.
\]

\begin{figure}[H]
\centering
\includegraphics[scale=0.4]{图5.5.png}
\caption{}
\label{figure:图5.5}
\end{figure}
{\color{blue}解法二:}
在计算过程中我们发现,如果取\( f(z) = \frac{\log z}{(1 + z^2)^2} \),则所需计算的积分将被抵消掉,这是取\( f(z) = \frac{\log^2 z}{(1 + z^2)^2} \)的原因。但若改变围道如\reffig{figure:图5.5}所示,那么取\( f(z) = \frac{\log z}{(1 + z^2)^2} \)也是可以的。这时,\( f \)在\( \Gamma \)围成的域中只有一个2阶极点\( z = \mathrm{i} \)。当\( z \in [-R, -\rho] \)时,\( \log z = \log|x| + \mathrm{i}\pi \)。对\( f \)在\( \Gamma \)上应用\hyperref[theorem:留数定理(残数定理)-定理5.4.9]{留数定理},可得
\begin{align}
&\int_{-R}^{-\rho}{\frac{\log |x|}{(1+x^2)^2}\mathrm{d}x}+\mathrm{i}\pi \int_{-R}^{-\rho}{\frac{\mathrm{d}x}{(1+x^2)^2}}+\int_{\gamma _{\rho}}{\frac{\log z}{(1+z^2)^2}\mathrm{d}z}\nonumber
\\
&\quad \quad+\int_{\rho}^R{\frac{\log x}{(1+x^2)^2}\mathrm{d}x}+\int_{\gamma _R}{\frac{\log z}{(1+z^2)^2}\mathrm{d}z}\nonumber
\\
&=2\pi \mathrm{iRes}\left( \frac{\log z}{(1+z^2)^2},\mathrm{i} \right) . \label{ex5.5.12_eq2}
\end{align}
与上面的做法一样,可证
\[
\lim_{R \to \infty} \int_{\gamma_R} \frac{\log z}{(1 + z^2)^2} \mathrm{d}z = 0,
\quad
\lim_{\rho \to 0} \int_{\gamma_{\rho}} \frac{\log z}{(1 + z^2)^2} \mathrm{d}z = 0,
\]
而
\[
\mathrm{Res}\left( \frac{\log z}{(1 + z^2)^2}, \mathrm{i} \right) = \frac{\pi}{8} + \frac{\mathrm{i}}{4}.
\]
在\(\eqref{ex5.5.12_eq2}\)式两端令\( \rho \to 0 \),\( R \to \infty \),得
\[
2\int_{0}^{\infty} \frac{\log x}{(1 + x^2)^2} \mathrm{d}x - \mathrm{i}\pi \int_{0}^{\infty} \frac{\mathrm{d}x}{(1 + x^2)^2} = 2\pi \mathrm{i} \left( \frac{\pi}{8} + \frac{\mathrm{i}}{4} \right),
\]
两边取实部,即得
\[
\int_{0}^{\infty} \frac{\log x}{(1 + x^2)^2} \mathrm{d}x = -\frac{\pi}{4}.
\]
与第一种方法所得的结果一样。
\end{solution}

\subsection{$\int_a^b{f\left( x \right) \mathrm{d}x}$型积分}

我们讨论两种重要类型的有穷限积分。一种是
\[
\int_{0}^{2\pi} R(\sin\theta, \cos\theta) \mathrm{d}\theta
\]
类型的积分,其中,\( R(X, Y) \)是两个变量\( X, Y \)的有理函数。这种类型的积分可以化为\( \int_{-\infty}^{\infty} f(x) \mathrm{d}x \)型积分来讨论。事实上,因为被积函数是周期为\( 2\pi \)的函数,所以
\[
\int_{0}^{2\pi} R(\sin\theta, \cos\theta) \mathrm{d}\theta = \int_{-\pi}^{\pi} R(\sin\theta, \cos\theta) \mathrm{d}\theta.
\]
作变换\( t = \tan \frac{\theta}{2} \),那么
\[
\sin\theta = \frac{2t}{1 + t^2},
\quad
\cos\theta = \frac{1 - t^2}{1 + t^2},
\quad
\mathrm{d}\theta = \frac{2dt}{1 + t^2},
\]
于是
\[
\int_{0}^{2\pi} R(\sin\theta, \cos\theta) \mathrm{d}\theta = 2 \int_{-\infty}^{\infty} R\left( \frac{2t}{1 + t^2}, \frac{1 - t^2}{1 + t^2} \right) \frac{1}{1 + t^2} dt.
\]
右端积分中的被积函数是\( t \)的有理函数,这是刚讨论过的积分。

计算这种积分的另外一种方法是把它化为单位圆周上的积分。设\( z = \mathrm{e}^{\mathrm{i}\theta} \),那么
\[
\cos\theta = \frac{1}{2} (\mathrm{e}^{\mathrm{i}\theta} + \mathrm{e}^{-\mathrm{i}\theta}) = \frac{1}{2} \left( z + \frac{1}{z} \right),
\]
\[
\sin\theta = \frac{1}{2\mathrm{i}} (\mathrm{e}^{\mathrm{i}\theta} - \mathrm{e}^{-\mathrm{i}\theta}) = \frac{1}{2\mathrm{i}} \left( z - \frac{1}{z} \right),
\]
\[
\mathrm{d}\theta = \frac{1}{\mathrm{i}z} \mathrm{d}z,
\]
于是
\[
\int_{0}^{2\pi} R(\sin\theta, \cos\theta) \mathrm{d}\theta = \int_{|z| = 1} R\left( \frac{1}{2\mathrm{i}} \left( z - \frac{1}{z} \right), \frac{1}{2} \left( z + \frac{1}{z} \right) \right) \frac{1}{\mathrm{i}z} \mathrm{d}z.
\]
右端积分中的被积函数是\( z \)的有理函数,积分在单位圆周上进行,因而可用残数定理来计算。

\begin{example}
计算积分
\[
\int_{0}^{2\pi} \frac{\mathrm{d}\theta}{3 + \cos\theta + 2\sin\theta}.
\]
\end{example}
\begin{solution}
令\( z = \mathrm{e}^{\mathrm{i}\theta} \),则
\[
\int_{0}^{2\pi} \frac{\mathrm{d}\theta}{3 + \cos\theta + 2\sin\theta} = 2 \int_{|z| = 1} \frac{\mathrm{d}z}{(\mathrm{i} + 2)z^2 + 6\mathrm{i}z + \mathrm{i} - 2}.
\]
右端积分中的被积函数有两个1阶极点
\[
a_1 = -\frac{1 + 2\mathrm{i}}{5}, \quad a_2 = -1 - 2\mathrm{i},
\]
但只有\( a_1 \)在单位圆内,被积函数在\( a_1 \)处的留数为\( \frac{1}{4\mathrm{i}} \),因而
\[
\int_{0}^{2\pi} \frac{\mathrm{d}\theta}{3 + \cos\theta + 2\sin\theta} = 4\pi \mathrm{i} \cdot \frac{1}{4\mathrm{i}}
= \pi.
\]
\end{solution}

用类似的方法可以计算积分
\[
\int_{0}^{2\pi} R(\sin n\theta, \cos n\theta) \mathrm{d}\theta,
\]
这是因为
\[
\int_{0}^{2\pi} R(\sin n\theta, \cos n\theta) \mathrm{d}\theta = \int_{|z| = 1} R\left( \frac{1}{2\mathrm{i}} \left( z^n - \frac{1}{z^n} \right), \frac{1}{2} \left( z^n + \frac{1}{z^n} \right) \right) \frac{1}{\mathrm{i}z} \mathrm{d}z,
\]
这里,\( n \)是整数。

如果要计算积分
\[
\int_{0}^{2\pi} R(\sin\theta, \cos\theta) \cos n\theta \mathrm{d}\theta
\]
或
\[
\int_{0}^{2\pi} R(\sin\theta, \cos\theta) \sin n\theta \mathrm{d}\theta,
\]
则先利用公式
\[
\int_{0}^{2\pi} R(\sin\theta, \cos\theta) \mathrm{e}^{\mathrm{i}n\theta} \mathrm{d}\theta = \int_{|z| = 1} R\left( \frac{1}{2\mathrm{i}} \left( z - \frac{1}{z} \right), \frac{1}{2} \left( z + \frac{1}{z} \right) \right) \frac{z^{n - 1}}{\mathrm{i}} \mathrm{d}z \label{formula10}
\]
算出左端的积分,然后取实部或虚部,即得上述两个积分。

另一种重要类型的有穷限积分是
\[
\int_{a}^{b} (x - a)^r (b - x)^s f(x) \mathrm{d}x,
\]
这里,\( -1 < r, s < 1 \),且\( r + s = -1, 0 \)或\( 1 \)。

\begin{lemma}\label{lemma:引理5.5.15}
设\( f \)在
\[
G = \{ z = \rho \mathrm{e}^{\mathrm{i}\theta} : \rho \geqslant R_0, \theta_0 \leqslant \theta \leqslant \theta_0 + \alpha \}
\]
中连续,如果\(\lim_{z \to \infty} z f(z) = A\),那么
\[
\lim_{\rho \to \infty} \int_{\gamma_{\rho}} f(z) \mathrm{d}z = \mathrm{i}A\alpha,
\]
这里,\(\gamma_{\rho} = \{ z = \rho \mathrm{e}^{\mathrm{i}\theta} : \theta_0 \leqslant \theta \leqslant \theta_0 + \alpha \}\),它的方向是沿着辐角增加的方向。
\end{lemma}
\begin{proof}
证明的方法与\reflem{lemma:引理5.5.9}完全一样.
\end{proof}

\begin{theorem}\label{theorem:定理5.5.14}
设\( f \)在\( \mathbb{C} \)中除去\( a_1, \cdots, a_n \)外是全纯的,\( a_1, \cdots, a_n \)都不在区间\([a, b]\)上;设\( -1 < r, s < 1 \),\( s \neq 0 \),且\( r + s \)是整数。如果
\[
\lim_{z \to \infty} z^{r + s + 1} f(z) = A \neq \infty,
\]
那么
\begin{align}
\int_{a}^{b} (x - a)^r (b - x)^s f(x) \mathrm{d}x
= -\frac{A\pi}{\sin s\pi} + \frac{\pi}{\mathrm{e}^{-s\pi \mathrm{i}} \sin s\pi} \sum_{k = 1}^{n} \mathrm{Res}(F, a_k), \label{thm5.5.14_eq}
\end{align}
这里,\( F(z) = (z - a)^r (b - z)^s f(z) \)。
\end{theorem}
\begin{figure}[H]
\centering
\includegraphics[scale=0.4]{图5.6.png}
\caption{}
\label{figure:图5.6}
\end{figure}
\begin{proof}
联接\( a \)和\( b \),我们证明在线段\([a, b]\)外部,\( F(z) = (z - a)^r (b - z)^s f(z) \)能分出单值全纯的分支。

事实上,记\( z - a = \rho_1 \mathrm{e}^{\mathrm{i}\theta_1} \),\( z - b = \rho_2 \mathrm{e}^{\mathrm{i}\theta_2} \),当\( z \)沿线段\([a, b]\)外部的任意简单闭曲线转一圈时,\( z - a \)和\( z - b \)的辐角都要增加\( 2\pi \),\( (z - a)^r (z - b)^s \)的值将由原来的\( \rho_1^r \rho_2^s \mathrm{e}^{\mathrm{i}(r\theta_1 + s\theta_2)} \)变为
\[
\rho_1^r \rho_2^s \mathrm{e}^{\mathrm{i}(r\theta_1 + s\theta_2) + 2\pi(r + s)\mathrm{i}} = \rho_1^r \rho_2^s \mathrm{e}^{\mathrm{i}(r\theta_1 + s\theta_2)},
\]
等式成立是因为\( r + s \)是整数,这就是说\( F(z) \)的值不变。

现取定在\([a, b]\)上岸
\[
\arg(z - a) = 0, \arg(b - z) = 0
\]
的一支来讨论。取\( R \)充分大,\( \varepsilon \)充分小,使得由圆周\( \Gamma = \{ z : |z| = R \} \)的内部以及圆周\( \gamma_1 = \{ z : |z - a| = \varepsilon \} \)和圆周\( \gamma_2 = \{ z : |z - b| = \varepsilon \} \)的外部所构成的域\( D \)包含\( f \)的全部奇点\( a_1, \cdots, a_n \)(见\reffig{figure:图5.6})。在域\( D \)上对函数\( F \)用\hyperref[theorem:留数定理(残数定理)-定理5.4.9]{留数定理},得
\begin{align}
\int_{\Gamma} F(z) \mathrm{d}z + \int_{\gamma_1} F(z) \mathrm{d}z + \int_{l_1} F(z) \mathrm{d}z + \int_{\gamma_2} F(z) \mathrm{d}z + \int_{l_2} F(z) \mathrm{d}z
= 2\pi \mathrm{i} \sum_{k = 1}^{n} \mathrm{Res}(F, a_k), \label{thm5.5.14_proof_eq1}
\end{align}
这里,\( l_1, l_2 \)分别是\([a, b]\)上、下岸上的一段。当\( z \in l_1 \)时,\(\arg(z - a) = 0, \arg(b - z) = 0\),所以
\begin{gather*}
(z - a)^r = \mathrm{e}^{r \log(z - a)} = \mathrm{e}^{r \log|z - a|}
= \mathrm{e}^{r \log(x - a)} = (x - a)^r,
\\
(b - z)^s = \mathrm{e}^{s \log(b - z)} = \mathrm{e}^{s \log|b - z|}
= \mathrm{e}^{s \log(b - x)} = (b - x)^s.
\end{gather*}
当\( z \in l_2 \)时,\(\arg(z - a) = 0, \arg(b - z) = -2\pi\),所以,
\[
(z - a)^r = (x - a)^r,
\]
\[
(b - z)^s = \mathrm{e}^{s(\log(b - x) + \mathrm{i} \arg(b - z))}
= \mathrm{e}^{-2s\pi \mathrm{i}} (b - x)^s.
\]
于是,\(\eqref{thm5.5.14_proof_eq1}\)式可写为
\[
\int_{\Gamma} F(z) \mathrm{d}z + \int_{\gamma_1} F(z) \mathrm{d}z + \int_{\gamma_2} F(z) \mathrm{d}z
+ (1 - \mathrm{e}^{-2s\pi \mathrm{i}}) \int_{a + \varepsilon}^{b - \varepsilon} (x - a)^r (b - x)^s f(x) \mathrm{d}x
= 2\pi \mathrm{i} \sum_{k = 1}^{n} \mathrm{Res}(F, a_k). \label{thm5.5.14_proof_eq2}
\]
因为\(-1\)的辐角取\(\pi\),所以
\[
\lim_{z \to \infty} z F(z) = \lim_{z \to \infty} z (z - a)^r (b - z)^s f(z)
= \mathrm{e}^{-s\pi \mathrm{i}} \lim_{z \to \infty} z^{r + s + 1} f(z)
= \mathrm{e}^{-s\pi \mathrm{i}} A,
\]
故由\reflem{lemma:引理5.5.15}得
\[
\lim_{R \to \infty} \int_{\Gamma} F(z) \mathrm{d}z = 2\pi \mathrm{i} \mathrm{e}^{-s\pi \mathrm{i}} A.
\]
由于\( r + 1 > 0, s + 1 > 0 \),所以
\[
\lim_{z \to a} (z - a) F(z) = \lim_{z \to a} (z - a)^{r + 1} (b - z)^s f(z)
= 0,
\]
\[
\lim_{z \to b} (b - z) F(z) = \lim_{z \to b} (z - a)^r (b - z)^{s + 1} f(z)
= 0,
\]
故由\reflem{lemma:引理5.5.9}得
\[
\lim_{\varepsilon \to 0} \int_{\gamma_1} F(z) \mathrm{d}z = 0,
\quad
\lim_{\varepsilon \to 0} \int_{\gamma_2} F(z) \mathrm{d}z = 0.
\]
在\(\eqref{thm5.5.14_proof_eq2}\)式中令\( R \to \infty \),\( \varepsilon \to 0 \),即得
\[
\int_{a}^{b} (x - a)^r (b - x)^s f(x) \mathrm{d}x
= -\frac{2\pi \mathrm{i} \mathrm{e}^{-s\pi \mathrm{i}} A}{1 - \mathrm{e}^{-2s\pi \mathrm{i}}} + \frac{2\pi \mathrm{i}}{1 - \mathrm{e}^{-2s\pi \mathrm{i}}} \sum_{k = 1}^{n} \mathrm{Res}(F, a_k)
= -\frac{\pi A}{\sin s\pi} + \frac{\pi}{\mathrm{e}^{-s\pi \mathrm{i}} \sin s\pi} \sum_{k = 1}^{n} \mathrm{Res}(F, a_k).
\]
这就是要证明的公式\(\eqref{thm5.5.14_eq}\)。
\end{proof}

\begin{example}
计算积分
\[
\int_{-1}^{1} \frac{\mathrm{d}x}{\sqrt[3]{(1 + x)^2 (1 - x)}}.
\]
\end{example}
\begin{solution}
题中,\( r = -\frac{2}{3}, s = -\frac{1}{3}, r + s = -1 \),是一个整数,\( f(z) \equiv 1 \),所以
\[
\lim_{z \to \infty} z^{r + s + 1} f(z) = 1.
\]
由公式\(\eqref{thm5.5.14_eq}\)即得
\[
\int_{-1}^{1} \frac{\mathrm{d}x}{\sqrt[3]{(1 + x)^2 (1 - x)}} = \frac{2}{\sqrt{3}} \pi.
\]
\end{solution}

\begin{example}
计算积分
\[
\int_{0}^{1} \frac{\sqrt[3]{x^2 (1 - x)}}{(1 + x)^3} \mathrm{d}x.
\]
\end{example}
\begin{solution}
题中,\( r = \frac{2}{3}, s = \frac{1}{3}, r + s = 1 \),\( f(z) = \frac{1}{(1 + z)^3} \),因而
\[
\lim_{z \to \infty} z^{r + s + 1} f(z) = \lim_{z \to \infty} \frac{z^2}{(1 + z)^3} = 0.
\]
\( f \)在全平面上只有一个3阶极点\( z = -1 \),于是由公式\(\eqref{thm5.5.14_eq}\)即得
\begin{align}
\int_{0}^{1} \frac{\sqrt[3]{x^2 (1 - x)}}{(1 + x)^3} \mathrm{d}x = \frac{\pi}{\sin \frac{\pi}{3}} \mathrm{e}^{\frac{\pi}{3} \mathrm{i}} \mathrm{Res}\left( \frac{z^{\frac{2}{3}} (1 - z)^{\frac{1}{3}}}{(1 + z)^3}, -1 \right). \label{ex5.5.17_eq1}
\end{align}
根据\refpro{proposition:命题5.4.2},有
\begin{align}
\mathrm{Res}\left( \frac{z^{\frac{2}{3}} (1 - z)^{\frac{1}{3}}}{(1 + z)^3}, -1 \right) = \frac{1}{2} \lim_{z \to -1} \frac{\mathrm{d}^2}{\mathrm{d}z^2} \left\{ z^{\frac{2}{3}} (1 - z)^{\frac{1}{3}} \right\}. \label{ex5.5.17_eq2}
\end{align}
易知
\[
\frac{\mathrm{d}^2}{\mathrm{d}z^2} \left\{ z^{\frac{2}{3}} (1 - z)^{\frac{1}{3}} \right\} = -\frac{2}{9} z^{-\frac{4}{3}} (1 - z)^{\frac{1}{3}} - \frac{4}{9} z^{-\frac{1}{3}} (1 - z)^{-\frac{2}{3}}
- \frac{2}{9} (1 - z)^{-\frac{5}{3}} z^{\frac{2}{3}},
\]
为了计算它在\( z = -1 \)处的值,注意当\( z = -1 \)时,\(\arg z = \pi\),\(\arg(1 - z) = 0\),于是
\[
\lim_{z \to -1} \frac{\mathrm{d}^2}{\mathrm{d}z^2} \left\{ z^{\frac{2}{3}} (1 - z)^{\frac{1}{3}} \right\} = -\frac{2}{9} \mathrm{e}^{-\frac{4}{3} \pi \mathrm{i}} \sqrt[3]{2} - \frac{4}{9} \mathrm{e}^{-\frac{\pi}{3} \mathrm{i}} \frac{1}{\sqrt[3]{4}}
- \frac{2}{9} \mathrm{e}^{\frac{2\pi}{3} \mathrm{i}} 2^{-\frac{5}{3}}.
\]
代入\(\eqref{ex5.5.17_eq2}\)式后再代入\(\eqref{ex5.5.17_eq1}\)式,即得
\[
\int_{0}^{1} \frac{\sqrt[3]{x^2 (1 - x)}}{(1 + x)^3} \mathrm{d}x = \frac{\sqrt[3]{2} \pi}{18 \sqrt{3}}.
\]
\end{solution}


\subsection{两个特殊的积分}

上面只是大致归纳了一下用残数定理计算积分的类型,但它适用的范围还是相当有限的。这里介绍的两个积分便不能用第2小节中的方法来计算。

(1) Fresnel积分\(\int_{0}^{\infty} \cos x^2 \mathrm{d}x\)和\(\int_{0}^{\infty} \sin x^2 \mathrm{d}x\)
\begin{figure}[H]
\centering
\includegraphics[scale=0.3]{图5.7.png}
\caption{}
\label{figure:图5.7}
\end{figure}
取函数\( f(z) = \mathrm{e}^{\mathrm{i}z^2} \),取围道如\reffig{figure:图5.7}所示。因为\( f \)是整函数,由\hyperref[theorem:Cauchy-Goursat定理(Cauchy积分定理)]{Cauchy积分定理},有
\begin{align}
\int_{0}^{R} \mathrm{e}^{\mathrm{i}x^2} \mathrm{d}x + \int_{\gamma_R} \mathrm{e}^{\mathrm{i}z^2} \mathrm{d}z + \int_{\gamma_2} \mathrm{e}^{\mathrm{i}z^2} \mathrm{d}z = 0. \label{fresnel_eq1}
\end{align}
当\( z \in \gamma_R \)时,\( z = R\mathrm{e}^{\mathrm{i}\theta}, 0 \leqslant \theta \leqslant \frac{\pi}{4} \),所以
\[
|\mathrm{e}^{\mathrm{i}z^2}| = \mathrm{e}^{-R^2 \sin 2\theta} \leqslant \mathrm{e}^{-\frac{4}{\pi} R^2 \theta}, 0 \leqslant \theta \leqslant \frac{\pi}{4}.
\]
于是,当\( R \to \infty \)时,有
\[
\left| \int_{\gamma_R} \mathrm{e}^{\mathrm{i}z^2} \mathrm{d}z \right| \leqslant \int_{0}^{\frac{\pi}{4}} \mathrm{e}^{-\frac{4}{\pi} R^2 \theta} R \mathrm{d}\theta
= \frac{\pi}{4R} (1 - \mathrm{e}^{-R^2})
\to 0.
\]
当\( z \in \gamma_2 \)时,\( z = r\mathrm{e}^{\mathrm{i}\frac{\pi}{4}}, 0 \leqslant r \leqslant R \),所以
\[
\int_{\gamma_2} \mathrm{e}^{\mathrm{i}z^2} \mathrm{d}z = -\mathrm{e}^{\mathrm{i}\frac{\pi}{4}} \int_{0}^{R} \mathrm{e}^{-r^2} \mathrm{d}r.
\]
在\(\eqref{fresnel_eq1}\)式中令\( R \to \infty \),即得
\begin{align}
\int_{0}^{\infty} \mathrm{e}^{\mathrm{i}x^2} \mathrm{d}x = \mathrm{e}^{\mathrm{i}\frac{\pi}{4}} \int_{0}^{\infty} \mathrm{e}^{-r^2} \mathrm{d}r
= \frac{\sqrt{\pi}}{2} \mathrm{e}^{\mathrm{i}\frac{\pi}{4}}. \label{fresnel_eq2}
\end{align}
这里,我们已经利用了已知的概率积分
\[
\int_{0}^{\infty} \mathrm{e}^{-r^2} \mathrm{d}r = \frac{\sqrt{\pi}}{2}.
\]
在\(\eqref{fresnel_eq2}\)式两边分别取实部和虚部,即得
\[
\int_{0}^{\infty} \cos x^2 \mathrm{d}x = \int_{0}^{\infty} \sin x^2 \mathrm{d}x = \frac{1}{2} \sqrt{\frac{\pi}{2}}.
\]

大家不难利用计算这两个积分的方法算出
\[
\int_{0}^{\infty} \cos x^n \mathrm{d}x \ (n > 1),
\]
和
\[
\int_{0}^{\infty} \sin x^n \mathrm{d}x \ (n > 1).
\]

(2) Poisson积分\(\int_{0}^{\infty} \mathrm{e}^{-ax^2} \cos bx \mathrm{d}x \ (a > 0)\)
\begin{figure}[H]
\centering
\includegraphics[scale=0.3]{图5.8.png}
\caption{}
\label{figure:图5.8}
\end{figure}
取函数\( f(z) = \mathrm{e}^{-az^2} \),取围道如\reffig{figure:图5.8}所示。因为\( f \)是整函数,由\hyperref[theorem:Cauchy-Goursat定理(Cauchy积分定理)]{Cauchy积分定理},有
\begin{align}
\int_{-R}^{R} \mathrm{e}^{-ax^2} \mathrm{d}x + \int_{\gamma_1} \mathrm{e}^{-az^2} \mathrm{d}z + \int_{\gamma_2} \mathrm{e}^{-az^2} \mathrm{d}z + \int_{\gamma_3} \mathrm{e}^{-az^2} \mathrm{d}z = 0. \label{poisson_eq1}
\end{align}
当\( z \in \gamma_1 \)时,\( z = R + \mathrm{i}y, 0 \leqslant y \leqslant \frac{b}{2a} \),所以
\[
\left| \int_{\gamma_1} \mathrm{e}^{-az^2} \mathrm{d}z \right| \leqslant \int_{0}^{\frac{b}{2a}} \mathrm{e}^{-a(R^2 - y^2)} \mathrm{d}y
\leqslant \mathrm{e}^{-aR^2} \cdot \mathrm{e}^{a\left( \frac{b}{2a} \right)^2} \cdot \frac{b}{2a}
\to 0 \ (R \to \infty).
\]
同样道理,有
\[
\int_{\gamma_3} \mathrm{e}^{-az^2} \mathrm{d}z \to 0 \ (R \to \infty).
\]
当\( z \in \gamma_2 \)时,\( z = x + \frac{b}{2a} \mathrm{i}, -R \leqslant x \leqslant R \),所以
\[
\int_{\gamma_2} \mathrm{e}^{-az^2} \mathrm{d}z = - \int_{-R}^{R} \mathrm{e}^{-a \left( x^2 - \frac{b^2}{4a^2} + \frac{b}{a} x \mathrm{i} \right)} \mathrm{d}x
\]
\[
= - \mathrm{e}^{\frac{b^2}{4a}} \int_{-R}^{R} \mathrm{e}^{-ax^2} \mathrm{e}^{-b x \mathrm{i}} \mathrm{d}x
\]
\[
= - \mathrm{e}^{\frac{b^2}{4a}} \int_{-R}^{R} \mathrm{e}^{-ax^2} \cos bx \mathrm{d}x.
\]
在\(\eqref{poisson_eq1}\)式中令\( R \to \infty \),即得
\[
\int_{-\infty}^{\infty} \mathrm{e}^{-ax^2} \mathrm{d}x - \mathrm{e}^{\frac{b^2}{4a}} \int_{-\infty}^{\infty} \mathrm{e}^{-ax^2} \cos bx \mathrm{d}x = 0.
\]
由概率积分可得
\[
\int_{-\infty}^{\infty} \mathrm{e}^{-ax^2} \mathrm{d}x = \sqrt{\frac{\pi}{a}},
\]
所以
\[
\int_{0}^{\infty} \mathrm{e}^{-ax^2} \cos bx \mathrm{d}x = \frac{1}{2} \sqrt{\frac{\pi}{a}} \mathrm{e}^{-\frac{b^2}{4a}}.
\]















































































\end{document}
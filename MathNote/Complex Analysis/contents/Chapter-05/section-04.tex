\documentclass[../../main.tex]{subfiles}% 注意这里的文件路径不能用 ./main.tex ,否则用latexmk编译子文件会报错
\graphicspath{{\subfix{./image/}}} % 指定图片目录,后续可以直接使用图片文件名
% 注意这里的文件路径不能用 ../../image/ ,否则用latexmk编译子文件会报错

% 例如:
% \begin{figure}[H]
% \centering
% \includegraphics[scale=0.3]{图.png}
% \caption{}
% \label{figure:图}
% \end{figure}
% 注意:上述\label{}一定要放在\caption{}之后,否则引用图片序号会只会显示??.

\begin{document}

\section{留数定理(残数定理)}

\begin{definition}
设 \( a \) 是 \( f \) 的一个孤立奇点,\( f \) 在 \( a \) 点的邻域 \( B(a,r) \) 中的Laurent展开为 \( f(z) = \sum_{n = -\infty}^{\infty} c_n (z - a)^n \),称 \( c_{-1} \) 为 \( f \) 在 \( a \) 点的\textbf{留数(残数)},记为
\[
\mathrm{Res}(f,a) = c_{-1}
\]
或
\[
\underset{z = a}{\mathrm{Res}} f = c_{-1}.
\]

若 \( z = \infty \) 是 \( f \) 的孤立奇点,即 \( f \) 在 \( R < |z| < \infty \) 中全纯,我们定义 \( f \) 在 \( z = \infty \) 处的\textbf{留数(残数)}为
\begin{align}
\mathrm{Res}(f,\infty) = -\frac{1}{2\pi \mathrm{i}} \int_{\gamma} f(z) \mathrm{d}z, \label{eq:::897897889---3}
\end{align}
这里,\( \gamma = \{ z: |z| = \rho \}, R < \rho < \infty \).
\end{definition}

\begin{proposition}\label{proposition:积分与留数的关系}
设 \( a \) 是 \( f \) 的一个孤立奇点,对于 \( a \) 点邻域中的任意可求长闭曲线 \( \gamma \),都有$$\int_{\gamma} f(z) \mathrm{d}z = 2\pi \mathrm{i} \mathrm{Res}(f,a).$$
\end{proposition}
\begin{remark}
如果\( f \) 在 \( a \) 点全纯,那么对于 \( a \) 点邻域中的任意可求长闭曲线 \( \gamma \),都有 \( \int_{\gamma} f(z) \mathrm{d}z = 0 \).如果 \( a \) 是 \( f \) 的孤立奇点,那么上述积分不一定总等于零,且积分值只与 \( f \) 和 \( a \) 有关,而与 \( \gamma \) 无关.
\end{remark}
\begin{proof}
设 \( f \) 在 \( a \) 点邻域中的Laurent展开式为
\[
f(z) = \sum_{n = -\infty}^{\infty} c_n (z - a)^n,
\]
这里
\[
c_n = \frac{1}{2\pi \mathrm{i}} \int_{\gamma} \frac{f(\zeta)}{(\zeta - a)^{n + 1}} \mathrm{d}\zeta, \, n = 0, \pm 1, \cdots.
\]
特别地,当 \( n = -1 \) 时,我们有
\begin{align}
c_{-1} = \frac{1}{2\pi \mathrm{i}} \int_{\gamma} f(\zeta) \mathrm{d}\zeta. \label{eq:::897897889---1}
\end{align}
原来所讨论的积分值就是 \( c_{-1} \) 的 \( 2\pi \mathrm{i} \) 倍(因此 \( c_{-1} \) 这个系数有它特殊的含义),根据\(\eqref{eq:::897897889---1}\)式,我们有
\begin{align}
\int_{\gamma} f(z) \mathrm{d}z = 2\pi \mathrm{i} \mathrm{Res}(f,a). \label{eq:::897897889---2}
\end{align}
这里,\( \gamma = \{ z: |z - a| = \rho \}, 0 < \rho < r \).

\end{proof}

\begin{proposition}\label{proposition:命题5.4.2}
若 \( a \) 是 \( f \) 的 \( m \) 阶极点,则
\[
\mathrm{Res}(f,a) = \frac{1}{(m - 1)!} \lim_{z \to a} \frac{\mathrm{d}^{m - 1}}{\mathrm{d}z^{m - 1}} \left\{ (z - a)^m f(z) \right\}.
\]
\end{proposition}
\begin{proof}
因为 \( a \) 是 \( f \) 的 \( m \) 阶极点,故由\rrefthe{theorem:m阶极点的充要条件}{theorem:m阶极点的充要条件-1}可知,在 \( a \) 点的邻域中有
\begin{align}
f(z) = \frac{1}{(z - a)^m} g(z), 
\label{eq::--1231231--4}
\end{align}
这里,\( g \) 在 \( a \) 点全纯,且 \( g(a) \neq 0 \).于是
\begin{align*}
f(z) = \frac{1}{(z - a)^m} \sum_{n = 0}^{\infty} \frac{g^{(n)}(a)}{n!} (z - a)^n = \sum_{n = 0}^{\infty} \frac{g^{(n)}(a)}{n!} (z - a)^{n - m}.
\end{align*}
这是一个Laurent展开式,\( (z - a)^{-1} \) 的系数为 \( \frac{g^{(m - 1)}(a)}{(m - 1)!} \).由\eqref{eq::--1231231--4}式知 \( g(z) = (z - a)^m f(z) \),因而得
\begin{align*}
\mathrm{Res}(f,a) = \frac{g^{(m - 1)}(a)}{(m - 1)!} = \frac{1}{(m - 1)!} \lim_{z \to a} \frac{\mathrm{d}^{m - 1}}{\mathrm{d}z^{m - 1}} \left\{ (z - a)^m f(z) \right\}.
\end{align*}

\end{proof}

\begin{proposition}\label{proposition:命题5.4.3}
若 \( a \) 是 \( f \) 的1阶极点,则
\[
\mathrm{Res}(f,a) = \lim_{z \to a} (z - a) f(z).
\]
\end{proposition}
\begin{proof}
这就是\refpro{proposition:命题5.4.2}中$n=1$的情形.

\end{proof}

\begin{example}
若 \( f(z) = \frac{1}{1 + z^2} \),\( z = \pm \mathrm{i} \) 都是 \( f \) 的1阶极点,求这个两个极点的留数.
\end{example}
\begin{solution}
由\refpro{proposition:命题5.4.3}即得
\[
\mathrm{Res}(f,\mathrm{i}) = \lim_{z \to \mathrm{i}} (z - \mathrm{i}) \frac{1}{1 + z^2} = \frac{1}{2\mathrm{i}},
\]
\[
\mathrm{Res}(f, -\mathrm{i}) = \lim_{z \to -\mathrm{i}} (z + \mathrm{i}) \frac{1}{1 + z^2} = -\frac{1}{2\mathrm{i}}.
\]

\end{solution}

\begin{proposition}\label{proposition:命题5.4.5}
设 \( f = \frac{g}{h} \),\( g \) 和 \( h \) 都在 \( a \) 处全纯,且 \( g(a) \neq 0 \),\( h(a) = 0 \),\( h'(a) \neq 0 \),那么
\[
\mathrm{Res}(f,a) = \frac{g(a)}{h'(a)}.
\]
\end{proposition}
\begin{proof}
在所设的条件下,\( a \) 是 \( f \) 的1阶极点,故由\refpro{proposition:命题5.4.3}即得
\begin{align*}
\mathrm{Res}(f,a) = \lim_{z \to a} (z - a) \frac{g(z)}{h(z)} = \lim_{z \to a} \frac{g(z)}{\frac{h(z) - h(a)}{z - a}} = \frac{g(a)}{h'(a)}.
\end{align*}

\end{proof}

\begin{example}
计算 \( f(z) = \frac{\mathrm{e}^z}{\sin z} \) 在 \( z = 0 \) 处的残数.
\end{example}
\begin{solution}
这时 \( g(z) = \mathrm{e}^z \),\( h(z) = \sin z \).于是 \( g(0) = 1 \),\( h(0) = 0 \),\( h'(0) = 1 \),因而由\refpro{proposition:命题5.4.5}得
\[
\mathrm{Res}(f,0) = 1.
\]

\end{solution}

\begin{example}
计算函数 \( f(z) = \frac{\mathrm{e}^{\mathrm{i}z}}{z(z^2 + 1)^2} \) 在 \( z = -\mathrm{i} \) 处的残数.
\end{example}
\begin{solution}
显然,\( z = -\mathrm{i} \) 是 \( f \) 的一个2阶极点,利用\refpro{proposition:命题5.4.2},得
\begin{align*}
\mathrm{Res}(f, -\mathrm{i}) = \lim_{z \to -\mathrm{i}} \frac{\mathrm{d}}{\mathrm{d}z} \left( \frac{\mathrm{e}^{\mathrm{i}z}}{z(z - \mathrm{i})^2} \right) = \frac{\mathrm{e}}{4}.
\end{align*}

\end{solution}

\begin{example}
计算 \( f(z) = \mathrm{e}^{z + \frac{1}{z}} \) 在 \( z = 0 \) 处的残数.
\end{example}
\begin{remark}
如果 \( a \) 是 \( f \) 的本性奇点,就没有像上面那种简单的计算残数的公式了,这时只能通过 \( f \) 的Laurent展开来得到 \( f \) 在 \( a \) 点的残数.
\end{remark}
\begin{solution}
因为
\begin{align*}
f(z) = \mathrm{e}^z \cdot \mathrm{e}^{\frac{1}{z}} = \left( 1 + z + \frac{z^2}{2!} + \cdots \right) \left( 1 + \frac{1}{z} + \frac{1}{2! z^2} + \cdots \right),
\end{align*}
在这个乘积中,\( \frac{1}{z} \) 的系数为
\[
1 + \frac{1}{2!} + \frac{1}{2! 3!} + \frac{1}{3! 4!} + \cdots,
\]
这就是要找的残数,即
\[
\mathrm{Res}(f,0) = \sum_{n = 0}^{\infty} \frac{1}{n! (n + 1)!}.
\]

\end{solution}

\begin{theorem}[留数定理(残数定理)]\label{theorem:留数定理(残数定理)-定理5.4.9}
设 \( D \) 是复平面上的一个有界区域,它的边界 \( \gamma \) 由一条或若干条简单闭曲线组成.如果 \( f \) 在 \( D \) 中除去孤立奇点 \( z_1, \cdots, z_n \) 外是全纯的,在闭域 \( \overline{D} \) 上除去 \( z_1, \cdots, z_n \) 外是连续的,那么
\begin{align}
\int_{\gamma} f(z) \mathrm{d}z = 2\pi \mathrm{i} \sum_{k = 1}^{n} \mathrm{Res}(f, z_k).\label{eq::---2837128-5}
\end{align}
\end{theorem}
\begin{remark}
这个定理称为\textbf{留数定理(残数定理)},它的主要贡献是把积分计算归结为残数的计算.而从\refpro{proposition:命题5.4.2}知道,计算残数是一个微分运算.因此,从实质上来说,残数定理把积分运算变成了微分运算,从而带来了方便.
\end{remark}
\begin{proof}
在 \( D \) 内以 \( z_k (k = 1, 2, \cdots, n) \) 为中心作一小圆周 \( \gamma_k \),使得所有 \( \gamma_k \) 都在 \( D \) 的内部,且每一个 \( \gamma_k \) 都在其余小圆周的外部.于是由\hyperref[theorem:定理3.2.5]{多连通区域的Cauchy积分定理},得
\[
\int_{\gamma} f(z) \mathrm{d}z = \sum_{k = 1}^{n} \int_{\gamma_k} f(z) \mathrm{d}z.
\]
再由\refpro{proposition:积分与留数的关系},即得所要证的公式\(\eqref{eq::---2837128-5}\).

\end{proof}

\begin{example}
计算积分
\[
\int_{\gamma} \frac{z}{(z^2 - 1)^2 (z^2 + 1)} \mathrm{d}z,
\]
这里,\( \gamma = \{ z: |z - 1| = \sqrt{3} \} \).
\end{example}
\begin{solution}
被积函数
\[
f(z) = \frac{z}{(z^2 - 1)^2 (z^2 + 1)}
\]
有两个1阶极点 \( z_1 = \mathrm{i}, z_2 = -\mathrm{i} \),以及两个2阶极点 \( z_3 = 1, z_4 = -1 \).容易看出,\( z_1, z_2, z_3 \) 都在 \( \gamma \) 的内部,\( z_4 \) 在 \( \gamma \) 的外部.由\hyperref[theorem:留数定理(残数定理)-定理5.4.9]{留数定理}得
\[
\int_{\gamma} f(z) \mathrm{d}z = 2\pi \mathrm{i} \sum_{k = 1}^{3} \mathrm{Res}(f, z_k).
\]
由\refpro{proposition:命题5.4.3}和\refpro{proposition:命题5.4.2},得
\begin{align*}
\mathrm{Res}(f, \mathrm{i}) = \lim_{z \to \mathrm{i}} (z - \mathrm{i}) f(z) = \lim_{z \to \mathrm{i}} \frac{z}{(z^2 - 1)^2 (z + \mathrm{i})} = \frac{1}{8},
\end{align*}
\begin{align*}
\mathrm{Res}(f, -\mathrm{i}) = \lim_{z \to -\mathrm{i}} (z + \mathrm{i}) f(z) = \lim_{z \to -\mathrm{i}} \frac{z}{(z^2 - 1)^2 (z - \mathrm{i})} = \frac{1}{8},
\end{align*}
\begin{align*}
\mathrm{Res}(f, 1) = \lim_{z \to 1} \frac{\mathrm{d}}{\mathrm{d}z} \left\{ (z - 1)^2 f(z) \right\} = \lim_{z \to 1} \frac{\mathrm{d}}{\mathrm{d}z} \left\{ \frac{z}{(z + 1)^2 (z^2 + 1)} \right\} = \lim_{z \to 1} \frac{-3z^3 - z^2 - z + 1}{(z + 1)^3 (z^2 + 1)^2} = -\frac{1}{8}.
\end{align*}
因而有
\begin{align*}
\int_{\gamma} f(z) \mathrm{d}z = 2\pi \mathrm{i} \left( \frac{1}{8} + \frac{1}{8} - \frac{1}{8} \right) = \frac{\pi \mathrm{i}}{4}.
\end{align*}

\end{solution}

\begin{example}
计算积分
\[
\int_{|z| = 1} \frac{z^2 \sin^2 z}{(1 - \mathrm{e}^z)^5} \mathrm{d}z.
\]
\end{example}
\begin{solution}
容易看出,被积函数
\[
f(z) = \frac{z^2 \sin^2 z}{(1 - \mathrm{e}^z)^5}
\]
在 \( |z| = 1 \) 内只有一个极点 \( z = 0 \).对于这种类型的函数,直接从Laurent展开来求残数更方便些:
\begin{align*}
\frac{z^2 \sin^2 z}{(1 - \mathrm{e}^z)^5} = \frac{z^2 \left( z - \frac{z^3}{3!} + \cdots \right)^2}{\left( -z - \frac{z^2}{2!} - \cdots \right)^5} = -\frac{z^4 \left( 1 - \frac{z^2}{3!} + \cdots \right)^2}{z^5 \left( 1 + \frac{z}{2!} + \cdots \right)^5}.
\end{align*}
因为 \( \frac{\left( 1 - \frac{z^2}{3!} + \cdots \right)^2}{\left( 1 + \frac{z}{2!} + \cdots \right)^5} \) 在 \( z = 0 \) 处全纯,且在 \( z = 0 \) 处等于1,故其Taylor展开可写为 \( 1 + c_1 z + \cdots \),于是得
\[
\frac{z^2 \sin^2 z}{(1 - \mathrm{e}^z)^5} = -\frac{1}{z} (1 + c_1 z + \cdots),
\]
因而 \( \mathrm{Res}(f, 0) = -1 \).由\hyperref[theorem:留数定理(残数定理)-定理5.4.9]{留数定理}即得
\[
\int_{|z| = 1} \frac{z^2 \sin^2 z}{(1 - \mathrm{e}^z)^5} \mathrm{d}z = -2\pi \mathrm{i}.
\]

\end{solution}

\begin{theorem}\label{theorem:定理5.4.12}
若 \( f \) 在 \( \mathbb{C} \) 中除去 \( z_1, \cdots, z_n \) 外是全纯的,则 \( f \) 在 \( z_1, \cdots, z_n \) 及 \( z = \infty \) 处的残数之和为零.
\end{theorem}
\begin{remark}
这个定理是\hyperref[theorem:留数定理(残数定理)-定理5.4.9]{留数定理}的另一种形式.
\end{remark}
\begin{proof}
取 \( R \) 充分大,使得 \( z_1, \cdots, z_n \) 都在 \( B(0,R) \) 中.于是,由\hyperref[theorem:留数定理(残数定理)-定理5.4.9]{留数定理}得
\begin{align}
\int_{|z| = R} f(z) \mathrm{d}z = 2\pi \mathrm{i} \sum_{k = 1}^{n} \mathrm{Res}(f, z_k).\label{eq:----12389749---6}
\end{align}
但由\(\eqref{eq:::897897889---3}\)式得
\begin{align}\label{eq:----12389749---7}
- \int_{|z| = R} f(z) \mathrm{d}z = 2\pi \mathrm{i} \mathrm{Res}(f, \infty).
\end{align}
由\(\eqref{eq:----12389749---6}\)式和\(\eqref{eq:----12389749---7}\)式即得所要证之结论.

\end{proof}












\end{document}
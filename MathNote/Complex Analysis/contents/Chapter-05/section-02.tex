\documentclass[../../main.tex]{subfiles}
\graphicspath{{\subfix{../../image/}}} % 指定图片目录,后续可以直接使用图片文件名。

% 例如:
% \begin{figure}[H]
% \centering
% \includegraphics[scale=0.4]{图.png}
% \caption{}
% \label{figure:图}
% \end{figure}
% 注意:上述\label{}一定要放在\caption{}之后,否则引用图片序号会只会显示??.

\begin{document}

\section{孤立奇点}

\begin{definition}
如果 \( f \) 在无心圆盘(即除去圆心后的圆盘)\( \{ z: 0 < |z - z_0| < R \} \) 中全纯,但在$z_0$处不全纯,就称 \( z_0 \) 是 \( f \) 的\textbf{孤立奇点}。

\( f \) 在孤立奇点 \( z_0 \) 附近可能有三种情形:

(i) \( \lim\limits_{z \to z_0} f(z) = a \),\( a \) 是一有限数,这时称 \( z_0 \) 是 \( f \) 的\textbf{可去奇点};

(ii) \( \lim\limits_{z \to z_0} f(z) = \infty \),这时称 \( z_0 \) 是 \( f \) 的\textbf{极点};

(iii) \( \lim\limits_{z \to z_0} f(z) \) 不存在,这时称 \( z_0 \) 是 \( f \) 的\textbf{本性奇点}。
\end{definition}

\begin{theorem}[Riemann定理]\label{theorem:Riemann定理-定理5.2.1}
\( z_0 \) 是 \( f \) 的可去奇点的充分必要条件是 \( f \) 在 \( z_0 \) 附近有界。
\end{theorem}
\begin{proof}
必要性是显然的,因为如果 \( z_0 \) 是 \( f \) 的可去奇点,那么 \( \lim\limits_{z \to z_0} f(z) = a \),\( f \) 在 \( z_0 \) 附近当然有界。现在设 \( f \) 在 \( z_0 \) 附近有界,即存在 \( \varepsilon > 0 \),使得当 \( z \) 满足 \( 0 < |z - z_0| < \varepsilon \) 时,\( |f(z)| < M \)。因为 \( f \) 在无心圆盘 \( D = \{ z: 0 < |z - z_0| < R \} \) 中全纯,根据\refthe{theorem:定理5.1.2},\( f \) 在 \( D \) 中有 Laurent 展开式:
\begin{align}
f(z) = \sum_{n = -\infty}^{\infty} a_n (z - z_0)^n, \, z \in D, \label{eq:11776311112155111111}
\end{align}
其中,\( a_n = \frac{1}{2\pi \mathrm{i}} \int_{\gamma_{\rho}} \frac{f(\zeta)}{(\zeta - z_0)^{n + 1}} \mathrm{d}\zeta \),\( 0 < \rho < R \),\( \gamma_{\rho} = \{ \zeta: |\zeta - z_0| = \rho \} \)。今取 \( 0 < \rho < \varepsilon \),故当 \( \zeta \in \gamma_{\rho} \) 时,\( |f(\zeta)| < M \)。于是,由\hyperref[proposition:长大不等式]{长大不等式}得
\begin{align*}
|a_{-n}| = \left| \frac{1}{2\pi \mathrm{i}} \int_{\gamma_{\rho}} \frac{f(\zeta)}{(\zeta - z_0)^{-n + 1}} \mathrm{d}\zeta \right| \leq \frac{M}{2\pi \rho^{-n + 1}} \cdot 2\pi \rho = M \rho^n,
\end{align*}
让 \( \rho \to 0 \),即得 \( a_{-n} = 0 \),\( n = 1, 2, \cdots \)。这说明在 \( f \) 的 Laurent 展开式 \eqref{eq:11776311112155111111} 中,所有负次幂的系数都是零,因而展开式 \eqref{eq:11776311112155111111} 是一个幂级数。所以 \( \lim\limits_{z \to z_0} f(z) = a_0 \),即 \( z_0 \) 是一个可去奇点。 
\end{proof}
\begin{remark}
从上面的证明可以看出,\( f \) 在可去奇点处的特征是Laurent展开式没有主要部分,只有全纯部分.在 \( z_0 \) 是 \( f \) 的可去奇点的情形下,\( f \) 在 \( \{ z: 0 < |z - z_0| < R \} \) 中的展开式为
\[
f(z) = \sum_{n = 0}^{\infty} a_n (z - z_0)^n,
\]
只要令 \( f(z_0) = a_0 \),上式便在圆盘 \( B(z_0, R) \) 中成立了,因而 \( f \) 在 \( z_0 \) 处全纯。换句话说,在这种情形下,只要适当定义 \( f \) 在 \( z_0 \) 处的值,便能使 \( f \) 在 \( z_0 \) 处全纯。这就是称 \( z_0 \) 为 \( f \) 的可去奇点的原因。
\end{remark}

\begin{proposition}\label{proposition:命题5.2.2}
\( z_0 \) 是 \( f \) 的极点的充分必要条件是 \( z_0 \) 为 \( \frac{1}{f} \) 的零点.
\end{proposition}
\begin{proof}
如果 \( z_0 \) 是 \( f \) 的极点,即 \( \lim\limits_{z \to z_0} f(z) = \infty \),那么存在 \( \varepsilon > 0 \),使得当 \( 0 < |z - z_0| < \varepsilon \) 时,\( f(z) \) 不等于零.故 \( \varphi(z) = \frac{1}{f(z)} \) 在上述无心圆盘中全纯,且 \( \lim\limits_{z \to z_0} \varphi(z) = 0 \),即 \( z_0 \) 是 \( \varphi \) 的可去奇点,且 \( \varphi(z_0) = 0 \).

反之,如果 \( z_0 \) 是 \( \varphi(z) = \frac{1}{f(z)} \) 的零点,则
\[
\lim\limits_{z \to z_0} f(z) = \lim\limits_{z \to z_0} \frac{1}{\varphi(z)} = \infty,
\]
即 \( z_0 \) 是 \( f \) 的极点.
\end{proof}
 
\begin{definition}
如果 \( z_0 \) 是 \( \frac{1}{f(z)} \) 的 \( m \) 阶零点,就称 \( z_0 \) 是 \( f \) 的 \textbf{\( m \) 阶极点}.
\end{definition}

\begin{theorem}\label{theorem:定理5.2.4}
\( z_0 \) 是 \( f \) 的 \( m \) 阶极点的充分必要条件是 \( f \) 在 \( z_0 \) 附近的Laurent展开式为
\begin{align}
f(z) = \frac{a_{-m}}{(z - z_0)^m} + \cdots + \frac{a_{-1}}{z - z_0} + a_0 + a_1(z - z_0) + \cdots, \label{eq---:::213907---2}
\end{align}
其中,\( a_{-m} \neq 0 \).
\end{theorem}
\begin{remark}
从这个定理可以看出,\( f \) 在极点处的特征是Laurent展开式的主要部分只有有限项.
\end{remark}
\begin{proof}
如果 \( z_0 \) 是 \( f \) 的 \( m \) 阶极点,根据定义,它是 \( \frac{1}{f} \) 的 \( m \) 阶零点.由\refpro{proposition:命题4.3.4},它在 \( z_0 \) 的邻域中可以表示为 \( \frac{1}{f(z)} = (z - z_0)^m g(z) \),这里,\( g \) 在 \( z_0 \) 处全纯,且 \( g(z_0) \neq 0 \),因而 \( \frac{1}{g} \) 也在 \( z_0 \) 处全纯.设 \( \frac{1}{g} \) 在 \( z_0 \) 处的Taylor展开为
\[
\frac{1}{g(z)} = \sum_{n=0}^{\infty} c_n (z - z_0)^n,
\]
这里,\( c_0 \neq 0 \),于是
\begin{align*}
f(z) = \frac{1}{(z - z_0)^m} \frac{1}{g(z)} = \sum_{n=0}^{\infty} c_n (z - z_0)^{n - m} = \frac{c_0}{(z - z_0)^m} + \cdots + \frac{c_{m - 1}}{z - z_0} + c_m + c_{m + 1}(z - z_0) + \cdots.
\end{align*}
记 \( a_n = c_{n + m}, n = -m, \cdots, -1, 0, 1, \cdots \),即得展开式\(\eqref{eq---:::213907---2}\).

反之,如果 \( f \) 在 \( z_0 \) 附近的Laurent展开式为\(\eqref{eq---:::213907---2}\)式,那么
\begin{align*}
(z - z_0)^m f(z) = a_{-m} + a_{-(m - 1)}(z - z_0) + \cdots + a_0(z - z_0)^m + \cdots.
\end{align*}
若记上式右端的幂级数为 \( \varphi(z) \),则 \( \varphi \) 在 \( z_0 \) 处全纯,且 \( \varphi(z_0) = a_{-m} \neq 0 \).因而 \( \frac{1}{\varphi} \) 也在 \( z_0 \) 处全纯,于是
\[
\frac{1}{f(z)} = (z - z_0)^m \frac{1}{\varphi(z)}
\]
在 \( z_0 \) 附近成立.由\refpro{proposition:命题4.3.4},\( z_0 \) 是 \( \frac{1}{f} \) 的 \( m \) 阶零点,所以是 \( f \) 的 \( m \) 阶极点.
\end{proof}

\begin{theorem}[Weierstrass定理]\label{theorem:Weierstrass-定理5.2.5}
设 \( z_0 \) 是 \( f \) 的本性奇点,那么对任意 \( A \in \mathbb{C}_{\infty} \),必存在趋于 \( z_0 \) 的点列 \( \{ z_n \} \),使得 \( \lim\limits_{n \to \infty} f(z_n) = A \).
\end{theorem}
\begin{remark}
\( f \) 在本性奇点处的特征是Laurent展开式的主要部分有无穷多项.实际上,这个定理证明了更深刻的结果.
\end{remark}
\begin{proof}
先设 \( A = \infty \).因为 \( z_0 \) 是 \( f \) 的本性奇点,故 \( f \) 在 \( z_0 \) 附近无界.于是对任意自然数 \( n \),总能找到 \( z_n \),使得 \( |z_n - z_0| < \frac{1}{n} \),但 \( |f(z_n)| > n \),这说明 \( \lim\limits_{n \to \infty} f(z_n) = \infty \).

再设 \( A \) 是一个有限数.令 \( \varphi(z) = \frac{1}{f(z) - A} \),我们证明 \( \varphi \) 在 \( z_0 \) 的邻域中无界.不然的话,\( z_0 \) 是 \( \varphi \) 的可去奇点,适当选择 \( \varphi(z_0) \) 的值,可使 \( \varphi \) 在 \( z_0 \) 处全纯.如果 \( \varphi(z_0) \neq 0 \),则因 \( f(z) = \frac{1}{\varphi(z)} + A \),\( f \) 也在 \( z_0 \) 处全纯,这不可能.故必有 \( \varphi(z_0) = 0 \),由\refpro{proposition:命题5.2.2}可知 \( z_0 \) 是 \( f(z) -A\) 的极点,也不可能.所以,\( \varphi \) 在 \( z_0 \) 的邻域中无界.于是,对任意自然数 \( n \),存在 \( z_n \),使得 \( |z_n - z_0| < \frac{1}{n} \),但 \( \frac{1}{|f(z) - A|} > n \),即 \( |f(z) - A| < \frac{1}{n} \).这就证明了 \( \lim\limits_{n \to \infty} f(z_n) = A \).
\end{proof}

\begin{theorem}[Picard定理]\label{ theorem:Picard定理-定理5.2.6}
全纯函数在本性奇点的邻域内无穷多次地取到每个有穷复值,最多只有一个例外.
\end{theorem}
\begin{remark}
例如,考虑函数 \( f(z) = \mathrm{e}^{\frac{1}{z}} \),它在 \( z = 0 \) 附近是全纯的.若让 \( z \) 沿着 \( x \) 轴分别从 \( 0 \) 的左边和右边趋于 \( 0 \),可得
\[
\lim_{\substack{z = x \to 0^-}} \mathrm{e}^{\frac{1}{z}} = \lim_{x \to 0^-} \mathrm{e}^{\frac{1}{x}} = 0,
\]
\[
\lim_{\substack{z = x \to 0^+}} \mathrm{e}^{\frac{1}{z}} = \lim_{x \to 0^+} \mathrm{e}^{\frac{1}{x}} = \infty.
\]
这说明 \( \lim_{z \to 0} \mathrm{e}^{\frac{1}{z}} \) 不存在,所以 \( z = 0 \) 是 \( \mathrm{e}^{\frac{1}{z}} \) 的本性奇点.对于任意复数 \( a \neq 0 \),若取 \( z_n = (\log a + 2n\pi \mathrm{i})^{-1} \),则 \( f(z_n) = \mathrm{e}^{\log a + 2n\pi \mathrm{i}} = a \).由于 \( z_n \to 0 \),这说明 \( \mathrm{e}^{\frac{1}{z}} \) 在 \( z = 0 \) 的邻域中可以无穷多次地取到非零值 \( a \),但 \( 0 \) 是它的唯一的例外值.
\end{remark}

\begin{definition}
如果 \( f \) 在无穷远点的邻域(不包括无穷远点) \( \{ z: 0 \leqslant R < |z| < \infty \} \) 中全纯,就称 \( \infty \) 是 \( f \) 的孤立奇点.
\end{definition}
\begin{remark}
在这种情形下,作变换 \( z = \frac{1}{\zeta} \),记
\[
g(\zeta) = f\left( \frac{1}{\zeta} \right),
\]
则 \( g \) 在 \( 0 < |\zeta| < \frac{1}{R} \) 中全纯,即 \( \zeta = 0 \) 是 \( g \) 的孤立奇点.
\end{remark}

\begin{definition}
设$g(\zeta) = f\left( \frac{1}{\zeta} \right),$如果 \( \zeta = 0 \) 是 \( g \) 的可去奇点、\( m \) 阶极点或本性奇点,那么我们相应地称 \( z = \infty \) 是 \( f \) 的可去奇点、\( m \) 阶极点或本性奇点.
\end{definition}

\begin{proposition}\label{proposition:无穷远点为孤立奇点时的Laurent展开}
设\( g \) 在原点的邻域中有Laurent展开:
\[
g(\zeta) = \sum_{n = -\infty}^{\infty} a_n \zeta^n, \, 0 < |\zeta| < \frac{1}{R},
\]
则\( f \) 在 \( R < |z| < \infty \) 中有下面的Laurent展开:
\[
f(z) = \sum_{n = -\infty}^{\infty} b_n z^n,
\]
其中,\( b_n = a_{-n}, n = 0, \pm 1, \cdots \).特别地,如果 \( z = \infty \) 是 \( f \) 的可去奇点或$f$在$z=\infty$处全纯,那么\( f \) 在 \( R < |z| < \infty \) 中有下面的Laurent展开式:
\begin{align}\label{eq:::--1232-3}
f(z) = \sum_{n = 0}^{\infty} b_{-n} z^{-n}.
\end{align}
如果 \( z = \infty \) 是 \( f \) 的 \( m \) 阶极点,那么 \( f \) 在 \( R < |z| < \infty \) 中有下面的Laurent展开式:
\begin{align}\label{eq:::--1232-4}
f(z) = b_m z^m + \cdots + b_1 z + b_0 + b_{-1} z^{-1} + \cdots,
\end{align}
如果 \( z = \infty \) 是 \( f \) 的 本性奇点,那么 \( f \) 在 \( R < |z| < \infty \) 中有下面的Laurent展开式:
\begin{align}\label{eq:::--1232-5}
f(z) = \cdots + b_m z^m + \cdots + b_1 z + b_0 + b_{-1} z^{-1} + \cdots.
\end{align}
这时,我们称 \( \sum_{n = 1}^{\infty} b_n z^n \) 为 \( f \) 的\textbf{主要部分}, \( \sum_{n = 0}^{\infty} b_{-n} z^{-n} \) 为 \( f \) 的\textbf{全纯部分}.
\end{proposition}
\begin{proof}
因为 \( g \) 在原点的邻域中有Laurent展开:
\[
g(\zeta) = \sum_{n = -\infty}^{\infty} a_n \zeta^n, \, 0 < |\zeta| < \frac{1}{R},
\]
所以 \( f \) 在 \( R < |z| < \infty \) 中有下面的Laurent展开:
\[
f(z) = \sum_{n = -\infty}^{\infty} b_n z^n,
\]
其中,\( b_n = a_{-n}, n = 0, \pm 1, \cdots \).

特别地,如果 \( z = \infty \) 是 \( f \) 的可去奇点,即 \( \zeta = 0 \) 是 \( g \) 的可去奇点,因而由\hyperref[theorem:Riemann定理-定理5.2.1]{Riemann定理的证明}可知 \( a_n = 0 \) (\( n = -1, -2, \cdots \)).如果$f$在$z=\infty$处全纯,由\hyperref[theorem:Riemann定理-定理5.2.1]{Riemann定理的注}可知 \( a_n = 0 \) (\( n = -1, -2, \cdots \)).所以此时 \( f \) 的Laurent展开为
\[
f(z) = \sum_{n = 0}^{\infty} b_{-n} z^{-n}.
\]

同样道理,如果 \( z = \infty \) 分别是 \( f \) 的 \( m \) 阶极点或本性奇点,那么 \( f \) 在 \( R < |z| < \infty \) 中分别有下面的Laurent展开式:
\[
f(z) = b_m z^m + \cdots + b_1 z + b_0 + b_{-1} z^{-1} + \cdots, 
\]
或
\[
f(z) = \cdots + b_m z^m + \cdots + b_1 z + b_0 + b_{-1} z^{-1} + \cdots. 
\]
\end{proof}






















\end{document}
\documentclass[../../main.tex]{subfiles}% 注意这里的文件路径不能用 ./main.tex ,否则用latexmk编译子文件会报错
\graphicspath{{\subfix{./image/}}} % 指定图片目录,后续可以直接使用图片文件名
% 注意这里的文件路径不能用 ../../image/ ,否则用latexmk编译子文件会报错

% 例如:
% \begin{figure}[H]
% \centering
% \includegraphics[scale=0.3]{图.png}
% \caption{}
% \label{figure:图}
% \end{figure}
% 注意:上述\label{}一定要放在\caption{}之后,否则引用图片序号会只会显示??.

\begin{document}

\section{孤立奇点}

\begin{definition}
设$z_0\in \mathbb{C}$,如果 \( f \) 在无心圆盘(即除去圆心后的圆盘)\( \{ z: 0 < |z - z_0| < R \} \) 中全纯,但在$z_0$处不全纯,就称 \( z_0 \) 是 \( f \) 的\textbf{孤立奇点}。不是孤立奇点的奇点称为\textbf{非孤立奇点}.

\( f \) 在孤立奇点 \( z_0 \) 附近可能有三种情形:

(i) \( \lim\limits_{z \to z_0} f(z) = a \),\( a \) 是一有限数,这时称 \( z_0 \) 是 \( f \) 的\textbf{可去奇点};

(ii) \( \lim\limits_{z \to z_0} f(z) = \infty \),这时称 \( z_0 \) 是 \( f \) 的\textbf{极点};

(iii) \( \lim\limits_{z \to z_0} f(z) \) 不存在,这时称 \( z_0 \) 是 \( f \) 的\textbf{本性奇点}或\textbf{本质奇点}.
\end{definition}

\begin{theorem}
$a\in \mathbb{C}$是函数$f(z)$的非孤立奇点的充要条件是存在一列奇点$\{z_n\}$,使得$\lim_{n\to \infty}z_n=a.$
\end{theorem}
\begin{proof}


\end{proof}

\begin{theorem}[Riemann定理]\label{theorem:Riemann定理-定理5.2.1}
如果$a\in \mathbb{C}$为函数$f(z)$的孤立奇点,则$a$为可去奇点的充要条件是下列三条中的任何一条成立:
\begin{enumerate}[(1)]
\item $f(z)$在点$a$的主要部分为零.即存在$a$的某个去心邻域$D=\{z:0<|z-a|<R\}(R\in \mathbb{R})$,使得\( f \)在$D$内可Laurent展开为一个幂级数
\begin{align*}
f(z)=\sum_{n=0}^{\infty}{a_n\left( z-a \right) ^n},\ z\in D.
\end{align*}
其中
\begin{align*}
a_n=\frac{1}{2\pi \mathrm{i}}\int_{\gamma _{\rho}}{\frac{f\left( \zeta \right)}{\left( \zeta -a \right) ^{n+1}}\mathrm{d}\zeta}=\frac{f^{\left( n \right)}\left( a \right)}{n!},\,\,n=0,1,\cdots .
\end{align*}
且$0<\rho <R,\gamma _{\rho}=\left\{ \zeta :\left| \zeta -a\right|=\rho \right\} .$

\item $\lim\limits_{z \to a}f(z)=b\neq\infty$.

\item 存在$a$的某个去心邻域$D=\{z:0<|z-a|<R\}(R\in \mathbb{R})$,使得\( f \)在$D$内有界.
\end{enumerate}
\end{theorem}
\begin{remark}
从上面定理可以看出,\( f \) 在可去奇点处的特征是Laurent展开式没有主要部分,只有全纯部分.在 \( a\) 是 \( f \) 的可去奇点的情形下,\( f \) 在 \( \{ z: 0 < |z - a| < R \} \) 中的展开式为
\[
f(z) = \sum_{n = 0}^{\infty} a_n (z - a)^n,
\]
只要令 \( f(a) = a_0 \),上式便在圆盘 \( B(a R) \) 中成立了,因而由\refthe{theorem:定理4.3.1}知 \( f \) 在 \( a \) 处全纯。换句话说,在这种情形下,只要补充定义 \( f \) 在 \( a \) 处的值,便能使 \( f \) 在 \(a \) 处全纯。这就是称 \(a\) 为 \( f \) 的可去奇点的原因。
\end{remark}
\begin{proof}
由可去奇点的定义知$a$为可去奇点与(2)等价.故只需证(1)(2)(3)等价即可.

(1)$\Longrightarrow $(2):由(1)知,可设$f$在$D=\{z:0<|z-a|<R\}(R\in \mathbb{R})$内,有
\begin{align*}
f(z)=a_0+c_1(z-a)+a_2(z-a)^2+\cdots\quad (0<|z-a|<R),
\end{align*}
于是
\begin{align*}
\lim\limits_{z \to a}f(z)=a_0\neq\infty.
\end{align*}

(2)$\Longrightarrow $(3):即\refpro{proposition:复变函数--例题1.31}.

(3)$\Longrightarrow $(1):设 \( f \) 在 \( a \) 附近有界,即存在 \( \varepsilon > 0 \),使得当 \( z \) 满足 \( 0 < |z - a| < \varepsilon \) 时,\( |f(z)| < M \)。因为$a$是$f$的孤立奇点,所以存在$a$的去心邻域\( D = \{ z: 0 < |z - a| < R \} \),使得\( f \) 在$D$中全纯.根据\refthe{theorem:定理5.1.2},\( f \) 在 \( D \) 中有 Laurent 展开式:
\begin{align}
f(z) = \sum_{n = -\infty}^{\infty} a_n (z - a)^n, \, z \in D, \label{eq:11776311112155111111}
\end{align}
其中,\( a_n = \frac{1}{2\pi \mathrm{i}} \int_{\gamma_{\rho}} \frac{f(\zeta)}{(\zeta - a)^{n + 1}} \mathrm{d}\zeta \),\( 0 < \rho < R \),\( \gamma_{\rho} = \{ \zeta: |\zeta - a| = \rho \} \)。任取 \( \rho\in (0,\varepsilon)  \),故当 \( \zeta \in \gamma_{\rho} \) 时,\( |f(\zeta)| < M \)。于是,由\hyperref[theorem:长大不等式]{长大不等式}得
\begin{align*}
|a_{-n}| = \left| \frac{1}{2\pi \mathrm{i}} \int_{\gamma_{\rho}} \frac{f(\zeta)}{(\zeta - a)^{-n + 1}} \mathrm{d}\zeta \right| \leqslant  \frac{M}{2\pi \rho^{-n + 1}} \cdot 2\pi \rho = M \rho^n,
\end{align*}
让 \( \rho \to 0 \),即得 \( a_{-n} = 0 \),\( n = 1, 2, \cdots \)。这说明在 \( f \) 的 Laurent 展开式 \eqref{eq:11776311112155111111} 中,所有负次幂的系数都是零,因而展开式 \eqref{eq:11776311112155111111} 是一个幂级数
\begin{align*}
f(z)=\sum_{n=0}^{\infty}{a_n\left( z-a \right) ^n},\ z\in D.
\end{align*}

\end{proof}

\begin{theorem}\label{theorem:复变函数----定理5.2.2}
函数$f(z)$的孤立奇点$a\in \mathbb{C}$为极点的充要条件是\( a\) 为 \( \frac{1}{f} \) 的零点.
\end{theorem}
\begin{proof}
如果 \( a \) 是 \( f \) 的极点,即 \( \lim\limits_{z \to a} f(z) = \infty \),那么存在 \( \varepsilon > 0 \),使得当 \( 0 < |z - a| < \varepsilon \) 时,\( f(z) \) 不等于零.故 \( \varphi(z) = \frac{1}{f(z)} \) 在无心圆盘$\{z:0<|z-a|<\varepsilon\}$中全纯,且 \( \lim\limits_{z \to a} \varphi(z) = 0 \),即 \( a \) 是 \( \varphi \) 的可去奇点,且 \( \varphi(a) = 0 \).

反之,如果 \( a \) 是 \( \varphi(z) = \frac{1}{f(z)} \) 的零点,则
\[
\lim\limits_{z \to a} f(z) = \lim\limits_{z \to a} \frac{1}{\varphi(z)} = \infty,
\]
即 \( a \) 是 \( f \) 的极点.

\end{proof}
 
\begin{definition}
如果 \( z_0 \) 是 \( \frac{1}{f} \) 的 \( m \) 阶零点,就称 \( z_0 \) 是 \( f \) 的 \textbf{\( m \) 阶极点}.
\end{definition}

\begin{theorem}\label{theorem:m阶极点的充要条件}
如果函数$f(z)$以点$z_0\in \mathbb{C}$为孤立奇点,则$a$为$m$阶极点的充要条件是下列两条中的任何一条成立:
\begin{enumerate}[(1)]
\item\label{theorem:m阶极点的充要条件-1} 存在$z_0$的某个去心邻域$D=\{z:0<|z-z_0|<R\}(R\in \mathbb{R})$,使得\( f \)在$D$内的Laurent展开式为
\begin{align}
f(z) = \frac{a_{-m}}{(z - z_0)^m} + \cdots + \frac{a_{-1}}{z - z_0} + a_0 + a_1(z - z_0) + \cdots, \label{eq---:::213907---2}
\end{align}
其中\( a_{-m} \neq 0 \),并且
\begin{align*}
a_n=\frac{1}{2\pi \mathrm{i}}\int_{\gamma _{\rho}}{\frac{f(\zeta )}{(\zeta -z_0)^{n+1}}\mathrm{d}\zeta},\,\,\,n=-m,\cdots ,-1,0,1,\cdots .
\end{align*}
而 \( \gamma_{\rho} = \{ \zeta: |\zeta - z_0| = \rho \} \)\( (r < \rho < R )\).

\item\label{theorem:m阶极点的充要条件-2} 存在$z_0$的某个去心邻域$D=\{z:0<|z-z_0|<R\}(R\in \mathbb{R})$,使得\( f \)在$D$内可以表示为
\begin{align}\label{eq::0jn2fgj4gfj2-wof8n23}
f(z) = \frac{g(z)}{(z-z_0)^m},
\end{align}
这里\( g \) 在 \( z_0 \) 点全纯且 \( g(z_0) \neq 0 \).
\end{enumerate}
\end{theorem}
\begin{remark}
从这个定理的(1)可以看出,\( f \) 在极点处的特征是Laurent展开式的主要部分只有有限项.
\end{remark}
\begin{proof}
\begin{enumerate}[(1)]
\item 如果 \( z_0 \) 是 \( f \) 的 \( m \) 阶极点,根据定义,它是 \( \frac{1}{f} \) 的 \( m \) 阶零点.由\refthe{theorem:复变函数-----定理4.3.4}知存在$z_0$的去心邻域$D=\{z:0<|z-a|<R\}(R\in\mathbb{R})$,使得$\frac{1}{f}$在$D$中可以表示为 \( \frac{1}{f(z)} = (z - z_0)^m g(z) \),这里,\( g \) 在 \( z_0 \) 处全纯,且 \( g(z_0) \neq 0 \),因而 \( \frac{1}{g} \) 也在 \( z_0 \) 处全纯.由\refthe{theorem:定理4.3.1},可设 \( \frac{1}{g} \) 在$D$的Taylor展开为
\[
\frac{1}{g(z)} = \sum_{n=0}^{\infty} c_n (z - z_0)^n,
\]
这里\( c_0 \neq 0 \),并且\( c_n = \frac{1}{2\pi \mathrm{i}} \int_{\gamma_{\rho}} \frac{f(\zeta)}{(\zeta - z_0)^{n + 1}} \mathrm{d}\zeta,n=0,1,\cdots \),而 \( \gamma_{\rho} = \{ \zeta: |\zeta - z_0| = \rho \} \)(\( r < \rho < R \)),.于是
\begin{align*}
f(z) = \frac{1}{(z - z_0)^m} \frac{1}{g(z)} = \sum_{n=0}^{\infty} c_n (z - z_0)^{n - m} = \frac{c_0}{(z - z_0)^m} + \cdots + \frac{c_{m - 1}}{z - z_0} + c_m + c_{m + 1}(z - z_0) + \cdots.
\end{align*}
记 \( a_n = c_{n + m}, n = -m, \cdots, -1, 0, 1, \cdots \),即得展开式\(\eqref{eq---:::213907---2}\).

反之,如果 \( f \) 在 \( z_0 \) 附近的Laurent展开式为\(\eqref{eq---:::213907---2}\)式,那么
\begin{align*}
(z - z_0)^m f(z) = a_{-m} + a_{-(m - 1)}(z - z_0) + \cdots + a_0(z - z_0)^m + \cdots.
\end{align*}
若记上式右端的幂级数为 \( \varphi(z) \),则 \( \varphi \) 在 \( z_0 \) 处全纯,且 \( \varphi(z_0) = a_{-m} \neq 0 \).因而 \( \frac{1}{\varphi} \) 也在 \( z_0 \) 处全纯,于是
\[
\frac{1}{f(z)} = (z - z_0)^m \frac{1}{\varphi(z)}
\]
在 \( z_0 \) 附近成立.由\refthe{theorem:复变函数-----定理4.3.4},\( z_0 \) 是 \( \frac{1}{f} \) 的 \( m \) 阶零点,所以是 \( f \) 的 \( m \) 阶极点.

\item 若\( z_0 \) 为 \( f \) 的 \( m \) 阶极点,则由\rrefthe{theorem:m阶极点的充要条件}{theorem:m阶极点的充要条件-1}知\( f \) 在 \( z_0 \) 的某个去心邻域内的Laurent展开式为
\begin{align*}
f(z) = \frac{a_{-m}}{(z - z_0)^m} + \cdots + \frac{a_{-1}}{z - z_0} + a_0 + a_1(z - z_0) + \cdots,
\end{align*}
其中\( a_{-m} \neq 0 \).令
\begin{align*}
g(z) = a_{-m} + a_{-(m - 1)}(z - z_0) + \cdots + a_0(z - z_0)^m + \cdots.
\end{align*}
则\( g \) 在 \( z_0 \) 点全纯,且 \( g(z_0) \neq 0 \).并且
\begin{align*}
f(z) = \frac{g(z)}{(z-z_0)^m}.
\end{align*}

反之,若\eqref{eq::0jn2fgj4gfj2-wof8n23}式成立,则
\begin{align*}
\frac{1}{f(z)}=\frac{(z-z_0)^m}{g(z)}
\end{align*}
在$z_0$处全纯,直接计算即知$z_0$是$\frac{1}{f(z)}$的$m$阶零点,即\( z_0 \) 为 \( f \) 的 \( m \) 阶极点.
\end{enumerate}

\end{proof}

\begin{theorem}
\begin{enumerate}[(1)]
\item 函数$f(z)$的孤立奇点$a$为本质奇点的充要条件是$f(z)$在$z=a$的主要部分有无穷多项负幂不等于零.即存在$a$的某个去心邻域$D=\{z\in \mathbb{C} :0<\left| z-a \right|<r\}(r\in \mathbb{R} )$,使得\( f \)在$D$内可Laurent展开为
\begin{align*}
f(z)=\sum_{n=-\infty}^{\infty}{a_n z^n},\ z\in D.
\end{align*}
其中
\begin{align*}
a_n=\frac{1}{2\pi \mathrm{i}}\int_{\gamma _{\rho}}{\frac{f\left( \zeta \right)}{\zeta^{n+1}}\mathrm{d}\zeta},\,\,n=0,\pm 1,\cdots .
\end{align*}
而$\gamma _{\rho}=\left\{ \zeta :\left| \zeta \right|=\rho \right\}(0<\rho <R) .$并且存在子列$\{n_k\}\subseteq \mathbb{N}$,使得$a_{n_k}\neq 0,\forall k\in \mathbb{N}.$

\item 若$z=a\in \mathbb{C}$为函数$f(z)$的一本质奇点,且在点$a$的充分小去心邻域内不为零,则$z=a$亦必为$\frac{1}{f(z)}$的本质奇点.
\end{enumerate}
\end{theorem}
\begin{proof}
\begin{enumerate}[(1)]
\item 
    
\item 令$\varphi(z)=\frac{1}{f(z)}$.由假设,$z=a$必为$\varphi(z)$的孤立奇点.若$z=a$为$\varphi(z)$的可去奇点或零点,则$z=a$必为$f(z)$的可去奇点或极点,此与假设矛盾!若$z=a$为$\varphi(z)$的极点,则$z=a$必为$f(z)$的零点,亦与假设矛盾.故$z=a$必为$\varphi(z)$的本质奇点.
\end{enumerate}

\end{proof}

\begin{definition}
如果 \( f \) 在无穷远点的邻域(不包括无穷远点) \( \{ z: 0 \leqslant R < |z| < +\infty \} \) 中全纯,就称 \( \infty \) 是 \( f \) 的\textbf{孤立奇点}.
\end{definition}
\begin{remark}
在这种情形下,作变换 \( z = \frac{1}{\zeta} \),记
\[
g(\zeta) = f\left( \frac{1}{\zeta} \right),
\]
则 \( g \) 在 \( 0 < |\zeta| < \frac{1}{R} \) 中全纯,即 \( \zeta = 0 \) 是 \( g \) 的孤立奇点.
\end{remark}

\begin{definition}
设$g(\zeta) = f\left( \frac{1}{\zeta} \right),$如果 \( \zeta = 0 \) 是 \( g \) 的可去奇点、\( m \) 阶极点或本性奇点,那么我们相应地称 \( z = \infty \) 是 \( f \) 的\textbf{可去奇点}、\( m \) \textbf{阶极点}或\textbf{本性(质)奇点}.
\end{definition}

\begin{theorem}
设$f$在$\infty$的某邻域$N\setminus \{\infty\}=\{z:R<|z|<+\infty\}$内全纯,则\( f \)在$N\setminus \{\infty\}$内有下面的Laurent展开:
\[
f(z) = \sum_{n = -\infty}^{\infty} b_n z^n,\,\,z\in N\setminus \{\infty\}.
\]
其中
\begin{align*}
b_n=\frac{1}{2\pi \mathrm{i}}\int_{\gamma _{\rho}}{\frac{f(\zeta )}{\zeta^{-n+1}}\mathrm{d}\zeta},\,\,n=0,\pm 1,\cdots .
\end{align*}
而 \( \gamma_{\rho} = \{ \zeta: |\zeta| = \rho \} \)\( (r < \rho < R )\).

这时,我们称 \( \sum_{n = 1}^{\infty} b_n z^n \) 为 \( f \) 的\textbf{主要部分}, \( \sum_{n = 0}^{\infty} b_{-n} z^{-n} \) 为 \( f \) 的\textbf{全纯部分}.
\end{theorem}
\begin{proof}
令$g\left( z \right) =f\left( \frac{1}{z} \right) $,则$g\in H(\{z:0<|z|<\frac{1}{R}\})$从而由\refthe{theorem:定理5.1.2}知\( g \) 在原点的邻域$\{z:0<|z|<\frac{1}{R}\}$中有Laurent展开:
\[
g(z) = \sum_{n = -\infty}^{\infty} a_n z^n, \,\, 0 < |z| < \frac{1}{R},
\]
其中
\begin{align*}
a_n = \frac{1}{2\pi \mathrm{i}} \int_{\gamma_{\rho}} \frac{f(\zeta)}{\zeta^{n + 1}} \mathrm{d}\zeta,\,\,n\in \mathbb{Z},
\end{align*}
而 \( \gamma_{\rho} = \{ \zeta: |\zeta| = \rho \} \)\(( r < \rho < R) \).于是\( f \) 在 \( N\setminus \{\infty\}\) 中有下面的Laurent展开:
\[
f(z) =g\left( \frac{1}{z} \right) =\sum_{n = -\infty}^{\infty} a_n \left(\frac{1}{z}\right)^n= \sum_{n = -\infty}^{\infty} b_n z^n,
\]
其中\( b_n = a_{-n}, n = 0, \pm 1, \cdots \),即
\begin{align*}
b_n=\frac{1}{2\pi \mathrm{i}}\int_{\gamma _{\rho}}{\frac{f(\zeta )}{\zeta^{-n+1}}\mathrm{d}\zeta},\,\,n=0,\pm 1,\cdots .
\end{align*}

\end{proof}

\begin{theorem}\label{theorem:无穷远点为可去奇点的充要条件}
函数$f(z)$的孤立奇点$z=\infty$为可去奇点的充要条件是下列三条中的任何一条成立:
\begin{enumerate}[(1)]
\item\label{theorem:无穷远点为可去奇点的充要条件-1} $f(z)$在$z=\infty$的主要部分为零.即存在$\infty$的某个去心邻域$N\setminus \{\infty\}=\{z:R<|z|<+\infty\}(R\in \mathbb{R})$,使得\( f \)在$N\setminus \{\infty\}$内可Laurent展开为一个幂级数
\begin{align*}
f(z)=\sum_{n=0}^{\infty}{b_n z^n},\ z\in N\setminus\{\infty\}.
\end{align*}
其中
\begin{align*}
b_n=\frac{1}{2\pi \mathrm{i}}\int_{\gamma _{\rho}}{\frac{f\left( \zeta \right)}{\zeta^{-n+1}}\mathrm{d}\zeta},\,\,n=0,1,\cdots .
\end{align*}
且$0<\rho <R,\gamma _{\rho}=\left\{ \zeta :\left| \zeta \right|=\rho \right\} .$

\item\label{theorem:无穷远点为可去奇点的充要条件-2} $\lim\limits_{z \to \infty}f(z)=b\neq\infty$.

\item\label{theorem:无穷远点为可去奇点的充要条件-3} 存在$\infty$的某个去心邻域$N\setminus \{\infty\}=\{z:R<|z|<+\infty\}(R\in \mathbb{R})$,使得\( f \)在$N\setminus \{\infty\}$内有界.
\end{enumerate}
\end{theorem}
\begin{proof}
令$g(z)=f\left( \frac{1}{z} \right) $,再利用\hyperref[theorem:Riemann定理-定理5.2.1]{Riemann定理}易证.

\end{proof}

\begin{theorem}\label{theorem:无穷远点为m阶极点的充要条件}
函数$f(z)$的孤立奇点$z=\infty$为$m$阶极点的充要条件是下列三条中的任何一条成立:
\begin{enumerate}[(1)]
\item\label{theorem:无穷远点为m阶极点的充要条件-1} 存在$\infty$的某个去心邻域$N\setminus \{\infty\}=\{z:R<|z|<+\infty\}(R\in \mathbb{R})$,使得\( f \)在$N\setminus \{\infty\}$内有下面的Laurent展开式:
\begin{align*}
f(z) = b_m z^m + \cdots + b_1 z + b_0 + b_{-1} z^{-1} + \cdots,\,\,z\in N\setminus \{\infty\}.
\end{align*}
其中$b_m\neq 0$,并且
\begin{align*}
b_n=\frac{1}{2\pi \mathrm{i}}\int_{\gamma _{\rho}}{\frac{f\left( \zeta \right)}{\zeta^{-n+1}}\mathrm{d}\zeta},\,\,n=\cdots,-1,0,1,\cdots,m .
\end{align*}
而$\gamma _{\rho}=\left\{ \zeta :\left| \zeta \right|=\rho \right\}(0<\rho <R) .$

\item\label{theorem:无穷远点为m阶极点的充要条件-2} 存在$\infty$的某个去心邻域$N\setminus \{\infty\}=\{z:R<|z|<+\infty\}(R\in \mathbb{R})$,使得\( f \)在$N\setminus \{\infty\}$内能表示成
\begin{align*}
f(z)=z^m h(z),
\end{align*}
其中$h(z)$在$z=\infty$的邻域$N$内解析,且$h(\infty)\neq0$.

\item\label{theorem:无穷远点为m阶极点的充要条件-3} $g(z)=\dfrac{1}{f(z)}$以$z=\infty$为$m$阶零点.
\end{enumerate}
\end{theorem}
\begin{proof}
令$g(z)=f\left( \frac{1}{z} \right) $,再利用\refthe{theorem:m阶极点的充要条件}易证.

\end{proof}

\begin{theorem}
函数$f(z)$的孤立奇点$\infty$为极点的充要条件是$\lim\limits_{z \to \infty}f(z)=\infty$.
\end{theorem}
\begin{proof}
令$g(z)=f\left( \frac{1}{z} \right) $,再利用\refthe{theorem:复变函数----定理5.2.2}易证.

\end{proof}

\begin{theorem}
函数$f(z)$的孤立奇点$\infty$为本质奇点的充要条件是下列两条中的任何一条成立:
\begin{enumerate}[(1)]
\item $f(z)$在$z=\infty$的主要部分有无穷多项正幂不等于零.即存在$\infty$的某个去心邻域$N\setminus \{\infty\}=\{z\in \mathbb{C}:r<|z|<+\infty\}(r\in \mathbb{R})$,使得\( f \)在$N\setminus \{\infty\}$内可Laurent展开为
\begin{align*}
f(z)=\sum_{n=-\infty}^{\infty}{b_n z^n},\ z\in N\setminus\{\infty\}.
\end{align*}
其中
\begin{align*}
b_n=\frac{1}{2\pi \mathrm{i}}\int_{\gamma _{\rho}}{\frac{f\left( \zeta \right)}{\zeta^{-n+1}}\mathrm{d}\zeta},\,\,n=0,\pm 1,\cdots .
\end{align*}
而$\gamma _{\rho}=\left\{ \zeta :\left| \zeta \right|=\rho \right\}(0<\rho <R) .$并且存在子列$\{n_k\}\subseteq \mathbb{N}$,使得$b_{n_k}\neq 0,\forall k\in \mathbb{N}.$

\item $\lim\limits_{z \to \infty}f(z)$不存在(即当$z$趋向于$\infty$时,$f(z)$不趋向于任何(有限或无穷)极限).
\end{enumerate}
\end{theorem}
\begin{proof}


\end{proof}

\begin{theorem}\label{theorem:零点的聚点是本性奇点}
若函数$f(z)$在$0<|z-a|<R$内解析,且不恒为零;又若$f(z)$有一列异于$a$但却以$a$为聚点的零点,则$a$必为$f(z)$的本质奇点.
\end{theorem}
\begin{proof}
$z=a$必是$f(z)$的孤立奇点且不能是可去奇点.否则$f(z)$于$|z-a|<R$内解析(令$f(a)=0$)且以$a$为非孤立的零点.由\refcor{corollary:复变函数---推论4.20}必有$f(z)$恒为零,这与假设矛盾.

其次,$z=a$也不能是$f(z)$的极点.否则,对任给$M>0$,有$\delta>0$,使当$0<|z-a|<\delta$时,$|f(z)|>M$,也与假设矛盾.

故$z=a$必为$f(z)$的本质奇点.

\end{proof}

\begin{theorem}[Weierstrass定理]\label{theorem:Weierstrass-定理5.2.5}
设 \( z_0 \) 是 \( f \) 的本性奇点,那么对任意 \( A \in \mathbb{C}_{\infty} \),必存在趋于 \( z_0 \) 的点列 \( \{ z_n \} \),使得 \( \lim\limits_{n \to \infty} f(z_n) = A \).
\end{theorem}
\begin{remark}
\( f \) 在本性奇点处的特征是Laurent展开式的主要部分有无穷多项.实际上,这个定理证明了更深刻的结果.
\end{remark}
\begin{proof}
先设 \( A = \infty \).因为 \( z_0 \) 是 \( f \) 的本性奇点,故 \( f \) 在 \( z_0 \) 附近无界.于是对任意自然数 \( n \),总能找到 \( z_n \),使得 \( |z_n - z_0| < \frac{1}{n} \),但 \( |f(z_n)| > n \),这说明 \( \lim\limits_{n \to \infty} f(z_n) = \infty \).

再设 \( A \) 是一个有限数.令 \( \varphi(z) = \frac{1}{f(z) - A} \),我们证明 \( \varphi \) 在 \( z_0 \) 的邻域中无界.不然的话,\( z_0 \) 是 \( \varphi \) 的可去奇点,适当选择 \( \varphi(z_0) \) 的值,可使 \( \varphi \) 在 \( z_0 \) 处全纯.如果 \( \varphi(z_0) \neq 0 \),则因 \( f(z) = \frac{1}{\varphi(z)} + A \),\( f \) 也在 \( z_0 \) 处全纯,这不可能.故必有 \( \varphi(z_0) = 0 \),由\refthe{theorem:复变函数----定理5.2.2}可知 \( z_0 \) 是 \( f(z) -A\) 的极点,也不可能.所以,\( \varphi \) 在 \( z_0 \) 的邻域中无界.于是,对任意自然数 \( n \),存在 \( z_n \),使得 \( |z_n - z_0| < \frac{1}{n} \),但 \( \frac{1}{|f(z) - A|} > n \),即 \( |f(z) - A| < \frac{1}{n} \).这就证明了 \( \lim\limits_{n \to \infty} f(z_n) = A \).

\end{proof}

\begin{theorem}[Picard定理]\label{ theorem:Picard定理-定理5.2.6}
每个非常数整函数都取遍每一个复数值,至多有一个例外.

此外,函数在任何本性奇点的邻域内,都可以无穷多次地取到每个有穷复值,最多只有一个例外.即若$a\in \mathbb{C}_{\infty}$为函数$f$的本性奇点,则对于每一个复数$A\neq \infty$,除掉一个可能值$A=A_0$外,必有趋于$a$的无限点列$\{z_n\}$,使$f(z_n)=A(n=1,2,\cdots)$.
\end{theorem}
\begin{remark}
例如,考虑函数 \( f(z) = \mathrm{e}^{\frac{1}{z}} \),它在 \( z = 0 \) 附近是全纯的.若让 \( z \) 沿着 \( x \) 轴分别从 \( 0 \) 的左边和右边趋于 \( 0 \),可得
\[
\lim_{\substack{z = x \to 0^-}} \mathrm{e}^{\frac{1}{z}} = \lim_{x \to 0^-} \mathrm{e}^{\frac{1}{x}} = 0,
\]
\[
\lim_{\substack{z = x \to 0^+}} \mathrm{e}^{\frac{1}{z}} = \lim_{x \to 0^+} \mathrm{e}^{\frac{1}{x}} = \infty.
\]
这说明 \( \lim_{z \to 0} \mathrm{e}^{\frac{1}{z}} \) 不存在,所以 \( z = 0 \) 是 \( \mathrm{e}^{\frac{1}{z}} \) 的本性奇点.对于任意复数 \( a \neq 0 \),若取 \( z_n = (\ln a + 2n\pi \mathrm{i})^{-1} \),则 \( f(z_n) = \mathrm{e}^{\ln a + 2n\pi \mathrm{i}} = a \).由于 \( z_n \to 0 \),这说明 \( \mathrm{e}^{\frac{1}{z}} \) 在 \( z = 0 \) 的邻域中可以无穷多次地取到非零值 \( a \),但 \( 0 \) 是它的唯一的例外值.
\end{remark}

\begin{proposition}\label{proposition:函数和积商的零点}
设$z_0$是函数$f(z)$的$m$阶零点,又是$g(z)$的$n$阶零点,则
\begin{enumerate}[(1)]
\item 当$m>n$时,$z_0$是$f(z)\pm  g(z)$的$n$阶零点;

当$m<n$时,$z_0$是$f(z)\pm  g(z)$的$m$阶极点;

当$m=n$时,若$f^{\left( m \right)}\left( z_0 \right) \pm g^{\left( m \right)}\left( z_0 \right) \neq0$,则$z_0$是$f(z)\pm  g(z)$的$m$阶零点;

若$f^{\left( m \right)}\left( z_0 \right) \pm g^{\left( m \right)}\left( z_0 \right)  = 0$,则$z_0$是$f(z)\pm g(z)$大于$m$阶的零点.

\item $z_0$为$f(z)\cdot g(z)$的$m+n$阶零点.

\item 当$m>n$时,$z_0$是$\frac{f(z)}{g(z)}$的$m-n$阶零点;

当$m<n$时,$z_0$是$\frac{f(z)}{g(z)}$的$n-m$阶极点;

当$m=n$时,$z_0$是$\frac{f(z)}{g(z)}$的可去奇点.
\end{enumerate}
\end{proposition}
\begin{proof}
因为$z_0$为$f(z)$的$m$阶零点,又是$g(z)$的$n$阶零点,所以由\refthe{theorem:定理4.3.1}知
\begin{align}\label{eq::0jr3rseegw3464ugaez}
\begin{gathered}
f(z)=a_m(z-z_0)^m+a_{m+1}(z-z_0)^{m+1}+\cdots
\\
g(z)=b_n(z-z_0)^n+b_{n+1}(z-z_0)^{n+1}+\cdots,
\end{gathered}
\end{align}
其中$a_m=\frac{f^{\left( m \right)}\left( z_0 \right)}{m!}\neq0$,$b_n=\frac{g^{\left( n \right)}\left( z_0 \right)}{n!}\neq0$.
\begin{enumerate}[(1)]
\item 如果$m>n$,那么由\eqref{eq::0jr3rseegw3464ugaez}式可得
\begin{align*}
f(z)\pm g(z)=(z-z_0)^n\left[\pm b_n\pm b_{n+1}(z-z_0)\pm \cdots\pm (a_m\pm b_m)(z-z_0)^{m-n}+\cdots\right],
\end{align*}
从而$z_0$为$f(z)\pm g(z)$的$n$阶零点.

如果$n>m$,那么同理可得$z_0$为$f(z)\pm g(z)$的$m$阶零点.

如果$m=n$,当$f^{\left( m \right)}\left( z_0 \right) \pm g^{\left( m \right)}\left( z_0 \right) \neq0$时,由\eqref{eq::0jr3rseegw3464ugaez}式可得
\begin{align*}
f(z)\pm g(z)=(z-z_0)^m\left[ \left( a_m\pm b_m \right) +\left( a_{m+1}\pm b_{m+1} \right) (z-z_0)+\cdots \right] ,
\end{align*}
从而此时$z_0$为$f(z)\pm g(z)$的$m$阶零点;当$f^{\left( m \right)}\left( z_0 \right) \pm g^{\left( m \right)}\left( z_0 \right) =0$时,此时零点$z_0$的阶数大于$m$.

\item 由\eqref{eq::0jr3rseegw3464ugaez}式可得
\begin{align*}
f(z)\cdot g(z)=a_mb_n(z-z_0)^{m+n}+(a_mb_{n+1}+a_{m+1}b_n)(z-z_0)^{m+n+1}+\cdots,
\end{align*}
故$z_0$为$f(z)\cdot g(z)$的$m+n$阶零点.

\item 由\eqref{eq::0jr3rseegw3464ugaez}式可得
\begin{align*}
\frac{f(z)}{g(z)}=\frac{(z-z_0)^m(a_m+a_{m+1}(z-z_0)+\cdots)}{(z-z_0)^n(b_n+b_{n+1}(z-z_0)+\cdots)}=(z-z_0)^{m-n}\frac{a_m+a_{m+1}(z-z_0)+\cdots}{b_n+b_{n+1}(z-z_0)+\cdots}.
\end{align*}

当$m>n$时,$z_0$为$\frac{f(z)}{g(z)}$的$m-n$阶零点.

当$m<n$时,$z_0$为$\frac{f(z)}{g(z)}$的$n-m$阶极点.

当$m=n$时,$z_0$为$\frac{f(z)}{g(z)}$的可去奇点.
\end{enumerate}

\end{proof}

\begin{proposition}\label{proposition:函数和积商的极点}
函数$f(z),g(z)$分别以$z=a$为$m$阶极点及$n$阶极点.则
\begin{enumerate}[(1)]
\item 当$m\neq n$时,$z=a$是$f(z)\pm g(z)$的$\max(m,n)$阶极点;

当$m=n$时,若$\left[ f(z)\left( z-a \right) ^m\pm g(z)\left( z-a \right) ^n \right] _{z=a}\neq0$,则$z=a$是$f(z)\pm g(z)$的低于$n$阶极点或可去奇点或高于$0$阶零点;

若$\left[ f(z)\left( z-a \right) ^m\pm g(z)\left( z-a \right) ^n \right] _{z=a}=0$,则$z=a$是$f(z)\pm g(z)$的低于$n$阶极点或可去奇点.

\item $z=a$是$f(z)g(z)$的$m+n$阶极点.

\item 当$m>n$时,$z=a$为$\frac{f(z)}{g(z)}$的$m-n$阶极点.

当$m<n$时,$z=a$为$\frac{f(z)}{g(z)}$的$n-m$阶零点.

当$m=n$时,$z=a$为$\frac{f(z)}{g(z)}$的可去奇点.
\end{enumerate}
\end{proposition}
\begin{proof}
因为$z=a$是$f(z)$与$g(z)$的$m$级与$n$级极点,所以由\rrefthe{theorem:m阶极点的充要条件}{theorem:m阶极点的充要条件-2}知
\begin{align}\label{eq::--sdfjon23fm2iocjwe}
f(z)=\frac{f_1(z)}{(z-a)^m},\quad g(z)=\frac{g_1(z)}{(z-a)^n},
\end{align}
其中$f_1(z)$与$g_1(z)$在$z=a$全纯,且$f_1(a)\neq0,g_1(a)\neq0$.并且$\left[ f(z)\left( z-a \right) ^m+g(z)\left( z-a \right) ^n \right] _{z=a}=f_1(a)+g_1(a).$
\begin{enumerate}[(1)]
\item 由\eqref{eq::--sdfjon23fm2iocjwe}可得
\begin{align*}
f(z)\pm g(z)=
\begin{cases}
\frac{f_1(z)\pm (z-a)^{m-n}g_1(z)}{(z-a)^m},&m>n,\\
\frac{(z-a)^{n-m}f_1(z)\pm g_1(z)}{(z-a)^n},&n>m,\\
\frac{f_1(z)\pm g_1(z)}{(z-a)^n},&m=n.
\end{cases}
\end{align*}

其中当$m>n$时,将$z=a$代入分子中得$f_1(a)\neq0$.当$n>m$时,将$z=a$代入分子中得$g_1(a)\neq0$.当$m=n$时,将$z=a$代入分子中得$f_1(a)\pm g_1(a)$,各个分子显然在$z=a$是全纯的,所以有以下结论:

当$m\neq n$时,点$a$是$f(z)\pm g(z)$的$\max(m,n)$阶极点;当$m=n$时,若$f_1(a)\pm g_1(a)\neq0$,点$a$是$f(z)\pm g(z)$的$n$阶极点;若$f_1(a)\pm g_1(a)=0$,设$a$是$f_1(z)\pm g_1(z)$的$k$阶零点,则由\refthe{theorem:复变函数-----定理4.3.4}知\( f_1(z)\pm g_1(z) \) 在 \( a \) 的邻域内可以表示为
\begin{align*}
f_1(z)\pm g_1(z) = (z - a)^k h(z),
\end{align*}
其中$h$在$a$点全纯且$h(a)\neq 0$.

从而此时
\begin{align*}
f(z)\pm g(z)=\frac{(z-a)^kh(z)}{(z-a)^n}=\begin{cases}
\frac{h(z)}{(z-a)^{n-k}},\quad &k<n,\\
(z-a)^{k-n}h(z),\quad &k\geqslant n.\\
\end{cases}
\end{align*}
因此当$k<n$时,$a$是$f(z)\pm g(z)$的$n-k$阶极点;当$k=n$时,$a$是$f(z)\pm g(z)$的可去奇点;当$k>n$时,$a$是$f(z)\pm g(z)$的$k-n$阶零点.
故此时点$a$是$f(z)\pm g(z)$的低于$n$阶极点或可去奇点或高于$0$阶零点.

\item 由\eqref{eq::--sdfjon23fm2iocjwe}可得$f(z)\cdot g(z)=\frac{f_1(z)g_1(z)}{(z-a)^{m+n}}$,因为$f_1(z)g_1(z)$在$z=a$解析,且$f_1(a)g_1(a)\neq0$,所以$z=a$是$f(z)g(z)$的$m+n$阶极点.

\item 由\eqref{eq::--sdfjon23fm2iocjwe}可得
\begin{align*}
\frac{f(z)}{g(z)}=
\begin{cases}
\frac{1}{(z-a)^{m-n}}\cdot\frac{f_1(z)}{g_1(z)},&m>n,\quad a\text{是}m-n\text{阶极点},\\
(z-a)^{n-m}\cdot\frac{f_1(z)}{g_1(z)},&m<n,\quad a\text{是}n-m\text{阶零点},\\
\frac{f_1(z)}{g_1(z)},&m=n,\quad a\text{是可去奇点}.
\end{cases}
\end{align*}
\end{enumerate}

\end{proof}

\begin{proposition}\label{proposition:本性奇点的积和差商仍是本性奇点}
设函数$f(z)$不恒为零且以$z=a$为解析点或极点,而函数$\varphi(z)$以$z=a$为本性奇点,则$z=a$是$\varphi(z)\pm f(z),\varphi(z)\cdot f(z)$及$\frac{\varphi(z)}{f(z)}$的本性奇点.
\end{proposition}
\begin{proof}
反证,如果$z=a$不是$\varphi(z)\pm f(z),\varphi(z)\cdot f(z)$及$\frac{\varphi(z)}{f(z)}$的本性奇点,则
\begin{align*}
\varphi (z)=\left[ \varphi (z)\pm f(z) \right] \mp f\left( x \right) ,\quad \varphi (z)=\frac{\varphi (z)}{f(z)}\cdot f\left( x \right) \quad ,\varphi (z)=\frac{\varphi (z)\cdot f(z)}{f\left( x \right)}.
\end{align*}
由\refpro{proposition:函数和积商的零点}和\refpro{proposition:函数和积商的极点}知$\varphi(z)$就以$z=a$为可去奇点或极点或零点,这与题设矛盾.

\end{proof}




















\end{document}
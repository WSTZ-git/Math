\documentclass[../../main.tex]{subfiles}
\graphicspath{{\subfix{../../image/}}} % 指定图片目录,后续可以直接使用图片文件名。

% 例如:
% \begin{figure}[H]
% \centering
% \includegraphics[scale=0.4]{图.png}
% \caption{}
% \label{figure:图}
% \end{figure}
% 注意:上述\label{}一定要放在\caption{}之后,否则引用图片序号会只会显示??.

\begin{document}

\section{整函数与亚纯函数}

\begin{theorem}\label{theorem:定理5.3.1}
如果整函数$f$在无穷远处全纯或无穷远点是$f$的可去奇点,那么$f$一定是常数.
\end{theorem}
\begin{proof}
由于\( f \) 在整个复平面 \( \mathbb{C} \) 上全纯,即 \( f \) 为整函数,则\( f \) 在 \( \mathbb{C} \) 上有Taylor展开式
\begin{align}
f(z) = \sum_{n = 0}^{\infty} a_n z^n. \label{equation::::::::::11111}
\end{align}
它当然在 \( R < |z| < \infty \) 中也成立,因此也可把它看成是无穷远点邻域中的Laurent展开式.

如果整函数 \( f \) 在 \( \infty \) 处全纯或无穷远点为$f$的可去奇点,那么根据\eqref{eq:::--1232-3},它在 \( \infty \) 处邻域中的Laurent展开式除去常数项外只有负次幂的项,因此在展开式\(\eqref{equation::::::::::11111}\)中必须有
\[
a_1 = a_2 = \cdots = 0,
\]
所以 \( f \) 是一常数.

\end{proof}

\begin{theorem}\label{theorem:定理5.3.2}
如果无穷远点是整函数 \( f \) 的一个 \( m \) 阶极点,那么 \( f \) 是一个 \( m \) 次多项式.
\end{theorem}
\begin{proof}
如果无穷远点是整函数 \( f \) 的一个 \( m \) 阶极点,那么根据\eqref{eq:::--1232-4},它在无穷远点邻域中的Laurent展开式除去一个 \( m \) 次多项式外只有负次幂的项,因此在展开式\(\eqref{equation::::::::::11111}\)中必须有
\[
a_{m + 1} = a_{m + 2} = \cdots = 0,
\]
所以 \( f \) 是一个 \( m \) 次多项式.

\end{proof}

\begin{definition}
不是常数和多项式的整函数称为\textbf{超越整函数}.
\end{definition}
\begin{remark}
如 \( \mathrm{e}^z,\sin z,\cos z \) 等,都是超越整函数.
\end{remark}

\begin{proposition}
无穷远点一定是超越整函数的本性奇点.
\end{proposition}
\begin{proof}
反证,设$f$是超越整函数,若无穷远点不是$f$的本性奇点,则当$f$在无穷远点全纯或无穷远点是$f$的可去奇点时,由\refthe{theorem:定理5.3.1}可知$f$为一个常数.当无穷远点是$f$的$m$阶极点时,由\refthe{theorem:定理5.3.2}可知$f$是一个$m$次多项式.这与$f$是超越整函数矛盾!

\end{proof}


\begin{definition}
如果 \( f \) 在整个复平面 \( \mathbb{C} \) 上除去极点外没有其他的奇点,就称 \( f \) 是一个\textbf{亚纯函数}.
\end{definition}

\begin{proposition}
(1)整函数是亚纯函数.

(2)有理函数
\[
f(z) = \frac{P_n(z)}{Q_m(z)}
\]
也是亚纯函数,这里,\( P_n \) 和 \( Q_m \) 是两个既约的多项式.
\end{proposition}
\begin{proof}


\end{proof}

\begin{proposition}
无穷远点或是有理函数的可去奇点,或是有理函数的极点.
\end{proposition}
\begin{proof}
若记
\[
P_n(z) = a_0 + a_1 z + \cdots + a_n z^n, \, a_n \neq 0,
\]
\[
Q_m(z) = b_0 + b_1 z + \cdots + b_m z^m, \, b_m \neq 0,
\]
那么
\[
f(z) = \frac{P_n(z)}{Q_m(z)} = \frac{1}{z^{m - n}} \frac{a_n + a_{n - 1} \frac{1}{z} + \cdots + a_0 \frac{1}{z^n}}{b_m + b_{m - 1} \frac{1}{z} + \cdots + b_0 \frac{1}{z^m}},
\]
所以
\[
\lim_{z \to \infty} f(z) = \begin{cases} 
\dfrac{a_n}{b_m}, & n = m; \\
\infty, & n > m; \\
0, & n < m.
\end{cases}
\]
这说明 \( z = \infty \) 或是 \( f \) 的可去奇点,或是 \( f \) 的极点.

\end{proof}

\begin{theorem}\label{theorem:定理5.3.3}
若 \( z = \infty \) 是亚纯函数 \( f \) 的可去奇点或极点,则 \( f \) 一定是有理函数.
\end{theorem}
\begin{proof}
因 \( z = \infty \) 是 \( f \) 的可去奇点或极点,故必存在 \( R > 0 \),使得 \( f \) 在 \( R < |z| < \infty \) 中全纯.在 \( |z| \leqslant R \) 中,\( f \) 最多只能有有限个极点.因若有无穷多个极点 \( z_j, j = 1, 2, \cdots \),则 \( \{ z_j \} \) 必有收敛的子列 \( \{ z_{k_j} \} \),设其极限为 \( a \),则 \( |a| \leqslant R \),显然若$a$是极点,则\( a \)不是孤立奇点,矛盾!若$a$不是极点,则由$f$是亚纯函数可知$f$在$a$处全纯,但$f$在$a$的任意邻域内必有极点,矛盾!今设 \( z_1, \cdots, z_n \) 为 \( f \) 在 \( |z| \leqslant R \) 中的有限个极点,它们的阶分别为 \( m_1, \cdots, m_n \).由\refthe{theorem:定理5.2.4}可知\( f \) 在 \( z_j (j = 1, \cdots, n) \) 附近的Laurent展开的主要部分为
\[
h_j(z) = \frac{c_{-1}^{(j)}}{z - z_j} + \frac{c_{-2}^{(j)}}{(z - z_j)^2} + \cdots + \frac{c_{-m_j}^{(j)}}{(z - z_j)^{m_j}}.
\]
设 \( f \) 在 \( \infty \) 的邻域内的Laurent展开的主要部分为 \( g \),由\refpro{proposition:无穷远点为孤立奇点时的Laurent展开}可知,当 \( z = \infty \) 是 \( f \) 的极点时,\( g \) 是一个多项式;当 \( z = \infty \) 是 \( f \) 的可去奇点时,\( g \equiv 0 \).令
\[
F(z) = f(z) - h_1(z) - \cdots - h_n(z) - g(z),
\]
显然,\( F \) 在 \( \mathbb{C}_{\infty} \) 中除 \( z_1, \cdots, z_n \) 和 \( \infty \) 外是全纯的,而在 \( z_1, \cdots, z_n \) 和 \( \infty \) 这些点上,\( f \) 的主要部分都已经被消去,因而也是全纯的.所以,\( F \) 是 \( \mathbb{C}_{\infty} \) 上的全纯函数,因而由\refthe{theorem:定理5.3.1},\( F \) 是一个常数 \( c \).于是
\[
f(z) = c + g(z) + \sum_{j = 1}^{n} h_j(z),
\]
所以 \( f \) 是有理函数.

\end{proof}
\begin{remark}
这里,我们顺便得到了这样一个结论:\textbf{任何有理函数一定能分解成部分分式之和,而且这种分解是唯一的.}这个结论在计算有理函数的不定积分时已经被多次用过.
\end{remark}

\begin{theorem}\label{theorem:定理5.3.4}
\( \mathrm{Aut}(\mathbb{C}) \) 由所有的一次多项式组成.
\end{theorem}
\begin{proof}
设 \( f(z) = az + b, a \neq 0 \),则显然 \( f \in \mathrm{Aut}(\mathbb{C}) \).反之,对于任意的 \( f \in \mathrm{Aut}(\mathbb{C}) \),因为 \( f \) 是整函数,如果 \( \infty \) 是它的可去奇点,则由\refthe{theorem:定理5.3.1},\( f \) 是一个常数,这不可能.如果 \( \infty \) 是 \( f \) 的本性奇点,则由\refthe{theorem:Weierstrass-定理5.2.5},对于任意 \( A \in \mathbb{C} \),必有 \( z_n \to \infty \),使得 \( \lim\limits_{n \to \infty} f(z_n) = A \).现在记 \( f(z_n) = w_n \),则 \( z_n = f^{-1}(w_n) \),两端令 \( n \to \infty \),即得 \( f^{-1}(A) = \infty \).这说明 \( A \) 是 \( f^{-1} \) 的一个极点,与 \( f^{-1} \) 是整函数相矛盾.由此可知 \( \infty \) 必为 \( f \) 的极点,由\refthe{theorem:定理5.3.2}知道,\( f \) 是一个多项式.又因为 \( f \) 在 \( \mathbb{C} \) 上是单叶的,所以 \( f \) 只能是一次多项式.

\end{proof}

\begin{theorem}\label{theorem:定理5.3.5}
\( \mathrm{Aut}(\mathbb{C}_{\infty}) \) 由所有的分式线性变换组成.
\end{theorem}
\begin{proof}
因为是在 \( \mathbb{C}_{\infty} \) 上讨论,\( \mathrm{Aut}(\mathbb{C}_{\infty}) \) 中的元素不再是全纯函数,而是亚纯函数.由\hyperref[section:2.2.5]{分式线性变换}的讨论知道,任何分式线性变换都是 \( \mathrm{Aut}(\mathbb{C}_{\infty}) \) 中的元素.现设 \( f \in \mathrm{Aut}(\mathbb{C}_{\infty}) \),则 \( f \) 必为亚纯函数,而且 \( \infty \) 必是 \( f \) 的可去奇点或极点.由\refthe{theorem:定理5.3.3},\( f \) 必为有理函数,再由它的单叶性,它只能是分式线性变换.

\end{proof}









\end{document}
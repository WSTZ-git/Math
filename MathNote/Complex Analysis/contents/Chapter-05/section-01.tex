\documentclass[../../main.tex]{subfiles}
\graphicspath{{\subfix{../../image/}}} % 指定图片目录,后续可以直接使用图片文件名。

% 例如:
% \begin{figure}[H]
% \centering
% \includegraphics[scale=0.4]{图.png}
% \caption{}
% \label{figure:图}
% \end{figure}
% 注意:上述\label{}一定要放在\caption{}之后,否则引用图片序号会只会显示??.

\begin{document}

\section{全纯函数的Laurent展开}

\begin{definition}
称级数
\begin{align}
\sum_{n = -\infty}^{\infty} a_n (z - z_0)^n = \sum_{n = 0}^{\infty} a_n (z - z_0)^n + \sum_{n = 1}^{\infty} a_{-n} (z - z_0)^{-n} \label{eq:1111111111111111111}
\end{align}
为 $\mathbf{Laurent}$ \textbf{级数},它由两部分组成,第一部分就是幂级数,第二部分是负幂项的级数。如果这两个级数都收敛,就称级数 \eqref{eq:1111111111111111111} 收敛。
\end{definition}

\begin{theorem}\label{theorem:定理5.1.1}
如果 Laurent 级数
\begin{align*}
\sum_{n = -\infty}^{\infty} a_n (z - z_0)^n = \sum_{n = 0}^{\infty} a_n (z - z_0)^n + \sum_{n = 1}^{\infty} a_{-n} (z - z_0)^{-n}
\end{align*}
的收敛域为圆环 \( D = \{ z: r < |z - z_0| < R \} \),那么它在 \( D \) 中绝对收敛且内闭一致收敛,它的和函数在 \( D \) 中全纯。

上述级数的幂级数部分称为该级数的\textbf{全纯部分},负幂项级数部分称为该级数的\textbf{主要部分}。
\end{theorem}
\begin{remark}
下面我们将看到,Laurent 级数的一些重要性质取决于它的主要部分。
\end{remark}
\begin{proof}
设第一个级数的收敛半径为 \( R \)。对第二个级数作变换 \( \zeta = \frac{1}{z - z_0} \),它对 \( \zeta \) 而言就是幂级数:
\begin{align*}
\sum_{n = 1}^{\infty} a_{-n} (z - z_0)^{-n} = \sum_{n = 1}^{\infty} a_{-n} \zeta^n
\end{align*}
设其收敛半径为 \( \rho \),则当 \( |\zeta| < \rho \),或者 \( |z - z_0| > \frac{1}{\rho} \) 时,上述级数收敛。记 \( r = \frac{1}{\rho} \),则当 \( r < |z - z_0| < \infty \) 时,级数 \eqref{eq:1111111111111111111} 中的负幂项级数收敛。

如果 \( R \leqslant  r \),则当 \( |z - z_0| < R \) 时,必有 \( |z - z_0| < r \),这时级数 \eqref{eq:1111111111111111111} 的第一个级数是收敛的,但第二个级数却发散了。当 \( |z - z_0| > r \) 时,必有 \( |z - z_0| > R \),这时级数 \eqref{eq:1111111111111111111} 的第二个级数收敛而第一个级数发散。所以,两者不能同时收敛。

如果 \( r < R \),则当 \( r < |z - z_0| < R \) 时,级数 \eqref{eq:1111111111111111111} 的两个级数都收敛,而且在这个圆环中内闭一致收敛,即级数 \eqref{eq:1111111111111111111} 在上述圆环中内闭一致收敛,根据 \hyperref[theorem:Weierstrass定理]{Weierstrass 定理},它的和是圆环中的全纯函数。
\end{proof}

\begin{theorem}\label{theorem:定理5.1.2}
设 \( D = \{ z: r < |z - z_0| < R \} \),如果 \( f \in H(D) \),那么 \( f \) 在 \( D \) 上可以展开为 Laurent 级数:
\begin{align}
f(z) = \sum_{n = -\infty}^{\infty} a_n (z - z_0)^n, \, z \in D, \label{eq:1111111111211111111}
\end{align}
其中,\( a_n = \frac{1}{2\pi \mathrm{i}} \int_{\gamma_{\rho}} \frac{f(\zeta)}{(\zeta - z_0)^{n + 1}} \mathrm{d}\zeta \),而 \( \gamma_{\rho} = \{ \zeta: |\zeta - z_0| = \rho \} \)(\( r < \rho < R \)),并且展开式 \eqref{eq:1111111111211111111} 是唯一的。
\end{theorem}
\begin{proof}
如\reffig{figure:图5.1} 所示,任意取定 \( z \in D \),取 \( r_1, r_2 \),使得
\[
r < r_1 < |z - z_0| < r_2 < R.
\]
记 \( \gamma_j = \{ \zeta: |\zeta - z_0| = r_j \}, j = 1, 2 \)。由\refthe{theorem:定理3.4.6},得
\begin{align}
f(z) = \frac{1}{2\pi \mathrm{i}} \int_{\gamma_2} \frac{f(\zeta)}{\zeta - z} \mathrm{d}\zeta - \frac{1}{2\pi \mathrm{i}} \int_{\gamma_1} \frac{f(\zeta)}{\zeta - z} \mathrm{d}\zeta. \label{eq:1111111311211111111}
\end{align}
\begin{figure}[H]
\centering
\includegraphics[scale=0.4]{图5.1.png}
\caption{}
\label{figure:图5.1}
\end{figure}
记 \( M_j = \sup \{ |f(\zeta)|: \zeta \in \gamma_j \}, j = 1, 2 \)。当 \( \zeta \in \gamma_1 \) 时,\( \left| \frac{\zeta - z_0}{z - z_0} \right| = \frac{r_1}{|z - z_0|} < 1 \),所以有
\begin{align*}
\frac{1}{\zeta - z} = -\frac{1}{z - z_0} \left( 1 - \frac{\zeta - z_0}{z - z_0} \right)^{-1} = -\sum_{n = 0}^{\infty} \frac{(\zeta - z_0)^n}{(z - z_0)^{n + 1}} = -\sum_{n = 1}^{\infty} \frac{(\zeta - z_0)^{n - 1}}{(z - z_0)^n},
\end{align*}
于是
\begin{align}
\frac{f(\zeta)}{\zeta - z} = -\sum_{n = 1}^{\infty} f(\zeta) \frac{(\zeta - z_0)^{n - 1}}{(z - z_0)^n}, \, \zeta \in \gamma_1. \label{eq:11114311211111111}
\end{align}
由于
\[
\left| \frac{f(\zeta)(\zeta - z_0)^{n - 1}}{(z - z_0)^n} \right| \leqslant  \frac{M_1}{|z - z_0|} \left( \frac{r_1}{|z - z_0|} \right)^{n - 1},
\]
并且右端是一收敛级数,故知级数 \eqref{eq:11114311211111111} 在 \( \gamma_1 \) 上一致收敛,因而可\hyperref[theorem:定理4.1.5]{逐项积分}:
\begin{align}
\frac{1}{2\pi \mathrm{i}} \int_{\gamma_1} \frac{f(\zeta)}{\zeta - z} \mathrm{d}\zeta = -\sum_{n = 1}^{\infty} \left( \frac{1}{2\pi \mathrm{i}} \int_{\gamma_1} \frac{f(\zeta) \mathrm{d}\zeta}{(\zeta - z_0)^{-n + 1}} \right) (z - z_0)^{-n}. \label{eq:111143112155111111}
\end{align}
当 \( \zeta \in \gamma_2 \) 时,\( \left| \frac{z - z_0}{\zeta - z_0} \right| = \frac{|z - z_0|}{r_2} < 1 \),所以有
\begin{align*}
\frac{1}{\zeta - z} = \frac{1}{\zeta - z_0} \left( 1 - \frac{z - z_0}{\zeta - z_0} \right)^{-1} = \sum_{n = 0}^{\infty} \frac{(z - z_0)^n}{(\zeta - z_0)^{n + 1}},
\end{align*}
于是
\begin{align}
\frac{f(\zeta)}{\zeta - z} = \sum_{n = 0}^{\infty} f(\zeta) \frac{(z - z_0)^n}{(\zeta - z_0)^{n + 1}}, \, \zeta \in \gamma_2. \label{eq:111163112155111111}
\end{align}
与上面的讨论一样,级数 \eqref{eq:111163112155111111} 在 \( \gamma_2 \) 上一致收敛,所以
\begin{align}
\frac{1}{2\pi \mathrm{i}} \int_{\gamma_2} \frac{f(\zeta)}{\zeta - z} \mathrm{d}\zeta = \sum_{n = 0}^{\infty} \left( \frac{1}{2\pi \mathrm{i}} \int_{\gamma_2} \frac{f(\zeta)}{(\zeta - z_0)^{n + 1}} \mathrm{d}\zeta \right) (z - z_0)^n. \label{eq:117763112155111111}
\end{align}
由\hyperref[theorem:定理3.2.5]{多连通域的 Cauchy 积分定理},得
\begin{align*}
\frac{1}{2\pi \mathrm{i}} \int_{\gamma_1} \frac{f(\zeta)}{(\zeta - z_0)^{-n + 1}} \mathrm{d}\zeta &= \frac{1}{2\pi \mathrm{i}} \int_{\gamma_{\rho}} \frac{f(\zeta)}{(\zeta - z_0)^{-n + 1}} \mathrm{d}\zeta = a_{-n}, \\
\frac{1}{2\pi \mathrm{i}} \int_{\gamma_2} \frac{f(\zeta)}{(\zeta - z_0)^{n + 1}} \mathrm{d}\zeta &= \frac{1}{2\pi \mathrm{i}} \int_{\gamma_{\rho}} \frac{f(\zeta)}{(\zeta - z_0)^{n + 1}} \mathrm{d}\zeta = a_n.
\end{align*}
把它们分别代入 \eqref{eq:111143112155111111} 式和 \eqref{eq:117763112155111111} 式,得
\begin{align*}
\frac{1}{2\pi \mathrm{i}} \int_{\gamma_1} \frac{f(\zeta)}{\zeta - z} \mathrm{d}\zeta &= -\sum_{n = 1}^{\infty} a_{-n} (z - z_0)^{-n}, \\
\frac{1}{2\pi \mathrm{i}} \int_{\gamma_2} \frac{f(\zeta)}{\zeta - z} \mathrm{d}\zeta &= \sum_{n = 0}^{\infty} a_n (z - z_0)^n.
\end{align*}
再把它们代入 \eqref{eq:1111111311211111111} 式,即得展开式 \eqref{eq:1111111111211111111}。

现在证明展开式 \eqref{eq:1111111111211111111} 是唯一的。如果另有展开式
\[
f(z) = \sum_{n = -\infty}^{\infty} a_n' (z - z_0)^n,
\]
因为级数在 \( \gamma_{\rho} \) 上一致收敛,逐项积分得
\begin{align*}
\frac{1}{2\pi \mathrm{i}} \int_{\gamma_{\rho}} \frac{f(\zeta)}{(\zeta - z_0)^{m + 1}} \mathrm{d}\zeta = \sum_{n = -\infty}^{\infty} a_n' \frac{1}{2\pi \mathrm{i}} \int_{\gamma_{\rho}} (\zeta - z_0)^{n - m - 1} \mathrm{d}\zeta \xlongequal{\text{\refexa{example:例3.1.4}}} a_m',
\end{align*}
所以这个展开式就是 \eqref{eq:1111111111211111111} 式。 
\end{proof}

\begin{example}
设 \( f(z) = \frac{1}{(z - 1)(z - 2)} \),试分别给出这个函数在 \( D_1 = \{ z: 1 < |z| < 2 \} \) 和 \( D_2 = \{ z: 2 < |z| < \infty \} \) 上的 Laurent 展开式。
\end{example}
\begin{solution}
当 \( z \in D_1 \) 时,由于 \( 1 < |z| < 2 \),所以
\begin{align*}
\frac{1}{(z - 1)(z - 2)} = \frac{1}{z - 2} - \frac{1}{z - 1} = -\frac{1}{2} \frac{1}{1 - \frac{z}{2}} - \frac{1}{z} \frac{1}{1 - \frac{1}{z}} = -\sum_{n = 0}^{\infty} \frac{1}{2^{n + 1}} z^n - \sum_{n = 1}^{\infty} \frac{1}{z^n}.
\end{align*}
当 \( z \in D_2 \) 时,由于 \( 2 < |z| < \infty \),所以
\begin{align*}
\frac{1}{(z - 1)(z - 2)} &= \frac{1}{z - 2} - \frac{1}{z - 1} = \frac{1}{z} \frac{1}{1 - \frac{2}{z}} - \frac{1}{z} \frac{1}{1 - \frac{1}{z}} \\
&= \sum_{n = 0}^{\infty} \frac{2^n}{z^{n + 1}} - \sum_{n = 0}^{\infty} \frac{1}{z^{n + 1}} = \sum_{n = 0}^{\infty} \frac{2^n - 1}{z^{n + 1}}.
\end{align*}
\end{solution}















\end{document}
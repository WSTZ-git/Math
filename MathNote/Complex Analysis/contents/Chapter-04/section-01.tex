\documentclass[../../main.tex]{subfiles}
\graphicspath{{\subfix{../../image/}}} % 指定图片目录,后续可以直接使用图片文件名。

% 例如:
% \begin{figure}[H]
% \centering
% \includegraphics[scale=0.4]{图.png}
% \caption{}
% \label{figure:图}
% \end{figure}
% 注意:上述\label{}一定要放在\caption{}之后,否则引用图片序号会只会显示??.

\begin{document}

\section{Weierstrass定理}

\begin{definition}
设 \( z_1, z_2, \cdots \) 是 \( \mathbb{C} \) 中的一列复数,称
\begin{align}
\sum_{n=1}^{\infty} z_n = z_1 + z_2 + \cdots + z_n + \cdots \label{equation-----::::1.1.1.1.}
\end{align}
为一个复数项级数。级数 \eqref{equation-----::::1.1.1.1.} 称为是收敛的,如果它的部分和数列 \( S_n = \sum_{k=1}^{n} z_k \) 收敛。如果 \( \{S_n\} \) 的极限为 \( S \),就说级数 \eqref{equation-----::::1.1.1.1.} 的和为 \( S \),记为 \( \sum_{n=1}^{\infty} z_n = S \)。
\end{definition}

\begin{theorem}\label{theorem:复数项级数收敛充要条件}
设 $\alpha_n = a_n + \mathrm{i}b_n$($n = 1,2,\cdots$),$a_n$ 及 $b_n$ 为实数,则复级数
\begin{align*}
\sum_{n=1}^{\infty}{\alpha _n}=\alpha _1+\alpha _2+\cdots +\alpha _n+\cdots 
\end{align*}
收敛于 $s = a + \mathrm{i}b$($a,b$ 为实数)的充要条件为:实级数 $\sum_{n = 1}^{\infty} a_n$ 及 $\sum_{n = 1}^{\infty} b_n$ 分别收敛于 $a$ 及 $b$。
\end{theorem}
\begin{proof}
设 $s_n = \sum_{k = 1}^{n} \alpha_k$,$A_n = \sum_{k = 1}^{n} a_k$,$B_n = \sum_{k = 1}^{n} b_k$,则
\[
s_n = A_n + \mathrm{i}B_n \quad (n = 1,2,\cdots)
\]
由\refthe{theorem:复数收敛的充要条件}知
\[
\lim_{n \to \infty} s_n = a + \mathrm{i}b
\]
的充要条件为
\[
\lim_{n \to \infty} A_n = a \text{ 及 } \lim_{n \to \infty} B_n = b.
\]

\end{proof}

\begin{theorem}[Cauchy收敛准则]
设 \( z_1, z_2, \cdots \) 是 \( \mathbb{C} \) 中的一列复数,级数$\sum_{n=1}^{\infty} z_n$收敛的充要条件是对任意 \( \varepsilon > 0 \),存在正整数 \( N \),使得当 \( n > N \) 时,不等式
\[
|z_{n + 1} + z_{n + 2} + \cdots + z_{n + p}| < \varepsilon
\]
对任意自然数 \( p \) 成立。
\end{theorem}
\begin{proof}
由数列的Cauchy收敛准则立得.

\end{proof}

\begin{corollary}\label{corollary:复幂级数收敛必要条件}
设 \( z_1, z_2, \cdots \) 是 \( \mathbb{C} \) 中的一列复数,\( \sum_{n=1}^{\infty} z_n \) 收敛的必要条件是 \( \lim_{n \to \infty} z_n = 0 \).
\end{corollary}

\begin{definition}
设 \( z_1, z_2, \cdots \) 是 \( \mathbb{C} \) 中的一列复数,如果级数 \( \sum_{n=1}^{\infty} |z_n| \) 收敛,就说级数 \( \sum_{n=1}^{\infty} z_n \) \textbf{绝对收敛}。
\end{definition}

\begin{proposition}
绝对收敛的级数一定收敛。反过来当然不成立。
\end{proposition}
\begin{proof}
由级数收敛的Cauchy收敛准则立得.

\end{proof}

\begin{definition}
设 \( E \) 是 \( \mathbb{C} \) 中的一个点集,\( f_n: E \to \mathbb{C} \) 是定义在 \( E \) 上的一个函数列,如果对于每一个 \( z \in E \),级数
\begin{align}
\sum_{n=1}^{\infty} f_n(z) = f_1(z) + \cdots + f_n(z) + \cdots\label{eq:------2.2.2....123}
\end{align}
收敛到 \( f(z) \),就说级数 \eqref{eq:------2.2.2....123} 在 \( E \) 上收敛,其和函数为 \( f \),记为 \( \sum_{n=1}^{\infty} f_n(z) = f(z) \)。
\end{definition}

\begin{definition}
设 \( \sum_{n=1}^{\infty} f_n(z) \) 是定义在点集 \( E \) 上的级数,我们说 \( \sum_{n=1}^{\infty} f_n(z) \) 在 \( E \) 上一致收敛到 \( f(z) \),是指对任意 \( \varepsilon > 0 \),存在正整数 \( N \),当 \( n > N \) 时,不等式
\[
|S_n(z) - f(z)| < \varepsilon
\]
对所有 \( z \in E \) 成立,这里,\( S_n(z) = \sum_{k=1}^{n} f_k(z) \) 是级数的部分和。
\end{definition}

\begin{theorem}\label{theorem:定理4.1.2}
级数 \( \sum_{n=1}^{\infty} f_n(z) \) 在 \( E \) 上一致收敛的充要条件是对任意 \( \varepsilon > 0 \),存在正整数 \( N \),当 \( n > N \) 时,不等式
\begin{align}
|f_{n + 1}(z) + \cdots + f_{n + p}(z)| < \varepsilon\label{eq::----333333}
\end{align}
对所有 \( z \in E \) 及任意自然数 \( p \) 成立。
\end{theorem}
\begin{proof}
设 \( \sum_{n=1}^{\infty} f_n(z) \) 在 \( E \) 上一致收敛到 \( f(z) \),那么按定义,对任意 \( \varepsilon > 0 \),存在 \( N \),使得当 \( n > N \) 时,不等式
\[
|S_n(z) - f(z)| < \frac{\varepsilon}{2},
\quad
|S_{n + p}(z) - f(z)| < \frac{\varepsilon}{2}
\]
在 \( E \) 上成立,这里,\( p \) 是任意自然数。因而
\begin{align*}
|f_{n + 1}(z) + \cdots + f_{n + p}(z)| = |S_{n + p}(z) - S_n(z)| \leqslant |S_{n + p}(z) - f(z)| + |S_n(z) - f(z)| < \varepsilon
\end{align*}
在 \( E \) 上成立,这就是不等式\eqref{eq::----333333}。

反之,如果不等式\eqref{eq::----333333} 对任意自然数 \( p \) 在 \( E \) 上成立,那么 \( \sum_{n=1}^{\infty} f_n(z) \) 在 \( E \) 上收敛,设其和为 \( f(z) \)。在不等式
\[
|S_{n + p}(z) - S_n(z)| < \varepsilon
\]
中令 \( p \to \infty \),即得
\[
|f(z) - S_n(z)| \leqslant \varepsilon.
\]
按定义,\( \sum_{n=1}^{\infty} f_n(z) \) 在 \( E \) 上一致收敛到 \( f(z) \)。

\end{proof}

\begin{theorem}[Weierstrass一致收敛判别法]\label{theorem:Weierstrass一致收敛判别法}
设 \( f_n: E \to \mathbb{C} \) 是定义在 \( E \) 上的函数列,且在 \( E \) 上满足 \( |f_n(z)| \leqslant a_n, n = 1, 2, \cdots \)。如果 \( \sum_{n=1}^{\infty} a_n \) 收敛,那么 \( \sum_{n=1}^{\infty} f_n(z) \) 在 \( E \) 上一致收敛。
\end{theorem}
\begin{proof}
因为 \( \sum_{n=1}^{\infty} a_n \) 收敛,故对任意 \( \varepsilon > 0 \),存在正整数 \( N \),使得当 \( n > N \) 时,不等式
\[
a_{n + 1} + \cdots + a_{n + p} < \varepsilon
\]
对任意自然数 \( p \) 成立。于是,当 \( n > N \) 时,不等式
\[
|f_{n + 1}(z) + \cdots + f_{n + p}(z)| \leqslant a_{n + 1} + \cdots + a_{n + p}
< \varepsilon
\]
对任意 \( z \in E \) 及任意自然数 \( p \) 成立。故由\refthe{theorem:定理4.1.2}知道,级数 \( \sum_{n=1}^{\infty} f_n(z) \) 在 \( E \) 上一致收敛。

\end{proof}

\begin{theorem}\label{theorem:定理4.1.4}
设级数 \( \sum_{n=1}^{\infty} f_n(z) \) 在点集 \( E \) 上一致收敛到 \( f(z) \),如果每个 \( f_n (n = 1, 2, \cdots) \) 都是 \( E \) 上的连续函数,那么 \( f \) 也是 \( E \) 上的连续函数。
\end{theorem}
\begin{proof}
任取 \( a \in E \),只要证明 \( f \) 在 \( a \) 处连续就可以了。因为 \( \sum_{n=1}^{\infty} f_n(z) \) 在 \( E \) 上一致收敛到 \( f(z) \),故对任意 \( \varepsilon > 0 \),存在正整数 \( N \),当 \( n > N \) 时,不等式
\[
|f(z) - S_n(z)| < \frac{\varepsilon}{3}
\]
对所有 \( z \in E \) 成立。取定 \( n_0 > N \),则因 \( S_{n_0}(z) = \sum_{k=1}^{n_0} f_k(z) \) 在 \( a \) 点连续,故对任意 \( \varepsilon > 0 \),存在 \( \delta > 0 \),当 \( z \in E \cap B(a, \delta) \) 时,有
\[
|S_{n_0}(z) - S_{n_0}(a)| < \frac{\varepsilon}{3}.
\]
于是,当 \( z \in E \cap B(z_0, \delta) \) 时,有
\begin{align*}
|f(z) - f(a)| &\leqslant |f(z) - S_{n_0}(z)| + |S_{n_0}(z) - S_{n_0}(a)| + |S_{n_0}(a) - f(a)| \\
&< \frac{\varepsilon}{3} + \frac{\varepsilon}{3} + \frac{\varepsilon}{3}= \varepsilon.
\end{align*}
这就证明了 \( f \) 在 \( a \) 处连续。

\end{proof}

\begin{theorem}\label{theorem:定理4.1.5}
设级数 \( \sum_{n=1}^{\infty} f_n(z) \) 在可求长曲线 \( \gamma \) 上一致收敛到 \( f(z) \),如果每个 \( f_n (n = 1, 2, \cdots) \) 都在 \( \gamma \) 上连续,那么
\begin{align}
\int_{\gamma} f(z) \mathrm{d}z = \sum_{n=1}^{\infty} \int_{\gamma} f(z) \mathrm{d}z.\label{eq:::----4431}
\end{align}
\end{theorem}
\begin{remark}
这个定理实际上证明了在上述的条件下,级数 \( \sum_{n=1}^{\infty} f_n(z) \) 可以沿 \( \gamma \) 逐项积分。
\end{remark}
\begin{proof}
由\refthe{theorem:定理4.1.4},\( f \) 在 \( \gamma \) 上连续。因为 \( \sum_{n=1}^{\infty} f_n(z) \) 在 \( \gamma \) 上一致收敛到 \( f(z) \),所以对任意 \( \varepsilon > 0 \),存在正整数 \( N \),当 \( n > N \) 时,不等式
\[
\left| \sum_{k=1}^{n} f_k(z) - f(z) \right| < \varepsilon
\]
对任意 \( z \in \gamma \) 成立。于是,当 \( n > N \) 时,由\hyperref[proposition:长大不等式]{长大不等式}得
\begin{align*}
\left| \sum_{k=1}^{n} \int_{\gamma} f_k(z) \mathrm{d}z - \int_{\gamma} f(z) \mathrm{d}z \right| = \left| \int_{\gamma} \left( \sum_{k=1}^{n} f_k(z) - f(z) \right) \mathrm{d}z \right| < \varepsilon |\gamma|.
\end{align*}
因而等式\eqref{eq:::----4431}成立。

\end{proof}

\begin{definition}
如果级数 \( \sum_{n=1}^{\infty} f_n(z) \) 在域 \( D \) 的任意紧子集上一致收敛,就称 \( \sum_{n=1}^{\infty} f_n(z) \) 在 \( D \) 中\textbf{内闭一致收敛}。
\end{definition}

\begin{remark}
如果 \( \sum_{n=1}^{\infty} f_n(z) \) 在域 \( D \) 上内闭一致收敛,那么它在 \( D \) 中的每一点都收敛,但不一定一致收敛。例如,级数 \( 1 + \sum_{k=1}^{\infty} (z^k - z^{k - 1}) \),它的部分和
\[
S_{n + 1}(z) = 1 + (z - 1) + \cdots + (z^n - z^{n - 1}) = z^n,
\]
显然它在单位圆盘中是内闭一致收敛的,但不一致收敛。
\end{remark}

\begin{proposition}
如果 \( \sum_{n=1}^{\infty} f_n(z) \) 在 \( D \) 中一致收敛,那么它一定内闭一致收敛。
\end{proposition}
\begin{note}
由此可知,内闭一致收敛比一致收敛要求低。
\end{note}
\begin{proof}
证明是显然的.

\end{proof}

\begin{definition}
如果 \( D \) 的子集 \( G \) 满足

(i) \( \overline{G} \subset D \);

(ii) \( \overline{G} \) 是紧的,

就说 \( G \) 相对于 \( D \) 是紧的,记为 \( G \subset\subset D \)。
\end{definition}

\begin{lemma}\label{lemma:引理4.1.8}
设 \( D \) 是 \( \mathbb{C} \) 中的域,\( K \) 是 \( D \) 中的紧子集,且包含在相对于 \( D \) 是紧的开集 \( G \) 中,即 \( K \subset G \subset\subset D \),那么对任意 \( f \in H(D) \),均有
\begin{align}
\sup\{|f^{(k)}(z)| : z \in K\} \leqslant C\sup\{|f(z)| : z \in G\},\label{eq::::--5}
\end{align}
这里,\( k \) 是任意自然数,\( C \) 是与 \( k, K, G \) 有关的常数。
\end{lemma}
\begin{note}
这个引理告诉我们,\( f^{(k)} \)(\( k \) 是任意自然数)在紧集 \( K \) 上的上确界可用 \( f \) 在 \( K \) 的邻域 \( G \) 上的上确界来控制。
\end{note}
\begin{proof}
由\refthe{theorem:紧集和闭集无交则距离大于0},\( \rho = d(K, \partial G) > 0 \)。所以,以 \( K \) 中任意点 \( a \) 为中心、\( \rho \) 为半径的圆盘都包含在 \( G \) 中。对圆盘 \( B(a, \rho) \) 用 \hyperref[theorem:Cauchy不等式-复变函数]{Cauchy 不等式},得
\begin{align*}
|f^{(k)}(a)| \leqslant \frac{k!}{\rho^k}\sup\{|f(z)| : z \in B(a, \rho)\} \leqslant \frac{k!}{\rho^k}\sup\{|f(z)| : z \in G\}.
\end{align*}
对 \( K \) 中的 \( a \) 取上确界,即得不等式\eqref{eq::::--5}。

\end{proof}

\begin{theorem}[Weierstrass定理]\label{theorem:Weierstrass定理}
设 \( D \) 是 \( \mathbb{C} \) 中的域,如果

(i) \( f_n \in H(D), n = 1, 2, \cdots \);

(ii) \( \sum_{n=1}^{\infty} f_n(z) \) 在 \( D \) 中内闭一致收敛到 \( f(z) \),

那么

(i) \( f \in H(D) \);

(ii) 对任意自然数 \( k \),\( \sum_{n=1}^{\infty} f_n^{(k)}(z) \) 在 \( D \) 中内闭一致收敛到 \( f^{(k)}(z) \)。
\end{theorem}
\begin{note}
从Weierstrass定理我们看到,由全纯函数构成的级数只要在域中内闭一致收敛,它的和函数就一定是域中的全纯函数,而且可以逐项求导任意次。这样的结果在实变函数中当然不成立。
\end{note}
\begin{proof}
任取 \( z_0 \in D \),只要证明 \( f \) 在 \( z_0 \) 的一个邻域中全纯就行了。选取 \( r > 0 \),使得 \( \overline{B(z_0, r)} \subset D \),由\refthe{theorem:定理4.1.4},\( f \) 在 \( B(z_0, r) \) 中连续。在 \( B(z_0, r) \) 中任取一可求长闭曲线 \( \gamma \),由\refthe{theorem:定理4.1.5}和 \hyperref[theorem:Cauchy-Goursat定理(Cauchy积分定理)]{Cauchy积分定理},得
\[
\int_{\gamma} f(z) \mathrm{d}z = \sum_{n=1}^{\infty} \int_{\gamma} f_n(z) \mathrm{d}z = 0.
\]
由\hyperref[theorem:Morera定理]{Morera定理},即知 \( f \) 在 \( B(z_0, r) \) 中全纯,所以 \( f \) 在 \( D \) 中全纯。

为了证明第二个结论,任取 \( D \) 中的紧子集 \( K \),记 \( \rho = d(K, \partial D) > 0 \)。令
\[
G = \bigcup\left\{ B\left(z, \frac{\rho}{2}\right) : z \in K \right\},
\]
则 \( K \subset G \subset\subset D \)。由于 \( \overline{G} \) 是紧集,所以 \( \sum_{n=1}^{\infty} f_n(z) \) 在 \( \overline{G} \) 上一致收敛到 \( f(z) \)。因而对任意 \( \varepsilon > 0 \),存在正整数 \( N \),当 \( n > N \) 时,不等式 \( |S_n(z) - f(z)| < \varepsilon \) 对 \( \overline{G} \) 上所有的 \( z \) 成立,这里,\( S_n(z) = \sum_{j=1}^{n} f_j(z) \)。于是由\reflem{lemma:引理4.1.8},对任意的自然数 \( k \),有
\begin{align*}
\sup\{|S_n^{(k)}(z) - f^{(k)}(z)| : z \in K\} \leqslant C\sup\{|S_n(z) - f(z)| : z \in G\} \leqslant C\varepsilon,
\end{align*}
这就证明了 \( \sum_{n=1}^{\infty} f_n^{(k)}(z) \) 在 \( K \) 上一致收敛到 \( f^{(k)}(z) \)。由于 \( K \) 是 \( D \) 的任意紧子集,所以 \( \sum_{n=1}^{\infty} f_n^{(k)}(z) \) 在 \( D \) 上内闭一致收敛到 \( f^{(k)}(z) \)

\end{proof}

\begin{example}
研究函数 \( \zeta(z) = \sum_{n=1}^{\infty} \frac{1}{n^z} \)。
\end{example}
\begin{solution}
因为 \( n^z = \mathrm{e}^{z \log n} \),若记 \( z = x + \mathrm{i}y \),则
\[
|n^z| = |\mathrm{e}^{x \log n} \cdot \mathrm{e}^{\mathrm{i}y \log n}| = n^x.
\]
当 \( \mathrm{Re}z = x \geqslant x_0 > 1 \) 时,\( \left| \frac{1}{n^z} \right| \leqslant \frac{1}{n^{x_0}} \),故由\hyperref[theorem:Weierstrass一致收敛判别法]{Weierstrass一致收敛判别法}可知,级数 \( \sum_{n=1}^{\infty} \frac{1}{n^z} \) 在 \( \mathrm{Re}z > 1 \) 中一致收敛,从而内闭一致收敛.由 \hyperref[theorem:Weierstrass定理]{Weierstrass 定理},\( \zeta \) 是半平面 \( \mathrm{Re}z > 1 \) 上的全纯函数。

\end{solution}




\end{document}
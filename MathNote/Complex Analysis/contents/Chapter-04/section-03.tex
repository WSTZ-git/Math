\documentclass[../../main.tex]{subfiles}
\graphicspath{{\subfix{../../image/}}} % 指定图片目录,后续可以直接使用图片文件名。

% 例如:
% \begin{figure}[H]
% \centering
% \includegraphics[scale=0.4]{图.png}
% \caption{}
% \label{figure:图}
% \end{figure}
% 注意:上述\label{}一定要放在\caption{}之后,否则引用图片序号会只会显示??.

\begin{document}

\section{全纯函数的Taylor展开}

\begin{theorem}\label{theorem:定理4.3.1}
若 \( f \in H(B(z_0,R)) \),则 \( f \) 可以在 \( B(z_0,R) \) 中展开成幂级数:
\begin{align}
f(z) = \sum_{n=0}^{\infty} \frac{f^{(n)}(z_0)}{n!} (z - z_0)^n, \quad z \in B(z_0,R). \label{eq:taylor312412342}
\end{align}
右端的级数称为 \( f \) 的$\mathbf{Taylor}$\textbf{级数},并且$f$的Taylor级数展开式是唯一的.
\end{theorem}
\begin{proof}
任意取定 \( z \in B(z_0,R) \),再取 \( \rho < R \),使得 \( |z - z_0| < \rho \)(见\reffig{figure:图4.4}). 记 \( \gamma_{\rho} = \{ \zeta : |\zeta - z_0| = \rho \} \),根据 \hyperref[theorem:定理3.4.1]{Cauchy 积分公式},得
\[
f(z) = \frac{1}{2\pi i} \int_{\gamma_{\rho}} \frac{f(\zeta)}{\zeta - z} d\zeta.
\]
\begin{figure}[H]
\centering
\includegraphics[scale=0.3]{图4.4.png}
\caption{}
\label{figure:图4.4}
\end{figure}
把 \( \frac{1}{\zeta - z} \) 展开成级数,为
\begin{align*}
\frac{1}{\zeta - z} = \frac{1}{(\zeta - z_0) - (z - z_0)} = \frac{1}{\zeta - z_0} \left( 1 - \frac{z - z_0}{\zeta - z_0} \right)^{-1} = \frac{1}{\zeta - z_0} \sum_{n=0}^{\infty} \left( \frac{z - z_0}{\zeta - z_0} \right)^n,
\end{align*}
最后一个等式成立是因为 \( \left| \frac{z - z_0}{\zeta - z_0} \right| = \frac{|z - z_0|}{\rho} < 1 \) 的缘故. 现在可得
\begin{align}
\frac{f(\zeta)}{\zeta - z} = \sum_{n=0}^{\infty} f(\zeta) \frac{(z - z_0)^n}{(\zeta - z_0)^{n+1}}. \label{eq:series13123132313}
\end{align}
因为 \( f \) 在 \( \gamma_{\rho} \) 上连续,记 \( M = \sup \{ |f(\zeta)| : \zeta \in \gamma_{\rho} \} \),于是当 \( \zeta \in \gamma_{\rho} \) 时,有
\[
\left| \frac{f(\zeta)(z - z_0)^n}{(\zeta - z_0)^{n+1}} \right| \leqslant  \frac{M}{\rho} \left( \frac{|z - z_0|}{\rho} \right)^n.
\]
右端是一收敛级数,故由 \hyperref[theorem:Weierstrass一致收敛判别法]{Weierstrass 判别法},级数\eqref{eq:series13123132313} 在 \( \gamma_{\rho} \) 上一致收敛,故由\refthe{theorem:定理4.1.5}可知,级数\eqref{eq:series13123132313}可逐项积分.又因为$f\in \overline{H(B(z_0,\rho))}$,所以再由\refthe{theorem:定理3.4.3}可得
\begin{align*}
f(z) = \frac{1}{2\pi i} \int_{\gamma_{\rho}} \sum_{n=0}^{\infty} f(\zeta) \frac{(z - z_0)^n}{(\zeta - z_0)^{n+1}} d\zeta = \sum_{n=0}^{\infty} \left( \frac{1}{2\pi i} \int_{\gamma_{\rho}} \frac{f(\zeta)}{(\zeta - z_0)^{n+1}} d\zeta \right) (z - z_0)^n = \sum_{n=0}^{\infty} \frac{f^{(n)}(z_0)}{n!} (z - z_0)^n.
\end{align*}
由于 \( z \) 是 \( B(z_0,R) \) 中的任意点,所以上式在 \( B(z_0,R) \) 中成立. 

\( f \) 的展开式\eqref{eq:taylor312412342}是唯一的. 因为若有展开式
\[
f(z) = \sum_{n=0}^{\infty} a_n (z - z_0)^n,
\]
那么由\refthe{theorem:幂级数在其收敛圆内确定一个全纯函数并且任意阶可导}可知
\[
f^{(k)}(z) = \sum_{n=k}^{\infty} n(n - 1) \cdots (n - k + 1) a_n (z - z_0)^{n - k}.
\]
在上式中令 \( z = z_0 \),即得 \( f^{(k)}(z_0) = k! a_k \),或者 \( a_k = \frac{f^{(k)}(z_0)}{k!} \),所以
\[
f(z) = \sum_{n=0}^{\infty} \frac{f^{(n)}(z_0)}{n!} (z - z_0)^n,
\]
这就是展开式 \eqref{eq:taylor312412342}.
\end{proof}

\begin{theorem}\label{theorem:定理4.3.2}
\( f \) 在点 \( z_0 \) 处全纯的充分必要条件是 \( f \) 在 \( z_0 \) 的邻域内可以展开成幂级数:
\[
f(z) = \sum_{n=0}^{\infty} a_n (z - z_0)^n.
\]
\end{theorem}
\begin{proof}
由\refthe{theorem:定理4.3.1}和\refthe{theorem:定理4.2.4}立得.
\end{proof}

\begin{definition}
设 \( f \) 在 \( z_0 \) 点全纯且不恒为零,如果
\[
f(z_0) = 0, \, f'(z_0) = 0, \, \cdots, \, f^{(m - 1)}(z_0) = 0, \, f^{(m)}(z_0) \neq 0,
\]
则称 \( z_0 \) 是 \( f \) 的 $\boldsymbol{m}$\textbf{阶零点}.
\end{definition}

\begin{proposition}\label{proposition:命题4.3.4}
\( z_0 \) 为 \( f \) 的 \( m \) 阶零点的充分必要条件是 \( f \) 在 \( z_0 \) 的邻域内可以表示为
\begin{align}
f(z) = (z - z_0)^m g(z), \label{eq:zero_point}
\end{align}
这里,\( g \) 在 \( z_0 \) 点全纯,且 \( g(z_0) \neq 0 \).
\end{proposition}
\begin{proof}
如果 \( z_0 \) 是 \( f \) 的 \( m \) 阶零点,则从 \( f \) 的 Taylor 展开可得
\begin{align*}
f(z) &= \sum_{n=0}^{\infty} \frac{f^{(n)}(z_0)}{n!} (z - z_0)^n = \sum_{n=m}^{\infty} \frac{f^{(n)}(z_0)}{n!} (z - z_0)^n \\
&= (z - z_0)^m \left\{ \frac{f^{(m)}(z_0)}{m!} + \frac{f^{(m + 1)}(z_0)}{(m + 1)!} (z - z_0) + \cdots \right\} \\
&= (z - z_0)^m g(z).
\end{align*}
这里,\( g(z) \) 就是花括弧中的幂级数,它当然在 \( z_0 \) 处全纯,而且
\[
g(z_0) = \frac{f^{(m)}(z_0)}{m!} \neq 0.
\]
反之,如果 \(\eqref{eq:zero_point}\) 式成立,\( f \) 当然在 \( z_0 \) 处全纯,通过直接计算即知 \( z_0 \) 是 \( f \) 的 \( m \) 阶零点.
\end{proof}

\begin{proposition}\label{proposition:命题4.3.5}
设 \( D \) 是 \( \mathbb{C} \) 中的域,\( f \in H(D) \),如果 \( f \) 在 \( D \) 中的小圆盘 \( B(z_0, \varepsilon) \) 上恒等于零,那么 \( f \) 在 \( D \) 上恒等于零.
\end{proposition}
\begin{proof}
在 \( D \) 中任取一点 \( a \),我们证明 \( f(a) = 0 \). 用 \( D \) 中的曲线 \( \gamma \) 连接 \( z_0 \) 和 \( a \),由\refthe{theorem:紧集和闭集无交则距离大于0},\( \rho = d(\gamma, \partial D) > 0 \). 在 \( \gamma \) 上依次取点 \( z_0, z_1, z_2, \cdots, z_n = a \),使得 \( z_1 \in B(z_0, \varepsilon) \),其他各点之间的距离都小于 \( \rho \),作圆盘 \( B(z_j, \rho) \),\( j = 1, \cdots, n \)(\reffig{figure:图4.5}). 由于 \( f \) 在 \( B(z_0, \varepsilon) \) 中恒为零,所以 \( f^{(n)}(z_1) = 0 \),\( n = 0, 1, \cdots \). 于是,\( f \) 在 \( B(z_1, \rho) \) 中的 Taylor 展开式的系数全为零,所以 \( f \) 在 \( B(z_1, \rho) \) 中恒为零. 由于 \( z_2 \in B(z_1, \rho) \),所以 \( f^{(n)}(z_2) = 0 \),\( n = 0, 1, \cdots \),用同样的方法推理,\( f \) 在 \( B(z_2, \rho) \) 中恒为零. 再往下推,即知 \( f \) 在 \( B(a, \rho) \) 中恒为零,所以 \( f(a) = 0 \).
\begin{figure}[H]
\centering
\includegraphics[scale=0.3]{图4.5.png}
\caption{}
\label{figure:图4.5}
\end{figure}
\end{proof}

\begin{proposition}\label{proposition:命题4.3.6}
设 \( D \) 是 \( \mathbb{C} \) 中的域,\( f \in H(D) \),\( f(z) \not\equiv 0 \),那么 \( f \) 在 \( D \) 中的零点是孤立的. 即若 \( z_0 \) 为 \( f \) 的零点,则必存在 \( z_0 \) 的邻域 \( B(z_0, \varepsilon) \),使得 \( f \) 在 \( B(z_0, \varepsilon) \) 中除了 \( z_0 \) 外不再有其他的零点.
\end{proposition}
\begin{proof}
由\refpro{proposition:命题4.3.5} 知,\( f \) 在 \( z_0 \) 的邻域中不能恒等于零,故不妨设 \( z_0 \) 为 \( f \) 的 \( m \) 阶零点. 由\refpro{proposition:命题4.3.4}知,\( f \) 在 \( z_0 \) 的邻域中可表示为 \( f(z) = (z - z_0)^m g(z) \),因 \( g \) 在 \( z_0 \) 处连续,且 \( g(z_0) \neq 0 \),故存在 \( z_0 \) 的邻域 \( B(z_0, \varepsilon) \),使得 \( g \) 在 \( B(z_0, \varepsilon) \) 中处处不为零,因而 \( f \) 在 \( B(z_0, \varepsilon) \) 中除了 \( z_0 \) 外不再有其他的零点.
\end{proof} 

\begin{theorem}[唯一性定理]\label{theorem:定理4.3.7}
设 \( D \) 是 \( \mathbb{C} \) 中的域,\( f_1, f_2 \in H(D) \). 如果存在 \( D \) 中的点列 \( \{ z_n \} \),使得 \( f_1(z_n) = f_2(z_n) \),\( n = 1, 2, \cdots \),且 \( \lim_{n \to \infty} z_n = a \in D \),那么在 \( D \) 中有 \( f_1(z) \equiv f_2(z) \).
\end{theorem}
\begin{remark}
这个定理说明,全纯函数由极限在域中的一列点上的值所完全确定,这是一个非常深刻的结果.
\end{remark}
\begin{remark}
必须注意,\(\lim_{n \to \infty} z_n = a\),\(a \in D\) 这个条件是不能去掉的,否则结果不成立. 例如,\(f(z) = \sin \frac{1}{1 - z}\) 在单位圆盘中全纯,令 \(z_n = 1 - \frac{1}{n\pi}\),则 \(f(z_n) = 0\),\(n = 1, 2, \cdots\),但 \(f(z) \not\equiv 0\),原因是 \(z_n \to 1\),而 \(1\) 不在单位圆盘中.
\end{remark}
\begin{proof}
令 \( g(z) = f_1(z) - f_2(z) \),则 \( g(z_n) = 0 \),\( n = 1, 2, \cdots \). 由于 \( g \in H(D) \),所以 \( g(a) = \lim_{n \to \infty} g(z_n) = 0 \),即 \( a \) 是 \( g \) 的一个零点. 由于 \( \{ z_n \} \) 也是 \( g \) 的零点,而且 \( z_n \to a \),因而零点 \( a \) 不是孤立的. 由\refpro{proposition:命题4.3.5},得 \( g(z) \equiv 0 \),即 \( f_1(z) \equiv f_2(z) \).
\end{proof}

\begin{proposition}[常用的初等函数的Taylor展开式]\label{proposition:常用的初等函数的Taylor展开式}
\begin{enumerate}[(1)]
\item $e^z = \sum_{n=0}^{\infty} \frac{z^n}{n!}, \quad z \in \mathbb{C}.$

\item $\cos z = \sum_{n=0}^{\infty} (-1)^n \frac{z^{2n}}{(2n)!},\quad z \in \mathbb{C}.$

\item $\sin z = \sum_{n=0}^{\infty} (-1)^n \frac{z^{2n + 1}}{(2n + 1)!},\quad z \in \mathbb{C}.$

\item $\log(1 + z) = \sum_{n=1}^{\infty} (-1)^{n - 1} \frac{z^n}{n}, \quad |z| < 1.$

\item $e^{\alpha \log(1 + z)} = \sum_{n=0}^{\infty} \binom{\alpha}{n} z^n, \quad \alpha \in \mathbb{R},|z| < 1.$
\end{enumerate}
\end{proposition}
\begin{proof}
\begin{enumerate}[(1)]
\item 指数函数 \( f(z) = e^z \),它是一个整函数,所以可以在圆盘 \( B(0,R) \) 中展开成幂级数,其中,\( R \) 是任意正数. 由于 \( f^{(n)}(z) = e^z \),\( f^{(n)}(0) = 1 \),所以
\begin{align}
e^z = \sum_{n=0}^{\infty} \frac{z^n}{n!}, \quad z \in \mathbb{C}. \label{eq:exp_taylor123}
\end{align}
公式\eqref{eq:exp_taylor123} 也可以由全纯函数的唯一性定理得到. 由直接计算知道,幂级数 \( \sum_{n=0}^{\infty} \frac{z^n}{n!} \) 的收敛半径 \( R = \infty \),所以 \( \varphi(z) = \sum_{n=0}^{\infty} \frac{z^n}{n!} \) 是一个整函数. 已知 \( e^z \) 是一个整函数,这两个整函数在实轴上相等,即
\[
e^x = \sum_{n=0}^{\infty} \frac{x^n}{n!}, \quad x \in \mathbb{R},
\]
故由唯一性定理知道这两个整函数在 \( \mathbb{C} \) 上处处相等,这就是公式 \eqref{eq:exp_taylor123}.

\item 由(1)同理可得.

\item 由(1)同理可得.

\item 由\refexa{example:4.2.10}我们已经得到
\[
- \log(1 - z) = \sum_{n=1}^{\infty} \frac{z^n}{n}, \quad |z| < 1,
\]
在上式中用 \( -z \) 代替 \( z \),立刻可得
\[
\log(1 + z) = \sum_{n=1}^{\infty} (-1)^{n - 1} \frac{z^n}{n}, \quad |z| < 1.
\]

\item 函数 \( f(z) = (1 + z)^{\alpha} \),\( \alpha \) 不是整数,我们考虑它的主支 \( f(z) = e^{\alpha \log(1 + z)} \) 在 \( z = 0 \) 处的 Taylor 展开式. 这个分支在 \( z = 0 \) 处的值为 1,它的各阶导数在 \( z = 0 \) 处的值为
\[
f^{(n)}(0) = \alpha(\alpha - 1) \cdots (\alpha - n + 1), \quad n = 1, 2, \cdots.
\]
如果记
\[
\binom{\alpha}{n} = \frac{\alpha(\alpha - 1) \cdots (\alpha - n + 1)}{n!}, \quad n = 1, 2, \cdots,
\]
\[
\binom{\alpha}{0} = 1,
\]
那么
\[
e^{\alpha \log(1 + z)} = \sum_{n=0}^{\infty} \binom{\alpha}{n} z^n, \quad |z| < 1.
\]
也可通过直接计算得到右端级数的收敛半径为 1. 上式对整数 \( \alpha \) 当然也成立,特别当 \( \alpha \) 为正整数时,右端为一多项式.
\end{enumerate}
\end{proof}













\end{document}
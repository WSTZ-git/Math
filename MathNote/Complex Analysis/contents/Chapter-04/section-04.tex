\documentclass[../../main.tex]{subfiles}
\graphicspath{{\subfix{../../image/}}} % 指定图片目录,后续可以直接使用图片文件名。

% 例如:
% \begin{figure}[H]
% \centering
% \includegraphics[scale=0.4]{图.png}
% \caption{}
% \label{figure:图}
% \end{figure}
% 注意:上述\label{}一定要放在\caption{}之后,否则引用图片序号会只会显示??.

\begin{document}

\section{辐角原理和Rouché定理}

\begin{theorem}\label{theorem:定理4.4.1}
设 \( f \in H(D) \),\( \gamma \) 是 \( D \) 中一条可求长的简单闭曲线,\( \gamma \) 的内部位于 \( D \) 中. 如果 \( f \) 在 \( \gamma \) 上没有零点,在 \( \gamma \) 内部有零点
\[
a_1, a_2, \cdots, a_n,
\]
它们的阶数分别为
\[
\alpha_1, \alpha_2, \cdots, \alpha_n,
\]
那么
\begin{align}
\frac{1}{2\pi i} \int_{\gamma} \frac{f'(z)}{f(z)} dz = \sum_{k=1}^{n} \alpha_k. \label{eq:zero_count}
\end{align}
\end{theorem}
\begin{remark}
公式 \(\eqref{eq:zero_count}\) 有明确的几何意义. 我们先作一个自然的约定:\textbf{如果 \( a \) 是 \( f \) 的 \( m \) 阶零点,我们就把 \( a \) 看成是 \( f \) 的 \( m \) 个重合的 1 阶零点.} 这样,公式 \(\eqref{eq:zero_count}\) 右边就表示 \( f \) 在 \( \gamma \) 内部的零点个数的总和,我们记之为 \( N \). 于是,公式 \(\eqref{eq:zero_count}\) 可写为
\begin{align}
\frac{1}{2\pi i} \int_{\gamma} \frac{f'(z)}{f(z)} dz = N. \label{eq:zero_count_geo}
\end{align}
\end{remark}
\begin{proof} 
取充分小的 \( \varepsilon > 0 \),作圆盘 \( B(a_k, \varepsilon) \),\( k = 1, \cdots, n \),使得这 \( n \) 个圆盘都在 \( \gamma \) 内部,且两两不相交. 于是,\( \frac{f'(z)}{f(z)} \) 在 \( D \setminus \bigcup_{k=1}^{n} B(a_k, \varepsilon) \) 中全纯.记$\gamma$的内部为$E$,则$f\in H\left(E\setminus \bigcup_{k=1}^{n} B(a_k, \varepsilon)\right)\cap C\left(\overline{E\setminus \bigcup_{k=1}^{n} B(a_k, \varepsilon)}\right)$. 应用\hyperref[theorem:定理3.2.5]{多连通域的 Cauchy 积分定理},得
\begin{align}
\frac{1}{2\pi i} \int_{\gamma} \frac{f'(z)}{f(z)} dz = \frac{1}{2\pi i} \int_{\gamma_1} \frac{f'(z)}{f(z)} dz + \cdots + \frac{1}{2\pi i} \int_{\gamma_n} \frac{f'(z)}{f(z)} dz, \label{equation:2222-------2}
\end{align}
其中,\( \gamma_k = \partial B(a_k, \varepsilon) \),\( k = 1, \cdots, n \).

因为 \( a_k \) 是 \( f \) 的 \( \alpha_k \) 阶零点,由\refpro{proposition:命题4.3.4} 知道,\( f \) 在 \( a_k \) 的邻域中可以写成
\[
f(z) = (z - a_k)^{\alpha_k} g_k(z),
\]
这里,\( g_k \) 在 \( a_k \) 的邻域中全纯,且 \( g_k(a_k) \neq 0 \). 于是
\[
f'(z) = \alpha_k (z - a_k)^{\alpha_k - 1} g_k(z) + (z - a_k)^{\alpha_k} g_k'(z),
\]
\[
\frac{f'(z)}{f(z)} = \frac{\alpha_k}{z - a_k} + \frac{g_k'(z)}{g_k(z)}.
\]
因为 \( \frac{g_k'}{g_k} \) 在 \( \overline{B(a_k, \varepsilon)} \) 中全纯,于是由\hyperref[theorem:Cauchy-Goursat定理(Cauchy积分定理)]{Cauchy积分定理}及\refexa{example:例3.1.4} 得
\[
\frac{1}{2\pi \mathrm{i}}\int_{\gamma _k}{\frac{f' (z)}{f(z)}dz}=\frac{1}{2\pi \mathrm{i}}\int_{\gamma _k}{\left[ \frac{\alpha _k}{z-a_k}+\frac{g_k' (z)}{g_k(z)} \right] dz}=\alpha _k,\quad k=1,\cdots ,n.
\]
把它代入 \eqref{equation:2222-------2} 式,即得公式 \(\eqref{eq:zero_count}\).
\end{proof}

\begin{theorem}[辐角原理]\label{theorem:辐角原理-定理4.4.2}
设 \( f \in H(D) \),\( \gamma \) 是 \( D \) 中的可求长简单闭曲线,\( \gamma \) 的内部位于 \( D \) 中。如果 \( f \) 在 \( \gamma \) 上没有零点,那么当 \( z \) 沿着 \( \gamma \) 的正方向转动一圈时,函数 \( f(z) \) 在相应的曲线 \( \gamma \) 上绕原点转动的总圈数恰好等于 \( f \) 在 \( \gamma \) 内部的零点的个数。
\end{theorem}
\begin{remark}
例如,设 \( f(z) = (z^2 + 1)(z - 1)^5 \),则当 \( z \) 沿着圆周 \( \{ z: |z| = 3 \} \) 的正方向转动一圈时,\( f(z) \) 在 \( w \) 平面上绕原点转动 7 圈。这是因为 \( f \) 在 \( B(0, 3) \) 中共有 7 个零点,其中,\( \pm \mathrm{i} \) 是 1 阶零点,而 1 则是 5 阶零点。
\end{remark}
\begin{note}
实际上,这个定理就是 \(\eqref{eq:zero_count_geo}\) 式左端积分的几何意义.
\end{note}
\begin{proof}
 设 \( \Gamma \) 是 \( w \) 平面上一段不通过原点的连续曲线,它的方程记为 \( w = \lambda(t) \),\( a \leq t \leq b \). 对于每个 \( t \),选取 \( \lambda(t) \) 的一个辐角 \( \theta(t) \),使得 \( \theta(t) \) 是 \( t \) 的连续函数,我们称 \( \theta(b) - \theta(a) \) 为 \( w \) 沿曲线 \( \Gamma \) 的\textbf{辐角的变化},记为
\[
\Delta_{\Gamma} \text{Arg} w = \theta(b) - \theta(a).
\]
今设 \( \Gamma \) 是一条不通过原点的可求长简单闭曲线,显然有
\[
\frac{1}{2\pi} \Delta_{\Gamma} \text{Arg} w = 
\begin{cases} 
1, & \text{如果原点在 } \Gamma \text{ 内部}; \\
0, & \text{如果原点不在 } \Gamma \text{ 内部}.
\end{cases}
\]
另一方面,由\refexa{example:例3.2.7}可知
\[
\frac{1}{2\pi i} \int_{\Gamma} \frac{1}{w} dw = 
\begin{cases} 
1, & \text{如果原点在 } \Gamma \text{ 内部}; \\
0, & \text{如果原点不在 } \Gamma \text{ 内部}.
\end{cases}
\]
于是得到
\begin{align}
\frac{1}{2\pi i} \int_{\Gamma} \frac{dw}{w} = \frac{1}{2\pi} \Delta_{\Gamma} \text{Arg} w. \label{eq:arg_int}
\end{align}
一般来说,当 \( \Gamma \) 是一条不通过原点的任意可求长闭曲线时,\( \frac{1}{2\pi i} \int_{\Gamma} \frac{dw}{w} \) 等于 \( \Gamma \) 绕原点的圈数,称为 \( \Gamma \) 关于原点的\textbf{环绕指数},因而 \(\eqref{eq:arg_int}\) 式对于一般的不通过原点的可求长闭曲线都是成立的.

现在让 \( z \) 在 \( z \) 平面上沿曲线 \( \gamma \) 的正方向走一圈,相应的函数 \( w = f(z) \) 的值在 \( w \) 平面上画出一条相应的闭曲线 \( \Gamma \)(见\reffig{figure:图4.6}). 根据 \(\eqref{eq:arg_int}\) 式,我们有
\begin{align}
\frac{1}{2\pi i} \int_{\gamma} \frac{f'(z)}{f(z)} dz = \frac{1}{2\pi} \Delta_{\gamma} \text{Arg} f(z). \label{eq:f_arg_int}
\end{align}
\begin{figure}[H]
\centering
\includegraphics[scale=0.4]{图4.6.png}
\caption{}
\label{figure:图4.6}
\end{figure}
由此可知,积分 \( \frac{1}{2\pi i} \int_{\gamma} \frac{f'(z)}{f(z)} dz \) 就表示当 \( z \) 沿着 \( \gamma \) 的正方向走一圈时,函数 \( f(z) \) 在 \( \Gamma \) 上的辐角变化再除以 \( 2\pi \). 由 \(\eqref{eq:zero_count_geo}\) 式和 \(\eqref{eq:f_arg_int}\) 式,我们得到
\begin{align}
\frac{1}{2\pi} \Delta_{\gamma} \text{Arg} f(z) = N. \label{eq:arg_zero}
\end{align}
\end{proof}

\begin{theorem}[Rouché(鲁歇)定理]\label{theorem:Rouché定理-定理4.4.3}
设 \( f, g \in H(D) \),\( \gamma \) 是 \( D \) 中可求长的简单闭曲线,\( \gamma \) 的内部位于 \( D \) 中。如果当 \( z \in \gamma \) 时,有不等式
\begin{align}\label{eq:-----898797777}
|f(z) - g(z)| < |f(z)|,
\end{align}
那么 \( f \) 和 \( g \) 在 \( \gamma \) 内部的零点个数相同。
\end{theorem}
\begin{proof}
由 \eqref{eq:-----898797777} 式知道,\( f \) 和 \( g \) 在 \( \gamma \) 上都没有零点。用 \( |f(z)| \) 去除 \eqref{eq:-----898797777} 式的两端,得
\[
\left| 1 - \frac{g(z)}{f(z)} \right| < 1.
\]
若记 \( w = \frac{g}{f} \),则有 \( |w - 1| < 1 \)。这说明当 \( z \) 在 \( \gamma \) 上变动时,\( w \) 落在以 1 为中心、半径为 1 的圆内,因而 \( \Delta_{\Gamma} \text{Arg} w = 0 \),于是由\refthe{theorem:复数辐角的性质}可得
\[
\Delta_{\gamma} \text{Arg} f(z) = \Delta_{\gamma} \text{Arg} g(z).
\]
由\hyperref[theorem:辐角原理-定理4.4.2]{辐角原理}即知 \( f \) 和 \( g \) 在 \( \gamma \) 内部的零点个数相同。 
\end{proof}

\begin{theorem}\label{theorem:定理4.4.4}
设 \( f \) 是域 \( D \) 中的全纯函数,\( z_0 \in D \),记 \( w_0 = f(z_0) \),如果 \( z_0 \) 是 \( f(z) - w_0 \) 的 \( m \) 阶零点,那么对于充分小的 \( \rho > 0 \),必存在 \( \delta > 0 \),使得对于任意 \( a \in B(w_0, \delta) \),\( f(z) - a \) 在 \( B(z_0, \rho) \) 中恰有 \( m \) 个零点。
\end{theorem}
\begin{proof}
根据全纯函数零点的孤立性,必存在充分小的 \( \rho > 0 \),使得 \( f(z) - w_0 \) 在 \( \overline{B(z_0, \rho)} \) 中除 \( z_0 \) 外没有其他的零点。记
\[
\min \{ |f(z) - w_0| : |z - z_0| = \rho \} = \delta > 0,
\]
于是当 \( |z - z_0| = \rho \) 时,\( |f(z) - w_0| \geq \delta \)。今任取 \( a \in B(w_0, \delta) \),则当 \( z \) 在圆周 \( |z - z_0| = \rho \) 上时,有
\begin{align}
|f(z) - w_0| \geq \delta > |w_0 - a|. \label{equation-----8888888}
\end{align}
若记 \( F(z) = f(z) - w_0 \),\( G(z) = f(z) - a \),则 \(\eqref{equation-----8888888}\) 式可写成
\[
|F(z)| > |F(z) - G(z)|.
\]
由 \hyperref[theorem:Rouché定理-定理4.4.3]{Rouché定理},\( F \) 和 \( G \) 在 \( B(z_0, \rho) \) 中的零点个数相同,因而 \( G(z) = f(z) - a \) 在 \( B(z_0, \rho) \) 中恰有 \( m \) 个零点。 
\end{proof}

\begin{corollary}\label{corollary:推论4.4.5}
设 \( f \in H(D) \),\( z_0 \in D \),\( w_0 = f(z_0) \),则对充分小的 \( \rho > 0 \),一定存在 \( \delta > 0 \),使得
\[
f(B(z_0, \rho)) \supset B(w_0, \delta).
\]
\end{corollary}
\begin{proof}
显然 \( z_0 \) 至少是 \( f(z) - w_0 \) 的 1 阶零点,由\refthe{theorem:定理4.4.4}可知,对充分小的 \( \rho > 0 \),一定存在 \( \delta > 0 \),使得对任意 \( a \in B(w_0, \delta) \),都有 \( f(z) - a \) 在 \( B(z_0, \rho) \) 中有一个零点 \( t_a \),即
\[
t_a \in B(z_0, \rho), \quad \text{且} \quad f(t_a) = a.
\]
再由 \( a \) 的任意性可知
\[
f(B(z_0, \rho)) \supset B(w_0, \delta).
\]
\end{proof}

\begin{theorem}[开映射定理]\label{theorem:定理4.4.6}
设 \( f \) 是域 \( D \) 上非常数的全纯函数,那么 \( f(D) \) 也是 \( \mathbb{C} \) 中的域。
\end{theorem}
\begin{remark}
如果 \( f \) 是域 \( D \) 上非常数的连续函数,那么 \( f(D) \) 未必是一个域。例如,函数 \( f(z) = |z| \) 是单位圆盘上的连续函数,它把单位圆盘映为线段 \([0, 1)\)。但是,由这个定理可知,域 \( D \) 上非常数的全纯函数则一定把域映为域。
\end{remark}
\begin{note}
这个定理说明\textbf{非常数的全纯函数把开集映为开集},因此称为\textbf{开映射定理}。
\end{note}
\begin{proof}
我们证明 \( f(D) \) 是 \( \mathbb{C} \) 中的连通开集。先证 \( f(D) \) 是开集。任取 \( w_0 \in f(D) \),由\refcor{corollary:推论4.4.5} 知,存在 \( \delta > 0 \),使得 \( B(w_0, \delta) \subset f(D) \),这说明 \( w_0 \) 是 \( f(D) \) 的内点,所以 \( f(D) \) 是开集。

再证 \( f(D) \) 是连通的。任取 \( w_1, w_2 \in f(D) \),则存在 \( z_1, z_2 \in D \),使得 \( f(z_1) = w_1 \),\( f(z_2) = w_2 \)。因为 \( D \) 是连通的,故在 \( D \) 中存在连续曲线 \( z = \gamma(t) \)(\( \alpha \leq t \leq \beta \))连接 \( z_1 \) 和 \( z_2 \),于是 \( w = f(\gamma(t)) \)(\( \alpha \leq t \leq \beta \))是 \( f(D) \) 中连接 \( w_1, w_2 \) 的曲线,因而 \( f(D) \) 是连通的。 
\end{proof}

\begin{theorem}\label{theorem:定理4.4.7}
如果 \( f \) 是域 \( D \) 中单叶的全纯函数,那么对于 \( D \) 内每一点 \( z \),有 \( f'(z) \neq 0 \)。
\end{theorem}
\begin{remark}
这个定理的逆定理是不成立的,即若 \( f' \) 在 \( D \) 中处处不为零,\( f \) 未必是 \( D \) 中的单叶函数。\( f(z) = \mathrm{e}^z \) 就是最简单的例子。
\end{remark}
\begin{proof}
用反证法。如果存在 \( z_0 \in D \),使得 \( f'(z_0) = 0 \),那么 \( z_0 \) 是 \( f(z) - f(z_0) \) 的 \( m \) 级零点,这里,\( m \geq 2 \)。因为$f'(z_0)\ne 0$,所以$f(z)$非常数.从而由\refpro{proposition:命题4.3.6},可取 \( \rho \) 充分小,使得 \( f'(z) \) 在 \( B(z_0, \rho) \) 中除了 \( z_0 \) 外不再有其他的零点。由\refthe{theorem:定理4.4.4},对于 \( 0 < \eta < \rho \),存在 \( \delta > 0 \),使得对任意 \( a \in B(f(z_0), \delta) \),\( f(z) - a \) 在 \( B(z_0, \eta) \) 中至少有两个零点,设为 \( z_1, z_2 \)。由于 \( f'(z_1) \neq 0 \),\( f'(z_2) \neq 0 \),故 \( z_1, z_2 \) 都是 \( f(z) - a \) 的 1 阶零点。这就是说,存在 \( z_1 \neq z_2 \),使得 \( f(z_1) = f(z_2) = a \),这与 \( f \) 的单叶性相矛盾。 
\end{proof}

\begin{theorem}\label{theorem:定理4.4.8}
设 \( f \) 是域 \( D \) 中的全纯函数,如果对于 \( z_0 \in D \),\( f'(z_0) \neq 0 \),那么 \( f \) 在 \( z_0 \) 的邻域中是单叶的。
\end{theorem}
\begin{proof}
因为 \( f'(z_0) \neq 0 \),所以 \( z_0 \) 是 \( f(z) - f(z_0) \) 的 1 阶零点。由\refthe{theorem:定理4.4.4},存在 \( \rho > 0 \) 和 \( \delta > 0 \),使得对于任意的 \( a \in B(f(z_0), \delta) \),\( f(z) - a \) 在 \( B(z_0, \rho) \) 中只有一个零点。由 \( f \) 的连续性,对$\delta>0$,存在\( \rho_1 < \rho \),使得对任意$z\in B(z_0,\rho_1)$,有
\[
|f(z)-f(z_0)|<\delta \iff f(z)\in B(f(z_0),\delta).
\]
即
\[
f(B(z_0, \rho_1)) \subset B(f(z_0), \delta),
\]
设$z_1,z_2\in B(z_0,\rho_1)$且$z_1\ne z_2$,则$f(z_1),f(z_2)\in f(B(z_0, \rho_1)) \subset B(f(z_0), \delta)$.又因为$f(z)-f(z_2)$在$B(z_0,\rho)$中只有一个零点$z_2$,所以
\[
f(z_1)\ne f(z_2).
\]
因而 \( f \) 在 \( B(z_0, \rho_1) \) 中是单叶的。 
\end{proof}

\begin{theorem}\label{theorem:定理4.4.9}
设 \( f \) 是域 \( D \) 上的单叶全纯函数,那么它的反函数 \( f^{-1} \) 是 \( G = f(D) \) 上的全纯函数,而且
\[
(f^{-1})'(w) = \frac{1}{f'(z)}, \, w \in G,
\]
其中,\( w = f(z) \)。
\end{theorem}
\begin{note}
由于这个定理,故单叶全纯函数也称为\textbf{双全纯函数}或\textbf{双全纯映射}。
\end{note}
\begin{proof}
先证明 \( f^{-1} \) 在 \( G \) 上连续。任取 \( w_0 \in G \),则存在 \( z_0 \in D \),使得 \( f(z_0) = w_0 \)。由\refthe{theorem:定理4.4.4}或\refcor{corollary:推论4.4.5},对于充分小的 \( \rho > 0 \),存在 \( \delta > 0 \),使得当 \( |w - w_0| < \delta \) 时,相应的 \( z \) 满足 \( |z - z_0| < \rho \),即 \( |f^{-1}(w) - f^{-1}(w_0)| < \rho \),这说明 \( f^{-1} \) 在 \( w_0 \) 处是连续的。现在
\[
\lim_{w \to w_0} \frac{f^{-1}(w) - f^{-1}(w_0)}{w - w_0} = \lim_{z \to z_0} \frac{z - z_0}{f(z) - f(z_0)} = \frac{1}{f'(z_0)},
\]
即
\[
(f^{-1})'(w_0) = \frac{1}{f'(z_0)}.
\]
这里,我们已经利用了\refthe{theorem:定理4.4.7}的结果。 
\end{proof}

\begin{theorem}[Hurwitz定理]\label{theorem:Hurwitz定理}
设 \( \{f_n\} \) 是域 \( D \) 中的一列全纯函数,它在 \( D \) 中内闭一致收敛到不恒为零的函数 \( f \)。设 \( \gamma \) 是 \( D \) 中一条可求长简单闭曲线,它的内部属于 \( D \),且不经过 \( f \) 的零点。那么必存在正整数 \( N \),当 \( n \geq N \) 时,\( f_n \) 与 \( f \) 在 \( \gamma \) 内部的零点个数相同。
\end{theorem}
\begin{proof}
由 \hyperref[theorem:Weierstrass定理]{Weierstrass 定理},\( f \) 在 \( D \) 中是全纯的。因为 \( f \) 在 \( \gamma \) 上没有零点,所以
\[
\min \{ |f(z)| : z \in \gamma \} = \varepsilon > 0.
\]
另一方面,对于上面的 \( \varepsilon > 0 \),存在正整数 \( N \),当 \( n \geq N \) 时,\( |f_n(z) - f(z)| < \varepsilon \) 在 \( \gamma \) 上成立。于是,当 \( n \geq N \) 时,在 \( \gamma \) 上有不等式
\[
|f(z)| \geq \varepsilon > |f_n(z) - f(z)|.
\]
根据 \hyperref[theorem:theorem:Rouché定理-定理4.4.3]{Rouché 定理},\( f \) 和 \( f_n \) 在 \( \gamma \) 内有相同个数的零点。
\end{proof}

\begin{theorem}\label{theorem:定理4.4.11 }
设 \( \{f_n\} \) 是域 \( D \) 上一列单叶的全纯函数,它在 \( D \) 上内闭一致收敛到 \( f \),如果 \( f \) 不是常数,那么 \( f \) 在 \( D \) 中也是单叶的全纯函数。
\end{theorem}
\begin{proof}
由 \hyperref[theorem:Weierstrass定理]{Weierstrass 定理},\( f \) 是 \( D \) 上的全纯函数。如果 \( f \) 在 \( D \) 上不是单叶的,那么一定存在 \( z_1, z_2 \),\( z_1 \neq z_2 \),使得 \( f(z_1) = f(z_2) \)。令
\[
F(z) = f(z) - f(z_1),
\]
那么 \( F \) 在 \( D \) 中有两个零点 \( z_1 \) 和 \( z_2 \)。因为 \( F \not\equiv 0 \),故由\refpro{proposition:命题4.3.6}可知\( z_1 \) 和 \( z_2 \) 是孤立的。选取充分小的 \( \varepsilon > 0 \),使得 \( B(z_1, \varepsilon) \cap B(z_2, \varepsilon) = \varnothing \),且 \( F \) 在 \( B(z_1, \varepsilon) \) 和 \( B(z_2, \varepsilon) \) 中除去 \( z_1 \) 和 \( z_2 \) 外不再有其他的零点。令
\[
F_n(z) = f_n(z) - f(z_1),
\]
则 \( F_n \) 在 \( D \) 中内闭一致收敛到 \( F \)。由 \hyperref[theorem:Hurwitz定理]{Hurwitz 定理},存在正整数 \( N \),当 \( n > N \) 时,\( F_n \) 在 \( B(z_1, \varepsilon) \) 和 \( B(z_2, \varepsilon) \) 中各有一个零点,设为 \( z_1' \) 和 \( z_2' \)。显然 \( z_1' \neq z_2' \),由此即得
\[
f_n(z_1') = f_n(z_2') = f(z_1).
\]
这与 \( f_n \) 在 \( D \) 内的单叶性相矛盾。 
\end{proof}

\begin{example}\label{example:例4.4.12}
求方程 \( z^8 - 4z^5 + z^2 - 1 = 0 \) 在单位圆内的零点个数。
\end{example}
\begin{solution}
令
\begin{gather*}
f(z) = -4z^5,
\\
g(z) = z^8 - 4z^5 + z^2 - 1.
\end{gather*}
在单位圆周上,\( |f(z)| = 4 \),于是
\[
|f(z) - g(z)| = |z^8 + z^2 - 1| \leq |z|^8 + |z|^2 + 1 = 3 < |f(z)|.
\]
根据 \hyperref[theorem:Rouché定理-定理4.4.3]{Rouché 定理},\( g \) 和 \( f \) 在 \( |z| < 1 \) 中的零点个数相同。而 \( f \) 在 \( z = 0 \) 处有 1 个 5 阶零点,因而原方程在 \( |z| < 1 \) 中有 5 个零点。 
\end{solution}

\begin{example}
试求方程 \( z^4 - 6z + 3 = 0 \) 在圆环 \( \{ z : 1 < |z| < 2 \} \) 中根的个数。
\end{example}
\begin{solution}
我们只需分别算出它在圆盘 \( |z| \leq 1 \) 和 \( |z| < 2 \) 中根的个数,二者之差即为在圆环 \( 1 < |z| < 2 \) 中根的个数。

利用\refexa{example:例4.4.12} 中的方法,容易知道原方程在 \( |z| < 1 \) 中只有 1 个根。而在圆周 \( |z| = 1 \) 上,由于
\[
|z^4 - 6z + 3| \geq 6 - |z^4 + 3| \geq 2,
\]
故在圆周 \( |z| = 1 \) 上不可能有零点。所以,在 \( |z| \leq 1 \) 中只有 1 个根。

为了计算 \( |z| < 2 \) 中根的个数,令 \( f(z) = z^4 \),\( g(z) = z^4 - 6z + 3 \),于是在圆周 \( |z| = 2 \) 上,有
\[
|f(z) - g(z)| \leq |6z| + 3 = 15 < 16 = |f(z)|.
\]
故由 \hyperref[theorem:Rouché定理-定理4.4.3]{Rouché 定理},\( g(z) = z^4 - 6z + 3 \) 和 \( f(z) = z^4 \) 在 \( |z| < 2 \) 中的零点个数相同,因而原方程在 \( |z| < 2 \) 中有 4 个根。由此即知原方程在圆环 \( 1 < |z| < 2 \) 中有 3 个根。 
\end{solution}

\begin{example}
证明:方程 \( z^4 + 2z^3 - 2z + 10 = 0 \) 在每个象限内各有一个根。
\end{example}
\begin{proof}
记
\[
P(z) = z^4 + 2z^3 - 2z + 10,
\]
我们直接用\hyperref[theorem:theorem:辐角原理-定理4.4.2]{辐角原理}来证明它在第一象限内只有一个零点。为此,取围道如\reffig{figure:图4.7}所示,为了应用\hyperref[theorem:theorem:辐角原理-定理4.4.2]{辐角原理},我们要证明 \( P \) 在 \( \gamma_1, \gamma_2, \gamma_3 \) 上都没有零点。当 \( R \) 充分大时,\( P \) 在 \( \gamma_2 \) 上没有零点是显然的。
\begin{figure}[H]
\centering
\includegraphics[scale=0.4]{图4.7.png}
\caption{}
\label{figure:图4.7}
\end{figure}
当 \( z \in \gamma_1 \) 时,\( z = x > 0 \),于是
\[
P(z) = P(x) = x^4 + 2x^3 - 2x + 10 = (x^2 - 1)(x + 1)^2 + 11.
\]
故当 \( x > 1 \) 时,\( P(x) > 11 \);当 \( 0 \leq x \leq 1 \) 时,\( P(x) \geq -2 + 11 = 9 \)。因此,\( P \) 在 \( \gamma_1 \) 上取正值。当 \( z \in \gamma_3 \) 时,\( z = iy \),\( y > 0 \),显然
\[
P(iy) = y^4 + 10 - 2iy(y^2 + 1) \neq 0.
\]
现在来计算 \( P \) 在 \( \gamma = \gamma_1 \cup \gamma_2 \cup \gamma_3 \) 上辐角的变化。由于 \( P \) 在 \( \gamma_1 \) 上取正值,所以
\begin{align}
\Delta_{\gamma_1} \text{Arg} P(z) = 0.\label{equation:0909090900999}
\end{align}
当 \( z \in \gamma_2 \) 时,有
\[
P(z) = z^4 \left( 1 + \frac{2z^3 - 2z + 10}{z^4} \right) = z^4 Q(z),
\]
这里,\( Q(z) = 1 + \frac{2z^3 - 2z + 10}{z^4} \)。当 \( |z| \) 充分大时,有
\[
|Q(z) - 1| = \left| \frac{2z^3 - 2z + 10}{z^4} \right| < 1,
\]
即 \( Q(z) \) 落在以 1 为中心、半径为 1 的圆内,所以 \( \Delta_{\gamma_2} \text{Arg} Q(z) = 0 \),于是
\begin{align}
\Delta_{\gamma_2} \text{Arg} P(z) = 4\Delta_{\gamma_2} \text{Arg} z + \Delta_{\gamma_2} \text{Arg} Q(z) = 2\pi. \label{equation:09090909009910}
\end{align}
当 \( z \in \gamma_3 \) 时,有
\begin{align}
\Delta_{\gamma_3} \text{Arg} P(z) &= \text{Arg} P(0) - \text{Arg} P(iR) \nonumber\\
&= -\text{Arg} \{ R^4 + 10 - 2iR(R^2 + 1) \} \nonumber\\
&= -\text{Arg} (R^4 + 10) \left( 1 - \frac{2iR(R^2 + 1)}{R^4 + 10} \right) \nonumber\\
&= -\text{Arg} \left( 1 - \frac{2iR(R^2 + 1)}{R^4 + 10} \right)\nonumber \\
&= 0 \, (R \text{ 充分大时}). \label{equation:09090909009911}
\end{align}
由 \eqref{equation:0909090900999},\eqref{equation:09090909009910} 和 \eqref{equation:09090909009911} 式即得
\[
\frac{1}{2\pi} \Delta_{\gamma} \text{Arg} P(z) = \frac{1}{2\pi} \Delta_{\gamma_2} \text{Arg} P(z) = 1.
\]
根据\hyperref[theorem:辐角原理-定理4.4.2]{辐角原理},\( P \) 在第一象限内只有一个零点。

由于 \( P \) 是实系数多项式,它的零点是共轭出现的,故在第四象限内也有一个零点。

用与前面相同的方法,可以证明 \( P \) 在负实轴上没有零点,因此剩下的两个零点当然就在第二、第三象限中了。 
\end{proof}





\end{document}
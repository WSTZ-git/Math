\documentclass[../../main.tex]{subfiles}
\graphicspath{{\subfix{../../image/}}} % 指定图片目录,后续可以直接使用图片文件名。

% 例如:
% \begin{figure}[H]
% \centering
% \includegraphics[scale=0.4]{图.png}
% \caption{}
% \label{figure:图}
% \end{figure}
% 注意:上述\label{}一定要放在\caption{}之后,否则引用图片序号会只会显示??.

\begin{document}

\section{幂级数}

\begin{definition}
所谓幂级数,是指形如
\begin{align}
\sum_{n=0}^{\infty} a_n(z - z_0)^n = a_0 + a_1(z - z_0) + \cdots  + a_n(z - z_0)^n + \cdots \label{eq::::1486}
\end{align}
的级数,它的通项是幂函数,这里,\( a_0, \cdots, a_n, \cdots \) 和 \( z_0 \) 都是复常数。
\end{definition}
\begin{remark}
为讨论简便起见,不妨假定 \( z_0 = 0 \),这时级数\eqref{eq::::1486} 成为
\begin{align}
\sum_{n=0}^{\infty} a_n z^n = a_0 + a_1 z + \cdots + a_n z^n + \cdots. \label{eq::::::::2222}
\end{align}
通常,只要作变换 \( w = z - z_0 \),就能把级数\eqref{eq::::1486}化为级数\eqref{eq::::::::2222}.
\end{remark}

\begin{definition}
如果存在常数 \( R \),使得当 \( |z| < R \) 时,级数 \eqref{eq::::::::2222} 收敛;当 \( |z| > R \) 时,级数\eqref{eq::::::::2222} 发散,就称 \( R \) 为级数 \eqref{eq::::::::2222} 的\textbf{收敛半径},\( \{z : |z| < R\} \) 称为级数\eqref{eq::::::::2222}的\textbf{收敛圆}。
\end{definition}

\begin{theorem}\label{theorem:定理4.2.2}
级数 \eqref{eq::::::::2222}存在收敛半径
\[
R=\frac{1}{\underset{n\rightarrow \infty}{\overline{\lim }}\sqrt[n]{|a_n|}}.
\]
\end{theorem}
\begin{proof}
我们要证明下列三件事:
\begin{enumerate}[(i)]
\item 先证(i) 当 \( R = 0 \) 时,\( \sum_{n=0}^{\infty} a_n z^n \) 只在 \( z = 0 \) 处收敛;

\item 当 \( R = \infty \) 时,\( \sum_{n=0}^{\infty} a_n z^n \) 在 \( \mathbb{C} \) 中处处收敛;

\item 当 \( 0 < R < \infty \) 时,\( \sum_{n=0}^{\infty} a_n z^n \) 在 \( \{z : |z| < R\} \) 中收敛,在 \( \{z : |z| > R\} \) 中发散。
\end{enumerate}
先证(i).级数 \( \sum_{n=0}^{\infty} a_n z^n \) 在 \( z = 0 \) 处收敛是显然的。现固定 \( z \neq 0 \),由于 \( \varlimsup_{n \to \infty} \sqrt[n]{|a_n|} = \infty \),故必有子列 \( n_k \),使得 \( \sqrt[n_k]{|a_{n_k}|} > \frac{1}{|z|} \),于是 \( |a_{n_k} z^{n_k}| > 1 \)。所以,由\refcor{corollary:复幂级数收敛必要条件}可知级数 \( \sum_{n=0}^{\infty} a_n z^n \) 发散。

再证(ii).任取 \( z \neq 0 \),因为 \( \varlimsup_{n \to \infty} \sqrt[n]{|a_n|} = 0 \),对于 \( \varepsilon = \frac{1}{2|z|} \),存在正整数 \( N \),当 \( n > N \) 时,\( \sqrt[n]{|a_n|} < \frac{1}{2|z|} \),于是 \( |a_n z^n| < \frac{1}{2^n} \)。所以,由\hyperref[theorem:Weierstrass一致收敛判别法]{Weierstrass一致收敛判别法}可知,级数 \( \sum_{n=1}^{\infty} a_n z^n \) 一致收敛,从而也收敛.

最后证(iii).取定 \( z \neq 0 \),\( z \in B(0, R) \)。选取 \( \rho \),使得 \( |z| < \rho < R \)。于是 \( \varlimsup_{n \to \infty} \sqrt[n]{|a_n|} = \frac{1}{R} < \frac{1}{\rho} \),因而存在 \( N \),当 \( n > N \) 时,\( \sqrt[n]{|a_n|} < \frac{1}{\rho} \),即 \( |a_n z^n| < \left( \frac{|z|}{\rho} \right)^n<1 \)。所以 由\hyperref[theorem:Weierstrass一致收敛判别法]{Weierstrass一致收敛判别法}可知,\( \sum_{n=0}^{\infty} |a_n z^n|\)一致收敛,从而\( \sum_{n=0}^{\infty} |a_n z^n| < \infty \).

再设 \( |z| > R \),选取 \( r \),使得 \( |z| > r > R \)。因而 \( \varlimsup_{n \to \infty} \sqrt[n]{|a_n|} = \frac{1}{R} > \frac{1}{r} \),故有 \( \{n_k\} \),使得 \( \sqrt[n_k]{|a_{n_k}|} > \frac{1}{r} \),即 \( |a_{n_k} z^{n_k}| > \left( \frac{|z|}{r} \right)^{n_k} > 1 \)。故由\refcor{corollary:复幂级数收敛必要条件}可知级数 \( \sum_{n=0}^{\infty} a_n z^n \) 发散。
\end{proof}

\begin{theorem}[Abel定理]\label{theorem:Abel定理-定理4.2.3}
如果 \( \sum_{n=0}^{\infty} a_n z^n \) 在 \( z = z_0 \neq 0 \) 处收敛,则必在 \( \{z : |z| < |z_0|\} \) 中内闭绝对一致收敛。
\end{theorem}
\begin{proof}
设 \( K \) 是 \( \{z : |z| < |z_0|\} \) 中的一个紧集,选取 \( r < |z_0| \),使得 \( K \subset B(0, r) \)。于是,当 \( z \in K \) 时,有 \( |z| < r \)。因为 \( \sum_{n=0}^{\infty} a_n z_0^n \) 收敛,所以由\refcor{corollary:复幂级数收敛必要条件}可知 \( |a_n z_0^n| < M \),这里,\( M \) 是一个常数。于是,当 \( z \in K \) 时,有
\begin{align*}
|a_n z^n| = \left| a_n z_0^n \frac{z^n}{z_0^n} \right| \leqslant M \frac{|z|^n}{|z_0|^n} \leqslant M \left( \frac{r}{|z_0|} \right)^n.
\end{align*}
因为 \( r < |z_0| \),所以由\hyperref[theorem:Weierstrass一致收敛判别法]{Weierstrass一致收敛判别法},\( \sum_{n=0}^{\infty} |a_n z^n| \) 在 \( K \) 中一致收敛。
\end{proof}

\begin{theorem}\label{theorem:定理4.2.4}
幂级数在其收敛圆内确定一个全纯函数,即幂级数的和函数在其收敛圆内必是全纯函数.
\end{theorem}
\begin{proof}
由\hyperref[theorem:Abel定理-定理4.2.3]{Abel定理}知道,幂级数在其收敛圆内是内闭一致收敛的。根据 \hyperref[theorem:Weierstrass定理]{Weierstrass 定理},它的和函数是收敛圆内的全纯函数。
\end{proof}

\begin{example}
级数 \( \sum_{n=0}^{\infty} z^n \) 的收敛半径为1,它在收敛圆周 \( |z| = 1 \) 上处处发散。
\end{example}

\begin{example}
级数 \( \sum_{n=1}^{\infty} \frac{z^n}{n^2} \) 的收敛半径为1,它在收敛圆周 \( |z| = 1 \) 上处处收敛。
\end{example}

\begin{example}
级数 \( \sum_{n=1}^{\infty} \frac{z^n}{n} \) 的收敛半径为1,它在 \( z = 1 \) 处是发散的,但在收敛圆周的其他点 \( z = \mathrm{e}^{\mathrm{i}\theta} \)(\( 0 < \theta < 2\pi \))处则是收敛的。
\end{example}
\begin{proof}
这是因为
\begin{align*}
\sum_{n=1}^{\infty} \frac{z^n}{n} = \sum_{n=1}^{\infty} \frac{\mathrm{e}^{\mathrm{i}n\theta}}{n} = \sum_{n=1}^{\infty} \frac{\cos n\theta}{n} + \mathrm{i}\sum_{n=1}^{\infty} \frac{\sin n\theta}{n},
\end{align*}
由 Dirichlet 判别法知道,实部和虚部的两个级数都是收敛的。
\end{proof}

\begin{theorem}\label{theorem:幂级数在其收敛圆内确定一个全纯函数并且任意阶可导}
设 \( \sum_{n=0}^{\infty} a_n(z - z_0)^n \) 的收敛半径为 \( R \),则其和函数
\[
f(z) = \sum_{n=0}^{\infty} a_n(z - z_0)^n
\]
是圆盘 \( B(z_0, R) \) 中的全纯函数,并且
\begin{gather*}
f'(z) = \sum_{n=1}^{\infty} n a_n(z - z_0)^{n - 1},
\\
\cdots\cdots\cdots\cdots,
\\
f^{(k)}(z) = \sum_{n=k}^{\infty} n(n - 1)\cdots(n - k + 1) a_n(z - z_0)^{n - k},
\\
\cdots\cdots\cdots\cdots.
\end{gather*}
\end{theorem}
\begin{proof}
由\refthe{theorem:定理4.2.4},和函数
\[
f(z) = \sum_{n=0}^{\infty} a_n(z - z_0)^n
\]
是圆盘 \( B(z_0, R) \) 中的全纯函数.\nrefpro{proposition:全纯函数必任意连续可微}{(3)}可知$f\in C^{\infty}$.再由\hyperref[theorem:Weierstrass定理]{Weierstrass定理},得
\begin{gather*}
f'(z) = \sum_{n=1}^{\infty} n a_n(z - z_0)^{n - 1},
\\
\cdots\cdots\cdots\cdots,
\\
f^{(k)}(z) = \sum_{n=k}^{\infty} n(n - 1)\cdots(n - k + 1) a_n(z - z_0)^{n - k},
\\
\cdots\cdots\cdots\cdots.
\end{gather*}
\end{proof}



现若 \( \sum_{n=0}^{\infty} a_n(z - z_0)^n \) 在收敛圆周 \( |z - z_0| = R \) 上某点 \( \zeta \) 处收敛,那么 \( \sum_{n=0}^{\infty} a_n(\zeta - z_0)^n \) 和 \( f \) 有什么关系呢?为了简化问题的讨论,作变换 \( w = \frac{z - z_0}{\zeta - z_0} \),那么
\begin{align*}
\sum_{n=0}^{\infty} a_n(z - z_0)^n = \sum_{n=0}^{\infty} a_n(\zeta - z_0)^n w^n = \sum_{n=0}^{\infty} b_n w^n,
\end{align*}
这里,\( b_n = a_n(\zeta - z_0)^n \).$\,\,$\( \sum_{n=0}^{\infty} b_n w^n \) 的收敛半径为
\begin{align*}
\frac{1}{\underset{n\rightarrow \infty}{\overline{\lim }}\sqrt[n]{|b_n|}}=\frac{1}{|\zeta -z_0|}\cdot \frac{1}{\underset{n\rightarrow \infty}{\overline{\lim }}\sqrt[n]{|a_n|}}=\frac{1}{R}\cdot R=1,
\end{align*}
且在 \( w = 1 \) 处收敛。因此,不妨就收敛半径为1,且在 \( z = 1 \) 处收敛的幂级数 \( \sum_{n=0}^{\infty} a_n z^n \) 来讨论。

\begin{definition}
设 \( g \) 是定义在单位圆中的函数,\( \mathrm{e}^{\mathrm{i}\theta_0} \) 是单位圆周上一点,\( S_{\alpha}(\mathrm{e}^{\mathrm{i}\theta_0}) \) 如\reffig{figure:图4.1}所示,其中 \( \alpha < \frac{\pi}{2} \)。如果当 \( z \) 在 \( S_{\alpha}(\mathrm{e}^{\mathrm{i}\theta_0}) \) 中趋于 \( \mathrm{e}^{\mathrm{i}\theta_0} \) 时,\( g(z) \) 有极限 \( l \),就称 \( g \) 在 \( \mathrm{e}^{\mathrm{i}\theta_0} \) 处有\textbf{非切向极限} \( l \),记为
\[
\lim_{\substack{z \to \mathrm{e}^{\mathrm{i}\theta_0} \\ z \in S_{\alpha}(\mathrm{e}^{\mathrm{i}\theta_0})}} g(z) = l.
\]
\end{definition}
\begin{figure}[H]
\centering
\includegraphics[scale=0.4]{图4.1.png}
\caption{}
\label{figure:图4.1}
\end{figure}

\begin{theorem}[Abel第二定理]\label{theorem:Abel第二定理}
设 \( f(z) = \sum_{n=0}^{\infty} a_n z^n \) 的收敛半径 \( R = 1 \),且级数在 \( z = 1 \) 处收敛于 \( S \),那么 \( f \) 在 \( z = 1 \) 处有非切向极限 \( S \),即
\begin{align}
\lim_{\substack{z \to 1 \\ z \in S_{\alpha}(1)}} f(z) = S .\label{eq::----4.4.4.4.3}
\end{align}
\end{theorem}
\begin{proof}
如\reffig{figure:图4.2}所示,只要能证明级数 \( \sum_{n=0}^{\infty} a_n z^n \) 在 \( S_{\alpha}(1) \cap B(1, \delta) \)(这里,\( \delta = \cos\alpha \))的闭包上一致收敛,那么由\hyperref[theorem:Weierstrass定理]{Weierstrass定理}可知 \( f(z) \) 便在 \( z = 1 \) 处连续,因而 \eqref{eq::----4.4.4.4.3} 式成立。
\begin{figure}[H]
\centering
\includegraphics[scale=0.3]{图4.2.png}
\caption{}
\label{figure:图4.2}
\end{figure}
记
\[
\sigma_{n,p} = a_{n+1} + \cdots + a_{n+p},
\]
因为 \( \sum_{n=0}^{\infty} a_n z^n \) 在 \( z = 1 \) 处收敛,即 \( \sum_{n=0}^{\infty} a_n \) 收敛,故对任给的 \( \varepsilon > 0 \),存在正整数 \( N \),当 \( n > N \) 时,\( |\sigma_{n,p}| < \varepsilon \) 对任意自然数 \( p \) 成立。注意
\[
\begin{aligned}
a_{n+1} z^{n+1} + \cdots + a_{n+p} z^{n+p} &= \sigma_{n,1} z^{n+1} + (\sigma_{n,2} - \sigma_{n,1}) z^{n+2} + \cdots + (\sigma_{n,p} - \sigma_{n,p-1}) z^{n+p} \\
&= \sigma_{n,1} z^{n+1} (1 - z) + \sigma_{n,2} z^{n+2} (1 - z)  + \cdots + \sigma_{n,p-1} z^{n+p-1} (1 - z) + \sigma_{n,p} z^{n+p}
\\
&= z^{n+1} (1 - z) (\sigma_{n,1} + \sigma_{n,2} z + \cdots  + \sigma_{n,p-1} z^{p-2}) + \sigma_{n,p} z^{n+p}.
\end{aligned}
\]
因而当 \( |z| < 1 \),\( p = 1, 2, \cdots \),\( n > N \) 时,便有
\begin{align}
|a_{n+1} z^{n+1} + \cdots + a_{n+p} z^{n+p}| < \varepsilon |1 - z| (1 + |z| + \cdots) + \varepsilon \xlongequal{\text{Taylor公式}}\varepsilon \left( \frac{|1 - z|}{1 - |z|} + 1 \right).\label{eq::4.4.4.4.44.4.4.}
\end{align}
现在任取 \( z \in S_{\alpha}(1) \cap B(1, \delta) \),记 \( |z| = r \),\( |1 - z| = \rho \),那么
\[
r^2 = 1 + \rho^2 - 2\rho \cos\theta.
\]
故有
\[
\begin{aligned}
\frac{|1 - z|}{1 - |z|} = \frac{\rho}{1 - r} = \frac{\rho (1 + r)}{1 - r^2} \leqslant  \frac{2\rho}{2\rho \cos\theta - \rho^2} = \frac{2}{2\cos\theta - \rho}.
\end{aligned}
\]
因为 \( z \in B(1, \delta) \),所以 \( \rho = |1 - z| < \delta = \cos\alpha \)。又因 \( \theta < \alpha \),所以
\[
\frac{|1 - z|}{1 - |z|} \leqslant  \frac{2}{2\cos\alpha - \rho} < \frac{2}{\cos\alpha}
\]
由 \eqref{eq::4.4.4.4.44.4.4.} 式便可得
\[
|a_{n+1} z^{n+1} + \cdots + a_{n+p} z^{n+p}| < \varepsilon \left( \frac{2}{\cos\alpha} + 1 \right)
\]
又当 \( z = 1 \) 时,有
\[
|a_{n+1} z^{n+1} + \cdots + a_{n+p} z^{n+p}| = |\sigma_{n,p}| < \varepsilon
\]
这样,我们就证明了级数 \( \sum_{n=0}^{\infty} a_n z^n \) 在 \( S_{\alpha}(1) \cap B(1, \delta) \) 的闭包上一致收敛,因而 \eqref{eq::----4.4.4.4.3} 式成立。
\end{proof}

\begin{example}\label{example:4.2.10}
计算级数$\sum_{n=1}^{\infty} \frac{z^n}{n}$的和.
\end{example}
\begin{solution}
容易知道该级数的收敛半径为1,所以它的和$f(z)$是单位圆盘中的全纯函数,因而有
\[
f(z) = \sum_{n=1}^{\infty} \frac{z^n}{n},
\quad
f'(z) = \sum_{n=1}^{\infty} z^{n-1} = \frac{1}{1 - z}.
\]
由此得
\[
f(z) = -\log(1 - z),
\]
即
\[
\sum_{n=1}^{\infty} \frac{z^n}{n} = -\log(1 - z),\ |z| < 1. 
\]

这个级数在收敛圆周上除了点$z = 1$外都收敛,故由\hyperref[theorem:Abel第二定理]{Abel第二定理},当$z = \mathrm{e}^{\mathrm{i}\theta}\ (0 < \theta < 2\pi)$时,有
\begin{align}
\sum_{n=1}^{\infty} \frac{\mathrm{e}^{\mathrm{i}n\theta}}{n} = -\log(1 - \mathrm{e}^{\mathrm{i}\theta}) = -\log|1 - \mathrm{e}^{\mathrm{i}\theta}| - \mathrm{i}\arg(1 - \mathrm{e}^{\mathrm{i}\theta}). \label{eq:2-=------2}
\end{align}
\end{solution}
\begin{remark}
\begin{figure}[H]
\centering
\includegraphics[scale=0.3]{图4.3.png}
\caption{}
\label{figure:图4.3}
\end{figure}
从\reffig{figure:图4.3}容易看出
\[
|1 - \mathrm{e}^{\mathrm{i}\theta}| = 2\sin \frac{\theta}{2},\quad
\arg(1 - \mathrm{e}^{\mathrm{i}\theta}) = -\varphi,
\]
但$2\varphi = \pi - \theta$, $\varphi = \frac{\pi - \theta}{2}$.这样,由\eqref{eq:2-=------2}式便可得
\[
\sum_{n=1}^{\infty} \frac{\cos n\theta}{n} = -\log\left(2\sin \frac{\theta}{2}\right),
\quad
\sum_{n=1}^{\infty} \frac{\sin n\theta}{n} = \frac{\pi - \theta}{2}.
\]
上面两个等式都在$0 < \theta < 2\pi$中成立.特别地,当$\theta = \pi$时,得
\[
\sum_{n=1}^{\infty} \frac{(-1)^{n-1}}{n} = \log 2;
\]
当$\theta = \frac{\pi}{2}$时,由于
\[
\sin \frac{n\pi}{2} = 
\begin{cases} 
0, & n = 2k; \\
(-1)^k, & n = 2k + 1,
\end{cases}
\]
所以得
\[
\sum_{k=0}^{\infty} \frac{(-1)^k}{2k + 1} = \frac{\pi}{4}.
\]
\end{remark}


















\end{document}
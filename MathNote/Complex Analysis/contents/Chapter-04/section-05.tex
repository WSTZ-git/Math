\documentclass[../../main.tex]{subfiles}
\graphicspath{{\subfix{../../image/}}} % 指定图片目录,后续可以直接使用图片文件名。

% 例如:
% \begin{figure}[H]
% \centering
% \includegraphics[scale=0.4]{图.png}
% \caption{}
% \label{figure:图}
% \end{figure}
% 注意:上述\label{}一定要放在\caption{}之后,否则引用图片序号会只会显示??.

\begin{document}

\section{最大模原理和Schwarz引理}

\begin{theorem}[最大模原理]\label{theorem:最大模原理-定理4.5.1}
设 \( f \) 是域 \( D \) 中非常数的全纯函数,那么 \( |f(z)| \) 不可能在 \( D \) 中取到最大值。
\end{theorem}
\begin{proof}
因为 \( f \) 是 \( D \) 上非常数的全纯函数,由\refthe{theorem:定理4.4.6},\( G = f(D) \) 是一个域。如果 \( |f(z)| \) 在 \( D \) 中某点 \( z_0 \) 处取到最大值,记 \( w_0 = f(z_0) \),则 \( w_0 \) 是 \( G \) 的一个内点,即有 \( \varepsilon > 0 \),使得 \( B(w_0, \varepsilon) \subset G \)。故必有 \( w_1 \in G \),使得 \( |w_1| > |w_0| \)。于是存在 \( z_1 \in D \),使得 \( |f(z_1)| = |w_1| > |w_0| = |f(z_0)| \)。这与 \( |f(z_0)| \) 是 \( |f(z)| \) 在 \( D \) 中的最大值相矛盾。 

\end{proof}

\begin{theorem}\label{theorem:定理4.5.2}
设 \( D \) 是 \( \mathbb{C} \) 中的有界域,如果非常数的函数 \( f \) 在 \( \overline{D} \) 上连续,在 \( D \) 内全纯,那么 \( f \) 的最大模在 \( D \) 的边界上而且只在 \( D \) 的边界上达到。
\end{theorem}
\begin{remark}
\refthe{theorem:定理4.5.2}中 \( D \) 的有界性条件不能去掉,否则定理可能不成立。例如,设
\[
D = \left\{ z : |\text{Im} z| < \frac{\pi}{2} \right\},
\quad
f(z) = \mathrm{e}^{\mathrm{e}^z}.
\]
当然 \( f \) 在 \( D \) 中全纯,在 \( \overline{D} \) 上连续,但它的最大模并不能在 \( \partial D \) 上达到。事实上,当 \( z \in \partial D \) 时,\( z = x \pm \frac{\pi}{2}i \),这时,\( \mathrm{e}^z = \mathrm{e}^x \mathrm{e}^{\pm \frac{\pi}{2}i} = \pm i\mathrm{e}^x \),所以 \( |\mathrm{e}^{\mathrm{e}^z}| = |\mathrm{e}^{\pm i\mathrm{e}^x}| = 1 \)。而当 \( z \in D \) 时,取 \( z = x \),即有 \( \mathrm{e}^{\mathrm{e}^x} \to \infty \)(\( x \to \infty \))。故\refthe{theorem:定理4.5.2}不成立。
\end{remark}
\begin{proof}
因为 \( \overline{D} \) 是紧集,其上的连续函数 \( |f| \) 一定有最大值,即存在 \( z_0 \in \overline{D} \),使得 \( |f(z_0)| \) 是 \( |f(z)| \) 在 \( \overline{D} \) 上的最大值。由\refthe{theorem:最大模原理-定理4.5.1} 知道,\( z_0 \) 不能属于 \( D \),因此只能有 \( z_0 \in \partial D \)。 

\end{proof}

\begin{theorem}[Schwarz引理]\label{theorem:Schwarz引理-定理4.5.3}
设 \( f \) 是单位圆盘 \( B(0,1) \) 中的全纯函数,且满足条件

(i) 当 \( z \in B(0,1) \) 时,\( |f(z)| \leqslant  1 \);

(ii) \( f(0) = 0 \),

那么下列结论成立:

(i) 对于任意 \( z \in B(0,1) \),均有 \( |f(z)| \leqslant  |z| \);

(ii) \( |f'(0)| \leqslant  1 \);

(iii) 如果存在某点 \( z_0 \in B(0,1) \),\( z_0 \neq 0 \),使得 \( |f(z_0)| = |z_0| \),或者 \( |f'(0)| = 1 \) 成立,那么存在实数 \( \theta \),使得对 \( B(0,1) \) 中所有的 \( z \),都有 \( f(z) = \mathrm{e}^{\mathrm{i}\theta} z \)。
\end{theorem}
\begin{proof}
因为 \( f \in H(B(0,1)) \),且 \( f(0) = 0 \),故 \( f \) 在 \( B(0,1) \) 中可展开为
\begin{align*}
f(z) = a_1 z + a_2 z^2 + \cdots = z(a_1 + a_2 z + \cdots) = z g(z),
\end{align*}
这里,\( g(0) = a_1 = f'(0) \)。取 \( 0 < r < 1 \),当 \( |z| = r \) 时,有
\[
|g(z)| = \frac{|f(z)|}{|z|} \leqslant  \frac{1}{r},
\]
故由最大模原理,在圆盘 \( B(0,r) \) 中也有
\[
|g(z)| \leqslant  \frac{1}{r} \, (\text{当} \ |z| < r \ \text{时}).
\]
让 \( r \to 1 \),即得 \( |g(z)| \leqslant  1 \)(\( z \in B(0,1) \)),即 \( |f(z)| \leqslant  |z| \),结论(i)成立。

从 \( |g(0)| \leqslant  1 \) 即得 \( |f'(0)| \leqslant  1 \),结论(ii)成立。

现若有 \( z_0 \in B(0,1) \),\( z_0 \neq 0 \),使得 \( |f(z_0)| = |z_0| \),即 \( |g(z_0)| = 1 \)。这说明全纯函数 \( g \) 在内点 \( z_0 \) 处取到了最大模 1,根据\hyperref[theorem:最大模原理-定理4.5.1]{最大模原理},\( g \) 必须是常数。设 \( g(z) \equiv c \),由 \( |g(z_0)| = 1 \),得 \( |c| = 1 \),所以 \( c = \mathrm{e}^{\mathrm{i}\theta} \),因而 \( f(z) = \mathrm{e}^{\mathrm{i}\theta} z \)。如果 \( |f'(0)| = 1 \),即 \( |g(0)| = 1 \),与上面一样讨论,即得 \( f(z) = \mathrm{e}^{\mathrm{i}\theta} z \)。结论(iii)成立。

\end{proof}

\begin{definition}
设 \( D \) 是 \( \mathbb{C} \) 中的域,如果 \( f \) 是 \( D \) 上的单叶全纯函数,且 \( f(D) = D \),就称 \( f \) 是 \( D \) 上的一个\textbf{全纯自同构}。\( D \) 上全纯自同构的全体记为 \( \text{Aut}(D) \)。
\end{definition}

\begin{proposition}
\( \text{Aut}(D) \) 在复合运算下构成一个群,称为 \( D \) 的\textbf{全纯自同构群}。
\end{proposition}
\begin{proof}
设 \( f, g \in \text{Aut}(D) \),那么 \( f \circ g \in \text{Aut}(D) \),且复合运算满足结合律。对于每个 \( f \in \text{Aut}(D) \),由\refthe{theorem:定理4.4.9},\( f^{-1} \in \text{Aut}(D) \)。\( f(z) = z \) 在复合运算下起着单位元素的作用。因而 \( \text{Aut}(D) \) 在复合运算下构成一个群.

\end{proof}

对于一般的域 \( D \),要确定 \( \text{Aut}(D) \) 是很困难的。但对于单位圆盘 \( B(0,1) \),应用 \hyperref[theorem:Schwarz引理-定理4.5.3]{Schwarz引理}不难定出其上的全纯自同构群。

对于 \( |a| < 1 \),记
\[
\varphi_a(z) = \frac{a - z}{1 - \overline{a} z},
\]
由\refexa{example:例2.5.16} 知道,它把 \( B(0,1) \) 一一地映为 \( B(0,1) \),因而 \( \varphi_a \in \text{Aut}(B(0,1)) \)。如果记 \( \rho_\theta(z) = \mathrm{e}^{\mathrm{i}\theta} z \),它是一个旋转变换,当然有 \( \rho_\theta \in \text{Aut}(B(0,1)) \)。下面我们将证明,\( \text{Aut}(B(0,1)) \) 中除了 \( \varphi_a, \rho_\theta \) 以及它们的复合外,不再有其他的变换。

\begin{theorem}\label{theorem:定理4.5.5}
设 \( f \in \text{Aut}(B(0,1)) \),且 \( f^{-1}(0) = a \),则必存在 \( \theta \in \mathbb{R} \),使得
\[
f(z) = \mathrm{e}^{\mathrm{i}\theta} \frac{a - z}{1 - \overline{a} z}.
\]
\end{theorem}
\begin{proof}
记 \( w = \varphi_a(z)= \frac{a - z}{1 - \overline{a} z} \),直接计算可得
\begin{align}\label{equation::::14865}
z = \varphi_a^{-1}(w) = \frac{a - w}{1 - \overline{a} w} = \varphi_a(w).
\end{align}
令 \( g(w) = f \circ \varphi_a(w) \),则由\refexa{example:例2.5.16} 知道 \( g \in \text{Aut}(B(0,1)) \),而且
\[
g(0) = f(\varphi_a(0)) = f(a) = 0,
\]
故由 \hyperref[theorem:Schwarz引理-定理4.5.3]{Schwarz引理}得
\begin{align}
|g'(0)| \leqslant  1.\label{eq::::1--1---1-1-1-1}
\end{align}
由于 \( g^{-1} \in \text{Aut}(B(0,1)) \),且 \( g^{-1}(0) = 0 \),故对 \( g^{-1} \) 用\hyperref[theorem:Schwarz引理-定理4.5.3]{Schwarz引理},得 \( |(g^{-1})'(0)| \leqslant  1 \)。但由\refthe{theorem:定理4.4.9},有
\[
|(g^{-1})'(0)| = \frac{1}{|g'(0)|},
\]
由此即得
\[
|g'(0)| \geqslant  1.
\]
与 \(\eqref{eq::::1--1---1-1-1-1}\) 式比较,即得 \( |g'(0)| = 1 \)。根据 \hyperref[theorem:Schwarz引理-定理4.5.3]{Schwarz引理的结论(iii)},存在实数 \( \theta \),使得 \( g(w) = \mathrm{e}^{\mathrm{i}\theta} w \),即 \( f \circ \varphi_a(w) = \mathrm{e}^{\mathrm{i}\theta} w \)。令 \( w = \varphi_a(z) \),再结合\eqref{equation::::14865}式即得
\[
f(z) = \mathrm{e}^{\mathrm{i}\theta} \frac{a - z}{1 - \overline{a} z}.
\] 

\end{proof}

\begin{theorem}[Schwarz-Pick定理]\label{theorem:Schwarz-Pick定理-定理4.5.6}
设 \( f: B(0,1) \to B(0,1) \) 是全纯函数,对于 \( a \in B(0,1) \),\( f(a) = b \)。那么

(i) 对任意 \( z \in B(0,1) \),有 \( |\varphi_b(f(z))| \leqslant  |\varphi_a(z)| \)其中$\varphi_a(z) = \frac{a - z}{1 - \overline{a} z},\varphi_b(z) = \frac{b - z}{1 - \overline{b} z}$;

(ii) \( |f'(a)| \leqslant  \frac{1 - |b|^2}{1 - |a|^2} \);

(iii) 如果存在某点 \( z_0 \in B(0,1) \),\( z_0 \neq a \),使得 \( |\varphi_b(f(z_0))| = |\varphi_a(z_0)| \),或者 \( |f'(a)| = \frac{1 - |b|^2}{1 - |a|^2} \) 成立,
那么 \( f \in \text{Aut}(B(0,1)) \)。其中$\varphi_a(z) = \frac{a - z}{1 - \overline{a} z},\varphi_b(z) = \frac{b - z}{1 - \overline{b} z}.$
\end{theorem}
\begin{proof}
令 \( g = \varphi_b \circ f \circ \varphi_a \),则 \( g \in H(B(0,1)) \),且 \( g(B(0,1)) \subset B(0,1) \),\( g(0) = \varphi_b \circ f \circ \varphi_a(0) = 0 \)。对 \( g \) 用\hyperref[theorem:Schwarz引理-定理4.5.3]{Schwarz引理},有
\begin{align}
|\varphi_b \circ f \circ \varphi_a(\zeta)| \leqslant  |\zeta|, \, \zeta \in B(0,1) \label{eq::::--2-2-2-2-2}
\end{align}
和
\begin{align}
|(\varphi_b \circ f \circ \varphi_a)'(0)| \leqslant  1. \label{eq::::--2-2-2-2-3}
\end{align}
令 \( z = \varphi_a(\zeta) \),则 \( \zeta = \varphi_a(z) \),于是 \(\eqref{eq::::--2-2-2-2-2}\) 式变成
\begin{align}
|\varphi_b(f(z))| \leqslant  |\varphi_a(z)|. \label{eq::::--2-2-2-2-4}
\end{align}
这就是(i)。

由于
\[
\varphi_a'(0) = - (1 - |a|^2),
\quad
\varphi_b'(b) = - \frac{1}{1 - |b|^2},
\]
由 \(\eqref{eq::::--2-2-2-2-3}\) 式即得
\begin{align}
|f'(a)| \leqslant  \frac{1 - |b|^2}{1 - |a|^2}. \label{eq::::--2-2-2-2-5}
\end{align}
这就是(ii)。

如果存在 \( z_0 \in B(0,1) \),\( z_0 \neq a \),使得 \(\eqref{eq::::--2-2-2-2-4}\) 式中的等号成立,令 \( \zeta_0 = \varphi_a(z_0) \),则 \( \zeta_0 \neq 0 \),且 \( \zeta_0 \) 使 \(\eqref{eq::::--2-2-2-2-2}\) 式中的等号成立。于是由 \hyperref[theorem:Schwarz引理-定理4.5.3]{Schwarz引理},\( g(z) = \mathrm{e}^{\mathrm{i}\theta} z \),即 \( g \in \text{Aut}(B(0,1)) \),于是 \( f = \varphi_b \circ g \circ \varphi_a \in \text{Aut}(B(0,1)) \)。

注意到当\(\eqref{eq::::--2-2-2-2-5}\) 式中的等号成立时,有
\begin{align*}
g' (0)&=\varphi _{b}'\left[ f\left( \varphi _a\left( 0 \right) \right) \right] \cdot f\prime \left( \varphi _a\left( 0 \right) \right) \cdot \varphi _{a}'\left( 0 \right) 
=\varphi _{b}'\left[ f\left( a \right) \right] \cdot f\prime \left( a \right) \cdot \varphi _{a}'\left( 0 \right) 
\\
&=\varphi _{b}'\left[ b \right] \cdot f\prime \left( a \right) \cdot \varphi _{a}'\left( 0 \right) 
=-\frac{1}{1-|b|^2}\cdot \frac{1-|b|^2}{1-|a|^2}\cdot \left[ -\left( 1-|a|^2 \right) \right] =1,
\end{align*}
由 \hyperref[theorem:Schwarz引理-定理4.5.3]{Schwarz引理},\( g(z) = \mathrm{e}^{\mathrm{i}\theta} z \),即 \( g \in \text{Aut}(B(0,1)) \),于是 \( f = \varphi_b \circ g \circ \varphi_a \in \text{Aut}(B(0,1)) \)。

\end{proof}





\end{document}
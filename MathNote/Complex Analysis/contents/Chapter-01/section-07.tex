\documentclass[../../main.tex]{subfiles}% 注意这里的文件路径不能用 ./main.tex ,否则用latexmk编译子文件会报错
\graphicspath{{\subfix{./image/}}} % 指定图片目录,后续可以直接使用图片文件名
% 注意这里的文件路径不能用 ../../image/ ,否则用latexmk编译子文件会报错

% 例如:
% \begin{figure}[H]
% \centering
% \includegraphics[scale=0.3]{图.png}
% \caption{}
% \label{figure:图}
% \end{figure}
% 注意:上述\label{}一定要放在\caption{}之后,否则引用图片序号会只会显示??.

\begin{document}

\section{复变函数的极限和连续性}

\begin{definition}
设 $E$ 是复平面上一点集,如果对每一个 $z \in E$,按照某一规则有一确定的复数 $w$ 与之对应,我们就说在 $E$ 上确定了一个\textbf{单值复变函数},记为 $w = f(z)$ 或 $f: E \to \mathbb{C}$。$E$ 称为 $f$ 的\textbf{定义域},点集 $\{ f(z): z \in E\}$ 称为 $f$ 的\textbf{值域}.$w=f(z)$的值域所在的平面称为$\boldsymbol{w}$\textbf{平面}.

如果对于 $z \in E$,对应的 $w$ 有几个或无穷多个,则称在 $E$ 上确定了一个\textbf{多值函数}.

复变函数是定义在复平面上的,它的值域也在复平面上,因此复变函数也称为\textbf{映射}或\textbf{变换},它把一个复平面上的平面点集映成另一个复平面上的平面点集。与 $z \in E$ 对应的点 $w = f(z)$ 称为 $z$ 在映射 $f$ 下的\textbf{像点},$z$ 就称为 $w$ 的\textbf{原像}。点集 $\{ f(z): z \in E\}$ 也称为 $E$ 在映射 $f$ 下的\textbf{像},记为 $f(E)$。如果 $f(E) \subseteq F$,就说 $f$ 把 $E$ \textbf{映入} $F$,或者说 $f$ 是 $E$ 到 $F$ 中的映射。如果 $f(E) = F$,就说 $f$ 把 $E$ \textbf{映为} $F$,或者说 $f$ 是 $E$ 到 $F$ 上的\textbf{满变换}。
\end{definition}
\begin{remark}
我们知道,任意一个复数$z(z \neq 0)$都有无穷多个辐角.因此,辐角函数$w = \mathrm{Arg}\, z$是一个多值函数.它的定义域是$\mathbb{C} \setminus \{0\}$(在$z=0$处辐角无意义).
\end{remark}

\begin{definition}
若$w=f(z)$是复平面点集$E$到$F$的满变换,且对$F$中的每一点$w$,在$E$中有一个(或至少有两个)点与之相对应,则在$F$上确定了一个单值(或多值)函数,记作$z=f^{-1}(w)$,它就称为函数$w=f(z)$的\textbf{反函数}或称为变换$w=f(z)$的\textbf{逆变换};若$z=f^{-1}(w)$也是$F$到$E$的单值变换,则称$w=f(z)$是$E$到$F$的\textbf{双方单值变换}或\textbf{一一变换}.
\end{definition}
\begin{remark}
从上述反函数的定义可以看出
\begin{align*}
w=f[f^{-1}(w)],\quad \forall w\in F.
\end{align*}
且当反函数也是单值函数时,还有
\begin{align*}
z=f^{-1}[f(z)],\quad \forall z\in E.
\end{align*}
\end{remark}

\begin{definition}\label{definition:单叶性区域}
设 \( f:D \to \mathbb{C} \) 是一个复变函数,如果对区域 \( D \) 中任意两点 \( z_1, z_2 \)(\( z_1 \neq z_2 \)),必有 \( f(z_1) \neq f(z_2) \),就称 \( f \) 在 \( D \) 中是\textbf{单叶的},\( D \) 称为 \( f \) 的\textbf{单叶性区域}.
\end{definition}

\begin{proposition}
如果 \( f \) 在 \( D \) 中是单叶的,\( f(D) = G \),那么 \( f \) 是 \( D \) 到 \( G \) 之上的一一映射.
\end{proposition}
\begin{proof}
由\hyperref[definition:单叶性区域]{单叶的定义}和双射的定义立得.

\end{proof}

\begin{theorem}
设 $z = x + \mathrm{i}y$,用 $u$ 和 $v$ 记 $w = f(z)$ 的实部和虚部,则有
\begin{align*}
w = f(z) = u(z) + \mathrm{i}v(z)= u(x, y) + \mathrm{i}v(x, y).
\end{align*}
\end{theorem}
\begin{remark}
这个定理表明:一个复变函数等价于两个二元的实变函数 $u = u(x, y)$ 和 $v = v(x, y)$。
\end{remark}
\begin{note}
例如 $w = z^2 = (x + \mathrm{i}y)^2 = x^2 - y^2 + 2\mathrm{i}xy$,它等价于 $u = x^2 - y^2$ 和 $v = 2xy$ 两个二元函数;

再如 $w = |z|$,它等价于 $u = \sqrt{x^2 + y^2}$ 和 $v = 0$ 这两个二元函数。
\end{note}

\begin{definition}
设 $f$ 是定义在点集 $E\subseteq \mathbb{C}$ 上的一个复变函数,$z_0$ 是 $E$ 的一个极限点,$a$ 是给定的一个复数。如果对任意的 $\varepsilon > 0$,存在与 $\varepsilon$ 有关的 $\delta > 0$,使得当 $z \in E$ 且 $0 < |z - z_0| < \delta$ 时有 $|f(z) - a| < \varepsilon$,就说当 $z \to z_0$ 时 $f(z)$ 有极限 $a$,记作 $\lim\limits_{z \to z_0} f(z) = a$。

上述极限的定义也可用邻域的语言叙述为:对于任给的 $\varepsilon > 0$,存在与 $\varepsilon$ 有关的正数 $\delta$,使得当 $z \in B(z_0, \delta) \cap E$ 且 $z \neq z_0$ 时有 $f(z) \in B(a, \varepsilon)$.

特别地,$\lim\limits_{z\rightarrow \infty}f\left( z \right) =a$定义为:对$\forall \varepsilon >0$,存在$R\left( \varepsilon \right) >0$,使得当$\left| z \right|>R\left( \varepsilon \right)$且$z\in E$时,有$f\left( z \right) \in B\left( a,\varepsilon \right).$

$\lim\limits_{z\rightarrow z_0}{\lim}f\left( z \right) =\infty$定义为:对$\forall \varepsilon >0$,存在$\delta \left( \varepsilon \right) >0$,使得当$z\in B\left( z_0,\delta \left( \varepsilon \right) \right) \cap E$时,有$\left| f\left( z \right) \right|>\varepsilon.$
\end{definition}

\begin{theorem}\label{theorem:复变函数极限存在充要条件1.7}
设 $f(z) = u(x, y) + \mathrm{i}v(x, y)$ 是定义在点集 $E\subseteq \mathbb{C}$ 上的一个复变函数,$z_0$ 是 $E$ 的一个极限点,$a$ 是给定的一个复数。$\lim\limits_{z \to z_0} f(z) = a$ 的充分必要条件为
\[
\lim_{\substack{x \to x_0 \\ y \to y_0}} u(x, y) = \alpha, \quad \lim_{\substack{x \to x_0 \\ y \to y_0}} v(x, y) = \beta.
\]
\end{theorem}
\begin{note}
由此可知,实变函数中有关极限的一些运算法则在复变函数中也成立.
\end{note}
\begin{proof}
设 $a = \alpha + \mathrm{i}\beta$,$z_0 = x_0 + \mathrm{i}y_0$,$f(z) = u(x, y) + \mathrm{i}v(x, y)$,由下面的不等式
\begin{align*}
\vert u(x, y) - \alpha \vert &\leqslant \vert f(z) - a \vert \leqslant \vert u(x, y) - \alpha \vert + \vert v(x, y) - \beta \vert, \\
\vert v(x, y) - \beta \vert &\leqslant \vert f(z) - a \vert \leqslant \vert u(x, y) - \alpha \vert + \vert v(x, y) - \beta \vert 
\end{align*}
知道,$\lim\limits_{z \to z_0} f(z) = a$ 的充分必要条件为
\[
\lim_{\substack{x \to x_0 \\ y \to y_0}} u(x, y) = \alpha, \quad \lim_{\substack{x \to x_0 \\ y \to y_0}} v(x, y) = \beta.
\]

\end{proof}

\begin{proposition}\label{proposition:复变函数--例题1.31}
设$\lim\limits_{z\to z_0}f(z)=\eta$,试证函数$f(z)$在点$z_0$的某一去心邻域内是有界的.
\end{proposition}
\begin{proof}
因
\begin{align*}
\lim\limits_{z\to z_0}f(z)=\eta,
\end{align*}
则对任给的$\varepsilon>0$,有$\delta>0$,只要$0<|z-z_0|<\delta$,就有
\begin{align*}
|f(z)-\eta|<\varepsilon,
\end{align*}
由此可得
\begin{align*}
|f(z)|-|\eta|<\varepsilon,
\end{align*}
于是
\begin{align*}
|f(z)|<|\eta|+\varepsilon,
\end{align*}
所以,在点$z_0$的去心邻域$N_\delta(z_0)\setminus\{z_0\}$内$f(z)$是有界的.

\end{proof}

\begin{definition}
我们说 $f$ 在点 $z_0 \in E\subseteq \mathbb{C}$ 连续,如果
\[
\lim_{z \to z_0} f(z) = f(z_0).
\]
如果 $f$ 在集 $E$ 中每点都连续,就说 $f$ 在集 $E$ 上\textbf{连续}.
\end{definition}

\begin{theorem}
复变函数$f(z) = u(x, y) + \mathrm{i}v(x, y)$ 在 $z_0 = x_0 + \mathrm{i}y_0$ 处连续的充要条件是 $u(x, y)$ 和 $v(x, y)$ 作为二元函数在 $(x_0, y_0)$ 处连续.
\end{theorem}
\begin{proof}
由\refthe{theorem:复变函数极限存在充要条件1.7}易得.

\end{proof}

\begin{proposition}\label{proposition:复变函数---命题1.32}
设函数$f(z)$在点$z_0$连续,且$f(z_0)\neq0$,试证$f(z)$在点$z_0$的某一邻域内恒不为零.
\end{proposition}
\begin{proof}
因$f(z)$在点$z_0$连续,则对任给的$\varepsilon>0$,有$\delta>0$,只要$|z-z_0|<\delta$,就有
\begin{align*}
|f(z)-f(z_0)|<\varepsilon.
\end{align*}
特别,取$\varepsilon=\frac{|f(z_0)|}{2}>0$,则由上面的不等式得
\begin{align*}
|f(z)|>|f(z_0)|-\varepsilon=|f(z_0)|-\frac{|f(z_0)|}{2}=\frac{|f(z_0)|}{2}>0,
\end{align*}
因此,$f(z)$在点$z_0$的$\delta$邻域$N_\delta(z_0)$内就恒不为零.

\end{proof}

\begin{definition}
$f$ 在 $E\subseteq \mathbb{C}$ 上\textbf{一致连续},是指对任意 $\varepsilon > 0$,存在只与 $\varepsilon$ 有关的 $\delta > 0$,对 $E$ 上任意的 $z_1, z_2$,只要 $|z_1 - z_2| < \delta$,就有 $|f(z_1) - f(z_2)| < \varepsilon$.
\end{definition}

\begin{theorem}\label{theorem:连续的复变函数在紧集上的性质}
设 $E$ 是 $\mathbb{C}$ 中的紧集,$f: E \to \mathbb{C}$ 在 $E$ 上连续,那么
\begin{enumerate}[(1)]
\item $f$ 在 $E$ 上有界;

\item $|f|$ 在 $E$ 上能取得最大值和最小值,即存在 $a, b \in E$,使得对每个 $z \in E$,都有
\[
|f(z)| \leqslant |f(a)|, \quad |f(z)| \geqslant |f(b)|;
\]

\item $f$ 在 $E$ 上一致连续.
\end{enumerate}
\end{theorem}
\begin{proof}
\begin{enumerate}[(1)]
\item 由\hyperref[theorem:Heine-Borel定理]{Heine-Borel定理}, 只需证明$f(E)$是紧集. 为此, 取$f(E)$的一个开覆盖$\mathcal{F}$. 任取开集$U\in\mathcal{F}$, 由$f$的连续性, $f^{-1}(U)$是开集, 因此$\{f^{-1}(U)|U\in\mathcal{F}\}$是$E$的开覆盖. 由$E$的紧性, 该开覆盖存在有限子覆盖$\{f^{-1}(U_k)|1\leqslant k\leqslant m\}$. 由此得$f(E)$的有限子覆盖$\{U_k|1\leqslant k\leqslant m\}$.

\item 记 $M = \sup\{|f(z)|: z \in E\}$,于是对每一自然数 $n$,必有 $z_n \in E$,使得
\begin{align}\label{equation------12----1}
M - \frac{1}{n} \leqslant |f(z_n)| \leqslant M.
\end{align}
因为 $E$ 是 $\mathbb{C}$ 中的紧集,由\hyperref[theorem:Heine-Borel定理]{Heine-Borel定理},$E$ 为有界闭集. 再由\hyperref[theorem:Bolzano-Weierstrass定理]{Bolzano-Weierstrass定理},$\{z_n\}$ 必有极限点,即有一收敛子列 $\{z_{n_k}\}$,设其极限为 $a$,则 $a \in E$. 把\eqref{equation------12----1}式写成
\[
M - \frac{1}{n_k} \leqslant |f(z_{n_k})| \leqslant M,
\]
让 $k \to \infty$,并注意到 $f$ 在 $a$ 处的连续性,即得 $|f(a)| = M$. 
同理可证,存在$b\in E$,使得$|f(b)|=\inf\{|f(z)|:z\in E\}.$

\item 任取$\varepsilon>0$. 对任意$z\in E$, 由$f$的连续性, $f^{-1}(B(f(z),\frac{\varepsilon}{2}))$为包含$z$的开集. 因此有$\delta_z>0$, 使
\begin{align}\label{eq::-o2jgegmi23jsmvismq}
B(z,\delta_z)\subseteq f^{-1}(B(f(z),\frac{\varepsilon}{2})).
\end{align}
显然集族$\{B(z,\frac{\delta_z}{2})\mid z\in E\}$是$E$的开覆盖. 由$E$的紧性, 存在有限子覆盖$\{B(z_k,\frac{\delta_k}{2})\mid 1\leqslant k\leqslant n\}\supseteq E$.

取$\delta=\frac{\min \{\delta _k\mid 1\leqslant k\leqslant n\}}{2}$. 对任意$p,q\in E$且满足$|p-q|<\delta$,不妨设$p\in B(z_1,\frac{\delta_1}{2})$. 由\refpro{proposition:复数基本不等式}, 有
\begin{align*}
|z_1-q|\leqslant |z_1-p|+|p-q|<\frac{\delta_1}{2}+\delta\leqslant \delta_1.
\end{align*}
因此$p,q\in B(z_1,\delta_1)$, 再由\eqref{eq::-o2jgegmi23jsmvismq}式知$f(p),f(q)\in B(f(z_1),\frac{\varepsilon}{2})$, 从而
\begin{align*}
|f(p)-f(q)|\leqslant |f(p)-f(z_1)|+|f(z_1)-f(q)|<\frac{\varepsilon}{2}+\frac{\varepsilon}{2}=\varepsilon.
\end{align*}
由此得$f$在$E$上一致连续.
\end{enumerate}

\end{proof}














\end{document}
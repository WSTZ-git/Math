\documentclass[../../main.tex]{subfiles}
\graphicspath{{\subfix{./image/}}} % 指定图片目录,后续可以直接使用图片文件名
% 注意这里的文件路径不能用 ../../image/ ,否则用latexmk编译子文件会报错

% 例如:
% \begin{figure}[H]
% \centering
% \includegraphics[scale=0.4]{图.png}
% \caption{}
% \label{figure:图}
% \end{figure}
% 注意:上述\label{}一定要放在\caption{}之后,否则引用图片序号会只会显示??.

\begin{document}

\section{复变函数的极限和连续性}

\begin{definition}
设 $E$ 是复平面上一点集,如果对每一个 $z \in E$,按照某一规则有一确定的复数 $w$ 与之对应,我们就说在 $E$ 上确定了一个\textbf{单值复变函数},记为 $w = f(z)$ 或 $f: E \to \mathbb{C}$。$E$ 称为 $f$ 的定义域,点集 $\{ f(z): z \in E\}$ 称为 $f$ 的值域。

如果对于 $z \in E$,对应的 $w$ 有几个或无穷多个,则称在 $E$ 上确定了一个\textbf{多值函数}。
\end{definition}
\begin{note}
例如,$w = |z|^2$,$w = z^3 + 1$ 都是确定在整个平面上的单值函数;而 $w = \sqrt[n]{z}$,$w = \text{Arg}z$ 则是多值函数。今后若非特别说明,我们所讲的函数都是指单值函数。
\end{note}
\begin{remark}
复变函数是定义在平面点集上的,它的值域也是一个平面点集,因此复变函数也称为\textbf{映射},它把一个平面点集映成另一个平面点集。与 $z \in E$ 对应的点 $w = f(z)$ 称为 $z$ 在映射 $f$ 下的像点,$z$ 就称为 $w$ 的原像。点集 $\{ f(z): z \in E\}$ 也称为 $E$ 在映射 $f$ 下的像,记为 $f(E)$。如果 $f(E) \subset F$,就说 $f$ 把 $E$ 映入 $F$,或者说 $f$ 是 $E$ 到 $F$ 中的映射。如果 $f(E) = F$,就说 $f$ 把 $E$ 映为 $F$,或者说 $f$ 是 $E$ 到 $F$ 上的映射。
\end{remark}

\begin{theorem}
设 $z = x + \mathrm{i}y$,用 $u$ 和 $v$ 记 $w = f(z)$ 的实部和虚部,则有
\begin{align*}
w = f(z) = u(z) + \mathrm{i}v(z)= u(x, y) + \mathrm{i}v(x, y).
\end{align*}
这就是说,一个复变函数等价于两个二元的实变函数 $u = u(x, y)$ 和 $v = v(x, y)$。
\end{theorem}
\begin{note}
例如 $w = z^2 = (x + \mathrm{i}y)^2 = x^2 - y^2 + 2\mathrm{i}xy$,它等价于 $u = x^2 - y^2$ 和 $v = 2xy$ 两个二元函数;

再如 $w = |z|$,它等价于 $u = \sqrt{x^2 + y^2}$ 和 $v = 0$ 这两个二元函数。
\end{note}

\begin{definition}
设 $f$ 是定义在点集 $E$ 上的一个复变函数,$z_0$ 是 $E$ 的一个极限点,$a$ 是给定的一个复数。如果对任意的 $\varepsilon > 0$,存在与 $\varepsilon$ 有关的 $\delta > 0$,使得当 $z \in E$ 且 $0 < |z - z_0| < \delta$ 时有 $|f(z) - a| < \varepsilon$,就说当 $z \to z_0$ 时 $f(z)$ 有极限 $a$,记作 $\lim\limits_{z \to z_0} f(z) = a$。

上述极限的定义也可用邻域的语言叙述为:对于任给的 $\varepsilon > 0$,存在与 $\varepsilon$ 有关的正数 $\delta$,使得当 $z \in B(z_0, \delta) \cap E$ 且 $z \neq z_0$ 时有 $f(z) \in B(a, \varepsilon)$,这后一种说法也适用于 $z = \infty$ 的情形。
\end{definition}

\begin{theorem}\label{theorem:复变函数极限存在充要条件1.7}
设 $f$ 是定义在点集 $E$ 上的一个复变函数,$z_0$ 是 $E$ 的一个极限点,$a$ 是给定的一个复数。$\lim\limits_{z \to z_0} f(z) = a$ 的充分必要条件为
\[
\lim_{\substack{x \to x_0 \\ y \to y_0}} u(x, y) = \alpha, \quad \lim_{\substack{x \to x_0 \\ y \to y_0}} v(x, y) = \beta.
\]
\end{theorem}
\begin{note}
由此可知,实变函数中有关极限的一些运算法则在复变函数中也成立.
\end{note}
\begin{proof}
设 $a = \alpha + \mathrm{i}\beta$,$z_0 = x_0 + \mathrm{i}y_0$,$f(z) = u(x, y) + \mathrm{i}v(x, y)$,由下面的不等式
\begin{align*}
\vert u(x, y) - \alpha \vert &\leqslant \vert f(z) - a \vert \leqslant \vert u(x, y) - \alpha \vert + \vert v(x, y) - \beta \vert, \\
\vert v(x, y) - \beta \vert &\leqslant \vert f(z) - a \vert \leqslant \vert u(x, y) - \alpha \vert + \vert v(x, y) - \beta \vert 
\end{align*}
知道,$\lim\limits_{z \to z_0} f(z) = a$ 的充分必要条件为
\[
\lim_{\substack{x \to x_0 \\ y \to y_0}} u(x, y) = \alpha, \quad \lim_{\substack{x \to x_0 \\ y \to y_0}} v(x, y) = \beta.
\]

\end{proof}

\begin{definition}
我们说 $f$ 在点 $z_0 \in E$ 连续,如果
\[
\lim_{z \to z_0} f(z) = f(z_0).
\]
如果 $f$ 在集 $E$ 中每点都连续,就说 $f$ 在集 $E$ 上连续.
\end{definition}

\begin{theorem}
复变函数$f(z) = u(x, y) + \mathrm{i}v(x, y)$ 在 $z_0 = x_0 + \mathrm{i}y_0$ 处连续的充要条件是 $u(x, y)$ 和 $v(x, y)$ 作为二元函数在 $(x_0, y_0)$ 处连续.
\end{theorem}
\begin{proof}
由\refthe{theorem:复变函数极限存在充要条件1.7}易得.

\end{proof}

\begin{definition}
$f$ 在 $E$ 上\textbf{一致连续},是指对任意 $\varepsilon > 0$,存在只与 $\varepsilon$ 有关的 $\delta > 0$,对 $E$ 上任意的 $z_1, z_2$,只要 $|z_1 - z_2| < \delta$,就有 $|f(z_1) - f(z_2)| < \varepsilon$.
\end{definition}

\begin{theorem}\label{theorem:连续的复变函数在紧集上的性质}
设 $E$ 是 $\mathbb{C}$ 中的紧集,$f: E \to \mathbb{C}$ 在 $E$ 上连续,那么

(i) $f$ 在 $E$ 上有界;

(ii) $|f|$ 在 $E$ 上能取得最大值和最小值,即存在 $a, b \in E$,使得对每个 $z \in E$,都有
\[
|f(z)| \leqslant |f(a)|, \quad |f(z)| \geqslant |f(b)|;
\]

(iii) $f$ 在 $E$ 上一致连续.
\end{theorem}
\begin{proof}
(i)

(ii)记 $M = \sup\{|f(z)|: z \in E\}$,于是对每一自然数 $n$,必有 $z_n \in E$,使得
\begin{align}\label{equation------12----1}
M - \frac{1}{n} \leqslant |f(z_n)| \leqslant M.
\end{align}
因为 $E$ 是 $\mathbb{C}$ 中的紧集,由\hyperref[theorem:Heine-Borel定理]{Heine-Borel定理},$E$ 为有界闭集. 再由\hyperref[theorem:Bolzano-Weierstrass定理]{Bolzano-Weierstrass定理},$\{z_n\}$ 必有极限点,即有一收敛子列 $\{z_{n_k}\}$,设其极限为 $a$,则 $a \in E$. 把\eqref{equation------12----1}式写成
\[
M - \frac{1}{n_k} \leqslant |f(z_{n_k})| \leqslant M,
\]
让 $k \to \infty$,并注意到 $f$ 在 $a$ 处的连续性,即得 $|f(a)| = M$. 

同理可证,存在$b\in E$,使得$|f(b)|=\inf\{|f(z)|:z\in E\}.$

(iii)

\end{proof}














\end{document}
\documentclass[../../main.tex]{subfiles}% 注意这里的文件路径不能用 ./main.tex ,否则用latexmk编译子文件会报错
\graphicspath{{\subfix{./image/}}} % 指定图片目录,后续可以直接使用图片文件名
% 注意这里的文件路径不能用 ../../image/ ,否则用latexmk编译子文件会报错

% 例如:
% \begin{figure}[H]
% \centering
% \includegraphics[scale=0.3]{图.png}
% \caption{}
% \label{figure:图}
% \end{figure}
% 注意:上述\label{}一定要放在\caption{}之后,否则引用图片序号会只会显示??.

\begin{document}

\section{复数的定义及其运算}

\begin{definition}[复数域]
我们把复数定义为一对有序的实数\((a,b)\),如果用\(\mathbb{R}\)记实数的全体,\(\mathbb{C}\)记复数的全体,那么
\[
\mathbb{C} = \{ (a,b) : a \in \mathbb{R}, b \in \mathbb{R} \}.
\]
在这个集合中定义加法和乘法两种运算:
\[
(a,b) + (c,d) = (a + c, b + d),
\]
\[
(a,b)(c,d) = (ac - bd, ad + bc).
\]
容易验证,加法和乘法都满足交换律和结合律;\((0,0)\)是零元素,\((-a,-b)\)是\((a,b)\)的负元素;\((1,0)\)是乘法的单位元素;每个非零元素\((a,b)\)有逆元素\(\left( \dfrac{a}{a^2 + b^2}, -\dfrac{b}{a^2 + b^2} \right)\);此外,\(\mathbb{C}\)中的加法和乘法还满足分配律:
\[
[(a,b) + (c,d)](e,f) = (a,b)(e,f) + (c,d)(e,f).
\]
因此\(\mathbb{C}\)在上面定义的加法和乘法运算下构成一个域,称为\textbf{复数域}.如果记
\[
\tilde{\mathbb{R}} = \{ (a,0) : a \in \mathbb{R} \},
\]
那么\(\tilde{\mathbb{R}}\)是\(\mathbb{C}\)的一个子域。显然,\((a,0) \to a\)是\(\tilde{\mathbb{R}}\)与\(\mathbb{R}\)之间的一个同构对应,因此实数域\(\mathbb{R}\)是\(\mathbb{C}\)的一个子域。我们直接记\((a,0) = a\)。
在\(\mathbb{C}\)中,\((0,1)\)这个元素有其特殊性,它满足
\[
(0,1)^2 = (0,1)(0,1) = (-1,0) = -1.
\]
专门用\(\mathrm{i}\)记\((0,1)\)这个元素,于是有\(\mathrm{i}^2 = -1\)。由于\((0,b) = (b,0) \cdot (0,1) = b\mathrm{i}\),于是每一个复数\((a,b)\)都可写成
\[
(a,b) = (a,0) + (0,b) = a + b\mathrm{i}.
\]
从现在开始,我们不再用实数对\((a,b)\)来记复数,而直接用\(z = a + b\mathrm{i}\)记复数,\(a\)称为\(z\)的\textbf{实部},\(b\)称为\(z\)的\textbf{虚部},分别记为\(a = \mathrm{Re}z\),\(b = \mathrm{Im}z\)。加法和乘法用现在的记号定义为:
\[
(a + b\mathrm{i}) + (c + d\mathrm{i}) = (a + c) + (b + d)\mathrm{i},
\]
\[
(a + b\mathrm{i})(c + d\mathrm{i}) = (ac - bd) + (ad + bc)\mathrm{i}.
\]
减法和除法分别定义为加法和乘法的逆运算:
\[
(a + b\mathrm{i}) - (c + d\mathrm{i}) = (a - c) + (b - d)\mathrm{i},
\]
\[
\begin{split}
\frac{a + b\mathrm{i}}{c + d\mathrm{i}} = (a + b\mathrm{i})\left( \frac{c - d\mathrm{i}}{c^2 + d^2} \right)= \frac{ac + bd}{c^2 + d^2} + \frac{bc - ad}{c^2 + d^2}\mathrm{i}.
\end{split}
\]

设\(z = a + b\mathrm{i}\)是一复数,定义
\[
|z| = \sqrt{a^2 + b^2},
\]
\[
\bar{z} = a - b\mathrm{i},
\]
\(|z|\)称为\(z\)的\textbf{模}或\textbf{绝对值},\(\bar{z}\)称为\(z\)的\textbf{共轭复数}。
\end{definition}

\begin{definition}[有序域]\label{definition:有序域}
域\(F\)称为\textbf{有序域},如果在\(F\)的元素间能确定一种关系(记为\(a < b\)),其满足下列要求:
\begin{enumerate}[(i)]
\item\label{有序域:(i)} 对\(F\)中任意两个元素\(a,b\),下述三个关系中必有而且只有一个成立:
\[
a < b, \quad a = b, \quad b < a;
\]

\item\label{有序域:(ii)} 如果\(a < b, b < c\),那么\(a < c\);

\item\label{有序域:(iii)} 如果\(a < b\),那么对任意\(c\),有\(a + c < b + c\);

\item\label{有序域:(iv)} 如果\(a < b, c > 0\),那么\(ac < bc\)。
\end{enumerate}
\end{definition}
\begin{note}
容易知道,实数域是有序域,而复数域则不是。
\end{note}

\begin{theorem}\label{theorem:复数域不是有序域}
复数域不是有序域.
\end{theorem}
\begin{proof}
如果\(\mathbb{C}\)是有序域,那么因为\(\mathrm{i} \neq 0\),\(\mathrm{i}\)和\(0\)之间必有\(\mathrm{i} > 0\)或\(\mathrm{i} < 0\)的关系.如果\(\mathrm{i} > 0\),则由\hyperref[definition:有序域]{有序域(iv)}得\(\mathrm{i} \cdot \mathrm{i} > \mathrm{i} \cdot 0\),即\(-1 > 0\),再由\hyperref[有序域:(iii)]{(iii)},两端都加\(1\),即得\(0 > 1\).另一方面,从\(-1 > 0\)还可得\((-1) \cdot (-1) > 0 \cdot (-1)\),即\(1 > 0\),这和刚才得到的\(0 > 1\)矛盾.如果\(\mathrm{i} < 0\),两端都加\(-\mathrm{i}\),得\(0 < -\mathrm{i}\),再由\hyperref[definition:有序域]{有序域(iv)},两端乘\(-\mathrm{i}\),得\(-1 > 0\).重复上面的讨论,即可得\(0 > 1\)和\(0 < 1\)的矛盾.所以,复数域不是有序域. 

\end{proof}

\begin{proposition}[复数运算性质]\label{proposition:复数运算性质}
设\(z\)和\(w\)是两个复数,那么
\begin{enumerate}[(1)]
\item\label{proposition:复数运算性质-1} \(\mathrm{Re}z = \frac{1}{2}(z + \overline{z})\),\(\quad \mathrm{Im}z = \frac{1}{2\mathrm{i}}(z - \overline{z})\),$\quad \mathrm{Re}z=\mathrm{Re}\overline{z}$,$\quad \mathrm{Im}z=-\mathrm{Im}\overline{z}$.

\item\label{proposition:复数运算性质-2} \(z\overline{z} = |z|^2\),\(\quad \overline{z} = \frac{|z|^2}{z}\).

\item\label{proposition:复数运算性质-3} \(\overline{z + w} = \overline{z} + \overline{w}\),\(\quad \overline{zw} = \overline{z}\,\overline{w}\).

\item\label{proposition:复数运算性质-4} \(|z| = |\overline{z}|\),$\quad |z|=|-z|$,$\quad |zw|=|z||w|$,$\quad \left| \frac{z}{w} \right|=\frac{\left| z \right|}{\left| w \right|}$.

\item\label{proposition:复数运算性质-5} $(z+w)^2=|z|^2+|w|^2+2\text{Re}(z\overline{w})$.

\item\label{proposition:复数运算性质-6} 若$|z|=\lambda|w|$,$\lambda>0$,则$|z-\lambda^2 w|=\lambda|z-w|.$
\end{enumerate}
\end{proposition}
\begin{proof}
\begin{enumerate}[(1)]
\item 

\item 

\item

\item

\item 由\rrefpro{proposition:复数运算性质}{proposition:复数运算性质-1}和\rrefpro{proposition:复数运算性质}{proposition:复数运算性质-2}可得$|z + w|^2 = (z + w)(\overline{z + w}) = |z|^2 + 2\mathrm{Re}(z\overline{w}) + |w|^2.$

\item 利用\rrefpro{proposition:复数运算性质}{proposition:复数运算性质-5}和\rrefpro{proposition:复数运算性质}{proposition:复数运算性质-4}可得
\begin{align*}
\left| z-\lambda ^2w \right|^2&=\left| z \right|^2+\lambda ^4\left| w \right|^2-2\lambda ^2\mathrm{Re}\left( z\overline{w} \right) 
\\
&=\lambda ^2\left| w \right|^2+\lambda ^4\left| w \right|^2-2\lambda ^2\mathrm{Re}\left( z\overline{w} \right) 
\\
&=\lambda ^2\left( \lambda ^2\left| w \right|^2+\left| w \right|^2-2\mathrm{Re}\left( z\overline{w} \right) \right) 
\\
&=\lambda ^2\left| \lambda w-w \right|^2=\lambda ^2\left| z-w \right|^2.
\end{align*}
\end{enumerate}

\end{proof}

\begin{proposition}[基本不等式]\label{proposition:复数基本不等式}
设\(z\)和\(w\)是两个复数,那么
\begin{enumerate}[(1)]
\item\label{proposition:复数基本不等式-1} \(|\mathrm{Re}z| \leqslant |z|\),等号成立当且仅当$z\in \mathbb{R}$.

\(|\mathrm{Im}z| \leqslant |z|\),等号成立当且仅当$z$是纯虚数.

\item $\frac{1}{\sqrt{2}}\left( \left| \mathrm{Re}z \right|+\left| \mathrm{Im}z \right| \right) \leqslant \left| z \right|\leqslant \left| \mathrm{Re}z \right|+\left| \mathrm{Im}z \right|.$

\item\label{proposition:复数基本不等式-2} \(|z + w| \leqslant |z| + |w|\),等号成立当且仅当存在某个实数\(t \geqslant 0\),使得\(z = tw\).

一般的,设\(z_1, \cdots, z_n\)是任意\(n\)个复数,则
\[
|z_1 + \cdots + z_n| \leqslant |z_1| + \cdots + |z_n|.
\]
等号成立当且仅当$z_i,\,i=1,2,\cdots,n$共线且同向.

\item\label{proposition:复数基本不等式-3} \(|z - w| \geqslant \big||z| - |w|\big|\),等号成立当且仅当存在某个实数\(t \geqslant 0\),使得\(z - w = tw\),即\(z = (t + 1)w\).
\end{enumerate}
\end{proposition}
\begin{proof}
\begin{enumerate}[(1)]
\item 从\(\mathrm{Re}z\),\(\mathrm{Im}z\)和\(|z|\)的定义马上知道不等式成立。

\item 设$z=a+b\mathrm{i},$则由均值不等式可得
\begin{align*}
\frac{1}{\sqrt{2}}\left( \left| \mathrm{Re}z \right|+\left| \mathrm{Im}z \right| \right) =\frac{a+b}{\sqrt{2}}\leqslant \left| z \right|=\sqrt{a^2+b^2}\leqslant a+b=\left| \mathrm{Re}z \right|+\left| \mathrm{Im}z \right|.
\end{align*}

\item 利用\rrefpro{proposition:复数运算性质}{proposition:复数运算性质-5}和\rrefpro{proposition:复数基本不等式}{proposition:复数基本不等式-1},即得
\begin{align*}
|z + w|^2  = |z|^2 + 2\mathrm{Re}(z\overline{w}) + |w|^2\leqslant |z|^2 + 2|z||w| + |w|^2 = (|z| + |w|)^2,
\end{align*}
由此即知结论成立。由上面的不等式可以看出,等式成立的充要条件是\(\mathrm{Re}(z\overline{w}) = |z\overline{w}|\),这等价于$z\overline{w}\in \mathbb{R}$且\(z\overline{w} \geqslant 0\)。不妨设\(w \neq 0\)(\(w = 0\)时,等号显然成立),则当$z\overline{w}\in (0,+\infty)$时,由于\(\overline{w} = \dfrac{|w|^2}{w}\),故\(z\overline{w}=\dfrac{z}{w}|w|^2 \in(0,+\infty)\).令\(t = \left( \dfrac{z}{w}|w|^2 \right) \dfrac{1}{|w|^2}\),则$t\in \mathbb{R}$且\(t \geqslant 0\),而且\(z = tw\)。反之,若存在$t\in (0,+\infty)$,使得$z=tw.$则$\frac{z}{w}=t\in (0,+\infty)$,从而\(z\overline{w}=\dfrac{z}{w}|w|^2 \in(0,+\infty)\).

一般的情形,利用数学归纳法易证.

\item 当\(| z | = | w |\)时,结论显然成立.

当\(| z | > | w |\)时,由\rrefpro{proposition:复数基本不等式}{proposition:复数基本不等式-2}可得
\[| z | = | (z - w) + w | \leqslant | z - w | + | w |,\]
移项可得\(| z - w | \geqslant | z | - | w | = | | z | - | w | |\).由\rrefpro{proposition:复数基本不等式}{proposition:复数基本不等式-2}可知,此时等号成立当且仅当存在某个实数\(t \geqslant 0\),使得\(z - w = tw\),即\(z = (t + 1)w\).

当\(| z | < | w |\)时,由\rrefpro{proposition:复数基本不等式}{proposition:复数基本不等式-2}可得
\[| w | = | (w - z) + z | \leqslant | w - z | + | z |,\]
移项可得\(| z - w | = | w - z | \geqslant | w | - | z | = | | z | - | w | |\).由\rrefpro{proposition:复数基本不等式}{proposition:复数基本不等式-2}可知,此时等号成立当且仅当存在某个实数\(t \geqslant 0\),使得\(w - z = tz\),即\(w = (t + 1)z\).
\end{enumerate}

\end{proof}

\begin{proposition}
设$z,a\in \mathbb{C},$且$|a|<1$,则
\begin{enumerate}[(1)]
\item $\left| \frac{z-a}{1-\overline{a}z} \right|=
\begin{cases}
1,&|z|=1,\\
<1,&|z|<1,\\
>1,&|z|>1.
\end{cases}$

\item $1-\left| \frac{z-a}{1-\overline{a}z} \right|^2=\frac{\left( 1-\left| a \right|^2 \right) \left( 1-\left| z \right|^2 \right)}{\left| 1-\overline{a}z \right|^2}.$

\item 若还有$|z|<1$,则
\begin{align*}
\frac{\left| \left| z \right|-\left| a \right| \right|}{1-\left| a \right|\left| z \right|}\leqslant \left| \frac{z-a}{1-\overline{a}z} \right|\leqslant \frac{\left| z \right|+\left| a \right|}{1+\left| a \right|\left| z \right|}.
\end{align*}
\end{enumerate}
\end{proposition}
\begin{proof}
\begin{enumerate}[(1)]
\item 当$|z|=1$时,由\rrefpro{proposition:复数运算性质}{proposition:复数运算性质-5}可得
\begin{align*}
\left| \frac{z-a}{1-\overline{a}z} \right|=1
\Longleftrightarrow |z-a|^2=|1-\overline{a}z|^2
\Longleftrightarrow |z|^2+|a|^2-2\mathrm{Re}\left( \overline{a}z \right) =1+|\overline{a}z|^2-2\mathrm{Re}\left( a\overline{z} \right) .
\end{align*}
由$|z|=1$和$\mathrm{Re}\left( \overline{a}z \right) =\mathrm{Re}\left( a\overline{z} \right)$知上面最后一个式子成立,故此时$\left| \frac{z-a}{1-\overline{a}z} \right|=1.$

当$|z|\ne 1$时,由\rrefpro{proposition:复数运算性质}{proposition:复数运算性质-5}可得
\begin{align*}
|z-a|^2-|1-\overline{a}z|^2
&=\left[ |z|^2+|a|^2-2\mathrm{Re}\left( \overline{a}z \right) \right] -\left[ 1+|\overline{a}z|^2-2\mathrm{Re}\left( a\overline{z} \right) \right] \\
&=|z|^2+|a|^2-1-|a|^2|z|^2
=\left( |z|^2-1 \right) \left( 1-|a|^2 \right) \\
&=
\begin{cases}
<0,&|z|<1,\\
>0,&|z|>1.
\end{cases}
\end{align*}
从而
\begin{align*}
\left| \frac{z-a}{1-\overline{a}z} \right|^2=\frac{|z-a|^2}{|1-\overline{a}z|^2}=
\begin{cases}
<1,&|z|<1,\\
>1,&|z|>1.
\end{cases}
\end{align*}
故
\begin{align*}
\left| \frac{z-a}{1-\overline{a}z} \right|=
\begin{cases}
<1,&|z|<1,\\
>1,&|z|>1.
\end{cases}
\end{align*}

\item 由\rrefpro{proposition:复数运算性质}{proposition:复数运算性质-5}可得
\begin{align*}
1-\left| \frac{z-a}{1-\overline{a}z} \right|^2
&=\frac{\left| 1-\overline{a}z \right|^2-\left| z-a \right|^2}{\left| 1-\overline{a}z \right|^2}=\frac{\left[ 1+\left| a \right|^2\left| z \right|^2-2\mathrm{Re}\left( \overline{a}z \right) \right] -\left[ \left| z \right|^2+\left| a \right|^2-2\mathrm{Re}\left( \overline{a}z \right) \right]}{\left| 1-\overline{a}z \right|^2}\\
&=\frac{1+\left| a \right|^2\left| z \right|^2-\left| z \right|^2-\left| a \right|^2}{\left| 1-\overline{a}z \right|^2}=\frac{\left( 1-\left| a \right|^2 \right) \left( 1-\left| z \right|^2 \right)}{\left| 1-\overline{a}z \right|^2}.
\end{align*}

\item 由\rrefpro{proposition:复数运算性质}{proposition:复数运算性质-4}和\rrefpro{proposition:复数运算性质}{proposition:复数运算性质-5}可得
\begin{align*}
&\frac{\left| \left| z \right|-\left| a \right| \right|}{1-\left| a \right|\left| z \right|}\leqslant \left| \frac{z-a}{1-\overline{a}z} \right|
=\frac{\left| z-a \right|}{\left| 1-\overline{a}z \right|}\Longleftrightarrow \left| \left| z \right|-\left| a \right| \right|\cdot \left| 1-\overline{a}z \right|\leqslant \left( 1-\left| a \right|\left| z \right| \right) \left| z-a \right|\\
&\Longleftrightarrow \left| \left| z \right|-\left| a \right| \right|^2\cdot \left| 1-\overline{a}z \right|^2\leqslant \left( 1-\left| a \right|\left| z \right| \right) ^2\left| z-a \right|^2\\
&\Longleftrightarrow \left( \left| z \right|^2+\left| a \right|^2-2\left| a \right|\left| z \right| \right) \left( 1+\left| a \right|^2\left| z \right|^2-2\mathrm{Re}\left( \overline{a}z \right) \right) \leqslant \left( 1+\left| a \right|^2\left| z \right|^2-2\left| a \right|\left| z \right| \right) \left( \left| z \right|^2+\left| a \right|^2-2\mathrm{Re}\left( \overline{a}z \right) \right) \\
&\Longleftrightarrow \left| a \right|\left| z \right|\left( \left| z \right|^2+\left| a \right|^2-1-\left| a \right|^2\left| z \right|^2 \right) -\mathrm{Re}\left( \overline{a}z \right) \left( \left| z \right|^2+\left| a \right|^2-1-\left| a \right|^2\left| z \right|^2 \right) \leqslant 0\\
&\Longleftrightarrow \left( \left| a \right|\left| z \right|-\mathrm{Re}\left( \overline{a}z \right) \right) \left( 1-\left| a \right|^2 \right) \left( \left| z \right|^2-1 \right) \leqslant 0\Longleftrightarrow \left| a \right|\left| z \right|\geqslant \mathrm{Re}\left( \overline{a}z \right) .
\end{align*}
\begin{align*}
&\frac{\left| z-a \right|}{\left| 1-\overline{a}z \right|}
=\left| \frac{z-a}{1-\overline{a}z} \right|\leqslant \frac{\left| z \right|+\left| a \right|}{1+\left| a \right|\left| z \right|}
\Longleftrightarrow \left| z-a \right|\left( 1+\left| a \right|\left| z \right| \right) \leqslant \left| 1-\overline{a}z \right|\left( \left| z \right|+\left| a \right| \right) \\
&\Longleftrightarrow \left| z-a \right|^2\left( 1+\left| a \right|\left| z \right| \right) ^2\leqslant \left| 1-\overline{a}z \right|^2\left( \left| z \right|+\left| a \right| \right) ^2\\
&\Longleftrightarrow \left( \left| z \right|^2+\left| a \right|^2-2\mathrm{Re}\left( \overline{a}z \right) \right) \left( 1+\left| a \right|^2\left| z \right|^2+2\left| a \right|\left| z \right| \right) \leqslant \left( 1+\left| a \right|^2\left| z \right|^2-2\mathrm{Re}\left( \overline{a}z \right) \right) \left( \left| z \right|^2+\left| a \right|^2+2\left| a \right|\left| z \right| \right) \\
&\Longleftrightarrow \left| a \right|\left| z \right|\left( \left| z \right|^2+\left| a \right|^2-1-\left| a \right|^2\left| z \right|^2 \right) +\mathrm{Re}\left( \overline{a}z \right) \left( \left| z \right|^2+\left| a \right|^2-1-\left| a \right|^2\left| z \right|^2 \right) \leqslant 0\\
&\Longleftrightarrow \left( \left| a \right|\left| z \right|+\mathrm{Re}\left( \overline{a}z \right) \right) \left( 1-\left| a \right|^2 \right) \left( \left| z \right|^2-1 \right) \leqslant 0\Longleftrightarrow \left| a \right|\left| z \right|\geqslant -\mathrm{Re}\left( \overline{a}z \right) .
\end{align*}
因此
\begin{align*}
\frac{\left| \left| z \right|-\left| a \right| \right|}{1-\left| a \right|\left| z \right|}\leqslant \left| \frac{z-a}{1-\overline{a}z} \right|\leqslant \frac{\left| z \right|+\left| a \right|}{1+\left| a \right|\left| z \right|}
\Longleftrightarrow -\mathrm{Re}\left( \overline{a}z \right) \leqslant \left| a \right|\left| z \right|\leqslant \mathrm{Re}\left( \overline{a}z \right) \Longleftrightarrow \left| \mathrm{Re}\left( \overline{a}z \right) \right|\leqslant \left| a \right|\left| z \right|.
\end{align*}
由\rrefpro{proposition:复数基本不等式}{proposition:复数基本不等式-1}知最后一个不等式成立,故结论成立.
\end{enumerate}

\end{proof}

\begin{theorem}
设$z_1,\cdots,z_n,w_1,\cdots,w_n$是任意$2n$个复数,证明复数形式的$\mathbf{Lagrange}$\textbf{恒等式}:
\begin{align*}
\left| \sum_{i=1}^n{z_iw_i} \right|^2=\left( \sum_{i=1}^n{|z_i|^2} \right) \left( \sum_{i=1}^n{|w_i|^2} \right) -\sum_{1\leqslant i<j\leqslant n}{\left| z_i\overline{w_j}-z_j\overline{w_i} \right|^2}.
\end{align*}
并由此推出$\mathbf{Cauchy}$\textbf{不等式}
\begin{align*}
\left| \sum_{i=1}^n{z_iw_i} \right|^2\leqslant \left( \sum_{i=1}^n{|z_i|^2} \right) \left( \sum_{i=1}^n{|w_i|^2} \right) .
\end{align*}
等号成立当且仅当$(z_1,z_2,\cdots,z_n)$与$(\overline{w_1},\overline{w_2},\cdots,\overline{w_n})$线性相关,也当且仅当存在$k\in \mathbb{C}$,使得$z_i=k\overline{w_i}\left( i=1,2,\cdots ,n \right)$或$\overline{w_i}=kz_i\left( i=1,2,\cdots ,n \right) $.
\end{theorem}
\begin{proof}
记
\begin{align*}
A=\begin{pmatrix}
z_1& z_2& \cdots& z_n\\
\overline{w_1}& \overline{w_2}& \cdots& \overline{w_n}
\end{pmatrix},
\end{align*}
则
\begin{align}
\left| A\overline{A}^T \right|
=\left| \begin{pmatrix}
z_1& z_2& \cdots& z_n\\
\overline{w_1}& \overline{w_2}& \cdots& \overline{w_n}
\end{pmatrix} \begin{pmatrix}
\overline{z_1}& w_1\\
\overline{z_2}& w_2\\
\vdots& \vdots\\
\overline{z_n}& w_n\\
\end{pmatrix} \right| 
=\left| \begin{pmatrix}
\sum_{i=1}^n{\left| z_i \right|^2} \sum_{i=1}^n{z_iw_i}
\sum_{i=1}^n{\overline{z_iw_i}} \sum_{i=1}^n{\left| w_i \right|^2}
\end{pmatrix} \right|.\label{eq:::--wf89h23fhsjoioszuv8jw23fpaow78ty4h3yoghvowz789eg21.1}
\end{align}
由\hyperref[Basis of Algebra-theorem:Cauchy-Binet公式]{Cauchy-Binet公式}可得\eqref{eq:::--wf89h23fhsjoioszuv8jw23fpaow78ty4h3yoghvowz789eg21.1}式左边等于
\begin{align*}
\sum_{1\leqslant i<j\leqslant n}{\left| \begin{pmatrix}
z_i& z_j\\
\overline{w_i}& \overline{w_j}\\
\end{pmatrix} \right| \left| \begin{pmatrix}
\overline{z_i}& w_i\\
\overline{z_j}& w_j\\
\end{pmatrix} \right|}
=\sum_{1\leqslant i<j\leqslant n}{\left( z_i\overline{w_j}-z_j\overline{w_i} \right) \left( \overline{z_i}w_j-\overline{z_j}w_i \right)}
=\sum_{1\leqslant i<j\leqslant n}{\left| z_i\overline{w_j}-z_j\overline{w_i} \right|^2}.
\end{align*}
而\eqref{eq:::--wf89h23fhsjoioszuv8jw23fpaow78ty4h3yoghvowz789eg21.1}式右边显然等于
\begin{align*}
\left( \sum_{i=1}^n{|z_i|^2} \right) \left( \sum_{i=1}^n{|w_i|^2} \right) -\left( \sum_{i=1}^n{z_iw_i} \right) \left( \sum_{i=1}^n{\overline{z_iw_i}} \right)
=\left( \sum_{i=1}^n{|z_i|^2} \right) \left( \sum_{i=1}^n{|w_i|^2} \right) -\left| \sum_{i=1}^n{z_iw_i} \right|^2.
\end{align*}
故
\begin{align*}
&\sum_{1\leqslant i<j\leqslant n}{\left| z_i\overline{w_j}-z_j\overline{w_i} \right|^2}=\left( \sum_{i=1}^n{|z_i|^2} \right) \left( \sum_{i=1}^n{|w_i|^2} \right) -\left| \sum_{i=1}^n{z_iw_i} \right|^2\\
&\Longleftrightarrow \left| \sum_{i=1}^n{z_iw_i} \right|^2=\left( \sum_{i=1}^n{|z_i|^2} \right) \left( \sum_{i=1}^n{|w_i|^2} \right) -\sum_{1\leqslant i<j\leqslant n}{\left| z_i\overline{w_j}-z_j\overline{w_i} \right|^2}.
\end{align*}
由此立得
\begin{align*}
\left| \sum_{i=1}^n{z_iw_i} \right|^2\leqslant \left( \sum_{i=1}^n{|z_i|^2} \right) \left( \sum_{i=1}^n{|w_i|^2} \right) .
\end{align*}
等号成立当且仅当
\begin{align*}
0=\sum_{1\leqslant i<j\leqslant n}{\left| z_i\overline{w_j}-z_j\overline{w_i} \right|^2}=\left| A\overline{A}^T \right|
\Longleftrightarrow 1\geqslant \mathrm{r}\left( A\overline{A}^T \right) =\mathrm{r}\left( A \right) .
\end{align*}
再记$\alpha =\left( z_1,z_2,\cdots ,z_n \right) ^T$,$\beta =\left( \overline{w_1},\overline{w_2},\cdots ,\overline{w_n} \right) ^T$,则由上式可知$A$不可逆,从而$A$的行向量$\alpha ^T$,$\beta ^T$线性相关,即存在$k\in \mathbb{C}$,使得
\begin{align*}
\alpha =k\beta \, \text{或} \, \beta =k\alpha
&\Longleftrightarrow z_i=k\overline{w_i}\left( i=1,2,\cdots ,n \right) \, \text{或} \, \overline{w_i}=kz_i\left( i=1,2,\cdots ,n \right) .
\end{align*}


\end{proof}

\begin{lemma}
设$z_1,\cdots,z_n$是任意$n$个复数,证明必有$\{1,2,\cdots,n\}$的子集$S$,使得
\begin{align*}
\left| \sum_{j\in S} z_j \right| \geqslant \frac{1}{\pi} \sum_{j=1}^{n} |z_j|.
\end{align*}
\end{lemma}
\begin{proof}
记$z_k = |z_k| \mathrm{e}^{\mathrm{i}\alpha_k}$。当$-\pi \leqslant \theta \leqslant \pi$时,设$S(\theta)$是所有使得$\cos(\alpha_k - \theta) > 0$的$k$组成的集,则
\begin{align*}
\left| \sum_{S(\theta)} z_k \right| &= \left| \sum_{S(\theta)} \mathrm{e}^{-\mathrm{i}\theta} z_k \right| \geqslant \mathrm{Re}\sum_{S(\theta)} \mathrm{e}^{-\mathrm{i}\theta} z_k = \sum_{k=1}^{n} |z_k| \cos^+(\alpha_k - \theta),
\end{align*}
其中
\begin{align*}
\cos ^+(\alpha _k-\theta )=\begin{cases}
\cos\mathrm{(}\alpha _k-\theta ),&\cos\mathrm{(}\alpha _k-\theta )>0,\\
0,&\cos\mathrm{(}\alpha _k-\theta )\leqslant 0.\\
\end{cases}.
\end{align*}
选取$\theta_0$,使得最后一个和式最大(连续函数在闭区间必有最值),并令$S = S(\theta_0)$。这个最大值至少是$[-\pi, \pi]$上和的平均值,由于对每个$\alpha$,有
\begin{align*}
\frac{1}{2\pi} \int_{-\pi}^{\pi} \cos^+(\alpha - \theta) \mathrm{d}\theta = \frac{1}{\pi},
\end{align*}
所以这个平均值就是$\frac{1}{\pi} \sum_{j=1}^{n} |z_j|.$故
\begin{align*}
\left| \sum_S{z_k} \right|=\left| \sum_{S(\theta _0)}{z_k} \right|=\underset{\theta \in \left[ -\pi ,\pi \right]}{\max}\sum_{k=1}^n{|z_k|\cos ^+(\alpha _k-\theta )}\geqslant \frac{1}{2\pi}\int_{-\pi}^{\pi}{\sum_{k=1}^n{|z_k|\cos ^+(\alpha _k-\theta )}}=\frac{1}{\pi}\sum_{k=1}^n{|z_k|}.
\end{align*}


\end{proof}



















\end{document}
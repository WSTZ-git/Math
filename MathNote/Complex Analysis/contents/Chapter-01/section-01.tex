\documentclass[../../main.tex]{subfiles}
\graphicspath{{\subfix{../../image/}}} % 指定图片目录,后续可以直接使用图片文件名。

% 例如:
% \begin{figure}[H]
% \centering
% \includegraphics[scale=0.4]{图.png}
% \caption{}
% \label{figure:图}
% \end{figure}
% 注意:上述\label{}一定要放在\caption{}之后,否则引用图片序号会只会显示??.

\begin{document}

\section{复数的定义及其运算}

\begin{definition}[复数域]
我们把复数定义为一对有序的实数\((a,b)\),如果用\(\mathbf{R}\)记实数的全体,\(\mathbf{C}\)记复数的全体,那么
\[
\mathbf{C} = \{ (a,b) : a \in \mathbf{R}, b \in \mathbf{R} \}.
\]
在这个集合中定义加法和乘法两种运算:
\[
(a,b) + (c,d) = (a + c, b + d),
\]
\[
(a,b)(c,d) = (ac - bd, ad + bc).
\]
容易验证,加法和乘法都满足交换律和结合律;\((0,0)\)是零元素,\((-a,-b)\)是\((a,b)\)的负元素;\((1,0)\)是乘法的单位元素;每个非零元素\((a,b)\)有逆元素\(\left( \dfrac{a}{a^2 + b^2}, -\dfrac{b}{a^2 + b^2} \right)\);此外,\(\mathbf{C}\)中的加法和乘法还满足分配律:
\[
[(a,b) + (c,d)](e,f) = (a,b)(e,f) + (c,d)(e,f).
\]
因此,\(\mathbf{C}\)在上面定义的加法和乘法运算下构成一个域,称为\textbf{复数域}.如果记
\[
\tilde{\mathbf{R}} = \{ (a,0) : a \in \mathbf{R} \},
\]
那么\(\tilde{\mathbf{R}}\)是\(\mathbf{C}\)的一个子域。显然,\((a,0) \to a\)是\(\tilde{\mathbf{R}}\)与\(\mathbf{R}\)之间的一个同构对应,因此,实数域\(\mathbf{R}\)是\(\mathbf{C}\)的一个子域。我们直接记\((a,0) = a\)。
在\(\mathbf{C}\)中,\((0,1)\)这个元素有其特殊性,它满足
\[
(0,1)^2 = (0,1)(0,1) = (-1,0) = -1.
\]
专门用\(\mathrm{i}\)记\((0,1)\)这个元素,于是有\(\mathrm{i}^2 = -1\)。由于\((0,b) = (b,0) \cdot (0,1) = b\mathrm{i}\),于是每一个复数\((a,b)\)都可写成
\[
(a,b) = (a,0) + (0,b) = a + b\mathrm{i}.
\]
从现在开始,我们不再用实数对\((a,b)\)来记复数,而直接用\(z = a + b\mathrm{i}\)记复数,\(a\)称为\(z\)的实部,\(b\)称为\(z\)的虚部,分别记为\(a = \mathrm{Re}z\),\(b = \mathrm{Im}z\)。加法和乘法用现在的记号定义为:
\[
(a + b\mathrm{i}) + (c + d\mathrm{i}) = (a + c) + (b + d)\mathrm{i},
\]
\[
(a + b\mathrm{i})(c + d\mathrm{i}) = (ac - bd) + (ad + bc)\mathrm{i}.
\]
减法和除法分别定义为加法和乘法的逆运算:
\[
(a + b\mathrm{i}) - (c + d\mathrm{i}) = (a - c) + (b - d)\mathrm{i},
\]
\[
\begin{split}
\frac{a + b\mathrm{i}}{c + d\mathrm{i}} = (a + b\mathrm{i})\left( \frac{c - d\mathrm{i}}{c^2 + d^2} \right)= \frac{ac + bd}{c^2 + d^2} + \frac{bc - ad}{c^2 + d^2}\mathrm{i}.
\end{split}
\]

设\(z = a + b\mathrm{i}\)是一复数,定义
\[
|z| = \sqrt{a^2 + b^2},
\]
\[
\bar{z} = a - b\mathrm{i},
\]
\(|z|\)称为\(z\)的\textbf{模}或\textbf{绝对值},\(\bar{z}\)称为\(z\)的\textbf{共轭复数}。
\end{definition}

\begin{definition}[有序域]\label{definition:有序域}
域\(F\)称为\textbf{有序域},如果在\(F\)的元素间能确定一种关系(记为\(a < b\)),其满足下列要求:
\begin{enumerate}[(i)]
\item\label{有序域:(i)} 对\(F\)中任意两个元素\(a,b\),下述三个关系中必有而且只有一个成立:
\[
a < b, \quad a = b, \quad b < a;
\]

\item\label{有序域:(ii)} 如果\(a < b, b < c\),那么\(a < c\);

\item\label{有序域:(iii)} 如果\(a < b\),那么对任意\(c\),有\(a + c < b + c\);

\item\label{有序域:(iv)} 如果\(a < b, c > 0\),那么\(ac < bc\)。
\end{enumerate}
\end{definition}
\begin{note}
容易知道,实数域是有序域,而复数域则不是。
\end{note}

\begin{theorem}\label{theorem:复数域不是有序域}
复数域不是有序域.
\end{theorem}
\begin{proof}
如果\(\mathbf{C}\)是有序域,那么因为\(\mathrm{i} \neq 0\),\(\mathrm{i}\)和\(0\)之间必有\(\mathrm{i} > 0\)或\(\mathrm{i} < 0\)的关系.如果\(\mathrm{i} > 0\),则由\hyperref[definition:有序域]{有序域(iv)}得\(\mathrm{i} \cdot \mathrm{i} > \mathrm{i} \cdot 0\),即\(-1 > 0\),再由\hyperref[有序域:(iii)]{(iii)},两端都加\(1\),即得\(0 > 1\).另一方面,从\(-1 > 0\)还可得\((-1) \cdot (-1) > 0 \cdot (-1)\),即\(1 > 0\),这和刚才得到的\(0 > 1\)矛盾.如果\(\mathrm{i} < 0\),两端都加\(-\mathrm{i}\),得\(0 < -\mathrm{i}\),再由\hyperref[definition:有序域]{有序域(iv)},两端乘\(-\mathrm{i}\),得\(-1 > 0\).重复上面的讨论,即可得\(0 > 1\)和\(0 < 1\)的矛盾.所以,复数域不是有序域. 

\end{proof}

\begin{proposition}[复数运算性质]\label{proposition:复数运算性质}
设\(z\)和\(w\)是两个复数,那么
\begin{enumerate}[(i)]
\item \(\mathrm{Re}z = \dfrac{1}{2}(z + \bar{z})\),\(\mathrm{Im}z = \dfrac{1}{2\mathrm{i}}(z - \bar{z})\);

\item \(z\bar{z} = |z|^2\),\(\overline{z} = \dfrac{|z|^2}{z}\);

\item \(\overline{z + w} = \bar{z} + \bar{w}\),\(\overline{zw} = \bar{z}\,\bar{w}\);

\item\(|zw| = |z||w|\),\(\left| \dfrac{z}{w} \right| = \dfrac{|z|}{|w|}\);

\item\(|z| = |\bar{z}|\),$|z|=|-z|$.
\end{enumerate}
\end{proposition}
\begin{proof}
\begin{enumerate}[(i)]
\item 

\item 

\item

\item 由(ii)可得$|zw|^2 = (zw)(\overline{zw}) = |z|^2|w|^2,$进而\(|zw| = |z||w|\).

\item
\end{enumerate}

\end{proof}

\begin{proposition}\label{proposition:复数基本不等式}
设\(z\)和\(w\)是两个复数,那么

(i) \(|\mathrm{Re}z| \leqslant |z|\),\(|\mathrm{Im}z| \leqslant |z|\);

(ii) \(|z + w| \leqslant |z| + |w|\),等号成立当且仅当存在某个实数\(t \geqslant 0\),使得\(z = tw\);

(iii) \(|z - w| \geqslant \big||z| - |w|\big|\)。
\end{proposition}
\begin{proof}
(i) 从\(\mathrm{Re}z\),\(\mathrm{Im}z\)和\(|z|\)的定义马上知道不等式成立。

(ii) 利用\hyperref[proposition:复数运算性质]{复数运算性质(ii)(i)}和这里的不等式(i),即得
\[
\begin{split}
|z + w|^2 &= (z + w)(\overline{z + w}) = |z|^2 + 2\mathrm{Re}(z\overline{w}) + |w|^2 \\
& \leqslant |z|^2 + 2|z||w| + |w|^2 = (|z| + |w|)^2,
\end{split}
\]
由此即知(ii)成立。由上面的不等式可以看出,等式成立的充要条件是\(\mathrm{Re}(z\overline{w}) = |z\overline{w}|\),这等价于$z\overline{w}\in \mathbb{R}$且\(z\overline{w} \geqslant 0\)。不妨设\(w \neq 0\)(\(w = 0\)时,等号显然成立),由于\(\overline{w} = \dfrac{|w|^2}{w}\),故\(z\overline{w}=\dfrac{z}{w}|w|^2 \geqslant 0\)。令\(t = \left( \dfrac{z}{w}|w|^2 \right) \dfrac{1}{|w|^2}\),则$t\in \mathbb{R}$且\(t \geqslant 0\),而且\(z = tw\)。

(iii) 当\(| z | = | w |\)时,结论显然成立.

当\(| z | > | w |\)时,由\((\mathrm{ii})\)可得
\[| z | = | (z - w) + w | \leqslant | z - w | + | w |,\]
移项可得\(| z - w | \geqslant | z | - | w | = | | z | - | w | |\).由\((\mathrm{ii})\)可知,此时等号成立当且仅当存在某个实数\(t \geqslant 0\),使得\(z - w = tw\),即\(z = (t + 1)w\).

当\(| z | < | w |\)时,由\((\mathrm{ii})\)可得
\[| w | = | (w - z) + z | \leqslant | w - z | + | z |,\]
移项可得\(| z - w | = | w - z | \geqslant | w | - | z | = | | z | - | w | |\).由\((\mathrm{ii})\)可知,此时等号成立当且仅当存在某个实数\(t \geqslant 0\),使得\(w - z = tz\),即\(w = (t + 1)z\).

\end{proof}

\begin{corollary}
设\(z_1, \cdots, z_n\)是任意\(n\)个复数,则
\[
|z_1 + \cdots + z_n| \leqslant |z_1| + \cdots + |z_n|.
\]
\end{corollary}
\begin{proof}
由\nrefpro{proposition:复数基本不等式}{(ii)}及数学归纳法易证.等号成立当且仅当$z_1,z_2,\cdots,z_n$线性相关.

\end{proof}







\end{document}
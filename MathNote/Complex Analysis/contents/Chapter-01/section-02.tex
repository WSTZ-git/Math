\documentclass[../../main.tex]{subfiles}
\graphicspath{{\subfix{../../image/}}} % 指定图片目录,后续可以直接使用图片文件名。

% 例如:
% \begin{figure}[H]
% \centering
% \includegraphics[scale=0.4]{图.png}
% \caption{}
% \label{figure:图}
% \end{figure}
% 注意:上述\label{}一定要放在\caption{}之后,否则引用图片序号会只会显示??.

\begin{document}

\section{复数的几何表示}\label{section:复数的几何表示}

在平面上取定一个直角坐标系,实数对\((a,b)\)就表示平面上的一个点,所以复数\(z = a + b\mathrm{i}\)可以看成平面上以\(a\)为横坐标、以\(b\)为纵坐标的一个点(\reffig{figure:图1.1})。这个点的极坐标设为\((r,\theta)\),那么
\begin{figure}[H]
\centering
\includegraphics[scale=0.3]{图1.1.png}
\caption{}
\label{figure:图1.1}
\end{figure}
\[
a = r\cos\theta, \quad b = r\sin\theta,
\]
因而复数\(z = a + b\mathrm{i}\)也可表示为
\[
z = r(\cos\theta + \mathrm{i}\sin\theta).
\]
这里,\(r = |z| = \sqrt{a^2 + b^2}\)就是前面定义过的\(z\)的模,\(\theta\)称为\(z\)的\textbf{辐角},记为\(\theta = \mathrm{Arg}z\)。容易看出,如果\(\theta\)是\(z\)的辐角,那么\(\theta + 2k\pi\)也是\(z\)的辐角,这里,\(k\)是任意的整数,因此\(z\)的\textbf{辐角有无穷多个}。但是在\(\mathrm{Arg}z\)中,只有一个\(\theta\)满足\(-\pi < \theta \leqslant \pi\),称这个\(\theta\)为\(z\)的\textbf{辐角的主值},把它记为\(\arg z\)。因而
\[
\mathrm{Arg}z = \arg z + 2k\pi, \ k \in \mathbf{Z},
\]
这里,\(\mathbf{Z}\)表示整数的全体。注意,\(0\)的辐角没有意义。

我们还可把复数\(z = a + b\mathrm{i}\)看成在\(x\)轴和\(y\)轴上的投影分别为\(a\)和\(b\)的一个向量,这时我们就把复数和向量作为同义语来使用。容易知道,由一向量经过平行移动所得的所有向量表示的是同一个复数。如果一个向量的起点和终点分别为复数\(z_1\)和\(z_2\),那么这个向量所表示的复数便是\(z_2 - z_1\),因而\(|z_2 - z_1|\)就表示\(z_1\)与\(z_2\)之间的距离。特别地,当一个向量的起点为原点时,它的终点所表示的复数和向量所表示的复数是一致的。

由此可以知道,前面定义的复数的加法和向量的加法是一致的:把两个不重合的非零向量\(z_1\)和\(z_2\)的起点取在原点,以\(z_1\)和\(z_2\)为两边作平行四边形,那么以原点为起点沿对角线所作的向量就表示\(z_1 + z_2\);以\(z_2\)为起点,\(z_1\)为终点的向量就表示\(z_1 - z_2\)(\reffig{figure:图1.2})。现在再来看\nrefpro{proposition:复数基本不等式}{(ii)}的不等式\(|z_1 + z_2| \leqslant |z_1| + |z_2|\),它实际上就是三角形两边之和大于第三边的最简单的几何命题。
\begin{figure}[H]
\centering
\includegraphics[scale=0.3]{图1.2.png}
\caption{}
\label{figure:图1.2}
\end{figure}

\begin{theorem}\label{theorem:复数辐角的性质}
设$z_1,z_2$是两个复数,则
\begin{enumerate}[(1)]
\item $|z_1z_2| = |z_1||z_2|,\mathrm{Arg}(z_1z_2) = \mathrm{Arg}z_1 + \mathrm{Arg}z_2.$

\item $\left| \frac{z_1}{z_2} \right| = \frac{|z_1|}{|z_2|},
\mathrm{Arg}\left( \frac{z_1}{z_2} \right) = \mathrm{Arg}z_1 - \mathrm{Arg}z_2.$
\end{enumerate}
\end{theorem}
\begin{note}
在(1)中,第一个等式在\nrefpro{proposition:复数运算性质}{(iv)}中已经证明过;第二个等式应该理解为两个集合的相等。这就是说,两个复数的乘积是这样一个复数,它的模是两个复数的模的乘积,它的辐角是两个复数的辐角之和。从几何上看,用复数\(w\)乘复数\(z\),相当于把\(z\)沿反时针方向转动大小为\(\arg w\)的角,再让\(z\)的长度伸长\(|w|\)倍。特别地,如果\(w\)是单位向量,那么\(w\)乘\(z\)的结果就是把\(z\)沿反时针方向转动大小为\(\arg w\)的角。例如,已知\(\mathrm{i}\)是单位向量,它的辐角为\(\frac{\pi}{2}\),因此\(\mathrm{i}z\)就是把\(z\)按反时针方向转动\(\dfrac{\pi}{2}\)角所得的向量。这种几何直观在考虑问题时非常有用。

在(2)中,第二个等式也理解为集合的相等。这说明向量\(z_1\)与\(z_2\)之间的夹角可以用\(\mathrm{Arg}\left( \frac{z_1}{z_2} \right)\)来表示,这一简单的事实在讨论某些几何问题时很有用。
\end{note}
\begin{proof}
为了说明复数乘法的几何意义,我们采用复数的三角表示式。设
\[
z_1 = r_1(\cos\theta_1 + \mathrm{i}\sin\theta_1),
\]
\[
z_2 = r_2(\cos\theta_2 + \mathrm{i}\sin\theta_2),
\]
\begin{enumerate}[(1)]
\item 注意到
\[
z_1z_2 = r_1r_2(\cos(\theta_1 + \theta_2) + \mathrm{i}\sin(\theta_1 + \theta_2)).
\]
由此立刻得到
\[
|z_1z_2| = |z_1||z_2|,
\]
\[
\mathrm{Arg}(z_1z_2) = \mathrm{Arg}z_1 + \mathrm{Arg}z_2.
\]

\item 再看复数的除法,由于
\[
\frac{z_1}{z_2} = \frac{r_1}{r_2}[\cos(\theta_1 - \theta_2) + \mathrm{i}\sin(\theta_1 - \theta_2)],
\]
所以
\[
\left| \frac{z_1}{z_2} \right| = \frac{|z_1|}{|z_2|},
\]
\[
\mathrm{Arg}\left( \frac{z_1}{z_2} \right) = \mathrm{Arg}z_1 - \mathrm{Arg}z_2.
\]
\end{enumerate}
\end{proof}

\begin{proposition}\label{proposition:复平面向量垂直与平行的充要条件}
\begin{enumerate}[(1)]
\item 向量\(z_1\)与\(z_2\)垂直的充要条件是\(\mathrm{Re}(z_1\bar{z}_2) = 0\)。

\item 向量\(z_1\)与\(z_2\)平行的充要条件为\(\mathrm{Im}(z_1\bar{z}_2) = 0\)。
\end{enumerate}
\end{proposition}
\begin{proof}
\begin{enumerate}[(1)]
\item 这是因为\(z_1\)与\(z_2\)垂直就是\(z_1\)与\(z_2\)之间的夹角为\(\pm \frac{\pi}{2}\),即\(\arg\left( \frac{z_1}{z_2} \right) = \pm \frac{\pi}{2}\),这说明\(\frac{z_1}{z_2}\)是一个纯虚数,因而\(z_1\bar{z}_2 = \frac{z_1}{z_2}|z_2|^2\)也是一个纯虚数,即\(\mathrm{Re}(z_1\bar{z}_2) = 0\)。

\item 这是因为\(z_1\)与\(z_2\)平行就是\(z_1\)与\(z_2\)之间的夹角为\(\pm \pi\),即\(\arg\left( \frac{z_1}{z_2} \right) = \pm \pi\),这说明\(\frac{z_1}{z_2}\)是一个实数,因而\(z_1\bar{z}_2 = \frac{z_1}{z_2}|z_2|^2\)也是一个实数,即\(\mathrm{Im}(z_1\bar{z}_2) = 0\)。
\end{enumerate}
\end{proof}

\begin{example}
在\reffig{figure:图1.3}的三角形中,\(AB = AC\),\(PQ = RS\),\(M\)和\(N\)分别是\(PR\)和\(QS\)的中点。证明:\(MN \perp BC\)。
\begin{figure}[H]
\centering
\includegraphics[scale=0.3]{图1.3.png}
\caption{}
\label{figure:图1.3}
\end{figure}
\end{example}
\begin{proof}
把\(A\)取作坐标原点,\(AB\)所在的直线取作\(x\)轴,那么\(P\),\(Q\)的坐标分别为\(a\)和\(a + h\)。如果用\(\mathrm{e}^{\mathrm{i}\theta}\)记\(\cos\theta + \mathrm{i}\sin\theta\),那么\(R\)点和\(S\)点可分别用复数\(r\mathrm{e}^{\mathrm{i}\theta}\)和\((r + h)\mathrm{e}^{\mathrm{i}\theta}\)表示。由于\(M\)和\(N\)分别是\(PR\)和\(SQ\)的中点,所以\(M\)和\(N\)可以分别用复数表示为
\[
M: \frac{1}{2}(a + r\mathrm{e}^{\mathrm{i}\theta}),
\]
\[
N: \frac{1}{2}[(a + h) + (r + h)\mathrm{e}^{\mathrm{i}\theta}].
\]
若记\(z_1 = \overrightarrow{MN}\),则
\[
\begin{split}
z_1 = \frac{1}{2}[(a + h) + (r + h)\mathrm{e}^{\mathrm{i}\theta}] - \frac{1}{2}(a + r\mathrm{e}^{\mathrm{i}\theta}) = \frac{h}{2}(1 + \mathrm{e}^{\mathrm{i}\theta}).
\end{split}
\]
如果记\(B\)的坐标为\(b\),因为\(AB = AC\),所以\(C\)的坐标为\(b\mathrm{e}^{\mathrm{i}\theta}\)。若记\(z_2 = \overrightarrow{BC}\),则
\[
z_2 = b\mathrm{e}^{\mathrm{i}\theta} - b = b(\mathrm{e}^{\mathrm{i}\theta} - 1).
\]
现在
\[
\begin{split}
z_1\bar{z}_2 = \frac{h}{2}(1 + \mathrm{e}^{\mathrm{i}\theta})b(\mathrm{e}^{-\mathrm{i}\theta} - 1) = \frac{bh}{2}(\mathrm{e}^{-\mathrm{i}\theta} - \mathrm{e}^{\mathrm{i}\theta}) = -\mathrm{i}bh\sin\theta,
\end{split}
\]
因而\(\mathrm{Re}(z_1\bar{z}_2) = 0\)。所以由\nrefpro{proposition:复平面向量垂直与平行的充要条件}{(1)}可知\(z_1\)垂直\(z_2\),即\(MN \perp BC\)。 
\end{proof}

\begin{example}
证明:平面上四点\(z_1, z_2, z_3, z_4\)共圆的充要条件为
\begin{align}\label{equation:::::-------1111.11}
\mathrm{Im}\left( \frac{z_1 - z_3}{z_1 - z_4} \bigg/ \frac{z_2 - z_3}{z_2 - z_4} \right) = 0.
\end{align}
\end{example}
\begin{proof}
从\reffig{figure:图1.4}可以看出,\(z_1, z_2, z_3, z_4\)四点共圆的充要条件是向量\(z_1 - z_3\)和\(z_1 - z_4\)的夹角等于向量\(z_2 - z_3\)和\(z_2 - z_4\)的夹角或互补(当\(z_2\)在\(z_3\)与\(z_4\)之间时),此时由\nrefpro{proposition:复平面向量垂直与平行的充要条件}{(2)}立得.即
\[
\begin{split}
\arg\left( \frac{z_1 - z_3}{z_1 - z_4} \bigg/ \frac{z_2 - z_3}{z_2 - z_4} \right) = \arg\left( \frac{z_1 - z_3}{z_1 - z_4} \right) - \arg\left( \frac{z_2 - z_3}{z_2 - z_4} \right) 
= 0 \text{ 或 } \pm \pi.
\end{split}
\]
这说明复数\(\frac{z_1 - z_3}{z_1 - z_4} \bigg/ \frac{z_2 - z_3}{z_2 - z_4}\)在实轴上,因而等式\eqref{equation:::::-------1111.11}成立。
\begin{figure}[H]
\centering
\includegraphics[scale=0.4]{图1.4.png}
\caption{}
\label{figure:图1.4}
\end{figure} 
\end{proof}

\begin{theorem}[De Moivre公式]\label{theorem:De Moivre公式}
对任意整数$n$,都有$(\cos\theta + \mathrm{i}\sin\theta)^n = \cos n\theta + \mathrm{i}\sin n\theta.$
\end{theorem}
\begin{proof}
设\(z_1 = r_1(\cos\theta_1 + \mathrm{i}\sin\theta_1), \cdots, z_n = r_n(\cos\theta_n + \mathrm{i}\sin\theta_n)\)是给定的\(n\)个复数,容易用数学归纳法证明:
\[
z_1 \cdots z_n = r_1 \cdots r_n[\cos(\theta_1 + \cdots + \theta_n) + \mathrm{i}\sin(\theta_1 + \cdots + \theta_n)].
\]

特别当\(z_1 = \cdots = z_n\)都是单位向量时,就有
\[
(\cos\theta + \mathrm{i}\sin\theta)^n = \cos n\theta + \mathrm{i}\sin n\theta,
\]
其实,对于负整数,上面的公式也成立:
\[
\begin{split}
(\cos\theta + \mathrm{i}\sin\theta)^{-n} &= \frac{1}{(\cos\theta + \mathrm{i}\sin\theta)^n} = \frac{1}{\cos n\theta + \mathrm{i}\sin n\theta} \\
&= \cos n\theta - \mathrm{i}\sin n\theta = \cos(-n)\theta + \mathrm{i}\sin(-n)\theta.
\end{split}
\]
\end{proof}

\begin{proposition}
设$w$是一个复数,则满足方程$z^n=w$的复数根有$n$个,即
\begin{align*}
z = \sqrt[n]{|w|}\left( \cos\frac{\theta + 2k\pi}{n} + \mathrm{i}\sin\frac{\theta + 2k\pi}{n} \right),\quad k = 0, 1, \cdots, n - 1.
\end{align*}
\end{proposition}
\begin{proof}
现在设\(w = r(\cos\theta + \mathrm{i}\sin\theta)\)是给定的,要求的\(z = \rho(\cos\varphi + \mathrm{i}\sin\varphi)\)。由 \hyperref[theorem:De Moivre公式]{De Moivre 公式},\(z^n = w\)等价于
\[
\rho^n(\cos n\varphi + \mathrm{i}\sin n\varphi) = r(\cos\theta + \mathrm{i}\sin\theta).
\]
由此即得\(\rho = \sqrt[n]{r}\),\(n\varphi = \theta + 2k\pi\),\(k = 0, 1, \cdots, n - 1\)。这就是说,共有\(n\)个复数满足\(z^n = w\),它们是
\[
\begin{split}
z = \sqrt[n]{|w|}\left( \cos\frac{\theta + 2k\pi}{n} + \mathrm{i}\sin\frac{\theta + 2k\pi}{n} \right),\quad k = 0, 1, \cdots, n - 1.
\end{split}
\]
这\(n\)个复数恰好是以原点为中心、\(\sqrt[n]{|w|}\)为半径的圆的内接正\(n\)边形的顶点。当\(w = 1\)时,若记\(\omega = \cos\frac{2\pi}{n} + \mathrm{i}\sin\frac{2\pi}{n}\),则\(\sqrt[n]{1}\)的\(n\)个值为
\[
1, \omega, \omega^2, \cdots, \omega^{n - 1},
\]
称为\(n\)个单位根。如果用\(\sqrt[n]{w}\)记\(w\)的任一\(n\)次根,那么\(w\)的\(n\)个\(n\)次根又可表示为
\[
\sqrt[n]{w}, \sqrt[n]{w}\omega, \cdots, \sqrt[n]{w}\omega^{n - 1}.
\]
\end{proof}










\end{document}
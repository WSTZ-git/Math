\documentclass[../../main.tex]{subfiles}% 注意这里的文件路径不能用 ./main.tex ,否则用latexmk编译子文件会报错
\graphicspath{{\subfix{./image/}}} % 指定图片目录,后续可以直接使用图片文件名
% 注意这里的文件路径不能用 ../../image/ ,否则用latexmk编译子文件会报错

% 例如:
% \begin{figure}[H]
% \centering
% \includegraphics[scale=0.3]{图.png}
% \caption{}
% \label{figure:图}
% \end{figure}
% 注意:上述\label{}一定要放在\caption{}之后,否则引用图片序号会只会显示??.

\begin{document}

\section{复数列的极限}

\begin{definition}
对于\(a \in \mathbb{C}\),\(r > 0\),称
\[
B(a, r) = \{ z \in \mathbb{C} : | z - a | < r \}
\]
为以\(a\)为中心、以\(r\)为半径的\textbf{圆盘}。特别当\(a = 0\),\(r = 1\)时,\(B(0, 1) = \{ z : | z | < 1 \}\)称为\textbf{单位圆盘}。\(B(a, r)\)也称为\(a\)点的一个\textbf{\(r\)邻域},或简称为\(a\)点的\textbf{邻域}。无穷远点\(z = \infty\)的邻域是指集合\(\{ z \in \mathbb{C} : | z | > R \}\),记为\(B(\infty, R)\)。
\end{definition}

\begin{definition}
我们说\(\mathbb{C}\)中的复数列\(\{ z_n \}\)收敛到\(\mathbb{C}\)中的点\(z_0\),是指对于任给的\(\varepsilon > 0\),存在正整数\(N\),当\(n > N\)时,\(| z_n - z_0 | < \varepsilon\),记作\(\lim\limits_{n \to \infty} z_n = z_0\)。或者从几何上来说,对任给的\(\varepsilon > 0\),当\(n\)充分大时,\(z_n \in B(z_0, \varepsilon)\).

我们称复数列\(\{ z_n \}\)收敛到\(\infty\),是指对任给的正数\(M > 0\),存在正整数\(N\),当\(n > N\)时,\(| z_n | > M\),记为\(\lim\limits_{n \to \infty} z_n = \infty\).或者从几何上来说,对任给的\(M > 0\),当\(n\)充分大时,\(z_n \in B(\infty, M)\).
\end{definition}

\begin{theorem}\label{theorem:复数收敛的充要条件}
设\(z_n = x_n + \mathrm{i}y_n\),\(z_0 = x_0 + \mathrm{i}y_0\),则\(\lim\limits_{n \to \infty} z_n = z_0\)的充分必要条件是\(\{ z_n \}\)的实部和虚部分别\(\lim\limits_{n \to \infty} x_n = x_0\)和\(\lim\limits_{n \to \infty} y_n = y_0\)。
\end{theorem}
\begin{proof}
设\(z_n = x_n + \mathrm{i}y_n\),\(z_0 = x_0 + \mathrm{i}y_0\),从等式
\[
| z_n - z_0 | = \sqrt{(x_n - x_0)^2 + (y_n - y_0)^2}
\]
马上可以得到:\(\lim\limits_{n \to \infty} z_n = z_0\)的充分必要条件是\(\{ z_n \}\)的实部和虚部分别\(\lim\limits_{n \to \infty} x_n = x_0\)和\(\lim\limits_{n \to \infty} y_n = y_0\)。

\end{proof}

\begin{theorem}\label{theorem:模长和辐角的极限等价于复数的极限}
设$z_0 \notin (-\infty,0]$,$z_n \neq 0$,$\forall\ n \in \mathbb{N}$. 证明: 复数列$\{z_n\}$收敛到$z_0$的充要条件是$\lim\limits_{n \to \infty}|z_n| = |z_0|$和$\lim\limits_{n \to \infty}\arg z_n = \arg z_0$.
\end{theorem}
\begin{proof}
{\heiti 必要性:}设$z_n=x_n+\mathrm{i}y_n$,若$\lim\limits_{n\rightarrow \infty}z_n=z_0=x_0+\mathrm{i}y_0$,则由定理知$\lim\limits_{n\rightarrow \infty}x_n=x_0$,$\lim\limits_{n\rightarrow \infty}y_n=y_0$.由二元函数$\sqrt{x^2+y^2}$的连续性知
\begin{align*}
\lim\limits_{n\rightarrow \infty}\left| z_n \right|=\lim\limits_{n\rightarrow \infty}\sqrt{x_{n}^{2}+y_{n}^{2}}=\sqrt{x_{0}^{2}+y_{0}^{2}}=\left| z_0 \right|.
\end{align*}
设$z_n=\left| z_n \right|e^{\mathrm{i}\theta _n}$,$z_0=\left| z_0 \right|e^{\mathrm{i}\theta _0}$,其中$\theta _0=\mathrm{arg}z_0\in \left( -\pi ,\pi \right)$,$\theta _n=\mathrm{arg}z_n$.由$\lim\limits_{n\rightarrow \infty}z_n=z_0\notin \left( -\infty ,0 \right]$知,存在$N>0$和一个与负实轴无交的邻域$U$,使得$z_n\in U$,$\forall n>N$.从而$\theta _n\in \left( -\pi ,\pi \right)$,$\forall n>N$.于是
\begin{align*}
\left| z_n-z_0 \right|^2=\left| z_n \right|^2+\left| z_0 \right|^2-2\left| z_n \right|\left| z_0 \right|\cos \left( \theta _n-\theta _0 \right),\quad \forall n>N.
\end{align*}
进而
\begin{align*}
\cos \left( \theta _n-\theta _0 \right) =\frac{\left| z_n \right|^2+\left| z_0 \right|^2-\left| z_n-z_0 \right|^2}{2\left| z_n \right|\left| z_0 \right|}.
\end{align*}
令$n\rightarrow \infty$得
\begin{align*}
\cos \left( \lim\limits_{n\rightarrow \infty}\left( \theta _n-\theta _0 \right) \right) =\lim\limits_{n\rightarrow \infty}\cos \left( \theta _n-\theta _0 \right) =\lim\limits_{n\rightarrow \infty}\frac{\left| z_n \right|^2+\left| z_0 \right|^2-\left| z_n-z_0 \right|^2}{2\left| z_n \right|\left| z_0 \right|}=1.
\end{align*}
又$\theta _n,\theta _0\in \left( -\pi ,\pi \right)$,$\forall n>N$,故$\theta _n-\theta _0\in \left( -2\pi ,2\pi \right)$,$\forall n>N$.因此$\lim\limits_{n\rightarrow \infty}\left( \theta _n-\theta _0 \right) \in \left( -2\pi ,2\pi \right)$,故必有$\lim\limits_{n\rightarrow \infty}\left( \theta _n-\theta _0 \right) =0$,即
\begin{align*}
\lim\limits_{n\rightarrow \infty}\mathrm{arg}z_n=\mathrm{arg}z_0.
\end{align*}

{\heiti 充分性:}设$z_n=\left| z_n \right|e^{\mathrm{i}\theta _n}$,$z_0=\left| z_0 \right|e^{\mathrm{i}\theta _0}$,其中$\theta _0=\mathrm{arg}z_0\in \left( -\pi ,\pi \right)$,$\theta _n=\mathrm{arg}z_n$,则由$\lim\limits_{n\rightarrow \infty}\left| z_n \right|=\left| z_0 \right|$和$\lim\limits_{n\rightarrow \infty}\mathrm{arg}z_n=\mathrm{arg}z_0$知,$\forall \varepsilon >0$,存在$N\in \mathbb{N}$,使得
\begin{align*}
\left| \left| z_n \right|-\left| z_0 \right| \right|<\varepsilon,\quad \left| \theta _n-\theta _0 \right|<\varepsilon,\quad \forall n>N.
\end{align*}
于是对$\forall n>N$,都有
\begin{align*}
\left| z_n-z_0 \right|^2&=\left| z_n \right|^2+\left| z_0 \right|^2-2\left| z_n \right|\left| z_0 \right|\cos \left( \theta _n-\theta _0 \right) \\
&\leqslant \left( \left| z_n \right|-\left| z_0 \right| \right) ^2+2\left| z_n \right|\left| z_0 \right|\left( 1-\cos \left( \theta _n-\theta _0 \right) \right) \\
&\leqslant \varepsilon ^2+4\left( \left| z_0 \right|+\varepsilon \right) ^2\sin ^2\frac{\theta _n-\theta _0}{2} \\
&\leqslant \varepsilon ^2+2\left( \left| z_0 \right|+\varepsilon \right) ^2\left| \theta _n-\theta _0 \right|^2 \\
&<\varepsilon ^2+2\left( \left| z_0 \right|+\varepsilon \right) ^2\varepsilon ^2.
\end{align*}
故
\begin{align*}
\left| z_n-z_0 \right|<\varepsilon \sqrt{1+2\left( \left| z_0 \right|+\varepsilon \right) ^2},\quad \forall n>N.
\end{align*}
此即$\lim\limits_{n\rightarrow \infty}z_n=z_0$.

\end{proof}

\begin{definition}
复数列\(\{ z_n \}\)称为\(\mathbf{Cauchy}\)\textbf{列},如果对任给的\(\varepsilon > 0\),存在正整数\(N\),当\(m, n > N\)时,有\(| z_n - z_m | < \varepsilon\)。
\end{definition}

\begin{theorem}\label{theorem:复数Cauchy列充要条件}
\(\{ z_n \}\)是Cauchy列的充分必要条件是它的实部\(\{ x_n \}\)和虚部\(\{ y_n \}\)都是实的Cauchy列.
\end{theorem}
\begin{proof}
设\(z_n = x_n + \mathrm{i}y_n\),\(z_m = x_m + \mathrm{i}y_m\),那么从等式
\[
| z_n - z_m | = \sqrt{(x_n - x_m)^2 + (y_n - y_m)^2}
\]
知道,\(\{ z_n \}\)是Cauchy列的充分必要条件是它的实部\(\{ x_n \}\)和虚部\(\{ y_n \}\)都是实的Cauchy列.

\end{proof}

\begin{theorem}[复数域的Cauchy收敛准则]\label{theorem:复数域的Cauchy收敛准则}
\(\{ z_n \}\)收敛的充要条件是\(\{ z_n \}\)为Cauchy列.
\end{theorem}
\begin{note}
由此知道复数域\(\mathbb{C}\)是\textbf{完备的}.
\end{note}
\begin{proof}
由\refthe{theorem:复数收敛的充要条件}和\refthe{theorem:复数Cauchy列充要条件},再结合实数域中的Cauchy收敛准则立刻得到复数域的Cauchy收敛准则.

\end{proof}

\begin{proposition}\label{proposition:复变版本的经典极限和Cacuhy命题}
\begin{enumerate}[(1)]
\item\label{proposition:复变版本的经典极限和Cacuhy命题-1} 设$z = x + \mathrm{i}y \in \mathbb{C}$,证明:
\begin{align*}
\lim\limits_{n \to \infty} \left(1 + \frac{z}{n}\right)^n = e^z = e^{x+\text{i}y} = \mathrm{e}^x (\cos y + \mathrm{i}\sin y).
\end{align*}

\item\label{proposition:复变版本的经典极限和Cacuhy命题-2} 证明:若$\lim\limits_{n \to \infty} z_n = z_0$,则
\begin{align*}
\lim\limits_{n \to \infty} \frac{z_1 + z_2 + \cdots + z_n}{n} = z_0.
\end{align*}
\end{enumerate}
\end{proposition}
\begin{proof}
\begin{enumerate}[(1)]
\item 当$z=0$时,结论显然成立.下设$z\neq 0$,则$e^x>0$,从而$e^z=e^x(\cos y+\text{i}\sin y)\notin (-\infty, 0]$.注意到
\begin{align*}
1+\frac{z}{n}=1+\frac{x+\mathrm{i}y}{n}=\left| 1+\frac{x+\mathrm{i}y}{n} \right|e^{\mathrm{iarg}\left( 1+\frac{x+\mathrm{i}y}{n} \right)}=\sqrt{\left( 1+\frac{x}{n} \right) ^2+\frac{y^2}{n^2}}e^{\mathrm{iarg}\left( 1+\frac{x+\mathrm{i}y}{n} \right)},
\end{align*}
故
\begin{align*}
\left( 1+\frac{z}{n} \right) ^n=\left[ \left( 1+\frac{x}{n} \right) ^2+\frac{y^2}{n^2} \right] ^{\frac{n}{2}}e^{\mathrm{i}n\mathrm{arg}\left( 1+\frac{x+\mathrm{i}y}{n} \right)}.
\end{align*}
于是
\begin{align*}
\lim\limits_{n\rightarrow \infty}\left| \left( 1+\frac{z}{n} \right) ^n \right|&=\lim\limits_{n\rightarrow \infty}\left[ \left( 1+\frac{x}{n} \right) ^2+\frac{y^2}{n^2} \right] ^{\frac{n}{2}}=\lim\limits_{n\rightarrow \infty}\left[ 1+\frac{2x}{n}+\frac{x^2+y^2}{n^2} \right] ^{\frac{n}{2}}\\
&=\lim\limits_{n\rightarrow \infty}e^{\frac{n}{2}\ln \left( 1+\frac{2x}{n}+\frac{x^2+y^2}{n^2} \right)}=\lim\limits_{n\rightarrow \infty}e^{\frac{n}{2}\left( \frac{2x}{n}+\frac{x^2+y^2}{n^2} \right)}=e^x.
\end{align*}
取充分大的$N$,使得$1+\frac{x}{n}>0,\forall n>N$.从而
\begin{align*}
\mathrm{arg}\left( 1+\frac{z}{n} \right) ^n=n\mathrm{arg}\left( 1+\frac{x+\mathrm{i}y}{n} \right) =n\arctan \frac{\frac{y}{n}}{1+\frac{x}{n}},\quad \forall n>N.
\end{align*}
令$n\rightarrow \infty$得
\begin{align*}
\lim\limits_{n\rightarrow \infty}\mathrm{arg}\left( 1+\frac{z}{n} \right) ^n&=\lim\limits_{n\rightarrow \infty}\left[ n\mathrm{arg}\left( 1+\frac{x+\mathrm{i}y}{n} \right) \right] =\lim\limits_{n\rightarrow \infty}\left( n\arctan \frac{\frac{y}{n}}{1+\frac{x}{n}} \right)\\
&=\lim\limits_{n\rightarrow \infty}\left( n\arctan \frac{y}{n+x} \right) =\lim\limits_{n\rightarrow \infty}\frac{ny}{n+x}=y.
\end{align*}
又$e^z\notin (-\infty, 0]$,故由\refthe{theorem:模长和辐角的极限等价于复数的极限}知
\begin{align*}
\lim\limits_{n\rightarrow \infty}\left( 1+\frac{z}{n} \right) ^n=e^{x+\mathrm{i}y}=e^x\left( \cos y+\mathrm{i}\sin y \right).
\end{align*}

\item 设$z_n=x_n+\mathrm{i}y_n,z_0=x_0+\mathrm{i}y_0$,则由\refthe{theorem:复数收敛的充要条件}知
\begin{align*}
\lim\limits_{n\rightarrow \infty}x_n=x_0,\quad \lim\limits_{n\rightarrow \infty}y_n=y_0.
\end{align*}
又因为
\begin{align*}
\frac{z_1+z_2+\cdots +z_n}{n}=\frac{x_1+x_2+\cdots +x_n}{n}+\mathrm{i}\frac{y_1+y_2+\cdots +y_n}{n}.
\end{align*}
所以由Stolz定理可得
\begin{align*}
\lim\limits_{n\rightarrow \infty}\frac{x_1+x_2+\cdots +x_n}{n}=\lim\limits_{n\rightarrow \infty}x_n=x_0,\quad \lim\limits_{n\rightarrow \infty}\frac{y_1+y_2+\cdots +y_n}{n}=\lim\limits_{n\rightarrow \infty}y_n=y_0.
\end{align*}
于是由\refthe{theorem:复数收敛的充要条件}知
\begin{align*}
\lim\limits_{n\rightarrow \infty}\frac{z_1+z_2+\cdots +z_n}{n}=z_0.
\end{align*}
\end{enumerate}

\end{proof}

\begin{theorem}[Toplitz定理]\label{theorem:复变版本-Toplitz定理}
设复无穷三角阵
\[
\begin{array}{ccc} 
a_{11} \\
a_{21} & a_{22} \\
a_{31} & a_{32} & a_{33} \\
\cdots & \cdots & \cdots
\end{array}
\]
满足
\begin{enumerate}[(i)]
\item 对$\forall k\in \mathbb{N}$,$\lim\limits_{n\rightarrow \infty}a_{nk}=0$;

\item $\lim\limits_{n\rightarrow \infty}\sum\limits_{k=1}^n a_{nk}=S$存在;

\item $\sum\limits_{k=1}^n |a_{nk}| \leqslant M < \infty,\forall\ n \in N$.
\end{enumerate}
证明:若复数列$\{z_n\}$收敛到$z_0$,则$\lim\limits_{n\rightarrow \infty}\sum\limits_{k=1}^n a_{nk}z_k=z_0S$.
\end{theorem}
\begin{proof}
对$\forall n\in \mathbb{N}$, 都有
\begin{align*}
\sum_{k=1}^n{a_{nk}z_k}=z_0\sum_{k=1}^n{a_{nk}}+\sum_{k=1}^n{a_{nk}\left( z_k-z_0 \right)}.
\end{align*}
由条件得$\lim\limits_{n\rightarrow \infty}z_0\sum_{k=1}^n{a_{nk}}=z_0S$, 因此只需证$\lim\limits_{n\rightarrow \infty}\sum_{k=1}^n{a_{nk}\left( z_k-z_0 \right)}=0$. 对$\forall N\in \mathbb{N}$, $n>N$, 有
\begin{align*}
\left| \sum_{k=1}^n{a_{nk}\left( z_k-z_0 \right)} \right|&\leqslant \left| \sum_{k=1}^N{a_{nk}\left( z_k-z_0 \right)} \right|+\left| \sum_{k=N+1}^n{a_{nk}\left( z_k-z_0 \right)} \right|
\\
&\leqslant \sum_{k=1}^N{\left| a_{nk}\left( z_k-z_0 \right) \right|}+\sup\limits_{k\geqslant N+1}\left| z_k-z_0 \right|\cdot \sum_{k=1}^n{\left| a_{nk} \right|}
\\
&\leqslant \sum_{k=1}^N{\left| a_{nk} \right|\cdot \left| z_k-z_0 \right|}+M\sup\limits_{k\geqslant N+1}\left| z_k-z_0 \right|.
\end{align*}
令$n\rightarrow +\infty$得
\begin{align*}
\varlimsup\limits_{n\rightarrow \infty}\left| \sum_{k=1}^n{a_{nk}\left( z_k-z_0 \right)} \right|\leqslant M\sup\limits_{k\geqslant N+1}\left| z_k-z_0 \right|.
\end{align*}
再令$N\rightarrow +\infty$得
\begin{align*}
\varlimsup\limits_{n\rightarrow \infty}\left| \sum_{k=1}^n{a_{nk}\left( z_k-z_0 \right)} \right|\leqslant M\varlimsup\limits_{N\rightarrow +\infty}\left| z_k-z_0 \right|=0.
\end{align*}
故结论得证.

\end{proof}

\begin{example}
证明:若$\lim\limits_{n \to \infty} z_n = z_0$,$\lim\limits_{n \to \infty} w_n = w_0$,则
\begin{align*}
\lim\limits_{n \to \infty} \frac{1}{n} \sum\limits_{k=1}^n z_k w_{n - k} = z_0 w_0.
\end{align*}
\end{example}
\begin{proof}
记$a_{nk}=\frac{w_{n-k}}{n}$, 则由条件和\rrefpro{proposition:复变版本的经典极限和Cacuhy命题}{proposition:复变版本的经典极限和Cacuhy命题-2}知
\begin{align*}
\lim\limits_{n\rightarrow \infty}a_{nk}=\lim\limits_{n\rightarrow \infty}\frac{w_{n-k}}{n}=0,\quad \lim\limits_{n\rightarrow \infty}\sum_{k=1}^n{a_{nk}}=\lim\limits_{n\rightarrow \infty}\sum_{k=1}^n{\frac{w_{n-k}}{n}}=\lim\limits_{n\rightarrow \infty}\frac{\sum\limits_{k=1}^n{w_k}}{n}=w_0.
\end{align*}
故由\hyperref[theorem:复变版本-Toplitz定理]{Toplitz定理}知
\begin{align*}
\lim\limits_{n\rightarrow \infty} \frac{1}{n}\sum_{k=1}^n{z_kw_{n-k}}=z_0w_0.
\end{align*}

\end{proof}











\end{document}
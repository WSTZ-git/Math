\documentclass[../../main.tex]{subfiles}
\graphicspath{{\subfix{../../image/}}} % 指定图片目录,后续可以直接使用图片文件名。

% 例如:
% \begin{figure}[H]
% \centering
% \includegraphics[scale=0.4]{图.png}
% \caption{}
% \label{figure:图}
% \end{figure}
% 注意:上述\label{}一定要放在\caption{}之后,否则引用图片序号会只会显示??.

\begin{document}

\section{复数列的极限}

\begin{definition}
对于\(a \in \mathbf{C}\),\(r > 0\),称
\[
B(a, r) = \{ z \in \mathbf{C} : | z - a | < r \}
\]
为以\(a\)为中心、以\(r\)为半径的\textbf{圆盘}。特别当\(a = 0\),\(r = 1\)时,\(B(0, 1) = \{ z : | z | < 1 \}\)称为\textbf{单位圆盘}。\(B(a, r)\)也称为\(a\)点的一个\textbf{\(r\)邻域},或简称为\(a\)点的\textbf{邻域}。无穷远点\(z = \infty\)的邻域是指集合\(\{ z \in \mathbf{C} : | z | > R \}\),记为\(B(\infty, R)\)。
\end{definition}

\begin{definition}
我们说\(\mathbf{C}\)中的复数列\(\{ z_n \}\)收敛到\(\mathbf{C}\)中的点\(z_0\),是指对于任给的\(\varepsilon > 0\),存在正整数\(N\),当\(n > N\)时,\(| z_n - z_0 | < \varepsilon\),记作\(\lim\limits_{n \to \infty} z_n = z_0\)。或者从几何上来说,对任给的\(\varepsilon > 0\),当\(n\)充分大时,\(z_n \in B(z_0, \varepsilon)\).

我们称复数列\(\{ z_n \}\)收敛到\(\infty\),是指对任给的正数\(M > 0\),存在正整数\(N\),当\(n > N\)时,\(| z_n | > M\),记为\(\lim\limits_{n \to \infty} z_n = \infty\).或者从几何上来说,对任给的\(M > 0\),当\(n\)充分大时,\(z_n \in B(\infty, M)\).
\end{definition}

\begin{theorem}\label{theorem:复数收敛的充要条件}
设\(z_n = x_n + \mathrm{i}y_n\),\(z_0 = x_0 + \mathrm{i}y_0\),则\(\lim\limits_{n \to \infty} z_n = z_0\)的充分必要条件是\(\{ z_n \}\)的实部和虚部分别\(\lim\limits_{n \to \infty} x_n = x_0\)和\(\lim\limits_{n \to \infty} y_n = y_0\)。
\end{theorem}
\begin{proof}
设\(z_n = x_n + \mathrm{i}y_n\),\(z_0 = x_0 + \mathrm{i}y_0\),从等式
\[
| z_n - z_0 | = \sqrt{(x_n - x_0)^2 + (y_n - y_0)^2}
\]
马上可以得到:\(\lim\limits_{n \to \infty} z_n = z_0\)的充分必要条件是\(\{ z_n \}\)的实部和虚部分别\(\lim\limits_{n \to \infty} x_n = x_0\)和\(\lim\limits_{n \to \infty} y_n = y_0\)。
\end{proof}

\begin{definition}
复数列\(\{ z_n \}\)称为\(\mathbf{Cauchy}\)\textbf{列},如果对任给的\(\varepsilon > 0\),存在正整数\(N\),当\(m, n > N\)时,有\(| z_n - z_m | < \varepsilon\)。
\end{definition}

\begin{theorem}\label{theorem:复数Cauchy列充要条件}
\(\{ z_n \}\)是Cauchy列的充分必要条件是它的实部\(\{ x_n \}\)和虚部\(\{ y_n \}\)都是实的Cauchy列.
\end{theorem}
\begin{proof}
设\(z_n = x_n + \mathrm{i}y_n\),\(z_m = x_m + \mathrm{i}y_m\),那么从等式
\[
| z_n - z_m | = \sqrt{(x_n - x_m)^2 + (y_n - y_m)^2}
\]
知道,\(\{ z_n \}\)是Cauchy列的充分必要条件是它的实部\(\{ x_n \}\)和虚部\(\{ y_n \}\)都是实的Cauchy列.
\end{proof}

\begin{theorem}[复数域的Cauchy收敛准则]\label{theorem:复数域的Cauchy收敛准则}
\(\{ z_n \}\)收敛的充要条件是\(\{ z_n \}\)为Cauchy列.
\end{theorem}
\begin{note}
由此知道复数域\(\mathbf{C}\)是\textbf{完备的}.
\end{note}
\begin{proof}
由\refthe{theorem:复数收敛的充要条件}和\refthe{theorem:复数Cauchy列充要条件},再结合实数域中的Cauchy收敛准则立刻得到复数域的Cauchy收敛准则.
\end{proof}






\end{document}
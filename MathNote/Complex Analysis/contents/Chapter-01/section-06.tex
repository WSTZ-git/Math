\documentclass[../../main.tex]{subfiles}
\graphicspath{{\subfix{../../image/}}} % 指定图片目录,后续可以直接使用图片文件名。

% 例如:
% \begin{figure}[H]
% \centering
% \includegraphics[scale=0.4]{图.png}
% \caption{}
% \label{figure:图}
% \end{figure}
% 注意:上述\label{}一定要放在\caption{}之后,否则引用图片序号会只会显示??.

\begin{document}

\section{曲线和域}

\begin{definition}[连续曲线]
所谓\textbf{连续曲线},是指定义在闭区间\([a, b]\)上的一个复值连续函数\(\gamma: [a, b] \to \mathbf{C}\),写为
\[
z = \gamma(t) = x(t) + \mathrm{i}y(t), \ a \leqslant t \leqslant b,
\]
这里,\(x(t)\),\(y(t)\)都是\([a, b]\)上的连续函数。

如果用\(\gamma^*\)记\(\gamma\)的像点所成的集合:
\[
\gamma^* = \{ \gamma(t) : a \leqslant t \leqslant b \},
\]
那么\(\gamma^*\)是\(\mathbf{C}\)上的紧集。曲线\(\gamma\)的方向就是参数\(t\)增加的方向,在这个意义下,\(\gamma(a)\)和\(\gamma(b)\)分别称为\(\gamma\)的\textbf{起点}和\textbf{终点}。

如果\(\gamma(a) = \gamma(b)\),即起点和终点重合,就称\(\gamma\)为\textbf{闭曲线}。

如果曲线\(\gamma\)仅当\(t_1 = t_2\)时才有\(\gamma(t_1) = \gamma(t_2)\),就称\(\gamma\)为\textbf{简单曲线}或 \textbf{Jordan曲线}。

如果只有当\(t_1 = a\),\(t_2 = b\)时才有\(\gamma(t_1) = \gamma(t_2)\),就称\(\gamma\)为\textbf{简单闭曲线}或\textbf{Jordan闭曲线},或简称\textbf{围道}。
\end{definition}

\begin{definition}\label{definition:复平面上的光滑曲线}
设\(z = \gamma(t)\)(\(a \leqslant t \leqslant b\))是一条曲线。对区间\([a, b]\)作分割$T:$ \(a = t_0 < t_1 < \cdots < t_n = b\),得到以\(z_k = \gamma(t_k)\)(\(k = 0, 1, \cdots, n\))为顶点的折线\(P\),那么\(P\)的长度为
\[
|P| = \sum_{k = 1}^n | \gamma(t_k) - \gamma(t_{k - 1}) |.
\]
如果不论如何分割区间\([a, b]\),所得折线的长度都是有界的,就称曲线\(\gamma\)是\textbf{可求长的},\(\gamma\)的长度定义为\(|P|\)的上确界,即$$\underset{T}{\mathrm{sup}}\sum_{k=1}^n{|\gamma (t_k)}-\gamma (t_{k-1})|.$$

如果\(\gamma'(t) = x'(t) + \mathrm{i}y'(t)\)存在,且\(\gamma'(t) \neq 0\),那么\(\gamma\)在每一点都有切线,\(\gamma'(t)\)就是曲线\(\gamma\)在\(\gamma(t)\)处的切向量,它与正实轴的夹角为\(\mathrm{Arg}\gamma'(t)\)。如果\(\gamma'(t)\)是连续函数,那么\(\gamma\)的切线随\(t\)而连续变动,这时称\(\gamma\)为\textbf{光滑曲线}。在这种情况下,\(\gamma\)的长度为
\[
\int_a^b \sqrt{(x'(t))^2 + (y'(t))^2} \mathrm{d}t = \int_a^b | \gamma'(t) | \mathrm{d}t.
\]
曲线\(\gamma\)称为\textbf{逐段光滑的},如果存在\(t_0, t_1, \cdots, t_n\),使得\(a = t_0 < t_1 < \cdots < t_n = b\),\(\gamma\)在每个参数区间\([t_{j - 1}, t_j]\)上是光滑的,在每个分点\(t_1, \cdots, t_{n - 1}\)处\(\gamma\)的左右导数存在。
\end{definition}

\begin{definition}
平面点集\(E\)称为是\textbf{连通的},如果对任意两个不相交的非空集\(E_1\)和\(E_2\),满足
\[
E = E_1 \cup E_2,
\]
那么\(E_1\)必含有\(E_2\)的极限点,或者\(E_2\)必含有\(E_1\)的极限点。也就是说,\(E_1 \cap \bar{E_2}\)和\(\bar{E_1} \cap E_2\)至少有一个非空。
\end{definition}

\begin{proposition}\label{proposition:集合连通的条件}
\(\mathbf{C}\)中的开集\(E\)是连通的充分必要条件是\(E\)不能表示为两个不相交的非空开集的并。
\end{proposition}
\begin{proof}
设开集\(E\)是连通的,如果存在不相交的非空开集\(E_1\)和\(E_2\),使得\(E = E_1 \cup E_2\)。由于\(E_1\)中的点都是\(E_1\)的内点,\(E_2\)中的点都是\(E_2\)的内点,因此\(E_1\)中没有\(E_2\)的极限点,\(E_2\)中也没有\(E_1\)的极限点,这与\(E\)的连通性相矛盾。这就证明了条件的必要性。反之,如果开集\(E\)是不连通的,则必存在不相交的非空集\(E_1\)和\(E_2\),使得\(E = E_1 \cup E_2\),且\(E_1\)中无\(E_2\)的极限点,\(E_2\)中无\(E_1\)的极限点。由此可见,\(E_1\)和\(E_2\)均为开集。这就证明了条件的充分性。 
\end{proof}

\begin{theorem}\label{theorem:连通集合的性质}
平面上的非空开集\(E\)是连通的充分必要条件是:\(E\)中任意两点可用位于\(E\)中的折线连接起来。
\end{theorem}
\begin{proof}
先证必要性。设\(E\)是平面上一个非空的连通的开集,任取\(a \in E\),定义\(E\)的子集\(E_1\),\(E_2\)如下:
\[
E_1 = \{ z \in E : z \text{ 和 } a \text{ 可用位于 } E \text{ 中的折线连接} \},
\]
\[
E_2 = \{ z \in E : z \text{ 和 } a \text{ 不能用位于 } E \text{ 中的折线连接} \}.
\]
显然,\(E = E_1 \cup E_2\),而且\(E_1 \cap E_2 = \varnothing\)。现在证明\(E_1\)和\(E_2\)都是开集。任取\(z_0 \in E_1\),因\(E\)是开集,故必有\(z_0\)的邻域\(B(z_0, \delta) \subset E\)。这一邻域中的所有点当然可用一条线段与\(z_0\)相连,因而可用位于\(E\)中的折线与\(a\)相连,即\(B(z_0, \delta) \subset E_1\),所以\(E_1\)是开集。再任取\(z_0' \in E_2\),则必有\(z_0'\)的邻域\(B(z_0', \delta') \subset E\),如果此邻域中有一点能用一条折线与\(a\)点相连,那么\(z_0'\)能用线段与该点相连,因而\(z_0'\)能用折线与\(a\)点相连,这与\(z_0'\)的定义矛盾。因而\(B(z_0', \delta') \subset E_2\),即\(E_2\)也是开集。由\(E\)的连通性知道,\(E_1\),\(E_2\)中必有一个是空集。由于\(a \in E_1\),故\(E_2\)是空集。因而\(E\)中所有点都能用折线与\(a\)相连,而\(E\)中任意两点可以用经过\(a\)的折线相连,这就证明了必要性。

再证条件的充分性。如果存在两个不相交的非空开集\(E_1\),\(E_2\),使得\(E = E_1 \cup E_2\)。任取\(z_1 \in E_1\),\(z_2 \in E_2\),由假定,这两点可用\(E\)中的折线连接,因而折线中必有一条线段把\(E_1\)中的一点与\(E_2\)中的一点连接起来。不妨设这条线段连接的就是\(z_1\)和\(z_2\),该线段的参数表示为
\[
z = z_1 + t(z_2 - z_1),
\]
其中,\(t \in [0, 1]\)。今设
\[
T_1 = \{ t \in (0, 1) : z_1 + t(z_2 - z_1) \in E_1 \},
\]
\[
T_2 = \{ t \in (0, 1) : z_1 + t(z_2 - z_1) \in E_2 \}.
\]
则\(T_1\),\(T_2\)是非空的不相交的开集,而且\(T_1 \cup T_2 = (0, 1)\),这与区间的连通性相矛盾。 
\end{proof}

\begin{definition}
非空的连通开集称为\textbf{域}.
\end{definition}
\begin{note}
从\refthe{theorem:连通集合的性质}知道,域中任意两点必可用位于域中的折线连接起来.

从几何上来看,一个域就是平面上连成一片的开集.例如,单位圆的内部、上半平面、下半平面等都是域的例子.
\end{note}

\begin{theorem}[Jordan定理]\label{theorem:Jordan定理}
一条简单闭曲线 $\gamma$ 把复平面分成两个域,其中一个是有界的,称为 $\gamma$ 的内部;另一个是无界的,称为 $\gamma$ 的外部,而 $\gamma$ 是这两个域的共同的边界.
\end{theorem}
\begin{note}
单位圆盘 $\{ z: \vert z \vert < 1\}$ 和圆环 $\{ z: 1 < \vert z \vert < 2\}$ 都是域,但它们从函数论的角度来看有很大的差别,原因是前者是单连通的,而后者则不是.
\end{note}
\begin{proof}

\end{proof}


\begin{definition}
域 $D$ 称为是\textbf{单连通的},如果 $D$ 内任意简单闭曲线的内部仍在 $D$ 内. 不是单连通的域称为是\textbf{多连通的}.
\end{definition}

\begin{definition}
如果域 $D$ 是由 $n$ 条简单闭曲线围成的,就称 $D$ 是 $n$ \textbf{连通的},简单闭曲线中也可以有退化成一条简单曲线或一点的.
\end{definition}

\begin{example}
单位圆盘是单连通的,圆环 $\{ z: 1 < \vert z \vert < 2\}$ 是二连通的,除去圆心的单位圆盘也是二连通的,除去圆心和线段 $\left[ \frac{1}{2}, \frac{2}{3} \right]$ 的单位圆盘则是一个三连通域.
\end{example}






\end{document}
\documentclass[../../main.tex]{subfiles}
\graphicspath{{\subfix{../../image/}}} % 指定图片目录,后续可以直接使用图片文件名。

% 例如:
% \begin{figure}[H]
% \centering
% \includegraphics[scale=0.4]{图.png}
% \caption{}
% \label{figure:图}
% \end{figure}
% 注意:上述\label{}一定要放在\caption{}之后,否则引用图片序号会只会显示??.

\begin{document}

\section{开集、闭集和紧集}

\begin{definition}
设\(E\)是一平面点集,\(\mathbf{C}\)中的点对\(E\)而言可以分为三类:

(i) 如果存在\(r > 0\),使得\(B(a, r) \subset E\),就称\(a\)为\(E\)的\textbf{内点};

(ii) 如果存在\(r > 0\),使得\(B(a, r) \subset E^c\),就称\(a\)为\(E\)的外点,这里,\(E^c\)是由所有不属于\(E\)的点构成的集,称为\(E\)的\textbf{余集}或\textbf{补集};

(iii) 如果对任意\(r > 0\),\(B(a, r)\)中既有\(E\)的点,也有\(E^c\)的点,就称\(a\)为\(E\)的\textbf{边界点}。
\end{definition}

\begin{definition}
\(E\)的内点的全体称为\(E\)的\textbf{内部},记为\(E^\circ\);

\(E\)的外点的全体称为\(E\)的\textbf{外部},它就是\(E\)的余集\(E^c\)的内部,即\((E^c)^\circ\);

\(E\)的边界点的全体称为\(E\)的\textbf{边界},记为\(\partial E\)。
\end{definition}
\begin{note}
由上面的定义可知,集\(E\)把复平面分成三个互不相交的部分:\(\mathbf{C} = E^\circ \cup (E^c)^\circ \cup \partial E\),即
\begin{align}\label{equation:::2318}
(\partial E)^c = E^\circ \cup (E^c)^\circ. 
\end{align}
\end{note}

\begin{example}[邻域的内部和边界]
\(B(a, r)\)中的所有点都是它的内点,即\(B(a, r) = (B(a, r))^\circ\),\(B(a, r)\)的边界\(\partial B = \{ z : | z - a | = r \}\),即是圆周,满足条件\(| z - a | > r\)的点\(z\)都是\(B(a, r)\)的外点。
\end{example}

\begin{definition}
如果\(E\)的所有点都是它的内点,即\(E = E^\circ\),就称\(E\)为\textbf{开集}。

如果\(E^c\)是开集,就称\(E\)为\textbf{闭集}。
\end{definition}

\begin{example}
\(B(a, r)\)是开集,闭圆盘\(\{ z : | z - a | \leqslant slant r \}\)是闭集,\(B(a, r)\)和它的上半圆周的并集既不是开集也不是闭集。
\end{example}

\begin{definition}
点\(a\)称为集\(E\)的\textbf{极限点}或\textbf{聚点},如果对任意\(r > 0\),\(B(a, r)\)中除\(a\)外总有\(E\)中的点。

集\(E\)的所有极限点构成的集称为\(E\)的\textbf{导集},记为\(E'\)。

\(E\)中不属于\(E'\)的点称为\(E\)的\textbf{孤立点}。

\(E\)和它的导集\(E'\)的并称为\(E\)的\textbf{闭包},记为\(\bar{E}\),即\(\bar{E} = E \cup E'\)。
\end{definition}

\begin{proposition}\label{proposition:闭包的性质}
对于任意集\(E\),有

(i) \(a \in \bar{E}\)的充要条件是对任意\(\varepsilon > 0\),有
\begin{align}
B(a, r) \cap E \neq \varnothing,\forall r>0.\label{equation-----21341}
\end{align}
这里,\(\varnothing\)表示空集;

(ii) \((\bar{E})^c = (E^c)^\circ\),\(\overline{E^c} = (E^\circ)^c\)。
\end{proposition}
\begin{proof}
(i) 若\(a \in \bar{E}\),则\(a \in E\)或\(a \in E'\),不论何者发生,总有\(B(a, r) \cap E \neq \varnothing\)。反之,若等式\eqref{equation-----21341}成立,这说明\(a\)或是\(E\)的极限点,或是\(E\)的孤立点,因而\(a \in \bar{E}\)。

(ii) 由(i)知,\(a \in (\bar{E})^c\)当且仅当存在\(\varepsilon > 0\),使得\(B(a, r) \cap E = \varnothing\),这说明\(a\)是\(E^c\)的内点,即\(a \in (E^c)^\circ\),因而\((\bar{E})^c = (E^c)^\circ\)。再看第二个等式,\(a \in (E^\circ)^c\)意味着\(a\)不是\(E\)的内点,即\(a\)是\(E\)的外点或边界点,因而对任意\(\varepsilon > 0\),总有\(B(a, r) \cap E^c \neq \varnothing\),由(i)知\(a \in \overline{E^c}\)。因而\(\overline{E^c} = (E^\circ)^c\)。 
\end{proof}

\begin{proposition}\label{proposition:开集,闭集的充要条件}
(i) \(E^\circ\)是开集,\(\partial E\)和\(\bar{E}\)是闭集;

(ii) \(E\)是闭集的充要条件是\(E = \bar{E}\);

(iii) \(E\)是闭集的充要条件是\(E' \subset E\)。
\end{proposition}
\begin{proof}
(i) 任取\(a \in E^\circ\),则由定义知道,存在\(\varepsilon > 0\),使得\(B(a, \varepsilon) \subset E\)。显然,\(B(a, \varepsilon)\)中的每一点都是\(E\)的内点,因而\(B(a, \varepsilon) \subset E^\circ\),即\(a\)是\(E^\circ\)的内点。由于\(a\)是任意取的,所以\(E^\circ\)是开集。由刚才所证,\(E^\circ\)和\((E^c)^\circ\)都是开集,两个开集的并当然也是开集,由等式\eqref{equation:::2318}知\((\partial E)^c\)是开集,因而\(\partial E\)是闭集。由于\((E^c)^\circ\)是开集,由\nrefpro{proposition:闭包的性质}{(ii)}知,\((\bar{E})^c\)是开集,所以\(\bar{E}\)是闭集。

(ii) 如果\(E = \bar{E}\),则由(i)知\(\bar{E}\)是闭集,所以\(E\)是闭集。反之,如果\(E\)是闭集,那么\(E^c\)是开集,因而\(E^c = (E^c)^\circ\)。另外,由\nrefpro{proposition:闭包的性质}{(ii)}得\((\bar{E})^c = (E^c)^\circ\),因而\(E^c = (\bar{E})^c\),即\(E = \bar{E}\)。

(iii) 从(ii)立刻可得。 
\end{proof}

\begin{definition}
点集\(E\)的\textbf{直径}定义为\(E\)中任意两点间距离的上确界,记为\(\mathrm{diam}E\),即
\[
\mathrm{diam}E = \sup\{ | z_1 - z_2 | : z_1, z_2 \in E \}.
\]
\end{definition}

\begin{theorem}[Cantor闭集套定理]\label{theorem:Cantor闭集套定理}
若非空闭集序列\(\{ F_n \}\)满足

(i) \(F_1 \supset F_2 \supset \cdots \supset F_n \supset \cdots\);

(ii) \(\mathrm{diam}F_n \to 0\)(当\(n \to \infty\)时),

那么\(\bigcap_{n = 1}^\infty F_n\)是一个独点集.
\end{theorem}
\begin{note}
这个定理是实数域中的区间套定理在复数域中的推广.
\end{note}
\begin{proof}
在每一个\(F_n\)中任取一点\(z_n\),我们证明\(\{ z_n \}\)是一个Cauchy点列。由于\(\lim\limits_{n \to \infty}\mathrm{diam}F_n = 0\),所以对任意\(\varepsilon > 0\),可取充分大的\(N\),使得\(\mathrm{diam}F_N < \varepsilon\)。今取\(m, n > N\),由条件(i),\(z_m, z_n \in F_N\),所以\(| z_n - z_m | \leqslant slant \mathrm{diam}F_N < \varepsilon\)。因而\(\{ z_n \}\)是一Cauchy序列,设其收敛于\(z_0\)。我们证明\(z_0 \in \bigcap_{n = 1}^\infty F_n\)。事实上,任取\(F_k\),则当\(n > k\)时,\(z_n\)便全部落入\(F_k\)中,因为\(F_k\)是闭的,由\nrefpro{proposition:开集,闭集的充要条件}{(iii)},\(\{ z_n \}\)的极限\(z_0 \in F_k\),所以\(z_0 \in \bigcap_{n = 1}^\infty F_n\)。如果还有另一点\(z_1\)也属于\(\bigcap_{n = 1}^\infty F_n\),那么必有\(| z_0 - z_1 | \leqslant slant \mathrm{diam}F_n \to 0\)(\(n \to \infty\)),因而\(z_1 = z_0\)。 
\end{proof}

\begin{definition}
设\(E\)是一个集,\(\mathscr{F} = \{ G \}\)是一个\textbf{开集族},即\(\mathscr{F}\)中的每一个元素都是开集。

如果\(E\)中每一点至少属于\(\mathscr{F}\)中的一个开集,就说\(\mathscr{F}\)是\(E\)的一个\textbf{开覆盖}。
\end{definition}

\begin{example}
\(E\)是任一点集,\(\varepsilon\)是一个给定的正数,那么
\[
\mathscr{F} = \{ B(a, \varepsilon) : a \in E \}
\]
便是\(E\)的一个开覆盖.
\end{example}

\begin{definition}
我们说点集\(E\)具有\textbf{有限覆盖性质},是指从\(E\)的任一个开覆盖中必能选出有限个开集\(G_1, \cdots, G_n\),使得这有限个开集的并就能覆盖\(E\),即
\[
E \subset \bigcup_{j = 1}^n G_j.
\]
具有有限覆盖性质的集称为\textbf{紧集}。
\end{definition}

\begin{example}
空集和有限集都是紧集,但单位圆盘\(B(0, 1) = \{ z \in \mathbf{C} : | z | < 1 \}\)却不是紧集,因为\(G_n = \left\{ z : | z | < 1 - \frac{1}{n} \right\}\),\(n = 2, 3, \cdots\),这一串同心圆构成\(B(0, 1)\)的一个开覆盖,但从中找不出有限个集覆盖\(B(0, 1)\)。
\end{example}

\begin{definition}
集\(E\)称为是\textbf{有界的},如果存在\(R > 0\),使得\(E \subset B(0, R)\)。
\end{definition}

\begin{theorem}[Heine-Borel定理]\label{theorem:Heine-Borel定理}
在\(\mathbf{C}\)中,\(E\)是紧集的充要条件为\(E\)是有界闭集;在\(\mathbf{C}_\infty\)中,\(E\)是紧集的充要条件为\(E\)是闭集。
\end{theorem}
\begin{proof}
我们先证明,如果\(E\)是\(\mathbf{C}_\infty\)中的闭集或\(\mathbf{C}\)中的有界闭集,那么\(E\)是紧集,即从\(E\)的任一开覆盖\(\mathscr{F}\)中,可以选出有限个开集覆盖\(E\)。先设\(E\)是\(\mathbf{C}_\infty\)中的闭集,如果\(z = \infty \notin E\),则因\(E\)是闭集,有\(E = \bar{E}\),即\(\infty \notin \bar{E}\),由\nrefpro{proposition:闭包的性质}{(i)},存在\(R > 0\),使得\(B(\infty, R) \cap E = \varnothing\),即\(E \subset \overline{B(0, R)}\),因而\(E\)是有界闭集。如果\(z = \infty \in E\),由开覆盖的定义,\(\infty\)属于\(\mathscr{F}\)中的某一个开集,而\(E\)在这个开集之外的部分是一有界闭集,只要再证明这个有界闭集的部分被有限个开集覆盖即可.总之,不论何种情况发生,只要考虑\(E\)是有界闭集的情形就够了。

现设\(E\)是有界闭集,如果它不是紧集,那么从\(E\)的开覆盖\(\mathscr{F}\)中不能取出有限个开集来覆盖\(E\)。因为\(E\)是有界的,它一定包含在一个充分大的闭正方形\(Q\)中:
\[
Q = \{ (x, y) : | x | \leqslant slant M, | y | \leqslant slant M \}.
\]
把这个正方形分成相等的四个小正方形,则其中必有一个小正方形\(Q_1\),使得\(Q_1 \cap E\)是有界闭集且不具有有限覆盖性质。再把\(Q_1\)分成四个相等的小正方形,其中必有一个小正方形\(Q_2\)具有上述同样的性质。这个过程可以无限地进行下去,得到一列闭正方形\(\{ Q_n \}\)。如果记\(F_n = Q_n \cap E\),那么\(F_n\)满足下列条件:

(i) \(F_n\)是有界闭集;

(ii) \(F_n \supset F_{n + 1}\),\(n = 1, 2, \cdots\);

(iii) 不能从\(\mathscr{F}\)中取出有限个开集来覆盖\(F_n\);

(iv) 当\(n \to \infty\)时,\(\mathrm{diam}F_n \leqslant slant \frac{M}{2^n} \sqrt{2} \to 0\)。

由(i),(ii),(iv)知道\(\{ F_n \}\)满足 \hyperref[theorem:Cantor闭集套定理]{Cantor闭集套定理}的条件,因而存在复数\(z_0\),使得\(\bigcap_{n = 1}^\infty F_n = \{ z_0 \}\)。由于\(z_0 \in F_n \subset E\),故在\(\mathscr{F}\)中必有一个开集\(G_0\),使得\(z_0 \in G_0\)。由于\(z_0\)是\(G_0\)的内点,故有\(z_0\)的邻域\(B(z_0, \varepsilon) \subset G_0\)。由于\(\mathrm{diam}F_n \to 0\),故当\(n\)充分大时\(F_n \subset B(z_0, \varepsilon) \subset G_0\),这就是说\(G_0\)覆盖了\(F_n\),这与(iii)矛盾。因而\(E\)是紧集。

现在证明必要性。只要对扩充平面的情形来证明就够了,因为如果一个集对扩充平面是闭的,它又不包含无穷远点,那么它必然是有界的。设\(E\)是一个紧集,我们要证明它是闭集,只要证明\(E^c\)是开集即可。为此,任取\(a \in E^c\),只要证明\(a\)是\(E^c\)的内点就行了。取这样的开集族\(\mathscr{F}\):凡是闭包不包含\(a\)点的开集都属于\(\mathscr{F}\)。因为\(a \in E^c\),因此对\(E\)中每一点\(z\),都能找到它的邻域\(B(z, \varepsilon)\),使得\(a \notin \overline{B(z, \varepsilon)}\),所以\(B(z, \varepsilon) \in \mathscr{F}\)。这就是说,\(\mathscr{F}\)是\(E\)的一个开覆盖。由于\(E\)是紧集,故能从\(\mathscr{F}\)中取出有限个开集\(G_1, \cdots, G_n\),使得\(E \subset \bigcup_{j = 1}^n G_j\)。但\(a \notin \overline{G_j}\),\(j = 1, \cdots, n\),所以\(a \in \bigcap_{j = 1}^n (\overline{G_j})^c\)。显然,\(\bigcap_{j = 1}^n (\overline{G_j})^c\)是一个开集,于是由开集和内点的定义可知,存在$r>0,$使得$B(a,r)\subset \bigcap_{j = 1}^n (\overline{G_j})^c$.而且从\nrefpro{proposition:闭包的性质}{(ii)}得
\[
B(a,r)\subset \bigcap_{j = 1}^n (\overline{G_j})^c = \bigcap_{j = 1}^n (G_j^c)^\circ \subset \bigcap_{j = 1}^n G_j^c = \left( \bigcup_{j = 1}^n G_j \right)^c \subset E^c,
\]
这就证明了\(a\)是\(E^c\)的内点,即\(E^c\)是开集。
\end{proof}

\begin{definition}
设\(E\),\(F\)是任意两个集,\(E\),\(F\)间的距离定义为
\[
d(E, F) = \inf\{ | z_1 - z_2 | : z_1 \in E, z_2 \in F \}.
\]
如果\(E = \{ a \}\)是由一个点所构成的集,那么\(a\)和\(F\)间的距离为
\[
d(a, F) = \inf\{ | a - z | : z \in F \}.
\]
\end{definition}

\begin{proposition}
\begin{enumerate}[(1)]
\item 如果\(F\)是闭集,\(a \notin F\),那么\(d(a, F) > 0\)。

\item 如果\(E\)是有限点集,且\(E \cap F = \varnothing\),当然也有\(d(E, F) > 0\)。
\end{enumerate}
\end{proposition}
\begin{proof}
\begin{enumerate}[(1)]
\item 此时必有\(\varepsilon > 0\),使得\(B(a, \varepsilon) \cap F = \varnothing\),因而\(d(a, F) \geqslant slant \varepsilon > 0\)。

\item 
\end{enumerate}
\end{proof}

\begin{theorem}\label{theorem:紧集和闭集无交则距离大于0}
设\(E\)是紧集,\(F\)是闭集,且\(E \cap F = \varnothing\),则$d(E, F) > 0.$
\end{theorem}
\begin{remark}
若\(E\)是无穷闭集,\(F\)也是闭集,但$E$不是紧集,且\(E \cap F = \varnothing\),这时\(d(E, F) > 0\)未必成立。

例如,\(E\)是整个实轴,\(F = \{ z = x + \mathrm{i}\mathrm{e}^x : -\infty < x < \infty \}\),则\(E\)和\(F\)都是\(\mathbf{C}\)中的闭集,而且\(E \cap F = \varnothing\),但\(d(E, F) = 0\)。
\end{remark}
\begin{note}
从这个定理可以看出,紧集之所以重要,在于它保留了大部分有限集的性质.
\end{note}
\begin{proof}
任取\(a \in E\),则\(a \notin F\),所以\(d(a, F) > 0\)。今以\(a\)为中心、\(\frac{1}{2}d(a, F)\)为半径作一圆盘,当\(a\)跑遍集\(E\)时,这些圆盘所组成的开集族就是\(E\)的一个开覆盖。因为\(E\)是紧的,故从这个开覆盖中能选出有限个开集\(G_1, \cdots, G_n\)来覆盖\(E\),其中,\(G_j = B\left( a_j, \frac{1}{2}d(a_j, F) \right)\),\(j = 1, \cdots, n\)。记
\[
\delta = \min\left\{ \frac{1}{2}d(a_1, F), \cdots, \frac{1}{2}d(a_n, F) \right\}.
\]
今任取\(z_1 \in E\),则必有某个\(G_j\),使得\(z_1 \in G_j\),因而
\[
| z_1 - a_j | < \frac{1}{2}d(a_j, F).
\]
任取\(z_2 \in F\),当然\(| z_2 - a_j | \geqslant slant d(a_j, F)\),于是
\[
\begin{split}
| z_1 - z_2 | \geqslant slant | z_2 - a_j | - | z_1 - a_j | \geqslant slant d(a_j, F) - \frac{1}{2}d(a_j, F) = \frac{1}{2}d(a_j, F) \geqslant slant \delta.
\end{split}
\]
所以
\[
\begin{split}
d(E, F) = \inf\{ | z_1 - z_2 | : z_1 \in E, z_2 \in F \}  \geqslant slant \delta > 0.
\end{split}
\] 
\end{proof}

\begin{theorem}[Bolzano-Weierstrass定理]\label{theorem:Bolzano-Weierstrass定理}
任一无穷点集至少有一个极限点。
\end{theorem}
\begin{remark}
这个定理也可以用证明\hyperref[theorem:Cantor闭集套定理]{Cantor闭集套定理}的方法给出另一个证明。
\end{remark}
\begin{proof}
设\(E\)是一个无穷点集,如果\(E\)是无界集,那么无穷远点便是它的极限点。今设\(E\)是有界集,如果它没有极限点,那么它是一个闭集。任取\(z \in E\),由于它不是\(E\)的极限点,故必存在\(\varepsilon > 0\),使得\(B(z, \varepsilon)\)中除\(z\)外不再有\(E\)中的点。$z$取遍整个$E$,由这种\(B(z, \varepsilon)\)构成的开集族便是\(E\)的一个开覆盖,由\hyperref[theorem:Heine-Borel定理]{Heine-Borel 定理},能从中选出有限个来覆盖\(E\)。因为每个开集只包含\(E\)的一个点,这说明\(E\)是一个有限集,与\(E\)是无穷点集的假定矛盾,因而\(E\)必有极限点。
\end{proof}













\end{document}
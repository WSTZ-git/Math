\documentclass[../../main.tex]{subfiles}% 注意这里的文件路径不能用 ./main.tex ,否则用latexmk编译子文件会报错
\graphicspath{{\subfix{./image/}}} % 指定图片目录,后续可以直接使用图片文件名
% 注意这里的文件路径不能用 ../../image/ ,否则用latexmk编译子文件会报错

% 例如:
% \begin{figure}[H]
% \centering
% \includegraphics[scale=0.3]{图.png}
% \caption{}
% \label{figure:图}
% \end{figure}
% 注意:上述\label{}一定要放在\caption{}之后,否则引用图片序号会只会显示??.

\begin{document}

\section{扩充平面和复数的球面表示}

\begin{definition}
为了今后讨论的需要,我们要在\(\mathbb{C}\)中引进一个新的数\(\infty\),这个数的模是\(\infty\),辐角没有意义,它和其他数的运算规则规定为:
\[
z \pm \infty = \infty, \quad z \cdot \infty = \infty \ (z \neq 0),
\]
\[
\frac{z}{\infty} = 0, \quad \frac{z}{0} = \infty \ (z \neq 0);
\]
\(0 \cdot \infty\)和\(\infty \pm \infty\)都不规定其意义。引进了\(\infty\)的复数系记为\(\mathbb{C}_\infty\),即\(\mathbb{C}_\infty = \mathbb{C} \cup \{ \infty \}\)。

在复平面上,没有一个点和\(\infty\)相对应,但我们想像有一个\textbf{无穷远点}和\(\infty\)对应,加上无穷远点的复平面称为\textbf{扩充平面}或\textbf{闭平面},不包括无穷远点的复平面也称为\textbf{开平面}。
\end{definition}
\begin{remark}
在复平面上,无穷远点和普通的点是不一样的,Riemann 首先引进了复数的球面表示,在这种表示中,\(\infty\)和普通的复数没有什么区别。
\end{remark}

\begin{theorem}\label{theorem:扩充复平面与单位球面对等}
证明:扩充平面和单位球面对等,即两者之间存在一个双射.并且我们可以具体地写出扩充平面到单位球面的双射
\begin{gather*}
f:C_{\infty}\longrightarrow \mathbb{R} ^3,
z\longmapsto \left( \frac{z+\bar{z}}{1+|z|^2},\frac{z-\bar{z}}{\mathrm{i}\left( 1+|z|^2 \right)},\frac{|z|^2-1}{1+|z|^2} \right) =\left( \frac{\mathrm{Re}z}{1+|z|^2},\frac{\mathrm{Im}z}{1+|z|^2},\frac{|z|^2-1}{1+|z|^2} \right) .
\\
f^{-1}:\mathbb{R} ^3\longrightarrow C_{\infty},
\left( x_1,x_2,x_3 \right) \longmapsto \frac{x_1+\mathrm{i}x_2}{1-x_3}.
\end{gather*}
这个双射通常称为\textbf{球极投影}.这个单位球面通常称为\textbf{黎曼球面}或\textbf{复数的球面表示}.复数$z$在黎曼球面上的像$f(z)$通常称为\textbf{球面像}.
\end{theorem}
\begin{proof}
设\(S\)是\(\mathbb{R}^3\)中的单位球面,即
\[
S = \{ (x_1, x_2, x_3) \in \mathbb{R}^3 : x_1^2 + x_2^2 + x_3^2 = 1 \}.
\]
把\(\mathbb{C}\)等同于平面:
\[
\mathbb{C} = \{ (x_1, x_2, 0) : x_1, x_2 \in \mathbb{R} \}.
\]
固定\(S\)的北极\(N\),即\(N = (0, 0, 1)\),对于\(\mathbb{C}\)上的任意点\(z\),联结\(N\)和\(z\)的直线必和\(S\)交于一点\(P\)(\reffig{figure:图1.5})。若\(|z| > 1\),则\(P\)在北半球上;若\(|z| < 1\),则\(P\)在南半球上;若\(|z| = 1\),则\(P\)就是\(z\)。容易看出,当\(z\)趋向\(\infty\)时,球面上对应的点\(P\)趋向于北极\(N\),自然地,我们就把\(\mathbb{C}_\infty\)中的\(\infty\)对应于北极\(N\)。这样一来,\(\mathbb{C}_\infty\)中的所有点(包括无穷远点在内)都被移植到球面上去了,这样我们就找到了一个扩充平面到单位球面的双射.而在球面上,\(N\)和其他的点是一视同仁的。
\begin{figure}[H]
\centering
\includegraphics[scale=0.2]{图1.5.png}
\caption{}
\label{figure:图1.5}
\end{figure}
现在给出这种对应的具体表达式。设\(z = x + \mathrm{i}y\),容易算出\(zN\)和球面\(S\)的交点的坐标为
\[
x_1 = \frac{2x}{x^2 + y^2 + 1}, \ x_2 = \frac{2y}{x^2 + y^2 + 1}, \ x_3 = \frac{x^2 + y^2 - 1}{x^2 + y^2 + 1}.
\]
直接用复数\(z\),可表示为
\[
x_1 = \frac{z + \bar{z}}{1 + |z|^2}, \ x_2 = \frac{z - \bar{z}}{\mathrm{i}(1 + |z|^2)}, \ x_3 = \frac{|z|^2 - 1}{|z|^2 + 1}.
\]
这样,从\(z\)便可算出它在球面上对应点的坐标。反过来,从球面上的点\((x_1, x_2, x_3)\)也可算出它在平面上的对应点\(z\)。事实上,从上面的表达式得
\[
\begin{cases}
x_1 + \mathrm{i}x_2 = \frac{2z}{1 + |z|^2}, \\
1 - x_3 = \frac{2}{1 + |z|^2},
\end{cases}
\]
由此即得
\[
z = \frac{x_1 + \mathrm{i}x_2}{1 - x_3}.
\]
这就是所需的计算公式。现在我们可以具体地写出扩充平面到单位球面的双射
\begin{gather*}
f:C_{\infty}\longrightarrow \mathbb{R} ^3,
z\longmapsto \left( \frac{z+\bar{z}}{1+|z|^2}, \frac{z-\bar{z}}{\mathrm{i}(1+|z|^2)}, \frac{|z|^2-1}{|z|^2+1} \right) .
\\
f^{-1}:\mathbb{R} ^3\longrightarrow C_{\infty},
\left( x_1,x_2,x_3 \right) \longmapsto \frac{x_1+\mathrm{i}x_2}{1-x_3}.
\end{gather*}

\end{proof}

\begin{proposition}\label{proposition:扩充复球面的一些基本性质}
\begin{enumerate}[(1)]
\item\label{proposition:扩充复球面的一些基本性质-1} 在复数的球面表示下, $z$ 和 $\frac{1}{\overline{z}}$ 的球面像关于复平面对称.

\item\label{proposition:扩充复球面的一些基本性质-2} 在复数的球面表示下, $z$ 和 $w$ 的球面像是直径对点当且仅当 $z\overline{w}=-1$.

\item\label{proposition:扩充复球面的一些基本性质-3} 在复数的球面表示下, $\mathbb{C}_\infty$ 中的点 $z$ 和 $w$ 的球面像间的距离为
\begin{align*}
\frac{2|z-w|}{\sqrt{(|z|^2+1)(|w|^2+1)}}.
\end{align*}

\item\label{proposition:扩充复球面的一些基本性质-4} 在复数的球面表示下, 若 $\begin{pmatrix} a & b \\ c & d \end{pmatrix}$ 是二阶酉方阵, 则 $\mathbb{C}_\infty$ 的变换 $w=\dfrac{az+b}{cz+d}$ 诱导了球面绕球心的一个旋转.

\item\label{proposition:扩充复球面的一些基本性质-5} 在复数的球面表示下, 球面上的圆周对应于复平面上的圆周或直线, 反之亦然.

\item\label{proposition:扩充复球面的一些基本性质-6} 在复数的球面表示下, 复平面上两条光滑曲线在交点处的夹角与它们的球面像在交点处的夹角相等.
\end{enumerate}
\end{proposition}
\begin{proof}
\begin{enumerate}[(1)]
\item 由\refthe{theorem:扩充复平面与单位球面对等}知复数$z$对应的球面像坐标为
\begin{align*}
\left( \frac{z+\bar{z}}{1+|z|^2}, \frac{z-\bar{z}}{\mathrm{i}(1+|z|^2)}, \frac{|z|^2-1}{|z|^2+1} \right);
\end{align*}
复数$\frac{1}{\overline{z}}$对应的球面像坐标为
\begin{align*}
\left( \frac{\frac{1}{\overline{z}}+\frac{1}{z}}{1+\frac{1}{\left| z \right|^2}},\frac{\frac{1}{\overline{z}}-\frac{1}{z}}{\mathrm{i(}1+\frac{1}{\left| z \right|^2})},\frac{\frac{1}{\left| z \right|^2}-1}{\frac{1}{\left| z \right|^2}+1} \right) =\left( \frac{z+\overline{z}}{1+\left| z \right|^2},\frac{z-\overline{z}}{\mathrm{i(}1+\left| z \right|^2)},-\frac{\left| z \right|^2-1}{\left| z \right|^2+1} \right) .
\end{align*}
故$z$ 和 $\frac{1}{\overline{z}}$的球面像关于复平面对称.

\item {\heiti 充分性:}若$z\overline{w}=-1$,则$w=-\dfrac{1}{\overline{z}}$.由\refthe{theorem:扩充复平面与单位球面对等}知复数$z$对应的球面像坐标为
\begin{align*}
\left( \frac{z+\overline{z}}{1+|z|^2},\frac{z-\overline{z}}{\mathrm{i}\left( 1+|z|^2 \right)},\frac{|z|^2-1}{|z|^2+1} \right) ;
\end{align*}
复数$w=-\dfrac{1}{\overline{z}}$对应的球面像坐标为
\begin{align*}
\left( \frac{-\frac{1}{\overline{z}}-\frac{1}{z}}{1+\frac{1}{\left| z \right|^2}},\frac{-\frac{1}{\overline{z}}+\frac{1}{z}}{\mathrm{i}\left( 1+\frac{1}{\left| z \right|^2} \right)},\frac{\frac{1}{\left| z \right|^2}-1}{\frac{1}{\left| z \right|^2}+1} \right) =\left( -\frac{z+\overline{z}}{1+\left| z \right|^2},-\frac{z-\overline{z}}{\mathrm{i}(1+\left| z \right|^2)},-\frac{\left| z \right|^2-1}{\left| z \right|^2+1} \right) .
\end{align*}
显然$z$和$w=-\dfrac{1}{\overline{z}}$对应的球面像坐标关于原点中心对称,故$z$和$w$的球面像是直径对点.

{\heiti 必要性:}若$z$和$w$的球面像是直径对点,则$z$和$w$的球面像坐标关于原点中心对称.
设$z$和$w$的球面像分别为$\left( x_1,x_2,x_3 \right)$和$\left( x_{1}^{\prime},x_{2}^{\prime},x_{3}^{\prime} \right)$,则
\begin{align*}
x_{1}^{\prime}=-x_1,\quad x_{2}^{\prime}=-x_2,\quad x_{3}^{\prime}=-x_3;
\end{align*}
\begin{align*}
x_{1}^{2}+x_{2}^{2}+x_{3}^{2}=1.
\end{align*}
于是由\refthe{theorem:扩充复平面与单位球面对等}知
\begin{align*}
z=\frac{x_1+\mathrm{i}x_2}{1-x_3},\quad w=\frac{x_{1}^{\prime}+\mathrm{i}x_{2}^{\prime}}{1-x_{3}^{\prime}}=-\frac{x_1+\mathrm{i}x_2}{1+x_3}.
\end{align*}
故
\begin{align*}
z\overline{w}=\frac{x_1+\mathrm{i}x_2}{1-x_3}\cdot \left( -\frac{x_1-\mathrm{i}x_2}{1+x_3} \right) =-\frac{x_{1}^{2}+x_{2}^{2}}{1-x_{3}^{2}}=-\frac{x_{1}^{2}+x_{2}^{2}}{x_{1}^{2}+x_{2}^{2}}=-1.
\end{align*}

\item 由\refthe{theorem:扩充复平面与单位球面对等}知复数$z$和$w$的球面像坐标分别为
\begin{align*}
\left( \frac{z+\bar{z}}{1+|z|^2},\frac{z-\bar{z}}{\mathrm{i}\left( 1+|z|^2 \right)},\frac{|z|^2-1}{|z|^2+1} \right) ,\quad \left( \frac{w+\overline{w}}{1+|w|^2},\frac{w-\overline{w}}{\mathrm{i}\left( 1+|w|^2 \right)},\frac{|w|^2-1}{|w|^2+1} \right) .
\end{align*}
从而这两个球面像之间的距离为
\begin{scriptsize}
\begin{align*}
&\,\,\quad \sqrt{\left( \frac{z+\bar{z}}{1+|z|^2}-\frac{w+\overline{w}}{1+|w|^2} \right) ^2+\left( \frac{z-\bar{z}}{\mathrm{i}\left( 1+|z|^2 \right)}-\frac{w-\overline{w}}{\mathrm{i}\left( 1+|w|^2 \right)} \right) ^2+\left( \frac{|z|^2-1}{|z|^2+1}-\frac{|w|^2-1}{|w|^2+1} \right) ^2}
\\
&=\sqrt{\frac{\left[ \left( z+\bar{z} \right) \left( 1+|w|^2 \right) -\left( w+\overline{w} \right) \left( 1+|z|^2 \right) \right] ^2}{\left( 1+|z|^2 \right) ^2\left( 1+|w|^2 \right) ^2}-\frac{\left[ \left( z-\bar{z} \right) \left( 1+|w|^2 \right) -\left( w-\overline{w} \right) \left( 1+|z|^2 \right) \right] ^2}{\left( 1+|z|^2 \right) ^2\left( 1+|w|^2 \right) ^2}+\frac{\left[ \left( |z|^2-1 \right) \left( 1+|w|^2 \right) -\left( |w|^2-1 \right) \left( 1+|z|^2 \right) \right] ^2}{\left( 1+|z|^2 \right) ^2\left( 1+|w|^2 \right) ^2}}
\\
&=\sqrt{\frac{\left[ 2z\left( 1+|w|^2 \right) -2w\left( 1+|z|^2 \right) \right] \left[ 2\bar{z}\left( 1+|w|^2 \right) -2\overline{w}\left( 1+|z|^2 \right) \right] +\left[ \left( |z|^2-1 \right) \left( 1+|w|^2 \right) -\left( |w|^2-1 \right) \left( 1+|z|^2 \right) \right] ^2}{\left( 1+|z|^2 \right) ^2\left( 1+|w|^2 \right) ^2}}
\\
&=\sqrt{\frac{\left( 4\left| z \right|^2+\left( |z|^2-1 \right) ^2 \right) \left( 1+|w|^2 \right) ^2+\left( 4\left| w \right|^2+\left( |w|^2-1 \right) ^2 \right) \left( 1+|z|^2 \right) ^2-2\left( 2\overline{z}w+2z\overline{w}+\left( |z|^2-1 \right) \left( |w|^2-1 \right) \right) \left( 1+|z|^2 \right) \left( 1+|w|^2 \right)}{\left( 1+|z|^2 \right) ^2\left( 1+|w|^2 \right) ^2}}
\\
&=\sqrt{\frac{2\left( 1+|z|^2 \right) ^2\left( 1+|w|^2 \right) ^2-2\left( \left| zw \right|^2+1+\overline{z}w+z\overline{w}-\left| z-w \right|^2 \right) \left( 1+|z|^2 \right) \left( 1+|w|^2 \right)}{\left( 1+|z|^2 \right) ^2\left( 1+|w|^2 \right) ^2}}
\\
&=\sqrt{\frac{2\left( \left( 1+|z|^2 \right) \left( 1+|w|^2 \right) -\left| zw \right|^2-1-\overline{z}w-z\overline{w}+\left| z-w \right|^2 \right) \left( 1+|z|^2 \right) \left( 1+|w|^2 \right)}{\left( 1+|z|^2 \right) ^2\left( 1+|w|^2 \right) ^2}}
\\
&=\sqrt{\frac{2\left( \left| z \right|^2+\left| w \right|^2-\overline{z}w-z\overline{w}+\left| z-w \right|^2 \right) \left( 1+|z|^2 \right) \left( 1+|w|^2 \right)}{\left( 1+|z|^2 \right) ^2\left( 1+|w|^2 \right) ^2}}
\\
&=\sqrt{\frac{4\left| z-w \right|^2\left( 1+|z|^2 \right) \left( 1+|w|^2 \right)}{\left( 1+|z|^2 \right) ^2\left( 1+|w|^2 \right) ^2}}=\frac{2\left| z-w \right|}{\sqrt{\left( 1+|z|^2 \right) \left( 1+|w|^2 \right)}}.
\end{align*}
\end{scriptsize}

\item 只需证明这个变换是正交(保距)变换且保定向即可.任取$z,z'\in \mathbb{C}$,记$w'=\frac{az'+b}{cz'+d}$,则由条件可得
\begin{align}
\left| \det \left( \begin{matrix}
a& b\\
c& d\\
\end{matrix} \right) \right|=\left| ad-bc \right|=1.\label{eq:21.1}
\end{align}
从而
\begin{align*}
\left| w-w' \right|=\left| \frac{az+b}{cz+d}-\frac{az'+b}{cz'+d} \right|=\frac{\left| \left( az+b \right) \left( cz'+d \right) -\left( az'+b \right) \left( cz+d \right) \right|}{\left| cz+d \right|\left| cz'+d \right|}=\frac{\left| ad-bc \right|\left| z-z' \right|}{\left| cz+d \right|\left| cz'+d \right|}=\frac{\left| z-z' \right|}{\left| cz+d \right|\left| cz'+d \right|};
\end{align*}
\begin{align*}
\left| w \right|^2+1=\frac{\left| az+b \right|^2+\left| cz+d \right|^2}{\left| cz+d \right|^2}=\frac{\left| a \right|^2\left| z \right|^2+\left| b \right|^2+\left| c \right|^2\left| z \right|^2+\left| d \right|^2}{\left| cz+d \right|^2}=\frac{\left| z \right|^2+1}{\left| cz+d \right|};
\end{align*}
\begin{align*}
\left| w' \right|^2+1=\frac{\left| az'+b \right|^2+\left| cz'+d \right|^2}{\left| cz'+d \right|^2}=\frac{\left| a \right|^2\left| z' \right|^2+\left| b \right|^2+\left| c \right|^2\left| z' \right|^2+\left| d \right|^2}{\left| cz'+d \right|^2}=\frac{\left| z' \right|^2+1}{\left| cz'+d \right|}.
\end{align*}
由\rrefpro{proposition:扩充复球面的一些基本性质}{proposition:扩充复球面的一些基本性质-3}可得$w,w'$之间的距离为
\begin{align*}
d\left( w,w' \right) =\frac{2\left| w-w' \right|}{\sqrt{\left( \left| w \right|^2+1 \right) \left( \left| w' \right|^2+1 \right)}}=\frac{\frac{2\left| z-z' \right|}{\left| cz+d \right|\left| cz'+d \right|}}{\sqrt{\frac{\left| z \right|^2+1}{\left| cz+d \right|}\cdot \frac{\left| z' \right|^2+1}{\left| cz'+d \right|}}}=\frac{2\left| z-z' \right|}{\sqrt{\left( \left| z \right|^2+1 \right) \left( \left| z' \right|^2+1 \right)}}=d\left( z,z' \right) .
\end{align*}
故$w$是保距变换.又显然$w$在$\mathbb{C}$上全纯,由\eqref{eq:21.1}式可得
\begin{align*}
w'=\frac{ad-bc}{\left( cz+d \right) ^2}=\pm \frac{1}{\left( cz+d \right) ^2}\ne 0,\quad \forall z\in \mathbb{C}.
\end{align*}
当$z=\infty$时,令$g\left( z \right) =w\left( \frac{1}{z} \right)$,则
\begin{align*}
w'\left( \infty \right) =g'\left( 0 \right) =\left. \left( \frac{a\frac{1}{z}+b}{c\frac{1}{z}+d} \right)' \right| _{z=0}=\left. \left( \frac{a+bz}{c+dz} \right)' \right| _{z=0}=\left. \frac{bc-ad}{\left( c+dz \right) ^2} \right| _{z=0}=\pm \frac{1}{c^2}\ne 0.
\end{align*}
故$w'$在$\mathbb{C} _{\infty}$上恒不为$0$,因此由\refcor{corollary:复变函数保角的充要条件是全纯且导数值不为0}知$w$在$\mathbb{C} _{\infty}$上是保角的,进而保定向.综上可知,$w$是$\mathbb{C} _{\infty}$上的旋转变换.

\item 设球面上的圆周$S$方程为
\begin{gather*}
x_1^2 + x_2^2 + x_3^2 = 1,\alpha_1x_1 + \alpha_2x_2 + \alpha_3x_3 = \alpha,
\\
\alpha_1^2 + \alpha_2^2 + \alpha_3^2 = 1.
\end{gather*}
化简得
\begin{align*}
(\alpha_3 - \alpha)z\overline{z} + (\alpha_1 - i\alpha_2)z + (\alpha_1 + i\alpha_2)\overline{z} + (-\alpha_3 - \alpha) = 0
\end{align*}
当$\alpha_3 \neq \alpha,$时,由\rrrefpro{proposition:复数的几何性质结论}{proposition:复数的几何性质结论-5}{proposition:复数的几何性质结论-5-2}知这时$S$为复平面上的圆周。当$\alpha_3 = \alpha$时,由\rrrefpro{proposition:复数的几何性质结论}{proposition:复数的几何性质结论-5}{proposition:复数的几何性质结论-5-2}知这时$S$为复平面上的直线。

反之,由\rrrefpro{proposition:复数的几何性质结论}{proposition:复数的几何性质结论-5}{proposition:复数的几何性质结论-5-2}可设$\mathbb{C}$上的圆或直线$S$为$A z\overline{z} + \overline{B}z + B\overline{z} + C = 0$,其中$A,C\in \mathbb{R},0\neq B=\alpha+\text{i}\beta \in \mathbb{C}$,则由\refthe{theorem:扩充复平面与单位球面对等},将$z$的球面像坐标代入得
\begin{align*}
A\frac{x_1^2 + x_2^2}{(1 - x_3)^2} + \overline{B}\frac{x_1 + \text{i}x_2}{1 - x_3} + B\frac{x_1 - \text{i}x_2}{1 - x_3} + C = 0,
\end{align*}
其中$x_1^2+x_2^2+x_3^2=1$.化简得
\begin{align*}
&\quad \quad A\frac{x_{1}^{2}+x_{2}^{2}}{(1-x_3)^2}+\overline{B}\frac{x_1+\mathrm{i}x_2}{1-x_3}+B\frac{x_1-\mathrm{i}x_2}{1-x_3}+C=0
\\
&\Longleftrightarrow A\frac{x_{1}^{2}+x_{2}^{2}}{(1-x_3)^2}+2\mathrm{Re}\left( B\frac{x_1-\mathrm{i}x_2}{1-x_3} \right) +C=0
\\
&\Longleftrightarrow A\frac{x_{1}^{2}+x_{2}^{2}}{(1-x_3)^2}+\frac{2\left( \alpha x_1+\beta x_2 \right)}{1-x_3}+C=0
\\
&\Longleftrightarrow A\frac{1-x_{3}^{2}}{(1-x_3)^2}+\frac{2\left( \alpha x_1+\beta x_2 \right)}{1-x_3}+C=0
\\
&\Longleftrightarrow A\frac{1+x_3}{1-x_3}+\frac{2\left( \alpha x_1+\beta x_2 \right)}{1-x_3}+C=0
\\
&\Longleftrightarrow \left( A-1 \right) x_3+2\alpha x_1+2\beta x_2+A+C=0.
\end{align*}
从而
\begin{align*}
S:\begin{cases}
\left( A-1 \right) x_3+2\alpha x_1+2\beta x_2+A+C=0,\\
x_{1}^{2}+x_{2}^{2}+x_{3}^{2}=1.\\
\end{cases}
\end{align*}
故$S$必为黎曼球面上的圆周.

\item 球极投影$f$将复平面上的点$z = x + \text{i}y$映射到单位球面$S^2$上的点$(X,Y,Z)$,由\refthe{theorem:扩充复平面与单位球面对等}可知
\begin{align*}
X = \frac{2x}{x^2 + y^2 + 1}, \quad Y = \frac{2y}{x^2 + y^2 + 1}, \quad Z = \frac{x^2 + y^2 - 1}{x^2 + y^2 + 1}.
\end{align*}
从而球极投影$f$的Jacobi矩阵为:
\begin{align*}
J = \frac{\partial(X,Y,Z)}{\partial(x,y)} = \begin{bmatrix}
\frac{2(y^2 + 1 - x^2)}{(x^2 + y^2 + 1)^2} & \frac{-4xy}{(x^2 + y^2 + 1)^2} \\
\frac{-4xy}{(x^2 + y^2 + 1)^2} & \frac{2(x^2 + 1 - y^2)}{(x^2 + y^2 + 1)^2} \\
\frac{4x}{(x^2 + y^2 + 1)^2} & \frac{4y}{(x^2 + y^2 + 1)^2}
\end{bmatrix}
\end{align*}
于是
\begin{align}\label{eq::--2hr89289hjoawjdijwofjo8ewuohj38wjfnv}
J^T J = \frac{4}{(x^2 + y^2 + 1)^2} \begin{bmatrix} 1 & 0 \\ 0 & 1 \end{bmatrix}=\frac{4}{(x^2 + y^2 + 1)^2}I_2.
\end{align}
设复平面上两条相交的光滑曲线$\gamma _1\left( t \right) =x_1\left( t \right) +\mathrm{i}y_1\left( t \right)$,$\gamma _2\left( t \right) =x_2\left( t \right) +\mathrm{i}y_2\left( t \right)$,交点不妨设为$z_0=x_0+\mathrm{i}y_0=\gamma _1\left( 0 \right) =\gamma _2\left( 0 \right)$,则$\gamma _1\left( t \right)$,$\gamma _2\left( t \right)$在交点处的切向量分别为
\begin{align*}
\overrightarrow{v_1}=\left( x_{1}^{\prime}\left( 0 \right) ,y_{1}^{\prime}\left( 0 \right) \right) ,\quad \overrightarrow{v_2}=\left( x_{2}^{\prime}\left( 0 \right) ,y_{2}^{\prime}\left( 0 \right) \right) .
\end{align*}
从而两条曲线在$z_0$处的夹角$\theta$的余弦为
\begin{align*}
\cos \theta =\frac{\overrightarrow{v_1}\cdot \overrightarrow{v_2}}{\left\| \overrightarrow{v_1} \right\| \cdot \left\| \overrightarrow{v_2} \right\|}.
\end{align*}
记曲线$\gamma _k\left( t \right)$,$k=1,2$在球极投影$f$下的球面像为$f\left( \gamma _k\left( t \right) \right) =f\left( x_k\left( t \right) +\mathrm{i}y_k\left( t \right) \right) =\left( X_k\left( t \right) ,Y_k\left( t \right) ,Z_k\left( t \right) \right)$,$k=1,2$.则$f\left( \gamma _k\left( t \right) \right)$的切向量为
\begin{align*}
\overrightarrow{w_k}=\left. \frac{\mathrm{d}}{\mathrm{d}t}f\left( \gamma _k\left( t \right) \right) \right| _{t=0}=\left. J \right| _{\left( x_0,y_0 \right)}\cdot \overrightarrow{v_k},\quad k=1,2.
\end{align*}
记$J_0=\left. J \right| _{\left( x_0,y_0 \right)}$,则$\overrightarrow{w_k}=J_0\overrightarrow{v_k}$,$k=1,2$.于是利用\eqref{eq::--2hr89289hjoawjdijwofjo8ewuohj38wjfnv}式可得$f\left( \gamma _1\left( t \right) \right)$,$f\left( \gamma _2\left( t \right) \right)$在交点$f\left( z_0 \right)$处的夹角$\theta'$的余弦为
\begin{align*}
\cos \theta' &=\frac{\overrightarrow{w_1}\cdot \overrightarrow{w_2}}{\left\| \overrightarrow{w_1} \right\| \cdot \left\| \overrightarrow{w_2} \right\|}=\frac{\left( J_0\overrightarrow{v_1} \right) \cdot \left( J_0\overrightarrow{v_2} \right)}{\left\| J_0\overrightarrow{v_1} \right\| \cdot \left\| J_0\overrightarrow{v_2} \right\|}=\frac{\overrightarrow{v_1}^TJ_{0}^{T}J_0\overrightarrow{v_2}}{\sqrt{\left\| J_0\overrightarrow{v_1} \right\| ^2}\cdot \sqrt{\left\| J_0\overrightarrow{v_2} \right\| ^2}} \\
&=\frac{\frac{4}{\left( x_{0}^{2}+y_{0}^{2}+1 \right) ^2}\cdot \overrightarrow{v_1}^T\overrightarrow{v_2}}{\sqrt{\left( J_0\overrightarrow{v_1} \right) ^T\left( J_0\overrightarrow{v_1} \right)}\cdot \sqrt{\left( J_0\overrightarrow{v_2} \right) ^T\left( J_0\overrightarrow{v_2} \right)}} \\
&=\frac{\frac{4}{\left( x_{0}^{2}+y_{0}^{2}+1 \right) ^2}\cdot \overrightarrow{v_1}^T\overrightarrow{v_2}}{\sqrt{\frac{4}{\left( x_{0}^{2}+y_{0}^{2}+1 \right) ^2}\overrightarrow{v_1}^T\overrightarrow{v_1}}\cdot \sqrt{\frac{4}{\left( x_{0}^{2}+y_{0}^{2}+1 \right) ^2}\overrightarrow{v_2}^T\overrightarrow{v_2}}} \\
&=\frac{\overrightarrow{v_1}\cdot \overrightarrow{v_2}}{\left\| \overrightarrow{v_1} \right\| \cdot \left\| \overrightarrow{v_2} \right\|}=\cos \theta.
\end{align*}

\end{enumerate}

\end{proof}






















\end{document}
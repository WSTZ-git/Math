\documentclass[../../main.tex]{subfiles}
\graphicspath{{\subfix{../../image/}}} % 指定图片目录,后续可以直接使用图片文件名。

% 例如:
% \begin{figure}[H]
% \centering
% \includegraphics[scale=0.4]{图.png}
% \caption{}
% \label{figure:图}
% \end{figure}
% 注意:上述\label{}一定要放在\caption{}之后,否则引用图片序号会只会显示??.

\begin{document}

\section{扩充平面和复数的球面表示}

\begin{definition}
为了今后讨论的需要,我们要在\(\mathbf{C}\)中引进一个新的数\(\infty\),这个数的模是\(\infty\),辐角没有意义,它和其他数的运算规则规定为:
\[
z \pm \infty = \infty, \quad z \cdot \infty = \infty \ (z \neq 0),
\]
\[
\frac{z}{\infty} = 0, \quad \frac{z}{0} = \infty \ (z \neq 0);
\]
\(0 \cdot \infty\)和\(\infty \pm \infty\)都不规定其意义。引进了\(\infty\)的复数系记为\(\mathbf{C}_\infty\),即\(\mathbf{C}_\infty = \mathbf{C} \cup \{ \infty \}\)。

在复平面上,没有一个点和\(\infty\)相对应,但我们想像有一个\textbf{无穷远点}和\(\infty\)对应,加上无穷远点的复平面称为\textbf{扩充平面}或\textbf{闭平面},不包括无穷远点的复平面也称为\textbf{开平面}。
\end{definition}
\begin{remark}
在复平面上,无穷远点和普通的点是不一样的,Riemann 首先引进了复数的球面表示,在这种表示中,\(\infty\)和普通的复数没有什么区别。
\end{remark}

\begin{proposition}
证明:扩充平面和单位球面对等,即两者之间存在一个双射.
\end{proposition}
\begin{proof}
设\(S\)是\(\mathbf{R}^3\)中的单位球面,即
\[
S = \{ (x_1, x_2, x_3) \in \mathbf{R}^3 : x_1^2 + x_2^2 + x_3^2 = 1 \}.
\]
把\(\mathbf{C}\)等同于平面:
\[
\mathbf{C} = \{ (x_1, x_2, 0) : x_1, x_2 \in \mathbf{R} \}.
\]
固定\(S\)的北极\(N\),即\(N = (0, 0, 1)\),对于\(\mathbf{C}\)上的任意点\(z\),联结\(N\)和\(z\)的直线必和\(S\)交于一点\(P\)(\reffig{figure:图1.5})。若\(|z| > 1\),则\(P\)在北半球上;若\(|z| < 1\),则\(P\)在南半球上;若\(|z| = 1\),则\(P\)就是\(z\)。容易看出,当\(z\)趋向\(\infty\)时,球面上对应的点\(P\)趋向于北极\(N\),自然地,我们就把\(\mathbf{C}_\infty\)中的\(\infty\)对应于北极\(N\)。这样一来,\(\mathbf{C}_\infty\)中的所有点(包括无穷远点在内)都被移植到球面上去了,这样我们就找到了一个扩充平面到单位球面的双射.而在球面上,\(N\)和其他的点是一视同仁的。
\begin{figure}[H]
\centering
\includegraphics[scale=0.2]{图1.5.png}
\caption{}
\label{figure:图1.5}
\end{figure}
现在给出这种对应的具体表达式。设\(z = x + \mathrm{i}y\),容易算出\(zN\)和球面\(S\)的交点的坐标为
\[
x_1 = \frac{2x}{x^2 + y^2 + 1}, \ x_2 = \frac{2y}{x^2 + y^2 + 1}, \ x_3 = \frac{x^2 + y^2 - 1}{x^2 + y^2 + 1}.
\]
直接用复数\(z\),可表示为
\[
x_1 = \frac{z + \bar{z}}{1 + |z|^2}, \ x_2 = \frac{z - \bar{z}}{\mathrm{i}(1 + |z|^2)}, \ x_3 = \frac{|z|^2 - 1}{|z|^2 + 1}.
\]
这样,从\(z\)便可算出它在球面上对应点的坐标。反过来,从球面上的点\((x_1, x_2, x_3)\)也可算出它在平面上的对应点\(z\)。事实上,从上面的表达式得
\[
\begin{cases}
x_1 + \mathrm{i}x_2 = \frac{2z}{1 + |z|^2}, \\
1 - x_3 = \frac{2}{1 + |z|^2},
\end{cases}
\]
由此即得
\[
z = \frac{x_1 + \mathrm{i}x_2}{1 - x_3}.
\]
这就是所需的计算公式。现在我们可以具体地写出扩充平面到单位球面的双射
\begin{gather*}
f:C_{\infty}\longrightarrow \mathbf{R} ^3,
z\longmapsto \left( \frac{z+\bar{z}}{1+|z|^2}, \frac{z-\bar{z}}{\mathrm{i(}1+|z|^2)}, \frac{|z|^2-1}{|z|^2+1} \right) .
\\
f^{-1}:\mathbf{R} ^3\longrightarrow C_{\infty},
\left( x_1,x_2,x_3 \right) \longmapsto \frac{x_1+\mathrm{i}x_2}{1-x_3}.
\end{gather*}
\end{proof}

























\end{document}
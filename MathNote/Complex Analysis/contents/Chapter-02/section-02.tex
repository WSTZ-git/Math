\documentclass[../../main.tex]{subfiles}
\graphicspath{{\subfix{../../image/}}} % 指定图片目录,后续可以直接使用图片文件名。

% 例如:
% \begin{figure}[H]
% \centering
% \includegraphics[scale=0.3]{image-01.01}
% \caption{图片标题}
% \label{figure:image-01.01}
% \end{figure}
% 注意:上述\label{}一定要放在\caption{}之后,否则引用图片序号会只会显示??.

\begin{document}

\section{Cauchy-Riemann方程}

\begin{definition}
设 \( f(z) = u(x,y) + \text{i}v(x,y) \) 是定义在域 \( D \) 上的函数,\( z_0 = x_0 + \text{i}y_0 \in D \)。我们说 \( f \) 在 \( z_0 \) 处\textbf{实可微},是指 \( u \) 和 \( v \) 作为 \( x,y \) 的二元函数在 \( (x_0,y_0) \) 处可微.
\end{definition}

\begin{proposition}
设 \( f:D \to \mathbb{C} \) 是定义在域 \( D \) 上的函数,\( z_0 \in D \),那么 \( f \) 在 \( z_0 \) 处实可微的充分必要条件是
\begin{align}\label{equation::::1-4}
f(z_0 + \Delta z) - f(z_0) &= \frac{\partial f}{\partial z}(z_0)\Delta z + \frac{\partial f}{\partial \overline{z}}(z_0)\overline{\Delta z} + o(|\Delta z|).
\end{align}
成立,其中
\begin{align*}
\frac{\partial}{\partial z} &= \frac{1}{2}\left( \frac{\partial}{\partial x} - \text{i}\frac{\partial}{\partial y} \right), \\
\frac{\partial}{\partial \overline{z}} &= \frac{1}{2}\left( \frac{\partial}{\partial x} + \text{i}\frac{\partial}{\partial y} \right).
\end{align*}
\end{proposition}
\begin{proof}
设 \( f \) 在 \( z_0 \) 处实可微,由二元实值函数可微的定义,有
\begin{align}
u(x_0 + \Delta x, y_0 + \Delta y) - u(x_0, y_0) &= \frac{\partial u}{\partial x}(x_0, y_0)\Delta x + \frac{\partial u}{\partial y}(x_0, y_0)\Delta y + o(|\Delta z|), \label{eq:-----2.2.1} \\
v(x_0 + \Delta x, y_0 + \Delta y) - v(x_0, y_0) &= \frac{\partial v}{\partial x}(x_0, y_0)\Delta x + \frac{\partial v}{\partial y}(x_0, y_0)\Delta y + o(|\Delta z|), \label{eq:-----2.2.2}
\end{align}
这里,\( |\Delta z| = \sqrt{(\Delta x)^2 + (\Delta y)^2} \)。于是
\begin{align*}
f(z_0 + \Delta z) - f(z_0) &= u(x_0 + \Delta x, y_0 + \Delta y) - u(x_0, y_0) + \text{i}(v(x_0 + \Delta x, y_0 + \Delta y) - v(x_0, y_0)) \\
&= \frac{\partial u}{\partial x}(x_0, y_0)\Delta x + \frac{\partial u}{\partial y}(x_0, y_0)\Delta y + o(|\Delta z|) + \text{i}\left( \frac{\partial v}{\partial x}(x_0, y_0)\Delta x + \frac{\partial v}{\partial y}(x_0, y_0)\Delta y + o(|\Delta z|) \right) \\
&= \left( \frac{\partial u}{\partial x}(x_0, y_0) + \text{i}\frac{\partial v}{\partial x}(x_0, y_0) \right)\Delta x + \left( \frac{\partial u}{\partial y}(x_0, y_0) + \text{i}\frac{\partial v}{\partial y}(x_0, y_0) \right)\Delta y + o(|\Delta z|) \\
&= \frac{\partial f}{\partial x}(x_0, y_0)\Delta x + \frac{\partial f}{\partial y}(x_0, y_0)\Delta y + o(|\Delta z|).
\end{align*}
把 \( \Delta x = \frac{1}{2}(\Delta z + \overline{\Delta z}) \),\( \Delta y = \frac{1}{2\text{i}}(\Delta z - \overline{\Delta z}) \) 代入上式,得
\begin{align*}
f(z_0 + \Delta z) - f(z_0) &= \frac{1}{2}\frac{\partial f}{\partial x}(x_0, y_0)(\Delta z + \overline{\Delta z}) - \frac{\text{i}}{2}\frac{\partial f}{\partial y}(x_0, y_0)(\Delta z - \overline{\Delta z}) + o(|\Delta z|) \\
&= \frac{1}{2}\left( \frac{\partial}{\partial x} - \text{i}\frac{\partial}{\partial y} \right)f(x_0, y_0)\Delta z + \frac{1}{2}\left( \frac{\partial}{\partial x} + \text{i}\frac{\partial}{\partial y} \right)f(x_0, y_0)\overline{\Delta z} + o(|\Delta z|).
\end{align*}
引进算子
\begin{gather}\label{eq:-----2.2.3}
\begin{aligned}
\frac{\partial}{\partial z} &= \frac{1}{2}\left( \frac{\partial}{\partial x} - \text{i}\frac{\partial}{\partial y} \right), \\
\frac{\partial}{\partial \overline{z}} &= \frac{1}{2}\left( \frac{\partial}{\partial x} + \text{i}\frac{\partial}{\partial y} \right),
\end{aligned}
\end{gather}
则上式可写为
\begin{align}\label{eq:-----2.2.4}
f(z_0 + \Delta z) - f(z_0) &= \frac{\partial f}{\partial z}(z_0)\Delta z + \frac{\partial f}{\partial \overline{z}}(z_0)\overline{\Delta z} + o(|\Delta z|).
\end{align}
容易看出,\eqref{eq:-----2.2.4}式和\eqref{eq:-----2.2.1},\eqref{eq:-----2.2.2}两式等价。
\end{proof}

\begin{remark}
为什么要像\eqref{eq:-----2.2.3}式那样来定义算子 \( \frac{\partial}{\partial z} \) 和 \( \frac{\partial}{\partial \overline{z}} \) 呢?这是因为如果把复变函数 \( f(z) \) 写成
\begin{align*}
f(x,y) = f\left( \frac{z + \overline{z}}{2}, -\text{i}\frac{z - \overline{z}}{2} \right),
\end{align*}
把 \( z,\overline{z} \) 看成独立变量,分别对 \( z \) 和 \( \overline{z} \) 求偏导数,则得
\begin{align*}
\frac{\partial f}{\partial z} &= \frac{\partial f}{\partial x}\frac{\partial x}{\partial z} + \frac{\partial f}{\partial y}\frac{\partial y}{\partial z} = \frac{1}{2}\left( \frac{\partial f}{\partial x} - \text{i}\frac{\partial f}{\partial y} \right), \\
\frac{\partial f}{\partial \overline{z}} &= \frac{\partial f}{\partial x}\frac{\partial x}{\partial \overline{z}} + \frac{\partial f}{\partial y}\frac{\partial y}{\partial \overline{z}} = \frac{1}{2}\left( \frac{\partial f}{\partial x} + \text{i}\frac{\partial f}{\partial y} \right).
\end{align*}
这就是表达式\eqref{eq:-----2.2.3}的来源。这说明在进行微分运算时,可以把 \( z,\overline{z} \) 看成独立的变量。

现在很容易得到 \( f \) 在 \( z_0 \) 处可微的条件了。
\end{remark}

\begin{theorem}\label{theorem:复变函数可微的充要条件1}
设 \( f \) 是定义在域 \( D \) 上的函数,\( z_0 \in D \),那么 \( f \) 在 \( z_0 \) 处可微的充要条件是 \( f \) 在 \( z_0 \) 处实可微且 \( \frac{\partial f}{\partial \overline{z}}(z_0) = 0 \)。在可微的情况下,\( f'(z_0) = \frac{\partial f}{\partial z}(z_0) \)。
\end{theorem}
\begin{proof}
如果 \( f \) 在 \( z_0 \) 处可微,由\eqref{eq:2.1.2}式得
\[
f(z_0 + \Delta z) - f(z_0) = f'(z_0)\Delta z + o(|\Delta z|)
\]
与\eqref{equation::::1-4}式比较就知道,\( f \) 在 \( z_0 \) 处是实可微的,而且 \( \frac{\partial f}{\partial \overline{z}}(z_0) = 0 \),\( f'(z_0) = \frac{\partial f}{\partial z}(z_0) \)。

反之,若 \( f \) 在 \( z_0 \) 处实可微,且 \( \frac{\partial f}{\partial \overline{z}}(z_0) = 0 \),则由\eqref{equation::::1-4}式得
\[
f(z_0 + \Delta z) - f(z_0) = \frac{\partial f}{\partial z}(z_0)\Delta z + o(|\Delta z|)
\]
由此即知 
\begin{align*}
\underset{\Delta z}{\lim}\frac{f\left( z_0+\Delta z \right) -f\left( z_0 \right)}{\Delta z}=\frac{\partial f}{\partial z}\left( z_0 \right).
\end{align*}
故\( f \) 在 \( z_0 \) 处可微,而且 \( f'(z_0) = \frac{\partial f}{\partial z}(z_0) \)。 
\end{proof}

\begin{definition}[Cauchy-Riemann 方程]
设$f$是定义在域$D$上的函数,$\frac{\partial f}{\partial \overline{z}} = 0$ 称为 $\mathbf{Cauchy}-\mathbf{Riemann}$\textbf{方程},从这个方程可以得到 $f$ 的实部和虚部应满足的条件。设 $f = u + iv$,则由\eqref{eq:-----2.2.3}式得
\begin{align*}
\frac{\partial f}{\partial \overline{z}}&=\frac{\partial u}{\partial \overline{z}} + \mathrm{i}\frac{\partial v}{\partial \overline{z}}\\
&=\frac{1}{2}\left(\frac{\partial u}{\partial x}+\mathrm{i}\frac{\partial u}{\partial y}\right)+\frac{i}{2}\left(\frac{\partial v}{\partial x}+\mathrm{i}\frac{\partial v}{\partial y}\right)\\
&=\frac{1}{2}\left(\frac{\partial u}{\partial x}-\frac{\partial v}{\partial y}\right)+\frac{i}{2}\left(\frac{\partial u}{\partial y}+\frac{\partial v}{\partial x}\right)
\end{align*}
因此,Cauchy-Riemann 方程 $\frac{\partial f}{\partial \overline{z}} = 0$ 就等价于
\begin{align}\label{equation:--2.5}
\begin{cases}
\frac{\partial u}{\partial x}=\frac{\partial v}{\partial y},\\
\frac{\partial u}{\partial y}=-\frac{\partial v}{\partial x}.
\end{cases}
\end{align}
\end{definition}

\begin{theorem}
设 \( f = u + iv \) 是定义在域 \( D \) 上的函数,\( z_0 = x_0 + iy_0 \in D \),那么 \( f \) 在 \( z_0 \) 处可微的充要条件是 \( u(x, y) \),\( v(x, y) \) 在 \( (x_0, y_0) \) 处可微,且在 \( (x_0, y_0) \) 处满足
\[
\begin{cases}
\dfrac{\partial u}{\partial x} = \dfrac{\partial v}{\partial y}, \\
\dfrac{\partial u}{\partial y} = -\dfrac{\partial v}{\partial x}
\end{cases}
\]
在可微的情况下,有
\begin{align*}
f'(z_0) = \dfrac{\partial u}{\partial x} + \mathrm{i}\dfrac{\partial v}{\partial x}
= \dfrac{\partial v}{\partial y} + \mathrm{i}\dfrac{\partial v}{\partial x} = \dfrac{\partial u}{\partial x} - \mathrm{i}\dfrac{\partial u}{\partial y} = \dfrac{\partial v}{\partial y} - \mathrm{i}\dfrac{\partial u}{\partial y}.
\end{align*}
这里的偏导数都在 \( (x_0, y_0) \) 处取值。
\end{theorem}
\begin{proof}
最后这个 \( f'(z_0) \) 的表达式是从 \refthe{theorem:复变函数可微的充要条件1}中的\( f'(z_0) = \dfrac{\partial f}{\partial z}(z_0) \) 和 Cauchy-Riemann 方程\eqref{equation:--2.5}得到的。
\end{proof}

\begin{definition}
\begin{enumerate}
\item 设 \( D \) 是 \( \mathbb{C} \) 中的域,我们用 \( C(D) \) 记 \( D \) 上连续函数的全体,用 \( H(D) \) 记 \( D \) 上全纯函数的全体.

\item 设 \( f = u + \mathrm{i}v \),记 \( \frac{\partial f}{\partial x} = \frac{\partial u}{\partial x} + \mathrm{i}\frac{\partial v}{\partial x} \),\( \frac{\partial f}{\partial y} = \frac{\partial u}{\partial y} + \mathrm{i}\frac{\partial v}{\partial y} \)。我们用 \( C^1(D) \) 记 \( \frac{\partial f}{\partial x} \),\( \frac{\partial f}{\partial y} \) 在 \( D \) 上连续的 \( f \) 的全体。

\item 用 \( C^k(D) \) 记在 \( D \) 上有 \( k \) 阶连续偏导数的函数的全体,\( C^\infty(D) \) 记在 \( D \) 上有任意阶连续偏导数的函数的全体。
\end{enumerate}
\end{definition}

\begin{proposition}
\begin{enumerate}[(1)]
\item \( H(D) \subset C(D) \).

\item $C^1(D) \subset C(D).$

\item 域 \( D \) 上的全纯函数在 \( D \) 上有任意阶的连续偏导数,并且有如下的包含关系:
\[
H(D) \subset C^\infty(D) \subset C^k(D) \subset C^1(D) \subset C(D)
\]
这里,\( k \) 是大于 1 的自然数。
\end{enumerate}
\end{proposition}
\begin{proof}
\begin{enumerate}[(1)]
\item \refpro{proposition:全纯函数必连续}告诉我们,\( H(D) \subset C(D) \)。

\item 设 \( f = u + \mathrm{i}iv \),记 \( \frac{\partial f}{\partial x} = \frac{\partial u}{\partial x} + \mathrm{i}\frac{\partial v}{\partial x} \),\( \frac{\partial f}{\partial y} = \frac{\partial u}{\partial y} + \mathrm{i}\frac{\partial v}{\partial y} \)。我们用 \( C^1(D) \) 记 \( \frac{\partial f}{\partial x} \),\( \frac{\partial f}{\partial y} \) 在 \( D \) 上连续的 \( f \) 的全体。进而$u,v$关于$x,y$的偏导在$D$上都连续,由多元微积分的知识知道,$u,v$在$D$上都可微。于是对于任意 \( f \in C^1(D) \),\( f \) 在 \( D \) 上实可微,从\eqref{eq:-----2.2.4}式知道
\begin{align*}
f(z_0 + \Delta z) - f(z_0) &= \frac{\partial f}{\partial z}(z_0)\Delta z + \frac{\partial f}{\partial \overline{z}}(z_0)\overline{\Delta z} + o(|\Delta z|).
\end{align*}
令$\Delta z\to 0$,则$\lim_{\Delta z\to 0}f(z_0+\Delta z)=f(z_0)$,
故\( f \) 在 \( D \) 上连续,因而
$C^1(D) \subset C(D).$

\item 
\end{enumerate}
\end{proof}

\begin{example}

\end{example}
\begin{proof}

\end{proof}











\end{document}
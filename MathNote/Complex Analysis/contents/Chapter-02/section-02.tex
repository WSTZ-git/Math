\documentclass[../../main.tex]{subfiles}
\graphicspath{{\subfix{../../image/}}} % 指定图片目录,后续可以直接使用图片文件名。

% 例如:
% \begin{figure}[H]
% \centering
% \includegraphics[scale=0.4]{image-01.01}
% \caption{图片标题}
% \label{figure:image-01.01}
% \end{figure}
% 注意:上述\label{}一定要放在\caption{}之后,否则引用图片序号会只会显示??.

\begin{document}

\section{Cauchy-Riemann方程}

\begin{definition}
设 \( f(z) = u(x,y) + \text{i}v(x,y) \) 是定义在域 \( D \) 上的函数,\( z_0 = x_0 + \text{i}y_0 \in D \)。我们说 \( f \) 在 \( z_0 \) 处\textbf{实可微},是指 \( u \) 和 \( v \) 作为 \( x,y \) 的二元函数在 \( (x_0,y_0) \) 处可微.
\end{definition}

\begin{proposition}
设 \( f:D \to \mathbb{C} \) 是定义在域 \( D \) 上的函数,\( z_0 \in D \),那么 \( f \) 在 \( z_0 \) 处实可微的充分必要条件是
\begin{align}\label{equation::::1-4}
f(z_0 + \Delta z) - f(z_0) &= \frac{\partial f}{\partial z}(z_0)\Delta z + \frac{\partial f}{\partial \overline{z}}(z_0)\overline{\Delta z} + o(|\Delta z|).
\end{align}
成立,其中
\begin{align*}
\frac{\partial}{\partial z} &= \frac{1}{2}\left( \frac{\partial}{\partial x} - \text{i}\frac{\partial}{\partial y} \right), \\
\frac{\partial}{\partial \overline{z}} &= \frac{1}{2}\left( \frac{\partial}{\partial x} + \text{i}\frac{\partial}{\partial y} \right).
\end{align*}
\end{proposition}
\begin{proof}
设 \( f \) 在 \( z_0 \) 处实可微,由二元实值函数可微的定义,有
\begin{align}
u(x_0 + \Delta x, y_0 + \Delta y) - u(x_0, y_0) &= \frac{\partial u}{\partial x}(x_0, y_0)\Delta x + \frac{\partial u}{\partial y}(x_0, y_0)\Delta y + o(|\Delta z|), \label{eq:-----2.2.1} \\
v(x_0 + \Delta x, y_0 + \Delta y) - v(x_0, y_0) &= \frac{\partial v}{\partial x}(x_0, y_0)\Delta x + \frac{\partial v}{\partial y}(x_0, y_0)\Delta y + o(|\Delta z|), \label{eq:-----2.2.2}
\end{align}
这里,\( |\Delta z| = \sqrt{(\Delta x)^2 + (\Delta y)^2} \)。于是
\begin{align*}
f(z_0 + \Delta z) - f(z_0) &= u(x_0 + \Delta x, y_0 + \Delta y) - u(x_0, y_0) + \text{i}(v(x_0 + \Delta x, y_0 + \Delta y) - v(x_0, y_0)) \\
&= \frac{\partial u}{\partial x}(x_0, y_0)\Delta x + \frac{\partial u}{\partial y}(x_0, y_0)\Delta y + o(|\Delta z|) + \text{i}\left( \frac{\partial v}{\partial x}(x_0, y_0)\Delta x + \frac{\partial v}{\partial y}(x_0, y_0)\Delta y + o(|\Delta z|) \right) \\
&= \left( \frac{\partial u}{\partial x}(x_0, y_0) + \text{i}\frac{\partial v}{\partial x}(x_0, y_0) \right)\Delta x + \left( \frac{\partial u}{\partial y}(x_0, y_0) + \text{i}\frac{\partial v}{\partial y}(x_0, y_0) \right)\Delta y + o(|\Delta z|) \\
&= \frac{\partial f}{\partial x}(x_0, y_0)\Delta x + \frac{\partial f}{\partial y}(x_0, y_0)\Delta y + o(|\Delta z|).
\end{align*}
把 \( \Delta x = \frac{1}{2}(\Delta z + \overline{\Delta z}) \),\( \Delta y = \frac{1}{2\text{i}}(\Delta z - \overline{\Delta z}) \) 代入上式,得
\begin{align*}
f(z_0 + \Delta z) - f(z_0) &= \frac{1}{2}\frac{\partial f}{\partial x}(x_0, y_0)(\Delta z + \overline{\Delta z}) - \frac{\text{i}}{2}\frac{\partial f}{\partial y}(x_0, y_0)(\Delta z - \overline{\Delta z}) + o(|\Delta z|) \\
&= \frac{1}{2}\left( \frac{\partial}{\partial x} - \text{i}\frac{\partial}{\partial y} \right)f(x_0, y_0)\Delta z + \frac{1}{2}\left( \frac{\partial}{\partial x} + \text{i}\frac{\partial}{\partial y} \right)f(x_0, y_0)\overline{\Delta z} + o(|\Delta z|).
\end{align*}
引进算子
\begin{gather}\label{eq:-----2.2.3}
\begin{aligned}
\frac{\partial}{\partial z} &= \frac{1}{2}\left( \frac{\partial}{\partial x} - \text{i}\frac{\partial}{\partial y} \right), \\
\frac{\partial}{\partial \overline{z}} &= \frac{1}{2}\left( \frac{\partial}{\partial x} + \text{i}\frac{\partial}{\partial y} \right),
\end{aligned}
\end{gather}
则上式可写为
\begin{align}\label{eq:-----2.2.4}
f(z_0 + \Delta z) - f(z_0) &= \frac{\partial f}{\partial z}(z_0)\Delta z + \frac{\partial f}{\partial \overline{z}}(z_0)\overline{\Delta z} + o(|\Delta z|).
\end{align}
容易看出,\eqref{eq:-----2.2.4}式和\eqref{eq:-----2.2.1},\eqref{eq:-----2.2.2}两式等价。
\end{proof}

\begin{remark}
为什么要像\eqref{eq:-----2.2.3}式那样来定义算子 \( \frac{\partial}{\partial z} \) 和 \( \frac{\partial}{\partial \overline{z}} \) 呢?这是因为如果把复变函数 \( f(z) \) 写成
\begin{align*}
f(x,y) = f\left( \frac{z + \overline{z}}{2}, -\text{i}\frac{z - \overline{z}}{2} \right),
\end{align*}
把 \( z,\overline{z} \) 看成独立变量,分别对 \( z \) 和 \( \overline{z} \) 求偏导数,则得
\begin{align*}
\frac{\partial f}{\partial z} &= \frac{\partial f}{\partial x}\frac{\partial x}{\partial z} + \frac{\partial f}{\partial y}\frac{\partial y}{\partial z} = \frac{1}{2}\left( \frac{\partial f}{\partial x} - \text{i}\frac{\partial f}{\partial y} \right), \\
\frac{\partial f}{\partial \overline{z}} &= \frac{\partial f}{\partial x}\frac{\partial x}{\partial \overline{z}} + \frac{\partial f}{\partial y}\frac{\partial y}{\partial \overline{z}} = \frac{1}{2}\left( \frac{\partial f}{\partial x} + \text{i}\frac{\partial f}{\partial y} \right).
\end{align*}
这就是表达式\eqref{eq:-----2.2.3}的来源。这说明在进行微分运算时,可以把 \( z,\overline{z} \) 看成独立的变量。

现在很容易得到 \( f \) 在 \( z_0 \) 处可微的条件了。
\end{remark}

\begin{theorem}\label{theorem:复变函数可微的充要条件1}
设 \( f \) 是定义在域 \( D \) 上的函数,\( z_0 \in D \),那么 \( f \) 在 \( z_0 \) 处可微的充要条件是 \( f \) 在 \( z_0 \) 处实可微且 \( \frac{\partial f}{\partial \overline{z}}(z_0) = 0 \)。在可微的情况下,\( f'(z_0) = \frac{\partial f}{\partial z}(z_0) \)。
\end{theorem}
\begin{proof}
如果 \( f \) 在 \( z_0 \) 处可微,由\eqref{eq:2.1.2}式得
\[
f(z_0 + \Delta z) - f(z_0) = f'(z_0)\Delta z + o(|\Delta z|)
\]
与\eqref{equation::::1-4}式比较就知道,\( f \) 在 \( z_0 \) 处是实可微的,而且 \( \frac{\partial f}{\partial \overline{z}}(z_0) = 0 \),\( f'(z_0) = \frac{\partial f}{\partial z}(z_0) \)。

反之,若 \( f \) 在 \( z_0 \) 处实可微,且 \( \frac{\partial f}{\partial \overline{z}}(z_0) = 0 \),则由\eqref{equation::::1-4}式得
\[
f(z_0 + \Delta z) - f(z_0) = \frac{\partial f}{\partial z}(z_0)\Delta z + o(|\Delta z|)
\]
由此即知 
\begin{align*}
\underset{\Delta z}{\lim}\frac{f\left( z_0+\Delta z \right) -f\left( z_0 \right)}{\Delta z}=\frac{\partial f}{\partial z}\left( z_0 \right).
\end{align*}
故\( f \) 在 \( z_0 \) 处可微,而且 \( f'(z_0) = \frac{\partial f}{\partial z}(z_0) \)。 
\end{proof}

\begin{definition}[Cauchy-Riemann 方程]
设$f$是定义在域$D$上的函数,$\frac{\partial f}{\partial \overline{z}} = 0$ 称为 $\mathbf{Cauchy}-\mathbf{Riemann}$\textbf{方程}.
\end{definition}
\begin{note}
从这个方程可以得到 $f$ 的实部和虚部应满足的条件。
\end{note}

\begin{proposition}[Cauchy-Riemann方程的等价定义]\label{proposition:Cauchy-Riemann方程的等价定义}
设 $z=x+\mathrm{i}y$,$f(z) = u(x,y) + \mathrm{i}v(x,y)$,则Cauchy-Riemann方程$\frac{\partial f}{\partial \overline{z}} = 0$ 等价于

(i)\begin{align}\label{equation:--2.5}
\begin{cases}
\frac{\partial u}{\partial x}=\frac{\partial v}{\partial y},\\
\frac{\partial u}{\partial y}=-\frac{\partial v}{\partial x}.
\end{cases}
\end{align}

(ii)$$\frac{\partial \overline{f}}{\partial z}=0.$$

(iii)令$x=r\cos \theta ,y=r\sin \theta $,进而$z=r(\cos\theta+\mathrm{i}\sin\theta),f(z) = u(r,\theta) + \mathrm{i}v(r,\theta)$,则Cauchy-Riemann方程$\frac{\partial f}{\partial \overline{z}} = 0$ 等价于
\begin{align*}
\begin{cases}
\dfrac{\partial u}{\partial r} = \dfrac{1}{r} \dfrac{\partial v}{\partial \theta}, \\
\dfrac{\partial v}{\partial r} = -\dfrac{1}{r} \dfrac{\partial u}{\partial \theta}.
\end{cases}
\end{align*}
\end{proposition}
\begin{proof}
(i)由\eqref{eq:-----2.2.3}式得
\begin{align*}
\frac{\partial f}{\partial \overline{z}}=\frac{\partial u}{\partial \overline{z}} + \mathrm{i}\frac{\partial v}{\partial \overline{z}}=\frac{1}{2}\left(\frac{\partial u}{\partial x}+\mathrm{i}\frac{\partial u}{\partial y}\right)+\frac{\mathrm{i}}{2}\left(\frac{\partial v}{\partial x}+\mathrm{i}\frac{\partial v}{\partial y}\right)=\frac{1}{2}\left(\frac{\partial u}{\partial x}-\frac{\partial v}{\partial y}\right)+\frac{\mathrm{i}}{2}\left(\frac{\partial u}{\partial y}+\frac{\partial v}{\partial x}\right).
\end{align*}
因此,Cauchy-Riemann 方程 $\frac{\partial f}{\partial \overline{z}} = 0$ 就等价于
\begin{align*}
\begin{cases}
\frac{\partial u}{\partial x}=\frac{\partial v}{\partial y},\\
\frac{\partial u}{\partial y}=-\frac{\partial v}{\partial x}.
\end{cases}
\end{align*}

(ii)又注意到
\begin{align*}
\frac{\partial \overline{f}}{\partial z}=\frac{\partial u}{\partial z}-\mathrm{i}\frac{\partial v}{\partial z}=\frac{1}{2}\left( \frac{\partial u}{\partial x}-\mathrm{i}\frac{\partial u}{\partial y} \right) -\frac{\mathrm{i}}{2}\left( \frac{\partial v}{\partial x}-\mathrm{i}\frac{\partial v}{\partial y} \right) =\frac{1}{2}\left( \frac{\partial u}{\partial x}-\frac{\partial v}{\partial y} \right) -\frac{\mathrm{i}}{2}\left( \frac{\partial u}{\partial y}+\frac{\partial v}{\partial x} \right) ,
\end{align*}
故Cauchy-Riemann 方程 $\frac{\partial f}{\partial \overline{z}} = 0$ 也等价于$\frac{\partial \overline{f}}{\partial z}=0.$

(iii)令 \( F = x - r\cos\theta, G = y - r\sin\theta \),则
\[
\left( \begin{matrix}
F_x & F_y & F_r & F_{\theta} \\
G_x & G_y & G_r & G_{\theta} \\
\end{matrix} \right) = \left( \begin{matrix}
1 & 0 & -\cos\theta & r\sin\theta \\
0 & 1 & -\sin\theta & -r\cos\theta \\
\end{matrix} \right).
\]
直接计算 Jacobi 行列式得
\[
J = \frac{\partial (F,G)}{\partial (r,\theta)} = \left| \begin{matrix}
-\cos\theta & r\sin\theta \\
-\sin\theta & -r\cos\theta \\
\end{matrix} \right| = r,
\]
于是
\[
\frac{\partial u}{\partial x} = \frac{\partial u}{\partial r} \cdot \frac{\partial r}{\partial x} + \frac{\partial u}{\partial \theta} \cdot \frac{\partial \theta}{\partial x} = \frac{\partial u}{\partial r} \cdot \left[ -\frac{1}{J}\frac{\partial (F,G)}{\partial (x,\theta)} \right] + \frac{\partial u}{\partial \theta} \cdot \left[ -\frac{1}{J}\frac{\partial (F,G)}{\partial (r,x)} \right] = \frac{\partial u}{\partial r} \cdot \cos\theta - \frac{\partial u}{\partial \theta} \cdot \frac{\sin\theta}{r};
\]
\[
\frac{\partial u}{\partial y} = \frac{\partial u}{\partial r} \cdot \frac{\partial r}{\partial y} + \frac{\partial u}{\partial \theta} \cdot \frac{\partial \theta}{\partial y} = \frac{\partial u}{\partial r} \cdot \left[ -\frac{1}{J}\frac{\partial (F,G)}{\partial (y,\theta)} \right] + \frac{\partial u}{\partial \theta} \cdot \left[ -\frac{1}{J}\frac{\partial (F,G)}{\partial (r,y)} \right] = \frac{\partial u}{\partial r} \cdot \sin\theta + \frac{\partial u}{\partial \theta} \cdot \frac{\cos\theta}{r};
\]
\[
\frac{\partial v}{\partial x} = \frac{\partial v}{\partial r} \cdot \frac{\partial r}{\partial x} + \frac{\partial v}{\partial \theta} \cdot \frac{\partial \theta}{\partial x} = \frac{\partial v}{\partial r} \cdot \left[ -\frac{1}{J}\frac{\partial (F,G)}{\partial (x,\theta)} \right] + \frac{\partial v}{\partial \theta} \cdot \left[ -\frac{1}{J}\frac{\partial (F,G)}{\partial (r,x)} \right] = \frac{\partial v}{\partial r} \cdot \cos\theta - \frac{\partial v}{\partial \theta} \cdot \frac{\sin\theta}{r};
\]
\[
\frac{\partial v}{\partial y} = \frac{\partial v}{\partial r} \cdot \frac{\partial r}{\partial y} + \frac{\partial v}{\partial \theta} \cdot \frac{\partial \theta}{\partial y} = \frac{\partial v}{\partial r} \cdot \left[ -\frac{1}{J}\frac{\partial (F,G)}{\partial (y,\theta)} \right] + \frac{\partial v}{\partial \theta} \cdot \left[ -\frac{1}{J}\frac{\partial (F,G)}{\partial (r,y)} \right] = \frac{\partial v}{\partial r} \cdot \sin\theta + \frac{\partial v}{\partial \theta} \cdot \frac{\cos\theta}{r}.
\]
从而
\[
\frac{\partial u}{\partial x} = \frac{\partial v}{\partial y} \Longleftrightarrow \frac{\partial u}{\partial r} \cdot \cos\theta - \frac{\partial u}{\partial \theta} \cdot \frac{\sin\theta}{r} = \frac{\partial v}{\partial r} \cdot \sin\theta + \frac{\partial v}{\partial \theta} \cdot \frac{\cos\theta}{r}
\]
\[
\Longleftrightarrow \frac{\partial u}{\partial r} \cdot r\cos\theta - \frac{\partial u}{\partial \theta} \cdot \sin\theta = \frac{\partial v}{\partial r} \cdot r\sin\theta + \frac{\partial v}{\partial \theta} \cdot \cos\theta,
\]
\[
\frac{\partial u}{\partial y} = -\frac{\partial v}{\partial x} \Longleftrightarrow \frac{\partial u}{\partial r} \cdot \sin\theta + \frac{\partial u}{\partial \theta} \cdot \frac{\cos\theta}{r} = -\frac{\partial v}{\partial r} \cdot \cos\theta + \frac{\partial v}{\partial \theta} \cdot \frac{\sin\theta}{r}
\]
\[
\Longleftrightarrow \frac{\partial u}{\partial r} \cdot r\sin\theta + \frac{\partial u}{\partial \theta} \cdot \cos\theta = -\frac{\partial v}{\partial r} \cdot r\cos\theta + \frac{\partial v}{\partial \theta} \cdot \sin\theta.
\]
由 (i) 可知 Cauchy-Riemann 方程等价于 \( \frac{\partial u}{\partial x} = \frac{\partial v}{\partial y}, \frac{\partial u}{\partial y} = -\frac{\partial v}{\partial x} \),故此时 Cauchy-Riemann 方程等价于
\[
\begin{cases}
\frac{\partial u}{\partial r} \cdot r\cos\theta - \frac{\partial u}{\partial \theta} \cdot \sin\theta = \frac{\partial v}{\partial r} \cdot r\sin\theta + \frac{\partial v}{\partial \theta} \cdot \cos\theta, \\
\frac{\partial u}{\partial r} \cdot r\sin\theta + \frac{\partial u}{\partial \theta} \cdot \cos\theta = -\frac{\partial v}{\partial r} \cdot r\cos\theta + \frac{\partial v}{\partial \theta} \cdot \sin\theta. \\
\end{cases}
\]
化简可得
\[
\begin{cases}
\frac{\partial u}{\partial r} = \frac{1}{r}\frac{\partial v}{\partial \theta}, \\
\frac{\partial v}{\partial r} = -\frac{1}{r}\frac{\partial u}{\partial \theta}. \\
\end{cases}
\]
\end{proof}

\begin{theorem}\label{theorem:复变函数可微的充要条件2}
设 \( f = u + \mathrm{i}v \) 是定义在域 \( D \) 上的函数,\( z_0 = x_0 + \mathrm{i}y_0 \in D \),那么 \( f \) 在 \( z_0 \) 处可微的充要条件是 \( u(x, y) \),\( v(x, y) \) 在 \( (x_0, y_0) \) 处可微,且在 \( (x_0, y_0) \) 处满足Cauchy-Riemann方程,即
\[
\frac{\partial f}{\partial \overline{z}} = 0,\quad \quad \frac{\partial \overline{f}}{\partial z}=0,\quad \begin{cases}
\dfrac{\partial u}{\partial x} = \dfrac{\partial v}{\partial y}, \\
\dfrac{\partial u}{\partial y} = -\dfrac{\partial v}{\partial x}
\end{cases}.
\]
在可微的情况下,有
\begin{align*}
f'(z_0) = \dfrac{\partial u}{\partial x} + \mathrm{i}\dfrac{\partial v}{\partial x}
= \dfrac{\partial v}{\partial y} + \mathrm{i}\dfrac{\partial v}{\partial x} = \dfrac{\partial u}{\partial x} - \mathrm{i}\dfrac{\partial u}{\partial y} = \dfrac{\partial v}{\partial y} - \mathrm{i}\dfrac{\partial u}{\partial y}.
\end{align*}
这里的偏导数都在 \( (x_0, y_0) \) 处取值。
\end{theorem}
\begin{proof}
最后这个 \( f'(z_0) \) 的表达式是从 \refthe{theorem:复变函数可微的充要条件1}中的\( f'(z_0) = \dfrac{\partial f}{\partial z}(z_0) \) 和 \hyperref[proposition:Cauchy-Riemann方程的等价定义]{Cauchy-Riemann方程的等价定义}得到的。
\end{proof}

\begin{definition}
\begin{enumerate}
\item 设 \( D \) 是 \( \mathbb{C} \) 中的域,我们用 \( C(D) \) 记 \( D \) 上连续函数的全体,用 \( H(D) \) 记 \( D \) 上全纯函数的全体.

\item 设 \( f = u + \mathrm{i}v \),记 \( \frac{\partial f}{\partial x} = \frac{\partial u}{\partial x} + \mathrm{i}\frac{\partial v}{\partial x} \),\( \frac{\partial f}{\partial y} = \frac{\partial u}{\partial y} + \mathrm{i}\frac{\partial v}{\partial y} \)。我们用 \( C^1(D) \) 记 \( \frac{\partial f}{\partial x} \),\( \frac{\partial f}{\partial y} \) 在 \( D \) 上连续的 \( f \) 的全体。

\item 用 \( C^k(D) \) 记在 \( D \) 上有 \( k \) 阶连续偏导数的函数的全体,\( C^\infty(D) \) 记在 \( D \) 上有任意阶连续偏导数的函数的全体。
\end{enumerate}
\end{definition}

\begin{proposition}
\begin{enumerate}[(1)]
\item \( H(D) \subset C(D) \).

\item $C^1(D) \subset C(D).$

\item 域 \( D \) 上的全纯函数在 \( D \) 上有任意阶的连续偏导数,并且有如下的包含关系:
\[
H(D) \subset C^\infty(D) \subset C^k(D) \subset C^1(D) \subset C(D).
\]
这里,\( k \) 是大于 1 的自然数。
\end{enumerate}
\end{proposition}
\begin{proof}
\begin{enumerate}[(1)]
\item \refpro{proposition:全纯函数必连续}告诉我们,\( H(D) \subset C(D) \)。

\item 设 \( f = u + \mathrm{i}v \),记 \( \frac{\partial f}{\partial x} = \frac{\partial u}{\partial x} + \mathrm{i}\frac{\partial v}{\partial x} \),\( \frac{\partial f}{\partial y} = \frac{\partial u}{\partial y} + \mathrm{i}\frac{\partial v}{\partial y} \)。我们用 \( C^1(D) \) 记 \( \frac{\partial f}{\partial x} \),\( \frac{\partial f}{\partial y} \) 在 \( D \) 上连续的 \( f \) 的全体。进而$u,v$关于$x,y$的偏导在$D$上都连续,由多元微积分的知识知道,$u,v$在$D$上都可微。于是对于任意 \( f \in C^1(D) \),\( f \) 在 \( D \) 上实可微,从\eqref{eq:-----2.2.4}式知道
\begin{align*}
f(z_0 + \Delta z) - f(z_0) &= \frac{\partial f}{\partial z}(z_0)\Delta z + \frac{\partial f}{\partial \overline{z}}(z_0)\overline{\Delta z} + o(|\Delta z|).
\end{align*}
令$\Delta z\to 0$,则$\lim_{\Delta z\to 0}f(z_0+\Delta z)=f(z_0)$,
故\( f \) 在 \( D \) 上连续,因而
$C^1(D) \subset C(D).$

\item 
\end{enumerate}
\end{proof}

\begin{example}
研究函数 $f(z) = z^n$,$n$ 是自然数.
\end{example}
\begin{solution}
显然,$\frac{\partial f}{\partial \overline{z}} = 0$,且 $f$ 在整个平面上是实可微的. 因而,$f$ 是 $\mathbb{C}$ 上的全纯函数,而且
\begin{align}
f'(z) &= \frac{\partial f}{\partial z} = nz^{n - 1}. \label{eq:2.2.5}
\end{align}
\end{solution}

\begin{example}
研究函数 $f(z) = \mathrm{e}^{-\vert z \vert^2}$.
\end{example}
\begin{solution}
把 $f$ 写为 $f(z) = \mathrm{e}^{-z\overline{z}}$,于是 $\frac{\partial f}{\partial \overline{z}} = -\mathrm{e}^{-z\overline{z}}z$,它只有在 $z = 0$ 处才等于零. 因此,$\mathrm{e}^{-\vert z \vert^2}$ 只有在 $z = 0$ 处可微,它在任何点处都不是全纯的. 但它对 $x, y$ 有任意阶连续偏导数,所以它是 $C^\infty(\mathbb{C})$ 中的函数.
\end{solution}

\begin{definition}[调和函数]
设 $u$ 是域 $D$ 上的实值函数,如果 $u \in C^2(D)$,且对任意 $z \in D$,有
\begin{align}
\Delta u(z) &= \frac{\partial^2 u(z)}{\partial x^2} + \frac{\partial^2 u(z)}{\partial y^2} = 0, \label{eq:2.2.7}
\end{align}
就称 $u$ 是 $D$ 中的\textbf{调和函数}. $\Delta = \frac{\partial^2}{\partial x^2} + \frac{\partial^2}{\partial y^2}$ 称为 $\mathbf{Laplace}$\textbf{算子}.
\end{definition}

\begin{proposition}
设 $u \in C^2(D)$,那么 $\Delta u = 4 \frac{\partial^2 u}{\partial z \partial \overline{z}}$.
\end{proposition}
\begin{proof}
由\eqref{eq:-----2.2.3}式,有
\begin{align*}
\frac{\partial}{\partial z} &= \frac{1}{2}\left( \frac{\partial}{\partial x} - \text{i}\frac{\partial}{\partial y} \right), \\
\frac{\partial}{\partial \overline{z}} &= \frac{1}{2}\left( \frac{\partial}{\partial x} + \text{i}\frac{\partial}{\partial y} \right),
\end{align*}
所以
\begin{align*}
\frac{\partial^2 u}{\partial z \partial \overline{z}} &= \frac{\partial}{\partial \overline{z}} \frac{\partial u}{\partial z} = \frac{1}{4} \left[ \frac{\partial}{\partial x} \left( \frac{\partial u}{\partial x} - \mathrm{i} \frac{\partial u}{\partial y} \right) + \mathrm{i} \frac{\partial}{\partial y} \left( \frac{\partial u}{\partial x} - \mathrm{i} \frac{\partial u}{\partial y} \right) \right] \\
&= \frac{1}{4} \left( \frac{\partial^2 u}{\partial x^2} + \frac{\partial^2 u}{\partial y^2} \right) = \frac{1}{4} \Delta u.
\end{align*}
\end{proof}

\begin{theorem}\label{theorem:全纯函数的实部和虚部都是调和函数}
设 $f = u + \mathrm{i}v \in H(D)$,那么 $u$ 和 $v$ 都是 $D$ 上的调和函数.
\end{theorem}
\begin{proof}
因为 $f \in H(D)$,由\refthe{theorem:复变函数可微的充要条件2},有
\begin{align*}
\frac{\partial f}{\partial \overline{z}} = 0, \quad
\frac{\partial \overline{f}}{\partial z} = 0.
\end{align*}
所以
\begin{align*}
\frac{\partial^2 f}{\partial z \partial \overline{z}} &= \frac{\partial^2 \overline{f}}{\partial z \partial \overline{z}} = 0.
\end{align*}
于是,由 $u = \frac{1}{2}(f + \overline{f})$ 即得
\begin{align*}
\Delta u &= 4 \frac{\partial^2 u}{\partial z \partial \overline{z}} = 0.
\end{align*}

同理可证 $\Delta v = 0$.
\end{proof}

\begin{definition}[共轭调和函数]
设 $u$ 和 $v$ 是 $D$ 上的一对调和函数,如果它们还满足 Cauchy-Riemann 方程
\begin{align}
\begin{cases} 
\displaystyle \frac{\partial u}{\partial x} = \frac{\partial v}{\partial y}, \\
\displaystyle \frac{\partial u}{\partial y} = -\frac{\partial v}{\partial x},
\end{cases} \label{eq:2.2.10}
\end{align}
就称 $v$ 为 $u$ 的\textbf{共轭调和函数}.
\end{definition}

\begin{proposition}
全纯函数的实部和虚部就构成一对共轭调和函数.
\end{proposition}
\begin{proof}
由\refthe{theorem:全纯函数的实部和虚部都是调和函数}和\refthe{theorem:复变函数可微的充要条件2}立得.
\end{proof}

\begin{theorem}
设 $u$ 是单连通域 $D$ 上的调和函数,则必存在 $u$ 的共轭调和函数 $v$,使得 $u + \mathrm{i}v$ 是 $D$ 上的全纯函数.
\end{theorem}
\begin{proof}
因为 $u$ 满足 Laplace 方程
\begin{align*}
\frac{\partial^2 u}{\partial x^2} + \frac{\partial^2 u}{\partial y^2} &= 0,
\end{align*}
若令 $P = -\frac{\partial u}{\partial y}, Q = \frac{\partial u}{\partial x}$,则
\begin{align*}
\frac{\partial Q}{\partial x} &= \frac{\partial^2 u}{\partial x^2} = -\frac{\partial^2 u}{\partial y^2} = \frac{\partial P}{\partial y},
\end{align*}
所以
\begin{align*}
P \mathrm{d}x + Q \mathrm{d}y &= -\frac{\partial u}{\partial y} \mathrm{d}x + \frac{\partial u}{\partial x} \mathrm{d}y
\end{align*}
是一个全微分,因而积分
\begin{align*}
\int_{(x_0, y_0)}^{(x, y)} -\frac{\partial u}{\partial y} \mathrm{d}x + \frac{\partial u}{\partial x} \mathrm{d}y
\end{align*}
与路径无关. 令
\begin{align*}
v(x, y) = \int_{(x_0, y_0)}^{(x, y)} -\frac{\partial u}{\partial y} \mathrm{d}x + \frac{\partial u}{\partial x} \mathrm{d}y,
\end{align*}
则
\begin{align*}
v(x,y)&=\int_{(x_0,y_0)}^{(x,y_0)}{-\frac{\partial u}{\partial y}\mathrm{d}x+\frac{\partial u}{\partial x}\mathrm{d}y}+\int_{(x,y_0)}^{(x,y)}{-\frac{\partial u}{\partial y}\mathrm{d}x+\frac{\partial u}{\partial x}\mathrm{d}y}
\\
&=\int_{x_0}^x{-\frac{\partial u}{\partial y}\mathrm{d}x}+0+0+\int_{y_0}^y{\frac{\partial u}{\partial x}\mathrm{d}y}
\\
&=\int_{x_0}^x{-\frac{\partial u}{\partial y}\mathrm{d}x}+\int_{y_0}^y{\frac{\partial u}{\partial x}\mathrm{d}y}.
\end{align*}
那么
\begin{align*}
\begin{cases} 
\displaystyle \frac{\partial v}{\partial x} = -\frac{\partial u}{\partial y}, \\
\displaystyle \frac{\partial v}{\partial y} = \frac{\partial u}{\partial x}.
\end{cases}
\end{align*}
所以,$v$ 就是要求的 $u$ 的共轭调和函数.
\end{proof}






\end{document}
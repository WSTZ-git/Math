\documentclass[../../main.tex]{subfiles}
\graphicspath{{\subfix{../../image/}}} % 指定图片目录,后续可以直接使用图片文件名。

% 例如:
% \begin{figure}[H]
% \centering
% \includegraphics[scale=0.4]{图.png}
% \caption{}
% \label{figure:图}
% \end{figure}
% 注意:上述\label{}一定要放在\caption{}之后,否则引用图片序号会只会显示??.

\begin{document}

\section{复变函数的导数}

\begin{definition}
设 \( f:D \to \mathbb{C} \) 是定义在域 \( D \) 上的函数,\( z_0 \in D \)。如果极限
\begin{align}
\lim_{z \to z_0} \frac{f(z) - f(z_0)}{z - z_0}. \label{eq:2.1.1}
\end{align}
存在,就说 \( f \) 在 \( z_0 \) 处\textbf{复可微}或\textbf{可微},这个极限称为 \( f \) 在 \( z_0 \) 处的\textbf{导数}或\textbf{微商},记作 \( f'(z_0) \)。如果 \( f \) 在 \( D \) 中每点都可微,就称 \( f \) 是域 \( D \) 中的\textbf{全纯函数}或\textbf{解析函数}。如果 \( f \) 在 \( z_0 \) 的一个邻域中全纯,就称 \( f \) 在 \( z_0 \) 处\textbf{全纯}。
\end{definition}

\begin{proposition}\label{proposition:全纯函数必连续}
若 \( f \) 在 \( z_0 \) 处可微,则必在 \( z_0 \) 处连续。
\end{proposition}
\begin{remark}
但反过来不成立,即若 \( f \) 在 \( z_0 \) 处连续,则 \( f \) 未必在 \( z_0 \) 处可微。
\end{remark}
\begin{proof}
设 \( f \) 在 \( z_0 \) 处可微。若记 \( \Delta z = z - z_0 \),则 \eqref{eq:2.1.1} 式可以写成
\begin{align*}
\lim_{\Delta z \to 0} \frac{f(z_0 + \Delta z) - f(z_0)}{\Delta z} = f'(z_0)
\end{align*}
或者
\begin{align}
f(z_0 + \Delta z) - f(z_0) = f'(z_0)\Delta z + o(|\Delta z|) \label{eq:2.1.2}
\end{align}
由此即得 \( \lim_{\Delta z \to 0} f(z_0 + \Delta z) = f(z_0) \),这说明 \( f \) 在 \( z_0 \) 处连续。
\end{proof}

\begin{example}
函数 \( f(z) = \overline{z} \) 在 \( \mathbb{C} \) 中处处不可微。
\end{example}
\begin{note}
但容易看出这个函数在 \( \mathbb{C} \) 中却是处处连续的,这是一个处处连续、处处不可微的例子。其实,在复变函数中这种例子很多,例如 \( f(z) = \text{Re}z \),\( f(z) = |z| \) 都是。但在实变函数中,要举一个这样的例子却是相当困难的。这说明在复变函数中可微的要求比实变函数中要强得多,因而得到的结论也强得多,这在以后的学习中将逐步揭示出来。
\end{note}
\begin{proof}
对于任意 \( z \in \mathbb{C} \),有
\begin{align*}
\frac{f(z + \Delta z) - f(z)}{\Delta z} = \frac{\overline{z + \Delta z} - \overline{z}}{\Delta z} = \frac{\overline{\Delta z}}{\Delta z}
\end{align*}
如果让 \( \Delta z \) 取实数,则 \( \frac{\overline{\Delta z}}{\Delta z} = 1 \);如果让 \( \Delta z \) 取纯虚数,则 \( \frac{\overline{\Delta z}}{\Delta z} = -1 \)。因此,当 \( \Delta z \to 0 \) 时上述极限不存在,因而在 \( \mathbb{C} \) 中处处不可导。
\end{proof}

\begin{proposition}\label{proposition:复变函数求导运算法则}
\begin{enumerate}[(1)]
\item 若 \( f \) 和 \( g \) 在域 \( D \) 中全纯,那么 \( f \pm g \),\( fg \) 也在 \( D \) 中全纯,而且
\begin{gather*}
(f(z) \pm g(z))' = f'(z) \pm g'(z), \\
(f(z)g(z))' = f'(z)g(z) + f(z)g'(z).
\end{gather*}
如果对每一点 \( z \in D \),\( g(z) \neq 0 \),那么 \( \frac{f}{g} \) 也是 \( D \) 中的全纯函数,而且
\begin{align*}
\left( \frac{f(z)}{g(z)} \right)' &= \frac{f'(z)g(z) - g'(z)f(z)}{(g(z))^2}.
\end{align*}

\item 设 \( D_1, D_2 \) 是 \( \mathbb{C} \) 中的两个域,且
\begin{align*}
f: D_1 &\to D_2, \\
g: D_2 &\to \mathbb{C}
\end{align*}
都是全纯函数,那么 \( h = g \circ f \) 是 \( D_1 \to \mathbb{C} \) 的全纯函数,而且 \( h'(z) = g'(f(z))f'(z) \)。这里,\( g \circ f \) 记 \( f \) 和 \( g \) 的复合函数:\( g \circ f(z) = g(f(z)) \)。
\end{enumerate}
\end{proposition}
\begin{proof}
\begin{enumerate}[(1)]
\item 

\item 
\end{enumerate}
\end{proof}










\end{document}
\documentclass[../../main.tex]{subfiles}
\graphicspath{{\subfix{../../image/}}} % 指定图片目录,后续可以直接使用图片文件名。

% 例如:
% \begin{figure}[H]
% \centering
% \includegraphics[scale=0.4]{图.png}
% \caption{}
% \label{figure:图}
% \end{figure}
% 注意:上述\label{}一定要放在\caption{}之后,否则引用图片序号会只会显示??.

\begin{document}

\section{导数的几何意义}

\begin{proposition}
过 \( z_0 \) 作一条光滑曲线 \( \gamma \),它的方程为  
\[ z = \gamma(t),\ a \leqslant t \leqslant b. \]  
设 \( \gamma(a) = z_0 \),且 \( \gamma'(a) \neq 0 \)。设 \( w = f(z) \) 把曲线 \( \gamma \) 映为 \( \sigma \),它的方程为  
\[ w = \sigma(t) = f(\gamma(t)),\ a \leqslant t \leqslant b. \]  
则 \( \sigma \) 在 \( w_0 = f(z_0) \) 处的切线与正实轴的夹角为  
\[ \mathrm{Arg}\sigma'(a) = \mathrm{Arg} f'(z_0) + \mathrm{Arg} \gamma'(a), \]  
或者写为  
\begin{align}
\mathrm{Arg} \sigma'(a) - \mathrm{Arg} \gamma'(a) = \mathrm{Arg} f'(z_0). \label{equation----:::2.1}
\end{align}
\( \mathrm{Arg} f'(z_0) \) 就称为映射 \( w = f(z) \) 在点 \( z_0 \) 处的\textbf{转动角}。
\end{proposition}
\begin{note}
这说明像曲线 \( \sigma \) 在 \( w_0 \) 处的切线与正实轴的夹角与原曲线 \( \gamma \) 在 \( z_0 \) 处的切线与正实轴的夹角之差总是 \( \mathrm{Arg} f'(z_0) \),而与曲线 \( \gamma \) 无关。
\end{note}
\begin{proof}
由\refdef{definition:复平面上的光滑曲线}可知,\( \gamma \) 在点 \( z_0 \) 处的切线与正实轴的夹角为 \( \mathrm{Arg} \gamma'(a) \)。由于 \( \sigma'(a) = f'(\gamma(a))\gamma'(a) = f'(z_0)\gamma'(a) \neq 0 \),所以再结合\nrefthe{theorem:复数辐角的性质}{(1)}可得 \( \sigma \) 在 \( w_0 = f(z_0) \) 处的切线与正实轴的夹角为  
\[ \mathrm{Arg}\sigma'(a) = \mathrm{Arg} f'(z_0) + \mathrm{Arg} \gamma'(a), \]  
\end{proof}

\begin{definition}
若\( w = f(z) \)是定义域为$D$的复变函数,并且在 \( z_0 \) 处满足:  
过 \( z_0 \) 的任意两条曲线 \( C_1, C_2 \) 经映射后得到的曲线 \( \Gamma_1, \Gamma_2 \),其夹角(包括大小和方向)与原曲线 \( C_1, C_2 \) 的夹角相等。  
则称 \( z_0 \) 为 \( f(z) \) 的\textbf{保角点},也称$f$在$z_0$点是\textbf{保角的}.若\(f(z)\)在$D$内所有点都是保角的,则称\(f(z)\)是$D$上的\textbf{保角变换}.
\end{definition}
\begin{remark}
夹角的“方向”指从曲线 \( C_1 \) 到 \( C_2 \) 的旋转方向,与映射后从 \( \Gamma_1 \) 到 \( \Gamma_2 \) 的旋转方向一致,即保持“定向”。
\end{remark}

\begin{theorem}\label{theorem:全纯函数在其导数不为零的点处是保角的}
全纯函数在其导数不为零的点处是保角的。
\end{theorem}
\begin{proof}
如果过 \( z_0 \) 点作两条光滑曲线 \( \gamma_1, \gamma_2 \),它们的方程分别为  
\[ z = \gamma_1(t),\ a \leqslant t \leqslant b \text{和}  
z = \gamma_2(t),\ a \leqslant t \leqslant b, \]  
且 \( \gamma_1(a) = \gamma_2(a) = z_0 \)(\reffig{figure:image-图2.1}(a))。映射 \( w = f(z) \) 把它们分别映为过 \( w_0 \) 点的两条光滑曲线 \( \sigma_1 \) 和 \( \sigma_2 \)(\reffig{figure:image-图2.1}(b)),它们的方程分别为  
\[ w = \sigma_1(t) = f(\gamma_1(t)),\ a \leqslant t \leqslant b \text{和} w = \sigma_2(t) = f(\gamma_2(t)),\ a \leqslant t \leqslant b. \]  

由 \eqref{equation----:::2.1}式可得  
\begin{align}
\mathrm{Arg} \sigma_1'(a) - \mathrm{Arg} \gamma_1'(a) = \mathrm{Arg} f'(z_0) \notag = \mathrm{Arg} \sigma_2'(a) - \mathrm{Arg} \gamma_2'(a), \notag
\end{align}
即  
\begin{align}
\mathrm{Arg} \sigma_2'(a) - \mathrm{Arg} \sigma_1'(a) = \mathrm{Arg} \gamma_2'(a) - \mathrm{Arg} \gamma_1'(a). \label{equation-----:::2.2}
\end{align}
\begin{figure}[H]
\centering
\includegraphics[scale=0.4]{
图2.1.png}
\caption{}
\label{figure:image-图2.1}
\end{figure}
上式左端是曲线 \( \sigma_1 \) 和 \( \sigma_2 \) 在 \( w_0 \) 处的夹角(两条曲线在某点的夹角定义为这两条曲线在该点的切线的夹角),右端是曲线 \( \gamma_1 \) 和 \( \gamma_2 \) 在 \( z_0 \) 处的夹角。\eqref{equation-----:::2.2}式说明,如果 \( f'(z_0) \neq 0 \),那么在映射 \( w = f(z) \) 的作用下,过 \( z_0 \) 点的任意两条光滑曲线的夹角的大小与旋转方向都是保持不变的.
\end{proof}

\begin{corollary}
设\( w = f(z) \)是定义域为$D$的复变函数,则$f$在$z_0$上是保角的的充要条件是\( f(z) \) 在$D$上全纯且 \( f'(z_0) \neq 0 \).
\end{corollary}
\begin{proof}
由\refthe{theorem:全纯函数在其导数不为零的点处是保角的}立得.
\end{proof}

\begin{definition}[伸缩率]
过 \( z_0 \) 作一条光滑曲线 \( \gamma \),它的方程为  
\[ z = \gamma(t),\ a \leqslant t \leqslant b. \]  
设 \( \gamma(a) = z_0 \),且 \( \gamma'(a) \neq 0 \)。设 \( w = f(z) \) 把曲线 \( \gamma \) 映为 \( \sigma \)(\reffig{figure:image-图2.2}),它的方程为  
\[ w = \sigma(t) = f(\gamma(t)),\ a \leqslant t \leqslant b. \] 
称 \( |f'(z_0)| \) 为 \( f \) 在 \( z_0 \) 处的\textbf{伸缩率}.
\end{definition}
\begin{note}
由于
\[
\lim_{z \to z_0} \frac{f(z) - f(z_0)}{z - z_0} = f'(z_0),
\]
所以,当 \( z \) 沿着 \( \gamma \) 趋于 \( z_0 \) 时,有
\[
\lim_{z \to z_0} \frac{|f(z) - f(z_0)|}{|z - z_0|} = \lim_{z \to z_0} \frac{|w - w_0|}{|z - z_0|} = |f'(z_0)|.
\]
这说明像点之间的距离与原像之间的距离之比只与 \( z_0 \) 有关,而与曲线 \( \gamma \) 无关. 
\end{note}

\begin{figure}[H]
\centering
\includegraphics[scale=0.4]{
图2.2.png}
\caption{}
\label{figure:image-图2.2}
\end{figure}




\end{document}
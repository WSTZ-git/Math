\documentclass[../../main.tex]{subfiles}
\graphicspath{{\subfix{../../image/}}} % 指定图片目录,后续可以直接使用图片文件名。

% 例如:
% \begin{figure}[H]
% \centering
% \includegraphics[scale=0.4]{图.png}
% \caption{}
% \label{figure:图}
% \end{figure}
% 注意:上述\label{}一定要放在\caption{}之后,否则引用图片序号会只会显示??.

\begin{document}

\section{分式线性变换}\label{section:2.2.5}






\begin{example}\label{example:例2.5.16}
求一分式线性变换,把单位圆的内部变成单位圆的内部,而且把圆内指定的点 \( a \) 变为圆心。
\end{example}
\begin{remark}
这个变换十分重要,它是把单位圆盘一一地变为自己的变换,称为单位圆盘的\textbf{全纯自同构}。以后我们将证明(\refthe{theorem:定理4.5.5}),把单位圆盘一一地变为自己的全纯映射只能是这种样子,再没有其他的变换。
\end{remark}
\begin{proof}
因为 \( a \) 关于单位圆的对称点是 \( \frac{1}{\overline{a}} \),所以这个变换把 \( a \) 和 \( \frac{1}{\overline{a}} \) 分别变为 \( 0 \) 和 \( \infty \),故这个变换可写成
\[
\begin{aligned}
w = \lambda \frac{z - a}{z - \frac{1}{\overline{a}}} = -\lambda \overline{a} \frac{z - a}{1 - \overline{a} z} = \mu \frac{z - a}{1 - \overline{a} z}.
\end{aligned}
\]
为了把单位圆周变成单位圆周,即将满足 \( |z| = 1 \) 的 \( z \) 变为满足 \( |w| = 1 \) 的 \( w \),\( \mu \) 必须满足
\[
\begin{aligned}
1 = |w| = |\mu| \frac{|z - a|}{|1 - \overline{a} z|} = |\mu| \frac{|z - a|}{|z| |\overline{z} - \overline{a}|} = |\mu|,
\end{aligned}
\]
即 \( \mu = \mathrm{e}^{\mathrm{i}\theta} \)。故所求的变换为
\[
w = \mathrm{e}^{\mathrm{i}\theta} \frac{z - a}{1 - \overline{a} z}.
\] 
\end{proof}

\end{document}
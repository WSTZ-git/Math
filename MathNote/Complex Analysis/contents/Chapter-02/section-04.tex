\documentclass[../../main.tex]{subfiles}
\graphicspath{{\subfix{../../image/}}} % 指定图片目录,后续可以直接使用图片文件名。

% 例如:
% \begin{figure}[H]
% \centering
% \includegraphics[scale=0.4]{图.png}
% \caption{}
% \label{figure:图}
% \end{figure}
% 注意:上述\label{}一定要放在\caption{}之后,否则引用图片序号会只会显示??.

\begin{document}

\section{初等全纯函数}

\subsection{指数函数}

\begin{definition}[指数函数]
设 \( z = x + \mathrm{i}y \),定义
\[
\mathrm{e}^z = \mathrm{e}^x (\cos y + \mathrm{i}\sin y).
\]
\end{definition}

\begin{proposition}\label{proposition:指数函数的性质}
\begin{enumerate}[(i)]
\item \( \mathrm{e}^z \) 是 \( \mathbb{C} \) 上的全纯函数,而且
\[
(\mathrm{e}^z)' = \mathrm{e}^z.
\]

\item 对任意\( z \in \mathbf{C} \),都有\( z = r(\cos\theta + \mathrm{i}\sin\theta)= r\mathrm{e}^{\mathrm{i}\theta} \).

\item 对于任意 \( z \in \mathbf{C} \),\( \mathrm{e}^z \neq 0 \). 

\item 对于任意 \( z_1, z_2 \),有
\[
\mathrm{e}^{z_1} \mathrm{e}^{z_2} = \mathrm{e}^{z_1 + z_2}.
\]

\item \( \mathrm{e}^z \) 是以 \( 2\pi\mathrm{i} \) 为周期的周期函数.
\end{enumerate}
\end{proposition}
\begin{proof}
\begin{enumerate}[(i)]
\item \( \mathrm{e}^z \) 在 \( \mathbb{C} \) 上每点实可微是显然的. 今验证它满足 Cauchy-Riemann 方程. 因为 \( u(x, y) = \mathrm{e}^x \cos y \),\( v(x, y) = \mathrm{e}^x \sin y \),所以
\[
\frac{\partial u}{\partial x} = \mathrm{e}^x \cos y = \frac{\partial v}{\partial y},
\]
\[
\frac{\partial u}{\partial y} = -\mathrm{e}^x \sin y = -\frac{\partial v}{\partial x}.
\]
故由\refthe{theorem:复变函数可微的充要条件2},\( \mathrm{e}^z \) 在 \( \mathbb{C} \) 上全纯,而且
\begin{align*}
(\mathrm{e}^z)' = \frac{\partial u}{\partial x} + \mathrm{i} \frac{\partial v}{\partial x}= \mathrm{e}^x \cos y + \mathrm{i} \mathrm{e}^x \sin y= \mathrm{e}^z.
\end{align*}

\item 当 \( z = x \) 时,即 \( y = 0 \),因而有 \( \mathrm{e}^z = \mathrm{e}^x \);当 \( z = \mathrm{i}y \) 时,\( \mathrm{e}^{\mathrm{i}y} = \cos y + \mathrm{i}\sin y \). 这样,复数的三角表示 \( z = r(\cos\theta + \mathrm{i}\sin\theta) \) 就可简单地写为 \( z = r\mathrm{e}^{\mathrm{i}\theta} \).

\item 这是因为$|\mathrm{e}^z| = \mathrm{e}^x > 0.$

\item 设 \( z_1 = x_1 + \mathrm{i}y_1 \),\( z_2 = x_2 + \mathrm{i}y_2 \),直接计算即得
\begin{align*}
\mathrm{e}^{z_1} \mathrm{e}^{z_2} &= \mathrm{e}^{x_1} (\cos y_1 + \mathrm{i}\sin y_1) \mathrm{e}^{x_2} (\cos y_2 + \mathrm{i}\sin y_2)\\
&= \mathrm{e}^{x_1 + x_2} (\cos(y_1 + y_2) + \mathrm{i}\sin(y_1 + y_2))\\
&= \mathrm{e}^{z_1 + z_2}.
\end{align*}

\item 注意到
\begin{align*}
\mathrm{e}^{z + 2\pi\mathrm{i}} = \mathrm{e}^{x + \mathrm{i}(y + 2\pi)}= \mathrm{e}^x (\cos(y + 2\pi) + \mathrm{i}\sin(y + 2\pi))= \mathrm{e}^z.
\end{align*}
\end{enumerate}
\end{proof}

\begin{definition}\label{definition:单叶性域}
设 \( f:D \to \mathbb{C} \) 是一个复变函数,如果对域 \( D \) 中任意两点 \( z_1, z_2 \)(\( z_1 \neq z_2 \)),必有 \( f(z_1) \neq f(z_2) \),就称 \( f \) 在 \( D \) 中是\textbf{单叶的},\( D \) 称为 \( f \) 的\textbf{单叶性域}.
\end{definition}

\begin{proposition}
如果 \( f \) 在 \( D \) 中是单叶的,\( f(D) = G \),那么 \( f \) 是 \( D \) 到 \( G \) 之上的一一映射.
\end{proposition}
\begin{proof}
由\hyperref[definition:单叶性域]{单叶的定义}和双射的定义立得.
\end{proof}

\begin{proposition}\label{proposition:复变函数的指数函数的单叶性域}
求 \( w = \mathrm{e}^z \) 的单叶性域.
\end{proposition}
\begin{solution}
如果 \( z_1 = x_1 + \mathrm{i}y_1 \),\( z_2 = x_2 + \mathrm{i}y_2 \) 使得 \( \mathrm{e}^{z_1} = \mathrm{e}^{z_2} \),即 \( \mathrm{e}^{x_1} \mathrm{e}^{\mathrm{i}y_1} = \mathrm{e}^{x_2} \mathrm{e}^{\mathrm{i}y_2} \),那么 \( x_1 = x_2 \),\( y_1 = y_2 + 2k\pi \),\( k \) 是任意整数,也即 \( z_1 - z_2 = 2k\pi \mathrm{i} \). 这就是说,凡是不包含满足条件 \( z_1 - z_2 = 2k\pi \mathrm{i} \) 的 \( z_1, z_2 \) 的域都是 \( w = \mathrm{e}^z \) 的单叶性域. 
\end{solution}

\begin{proposition}
域
\[
\{ z = x + \mathrm{i}y : 2k\pi < y < 2(k + 1)\pi \}, \, k = 0, \pm 1, \cdots
\]
都是 \( \mathrm{e}^z \) 的单叶性域,它是平行于实轴、宽度为 \( 2\pi \) 的带状域.
\end{proposition}
\begin{note}
由于 \( \mathrm{e}^z \) 是以 \( 2\pi \mathrm{i} \) 为周期的函数,我们只要弄清 \( \mathrm{e}^z \) 在域 \( \{ z = x + \mathrm{i}y : 0 < y < 2\pi \} \) 中的映射性质,那么在其他带状域中的性质是一样的.
\end{note}
\begin{proof}
由\refpro{proposition:复变函数的指数函数的单叶性域}不难验证.
\end{proof}

\begin{proposition}
设$w=e^z$,考虑平行于实轴的直线 \( \mathrm{Im}z = y_0 \). 这条直线上的点的方程为
\[
z = x + \mathrm{i}y_0, \, -\infty < x < \infty,
\]
则\( w = \mathrm{e}^z \) 把平行于实轴的直线\( \mathrm{Im}z = y_0 \)变成一条从原点出发的半射线,它与实轴正方向的夹角是 \( y_0 \)(\reffig{figure:图2.3});

\( w = \mathrm{e}^z \) 把带状域 \( \{ z = x + \mathrm{i}y : 0 < y < 2\pi \} \) 变成全平面除掉正实轴的域 \( \mathbb{C} \setminus \{ z : z \geqslant  0 \} \);

\( w = \mathrm{e}^z \) 把直线 \( \mathrm{Im}z = 0 \) 变成正实轴的上岸,直线 \( \mathrm{Im}z = 2\pi \) 变成正实轴的下岸;

\( w = \mathrm{e}^z \) 把带状域 \( \{ z = x + \mathrm{i}y : 0 < y < \pi \} \) 变成上半平面,带状域 \( \{ z = x + \mathrm{i}y : \pi < y < 2\pi \} \) 变成下半平面. 

一般来说,\( w = \mathrm{e}^z \) 把带状域 \( \{ z = x + \mathrm{i}y : \alpha < y < \beta, 0 < \alpha < \beta \leqslant  2\pi \} \) 变成角状域 \( \alpha < \arg w < \beta \).
\end{proposition}
\begin{proof}
由条件可知
\[
w = \mathrm{e}^z = \mathrm{e}^x \mathrm{e}^{\mathrm{i}y_0}.
\]
这是一条从原点出发的半射线,它与实轴正方向的夹角是 \( y_0 \)(\reffig{figure:图2.3}). 当 \( y_0 \) 从 \( 0 \) 变到 \( 2\pi \) 时,这条半射线的辐角也从 \( 0 \) 变到 \( 2\pi \). 因此,\( w = \mathrm{e}^z \) 把带状域 \( \{ z = x + \mathrm{i}y : 0 < y < 2\pi \} \) 变成全平面除掉正实轴的域 \( \mathbb{C} \setminus \{ z : z \geqslant  0 \} \),直线 \( \mathrm{Im}z = 0 \) 变成正实轴的上岸,直线 \( \mathrm{Im}z = 2\pi \) 变成正实轴的下岸;带状域 \( \{ z = x + \mathrm{i}y : 0 < y < \pi \} \) 变成上半平面,带状域 \( \{ z = x + \mathrm{i}y : \pi < y < 2\pi \} \) 变成下半平面. 一般来说,\( w = \mathrm{e}^z \) 把带状域 \( \{ z = x + \mathrm{i}y : \alpha < y < \beta, 0 < \alpha < \beta \leqslant  2\pi \} \) 变成角状域 \( \alpha < \arg w < \beta \).
\end{proof}

\begin{figure}[H]
\centering
\includegraphics[scale=0.3]{图2.3.png}
\caption{}
\label{figure:图2.3}
\end{figure}



\subsection{对数函数}

\begin{definition}
对于给定的 \( z \in \mathbb{C} \),满足方程 \( \mathrm{e}^w = z \) 的 \( w \) 称为 \( z \) 的\textbf{对数},记为 \( w = \mathrm{Log} z \)。
\end{definition}

\begin{proposition}
设 \( z = r\mathrm{e}^{\mathrm{i}\theta} \),\( w = u + \mathrm{i}v \),并且满足方程\( \mathrm{e}^w = z \) ,则\( \mathrm{e}^u = r \),\( v = \theta + 2k\pi \)。于是
\begin{align*}
\mathrm{Log} z = \log|z| + \mathrm{i}\arg z + 2k\pi\mathrm{i} = \log|z| + \mathrm{i}\mathrm{Arg} z.
\end{align*}
由此可知,\( \mathrm{Log} z \) 是一个多值函数,它的多值性是由 \( z \) 的辐角 \( \mathrm{Arg} z \) 的多值性产生的.
\end{proposition}
\begin{proof}
由条件可知\( \mathrm{e}^{u + \mathrm{i}v} = r\mathrm{e}^{\mathrm{i}\theta} \),因而 \( \mathrm{e}^u = r \),\( v = \theta + 2k\pi \)。于是
\begin{align*}
\mathrm{Log} z = \log|z| + \mathrm{i}\arg z + 2k\pi\mathrm{i} = \log|z| + \mathrm{i}\mathrm{Arg} z.
\end{align*}
\end{proof}

\begin{theorem}\label{theorem:单值全纯分支}
如果 \( D \) 是不包含原点和无穷远点的单连通域,则必在 \( D \) 上存在无穷多个单值全纯函数 \( \varphi_k, k = 0, \pm 1, \cdots \),使得在 \( D \) 上成立
\[
\mathrm{e}^{\varphi_k(z)} = z, \, k = 0, \pm 1, \cdots;
\]
而且对每一个 \( k \),有 \( \varphi_k'(z) = \frac{1}{z} \)。其中的每一个 \( \varphi_k \) 都称为 \( \mathrm{Log} z \) 在 \( D \) 上的\textbf{单值全纯分支}。
\end{theorem}
\begin{remark}
现在说明为什么要求 \( D \) 不包含原点和无穷远点。如果 \( D \) 包含原点,那么 \( D \) 中就包含绕原点 \( z = 0 \) 的简单闭曲线 \( \gamma \),当 \( z \) 从 \( \gamma \) 上的一点 \( z_0 \) 沿 \( \gamma \) 的正方向(即反时针方向)回到 \( z_0 \) 时,\( z \) 的辐角增加了 \( 2\pi \),\( \varphi_{k_0} \) 的值从 \( \varphi_{k_0}(z_0) \) 连续地变为 \( \varphi_{k_0 + 1}(z_0) \),而不再回到原来的值 \( \varphi_{k_0}(z_0) \)。因此,在这样的域中就不可能从 \( \mathrm{Log} z \) 中分出单值的全纯分支。因为 \( D \) 内任意一条绕原点的简单闭曲线也可以看作是绕无穷远点的简单闭曲线,因此 \( D \) 也不能包含无穷远点.
\end{remark}
\begin{proof}
对给定的 \( z \),选定它的辐角 \( \theta = \theta_0 + 2k_0\pi \),则$z=r\mathrm{e}^{\mathrm{i}\theta}$.这里,\( \theta_0 \) 是 \( z \) 的辐角的主值,即 \( \theta_0 = \arg z \),\( k_0 \) 是任意一个给定的整数。在 \( D \) 上定义
\[
\varphi_{k_0}(z) = \log|z| + \mathrm{i}(\theta_0 + 2k_0\pi) = \log r + \mathrm{i}\theta,
\]
这时,\( u = \log r \),\( v = \theta \)。容易验证这时有
\[
\frac{\partial u}{\partial r} = \frac{1}{r} \frac{\partial v}{\partial \theta},\quad 
\frac{\partial u}{\partial \theta} = -r \frac{\partial v}{\partial r},
\]
因此由\nrefpro{proposition:Cauchy-Riemann方程的等价定义}{(iii)}知道,\( \varphi_{k_0} \) 是 \( D \) 上的全纯函数,而且由\refthe{theorem:复变函数可微的充要条件2}及\nrefpro{proposition:Cauchy-Riemann方程的等价定义}{(iii)的证明过程}可得
\begin{align*}
\varphi _{k_0}'(z)&=\frac{\partial u}{\partial x}+\mathrm{i}\frac{\partial v}{\partial x}=\left( \frac{\partial u}{\partial r}\cdot \frac{\partial r}{\partial x}+\frac{\partial u}{\partial \theta}\cdot \frac{\partial \theta}{\partial x} \right) +\mathrm{i}\left( \frac{\partial v}{\partial r}\cdot \frac{\partial r}{\partial x}+\frac{\partial v}{\partial \theta}\cdot \frac{\partial \theta}{\partial x} \right) 
\\
&=\left( \frac{\partial u}{\partial r}\cdot \cos \theta -\frac{\partial u}{\partial \theta}\cdot \frac{\sin \theta}{r} \right) +\mathrm{i}\left( \frac{\partial v}{\partial r}\cdot \cos \theta -\frac{\partial v}{\partial \theta}\cdot \frac{\sin \theta}{r} \right) 
\\
&=\left( \frac{\partial u}{\partial r}\cdot \cos \theta \right) -\mathrm{i}\left( \frac{\partial v}{\partial \theta}\cdot \frac{\sin \theta}{r} \right) =\frac{\cos \theta}{r}-\mathrm{i}\cdot \frac{\sin \theta}{r}
\\
&=\frac{1}{r}\left( \cos \theta -\mathrm{i}\sin \theta \right) =\frac{1}{r}\cdot \mathrm{e}^{-\mathrm{i}\theta}
\\
&=\frac{1}{r}\cdot \frac{r}{z}=\frac{1}{z}.
\end{align*}
此外
\begin{align*}
\mathrm{e}^{\varphi_k(z)} = \mathrm{e}^{\log|z| + \mathrm{i}(\theta_0 + 2k_0\pi)} = |z| \mathrm{e}^{\mathrm{i}\theta_0} = z,
\end{align*}
对每一点 \( z \in D \) 成立。
\end{proof}

\begin{definition}
如果当 \( z \) 沿着 \( z_0 \) 的充分小邻域中的任意简单闭曲线绕一圈时,多值函数的值就从一支变到另一支,那么称 \( z_0 \) 为该多值函数的一个\textbf{支点}。
\end{definition}
\begin{note}
以对数函数为例,\( z = 0 \) 和 \( z = \infty \) 便是 \( \mathrm{Log} z \) 的支点。
\end{note}

现在讨论 \( \mathrm{Log} z \) 的映射性质。根据\refthe{theorem:单值全纯分支},我们取 \( D \) 为 \( \mathbb{C} \) 除去负实轴后所得的域,它是不包含原点和无穷远点的单连通域,因而可以分出无穷多个单值的全纯分支。我们把 \( k_0 = 0 \) 的那一支称为它的主支,这时取 \( \mathrm{Arg} z \) 的主值为 \( -\pi < \arg z < \pi \),于是
\[
w = \varphi_0(z) = \log|z| + \mathrm{i}\arg z
\]
把 \( D \) 单叶地映为带状域 \( -\pi < \mathrm{Im} w < \pi \)。其他各分支,例如 \( w = \varphi_k(z)=\log|z| + \mathrm{i}(\arg z + 2k\pi) \),就把 \( D \) 单叶地映为带状域 \( (2k - 1)\pi < \mathrm{Im} w < (2k + 1)\pi \)。一般来说,\( w = \varphi_0(z) \) 把角状域 \( -\pi \leqslant \alpha < \arg z < \beta \leqslant \pi \) 单叶地映为带状域 \( \alpha < \mathrm{Im} w < \beta \)。今后,我们就把 \( \mathrm{Log} z \) 的主支 \( \varphi_0(z) \) 记为 \( \log z \)。

有时,为了方便起见,也可把 \( \mathbb{C} \) 去掉正实轴以后的域取为 \( D \),它同样是不包含原点和无穷远点的单连通域,但这时辐角的主值范围应取为 \( 0 < \arg z < 2\pi \)。\( \mathrm{Log} z \) 的主支是
\[
\log z = \log|z| + \mathrm{i}\arg z, \ 0 < \arg z < 2\pi,
\]
它把 \( D \) 单叶地映为带状域 \( 0 < \mathrm{Im} w < 2\pi \)。

\begin{definition}[割线]
我们把这种从原点出发并伸向无穷远的曲线叫做\textbf{割线}。
\end{definition}

一般来说,还可以用一条从原点出发并伸向无穷远的曲线代替上面的负实轴或正实轴,这样得到的域 \( D \) 同样满足\refthe{theorem:单值全纯分支}的条件。通常,为了便于表达出 \( \mathrm{Log} z \) 的单值分支 \( \varphi_k(z) \),常取从原点出发的一条射线作为割线,特别是取负实轴或正实轴。



\subsection{幂函数}

\begin{definition}[整函数]
在$\mathbb{C}$上每点都全纯的函数称为\textbf{整函数}.
\end{definition}

\begin{definition}[幂函数]
设$z$是任意复数,$w = z^{\mu}$称为\textbf{幂函数},这里,$\mu = a + b\text{i}$是一个复数.
\end{definition}
我们分几种情形来讨论.

(1)$\mu = n$,是一个自然数.
按导数的定义,可以直接算出
\begin{align*}
(z^n)' = nz^{n - 1}
\end{align*}
所以,$w = z^n$在$\mathbb{C}$上每点都是全纯的.
所以,$w = z^n$是一个整函数.由于它的导数除原点外都不为零,因此除原点外它是一个保角变换,保角性在原点不成立.考虑从原点出发的射线,它与正实轴的夹角为$\theta$,这条射线的方程可写为$\arg z = \theta$.由于$w = z^n$,所以
\begin{align*}
\arg w = n\arg z = n\theta.
\end{align*}
这就是说,这条射线的像也是一条过原点的射线,但它与正实轴的夹角是$n\theta$,已经比原来的夹角扩大了$n$倍.这一事实在作具体变换时却很有用.例如,$w = z^2$能把第一象限变成上半平面,$w = z^3$能把角状域$\left\{ z: 0 < \arg z < \dfrac{\pi}{3} \right\}$变成上半平面,等等.

现在来看$w = z^n$的单叶性域.设$z_1 = r_1\text{e}^{\text{i}\theta_1}$,$z_2 = r_2\text{e}^{\text{i}\theta_2}$,如果$z_1 \neq z_2$,但$z_1^n = z_2^n$,即$r_1^n\text{e}^{\text{i}n\theta_1} = r_2^n\text{e}^{\text{i}n\theta_2}$,因而$r_1 = r_2$,$\theta_1 = \theta_2 + \dfrac{2k\pi}{n}$.因此,只要域中不出现这样两个点,它们的辐角差等于$\dfrac{2\pi}{n}$,这样的域便是$w = z^n$的单叶性域.例如,$\left\{ z: 0 < \arg z < \dfrac{2\pi}{n} \right\}$便是它的一个单叶性域.一般来说,域
\[
\left\{ z: \alpha < \arg z < \beta, 0 < \beta - \alpha \leqslant \dfrac{2\pi}{n} \right\}
\]
是它的单叶性域,它在$w = z^n$映射下的像是
\[
\{ w: n\alpha < \arg w < n\beta \}.
\]
(2) $\mu = \dfrac{1}{n}$,$n$是一个自然数.

$w = z^{\frac{1}{n}}$是$w = z^n$的反函数.因为对于一个给定的$z$,$z^{\frac{1}{n}}$有$n$个值,所以它是一个多值函数.由\hyperref[section:复数的几何表示]{第一章的知识}知道,它的多值性也是由$\text{Arg}z$的多值性产生的,所以$z = 0$和$z = \infty$是它的支点.因而,在$\mathbb{C}$去掉正实轴后所成的域上可以分出$n$个单值的全纯分支,它们是
\[
w = \varphi_k(z) = \sqrt[n]{|z|} \left( \cos \dfrac{\theta + 2k\pi}{n} + \text{i}\sin \dfrac{\theta + 2k\pi}{n} \right),
\]
\[
k = 0,1,\cdots,n - 1.
\]
这里,$\theta = \arg z$,它的变化范围是$0 < \arg z < 2\pi$.$k = 0$的那一支称为它的主支,直接记为$w = \sqrt[n]{z}$.

现在来看它的主支的映射性质.容易看出,它把从原点发出的射线$\arg z = \theta$变为从原点发出的射线$\arg w = \dfrac{\theta}{n}$.由此可知,$w = \sqrt[n]{z}$把除去正实轴以后的全平面单叶地映为角状域$\left\{ z: 0 < \arg z < \dfrac{2\pi}{n} \right\}$.例如,$w = \sqrt{z}$,$w = \sqrt[4]{z}$分别把除去正实轴的全平面单叶地映为上半平面和第一象限.

(3) $\mu = a + b\text{i}$,是一个复数.

一般的幂函数$w = z^\mu$定义为
\[
w = z^\mu = \text{e}^{\mu \text{Log} z},
\]
显然,它是一个多值函数.用$\text{Log} z$的表达式代入上式,可得
\[
w = z^\mu = \text{e}^{(a + b\text{i})(\log |z| + \text{i}\arg z + 2k\pi \text{i})}
= \text{e}^{a\log |z| - b(\arg z + 2k\pi)} \text{e}^{\text{i}[b\log |z| + a(\arg z + 2k\pi)]},
k = 0, \pm 1, \cdots.
\]

(i) 若$b = 0$,$a = n$是一个整数,这时$w = z^n$是一个单值函数.

(ii) 若$b = 0$,$a = \dfrac{p}{q}$是一个有理数,不妨设$p < q$,这时
\[
w = z^\mu = z^{\frac{p}{q}} = |z|^{\frac{p}{q}} \text{e}^{\text{i}\frac{p}{q}(\arg z + 2k\pi)}.
\]
当$k = 0,1,\cdots,q - 1$时,$z^{\frac{p}{q}}$有$q$个不同的值,因此是一个$q$值函数.

(iii) 若$b = 0$,$a$是一个无理数,这时
\[
w = z^\mu = |z|^a \text{e}^{\text{i}a\arg z} \text{e}^{\text{i}2k\pi a}.
\]
因为$a$是无理数,不论$k$取什么整数值,都不能使$ka$为一整数,因此$z^a$是一个无穷值函数.

(iv) 若$b \neq 0$,则$w = z^\mu$是一无穷值函数.

总之,在上面的情况(ii),(iii),(iv)下,$z^\mu$都是一个多值函数.它的多值性是由$\text{Log} z$的多值性引起的,因此$z = 0$和$z = \infty$是它的支点,而且在$\text{Log} z$可以分出单值全纯分支的域内,$z^\mu$也能分出单值全纯分支.设$\varphi_k(z)$是$\text{Log} z$在域$D$中的单值全纯分支,$w_k(z)$是$z^\mu$的单值全纯分支,按定义,有
\[
w_k(z) = \text{e}^{\mu \varphi_k(z)}.
\]
其中
\[
w_0(z) = \text{e}^{\mu \varphi_0(z)} = \text{e}^{\mu \text{log} z},
\]
称为$z^\mu$的主支.因为$\varphi_k(z)$和$\varphi_{k + 1}(z)$相差$2\pi \text{i}$,所以$w_k(z)$和$w_{k + 1}(z)$相差$\text{e}^{2\mu \pi \text{i}}$.由于$\varphi_k'(z) = \dfrac{1}{z}$,所以
\begin{align}\label{equation:幂函数求导}
w_k'(z) = \text{e}^{\mu \varphi_k(z)} \mu \varphi_k'(z)
= \mu z^{\mu - 1}
= \mu \text{e}^{(\mu - 1)\varphi_k(z)}.
\end{align}

\begin{example}\label{example:例2.4.5}
求一保角变换,把除去线段$\{ z = a + \text{i}y: 0 < y < h \}$的上半平面变为上半平面.
\end{example}
\begin{solution}
初看起来,解这样的题目很困难,因为并没有一个现成的变换可以达到上述目的.我们的想法是把整个变换过程分解成若干个简单的步骤,而每一个步骤都可用我们已知的变换来实现,把这些变换复合起来,就是我们要找的变换.\reffig{figure:图2.4}就是整个变换的分解过程.所以,要找的变换就是
\[
w = \sqrt{z_3}
= \sqrt{z_2 + h^2}
= \sqrt{z_1^2 + h^2}
= \sqrt{(z - a)^2 + h^2}.
\]
\begin{figure}[H]
\centering
\includegraphics[scale=0.3]{图2.4.png}
\caption{}
\label{figure:图2.4}
\end{figure}
\end{solution}

\begin{example}
求一保角变换,把除去割线$\{ z = x + \text{i}: -\infty < x < -1 \}$后的带状域$\{ z: 0 < \text{Im} z < 2 \}$变为上半平面.
\end{example}
\begin{solution}
\reffig{figure:图2.5}是变换的分解过程.由此可见,要找的变换就是
\[
w = \sqrt{\text{e}^{\pi z} + \text{e}^{-\pi}}.
\]
这里,最后一个步骤用到了\refexa{example:例2.4.5}的结果.
\begin{figure}[H]
\centering
\includegraphics[scale=0.3]{图2.5.png}
\caption{}
\label{figure:图2.5}
\end{figure}
\end{solution}



\subsection{三角函数}

\begin{definition}
设$z$是任意复数,定义
\[
\cos z = \dfrac{1}{2}(\text{e}^{\text{i}z} + \text{e}^{-\text{i}z}),
\]
\[
\sin z = \dfrac{1}{2\text{i}}(\text{e}^{\text{i}z} - \text{e}^{-\text{i}z}).
\]
\end{definition}

\begin{proposition}[三角函数的性质]\label{proposition:三角函数的性质}
\begin{enumerate}[(i)]
\item $\cos z$和$\sin z$都是整函数,并且$(\cos z)' = -\sin z,
\quad
(\sin z)' = \cos z.$

\item $\cos z$和$\sin z$都以$2\pi$为周期.

\item $\cos z$是偶函数,$\sin z$是奇函数,即
\[
\cos(-z) = \cos z,
\quad
\sin(-z) = -\sin z.
\]

\item 对任意复数$z_1$和$z_2$,有
\begin{align}
\cos(z_1 + z_2) = \cos z_1 \cos z_2 - \sin z_1 \sin z_2, \label{equation----2.2.2.2.1}
\\
\sin(z_1 + z_2) = \sin z_1 \cos z_2 + \cos z_1 \sin z_2.
\label{equation----2.2.2.2.2} 
\end{align}

\item 对任意复数$z$,有
\[
\cos^2 z + \sin^2 z = 1,
\quad
\sin 2z = 2\sin z \cos z.
\]

\item $\sin z$仅在$z = k\pi$处为零,$\cos z$仅在$z = k\pi + \dfrac{\pi}{2}$处为零,这里,$k = 0, \pm 1, \cdots$.

\item $\cos z$和$\sin z$不是有界函数.
\end{enumerate}
\end{proposition}
\begin{remark}
第(vii)条性质与实变函数中的正、余弦函数不一样,其余性质都相同.
\end{remark}
\begin{proof}
\begin{enumerate}[(i)]
\item 因为$\text{e}^{\text{i}z}$,$\text{e}^{-\text{i}z}$是整函数,所以$\cos z$和$\sin z$也都是整函数.由\nrefpro{proposition:指数函数的性质}{(i)}及\nrefpro{proposition:复变函数求导运算法则}{(1)}可得
\begin{gather*}
(\cos z)' =\left[ \frac{1}{2}\left( \mathrm{e}^{\mathrm{i}z}+\mathrm{e}^{-\mathrm{i}z} \right) \right] ' =\frac{\mathrm{i}}{2}\left( \mathrm{e}^{\mathrm{i}z}-\mathrm{e}^{-\mathrm{i}z} \right) =-\frac{1}{2\mathrm{i}}\left( \mathrm{e}^{\mathrm{i}z}-\mathrm{e}^{-\mathrm{i}z} \right) =-\sin z,
\\
(\sin z)' =\left[ \frac{1}{2\mathrm{i}}\left( \mathrm{e}^{\mathrm{i}z}-\mathrm{e}^{-\mathrm{i}z} \right) \right]'=\frac{1}{2}\left( \mathrm{e}^{\mathrm{i}z}+\mathrm{e}^{-\mathrm{i}z} \right) =\cos z.
\end{gather*}

\item 由于$\text{e}^{\text{i}z}$和$\text{e}^{-\text{i}z}$都以$2\pi$为周期,所以$\cos z$和$\sin z$也都以$2\pi$为周期.

\item 注意到
\begin{gather*}
\cos\mathrm{(}-z)=\frac{1}{2}\left( \mathrm{e}^{-\mathrm{i}z}+\mathrm{e}^{\mathrm{i}z} \right) \cos z,
\\
\sin\mathrm{(}-z)=\frac{1}{2\mathrm{i}}\left( \mathrm{e}^{-\mathrm{i}z}-\mathrm{e}^{\mathrm{i}z} \right) =-\frac{1}{2}\left( \mathrm{e}^{\mathrm{i}z}-\mathrm{e}^{-\mathrm{i}z} \right) =-\sin z.
\end{gather*}

\item 根据定义直接验证即得.

\item 在\eqref{equation----2.2.2.2.1}式中令$z_1 = z$,$z_2 = -z$,即得
\[
\cos^2 z + \sin^2 z = 1.
\]
在\eqref{equation----2.2.2.2.2}式令$z_1 = z_2 = z$,即得
\[
\sin 2z = 2\sin z \cos z.
\]

\item 注意到
\[
\sin z = \dfrac{1}{2\text{i}}(\text{e}^{\text{i}z} - \text{e}^{-\text{i}z})
= \dfrac{1}{2\text{i}\text{e}^{\text{i}z}}(\text{e}^{2\text{i}z} - 1),
\]
$\sin z = 0$当且仅当$\text{e}^{2\text{i}z} - 1 = 0$,而这只有当$z = k\pi$($k = 0, \pm 1, \pm 2, \cdots$)时才能成立.又由(iv)可得$\cos z = \sin\left( \dfrac{\pi}{2} - z \right)$,$\cos z = 0$当且仅当$\sin\left( \dfrac{\pi}{2} - z \right) = 0$,所以$z = \dfrac{\pi}{2} + k\pi$.

\item 若取$z = \text{i}y$,$y$是实数,则$y=-\text{i}z$.从而
\[
\cos z = \dfrac{1}{2}(\text{e}^{\text{i}z} + \text{e}^{-\text{i}z})
= \dfrac{1}{2}(\text{e}^{-y} + \text{e}^y)
\rightarrow \infty \ (\text{当} \ y \rightarrow \infty \ \text{时}).
\]
对于$\sin z$,取$z = \dfrac{\pi}{2} + \text{i}y$,则有
\[
\sin\left( \dfrac{\pi}{2} + \text{i}y \right) = \cos \text{i}y \rightarrow \infty \ (\text{当} \ y \rightarrow \infty \ \text{时}).
\]
\end{enumerate}
\end{proof}

\begin{definition}
设$z$是任意的复数,我们定义
\[
\text{tg} z = \dfrac{\sin z}{\cos z},
\quad 
\text{ctg} z = \dfrac{\cos z}{\sin z}.
\]
\end{definition}

\begin{proposition}
$\text{tg} z$在除掉$z = \dfrac{\pi}{2} + k\pi$($k = 0, \pm 1, \cdots$)的开平面上是全纯的,$\text{ctg} z$在除掉$z = k\pi$($k = 0, \pm 1, \cdots$)的开平面上全纯.
\end{proposition}
\begin{proof}
由\refpro{proposition:三角函数的性质}易证.
\end{proof}



\subsection{多值函数}

这一小段我们讨论多值函数
\[
w = \sqrt[n]{(z - a_1)^{\beta_1} \cdots (z - a_m)^{\beta_m}},
\]
这里,$a_1, \cdots, a_m$是复数,$\beta_1, \cdots, \beta_m$是整数,$n$是正整数.在什么样的域中,从它能分出单值的全纯分支呢?任取$z_0 \neq a_j, j = 1, \cdots, m$,取充分小的简单闭曲线$\gamma_0$,使$z_0$在其内部,$a_1, \cdots, a_m$都在其外部.当$z$沿着$\gamma_0$的正方向走一圈时,$z - a_1, \cdots, z - a_m$的辐角都不变,故$z_0$不是支点.再看$a_j$是不是支点,以$a_1$为例,记$z - a_j = r_j \text{e}^{\text{i}\theta_j}, j = 1, \cdots, m$,于是$w$可写为
\[
w = \sqrt[n]{r_1^{\beta_1} \cdots r_m^{\beta_m}} \text{e}^{\text{i}\frac{\beta_1 \theta_1 + \cdots + \beta_m \theta_m}{n}}.
\]
取简单闭曲线$\gamma_1$,使$a_1$在其内部,$a_2, \cdots, a_m$都在其外部.当$z$沿着$\gamma_1$的正方向走一圈时,$\theta_1$增加$2\pi$,$\theta_2, \cdots, \theta_m$都不变,$w$就变成
\[
\sqrt[n]{r_1^{\beta_1} \cdots r_m^{\beta_m}} \text{e}^{\text{i}\frac{\beta_1 \theta_1 + \cdots + \beta_m \theta_m + 2\pi \beta_1}{n}}
\]
\[
= \text{e}^{\text{i}\frac{2\pi \beta_1}{n}} \sqrt[n]{r_1^{\beta_1} \cdots r_m^{\beta_m}} \text{e}^{\text{i}\frac{\beta_1 \theta_1 + \cdots + \beta_m \theta_m}{n}}.
\]
因此,只有当$\beta_1$是$n$的倍数时,$w$的值才不变.其他$a_2, \cdots, a_m$点的情况也一样.于是得到结论:如果$\beta_j$不是$n$的倍数,那么$a_j$是它的支点.再看无穷远点,取充分大的圆周,使$a_1, \cdots, a_m$都在其内部.当$z$沿着这个圆周转一圈时,$z - a_1, \cdots, z - a_m$的辐角都要增加$2\pi$,$w$就变成
\[
\text{e}^{\text{i}\frac{2\pi (\beta_1 + \cdots + \beta_m)}{n}} \sqrt[n]{r_1^{\beta_1} \cdots r_m^{\beta_m}} \text{e}^{\text{i}\frac{\beta_1 \theta_1 + \cdots + \beta_m \theta_m}{n}}.
\]
因而,只有当$\beta_1 + \cdots + \beta_m$不是$n$的倍数时,$z = \infty$是支点.用同样的方法讨论,可以知道,如果简单闭曲线的内部包含$a_{j_1}, \cdots, a_{j_r}$,与它们相应的和$\beta_{j_1} + \cdots + \beta_{j_r}$是$n$的倍数,那么当$z$沿该曲线转一圈后$w$的值不变.根据这些考察,我们得到下面的定理.

\begin{theorem}\label{theorem:多值函数分出单值全纯分支}
如果域$D$只包含这样的简单闭曲线,它的内部或者不含有任何支点,或者包含一组支点$a_{j_1}, \cdots, a_{j_r}$,但与它们相应的和$\beta_{j_1} + \cdots + \beta_{j_r}$是$n$的倍数,那么$w = \sqrt[n]{(z - a_1)^{\beta_1} \cdots (z - a_m)^{\beta_m}}$在$D$中能分出单值的全纯分支.
\end{theorem}
\begin{proof}

\end{proof}

\begin{example}
在怎样的域中,$w = \sqrt{z^2 - 1}$能分出单值的全纯分支?
\end{example}
\begin{solution}
由于
\[
w = \sqrt{z^2 - 1} = \sqrt{(z - 1)(z + 1)},
\]
这时$a_1 = 1$,$a_2 = -1$,$\beta_1 = \beta_2 = 1$,$n = 2$.所以,$1$和$-1$都是它的支点,但无穷远点不是支点.因而,在除去线段$[-1,1]$的全平面(\nreffig{figure:图2.6}{左})上,或者在除去两条割线$\{ z: -\infty < z < -1 \}$和$\{ z: 1 < z < \infty \}$的全平面(\nreffig{figure:图2.6}{右})上,都能分出单值的全纯分支.

\begin{figure}[H]
\centering
\includegraphics[scale=0.4]{图2.6.png}
\caption{}
\label{figure:图2.6}
\end{figure}

\end{solution}

\begin{example}
设$f(z) = \sqrt{z^{-1}(1 - z)^3} (z + 1)^{-1}$,试确定$f$在$[0,1]$的上岸取正值的单值全纯分支$f_0$,并计算$f_0(-\text{i})$.
\end{example}
\begin{solution}
多值性主要发生在带根号的函数上,与$(z + 1)^{-1}$无关.令$\varphi(z) = \sqrt{z^{-1}(1 - z)^3}$,这时$z = 0$和$z = 1$都是$\varphi$的支点,但$z = \infty$不是.由\refthe{theorem:多值函数分出单值全纯分支},$\varphi$能在除去线段$[0,1]$的全平面上分出单值全纯的分支.为了确定出在$[0,1]$上岸取正值的分支,记$z = r_1 \text{e}^{\text{i}\theta_1}$,$1 - z = r_2 \text{e}^{\text{i}\theta_2}$(\reffig{figure:图2.7}),则
\[
\sqrt{z^{-1}(1 - z)^3} = \sqrt{r_1^{-1} r_2^3} \text{e}^{\text{i}\left( \frac{3\theta_2 - \theta_1}{2} + k\pi \right)}, \ k = 0,1.
\]

\begin{figure}[H]
\centering
\includegraphics[scale=0.4]{图2.7.png}
\caption{}
\label{figure:图2.7}
\end{figure}

当$z$在$[0,1]$的上岸时,有
\[
\theta_1 = \theta_2 = 0, \ r_1 = x, \ r_2 = 1 - x.
\]
显然,$k = 0$的那一支在上岸取正值,记为$\varphi_0$,即
\[
\varphi_0(z) = \sqrt{r_1^{-1} r_2^3} \text{e}^{\text{i}\frac{3\theta_2 - \theta_1}{2}}.
\]

现在计算$\varphi_0(-\text{i})$.若让$z$从原点的左边到达$-\text{i}$,则
\[
\theta_1 = \dfrac{3}{2}\pi, \ \theta_2 = \dfrac{\pi}{4}, \ r_1 = 1, \ r_2 = \sqrt{2}.
\]
所以
\[
\varphi_0(-\text{i}) = 2^{\frac{3}{4}} \text{e}^{-\frac{3\pi}{8} \text{i}},
\]
故
\[
f_0(-\text{i}) = \dfrac{1}{1 - \text{i}} 2^{\frac{3}{4}} \text{e}^{-\frac{3\pi}{8} \text{i}}
= 2^{\frac{1}{4}} \text{e}^{-\frac{\pi}{8} \text{i}}.
\]
若让$z$从$1$的右边到达$-\text{i}$,则
\[
\theta_1 = -\dfrac{\pi}{2}, \ \theta_2 = -\dfrac{7}{4}\pi, \ r_1 = 1, \ r_2 = \sqrt{2}.
\]
这时
\[
\varphi_0(-\text{i}) = 2^{\frac{3}{4}} \text{e}^{-\frac{19}{8}\pi \text{i}}
= 2^{\frac{3}{4}} \text{e}^{-\left( 2\pi + \frac{3}{8}\pi \right) \text{i}}
= 2^{\frac{3}{4}} \text{e}^{-\frac{3}{8}\pi \text{i}}.
\]
所得结果和刚才的完全一样.
\end{solution}








\end{document}
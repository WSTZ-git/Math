\documentclass[../../main.tex]{subfiles}
\graphicspath{{\subfix{../../image/}}} % 指定图片目录,后续可以直接使用图片文件名。

% 例如:
% \begin{figure}[H]
% \centering
% \includegraphics[scale=0.4]{图.png}
% \caption{}
% \label{figure:图}
% \end{figure}
% 注意:上述\label{}一定要放在\caption{}之后,否则引用图片序号会只会显示??.

\begin{document}

\section{全纯函数的原函数}

\begin{definition}
设 \( f:D \to C \) 是定义在域 \( D \) 上的一个函数,如果存在 \( F \in H(D) \),使得 \( F'(z) = f(z) \) 在 \( D \) 上成立,就称 \( F \) 是 \( f \) 的一个\textbf{原函数}.
\end{definition}

如果 \( f \in H(D) \),是否一定存在原函数呢?答案是否定的. 例如,若 \( D \) 是除去原点的单位圆盘,\( f(z) = \frac{1}{z} \),\( f \) 当然是 \( D \) 上的全纯函数. 如果它在 \( D \) 上存在原函数 \( F \),则有 \( F'(z) = \frac{1}{z} \) 在 \( D \) 上成立,但这是不可能的. 因为若上式成立,在 \( D \) 中取光滑闭曲线 \( \gamma: [a,b] \to D \),则有 \( \gamma(a) = \gamma(b) \),于是
\begin{align*}
\int_{\gamma} \frac{\mathrm{d}z }{z} = \int_{\gamma} F'(z) \mathrm{d}z  = \int_{a}^{b} F'(\gamma(t)) \gamma'(t) \mathrm{d}t = F(\gamma(b)) - F(\gamma(a)) = 0.
\end{align*}
但由\refexa{example:例3.1.4}知道 \( \int_{\gamma} \frac{\mathrm{d}z }{z} = 2\pi \mathrm{i} \). 这一矛盾说明 \( \frac{1}{z} \) 在 \( D \) 上不存在原函数. 问题出在 \( D \) 不是单连通域. 实际上,对于单连通域上的全纯函数,一定存在原函数.

\begin{theorem}\label{theorem:定理3.3.2}
设 \( f \) 在域 \( D \) 中连续,且对 \( D \) 中任意可求长闭曲线 \( \gamma \),均有 \( \int_{\gamma} f(z) \mathrm{d}z  = 0 \),那么
\[
F(z) = \int_{z_0}^{z} f(\zeta) \mathrm{d}\zeta 
\]
是 \( D \) 中的全纯函数,且在 \( D \) 中有 \( F'(z) = f(z) \),这里,\( z_0 \) 是 \( D \) 中一固定点.
\end{theorem}
\begin{proof}
由于 \( f \) 沿任意可求长闭曲线的积分为零,\( f \) 的积分与路径无关,因而 \( F \) 是一单值函数. 任取 \( a \in D \),我们证明 \( F'(a) = f(a) \). 因为 \( f \) 在 \( a \) 点连续,故对任意 \( \varepsilon > 0 \),存在 \( \delta > 0 \),当 \( |z - a| < \delta \) 时,有 \( |f(z) - f(a)| < \varepsilon \). 今取 \( z \in B(a, \delta) \)(\reffig{figure:图3.6}),显然
\begin{align*}
F(z) - F(a) = \int_{z_0}^{z} f(\zeta) \mathrm{d}\zeta  - \int_{z_0}^{a} f(\zeta) \mathrm{d}\zeta  = \int_{a}^{z} f(\zeta) \mathrm{d}\zeta .
\end{align*}
这里,积分在线段 \( [a, z] \) 上进行,于是
\begin{align*}
\left| \frac{F(z) - F(a)}{z - a} - f(a) \right| 
&= \frac{1}{|z - a|} \left| F(z) - F(a) - f(a)(z - a) \right| \\
&= \frac{1}{|z - a|} \left| \int_{a}^{z} f(\zeta) \mathrm{d}\zeta  - \int_{a}^{z} f(a) \mathrm{d}\zeta  \right| \\
&= \frac{1}{|z - a|} \left| \int_{a}^{z} (f(\zeta) - f(a)) \mathrm{d}\zeta  \right|.
\end{align*}
由\hyperref[proposition:长大不等式]{长大不等式},即知上式右端小于 \( \varepsilon \),这就证明了 \( F'(a) = f(a) \).
\begin{figure}[H]
\centering
\includegraphics[scale=0.3]{图3.6.png}
\caption{}
\label{figure:图3.6}
\end{figure}

\end{proof}

\begin{theorem}\label{theorem:定理3.3.3}
设 \( D \) 是 \( \mathbb{C} \) 中的单连通域,\( f \in H(D) \),那么 \( F(z) = \int_{z_0}^{z} f(\zeta) \mathrm{d}\zeta  \) 是 \( f \) 在 \( D \) 中的一个原函数.
\end{theorem}
\begin{proof}
在定理的假定下,由\hyperref[theorem:Cauchy-Goursat定理(Cauchy积分定理)]{Cauchy积分定理}知道,\( f \) 沿 \( D \) 中任意可求长闭曲线的积分为零,由\refthe{theorem:定理3.3.2} 即得本定理.

\end{proof}

\begin{theorem}\label{theorem:复积分的微分学基本定理}
设 \( D \) 是 \( \mathbb{C} \) 中的单连通域,\( f \in H(D) \),\( \Phi \) 是 \( f \) 的任一原函数,那么
\[
\int_{z_0}^{z} f(\zeta) \mathrm{d}\zeta  = \Phi(z) - \Phi(z_0).
\]
\end{theorem}
\begin{proof}
证明方法也与微积分中一样. 由\refthe{theorem:定理3.3.3}知,由变上限积分确定的函数 \( F \) 是 \( f \) 的一个原函数,因而
\[
(\Phi(z) - F(z))' = f(z) - f(z) = 0.
\]
故由\refpro{proposition:复变函数导数为0必是常函数}知道 \( \Phi(z) - F(z) \) 是一个常数,因而
\begin{align*}
\int_{z_0}^{z} f(\zeta) \mathrm{d}\zeta  = F(z) - F(z_0) = \Phi(z) - \Phi(z_0).
\end{align*}

\end{proof}

多连通域相关内容
现在设 \( D \) 是多连通域,\( f \in H(D) \),一般来说
\[
F(z) = \int_{z_0}^{z} f(\zeta) \mathrm{d}\zeta 
\]
是一个多值函数,它在 \( z \) 点的值将随着连接 \( z_0 \) 和 \( z \) 的曲线变化而变动. 下面看一个具体的例子:

设 \( D = \mathbb{C} \setminus \{0\} \),则 \( D \) 是一个二连通域,\( f(z) = \frac{1}{z} \) 是 \( D \) 中的全纯函数,我们来研究积分
\[
\int_{1}^{z} \frac{1}{\zeta} \mathrm{d}\zeta .
\]
如果连接 \( 1 \) 和 \( z \) 的曲线 \( \gamma \) 不围绕原点(\nreffig{figure:图3.7}{左}),那么 \( \frac{1}{\zeta} \) 沿 \( \gamma \) 的积分等于在实轴上从 \( 1 \) 到 \( |z| \) 的积分与圆弧 \( \gamma' \) 上的积分之和,即
\begin{align*}
\int_{\gamma} \frac{\mathrm{d}\zeta }{\zeta} = \int_{1}^{|z|} \frac{\mathrm{d}x}{x} + \int_{0}^{\arg z} \frac{\mathrm{i}|z|e^{i\theta}}{|z|e^{\mathrm{i}\theta}} d\theta = \log|z| + \mathrm{i}\arg z = \log z.
\end{align*}
如果连接 \( 1 \) 和 \( z \) 的曲线 \( \gamma \) 绕原点沿反时针方向转了 2 圈(\nreffig{figure:图3.7}{右}),这时沿 \( \gamma \) 的积分可以分解为沿 \( \wideparen{az} \),\( \wideparen{abea} \) 和 \( \wideparen{bcdb} \) 的积分,即
\begin{align}
\int_{\gamma} \frac{\mathrm{d}\zeta }{\zeta} = \int_{\wideparen{az}} \frac{\mathrm{d}\zeta }{\zeta} + \int_{\wideparen{abea}} \frac{\mathrm{d}\zeta }{\zeta} + \int_{\wideparen{bcdb}} \frac{\mathrm{d}\zeta }{\zeta}. \label{equation-------,::1}
\end{align}
由于 \( \wideparen{abea} \) 和 \( \wideparen{bcdb} \) 是两条围绕原点的简单闭曲线,故由\refexa{example:例3.2.7},\((1)\) 式右端的后两个积分都等于 \( 2\pi \mathrm{i} \). 根据上面的计算,\eqref{equation-------,::1}式右端的第一个积分为 \( \log z \),因而得
\[
\int_{\gamma} \frac{\mathrm{d}\zeta }{\zeta} = \log z + 4\pi \mathrm{i}.
\]
由此可见,随着 \( \gamma \) 绕原点圈数的不同,一般可得
\[
\int_{1}^{z} \frac{\mathrm{d}\zeta }{\zeta} = \log z + 2k\pi \mathrm{i}, \, k = 0, \pm 1, \cdots,
\]
这恰好是对数函数 \( \text{Log} z \). 所以,对数函数也可用变上限的积分来定义. 

\begin{figure}[H]
\centering
\includegraphics[scale=0.3]{图3.7.png}
\caption{}
\label{figure:图3.7}
\end{figure}

\begin{theorem}\label{theorem:复积分的分部积分法}
设函数 $f(z)$,$g(z)$ 在单连通区域 $D$ 内解析,$\alpha$,$\beta$ 是 $D$ 内两点,试证
\begin{align*}
\int_{\alpha}^{\beta} f(z) g'(z) \mathrm{d}z &= f(z) g(z) \Big|_{\alpha}^{\beta} - \int_{\alpha}^{\beta} g(z) f'(z) \mathrm{d}z. 
\end{align*}
\end{theorem}
\begin{proof}
由\refthe{theorem:定理3.4.3}知\( f', g' \in H(D) \)。从而由\refpro{proposition:复变函数求导运算法则}知,\( f'g, fg' \in H(D) \subset C(D) \)。注意到
\begin{align*}
\left[ f(z)g(z) \right]' = f'(z)g(z) + f(z)g'(z),
\end{align*}
于是由\hyperref[proposition:复积分的基本性质]{复积分的基本性质}和\refthe{theorem:复积分的微分学基本定理}可得
\begin{align*}
f(z)g(z) \Big|_{\alpha}^{\beta} = \int_{\alpha}^{\beta} \left[ f(z)g(z) \right]' \mathrm{d}z = \int_{\alpha}^{\beta} \left[ f'(z)g(z) + f(z)g'(z) \right] \mathrm{d}z = \int_{\alpha}^{\beta} f'(z)g(z) \mathrm{d}z + \int_{\alpha}^{\beta} f(z)g'(z) \mathrm{d}z.
\end{align*}
移项即得.

\end{proof}












\end{document}
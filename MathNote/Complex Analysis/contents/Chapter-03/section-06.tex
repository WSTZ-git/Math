\documentclass[../../main.tex]{subfiles}
\graphicspath{{\subfix{../../image/}}} % 指定图片目录,后续可以直接使用图片文件名。

% 例如:
% \begin{figure}[H]
% \centering
% \includegraphics[scale=0.4]{图.png}
% \caption{}
% \label{figure:图}
% \end{figure}
% 注意:上述\label{}一定要放在\caption{}之后,否则引用图片序号会只会显示??.

\begin{document}

\section{非齐次Cauchy积分公式}

\begin{definition}
设$z=x+\mathrm{i}y$是一个复数,把 \( z,\bar{z} \) 看成独立变量,定义微分 \( \mathrm{d}z,d\bar{z} \) 的\textbf{外积}为
\begin{align*}
\mathrm{d}z \wedge \mathrm{d}z = 0, 
d\bar{z} \wedge d\bar{z} = 0, 
\mathrm{d}z \wedge d\bar{z} = -d\bar{z} \wedge \mathrm{d}z.
\end{align*}
\( \mathrm{d}x,\mathrm{d}y \) 的\textbf{外积}定义与 \( \mathrm{d}z,d\bar{z} \) 的外积定义一样,即 
\[
\mathrm{d}x \wedge \mathrm{d}x = 0 , \mathrm{d}y \wedge \mathrm{d}y = 0 , \mathrm{d}x \wedge \mathrm{d}y = -\mathrm{d}y \wedge \mathrm{d}x .
\]

定义\textbf{面积元素}\( dA =\mathrm{d}x \wedge \mathrm{d}y .\) 
\end{definition}

\begin{proposition}
设$z=x+\mathrm{i}y$是一个复数,则
\begin{align*}
\mathrm{d}z \wedge d\bar{z} = -2i\mathrm{d}x \wedge \mathrm{d}y = -2idA.
\end{align*}
\end{proposition}
\begin{proof}
由于 \( \mathrm{d}z = \mathrm{d}x + i\mathrm{d}y \),\( d\bar{z} = \mathrm{d}x - i\mathrm{d}y \),所以
\begin{align*}
\mathrm{d}z \wedge d\bar{z} &= (\mathrm{d}x + i\mathrm{d}y) \wedge (\mathrm{d}x - i\mathrm{d}y) \\
&= i\mathrm{d}y \wedge \mathrm{d}x - i\mathrm{d}x \wedge \mathrm{d}y \\
&= -2i\mathrm{d}x \wedge \mathrm{d}y = -2idA.
\end{align*}
这里,\( dA \) 是面积元素.
\end{proof}

\begin{definition}
称 \( z,\bar{z} \) 的函数 \( f(z,\bar{z}) \) 为\textbf{零次微分形式},\( f_1(z,\bar{z})\mathrm{d}z + f_2(z,\bar{z})d\bar{z} \) 为\textbf{一次微分形式},\( f(z,\bar{z})\mathrm{d}z \wedge d\bar{z} \) 为\textbf{二次微分式}。
\end{definition}

\begin{definition}
定义算子 \( \partial,\bar{\partial} \) 如下:
\begin{align*}
\partial f = \frac{\partial f}{\partial z}\mathrm{d}z, 
\bar{\partial} f = \frac{\partial f}{\partial \bar{z}}d\bar{z},
\end{align*}
这里
\begin{align*}
\frac{\partial}{\partial z} &= \frac{1}{2}\left( \frac{\partial}{\partial x} - \text{i}\frac{\partial}{\partial y} \right), \\
\frac{\partial}{\partial \overline{z}} &= \frac{1}{2}\left( \frac{\partial}{\partial x} + \text{i}\frac{\partial}{\partial y} \right).
\end{align*}
定义算子 \( d = \partial + \bar{\partial} \),即
\begin{align}
df = \partial f + \bar{\partial} f = \frac{\partial f}{\partial z}\mathrm{d}z + \frac{\partial f}{\partial \bar{z}}d\bar{z}. \label{eq:1}
\end{align}
\end{definition}

\begin{definition}
算子 \( \partial,\bar{\partial} \) 对一次微分形式的作用定义为
\begin{gather*}
\partial(f_1(z,\bar{z})\mathrm{d}z + f_2(z,\bar{z})d\bar{z}) = \frac{\partial f_1}{\partial z}\mathrm{d}z \wedge \mathrm{d}z + \frac{\partial f_2}{\partial z}\mathrm{d}z \wedge d\bar{z} = \frac{\partial f_2}{\partial z}\mathrm{d}z \wedge d\bar{z},
\\
\bar{\partial}(f_1(z,\bar{z})\mathrm{d}z + f_2(z,\bar{z})d\bar{z}) = \frac{\partial f_1}{\partial \bar{z}}d\bar{z} \wedge \mathrm{d}z + \frac{\partial f_2}{\partial \bar{z}}d\bar{z} \wedge d\bar{z} = -\frac{\partial f_1}{\partial \bar{z}}\mathrm{d}z \wedge d\bar{z}.
\end{gather*}
所以
\begin{align}
d(f_1(z,\bar{z})\mathrm{d}z + f_2(z,\bar{z})d\bar{z}) &= \left( \frac{\partial f_2}{\partial z} - \frac{\partial f_1}{\partial \bar{z}} \right) \mathrm{d}z \wedge d\bar{z}. \label{eq:2}
\end{align}
算子\( \partial,\bar{\partial} \) 作用在二次微分形式上的结果定义为零:
\begin{align*}
\partial(f(z,\bar{z})\mathrm{d}z \wedge d\bar{z}) &= \frac{\partial f}{\partial z}\mathrm{d}z \wedge \mathrm{d}z \wedge d\bar{z} = 0, \\
\bar{\partial}(f(z,\bar{z})\mathrm{d}z \wedge d\bar{z}) &= \frac{\partial f}{\partial \bar{z}}d\bar{z} \wedge \mathrm{d}z \wedge d\bar{z} = 0,
\end{align*}
因而
\begin{align}
d(f(z,\bar{z})\mathrm{d}z \wedge d\bar{z}) &= 0. \label{eq:3}
\end{align}
\end{definition}

\begin{definition}
定义 \( d^2\omega = d(d\omega),\quad \partial^2\omega = \partial(\partial\omega),\quad \bar{\partial}^2\omega = \bar{\partial}(\bar{\partial}\omega),\quad \partial\bar{\partial}\omega = \partial(\bar{\partial}\omega) ,\quad \bar{\partial}\partial\omega = \bar{\partial}(\partial\omega) \)。
\end{definition}

\begin{proposition}
$d^2 = 0 , \quad\partial^2 = 0, \quad
\bar{\partial}^2 = 0, \quad
\bar{\partial}\partial + \partial\bar{\partial} = 0.$
\end{proposition}
\begin{proof}
当 \( \omega \) 是一 \( C^2 \) 函数时,由 \(\eqref{eq:1}\) 式和 \(\eqref{eq:2}\) 式,得
\begin{align*}
d^2\omega = d(d\omega) = d\left( \frac{\partial \omega}{\partial z}\mathrm{d}z + \frac{\partial \omega}{\partial \bar{z}}d\bar{z} \right) = \left( \frac{\partial^2 \omega}{\partial \bar{z}\partial z} - \frac{\partial^2 \omega}{\partial z\partial \bar{z}} \right) \mathrm{d}z \wedge d\bar{z} = 0. 
\end{align*}
当 \( \omega \) 是一个一次微分形式时,由 \(\eqref{eq:2}\) 式知 \( d\omega \) 是一个二次微分形式,由 \(\eqref{eq:3}\) 式即知 \( d^2\omega = 0 \)。当 \( \omega \) 是一个二次微分形式时,由 \(\eqref{eq:3}\) 式知 \( d^2\omega = 0 \)。总之,不论 \( \omega \) 是零次、一次或二次微分形式,都有 \( d^2\omega = 0 \),所以 \( d^2 = 0 \)。

根据上述证明,同样可以证明
\begin{align*}
\partial^2 = 0, \quad
\bar{\partial}^2 = 0, \quad
\bar{\partial}\partial + \partial\bar{\partial} = 0. 
\end{align*}
\end{proof}

\begin{theorem}[Green公式]\label{theorem:Green公式-复变函数形式}
设 \( \gamma_0, \gamma_1, \cdots, \gamma_n \) 是 \( n + 1 \) 条可求长简单闭曲线,\( \gamma_1, \cdots, \gamma_n \) 都在 \( \gamma_0 \) 的内部,\( \gamma_1, \cdots, \gamma_n \) 中的每一条都在其他 \( n - 1 \) 条的外部,\( D \) 是由这 \( n + 1 \) 条曲线围成的域,用 \(\partial D \) 记 \( D \) 的边界.如果 \( \omega = f_1(z,\bar{z})\mathrm{d}z + f_2(z,\bar{z})d\bar{z} \) 是域 \( D \) 上的一个一次微分形式,这里,\( f_1,f_2 \in C^1(\bar{D}) \),那么
\begin{align}
\int_{\partial D} \omega &= \int_D d\omega. \label{eq:6}
\end{align}
\end{theorem}
\begin{note}
公式 \(\eqref{eq:6}\) 在高维空间中也成立,通常称为 $\mathbf{Stokes}$\textbf{公式},这里只是它的一个特例。
\end{note}
\begin{proof}
记 \( f_1 = u_1 + iv_1 \),\( f_2 = u_2 + iv_2 \),这里,\( u_1,v_1,u_2,v_2 \) 是 \( z,\bar{z} \) 的实值函数,于是
\begin{align}\label{eq:7}
\omega &= f_1 \mathrm{d}z + f_2 d\bar{z} = (u_1 + iv_1)(\mathrm{d}x + i\mathrm{d}y) + (u_2 + iv_2)(\mathrm{d}x - i\mathrm{d}y) \nonumber \\
&= \{(u_1 + u_2)\mathrm{d}x + (-v_1 + v_2)\mathrm{d}y\}+ i\{(v_1 + v_2)\mathrm{d}x + (u_1 - u_2)\mathrm{d}y\}.
\end{align}
由\eqref{eq:2}式,得
\begin{align}
d\omega &= \left( \frac{\partial f_2}{\partial z} - \frac{\partial f_1}{\partial \bar{z}} \right) \mathrm{d}z \wedge d\bar{z} = -\left\{ \frac{1}{2} \left( \frac{\partial}{\partial x} - i \frac{\partial}{\partial y} \right)(u_2 + iv_2) \right. \left. - \frac{1}{2} \left( \frac{\partial}{\partial x} + i \frac{\partial}{\partial y} \right)(u_1 + iv_1) \right\} 2idA \nonumber \\
&= \left\{ \left( \frac{\partial v_2}{\partial x} - \frac{\partial v_1}{\partial x} - \frac{\partial u_2}{\partial y} - \frac{\partial u_1}{\partial y} \right) \right. \left. + i \left( \frac{\partial u_1}{\partial x} - \frac{\partial u_2}{\partial x} - \frac{\partial v_1}{\partial y} - \frac{\partial v_2}{\partial y} \right) \right\} dA. \label{eq:8}
\end{align}
因为\( f_1,f_2 \in C^1(\bar{D}) \),所以$u_1,u_2,v_1,v_2\in C^1(\bar{D})$.于是根据实值函数的Green 公式,我们有
\begin{gather}
\int_{\partial D} (u_1 + u_2)\mathrm{d}x + (-v_1 + v_2)\mathrm{d}y = \int_D \left\{ \frac{\partial}{\partial x}(-v_1 + v_2) - \frac{\partial}{\partial y}(u_1 + u_2) \right\} dA, \label{eq:9}
\\
\int_{\partial D} (v_1 + v_2)\mathrm{d}x + (u_1 - u_2)\mathrm{d}y = \int_D \left\{ \frac{\partial}{\partial x}(u_1 - u_2) - \frac{\partial}{\partial y}(v_1 + v_2) \right\} dA. \label{eq:10}
\end{gather}
由等式\eqref{eq:7},\eqref{eq:8},\eqref{eq:9},\eqref{eq:10} 即得我们要证明的公式 \eqref{eq:6}.
\end{proof}

\begin{theorem}[非齐次Cauchy积分公式(Pompeiu公式)]\label{theorem:非齐次Cauchy积分公式(Pompeiu公式)}
设 \( \gamma_0, \gamma_1, \cdots, \gamma_n \) 是 \( n + 1 \) 条可求长简单闭曲线,\( \gamma_1, \cdots, \gamma_n \) 都在 \( \gamma_0 \) 的内部,\( \gamma_1, \cdots, \gamma_n \) 中的每一条都在其他 \( n - 1 \) 条的外部,\( D \) 是由这 \( n + 1 \) 条曲线围成的域,用 \(\partial D \) 记 \( D \) 的边界.如果 \( f \in C^1(\bar{D}) \),那么对任意 \( z \in D \),有
\begin{align}
f(z) &= \frac{1}{2\pi i} \int_{\partial D} \frac{f(\zeta)}{\zeta - z} d\zeta + \frac{1}{2\pi i} \int_D \frac{\partial f(\zeta)}{\partial \bar{\zeta}} \frac{1}{\zeta - z} d\zeta \wedge d\bar{\zeta}. \label{eq:11}
\end{align}
\end{theorem}
\begin{note}
如果 \( f \in H(D) \),那么由 Cauchy-Riemann 方程,\( \frac{\partial f}{\partial \bar{\zeta}} = 0 \),这时公式 \(\eqref{eq:11}\) 右端的第二项就消失了,公式 \(\eqref{eq:11}\) 就是 Cauchy积分公式。所以,公式 \(\eqref{eq:11}\) 是 Cauchy 积分公式在 \( C^1 \) 函数类中的推广,有时也称为\textbf{非齐次}$\mathbf{Cauchy}$\textbf{积分公式}。

公式 \(\eqref{eq:11}\) 首先是由 Pompeiu 在 1912 年证明的(所以有时也称之为 Pompeiu 公式),但长期以来似乎被人们遗忘了。直到 1950 年,Grothendieck 和 Dolbeault 用它来解 \( \bar{\partial} \) 方程时,人们才发现它的意义所在。
\end{note}
\begin{proof}
不妨设 \( D \) 是\reffig{figure:图3.10}所示的二连通域,\( D \) 的边界 \( \partial D \) 由 \( \gamma_0 \) 和 \( \gamma_1 \) 组成。任取 \( z \in D \),因为 \( f \) 在 \( z \) 点连续,故对任意 \( \varepsilon > 0 \),存在 \( \delta > 0 \),当 \( |\zeta - z| < \delta \) 时,\( |f(\zeta) - f(z)| < \varepsilon \)。记 \( \rho = \inf_{\zeta \in \partial D} |\zeta - z| > 0 \),取 \( \eta \),使得 \( 0 < \eta < \min(\rho,\delta) \),于是 \( \overline{B(z,\eta)} \subset D \)。记 \( B_\eta = B(z,\eta) \),令 \( G_\eta = D \setminus \overline{B_\eta} \),则 \( G_\eta \) 的边界 \( \partial G_\eta \) 由 \( \gamma_0,\gamma_1 \) 和 \( \partial B_\eta \) 三条曲线组成。考虑一次微分形式
\[
\omega = \frac{f(\zeta)}{\zeta - z} d\zeta,
\]
它在域 \( G_\eta \) 上满足\hyperref[theorem:Green公式-复变函数形式]{Green公式}的条件,因而有
\begin{align}
\int_{\partial G_\eta} \omega &= \int_{G_\eta} d\omega. \label{eq:12}
\end{align}
由于 \( \frac{1}{\zeta - z} \) 在 \( G_\eta \) 中全纯,所以由\refthe{theorem:复变函数可微的充要条件2}可知$\frac{\partial}{\partial \overline{\zeta }}\left( \frac{1}{\zeta -z} \right) =0$.因此
\begin{align*}
\bar{\partial} \omega = \frac{\partial}{\partial \bar{\zeta}} \left( \frac{f(\zeta)}{\zeta - z} \right) d\bar{\zeta} \wedge d\zeta \xlongequal{\text{乘积求导法则}} \left\{ f(\zeta) \frac{\partial}{\partial \bar{\zeta}} \left( \frac{1}{\zeta - z} \right) + \frac{\partial f(\zeta)}{\partial \bar{\zeta}} \frac{1}{\zeta - z} \right\} d\bar{\zeta} \wedge d\zeta = \frac{\partial f(\zeta)}{\partial \bar{\zeta}} \frac{1}{\zeta - z} d\bar{\zeta} \wedge d\zeta.
\end{align*}
易知
\[
\partial \omega = \frac{\partial}{\partial \zeta} \left( \frac{f(\zeta)}{\zeta - z} \right) d\zeta \wedge d\zeta = 0,
\]
因而
\[
d\omega = \partial \omega + \bar{\partial} \omega = \frac{\partial f(\zeta)}{\partial \bar{\zeta}} \frac{1}{\zeta - z} d\bar{\zeta} \wedge d\zeta.
\]
这样,\(\eqref{eq:12}\) 式可以写成
\begin{align}
\int_{\partial D} \frac{f(\zeta)}{\zeta - z} d\zeta - \int_{\partial B_\eta} \frac{f(\zeta)}{\zeta - z} d\zeta &= \int_{G_\eta} \frac{\partial f(\zeta)}{\partial \bar{\zeta}} \frac{1}{\zeta - z} d\bar{\zeta} \wedge d\zeta. \label{eq:13}
\end{align}
注意
\begin{align*}
\int_{\partial B_\eta} \frac{f(\zeta)}{\zeta - z} d\zeta = \int_{\partial B_\eta} \frac{f(\zeta) - f(z)}{\zeta - z} d\zeta + f(z) \int_{\partial B_\eta} \frac{d\zeta}{\zeta - z} \xlongequal{\text{\refexa{example:例3.1.4}}} \int_{\partial B_\eta} \frac{f(\zeta) - f(z)}{\zeta - z} d\zeta + 2\pi i f(z),
\end{align*}
而
\begin{align*}
\left| \int_{\partial B_\eta} \frac{f(\zeta) - f(z)}{\zeta - z} d\zeta \right| \leqslant \int_{\partial B_\eta} \frac{|f(\zeta) - f(z)|}{|\zeta - z|} |d\zeta| \leqslant \frac{\varepsilon}{\eta} \cdot 2\pi \eta = 2\pi \varepsilon,
\end{align*}
由此即得
\begin{align}
\lim_{\eta \to 0} \int_{\partial B_\eta} \frac{f(\zeta)}{\zeta - z} d\zeta &= 2\pi i f(z). \label{eq:14}
\end{align}
另一方面,由\( f \in C^1(\bar{D}) \)可知 \( \frac{\partial f}{\partial \bar{\zeta}} \) 在 \( \bar{B}_\eta \) 上连续,故有常数 \( M \),使得 \( \left| \frac{\partial f}{\partial \bar{\zeta}} \right| \leqslant M \) 在 \( \bar{B}_\eta \) 上成立。于是
\begin{align*}
\left| \int_{B_\eta} \frac{\partial f}{\partial \bar{\zeta}} \frac{1}{\zeta - z} d\bar{\zeta} \wedge d\zeta \right| \leqslant 2 \int_{B_\eta} \left| \frac{\partial f}{\partial \bar{\zeta}} \right| \frac{1}{|\zeta - z|} dA \leqslant 4M \pi \eta \to 0 \ (\eta \to 0).
\end{align*}
因而
\begin{align}
\lim_{\eta \to 0} \int_{G_\eta} \frac{\partial f}{\partial \bar{\zeta}} \frac{1}{\zeta - z} d\bar{\zeta} \wedge d\zeta = \lim_{\eta \to 0} \left\{ \int_D \frac{\partial f}{\partial \bar{\zeta}} \frac{1}{\zeta - z} d\bar{\zeta} \wedge d\zeta - \int_{B_\eta} \frac{\partial f}{\partial \bar{\zeta}} \frac{1}{\zeta - z} d\bar{\zeta} \wedge d\zeta \right\} = \int_D \frac{\partial f}{\partial \bar{\zeta}} \frac{1}{\zeta - z} d\bar{\zeta} \wedge d\zeta. \label{eq:15}
\end{align}
在等式 \(\eqref{eq:13}\) 两端令 \( \eta \to 0 \),并利用 \(\eqref{eq:14}\) 式和 \(\eqref{eq:15}\) 式,即得所要证明的公式 \(\eqref{eq:11}\)。
\end{proof}

\begin{figure}[H]
\centering
\includegraphics[scale=0.3]{图3.10.png}
\caption{}
\label{figure:图3.10}
\end{figure}




























\end{document}
\documentclass[../../main.tex]{subfiles}
\graphicspath{{\subfix{../../image/}}} % 指定图片目录,后续可以直接使用图片文件名。

% 例如:
% \begin{figure}[H]
% \centering
% \includegraphics[scale=0.4]{图.png}
% \caption{}
% \label{figure:图}
% \end{figure}
% 注意:上述\label{}一定要放在\caption{}之后,否则引用图片序号会只会显示??.

\begin{document}

\section{复变函数的积分}

\begin{definition}
设 \( z = \gamma(t) \)(\( a \leqslant t \leqslant b \))是一条可求长曲线,\( f \) 是定义在 \( \gamma \) 上的函数,沿 \( \gamma \) 的正方向取分点 \( \gamma(a) = z_0, z_1, z_2, \cdots, z_n = \gamma(b) \),在 \( \gamma \) 中从 \( z_{k - 1} \) 到 \( z_k \) 的弧段上任取点 \( \zeta_k \),\( k = 1, \cdots, n \)(见\reffig{figure:图3.1}),作 Riemann 和
\begin{align}\label{equation----3.1.1.1.1}
\sum_{k = 1}^n f(\zeta_k)(z_k - z_{k - 1}). 
\end{align}
用 \( s_k \) 记弧段 \( \wideparen{z_{k - 1}z_k} \) 的长度,如果当 \( \lambda = \max\{s_k: 1 \leqslant k \leqslant n\} \to 0 \) 时,不论 \( \zeta_k \) 的取法如何,和式 \eqref{equation----3.1.1.1.1}总有一确定的极限,就称此极限为 \( f \) 沿 \( \gamma \) 的积分,记为 \( \int_\gamma f(z)\mathrm{d}z \),即
\[
\int_\gamma f(z)\mathrm{d}z = \lim_{\lambda \to 0} \sum_{k = 1}^n f(\zeta_k)(z_k - z_{k - 1}).
\]
\end{definition}

\begin{figure}[H]
\centering
\includegraphics[scale=0.4]{图3.1.png}
\caption{}
\label{figure:图3.1}
\end{figure}

\begin{proposition}\label{proposition:连续复变函数积分必存在}
设 \( f = u + \mathrm{i}v \) 在可求长曲线 \( \gamma \) 上连续,则\( f \) 沿 \( \gamma \) 的积分必存在,并且
\[
\int_\gamma f(z)\mathrm{d}z = \int_\gamma u \mathrm{d}x - v \mathrm{d}y + \mathrm{i}\int_\gamma v \mathrm{d}x + u \mathrm{d}y.
\]
\end{proposition}
\begin{proof}
记 \( z_k = x_k + \mathrm{i}y_k \),\( \zeta_k = \xi_k + \mathrm{i}\eta_k \),\( f(\zeta_k) = u(\xi_k, \eta_k) + \mathrm{i}v(\xi_k, \eta_k) \),于是$f$的Riemann和可写成
\begin{align*}
&\sum_{k=1}^n{f(\zeta _k)(z_k}-z_{k-1})=\sum_{k=1}^n{\left[ u\left( \xi _k,\eta _k \right) +\mathrm{i}v\left( \xi _k,\eta _k \right) \right] \left( \Delta x_k+\mathrm{i}\Delta y_k \right)}
\\
&=\sum_{k=1}^n{\left\{ u(\xi _k,\eta _k)\Delta x_k-v(\xi _k,\eta _k)\Delta y_k \right\}}+\mathrm{i}\sum_{k=1}^n{\left\{ v(\xi _k,\eta _k)\Delta x_k+u(x_k,y_k)\Delta y_k \right\}},
\end{align*}
这里,\( \Delta x_k = x_k - x_{k - 1} \),\( \Delta y_k = y_k - y_{k - 1} \). 当 \( u, v \) 在 \( \gamma \) 上连续时,上述和式当 \( \lambda \to 0 \) 时趋于曲线积分
\[
\int_\gamma u \mathrm{d}x - v \mathrm{d}y + \mathrm{i}\int_\gamma v \mathrm{d}x + u \mathrm{d}y.
\]
\end{proof}

\begin{proposition}\label{proposition:复变函数积分参数方程形式}
如果 \( z = \gamma(t) \)(\( a \leqslant t \leqslant b \))是光滑曲线,\( f \) 在 \( \gamma \) 上连续,那么\( f \) 沿 \( \gamma \) 的积分必存在,并且
\[
\int_\gamma f(z)\mathrm{d}z = \int_a^b f(\gamma(t))\gamma'(t)\mathrm{d}t.
\]
\end{proposition}
\begin{proof}
设 \( z = \gamma(t) = x(t) + \mathrm{i}y(t) \),在所设的条件下,有
\[
\int_\gamma u \mathrm{d}x - v \mathrm{d}y
= \int_a^b \left\{ u(x(t), y(t))x'(t) - v(x(t), y(t))y'(t) \right\} \mathrm{d}t,
\]
\[
\int_\gamma v \mathrm{d}x + u \mathrm{d}y
= \int_a^b \left\{ v(x(t), y(t))x'(t) + u(x(t), y(t))y'(t) \right\} \mathrm{d}t.
\]
由\refpro{proposition:连续复变函数积分必存在}可知,第二式乘 \( \mathrm{i} \) 后与第一式相加,即得
\[
\int_\gamma f(z)\mathrm{d}z
= \int_a^b \left\{ [u(x(t), y(t)) + \mathrm{i}v(x(t), y(t))](x'(t) + \mathrm{i}y'(t)) \right\} \mathrm{d}t
= \int_a^b f(\gamma(t))\gamma'(t)\mathrm{d}t.
\]
\end{proof}

\begin{proposition}
如果 \( f, g \) 在可求长曲线 \( \gamma \) 上连续,那么
\begin{enumerate}[(i)]
\item \( \int_{\gamma^-} f(z)\mathrm{d}z = -\int_\gamma f(z)\mathrm{d}z \),这里,\( \gamma^- \) 是指与 \( \gamma \) 方向相反的曲线;

\item \( \int_\gamma (\alpha f(z) + \beta g(z))\mathrm{d}z = \alpha \int_\gamma f(z)\mathrm{d}z + \beta \int_\gamma g(z)\mathrm{d}z \),这里,\( \alpha, \beta \) 是两个复常数;

\item \( \int_\gamma f(z)\mathrm{d}z = \int_{\gamma_1} f(z)\mathrm{d}z + \int_{\gamma_2} f(z)\mathrm{d}z \),这里,\( \gamma \) 是由 \( \gamma_1 \) 和 \( \gamma_2 \) 组成的曲线.
\end{enumerate}
\end{proposition}
\begin{proof}
由\refpro{proposition:连续复变函数积分必存在}和实变函数第二型曲面积分的性质不难证明.
\end{proof}

\begin{example}\label{example:例3.1.3}
设可求长曲线 \( z = \gamma(t) \)(\( a \leqslant t \leqslant b \))的起点为 \( \alpha \),终点为 \( \beta \),证明
\[
\int_\gamma \mathrm{d}z = \beta - \alpha,
\quad
\int_\gamma z \mathrm{d}z = \frac{1}{2}(\beta^2 - \alpha^2).
\]
\end{example}
\begin{proof}
若 \( \gamma \) 是光滑曲线,由\refpro{proposition:复变函数积分参数方程形式},得
\[
\int_\gamma \mathrm{d}z = \int_a^b \gamma'(t)\mathrm{d}t
= \gamma(b) - \gamma(a)
= \beta - \alpha,
\]
\[
\int_\gamma z \mathrm{d}z = \int_a^b \gamma(t)\gamma'(t)\mathrm{d}t
= \left. \frac{1}{2}\gamma^2(t) \right|_a^b
= \frac{1}{2}(\gamma^2(b) - \gamma^2(a))
= \frac{1}{2}(\beta^2 - \alpha^2).
\]

如果 \( \gamma \) 不是光滑曲线,可直接按积分的定义计算:
\[
\int_\gamma \mathrm{d}z = \lim_{\lambda \to 0} \sum_{k = 1}^n (z_k - z_{k - 1})
= z_n - z_0
= \beta - \alpha.
\]
\[
\int_\gamma z \mathrm{d}z = \lim_{\lambda \to 0} \sum_{k = 1}^n z_k(z_k - z_{k - 1}),
\quad
\int_\gamma z \mathrm{d}z = \lim_{\lambda \to 0} \sum_{k = 1}^n z_{k - 1}(z_k - z_{k - 1}),
\]
把两式加起来,得
\[
\int_\gamma z \mathrm{d}z = \frac{1}{2} \lim_{\lambda \to 0} \sum_{k = 1}^n (z_k^2 - z_{k - 1}^2)
= \frac{1}{2}(z_n^2 - z_0^2)
= \frac{1}{2}(\beta^2 - \alpha^2).
\]
\end{proof}

\begin{example}\label{example:例3.1.4}
计算积分 \( \int_\gamma \frac{\mathrm{d}z}{(z - a)^n} \),这里,\( n \) 是任意整数,\( \gamma \) 是以 \( a \) 为中心、以 \( r \) 为半径的圆周.
\end{example}
\begin{solution}
\( \gamma \) 的参数方程为 \( z = a + r\mathrm{e}^{\mathrm{i}t}, 0 \leqslant t \leqslant 2\pi \). 由\refpro{proposition:复变函数积分参数方程形式},得
\[
\int_\gamma \frac{\mathrm{d}z}{(z - a)^n} = \int_0^{2\pi} \frac{r\mathrm{i}\mathrm{e}^{\mathrm{i}t}}{r^n\mathrm{e}^{\mathrm{i}nt}}\mathrm{d}t
= r^{1 - n}\mathrm{i} \int_0^{2\pi} \mathrm{e}^{\mathrm{i}(1 - n)t}\mathrm{d}t.
\]
所以,上述积分当 \( n \neq 1 \) 时为零,当 \( n = 1 \) 时为 \( 2\pi\mathrm{i} \),即
\[
\int_\gamma \frac{\mathrm{d}z}{(z - a)^n} = 
\begin{cases} 
0, & n \neq 1; \\
2\pi\mathrm{i}, & n = 1.
\end{cases}
\]
\end{solution}

\begin{proposition}[长大不等式]\label{proposition:长大不等式}
如果 \( \gamma \) 的长度为 \( L \),\( M = \sup_{z \in \gamma} |f(z)| \),那么
\[
\left| \int_\gamma f(z)\mathrm{d}z \right| \leqslant ML. \tag{4}
\]
\end{proposition}
\begin{remark}
这个不等式很简单,但很重要,是我们今后估计积分的主要工具,可简称为\textbf{长大不等式}.
\end{remark}
\begin{proof}
\( f \) 在 \( \gamma \) 上的 Riemann 和有不等式
\[
\left| \sum_{k = 1}^n f(\zeta_k)(z_k - z_{k - 1}) \right| \leqslant \sum_{k = 1}^n |f(\zeta_k)| |z_k - z_{k - 1}|
\leqslant M \sum_{k = 1}^n |z_k - z_{k - 1}|
\leqslant ML,
\]
令 \( \lambda = \max_{1 \leqslant k \leqslant n} s_k \to 0 \),即得所要的不等式.
\end{proof}
























\end{document}
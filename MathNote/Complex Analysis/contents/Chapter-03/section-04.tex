\documentclass[../../main.tex]{subfiles}% 注意这里的文件路径不能用 ./main.tex ,否则用latexmk编译子文件会报错
\graphicspath{{\subfix{./image/}}} % 指定图片目录,后续可以直接使用图片文件名
% 注意这里的文件路径不能用 ../../image/ ,否则用latexmk编译子文件会报错

% 例如:
% \begin{figure}[H]
% \centering
% \includegraphics[scale=0.3]{图.png}
% \caption{}
% \label{figure:图}
% \end{figure}
% 注意:上述\label{}一定要放在\caption{}之后,否则引用图片序号会只会显示??.

\begin{document}

\section{Cauchy积分公式}

\begin{theorem}[Cauchy积分公式]\label{theorem:定理3.4.1}
设 \( D \) 是由可求长简单闭曲线 \( \gamma \) 围成的区域,如果 \( f \in H(D) \cap C(\overline{D}) \),那么对任意 \( z \in D \),均有
\begin{align}\label{equation:Cauchy积分公式1}
f(z) = \frac{1}{2\pi \mathrm{i}} \int_{\gamma} \frac{f(\zeta)}{\zeta - z} \mathrm{d}\zeta.
\end{align}
等式\eqref{equation:Cauchy积分公式1}称为$\mathbf{Cauchy}$\textbf{积分公式}.
\end{theorem}
\begin{note}
上述定理表明全纯函数在区域中的值由它在边界上的值所完全确定. \eqref{corollary:定理3.2.5推论}式是全纯函数的一种积分表示,通过这种表示,我们可以证明全纯函数有任意阶导数.
\end{note}
\begin{proof}
任取 \( z \in D \),因为 \( f \) 在 \( z \) 点连续,故对任意 \( \varepsilon > 0 \),存在 \( \delta > 0 \),使得当 \( |\zeta - z| < \delta \) 时,有 \( |f(\zeta) - f(z)| < \varepsilon \). 今取 \( \rho < \delta \),使得 \( B(z, \rho) \subset D \). 记 \( \gamma_{\rho} = \{\zeta: |\zeta - z| = \rho\} \),由 \( \gamma \) 和 \( \gamma_{\rho} \) 围成的二连通区域记为 \( D' \)(\reffig{figure:图3.8}),则 \( \frac{f(\zeta)}{\zeta - z} \) 在 \( D' \) 中全纯,在 \( \overline{D'} \) 上连续. 于是,由\refcor{corollary:定理3.2.5推论}得
\begin{align}\label{equation:Cauchy积分公式2}
\frac{1}{2\pi \mathrm{i}} \int_{\gamma} \frac{f(\zeta)}{\zeta - z} \mathrm{d}\zeta = \frac{1}{2\pi \mathrm{i}} \int_{\gamma_{\rho}} \frac{f(\zeta)}{\zeta - z} \mathrm{d}\zeta.
\end{align}
又由\refpro{proposition:例3.1.4}可知\( \frac{1}{2\pi \mathrm{i}} \int_{\gamma_{\rho}} \frac{\mathrm{d}\zeta}{\zeta - z} = 1 \),所以
\begin{align}
f(z) = \frac{1}{2\pi \mathrm{i}} \int_{\gamma_{\rho}} \frac{f(z)}{\zeta - z} \mathrm{d}\zeta. \label{equation:Cauchy积分公式3}
\end{align}
于是,由\eqref{equation:Cauchy积分公式2}式、\eqref{equation:Cauchy积分公式3}式及\hyperref[proposition:长大不等式]{长大不等式}即得
\begin{align*}
\left| f(z) - \frac{1}{2\pi \mathrm{i}} \int_{\gamma} \frac{f(\zeta)}{\zeta - z} \mathrm{d}\zeta \right| 
&= \left| \frac{1}{2\pi \mathrm{i}} \int_{\gamma_{\rho}} \frac{f(z)}{\zeta - z} \mathrm{d}\zeta - \frac{1}{2\pi \mathrm{i}} \int_{\gamma_{\rho}} \frac{f(\zeta)}{\zeta - z} \mathrm{d}\zeta \right| \\
&= \frac{1}{2\pi} \left| \int_{\gamma_{\rho}} \frac{f(z) - f(\zeta)}{\zeta - z} \mathrm{d}\zeta \right| \\
& \leqslant \frac{1}{2\pi} \cdot \frac{\varepsilon}{\rho} \cdot 2\pi \rho = \varepsilon.
\end{align*}
让 \( \varepsilon \to 0 \),即得所要证的等式 \eqref{corollary:定理3.2.5推论}. 
\begin{figure}[H]
\centering
\includegraphics[scale=0.3]{图3.8.png}
\caption{}
\label{figure:图3.8}
\end{figure}

\end{proof}

\begin{theorem}[解析函数的平均值定理]\label{theorem:解析函数的平均值定理}
如果函数$f(z)$在圆$|\zeta - z_0| < R$内解析,在闭圆$|\zeta - z_0| \leqslant R$上连续,则
\begin{align*}
f(z_0) = \frac{1}{2\pi} \int_{0}^{2\pi} f(z_0 + R \mathrm{e}^{\mathrm{i}\varphi}) \mathrm{d}\varphi,
\end{align*}
即$f(z)$在圆心$z_0$的值等于它在圆周上的值的算术平均数.
\end{theorem}
\begin{proof}
设$C$表示圆周$|\zeta - z_0| = R$(如\reffig{figure:复变函数--图3.13}),则
\begin{align*}
\zeta - z_0 = R \mathrm{e}^{\mathrm{i}\varphi}, \quad 0 \leqslant \varphi \leqslant 2\pi,
\end{align*}
或
\begin{align*}
\zeta = z_0 + R \mathrm{e}^{\mathrm{i}\varphi},
\end{align*}
由此
\begin{align*}
\mathrm{d}\zeta = \mathrm{i}R \mathrm{e}^{\mathrm{i}\varphi} \mathrm{d}\varphi,
\end{align*}
根据\hyperref[theorem:定理3.4.1]{Cauchy积分公式}可得
\begin{align*}
f(z_0) &= \frac{1}{2\pi\mathrm{i}} \int_{C} \frac{f(\zeta)}{\zeta - z_0} \mathrm{d}\zeta = \frac{1}{2\pi\mathrm{i}} \int_{0}^{2\pi} \frac{f(z_0 + R \mathrm{e}^{\mathrm{i}\varphi})}{R \mathrm{e}^{\mathrm{i}\varphi}} \cdot \mathrm{i}R \mathrm{e}^{\mathrm{i}\varphi} \mathrm{d}\varphi \\
&= \frac{1}{2\pi} \int_{0}^{2\pi} f(z_0 + R \mathrm{e}^{\mathrm{i}\varphi}) \mathrm{d}\varphi.
\end{align*}
\begin{figure}[H]
\centering
\includegraphics[scale=0.3]{复变函数--图3.13.png}
\caption{}
\label{figure:复变函数--图3.13}
\end{figure}

\end{proof}

\begin{definition}
设 \( \gamma \) 是 \( \mathbb{C} \) 中一条可求长曲线(不一定是闭的),\( g \) 是 \( \gamma \) 上的连续函数,如果 \( z \in \mathbb{C} \setminus \gamma \),那么由\refthe{theorem:连续复变函数积分必存在}可知积分
\[
\frac{1}{2\pi \mathrm{i}} \int_{\gamma} \frac{g(\zeta)}{\zeta - z} \mathrm{d}\zeta
\]
是存在的,它定义了 \( \mathbb{C} \setminus \gamma \) 上的一个函数 \( G(z) \),即
\[
G(z) = \frac{1}{2\pi \mathrm{i}} \int_{\gamma} \frac{g(\zeta)}{\zeta - z} \mathrm{d}\zeta,
\]
称它为 $\mathbf{Cauchy}$\textbf{型积分}. 
\end{definition}
\begin{note}
由Cauchy型积分确定的函数有很好的性质. 
\end{note}

\begin{theorem}\label{theorem:定理3.4.2}
设 \( \gamma \) 是 \( \mathbb{C} \) 中的可求长曲线,\( g \) 是 \( \gamma \) 上的连续函数,那么由 Cauchy型积分确定的函数
\[
G(z) = \frac{1}{2\pi \mathrm{i}} \int_{\gamma} \frac{g(\zeta)}{\zeta - z} \mathrm{d}\zeta
\]
在 \( \mathbb{C} \setminus \gamma \) 上有任意阶导数,而且
\begin{align}
G^{(n)}(z) = \frac{n!}{2\pi \mathrm{i}} \int_{\gamma} \frac{g(\zeta)}{(\zeta - z)^{n + 1}} \mathrm{d}\zeta, \, n = 1, 2, \cdots. \label{equation-----::::::4}
\end{align}
\end{theorem}
\begin{note}
这个定理实际上证明了在现在的情况下,微分运算和积分运算可以交换,公式很便于记忆.
\end{note}
\begin{proof}
我们用数学归纳法来证明等式 \eqref{equation-----::::::4}. 先设 \( n = 1 \),我们要证明
\begin{align}\label{equation-----::::::5}
G'(z) = \frac{1}{2\pi \mathrm{i}} \int_{\gamma} \frac{g(\zeta)}{(\zeta - z)^2} \mathrm{d}\zeta, \, z \in \mathbb{C} \setminus \gamma. 
\end{align}
任意取定 \( z_0 \in \mathbb{C} \setminus \gamma \),记 \( \rho = \inf_{\zeta \in \gamma} |\zeta - z_0| > 0 \),\( \delta = \min\left(1, \frac{\rho}{2}\right) \),则当 \( \zeta \in \gamma \),\( z \in B(z_0, \delta) \) 时,有 \( \left| \frac{z - z_0}{\zeta - z_0} \right| < \frac{1}{2} \). 于是由Taylor公式可得
\begin{align}\label{equation-----::::::6}
\frac{1}{\zeta -z}=\frac{1}{\zeta -z_0}\cdot \frac{1}{1-\frac{z-z_0}{\zeta -z_0}}=\frac{1}{\zeta -z_0}\cdot \sum_{n=0}^{\infty}{\left( \frac{z-z_0}{\zeta -z_0} \right) ^n}=\frac{1}{\zeta -z_0}\left( 1+\frac{z-z_0}{\zeta -z_0}+h(z,\zeta ) \right) ,
\end{align}
其中$h(z,\zeta)=\sum_{n=2}^{\infty}{\left( \frac{z-z_0}{\zeta -z_0} \right)^n}$,从而
\begin{align}\label{equation-----::::::7}
|h(z,\zeta )|\leqslant \sum_{n=2}^{\infty}{\left| \frac{z-z_0}{\zeta -z_0} \right|^n}=\left| \frac{z-z_0}{\zeta -z_0} \right|^2\sum_{n=0}^{\infty}{\left| \frac{z-z_0}{\zeta -z_0} \right|^n}<\frac{|z-z_0|^2}{\rho ^2}\sum_{n=0}^{\infty}{\left( \frac{1}{2} \right) ^n}=\frac{2}{\rho ^2}|z-z_0|^2.
\end{align}
这样,由 \eqref{equation-----::::::6}式便得
\begin{align*}
G(z) = \frac{1}{2\pi \mathrm{i}} \int_{\gamma} \frac{g(\zeta)}{\zeta - z} \mathrm{d}\zeta = \frac{1}{2\pi \mathrm{i}} \int_{\gamma} \frac{g(\zeta)}{\zeta - z_0} \mathrm{d}\zeta + \frac{z - z_0}{2\pi \mathrm{i}} \int_{\gamma} \frac{g(\zeta)}{(\zeta - z_0)^2} \mathrm{d}\zeta   + \frac{1}{2\pi \mathrm{i}} \int_{\gamma} \frac{g(\zeta)h(z, \zeta)}{\zeta - z_0} \mathrm{d}\zeta,
\end{align*}
又注意到$h(z_0,\zeta)=0$,因而有
\begin{align}
\frac{G(z) - G(z_0)}{z - z_0} - \frac{1}{2\pi \mathrm{i}} \int_{\gamma} \frac{g(\zeta)}{(\zeta - z_0)^2} \mathrm{d}\zeta  = \frac{1}{2\pi \mathrm{i} (z - z_0)} \int_{\gamma} \frac{g(\zeta)h(z, \zeta)}{\zeta - z_0} \mathrm{d}\zeta. \label{equation-----::::::8}
\end{align}
若记 \( M = \sup_{\zeta \in \gamma} |g(\zeta)| \),由 \eqref{equation-----::::::7}式便知\eqref{equation-----::::::8}式右端的绝对值不超过
\[
\frac{M |\gamma|}{\pi \rho^3 |z - z_0|} \cdot |z - z_0|^2 = \frac{M |\gamma|}{\pi \rho^3} |z - z_0|.
\]
在 \eqref{equation-----::::::8}式两端令 \( z \to z_0 \),即得
\[
G'(z_0) = \frac{1}{2\pi \mathrm{i}} \int_{\gamma} \frac{g(\zeta)}{(\zeta - z_0)^2} \mathrm{d}\zeta.
\]
现设 \( n = k \) 时 \eqref{equation-----::::::4}式成立,即
\[
G^{(k)}(z) = \frac{k!}{2\pi \mathrm{i}} \int_{\gamma} \frac{g(\zeta)}{(\zeta - z)^{k + 1}} \mathrm{d}\zeta,
\]
要证明
\[
G^{(k + 1)}(z) = \frac{(k + 1)!}{2\pi \mathrm{i}} \int_{\gamma} \frac{g(\zeta)}{(\zeta - z)^{k + 2}} \mathrm{d}\zeta.
\]
由 \eqref{equation-----::::::6}式和二项式定理,可得
\begin{align*}
\frac{1}{(\zeta - z)^{k + 1}} = \frac{1}{(\zeta - z_0)^{k + 1}} \left( 1 + \frac{z - z_0}{\zeta - z_0} + h(z, \zeta) \right)^{k + 1} = \frac{1}{(\zeta - z_0)^{k + 1}} \left( 1 + (k + 1) \frac{z - z_0}{\zeta - z_0} + H(z, \zeta) \right),
\end{align*}
由 \eqref{equation-----::::::7}式便得
\begin{align}
|H(z, \zeta)| \leqslant C |z - z_0|^2, \label{equation-----::::::9}
\end{align}
这里,\( C \) 是一个常数. 于是
\begin{align*}
G^{(k)}(z) = \frac{k!}{2\pi \mathrm{i}} \int_{\gamma} \frac{g(\zeta)}{(\zeta - z_0)^{k + 1}} \mathrm{d}\zeta + \frac{(k + 1)!}{2\pi \mathrm{i}} (z - z_0) \int_{\gamma} \frac{g(\zeta)}{(\zeta - z_0)^{k + 2}} \mathrm{d}\zeta + \frac{k!}{2\pi \mathrm{i}} \int_{\gamma} \frac{g(\zeta)H(z, \zeta)}{(\zeta - z_0)^{k + 1}} \mathrm{d}\zeta,
\end{align*}
即
\begin{align}
\frac{G^{(k)}(z) - G^{(k)}(z_0)}{z - z_0} - \frac{(k + 1)!}{2\pi \mathrm{i}} \int_{\gamma} \frac{g(\zeta)}{(\zeta - z_0)^{k + 2}} \mathrm{d}\zeta = \frac{k!}{2\pi \mathrm{i} (z - z_0)} \int_{\gamma} \frac{g(\zeta)H(z, \zeta)}{(\zeta - z_0)^{k + 1}} \mathrm{d}\zeta. \label{equation-----::::::10}
\end{align}
由 \eqref{equation-----::::::9}式便知 \eqref{equation-----::::::10}式右端的绝对值不超过 \( K |z - z_0| \),这里,\( K \) 是一个常数. 在 \eqref{equation-----::::::10} 式中令 \( z \to z_0 \),即得
\[
G^{(k + 1)}(z_0) = \frac{(k + 1)!}{2\pi \mathrm{i}} \int_{\gamma} \frac{g(\zeta)}{(\zeta - z_0)^{k + 2}} \mathrm{d}\zeta.
\]
由于 \( z_0 \) 是 \( D \) 中的任意点,归纳法证明完毕. 

\end{proof}

\begin{theorem}\label{theorem:定理3.4.3}
设 \( D \) 是由可求长简单闭曲线 \( \gamma \) 围成的区域,如果 \( f \in H(D) \cap C(\overline{D}) \),那么 \( f \) 在 \( D \) 上有任意阶导数,而且对任意 \( z \in D \),有
\[
f^{(n)}(z) = \frac{n!}{2\pi \mathrm{i}} \int_{\gamma} \frac{f(\zeta)}{(\zeta - z)^{n + 1}} \mathrm{d}\zeta, \, n = 1, 2, \cdots.
\]
\end{theorem}
\begin{proof}
{\color{blue}证法一:}
由\refthe{theorem:定理3.4.1},\( f \) 可写为 Cauchy 型积分
\[
f(z) = \frac{1}{2\pi \mathrm{i}} \int_{\gamma} \frac{f(\zeta)}{\zeta - z} \mathrm{d}\zeta.
\]
由于 \( f \) 在 \( \gamma \) 上连续,故由\refthe{theorem:定理3.4.2}即得所要证的结果. 

{\color{blue}证法二:}
设\( f(z) = u(x,y) + \mathrm{i}v(x,y) \),则由\refthe{theorem:复变函数可微的充要条件2}知
\begin{align*}
\frac{\partial u}{\partial x} = \frac{\partial v}{\partial y},\quad \frac{\partial v}{\partial x} = -\frac{\partial u}{\partial y},
\end{align*}
\( f'(z) = U(x,y) + \mathrm{i}V(x,y) \),
其中\( U(x,y) = \frac{\partial u}{\partial x} \),\( V(x,y) = \frac{\partial v}{\partial x} \)。又由\nrefpro{proposition:全纯函数必任意连续可微}{(3)}知,\( f \)有任意阶的连续偏导数。从而\( u(x,y) \),\( v(x,y) \)任意阶可微,并且
\begin{align*}
\frac{\partial U}{\partial x} = \frac{\partial^2 u}{\partial x^2} = \frac{\partial^2 v}{\partial y \partial x} = \frac{\partial^2 v}{\partial x \partial y} = \frac{\partial V}{\partial y},
\end{align*}
\begin{align*}
\frac{\partial V}{\partial x} = \frac{\partial^2 v}{\partial x^2} = -\frac{\partial^2 u}{\partial y \partial x} = -\frac{\partial^2 u}{\partial x \partial y} = -\frac{\partial U}{\partial y}.
\end{align*}
因此\( f' \)满足Cauchy-Riemann方程。故由\refthe{theorem:复变函数可微的充要条件2}知\( f' \in H(D) \)。再利用数学归纳法易知,对\( \forall n \in \mathbb{N} \),都有\( f^{(n)} \in H(D) \)。

\end{proof}

\begin{theorem}\label{theorem:定理3.4.4}
如果 \( f \) 是区域 \( D \) 上的全纯函数,那么 \( f \) 在 \( D \) 上有任意阶导数.
\end{theorem}
\begin{proof}
任取 \( z_0 \in D \),取充分小的 \( \delta \),使得 \( \overline{B(z_0, \delta)} \subset D \). 由\refthe{theorem:定理3.4.3},\( f \) 在 \( B(z_0, \delta) \) 中有任意阶导数,又由于 \( z_0 \) 是任意的,所以 \( f \) 在 \( D \) 中有任意阶导数.

\end{proof}

\begin{example}
计算积分
\[
\int_{|z| = 2} \frac{\mathrm{d}z}{z^2(z^2 + 16)}.
\]
\end{example}
\begin{solution}
令 \( f(z) = \frac{1}{z^2 + 16} \),则 \( f \) 在 \( \{z: |z| \leqslant 2\} \) 中全纯,根据\refthe{theorem:定理3.4.3},有
\begin{align*}
\int_{|z| = 2} \frac{\mathrm{d}z}{z^2(z^2 + 16)} = 2\pi \mathrm{i} \left( \frac{1}{z^2 + 16} \right)' \bigg|_{z = 0} = 0.
\end{align*}

也可以这样计算:
\begin{align*}
\int_{|z| = 2} \frac{\mathrm{d}z}{z^2(z^2 + 16)} = \frac{1}{16} \left\{ \int_{|z| = 2} \frac{\mathrm{d}z}{z^2} - \int_{|z| = 2} \frac{\mathrm{d}z}{z^2 + 16} \right\} = 0.
\end{align*}
这是因为,由\refpro{proposition:例3.1.4},第一个积分为零;由 \hyperref[theorem:Cauchy-Goursat定理(Cauchy积分定理)]{Cauchy 积分定理},第二个积分为零. 

\end{solution}

\begin{theorem}\label{theorem:定理3.4.6}
设 \( \gamma_0, \gamma_1, \cdots, \gamma_k \) 是 \( k + 1 \) 条可求长简单闭曲线,\( \gamma_1, \cdots, \gamma_k \) 都在 \( \gamma_0 \) 的内部,\( \gamma_1, \cdots, \gamma_k \) 中的每一条都在其他 \( k - 1 \) 条的外部,\( D \) 是由这 \( k + 1 \) 条曲线围成的区域,\( D \) 的边界 \( \gamma \) 由 \( \gamma_0, \gamma_1, \cdots, \gamma_k \) 所组成. 如果 \( f \in H(D) \cap C(\overline{D}) \),则对任意 \( z \in D \),有
\[
f(z) = \frac{1}{2\pi \mathrm{i}} \int_{\gamma} \frac{f(\zeta)}{\zeta - z} \mathrm{d}\zeta.
\]
\( f \) 在 \( D \) 内有任意阶导数,且
\[
f^{(n)}(z) = \frac{n!}{2\pi \mathrm{i}} \int_{\gamma} \frac{f(\zeta)}{(\zeta - z)^{n + 1}} \mathrm{d}\zeta, \, n = 1, 2, \cdots.
\]
\end{theorem}
\begin{proof}
定理的证明和前面的一样,不再重复.根据\refthe{theorem:定理3.4.2}的结论,再利用\refthe{theorem:定理3.2.5}的证明思路进行证明即可.

\end{proof}

\begin{example}
计算积分
\[
\int_{|z| = 2} \frac{\mathrm{d}z}{(z^3 - 1)(z + 4)^2}.
\]
\end{example}
\begin{solution}
注意到\( f(z) = \frac{1}{z^3 - 1} \)在$1,e^{\frac{2\pi \mathrm{i}}{3}},e^{\frac{4\pi \mathrm{i}}{3}}$处不解析.作一个中心在原点、半径为 \( R(R > 4) \) 的大圆(\reffig{figure:图3.9}),则在闭圆环
\[
\{z: 2 \leqslant |z| \leqslant R\}
\]
上,\( f(z) = \frac{1}{z^3 - 1} \) 是全纯的. 于是,由\refthe{theorem:定理3.4.6} 得
\begin{align*}
\int_{\gamma_1^-} \frac{\mathrm{d}z}{(z^3 - 1)(z + 4)^2} + \int_{\gamma_2} \frac{\mathrm{d}z}{(z^3 - 1)(z + 4)^2} 
= 2\pi \mathrm{i} \left( \frac{1}{z^3 - 1} \right)' \bigg|_{z = -4} = -\frac{32}{1323}\pi \mathrm{i},
\end{align*}
其中$\gamma_1=\{z:|z|=2\}$,$\gamma_2=\{z:|z|=R\}$.所以
\begin{align}
-\int_{|z|=2}{\frac{\mathrm{d}z}{(z^3-1)(z+4)^2}}+\int_{|z|=R}{\frac{\mathrm{d}z}{(z^3-1)(z+4)^2}}=-\frac{32\pi \mathrm{i}}{1323} \nonumber
\\
\Longleftrightarrow \int_{|z|=2}{\frac{\mathrm{d}z}{(z^3-1)(z+4)^2}}=\frac{32\pi \mathrm{i}}{1323}+\int_{|z|=R}{\frac{\mathrm{d}z}{(z^3-1)(z+4)^2}}.\label{equation---::11}
\end{align}
由于当 \( |z| = R \) 时,有
\[
|(z^3 - 1)(z + 4)^2| \geqslant (R^3 - 1)(R - 4)^2,
\]
所以由\hyperref[proposition:长大不等式]{长大不等式}得
\[
\left| \int_{|z| = R} \frac{\mathrm{d}z}{(z^3 - 1)(z + 4)^2} \right| \leqslant \frac{2\pi R}{(R^3 - 1)(R - 4)^2} \to 0 \, (R \to \infty).
\]
故在\eqref{equation---::11}式中令 \( R \to \infty \),即得
\[
\int_{|z| = 2} \frac{\mathrm{d}z}{(z^3 - 1)(z + 4)^2} = \frac{32\pi \mathrm{i}}{1323}.
\]
\begin{figure}[H]
\centering
\includegraphics[scale=0.3]{图3.9.png}
\caption{}
\label{figure:图3.9}
\end{figure}

\end{solution}

\begin{theorem}[Schwarz积分公式]\label{theorem:Schwarz积分公式}
设 \( f \in H(B(0,R)) \cap C(\overline{B(0,R)}) \),\( f = u + \mathrm{i}v \)。证明:\( f \) 可用实部表示为
\[
f(z) = \frac{1}{2\pi} \int_{0}^{2\pi} \frac{Re^{\mathrm{i}\theta} + z}{Re^{\mathrm{i}\theta} - z} u(Re^{\mathrm{i}\theta}) \mathrm{d}\theta + \mathrm{i}v(0).
\]
\end{theorem}
\begin{proof}


\end{proof}















\end{document}
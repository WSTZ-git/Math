\documentclass[../../main.tex]{subfiles}
\graphicspath{{\subfix{./image/}}} % 指定图片目录,后续可以直接使用图片文件名
% 注意这里的文件路径不能用 ../../image/ ,否则用latexmk编译子文件会报错

% 例如:
% \begin{figure}[H]
% \centering
% \includegraphics[scale=0.4]{图.png}
% \caption{}
% \label{figure:图}
% \end{figure}
% 注意:上述\label{}一定要放在\caption{}之后,否则引用图片序号会只会显示??.

\begin{document}

\section{Cauchy积分定理}

\begin{theorem}[Cauchy定理]\label{theorem:Cauchy定理}
设 \( D \) 是 \( \mathbb{C} \) 中的单连通区域,\( f \in H(D) \),且 \( f' \) 在 \( D \) 中连续,则对 \( D \) 中任意的可求长闭曲线 \( \gamma \),均有
\[
\int_\gamma f(z)\mathrm{d}z = 0.
\]
\end{theorem}
\begin{proof}  
由 \( \gamma \) 围成的区域记为 \( G \),因为 \( f' \) 连续,即 \( \frac{\partial u}{\partial x}, \frac{\partial v}{\partial x}, \frac{\partial u}{\partial y}, \frac{\partial v}{\partial y} \) 连续,故可用 Green 公式. 又因 \( f \) 在 \( D \) 中全纯,故由\refthe{theorem:复变函数可微的充要条件2}可知 Cauchy-Riemann 方程成立. 于是由 Green公式可得
\[
\int_\gamma u \mathrm{d}x - v \mathrm{d}y = \iint_G \left( -\frac{\partial v}{\partial x} - \frac{\partial u}{\partial y} \right) \mathrm{d}x\mathrm{d}y = 0,
\]
\[
\int_\gamma v \mathrm{d}x + u \mathrm{d}y = \iint_G \left( \frac{\partial u}{\partial x} - \frac{\partial v}{\partial y} \right) \mathrm{d}x\mathrm{d}y = 0.
\]
由\refpro{proposition:连续复变函数积分必存在},即得
\[
\int_\gamma f(z)\mathrm{d}z = 0.
\]

\end{proof}

\begin{lemma}\label{lemma:引理:3.2.2}
设 \( f \) 是区域 \( D \) 中的连续函数,\( \gamma \) 是 \( D \) 内的可求长曲线. 对于任给的 \( \varepsilon > 0 \),一定存在一条 \( D \) 中的折线 \( P \),使得

(i) \( P \) 和 \( \gamma \) 有相同的起点和终点,\( P \) 中其他的顶点都在 \( \gamma \) 上;

(ii) \( \left| \int_\gamma f(z)\mathrm{d}z - \int_P f(z)\mathrm{d}z \right| < \varepsilon \).
\end{lemma}
\begin{proof}
因为 \( \partial D \) 是一个闭集,\( \gamma \) 是一个紧集,且两者不相交,根据\refthe{theorem:紧集和闭集无交则距离大于0},\( d(\gamma, \partial D) = \rho > 0 \). 作有界的区域\( G \),使得 \( \gamma \subset \overline{G} \subset D \). 因为 \( f \) 在紧集\( \overline{G} \) 上连续,故由\nrefthe{theorem:连续的复变函数在紧集上的性质}{(iii)}可知$f$必一致连续. 于是,对任意 \( \varepsilon > 0 \),存在 \( \delta > 0 \),当 \( z', z'' \in \overline{G} \),\( |z' - z''| < \delta \) 时,\( |f(z') - f(z'')| < \frac{\varepsilon}{2L} \),这里,\( L \) 是 \( \gamma \) 的长度. 现取 \( \eta = \min(\rho, \delta) \). 在 \( \gamma \) 上取分点 \( z_0, z_1, \cdots, z_n \),使得每一个弧段 \( \wideparen{z_{k - 1}z_k} \) 的长度都小于 \( \eta \),这里,\( z_0, z_n \) 分别记为 \( \gamma \) 的起点和终点. 连接 \( z_{k - 1} \) 和 \( z_k \)(\( k = 1, \cdots, n \)),就得到一条折线 \( P \),它与 \( \gamma \) 有相同的起点和终点,且其他顶点都在 \( \gamma \) 上. 由于 \( |z_{k - 1} - z_k| < \eta \leqslant \rho \),所以线段 \( \overline{z_{k - 1}z_k} \) 都在 \( D \) 内,即折线 \( P \) 都在 \( D \) 内.

现在估计下面的积分差,记 \( \gamma_k = \wideparen{z_{k - 1}z_k} \),\( P_k = \overline{z_{k - 1}z_k} \),则有
\begin{align*}
\left| \int_{\gamma_k} f(z)\mathrm{d}z - \int_{P_k} f(z)\mathrm{d}z \right|
&\leqslant \left| \int_{\gamma_k} f(z)\mathrm{d}z - f(z_{k - 1})(z_k - z_{k - 1}) \right|
+ \left| \int_{P_k} f(z)\mathrm{d}z - f(z_{k - 1})(z_k - z_{k - 1}) \right|
\\
&= \left| \int_{\gamma_k} f(z)\mathrm{d}z - \int_{\gamma_k} f(z_{k - 1})\mathrm{d}z \right| + \left| \int_{P_k} f(z)\mathrm{d}z - \int_{P_k} f(z_{k - 1})\mathrm{d}z \right|
\\
&= \left| \int_{\gamma_k} (f(z) - f(z_{k - 1}))\mathrm{d}z \right| + \left| \int_{P_k} (f(z) - f(z_{k - 1}))\mathrm{d}z \right|.
\end{align*}
当 \( z \in \gamma_k \) 或 \( P_k \) 时,都有 \( |z - z_{k - 1}| < \eta \leqslant \delta \),因而 \( |f(z) - f(z_{k - 1})| < \frac{\varepsilon}{2L} \). 对上面两个积分用\hyperref[proposition:长大不等式]{长大不等式},它们都不超过 \( \frac{\varepsilon}{2L}|\gamma_k| \),因而
\[
\left| \int_\gamma f(z)\mathrm{d}z - \int_P f(z)\mathrm{d}z \right| \leqslant \sum_{k = 1}^n \left| \int_{\gamma_k} f(z)\mathrm{d}z - \int_{P_k} f(z)\mathrm{d}z \right|
< \frac{\varepsilon}{L} \sum_{k = 1}^n |\gamma_k|
= \varepsilon.
\]
故折线 \( P \) 完全符合定理的要求.

\end{proof}

\begin{theorem}[Cauchy-Goursat定理(Cauchy积分定理)]\label{theorem:Cauchy-Goursat定理(Cauchy积分定理)}
设 \( D \) 是 \( \mathbb{C} \) 中的单连通区域,如果 \( f \in H(D) \),那么对 \( D \) 中任意的可求长闭曲线 \( \gamma \),均有
\[
\int_\gamma f(z)\mathrm{d}z = 0.
\]
\end{theorem}
\begin{remark}
注意,对于非单连通的区域,定理不一定成立. 例如,\( D \) 是除去原点的单位圆盘,\( f(z) = \frac{1}{z} \) 当然在 \( D \) 中全纯,若设 \( \gamma = \{ z: |z| = r < 1 \} \),则由\refpro{proposition:例3.1.4}知,\( \int_\gamma \frac{\mathrm{d}z}{z} = 2\pi \mathrm{i} \neq 0 \).
\end{remark}
\begin{proof}
证明分为下面三步:

(1) 先假定 \( \gamma \) 是一个三角形的边界.

如果 \( \left| \int_\gamma f(z)\mathrm{d}z \right| = M \),我们证明 \( M = 0 \). 连接三角形三边的中点,把三角形分成四个全等的小三角形(\reffig{figure:图3.2}),
\begin{figure}[H]
\centering
\includegraphics[scale=0.3]{图3.2.png}
\caption{}
\label{figure:图3.2}
\end{figure}
这四个小三角形的边界分别记为 \( \gamma^{(1)}, \gamma^{(2)}, \gamma^{(3)} \) 和 \( \gamma^{(4)} \). 让 \( f \) 沿这四个小三角形的边界积分,从图中可以看出,中间那个小三角形的边界被来回走了两次,\( f \) 在其上的积分恰好抵消,剩下的积分的和正好等于大三角形边界上的积分,即
\[
\int_\gamma f(z)\mathrm{d}z = \int_{\gamma^{(1)}} f(z)\mathrm{d}z + \int_{\gamma^{(2)}} f(z)\mathrm{d}z
+ \int_{\gamma^{(3)}} f(z)\mathrm{d}z + \int_{\gamma^{(4)}} f(z)\mathrm{d}z,
\]
或者
\[
M = \left| \int_\gamma f(z)\mathrm{d}z \right|
\leqslant \left| \int_{\gamma^{(1)}} f(z)\mathrm{d}z \right| + \left| \int_{\gamma^{(2)}} f(z)\mathrm{d}z \right|
+ \left| \int_{\gamma^{(3)}} f(z)\mathrm{d}z \right| + \left| \int_{\gamma^{(4)}} f(z)\mathrm{d}z \right|.
\]
因此上述四个小三角形中必有一个小三角形 \( \Delta_1 \),它的边界记为 \( \gamma_1 \),\( f \) 在其上的积分满足 \( \left| \int_{\gamma_1} f(z)\mathrm{d}z \right| \geqslant \frac{M}{4} \). 把 \( \Delta_1 \) 再分成四个全等的小三角形,按照同样的推理,其中又有一个小三角形 \( \Delta_2 \),它的边界记为 \( \gamma_2 \),\( f \) 在其上的积分满足 \( \left| \int_{\gamma_2} f(z)\mathrm{d}z \right| \geqslant \frac{M}{4^2} \). 这个过程可以一直进行下去,我们得到一串三角形 \( \Delta_n \),记它们的边界为 \( \gamma_n \),这串三角形具有下列性质:

(i) \( \Delta \supset \Delta_1 \supset \cdots \supset \Delta_n \supset \cdots \);

(ii) \( \text{diam}\Delta_n \to 0 \)(\( n \to \infty \));

(iii) \( |\gamma_n| = \frac{L}{2^n} \),\( n = 1, 2, \cdots \),这里,\( L \) 为 \( \gamma \) 的长度;

(iv) \( \left| \int_{\gamma_n} f(z)\mathrm{d}z \right| \geqslant \frac{M}{4^n} \),\( n = 1, 2, \cdots \).

由 (i) 和 (ii),根据\hyperref[theorem:复变函数--Cantor闭集套定理]{Cantor闭集套定理},存在唯一的 \( z_0 \in \Delta_n \)(\( n = 1, 2, \cdots \)). 因为 \( D \) 是单连通的,所以 \( z_0 \in D \). 由于 \( f \) 在 \( z_0 \) 处全纯,故对任意 \( \varepsilon > 0 \),存在 \( \delta > 0 \),当 \( 0 < |z - z_0| < \delta \) 时,成立
\[
\left| \frac{f(z) - f(z_0)}{z - z_0} - f'(z_0) \right| < \varepsilon,
\]
即
\begin{align}\label{equation--------::1}
|f(z) - f(z_0) - f'(z_0)(z - z_0)| < \varepsilon |z - z_0|.
\end{align}
取 \( n \) 充分大,使得 \( \Delta_n \subset B(z_0, \delta) \),故当 \( z \in \gamma_n \) 时,\eqref{equation--------::1}式成立. 显然,\( z \in \gamma_n \) 时,\( |z - z_0| < |\gamma_n| = \frac{L}{2^n} \). 因而,当 \( z \in \gamma_n \) 时,有
\begin{align}\label{equation--------::2}
|f(z) - f(z_0) - f'(z_0)(z - z_0)| < \frac{\varepsilon L}{2^n}. 
\end{align}
因为 \( \gamma_n \) 是闭曲线,由\refexa{example:例3.1.3}知道,有
\[
\int_{\gamma_n} \mathrm{d}z = 0,
\quad
\int_{\gamma_n} z \mathrm{d}z = 0.
\]
于是有
\[
\int_{\gamma_n} [f(z) - f(z_0) - f'(z_0)(z - z_0)]\mathrm{d}z
= \int_{\gamma_n} f(z)\mathrm{d}z - f(z_0)\int_{\gamma_n} \mathrm{d}z - f'(z_0)\int_{\gamma_n} z \mathrm{d}z + z_0 f'(z_0)\int_{\gamma_n} \mathrm{d}z
= \int_{\gamma_n} f(z)\mathrm{d}z.
\]
利用\eqref{equation--------::2}式、(iii) 和\hyperref[proposition:长大不等式]{长大不等式},即得
\[
\left| \int_{\gamma_n} f(z)\mathrm{d}z \right| \leqslant \frac{\varepsilon L}{2^n} |\gamma_n| = \varepsilon \left( \frac{L}{2^n} \right)^2.
\]
再由 (iv),可得 \( M \leqslant \varepsilon L^2 \). 又因为 \( \varepsilon \) 是任意小的正数,所以 \( M = 0 \).

(2) 假定 \( \gamma \) 是一个多边形的边界.
\begin{figure}[H]
\centering
\includegraphics[scale=0.3]{图3.3.png}
\caption{}
\label{figure:图3.3}
\end{figure}
从\reffig{figure:图3.3}可以看出,我们可以把多边形分解成若干个三角形. 与刚才的道理一样,\( f \) 沿 \( \gamma \) 的积分等于沿各个三角形边界积分的和,由 \eqref{equation--------::1} 已知沿三角形边界的积分为零,因而
\[
\int_\gamma f(z)\mathrm{d}z = 0.
\]

(3) 假定 \( \gamma \) 是一般的可求长闭曲线.

根据\reflem{lemma:引理:3.2.2},在 \( D \) 内存在闭折线 \( P \),使得
\begin{align}\label{equation--------::3}
\left| \int_\gamma f(z)\mathrm{d}z - \int_P f(z)\mathrm{d}z \right| < \varepsilon,
\end{align}
这里,\( \varepsilon \) 是任意事先给定的正数. 由\eqref{equation--------::3}式和(2)即知
\[
\int_\gamma f(z)\mathrm{d}z = 0.
\]

\end{proof}

\begin{corollary}\label{corollary:复积分的积分与路径无关}
设函数 $f(z)$ 在 $z$ 平面上的单连通区域 $D$ 内解析,则 $f(z)$ 在 $D$ 内积分与路径无关.即对 $D$ 内任意两点 $z_0$ 与 $z_1$,积分
$$\int_{z_0}^{z_1} f(z) \mathrm{d}z$$
之值,不依赖于 $D$ 内连接起点 $z_0$ 与终点 $z_1$ 的曲线.
\end{corollary}
\begin{proof}
设 $C_1$ 与 $C_2$ 是 $D$ 内连接起点 $z_0$ 与终点 $z_1$ 的任意两条曲线(如\reffig{figure:图3...4}).则正方向曲线 $C_1$ 与负方向曲线 $C_2^-$ 就衔接成 $D$ 内的一条闭曲线 $C$.于是,由\hyperref[theorem:Cauchy-Goursat定理(Cauchy积分定理)]{Cauchy积分定理}与\hyperref[proposition:复积分的基本性质]{复积分的基本性质(3)},有
\begin{align*}
0 &= \int_{C} f(z) \mathrm{d}z = \int_{C_1} f(z) \mathrm{d}z + \int_{C_2^-} f(z) \mathrm{d}z, 
\end{align*}
因而
\begin{align*}
\int_{C_1} f(z) \mathrm{d}z &= \int_{C_2} f(z) \mathrm{d}z. 
\end{align*}
\begin{figure}[H]
\centering
\includegraphics[scale=0.4]{图3...4.png}
\caption{}
\label{figure:图3...4}
\end{figure}

\end{proof}

\begin{theorem}\label{theorem:定理3.2.4}
设 \( D \) 是可求长简单闭曲线 \( \gamma \) 的内部,若 \( f \in H(D) \cap C(\overline{D}) \),则
\[
\int_\gamma f(z)\mathrm{d}z = 0.
\]
\end{theorem}
\begin{remark}
这里已不再假定 \( f \) 在积分路径 \( \gamma \) 上全纯,而代之以在闭域 \( \overline{D} \) 上连续,条件确实是减弱了. 一般地,证明这个定理还需要一些其他的知识,我们这里对 \( \gamma \) 附加两个条件:

(i) \( \gamma \) 是逐段光滑的;

(ii) 在 \( D \) 中存在点 \( z_0 \),使得从 \( z_0 \) 出发的每条射线与 \( \gamma \) 只有一个交点. 例如,凸多边形和圆盘都满足这两个条件.
\end{remark}
\begin{proof}
在所设的两个条件下,\( \gamma \) 的方程可以写成
\[
z = z_0 + \lambda(t),\ a \leqslant t \leqslant b.
\]
记
\[
p = \max\{ |\lambda(t)| : a \leqslant t \leqslant b \},
\quad
q = \max\{ |\lambda'(t)| : a \leqslant t \leqslant b \}.
\]
由于 \( f \) 在 \( \overline{D} \) 上连续,故必一致连续,故对任意的 \( \varepsilon > 0 \),存在 \( \delta > 0 \),当 \( z_1, z_2 \in \overline{D} \),且 \( |z_1 - z_2| < \delta \) 时,有 \( |f(z_1) - f(z_2)| < \varepsilon \). 今取 \( \delta_0 < \min(\delta, p) \),于是 \( \frac{\delta_0}{p} < 1 \). 取 \( \rho \),使得 \( 1 - \frac{\delta_0}{p} < \rho < 1 \). 记 \( \gamma_\rho \) 为曲线
\[
z = z_0 + \rho\lambda(t),\ a \leqslant t \leqslant b,
\]
则显然有 \( \gamma_\rho \subset D \). 由\hyperref[theorem:Cauchy-Goursat定理(Cauchy积分定理)]{Cauchy-Goursat定理},成立
\[
\int_{\gamma_\rho} f(z)\mathrm{d}z = \int_a^b f(z_0 + \rho\lambda(t))\rho\lambda'(t)\mathrm{d}t = 0,
\]
即
\[
\int_a^b f(z_0 + \rho\lambda(t))\lambda'(t)\mathrm{d}t = 0.
\]
由于
\[
|(z_0 + \rho\lambda(t)) - (z_0 + \lambda(t))| = (1 - \rho)|\lambda(t)|
\leqslant (1 - \rho)p
< \delta_0 < \delta,
\]
所以
\[
|f(z_0 + \lambda(t)) - f(z_0 + \rho\lambda(t))| < \varepsilon.
\]
于是
\begin{align*}
\left| \int_\gamma f(z)\mathrm{d}z \right| &= \left| \int_a^b f(z_0 + \lambda(t))\lambda'(t)\mathrm{d}t \right|
= \left| \int_a^b [f(z_0 + \lambda(t)) - f(z_0 + \rho\lambda(t))]\lambda'(t)\mathrm{d}t \right|
\\
&\leqslant \int_a^b |f(z_0 + \lambda(t)) - f(z_0 + \rho\lambda(t))||\lambda'(t)|\mathrm{d}t
< \varepsilon q(b - a).
\end{align*}
由于 \( \varepsilon > 0 \) 是任意的,所以
\[
\int_\gamma f(z)\mathrm{d}z = 0.
\]

\end{proof}

\begin{theorem}[多连通区域的Cauchy积分定理]\label{theorem:定理3.2.5}
设 \( \gamma_0, \gamma_1, \cdots, \gamma_n \) 是 \( n + 1 \) 条可求长简单闭曲线,\( \gamma_1, \cdots, \gamma_n \) 都在 \( \gamma_0 \) 的内部,\( \gamma_1, \cdots, \gamma_n \) 中的每一条都在其他 \( n - 1 \) 条的外部,\( D \) 是由这 \( n + 1 \) 条曲线围成的域,用 \( \gamma \) 记 \( D \) 的边界. 如果 \( f \in H(D) \cap C(\overline{D}) \),那么
\begin{align}\label{equation-------..4}
\int_\gamma f(z)\mathrm{d}z = 0,
\end{align}
这里,积分沿 \( \gamma \) 的正方向进行,并且$\gamma = \gamma_0+\gamma_1^-+\cdots+\gamma_n^-$.\eqref{equation-------..4}式也可写为
\begin{align}\label{equation-------..5}
\int_{\gamma_0} f(z)\mathrm{d}z = \int_{\gamma_1} f(z)\mathrm{d}z + \cdots + \int_{\gamma_n} f(z)\mathrm{d}z,
\end{align}
\eqref{equation-------..5}式右端的积分分别沿 \( \gamma_1, \cdots, \gamma_n \) 的逆时针方向进行.
\end{theorem}
\begin{proof}
如\reffig{figure:图3.4}所示,我们用一些辅助线把几个“洞”连接起来,这样,\( D \) 就被分成若干个单连通区域. 由\refthe{theorem:定理3.2.4},沿每个单连通区域的边界的积分为零,若干个单连通区域的边界积分之和仍为零. 由于在辅助线上的积分来回各进行一次,正好抵消,所以总和恰好就是 \( \gamma \) 上的积分,因而 \eqref{equation-------..4}式成立. 而
\[
\int_\gamma f(z)\mathrm{d}z = \int_{\gamma_0} f(z)\mathrm{d}z + \int_{\gamma_1^-} f(z)\mathrm{d}z + \cdots + \int_{\gamma_n^-} f(z)\mathrm{d}z,
\]
移项即得\eqref{equation-------..5}式.
\begin{figure}[H]
\centering
\includegraphics[scale=0.4]{图3.4.png}
\caption{}
\label{figure:图3.4}
\end{figure}

\end{proof}

\begin{corollary}\label{corollary:定理3.2.5推论}
设 \( \gamma_0 \) 和 \( \gamma_1 \) 是两条可求长的简单闭曲线,\( \gamma_1 \) 在 \( \gamma_0 \) 的内部,\( D \) 是由 \( \gamma_0 \) 和 \( \gamma_1 \) 围成的区域. 如果 \( f \in H(D) \cap C(\overline{D}) \),那么
\[
\int_{\gamma_0} f(z)\mathrm{d}z = \int_{\gamma_1} f(z)\mathrm{d}z.
\]
\end{corollary}
\begin{proof}
由\refthe{theorem:定理3.2.5}中$n=1$的情况立得.

\end{proof}

\begin{example}\label{example:例3.2.7}
设 \( \gamma \) 是一可求长简单闭曲线,\( a \notin \gamma \),试计算积分
\[
\int_\gamma \frac{\mathrm{d}z}{z - a}.
\]
\end{example}
\begin{solution}
若 \( a \) 在 \( \gamma \) 的外部,则因 \( \frac{1}{z - a} \) 在 \( \gamma \) 围成的闭域上全纯,所以由 \hyperref[theorem:Cauchy-Goursat定理(Cauchy积分定理)]{Cauchy 积分定理},\( \int_\gamma \frac{\mathrm{d}z}{z - a} = 0 \).

若 \( a \) 在 \( \gamma \) 的内部,则有充分小的 \( r > 0 \),使得 \( B(a, r) \) 落在 \( \gamma \) 的内部(\reffig{figure:图3.5}). 记 \( B(a, r) \) 的边界为 \( \gamma_1 \),由 \( \gamma \) 和 \( \gamma_1 \) 围成的区域记为 \( D \),则 \( \frac{1}{z - a} \) 在 \( \overline{D} \) 上全纯,因而由\refcor{corollary:定理3.2.5推论},得
\[
\int_\gamma \frac{\mathrm{d}z}{z - a} = \int_{\gamma_1} \frac{\mathrm{d}z}{z - a} = 2\pi \mathrm{i}.
\]
最后的等式利用了\refpro{proposition:例3.1.4}的结果.

\begin{figure}[H]
\centering
\includegraphics[scale=0.4]{图3.5.png}
\caption{}
\label{figure:图3.5}
\end{figure}

\end{solution}

\begin{example}
设 \( \gamma \) 是一可求长简单闭曲线,\( a, b \notin \gamma \),试计算积分
\[
I = \int_\gamma \frac{\mathrm{d}z}{(z - a)(z - b)}.
\]
\end{example}
\begin{solution}
上面的积分可写为
\[
I = \int_\gamma \frac{\mathrm{d}z}{(z - a)(z - b)}
= \frac{1}{a - b} \left( \int_\gamma \frac{\mathrm{d}z}{z - a} - \int_\gamma \frac{\mathrm{d}z}{z - b} \right).
\]
由\refexa{example:例3.2.7}即可得
\[
I = 
\begin{cases} 
0&,  \text{若 } a, b \text{ 都在 } \gamma \text{ 的外部}; \\
\frac{2\pi \mathrm{i}}{a - b}&,  \text{若 } a \text{ 在 } \gamma \text{ 的内部}, b \text{ 在 } \gamma \text{ 的外部}; \\
-\frac{2\pi \mathrm{i}}{a - b}&,  \text{若 } a \text{ 在 } \gamma \text{ 的外部}, b \text{ 在 } \gamma \text{ 的内部}; \\
0&,  \text{若 } a, b \text{ 都在 } \gamma \text{ 的内部}.
\end{cases}
\]

\end{solution}










\end{document}
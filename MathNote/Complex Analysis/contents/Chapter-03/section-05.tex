\documentclass[../../main.tex]{subfiles}
\graphicspath{{\subfix{../../image/}}} % 指定图片目录,后续可以直接使用图片文件名。

% 例如:
% \begin{figure}[H]
% \centering
% \includegraphics[scale=0.4]{image-01.01}
% \caption{图片标题}
% \label{figure:image-01.01}
% \end{figure}
% 注意:上述\label{}一定要放在\caption{}之后,否则引用图片序号会只会显示??.

\begin{document}

\section{Cauchy积分公式的一些重要推论}

\begin{theorem}[Cauchy不等式]\label{theorem:Cauchy不等式-复变函数}
设 \( f \) 在 \( B(a,R) \) 中全纯,且对任意 \( z \in B(a,R) \),有 \( |f(z)| \leqslant M \),那么
\begin{align}
|f^{(n)}(a)| &\leqslant \frac{n!M}{R^n}, \quad n = 1,2,\cdots. \label{eq:3.5.1}
\end{align}
\end{theorem}
\begin{note}
这个不等式给出了圆盘上全纯函数的各阶导数在圆心处值的估计.
\end{note}
\begin{proof}
取 \( 0 < r < R \),则 \( f \) 在闭圆盘 \( \overline{B(a,r)} \) 中全纯,由\refthe{theorem:定理3.4.3},得
\[
f^{(n)}(a) = \frac{n!}{2\pi \mathrm{i}} \int_{|\zeta - a| = r} \frac{f(\zeta)}{(\zeta - a)^{n + 1}} \mathrm{d}\zeta.
\]
于是,由\hyperref[proposition:长大不等式]{长大不等式}得
\[
|f^{(n)}(a)| \leqslant \frac{n!}{2\pi} \cdot \frac{M}{r^{n + 1}} \cdot 2\pi r = \frac{n!M}{r^n}.
\]
让 \( r \to R \),即得所要证的不等式 \eqref{eq:3.5.1}。
\end{proof}

\begin{theorem}[Liouville定理]\label{theorem:Liouville(刘维尔)定理}
有界整函数必为常数。
\end{theorem}
\begin{proof}
设 \( f \) 为一有界整函数,其模的上界设为 \( M \),即对任意 \( z \in \mathbb{C} \),有 \( |f(z)| \leqslant M \)。任取 \( a \in \mathbb{C} \),以 \( a \) 为中心、\( R \) 为半径作圆,因为 \( f \) 为整函数,故由 \hyperref[theorem:Cauchy不等式-复变函数]{Cauchy 不等式}可得
\[
|f'(a)| \leqslant \frac{M}{R}.
\]
这个不等式对任意 \( R > 0 \) 都成立,让 \( R \to \infty \),即得 \( f'(a) = 0 \)。因为 \( a \) 是任意的,所以在全平面上有 \( f'(z) \equiv 0 \),因而由\refpro{proposition:复变函数导数为0必是常函数}可知\( f \) 是常数。
\end{proof}

\begin{theorem}[代数学基本定理]\label{theorem:代数学基本定理}
任意复系数多项式
\[
P(z) = a_0 z^n + a_1 z^{n - 1} + \cdots + a_n, \quad a_0 \neq 0
\]
在 \( \mathbb{C} \) 中必有零点。
\end{theorem}
\begin{note}
考虑到实系数多项式在实数域中未必有零点,这个定理给出了复数域的又一重要性质。
\end{note}
\begin{proof}
如果 \( P(z) \) 在 \( \mathbb{C} \) 中没有零点,那么 \( f(z) = \frac{1}{P(z)} \) 是一个整函数。由于 \( \lim_{z \to \infty} P(z) = \infty \),故当 \( |z| > R \) 时,\( |f(z)| \leqslant 1 \);而当 \( |z| \leqslant R \) 时,\( f \) 是有界的,因而 \( f \) 是一有界整函数。由\hyperref[theorem:Liouville(刘维尔)定理]{Liouville 定理},\( f \) 应是一常数。这个矛盾证明了 \( P \) 在 \( \mathbb{C} \) 中必有零点。
\end{proof}

\begin{theorem}[Morera定理]\label{theorem:Morera(莫雷拉)定理}
如果 \( f \) 是域 \( D \) 上的连续函数,且沿 \( D \) 内任一可求长闭曲线的积分为零,那么 \( f \) 在 \( D \) 上全纯。
\end{theorem}
\begin{note}
这个定理是\hyperref[theorem:Cauchy-Goursat定理(Cauchy积分定理)]{Cauchy积分定理的逆定理}.
\end{note}
\begin{proof}
由\refthe{theorem:定理3.3.2},存在 \( F \in H(D) \),使得 \( F'(z) = f(z) \) 在 \( D \) 中成立。由\refthe{theorem:定理3.4.4},\( F' \) 是 \( D \) 中的全纯函数,所以\(F' = f \) 也是全纯函数。
\end{proof}

















\end{document}
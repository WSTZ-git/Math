\documentclass[../../main.tex]{subfiles}
\graphicspath{{\subfix{../../image/}}} % 指定图片目录,后续可以直接使用图片文件名。

% 例如:
% \begin{figure}[H]
% \centering
% \includegraphics[scale=0.4]{图.png}
% \caption{}
% \label{figure:图}
% \end{figure}
% 注意:上述\label{}一定要放在\caption{}之后,否则引用图片序号会只会显示??.

\begin{document}

\section{一维$\overline{\partial }$的解}

\begin{definition}
所谓一维 \( \bar{\partial} \) 问题,是指在域 \( D \) 上给定一个函数 \( f \),要求函数 \( u \),使得在 \( D \) 上有
\[
\frac{\partial u(z)}{\partial \bar{z}} = f(z), \quad z \in D.
\]
\( u \) 就称为 \( \bar{\partial} \)问题的解.
\end{definition}

\begin{definition}
设 \( \varphi \) 是 \( \mathbb{C} \) 上的函数,使 \( \varphi \) 不取零值的点集的闭包称为 \( \varphi \) 的\textbf{支集},记为 \( \text{supp}\varphi \),即
\[
\text{supp}\varphi = \overline{\{ z \in \mathbb{C} : \varphi(z) \neq 0 \}}.
\]
\end{definition}

\begin{lemma}\label{lemma:引理3.7.2}
设 \( a \) 是 \( \mathbb{C} \) 中任意一点,\( 0 < r < R \),则必存在 \( \varphi \),满足下列条件:

(i) \( \varphi \in C^\infty(\mathbb{C}) \);

(ii) \( \text{supp}\varphi \subset \overline{B(a,R)} \);

(iii) 当 \( z \in \overline{B(a,r)} \) 时,\( \varphi(z) \equiv 1 \);

(iv) 对于任意 \( z \in \mathbb{C} \),\( 0 \leqslant \varphi(z) \leqslant 1 \)。
\end{lemma}
\begin{proof}
令 \( r < R_1 < R \) 和
\[
h_1(z) = \begin{cases} 
\displaystyle e^{\frac{1}{|z - a|^2 - R_1^2}}, & z \in B(a,R_1); \\
0, & z \notin B(a,R_1),
\end{cases}
\]
\[
h_2(z) = \begin{cases} 
0, & z \in \overline{B(a,r)}; \\
\displaystyle e^{\frac{1}{r^2 - |z - a|^2}}, & z \notin \overline{B(a,r)},
\end{cases}
\]
那么 \( h_1, h_2 \in C^\infty(\mathbb{C}) \)。又令
\[
\varphi(z) = \frac{h_1(z)}{h_1(z) + h_2(z)},
\]
则 \( \varphi \in C^\infty(\mathbb{C}) \)。而且当 \( z \in \overline{B(a,r)} \) 时,\( \varphi(z) \equiv 1 \);当 \( z \notin B(a,R_1) \) 时,\( \varphi(z) \equiv 0 \),即 \( \text{supp}\varphi \subset B(a,R) \)。对于任意 \( z \in \mathbb{C} \),\( 0 \leqslant \varphi(z) \leqslant 1 \) 显然成立。\( \varphi \) 即为所求的函数。
\end{proof}

\begin{theorem}
设 \( D \) 是 \( \mathbb{C} \) 中的域,\( f \in C^1(D) \)。令
\begin{align}
u(z) &= \frac{1}{2\pi i} \int_D \frac{f(\zeta)}{\zeta - z} d\zeta \wedge d\bar{\zeta}, \quad z \in D, \label{eq:1---------1}
\end{align}
则 \( u \in C^1(D) \),且对任意 \( z \in D \),有 \( \frac{\partial u(z)}{\partial \bar{z}} = f(z) \)。
\end{theorem}
\begin{note}
在上面的证明中,容易看出,如果 \( f \in C^\infty(D) \),那么 \( \bar{\partial} \) 问题的解 \( u \in C^\infty(D) \)。
\end{note}
\begin{proof}
把 \( f \) 的定义扩充到整个复平面,对于 \( z \notin D \),定义 \( f(z) = 0 \)。这时,\(\eqref{eq:1---------1}\) 式可写为
\begin{align*}
u(z) = \frac{1}{2\pi i} \int_{\mathbb{C}} \frac{f(\zeta)}{\zeta - z} d\zeta \wedge d\bar{\zeta} \xlongequal{\text{令}\zeta=z+\eta} \frac{1}{2\pi i} \int_{\mathbb{C}} f(\zeta + \eta) \frac{1}{\eta} d\eta \wedge d\bar{\eta}.
\end{align*}
由 \( f \in C^1(D) \),可得 \( u \in C^1(D) \)。

现固定 \( a \in D \),我们证明
\[
\frac{\partial u(a)}{\partial \bar{z}} = f(a).
\]
为此,取 \( 0 < \varepsilon < r \),使得 \( B(a,\varepsilon) \subset B(a,r) \subset D \)。根据\reflem{lemma:引理3.7.2},存在 \( \varphi \in C^\infty(\mathbb{C}) \),使得当 \( z \in B(a,\varepsilon) \) 时,\( \varphi(z) \equiv 1 \);而当 \( z \notin B(a,r) \) 时,\( \varphi(z) \equiv 0 \)。记
\[
u_1(z) = \frac{1}{2\pi i} \int_{\mathbb{C}} \frac{\varphi(\zeta)f(\zeta)}{\zeta - z} d\zeta \wedge d\bar{\zeta},
\]
\[
u_2(z) = \frac{1}{2\pi i} \int_{\mathbb{C}} \frac{f(\zeta) - \varphi(\zeta)f(\zeta)}{\zeta - z} d\zeta \wedge d\bar{\zeta},
\]
那么 \( u = u_1 + u_2 \)。由于当 \( \zeta \in B(a,\varepsilon) \) 时,\( f(\zeta) - \varphi(\zeta)f(\zeta) \equiv 0 \),所以
\[
u_2(z) = \frac{1}{2\pi i} \int_{\mathbb{C} \setminus \overline{B(a,\varepsilon)}} \frac{(1 - \varphi(\zeta))f(\zeta)}{\zeta - z} d\zeta \wedge d\bar{\zeta}.
\]
因而,当 \( z \in B(a,\varepsilon) \) 时,\( u_2 \) 是全纯函数,所以由\refthe{theorem:复变函数可微的充要条件2}可知\( \frac{\partial u_2}{\partial \bar{z}} = 0 \)。于是,在小圆盘 \( B(a,\varepsilon) \) 上就有
\begin{align}
\frac{\partial u}{\partial \bar{z}}&=\frac{\partial u_1}{\partial \bar{z}}\xlongequal{\text{令}\zeta =z+\eta}\frac{\partial}{\partial \bar{z}}\left\{ \frac{1}{2\pi \mathrm{i}}\int_{\mathbb{C}}{\frac{\varphi (z+\eta )f(z+\eta )}{\eta}d\eta}\land d\bar{\eta} \right\}
\nonumber 
\\
&=\frac{1}{2\pi \mathrm{i}}\int_{\mathbb{C}}{\left\{ \frac{\partial (\varphi f)}{\partial \zeta}\frac{\partial \zeta}{\partial \bar{z}}+\frac{\partial (\varphi f)}{\partial \bar{\zeta}}\frac{\partial \bar{\zeta}}{\partial \bar{z}} \right\} \frac{1}{\eta}d\eta}\land d\bar{\eta}
\nonumber 
\\
&=\frac{1}{2\pi \mathrm{i}}\int_{\mathbb{C}}{\left\{ \frac{\partial (\varphi f)}{\partial \zeta}\frac{\partial \left( z+\eta \right)}{\partial \bar{z}}+\frac{\partial (\varphi f)}{\partial \bar{\zeta}}\frac{\partial \left( \bar{z}+\bar{\eta} \right)}{\partial \bar{z}} \right\} \frac{1}{\eta}d\eta}\land d\bar{\eta}
\nonumber 
\\
&=\frac{1}{2\pi \mathrm{i}}\int_{\mathbb{C}}{\left\{ \frac{\partial (\varphi f)}{\partial \zeta}\cdot \left( 0+0 \right) +\frac{\partial (\varphi f)}{\partial \bar{\zeta}}\cdot \left( 1+0 \right) \right\} \frac{1}{\eta}d\eta}\land d\bar{\eta}
\nonumber 
\\
&=\frac{1}{2\pi \mathrm{i}}\int_{\mathbb{C}}{\frac{\partial (\varphi f)}{\partial \bar{\zeta}}\frac{1}{\eta}d\eta}\land d\bar{\eta}
\nonumber 
\\
&=\frac{1}{2\pi \mathrm{i}}\int_{\mathbb{C}}{\frac{\partial (\varphi f)}{\partial \bar{\zeta}}\frac{1}{\zeta -z}d\zeta}\land d\bar{\zeta}
\nonumber 
\\
&=\frac{1}{2\pi \mathrm{i}}\int_{B(a,r)}{\frac{\partial (\varphi f)}{\partial \bar{\zeta}}\frac{1}{\zeta -z}d\zeta}\land d\bar{\zeta}.\label{eq:2---------2}
\end{align}
最后一个等式成立是因为当 \( \zeta \in \mathbb{C} \setminus \overline{B(a,r)} \) 时 \( \varphi(\zeta) \equiv 0 \)。又因为当 \( \zeta \in \partial B(a,r) \) 时 \( \varphi(\zeta) \equiv 0 \),所以根据\hyperref[theorem:非齐次Cauchy积分公式(Pompeiu公式)]{非齐次Cauchy积分公式},有
\[
\varphi(z)f(z) = \frac{1}{2\pi i} \int_{B(a,r)} \frac{\partial(\varphi f)}{\partial \bar{\zeta}} \frac{1}{\zeta - z} d\zeta \wedge d\bar{\zeta}.
\]
因为当 \( z \in B(a,\varepsilon) \) 时 \( \varphi(z) = 1 \),所以
\begin{align}
f(z) &= \frac{1}{2\pi i} \int_{B(a,r)} \frac{\partial(\varphi f)}{\partial \bar{\zeta}} \frac{1}{\zeta - z} d\zeta \wedge d\bar{\zeta}. \label{eq:3---------3}
\end{align}
比较 \(\eqref{eq:2---------2}\) 式和 \(\eqref{eq:3---------3}\) 式,即得
\[
\frac{\partial u(z)}{\partial \bar{z}} = f(z).
\]
特别地,取 \( z = a \),即得
\[
\frac{\partial u(a)}{\partial \bar{z}} = f(a).
\]
由于 \( a \) 是 \( D \) 中的任意点,所以 \( \frac{\partial u(z)}{\partial \bar{z}} = f(z) \) 在 \( D \) 上成立。
\end{proof}















\end{document}
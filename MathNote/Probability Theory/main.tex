\documentclass[lang=cn,newtx,10pt,scheme=chinese]{../../Template/elegantbook}

\title{概率论}
\subtitle{\,\,}

\author{邹文杰}
\institute{无}
\date{2024/10/25}
\version{ElegantBook-4.5}
\bioinfo{自定义}{信息}

\extrainfo{宠辱不惊,闲看庭前花开花落;
\\
去留无意,漫随天外云卷云舒.}


\setcounter{tocdepth}{3}

\logo{logo-blue.png}
\cover{cover.png}

%% 本文档额外使用的宏包和命令
\usepackage{../../Styles/mystyle-elegantbook}

\begin{document}

\maketitle
\frontmatter

\tableofcontents

\mainmatter% 将行为改回预期版本,并重置页码

\chapter{大数定律和中心极限定理}

\section{大数定律}

\begin{definition}[(弱)大数定律]\label{definition:(弱)大数定律}
    称随机变量序列$\left\{ X_n,n\ge 1 \right\} $服从\textbf{(弱)大数定律},若存在数列$\left\{a_n,n\geq 1\right\}$和$\left\{b_n,n\geq 1\right\}$,满足$b_n \to +\infty$,使得当$n\to +\infty$时,有
    \begin{align*}
        \frac{S_n-a_n}{b_n}\overset{P}{\rightarrow}0
    \end{align*}
    成立,其中$S_n=\sum\limits_{k=1}^n{X_k}$.
\end{definition}
\begin{note}
    通常不作特别说明的时候,我们称随机变量序列$\{X_n,n\geq 1\}$服从\textbf{(弱)大数定律},都是指$a_n=\boldsymbol{E}S_n,b_n=n$时的规范形式.即下面的定义.
\end{note}

\begin{definition}[(弱)大数定律的规范形式]\label{definition:(弱)大数定律的规范形式}
    称随机变量序列$\left\{ X_n,n\ge 1 \right\}$服从\textbf{(弱)大数定律},若当$n\to +\infty$时,有
    \begin{align*}
        \frac{S_n}{n}-\frac{\boldsymbol{E}S_n}{n}\overset{P}{\rightarrow}0
    \end{align*}
    成立,其中$S_n=\sum\limits_{k=1}^n{X_k}$.
    
    即
    \begin{align*}
        \underset{n\rightarrow +\infty}{\lim}P\left( \left| \frac{1}{n}\sum_{k=1}^n{X_k}-\frac{1}{n}\boldsymbol{E}\sum_{k=1}^n{X_k} \right|<\varepsilon \right) =1.
    \end{align*}
\end{definition}

\begin{theorem}[Bernoulli大数定律]\label{theorem:Bernoulli大数定律}
    设在一次试验中事件$A$发生的概率为$p$,记$S_n$表示$n$次独立的这种试验$A$发生的次数,则当$n\to +\infty$时,
    \begin{align*}
        \frac{S_n}{n}\overset{P}{\rightarrow}p.
    \end{align*}
\end{theorem}

\begin{theorem}[Markov大数定律]\label{theorem:Markov大数定律}
    若随机变量序列$\left\{ X_n,n\ge 1 \right\}$满足$Markov$条件,即
    \begin{align*}
        \underset{n\rightarrow +\infty}{\lim}\frac{1}{n^2}\boldsymbol{D}\sum_{k=1}^n{X_k}=0.
    \end{align*}
    则$\left\{ X_n,n\ge 1 \right\}$服从(弱)大数定律.
\end{theorem}
\begin{proof}
    对$\forall \varepsilon>0$,由$Chebeshev$不等式可知
    \begin{align*}
        1\geqslant P\left( \left| \frac{1}{n}\sum_{k=1}^n{X_k}-\frac{1}{n}\boldsymbol{E}\sum\limits_{k=1}^n{X_k} \right|<\varepsilon \right) \geqslant 1-\frac{\boldsymbol{D}\left( \frac{1}{n}\sum\limits_{k=1}^n{X_k} \right)}{\varepsilon ^2}=1-\frac{\boldsymbol{D}\sum\limits_{k=1}^n{X_k}}{n^2\varepsilon ^2}.
    \end{align*}
    令$n\to +\infty$,可得
    \begin{align*}
        \underset{n\rightarrow +\infty}{\lim}P\left( \left| \frac{1}{n}\sum_{k=1}^n{X_k}-\frac{1}{n}\boldsymbol{E}\sum_{k=1}^n{X_k} \right|<\varepsilon \right) =1.
    \end{align*}
\end{proof}

\begin{theorem}[Chebeshev大数定律]\label{theorem:Chebeshev大数定律}
    若随机变量序列$\left\{ X_n,n\ge 1 \right\}$两两不相关,且方差一致有界,
    则$\left\{ X_n,n\ge 1 \right\}$服从(弱)大数定律.
\end{theorem}
\begin{proof}
    因为随机变量序列$\left\{ X_n,n\ge 1 \right\}$方差一致有界,所以存在$C>0$,使得
    \begin{align*}
        \boldsymbol{D}X_k\leqslant C,k\in \mathbb{N}_+.
    \end{align*}
    又由于随机变量序列$\left\{ X_n,n\ge 1 \right\}$两两不相关,因此
    \begin{align*}
        \boldsymbol{D}\left( \frac{1}{n}\sum_{k=1}^n{X_k} \right) =\frac{1}{n^2}\boldsymbol{D}\sum_{k=1}^n{X_k}=\frac{1}{n^2}\sum_{k=1}^n{\boldsymbol{D}X_k}.
    \end{align*}
    根据$Chebeshev$不等式可得
    \begin{align*}
        1\geqslant P\left( \left| \frac{1}{n}\sum_{k=1}^n{X_k}-\frac{1}{n}\boldsymbol{E}\sum_{k=1}^n{X_k} \right|<\varepsilon \right) \geqslant 1-\frac{\boldsymbol{D}\left( \frac{1}{n}\sum\limits_{k=1}^n{X_k} \right)}{\varepsilon ^2}\geqslant 1-\frac{C}{n}.
    \end{align*}
    令$n\to +\infty$,可得
    \begin{align*}
        \underset{n\rightarrow +\infty}{\lim}P\left( \left| \frac{1}{n}\sum_{k=1}^n{X_k}-\frac{1}{n}\boldsymbol{E}\sum_{k=1}^n{X_k} \right|<\varepsilon \right) =1.
    \end{align*}
\end{proof}

\begin{theorem}[Khinchin大数定律]\label{theorem:Khinchin大数定律}
    若随机变量序列$\left\{ X_n,n\ge 1 \right\}$独立同分布,且数学期望存在,
    则$\left\{ X_n,n\ge 1 \right\}$服从(弱)大数定律.
\end{theorem}
\begin{proof}
    
\end{proof}

\begin{theorem}[依概率收敛的性质]\label{theorem:依概率收敛的性质}
    设\(\{ X_n, n \geqslant 1\}\)是概率空间\((\varOmega, \mathcal{F}, P)\)中的随机变量序列.

若\(X_n \overset{P}{\rightarrow} c\),且\(c \neq 0\),\(X_n \neq 0\),则\(\frac{1}{X_n} \overset{P}{\rightarrow} \frac{1}{c}\).
\end{theorem}
\begin{proof}
    对\(\forall \varepsilon > 0\),由全概率公式可知
\begin{equation}\label{equation:1.1}
\begin{aligned}
   &P\left(\left|\frac{1}{X_n} - \frac{1}{c}\right| \geqslant \varepsilon\right) = P\left(\left|\frac{X_n - c}{cX_n}\right| \geqslant \varepsilon\right) = P\left(\left|\frac{X_n - c}{cX_n - c^2 + c^2}\right| \geqslant \varepsilon\right) = P\left(\left|\frac{X_n - c}{c(X_n - c) + c^2}\right| \geqslant \varepsilon\right)
   \\
   &= P\left(\left|\frac{X_n - c}{c(X_n - c) + c^2}\right| \geqslant \varepsilon,\left|X_n - c\right| < \varepsilon\right) + P\left(\left|\frac{X_n - c}{c(X_n - c) + c^2}\right| \geqslant \varepsilon,\left|X_n - c\right| \geqslant \varepsilon\right)
   \\
   &\leqslant P\left(\left|\frac{X_n - c}{c(X_n - c) + c^2}\right| \geqslant \varepsilon,\left|X_n - c\right| < \varepsilon\right) + P\left(\left|X_n - c\right| \geqslant \varepsilon\right).
\end{aligned}
\end{equation}
当\(\left|\frac{X_n - c}{c(X_n - c) + c^2}\right| \geqslant \varepsilon\)且\(\left|X_n - c\right| < \varepsilon\)时,由\(\left|X_n - c\right| < \varepsilon\)可知\(-\varepsilon < X_n - c < \varepsilon\).
从而当\(c > 0\)时,\(c(X_n - c) > -c\varepsilon\);当\(c < 0\)时,\(c(X_n - c) > c\varepsilon\).因此对\(\forall c \neq 0\),都有\(c(X_n - c) > -|c|\varepsilon\).
于是此时\(\left|\frac{X_n - c}{c(X_n - c) + c^2}\right| < \left|\frac{X_n - c}{c^2 - |c|\varepsilon}\right|\).故
\begin{align*}
   \left\{\left|\frac{X_n - c}{c(X_n - c) + c^2}\right| \geqslant \varepsilon,\left|X_n - c\right| < \varepsilon\right\} \subseteq \left\{\left|\frac{X_n - c}{c^2 - |c|\varepsilon}\right| \geqslant \varepsilon,\left|X_n - c\right| < \varepsilon\right\}.
\end{align*}
进而结合\eqref{equation:1.1}式,再根据概率的单调性可得
\begin{align*}
    &P\left(\left|\frac{1}{X_n} - \frac{1}{c}\right| \geqslant \varepsilon\right) \leqslant P\left(\left|\frac{X_n - c}{c(X_n - c) + c^2}\right| \geqslant \varepsilon,\left|X_n - c\right| < \varepsilon\right) + P\left(\left|X_n - c\right| \geqslant \varepsilon\right)
    \\
    &\leqslant P\left(\left|\frac{X_n - c}{c^2 - |c|\varepsilon}\right| \geqslant \varepsilon,\left|X_n - c\right| < \varepsilon\right) + P\left(\left|X_n - c\right| \geqslant \varepsilon\right) \leqslant P\left(\left|\frac{X_n - c}{c^2 - |c|\varepsilon}\right| \geqslant \varepsilon\right) + P\left(\left|X_n - c\right| \geqslant \varepsilon\right)
    \\
    &= P\left(\left|X_n - c\right| \geqslant \varepsilon(c^2 - |c|\varepsilon)\right) + P\left(\left|X_n - c\right| \geqslant \varepsilon\right).
\end{align*}
令\(n \to +\infty\),则由\(\frac{1}{X_n} \overset{P}{\rightarrow} \frac{1}{c}\)可知
\begin{align*}
   0 \leqslant P\left(\left|\frac{1}{X_n} - \frac{1}{c}\right| \geqslant \varepsilon\right) \leqslant P\left(\left|X_n - c\right| \geqslant \varepsilon(c^2 - |c|\varepsilon)\right) + P\left(\left|X_n - c\right| \geqslant \varepsilon\right) \to 0,n \to +\infty .
\end{align*}
故\(\lim_{n \to +\infty}P\left(\left|\frac{1}{X_n} - \frac{1}{c}\right| \geqslant \varepsilon\right) = 0\),即\(\frac{1}{X_n} \overset{P}{\rightarrow} \frac{1}{c}\).
\end{proof}

\begin{theorem}[依概率收敛与依分布收敛的关系]\label{theorem:依概率收敛与依分布收敛的关系}
    设\(\{ X_n, n \geqslant 1\}\)和$X$分别是概率空间\((\varOmega, \mathcal{F}, P)\)中的随机变量序列和随机变量.

   (1)$X_n\overset{P}{\rightarrow}X\text{蕴含}X_n\overset{d}{\rightarrow}X.$
    
\end{theorem}
\begin{proof}
    (1)
记\(X_n\)(\(n\geq 1\))与\(X\)的分布函数分别为\(F_n(x)\)(\(n\geq 1\))与\(F\).
设\(X_n\overset{P}{\rightarrow}X\),往证\(F_n(x) \to F(x)\),\(n \to +\infty\),对所有\(F\)中连续点都成立.
为此只需证明对\(\forall x\in \mathbb{R}\),有
\begin{align*}
    F(x - 0) \leqslant \varliminf_{n \to +\infty}F_n(x) \leqslant \varlimsup_{n \to +\infty}F_n(x) \leqslant F(x + 0).
\end{align*}
任取\(y < x\),由全概率公式可知
\begin{align*}
    &F(y) = P(X\leqslant y) = P(X\leqslant y, X_n\leqslant x) + P(X\leqslant y, X_n > x)
    \\
    &\leqslant P(X_n\leqslant x) + P(|X_n - X|\geqslant x - y)
    \\
    &= F_n(x) + P(|X_n - X|\geqslant x - y)
\end{align*}
令\(n \to +\infty\),由\(X_n\overset{P}{\rightarrow}X\)可得\(F(y) \leqslant \varliminf_{n \to +\infty}F_n(x)\),对\(\forall y < x\)都成立.由于分布函数左右极限都存在,因此再令\(y \to x^-\),得到\(F(x - 0) \leqslant \varliminf_{n \to +\infty}F_n(x)\).

同理,任取\(y > x\),由全概率公式可知
\begin{align*}
   &F_n(x) = P(X_n\leqslant x) = P(X_n\leqslant x, X\leqslant y) + P(X_n\leqslant x, X > y)
   \\
   &\leqslant P(X\leqslant y) + P(|X_n - X|\geqslant y - x)
   \\
   &= F(y) + P(|X_n - X|\geqslant y - x)
\end{align*}
令\(n \to +\infty\),由\(X_n\overset{P}{\rightarrow}X\)可得\(\varlimsup_{n \to +\infty}F_n(x) \leqslant F(y)\),对\(\forall y < x\)都成立.由于分布函数左右极限都存在,因此再令\(y \to x^+\),得到\(\varlimsup_{n \to +\infty}F_n(x) \leqslant F(x + 0)\).

又显然有\(\varliminf_{n \to +\infty}F_n(x) \leqslant \varlimsup_{n \to +\infty}F_n(x)\).
综上所述,我们有
\begin{align*}
  F(x - 0) \leqslant \varliminf_{n \to +\infty}F_n(x) \leqslant \varlimsup_{n \to +\infty}F_n(x) \leqslant F(x + 0),\forall x\in \mathbb{R} .
\end{align*}
故结论得证.
\end{proof}

\begin{theorem}[特征函数性质]\label{theorem:特征函数性质}
    特征函数在$(-\infty,+\infty)$上一致连续.
\end{theorem}
\begin{proof}
设随机变量\(X\)的概率密度函数为\(p(x)\),则\(\int_{-\infty}^{+\infty}p(x)dx = 1\).其特征函数为\(f(t) = \int_{-\infty}^{+\infty}e^{itx}p(x)dx\).

由反常积分收敛的柯西收敛准则,可知\(\forall \varepsilon > 0\),存在\(A > 0\),使得\(\int_{|x| > A}p(x)dx < \frac{\varepsilon}{4}\).

于是对\(\forall t \in (-\infty, +\infty)\),取\(h = \frac{\varepsilon}{4A}\),我们有
\begin{equation}\label{equation:1.1123}
\begin{aligned}
|f(t + h) - f(t)| &= \left|\int_{-\infty}^{+\infty}(e^{i(t + h)x} - e^{itx})p(x)dx\right|
= \left|\int_{-\infty}^{+\infty}e^{itx}(e^{ihx} - 1)p(x)dx\right|\\
&\leqslant \int_{-\infty}^{+\infty}|e^{itx}||(e^{ihx} - 1)p(x)|dx
= \int_{-\infty}^{+\infty}|(e^{ihx} - 1)|p(x)dx
\end{aligned}
\end{equation}
又由欧拉公式,可得
\begin{equation}\label{equation:1.23434}
    \begin{aligned}
|(e^{ihx} - 1)| &= |\cos(hx) - 1 + i\sin(hx)|
= \sqrt{(\cos(hx) - 1)^2 + \sin^2(hx)}\\
&= \sqrt{2 - 2\cos(hx)}
= \sqrt{4\sin^2\frac{hx}{2}} \leqslant 2\left|\sin\frac{hx}{2}\right|
\end{aligned}
\end{equation}
从而由\eqref{equation:1.1123}\eqref{equation:1.23434}式,可得
\begin{align*}
|f(t + h) - f(t)| &\leqslant \int_{-\infty}^{+\infty}|(e^{ihx} - 1)|p(x)dx
\leqslant 2\int_{-\infty}^{+\infty}\left|\sin\frac{hx}{2}\right|p(x)dx\\
&= 2\int_{|x| > A}\left|\sin\frac{hx}{2}\right|p(x)dx + 2\int_{-A}^{A}\left|\sin\frac{hx}{2}\right|p(x)dx\\
&\leqslant 2\int_{|x| > A}p(x)dx + 2\int_{-A}^{A}|hx|p(x)dx
\leqslant \frac{\varepsilon}{2} + 2hA\int_{-A}^{A}p(x)dx\\
&\leqslant \frac{\varepsilon}{2} + 2hA\int_{-\infty}^{+\infty}p(x)dx
= \frac{\varepsilon}{2} + 2hA < \varepsilon
\end{align*}
故特征函数\(f(t)\)在\((-\infty, +\infty)\)上一致连续. 
\end{proof}



















\end{document}
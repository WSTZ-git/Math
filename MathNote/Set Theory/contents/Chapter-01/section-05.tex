\documentclass[../../main.tex]{subfiles}
\graphicspath{{\subfix{./image/}}} % 指定图片目录,后续可以直接使用图片文件名
% 注意这里的文件路径不能用 ../../image/ ,否则用latexmk编译子文件会报错

% 例如:
% \begin{figure}[H]
% \centering
% \includegraphics[scale=0.4]{图.png}
% \caption{}
% \label{figure:图}
% \end{figure}
% 注意:上述\label{}一定要放在\caption{}之后,否则引用图片序号会只会显示??.

\begin{document}

\section{集合的基数(势)}

\begin{definition}[集合的对等]
设 $A, B$ 为非空集, 若存在从 $A$ 到 $B$ 的一一映射, 则称 $A$ 与 $B$ \textbf{对等}, 记为 $A \sim B$. 规定 $\varnothing \sim \varnothing$.
\end{definition}
\begin{note}
$A$ 与 $B$ 对等就是两个集合的元素可以建立一一对应的关系. 
\end{note}

\begin{theorem}\label{theorem:对等关系也是等价关系}
对等关系也是等价关系,即具有如下性质:
\begin{enumerate}[(1)]
\item (反身性)$A \sim A$;
\item (对称性) 若 $A \sim B$, 则 $B \sim A$;
\item (传递性) 若 $A \sim B$, $B \sim C$, 则 $A \sim C$.
\end{enumerate} 
\end{theorem}
\begin{proof}
证明是显然的.

\end{proof}

\begin{proposition}\label{proposition:对等集合的积也对等}
\begin{enumerate}[(1)]
\item 设 $A, B, C, D$ 都是非空集合, 若 $A \sim C$, $B \sim D$, 则 $A \times B \sim C \times D$.

\item 设 $A_i, B_i$ 都是非空集合,其中$i=1,2,\cdots,n$. 若 $A_i \sim B_i$,则 $A_1\times A_2\times \cdots \times A_n\sim B_1\times B_2\times \cdots \times B_n$.
\end{enumerate}
\end{proposition}
\begin{proof}
\begin{enumerate}[(1)]
\item 由 $A \sim C$, $B \sim D$ 可知, 存在双射 $f : A \rightarrow C$ 和 $g : B \rightarrow D$. 于是令
\begin{align*}
\varphi : A \times B \rightarrow C \times D, \quad (a, b) \mapsto (f(a), g(b))
\end{align*}
对 $\forall (a, b) \in A \times B$, 由 $f(a) \in C$, $g(b) \in D$ 可知 $(f(a), g(b)) \in C \times D$. 故 $\varphi$ 是良定义的. 由 $f$, $g$ 都是双射易知 $\varphi$ 也是双射. 故 $A \times B \sim C \times D$. 


\item 根据(1)的结论,再利用数学归纳法不难证明.
\end{enumerate}

\end{proof}

\begin{example}
自然数集 $\mathbb{N} \sim$ 正偶数集 $\sim$ 正奇数集 $\sim$ 整数集 $\mathbb{Z}$.
\end{example}
\begin{proof}
正偶数集 $= \{2n : n \in \mathbb{N}\}$; 正奇数集 $= \{2n - 1 : n \in \mathbb{N}\}$;
\begin{align*}
\mathbb{Z} = \{(-1)^{n + 1}[n/2] : n \in \mathbb{N}\}.
\end{align*}

\end{proof}

\begin{example}\label{example-214234210.2}
$(-1, 1) \sim \mathbb{R}$.
\end{example}
\begin{proof}
$f(x) = \tan \frac{\pi}{2}x$.

\end{proof}

\begin{example}
\{去掉一点的圆周\} $\sim \mathbb{R}$.
\end{example}
\begin{note}
类似地,球面去掉一点与平面上的点集对等.
\end{note}
\begin{proof}
如\reffig{figure:去掉一点的圆周与实轴对等}, 设圆周为 $C$, 从除去的点 $P$ 作过圆心的直线, 取与该直线垂直且与圆周相切 (不过点 $P$ ) 的直线表示实轴 $\mathbb{R}$. 对于 $C\backslash  \{P\}$ 上的每一点 $c$, 从点 $P$ 作过点 $c$ 的直线必与实轴相交于某点, 记为 $x$. 建立一一对应: $s : \mathbb{R} \to C\backslash  \{P\}$ 为 $s(x) = c$. (点 $P$ 对应 $\infty$) 
\begin{figure}[H]
\centering
\includegraphics[scale=0.4]{去掉一点的圆周与实轴对等.png}
\caption{去掉一点的圆周与实轴对等}
\label{figure:去掉一点的圆周与实轴对等}
\end{figure}

\end{proof}

\begin{lemma}[映射分解定理]\label{lemma:映射定义域与陪域的无交分解}
设 $A, B$ 为非空集, 若 $f : A \to B$, $g : B \to A$, 则 $A$ 与 $B$ 存在如下分解:
\begin{enumerate}[(i)]
\item $A = A_1 \cup A_2$, $B = B_1 \cup B_2$;
\item $A_1 \cap A_2 = \varnothing$, $B_1 \cap B_2 = \varnothing$;
\item $f(A_1) = B_1$, $g(B_2) = A_2$.
\end{enumerate}
\end{lemma}
\begin{proof}
作集族
\begin{align*}
\Gamma = \{E \subseteq A : E \cap g(B\backslash  f(E)) = \varnothing\}.
\end{align*}
令 $A_1 = \underset{E \in \Gamma}{\bigcup} E$, 则 $A_1 \in \Gamma$. 实际上, 对任意的 $E \in \Gamma$, 都有 $E \subseteq A_1$, 再由 $E \cap g(B\backslash  f(E)) = \varnothing$ 知
\begin{align*}
E \cap g(B\backslash  f(A_1)) = \varnothing
\end{align*}
从而
\begin{align*}
A_1 \cap g(B\backslash  f(A_1)) = \underset{E \in \Gamma}{\cup}[E \cap g(B\backslash  f(E))] = \varnothing
\end{align*}
因此, $A_1$ 是 $A$ 中隔离集, 且是 $\Gamma$ 中的最大元.

现在令 $B_1 = f(A_1)$, $B_2 = B\backslash  B_1$, $A_2 = g(B_2)$, 则
\begin{align*}
A_1 \cap A_2 = A_1 \cap g(B\backslash  f(A_1)) = \varnothing
\end{align*}
下面只需验证 $A_1 \cup A_2 = A$.

若不然, 则存在 $x_0 \in A$, $x_0 \notin A_1 \cup A_2$. 令 $A_0 = A_1 \cup \{x_0\}$, 由于 $B_1 = f(A_1) \subseteq f(A_0)$, 故
\begin{align*}
B\backslash  f(A_0) \subseteq B\backslash  B_1 = B_2
\end{align*}
从而
\begin{align*}
g(B\backslash  f(A_0)) \subseteq g(B_2) = A_2
\end{align*}
再由 $A_1 \cap A_2 = \varnothing$ 以及 $x_0 \notin A_2$ 知
\begin{align*}
A_1 \cap g(B\backslash  f(A_0)) = \varnothing, \quad x_0 \notin g(B\backslash  f(A_0))
\end{align*}
因此
\begin{align*}
A_0 \cap g(B\backslash  f(A_0)) = \varnothing
\end{align*}
故 $A_0 \in \Gamma$. 这与 $A_1$ 是 $\Gamma$ 中的最大元矛盾. 

\end{proof}

\begin{definition}[集合的基数(势)]\label{definition:集合的基数(势)}
设 $A, B$ 为两个集合, 若 $A \sim B$, 则称 $A$ 与 $B$ 的\textbf{基数}或\textbf{势}相同, 记为 $\overline{\overline{A}} = \overline{\overline{B}}$.
\end{definition}
\begin{note}
基数反映了对等集在元素数量级别上的共性. 
\end{note}

\begin{theorem}
\begin{enumerate}[(1)]
\item 自反性:$\overline{\overline{A}} = \overline{\overline{A}}$ 。

\item 对称性:若 $\overline{\overline{A}} = \overline{\overline{B}}$,则 $\overline{\overline{B}} = \overline{\overline{A}}$ 。

\item 传递性:若 $\overline{\overline{A}} = \overline{\overline{B}}$,$\overline{\overline{B}} = \overline{\overline{C}}$,则 $\overline{\overline{A}} = \overline{\overline{C}}$ 。
\end{enumerate}
\end{theorem}
\begin{proof}
由\refthe{theorem:对等关系也是等价关系}及\hyperref[definition:集合的基数(势)]{集合的基数(势)的定义}可直接得到.

\end{proof}

\begin{definition}\label{definition:集合的基数大小关系}
对于集合 $A, B$,若有 $B_0 \subseteq B$,$A \sim B_0$,则称 $A$ 的基数小于等于 $B$ 的基数,记作 $\overline{\overline{A}} \leqslant\overline{\overline{B}}$。

若 $\overline{\overline{A}}\leqslant \overline{\overline{B}}$ 且 $A$ 与 $B$ 不对等,则称 $A$ 的基数小于 $B$ 的基数,记作 $\overline{\overline{A}} < \overline{\overline{B}}$。

同理可定义$\overline{\overline{A}}\geqslant \overline{\overline{B}}$和$\overline{\overline{A}}> \overline{\overline{B}}$.
\end{definition}

\begin{proposition}[映射与基数之间的关系]\label{proposition:映射与基数之间的关系}
\begin{enumerate}[(1)]
\item 若存在从 $A$ 到 $B$ 的单射, 则 $\overline{\overline{A}} \leqslant \overline{\overline{B}}$;
\item 若存在从 $A$ 到 $B$ 的满射, 则 $\overline{\overline{A}} \geqslant \overline{\overline{B}}$;
\item 若存在从 $A$ 到 $B$ 的一一映射, 则 $\overline{\overline{A}} = \overline{\overline{B}}$.
\end{enumerate}
\end{proposition}
\begin{proof}
\begin{enumerate}[(1)]
\item 

\item 

\item 
\end{enumerate}

\end{proof}

\begin{theorem}[Bernstein定理]\label{theorem:Bernstein定理}
Bernstein定理的两个等价形式:
\begin{enumerate}[(1)]
\item 若 $A$ 与 $B$ 的某子集对等, $B$ 与 $A$ 的某子集对等, 则 $A \sim B$.

\item 若集合 $A, B$ 满足 $\overline{\overline{A}} \leqslant \overline{\overline{B}}$, $\overline{\overline{B}} \leqslant \overline{\overline{A}}$, 则 $\overline{\overline{A}} = \overline{\overline{B}}$. 
\end{enumerate}
\end{theorem}
\begin{proof}
由题设存在单射 $f : A \to B$, $g : B \to A$, 利用\hyperref[lemma:映射定义域与陪域的无交分解]{映射分解定理}可得到 $A$ 与 $B$ 的分解
\begin{align*}
A = A_1 \cup A_2, \quad B = B_1 \cup B_2\\
f(A_1) = B_1, \quad g(B_2) = A_2
\end{align*}
其中, $A_1 \cap A_2 = \varnothing$, $B_1 \cap B_2 = \varnothing$. 注意到 $f : A_1 \to B_1$, $g^{-1} : A_2 \to B_2$ 是一一映射, 作映射
\begin{align*}
F(x) = 
\begin{cases}
f(x), & x \in A_1\\
g^{-1}(x), & x \in A_2
\end{cases}
\end{align*}
则 $F : A \to B$ 是一一映射, 从而 $A \sim B$.故(1)得证.
再由\refdef{definition:集合的基数(势)}和\refdef{definition:集合的基数大小关系}可知(2)$\Leftrightarrow$(1).

\end{proof}

\begin{example}\label{example:[-1,1]与R对等}
$[-1, 1] \sim \mathbb{R}$.
\end{example}
\begin{proof}
由\refexa{example-214234210.2}可知, $\mathbb{R} \sim (-1, 1) \subseteq [-1, 1]$; 又 $[-1, 1] \sim [-1, 1] \subseteq \mathbb{R}$. 由\hyperref[theorem:Bernstein定理]{Bernstein定理(1)}可知, $[-1, 1] \sim \mathbb{R}$.

\end{proof}

\begin{theorem}
对于集合 $A, B$,$\overline{\overline{A}} < \overline{\overline{B}}$,$\overline{\overline{A}} = \overline{\overline{B}}$,$\overline{\overline{A}} > \overline{\overline{B}}$ 中的任意两个不会同时成立。
\end{theorem}
\begin{proof}
由\refdef{definition:集合的基数大小关系}可知,若 $\overline{\overline{A}} = \overline{\overline{B}}$,则$A$ 与 $B$ 对等,另外两个不会成立;假设$\overline{\overline{A}} < \overline{\overline{B}}$ 与 $\overline{\overline{A}} > \overline{\overline{B}}$ 同时成立,则存在$B_0 \subseteq B$,$A_0 \subseteq A$,使得$A \sim B_0$,$B \sim A_0$.使用\hyperref[theorem:Bernstein定理]{Bernstein定理(1)}得出$A \subseteq B$,进而$\overline{\overline{A}} = \overline{\overline{B}}$,显然矛盾,证毕。

\end{proof}

\begin{definition}[有限集与无限集]\label{definition:有限集与无限集}
设$A$是一个非空集合,若存在自然数 $n$, 使得 $A \sim \{1, 2, \cdots, n\}$, 则称 $A$ 为\textbf{有限集}, 并记 $\overline{\overline{A}} = n$. 若$A$不是有限集,则称$A$为\textbf{无限集}.特别地,规定 $\overline{\overline{\varnothing}} = 0$.
\end{definition}
\begin{note}
由上述定义可知有限集的基数等于该集合元素的个数. 实际上, 集合的基数就是有限集元素个数的推广.
\end{note}

\begin{theorem}
设$A$是非空集合,则
\begin{enumerate}[(1)]
\item $A$是有限集的充要条件是$A$不与其任何真子集对等.

\item $A$是无限集的充要条件是$A$与其某个真子集对等.
\end{enumerate}
\end{theorem}
\begin{note}
这就是有限集与无限集的本质区别.
\end{note}
\begin{proof}
\begin{enumerate}[(1)]
\item {\heiti 必要性:}设$\overline{\overline{A}} = n$,用数学归纳法证明, $n = 1$, 显然. 假设 $n = k$ 时, 结论成立.

当 $n = k + 1$ 时, 若存在 $A$ 的某个真子集 $A_0$ 使得 $A \sim A_0$, 则存在一一映射 $\varphi : A \to A_0$. 下面分两种情况:

(i) $\exists a \in A$, 使得 $\varphi(a) = a$.

令 $A_1 = A\backslash  \{a\}$, $A_2 = A_0\backslash  \{a\}$, 则 $A_2$ 是 $A_1$ 的真子集, $\overline{\overline{A_1}} = k$. 而 $\varphi|_{A_1}$ 是 $A_1$ 到 $A_2$ 的一一映射, 故 $A_1 \sim A_2$. 这与假设矛盾.

(ii) $\forall a \in A$, 都有 $\varphi(a) \neq a$.

$A_0$ 是 $A$ 的真子集, 则存在 $x_0 \in A$, $x_0 \notin A_0$. 令
\begin{align*}
A_3 = A\backslash  \{x_0\}, \quad A_4 = A_0\backslash  \{\varphi(x_0)\}
\end{align*}
注意到 $x_0 \notin A_0$ 以及 $A_0$ 是 $A$ 的真子集, 则 $A_4$ 是 $A_3$ 的子集. 又由于
\begin{align*}
\varphi(x_0) \in A_0 \subseteq A, \quad \varphi(x_0) \neq x_0
\end{align*}
故 $\varphi(x_0) \in A\backslash  \{x_0\} = A_3$, 而 $\varphi(x_0) \notin A_0\backslash  \{\varphi(x_0)\} = A_4$. 从而 $A_4$ 是 $A_3$ 的真子集, 于是 $\overline{\overline{A_3}} = k$, $\varphi|_{A_3}$ 是 $A_3$ 到 $A_4$ 的一一映射, 故 $A_3 \sim A_4$. 这与假设矛盾. 

{\heiti 充分性:}设$A$不与其任何真子集对等,假设$A$是无限集,则与由(2)的必要性矛盾!因此$A$不是无限集,故$A$是有限集.

\item {\color{blue}证法一:}{\heiti 必要性:}设$A$是无限集.先证明在任一无限集 $A$ 中, 一定能取出一列互不相同的元素 $a_1, a_2, \cdots$. 事实上, 在 $A$ 中任取一个元素, 记为 $a_1$. 因为 $A$ 是无限集, 集 $A\backslash  \{a_1\}$ 显然不空, 这时再从集 $A\backslash  \{a_1\}$ 取一个元素 $a_2$, 同样, $A\backslash  \{a_1, a_2\}$ 决不空. 可以继续做下去, 将从 $A$ 中取出一列互不相同的元素 $a_1, a_2, \cdots$, 记余集为 $\hat{A} = A\backslash  \{a_n | n = 1, 2, \cdots\}$. 在 $A$ 中取出一个真子集
\begin{align*}
\{a_2, a_3, \cdots\} \cup \hat{A} = \tilde{A}
\end{align*}
现作 $A$ 与 $\tilde{A}$ 之间的映射 $\varphi$:
\begin{align*}
&\varphi(a_i) = a_{i + 1}, \quad i = 1, 2, \cdots\\
&\varphi(x) = x, \quad x \in \hat{A}
\end{align*}
显然, $\varphi$ 是 $A$ 到 $\tilde{A}$ 上的一一对应, 证毕. 

{\heiti 充分性:}设$A$与其某个真子集对等,假设$A$是有限集,则与(1)的必要性矛盾!因此$A$是不是有限集.故$A$一定是无限集.

{\color{blue}证法二:}因为有限集是不与其真子集对等的,所以充分性是成立的。现在取\(A\)中一个非空有限子集\(B\),则由\rrefthe{theorem:可列集的性质}{theorem:可列集的性质(6)}立即可知
\begin{align*}
\overline{\overline{A}}=\overline{\overline{\left( \left( A\setminus B \right) \cup B \right) }}=\overline{\overline{\left( A\setminus B \right) }}.
\end{align*}
故$A\sim \left( A\setminus B \right) $.
\end{enumerate}

\end{proof}




\end{document}
\documentclass[../../main.tex]{subfiles}
\graphicspath{{\subfix{../../image/}}} % 指定图片目录,后续可以直接使用图片文件名。

% 例如:
% \begin{figure}[H]
% \centering
% \includegraphics[scale=0.4]{图.png}
% \caption{}
% \label{figure:图}
% \end{figure}
% 注意:上述\label{}一定要放在\caption{}之后,否则引用图片序号会只会显示??.

\begin{document}

\section{整数}

\begin{theorem}[逆归定理]\label{theorem:逆归定理}
假设$S$是一个集合,$a \in S$,并且对于每个$n \in \boldsymbol{N}$,$f_n: S \to S$均是函数,则存在唯一的函数$\varphi: \boldsymbol{N} \to S$,使得$\varphi(0) = a$并且$\varphi(n + 1) = f_n(\varphi(n)) (\forall n \in \boldsymbol{N})$。
\end{theorem}
\begin{proof}
我们将构作$\boldsymbol{N} \times S$上的一个关系$R$,使得它是满足上述性质的函数$\varphi: \boldsymbol{N} \to S$的图象,令
\[
\mathcal{F} = \{ Y \subset \boldsymbol{N} \times S \mid (0, a) \in Y, \text{并且} (n, x) \in Y \Rightarrow (n + 1, f_n(x)) \in Y (\forall n \in \boldsymbol{N}) \}
\]
由于$\boldsymbol{N} \times S \in \mathcal{F}$,从而$\mathcal{F} \neq \varnothing$。令$R = \bigcap_{Y \in \mathcal{F}} Y$,则$R \in \mathcal{F}$。又设$M$为子集合
\[
\{ n \in \boldsymbol{N} \mid \text{存在唯一的} x_n \in S, \text{使得} (n, x_n) \in R \}
\]
我们归纳证明$M = \boldsymbol{N}$。如果$0 \notin M$,则有$(0, b) \in R$,其中$b \neq a$,并且集合$R - \{ (0, b) \} \subset \boldsymbol{N} \times S$属于$\mathcal{F}$。从而$R = \bigcap_{Y \in \mathcal{F}} Y \subset R - \{ (0, b) \}$,这就导致矛盾。因此$0 \in M$。现在假定$n \in M$(即有唯一的$x_n \in S$,使得$(n, x_n) \in R$),则$(n + 1, f_n(x_n)) \in R$。如果又有$(n + 1, c) \in R$,而$c \neq f_n(x_n)$,则$R - \{ (n + 1, c) \} \in \mathcal{F}$(验证!),由此又可象上面那样导致矛盾。因此$x_{n + 1} = f_n(x_n)$是$S$中唯一的元素,使得$(n + 1, x_{n + 1}) \in R$。于是由归纳法(定理6.1)可知$\boldsymbol{N} = M$,即$n \longmapsto x_n$定义了一个函数$\varphi: \boldsymbol{N} \to S$,它的图象为$R$。由于$(0, a) \in R$,从而$\varphi(0) = a$。对于每个$n \in \boldsymbol{N}$,$(n, x_n) = (n, \varphi(n)) \in R$。由于$R \in \mathcal{F}$,从而$(n + 1, f_n(\varphi(n))) \in R$。但是$(n + 1, x_{n + 1}) \in R$。由$x_{n + 1}$的唯一性推出$\varphi(n + 1) = x_{n + 1} = f_n(\varphi(n))$。
\end{proof}

\begin{remark}
如果$A$是非空集合,$A$中的序列是一个函数$\boldsymbol{N} \to A$。一个序列通常表示成$\{a_0, a_1, \dots\}$,$\{a_i\}_{i \in \boldsymbol{N}}$或者$\{a_i\}$,其中$a_i \in A$是$i \in \boldsymbol{N}$的象。类似地,函数$\boldsymbol{N}^* \to A$也称作序列,并且表示成$\{a_1, a_2, \dots\}$,$\{a_i\}_{i \in \boldsymbol{N}^*}$或者$\{a_i\}$,这些符号在课文中不会引起混乱。
\end{remark}
















\end{document}
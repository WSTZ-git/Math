\documentclass[../../main.tex]{subfiles}
\graphicspath{{\subfix{../../image/}}} % 指定图片目录,后续可以直接使用图片文件名。

% 例如:
% \begin{figure}[H]
% \centering
% \includegraphics[scale=0.4]{图.png}
% \caption{}
% \label{figure:图}
% \end{figure}
% 注意:上述\label{}一定要放在\caption{}之后,否则引用图片序号会只会显示??.

\begin{document}

\section{集合及其运算}

\subsection{集合的基本概念}

\subsubsection*{集合的定义}

\begin{definition}[集合]
具有确定内容或满足一定条件的事物的全体称为\textbf{集合} (或\textbf{集}),通常用大写字母如 \(A, B, C\) 等表示。构成一个集合的那些事物称为集合的元素 (或元),通常用小写字母如 \(a, b, c\) 等表示。
\end{definition}

若 \(a\) 是集合 \(A\) 的元素,则称 \(a\) 属于 \(A\),记为 \(a \in A\);若 \(a\) 不是集合 \(A\) 的元素,则称 \(a\) 不属于 \(A\),记为 \(a \notin A\)。对于给定的集合,任一元素要么属于它,要么不属于它,二者必居其一。

不含任何元素的集合称为空集,用 \(\emptyset\) 表示;只含有限个元素的集合称为有限集;不是有限集的集合称为无限集。

我们用 \(\mathbb{Z}, \mathbb{N}, \mathbb{Q}, \mathbb{R}\) 分别表示整数集、自然数集(不包含0)、有理数集和实数集.特别地,我们用$\mathbb{N}_0$表示$\mathbb{N}\cup 0$.

\subsubsection*{集合的表示方法}

(1) 列举法 —— 列出给定集合的全部元素. 例如
\begin{align*}
A &= \{a, b, c\}, B = \{1, 3, \cdots, 2n - 1\}
\end{align*}

(2) 描述法 —— \(A = \{x : x \text{ 具有性质 } P\}\). 例如
\begin{align*}
\ker f &= \{x : f(x) = 0\}
\end{align*}

\subsubsection*{集合的相等与包含}

若集合 \(A\) 和 \(B\) 具有完全相同的元素,则称 \(A\) 与 \(B\) 相等,记为 \(A = B\).
若 \(A\) 中的每个元素都是 \(B\) 的元素,则称 \(A\) 为 \(B\) 的子集,记为 \(A \subset B\) 或 \(B \supset A\).
若 \(A \subset B\) 且 \(A \neq B\),则称 \(A\) 为 \(B\) 的真子集,记为 \(A \subsetneq B\).

\begin{remark}
\(A = B \Longleftrightarrow A \subset B\) 且 \(B \subset A\). (经常用于证明两个集合相等)
集合 \(A\) 的所有子集的全体,称为 \(A\) 的幂集,记为 \(2^A\). 
\end{remark}

\subsection{集合的运算}

\subsubsection*{交与并}

设 \(A, B\) 为两个集合,由属于 \(A\) 或属于 \(B\) 的所有元素构成的集合,称为 \(A\) 与 \(B\) 的并,记为 \(A \cup B\),即
\begin{align*}
A \cup B &= \{x : x \in A \text{ 或 } x \in B\}
\end{align*}
由既属于 \(A\) 又属于 \(B\) 的元素构成的集合,称为 \(A\) 与 \(B\) 的交,记为 \(A \cap B\),即
\begin{align*}
A \cap B &= \{x : x \in A \text{ 且 } x \in B\}
\end{align*}
若 \(A \cap B = \emptyset\),则称 \(A\) 与 \(B\) 互不相交。

\subsubsection*{集族}

\(\{A_{\alpha}\}_{\alpha \in \varGamma}\) 称为集族,其中 \(\varGamma\) 为指标集 (有限或无限),\(\alpha\) 为指标. 特别地,当 \(\varGamma = \mathbb{N}\) 时,集族称为列集,记为 \(\{A_n\}_{n = 1}^{\infty}\) 或 \(\{A_n\}\)。

\paragraph{集族的并:}
\begin{align*}
\bigcup_{\alpha \in \varGamma} A_{\alpha} &= \{x : \exists \alpha_0 \in \varGamma \text{ 使得 } x \in A_{\alpha_0}\}
\end{align*}

\paragraph{集族的交:}
\begin{align*}
\bigcap_{\alpha \in \varGamma} A_{\alpha} &= \{x : x \in A_{\alpha}, \forall \alpha \in \varGamma\}
\end{align*}

\subsubsection*{差与余}
由属于 \(A\) 但不属于 \(B\) 的元素构成的集合,称为 \(A\) 与 \(B\) 的差,记为 \(A - B\) 或 \(A \setminus B\),即
\begin{align*}
A - B &= \{x : x \in A \text{ 且 } x \notin B\}
\end{align*}
通常所讨论的集合都是某一固定集 \(X\) 的子集,\(X\) 称为全集或基本集. 全集 \(X\) 与子集 \(A\) 的差集 \(X - A\),称为 \(A\) 的余集,记为 \(A^c\),即
\begin{align*}
A^c &= \{x : x \notin A\}
\end{align*}
\begin{remark}
补集是相对概念,若 \(A \subset B\),则 \(B - A\) 称为 \(A\) 关于 \(B\) 的补集。特别地,余集是集合关于全集的补集。
\end{remark}

\subsubsection*{笛卡尔积}
\begin{align*}
A \times B &= \{(x, y) : x \in A, y \in B\}\\
A_1 \times \cdots \times A_n &= \{(x_1, \cdots, x_n) : x_i \in A_i, i = 1, \cdots, n\}
\end{align*}
例如,\(n\) 维欧氏空间 \(\mathbb{R}^n = \underbrace{\mathbb{R} \times \cdots \times \mathbb{R}}_{n \text{ 个}}\)。

\begin{theorem}[集合的运算及性质]\label{theorem:集合的运算及性质}
\begin{enumerate}[(1)]
\item \(A \cup A = A\),\(A \cap A = A\);

\item \(A \cup \emptyset = A\),\(A \cap \emptyset = \emptyset\);

\item \(A \cup B = B \cup A\),\(A \cap B = B \cap A\);

\item \((A \cup B) \cup C = A \cup (B \cup C)\),\((A \cap B) \cap C = A \cap (B \cap C)\);

\item \(A \cap (B \cup C) = (A \cap B) \cup (A \cap C)\),\(A \cup (B \cap C) = (A \cup B) \cap (A \cup C)\),

\(A \cap (\bigcup_{\alpha \in \varGamma} B_{\alpha}) = \bigcup_{\alpha \in \varGamma}(A \cap B_{\alpha})\),\(A \cup (\bigcap_{\alpha \in \varGamma} B_{\alpha}) = \bigcap_{\alpha \in \varGamma}(A \cup B_{\alpha})\);

\item \(A \cup A^c = X\),\(A \cap A^c = \emptyset\);

\item \(X^c = \emptyset\),\(\emptyset^c = X\);

\item \(A - B = A \cap B^c\)。
\end{enumerate}
\end{theorem}


\begin{theorem}[De Morgan定律]\label{theorem:De Morgan定律}
设 \(\{A_{\alpha}\}_{\alpha \in \varGamma}\) 为一集族,则

(i) \((\bigcup_{\alpha \in \varGamma} A_{\alpha})^c = \bigcap_{\alpha \in \varGamma} A_{\alpha}^c\);

(ii) \((\bigcap_{\alpha \in \varGamma} A_{\alpha})^c = \bigcup_{\alpha \in \varGamma} A_{\alpha}^c\)。
\end{theorem}
\begin{proof}
(i) 设 \(x \in (\bigcup_{\alpha \in \varGamma} A_{\alpha})^c\),则 \(x \notin \bigcup_{\alpha \in \varGamma} A_{\alpha}\),故对 \(\forall \alpha \in \varGamma\),\(x \notin A_{\alpha}\),即 \(\forall \alpha \in \varGamma\),\(x \in A_{\alpha}^c\)。从而 \(x \in \bigcap_{\alpha \in \varGamma} A_{\alpha}^c\),因此,\((\bigcup_{\alpha \in \varGamma} A_{\alpha})^c \subset \bigcap_{\alpha \in \varGamma} A_{\alpha}^c\)。上述推理反过来也成立,故 \(\bigcap_{\alpha \in \varGamma} A_{\alpha}^c \subset (\bigcup_{\alpha \in \varGamma} A_{\alpha})^c\)。因此,\((\bigcup_{\alpha \in \varGamma} A_{\alpha})^c = \bigcap_{\alpha \in \varGamma} A_{\alpha}^c\)。

(ii) 类似可证。 
\end{proof}

\begin{definition}[对称差集]
设 \( A,B \) 为两个集合,称集合 \( (A \setminus B) \cup (B \setminus A) \) 为 \( A \) 与 \( B \) 的\textbf{对称差集},记为 \( A \triangle B \).
\end{definition}
\begin{note}
对称差集是由既属于 \( A,B \) 之一但又不同时属于两者的一切元素构成的集合.
\end{note}

\begin{theorem}[集合对称差的性质]\label{theorem:集合对称差的性质}
\begin{enumerate}[(i)]
\item \(A \triangle B = \left( A\cup B \right) \backslash \left( A\cap B \right) .\)

\item \( A \triangle \varnothing = A \),\( A \triangle A = \varnothing \),\( A \triangle A^c = X \),\( A \triangle X = A^c \)。

\item 交换律:\( A \triangle B = B \triangle A \)。

\item 结合律:
\(
(A \triangle B) \triangle C = A \triangle (B \triangle C).
\)

\item 交与对称差满足分配律:$A \cap (B \triangle C) = (A \cap B) \triangle (A \cap C).$

\item \( A^c \triangle B^c = A \triangle B \);\( A = A \triangle B \) 当且仅当 \( B = \varnothing \)。

\item 对任意的集合 \( A \) 与 \( B \),存在唯一的集合 \( E \),使得 \( E \triangle A = B \)(实际上 \( E = B \triangle A \))。
\end{enumerate}
\end{theorem}
\begin{proof}
\begin{enumerate}[(i)]
\item 由对称差集的定义及\hyperref[theorem:集合的运算及性质]{集合的运算及性质(5)}可得
\begin{align*}
A\bigtriangleup B&=\left( A\cap B^c \right) \cup \left( B\cap A^c \right) =\left[ \left( A\cap B^c \right) \cup B \right] \cap \left[ \left( A\cap B^c \right) \cup A^c \right] 
\\
&=\left( A\cup B \right) \cap \left( A^c\cap B^c \right) =\left( A\cup B \right) \cap \left( A\cap B \right) ^c
\\
&=\left( A\cup B \right) \backslash \left( A\cap B \right) .
\end{align*}

\item 

\item 

\item 

\item 

\item 

\item 
\end{enumerate}
\end{proof}

\subsection{上限集与下限集}

设 $\{A_n\}$ 为单调集列,若 $\{A_n\}$ 单调递增,即
\begin{align*}
A_1 \subset A_2 \subset \cdots \subset A_n \subset \cdots
\end{align*}
则 $\{A_n\}$ 收敛,且 $\lim_{n \to \infty} A_n = \bigcup_{n = 1}^{\infty} A_n$。若 $\{A_n\}$ 单调递减,即
\begin{align*}
A_1 \supset A_2 \supset \cdots \supset A_n \supset \cdots
\end{align*}
则 $\{A_n\}$ 收敛,且 $\lim_{n \to \infty} A_n = \bigcap_{n = 1}^{\infty} A_n$。

\begin{definition}[上限集和下限集]
对于一般的集列 $\{A_n\}$
\begin{align*}
\bigcup_{k = 1}^{\infty} A_k \supset \bigcup_{k = 2}^{\infty} A_k \supset \cdots \supset \bigcup_{k = n}^{\infty} A_k \supset \cdots
\end{align*}
记 $C_n = \bigcup_{k = n}^{\infty} A_k$,则 $\{C_n\}$ 单调递减,故 $\{C_n\}$ 收敛且
\begin{align*}
\lim_{n \to \infty} C_n = \bigcap_{n = 1}^{\infty} C_n = \bigcap_{n = 1}^{\infty} \bigcup_{k = n}^{\infty} A_k
\end{align*}
称 $\bigcap_{n = 1}^{\infty} \bigcup_{k = n}^{\infty} A_k$ 为 $\{A_n\}$ 的\textbf{上限集},记为 $\limsup_{n \to \infty} A_n$ 或 $\varlimsup_{n \to \infty} A_n$。又
\begin{align*}
\bigcap_{k = 1}^{\infty} A_k \subset \bigcap_{k = 2}^{\infty} A_k \subset \cdots \subset \bigcap_{k = n}^{\infty} A_k \subset \cdots
\end{align*}
记 $D_n = \bigcap_{k = n}^{\infty} A_k$,则 $\{D_n\}$ 单调递增,故 $\{D_n\}$ 收敛且
\begin{align*}
\lim_{n \to \infty} D_n = \bigcup_{n = 1}^{\infty} D_n = \bigcup_{n = 1}^{\infty} \bigcap_{k = n}^{\infty} A_k
\end{align*}
称 $\bigcup_{n = 1}^{\infty} \bigcap_{k = n}^{\infty} A_k$ 为 $\{A_n\}$ 的下限集,记为 $\liminf_{n \to \infty} A_n$ 或 $\varliminf_{n \to \infty} A_n$。显然有如下关系
\begin{align*}
\bigcap_{n = 1}^{\infty} A_n \subset \liminf_{n \to \infty} A_n \subset \limsup_{n \to \infty} A_n \subset \bigcup_{n = 1}^{\infty} A_n
\end{align*}
若 $\liminf_{n \to \infty} A_n = \limsup_{n \to \infty} A_n$,则称 $\{A_n\}$ 收敛,其极限记为 $\lim_{n \to \infty} A_n$。 
\end{definition}

\begin{proposition}
设 $\{A_n\}$ 为一集列,则
\begin{align*}
\limsup_{n \to \infty} A_n = \bigcap_{n = 1}^{\infty} \bigcup_{k = n}^{\infty} A_k 
&= \{x: x \text{ 属于无穷多个 } A_n\}\\
&= \{x: \text{ 对 } \forall k \in \mathbb{N}, \text{ 都存在 } n_k \text{ 使得 } x \in A_{n_k}\}
\end{align*}

\begin{align*}
\liminf_{n \to \infty} A_n = \bigcup_{n = 1}^{\infty} \bigcap_{k = n}^{\infty} A_k 
&= \{x: \text{ 除有限个 } A_n \text{ 外, 都含有 } x\}\\
&= \{x: \exists n_0 \in \mathbb{N}, \text{ 使得 } x \in A_n, \forall n \geqslant n_0\}
\end{align*}
\end{proposition}
\begin{proof}
(1) 设 $x \in \bigcap_{n = 1}^{\infty} \bigcup_{k = n}^{\infty} A_k$,则对 $n = 1$, 有 $x \in \bigcup_{k = 1}^{\infty} A_k$,故 $\exists n_1 \in \mathbb{N}$, 使得 $x \in A_{n_1}$;对 $n = n_1 + 1$, 有 $x \in \bigcup_{k = n_1 + 1}^{\infty} A_k$,故 $\exists n_2 > n_1$, 使得 $x \in A_{n_2}$;以此类推,得到一列 $\{n_k\}$ 满足 $n_1 < n_2 < \cdots$, 且 $x \in A_{n_k}, \forall k$. 因此,$x$ 属于无穷多个 $A_n$.

反之,若 $x$ 属于无穷多个 $A_n$, 不妨设 $x \in A_{n_k}, k = 1, 2, \cdots$, 且 $n_1 < n_2 < \cdots$, 则对 $\forall n \in \mathbb{N}$, 都存在 $n_k > n$. 从而 $x \in A_{n_k} \subset \bigcup_{k = n}^{\infty} A_k$. 因此,$x \in \bigcap_{n = 1}^{\infty} \bigcup_{k = n}^{\infty} A_k$.

(2) $x \in \bigcup_{n = 1}^{\infty} \bigcap_{k = n}^{\infty} A_k \Longleftrightarrow \exists n_0 \in \mathbb{N}$, 使得 $x \in \bigcap_{k = n_0}^{\infty} A_k \Longleftrightarrow x \in A_n, \forall n \geqslant n_0$. 
\end{proof}

\begin{example}
设 $A_{2n + 1} = [0, 2 - 1/(2n + 1)], n = 0, 1, 2, \cdots$, $A_{2n} = [0, 1 + 1/2n], n = 1, 2, \cdots$, 求 $\liminf_{n \to \infty} A_n$ 与 $\limsup_{n \to \infty} A_n$.
\end{example}
\begin{solution}
注意到
\begin{align*}
[0, 1] \subset \bigcap_{n = 1}^{\infty} A_n \subset \liminf_{n \to \infty} A_n \subset \limsup_{n \to \infty} A_n \subset \bigcup_{n = 1}^{\infty} A_n \subset [0, 2)
\end{align*}
故只需考察 $(1, 2)$ 中的点. 对 $\forall x \in (1, 2)$, 存在 $n_0 \in \mathbb{N}$(与 $x$ 有关), 使得
\begin{align*}
1 + \frac{1}{2n} < x < 2 - \frac{1}{2n + 1}, \quad \forall n \geqslant n_0
\end{align*}
即当 $n \geqslant n_0$ 时, 有 $x \notin A_{2n}, x \in A_{2n + 1}$. 这说明: (i) $x$ 不能 “除有限个 $A_n$ 外, 都含有 $x$”, 即 $x \notin \liminf_{n \to \infty} A_n$; (ii) “$x$ 属于无穷多个 $A_n$”, 故 $x \in \limsup_{n \to \infty} A_n$. 因此, $\liminf_{n \to \infty} A_n = [0, 1]$, $\limsup_{n \to \infty} A_n = [0, 2)$.
\end{solution}

\begin{example}
设 $f_n(x), f(x)$ 为 $\mathbb{R}$ 上的实值函数, 则所有 $\{f_n(x)\}$ 不收敛于 $f(x)$ 的点 $x$ 构成的集合 $D$ 可表示为
\begin{align*}
D = \bigcup_{k = 1}^{\infty} \bigcap_{N = 1}^{\infty} \bigcup_{n = N}^{\infty} \{x : |f_n(x) - f(x)| \geqslant \frac{1}{k}\}
\end{align*}
\end{example}
\begin{proof}
若 $x \in D$, 则 “$f_n(x) \nrightarrow f(x)$”, 即 $\exists \varepsilon_0 > 0$, 对 $\forall k \in \mathbb{N}$, $\exists n_k \geqslant k$, 使得
\begin{align*}
|f_{n_k}(x) - f(x)| \geqslant \varepsilon_0
\end{align*}
记 $E_n(\varepsilon_0) = \{x : |f_n(x) - f(x)| \geqslant \varepsilon_0\}$, 则由命题 1.1 知,
\begin{align*}
D = \bigcap_{N = 1}^{\infty} \bigcup_{n = N}^{\infty} E_n(\varepsilon_0), \quad \exists \varepsilon_0 > 0
\end{align*}
考虑到 $\varepsilon_0$ 的取法, 不妨设 $\varepsilon_0 = 1/k_0, k_0 \in \mathbb{N}$. 因此
\begin{align*}
D = \bigcup_{k = 1}^{\infty} \bigcap_{N = 1}^{\infty} \bigcup_{n = N}^{\infty} E_n\left(\frac{1}{k}\right)
= \bigcup_{k = 1}^{\infty} \bigcap_{N = 1}^{\infty} \bigcup_{n = N}^{\infty} \{x : |f_n(x) - f(x)| \geqslant \frac{1}{k}\}.
\end{align*}
\end{proof}
\begin{remark}
由于收敛点集是不收敛点集的余集, 由德・摩根公式, 所有 $\{f_n(x)\}$ 收敛于 $f(x)$ 的点 $x$ 构成的集合 $C$ 可表示为
\begin{align*}
C = \bigcap_{k = 1}^{\infty} \bigcup_{N = 1}^{\infty} \bigcap_{n = N}^{\infty} \{x : |f_n(x) - f(x)| < \frac{1}{k}\}
\end{align*} 
\end{remark}










\end{document}
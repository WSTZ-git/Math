\documentclass[../../main.tex]{subfiles}% 注意这里的文件路径不能用 ./main.tex ,否则用latexmk编译子文件会报错
\graphicspath{{\subfix{./image/}}} % 指定图片目录,后续可以直接使用图片文件名
% 注意这里的文件路径不能用 ../../image/ ,否则用latexmk编译子文件会报错

% 例如:
% \begin{figure}[H]
% \centering
% \includegraphics[scale=0.3]{图.png}
% \caption{}
% \label{figure:图}
% \end{figure}
% 注意:上述\label{}一定要放在\caption{}之后,否则引用图片序号会只会显示??.

\begin{document}

\section{映射}

\begin{definition}[映射]
设 $A, B$ 为非空集, 若存在对应法则 $f$, 使得对每个 $x \in A$ 都有唯一确定的 $y \in B$ 与之对应, 则称对应法则 $f$ 为从 $A$ 到 $B$ 的映射. 记为 $f : A \to B$, 其表达形式为 $y = f(x), x \in A$.

$A$ 称为 $f$ 的\textbf{定义域}, 记为 $D(f)$.$B$称为$f$的\textbf{陪域}.
$A$ 在 $f$ 下的象称为 $f$ 的\textbf{值域}, 记为 $R(f)$, 即
\begin{align*}
R(f) = f(A) = \{f(x) : x \in A\}
\end{align*}

集合 $B_0 \subseteq B$ 在 $f$ 下的\textbf{原象}, 记为 $f^{-1}(B_0)$, 即
\begin{align*}
f^{-1}(B_0) = \{x \in A : f(x) \in B_0\}
\end{align*}
\end{definition}
\begin{remark}
关于映射的定义, 我们作以下几点说明:

(1) $f$的一个像可以存在多个原像;但对每一个 $x \in A$, 只能有唯一的 $y \in B$ 与它对应,因此今后如果构造映射$f:A\to B$,就必须先验证其良定义性,即$\forall x_1=x_2\in A$,则$f(x_1)=f(x_2)\in B$.也即定义域中的每个元素只能有一个像.

(2) $f(A) \subseteq B$, 不一定有 $f(A) = B$;

(3) 映射是由定义域和对应法则共同确定的, 但对应法则的表达形式可能不唯一. 例如
\begin{align*}
y = |x|, x \in \mathbb{R} \quad \text{ 和 } \quad y = \sqrt{x^2}, x \in \mathbb{R}
\end{align*}

(4) 值域中的元可以是集合. 例如 $A = \mathbb{N}, B = \{B_n\}_{n = 1}^{\infty}, \phi(n) = B_n$;

(5) 定义域中的元也可以是集合. 例如 $A$ 可列, $\mathcal{D} \subseteq 2^A$, 定义 $\phi : \mathcal{D} \to \{0\} \cup \mathbb{N}$ 为
\begin{align*}
\phi(A_0) = A_0 \text{ 中元素个数}, \quad A_0 \in \mathcal{D}.
\end{align*} 
\end{remark}

\begin{definition}[单射、满射和双射]
设$f:A\to B$,则
\begin{enumerate}
\item 若$B$中每个元素最多只有一个原像,即对$\forall y\in B$,$f^{-1}(y)$所含元素个数为0或1,则称 $f$ 为\textbf{单射}或\textbf{一一映射}或\textbf{一一的}.

\item 若 $f(A) = B$, 即 $\forall y \in B$, 都存在 $x \in A$ 使得 $f(x) = y$,亦即$f^{-1}(y)=\varnothing ,\forall y\in B,$ 则称 $f$ 为\textbf{满射}或\textbf{映上的}.

\item 若$f$既是单射又是满射,则称$f$为\textbf{双射}或\textbf{一一对应}.
\end{enumerate}
\end{definition}

\begin{definition}[逆映射]
设 $f : A \to B$ 为一一映射, 则对每个 $y \in B$, 都有唯一确定的 $x \in A$ 满足 $y = f(x)$. 定义 $f^{-1} : B \to A$ 为 $f^{-1}(y) = x$, 则称 $f^{-1}$ 为 $f$ 的\textbf{逆映射}.自然$f$也是$f^{-1}$的逆映射,即$(f^{-1})^{-1}=f.$
\end{definition}
\begin{note}
由逆映射的定义可直接得到
\begin{align*}
f^{-1}(f(x)) = x, \quad x \in A,\quad
f(f^{-1}(y)) = y, \quad y \in B.
\end{align*} 
\end{note}

\begin{axiom}[选择公理(AC)]\label{axiom:选择公理}
设 $\mathcal{F}$ 是一个非空集合族, 且 $\mathcal{F}$ 中的每个元素都是非空集合。那么存在一个函数
\begin{align*}
f : \mathcal{F} \to \bigcup_{A\in\mathcal{F}} A
\end{align*}
使得对任意 $A \in \mathcal{F}$, 有 $f(A) \in A$。
这样的函数 $f$ 称为\textbf{选择函数} (choice function).
\end{axiom}

\begin{theorem}
设$f:A\to B$,其中$A,B\neq \varnothing$,则
\begin{enumerate}[(1)]
\item $f$为单射的充要条件是满足下列条件之一.
\begin{enumerate}[(i)]
\item $f(x_1)=f(x_2)\iff x_1=x_2,\quad \forall x_1,x_2\in A.$

\item $f(x_1)=f(x_2)\Longrightarrow x_1=x_2,\quad \forall x_1,x_2\in A.$

\item $x_1\ne x_2\Longrightarrow f(x_1)\ne f(x_2),\quad \forall x_1,x_2\in A.$

\item 存在$g:B\to A$,使得$gf=\text{id}_A.$
\end{enumerate}

\item $f$为满射的充要条件是满足下列条件之一.
\begin{enumerate}[(i)]
\item $f(A)=B.$

\item 存在$g:B\to A$,使得$fg=\text{id}_B.$
\end{enumerate}

\item $f$为双射的充要条件是存在$g:B\to A$,使得
\begin{align*}
gf=\text{id}_A,\quad fg=\text{id}_B.
\end{align*}
此时必有$g=f^{-1}$.即有两个交换图,如\reffig{figure:交换图fiou3nr4-1}所示.
\end{enumerate}
\end{theorem}
\begin{note}
\begin{figure}[H]
\centering
% https://q.uiver.app/#q=WzAsNixbMCwwLCJBIl0sWzIsMCwiQiJdLFswLDIsIkEiXSxbNCwwLCJCIl0sWzYsMCwiQSJdLFs0LDIsIkIiXSxbMCwxLCJmIl0sWzAsMiwiXFx0ZXh0e2lkfV9BIiwyXSxbMSwyLCJmXnstMX0iXSxbMyw0LCJmXnstMX0iXSxbNCw1LCJmIl0sWzMsNSwiXFx0ZXh0e2lkfV9CIiwyXV0=
\begin{tikzcd}
A && B && B && A \\
\\
A &&&& B
\arrow["f", from=1-1, to=1-3]
\arrow["{\text{id}_A}"', from=1-1, to=3-1]
\arrow["{f^{-1}}", from=1-3, to=3-1]
\arrow["{f^{-1}}", from=1-5, to=1-7]
\arrow["{\text{id}_B}"', from=1-5, to=3-5]
\arrow["f", from=1-7, to=3-5]
\end{tikzcd}
\caption{}
\label{figure:交换图fiou3nr4-1}
\end{figure}
\end{note}
\begin{proof}


\end{proof}

\begin{definition}[映射的乘积]
设映射 $f : A \to B$, $g : B \to C$, 定义 $g \circ f : A \to C$ 为
\begin{align*}
(g \circ f)(x) = g(f(x)), \quad x \in A
\end{align*}
称为 $g$ 与 $f$ 的\textbf{复合映射}或\textbf{乘积}. $g \circ f$也常简记为$gf.$
\end{definition}

\begin{theorem}[映射的乘法满足结合律]
若有 \( f: A \to B \), \( g: B \to C \), \( h: C \to D \), 则有
\begin{align*}
h(gf) &= (hg)f. 
\end{align*}
\end{theorem}
\begin{note}
由映射的乘法满足结合律可知在多个映射相乘时, 可以不加括号. 特别地, \( h(gf) \) 与 \( (hg)f \) 均可简记作 \( hgf \).
\end{note}
\begin{proof}
事实上, 对 \( \forall x \in A \) 有
\begin{align*}
(h(gf))(x) = h((gf)(x)) = h(g(f(x))) = (hg)(f(x)) = ((hg)f)(x).
\end{align*}

\end{proof}

\begin{definition}
设映射 $f : A \to B$, $A_0 \subseteq A$, 定义映射 \( i:A_0 \to A \)满足
\begin{align*}
i(x)=x ,\quad \forall x \in A_0.
\end{align*}
称$i$为 \( A_0 \) 到 \( A \) 中的\textbf{嵌入映射}. 自然,嵌入映射是单射.

又若映射$f : A \to B$与映射$g : A_0 \to B$ 满足$gi=f$,即
\begin{align*}
g(x) = f(x), \quad \forall  x \in A_0.
\end{align*}
则称 $g$ 为 $f$ 在 $A_0$ 上的\textbf{限制}, 记为 $g = f|_{A_0}$, 也称 $f$ 为 $g$ 在 $A$ 上的\textbf{延拓}或\textbf{开拓}. 即\reffig{figure:嵌入映射交换图}为交换图.
\end{definition}
\begin{note}
\begin{figure}[H]
\centering
% https://q.uiver.app/#q=WzAsMyxbMCwwLCJBXzAiXSxbMiwwLCJBIl0sWzAsMiwiQiJdLFswLDEsImkiXSxbMCwyLCJmIiwyXSxbMSwyLCJnIl1d
\begin{tikzcd}
{A_0} && A \\
\\
B
\arrow["i", from=1-1, to=1-3]
\arrow["f"', from=1-1, to=3-1]
\arrow["g", from=1-3, to=3-1]
\end{tikzcd}
\caption{}
\label{figure:嵌入映射交换图}
\end{figure}
\end{note}


\begin{proposition}\label{proposition:映射的复合保持单射、满射和双射}
设一列映射 $f_i:A_i\rightarrow A_{i + 1}$, 其中 $i = 1, 2, \cdots, n$.
\begin{enumerate}[(1)]
\item 若 $f_i$ ($i = 1, 2, \cdots, n$) 都是单射, 则 $f_n\circ \cdots \circ f_2\circ f_1$ 也是单射.

\item 若 $f_i$ ($i = 1, 2, \cdots, n$) 都是满射, 则 $f_n\circ \cdots \circ f_2\circ f_1$ 也是满射.

\item 若 $f_i$ ($i = 1, 2, \cdots, n$) 都是双射, 则 $f_n\circ \cdots \circ f_2\circ f_1$ 也是双射.

\item 若$f_i(i=1,2,\cdots,n)$都是双射,则$(f_n\cdots f_2f_1)^{-1}=f_1^{-1}f_2^{-1}\cdots f_n^{-1}$.
\end{enumerate} 
\end{proposition}
\begin{proof}
\begin{enumerate}[(1)]
\item 

\item 

\item 
\end{enumerate}

\end{proof}

\begin{proposition}[映射的基本性质]\label{proposition:映射的基本性质}
对于映射$f : X \to Y$,$A\subseteq X,B\subseteq Y$,$\{A_\alpha\}_{\alpha \in I}\subseteq X,\{B_\alpha\}_{\alpha \in I}\subseteq Y$,则下列事实成立:
\begin{enumerate}[(1)]
\item\label{proposition:映射的基本性质-1} 若 $A \subseteq B$, 则 $f(A) \subseteq f(B)$;若$B_1\subseteq B_2\subseteq Y$,则$f^{-1}(B_1)\subseteq f^{-1}(B_2).$

\item\label{proposition:映射的基本性质-2} $f\left(\bigcup_{\alpha \in I} A_{\alpha}\right) = \bigcup_{\alpha \in I} f(A_{\alpha})$;$f^{-1}\left(\bigcup_{\alpha \in I} B_{\alpha}\right) = \bigcup_{\alpha \in I} f(B_{\alpha})$.

\item\label{proposition:映射的基本性质-3} $f^{-1}\left(\bigcap_{\alpha\in I}B_{\alpha}\right)=\bigcap_{\alpha\in I}f^{-1}(B_{\alpha})$;

$f\left(\bigcap_{\alpha \in I} A_{\alpha}\right) \subseteq \bigcap_{\alpha \in I} f(A_{\alpha})$, 当且仅当 $f$ 为单射时“$=$” 成立.

\item\label{proposition:映射的基本性质-4} $f\left(f^{-1}(B)\right) \subseteq B$, 当且仅当 $f$ 为满射时“$=$” 成立;

$A \subseteq f^{-1}(f(A))$, 当且仅当 $f$ 为单射时“$=$” 成立.

\item\label{proposition:映射的基本性质-5} $f\left( X\setminus B \right) =f^{-1}(B^c)=\left( f^{-1}(B) \right) ^c=Y\setminus f^{-1}(B)$成立;

但$f\left( X\setminus A \right) =f\left( A^c \right) =f\left( A \right) ^c=f\left( X \right) \setminus f\left( A \right)$不一定成立,只有$f(X)\setminus f(A)\subseteq f(X\setminus A).$
\end{enumerate}
\end{proposition}
\begin{note}
\ref{proposition:映射的基本性质-4}中第一条的直观理解是:$B$中某些元素不一定有原像(即$f$可能不是满射).

\ref{proposition:映射的基本性质-4}中第二条的直观理解是:$X\setminus A$中的某些元素的像也可能在$f(A)$(即$f$可能不是单射).
\end{note}
\begin{proof}
\begin{enumerate}[(1)]
\item 显然.

\item 显然.

\item 只证明两个集合的情形. 注意到 $A_1 \cap A_2 \subseteq A_1$, $A_1 \cap A_2 \subseteq A_2$, 由 (i) 知
\begin{align*}
f(A_1 \cap A_2) \subseteq f(A_1), \quad f(A_1 \cap A_2) \subseteq f(A_2)
\end{align*}
故 $f(A_1 \cap A_2) \subseteq f(A_1) \cap f(A_2)$.

设 $y \in f(A_1) \cap f(A_2)$, 则 $y \in f(A_1)$ 且 $y \in f(A_2)$, 故存在 $x_1 \in A_1$, $x_2 \in A_2$ 使得 $y = f(x_1) = f(x_2)$. 由于 $f$ 是单射, 则必有 $x_1 = x_2 = x$. 所以
\begin{align*}
x \in A_1 \cap A_2, \quad y = f(x) \in f(A_1 \cap A_2)
\end{align*}
故
\begin{align*}
f(A_1) \cap f(A_2) \subseteq f(A_1 \cap A_2)
\end{align*}
因此, “$=$” 成立.

反之, 假设 $f$ 不是单射, 则存在 $x_1 \neq x_2$ 使得 $f(x_1) = f(x_2)$. 构造集合 $A_1 = \{x_1\}$, $A_2 = \{x_2\}$, 则 $A_1 \cap A_2 = \varnothing$, 从而 $f(A_1 \cap A_2) = \varnothing$. 而
\begin{align*}
f(A_1) \cap f(A_2) = \{f(x_1)\} \neq \varnothing
\end{align*}
故 $f(A_1 \cap A_2) \neq f(A_1) \cap f(A_2)$. 矛盾.

\item (i) 设 $y \in f(f^{-1}(B))$, 则存在 $x \in f^{-1}(B)$ 使得 $y = f(x)$. 故 $y = f(x) \in B$. 因此, $f(f^{-1}(B)) \subseteq B$.

设 $y \in B$, $f$ 为满射, 则存在 $x \in A$ 使得 $y = f(x)$. 故 $x \in f^{-1}(y) \subseteq f^{-1}(B)$, 从而 $y = f(x) \in f(f^{-1}(B))$, 于是 $B \subseteq f(f^{-1}(B))$, 因此, “$=$” 成立.

反之, 假设 $f$ 不是满射, 则 $f(A) \subsetneq B$. 由于 $f^{-1}(B) \subseteq A$, 故
\begin{align*}
f(f^{-1}(B)) \subseteq f(A) \subsetneq B
\end{align*}
与 $B = f(f^{-1}(B))$ 矛盾.

(ii) 设 $x \in A$, 则 $f(x) \in f(A)$, 故 $x \in f^{-1}(f(A))$. 因此, $A \subseteq f^{-1}(f(A))$.

设 $x \in f^{-1}(f(A))$, 则 $f(x) \in f(A)$. 再由 $f$ 是单射, 则必有 $x \in A$. 从而 $f^{-1}(f(A)) \subseteq A$. 因此, “$=$” 成立.

反之, 假设 $f$ 不是单射, 则存在 $x_1 \neq x_2$ 使得 $f(x_1) = f(x_2)$. 构造集合 $A = \{x_1\}$, 则 $f(A) = \{f(x_1)\}$. 故 $\{x_1, x_2\} \subseteq f^{-1}(f(A))$, 从而 $A \neq f^{-1}(f(A))$. 矛盾. 

\item  $f^{-1}(Y \setminus B) = f^{-1}(Y) \setminus f^{-1}(B)$成立.

{\color{blue}解法一:}
\begin{align*}
x \in f^{-1}(Y \setminus B) &\iff  f(x) \in Y \setminus B \\
&\iff  f(x) \in Y, \, f(x) \notin B \\
&\iff  x \in f^{-1}(Y) = X, \, x \notin f^{-1}(B) \\
&\iff  x \in f^{-1}(Y) \setminus f^{-1}(B).
\end{align*}

{\color{blue}解法二:}
\begin{align*}
f^{-1}(Y \setminus B) &= \{x \in X \mid f(x) \in Y \setminus B\} = X \setminus \{x \in X \mid f(x) \in B\} \\
&= X \setminus f^{-1}(B) = f^{-1}(Y) \setminus f^{-1}(B).
\end{align*}

{\color{blue}解法三:}
\begin{align*}
f^{-1}(Y \setminus B) &= f^{-1}(Y \cap B^c) = f^{-1}(Y) \cap f^{-1}(B^c) = f^{-1}(Y) \cap (f^{-1}(B))^c \\
&= f^{-1}(Y) \setminus f^{-1}(B).
\end{align*}

$f(X \setminus A) = f(X) \setminus f(A)$未必成立.只有$f(X) \setminus f(A) \subseteq f(X \setminus A)$.
事实上,
\begin{align*}
y \in f(X) \setminus f(A) &\iff  y \in f(X), \, y \notin f(A) \\
&\Rightarrow \exists x \in X \text{但} \, x \notin A, \text{s. t.} \, y = f(x) \\
&\iff  \exists x \in X \setminus A, \text{s. t.} \, y = f(x) \\
&\iff  y \in f(X \setminus A).
\end{align*}
因此,$f(X) \setminus f(A) \subseteq f(X \setminus A)$.

但是,$f(X) \setminus f(A) \nsupseteq f(X \setminus A)$.从而,$f(X \setminus A) \neq f(X) \setminus f(A)$.

反例:设$X$为多于两点的集合,$A = \{a\} \subset X$, $f: X \to Y$为常值映射,$f(x) \equiv y_0 \in Y$, $\forall x \in X$.于是
\begin{align*}
f(X) \setminus f(A) = \{y_0\} \setminus \{y_0\} = \varnothing \nsupseteq \{y_0\} = f(X \setminus A).
\end{align*}
\end{enumerate}

\end{proof}

\begin{proposition}[单调映射的不动点]\label{proposition:单调映射的不动点}
设\(X\)是一个非空集合,且有\(f:\mathcal{P}(X)\to\mathcal{P}(X)\)。若对\(\mathcal{P}(X)\)中满足\(A\subseteq B\)的任意\(A,B\),必有\(f(A)\subseteq f(B)\),则存在\(T\subset\mathcal{P}(X)\),使得\(f(T)=T\)。
\end{proposition}
\begin{proof}
作集合\(S,T\):
\begin{align*}
S=\{A:A\in\mathcal{P}(X)\text{ 且 }A\subseteq f(A)\},\quad T=\bigcup_{A\in S}A(\in\mathcal{P}(X)),
\end{align*}
则有\(f(T)=T\).

事实上,因为由\(A\in S\)可知\(A\subseteq f(A)\),从而由\(A\subseteq T\)可得\(f(A)\subseteq f(T)\)。根据\(A\in S\)推出\(A\subseteq f(T)\),这就导致
\[
\bigcup_{A\in S}A\subseteq f(T),\quad T\subseteq f(T).
\]

另一方面,又从\(T\subseteq f(T)\)可知\(f(T)\subseteq f(f(T))\)。这说明\(f(T)\in S\),我们又有\(f(T)\subseteq T\)。

\end{proof}

\begin{definition}[特征函数(示性函数)]
集合 $E$ 的\textbf{特征函数}(\textbf{示性函数})定义为
\begin{align*}
\chi_E(x) = 
\begin{cases}
1, & x \in E\\
0, & x \notin E
\end{cases}
\end{align*}
\end{definition}
\begin{note}
特征函数 $\chi_E$ 在一定意义上反映出集合 $E$ 本身的特征, 可以通过它来表示各种集合关系.
\end{note}

\begin{proposition}[特征函数的基本性质]\label{proposition:特征函数的基本性质}
设$X$为固定的集合,$A,B\subset X$,集族$\{A_\alpha\}_{\alpha\in \Gamma}$和集列$\{A_n\}_{n=1}^{\infty}$都为$X$的子集,则
\begin{enumerate}[(1)]
\item\label{proposition:特征函数的基本性质-1}  $A = B \iff \chi_A(x) = \chi_B(x)$;

$A\neq B\iff \chi_A(x)\neq \chi_B(x)$;

$A\bigtriangleup B=\left\{ x\mid \chi _A\left( x \right) \ne \chi _B\left( x \right) \right\}$.

特别地,$A=X\iff \chi_A(x)\equiv1$;$\quad A=\varnothing\iff \chi_A(x)\equiv0$.

\item\label{proposition:特征函数的基本性质-2} $A \subseteq B \iff \chi_A(x) \leqslant \chi_B(x)$.

\item\label{proposition:特征函数的基本性质-3}$\chi_{A \cap B}(x) = \chi_A(x) \cdot \chi_B(x)$.

\item\label{proposition:特征函数的基本性质-4} $\chi_{A \cup B}(x) = \chi_A(x) + \chi_B(x) - \chi_A(x) \cdot \chi_B(x)$.

\item\label{proposition:特征函数的基本性质-5}\(\chi_{A\setminus B}(x)=\chi_A(x)[1 - \chi_B(x)]\).

\item\label{proposition:特征函数的基本性质-6} \(\chi_{A\triangle B}(x)=\vert\chi_A(x)-\chi_B(x)\vert\).

\item\label{proposition:特征函数的基本性质-7} $\chi_{\bigcup_{\alpha\in\Gamma}A_\alpha}(x)=\max\limits_{\alpha\in\Gamma}\chi_{A_\alpha}(x)$;$\quad \chi_{\bigcap_{\alpha\in\Gamma}A_\alpha}(x)=\min\limits_{\alpha\in\Gamma}\chi_{A_\alpha}(x)$.
 
\item\label{proposition:特征函数的基本性质-8} $\lim\limits_{n\to+\infty}A_n\text{存在}\iff \lim\limits_{n\to+\infty}\chi_{A_n}(x)\text{存在}.$
而且当极限存在时,有
$\chi_{\lim\limits_{n\to+\infty}A_n}(x)=\lim\limits_{n\to+\infty}\chi_{A_n}(x).$
\end{enumerate} 
\end{proposition}
\begin{proof}
\begin{enumerate}[(1)]
\item 
$A=B \iff  x\in A$ 等价于 $x\in B$
$\iff  \chi_A(x)=1$ 等价于 $\chi_B(x)=1$
$\iff  \chi_A=\chi_B$.

$\chi_A\neq\chi_B \iff  \exists x_0\in X,\text{s.t.}\ \chi_A(x_0)\neq\chi_B(x_0)$
$\iff  \exists x_0\in X,\text{s.t.}\ \chi_A(x_0)=0 \text{ 且 } \chi_B(x_0)=1,\text{或者}\chi_A(x_0)=1 \text{且}\chi_B(x_0)=0$
$\iff  \exists x_0\in X,\text{s.t.}\ x_0\notin A \text{ 且 } x_0\in B,\text{或者 } x_0\in A \text{且}x_0\notin B$
$\iff  A\neq B$.

$x\in A\triangle B = (A-B)\cup(B-A)$
$\iff  x\in A-B \text{ 或 } x\in B-A$
$\iff  x\in A,x\notin B,\text{或 } x\in B,x\notin A$
$\iff  \chi_A(x)=1,\chi_B(x)=0,\text{或}\chi_B(x)=1,\chi_A(x)=0$
$\iff  \chi_A(x)\neq\chi_B(x),$

即
\begin{align*}
A\triangle B = \{x \mid \chi_A(x) \neq \chi_B(x)\}.
\end{align*}

\item 
$A\subseteq B \iff \text{对 } \forall x\in X,x\in A \text{ 必有 } x\in B$
$\iff \text{对 } \forall x\in X,\chi_A(x)=1 \text{ 必有 } \chi_B(x)=1$
$\iff  \chi_A(x)\leqslant\chi_B(x),\forall x\in X.$

\item 
当 $x\in A\cap B$ 时,有 $x\in A$ 且 $x\in B$,故
\begin{align*}
\chi_{A\cap B}(x) = 1 = 1\cdot 1 = \chi_A(x)\cdot \chi_B(x);
\end{align*}
当 $x\notin A\cap B$ 时,必有 $x\notin A$,则
\begin{align*}
\chi_{A\cap B}(x) = 0 = 0\cdot \chi_B(x) = \chi_A(x)\cdot \chi_B(x);
\end{align*}
或 $x\notin B$,则
\begin{align*}
\chi_{A\cap B}(x) = 0 = \chi_A(x)\cdot 0 = \chi_A(x)\cdot \chi_B(x).
\end{align*}

\item 
当 $x\in A\cap B\subseteq A\cup B$ 时,有
\begin{align*}
\chi_A(x) + \chi_B(x) - \chi_{A\cap B}(x) = 1+1-1 = 1 = \chi_{A\cup B}(x).
\end{align*}
当 $x\in A\setminus B(\iff  x\in A,x\notin B\Rightarrow x\notin A\cap B,x\in A\cup B)$ 时,有
\begin{align*}
\chi_A(x) + \chi_B(x) - \chi_{A\cap B}(x) = 1+0-0 = 1 = \chi_{A\cup B}(x).
\end{align*}
当 $x\in B\setminus A$ 时,同理有
\begin{align*}
\chi_A(x) + \chi_B(x) - \chi_{A\cap B}(x) = 0+1-0 = 1 = \chi_{A\cup B}(x).
\end{align*}
当 $x\in (A\cup B)^c$ 时,有
\begin{align*}
\chi_A(x) + \chi_B(x) - \chi_{A\cap B}(x) = 0+0-0 = 0 = \chi_{A\cup B}(x).
\end{align*}
再由\rrefpro{proposition:特征函数的基本性质}{proposition:特征函数的基本性质-3}可知结论成立.

\item 
当 $x\in A-B$ 时,即 $x\in A,x\notin B$,有
\begin{align*}
\chi_{A-B}(x) = 1 = 1\cdot (1-0) = \chi_A(x)[1 - \chi_B(x)];
\end{align*}
当 $x\notin A-B$ 时,必有 $x\notin A$,则
\begin{align*}
\chi_{A-B}(x) = 0 = 0\cdot [1 - \chi_B(x)] = \chi_A(x)[1 - \chi_B(x)];
\end{align*}
或 $x\in A$,且 $x\in B$,则
\begin{align*}
\chi_{A-B}(x) = 0 = 1\cdot (1-1) = \chi_A(x)[1 - \chi_B(x)].
\end{align*}

\item 
\begin{align*}
\chi_{A\triangle B}(x)=
\begin{cases}
0 = |0-0|, & \text{当 } x \in (A \cup B)^c, \\
0 = |1-1|, & \text{当 } x \in A \cap B, \\
1 = |1-0|, & \text{当 } x \in A - B, \\
1 = |0-1|, & \text{当 } x \in B - A
\end{cases}
= | \chi_A(x) - \chi_B(x) |.
\end{align*}

\item {\color{blue}证法一:}
$$\chi_{\bigcup_{\alpha\in\Gamma}A_\alpha}(x)=1\iff  x\in\bigcup_{\alpha\in\Gamma}A_\alpha\iff \exists\alpha_0\in\Gamma,\text{s.t.}x\in A_{\alpha_0}$$
$$\iff \exists\alpha_0\in\Gamma,\text{s.t.}\chi_{A_{\alpha_0}}(x)=1\iff \max\limits_{\alpha\in\Gamma}\chi_{A_\alpha}(x)=1,$$
再由$\chi_{\bigcup_{\alpha\in\Gamma}A_\alpha}(x)$与$\max\limits_{\alpha\in\Gamma}\chi_{A_\alpha}(x)$或为1或为0知
\begin{align}\label{eq::90jrowj290jfk039wf}
\chi_{\bigcup_{\alpha\in\Gamma}A_\alpha}(x)=\max\limits_{\alpha\in\Gamma}\chi_{A_\alpha}(x).
\end{align}
或者再从
$$\chi_{\bigcup_{\alpha\in\Gamma}A_\alpha}(x)=0\iff  x\notin\bigcup_{\alpha\in\Gamma}A_\alpha\iff \forall\alpha\in\Gamma,x\notin A_\alpha$$
$$\iff \forall\alpha\in\Gamma,\chi_{A_\alpha}(x)=0\iff \max\limits_{\alpha\in\Gamma}\chi_{A_\alpha}(x)=0$$
推出上面等式.于是
\begin{align*}
\chi _{\bigcap_{\alpha \in \Gamma}{A_{\alpha}}}(x)&=1-\chi _{(\bigcap_{\alpha \in \Gamma}{A_{\alpha})^c}}(x)\xlongequal{\mathrm{De} \text{\,\,Morgan公式}}1-\chi _{\bigcup_{\alpha \in \Gamma}{A_{\alpha}^{c}}}(x)
\\
&\xlongequal{\text{\eqref{eq::90jrowj290jfk039wf}式}}1-\max_{\alpha \in \Gamma} \chi _{A_{\alpha}^{c}}(x)=1-\max_{\alpha \in \Gamma} (1-\chi _{A_{\alpha}}(x))
\\
&=1-(1-\min_{\alpha \in \Gamma} \chi _{A_{\alpha}}(x))=\min_{\alpha \in \Gamma} \chi _{A_{\alpha}}(x).
\end{align*}

{\color{blue}证法二:}
$$\chi_{\bigcap_{\alpha\in\Gamma}A_\alpha}(x)=1\iff  x\in\bigcap_{\alpha\in\Gamma}A_\alpha\iff \forall\alpha\in\Gamma,x\in A_\alpha$$
$$\iff \forall\alpha\in\Gamma,\chi_{A_\alpha}(x)=1\iff \min\limits_{\alpha\in\Gamma}\chi_{A_\alpha}(x)=1.$$
再由$\chi_{\bigcap_{\alpha\in\Gamma}A_\alpha}(x)$与$\min\limits_{\alpha\in\Gamma}\chi_{A_\alpha}(x)$或为1或为0知
\begin{align}\label{eq::90jrowj290jfwk039wf}
\chi_{\bigcap_{\alpha\in\Gamma}A_\alpha}(x)=\min\limits_{\alpha\in\Gamma}\chi_{A_\alpha}(x).
\end{align}
或者再从
$$\chi_{\bigcap_{\alpha\in\Gamma}A_\alpha}(x)=0\iff  x\notin\bigcap_{\alpha\in\Gamma}A_\alpha\iff \exists\alpha_0\in\Gamma,x\notin A_{\alpha_0}$$
$$\iff \exists\alpha_0\in\Gamma,\ \chi_{A_{\alpha_0}}(x)=0\iff \min\limits_{\alpha\in\Gamma}\chi_{A_\alpha}(x)=0$$
推出上面等式.于是
\begin{align*}
\chi _{\bigcup_{\alpha \in \Gamma}{A_{\alpha}}}(x)&=1-\chi _{(\bigcup_{\alpha \in \Gamma}{A_{\alpha})^c}}(x)\xlongequal{\mathrm{De} \text{\,\,Morgan公式}}1-\chi _{\bigcap_{\alpha \in \Gamma}{A_{\alpha}^{c}}}(x)
\\
&\xlongequal{\text{\eqref{eq::90jrowj290jfwk039wf}式}}1-\min_{\alpha \in \Gamma} \chi _{A_{\alpha}^{c}}(x)=1-\min_{\alpha \in \Gamma} (1-\chi _{A_{\alpha}}(x))
\\
&=1-(1-\max_{\alpha \in \Gamma} \chi _{A_{\alpha}}(x))=\max_{\alpha \in \Gamma} \chi _{A_{\alpha}}(x).
\end{align*}

\item 
\begin{align*}
&\quad \quad \lim_{n\rightarrow +\infty} A_n\text{存在}\Longleftrightarrow \underset{n\rightarrow \infty}{\overline{\lim }}A_n=\underset{n\rightarrow \infty}{\underline{\lim }}A_n\Longleftrightarrow \underset{n\rightarrow \infty}{\overline{\lim }}A_n\subseteq \underset{n\rightarrow \infty}{\underline{\lim }}A_n\\
&\Longleftrightarrow \text{如果}x\in \underset{n\rightarrow \infty}{\overline{\lim }}A_n,\text{即有无穷个}n,\text{使得}x\in A_n,\text{则必有}n_0\in \mathbb{N} ,\text{当}n>n_0\text{时},x\in A_n\\
&\Longleftrightarrow \text{如果}x\in \underset{n\rightarrow \infty}{\overline{\lim }}A_n,\text{即有无穷个}n,\text{使得}\chi _{A_n}(x)=1,\text{则必有}n_0\in \mathbb{N} ,\text{当}n>n_0\text{时},\chi _{A_n}(x)=1\\
&\Longleftrightarrow \forall x\in X,\text{若}x\in \underset{n\rightarrow \infty}{\overline{\lim }}A_n,\text{则}\exists n_0\in \mathbb{N} ,\text{当}n>n_0\text{时},\chi _{A_n}(x)\equiv 1;\text{若}x\notin \underset{n\rightarrow \infty}{\overline{\lim }}A_n,\text{则此时}\chi _{A_n}(x)\equiv 0\\
&\Longleftrightarrow \text{对}\forall x\in X,\text{都有}\lim_{n\rightarrow +\infty} \chi _{A_n}(x)\text{存在}.
\end{align*}
当上述极限存在时,由上式有
\begin{align*}
\chi_{\lim\limits_{n\to+\infty}A_n}(x)=\begin{cases}1,\ &x\in\lim\limits_{n\to+\infty}A_n,\\0,\ &x\notin\lim\limits_{n\to+\infty}A_n\end{cases}=\lim\limits_{n\to+\infty}\chi_{A_n}(x).
\end{align*}
\end{enumerate}

\end{proof}

\begin{example}
设$\{f_n\}(n=1,2,\cdots)$为定义在$[a,b]$上的实函数列,$E\subseteq [a,b]$,且有
\begin{align*}
\lim\limits_{n \to +\infty} f_n(x) = \chi_{[a,b]\setminus E}(x).
\end{align*}
若令$E_n = \left\{x \in [a,b] \mid f_n(x) \geqslant \frac{1}{2} \right\}$,求集合$\lim\limits_{n \to +\infty} E_n$.
\end{example}
\begin{solution}
{\color{blue}解法一:}由
\begin{align*}
\varlimsup_{n \to +\infty} E_n &= \left\{ x \in [a,b] \mid \exists \text{无穷个} \, n \in \mathbb{N}, \text{s. t.} \, x \in E_n \right\} \\
&= \left\{ x \in [a,b] \mid \exists \text{无穷个} \, n \in \mathbb{N}, \text{s. t.} \, f_n(x) \geqslant \frac{1}{2} \right\} \\
&= \left\{ x \in [a,b] \mid \chi_{[a,b]\setminus E}(x) = \lim\limits_{n \to +\infty} f_n(x) = 1 \right\} \\
&= \left\{ x \in [a,b] \mid x \in [a,b] \setminus E \right\} = [a,b] \setminus E \\
&= \left\{ x \in [a,b] \mid x \in [a,b] \setminus E \right\} = \left\{ x \in [a,b] \mid \chi_{[a,b]\setminus E}(x) = \lim\limits_{n \to +\infty} f_n(x) = 1 \right\} \\
&= \left\{ x \in [a,b] \mid \exists n_0 \in \mathbb{N}, \text{当} \, n > n_0 \text{时}, f_n(x) \geqslant \frac{1}{2} \right\} \\
&= \left\{ x \in [a,b] \mid \exists n_0 \in \mathbb{N}, \text{当} \, n > n_0 \text{时}, x \in E_n \right\} = \varliminf_{n \to +\infty} E_n
\end{align*}
知,$\lim\limits_{n \to +\infty} E_n = \varlimsup_{n \to +\infty} E_n = \varliminf_{n \to +\infty} E_n = [a,b] \setminus E$.

{\color{blue}解法二:}
\begin{align*}
x \in [a,b] \setminus E &\iff  1 = \chi_{[a,b]\setminus E}(x) = \lim\limits_{n \to +\infty} f_n(x) \\
&\iff  \exists n_0 \in \mathbb{N}, \text{当} \, n > n_0 \text{时}, f_n(x) \geqslant \frac{1}{2} \\
&\iff  \exists n_0 \in \mathbb{N}, \text{当} \, n > n_0 \text{时}, x \in E_n \\
&\iff  x \in \varliminf_{n \to +\infty} E_n \Rightarrow x \in \varlimsup_{n \to +\infty} E_n \\
&\iff  \text{有无穷个} \, n \in \mathbb{N}, \text{s. t.} \, x \in E_n \\
&\iff  \text{有无穷个} \, n \in \mathbb{N}, \text{s. t.} \, f_n(x) \geqslant \frac{1}{2} \\
&\iff  \chi_{[a,b]\setminus E}(x) = \lim\limits_{n \to +\infty} f_n(x) \geqslant \frac{1}{2} \\
&\iff  \chi_{[a,b]\setminus E}(x) = 1 \iff  x \in [a,b] \setminus E.
\end{align*}
由此推出
\begin{align*}
\lim\limits_{n \to +\infty} E_n = \varlimsup_{n \to +\infty} E_n = \varliminf_{n \to +\infty} E_n = [a,b] \setminus E.
\end{align*}

\end{solution}








\end{document}
\documentclass[../../main.tex]{subfiles}
\graphicspath{{\subfix{../../image/}}} % 指定图片目录,后续可以直接使用图片文件名。

% 例如:
% \begin{figure}[h]
% \centering
% \includegraphics{image-01.01}
% \caption{图片标题}
% \label{fig:image-01.01}
% \end{figure}
% 注意:上述\label{}一定要放在\caption{}之后,否则引用图片序号会只会显示??.

\begin{document}

\section{可列集与不可列集}

\subsection{可列集}

\begin{definition}[可列集]
与自然数集 $\mathbb{N}$ 对等的集合称为\textbf{可列集}, 其基数记为 $\aleph_0$(读作阿列夫零). 有限集和可列集统称为\textbf{可数集}.
\end{definition}

\begin{proposition}\label{proposition:可列集的表示方式}
$A$ 是可列集当且仅当 $A$ 可以写成 $A = \{a_n\}_{n = 1}^{\infty}$.
\end{proposition}
\begin{proof}
$A$ 可列, 则存在 $\mathbb{N}$ 到 $A$ 的一一映射 $\varphi$, 记为 $\varphi(n) = a_n, n \in \mathbb{N}$, 则 $A = \{a_n\}_{n = 1}^{\infty}$. 反过来, 若 $A = \{a_n\}_{n = 1}^{\infty}$, 将每个 $a_n$ 与其下标 $n$ 建立一一对应, 则 $A$ 与 $\mathbb{N}$ 对等, 从而是可列集
\end{proof}

\begin{proposition}[可列集的性质]\label{proposition:可列集的性质}
\begin{enumerate}[(1)]
\item\label{proposition:可列集的性质(1)} 任何无限集必包含一个可列子集.

\item\label{proposition:可列集的性质(2)} 可列集的任何无限子集都是可列集.

\item\label{proposition:可列集的性质(3)} 有限集与可列集的并集是可列集.

\item\label{proposition:可列集的性质(4)} 有限个可列集的并集是可列集.

\item\label{proposition:可列集的性质(5)} 可列个可列集的并集是可列集.

\item\label{proposition:可列集的性质(6)} 若 $A$ 为无限集, $B$ 为有限集或可列集, 则 $\overline{\overline{A \cup B}} = \overline{\overline{A}}$.

\item\label{proposition:可列集的性质(7)} 设 $A, B$ 为可列集, 则 $A \times B$ 是可列集.

\item\label{proposition:可列集的性质(8)} 若 $A_1, A_2, \cdots, A_n$ 可列, 则$A_1 \times A_2 \times \cdots \times A_n$可列. 
\end{enumerate}
\end{proposition}
\begin{note}
(1)也说明, 众多无限集中, 最小的基数是可列集的基数 $\aleph_0$.
\end{note}
\begin{proof}
\begin{enumerate}[(1)]
\item 设 $A$ 为无限集. 从 $A$ 中任取一元 $a_1$; 由于 $A - \{a_1\} \neq \varnothing$, 取 $a_2 \in A - \{a_1\}$; 又 $A - \{a_1, a_2\} \neq \varnothing$, 取 $a_3 \in A - \{a_1, a_2\}$; $\cdots\cdots$, 因为 $A$ 是无限集, 这一过程可以一直继续下去, 从而得到 $A$ 的一个可列子集 $\{a_n\}_{n = 1}^{\infty}$.

\item 设 $A = \{a_1, a_2, \cdots, a_n, \cdots\}$. $B$ 是 $A$ 的无限子集. 按照 $A$ 中元素的次序依次寻找 $B$ 中元素, 分别记为 $a_{n_1}, a_{n_2}, \cdots$, 则 $B = \{a_{n_1}, a_{n_2}, \cdots, a_{n_k}, \cdots\}$ 为可列集.

\item 设 $A = \{a_1, a_2, \cdots, a_n\}$, $B = \{b_1, b_2, \cdots\}$. 不妨设 $A \cap B = \varnothing$, 则
\begin{align*}
A \cup B = \{a_1, a_2, \cdots, a_n, b_1, b_2, \cdots\}
\end{align*}
可列.

\item 设 $A_k = \{a_1^{(k)}, a_2^{(k)}, a_3^{(k)}, \cdots\}$, $k = 1, 2, \cdots, n$ 为可列集, 则 $\bigcup_{k = 1}^{n} A_k$ 的元素可以按下面的方式编号排序
\begin{align*}
A_1 &= \{a_1^{(1)} \quad a_2^{(1)} \quad a_3^{(1)} \quad \cdots\}\\
A_2 &= \{a_1^{(2)} \quad a_2^{(2)} \quad a_3^{(2)} \quad \cdots\}\\
&\vdots\\
A_n &= \{a_1^{(n)} \quad a_2^{(n)} \quad a_3^{(n)} \quad \cdots\}
\end{align*}
必要时删掉后续的重复元(实际上,取集合后就自动删去了重复元,因为集合内不含重复元), 可得到
\begin{align*}
\bigcup_{k = 1}^{n} A_k = \{a_1^{(1)}, \cdots, a_1^{(n)}, a_2^{(1)}, \cdots, a_2^{(n)}, a_3^{(1)}, \cdots\}
\end{align*}
可列.

\item 设 $\{A_n\}_{n = 1}^{\infty}$ 为一列可列集, 则 $\bigcup_{n = 1}^{\infty} A_n$ 的元素可以按下面的方式编号排序
\begin{align*}
A_1 &= \{a_1^{(1)} \to a_2^{(1)} \to a_3^{(1)} \to a_4^{(1)} \cdots\}\\
A_2 &= \{a_1^{(2)} \quad a_2^{(2)} \quad a_3^{(2)} \quad a_4^{(2)} \cdots\}\\
A_3 &= \{a_1^{(3)} \quad a_2^{(3)} \quad a_3^{(3)} \quad a_4^{(3)} \cdots\}\\
&\vdots\\
A_n &= \{a_1^{(n)} \quad a_2^{(n)} \quad a_3^{(n)} \quad a_4^{(n)} \cdots\}\\
&\vdots
\end{align*}
必要时删掉后续的重复元(实际上,取集合后就自动删去了重复元,因为集合内不含重复元), 可得到
\begin{align*}
\bigcup_{n = 1}^{\infty} A_n = \{a_1^{(1)}, a_1^{(2)}, a_2^{(1)}, \cdots, a_{2n + 1}^{(1)}, a_{2n}^{(2)}, \cdots, a_{2n + 1}^{(n)}, \cdots\}
\end{align*}
(依次是下标之和等于 $2, 3, \cdots, 2n + 2, \cdots$ ) 可列.


\item 不妨设 $A \cap B = \varnothing$, 否则用 $B - A$ 代替 $B$ 即可. $A$ 为无限集, 由(1)可知, $A$ 包含一个可列子集 $A_1$. 由于 $A_1 \cup B$ 是可列集, 故 $A_1 \cup B \sim A_1$. 注意到 $(A - A_1) \cap (A_1 \cup B) = \varnothing$, 则有
\begin{align*}
A \cup B = (A - A_1) \cup A_1 \cup B
= (A - A_1) \cup (A_1 \cup B) \sim (A - A_1) \cup A_1 = A.
\end{align*}
因此, $\overline{\overline{A \cup B}} = \overline{\overline{A}}$. 

\item 由\refproposition{proposition:可列集的表示方式}可设$A=\left\{ x_i \right\} _{i=1}^{\infty},B=\left\{ y_i \right\} _{i=1}^{\infty}$,则
\begin{align*}
A\times B&=\{(x,y):x\in A,y\in B\}=\bigcup_{x\in A}{\{(x,y):y}\in B\}
\\
&=\bigcup_{x\in A}{\bigcup_{y\in B}{\left( x,y \right)}}=\bigcup_{i=1}^{\infty}{\bigcup_{j=1}^{\infty}{\left( x_i,y_j \right)}}.
\end{align*}
由\ref{proposition:可列集的性质(5)}可知,对$\forall i\in \mathbb{N}$,$\bigcup_{j=1}^{\infty}{\left( x_i,y_j \right)}$都可列.于是再由\ref{proposition:可列集的性质(5)}可知$\bigcup_{i=1}^{\infty}{\bigcup_{j=1}^{\infty}{\left( x_i,y_j \right)}}$也可列.

\item 利用(7)及数学归纳法不难证明.
\end{enumerate}
\end{proof}

\begin{example}
有理数集 $\mathbb{Q}$ 是可列集.
\end{example}
\begin{proof}
$\mathbb{Q} = \mathbb{Q}^+ \cup \{0\} \cup \mathbb{Q}^-$, 其中 $\mathbb{Q}^+, \mathbb{Q}^-$ 分别表示正、负有理数集. 由对称性以及\hyperref[proposition:可列集的性质(3)]{命题\ref{proposition:可列集的性质}\ref{proposition:可列集的性质(3)}}和\hyperref[proposition:可列集的性质(4)]{命题\ref{proposition:可列集的性质}\ref{proposition:可列集的性质(4)}}, 只需证明 $\mathbb{Q}^+$ 可列.

对每个 $n \in \mathbb{N}$, 令
\begin{align*}
A_n = \left\{\frac{1}{n}, \frac{2}{n}, \frac{3}{n}, \cdots\right\}
\end{align*}
则 $A_n$ 可列. 又 $\mathbb{Q}^+ = \bigcup_{n = 1}^{\infty} A_n$(除去重复元), 由可列集性质知 $\mathbb{Q}^+$ 可列.
\end{proof}

\begin{example}
实轴上互不相交的开区间至多有可列个.
\end{example}
\begin{proof}
开区间的长度大于 $0$, 故必含有有理数, 在每一个开区间内取出一个有理数. 因为开区间互不相交, 所以取出的有理数都不相等, 从而这些有理数构成 $\mathbb{Q}$ 的一个子集. 又 $\mathbb{Q}$ 是可列集, 故这样的开区间至多有可列个.
\end{proof}

\begin{example}
整系数多项式的全体 $\mathbf{P}$ 是可列集.
\end{example}
\begin{proof}
对每个 $n \in \{0\} \cup \mathbb{N}$, 令
\begin{align*}
P_n = \{a_0x^n + a_1x^{n - 1} + \cdots + a_n : a_i \in \mathbb{Z}, i = 0, 1, \cdots, n, a_0 \neq 0\}
\end{align*}
则
\begin{align*}
P_n \sim \mathbb{Z} - \{0\} \times \underbrace{\mathbb{Z} \times \cdots \times \mathbb{Z}}_{n 个}
\end{align*}
由\hyperref[proposition:可列集的性质(8)]{命题\ref{proposition:可列集的性质}\ref{proposition:可列集的性质(8)}}知 $P_n$ 可列. 又 $\mathbf{P} = \bigcup_{n = 0}^{\infty} P_n$, 由可列集性质知 $\mathbf{P}$ 可列.

整系数多项式的根称为代数数, 由于每个多项式只有有限个根, 故代数数的全体构成一可列集.
\end{proof}

\begin{example}
$\mathbb{R}$ 上单调函数的不连续点至多有可列个.
\end{example}
\begin{proof}
不妨设 $f$ 单调递增, 若 $x_0$ 是 $f$ 的不连续点, 则
\begin{align*}
f(x_0 - 0) = \lim_{x \to x_0^-} f(x) < \lim_{x \to x_0^+} f(x) = f(x_0 + 0)
\end{align*}
故 $x_0$ 就对应着一个开区间 $(f(x_0 - 0), f(x_0 + 0))$. 显然, 若 $x_1, x_2$ 是 $f$ 的不同不连续点, 则对应区间 $(f(x_1 - 0), f(x_1 + 0))$ 与 $(f(x_2 - 0), f(x_2 + 0))$ 互不相交. 而 $\mathbb{R}$ 上互不相交的开区间至多有可列个, 故 $f$ 的不连续点也至多有可列个. 
\end{proof}






\subsection{不可列集}

\begin{definition}[不可列集]
不是可列集的无限集称为\textbf{不可列集}.
\end{definition}

\begin{theorem}
$[0, 1]$ 是不可列集.
\end{theorem}
\begin{proof}
假设 $[0, 1]$ 可列, 则可表示为 $[0, 1] = \{x_n\}_{n = 1}^{\infty}$.
把 $[0, 1]$ 三等分为: $[0, 1/3]$, $[1/3, 2/3]$, $[2/3, 1]$, 则其中至少有一个闭区间不包含 $x_1$, 记该区间为 $I_1$, 则 $x_1 \notin I_1$; 把 $I_1$ 三等分, 则其中至少有一个闭区间不包含 $x_2$, 记该区间为 $I_2$, 则 $x_2 \notin I_2$, $I_2 \subset I_1$; $\cdots\cdots$, 依次做下去, 可得到一列闭区间 $\{I_n\}$ 满足:
\begin{enumerate}[(i)]
\item $I_1 \supset I_2 \supset \cdots \supset I_n \supset \cdots$;
\item $x_n \notin I_n$, $n \in \mathbb{N}$;
\item $I_n$ 的长度为 $1/3^n \to 0$, $n \to \infty$.
\end{enumerate}
由闭区间套定理, 存在 $\xi \in \bigcap_{n = 1}^{\infty} I_n$. 由于 $\xi \in [0, 1] = \{x_n\}_{n = 1}^{\infty}$, 则必存在 $n_0 \in \mathbb{N}$ 使得 $\xi = x_{n_0}$. 而 $x_{n_0} \notin I_{n_0}$, 这与 $\xi \in \bigcap_{n = 1}^{\infty} I_n$ 矛盾. 
\end{proof}

\begin{definition}
若 $A \sim [0, 1]$, 则称 $A$ 具有\textbf{连续基数}, 记 $\overline{\overline{A}} = \aleph$.
\end{definition}

\begin{theorem}
对$\forall a,b\in \mathbb{R}$,都有
$\overline{\overline{[a, b]}} = \overline{\overline{(a, b)}} = \overline{\overline{(a, b]}} = \overline{\overline{\mathbb{R}}} = \aleph$.
\end{theorem}
\begin{proof}
对$\forall a,b\in \mathbb{R}$,映射 $f(x) = a + (b - a)x$ 建立了 $[0, 1]$ 与 $[a, b]$ 之间的一一对应, 故 $\overline{\overline{[a, b]}} = \aleph$. 又 $(a, b)$ 和 $(a, b]$ 与 $[a, b]$ 分别只差一个点和两个点, 由\refpropositionnumber{proposition:可列集的性质}{proposition:可列集的性质(6)}知 $\overline{\overline{(a, b)}} = \overline{\overline{(a, b]}} = \overline{\overline{[a, b]}} = \aleph$. 最后, 由 $\S$\refexample{example:[-1,1]与R对等}以及刚证明的结论可得, $\overline{\overline{\mathbb{R}}} = \overline{\overline{[-1, 1]}} = \aleph$.
\end{proof}

\begin{corollary}
无理数的基数为 $\aleph$.
\end{corollary}
\begin{proof}
记无理数集为 $\mathbb{I}$, 注意到 $\mathbb{I} \cup \mathbb{Q} = \mathbb{R}$, 且 $\mathbb{Q}$ 可列, 由\refpropositionnumber{proposition:可列集的性质}{proposition:可列集的性质(6)}可得 $\overline{\overline{\mathbb{I}}} = \overline{\overline{\mathbb{I} \cup \mathbb{Q}}} = \overline{\overline{\mathbb{R}}} = \aleph$. 
\end{proof}

\begin{theorem}
设 $\{A_n\}$ 为一集列, 若对每个 $n$ 都有 $\overline{\overline{A_n}} = \aleph$, 则 $\overline{\overline{\bigcup_{n = 1}^{\infty} A_n}} = \aleph$.
\end{theorem}
\begin{proof}
不妨设 $A_i \cap A_j = \varnothing$. 由于 $\overline{\overline{A_n}} = \aleph$, 则 $A_n \sim (n, n + 1]$, 从而 $\bigcup_{n = 1}^{\infty} A_n \sim [1, \infty) \sim \mathbb{R}$.
\end{proof}

\begin{definition}
设 $A$ 为集合, 记 $2^A$ 为 $A$ 的幂集. 若 $A$ 为含有 $n$ 个元素的有限集, 则 $2^A$ 由 $1$ 个空集, $\mathrm{C}_{n}^1$ 个单元素集, $\mathrm{C}_{n}^2$ 个两元素集, $\cdots\cdots$, $\mathrm{C}_{n}^n$ 个 $n$ 元素集, 所以, $2^A$ 中元素的个数为
\begin{align*}
1 + \mathrm{C}_{n}^1 + \mathrm{C}_{n}^2 + \cdots + \mathrm{C}_{n}^n = (1 + 1)^n = 2^n = 2^{\overline{\overline{A}}}
\end{align*}
更一般地, 设 $\overline{\overline{A}} = \mu$, 定义 $\overline{\overline{2^A}} = 2^{\mu}$.
\end{definition}

\begin{proposition}\label{proposition:两集合对等的充要条件是幂集也对等}
设$A,B$都是非空集合,则$A\sim B$的充要条件是$2^A\sim 2^B$.
\end{proposition}
\begin{proof}
必要性:
由$A\sim B$可知$\overline{\overline{A}}=\overline{\overline{B}}$.于是$\overline{\overline{2^A}}=2^{\overline{\overline{A}}}=2^{\overline{\overline{B}}}=\overline{\overline{2^B}}.$故$2^A\sim 2^B$.

充分性:假设$A$与$B$不对等,则不妨设$\overline{\overline{A}}>\overline{\overline{B}}$,则$\overline{\overline{2^A}}=2^{\overline{\overline{A}}}>2^{\overline{\overline{B}}}=\overline{\overline{2^B}}$,这与$2^A\sim 2^B$矛盾!故$A\sim B$.
\end{proof}

\begin{lemma}\label{lemma:2^A与A上所有特征函数全体对等}
设$A$是一个非空集合,则$A$ 上所有特征函数的全体$\mathcal{F}_{A}$与 $2^A$ 对等,即$\mathcal{F}_{A}\sim 2^A$.进而$\overline{\overline{\mathcal{F}_{A}}}=\overline{\overline{2^A}}=2^{\overline{\overline{A}}}.$
\end{lemma}
\begin{proof}
对于每个 $E \in 2^A$, 可以唯一的对应一个特征函数
\begin{align*}
\chi_E(x) = 
\begin{cases}
1, & x \in E\\
0, & x \in A - E
\end{cases}
\end{align*}
反之亦然. 这说明 $A$ 上所有特征函数的全体$\mathcal{F}_{A}$与 $2^A$ 对等. 
\end{proof}

\begin{theorem}\label{theorem:aleph = 2^{aleph_0}}
$\aleph = 2^{\aleph_0}$.
\end{theorem}
\begin{proof}
用 $\mathcal{F}_{\mathbb{N}}$ 表示 $\mathbb{N}$ 上特征函数的全体, 只需证 $\mathcal{F}_{\mathbb{N}}$ 与 $(0, 1]$ 对等.

对任意的 $\varphi \in \mathcal{F}_{\mathbb{N}}$, 作映射
\begin{align*}
f:\varphi \rightarrow \sum_{n=1}^{\infty}{\frac{\varphi (n)}{3^n}},\varphi \left( n \right) \in \left\{ 0,1 \right\} .
\end{align*}
易知, $f$ 是从 $\mathcal{F}_{\mathbb{N}}$ 到 $(0, 1]$ 的单射, 故\refproposition{definition:集合的基数大小关系}可知$\overline{\overline{\mathcal{F}_{\mathbb{N}}}} \leqslant \overline{\overline{(0, 1]}}$.

另一方面, 对每一个 $x \in (0, 1]$, 用 $2$ 进制表示为 (不进位)
\begin{align*}
x = \sum_{n = 1}^{\infty} \frac{a_n}{2^n}, a_n \in \{0, 1\}.
\end{align*}
定义映射
\begin{align*}
g : x \to \varphi \in \mathcal{F}_{\mathbb{N}}, \quad \varphi(n) = a_n, \quad n = 1, 2, \cdots
\end{align*}
易知, $g$ 是从 $(0, 1]$ 到 $\mathcal{F}_{\mathbb{N}}$ 的单射, 故由\refproposition{definition:集合的基数大小关系}可知$\overline{\overline{(0, 1]}} \leqslant \overline{\overline{\mathcal{F}_{\mathbb{N}}}}$.

由\hyperref[theorem:Bernstein定理]{Bernstein定理}可知 $\overline{\overline{(0, 1]}} = \overline{\overline{\mathcal{F}_{\mathbb{N}}}}$. 再由\reflemma{lemma:2^A与A上所有特征函数全体对等}可得$\aleph =\overline{\overline{(0,1]}}=\overline{\overline{\mathcal{F} _{\mathbb{N}}}}=\overline{\overline{2^{\mathbb{N}}}}=2^{\aleph _0}.$
\end{proof}

\begin{example}
$\mathbb{R}^2 \sim \mathbb{R}$.
\end{example}
\begin{proof}
由\reftheorem{theorem:aleph = 2^{aleph_0}}及\refproposition{proposition:两集合对等的充要条件是幂集也对等}和$\mathbb{N}\sim \mathbb{Z}-\mathbb{N}$可知$\mathbb{R} \sim 2^{\mathbb{N}} \sim 2^{\mathbb{Z} - \mathbb{N}}$, 故再由\refproposition{proposition:对等集合的积也对等}可得$\mathbb{R} \times \mathbb{R} \sim 2^{\mathbb{N}} \times 2^{\mathbb{Z} - \mathbb{N}} \sim 2^{\mathbb{Z}} \sim 2^{\mathbb{N}} \sim \mathbb{R}$.
\end{proof}

\begin{example}
用 $M$ 表示 $[0, 1]$ 上实值有界函数的全体, 则 $\overline{\overline{M}} = 2^{\aleph}$.
\end{example}
\begin{proof}
设 $E$ 为 $[0, 1]$ 的任一子集, 则 $E$ 唯一对应一个特征函数
\begin{align*}
\chi_E(x) = 
\begin{cases}
1, & x \in E\\
0, & x \in [0, 1] - E
\end{cases}
\end{align*}
显然, $\chi_E \in M$. 故 $\overline{\overline{M}} \geqslant \overline{\overline{2^{[0, 1]}}} = 2^{\aleph}$.

另一方面, 对每一个 $f \in M$, 其图像 $\{(x, f(x)) : x \in [0, 1]\}$ 为平面上的一有界子集, 两者构成一一对应关系, 故 $\overline{\overline{M}} \leqslant \overline{\overline{2^{\mathbb{R}^2}}} = \overline{\overline{2^{\mathbb{R}}}} = 2^{\aleph}$. 由伯恩斯坦定理, $\overline{\overline{M}} = 2^{\aleph}$. 
\end{proof}

\begin{theorem}[无最大基数定理]\label{theorem:无最大基数定理}
设 $A$ 为非空集, 则 $\overline{\overline{A}} < \overline{\overline{2^A}}$.
\end{theorem}
\begin{proof}
由于 $2^A$ 中的单元素集与 $A$ 对等, 故 $\overline{\overline{A}} \leqslant \overline{\overline{2^A}}$.

若存在集合 $A$ 满足 $\overline{\overline{A}} = \overline{\overline{2^A}}$, 则存在 $f : A \to 2^A$ 为一一映射. 令
\begin{align*}
B = \{x \in A : x \notin f(x)\}
\end{align*}
注意到 $\varnothing \in 2^A$, 则存在 $x_0 \in A$ 使得 $f(x_0) = \varnothing$, 故 $x_0 \notin f(x_0)$. 这说明 $x_0 \in B$, 从而 $B \neq \varnothing$.

又 $B \in 2^A$, 则存在 $x_B \in A$ 使得 $f(x_B) = B$. 下面考察 $x_B$ 与 $B$ 的关系: 若 $x_B \in B$, 则 $x_B \notin f(x_B) = B$, 矛盾; 若 $x_B \notin B$, 即 $x_B \notin f(x_B)$, 这又蕴涵 $x_B \in B$, 矛盾. 
\end{proof}





\end{document}
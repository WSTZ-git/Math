\documentclass[../../main.tex]{subfiles}% 注意这里的文件路径不能用 ./main.tex ,否则用latexmk编译子文件会报错
\graphicspath{{\subfix{./image/}}} % 指定图片目录,后续可以直接使用图片文件名
% 注意这里的文件路径不能用 ../../image/ ,否则用latexmk编译子文件会报错

% 例如:
% \begin{figure}[H]
% \centering
% \includegraphics[scale=0.3]{图.png}
% \caption{}
% \label{figure:图}
% \end{figure}
% 注意:上述\label{}一定要放在\caption{}之后,否则引用图片序号会只会显示??.

\begin{document}

\section{集类.环、$\sigma$环、代数、$\sigma$代数、单调类}

\begin{definition}
设$X$为取定的集合,以$X$的某些子集为元素所成的集称为$X$上的\textbf{集类}.而$X$称为\textbf{基本空间}.集类用花体大写字母或希腊字母表示.例如:$\mathscr{A},\mathscr{B},\mathscr{C},\mathscr{D},\mathscr{E},\mathscr{F},\mathscr{R}$;$\tau,\mu,\nu$等.
\end{definition}

\begin{definition}
设$X$为一个集合,$\mathscr{R}$为$X$上的一个非空集类,如果对$\forall E_1,E_2 \in \mathscr{R}$,都有
\begin{align*}
E_1 \cup E_2 \in \mathscr{R}, \quad E_1 \setminus E_2 \in \mathscr{R},
\end{align*}
则称$\mathscr{R}$为$X$上的一个环.特别地,如果还有$X \in \mathscr{R}$,就称$\mathscr{R}$为$X$上的一个\textbf{代数},或称为\textbf{域}.

如果对任何$E,F \in \mathscr{R}$,有$E \setminus F \in \mathscr{R}$;且对任何一列$E_i \in \mathscr{R}(i=1,2,\cdots)$,都有
\begin{align*}
\bigcup_{i=1}^{\infty} E_i \in \mathscr{R},
\end{align*}
则称$\mathscr{R}$为$X$上的一个$\boldsymbol{\sigma }$\textbf{环}.如果还有$X \in \mathscr{R}$,则称$\mathscr{R}$为$X$上的$\boldsymbol{\sigma }$\textbf{代数},或称为$\boldsymbol{\sigma }$\textbf{域}.
\end{definition}
\begin{remark}
设$\mathscr{R}$为$X$上的$\sigma$代数,对$\forall E_1,E_2 \in \mathscr{R}$,取$E_i = E_2(i \geqslant 3)$,则$E_1 \cup E_2 = \bigcup_{i=1}^{\infty} E_i \in \mathscr{R}$,$E_1 \setminus E_2 \in \mathscr{R}$.所以,\textbf{$\boldsymbol{\sigma}$环必为环,$\boldsymbol{\sigma}$代数必为代数.}

由定义可知,环是对集的“$\cup$”及“$\setminus$”运算封闭的非空集类.而代数是对“余或补”运算也封闭的环(因为$\mathscr{R}$为非空集类,故有$E \in \mathscr{R}$,从而$E^c = X \setminus E \in \mathscr{R}$).$\sigma$环是对集的“$\bigcup_{i=1}^{\infty}$”及“$\setminus$”运算封闭的非空集类.而$\sigma$代数是对“余或补”运算也封闭的$\sigma$环.
\end{remark}

\begin{theorem}\label{theorem:集合论定理1.3..948h3}
设$\mathscr{R}$为环,则
\begin{enumerate}[(1)]
\item\label{theorem:集合论定理1.3..948h3-1} 空集$\varnothing \in \mathscr{R}$.
\item\label{theorem:集合论定理1.3..948h3-2} $\mathscr{R}$对“$\cap$”运算封闭.
\item\label{theorem:集合论定理1.3..948h3-3} 如果$E_i \in \mathscr{R}(i=1,2,\cdots,n)$,则$\bigcup_{i=1}^{n} E_i \in \mathscr{R}$.
\item\label{theorem:集合论定理1.3..948h3-4} 设$\mathscr{R}_\alpha(\alpha \in \Gamma)$为环(代数),则$\bigcap_{\alpha \in \Gamma} \mathscr{R}_\alpha$仍为环(代数).
\end{enumerate}
\end{theorem}
\begin{proof}
\begin{enumerate}[(1)]
\item 因环$\mathscr{R}$为非空集类,故$\exists E \in \mathscr{R}$,根据环的定义有$\varnothing = E \setminus E \in \mathscr{R}$.

\item 设$E_1,E_2 \in \mathscr{R}$,则
\begin{align*}
E_1 \cap E_2 = (E_1 \cup E_2) \setminus (E_1 \setminus E_2) \setminus (E_2 \setminus E_1) \in \mathscr{R}.
\end{align*}

\item 由环对“$\cup$”封闭和归纳法立即推得.

\item 设$E_1,E_2 \in \mathscr{R}_\alpha(\alpha \in \Gamma)$,由$\mathscr{R}_\alpha$都为环,故$E_1 \cup E_2 \in \mathscr{R}_\alpha$, $E_1 \setminus E_2 \in \mathscr{R}_\alpha(\alpha \in \Gamma)$,从而$E_1 \cup E_2 \in \bigcap_{\alpha \in \Gamma} \mathscr{R}_\alpha$, $E_1 \setminus E_2 \in \bigcap_{\alpha \in \Gamma} \mathscr{R}_\alpha$,即$\bigcap_{\alpha \in \Gamma} \mathscr{R}_\alpha$为环.进一步,如果$\mathscr{R}_\alpha(\alpha \in \Gamma)$为代数,则$X \in \mathscr{R}_\alpha(\alpha \in \Gamma)$,即$X \in \bigcap_{\alpha \in \Gamma} \mathscr{R}_\alpha$.这说明了$\bigcap_{\alpha \in \Gamma} \mathscr{R}_\alpha$为代数.
\end{enumerate}

\end{proof}

\begin{theorem}\label{theorem:sigma环或代数的基本性质}
设$\mathscr{R}$为$\sigma$环,则:
\begin{enumerate}[(1)]
\item\label{theorem:sigma环或代数的基本性质-1} $\mathscr{R}$为环.
\item\label{theorem:sigma环或代数的基本性质-2} $\mathscr{R}$对“$\bigcap_{i=1}^{\infty}$”运算封闭.
\item\label{theorem:sigma环或代数的基本性质-3} $\mathscr{R}$对“$\varlimsup_{k \to +\infty}$”,“$\varliminf_{k \to +\infty}$”,“$\lim_{k \to +\infty}$”运算封闭.

\item\label{theorem:sigma环或代数的基本性质-4} 设$\mathscr{R}_\alpha(\alpha \in \Gamma)$为$\sigma$环($\sigma$代数),则$\bigcap_{\alpha \in \Gamma} \mathscr{R}_\alpha$仍为$\sigma$环($\sigma$代数).
\end{enumerate}
\end{theorem}
\begin{proof}
\begin{enumerate}[(1)]
\item 由$\varnothing = E \setminus E \in \mathscr{R}$,当$E_1,E_2 \in \mathscr{R}$时,有
\begin{align*}
E_1 \cup E_2 = \bigcup_{i=1}^{\infty} E_i \in \mathscr{R},
\end{align*}
其中$E_i = \varnothing(i \geqslant 3)$.从而,$\mathscr{R}$为环.

\item 从
\begin{align*}
\bigcap_{i=1}^{\infty} E_i = \bigcup_{i=1}^{\infty} E_i \setminus \bigcup_{i=1}^{\infty} \left( \bigcup_{j=1}^{\infty} E_j \setminus E_i \right) \in \mathscr{R}
\end{align*}
可看出$\mathscr{R}$对“$\bigcap_{i=1}^{\infty}$”运算封闭.

\item 设$E_k \in \mathscr{R},k=1,2,\cdots$.根据\rrefthe{theorem:sigma环或代数的基本性质}{theorem:sigma环或代数的基本性质-2}和$\mathscr{R}$为$\sigma$环知
\begin{align*}
\varlimsup_{k \to +\infty} E_k = \bigcap_{n=1}^{\infty} \bigcup_{k=n}^{\infty} E_k \in \mathscr{R}, \quad \varliminf_{k \to +\infty} E_k = \bigcup_{k=1}^{\infty} \bigcap_{k=n}^{\infty} E_k \in \mathscr{R}.
\end{align*}
因此,$\mathscr{R}$对“$\varlimsup_{k \to +\infty}$”,“$\varliminf_{k \to +\infty}$”,“$\lim_{k \to +\infty}$”封闭.

\item 利用定义,容易验证.
\end{enumerate}

\end{proof}

\begin{proposition}\label{proposition:集合论----环和代数的例子}
\begin{enumerate}[(1)]
\item\label{proposition:集合论----环和代数的例子-1} 设$X$为任意集合,$X$的所有子集全体所成的集类$\mathscr{A}=2^X$为$\sigma$代数(当然也为代数).

\item\label{proposition:集合论----环和代数的例子-2} 设$X$为任意集合,$X$的有限子集(包括空集$\varnothing$)全体所成的集类$\mathscr{A}$为一个环.当且仅当$X$为有限集时,$\mathscr{A}$为代数.

\item\label{proposition:集合论----环和代数的例子-3} 设$X$为任意集合,$X$的至多可数集的全体所成的集类$\mathscr{A}$为一个$\sigma$环.当且仅当$X$为至多可数集时,$\mathscr{A}$为$\sigma$代数.
\end{enumerate}
\end{proposition}
\begin{proof}
\begin{enumerate}[(1)]
\item 

\item 

\item 
\end{enumerate}

\end{proof}

\begin{example}\label{example:12213反例1}
设$\mathbb{N}$为自然数集,$\mathbb{N}$的有限子集全体所成的集类$\mathscr{A}$为一个环.显然,$\mathbb{N} \notin \mathscr{A}$,故$\mathscr{A}$不是一个代数.又因$\bigcup_{i=1}^{\infty} \{i\} = \mathbb{N} \notin \mathscr{A}$,故$\mathscr{A}$不是一个$\sigma$代数.
\end{example}

\begin{example}\label{example:12213反例2}
设$\mathbb{R}$为实数集(它是不可数集,即不是至多可数集),$\mathbb{R}$的至多可数的子集的全体所成的集类$\mathscr{A}$为$\sigma$环.显然,$\mathbb{R} \notin \mathscr{A}$,故$\mathscr{A}$不是一个$\sigma$代数.
\end{example}

\begin{example}\label{example:12213反例3}
设$\mathbb{N}$为自然数集,$\mathbb{N}$的有限子集及其余集的全体所成的集类$\mathscr{A}$为一个代数($\mathbb{N} = \varnothing^c \in \mathscr{A}$).显然,$\bigcup_{i=1}^{\infty} \{2i\} = \{2i \mid i \in \mathbb{N}\} \notin \mathscr{A}$,故$\mathscr{A}$不是一个$\sigma$环,从而不是一个$\sigma$代数.
\end{example}

\begin{example}\label{example::--23r8u3r89uj2jr3}
设$\mathbb{R}^1 = \mathbb{R}$为实数集,则由$\mathbb{R}^1$中的有限个左开右闭的有限区间的并集
\begin{align*}
A = \bigcup_{i=1}^{n} (a_i, b_i]
\end{align*}
的全体所成的集类$\mathscr{R}_0$为一个环,但不是代数,也不是$\sigma$环.
\end{example}
\begin{remark}
$\mathscr{R}_0$中的元素都可表示成有限个两两不相交的左开右闭区间的并,但表示法并不惟一.如:$(0,1] = \left(0, \frac{1}{2}\right] \cup \left( \frac{1}{2}, 1 \right] = \left(0, \frac{1}{3}\right] \cup \left( \frac{1}{3}, 1 \right].$
\end{remark}
\begin{remark}
由有限个开区间(或闭区间)的并集的全体所组成的集类并不是一个环.这是因为$(0,2) \setminus (0,1) = [1,2)$(或$[0,2] \setminus [0,1] = (1,2]$)不是有限个开(或闭)区间的并,故该集类不是一个环.
\end{remark}
\begin{proof}
显然,$\mathscr{R}_0$对运算“$\cup$”是封闭的.再证$\mathscr{R}_0$对运算“$\setminus$”也是封闭的.首先,$\varnothing = (a, a] \in \mathscr{R}_0$.而任何两个左开右闭区间$(a,b],(c,d]$的差$(a,b] \setminus (c,d]$只可能发生如下3种情况:\one 空集;\two 左开右闭的区间;\three 两个不相交的左开右闭区间的并.任何情况都表明$(a,b] \setminus (c,d] \in \mathscr{R}_0$.于是,对$\mathscr{R}_0$中任何
\begin{align*}
A = \bigcup_{i=1}^{n} (a_i, b_i], \quad B = \bigcup_{j=1}^{m} (c_j, d_j],
\end{align*}
有
\begin{align*}
A \setminus B = \bigcup_{i=1}^{n} (a_i, b_i] \setminus \bigcup_{j=1}^{m} (c_j, d_j] = \bigcup_{i=1}^{n} \left( (a_i, b_i] \setminus (c_m, d_m] \right) \setminus \bigcup_{j=1}^{m-1} (c_j, d_j].
\end{align*}
从$\bigcup_{i=1}^{n} \left( (a_i, b_i] \setminus (c_m, d_m] \right) \in \mathscr{R}_0$和数学归纳法知,$A \setminus B \in \mathscr{R}_0$.而
\begin{align*}
A \cup B = \left( \bigcup_{i=1}^{n} (a_i, b_i) \right) \cup \left( \bigcup_{j=1}^{m} (c_j, d_j] \right) \in \mathscr{R}_0
\end{align*}
是显然的.因此,$\mathscr{R}_0$为一个环.
因为$\mathbb{R}^1 = \bigcup_{i \in \mathbb{Z}} (i, i+1] \notin \mathscr{R}_0$,故$\mathscr{R}_0$不是代数,也不是$\sigma$环.

\end{proof}

\begin{example}
当$a \leqslant b, c \leqslant d$时,称
\begin{align*}
\{(x,y) \mid a < x \leqslant b, c < y \leqslant d\} = (a,b] \times (c,d] \subset \mathbb{R}^2
\end{align*}
为左下开右上闭的区间(或矩形).类似\refexa{example::--23r8u3r89uj2jr3},由有限个左下开右上闭的区间(矩形)的并集全体所成的集类$\mathscr{R}_0$是一个环.但不是代数,也不是$\sigma$环.

对于$n$维Euclid空间$\mathbb{R}^n$,由有限个形如
\begin{align*}
\{(x_1,x_2,\cdots,x_n) \mid a_i < x_i \leqslant b_i, i=1,2,\cdots,n\} = \bigotimes_{i=1}^{n} (a_i,b_i] \subset \mathbb{R}^n
\end{align*}
的区间的并集全体所成的集类$\mathscr{R}_0$是一个环.但不是代数,也不是$\sigma$环.
\end{example}
\begin{proof}


\end{proof}

环、$\sigma$环、代数、$\sigma$代数之间的关系如下图所示:
\begin{figure}[H]
\centering
\begin{tikzcd}[row sep=large, column sep=large] 
{\text{$\sigma$ 代数}} && {\text{$\sigma$ 环}} \\
\\
{\text{代数}} && {\text{环}}
\arrow[shift left, Rightarrow, from=1-1, to=1-3]
\arrow[shift left, Rightarrow, from=1-1, to=3-1]
\arrow["{\text{\refexa{example:12213反例2}}}"{inner sep=.8ex}, "\shortmid"{marking}, shift left, Rightarrow, from=1-3, to=1-1]
\arrow["{\text{\refexa{example:12213反例1}\refexa{example::--23r8u3r89uj2jr3}}}"{inner sep=.8ex}, "\shortmid"{marking}, shift left, Rightarrow, from=1-3, to=3-3]
\arrow["{\text{\refexa{example:12213反例3}}}"{inner sep=.8ex}, "\shortmid"{marking}, shift left, Rightarrow, from=3-1, to=1-1]
\arrow["{\text{\refexa{example:12213反例1}\refexa{example::--23r8u3r89uj2jr3}}}"{inner sep=.8ex}, "\shortmid"{marking}, shift left, Rightarrow, from=3-1, to=3-3]
\arrow[shift left, Rightarrow, from=3-3, to=1-3]
\arrow[shift left, Rightarrow, from=3-3, to=3-1]
\end{tikzcd}
\caption{}
\label{figure:代数和环的关系图}
\end{figure}

\begin{theorem}\label{theorem:集合论--定理1.3.7}
设$\mathscr{E}$为由集合$X$的某些子集组成的集类,则存在惟一的环(或代数、或$\sigma$环、或$\sigma$代数)$\mathscr{R}$,使得
\begin{enumerate}[(1)]
\item $\mathscr{E} \subseteq \mathscr{R}$;
\item 任何包含$\mathscr{E}$的环(或代数、或$\sigma$环、或$\sigma$代数)$\mathscr{R}'$必有$\mathscr{R} \subseteq \mathscr{R}'$.换言之,$\mathscr{R}$是包含$\mathscr{E}$的最小环(或代数、或$\sigma$环、或$\sigma$代数).
\end{enumerate}
这样的环(或代数、或$\sigma$环、或$\sigma$代数)$\mathscr{R}$称为\textbf{由集类$\mathscr{E}$所生成(或张成)的环(或代数、或$\sigma$环、或$\sigma$代数)},并用$\mathscr{R}(\mathscr{E})$(或$\mathscr{A}(\mathscr{E})$,或$\mathscr{R}_\sigma(\mathscr{E})$,或$\mathscr{A}_\sigma(\mathscr{E})$)表示.
\end{theorem}
\begin{remark}
设$\mathscr{E}$为非空集类.易见,$\mathscr{R}(\mathscr{E})$就是由$\mathscr{E}$中任取有限个元素$E_1,E_2,\cdots,E_n$经过有限次“$\cup$”,“$\setminus$”运算后所得的集的全体.

显然,$\mathscr{A}(\mathscr{E}) = \mathscr{R}(\mathscr{E} \cup \{X\})$.也就是说,$\mathscr{A}(\mathscr{E})$是由$\mathscr{E} \cup \{X\}$中任取有限个元素$E_1,E_2,\cdots,E_n$经过有限次“$\cup$”,“$\setminus$”运算后所得的集的全体.

类似地,$\mathscr{R}_\sigma(\mathscr{E})$就是由$\mathscr{E}$中任取至多可数个元素$E_1,E_2,\cdots,E_n,\cdots$经过至多可数次“$\bigcup_{n=1}^{\infty}$”,“$\setminus$”运算后所得的集的全体.

显然,$\mathscr{A}_\sigma(\mathscr{E}) = \mathscr{R}_\sigma(\mathscr{E} \cup \{X\})$.也就是说,$\mathscr{A}_\sigma(\mathscr{E})$是由$\mathscr{E} \cup \{X\}$中任取至多可数个元素$E_1,E_2,\cdots,E_n,\cdots$经过至多可数次“$\bigcup_{n=1}^{\infty}$”,“$\setminus$”运算后所得的集的全体.
\end{remark}
\begin{proof}
首先,$X$的子集全体$2^X$是一个环(或代数、或$\sigma$环、或$\sigma$代数).当然,$\mathscr{E} \subseteq 2^X$.因此,包含$\mathscr{E}$的环(或代数、或$\sigma$环、或$\sigma$代数)确实是存在的.取环(或代数、或$\sigma$环、或$\sigma$代数)族
\begin{align*}
\mu = \{\mathscr{R}' \mid \mathscr{E} \subseteq \mathscr{R}' \subseteq 2^X, \mathscr{R}' \text{为环(或代数、或$\sigma$环、或$\sigma$代数)}\}.
\end{align*}
根据\rrefthe{theorem:集合论定理1.3..948h3}{theorem:集合论定理1.3..948h3-4}(或\rrefthe{theorem:sigma环或代数的基本性质}{theorem:sigma环或代数的基本性质-4}),
\begin{align*}
\mathscr{R} = \bigcap_{\mathscr{R}' \in \mu} \mathscr{R}'
\end{align*}
为环(或代数、或$\sigma$环、或$\sigma$代数).显然,还有$\mathscr{E} \subseteq \mathscr{R}$.由$\mathscr{R}$的定义知,性质(2)成立.

如果环(或代数、或$\sigma$环、或$\sigma$代数)$\widetilde{\mathscr{R}}$也满足(1),(2),则$\widetilde{\mathscr{R}} \subseteq \mathscr{R}$.因为$\mathscr{R}$满足(1),(2),故$\mathscr{R} \subseteq \widetilde{\mathscr{R}}$.因此,$\widetilde{\mathscr{R}} = \mathscr{R}$.这就证明了满足(1),(2)的环(或代数、或$\sigma$环、或$\sigma$代数)是惟一的.
\end{proof}

\begin{theorem}
$\mathscr{R}_\sigma(\mathscr{E}) = \mathscr{R}_\sigma(\mathscr{R}(\mathscr{E})).$
\end{theorem}
\begin{proof}
因为$\mathscr{R}_\sigma(\mathscr{E}) \supseteq \mathscr{E}$,所以$\mathscr{R}_\sigma(\mathscr{E}) \supseteq \mathscr{R}(\mathscr{E})$.由此推得$\mathscr{R}_\sigma(\mathscr{E}) \supseteq \mathscr{R}_\sigma(\mathscr{R}(\mathscr{E})).$

反之,由于$\mathscr{E} \subseteq \mathscr{R}(\mathscr{E})$,所以$\mathscr{R}_\sigma(\mathscr{E}) \subseteq \mathscr{R}_\sigma(\mathscr{R}(\mathscr{E})).$这就证明了$\mathscr{R}_\sigma(\mathscr{E}) = \mathscr{R}_\sigma(\mathscr{R}(\mathscr{E})).$

\end{proof}

\begin{example}
设$X$为一个非空集合,$\mathscr{E}$为$X$的单元素(独点)子集全体所成的集类.则
$\mathscr{R}(\mathscr{E})$就是$X$的有限子集(包括空集)全体所成的集类(见\rrefpro{proposition:集合论----环和代数的例子}{proposition:集合论----环和代数的例子-2}),它是一个环.
$\mathscr{R}_\sigma(\mathscr{E})$就是$X$的至多可数子集全体所成的集类(见\rrefpro{proposition:集合论----环和代数的例子}{proposition:集合论----环和代数的例子-3}),它是一个$\sigma$环.

如果$X$为有限集,则$\mathscr{R}(\mathscr{E})=\mathscr{A}(\mathscr{E})=\mathscr{R}_\sigma(\mathscr{E})=\mathscr{A}_\sigma(\mathscr{E})$,它是$X$的有限子集全体所成的集类,它是$\sigma$代数.

如果$X=\{a_n|n\in\mathbb{N}\}$为可数集,则$\mathscr{R}(\mathscr{E})$为$X$的有限子集全体所成的集类,这是一个环,不是代数,也不是$\sigma$环,更不是$\sigma$代数.而$\mathscr{R}_\sigma(\mathscr{E})=\mathscr{A}_\sigma(\mathscr{E})$是$X$的至多可数子集的全体所成的集类,它是$\sigma$环,是代数,是$\sigma$代数.易见,$X$的有限子集及其余集的全体所成的集类就是$\mathscr{A}(\mathscr{E})$,它是一个代数.但不是$\sigma$环$\left(\bigcup_{n=1}^{\infty}\{a_{2n}\}=\{a_{2n}|n\in\mathbb{N}\}\notin\mathscr{A}(\mathscr{E})\right)$,更不是$\sigma$代数.

如果$X$为不可数集,则$\mathscr{R}(\mathscr{E})$是$X$的有限子集的全体所成的集类,它是一个环,不是代数,不是$\sigma$环,更不是$\sigma$代数.$\mathscr{R}_\sigma(\mathscr{E})$是$X$的至多可数子集的全体所成的集类,它是$\sigma$环,但不是$\sigma$代数.$\mathscr{A}(\mathscr{E})$是$X$的有限子集及其余集的全体所成的集类.它是代数,但不是$\sigma$环(因为$X$的可数子集$\bigcup_{n=1}^{\infty}\{a_n\}=\{a_n|n\in\mathbb{N}\}\notin\mathscr{A}(\mathscr{E})$),更不是$\sigma$代数.$\mathscr{A}_\sigma(\mathscr{E})$是$X$的至多可数子集及其余集的全体(未必是$X$的所有子集形成的集类!例如:$X=\mathbb{R}$,则$(-\infty,0)\notin\mathscr{A}_\sigma(\mathscr{E})$)所成的集类.它是$\sigma$代数.
\end{example}

\begin{example}
设$\mathscr{P}$为$\mathbb{R}^1$上左开右闭区间$(a,b](-\infty<a<b<+\infty)$全体所成的集类,则$\mathscr{R}(\mathscr{P})=\mathscr{R}_0$(\refexa{example::--23r8u3r89uj2jr3}).
$\mathscr{A}(\mathscr{P})=\mathscr{R}(\mathscr{P}\cup\{\mathbb{R}^1\})$(有限个左开右闭区间的并及其余集所形成的集类).

显然,$\mathscr{R}_\sigma(\mathscr{P})=\mathscr{A}_\sigma(\mathscr{P})=\mathscr{R}_\sigma(\mathscr{R}_0)=\mathscr{A}_\sigma(\mathscr{R}_0)$(因为$\mathbb{R}^1=\bigcup_{i\in\mathbb{Z}}(i,i+1]$).注意:$(-\infty,0]=\bigcup_{n=1}^{\infty}(-n,-n+1]\in\mathscr{R}_\sigma(\mathscr{P})\setminus\mathscr{R}(\mathscr{P})$,所以$\mathscr{R}(\mathscr{P})\varsubsetneq\mathscr{R}_\sigma(\mathscr{P})$.
\end{example}

\begin{theorem}\label{theorem:集合论--定理1.3.5}
设$X$为非空集合,$\mathscr{R}$为$X$上的一个集类,则$\mathscr{R}$为$\sigma$代数的充要条件是同时满足以下条件
\begin{enumerate}[(1)]
\item $\varnothing \in \mathscr{R}$;
\item 若$E \in \mathscr{R}$,则$E^c \in \mathscr{R}$;
\item 若$E_i \in \mathscr{R},i=1,2,\cdots$,则$\bigcup_{i=1}^{\infty} E_i \in \mathscr{R}$.
\end{enumerate}
\end{theorem}
\begin{proof}
$(\Rightarrow)$因为$\mathscr{R}$为$\sigma$代数,故$\mathscr{R}$为非空集类,从而$\exists E \in \mathscr{R}$.由此得到$\varnothing = E \setminus E \in \mathscr{R}$.这就证明了(1).

因为$\mathscr{R}$为$\sigma$代数,故$X \in \mathscr{R}$.如果$E \in \mathscr{R}$,根据$\mathscr{R}$为环,所以$E^c = X \setminus E \in \mathscr{R}$.这就证明了(2).

(3)就是$\sigma$代数定义的第1条.

$(\Leftarrow)$从右边条件(1),(2)立知,$X = \varnothing^c \in \mathscr{R}$.右边条件(3)就是$\sigma$代数定义中的第1个条件.

如果$E_1,E_2 \in \mathscr{R}$,由右边条件(2)知,$E_1^c,E_2^c \in \mathscr{R}$.于是,由(1),(2),(3)得到
\begin{align*}
E_1 \setminus E_2 &= E_1 \cap E_2^c = \left( (E_1 \cap E_2^c)^c \right)^c = \left( E_1^c \cup E_2 \right)^c \\
&= \left( E_1^c \cup E_2 \cup \varnothing \cup \varnothing \cup \cdots \right)^c \in \mathscr{R}.
\end{align*}

综上所述,$\mathscr{R}$为$\sigma$代数.

\end{proof}

\begin{definition}[单调类]
设$\mathscr{M}$为由$X$的某些子集所成的集类、如果对$\mathscr{M}$中任何单调集列$\{E_n\}$,都必有$\lim\limits_{n \to +\infty} E_n \in \mathscr{M}$,则称$\mathscr{M}$为\textbf{单调类}.因此,单调类就是对单调集列的极限运算封闭的集类.
\end{definition}

\begin{example}
设$X=\mathbb{R}^1$,则$\mathscr{M}=\{[0,1],[2,3]\}$为单调类($\mathscr{M}$中任何单调类$\{E_n\}$,必有$n_0 \in \mathbb{N}$,当$n>n_0$时,有$E_n=[0,1]$.因此,$\lim\limits_{n \to +\infty} E_n=[0,1] \in \mathscr{M}$;或者,必有$n_0 \in \mathbb{N}$,当$n>n_0$时,有$E_n=[2,3]$.因此,$\lim\limits_{n \to +\infty} E_n=[2,3] \in \mathscr{M}$).但$\mathscr{M}$对“$\cup$”不封闭($[0,1] \cup [2,3] \notin \mathscr{M}$),故$\mathscr{M}$不为环.
\end{example}

\begin{theorem}\label{theorem:单调类的交仍是单调类}
设$\mathscr{M}_\alpha$为单调类,$\alpha \in \Gamma$,则$\bigcap_{\alpha \in \Gamma} \mathscr{M}_\alpha$也为单调类.
\end{theorem}
\begin{proof}
设$\{E_n\}$为$\bigcap_{\alpha \in \Gamma} \mathscr{M}_\alpha$中的单调集列,则它也是$\mathscr{M}_\alpha$中的单调集列.根据单调类的定义,$\lim\limits_{n \to +\infty} E_n \in \mathscr{M}_\alpha(\alpha \in \Gamma)$.所以,$\lim\limits_{n \to +\infty} E_n \in \bigcap_{\alpha \in \Gamma} \mathscr{M}_\alpha$.这就证明了$\bigcap_{\alpha \in \Gamma} \mathscr{M}_\alpha$也为单调类.

\end{proof}

\begin{theorem}
设$\mathscr{E}$是由集合$X$的某些子集所成的集类,则存在惟一的单调类$\mathscr{M}$,使得
\begin{enumerate}[(1)]
\item $\mathscr{E} \subseteq \mathscr{M}$;
\item 任何包含$\mathscr{E}$的单调类$\mathscr{M}'$,必有$\mathscr{M} \subseteq \mathscr{M}'$.
\end{enumerate}
换言之,$\mathscr{M}$是包含$\mathscr{E}$的最小单调类.这样的单调类$\mathscr{M}$称为\textbf{由集类$\mathscr{E}$所张成的单调类},记作$\mathscr{M}(\mathscr{E})$.
\end{theorem}
\begin{proof}
首先,$X$的子集的全体$2^X$是一个单调类,当然,$\mathscr{E} \subseteq 2^X$.因此,包含$\mathscr{E}$的单调类确实是存在的.取单调类族
\begin{align*}
\Gamma = \{\mathscr{M}' \mid \mathscr{E} \subseteq \mathscr{M}' \subseteq 2^X, \mathscr{M}' \text{为单调类}\}.
\end{align*}
根据\refthe{theorem:单调类的交仍是单调类},$\mathscr{M} = \bigcap_{\mathscr{M}' \in \Gamma} \mathscr{M}'$为单调类.显然,还有$\mathscr{E} \subseteq \mathscr{M}$,故$\mathscr{M}$满足(1).由$\mathscr{M}$的定义知,性质(2)成立.

如果单调类$\widetilde{\mathscr{M}}$也满足(1),(2),则$\widetilde{\mathscr{M}} \subseteq \mathscr{M}$.因为$\mathscr{M}$满足(1),(2),故$\mathscr{M} \subseteq \widetilde{\mathscr{M}}$.所以,$\widetilde{\mathscr{M}} = \mathscr{M}$.这就证明了满足(1),(2)的单调类是惟一的.

\end{proof}

\begin{theorem}\label{theorem:集合论--定理1.3.8}
设$\mathscr{M}$为集合$X$的集类.则$\mathscr{M}$为$\sigma$环$\iff  \mathscr{M}$为单调环(既是单调类又是环).
\end{theorem}
\begin{proof}
$(\Rightarrow)$由$\sigma$环定义知,$\sigma$环$\mathscr{M}$对“$\bigcup_{n=1}^{\infty}$”运算封闭.再由\rrefthe{theorem:sigma环或代数的基本性质}{theorem:sigma环或代数的基本性质-2},$\sigma$环$\mathscr{M}$对“$\bigcap_{n=1}^{\infty}$”运算也封闭.再根据$\mathscr{M}$的单调增(减)集列$\{E_n\}$的极限的定义知$\lim\limits_{n \to +\infty} E_n = \bigcup_{n=1}^{\infty} E_n$(或$\bigcap_{n=1}^{\infty} E_n) \in \mathscr{M}$.故$\mathscr{M}$为单调类,又因为$\mathscr{M}$为$\sigma$环,所以$\mathscr{M}$为单调环.

$(\Leftarrow)$设$\mathscr{M}$为单调环,即$\mathscr{M}$既是单调类又是环.要证$\mathscr{M}$为$\sigma$环,只须证$\mathscr{M}$对“$\bigcup_{n=1}^{\infty}$”运算封闭.事实上,对$\forall E_n \in \mathscr{M}(n=1,2,\cdots)$,由于$\mathscr{M}$为一个环,所以$\bigcup_{i=1}^{n} E_i \in \mathscr{M}$.而$\{\bigcup_{i=1}^{n} E_i \mid n \in \mathbb{N}\}$为单调增集列,因此
\begin{align*}
\bigcup_{n=1}^{\infty} E_n = \bigcup_{n=1}^{\infty} \left( \bigcup_{i=1}^{n} E_i \right) = \lim\limits_{n \to +\infty} \bigcup_{i=1}^{n} E_i \stackrel{\text{单调类定义}}{\in} \mathscr{M}.
\end{align*}

\end{proof}

\begin{theorem}\label{theorem:集合论--定理1.3.9}
设$\mathscr{E}$为集合$X$的某些子集所成的环,则$\mathscr{R}_\sigma(\mathscr{E}) = \mathscr{M}(\mathscr{E})$.
\end{theorem}
\begin{proof}
因为$\mathscr{R}_\sigma(\mathscr{E})$是包含$\mathscr{E}$的$\sigma$环,根据\refthe{theorem:集合论--定理1.3.8},它是单调类.但$\mathscr{M}(\mathscr{E})$是包含$\mathscr{E}$的最小单调类,所以$\mathscr{M}(\mathscr{E}) \subseteq \mathscr{R}_\sigma(\mathscr{E})$.

下面可以证明$\mathscr{M}(\mathscr{E})$为环.于是,$\mathscr{M}(\mathscr{E})$为一个单调环.根据\refthe{theorem:集合论--定理1.3.8},$\mathscr{M}(\mathscr{E})$为$\sigma$环.但$\mathscr{R}_\sigma(\mathscr{E})$是包含$\mathscr{E}$的最小$\sigma$环,因此$\mathscr{R}_\sigma(\mathscr{E}) \subseteq \mathscr{M}(\mathscr{E})$.这就证明了$\mathscr{R}_\sigma(\mathscr{E}) = \mathscr{M}(\mathscr{E})$.

现在来证明$\mathscr{M}(\mathscr{E})$为一个环.对$\forall E \subseteq X$,作集类
\begin{align*}
\mathscr{K}(E) = \{F \mid F \in \mathscr{M}(\mathscr{E}), \text{且} \, F \setminus E, E \setminus F, E \cup F \in \mathscr{M}(\mathscr{E})\}.
\end{align*}

先证$\mathscr{K}(E)$为单调类.事实上,设$\{F_n\}$为$\mathscr{K}(E)$中的任一单调集列.因为$F_n \setminus E, E \setminus F_n, E \cup F_n \in \mathscr{M}(\mathscr{E})$,且$\{F_n \setminus E\},\{E \setminus F_n\},\{E \cup F_n\}$都仍为单调集列.于是
\begin{align*}
\lim\limits_{n \to +\infty} F_n \setminus E &= \lim\limits_{n \to +\infty} (F_n \setminus E) \in \mathscr{M}(\mathscr{E}), \\
E \setminus \lim\limits_{n \to +\infty} F_n &= \lim\limits_{n \to +\infty} (E \setminus F_n) \in \mathscr{M}(\mathscr{E}), \\
E \cup \lim\limits_{n \to +\infty} F_n &= \lim\limits_{n \to +\infty} (E \cup F_n) \in \mathscr{M}(\mathscr{E}).
\end{align*}
由此与$\{F_n\}$为$\mathscr{M}(\mathscr{E})$中的单调集列知,$\lim\limits_{n \to +\infty} F_n \in \mathscr{M}(\mathscr{E})$, $\lim\limits_{n \to +\infty} F_n \in \mathscr{K}(E)$.这就证明了$\mathscr{K}(E)$为单调类.

特别,当$E \in \mathscr{E} \subseteq \mathscr{M}(\mathscr{E})$时,由于$\mathscr{E}$为环,故$\mathscr{E} \subseteq \mathscr{K}(E) \subseteq \mathscr{M}(\mathscr{E})$.又因为$\mathscr{K}(E)$为包含$\mathscr{E}$的单调类,从而$\mathscr{M}(\mathscr{E}) \subseteq \mathscr{K}(E)$.因此,$\mathscr{K}(E) = \mathscr{M}(\mathscr{E})$.这就表明:当$E \in \mathscr{E}$时,对$\forall F \in \mathscr{M}(\mathscr{E})$,总有$F \setminus E, E \setminus F, E \cup F \in \mathscr{M}(\mathscr{E})$.

对$\forall E \in \mathscr{M}(\mathscr{E})$,根据上述证明,当$F \in \mathscr{E}$时,$E \setminus F, F \setminus E, E \cup F \in \mathscr{M}(\mathscr{E})$,从而$\mathscr{E} \subseteq \mathscr{K}(E) \subseteq \mathscr{M}(\mathscr{E})$.但$\mathscr{K}(E)$为包含$\mathscr{E}$的单调类,所以,包含$\mathscr{E}$的最小单调类$\mathscr{M}(\mathscr{E}) \subseteq \mathscr{K}(E)$.由此得到$\mathscr{K}(E) = \mathscr{M}(\mathscr{E})$.

对$\forall E,F \in \mathscr{M}(\mathscr{E}) = \mathscr{K}(E)$,由$F \in \mathscr{K}(E)$知,$F \setminus E, E \setminus F, E \cup F \in \mathscr{M}(\mathscr{E})$.这就证明了$\mathscr{M}(\mathscr{E})$为环.

\end{proof}

\begin{corollary}
设$\mathscr{M},\mathscr{E}$为集合$X$上的两个集类.如果$\mathscr{M}$为单调类,$\mathscr{E}$为环,且$\mathscr{M} \supseteq \mathscr{E}$,则$\mathscr{M} \supseteq \mathscr{R}_\sigma(\mathscr{E})$.
\end{corollary}

\begin{proof}
因为$\mathscr{E}$为环,根据\refthe{theorem:集合论--定理1.3.9},$\mathscr{M}(\mathscr{E}) = \mathscr{R}_\sigma(\mathscr{E})$.再由$\mathscr{M}$为包含$\mathscr{E}$的单调类,而$\mathscr{M}(\mathscr{E})$为包含$\mathscr{E}$的最小单调类.从\refthe{theorem:集合论--定理1.3.7}知,$\mathscr{M} \supseteq \mathscr{M}(\mathscr{E}) = \mathscr{R}_\sigma(\mathscr{E})$.

\end{proof}

\begin{example}
设$\mathscr{E}$为集合$X$上的一个非空集类.证明:对$\forall F \in \mathscr{R}_\sigma(\mathscr{E})$,必$\exists E_i \in \mathscr{E}(i=1,2,\cdots)$,s.t. $F \subseteq \bigcup_{i=1}^{\infty} E_i$.
\end{example}

\begin{proof}
设$X$上的集类
\begin{align*}
\mathscr{R} = \{F \mid F \subseteq X, \exists E_i \in \mathscr{E}, \, i=1,2,\cdots,\text{s.t.} \, F \subseteq \bigcup_{i=1}^{\infty} E_i\},
\end{align*}
则$\mathscr{R}$为$X$上的一个$\sigma$环.事实上,对$\forall F_i \in \mathscr{R},i=1,2,\cdots$,必有$E_{ij} \in \mathscr{R},j=1,2,\cdots$,s.t. $F_i \subseteq \bigcup_{j=1}^{\infty} E_{ij}$.于是
\begin{align*}
F_1 \setminus F_2 \subseteq F_1 \subseteq \bigcup_{j=1}^{\infty} E_{1j}, \quad F_1 \setminus F_2 \in \mathscr{R}, \\
\bigcup_{i=1}^{\infty} F_i \subseteq \bigcup_{i=1}^{\infty} \left( \bigcup_{j=1}^{\infty} E_{ij} \right), \quad \bigcup_{i=1}^{\infty} F_i \in \mathscr{R}.
\end{align*}
因此,$\mathscr{R}$为$X$上包含$\mathscr{E}$的一个$\sigma$环.又因为$\mathscr{R}_\sigma(\mathscr{E})$为包含$\mathscr{E}$的最小$\sigma$环,故$\mathscr{R}_\sigma(\mathscr{E}) \subseteq \mathscr{R}$.从而,对$\forall F \in \mathscr{R}_\sigma(\mathscr{E})$,必$\exists E_i \in \mathscr{E},i=1,2,\cdots$,s.t. $F \subseteq \bigcup_{i=1}^{\infty} E_i$.

\end{proof}






\end{document}
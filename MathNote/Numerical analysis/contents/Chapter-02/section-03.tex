\documentclass[../../main.tex]{subfiles}
\graphicspath{{\subfix{../../image/}}} % 指定图片目录,后续可以直接使用图片文件名。

% 例如:
% \begin{figure}[H]
% \centering
% \includegraphics[scale=0.4]{图.png}
% \caption{}
% \label{figure:图}
% \end{figure}
% 注意:上述\label{}一定要放在\caption{}之后,否则引用图片序号会只会显示??.

\begin{document}

\section{牛顿插值多项式}

\subsection{牛顿均差插值多项式}

\begin{definition}[均差]
称 $f[x_0, x_k] = \frac{f(x_k) - f(x_0)}{x_k - x_0}$ 为函数 $f(x)$ 关于点 $x_0, x_k$ 的\textbf{一阶均差}。$f[x_0, x_1, x_k] = \frac{f[x_0, x_k] - f[x_0, x_1]}{x_k - x_1}$ 称为 $f(x)$ 的\textbf{二阶均差}。一般地,称 
\begin{align}
f[x_0, x_1, \cdots, x_k] = \frac{f[x_0, \cdots, x_{k - 2}, x_k] - f[x_0, x_1, \cdots, x_{k - 1}]}{x_k - x_{k - 1}} \label{eq:数值分析-3.3}
\end{align}
为 $f(x)$ 的 $\boldsymbol{k}$ \textbf{阶均差}(均差也称为\textbf{差商})。
\end{definition}

\begin{theorem}[均差的基本性质]\label{theorem:均差的基本性质}
\begin{enumerate}[(1)]
\item $k$ 阶均差可表示为函数值 $f(x_0), f(x_1), \cdots, f(x_k)$ 的线性组合,即 
\begin{align}
f[x_0, x_1, \cdots, x_k] = \sum_{j = 0}^k \frac{f(x_j)}{(x_j - x_0) \cdots (x_j - x_{j - 1})(x_j - x_{j + 1}) \cdots (x_j - x_k)}. \label{eq:数值分析-3.4}
\end{align}
这个性质也表明均差与节点的排列次序无关,称为\textbf{均差的对称性},即 
\[
f[x_0, x_1, \cdots, x_k] = f[x_1, x_0, x_2, \cdots, x_k] = \cdots = f[x_1, \cdots, x_k, x_0]
\]

\item \begin{align}
f[x_0, x_1, \cdots, x_k] = \frac{f[x_1, x_2, \cdots, x_k] - f[x_0, x_1, \cdots, x_{k - 1}]}{x_k - x_0}. \label{eq:数值分析-3.3'}
\end{align}

\item 若 $f(x)$ 在 $[a, b]$ 上存在 $n$ 阶导数,且节点 $x_0, x_1, \cdots, x_n \in [a, b]$,则 $n$ 阶均差与导数的关系为 
\begin{align}
f[x_0, x_1, \cdots, x_n] = \frac{f^{(n)}(\xi)}{n!}, \quad \xi \in [a, b].\label{eq:数值分析-3.5}
\end{align}
\end{enumerate}
\end{theorem}
\begin{proof}
\begin{enumerate}[(1)]
\item 利用数学归纳法证明即可。

\item 由性质(1)及 \eqref{eq:数值分析-3.3} 式立得.

\item 反复使用Rolle定理证明即可。
\end{enumerate}

\end{proof}
\begin{table}[H]
\centering
\caption{均差表}
\label{table:均差表}
\begin{tabular}{c|c|c|c|c|c}
\hline
$x_k$ & $f(x_k)$ & 一阶均差 & 二阶均差 & 三阶均差 & 四阶均差 \\
\hline
$x_0$ & $\underline{f(x_0)}$ &  &  &  &  \\
$x_1$ & $f(x_1)$ & $\underline{f[x_0, x_1]}$ &  &  &  \\
$x_2$ & $f(x_2)$ & $f[x_1, x_2]$ & $\underline{f[x_0, x_1, x_2]}$ &  &  \\
$x_3$ & $f(x_3)$ & $f[x_2, x_3]$ & $f[x_1, x_2, x_3]$ & $\underline{f[x_0, x_1, x_2, x_3]}$ &  \\
$x_4$ & $f(x_4)$ & $f[x_3, x_4]$ & $f[x_2, x_3, x_4]$ & $f[x_1, x_2, x_3, x_4]$ & $\underline{f[x_0, x_1, x_2, x_3, x_4]}$ \\
$\vdots$ & $\vdots$ & $\vdots$ & $\vdots$ & $\vdots$ & $\vdots$ \\
\hline
\end{tabular}
\end{table}

\begin{theorem}[牛顿均差插值多项式]\label{theorem:牛顿均差插值多项式}
若已知 $f$ 在插值点 $x_i$ ($i = 0, 1, \cdots, n$) 上的值为 $f(x_i)$ ($i = 0, 1, \cdots, n$),记$f$的 $n$ 次插值多项式为$P_n(x)$,且满足条件
\begin{align}
P_n(x_i) = f(x_i), \quad i = 0, 1, \cdots, n. \label{eq:数值分析-3.1}
\end{align}
则一次插值多项式可表示为
\begin{align*}
P_1(x) = f(x_0) + f[x_0, x_1](x - x_0),
\end{align*}
二次插值多项式可表示为
\begin{align*}
P_2(x) = f(x_0) + f[x_0, x_1](x - x_0) + f[x_0, x_1, x_2](x - x_0)(x - x_1).
\end{align*}
将 $x$ 看成 $[a, b]$ 上一点,可得
\begin{align*}
f(x) &= f(x_0) + f[x_0, x_1](x - x_0) + f[x_0, x_1, x_2](x - x_0)(x - x_1) + \cdots \\
&\quad + f[x_0, x_1, \cdots, x_n](x - x_0) \cdots (x - x_{n-1}) \\
&\quad + f[x, x_0, \cdots, x_n]\omega_{n+1}(x) = P_n(x) + R_n(x),
\end{align*}
其中$n$次插值多项式和余项分别为
\begin{align}
P_n(x) &= f(x_0) + f[x_0, x_1](x - x_0) + f[x_0, x_1, x_2](x - x_0)(x - x_1) + \cdots \nonumber  \\
&\quad + f[x_0, x_1, \cdots, x_n](x - x_0) \cdots (x - x_{n-1}),\label{eq:数值分析-3.6} \\
R_n(x) &= f(x) - P_n(x) = f[x, x_0, \cdots, x_n]\omega_{n+1}(x),\label{eq:数值分析-3.7}
\end{align}
其中 $\omega_{n+1}(x)$ 由\eqref{eq:数值分析-2.10} 式定义.我们称 $P_n(x)$ 为\textbf{牛顿均差插值多项式}. 系数就是\hyperref[table:均差表]{均差表}中加横线的各阶均差.
\end{theorem}
\begin{remark}
\eqref{eq:数值分析-3.7} 式为插值余项,由插值多项式唯一性知,它与\eqref{eq:数值分析-2.12}式是等价的,事实上,利用均差与导数关系式 \eqref{eq:数值分析-3.5} 可由 \eqref{eq:数值分析-3.7} 式推出\eqref{eq:数值分析-2.12}式. 但 \eqref{eq:数值分析-3.7} 式更有一般性,它对 $f$ 是由离散点给出的情形或 $f$ 导数不存在时均适用.
\end{remark}
\begin{remark}
牛顿插值比拉格朗日插值计算量省,且便于程序设计.
\end{remark}
\begin{proof}
借助均差的定义,一次插值多项式可表示为
\begin{align*}
P_1(x) &= P_0(x) + f[x_0, x_1](x - x_0) = f(x_0) + f[x_0, x_1](x - x_0),
\end{align*}
而二次插值多项式可表示为
\begin{align*}
P_2(x) &= P_1(x) + f[x_0, x_1, x_2](x - x_0)(x - x_1) \\
&= f(x_0) + f[x_0, x_1](x - x_0) + f[x_0, x_1, x_2](x - x_0)(x - x_1).
\end{align*}
实际上,根据均差定义,将 $x$ 看成 $[a, b]$ 上一点,可得
\begin{align*}
f(x) &= f(x_0) + f[x, x_0](x - x_0), \\
f[x, x_0] &= f[x_0, x_1] + f[x, x_0, x_1](x - x_1), \\
&\vdots \\
f[x, x_0, \cdots, x_{n-1}] &= f[x_0, x_1, \cdots, x_n] + f[x, x_0, \cdots, x_n](x - x_n).
\end{align*}
只要把后一式依次代入前一式,就得到
\begin{align*}
f(x) &= f(x_0) + f[x_0, x_1](x - x_0) + f[x_0, x_1, x_2](x - x_0)(x - x_1) + \cdots \\
&\quad + f[x_0, x_1, \cdots, x_n](x - x_0) \cdots (x - x_{n-1}) \\
&\quad + f[x, x_0, \cdots, x_n]\omega_{n+1}(x) = P_n(x) + R_n(x),
\end{align*}
其中
\begin{align}
P_n(x) &= f(x_0) + f[x_0, x_1](x - x_0) + f[x_0, x_1, x_2](x - x_0)(x - x_1) + \cdots \nonumber \\
&\quad + f[x_0, x_1, \cdots, x_n](x - x_0) \cdots (x - x_{n-1}),\label{eq:数值分析-3..6} \\
R_n(x) &= f(x) - P_n(x) = f[x, x_0, \cdots, x_n]\omega_{n+1}(x), \nonumber 
\end{align}
其中 $\omega_{n+1}(x)$ 由\eqref{eq:数值分析-2.10} 式定义.

由 \eqref{eq:数值分析-3..6} 式确定的多项式 $P_n(x)$ 显然满足插值条件 \eqref{eq:数值分析-3.1},且次数不超过 $n$,若记
\begin{align*}
P_n(x)=a_0+a_1(x-x_0)+a_2(x-x_0)(x-x_1)+\cdots+a_n(x-x_0)\cdots(x-x_{n-1}),
\end{align*}
则其系数为
\begin{align*}
a_k = f[x_0, x_1, \cdots, x_k], \quad k = 0, 1, \cdots, n.
\end{align*}

\end{proof}

\begin{example}
给出 $f(x)$ 的函数表(见表 2-2),求 4 次牛顿插值多项式,并由此计算 $f(0.596)$ 的近似值.
\end{example}
\begin{solution}
首先根据给定函数表造出均差表.
\begin{table}[H]
\centering
\caption{函数及均差表}
\label{table:例题1均差表}
\begin{tabular}{c|c|c|c|c|c|c}
\hline
0.40  & \underline{0.410 75} &          &          &          &          &          \\
0.55  & 0.578 15          & \underline{1.116 00} &          &          &          &          \\
0.65  & 0.696 75          & 1.186 00          & \underline{0.280 00} &          &          &          \\
0.80  & 0.888 11          & 1.275 73          & 0.358 93          & \underline{0.197 33} &          &          \\
0.90  & 1.026 52          & 1.384 10          & 0.433 48          & 0.213 00          & \underline{0.031 34} &          \\
1.05  & 1.253 82          & 1.515 33          & 0.524 93          & 0.228 63          & 0.031 26          & $-0.000 12$ \\
\hline
\end{tabular}
\end{table}
从均差表看到 4 阶均差近似常数,故取 4 次插值多项式 $P_4(x)$ 做近似即可.
\begin{align*}
P_4(x) &= 0.41075 + 1.116(x - 0.4) + 0.28(x - 0.4)(x - 0.55) \\
&\quad + 0.19733(x - 0.4)(x - 0.55)(x - 0.65) \\
&\quad + 0.03134(x - 0.4)(x - 0.55)(x - 0.65)(x - 0.8),
\end{align*}
于是
\[
f(0.596) \approx P_4(0.596) = 0.63192,
\]
截断误差
\[
|R_4(x)| \approx |f[x_0, x_1, \cdots, x_5]\omega_5(0.596)| \leqslant 3.63 \times 10^{-9}.
\]
这说明截断误差很小,可忽略不计.

此例的截断误差估计中,5 阶均差 $f[x, x_0, \cdots, x_4]$ 用 $f[x_0, x_1, \cdots, x_5] = -0.00012$ 近似. 另一种方法是取 $x = 0.596$,由 $f(0.596) \approx 0.63192$,可求得 $f[x, x_0, \cdots, x_4]$ 的近似值,再根据均差的定义,可求得 $|R_4(x)|$ 的近似.

\end{solution}


\subsection{差分形式的牛顿插值多项式}

\begin{definition}\label{definition:差分定义}
$n$个节点$x_0,x_1,\cdots,x_n$称为\textbf{等距节点},即$x_k = x_0 + kh$ ($k = 0, 1, \cdots, n$),这里的 $h$ 称为\textbf{步长}.

设 $x_k$ 点的函数值为 $f_k = f(x_k)$ ($k = 0, 1, \cdots, n$),称 $\Delta f_k = f_{k+1} - f_k$ 为 $x_k$ 处以 $h$ 为步长的\textbf{一阶(向前)差分}. 类似地称 $\Delta^2 f_k = \Delta f_{k+1} - \Delta f_k$ 为 $x_k$ 处的\textbf{二阶差分}. 一般地,称
\begin{align}
\Delta^n f_k = \Delta^{n-1} f_{k+1} - \Delta^{n-1} f_k \label{eq:数值分析-3.8}
\end{align}
为 $x_k$ 处的 $\boldsymbol{n}$ \textbf{阶差分}. 为了表示方便,再引入两个常用算子符号:
\[
\mathrm{I}f_k = f_k, \quad \mathrm{E}f_k = f_{k+1},
\]
$\mathrm{I}$ 称为\textbf{不变算子},$\mathrm{E}$ 称为步长为 $h$ 的\textbf{位移算子}.
\end{definition}

由给定函数表计算各阶差分可由以下形式差分表给出.
\begin{table}[H]
\scriptsize
\centering
\caption{}
\label{table:差分表}
\begin{tikzcd}
{f_0} \\
&& {\Delta f_0} \\
{f_1} &&&& {\Delta^2 f_0} \\
&& {\Delta f_1} &&&& {\Delta^3 f_0} \\
{f_2} &&&& {\Delta^2 f_1} \\
&& {\Delta f_2} \\
{f_3}
\arrow[from=1-1, to=2-3]
\arrow[from=2-3, to=3-5]
\arrow[from=3-1, to=2-3]
\arrow[from=3-1, to=4-3]
\arrow[from=3-5, to=4-7]
\arrow[from=4-3, to=3-5]
\arrow[from=4-3, to=5-5]
\arrow[from=5-1, to=4-3]
\arrow[from=5-1, to=6-3]
\arrow[from=5-5, to=4-7]
\arrow[from=6-3, to=5-5]
\arrow[from=7-1, to=6-3]
\end{tikzcd}
\end{table}

\begin{theorem}\label{theorem:差分的基本性质}
设$n$个节点$x_0,x_1,\cdots,x_n$,步长为$h$,$x_k$ 点的函数值为 $f_k = f(x_k)$ ($k = 0, 1, \cdots, n$),则
\[
\Delta f_k = f_{k+1} - f_k = \mathrm{E}f_k - \mathrm{I}f_k = (\mathrm{E} - \mathrm{I})f_k,
\]
\begin{align}
\Delta^n f_k = (\mathrm{E} - \mathrm{I})^n f_k = \sum_{j=0}^n (-1)^j \binom{n}{j} \mathrm{E}^{n-j} f_k = \sum_{j=0}^n (-1)^j \binom{n}{j} f_{n+k-j}, \label{eq:数值分析-3.9}
\end{align}
\eqref{eq:数值分析-3.9} 式表示各阶差分均可用函数值给出.

反之也可用各阶差分表示函数值如下
\begin{align}
f_{n+k} = \sum_{j=0}^n \binom{n}{j} \Delta^j f_k. \label{eq:数值分析-3.10}
\end{align}
还可导出均差与差分的关系:
\begin{align}
f[x_k, \cdots, x_{k+m}] = \frac{1}{m!} \frac{1}{h^m} \Delta^m f_k, \quad m = 1, 2, \cdots, n. \label{eq:数值分析-3.11}
\end{align}
还可得到差分与导数的关系:
\begin{align}
\Delta^n f_k = h^n f^{(n)}(\xi), \quad \text{其中 } \xi \in (x_k, x_{k+n}). \label{eq:数值分析-3.12}
\end{align}
\end{theorem}
\begin{proof}
由\refdef{definition:差分定义}可推出:
\[
\Delta f_k = f_{k+1} - f_k = \mathrm{E}f_k - \mathrm{I}f_k = (\mathrm{E} - \mathrm{I})f_k,
\]
\begin{align*}
\Delta^n f_k = (\mathrm{E} - \mathrm{I})^n f_k = \sum_{j=0}^n (-1)^j \binom{n}{j} \mathrm{E}^{n-j} f_k = \sum_{j=0}^n (-1)^j \binom{n}{j} f_{n+k-j},
\end{align*}
其中 $\binom{n}{j} = \frac{n(n-1)\cdots(n-j+1)}{j!}$ 为二项式展开系数.

由\refdef{definition:差分定义}可推出:
\begin{gather*}
\Delta f_k=f_{k+1}-f_k=f_{k+1}-\mathrm{I}f_k\Longrightarrow f_{k+1}=\mathrm{E}f_k=(\mathrm{I}+\Delta )f_k,
\\
f_{n+k}=\mathrm{E}^nf_k=(\mathrm{I}+\Delta )^nf_k=\left[ \sum_{j=0}^n{\left( \begin{array}{c}
n\\
j\\
\end{array} \right) \Delta ^j} \right] f_k,
\end{gather*}
于是
\begin{align*}
f_{n+k} = \sum_{j=0}^n \binom{n}{j} \Delta^j f_k.
\end{align*}
根据均差的定义就可导出均差与差分的关系:
\[
f[x_k, x_{k+1}] = \frac{f_{k+1} - f_k}{x_{k+1} - x_k} = \frac{\Delta f_k}{h},
\]
\[
f[x_k, x_{k+1}, x_{k+2}] = \frac{f[x_{k+1}, x_{k+2}] - f[x_k, x_{k+1}]}{x_{k+2} - x_k} = \frac{1}{2h^2} \Delta^2 f_k.
\]
一般地,有
\begin{align*}
f[x_k, \cdots, x_{k+m}] = \frac{1}{m!} \frac{1}{h^m} \Delta^m f_k, \quad m = 1, 2, \cdots, n. 
\end{align*}
由 \eqref{eq:数值分析-3.11} 式及\eqref{eq:数值分析-3.5}式就可得到差分与导数的关系:
\begin{align*}
\Delta^n f_k = h^n f^{(n)}(\xi), \quad \text{其中 } \xi \in (x_k, x_{k+n}).
\end{align*}

\end{proof}

\begin{theorem}
若已知$n+1$阶可导函数$f$ 在步长为$h$的等距插值点 $x_i$ ($i = 0, 1, \cdots, n$) 上的值为 $f(x_i)$ ($i = 0, 1, \cdots, n$),记 $n$ 次插值多项式 $P_n(x)$ 满足条件
\begin{align*}
P_n(x_i) = f(x_i), \quad i = 0, 1, \cdots, n. 
\end{align*}
令$x=x_0+th$,则插值多项式可表示为
\begin{align}
P_n(x_0 + th) &= f_0 + t\Delta f_0 + \frac{t(t - 1)}{2!} \Delta^2 f_0 + \cdots + \frac{t(t - 1) \cdots (t - n + 1)}{n!} \Delta^n f_0, \label{eq:数值分析-3.13}
\end{align}
\eqref{eq:数值分析-3.13}式称为\textbf{牛顿前插公式}.其余项为
\begin{align}
R_n(x) = \frac{t(t - 1) \cdots (t - n)}{(n + 1)!} h^{n+1} f^{(n+1)}(\xi), \quad \xi \in (x_0, x_n). \label{eq:数值分析-3.14}
\end{align}
\end{theorem}
\begin{proof}
在牛顿插值公式 \eqref{eq:数值分析-3.6} 中,用 \eqref{eq:数值分析-3.11} 式的差分代替均差,并令 $x = x_0 + th$,则得
\begin{align*}
P_n(x_0 + th) &= f_0 + t\Delta f_0 + \frac{t(t - 1)}{2!} \Delta^2 f_0 + \cdots + \frac{t(t - 1) \cdots (t - n + 1)}{n!} \Delta^n f_0,
\end{align*}
由 \eqref{eq:数值分析-3.7} 式和\eqref{eq:数值分析-3.5}式得其余项为
\begin{align*}
R_n(x) = \frac{t(t - 1) \cdots (t - n)}{(n + 1)!} h^{n+1} f^{(n+1)}(\xi), \quad \xi \in (x_0, x_n). 
\end{align*}

\end{proof}

\begin{example}
给出 $f(x) = \cos x$ 在 $x_k = kh$, $k = 0, 1, \cdots, 5$, $h = 0.1$ 处的函数值,试用 4 次牛顿前插公式计算 $f(0.048)$ 的近似值并估计误差.
\end{example}
\begin{solution}
先构造差分表(见\reftab{table:差分表})并用牛顿前插公式 \eqref{eq:数值分析-3.13} 求 $f(0.048)$ 的近似值.
\begin{table}[H]
\centering
\caption{差分表}
\label{tab:difference-table12312344343452345634456}
\begin{tabular}{c c c c c c c}
\toprule
$x_k$ & $f(x_k)$ & $\Delta f$ & $\Delta^2 f$ & $\Delta^3 f$ & $\Delta^4 f$ & $\Delta^5 f$ \\
\midrule
0.00 & 1.00000 &  &  &  &  &  \\
&  & $-0.00500$ &  &  &  &  \\
0.10 & 0.99500 &  & $-0.00993$ &  &  &  \\
&  & $-0.01493$ &  & $0.00013$ &  &  \\
0.20 & 0.98007 &  & $-0.00980$ &  & $0.00012$ &  \\
&  & $-0.02473$ &  & $0.00025$ &  & $-0.00002$ \\
0.30 & 0.95534 &  & $-0.00955$ &  & $0.00010$ &  \\
&  & $-0.03428$ &  & $0.00035$ &  &  \\
0.40 & 0.92106 &  & $-0.00920$ &  &  &  \\
&  & $-0.04348$ &  &  &  &  \\
0.50 & 0.87758 &  &  &  &  &  \\
\bottomrule
\end{tabular}
\end{table}
取 $x = 0.048$, $h = 0.1$, $t = \frac{x - 0}{h} = 0.48$, 得
\begin{align*}
P_4(0.048) &= 1.00000 + 0.48 \times (-0.00500) + \frac{(0.48)(0.48 - 1)}{2}(-0.00993) \\
&\quad + \frac{1}{3!}(0.48)(0.48 - 1)(0.48 - 2)(0.00013) \\
&\quad + \frac{1}{4!}(0.48)(0.48 - 1)(0.48 - 2)(0.48 - 3)(0.00012) \\
&= 0.99885 \approx \cos 0.048,
\end{align*}
由 \eqref{eq:数值分析-3.14} 式可得误差估计为
\[
|R_4(0.048)| \leqslant \frac{M_5}{5!} | t(t - 1)(t - 2)(t - 3)(t - 4) | h^5 \leqslant 1.5845 \times 10^{-7},
\]
其中 $M_5 = |\sin 0.6| \leqslant 0.565$.

\end{solution}






















































\end{document}
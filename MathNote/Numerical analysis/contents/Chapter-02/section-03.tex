\documentclass[../../main.tex]{subfiles}
\graphicspath{{\subfix{../../image/}}} % 指定图片目录,后续可以直接使用图片文件名。

% 例如:
% \begin{figure}[H]
% \centering
% \includegraphics[scale=0.4]{图.png}
% \caption{}
% \label{figure:图}
% \end{figure}
% 注意:上述\label{}一定要放在\caption{}之后,否则引用图片序号会只会显示??.

\begin{document}

\section{均差与牛顿插值多项式}

\begin{definition}[均差]
称 $f[x_0, x_k] = \frac{f(x_k) - f(x_0)}{x_k - x_0}$ 为函数 $f(x)$ 关于点 $x_0, x_k$ 的\textbf{一阶均差}。$f[x_0, x_1, x_k] = \frac{f[x_0, x_k] - f[x_0, x_1]}{x_k - x_1}$ 称为 $f(x)$ 的\textbf{二阶均差}。一般地,称 
\begin{align}
f[x_0, x_1, \cdots, x_k] = \frac{f[x_0, \cdots, x_{k - 2}, x_k] - f[x_0, x_1, \cdots, x_{k - 1}]}{x_k - x_{k - 1}} \label{eq:数值分析-3.3}
\end{align}
为 $f(x)$ 的 $\boldsymbol{k}$ \textbf{阶均差}(均差也称为\textbf{差商})。
\end{definition}

\begin{theorem}[均差的基本性质]\label{theorem:均差的基本性质}
\begin{enumerate}[(1)]
\item $k$ 阶均差可表示为函数值 $f(x_0), f(x_1), \cdots, f(x_k)$ 的线性组合,即 
\begin{align}
f[x_0, x_1, \cdots, x_k] = \sum_{j = 0}^k \frac{f(x_j)}{(x_j - x_0) \cdots (x_j - x_{j - 1})(x_j - x_{j + 1}) \cdots (x_j - x_k)}. \label{eq:数值分析-3.4}
\end{align}
这个性质也表明均差与节点的排列次序无关,称为\textbf{均差的对称性},即 
\[
f[x_0, x_1, \cdots, x_k] = f[x_1, x_0, x_2, \cdots, x_k] = \cdots = f[x_1, \cdots, x_k, x_0]
\]

\item \begin{align}
f[x_0, x_1, \cdots, x_k] = \frac{f[x_1, x_2, \cdots, x_k] - f[x_0, x_1, \cdots, x_{k - 1}]}{x_k - x_0}. \label{eq:数值分析-3.3'}
\end{align}

\item 若 $f(x)$ 在 $[a, b]$ 上存在 $n$ 阶导数,且节点 $x_0, x_1, \cdots, x_n \in [a, b]$,则 $n$ 阶均差与导数的关系为 
\[
f[x_0, x_1, \cdots, x_n] = \frac{f^{(n)}(\xi)}{n!}, \quad \xi \in [a, b].
\]
\end{enumerate}
\end{theorem}
\begin{proof}
\begin{enumerate}[(1)]
\item 利用数学归纳法证明即可。

\item 由性质(1)及 \eqref{eq:数值分析-3.3} 式立得.

\item 反复使用Rolle定理证明即可。
\end{enumerate}
\end{proof}








\end{document}
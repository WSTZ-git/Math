\documentclass[../../main.tex]{subfiles}
\graphicspath{{\subfix{../../image/}}} % 指定图片目录,后续可以直接使用图片文件名。

% 例如:
% \begin{figure}[H]
% \centering
% \includegraphics[scale=0.4]{图.png}
% \caption{}
% \label{figure:图}
% \end{figure}
% 注意:上述\label{}一定要放在\caption{}之后,否则引用图片序号会只会显示??.

\begin{document}

\section{多项式插值}

\begin{definition}
设函数 $y = f(x)$ 在区间 $[a, b]$ 上有定义,且已知在点 $a \leqslant x_0 < x_1 < \cdots < x_n \leqslant b$ 上的值 $y_0, y_1, \cdots, y_n$,若存在一简单函数 $P(x)$,使 
\[
P(x_i) = y_i, \quad i = 0, 1, \cdots, n .
\]
成立,就称 $P(x)$ 为 $f(x)$ 的\textbf{插值函数},点 $x_0, x_1, \cdots, x_n$ 称为\textbf{插值节点},包含插值节点的区间 $[a, b]$ 称为\textbf{插值区间},求插值函数 $P(x)$ 的方法称为\textbf{插值法}。若 $P(x)$ 是次数不超过 $n$ 的代数多项式,即 
\[
P(x) = a_0 + a_1 x + \cdots + a_n x^n .
\]
其中 $a_i$ 为实数,就称 $P(x)$ 为\textbf{插值多项式},相应的插值法称为\textbf{多项式插值}。若 $P(x)$ 为分段的多项式,就称为\textbf{分段插值}。若 $P(x)$ 为三角多项式,就称为\textbf{三角插值}.
\end{definition}

\begin{theorem}
设在区间 $[a, b]$ 上给定 $n + 1$ 个点 
\[
a \leqslant x_0 < x_1 < \cdots < x_n \leqslant b
\]
上的函数值 $y_i = f(x_i) (i = 0, 1, \cdots, n)$,求次数不超过 $n$ 的多项式$P(x)$,使 
\begin{align}
P(x_i) = y_i, \quad i = 0, 1, \cdots, n.\label{eq:::---993882}
\end{align}
证明:满足上述条件的插值多项式 $P(x)$ 是存在唯一的。
\end{theorem}
\begin{remark}
显然直接求解方程组\eqref{eq:::::----8791242---979273}就可以得到插值多项式$P(x)$,但这是求插值多项式最繁杂的方法,一般是不用的.
\end{remark}
\begin{proof}
由\eqref{eq:::---993882}式可得到关于系数 $a_0, a_1, \cdots, a_n$ 的 $n + 1$ 元线性方程组
\begin{align}
\begin{cases} 
a_0 + a_1 x_0 + \cdots + a_n x_0^n = y_0, \\ 
a_0 + a_1 x_1 + \cdots + a_n x_1^n = y_1, \\ 
\vdots \\ 
a_0 + a_1 x_n + \cdots + a_n x_n^n = y_n 
\end{cases} \label{eq:::::----8791242---979273}
\end{align}
此方程组的系数矩阵为 
\[
\boldsymbol{A} = \begin{pmatrix} 
1 & x_0 & \cdots & x_0^n \\ 
1 & x_1 & \cdots & x_1^n \\ 
\vdots & \vdots & & \vdots \\ 
1 & x_n & \cdots & x_n^n 
\end{pmatrix}
\]
称为范德蒙德(Vandermonde)矩阵,由于 $x_i (i = 0, 1, \cdots, n)$ 互异,故 
\[
\det \boldsymbol{A} = \prod_{\substack{0 \leqslant j < i \leqslant n}} (x_i - x_j) \neq 0
\]
因此,线性方程组 \eqref{eq:::::----8791242---979273} 的解 $a_0, a_1, \cdots, a_n$ 存在且唯一,于是定理得证。
\end{proof}
















\end{document}
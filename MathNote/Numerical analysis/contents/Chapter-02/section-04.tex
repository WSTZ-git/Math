\documentclass[../../main.tex]{subfiles}
\graphicspath{{\subfix{../../image/}}} % 指定图片目录,后续可以直接使用图片文件名。

% 例如:
% \begin{figure}[H]
% \centering
% \includegraphics[scale=0.4]{图.png}
% \caption{}
% \label{figure:图}
% \end{figure}
% 注意:上述\label{}一定要放在\caption{}之后,否则引用图片序号会只会显示??.

\begin{document}

\section{Hermite(埃尔米特)插值}

\subsection{重节点均差与Taylor(泰勒)插值}

\begin{theorem}
设 $f \in C^n[a, b]$, $x_0, x_1, \cdots, x_n$ 为 $[a, b]$ 上的相异节点,则 $f[x_0, x_1, \cdots, x_n]$ 是其变量的连续函数.
\end{theorem}

\begin{definition}[重节点均差]
如果 $[a, b]$ 上的节点$x_0,x_1,\cdots,x_n$互异,根据均差定义,若 $f \in C^1[a, b]$,则有
\[
\lim_{x \to x_0} f[x_0, x] = \lim_{x \to x_0} \frac{f(x) - f(x_0)}{x - x_0} = f'(x_0).
\]
由此定义\textbf{重节点均差}
\[
f[x_0, x_0] = \lim_{x \to x_0} f[x_0, x] = f'(x_0).
\]
类似地可定义\textbf{重节点的二阶均差},当 $x_1 \neq x_0$ 时,有
\[
f[x_0, x_0, x_1] = \frac{f[x_0, x_1] - f[x_0, x_0]}{x_1 - x_0}.
\]
当 $x_1 \to x_0$ 时,有
\[
f[x_0, x_0, x_0] = \lim_{\substack{x_1 \to x_0 \\ x_2 \to x_0}} f[x_0, x_1, x_2] = \frac{1}{2} f''(x_0).
\]
一般地,可定义 $\boldsymbol{n}$ \textbf{阶重节点的均差},由\eqref{eq:数值分析-3.5}式则得
\begin{align}
f[x_0, x_0, \cdots, x_0] = \lim_{\substack{x_1 \to x_0 \\ \vdots \\ x_n \to x_0}} f[x_0, x_1, \cdots, x_n] = \frac{1}{n!} f^{(n)}(x_0). \label{eq:数值分析-4.1}
\end{align}
\end{definition}

\begin{theorem}
设 $f(x)$ 在 $[a, b]$ 上存在 $n$ 阶连续导数,且 $(a, b)$ 上存在 $n + 1$ 阶导数,$x_0$ 为 $[a, b]$ 内一定点,
则对于任意的 $x \in [a, b]$,在 $x, x_0$ 之间存在一个数 $\xi$ 使得
\begin{align*}
f(x) = P_n(x)+R_n(x),
\end{align*}
其中
\begin{gather}
P_n(x) = f(x_0) + f'(x_0)(x - x_0) + \cdots + \frac{f^{(n)}(x_0)}{n!}(x - x_0)^n,\label{eq:数值分析-4.2}
\\
R_n(x) = \frac{f^{(n+1)}(\xi)}{(n+1)!}(x - x_0)^{n+1}, \quad \xi \in (a, b).\label{eq:数值分析-4.4}
\end{gather}
称 \eqref{eq:数值分析-4.2} 式为$\mathbf{Taylor}$\textbf{(泰勒)插值多项式},它就是一个$\mathbf{Hermite}$\textbf{(埃尔米特)插值多项式}.
\end{theorem}
\begin{remark}
实际上,上述Taylor插值多项式和余项之和就是\hyperref[Basis of Analytics-theorem:带各种余项的Taylor公式]{$f$在$x_0$点带Lagrange余项的Taylor展开式}.
\end{remark}
\begin{remark}
$P_n(x)$实际上是在点 $x_0$ 附近逼近 $f(x)$ 的一个带导数的插值多项式,它满足条件
\begin{align}
P_n^{(k)}(x_0) = f^{(k)}(x_0), \quad k = 0, 1, \cdots, n. \label{eq:数值分析-4.3}
\end{align}
实际上Taylor(泰勒)插值是牛顿插值的极限形式,是只在一点 $x_0$ 处给出 $n + 1$ 个插值条件 \eqref{eq:数值分析-4.3} 得到的 $n$ 次埃尔米特插值多项式.

一般地只要给出 $m + 1$ 个插值条件(含函数值和导数值)就可造出次数不超过 $m$ 次的埃尔米特插值多项式,由于导数条件各不相同,这里就不给出一般的埃尔米特插值公式.
\end{remark}
\begin{proof}
任取$x_0$邻域中$n+1$个互异点$x_0,x_1,\cdots,x_n$作为插值点,根据\refthe{theorem:牛顿均差插值多项式}可得到相应的牛顿均差插值多项式
\begin{align*}
P_n(x) &= f(x_0) + f[x_0, x_1](x - x_0) + f[x_0, x_1, x_2](x - x_0)(x - x_1) + \cdots  \\
&\quad + f[x_0, x_1, \cdots, x_n](x - x_0) \cdots (x - x_{n-1}),\\
R_n(x) &= f(x) - P_n(x) = f[x, x_0, \cdots, x_n]\omega_{n+1}(x),
\end{align*}
在上述牛顿均差插值多项式中,若令 $x_i \to x_0$ ($i = 1, 2, \cdots, n$),则由 \eqref{eq:数值分析-4.1} 式可得Taylor(泰勒)多项式
\begin{align*}
P_n(x) = f(x_0) + f'(x_0)(x - x_0) + \cdots + \frac{f^{(n)}(x_0)}{n!}(x - x_0)^n, 
\end{align*}
其余项为
\begin{align*}
R_n(x) = \frac{f^{(n+1)}(\xi)}{(n+1)!}(x - x_0)^{n+1}, \quad \xi \in (a, b). 
\end{align*}
\end{proof}


\subsection{两个典型的Hermite插值}

\begin{theorem}
若已知四阶可导函数$f$在插值点 $x_i$ ($i = 0, 1, 2$) 上的值为 $f(x_i)$ ($i = 0, 1, 2$)及一个导数值$f'(x_1)$,记$f$的三次插值多项式为$P(x)$,且满足条件
\begin{align*}
P(x_i) = f(x_i), \quad i = 0, 1,2\text{及}P'(x_1)=f'(x_1).
\end{align*}
则插值多项式可表示为
\begin{align*}
P(x) = f(x_0) + f[x_0, x_1](x - x_0) + f[x_0, x_1, x_2](x - x_0)(x - x_1) + A(x - x_0)(x - x_1)(x - x_2),
\end{align*}
其中\[
A = \frac{f'(x_1) - f[x_0, x_1] - (x_1 - x_0)f[x_0, x_1, x_2]}{(x_1 - x_0)(x_1 - x_2)}.
\]
余项表达式为
\begin{align}
R(x) = \frac{1}{4!}f^{(4)}(\xi)(x - x_0)(x - x_1)^2(x - x_2), \label{eq:数值分析-4.5}
\end{align}
式中 $\xi$ 位于 $x_0, x_1, x_2$ 和 $x$ 所界定的范围内.
\end{theorem}
\begin{remark}
一般上述插值多项式的系数$A$都是用待定系数法求解,并不直接套用上述$A$的公式.即先待定$A$,得到插值多项式$P(x)$,再代入$P'(x_1)=f'(x_1)$中解出$A$.
\end{remark}
\begin{proof}
由给定条件及\hyperref[theorem:牛顿均差插值多项式]{牛顿均差插值多项式},可确定次数不超过 3 的插值多项式. 由于此多项式通过点 $(x_0, f(x_0))$, $(x_1, f(x_1))$ 及 $(x_2, f(x_2))$, 故其形式为
\begin{align*}
P(x) = f(x_0) + f[x_0, x_1](x - x_0) + f[x_0, x_1, x_2](x - x_0)(x - x_1) + A(x - x_0)(x - x_1)(x - x_2),
\end{align*}
其中 $A$ 为待定常数,可由条件 $P'(x_1) = f'(x_1)$ 确定,通过计算可得
\[
A = \frac{f'(x_1) - f[x_0, x_1] - (x_1 - x_0)f[x_0, x_1, x_2]}{(x_1 - x_0)(x_1 - x_2)}.
\]
为了求出余项 $R(x) = f(x) - P(x)$ 的表达式,可设
\[
R(x) = f(x) - P(x) = k(x)(x - x_0)(x - x_1)^2(x - x_2),
\]
其中 $k(x)$ 为待定函数. 构造
\[
\varphi(t) = f(t) - P(t) - k(x)(t - x_0)(t - x_1)^2(t - x_2),
\]
显然 $\varphi(x_j) = 0$ ($j = 0, 1, 2$), 且 $\varphi'(x_1) = 0$, $\varphi(x) = 0$. 故 $\varphi(t)$ 在 $(a, b)$ 内有 5 个零点(二重根算两个). 假设 $f$ 具有较好的可微性,反复应用罗尔定理,得 $\varphi^{(4)}(t)$ 在 $(a, b)$ 内至少有一个零点 $\xi$, 故
\[
\varphi^{(4)}(\xi) = f^{(4)}(\xi) - 4!k(x) = 0,
\]
于是
\[
k(x) = \frac{1}{4!}f^{(4)}(\xi),
\]
余项表达式为
\begin{align*}
R(x) = \frac{1}{4!}f^{(4)}(\xi)(x - x_0)(x - x_1)^2(x - x_2),
\end{align*}
式中 $\xi$ 位于 $x_0, x_1, x_2$ 和 $x$ 所界定的范围内.
\end{proof}

\begin{example}
给定 $f(x) = x^{3/2}$, $x_0 = \frac{1}{4}$, $x_1 = 1$, $x_2 = \frac{9}{4}$, 试求 $f(x)$ 在 $\left[ \frac{1}{4}, \frac{9}{4} \right]$ 上的三次埃尔米特插值多项式 $P(x)$, 使它满足 $P(x_i) = f(x_i)$ ($i = 0, 1, 2$), $P'(x_1) = f'(x_1)$, 并写出余项表达式.
\end{example}
\begin{solution}
由所给节点可求出
\[
f_0 = f\left( \frac{1}{4} \right) = \frac{1}{8}, \quad f_1 = f(1) = 1, \quad f_2 = f\left( \frac{9}{4} \right) = \frac{27}{8},
\]
\[
f'(x) = \frac{3}{2} x^{1/2}, \quad f'(1) = \frac{3}{2}.
\]
利用牛顿均差插值,先求均差表如\reftab{table:均差表111}.
\begin{table}[H]
\centering
\caption{均差表}
\label{table:均差表111}
\begin{tabular}{c c c c}
\toprule
$x_i$ & $f_i$ &  &  \\
\midrule
$\frac{1}{4}$ & $\frac{1}{8}$ &  &  \\
$1$ & $1$ & $\frac{7}{6}$ & $\frac{11}{30}$ \\
$\frac{9}{4}$ & $\frac{27}{8}$ & $\frac{19}{10}$ &  \\
\bottomrule
\end{tabular}
\end{table}

于是有 $f[x_0, x_1] = \frac{7}{6}$, $f[x_0, x_1, x_2] = \frac{11}{30}$.
故可令
\begin{align*}
P(x) &= \frac{1}{8} + \frac{7}{6} \left( x - \frac{1}{4} \right) + \frac{11}{30} \left( x - \frac{1}{4} \right)(x - 1) \\
&\quad + A \left( x - \frac{1}{4} \right)(x - 1) \left( x - \frac{9}{4} \right).
\end{align*}
再由条件 $P'(1) = f'(1) = \frac{3}{2}$ 可得
\[
P'(1) = \frac{7}{6} + \frac{11}{30} \cdot \frac{3}{4} + A \cdot \frac{3}{4} \left( -\frac{5}{4} \right) = \frac{3}{2},
\]
解出
\[
A = -\frac{16}{15} \left( \frac{3}{2} - \frac{7}{6} - \frac{11}{40} \right) = -\frac{14}{225}.
\]
于是所求的三次埃尔米特多项式为
\begin{align*}
P(x) &= \frac{1}{8} + \frac{7}{6} \left( x - \frac{1}{4} \right) + \frac{11}{30} \left( x - \frac{1}{4} \right)(x - 1) - \frac{14}{225} \left( x - \frac{1}{4} \right)(x - 1) \left( x - \frac{9}{4} \right) \\
&= -\frac{14}{225} x^3 + \frac{263}{450} x^2 + \frac{233}{450} x - \frac{1}{25},
\end{align*}
余项为
\begin{align*}
R(x) &= f(x) - P(x) = \frac{f^{(4)}(\xi)}{4!} \left( x - \frac{1}{4} \right)(x - 1)^2 \left( x - \frac{9}{4} \right) \\
&= \frac{1}{4!} \cdot \frac{9}{16} \xi^{-5/2} \left( x - \frac{1}{4} \right)(x - 1)^2 \left( x - \frac{9}{4} \right), \quad \xi \in \left( \frac{1}{4}, \frac{9}{4} \right).
\end{align*}
\end{solution}

\begin{theorem}[两点三次插值多项式]
若已知四阶可导函数${\textstyle f}$在插值点 ${\textstyle x_k,x_{k+1}}$上的值为 ${\textstyle y_k=f(x_k),\,\, y_{k+1}=f(x_{k+1})}$及导数值为${\textstyle m_k=f'(x_k),\,\, m_{k+1}=f'(x_{k+1})}$,记$f$的三次插值多项式为$H_3(x)$,且满足条件
\begin{align}
\left. \begin{array}{r}
H_3(x_k)=y_k,\quad H_3(x_{k+1})=y_{k+1},\\
H_3'(x_k)=m_k,\quad H_3' (x_{k+1})=m_{k+1}.\\
\end{array} \right\} \label{eq:数值分析-4.6}
\end{align}
则插值多项式可表示为
\begin{align}
H_3(x) &= \left( 1 + 2 \frac{x - x_k}{x_{k+1} - x_k} \right) \left( \frac{x - x_{k+1}}{x_k - x_{k+1}} \right)^2 y_k + \left( 1 + 2 \frac{x - x_{k+1}}{x_k - x_{k+1}} \right) \left( \frac{x - x_k}{x_{k+1} - x_k} \right)^2 y_{k+1} \nonumber \\
&\quad + (x - x_k) \left( \frac{x - x_{k+1}}{x_k - x_{k+1}} \right)^2 m_k + (x - x_{k+1}) \left( \frac{x - x_k}{x_{k+1} - x_k} \right)^2 m_{k+1}, \label{eq:数值分析-4.12}
\end{align}
其余项 $R_3(x) = f(x) - H_3(x)$可表示为
\begin{align}
R_3(x) = \frac{1}{4!} f^{(4)}(\xi) (x - x_k)^2 (x - x_{k+1})^2, \quad \xi \in (x_k, x_{k+1}). \label{eq:数值分析-4.13}
\end{align}
\end{theorem}
\begin{proof}
令
\begin{align}
H_3(x) = \alpha_k(x)y_k + \alpha_{k+1}(x)y_{k+1} + \beta_k(x)m_k + \beta_{k+1}(x)m_{k+1}, \label{eq:数值分析-4.7}
\end{align}
其中 $\alpha_k(x), \alpha_{k+1}(x), \beta_k(x), \beta_{k+1}(x)$ 是关于节点 $x_k$ 及 $x_{k+1}$ 的三次埃尔米特插值基函数,它们应分别满足条件
\[
\alpha_k(x_k) = 1, \quad \alpha_k(x_{k+1}) = 0, \quad \alpha_k'(x_k) = \alpha_k'(x_{k+1}) = 0;
\]
\[
\alpha_{k+1}(x_k) = 0, \quad \alpha_{k+1}(x_{k+1}) = 1, \quad \alpha_{k+1}'(x_k) = \alpha_{k+1}'(x_{k+1}) = 0;
\]
\[
\beta_k(x_k) = \beta_k(x_{k+1}) = 0, \quad \beta_k'(x_k) = 1, \quad \beta_k'(x_{k+1}) = 0;
\]
\[
\beta_{k+1}(x_k) = \beta_{k+1}(x_{k+1}) = 0, \quad \beta_{k+1}'(x_k) = 0, \quad \beta_{k+1}'(x_{k+1}) = 1.
\]
根据给定条件可令
\[
\alpha_k(x) = (ax + b) \left( \frac{x - x_{k+1}}{x_k - x_{k+1}} \right)^2,
\]
显然
\[
\alpha_k(x_{k+1}) = \alpha_k'(x_{k+1}) = 0.
\]
再利用
\[
\alpha_k(x_k) = ax_k + b = 1,
\]
及
\[
\alpha_k'(x_k) = 2 \frac{ax_k + b}{x_k - x_{k+1}} + a = 0,
\]
解得
\[
a = -\frac{2}{x_k - x_{k+1}}, \quad b = 1 + \frac{2x_k}{x_k - x_{k+1}},
\]
于是求得
\begin{align}
\alpha_k(x) = \left( 1 + 2 \frac{x - x_k}{x_{k+1} - x_k} \right) \left( \frac{x - x_{k+1}}{x_k - x_{k+1}} \right)^2. \label{eq:数值分析-4.8}
\end{align}
同理可求得
\begin{align}
\alpha_{k+1}(x) = \left( 1 + 2 \frac{x - x_{k+1}}{x_k - x_{k+1}} \right) \left( \frac{x - x_k}{x_{k+1} - x_k} \right)^2. \label{eq:数值分析-4.9}
\end{align}
为求 $\beta_k(x)$, 由给定条件可令
\[
\beta_k(x) = a(x - x_k) \left( \frac{x - x_{k+1}}{x_k - x_{k+1}} \right)^2,
\]
直接由 $\beta_k'(x_k) = a = 1$ 得到
\begin{align}
\beta_k(x) = (x - x_k) \left( \frac{x - x_{k+1}}{x_k - x_{k+1}} \right)^2. \label{eq:数值分析-4.10}
\end{align}
同理有
\begin{align}
\beta_{k+1}(x) = (x - x_{k+1}) \left( \frac{x - x_k}{x_{k+1} - x_k} \right)^2. \label{eq:数值分析-4.11}
\end{align}
将 \eqref{eq:数值分析-4.8} 式~ \eqref{eq:数值分析-4.11} 式的结果代入 \eqref{eq:数值分析-4.7} 式得
\begin{align*}
H_3(x) &= \left( 1 + 2 \frac{x - x_k}{x_{k+1} - x_k} \right) \left( \frac{x - x_{k+1}}{x_k - x_{k+1}} \right)^2 y_k + \left( 1 + 2 \frac{x - x_{k+1}}{x_k - x_{k+1}} \right) \left( \frac{x - x_k}{x_{k+1} - x_k} \right)^2 y_{k+1} \\
&\quad + (x - x_k) \left( \frac{x - x_{k+1}}{x_k - x_{k+1}} \right)^2 m_k + (x - x_{k+1}) \left( \frac{x - x_k}{x_{k+1} - x_k} \right)^2 m_{k+1}, 
\end{align*}
其余项 $R_3(x) = f(x) - H_3(x)$. 类似 \eqref{eq:数值分析-4.5} 式可得
\begin{align*}
R_3(x) = \frac{1}{4!} f^{(4)}(\xi) (x - x_k)^2 (x - x_{k+1})^2, \quad \xi \in (x_k, x_{k+1}). 
\end{align*}
\end{proof}




















\end{document}
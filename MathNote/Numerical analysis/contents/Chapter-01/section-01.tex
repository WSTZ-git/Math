\documentclass[../../main.tex]{subfiles}
\graphicspath{{\subfix{../../image/}}} % 指定图片目录,后续可以直接使用图片文件名。

% 例如:
% \begin{figure}[H]
% \centering
% \includegraphics[scale=0.4]{图.png}
% \caption{}
% \label{figure:图}
% \end{figure}
% 注意:上述\label{}一定要放在\caption{}之后,否则引用图片序号会只会显示??.

\begin{document}

\section{数值计算的误差}

用计算机解决科学计算问题首先要建立数学模型,它是对被描述的实际问题进行抽象、简化而得到的,因而是近似的。我们把数学模型与实际问题之间出现的这种误差称为\textbf{模型误差}。只有实际问题提法正确,建立数学模型时又抽象、简化得合理,才能得到好的结果。

由于这种误差难于用数量表示,通常都假定数学模型是合理的,这种误差可忽略不计,在“数值分析”中不予讨论。在数学模型中往往还有一些根据观测得到的物理量,如温度、长度、电压等,这些参量显然也包含误差。这种由观测产生的误差称为\textbf{观测误差},在“数值分析”中也不讨论这种误差。数值分析只研究用数值方法求解数学模型产生的误差。

当数学模型不能得到精确解时,通常要用数值方法求它的近似解,其近似解与精确解之间的误差称为\textbf{截断误差}或\textbf{方法误差}.

\begin{definition}[误差和误差限]
设 $x$ 为准确值,$x^*$ 为 $x$ 的一个近似值,称 $e^* = x^* - x$ 为近似值的\textbf{绝对误差},简称\textbf{误差}。

误差$e^*$绝对值的一个上界。$\varepsilon^*$ 叫做近似值的\textbf{绝对误差限}或\textbf{误差限},它总是正数。

对于一般情形,$|x^* - x| \leqslant \varepsilon^*$,即 
\[
x^* - \varepsilon^* \leqslant x \leqslant x^* + \varepsilon^*
\]
这个不等式有时也表示为 $x = x^* \pm \varepsilon^*$。
\end{definition}

\begin{definition}[相对误差和相对误差限]
$x$ 本身的大小。我们把近似值的误差 $e^*$ 与准确值 $x$ 的比值 
\[
\frac{e^*}{x} = \frac{x^* - x}{x}
\]
称为近似值 $x^*$ 的\textbf{相对误差},记作 $e_r^*$。在实际计算中,通常取 
\[
e_r^* = \frac{e^*}{x^*} = \frac{x^* - x}{x^*}.
\]

相对误差也可正可负,它的绝对值上界叫做\textbf{相对误差限},记作 $\varepsilon_r^*$,即 $\varepsilon_r^* = \frac{\varepsilon^*}{|x^*|}$。
\end{definition}
\begin{remark}
相对误差与相对误差限是无量纲的,而绝对误差与误差限是有量纲的。
\end{remark}
\begin{remark}
相对误差$e_r^*$取$\frac{e^*}{x^*}$的原因:在实际计算中,由于真值 $x$ 总是不知道的,通常取 
\[
e_r^* = \frac{e^*}{x^*} = \frac{x^* - x}{x^*}
\]
作为 $x^*$ 的相对误差,条件是 $e_r^* = \frac{e^*}{x^*}$ 较小,此时 
\[
\frac{e^*}{x} - \frac{e^*}{x^*} = \frac{e^* (x^* - x)}{x^* x} = \frac{(e^*)^2}{x^* (x^* - e^*)} = \frac{(e^*/x^*)^2}{1 - (e^*/x^*)}
\]
是 $e_r^*$ 的平方项级,故可忽略不计。
\end{remark}

\begin{example}
当准确值 $x$ 有多位数时,常常按四舍五入的原则得到 $x$ 的前几位近似值 $x^*$,例如 
\[
x = \pi = 3.14159265\cdots
\]

取 3 位  $x_3^* = 3.14$,$\varepsilon_3^* \leqslant 0.002$,

取 5 位  $x_5^* = 3.1416$,$\varepsilon_5^* \leqslant 0.000008$,
\end{example}

\begin{definition}[有效数字]
若近似值 $x^*$ 的误差限是某一位的半个单位,该位到 $x^*$ 的第一位非零数字共有 $n$ 位,就说 $x^*$ 有 $n$ 位\textbf{有效数字}。它可表示为 
\begin{align}
x^* = \pm 10^m \times (a_1 + a_2 \times 10^{-1} + \cdots + a_n \times 10^{-(n - 1)}),\label{eq::--2.....111111241}
\end{align}
其中 $a_i (i = 1, 2, \cdots, n)$ 是 0 到 9 中的一个数字,$a_1 \neq 0$,$m$ 为整数,且 
\begin{align}
| x - x^* | \leqslant \frac{1}{2} \times 10^{m - n + 1}.\label{eq::--2.....111111242}
\end{align}
\end{definition}

\begin{example}
按四舍五入原则写出下列各数的具有 5 位有效数字的近似数:187.9325,0.03785551,8.000033,2.7182818。
\end{example}
\begin{solution}
按定义,上述各数的具有 5 位有效数字的近似数分别是 
\[
187.93, 0.037856, 8.0000, 2.7183
\]

注意 $x = 8.000033$ 的 5 位有效数字近似数是 8.0000 而不是 8,因为 8 只有 1 位有效数字。
\end{solution}

\begin{example}
如果以 $\mathrm{m/s^2}$ 为单位,重力常数 $g \approx 9.80\ \mathrm{m/s^2}$;若以 $\mathrm{km/s^2}$ 为单位,$g \approx 0.00980\ \mathrm{km/s^2}$,它们都具有 3 位有效数字,因为按第一种写法 
\[
| g - 9.80 | \leqslant \frac{1}{2} \times 10^{-2}
\]
根据 \eqref{eq::--2.....111111241} 式,这里 $m = 0$,$n = 3$;按第二种写法 
\[
| g - 0.00980 | \leqslant \frac{1}{2} \times 10^{-5}
\]
这里 $m = -3$,$n = 3$。它们虽然写法不同,但都具有 3 位有效数字。至于绝对误差限,由于单位不同结果也不同,$\varepsilon_1^* = \frac{1}{2} \times 10^{-2}\ \mathrm{m/s^2}$,$\varepsilon_2^* = \frac{1}{2} \times 10^{-5}\ \mathrm{km/s^2}$。而相对误差相同,因为 
\[
\varepsilon_r^* = 0.005 / 9.80 = 0.000005 / 0.00980.
\]
\end{example}
\begin{note}
这个例题说明有效位数与小数点后有多少位数无关。
\end{note}

\begin{theorem}
设近似数 $x^*$ 表示为 
\[
x^* = \pm 10^m \times (a_1 + a_2 \times 10^{-1} + \cdots + a_l \times 10^{-(l - 1)}) \label{eq::::::----2487-2.1'}
\]
其中 $a_i (i = 1, 2, \cdots, l)$ 是 0 到 9 中的一个数字,$a_1 \neq 0$,$m$ 为整数。若 $x^*$ 具有 $n$ 位有效数字,则其绝对误差限
\[
\varepsilon^* = \frac{1}{2} \times 10^{m - n + 1}.
\]
反之,若 $x^*$ 的绝对误差限 $\varepsilon^* = \frac{1}{2} \times 10^{m - n + 1}$,则 $x^*$具有 $n$ 位有效数字。
\end{theorem}
\begin{note}
这个定理说明,在 $m$ 相同的情况下,$n$ 越大则 $10^{m - n + 1}$ 越小,故有效位数越多,绝对误差限越小。
\end{note}
\begin{proof}
从 \eqref{eq::--2.....111111242} 式可得到具有 $n$ 位有效数字的近似数 $x^*$,其绝对误差限为 
\[
\varepsilon^* = \frac{1}{2} \times 10^{m - n + 1}.
\]
反之,若 $x^*$ 的绝对误差限 $\varepsilon^* = \frac{1}{2} \times 10^{m - n + 1}$,则
\begin{align*}
|x-x^*|=\varepsilon^*=\frac{1}{2} \times 10^{m - n + 1}.
\end{align*}
故$x^*$ 具有 $n$ 位有效数字.
\end{proof}

\begin{theorem}
设近似数 $x^*$ 表示为
\begin{align}
x^* = \pm 10^m \times (a_1 + a_2 \times 10^{-1} + \cdots + a_l \times 10^{-(l - 1)}), \label{eq::::::----2487-2.1'}
\end{align}
其中 $a_i (i = 1, 2, \cdots, l)$ 是 0 到 9 中的一个数字,$a_1 \neq 0$,$m$ 为整数。若 $x^*$ 具有 $n$ 位有效数字,则其相对误差限 
\[
\varepsilon_r^* \leqslant \frac{1}{2a_1} \times 10^{-(n - 1)}.
\]
反之,若 $x^*$ 的相对误差限 $\varepsilon_r^* \leqslant \frac{1}{2(a_1 + 1)} \times 10^{-(n - 1)}$,则 $x^*$ 至少具有 $n$ 位有效数字。
\end{theorem}
\begin{note}
这个定理说明,有效位数越多,相对误差限越小。
\end{note}
\begin{proof}
由 \eqref{eq::::::----2487-2.1'} 式可得 
\[
a_1 \times 10^m \leqslant | x^* | < (a_1 + 1) \times 10^m
\]
当 $x^*$ 具有 $n$ 位有效数字时 
\[
\varepsilon_r^* = \frac{| x - x^* |}{| x^* |} \leqslant \frac{0.5 \times 10^{m - n + 1}}{a_1 \times 10^m} = \frac{1}{2a_1} \times 10^{-n + 1}
\]
反之,由 
\[
| x - x^* | = | x^* | \varepsilon_r^* < (a_1 + 1) \times 10^m \times \frac{1}{2(a_1 + 1)} \times 10^{-n + 1} = 0.5 \times 10^{m - n + 1}
\]
故 $x^*$ 至少具有 $n$ 位有效数字。证毕。
\end{proof}

\begin{theorem}
设两个近似数 $x_1^*$ 与 $x_2^*$ 的误差限分别为 $\varepsilon(x_1^*)$ 及 $\varepsilon(x_2^*)$,则它们进行加、减、乘、除运算得到的误差限分别满足不等式 
\[
\varepsilon(x_1^* \pm x_2^*) \leqslant \varepsilon(x_1^*) + \varepsilon(x_2^*),
\]
\[
\varepsilon(x_1^* x_2^*) \leqslant | x_1^* | \varepsilon(x_2^*) + | x_2^* | \varepsilon(x_1^*),
\]
\[
\varepsilon(x_1^* / x_2^*) \leqslant \frac{| x_1^* | \varepsilon(x_2^*) + | x_2^* | \varepsilon(x_1^*)}{| x_2^* |^2}, \quad x_2^* \neq 0.
\]
\end{theorem}
\begin{proof}

\end{proof}

\begin{theorem}
\begin{enumerate}
\item 设 $f(x)$ 是一元可微函数,$x$ 的近似值为 $x^*$,以 $f(x^*)$ 近似 $f(x)$,其误差界记作 $\varepsilon(f(x^*))$,则函数的误差限 
\[
\varepsilon(f(x^*)) \approx | f'(x^*) | \varepsilon(x^*).
\]

\item 设$f$ 为多元函数时,令$A = f(x_1, x_2, \cdots, x_n)$。如果 $x_1, x_2, \cdots, x_n$ 的近似值为 $x_1^*, x_2^*, \cdots, x_n^*$,记$A$ 的近似值为 $A^* = f(x_1^*, x_2^*, \cdots, x_n^*)$,则$A^*$ 的误差 $e(A^*)$ 为 
\[
e(A^*) =\sum_{k=1}^n \left( \frac{\partial f}{\partial x_k} \right)^* e_k^*.
\]
$A^*$ 的误差限 
\begin{align}
\varepsilon(A^*) \approx \sum_{k=1}^n \left| \left( \frac{\partial f}{\partial x_k} \right)^* \right| \varepsilon(x_k^*) .\label{eq:::---20...3}
\end{align}
而 $A^*$ 的相对误差限为 
\[
\varepsilon_r^* = \varepsilon_r(A^*) = \frac{\varepsilon(A^*)}{| A^* |} \approx \sum_{k=1}^n \left| \left( \frac{\partial f}{\partial x_k} \right)^* \right| \frac{\varepsilon(x_k^*)}{| A^* |} .
\]
\end{enumerate}
\end{theorem}
\begin{proof}
\begin{enumerate}
\item 由泰勒展开 
\[
f(x) - f(x^*) = f'(x^*)(x - x^*) + \frac{f''(\xi)}{2}(x - x^*)^2, \quad \xi \text{ 介于 } x, x^* \text{ 之间},
\]
取绝对值得 
\[
| f(x) - f(x^*) | \leqslant | f'(x^*) | \varepsilon(x^*) + \frac{| f''(\xi) |}{2} \varepsilon^2(x^*).
\]
假定 $f'(x^*)$ 与 $f''(x^*)$ 的比值不太大,可忽略 $\varepsilon(x^*)$ 的高阶项,于是可得计算函数的误差限 
\[
\varepsilon(f(x^*)) \approx | f'(x^*) | \varepsilon(x^*).
\]

\item 由泰勒展开得函数值 $A^*$ 的误差 $e(A^*)$ 为 
\begin{align*}
e(A^*)&=A^*-A=f(x_{1}^{*},x_{2}^{*},\cdots ,x_{n}^{*})-f(x_1,x_2,\cdots ,x_n)
\\
&\approx \sum_{k=1}^n{\left( \frac{\partial f(x_{1}^{*},x_{2}^{*},\cdots ,x_{n}^{*})}{\partial x_k} \right) (x_{k}^{*}}-x_k)=\sum_{k=1}^n{\left( \frac{\partial f}{\partial x_k} \right) ^*e_{k}^{*}},
\end{align*}
于是误差限 
\[
\varepsilon(A^*) \approx \sum_{k=1}^n \left| \left( \frac{\partial f}{\partial x_k} \right)^* \right| \varepsilon(x_k^*) ,
\]
而 $A^*$ 的相对误差限为 
\[
\varepsilon_r^* = \varepsilon_r(A^*) = \frac{\varepsilon(A^*)}{| A^* |} \approx \sum_{k=1}^n \left| \left( \frac{\partial f}{\partial x_k} \right)^* \right| \frac{\varepsilon(x_k^*)}{| A^* |} .
\]
\end{enumerate}
\end{proof}

\begin{example}
已测得某场地长 $l$ 的值为 $l^* = 110\ \mathrm{m}$,宽 $d$ 的值为 $d^* = 80\ \mathrm{m}$,已知 $| l - l^* | \leqslant 0.2\ \mathrm{m}$,$| d - d^* | \leqslant 0.1\ \mathrm{m}$。试求面积 $s = ld$ 的绝对误差限与相对误差限。
\end{example}
\begin{solution}
因 $s = ld$,$\frac{\partial s}{\partial l} = d$,$\frac{\partial s}{\partial d} = l$,由 \eqref{eq:::---20...3} 式知 
\[
\varepsilon(s^*) \approx \left| \left( \frac{\partial s}{\partial l} \right)^* \right| \varepsilon(l^*) + \left| \left( \frac{\partial s}{\partial d} \right)^* \right| \varepsilon(d^*),
\]
其中 
\[
\left( \frac{\partial s}{\partial l} \right)^* = d^* = 80\ \mathrm{m}, \quad \left( \frac{\partial s}{\partial d} \right)^* = l^* = 110\ \mathrm{m}.
\]
而 $\varepsilon(l^*) = 0.2\ \mathrm{m}$,$\varepsilon(d^*) = 0.1\ \mathrm{m}$,于是绝对误差限 
\[
\varepsilon(s^*) \approx 80 \times (0.2) + 110 \times (0.1) = 27\ (\mathrm{m^2}),
\]
相对误差限 
\[
\varepsilon_r(s^*) = \frac{\varepsilon(s^*)}{| s^* |} = \frac{\varepsilon(s^*)}{l^* d^*} \approx \frac{27}{8800} = 0.31\%.
\]
\end{solution}













\end{document}
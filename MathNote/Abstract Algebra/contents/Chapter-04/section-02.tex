\documentclass[../../main.tex]{subfiles}% 注意这里的文件路径不能用 ./main.tex ,否则用latexmk编译子文件会报错
\graphicspath{{\subfix{./image/}}} % 指定图片目录,后续可以直接使用图片文件名
% 注意这里的文件路径不能用 ../../image/ ,否则用latexmk编译子文件会报错

% 例如:
% \begin{figure}[H]
% \centering
% \includegraphics[scale=0.3]{图.png}
% \caption{}
% \label{figure:图}
% \end{figure}
% 注意:上述\label{}一定要放在\caption{}之后,否则引用图片序号会只会显示??.

\begin{document}

\section{有限扩张}

\begin{definition}
设 $K$ 是域 $F$ 的扩域, $K$ 作为域 $F$ 上线性空间的维数称为 $K$ 对 $F$ 的\textbf{扩张次数}, 记为 $[K:F]$. 若 $[K:F]<+\infty$, 则称 $K$ 是 $F$ 的\textbf{有限扩张}.
\end{definition}

\begin{proposition}
设 $F(\alpha)$ 是 $F$ 的单扩张. $\alpha$ 是 $F$ 上代数元时, $F(\alpha)$ 是 $F$ 的有限扩张且 $[F(\alpha):F]=\deg(\alpha,F)$;

当 $\alpha$ 是 $F$ 上超越元时, $F(\alpha)$ 是 $F$ 的无限扩张.
\end{proposition}
\begin{proof}


\end{proof}

\begin{definition}
设 $K$ 是域 $F$ 的扩域. 若 $\forall \alpha\in K$, $\alpha$ 都是 $F$ 上代数元, 则称 $K$ 是域 $F$ 的\textbf{代数扩张}, 否则称为\textbf{超越扩张}.
\end{definition}

\begin{theorem}\label{theorem:定理3.2.1--抽象代数}
设 $K$ 是域 $F$ 的扩域, $\alpha\in K$, 则下列条件等价:
\begin{enumerate}[(1)]
\item $F(\alpha)$ 是 $F$ 的代数扩张;
\item $\alpha$ 在 $F$ 上是代数的;
\item $F(\alpha)$ 是 $F$ 的有限扩张.
\end{enumerate}
\end{theorem}
\begin{proof}
(1) $\Rightarrow$ (2). $F(\alpha)$ 是 $F$ 的代数扩张, 而 $\alpha\in F(\alpha)$, 故 $\alpha$ 在 $F$ 上是代数的.

(2) $\Rightarrow$ (3). $\alpha$ 在 $F$ 上是代数的, 故由\refthe{theorem:抽象代数--定理3.1.4}知 $[F(\alpha):F]=\deg(\alpha,F)<+\infty$, 因而 $F(\alpha)$ 是 $F$ 的有限扩张.

(3) $\Rightarrow$ (1). 设 $[F(\alpha):F]=n<+\infty$. $\forall \beta\in F(\alpha)$, 则 $1,\beta,\cdots,\beta^n$ 是 $F(\alpha)$ 中 $n+1$ 个元素, 一定线性相关, 即存在不全为零的 $a_0,a_1,\cdots,a_n\in F$, 使 $\sum\limits_{i=0}^{n}a_i\beta^i=0$. 故$\beta$ 是 $F$ 上代数元, 故 $F(\alpha)$ 是 $F$ 的代数扩张.


\end{proof}

\begin{corollary}
若 $K$ 是 $F$ 的有限扩张, 则 $K$ 一定是 $F$ 的代数扩张.
\end{corollary}
\begin{proof}
设 $[K:F]=n<+\infty$. $\forall \beta\in K$, 则 $1,\beta,\cdots,\beta^n$ 是 $K$ 中 $n+1$ 个元素, 一定线性相关, 即存在不全为零的 $a_0,a_1,\cdots,a_n\in F$, 使 $\sum\limits_{i=0}^{n}a_i\beta^i=0$. 故$\beta$ 是 $F$ 上代数元, 故 $K$ 是 $F$ 的代数扩张.


\end{proof}

\begin{theorem}\label{theorem:定理3.2.2--抽象代数}
设 $E$ 是域 $F$ 的扩域, $K$ 是域 $E$ 的扩域, 即有 $K\supseteq E\supseteq F$, 则当且仅当 $[K:F]<+\infty$ 时有 $[K:E]<+\infty$, $[E:F]<+\infty$, 而且此时有
\begin{align}
[K:F]=[K:E][E:F].\label{eq::--2h8iofn8ow3r:3.2.1}
\end{align}
\end{theorem}
\begin{proof}
设 $[K:F]<+\infty$, 由条件知 $E$ 是 $F$ 上线性空间 $K$ 的子空间, 则$[E:F]<+\infty$. 设 $\alpha_1,\alpha_2,\cdots,\alpha_n$ 是 $K$ 对 $F$ 的基, 即 $\forall \alpha\in K$, $\exists x_i\in F\subseteq E(1\leqslant i\leqslant n)$, 使 $\alpha=\sum\limits_{i=1}^{n}x_i\alpha_i$, 故 $\alpha_1,\alpha_2,\cdots,\alpha_n$ 也是 $E$ 上线性空间 $K$ 的一组生成元, 故 $[K:E]<+\infty$.

反之, 设 $[K:E]=r$, $[E:F]=s$. 又在 $E$ 上的线性空间 $K$ 中取基 $\alpha_1,\alpha_2,\cdots,\alpha_r$, 在 $F$ 上的线性空间 $E$ 中取基 $\beta_1,\beta_2,\cdots,\beta_s$. 设 $\alpha\in K$, 则 $\exists x_i\in E(1\leqslant i\leqslant r)$, 使得
\begin{align*}
\alpha=\sum\limits_{i=1}^{r}x_i\alpha_i.
\end{align*}
又对每个 $x_i\in E$, $\exists y_{ij}\in F(1\leqslant j\leqslant s)$, 使得 $x_i=\sum\limits_{j=1}^{s}y_{ij}\beta_j$, 因而
\begin{align*}
\alpha=\sum\limits_{i=1}^{r}\sum\limits_{j=1}^{s}y_{ij}\alpha_i\beta_j.
\end{align*}
故 $\{\alpha_i\beta_j\mid 1\leqslant i\leqslant r,1\leqslant j\leqslant s\}$ 是 $F$ 上线性空间 $K$ 的一组生成元. 设 $y_{ij}\in F$, 而
\begin{align*}
\sum\limits_{i=1}^{r}\sum\limits_{j=1}^{s}y_{ij}\alpha_i\beta_j=0,
\end{align*}
即
\begin{align*}
\sum\limits_{i=1}^{r}\left(\sum\limits_{j=1}^{s}y_{ij}\beta_j\right)\alpha_i=0.
\end{align*}
由 $\sum\limits_{j=1}^{s}y_{ij}\beta_j\in E$ 且 $\alpha_1,\alpha_2,\cdots,\alpha_r$ 是 $E$ 上线性空间 $K$ 的基知 $\sum\limits_{j=1}^{s}y_{ij}\beta_j=0(1\leqslant i\leqslant r)$. 又 $\beta_1,\beta_2,\cdots,\beta_s$ 是 $F$ 上线性空间 $E$ 的基, 故 $y_{ij}=0(1\leqslant i\leqslant r,1\leqslant j\leqslant s)$. 于是 $\{\alpha_i\beta_j\mid 1\leqslant i\leqslant r,1\leqslant j\leqslant s\}$ 是 $F$ 上线性空间 $K$ 的一组基, 于是式 \eqref{eq::--2h8iofn8ow3r:3.2.1} 成立.


\end{proof}

\begin{corollary}
设$K$是域$F$的扩域,若 $[K:F]$ 是素数, 则 $K$ 与 $F$ 之间无中间域, 即不存在域 $E$, 使得 $K\supset E\supset F$.
\end{corollary}
\begin{proof}
若 $E$ 为中间域, 即 $K\supset E\supset F$, 则由\refthe{theorem:定理3.2.2--抽象代数}知
\begin{align*}
[K:F]=[K:E][E:F].
\end{align*}
因 $[K:F]$ 是素数, 故 $[E:F]=1$ 或 $[E:F]=[K:F]$. 若 $[E:F]=1$, 则 $E=F$; 若 $[E:F]=[K:F]$, 则 $E\cong K$,即$E=K$. 这都导出矛盾, 故本推论成立.

\end{proof}

\begin{corollary}\label{corollary:抽象代数--推论3.2.111}
设 $K$ 是域 $F$ 的有限扩张, 则有中间域的升链
\begin{align*}
F=F_0\subseteq F_1\subseteq F_2\subseteq\cdots\subseteq F_r=K,
\end{align*}
其中$F_{i+1}$ 是 $F_i$ 的单代数扩张, $i=0,1,\cdots,r-1$ (此升链称为\textbf{单代数扩张升链}).

反之, 若 $K$ 有单代数扩张升链, 则 $K$ 是 $F$ 的有限扩张.
\end{corollary}

\begin{proof}
设 $[K:F]<+\infty$ 且 $K\neq F$ ($K=F$ 时推论显然成立). 取 $\alpha_1\in K\setminus F$, $F_1=F(\alpha_1)$. 于是 $F=F_0\subseteq F_1$. 若 $F_1=K$, 则 $r=1$. 若 $F_1\neq K$, 由 $[K:F_1]<[K:F]$ 重复上面的做法, 有限次后得到$F_{i+1}$ 是 $F_i$ 的单代数扩张,$F_{i+1}\subseteq F_i$且$[K:F_r]=1$,从而由\refcor{corollary:抽象代数--推论3.2.111}可得$F_r=K$.此即所求的单代数扩张升链.

反之, 由于 $F_{i+1}$ 是 $F_i$ 的单代数扩张, 故由\refthe{theorem:定理3.2.1--抽象代数}知 $[F_{i+1}:F_i]<+\infty$, 由\refthe{theorem:定理3.2.2--抽象代数}知
\begin{align*}
[K:F]=[K:F_{r-1}][F_{r-1}:F_{r-2}]\cdots[F_1:F]<+\infty.
\end{align*}


\end{proof}

\begin{corollary}\label{corollary:抽象代数--推论3.2.4}
设 $K$ 是域 $F$ 的扩域, $E$ 为中间域. 若 $E$ 是 $F$ 的代数扩张, $K$ 是 $E$ 的代数扩张, 则 $K$ 是 $F$ 的代数扩张.
\end{corollary}
\begin{proof}
任取 $\alpha\in K$, 即 $\alpha$ 是 $E$ 上的代数元, 于是 $\mathrm{Irr}(\alpha,E)=x^n+a_1x^{n-1}+\cdots+a_n$, 其中, $a_i\in E(i=1,2,\cdots,n)$. 由 $E$ 是 $F$ 的代数扩张, 故 $a_i$ 是 $F$ 上的代数元. 令
\begin{align*}
F_0=F,\quad F_i=F_0(a_1,a_2,\cdots,a_i),\ 1\leqslant i\leqslant n,\quad F_{n+1}=F_n(\alpha).
\end{align*}
于是 $F_0\subseteq F_1\subseteq\cdots\subseteq F_{n+1}$ 是单代数扩张升链. 故由\refcor{corollary:抽象代数--推论3.2.111}知 $F_{n+1}$ 是 $F$ 上的有限扩张, 故由\refthe{theorem:定理3.2.1--抽象代数}知$F_{n+1}$为代数扩张, $\alpha$ 是 $F$ 上的代数元, 故 $K$ 是 $F$ 的代数扩张.


\end{proof}

\begin{definition}
设 $K$ 是域 $F$ 的扩域, $K$ 中在 $F$ 上为代数元的元素集合 $K_0$ 称为 $K$ 在 $F$ 上的\textbf{代数闭包}.
\end{definition}

\begin{theorem}
设 $K$ 是域 $F$ 的扩域, $K_0$ 为 $K$ 在 $F$ 上的代数闭包, 则 $K_0$ 是含于 $K$ 的 $F$ 的最大代数扩张且 $\forall \delta\in K\setminus K_0$, $\delta$ 在 $K_0$ 上是超越的.
\end{theorem}
\begin{proof}
只需证明 $K_0$ 是 $F$ 的扩域. 由定义就知 $K_0$ 是 $K$ 中 $F$ 的最大代数扩张. 显然 $F\subseteq K_0$. 设 $\alpha,\beta\in K_0$ 且 $\beta\neq 0$, 于是有 $\alpha\pm\beta,\alpha\beta^{\pm 1}\in F(\alpha,\beta)=F(\alpha)(\beta)$, 而 $F(\alpha)(\beta)$ 是 $F(\alpha)$ 的代数扩张, $F(\alpha)$ 是 $F$ 的代数扩张. 于是由\refcor{corollary:抽象代数--推论3.2.4}知 $F(\alpha,\beta)$ 是 $F$ 的代数扩张, 因而 $\alpha\pm\beta,\alpha\beta^{\pm 1}$ 均为 $F$ 上的代数元, 即 $\alpha\pm\beta,\alpha\beta^{\pm 1}\in K_0$, 故$K_0$ 是 $K$ 的子域.又$K_0\subseteq F$,因此$K_0$ 是 $F$ 的扩域.

$\forall \delta\in K\setminus K_0$, 假设 $\delta$ 是 $K_0$ 上的代数元, 则由 $K_0(\delta)\supset K_0\supseteq F$ 知 $K_0(\delta)$ 是 $F$ 的代数扩张, 故由\refthe{theorem:定理3.2.1--抽象代数}知$\delta$ 是 $F$ 上的代数元, 即 $\delta\in K_0$, 与已知矛盾, 故 $\delta$ 是 $K_0$ 上的超越元.

\end{proof}














\end{document}
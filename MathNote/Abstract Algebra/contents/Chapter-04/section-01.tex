\documentclass[../../main.tex]{subfiles}% 注意这里的文件路径不能用 ./main.tex ,否则用latexmk编译子文件会报错
\graphicspath{{\subfix{./image/}}} % 指定图片目录,后续可以直接使用图片文件名
% 注意这里的文件路径不能用 ../../image/ ,否则用latexmk编译子文件会报错

% 例如:
% \begin{figure}[H]
% \centering
% \includegraphics[scale=0.3]{图.png}
% \caption{}
% \label{figure:图}
% \end{figure}
% 注意:上述\label{}一定要放在\caption{}之后,否则引用图片序号会只会显示??.

\begin{document}

\section{域的单扩张}

\begin{definition}[环的特征]
设$R$为环. 如果存在最小的正整数$n$, 使得对所有的$a \in R$, 有$na = 0$, 则称$n$为环$R$的\textbf{特征}. 如果这样的正整数不存在, 则称环$R$的\textbf{特征}为0. 环$R$的特征记作$\mathrm{Char}\, R$.
\end{definition}

\begin{example}
$\mathbb{Z}$, $\mathbb{Q}$, $\mathbb{R}$, $\mathbb{C}$的特征都等于0.
\end{example}

\begin{proposition}\label{proposition:模m剩余类环及其多项式环的特征就是m}
设$\mathbb{Z}_m$是模$m$剩余类环, 则$\mathrm{Char}\,\mathbb{Z}_m = m$,$\mathrm{Char}\,\mathbb{Z}_m[x] = m$.
\end{proposition}
\begin{proof}
对每个$\overline{n} \in \mathbb{Z}_m$, 有
\begin{align*}
m\overline{n} = \overline{mn} = \overline{0}.
\end{align*}
而对于任何正整数$k < m$, 有
\begin{align*}
k\overline{1} = \overline{k} \neq \overline{0},
\end{align*}
所以$\mathrm{Char}\,\mathbb{Z}_m = m$. 类似地可以证明, 对于$\mathbb{Z}_m$上的一元多项式环$\mathbb{Z}_m[x]$, 也有$\mathrm{Char}\,\mathbb{Z}_m[x] = m$.

\end{proof}

\begin{theorem}\label{theorem:环的特征的基本性质-1}
设$R$是有单位元$e$的环. 如果$e$关于加法的阶为无穷大, 那么$R$的特征等于0. 如果$e$关于加法的阶等于$n$, 那么$\mathrm{Char}\, R = n$.
\end{theorem}
\begin{proof}
如果$e$关于加法的阶为无穷大, 那么不存在正整数$n$, 使得$ne = 0$. 所以由特征的定义知, $R$的特征等于0.

如果$e$关于加法的阶等于正整数$n$, 则$ne = 0$. 而且$n$是满足这一性质的最小正整数. 因此, 对于任意的$a \in R$,有
\begin{align*}
na = n(e \cdot a) = (ne) \cdot a = 0 \cdot a = 0.
\end{align*}
于是$R$的特征等于$n$.

\end{proof}

\begin{theorem}
整环的特征是0或者是一个素数.
\end{theorem}
\begin{proof}
由\refthe{theorem:环的特征的基本性质-1}, 只要证明, 如果整环$R$的单位元$e$关于加法的阶有限, 则它必为素数.

设$e$关于加法的阶为$n$. 显然$n > 1$. 假设$n = p_1p_2\cdots p_s$, $1 \leqslant p_i \leqslant n$且$p_i$都是素数.则
\begin{align*}
0=ne=(p_1p_2\cdots p_s)e=\left( p_1e \right) \cdot \left( p_2e \right) \cdots \left( p_se \right) .
\end{align*}
由$R$是整环和\rrefpro{proposition:整环的一些性质}{proposition:整环的一些性质-3}可知存在$i_0\in [1,s]\cap \mathbb{N}$,使$p_{i_0}e=0$. 因为$n$是使得$ne = 0$成立的最小正整数, 所以$p_{i_0} = n$. 因此$n$是素数.

\end{proof}

\begin{definition}[素域/素体]
不包含任何平凡子体的体称为\textbf{素体}或\textbf{素域},即子体只有自身的体.
\end{definition}

\begin{theorem}\label{theorem:每个体必包含唯一的一个素体为其子体}
设$K$是一个体,则$K$的所有子体之交就是$K$包含的唯一素域(素体),也是$K$的子体.
\end{theorem}
\begin{remark}
这个定理表明:每个体可以看成是某个素域(素体)的扩张.
\end{remark}
\begin{proof}
记$K$的所有子体之交为$R$,则由\rrefpro{proposition:域和体的一些显然的性质}{proposition:域和体的一些显然的性质-2}知$R$仍为$K$的子体.设$R_1$是$R$的子体且$R_1\subseteq R$,则$R_1$也是$K$的子体,从而由$R$的定义知$R\subseteq R_1$,故$R_1=R$.故$R$是$K$的素域.

若$K$还包含一个素域$R'$,则$R'$也是$K$的子体,从而$R'\supseteq R$.又因为$R'$是素域,所以$R'=R$.故唯一性得证.

\end{proof}

\begin{theorem}\label{theorem:抽象代数--定理3.1.1}
\begin{enumerate}[(1)]
\item $\mathbb{Z}_p,\mathbb{Q}$都是素域(素体).

\item 设$\Pi$是一个素域(素体),则$\Pi\cong\mathbb{Z}_p$($p$为素数)或$\Pi\cong\mathbb{Q}$.进而素域(素体)一定是域.
\end{enumerate}
\end{theorem}
\begin{proof}
\begin{enumerate}[(1)]
\item $\mathbb{Z}_p$对于加法是素数阶群.由\hyperref[theorem:Lagrange定理--抽象代数]{Lagrange定理}知$\mathbb{Z}_p$无非平凡子群,故$\mathbb{Z}_p$无非平凡子体,因而$\mathbb{Z}_p$是素域(素体).

若$F$为域$\mathbb{Q}$的子体,于是$1\in F$,从而$\mathbb{Z}\subset F$,由\refpro{proposition:mZ的分式域都是Q}知$\mathbb{Z}$的分式域是$\mathbb{Q}$.因而由\refthe{theorem:分式域是包含R的最小域且唯一}知$\mathbb{Q}\subseteq F$,故$F=\mathbb{Q}$,所以$\mathbb{Q}$为素域(素体).

\item 设$e$为$\Pi$的幺元,于是易知$\mathbb{Z}e=\{ne|n\in\mathbb{Z}\}$为$\Pi$的可交换子环且有$\mathbb{Z}$到$\mathbb{Z}e$的同态$\pi:\pi(n)=ne(n\in\mathbb{Z})$.于是由\hyperref[theorem:环的同态基本定理]{环的同态基本定理}知
\begin{align*}
\mathbb{Z}e\cong \mathbb{Z}/\ker\pi.
\end{align*}
由于$\mathbb{Z}$为Euclid环,进而也是主理想整环,故有$p\in \mathbb{Z}$,使得$\ker\pi=\langle p\rangle $.
因为$\Pi$为体,故由\rrefpro{proposition:域和体的一些显然的性质}{proposition:域和体的一些显然的性质-1}知$\mathbb{Z}e$为交换整环,即$\mathbb{Z}/\ker\pi=\mathbb{Z}/\langle p\rangle$也是交换整环.
因此由\rrefthe{theorem:抽象代数--定理2.8.1}{theorem:抽象代数--定理2.8.1-1}知$\langle p\rangle$为素理想,再由\refthe{theorem:抽象代数--定理2.8.3}知$p$为$\mathbb{Z}$中的素元素,进而$p$只能为素数或零.

当$p$为素数时,由\refpro{proposition:Z_p是域且非数域}知$\mathbb{Z}e\cong \mathbb{Z}/\langle p\rangle=\mathbb{Z}/p\mathbb{Z} = \mathbb{Z}_p$为域,从而$\mathbb{Z}e$是$\Pi$的子体.又$\Pi$无非平凡子体,故$\Pi=\mathbb{Z}e\cong\mathbb{Z}_p$.

当$p=0$时,有$\mathbb{Z}e\cong \mathbb{Z}/\langle 0\rangle=\mathbb{Z}$.由\refpro{proposition:mZ的分式域都是Q}知$\mathbb{Z}$的分式域是$\mathbb{Q}$.记$\mathbb{Z}e$的分式域为$F$,则由\refcor{corollary:分式域同构等价于生成环同构}知$F\cong \mathbb{Q}$.
又$\mathbb{Z}e\subset \Pi$,故由\refthe{theorem:分式域是包含R的最小域且唯一}知$F\subseteq \Pi$.再由$\Pi$是素域知$\Pi=F\cong\mathbb{Q}$.
\end{enumerate}

\end{proof}

\begin{definition}[体的特征]
若体$K$包含的素域与$\mathbb{Q}$同构,则称$K$的\textbf{特征}为零.若体$K$包含的素域与$\mathbb{Z}_p$($p$为素数)同构,则称$K$的\textbf{特征}为$p$.记$K$的特征为$\mathrm{ch}\,K$或$\text{Char}\,K$.
\end{definition}
\begin{remark}
由\refthe{theorem:每个体必包含唯一的一个素体为其子体}知$K$只包含唯一的素域,又由\refthe{theorem:抽象代数--定理3.1.1}知$K$的素域只可能同构于$\mathbb{Q}$或$\mathbb{Z}_p$($p$为素数),因此$\mathrm{ch}\,K$只能是$0$或素数.
\end{remark}

\begin{theorem}\label{theorem:抽象代数--定理3.1.2}
设$K$是一个体,$p$为素数,则
\begin{enumerate}[(1)]
\item $\mathrm{ch}\,K=p\iff pa=0,\forall a\in K$;

\item $\mathrm{ch}\,K=0\iff na\neq0,\forall n\in\mathbb{N},a\in K^*=K\setminus\{0\}$.
\end{enumerate}
\end{theorem}
\begin{proof}
记$K$的幺元为$e$,$K$中素域为$\Pi$.显然$\mathbb{Z}e=\{ne|n\in\mathbb{Z}\}$为$\Pi$的可交换子环且有$\mathbb{Z}$到$\mathbb{Z}e$的同态$\pi:\pi(n)=ne(n\in\mathbb{Z})$.于是由\hyperref[theorem:环的同态基本定理]{环的同态基本定理}知
\begin{align}\label{eq:::0fj290r2fsgehw3}
\mathbb{Z}e\cong \mathbb{Z}/\ker\pi.
\end{align}
由于$\mathbb{Z}$为Euclid环,进而也是主理想整环,故有$p'\in \mathbb{Z}$,使得$\ker\pi=\langle p'\rangle $.
因为$\Pi$为体,故由\rrefpro{proposition:域和体的一些显然的性质}{proposition:域和体的一些显然的性质-1}知$\mathbb{Z}e$为交换整环,即$\mathbb{Z}/\ker\pi=\mathbb{Z}/\langle p'\rangle$也是交换整环.
因此由\rrefthe{theorem:抽象代数--定理2.8.1}{theorem:抽象代数--定理2.8.1-1}知$\langle p'\rangle$为素理想,再由\refthe{theorem:抽象代数--定理2.8.3}知$p'$为$\mathbb{Z}$中的素元素,进而$p'$只能为素数或零.
\begin{enumerate}[(1)]
\item 若$\mathrm{ch}\,K=p$,即$\Pi\cong\mathbb{Z}_p$,又因为在$\mathbb{Z}_p$中有$p\cdot 1=0$,所以在$\Pi$中有$pe=0$,因而$pa=pe\cdot a=0,\forall a\in K$.

反之,若$pa=0,\forall a\in K$,则$pe=0$.从而对$\forall z\in \mathbb{Z}$,有$\pi \left( pz \right) =pze=z\cdot pe =0$.故$\langle p\rangle=p\mathbb{Z}\subseteq \ker \pi.$若$p'\neq p$,则$\langle p\rangle \nsubseteq \langle p' \rangle =\mathrm{ker}\pi $矛盾!因此$p=p'$,即
\begin{align*}
\ker \pi =\langle p\rangle.
\end{align*}
又因为$p$为素数,所以由\eqref{eq:::0fj290r2fsgehw3}式及\refpro{proposition:Z_p是域且非数域}知$\mathbb{Z}e\cong \mathbb{Z}/\langle p\rangle=\mathbb{Z}/p\mathbb{Z} = \mathbb{Z}_p$为域,从而$\mathbb{Z}e$是$\Pi$的子域,也是子体.又$\Pi$无非平凡子体,故$\Pi=\mathbb{Z}e\cong\mathbb{Z}_p$,即$\mathrm{ch}\,K=p$.

\item 若$\mathrm{ch}\,K=0$,即$\Pi \cong \mathbb{Q}$.记$\mathbb{Z}e$的分式域为$F$,又$\mathbb{Z}e\subset \Pi$,故由\refthe{theorem:分式域是包含R的最小域且唯一}知$F\subseteq \Pi$.再由$\Pi$是素域知$\Pi=F\cong \mathbb{Q}$.于是由\refcor{corollary:分式域同构等价于生成环同构}知$\mathbb{Z}\cong\mathbb{Z}e$.
因此由$n\cdot 1\neq 0,\forall n\in \mathbb{N}$知$ne\neq0,\forall n\in\mathbb{N}$.又由\rrefpro{proposition:域和体的一些显然的性质}{proposition:域和体的一些显然的性质-1}知$K$是整环,故$\forall a\in K^*$,$na=ne\cdot a\neq0$.

反之,$\forall n\in\mathbb{N},a\in K^*$有$na\neq0$.特别地,$\pi(n)=ne\neq0,\pi(-n)=-ne\neq 0$.于是$\ker \pi =\{0\}=\langle0\rangle$.
故由\eqref{eq:::0fj290r2fsgehw3}式知$\mathbb{Z}e\cong\mathbb{Z}/\langle0\rangle=\mathbb{Z}$,即$\mathrm{ch}\,K=0$.
\end{enumerate}

\end{proof}

\begin{corollary}\label{corollary:数域的特征都是零}
数域的特征都是零.
\end{corollary}
\begin{proof}


\end{proof}

\begin{definition}
设$K$为域$F$的扩域,$S$为$K$的子集.$K$中所有包含$F\cup S$的子域之交,称为\textbf{由$F$与$S$生成的子域},也称为\textbf{$F$上添加$S$所得的域},亦称为\textbf{$S$在$F$上生成的域},记为$F(S)$.

显然有$K=F(K)$,因而讨论域的扩张实质上是讨论在域上添加一个集合所得的域.为清楚起见,以$F[S]$表示下列形式的一切有限和:
\begin{align*}
\sum\limits_{i_1,i_2,\cdots,i_n\geqslant0}\alpha_{i_1i_2\cdots i_n}a_1^{i_1}a_2^{i_2}\cdots a_n^{i_n},
\end{align*}
其中$\alpha_j\in S$,$j=1,2,\cdots,n$,$\alpha_{i_1i_2\cdots i_n}\in F$所构成的集合,显然$F[S]$是$K$的子环,它的分式域恰为$F(S)$.特别当$S=\{\alpha_1,\alpha_2,\cdots,\alpha_n\}$为有限集时,分别记
\begin{align*}
F[S]=F[\alpha_1,\alpha_2,\cdots,\alpha_n],\quad F(S)=F(\alpha_1,\alpha_2,\cdots,\alpha_n).
\end{align*}
\end{definition}

\begin{theorem}
设$K$为域$F$的扩域,$S\subseteq K$,则
\begin{enumerate}[(1)]
\item $F(S)=\bigcup\limits_{S'\subseteq S}F(S')$,此处$S'$取遍$S$的所有有限子集;

\item 若$S=S_1\cup S_2$,则$F(S)=F(S_1)(S_2)$.
\end{enumerate}
\end{theorem}

\begin{proof}
\begin{enumerate}[(1)]
\item 显然$F(S')\subseteq F(S)$,因而
\begin{align*}
\bigcup\limits_{S'\subseteq S}F(S')\subseteq F(S).
\end{align*}
反之,$\forall a\in F(S)$有$a=\frac{f}{g}$,$f,g\in F[S]$,由于$f,g$的表达式均为有限和的形式,因而存在$S$的有限子集$S'_0$,使$f,g\in F[S'_0]$.于是$a=\frac{f}{g}\in F[S'_0]\subseteq\bigcup\limits_{S'\subseteq S}F(S')$,故结论(1)成立.

\item 由于$F(S_1\cup S_2)$是$K$中包含$F,S_1\cup S_2$的最小子域,而$F,S_1,S_2\subseteq F(S_1)(S_2)$,故有
\begin{align*}
F(S_1\cup S_2)\subseteq F(S_1)(S_2).
\end{align*}
反之,$F(S_1)(S_2)$是包含$F(S_1),S_2$的最小子域,而
\begin{align*}
F(S_1)\subseteq F(S_1\cup S_2),\quad S_2\subseteq F(S_1\cup S_2),
\end{align*}
故$F(S_1)(S_2)\subseteq F(S_1\cup S_2)$,因而结论(2)成立.
\end{enumerate}

\end{proof}

\begin{corollary}
$F(\alpha_1,\alpha_2,\cdots,\alpha_n)=F(\alpha_1)(\alpha_2)\cdots(\alpha_n)$.
\end{corollary}

从定理3.1.3及推论3.1.2可知在一个域上添加一个有限集合$S$可转化成添加有限个元素的问题,而添加有限个元素的问题,可转化成添加一个元素的问题.

\begin{definition}
设$K$是$F$的扩域.若$\exists\alpha\in K$,使得
\begin{align*}
K=F(\alpha),
\end{align*}
称$K$是$F$的单扩域.

若$\alpha$是$F$上的代数元,称$K=F(\alpha)$为$F$的单代数扩域.

若$\alpha$是$F$上的超越元,称$K=F(\alpha)$为$F$的单超越扩域.

从定理2.8.5可知当$\alpha$为超越元时,$F(\alpha)$同构于$F$上一元多项式环$F[x]$的分式域$F(x)$.由2.1节知此分式域存在且唯一,故一个域$F$的单超越扩张存在且唯一,因而$F(\alpha)$就是$F$上的一元多项式环的分式域.今后将侧重讨论单代数扩张的情形.

同样从定理2.8.5可知当$\alpha$为代数元时,
\begin{align*}
F(\alpha)\cong F(x)/\langle p(x)\rangle,
\end{align*}
其中,$p(x)$是$F[x]$中的不可约多项式且满足$p(\alpha)=0$.此时$F[\alpha]$是域,因而有$F[\alpha]=F(\alpha)$.由于$F$是域,故可知$p(x)$与一个首一多项式相伴,因而不妨设$p(x)$为首一多项式且这样的$p(x)$由$\alpha$唯一确定.
\end{definition}

\begin{definition}
设$K$是$F$的扩域,$\alpha\in K$,$\alpha$是$F$上的代数元.$F[x]$中以$\alpha$为根的不可约首一多项式称为$\alpha$在$F$上的不可约多项式,记为$\mathrm{Irr}(\alpha,F)$.它的次数称为$\alpha$在$F$上的次数,记为$\deg(\alpha,F)$,即$\deg(\alpha,F)=\deg(\mathrm{Irr}(\alpha,F))$.

由2.2节与2.8节的讨论知若$\alpha$是$F$上的代数元,则
\begin{align*}
\langle\mathrm{Irr}(\alpha,F)\rangle=\{f(x)\in F[x]|f(\alpha)=0\}=\{f(x)\in F[x]|\mathrm{Irr}(\alpha,F)|f(x)\}.
\end{align*}
\end{definition}

\begin{theorem}
设$F(\alpha)$是$F$的单代数扩张,又$\deg(\alpha,F)=n$,则$F(\alpha)$是$F$上的$n$维线性空间且$1,\alpha,\alpha^2,\cdots,\alpha^{n-1}$是一组基\textsuperscript{①}.
\end{theorem}

\begin{proof}
根据1.6节中域$F$上的线性空间的定义,可直接验证$F(\alpha)$是$F$上的线性空间.回忆2.2节曾指出在$F[x]$与$F[\alpha]=F(\alpha)$之间有满同态$\eta$满足
\begin{align*}
\eta(f(x))=f(\alpha),\quad \forall f(x)\in F[x],
\end{align*}
而
\begin{align*}
\ker\eta=\langle\mathrm{Irr}(\alpha,F)\rangle.
\end{align*}
由$\deg(\alpha,F)=n$,故$\exists q(x),r(x)\in F[x]$,使得
\begin{align*}
f(x)=q(x)\mathrm{Irr}(\alpha,F)+r(x),\quad \deg r(x)<\deg(\alpha,F)
\end{align*}
(注意$\deg0=-\infty$),因而$f(\alpha)=r(\alpha)$.于是$1,\alpha,\alpha^2,\cdots,\alpha^{n-1}$生成$F(\alpha)$.又若$\deg s(x)<\deg(\alpha,F)$,而$s(\alpha)=0$,则$\eta(s(x))=0$.故$\mathrm{Irr}(\alpha,F)|s(x)$,因而$s(x)=0$,即$1,\alpha,\alpha^2,\cdots,\alpha^{n-1}$线性无关,故$1,\alpha,\alpha^2,\cdots,\alpha^{n-1}$是$F(\alpha)$的一组基,故$F(\alpha)$的维数为$n$.

\end{proof}

\begin{definition}
设$K_1,K_2$都是域$F$的扩域.若有$K_1$到$K_2$上的同构$\eta$,使$\eta|_F=\mathrm{id}_F$,则称$K_1$与$K_2$是$F$的等价扩张,$\eta$称为$F$同构.特别地,若$K_1=K_2$,则称$\eta$为$F$自同构.
\end{definition}

\begin{example}
$F(\alpha),F(\beta)$都是$F$的单超越扩张,则$F(\alpha)$与$F(\beta)$是$F$的等价扩张.这时它们与一元多项式环$F[x]$的分式域都是$F$的等价扩张.
\end{example}

\begin{example}
$F(\alpha),F(\beta)$都是$F$的单代数扩张且$\mathrm{Irr}(\alpha,F)=\mathrm{Irr}(\beta,F)$,则$F(\alpha)$与$F(\beta)$是$F$的等价扩张.

事实上,记$p(x)=\mathrm{Irr}(\alpha,F)=\mathrm{Irr}(\beta,F)$,则$F(\alpha),F(\beta)$与$F[x]/\langle p(x)\rangle$都是$F$的等价扩张,故$F(\alpha)$与$F(\beta)$是$F$的等价扩张.
\end{example}

由此例知对$F[x]$中的任一不可约多项式$p(x)$在等价定义下存在唯一单代数扩张.事实上,$F[x]/\langle p(x)\rangle=F(x+\langle p(x)\rangle)$.令$\alpha=x+\langle p(x)\rangle$,则$\mathrm{Irr}(\alpha,F)$与$p(x)$相伴.

但是,对$F$的两个等价的单代数扩张$F(\alpha),F(\beta)$,不一定有$\mathrm{Irr}(\alpha,F)=\mathrm{Irr}(\beta,F)$.

\begin{example}
设$F=\mathbb{R}$,$\alpha=\sqrt{-1}$,$\beta=1+\sqrt{-1}$.显然$F(\alpha)=\mathbb{C}$,$F(\beta)=\mathbb{C}$,故$F(\alpha)$与$F(\beta)$是$F$的等价扩张,但$\mathrm{Irr}(\alpha,F)=x^2+1$,$\mathrm{Irr}(\beta,F)=x^2-2x+2$,故$\mathrm{Irr}(\alpha,F)\neq\mathrm{Irr}(\beta,F)$.
\end{example}

\begin{example}
定义$\mathbb{C}$到$\mathbb{C}$的映射$\tau$:
\begin{align*}
\tau(a+b\sqrt{-1})=a-b\sqrt{-1},\quad \forall a,b\in\mathbb{R},
\end{align*}
则容易验证$\tau$是$\mathbb{C}$的$\mathbb{R}$自同构.
\end{example}

比等价扩张更特殊一点的概念是共轭.

\begin{definition}
设$K,K_1,K_2$都是域$F$的扩域且
\begin{align*}
K\supseteq K_i\supseteq F,\quad i=1,2.
\end{align*}
若$K_1$与$K_2$是$F$等价扩张,则称$K_1,K_2$是$K$(对$F$)共轭的子域.

又若$\alpha,\beta\in K$且$\mathrm{Irr}(\alpha,F)=\mathrm{Irr}(\beta,F)$,则称$\alpha$与$\beta$是(对$F$)的共轭元素.

从例3.1.2知$\alpha,\beta$是共轭元素,则$F(\alpha)$与$F(\beta)$共轭.从例3.1.3知反之不成立.
\end{definition}



\end{document}
\documentclass[../../main.tex]{subfiles}
\graphicspath{{\subfix{../../image/}}} % 指定图片目录,后续可以直接使用图片文件名。

% 例如:
% \begin{figure}[H]
% \centering
% \includegraphics[scale=0.4]{图.png}
% \caption{}
% \label{figure:图}
% \end{figure}
% 注意:上述\label{}一定要放在\caption{}之后,否则引用图片序号会只会显示??.

\begin{document}

\section{多项式环}

\begin{theorem}
设\( \widetilde{R} \)是一个交换幺环,\( R \)是\( \widetilde{R} \)的子环且\( 1 \in R \)。又设\( u \in \widetilde{R} \),\( \widetilde{R} \)中由\( R \)与\( u \)生成的子环,即包含\( R \)与\( u \)的最小子环记为\( R[u] \)。则
\[
R[u] = \{a_0 + a_1 u + \cdots + a_n u^n \mid a_i \in R, n \in \mathbf{N}\cup\{0\}\},
\]
也称\( R[u] \)为\textbf{\( R \)上添加\( u \)生成的子环}。
\end{theorem}
\begin{proof}
记 $S = \{a_0 + a_1 u + \cdots + a_n u^n \mid a_i \in R, \,  n \in \mathbf{N}\cup\{0\}\}$。
首先证明 $S \subseteq R[u]$.
由于 $R[u]$ 是包含 $R$ 和 $u$ 的子环,而 $S$ 中的所有元素都可以通过有限次运算(加法、乘法、取逆)从 $R$ 和 $u$ 得到,因此 $S \subseteq R[u]$。

接下来证明 $R[u] \subseteq S$.
设 $f(u) = a_0 + a_1 u + \cdots + a_m u^m \in S$,$g(u) = b_0 + b_1 u + \cdots + b_n u^n \in S$,不妨设$m\leqslant n$,再令$a_{m+1}=\cdots=a_n=0$,则
$$f(u) + g(u) = \sum_{i=0}^n (a_i + b_i)u^i \in S.$$
令$-f(u) \triangleq  (-a_0) + (-a_1)u + \cdots + (-a_m)u^m \in S,$则$f(u)+(-f(u))=0.$因此$S$对加法封闭且有加法逆元.又$\widetilde{R}$是交换幺环且$S\subseteq \widetilde{R}$,故$S$对加法满足结合律和交换律.于是$S$对加法构成$\widetilde{R}$的Abel群.

由于 $\widetilde{R}$ 是交换环,故
$$
f\left( u \right) g\left( u \right) =\left( \sum_{i=1}^n{a_iu^i} \right) \left( \sum_{i=1}^n{b_iu^i} \right) =\sum_{k=0}^{m+n}{\left( \sum_{i+j=k}{a_ib_j} \right) u^k}\in S.
$$
令$a_0=1,n=0$,则有$1\in S.$
因此$S$对乘法封闭且含幺元$1$.又$\widetilde{R}$是交换幺环且$S\subseteq \widetilde{R}$,故$S$对乘法满足结合律.于是$S$对乘法构成$\widetilde{R}$的幺半群.
由于$S$对加法和乘法封闭,$\widetilde{R}$为交换幺环且$S\subseteq \widetilde{R}$,故$S$的加法与乘法间自然满足分配律.
因此$S$是交换幺环$\widetilde{R}$的子环.

对于任意 $r \in R$,可取 $r = r + 0\cdot u + 0\cdot u^2 + \cdots \in S$,故 $R \subseteq S$.
同时 $u = 0 + 1\cdot u + 0\cdot u^2 + \cdots \in S$。
再设 $T$ 是 $\widetilde{R}$ 的任一包含 $R$ 和 $u$ 的子环,则 $T$ 必然包含所有的 $a_i u^i$ ($a_i \in R$) 以及它们的有限和,即 $S \subseteq T$。
因此$S$ 是包含 $R$ 和 $u$ 的最小子环.

综上可知$R[u] = S$。

\end{proof}

\begin{definition}
如果在\( R \)中存在有限多个元素\( a_0, a_1, \cdots, a_n \)且\( a_n \neq 0 \),使得
\[
a_0 + a_1 u + \cdots + a_n u^n = 0,
\]
那么称\( u \)为\( R \)上的\textbf{代数元},使上述关系成立的最小正整数\( n \)称为代数元\( u \)的\textbf{次数},记为\( \deg(u, R) \)。
\end{definition}

\begin{example}
令$\widetilde{R}=\mathbf{C}$,则$\sqrt{-1}$为$\mathbf{Z}$上的代数元,
\begin{align*}
\mathbf{Z}[\sqrt{-1}]=\{m+n\sqrt{-1}\mid m,n\in\mathbf{Z}\}
\end{align*}
称为$\mathbf{Gauss}$\textbf{的整数环},$\deg(\sqrt{-1},\mathbf{Z})=2$。
同样$\sqrt{-1}$为$\mathbf{Q}$上的代数元,$\deg(\sqrt{-1},\mathbf{Q})=2$。
\end{example}
\begin{proof}


\end{proof}

\begin{example}
令$\widetilde{R}=\mathbf{Q}$,则$\frac{1}{2}$是$\mathbf{Z}$上代数元且$\mathbf{Z}\subset\mathbf{Z}\left[\frac{1}{2}\right]\subset\mathbf{Q}$,$\deg\left(\frac{1}{2},\mathbf{Z}\right)=1$。
\end{example}
\begin{proof}


\end{proof}

\begin{definition}
设$R$是交换幺环$\widetilde{R}$的包含幺元$1$的子环,\( u \in \widetilde{R} \),$R[u]$为$R$添加$u$生成的$\widetilde{R}$的子环,若满足$a_0,a_1,\cdots,a_n$不全为0时,
\begin{align*}
a_0+a_1u+\cdots+a_nu^n\neq 0,
\end{align*}
则称$u$为$R$上的\textbf{超越元}或\textbf{不定元}。$R[u]$中的一个元素$f(u)=a_0+a_1u+\cdots+a_nu^n$称为$u$的(系数在$R$中的)一个\textbf{多项式}。若$a_n\neq 0$,则称$n$为$f(u)$的次数,记为$\deg f(u)$。$R[u]$称为$R$上的一个\textbf{一元多项式环}。
\end{definition}

\begin{example}
设$\mathbf{P}$是一个数域,$x$是一个文字,则$\mathbf{P}[x]$是$\mathbf{P}$上的一个一元多项式环,$x$是$\mathbf{P}$上的超越元。
\end{example}
\begin{proof}


\end{proof}

\begin{theorem}\label{theorem:抽象代数--定理2.2.1}
交换幺环$R$上的一元多项式环一定存在.
\end{theorem}
\begin{proof}
令
\begin{align*}
\widetilde{R}=\{(a_0,a_1,\cdots)\mid a_i\in R\text{ 且仅有有限个 }a_i\neq 0\}.
\end{align*}
自然$\widetilde{R}$中元素$(a_0,a_1,\cdots)=(b_0,b_1,\cdots)$当且仅当$a_i=b_i(i=0,1,\cdots)$. 在$\widetilde{R}$中定义加法与乘法
\begin{gather}
(a_0,a_1,\cdots)+(b_0,b_1,\cdots)=(a_0+b_0,a_1+b_1,\cdots),\label{eq::::---03j8g4yu46yrsu64wijh2.2.1}\\
(a_0,a_1,\cdots)\cdot(b_0,b_1,\cdots)=(c_0,c_1,\cdots).\label{eq::::---03j8g4yu46yrsu64wijh2.2.2}
\end{gather}
其中,
\begin{align}
c_n&=a_0b_n+a_1b_{n-1}+\cdots+a_{n-1}b_1+a_nb_0\nonumber\\
&=\sum\limits_{i+j=n}a_ib_j,\quad n=0,1,\cdots.\label{eq::::---03j8g4yu46yrsu64wijh2.2.3}
\end{align}
由于$(a_0,a_1,\cdots),(b_0,b_1,\cdots)\in\widetilde{R}$,故$\exists m\in\mathbf{N}$,使$n>m$时,$a_n=b_n=0$. 于是$a_n+b_n=0$,故$(a_0+b_0,a_1+b_1,\cdots)\in \widetilde{R}$.而当$n>2m$时,$c_n=\sum\limits_{i+j=n}a_ib_j=0$,故$(c_0,c_1,\cdots)\in \widetilde{R}$.由此知上面定义的加法与乘法是良定义的.

容易验证$\widetilde{R}$对加法为Abel群,它的零元素为$0=(0,0,\cdots)$且$-(a_0,a_1,\cdots)=(-a_0,-a_1,\cdots)$.
同样容易验证$\widetilde{R}$对乘法是可交换的且有幺元$(1,0,\cdots)$. 下面验证乘法的结合律. 设
\begin{align*}
f=(a_0,a_1,\cdots),\quad g=(b_0,b_1,\cdots),\quad h=(c_0,c_1,\cdots),
\end{align*}
则$(fg)h$的第$k$个元素为
\begin{align*}
\sum\limits_{s+r=k}\left(\sum\limits_{i+j=s}a_ib_j\right)c_r=\sum\limits_{i+j+r=k}a_ib_jc_r
=\sum\limits_{i+t=k}a_i\left(\sum\limits_{j+r=t}b_jc_r\right),
\end{align*}
这也是$f(gh)$的第$k$个元素. 故$\widetilde{R}$对乘法为交换幺半群.
又注意到$(f+g)h$的$k$个元素为
\begin{align*}
\sum_{i+j=k}{\left( a_i+b_i \right) c_j}=\sum_{i+j=k}{a_ic_j}+\sum_{i+j=k}{b_ic_j},
\end{align*}
这也是$fh+gh$的第$k$个元素.$h(f+g)$的$k$个元素为
\begin{align*}
\sum_{i+j=k}{c_i\left( a_j+b_j \right)}=\sum_{i+j=k}{c_ia_j}+\sum_{i+j=k}{c_ib_j},
\end{align*}
这也是$hf+hg$的第$k$个元素.
因此$\widetilde{R}$中加法与乘法间的分配律成立,故$\widetilde{R}$为交换幺环.

令$R_0=\{(a_0,0,0,\cdots):a_0\in R\}$,则$R_0$显然是$R$的子环.由
\begin{align*}
(a_0,0,\cdots)+(b_0,0,\cdots)&=(a_0+b_0,0,\cdots),\\
(a_0,0,\cdots)\cdot(b_0,0,\cdots)&=(a_0b_0,0,\cdots)
\end{align*}
知$a_0\to(a_0,0,\cdots)$是$R$到$R_0$上的同构映射. 为方便计,将$R_0$中元素$(a_0,0,\cdots)$记为$a_0$,即可将$R$视为$\widetilde{R}$的子环. $R$的幺元$1$恰为$\widetilde{R}$的幺元$(1,0,\cdots)$.

最后证明$\widetilde{R}$是$R$上的一元多项式环. 令
\begin{align*}
u=(0,1,0,\cdots),
\end{align*}
则不难验证
\begin{align*}
u^k&=(\underbrace{0,\cdots,0}_{k},1,0,\cdots),\\
a_ku^k&=(\underbrace{0,\cdots,0}_{k},a_k,0,\cdots),\quad a_k\in R=R_0.
\end{align*}
若$f=(a_0,a_1,\cdots)\in\widetilde{R}$,则有$n$,使$a_{n+1}=a_{n+2}=\cdots=0$. 于是
\begin{align*}
f=a_0+a_1u+\cdots+a_nu^n,
\end{align*}
因而有$\widetilde{R}=R_0[u]=R[u]$. 又若
\begin{align*}
a_0+a_1u+\cdots+a_nu^n=0,
\end{align*}
即
\begin{align*}
(a_0,a_1,\cdots,a_n,0,\cdots)=(0,0,\cdots),
\end{align*}
则$a_0=a_1=\cdots=a_n=0$,即$u$是$R$上的超越元,因而$\widetilde{R}=R[u]$是$R$上的一元多项式环.

\end{proof}

\begin{theorem}\label{theorem:抽象代数--定理2.2.2}
设$R,S$都是交换幺环,它们的幺元分别是$1,1'$. 又若$\eta$是$R$到$S$的同态且$\eta(1)=1'$,则$\forall u\in S$,$\eta$可唯一地扩充为$R$上的一元多项式环$R[x]$到$S$的同态$\eta_u$,使得
\begin{align*}
\eta_u(x)=u.
\end{align*}
即对$\forall u\in S$,$\eta$存在唯一的在$R$上的开拓$\eta_u:R[x]\to S$满足
\begin{align}\label{eq::::---03j8g4yu46yrsu64wijh2.2.4}
\eta _u|_R=\eta ,\quad \eta _u(x)=u.
\end{align}
且$\eta_u$是环同态.
\end{theorem}
\begin{proof}
因$R[x]$为$R$上的一元多项式环,故$R[x]=\{a_0+a_1x+\cdots+a_nx^n\mid a_i\in R\}$. 定义$\eta_u$,
\begin{align}
\eta_u(a_0+a_1x+\cdots+a_nx^n)=\eta(a_0)+\eta(a_1)u+\cdots+\eta(a_n)u^n\label{eq::::---03j8g4yu46yrsu64wijh2.2.5}
\end{align}
于是$\eta_u$是$R[x]$到$S$的映射. 直接计算可知$\eta_u$为满足式\eqref{eq::::---03j8g4yu46yrsu64wijh2.2.4}的扩充,并为同态映射.

现设$\eta'$也是$\eta$的扩充且$\eta'(x)=u$,于是
\begin{align*}
\eta \prime \left( \sum_{i=0}^n{a_ix^i} \right) =\sum_{i=0}^n{\eta \prime (a_i)u^i}=\sum_{i=0}^n{\eta (a_i)u^i}=\eta _u\left( \sum_{i=0}^n{a_ix^i} \right) ,
\end{align*}
故$\eta'=\eta_u$,即$\eta_u$是满足条件的唯一扩充.

\end{proof}

\begin{corollary}
设$R$是交换幺环,$R[x]$与$R[y]$都是$R$上的一元多项式环,则$R[x]$与$R[y]$是同构的.
\end{corollary}
\begin{note}
这个推论说明:任何交换幺环上的一元多项式环在同构意义下唯一.
\end{note}
\begin{proof}
事实上,容易验证$R$到$R[y]$的嵌入映射$i(a)=a(\forall a\in R)$是$R$到$R[y]$的环同态,于是由\refthe{theorem:抽象代数--定理2.2.2}知有$R[x]$到$R[y]$的同态$i_y$满足
\begin{align*}
i_y|_R=i,\quad i_y(x)=y.
\end{align*}
从而任取$a_0+a_1y+\cdots+a_ny^n\in R[y]$,都有
\begin{align*}
i_y(a_0+a_1x+\cdots+a_nx^n)=a_0+a_1y+\cdots+a_ny^n,
\end{align*}
故$i_y$是满同态. 由$y$是$R$上超越元知$\ker\,i_y=\{0\}$,因此由\refpro{proposition:单同态的充要条件是核为平凡零子空间}知$i_y$是单同态. 故$i_y$是同构映射.

\end{proof}

\begin{corollary}\label{corollary:抽象代数--推论2.2.2}
设$R$是交换幺环$\widetilde{R}$的包含幺元$1$的子环,$R[x]$为$R$上的一元多项式环,又设$u\in\widetilde{R}$,则有$R[x]$中的理想$I$满足$R\cap I=\{0\}$,$R[u]\cong R[x]/I$,并且当且仅当$I\neq\{0\}$时,$u$为代数元.
\end{corollary}
\begin{proof}
考虑$R$到$R[u]$的嵌入映射$i$,则不难验证$i$是$R$到$R[u]$上的同态. 于是由\refthe{theorem:抽象代数--定理2.2.2}知可将$i$扩充为环同态$i_u:R[x]\to R[u]$满足
\begin{align*}
i_u|_R=i,\quad i_u(x)=u.
\end{align*}
注意到$i_u(R[x])=R[u]$,故$i_u$是满同态.于是由\hyperref[theorem:环的同态基本定理]{环的同态基本定理}知$I=\ker i_u$为$R[x]$中理想,$R[u]\cong R[x]/I$. 又若$a\in R\cap I$,则$0=i_u(a)=i(a)=a$,故$R\cap I=\{0\}$. 由于$u$为$R$上代数元当且仅当存在$a_n\ne 0$,使得$\sum\limits_{i=0}^n a_iu^i=0$.这也当且仅当
\begin{align*}
i_u\left(\sum\limits_{i=0}^n a_ix^i\right)=\sum\limits_{i=0}^n a_iu^i=0 \iff 0\ne \sum\limits_{i=0}^n a_ix^i\in I \iff I\neq\{0\}.
\end{align*}

\end{proof}

\begin{corollary}\label{corollary:抽象代数--推论2.2.3}
设$R$是交换幺环,$R[x]$是$R$上一元多项式环. 又若$I$是$R[x]$的理想且$R\cap I=\{0\}$,$I\neq\{0\}$,则$R[x]/I$是$R$添加一个代数元所得的环.
\end{corollary}
\begin{proof}
设$\pi$是$R[x]$到$R[x]/I$的自然同态,于是$\pi(R)$是$R[x]/I$中的子环. 由\rrefthe{theorem:子环和理想的基本性质}{theorem:子环和理想的基本性质-2}知$I$也是$R$的理想,从而再由\rrefthe{theorem:抽象代数--定理1.7.6}{theorem:抽象代数--定理1.7.6-3}知
\begin{align*}
\pi (R)=R/I=\left( R+I \right) /I\cong R/(R\bigcap{I)}=R/\{0\}=R+0=R,
\end{align*}
故可将$R$视为$R[x]/I$的子环,令$u=\pi(x)$,则$u\in R[x]/I$,于是$R[u]\subseteq R[x]/I$.注意到
\begin{align*}
\pi(a_0+a_1x+\cdots+a_nx^n)=\pi(a_0)+\pi(a_1)u+\cdots+\pi(a_n)u^n,
\end{align*}
故再结合$\pi$是满同态可得
$$
R[x]/I=\pi(R[x])\subseteq R[u]\subseteq R[x]/I,
$$
即$R[x]/I=R[u]$. 又由$I\neq\{0\}$,故$I$中有非零元素$a_0+a_1x+\cdots+a_nx^n$,其中$a_n\ne 0$,又因为$\pi(R)\cong R$,所以$\pi(a_n)\ne 0$.而
\begin{align*}
\pi (a_0+a_1x+\cdots +a_nx^n)=\pi \left( a_0 \right) +\pi \left( a_1 \right) u+\cdots +\pi \left( a_n \right) u^n=0,
\end{align*}
故$u$为$R$上的代数元.

\end{proof}

\begin{theorem}
设 \( R \) 是交换幺环 \( \widetilde{R} \) 的包含幺元 \( 1 \) 的子环. 又设 \( u_1, u_2, \cdots, u_n \in \widetilde{R} \), 则 \( \widetilde{R} \) 中包含 \( R \) 与 \( u_1, u_2, \cdots, u_n \) 的最小子环为
\[
R[u_1, u_2, \cdots, u_n] = \left\{ \sum_{k_1 k_2 \cdots k_n} a_{k_1 k_2 \cdots k_n} u_1^{k_1} u_2^{k_2} \cdots u_n^{k_n} \bigg| a_{k_1 k_2 \cdots k_n} \in R, a_{k_1 k_2 \cdots k_n} \text{ 中仅有限个不为 } 0 \right\}
\]
称为 \textbf{\( R \) 添加 \( u_1, u_2, \cdots, u_n \) 所得的环}.
\end{theorem}
\begin{proof}
记
\[
S  = \left\{ \sum_{k_1, k_2, \ldots, k_n} a_{k_1 k_2 \cdots k_n} u_1^{k_1} u_2^{k_2} \cdots u_n^{k_n} \bigg| a_{k_1 k_2 \cdots k_n} \in R, \text{仅有限个 } a_{k_1 k_2 \cdots k_n} \neq 0 \right\}.
\]
首先,证明 \( S \) 是 \( \widetilde{R} \) 的子环.
设
\[
x = \sum_{k_1, \ldots, k_n} a_{k_1 \cdots k_n} u_1^{k_1} \cdots u_n^{k_n}, \quad y = \sum_{k_1, \ldots, k_n} b_{k_1 \cdots k_n} u_1^{k_1} \cdots u_n^{k_n} \in S,
\]
则
\[
x + y = \sum_{k_1, \ldots, k_n} (a_{k_1 \cdots k_n} + b_{k_1 \cdots k_n}) u_1^{k_1} \cdots u_n^{k_n}.
\]
由于 \( a_{k_1 \cdots k_n} \) 和 \( b_{k_1 \cdots k_n} \) 中仅有限个非零,故 \( a_{k_1 \cdots k_n} + b_{k_1 \cdots k_n} \) 中也仅有限个非零,因此 \( x + y \in S \)。并且有
\[
-x \triangleq  \sum_{k_1, \ldots, k_n}(-a_{k_1 \cdots k_n}) u_1^{k_1} \cdots u_n^{k_n} \in S,
\]
使得$x+(-x)=0.$因此$S$对加法封闭且有加法逆元.又因为$\widetilde{R}$是交换幺环且$S\subseteq \widetilde{R}$,所以$S$对加法也有结合律和交换律.故$S$对加法构成Abel群.

由于 \( \widetilde{R} \) 交换,有
\[
xy = \left( \sum_{k_1, \ldots, k_n} a_{k_1 \cdots k_n} u_1^{k_1} \cdots u_n^{k_n} \right) \left( \sum_{l_1, \ldots, l_n} b_{l_1 \cdots l_n} u_1^{l_1} \cdots u_n^{l_n} \right)
= \sum_{k_1, \ldots, k_n, l_1, \ldots, l_n} a_{k_1 \cdots k_n} b_{l_1 \cdots l_n} u_1^{k_1 + l_1} \cdots u_n^{k_n + l_n}.
\]
令 \( m_i = k_i + l_i \),则
\[
xy = \sum_{m_1, \ldots, m_n} \left( \sum_{k_1 + l_1 = m_1, \ldots, k_n + l_n = m_n} a_{k_1 \cdots k_n} b_{l_1 \cdots l_n} \right) u_1^{m_1} \cdots u_n^{m_n}.
\]
由于 \( a_{k_1 \cdots k_n} \) 和 \( b_{l_1 \cdots l_n} \) 中仅有限个非零,故\( xy \in S \)。
取 \( a_{0 \cdots 0} = 1 \in R \),其余系数为 0,则 \( 1 = 1 \cdot u_1^0 \cdots u_n^0 \in S \)。因此$S$对乘法封闭且含幺元.又因为$\widetilde{R}$是交换幺环且$S\subseteq \widetilde{R}$,所以$S$对乘法也有结合律和交换律.故$S$对乘法构成交换幺半群.
由于$S$对加法和乘法封闭,$\widetilde{R}$为交换幺环且$S\subseteq \widetilde{R}$,故$S$的加法与乘法间自然满足分配律.
因此$S$是交换幺环$\widetilde{R}$的子环.

其次,证明 \( S \) 包含 \( R \) 和 \( u_1, u_2, \cdots, u_n \).
对任意 \( a \in R \),取 \( a_{0 \cdots 0} = a \),其余系数为 0,则 \( a = a \cdot u_1^0 \cdots u_n^0 \in S \)。
对每个 \( u_i \),取 \( k_i = 1 \),其余指数为 0,且$a_{0\cdots k_i\cdots 0}=1$,其余系数均为0,则 \( u_i = 1 \cdot u_1^0 \cdots u_i^1 \cdots u_n^0 \in S \)。

最后,证明 \( S \) 是包含 \( R \) 和 \( u_1, u_2, \cdots, u_n \) 的最小子环.设 \( T \) 是 \( \widetilde{R} \) 的任意子环,且 \( T \) 包含 \( R \) 和 \( u_1, u_2, \cdots, u_n \)。由于 \( T \) 对乘法封闭,对任意非负整数 \( k_1, \ldots, k_n \),有 \( u_1^{k_1} \cdots u_n^{k_n} \in T \)。又因 \( T \) 包含 \( R \),对任意 \( a_{k_1 \cdots k_n} \in R \),有 \( a_{k_1 \cdots k_n} u_1^{k_1} \cdots u_n^{k_n} \in T \)。再由 \( T \) 对加法封闭,任意有限和 \( \sum a_{k_1 \cdots k_n} u_1^{k_1} \cdots u_n^{k_n} \in T \)。又 \( S \) 中元素均为此类有限和(因系数仅有限个非零),故 \( S \subseteq T \)。因此 \( S \) 是包含 \( R \) 和 \( u_1, u_2, \cdots, u_n \) 的最小子环。

综上,\( S = R[u_1, u_2, \cdots, u_n] \) 即为所求。

\end{proof}

\begin{definition}
如果 \( R \) 中有有限多个 \( a_{k_1 k_2 \cdots k_n} \neq 0 \), 使
\[
\sum_{k_1 k_2 \cdots k_n} a_{k_1 k_2 \cdots k_n} u_1^{k_1} u_2^{k_2} \cdots u_n^{k_n} = 0,
\]
则称 \( u_1, u_2, \cdots, u_n \) 在 \( R \) 上是\textbf{代数相关}的, 否则称 \( u_1, u_2, \cdots, u_n \) 在 \( R \) 上是\textbf{代数无关}的.

若 \( u_1, u_2, \cdots, u_n \) 在 \( R \) 上是代数无关的, 则称 \( R[u_1, u_2, \cdots, u_n] \) 为 \( R \) 上的\textbf{$n$元多项式环}, 其元素称为 \( R \) 上的\textbf{$n$元多项式}.

交换幺环 \( R \) 上的 \( n \) 元多项式环\( R[x_1, x_2, \cdots, x_n] \)中, 形如 \( a x_1^{k_1} x_2^{k_2} \cdots x_n^{k_n} (a \in R, a \neq 0) \) 的元素称为一个\textbf{单项式}, \( a \) 称为此单项式的\textbf{系数}, \( k_1 + k_2 + \cdots + k_n \) 称为此单项式的\textbf{次数}.

$n$元多项式 \( \sum_{k_1 k_2 \cdots k_n} a_{k_1 k_2 \cdots k_n} x_1^{k_1} x_2^{k_2} \cdots x_n^{k_n} \neq 0 \)的\textbf{次数}定义为所含单项式的最高次数, 即
\begin{align*}
\deg \left( \sum_{k_1 k_2 \cdots k_n} a_{k_1 k_2 \cdots k_n} x_1^{k_1} x_2^{k_2} \cdots x_n^{k_n} \right) &= \max \{ k_1 + k_2 + \cdots + k_n \mid a_{k_1 k_2 \cdots k_n} x_1^{k_1} x_2^{k_2} \cdots x_n^{k_n} \neq 0 \}.
\end{align*}
\end{definition}

\begin{theorem}
交换幺环 \( R \) 上的 \( n \) 元多项式环一定存在.
\end{theorem}
\begin{proof}
对 \( n \) 用数学归纳法证明.
当 \( n = 1 \) 时, 由\refthe{theorem:抽象代数--定理2.2.1}知本定理成立.
现设 \( n-1 \) 时本定理成立, 即有 \( R \) 上的 \( n-1 \) 元多项式环 \( R[x_1, x_2, \cdots, x_{n-1}] \). 这也是交换幺环且
\[
1 \in R \subset R[x_1, x_2, \cdots, x_{n-1}].
\]
再由\refthe{theorem:抽象代数--定理2.2.1}, 可构造 \( R[x_1, x_2, \cdots, x_{n-1}] \) 上的一元多项式环
\[
(R[x_1, x_2, \cdots, x_{n-1}])[x_n] \supset R[x_1, x_2, \cdots, x_{n-1}] \supset R \ni 1.
\]
显然有
\[
R[x_1, x_2, \cdots, x_{n-1}, x_n] \subseteq (R[x_1, x_2, \cdots, x_{n-1}])[x_n].
\]
若 \( f \in (R[x_1, x_2, \cdots, x_{n-1}])[x_n] \), 于是有 \( f_0, f_1, \cdots, f_k \in R[x_1, x_2, \cdots, x_{n-1}] \), 使得
\[
f = f_0 + f_1 x_n + \cdots + f_k x_n^k,
\]
而
\[
f_i = \sum_{k_1 k_2 \cdots k_{n-1}} a_{k_1 k_2 \cdots k_{n-1} i} x_1^{k_1} x_2^{k_2} \cdots x_{n-1}^{k_{n-1}}.
\]
于是 \( f \in R[x_1, x_2, \cdots, x_{n-1}, x_n] \), 故知
\[
R[x_1, x_2, \cdots, x_{n-1}, x_n] = (R[x_1, x_2, \cdots, x_{n-1}])[x_n].
\]

下面证明 \( x_1, x_2, \cdots, x_{n-1}, x_n \) 在 \( R \) 上是代数无关的. 假设$R$中有有限多个 \( a_{k_1 k_2 \cdots k_n} \ne 0 \), 使
\[
\sum_{k_1 k_2 \cdots k_n} a_{k_1 k_2 \cdots k_n} x_1^{k_1} x_2^{k_2} \cdots x_n^{k_n} = 0.
\]
记$m\triangleq \max \left\{ k_n:a_{k_1\cdots k_n}\ne 0 \right\}$,令
\[
f_i = \sum_{k_1 k_2 \cdots k_{n-1}} a_{k_1 k_2 \cdots k_{n-1} i} x_1^{k_1} x_2^{k_2} \cdots x_{n-1}^{k_{n-1}}, \quad 0 \leqslant i \leqslant m,
\]
则有 \( f_i \in R[x_1, x_2, \cdots, x_{n-1}] \) 且满足 \( \sum_{i=0}^m{f_ix_{n}^{i}}=0 \). 由于 \( x_n \) 是 \( R[x_1, x_2, \cdots, x_{n-1}] \) 上的超越元, 故有 \( f_i = 0 \), 即
\[
\sum_{k_1 k_2 \cdots k_{n-1}} a_{k_1 k_2 \cdots k_{n-1} i} x_1^{k_1} x_2^{k_2} \cdots x_{n-1}^{k_{n-1}} = 0, \quad 0 \leqslant i \leqslant m.
\]
由于 \( x_1, x_2, \cdots, x_{n-1} \) 在 \( R \) 上是代数无关的, 故$a_{k_1 k_2 \cdots k_{n-1} i} = 0.$
这样证明了 \( R[x_1, x_2, \cdots, x_n] \) 是 \( R \) 上的 \( n \) 元多项式环.

\end{proof}

\begin{theorem}\label{theorem:n元多项式环开拓定理}
设 \( R[x_1, x_2, \cdots, x_n] \) 是交换幺环 \( R \) 上的 \( n \) 元多项式环, \( S \) 是一个交换幺环, \( \eta \) 是 \( R \) 到 \( S \) 的环同态映射且 \( \eta(1) = 1' \), 其中, \( 1, 1' \) 分别为 \( R, S \) 的幺元. 又设 \( u_1, u_2, \cdots, u_n \in S \), 则 \( \eta \) 可唯一地开拓为 \( R[x_1, x_2, \cdots, x_n] \) 到 \( S \) 的同态 \( \eta_n \), 使得
\[
\eta_n(x_i) = u_i, \quad i = 1, 2, \cdots, n.
\]
即对$\forall u_1, u_2, \cdots, u_n \in S$,$\eta$存在唯一的在$R$上的开拓$\eta_u:R[x_1,x_2,\cdots,x_n]\to S$满足
\begin{align*}
\eta_n|_{R}=\eta,\quad \eta_n(x_i) = u_i, \quad i = 1, 2, \cdots, n.
\end{align*}
且$\eta_n$是环同态.
\end{theorem}

\begin{proof}
事实上, \( \eta_n \) 可定义为
\begin{align*}
\eta_n \left( \sum_{k_1 k_2 \cdots k_n} a_{k_1 k_2 \cdots k_n} x_1^{k_1} x_2^{k_2} \cdots x_n^{k_n} \right) &= \sum_{k_1 k_2 \cdots k_n} \eta(a_{k_1 k_2 \cdots k_n}) u_1^{k_1} u_2^{k_2} \cdots u_n^{k_n}.
\end{align*}
不难验证 \( \eta_n \) 是 \( R[x_1, x_2, \cdots, x_n] \) 到 \( S \) 的同态映射且 \( \eta_n(x_i) = u_i (i = 1, 2, \cdots, n) \).

又若 \( \eta' \) 也满足此性质, 则
\begin{align*}
\eta' \left( \sum_{k_1 k_2 \cdots k_n} a_{k_1 k_2 \cdots k_n} x_1^{k_1} x_2^{k_2} \cdots x_n^{k_n} \right) &= \sum_{k_1 k_2 \cdots k_n} \eta'(a_{k_1 k_2 \cdots k_n}) \eta'(x_1)^{k_1} \eta'(x_2)^{k_2} \cdots \eta'(x_n)^{k_n} \\
&= \sum_{k_1 k_2 \cdots k_n} \eta(a_{k_1 k_2 \cdots k_n}) u_1^{k_1} u_2^{k_2} \cdots u_n^{k_n} \\
&= \eta_n \left( \sum_{k_1 k_2 \cdots k_n} a_{k_1 k_2 \cdots k_n} x_1^{k_1} x_2^{k_2} \cdots x_n^{k_n} \right).
\end{align*}
由此可知定理成立.

\end{proof}

\begin{corollary}
交换幺环 \( R \) 上的任意两个 \( n \) 元多项式环是同构的.
\end{corollary}
\begin{note}
这个推论说明:任何交换幺环上的$n$元多项式环在同构意义下唯一.
\end{note}
\begin{proof}
设 \( R[x_1, x_2, \cdots, x_n] \) 与 \( R[y_1, y_2, \cdots, y_n] \) 为 \( R \) 上的两个 \( n \) 元多项式环. 令$i$为\( R \) 到 \( R[y_1, y_2, \cdots, y_n] \) 的嵌入映射,满足\( i(a) = a (\forall a \in R) \).容易验证$i$是\( R \) 到 \( R[y_1, y_2, \cdots, y_n] \) 的环同态映射.
由\refthe{theorem:n元多项式环开拓定理}知,可将$i$开拓为\( R[x_1, x_2, \cdots, x_n] \) 到 \( R[y_1, y_2, \cdots, y_n] \) 上的同态\( i_n \), 使 
\begin{align*}
i_n|_{R[x_1, x_2, \cdots, x_n]} = i ,\quad i_n(x_k) = y_k (k = 1, 2, \cdots, n).
\end{align*}
任取$\sum_{k_1 k_2 \cdots k_n} a_{k_1 k_2 \cdots k_n} y_1^{k_1} y_2^{k_2} \cdots y_n^{k_n} \in  R[y_1, y_2, \cdots, y_n]$,则
\begin{align*}
i_n \left( \sum_{k_1 k_2 \cdots k_n} a_{k_1 k_2 \cdots k_n} x_1^{k_1} x_2^{k_2} \cdots x_n^{k_n} \right) = \sum_{k_1 k_2 \cdots k_n} a_{k_1 k_2 \cdots k_n} y_1^{k_1} y_2^{k_2} \cdots y_n^{k_n},
\end{align*}
故 \( i_n \) 是满同态. 由 \( y_1, y_2, \cdots, y_n \) 是 \( R \) 上代数无关的知$\ker i_n=\{0\}$,故由\refpro{proposition:单同态的充要条件是核为平凡零子空间}知\( i_n \) 也是单同态, 即 \( i_n \) 为同构映射.

\end{proof}

\begin{corollary}
设 \( R \) 是交换幺环 \( \widetilde{R} \) 的包含幺元 \( 1 \) 的子环, \( R \) 上的 \( n \) 元多项式环 \( R[x_1, x_2, \cdots, x_n] \), 又 \( u_1, u_2, \cdots, u_n \in \widetilde{R} \), 则有
\begin{enumerate}[(1)]
\item 存在 \( R[x_1, x_2, \cdots, x_n] \) 中理想 \( I \), 满足
\[
R \cap I = \{0\}, \quad R[u_1, u_2, \cdots, u_n] \cong R[x_1, x_2, \cdots, x_n]/I;
\]
\item \( u_1, u_2, \cdots, u_n \) 代数相关当且仅当 \( I \neq \{0\} \).
\end{enumerate}
\end{corollary}
\begin{proof}
\begin{enumerate}[(1)]
\item 考虑$R$到$R[u_1,u_2,\cdots ,u_n]$的嵌入映射$i$,则不难验证$i$是$R$到$R[u_1,u_2,\cdots ,u_n]$上的环同态. 于是由\refthe{theorem:n元多项式环开拓定理}知可将$i$开拓为环同态$i_u:R[x_1, x_2, \cdots, x_n]\to R[u_1,u_2,\cdots ,u_n]$满足
\begin{align*}
i_u|_R=i,\quad i_u(x_k)=u_k(k=1,2,\cdots,n).
\end{align*}
注意到$i_u(R[x_1, x_2, \cdots, x_n])=R[u_1,u_2,\cdots ,u_n]$,故$i_u$是满同态.于是由\hyperref[theorem:环的同态基本定理]{环的同态基本定理}知$I=\ker i_u$为$R$[$x_1$, $x_2, \cdots, x_n]$的理想,$R[u_1,u_2,\cdots ,u_n]\cong R[x_1, x_2, \cdots, x_n]/I$. 

\item 若$a\in R\cap I$,则$0=i_u(a)=i(a)=a$,故$R\cap I=\{0\}$. 
由于 \( u_1, u_2, \cdots, u_n \) 代数相关当且仅当存在有限多个 \( a_{k_1 k_2 \cdots k_n} \neq 0 \), 使
\[
\sum_{k_1 k_2 \cdots k_n} a_{k_1 k_2 \cdots k_n} u_1^{k_1} u_2^{k_2} \cdots u_n^{k_n} = 0.
\]
这也当且仅当
\begin{align*}
&i_u\left( \sum_{k_1k_2\cdots k_n}{a_{k_1k_2\cdots k_n}x_{1}^{k_1}x_{2}^{k_2}}\cdots x_{n}^{k_n} \right) =\sum_{k_1k_2\cdots k_n}{a_{k_1k_2\cdots k_n}u_{1}^{k_1}u_{2}^{k_2}}\cdots u_{n}^{k_n}=0
\\
&\Longleftrightarrow 0\ne \sum_{k_1k_2\cdots k_n}{a_{k_1k_2\cdots k_n}x_{1}^{k_1}x_{2}^{k_2}}\cdots x_{n}^{k_n}\in I\Longleftrightarrow I\ne \{0\}.
\end{align*}
\end{enumerate}

\end{proof}

\begin{corollary}
设 \( R \) 是交换幺环, \( R[x_1, x_2, \cdots, x_n] \) 是 \( R \) 上 \( n \) 元多项式环, \( I \) 为 \( R[x_1, x_2, \cdots, x_n] \) 的理想且 \( R \cap I = \{0\}, I \neq \{0\} \), 则 \( R[x_1, x_2, \cdots, x_n]/I \) 是 \( R \) 添加 \( n \) 个代数相关元所得的环.
\end{corollary}
\begin{proof}
设$\pi$是$R[x_1, x_2, \cdots, x_n]$到$R[x_1, x_2, \cdots, x_n]/I$的自然同态,于是$\pi(R)$是$R[x_1, x_2, \cdots, x_n]/I$中的子环. 由\rrefthe{theorem:子环和理想的基本性质}{theorem:子环和理想的基本性质-2}知$I$也是$R$的理想,从而再由\rrefthe{theorem:抽象代数--定理1.7.6}{theorem:抽象代数--定理1.7.6-3}知
\begin{align*}
\pi (R)=R/I=\left( R+I \right) /I\cong R/(R\bigcap{I)}=R/\{0\}=R+0=R,
\end{align*}
故可将$R$视为$R[x_1, x_2, \cdots, x_n]/I$的子环,令$u_i=\pi(x_i)(i=1,2,\cdots,n)$,则$u_i\in R[x_1, x_2, \cdots, x_n]/I$,于是$$R[u_1,u_2,\cdots ,u_n]\subseteq R[x_1, x_2, \cdots, x_n]/I,$$.注意到
\begin{align*}
\pi \left( \sum_{k_1k_2\cdots k_n}{a_{k_1k_2\cdots k_n}x_{1}^{k_1}x_{2}^{k_2}}\cdots x_{n}^{k_n} \right) =\sum_{k_1k_2\cdots k_n}{\pi \left( a_{k_1k_2\cdots k_n} \right) u_{1}^{k_1}u_{2}^{k_2}}\cdots u_{n}^{k_n},
\end{align*}
故再结合$\pi$是满同态可得
$$
R[x_1, x_2, \cdots, x_n]/I=\pi(R[x_1, x_2, \cdots, x_n])\subseteq R[u_1,u_2,\cdots ,u_n]\subseteq R[x_1, x_2, \cdots, x_n]/I,
$$
即$R[x_1, x_2, \cdots, x_n]/I=R[u_1,u_2,\cdots ,u_n]$. 又由$I\neq\{0\}$,故$I$中有非零元素
\begin{align*}
\sum_{k_1k_2\cdots k_n}{a_{k_1k_2\cdots k_n}x_{1}^{k_1}x_{2}^{k_2}}\cdots x_{n}^{k_n},
\end{align*}
其中有有限多个 \( a_{k_1 k_2 \cdots k_n} \neq 0 \).而
\begin{align*}
\pi \left( \sum_{k_1k_2\cdots k_n}{a_{k_1k_2\cdots k_n}x_{1}^{k_1}x_{2}^{k_2}}\cdots x_{n}^{k_n} \right) =\sum_{k_1k_2\cdots k_n}{a_{k_1k_2\cdots k_n}u_{1}^{k_1}u_{2}^{k_2}}\cdots u_{n}^{k_n}=0.
\end{align*}
又因为$\pi(R)\cong R$,所以上式有有限个$\pi(a_{k_1 k_2 \cdots k_n})\ne 0$.
故$u_1,u_2,\cdots,u_n$是代数相关的.

\end{proof}

















\end{document}
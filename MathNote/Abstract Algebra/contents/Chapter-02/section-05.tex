\documentclass[../../main.tex]{subfiles}
\graphicspath{{\subfix{../../image/}}} % 指定图片目录,后续可以直接使用图片文件名。

% 例如:
% \begin{figure}[H]
% \centering
% \includegraphics[scale=0.4]{图.png}
% \caption{}
% \label{figure:图}
% \end{figure}
% 注意:上述\label{}一定要放在\caption{}之后,否则引用图片序号会只会显示??.

\begin{document}

\section{主理想整环与Euclid环}

\begin{definition}
若交换幺环的每个理想都是主理想,则称此环为\textbf{主理想环}.一个主理想环若又是整环,则称此环为\textbf{主理想整环},记为p.i.d..
\end{definition}

\begin{example}
整环$\mathbf{Z}$是主理想整环.
\end{example}
\begin{proof}
事实上,设$I$为$\mathbf{Z}$的一个非平凡理想,于是$\exists m \in I$满足
\begin{align*}
m=\min\{|k| \mid k \in I, k \neq 0\}.
\end{align*}
$\forall k \in I$,若$k=0$,则$k=0 \cdot m$;若$k \neq 0$,则$\exists q, r \in \mathbf{Z}$,满足$k=q m + r$($0 \leqslant r < m$),由$I\lhd \mathbf{Z}$和$m\in I$知$qm\in I$,于是$r \in I$.由$m$的取法知$r=0$,即$k=q m$,否则与$m$的最小值定义矛盾!故$I=\{xm\mid x\in \mathbf{Z}\}=\langle m \rangle$,因而$\mathbf{Z}$是主理想整环.

\end{proof}

\begin{example}
$\mathbf{Z}[x]$不是主理想整环.
\end{example}
\begin{proof}
事实上,若$\mathbf{Z}[x]$是主理想整环,则有$g(x)$,使得$\langle 2, x^2 + 1\rangle=\langle g(x) \rangle$.由\rrefthe{theorem:主理想和有限生成理想的形状}{theorem:主理想和有限生成理想的形状-2}知
\begin{align}\label{eq::90j3g434}
\left\{ 2u\left( x \right) +\left( x^2+1 \right) v\left( x \right) \mid u\left( x \right) ,v\left( x \right) \in \mathbf{Z}\left[ x \right] \right\} =\langle 2,x^2+1\rangle =\langle g(x)\rangle =\left\{ u\left( x \right) g\left( x \right) \mid u\left( x \right) \in \mathbf{Z}\left[ x \right] \right\} .
\end{align}
因为$2\in \langle 2,x^2+1\rangle$,所以由\eqref{eq::90j3g434}式知存在$f(x)\in \mathbf{Z}\left[ x \right],$使$2=f(x)g(x)$,即$g(x) \mid 2$,故$g(x)=\pm 1, \pm 2$.另一方面,由$g(x) \in \langle 2, x^2 + 1 \rangle$,故由\eqref{eq::90j3g434}式有$u(x), v(x) \in \mathbf{Z}[x]$,使得
\begin{align*}
g(x)=2 u(x) + (x^2 + 1) v(x).
\end{align*}
令$x=1$,则有$g(1)=2(u(1) + v(1))$.于是$g(x)=\pm 2$,但$\pm 2 \nmid (x^2 + 1)$,即$g(x)\nmid (x^2 + 1)$,从而$x^2 + 1 \notin \langle g(x) \rangle$.这与\eqref{eq::90j3g434}式矛盾!因而$\mathbf{Z}[x]$不是主理想整环.

\end{proof}

\begin{theorem}\label{theorem:整除与主理想的关系}
设$R$是交换整环,则
\begin{enumerate}[(1)]
\item\label{theorem:整除与主理想的关系-1} $a \mid b\iff \langle a \rangle \supseteq \langle b \rangle$.

\item\label{theorem:整除与主理想的关系-2} $a \sim b\iff \langle a \rangle = \langle b \rangle$.

\item\label{theorem:整除与主理想的关系-3} $a \sim 1\iff \langle a \rangle = \langle 1 \rangle = R$.

\item\label{theorem:整除与主理想的关系-4} $R$满足因子链条件当且仅当$R$满足\textbf{主理想的升链条件},即任一\textbf{主理想升链}
\begin{align*}
\langle a_1 \rangle \subseteq \langle a_2 \rangle \subseteq \cdots \subseteq \langle a_n \rangle \subseteq \langle a_{n+1} \rangle \subseteq \cdots,
\end{align*}
一定存在$m\in \mathbf{N}$,使得当$n \geqslant m$时,$\langle a_n \rangle = \langle a_m \rangle$.
\end{enumerate}
\end{theorem}
\begin{proof}
\begin{enumerate}[(1)]
\item 若$a \mid b$,则存在$r_1 \in R$,使$b = r_1a$。从而由\rrefthe{theorem:主理想和有限生成理想的形状}{theorem:主理想和有限生成理想的形状-2}知
\begin{align*}
\langle b \rangle = \{ rb \mid r \in R \} = \{ rr_1a \mid r \in R \} \subseteq \{ ra \mid r \in R \} = \langle a \rangle.
\end{align*}

若$\langle b \rangle \subseteq \langle a \rangle$,则由\rrefthe{theorem:主理想和有限生成理想的形状}{theorem:主理想和有限生成理想的形状-2}知
\begin{align*}
\langle b \rangle = \{ rb \mid r \in R \} \subseteq \{ ra \mid r \in R \} = \langle a \rangle.
\end{align*}
于是由$b \in \langle b \rangle$知存在$r_1 \in R$,使得$b = r_1a$,故$a \mid b$.

\item 这就是\ref{theorem:整除与主理想的关系-1}的直接推论.

\item 这就是\ref{theorem:整除与主理想的关系-2}的直接推论.

\item 对任一主理想升链
\begin{align*}
\langle a_1 \rangle \subseteq \langle a_2 \rangle \subseteq \cdots \subseteq \langle a_n \rangle \subseteq \langle a_{n+1} \rangle \subseteq \cdots,
\end{align*}
由\hyperref[theorem:整除与主理想的关系-2]{结论\ref{theorem:整除与主理想的关系-2}}知$a_{n+1}\mid a_n(n\in \mathbf{N})$.故
\begin{align*}
a_1,a_2,\cdots ,a_n,a_{n+1},\cdots 
\end{align*}
是$R$的因子链.
若$R$满足因子链条件,则存在$m\in \mathbf{N},$使得当$n\geqslant m$时,有$a_n\sim a_m$.由\hyperref[theorem:整除与主理想的关系-2]{结论\ref{theorem:整除与主理想的关系-2}},此即$\langle a_n \rangle = \langle a_m \rangle$.

若存在$m\in \mathbf{N}$,使得当$n \geqslant m$时,$\langle a_n \rangle = \langle a_m \rangle$.由\hyperref[theorem:整除与主理想的关系-2]{结论\ref{theorem:整除与主理想的关系-2}},此即$a_n\sim a_m$.故此时$R$满足因子链条件.

\end{enumerate}

\end{proof}

\begin{theorem}
主理想整环一定是唯一析因环.
\end{theorem}
\begin{proof}
由\rrefthe{theorem:整除与主理想的关系}{theorem:整除与主理想的关系-4}与\rrefthe{theorem:UFD的充要条件}{theorem:UFD的充要条件-3},只需证一个主理想整环$R$满足主理想升链条件与最大公因子条件.设
\begin{align*}
\langle a_1 \rangle \subseteq \langle a_2 \rangle \subseteq \cdots \subseteq \langle a_n \rangle \subseteq \cdots
\end{align*}
是$R$中一个主理想升链.令$I = \bigcup\limits_{i=1}^{\infty} \langle a_i \rangle$.若$a, b \in I$,则$\exists i, j \in \mathbf{N}$,使$a \in \langle a_i \rangle$,$b \in \langle a_j \rangle$.不妨设$j \geqslant i$.由此知$a - b \in \langle a_j \rangle \subseteq I$,故$I$是$R$中加法子群,也是Abel群.显然$I$对乘法封闭且满足结合律,故$I$是$R$的子环.又由\rrefthe{theorem:主理想和有限生成理想的形状}{theorem:主理想和有限生成理想的形状-2}知$\forall c \in R$,$ca \in \langle a_i \rangle \subseteq I$,故$I$是$R$中理想.由$R$是主理想整环知$\exists d \in R$,使$I = \langle d \rangle$.因$d \in I$,故$\exists m \in \mathbf{N}$,使$d \in \langle a_m \rangle$,因而当$n \geqslant m$时,由\refthe{theorem:主理想和有限生成理想的形状}有
\begin{align*}
I = \langle d \rangle \subseteq \langle a_m \rangle \subseteq \langle a_n \rangle \subseteq \bigcup\limits_{i=1}^{\infty} \langle a_i \rangle = I,
\end{align*}
即$\langle a_n \rangle = \langle a_m \rangle = I$.这就证明了$R$满足主理想升链条件.

其次,设$a, b \in R^*$.由\rrefthe{theorem:抽象代数--定理1.7.6}{theorem:抽象代数--定理1.7.6-1}知$\langle a \rangle + \langle b \rangle$是$R$的子环,利用\refthe{theorem:主理想和有限生成理想的形状}显然有$R(\langle a \rangle + \langle b \rangle)\subseteq \langle a \rangle + \langle b \rangle$,故$\langle a \rangle + \langle b \rangle$是$R$中理想.由$R$是主理想整环知$\exists d \in R$,使$\langle a \rangle + \langle b \rangle = \langle d \rangle$,因而有$\langle a \rangle \subseteq \langle d \rangle$,$\langle b \rangle \subseteq \langle d \rangle$,由\rrefthe{theorem:整除与主理想的关系}{theorem:整除与主理想的关系-1}知$d \mid a$,$d \mid b$,即$d$为$a, b$的公因子.又若$c \mid a$,$c \mid b$,则由\rrefthe{theorem:整除与主理想的关系}{theorem:整除与主理想的关系-1}知$\langle a \rangle \subseteq \langle c \rangle$,$\langle b \rangle \subseteq \langle c \rangle$,故$\langle d \rangle = \langle a \rangle + \langle b \rangle \subseteq \langle c \rangle$,由\rrefthe{theorem:整除与主理想的关系}{theorem:整除与主理想的关系-1}知$c \mid d$,故$d$为$a, b$的最大公因子.

综上知$R$为唯一析因环.

\end{proof}

\begin{corollary}\label{corollary:抽象代数--推论2.5.1}
设$R$是主理想整环,若$d$为$a, b$的最大公因子,则存在$u, v \in R$,使得
\begin{align*}
d = au + bv. 
\end{align*}
\end{corollary}
\begin{proof}
设$a, b \in R^*$.由\rrefthe{theorem:抽象代数--定理1.7.6}{theorem:抽象代数--定理1.7.6-1}知$\langle a \rangle + \langle b \rangle$是$R$的子环,利用\refthe{theorem:主理想和有限生成理想的形状}显然有$R(\langle a \rangle + \langle b \rangle)\subseteq \langle a \rangle + \langle b \rangle$,故$\langle a \rangle + \langle b \rangle$是$R$中理想.由$R$是主理想整环知$\exists d_1 \in R$,使$\langle a \rangle + \langle b \rangle = \langle d_1 \rangle$,因而有$\langle a \rangle \subseteq \langle d_1 \rangle$,$\langle b \rangle \subseteq \langle d_1 \rangle$,由\rrefthe{theorem:整除与主理想的关系}{theorem:整除与主理想的关系-1}知$d_1 \mid a$,$d_1 \mid b$,即$d_1$为$a, b$的公因子.又若$c \mid a$,$c \mid b$,则由\rrefthe{theorem:整除与主理想的关系}{theorem:整除与主理想的关系-1}知$\langle a \rangle \subseteq \langle c \rangle$,$\langle b \rangle \subseteq \langle c \rangle$,故$\langle d_1 \rangle = \langle a \rangle + \langle b \rangle \subseteq \langle c \rangle$,由\rrefthe{theorem:整除与主理想的关系}{theorem:整除与主理想的关系-1}知$c \mid d_1$,故$d_1$为$a, b$的最大公因子.从而$d_1\sim d$,再由\rrefthe{theorem:整除与主理想的关系}{theorem:整除与主理想的关系-2}知$\langle d \rangle=\langle d_1 \rangle = \langle a \rangle + \langle b \rangle$.由\refthe{theorem:主理想和有限生成理想的形状}知
\begin{align*}
\langle d\rangle =\langle a\rangle +\langle b\rangle =\left\{ ua+bv\mid u,v\in R \right\} .
\end{align*}
又$d\in \langle d\rangle$,故存在$u, v \in R$,使得
\begin{align*}
d = au + bv. 
\end{align*}

\end{proof}

\begin{corollary}
设$R$是主理想整环,$a, b$互素(即$(a, b) \sim 1$)的充要条件是$\exists u, v \in R$,使得
\begin{align*}
au + bv = 1. 
\end{align*}
\end{corollary}
\begin{proof}
必要性已含于\refcor{corollary:抽象代数--推论2.5.1}中.下证充分性.设$au + bv = 1(u,v\in R)$,若$d = (a, b)$,则$d \mid a$,$d \mid b$,故$d \mid au + bv$,因而$d \mid 1$,故$d \sim 1$.

\end{proof}

\begin{definition}
设$R$为交换整环. 若存在$R$到非负整数集$\mathbf{N}\cup\{0\}$的映射$\delta$, 使得$\forall\ a,b\in R$, $b\neq 0$, $\exists q,r\in R$满足
\begin{align}
a=qb+r,\quad \delta(r)<\delta(b),\label{eq:2.5.3}
\end{align}
则称$R$为Euclid环.
\end{definition}

\begin{example}
$\mathbf{Z}$是Euclid环.

事实上, 只需取$\delta(m)=|m|$即可验证$\delta$满足定义.
\end{example}

\begin{example}
设$\mathbf{P}$为数域, 则$\mathbf{P}[x]$是Euclid环. 定义2.5.2中的$\delta$可定义为
\begin{align*}
\delta(f(x))=
\begin{cases}
2^{\deg f(x)},\ &f(x)\neq 0,\\
0,\ &f(x)=0.
\end{cases}
\end{align*}

不难验证$\delta$满足所要求的条件.
\end{example}

\begin{example}
Gauss整数环$\mathbf{Z}[\sqrt{-1}]=\{a+b\sqrt{-1}\mid a,b\in\mathbf{Z}\}$是Euclid环.

事实上, 令$\delta(a+b\sqrt{-1})=a^2+b^2$, 则显然有
\begin{align*}
\delta(\alpha\beta)=\delta(\alpha)\delta(\beta),\quad \forall\alpha,\beta\in\mathbf{Z}[\sqrt{-1}].
\end{align*}
设$\beta\neq 0$. 不难看出$\beta^{-1}\in\mathbf{Q}[\sqrt{-1}]$, 即有
\begin{align*}
\alpha\beta^{-1}=\mu+\nu\sqrt{-1},\quad \mu,\nu\in\mathbf{Q}.
\end{align*}
于是$\exists c,d\in\mathbf{Z}$, 使得$|c-\mu|\leqslant 1/2$, $|d-\nu|\leqslant 1/2$. 令$\varepsilon=\mu-c$, $\eta=\nu-d$, 则有$|\varepsilon|\leqslant 1/2$, $|\eta|\leqslant 1/2$, 而
\begin{align*}
\alpha=\beta((c+\varepsilon)+(d+\eta)\sqrt{-1})=\beta q+r,
\end{align*}
其中, $q=c+d\sqrt{-1}\in\mathbf{Z}[\sqrt{-1}]$, $r=\beta(\varepsilon+\eta\sqrt{-1})=\alpha-\beta q\in\mathbf{Z}[\sqrt{-1}]$. 又
\begin{align*}
\delta(r)=|r|^2=\delta(\beta)(\varepsilon^2+\eta^2)\leqslant\delta(\beta)(1/4+1/4)<\delta(\beta),
\end{align*}
故$\mathbf{Z}[\sqrt{-1}]$为Euclid环.
\end{example}

\begin{theorem}
Euclid环是主理想环.
\end{theorem}

\begin{proof}
设$I$是Euclid环$R$中的一个理想. 若$I=\{0\}$, 显然是主理想, 故假设$I\neq\{0\}$. 取$I$中元素$b$, 使得
\begin{align}
\delta(b)=\min\{\delta(c)\mid c\in I,c\neq 0\}.\label{eq:2.5.4}
\end{align}
设$a\in I$, 则有$q,r\in R$, 使式\eqref{eq:2.5.3}成立. 因$a,b\in I$, 故$r=a-qb\in I$. 由$b$的取法知$r\notin I\setminus\{0\}$. 故$r=0$, 因而$a\in\langle b\rangle$, 故$I=\langle b\rangle$. 即$R$为主理想环.

\end{proof}

\begin{corollary}
Euclid环是主理想整环, 因而也是唯一析因环.
\end{corollary}

在Euclid环中, 可用辗转相除法来求两个元素的最大公因子.

设$a,b\in R^*$. 不妨设$\delta(a)\geqslant\delta(b)$, 并记$a=a_1$, $b=a_2$. 于是$\exists q_1,a_3\in R$, 使
\begin{align*}
a_1=q_1a_2+a_3,\quad \delta(a_3)<\delta(a_2).
\end{align*}
若$a_3=0$, 则$(a_1,a_2)\sim a_2$, 设$a_3\neq 0$, 则$(a_1,a_2)\sim(a_2,a_3)$. 再对$a_2,a_3$作除法运算
\begin{align*}
a_2=q_2a_3+a_4,\quad \delta(a_4)<\delta(a_3).
\end{align*}

若$a_4=0$, 则$(a_1,a_2)\sim(a_2,a_3)\sim a_3$, 若$a_4\neq 0$, 则$(a_1,a_2)\sim(a_2,a_3)\sim(a_3,a_4)$. 再继续下去有
\begin{align*}
\delta(a_1)\geqslant\delta(a_2)>\delta(a_3)>\delta(a_4)>\cdots,
\end{align*}
因而在有限步后必然终止, 即有$a_n\neq 0$, 而$a_{n+1}=0$. 于是$(a_1,a_2)\sim a_n$.

在初等数论与数域上一元多项式理论中, 此算法是熟知的, 故不再赘述.

Euclid环的定义在许多书上是各不相同的, 但有一点是共同的, 即辗转相除法可行. 现将常见的几种定义列出以供参考. 总假定$R$为交换整环.

\begin{enumerate}
\item 若有$R^*$到$\mathbf{N}\cup\{0\}$的映射$\delta$满足
  \begin{enumerate}
  \item $a\neq 0$,$b\neq 0$,$\delta(ab)\geqslant\delta(a)$;
  \item $\forall\ a,b\in R$, $b\neq 0$有$a=qb+r$, 其中, $r=0$或者$\delta(r)<\delta(b)$,
  \end{enumerate}
  则称$R$为Euclid环$^{[3]}$.
\item 若有$R$到$\mathbf{N}\cup\{0\}$的映射$\delta$满足
\begin{enumerate}
\item $\delta(a)\geqslant 0$,$\delta(a)=0$ iff $a=0$;
\item $\delta(ab)=\delta(a)\delta(b)$;
\item 若$b\neq 0$,$\forall\ a\in R$, 则$\exists q,r\in R$, 使$a=bq+r$, 其中, $\delta(r)<\delta(b)$,
\end{enumerate}
则称$R$为Euclid环$^{[4]}$.
\item 若有$R$到$\mathbf{N}\cup\{0\}$的映射$\delta$满足
\begin{enumerate}
\item 若$b|a$, 则$\delta(b)\leqslant\delta(a)$;
\item $a,b\in R$,$b\neq 0$, 则$\exists q,r\in R$, 使得
\begin{align*}
a=bq+r,\quad \delta(r)<\delta(b),
\end{align*}
\end{enumerate}
则称$R$为Euclid环.
\end{enumerate}

在此不讨论这些定义间的关系, 但要指出, 确有主理想整环不是Euclid环. 例如, 环
\begin{align*}
D=\left\{a+\frac{b}{2}(1+\sqrt{-19})\mid a,b\in\mathbf{Z}\right\}
\end{align*}
是一个主理想整环, 但不是Euclid环.




\end{document}
\documentclass[../../main.tex]{subfiles}
\graphicspath{{\subfix{../../image/}}} % 指定图片目录,后续可以直接使用图片文件名。

% 例如:
% \begin{figure}[h]
% \centering
% \includegraphics{image-01.01}
% \caption{图片标题}
% \label{fig:image-01.01}
% \end{figure}
% 注意:上述\label{}一定要放在\caption{}之后,否则引用图片序号会只会显示??.

\begin{document}

\section{环的局部化}

\begin{definition}[乘法子集]
设 $(R, +, \cdot)$ 是一个交换环,而 $S \subset R$。则我们称 $S$ 是一个\textbf{乘法子集},若 $S$ 是 $(R \setminus \{0\}, \cdot)$ 的(乘法)子幺半群,即
\begin{gather*}
1 \in S ,\\
\forall a, b \in S, ab \in S .
\end{gather*}
\end{definition}


\begin{definition}[环的局部化]
设 $(R, +, \cdot)$ 是一个交换环,而 $S$ 是乘法子集。则 $R$ 对 $S$ 的\textbf{局部化},记作 $(S^{-1}R, +, \cdot)$,定义为
\[S^{-1}R = \left\{\frac{r}{s} : r \in R, s \in S\right\} / \sim\]
其中
\[\frac{r}{s} \sim \frac{r'}{s'} \iff \exists t \in S, t(rs' - r's) = 0\]
若 $r, r' \in R, s, s' \in S$,我们定义
\begin{align*}
\frac{r}{s} + \frac{r'}{s'} &= \frac{rs' + sr'}{ss'},\\
\frac{r}{s}\frac{r'}{s'} &= \frac{rr'}{ss'}.
\end{align*}
\end{definition}
\begin{proof}
我们先证明 $\sim$ 是个等价关系,再证明加法和乘法是良定义的.

第一,我们来证明 $\sim$ 是个等价关系。$r/s \sim r/s$ 是显然的,这是因为 $1(rs - rs) = 0(1 \in S)$。若 $r/s \sim r'/s'$,则存在 $t \in S$,使得
\[t(rs' - r's) = 0 .\]
则
\[(-t)(r's - rs') = 0.\]
故 $r'/s' \sim r/s$。最后,如果 $r/s \sim r'/s'$,$r'/s' \sim r''/s''$,只须证明 $r/s \sim r''/s''$。我们取 $t, t' \in S$,使
\begin{gather*}
t(rs' - r's) = 0 ,\\
t'(r's'' - r''s') = 0 .
\end{gather*}
则我们可以通过不断的尝试,凑出一个美妙的 $t'' = tt's'$。于是 $t''(rs'' - r''s) = 0$,这是因为
\begin{align*}
(tt's')rs'' &= t's''(trs') = t's''(t'r's) = ts(t'r's'') = ts(t'r''s') = (tt's')r''s .
\end{align*}
由于 $S$ 是乘法子群,故 $t'' = tt's' \in S$。接下来,即使局部化中的每个元素实际上是等价类,我们还是为了方便起见,用等号来代替所有的等价号。

第二,我们来证明加法和乘法是良定义的。假设
\begin{align*}
\frac{r_1}{s_1} &\sim \frac{r_1'}{s_1'} .\\
\frac{r_2}{s_2} &\sim \frac{r_2'}{s_2'} .
\end{align*}
故存在 $t, t' \in S$,使得
\begin{align*}
t(r_1s_1' - r_1's_1) &= 0 ,\\
t'(r_2s_2' - r_2's_2) &= 0 .
\end{align*}
我们只须证明
\begin{align*}
\frac{r_1}{s_1} + \frac{r_2}{s_2} = \frac{r_1s_2 + r_2s_1}{s_1s_2} &\sim \frac{r_1's_2' + r_2's_1'}{s_1's_2'} = \frac{r_1'}{s_1'} + \frac{r_2'}{s_2'} ,\\
\frac{r_1}{s_1} \cdot \frac{r_2}{s_2} = \frac{r_1r_2}{s_1s_2} &\sim \frac{r_1'r_2'}{s_1's_2'} = \frac{r_1'}{s_1'} \cdot \frac{r_2'}{s_2'} .
\end{align*}
重新分组,对于 $tt' \in S$,我们有
\begin{align*}
&tt'\left((r_1s_2 + r_2s_1)s_1's_2' - (r_1's_2' + r_2's_1')s_1s_2\right)\\
=&tt'\left((r_1s_1' - r_1's_1)s_2s_2' + (r_2s_2' - r_2's_2)s_2s_1'\right)\\
=&0 + 0 = 0 .
\end{align*}
根据拆项补项,同样对于 $tt' \in S$,我们有
\begin{align*}
&tt'(r_1r_2s_1's_2' - r_1'r_2's_1s_2)\\
=&tt'(r_1r_2s_1's_2' - r_1'r_2s_1s_2') + tt'(r_1'r_2s_1s_2' - r_1'r_2's_1s_2)\\
=&tt'(r_1s_1' - r_1's_1)r_2s_2' + tt'(r_2s_2' - r_2's_2)r_1's_1\\
=&0 + 0 = 0 。
\end{align*}
\end{proof}

\begin{proposition}
设 $(R, +, \cdot)$ 是一个交换环,而 $S$ 是乘法子集。则 $R$ 对 $S$ 的局部化,即 $(S^{-1}R, +, \cdot)$,是个交换环。
\end{proposition}
\begin{proof}
根据定义,加法和乘法的封闭性和交换律是显然的。而加法单位元是 $0/1$,乘法单位元是 $1/1$,因为对于任何 $r/s \in S^{-1}R$,我们有
\begin{gather*}
\frac{0}{1} + \frac{r}{s} = \frac{0s + 1r}{1s} = \frac{r}{s},\\
\frac{1}{1} \cdot \frac{r}{s} = \frac{1r}{1s} = \frac{r}{s}.
\end{gather*}
乘法的结合律是显然的,而加法的结合律也很简单,我们很容易检验
\[
\left(\frac{r_1}{s_1} + \frac{r_2}{s_2}\right) + \frac{r_3}{s_3} = \frac{r_1s_2s_3 + s_1r_2s_3 + s_1s_2r_3}{s_1s_2s_3} = \frac{r_1}{s_1} + \left(\frac{r_2}{s_2} + \frac{r_3}{s_3}\right).
\]
加法的逆元也是显然的。$r/s$ 的加法逆元当然是 $(-r)/s$。

最后,我们只须证明乘法对加法的分配律。令 $r_1/s_1, r_2/s_2, r_3/s_3 \in S^{-1}R$,则我们很容易检验
\[
\frac{r_1}{s_1} \cdot \left(\frac{r_2}{s_2} + \frac{r_3}{s_3}\right) = \frac{r_1(r_2s_3 + r_3s_2)}{s_1s_2s_3} = \frac{r_1r_2s_3}{s_1s_2s_3} + \frac{r_1r_3s_2}{s_1s_2s_3} = \frac{r_1}{s_1} \cdot \frac{r_2}{s_2} + \frac{r_1}{s_1} \cdot \frac{r_3}{s_3}.
\]
综上所述,我们就证明了 $R$ 对 $S$ 的局部化 $S^{-1}R$ 是个交换环。 
\end{proof}

\begin{proposition}
设 $(R, +, \cdot)$ 是一个整环,而 $S$ 是一个乘法子集,则
\[\frac{a}{b} \sim \frac{c}{d} \iff ad - bc = 0 .\]
\end{proposition}
\begin{proof}
先证充分性。假如 $ad - bc = 0$,那么我们取 $s = 1$,则 $1(ad - bc) = 1 \cdot 0 = 0$,这就证明了
\[\frac{a}{b} \sim \frac{c}{d} .\]

再证必要性。假如 $a/b = c/d$,则存在 $s \in S$(于是 $s \neq 0$),使得 $s(ad - bc) = 0$。可是因为 $R$ 是个整环,而且 $s \neq 0$,所以 $ad - bc = 0$。

这就证明了这个命题。
\end{proof}

\begin{definition}
设$(R, +, \cdot)$ 是一个整环,我们定义 $R$ 上的分式域,记作 $\mathrm{Frac}(R)$,定义为 $S^{-1}(R)$,其中 $S = R \setminus \{0\}$。
\end{definition}

\begin{proposition}
设 $(R, +, \cdot)$ 是一个整环,则 $R$ 上的分式域 $\mathrm{Frac}(R)$ 是个域.
\end{proposition}
\begin{proof}
令 $S = R \setminus \{0\}$。

我们已经证明了 $\mathrm{Frac}(R) = S^{-1}R$ 是个交换环,因此只须证明对任意非零元素
\[\frac{r}{s} \in S^{-1}R\]
我们都能找到逆元即可。

而这是因为由
\[\frac{r}{s} \neq 0\]
我们可以得知 $r \neq 0$。

因此,
\[\frac{s}{r} \in S^{-1}R\]

而且
\[\frac{r}{s}\frac{s}{r} = \frac{s}{r}\frac{r}{s} = \frac{sr}{sr} = \frac{1}{1} = 1\]

这就证明了 $\mathrm{Frac}(R)$ 是个域。
\end{proof}

\begin{lemma}
设 $(R, +, \cdot)$ 是一个交换环,而 $\mathfrak{p} \triangleleft R$ 是一个素理想,则 $S = R \setminus \mathfrak{p}$ 是一个乘法子集。
\end{lemma}
\begin{proof}
首先,若 $1 \in \mathfrak{p}$,则 $\mathfrak{p} = R$,可是素理想根据定义是不能等于整个环的。因此 $1 \in S$。

接着,假设 $s_1, s_2 \in S$,我们只须证明 $s_1s_2 \in S$。

因为 $s_1 \notin \mathfrak{p}$,$s_2 \notin \mathfrak{p}$,所以根据逆否命题,
\[s_1s_2 \notin \mathfrak{p}\]

也就是说,
\[s_1s_2 \in S\]

这就巧妙地证明了素理想的补集是一个乘法子集。
\end{proof}




\end{document}
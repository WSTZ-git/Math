\documentclass[../../main.tex]{subfiles}
\graphicspath{{\subfix{../../image/}}} % 指定图片目录,后续可以直接使用图片文件名。

% 例如:
% \begin{figure}[h]
% \centering
% \includegraphics{image-01.01}
% \caption{图片标题}
% \label{fig:image-01.01}
% \end{figure}
% 注意:上述\label{}一定要放在\caption{}之后,否则引用图片序号会只会显示??.

\begin{document}

\section{环的局部化}

\begin{definition}[乘法子集]
设 $(R, +, \cdot)$ 是一个交换环,而 $S \subset R$。则我们称 $S$ 是一个\textbf{乘法子集},若 $S$ 是 $(R \setminus \{0\}, \cdot)$ 的(乘法)子幺半群,即
\begin{gather*}
0 \notin S, 1 \in S ,\\
\forall a, b \in S, ab \in S .
\end{gather*}
\end{definition}


\begin{definition}[(交换)环的局部化]\label{definition:(交换)环的局部化}
设 $(R, +, \cdot)$ 是一个交换环,而 $S$ 是乘法子集。则 $R$ 对 $S$ 的\textbf{局部化},记作 $(S^{-1}R, +, \cdot)$,定义为
\[S^{-1}R = \left\{\frac{r}{s} : r \in R, s \in S\right\} / \sim\]
其中
\[\frac{r}{s} \sim \frac{r'}{s'} \iff \exists t \in S, t(rs' - r's) = 0\]
若 $r, r' \in R, s, s' \in S$,我们定义
\begin{align*}
\frac{r}{s} + \frac{r'}{s'} &= \frac{rs' + sr'}{ss'},\\
\frac{r}{s}\frac{r'}{s'} &= \frac{rr'}{ss'}.
\end{align*}
\end{definition}
\begin{proof}
我们先证明 $\sim$ 是个等价关系,再证明加法和乘法是良定义的.

第一,我们来证明 $\sim$ 是个等价关系。$r/s \sim r/s$ 是显然的,这是因为 $1(rs - rs) = 0(1 \in S)$。若 $r/s \sim r'/s'$,则存在 $t \in S$,使得
\[t(rs' - r's) = 0 .\]
则
\[(-t)(r's - rs') = 0.\]
故 $r'/s' \sim r/s$。最后,如果 $r/s \sim r'/s'$,$r'/s' \sim r''/s''$,只须证明 $r/s \sim r''/s''$。我们取 $t, t' \in S$,使
\begin{gather*}
t(rs' - r's) = 0 ,\\
t'(r's'' - r''s') = 0 .
\end{gather*}
则我们可以通过不断的尝试,凑出一个美妙的 $t'' = tt's'$。于是 $t''(rs'' - r''s) = 0$,这是因为
\begin{align*}
(tt's')rs'' &= t's''(trs') = t's''(t'r's) = ts(t'r's'') = ts(t'r''s') = (tt's')r''s .
\end{align*}
由于 $S$ 是乘法子群,故 $t'' = tt's' \in S$。接下来,即使局部化中的每个元素实际上是等价类,我们还是为了方便起见,用等号来代替所有的等价号。

第二,我们来证明加法和乘法是良定义的。假设
\begin{align*}
\frac{r_1}{s_1} &\sim \frac{r_1'}{s_1'} .\\
\frac{r_2}{s_2} &\sim \frac{r_2'}{s_2'} .
\end{align*}
故存在 $t, t' \in S$,使得
\begin{align*}
t(r_1s_1' - r_1's_1) &= 0 ,\\
t'(r_2s_2' - r_2's_2) &= 0 .
\end{align*}
我们只须证明
\begin{align*}
\frac{r_1}{s_1} + \frac{r_2}{s_2} = \frac{r_1s_2 + r_2s_1}{s_1s_2} &\sim \frac{r_1's_2' + r_2's_1'}{s_1's_2'} = \frac{r_1'}{s_1'} + \frac{r_2'}{s_2'} ,\\
\frac{r_1}{s_1} \cdot \frac{r_2}{s_2} = \frac{r_1r_2}{s_1s_2} &\sim \frac{r_1'r_2'}{s_1's_2'} = \frac{r_1'}{s_1'} \cdot \frac{r_2'}{s_2'} .
\end{align*}
重新分组,对于 $tt' \in S$,我们有
\begin{align*}
&tt'\left((r_1s_2 + r_2s_1)s_1's_2' - (r_1's_2' + r_2's_1')s_1s_2\right)\\
=&tt'\left((r_1s_1' - r_1's_1)s_2s_2' + (r_2s_2' - r_2's_2)s_2s_1'\right)\\
=&0 + 0 = 0 .
\end{align*}
根据拆项补项,同样对于 $tt' \in S$,我们有
\begin{align*}
&tt'(r_1r_2s_1's_2' - r_1'r_2's_1s_2)\\
=&tt'(r_1r_2s_1's_2' - r_1'r_2s_1s_2') + tt'(r_1'r_2s_1s_2' - r_1'r_2's_1s_2)\\
=&tt'(r_1s_1' - r_1's_1)r_2s_2' + tt'(r_2s_2' - r_2's_2)r_1's_1\\
=&0 + 0 = 0 。
\end{align*}
\end{proof}

\begin{proposition}[(交换)环的局部化的基本性质]\label{proposition:(交换)环的局部化的基本性质}
设 $(R, +, \cdot)$ 是一个交换环,而 $S$ 是乘法子集。则 $R$ 对 $S$ 的局部化,即 $(S^{-1}R, +, \cdot)$满足
\begin{enumerate}[(1)]
\item 若$s\in S$,则$\frac{s}{s}\sim \frac{1}{1}.$

\item 若$r,s,s'\in S$,则$\frac{rs'}{ss'}\sim \frac{r}{s}.$
\end{enumerate}
\end{proposition}
\begin{proof}
\begin{enumerate}[(1)]
\item 因为$1\cdot (s\cdot1-1\cdot s)=0$,所以根据\hyperref[definition:(交换)环的局部化]{(交换)环的局部化的定义}可知$\frac{s}{s}\sim \frac{1}{1}.$

\item 因为$1\cdot (rs's-ss'r)=0$,所以根据\hyperref[definition:(交换)环的局部化]{(交换)环的局部化的定义}可知$\frac{rs'}{ss'}\sim \frac{r}{s}.$
\end{enumerate}
\end{proof}

\begin{proposition}[(交换)环的局部化还是交换环]\label{proposition:(交换)环的局部化还是交换环}
设 $(R, +, \cdot)$ 是一个交换环,而 $S$ 是乘法子集。则 $R$ 对 $S$ 的局部化,即 $(S^{-1}R, +, \cdot)$,是个交换环。
\end{proposition}
\begin{proof}
根据定义,加法和乘法的封闭性和交换律是显然的。而加法单位元是 $0/1$,乘法单位元是 $1/1$,因为对于任何 $r/s \in S^{-1}R$,我们有
\begin{gather*}
\frac{0}{1} + \frac{r}{s} = \frac{0s + 1r}{1s} = \frac{r}{s},\\
\frac{1}{1} \cdot \frac{r}{s} = \frac{1r}{1s} = \frac{r}{s}.
\end{gather*}
乘法的结合律是显然的,而加法的结合律也很简单,我们很容易检验
\[
\left(\frac{r_1}{s_1} + \frac{r_2}{s_2}\right) + \frac{r_3}{s_3} = \frac{r_1s_2s_3 + s_1r_2s_3 + s_1s_2r_3}{s_1s_2s_3} = \frac{r_1}{s_1} + \left(\frac{r_2}{s_2} + \frac{r_3}{s_3}\right).
\]
加法的逆元也是显然的。$r/s$ 的加法逆元当然是 $(-r)/s$。

最后,我们只须证明乘法对加法的分配律。令 $r_1/s_1, r_2/s_2, r_3/s_3 \in S^{-1}R$,则我们很容易检验
\[
\frac{r_1}{s_1} \cdot \left(\frac{r_2}{s_2} + \frac{r_3}{s_3}\right) = \frac{r_1(r_2s_3 + r_3s_2)}{s_1s_2s_3} = \frac{r_1r_2s_3}{s_1s_2s_3} + \frac{r_1r_3s_2}{s_1s_2s_3} = \frac{r_1}{s_1} \cdot \frac{r_2}{s_2} + \frac{r_1}{s_1} \cdot \frac{r_3}{s_3}.
\]
综上所述,我们就证明了 $R$ 对 $S$ 的局部化 $S^{-1}R$ 是个交换环。 
\end{proof}

\begin{proposition}\label{proposition整环的局部化中两个元素等价的充要条件}
设 $(R, +, \cdot)$ 是一个整环,而 $S$ 是一个乘法子集,则
\[\frac{a}{b} \sim \frac{c}{d} \iff ad - bc = 0 .\]
\end{proposition}
\begin{proof}
先证充分性。假如 $ad - bc = 0$,那么我们取 $s = 1$,则 $1(ad - bc) = 1 \cdot 0 = 0$,这就证明了
\[\frac{a}{b} \sim \frac{c}{d} .\]

再证必要性。假如 $a/b = c/d$,则存在 $s \in S$,使得 $s(ad - bc) = 0$。由$S$是一个乘法子集可知,$s \neq 0$.可是因为 $R$ 是个整环,所以 $ad - bc = 0$。

这就证明了这个命题。
\end{proof}

\begin{proposition}\label{proposition:R/0一定是整环R的乘法子集}
设$(R, +, \cdot)$ 是一个整环,则$R/\{0\}$是$R$的一个乘法子集.
\end{proposition}
\begin{proof}
显然$0\notin R/\{0\}$.因为$R$是整环,所以$R\ne \{0\}$,从而由\refpro{proposition:零环的充要条件}可知$0\ne 1$,故$1\in R/\{0\}.$对$\forall a,b\in R/\{0\}$,都有$a,b\ne 0.$于是$ab\ne 0$,否则,由$R$是整环可知一定有$a=0$或$b=0$矛盾!又因为$a,b\in R$,而$R$对乘法封闭,所以$ab\in R$.又$ab\ne 0$,故$ab\in R/\{0\}.$因此$R/\{0\}$是$R$的一个乘法子集.
\end{proof}

\begin{definition}[分式域]
设$(R, +, \cdot)$ 是一个整环,我们定义 $R$ 上的\textbf{分式域},记作 $\mathrm{Frac}(R)$,定义为 $(S^{-1}R,+,\cdot)$,其中 $S = R \setminus \{0\}$。
\end{definition}
\begin{note}
由\refpro{proposition:R/0一定是整环R的乘法子集}可知$R/\{0\}$是$R$的一个乘法子集,从而$\mathrm{Frac}(R)$实际上就是$R$对$R/\{0\}$的局部化(最大的局部化).
\end{note}

\begin{proposition}[分式域是域]
设 $(R, +, \cdot)$ 是一个整环,则 $R$ 上的分式域 $\mathrm{Frac}(R)$ 是个域.进而,$\mathrm{Frac}(R)$ 是$R$的子环.
\end{proposition}
\begin{proof}
令 $S = R \setminus \{0\}$。

由\hyperref[proposition:(交换)环的局部化还是交换环]{(交换)环的局部化还是交换环}可知   $\mathrm{Frac}(R) = S^{-1}R$ 是个交换环,因此只须证明对任意非零元素
\[\frac{r}{s} \in S^{-1}R\]
我们都能找到逆元即可。

而这是因为由
\[\frac{r}{s} \neq 0\]
我们可以得知 $r \neq 0$。

因此,
\[\frac{s}{r} \in S^{-1}R\]

而且
\[\frac{r}{s}\frac{s}{r} = \frac{s}{r}\frac{r}{s} = \frac{sr}{sr} = \frac{1}{1} = 1\]

这就证明了 $\mathrm{Frac}(R)$ 是个域。进而, $\mathrm{Frac}(R)$ 对单位元、加法、乘法和逆元都封闭.又因为$S\subset R$,所以$S^{-1}R\subset R$,故 $\mathrm{Frac}(R)$ 就是$R$的一个子环.
\end{proof}

\begin{lemma}\label{lemma:由整环中非零元素生成的子幺半群都是乘法子集}
设$(R,+,\cdot)$是一个整环,$s\in R/\{0\}$,则由$s$生成的乘法子幺半群$\langle s\rangle$是$R$的一个乘法子集.
\end{lemma}
\begin{proof}
由\refpro{proposition:由单个元素生成的子幺半群的集合表示}可知$\langle s \rangle=\{1,s,s^2,\cdots\}$.由于$\langle s\rangle$是$(R,\cdot)$的子幺半群,因此我们只需证$0\notin \langle s \rangle=\{1,s,s^2,\cdots\}$(即证$s$不是幂零的).已知$s\ne 0$,假设$s^n\ne 0$,则由$R$是整环可知
\begin{align*}
s^{n+1}=s\cdot s^n\ne 0.
\end{align*}
故由数学归纳法可知$s^n\ne 0 ,\forall n\in \mathbb{N}_0$.因此$0\notin \langle s \rangle$.于是$\langle s\rangle$是$R$的一个乘法子集.
\end{proof}

\begin{corollary}
整环中的任意非零元素都不是幂零的.
\end{corollary}
\begin{proof}
设$(R,+,\cdot)$是一个整环,$s\in R/\{0\}$,则由\hyperref[lemma:由整环中非零元素生成的子幺半群都是乘法子集]{引理\ref{lemma:由整环中非零元素生成的子幺半群都是乘法子集}的证明}可知$s^n\ne 0 ,\forall n\in \mathbb{N}_0$.因此结论得证.
\end{proof}

\begin{definition}
设$(R,+,\cdot)$是一个整环,$s\in R/\{0\}$,则我们称$(\langle s \rangle^{-1}R,+,\cdot)$是\textbf{$R$对$s$的局部化}.
\end{definition}
\begin{remark}
由\reflem{lemma:由整环中非零元素生成的子幺半群都是乘法子集}可知$\langle s \rangle$是$R$的一个乘法子集,故上述定义是良定义的.并且由\refpro{proposition:由单个元素生成的子幺半群的集合表示}可知$\langle s \rangle=\{1,s,s^2,\cdots\}$,从而
\begin{align*}
\langle s\rangle ^{-1}R=\left\{ \frac{r}{s}:r\in R,s\in \langle s\rangle \right\} =\left\{ \frac{r}{s^n}:r\in R,n\in \mathbb{N} _0 \right\} .
\end{align*}
\end{remark}
\begin{note}
整数环$\mathbb{Z}$对$3$的局部化就是$\langle 3\rangle^{-1}\mathbb{Z}=\{\frac{m}{3^n}:n\in \mathbb{N}_0\}.$
\end{note}

\begin{lemma}\label{lemma:交换环上的素理想的补集是乘法子集}
设 $(R, +, \cdot)$ 是一个交换环,而 $\mathfrak{p} \lhd  R$ 是一个素理想,则 $S = R \setminus \mathfrak{p}$ 是一个乘法子集。
\end{lemma}
\begin{proof}
首先,由$\mathfrak{p} \lhd  R$ 是一个素理想可知$(\mathfrak{p},+)<(R,+)$,从而$0\in \mathfrak{p}$,于是$0\notin R/\mathfrak{p}=S$.

其次,若 $1 \in \mathfrak{p}$,则由\refpro{lemma:理想是整个环的充要条件}可知$\mathfrak{p} = R$,可是素理想根据定义是不能等于整个环的。因此 $1 \in R/\mathfrak{p}=S$。

接着,设 $s_1, s_2 \in S=R/\mathfrak{p}$,我们只须证明 $s_1s_2 \in S$。

因为 $s_1 \notin \mathfrak{p}$,$s_2 \notin \mathfrak{p}$,所以根据$\mathfrak{p} \lhd  R$ 是一个素理想及素理想(第一条性质)的逆否命题,
\[s_1s_2 \notin \mathfrak{p}\]

也就是说,
\[s_1s_2 \in R/\mathfrak{p}=S\]

这就证明了素理想的补集是一个乘法子集。
\end{proof}

\begin{definition}
设 $(R, +, \cdot)$ 是一个交换环,而 $\mathfrak{p} \lhd  R$ 是一个素理想,则我们称$((R\setminus \mathfrak{p})^{-1}R,+,\cdot)$是\textbf{$R$在素理想$\mathfrak{p}$上的局部化},记作$R_{\mathfrak{p}}.$
\end{definition}
\begin{remark}
由\reflem{lemma:交换环上的素理想的补集是乘法子集}可知$R/\mathfrak{p}$是$R$的一个乘法子集,故上述定义是良定义的.
\end{remark}
\begin{note}
整数环$\mathbb{Z}$在其素理想$3\mathbb{Z}$上的局部化就是$\{\frac{m}{n}:m\in \mathbb{Z},n\notin 3\mathbb{Z}\}.$
\end{note}








\end{document}
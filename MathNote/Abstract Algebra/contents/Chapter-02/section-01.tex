\documentclass[../../main.tex]{subfiles}
\graphicspath{{\subfix{./image/}}} % 指定图片目录,后续可以直接使用图片文件名
% 注意这里的文件路径不能用 ../../image/ ,否则用latexmk编译子文件会报错

% 例如:
% \begin{figure}[H]
% \centering
% \includegraphics[scale=0.4]{图.png}
% \caption{}
% \label{figure:图}
% \end{figure}
% 注意:上述\label{}一定要放在\caption{}之后,否则引用图片序号会只会显示??.

\begin{document}

\section{分式域}

\begin{definition}[分式域]
若交换整环 \( R \) 是域 \( F \) 的子环且 \( \forall a \in F, \exists b, c \in R \)且$c\neq 0$,使得
\[
a=bc^{-1},\text{其中}c^{-1}\text{是}c\text{在域}F\text{中的乘法逆元},
\]
则称 \( F \) 为 \( R \) 的\textbf{分式域},记为$F=\mathrm{Frac}\left( R \right)$.
\end{definition}

\begin{theorem}\label{theorem:抽象代数--定理2.1.1}
设 \( R \) 为交换整环,则 \( R \) 的分式域一定存在.
\end{theorem}
\begin{remark}
关于 \( R \) 的条件可放宽为 \( R \) 是无零因子交换环,即 \( R \) 中不必有幺元.
\end{remark}
\begin{proof}
令 \( R^* = R \setminus \{0\} \),在集合 \( R \times R^* \) 中定义加法与乘法,\( \forall (a, b), (c, d) \in R \times R^* \),
\begin{align}
(a, b) + (c, d) &= (ad + bc, bd), \label{eq:fw3t4fevwf2.1.1} \\
(a, b)(c, d) &= (ac, bd). \label{eq:fw3t4fevwf2.1.2}
\end{align}

易验证 \( R \times R^* \) 对上述加法与乘法都是交换幺半群,它们的零元素及幺元分别为 \( (0, 1) \),\( (1, 1) \)。在 \( R \times R^* \) 中定义一个关系“\( \sim \)”,
\[
(a, b) \sim (c, d), \quad \text{若 } ad = bc.
\]

先证明关系 \( \sim \) 是等价关系。事实上,由 \( ab = ab \) 知 \( (a, b) \sim (a, b) \)。又若 \( (a, b) \sim (c, d) \),即 \( ad = cb \),因而 \( (c, d) \sim (a, b) \)。最后,假设 \( (a, b) \sim (c, d) \),\( (c, d) \sim (e, f) \),则 \( adf = bcf = bde \)。由 \( R \) 是交换整环,\( d \neq 0 \),于是 \( af = be \),即 \( (a, b) \sim (e, f) \)。

其次证明关系 \( \sim \) 对于 \( R \times R^* \) 中的乘法是同余关系,设
\[
(a, b) \sim (c, d), \quad (e, f) \sim (g, h).
\]
于是由式 \(\eqref{eq:fw3t4fevwf2.1.2}\) 知
\[
(a, b)(e, f) = (ae, bf), \quad (c, d)(g, h) = (cg, dh),
\]
而由\( R \)是交换整环可得\( (ae)(dh) = adeh = bcfg = (bf)(cg) \),即有
\[
(a, b)(e, f) \sim (c, d)(g, h).
\]

再次证明关系 \( \sim \) 对于 \( R \times R^* \) 中的加法是同余关系。设
\[
(a, b) \sim (c, d), \quad (e, f) \sim (g, h),
\]
则由式 \(\eqref{eq:fw3t4fevwf2.1.1}\) 知
\[
(a, b) + (e, f) = (af + be, bf), \quad (c, d) + (g, h) = (ch + dg, dh).
\]
这时由\( R \)是交换整环可得
\[
(af + be)dh = adfh + bedh = bcfh + fgbd = (ch + dg)bf,
\]
因而 \( ((a, b) + (e, f)) \sim ((c, d) + (g, h)) \)。

令 \( F = R \times R^*/\sim \) 为商集合,以 \( \frac{a}{b} \) 表示 \( (a, b) \) 所在等价类。于是由\refthe{Set Theory-theorem:同余关系诱导商集中的乘法-定理1.1.3},在 \( F \) 中有加法与乘法运算如下:
\begin{align*}
\frac{a}{b} + \frac{c}{d} &= \frac{ad + bc}{bd}, \\
\frac{a}{b} \cdot \frac{c}{d} &= \frac{ac}{bd}.
\end{align*}
再由\refthe{theorem:抽象代数--半群中的同余关系可导出商集合也是商半群}知$F$对加法与乘法都是交换幺半群。零元素与幺元素为 \( \frac{0}{1} \),\( \frac{1}{1} \),记 \( 0 = \frac{0}{1} \),\( 1 = \frac{1}{1} \).对$\forall d\in R$,由于 \( 0 \cdot d = 0 \cdot 1 \),故有 \( (0, 1) \) 与 \( (0, d) \) 等价,即 \( \frac{0}{1} = \frac{0}{d} \)。又由 \( 1 \cdot d = 1 \cdot d \) 知 \( \frac{1}{1} = \frac{d}{d} = 1 \)。

由
\[
\frac{a}{b} + \frac{-a}{b} = \frac{ab - ab}{b^2} = \frac{0}{b^2} = 0
\]
知 \( F \) 对加法为交换群。

又若 \( \frac{a}{b} \neq 0 \),即 \( a \neq 0 \),则 \( (b, a) \in R \times R^* \),即 \( \frac{b}{a} \in F \)。这时
\[
\frac{a}{b} \cdot \frac{b}{a} = \frac{ab}{ab} = 1,
\]
故 \( F^* = F \setminus \{0\} \) 对乘法为交换群且 \( \left( \frac{a}{b} \right)^{-1} = \frac{b}{a} \)。又由
\begin{align*}
\left( \frac{a}{b} + \frac{c}{d} \right) \cdot \frac{f}{e} &= \frac{ad + bc}{bd} \cdot \frac{f}{e} = \frac{adf + bcf}{bde} \\
&= \frac{adef + bcef}{bdee} = \frac{af}{be} + \frac{cf}{de} \\
&= \frac{a}{b} \cdot \frac{f}{e} + \frac{c}{d} \cdot \frac{f}{e}.
\end{align*}
知 \( F \) 中加法与乘法间分配律成立,故 \( F \) 为域。

记$R_1 \triangleq \left\{ \frac{a}{1}:a\in R \right\} $,则
\[
\frac{a}{1} + \frac{b}{1} = \frac{a + b}{1}, \quad \frac{a}{1} \cdot \frac{b}{1} = \frac{ab}{1},
\]
故$R_1$是$F$的子环.由于 \( \frac{a}{1} = \frac{b}{1} \) 当且仅当 \( a = b \),故$\frac{a}{1}\to a$是$R_1$到$R$上的一个良定义的映射,不难验证其也是同构映射,因此可将 \( R \) 作为 \( F \) 的子环。而对 \( F \) 中任一元素 \( \frac{a}{b} \) 有
\[
\frac{a}{b} = \frac{a}{1} \cdot \frac{1}{b} = \frac{a}{1} \cdot \left( \frac{b}{1} \right)^{-1},
\]
故 \( F \) 是 \( R \) 的分式域.综上可知
\begin{align*}
F=R\times R^*/\sim =\left\{ \frac{a}{b}\mid a\in R,b\in R^* \right\} 
\end{align*}
是$R$的分式域.

\end{proof}

\begin{theorem}\label{theorem:分式域是包含R的最小域且唯一}
交换整环 \( R \) 的分式域 \( F \) 是以 \( R \) 为子环的最小体,进而\( F \) 是以 \( R \) 为子环的最小域,因而 \( R \) 的分式域唯一.
\end{theorem}
\begin{remark}
关于 \( R \) 的条件可放宽为 \( R \) 是无零因子交换环,即 \( R \) 中不必有幺元.
\end{remark}
\begin{proof}
设 \( F' \) 是体且以 \( R \) 为子环,则 \( F' \) 中子集
\[
F_1 = \{ab^{-1} \mid a, b \in R, b \neq 0\}
\]
是 \( F' \) 的子体,事实上,对$\forall ab^{-1}, cd^{-1} \in F_1$,有
\[
ab^{-1} + cd^{-1} = (ad + cd)(bd)^{-1}, \quad -(ab^{-1}) = (-a)b^{-1},
\]
故 \( F_1 \) 对加法为 \( F' \) 的子群. 又若 \( ab^{-1}, cd^{-1} \in F_1 \setminus \{0\} \),则
\[
(ab^{-1})(cd^{-1})^{-1} = (ad)(bc)^{-1},
\]
故 \( F_1 \setminus \{0\} \) 对乘法为\( F' \setminus \{0\} \) 的子群,因此 \( F_1 \) 是 \( F' \) 的子体. 由\refthe{theorem:抽象代数--定理2.1.1}知,记$\frac{a}{b}\triangleq \left\{ \left( c,d \right) \in R\times R^*\mid ad=bc \right\}$,则$R$的一个分式域为
\begin{align*}
F=\left\{ \frac{a}{b}\mid a\in R,b\in R^* \right\} .
\end{align*}
又 \( \frac{a}{b} \to ab^{-1} \) 是 \( R \) 的分式域 \( F \) 到 \( F_1 \) 上的同构,故可将 \( F \) 与 \( F_1 \) 等同,因而 \( F \subseteq F' \).

再设$M$是以$R$为子环的域,则$M$也是体,从而$F\subseteq M$.故$F$是以$R$为子环的最小域.因而$R$的分式域唯一,否则与$F$是以$R$为子环的最小域矛盾!

\end{proof}

\begin{corollary}\label{corollary:分式域同构等价于生成环同构}
设$R_1,R_2$都是交换整环,$F_1,F_2$分别是$R_1,R_2$的分式域,则$R_1\cong R_2\iff F_1\cong F_2$.
\end{corollary}
\begin{proof}


\end{proof}

\begin{corollary}\label{corollary:交换整环的分式与的具体形式}
设 \( R \) 为交换整环,$R^*=R\setminus \{0\}$,则在$R \times R^*$中定义一个关系“$\sim$”
\begin{align*}
\left( a,b \right) \sim \left( c,d \right) ,\,\,\text{若}ad=bc.
\end{align*}
则$\sim$是同余关系.
以$\frac{a}{b}$表示$(a,b)$所在等价类,即
\begin{align*}
\frac{a}{b} =\left\{ (c,d) \in R \times R^* \mid (a,b)\sim (c,d) \right\}= \left\{ (c,d) \in R \times R^* \mid ad = bc \right\}.
\end{align*}
则
\begin{align*}
F = R \times R^*/\sim = \left\{ \frac{a}{b} \mid a \in R, b \in R^* \right\}
\end{align*}
是$R$的分式域,也是以$R$为子环的最小域,进而$F$就是$R$的唯一分式域。并且记$a=\frac{a}{1},\forall a\in R.$其中$F$的加法和乘法分别定义为
\begin{align*}
\frac{a}{b}+\frac{c}{d}=\frac{ad+bc}{bd},\quad \frac{a}{b}\cdot \frac{c}{d}=\frac{ac}{bd},\quad \forall \frac{a}{b},\frac{c}{d}\in F.
\end{align*}
进而$\frac{a}{b}$加法和乘法逆元分别为
\begin{align*}
-\frac{a}{b},\quad \frac{b}{a}.
\end{align*}
\end{corollary}
\begin{proof}
由\refthe{theorem:抽象代数--定理2.1.1}和\refthe{theorem:分式域是包含R的最小域且唯一}立得.

\end{proof}

\begin{example}
设 \( \mathbb{P} \) 是任一数域,设 \( \mathbb{P}[x] \) 的分式域为 \( \mathbb{P}(x) \),则
\[
\mathbb{P}(x) = \left\{ \frac{f(x)}{g(x)} \mid f(x), g(x) \in \mathbb{P}[x], g(x) \neq 0 \right\}.
\]
\end{example}
\begin{proof}


\end{proof}

\begin{proposition}\label{proposition:mZ的分式域都是Q}
设 \( m \) 是非零整数,则 \( m\mathbb{Z} \) 的分式域为 \( \mathbb{Q} \).
\end{proposition}
\begin{proof}


\end{proof}










\end{document}
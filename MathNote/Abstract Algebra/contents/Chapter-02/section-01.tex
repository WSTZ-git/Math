\documentclass[../../main.tex]{subfiles}
\graphicspath{{\subfix{../../image/}}} % 指定图片目录,后续可以直接使用图片文件名。

% 例如:
% \begin{figure}[h]
% \centering
% \includegraphics{image-01.01}
% \caption{图片标题}
% \label{fig:image-01.01}
% \end{figure}
% 注意:上述\label{}一定要放在\caption{}之后,否则引用图片序号会只会显示??.

\begin{document}

\section{环}

\begin{definition}[环]
我们称$(R, +, \cdot)$ 是一个\textbf{环},当 $(R, +)$ 是个阿贝尔群,$(R, \cdot)$ 是个幺半群,且乘法对加法有左右分配律,即
\begin{align*}
&\forall a,b,c\in R, a(b + c)=ab + ac ,\\
&\forall a,b,c\in R, (a + b)c=ac + bc .
\end{align*} 
\end{definition}
\begin{note}
最常见的环是
\end{note}

\begin{definition}[交换环]
设$(R, +, \cdot)$ 是一个环,我们称 $R$ 是一个\textbf{交换环},当 $R$ 对乘法有交换律,即
\begin{align*}
\forall a,b\in R, ab = ba.
\end{align*}
\end{definition}

\begin{example}

\end{example}
\begin{proof}

\end{proof}

\begin{proposition}
设$(R, +, \cdot)$ 是一个环,而 $a,b,c\in R$,则
\begin{enumerate}[(1)]
\item $a0 = 0a = 0,$

\item $a(-b)=(-a)b=-(ab),$

\item $(-a)(-b)=ab.$
\end{enumerate}
\end{proposition}
\begin{proof}
\begin{enumerate}[(1)]
\item 首先,利用分配律,
\begin{align*}
a0 = a(0 + 0)=a0 + a0.
\end{align*}
因此 $a0 = 0$。根据对称性,$0a = a$.

\item 根据对称性,我们只须证明 $a(-b)=-(ab)$.而这是因为
\begin{align*}
a(-b)+ab = a(-b + b)=a0 = 0.
\end{align*}

\item 利用两次(2),我们就得到
\begin{align*}
(-a)(-b)=-(a(-b))=-(-(ab)) = ab.
\end{align*} 
\end{enumerate}
\end{proof}











\end{document}
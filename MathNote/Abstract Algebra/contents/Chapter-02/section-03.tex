\documentclass[../../main.tex]{subfiles}
\graphicspath{{\subfix{../../image/}}} % 指定图片目录,后续可以直接使用图片文件名。

% 例如:
% \begin{figure}[h]
% \centering
% \includegraphics{image-01.01}
% \caption{图片标题}
% \label{fig:image-01.01}
% \end{figure}
% 注意:上述\label{}一定要放在\caption{}之后,否则引用图片序号会只会显示??.

\begin{document}

\section{理想}

\begin{definition}[由子集生成的理想]
设 $(R, +, \cdot)$ 是一个环,而 $A \subset R$。则 $(A)$,称为\textbf{由 $\boldsymbol{A}$ 生成的理想},定义为所有 $R$ 中包含 $A$ 的理想的交集,即
\begin{align*}
(A) = \bigcap \{I \subset R : I \supset A, I \lhd R\}.
\end{align*} 
\end{definition}
\begin{note}
因为$R\lhd R$且$A\subset R$,所以$R\subset (A)$.故$(A)\ne \varnothing.$
\end{note}

\begin{proposition}[生成的理想还是理想]
设 $(R, +, \cdot)$ 是一个环,而 $A \subset R$,则 $(A) \lhd R$。
\end{proposition}
\begin{proof}
 首先,取交集的集族非空,因为整个环 $R$ 是包含了 $A$ 的一个理想(对加法构成子群,且 “吸收” 了乘法)。

由于集族中每一个理想都是加法子群。因此根据\refproposition{proposition:子群的任意交仍是子群}可知,它们的交还是加法子群。我们只须检验乘法的 “吸收” 性,即 $R(A) \subset (A)$,及 $(A)R \subset (A)$。根据对称性,我们证明第一个包含关系。假设 $r \in R$,$a \in (A)$,则对于任意集族中的理想 $I$,我们都有 $a \in I$。故 $ra \in I$。这对于任意这样的理想 $I$ 都是成立的,因此 $ra \in (A)$。这就证明了 $(A)$ 是 $R$ 的子环。 
\end{proof}

\begin{definition}
设 $(R, +, \cdot)$ 是一个环,而 $a \in R$,则我们定义
\begin{align*}
(a) = (\{a\}).
\end{align*}
称为\textbf{由 $\boldsymbol{a}$ 生成的主理想}。一般地,若一个理想能被一个元素生成,我们就称其为\textbf{主理想}。

对于 $a_1, \cdots, a_n \in R$,我们定义
\begin{align*}
(a_1, \cdots, a_n) = (\{a_1, \cdots, a_n\}).
\end{align*}
一般地,若一个理想能被有限个元素生成,我们就称其为\textbf{有限生成的理想}。 
\end{definition}

\begin{proposition}
设$(R, +, \cdot)$ 是一个交换环,而 $a \in R$,则
\begin{align*}
(a) = Ra = \{ra : r \in R\}.
\end{align*}
一般地,若 $a_1, \cdots, a_n \in R$,则
\begin{align*}
(a_1, \cdots, a_n) = Ra_1 + \cdots + Ra_n = \{r_1a_1 + \cdots + r_na_n : r_1, \cdots, r_n \in R\} .
\end{align*}
\end{proposition}
\begin{proof}
显然有限生成的理想是主理想的特例,故我们只须证明第二个等式。

要证明 $(A) = I$,我们只须证明两点。一,$I$ 是包含 $A$ 的理想;二,每一个包含 $A$ 的理想都会包含 $A$。

首先,要证明 $Ra_1 + \cdots + Ra_n$ 是个理想。对加法而言,$0 = 0a_1 + \cdots + 0a_n \in Ra_1 + \cdots + Ra_n$,而且对 $r_1a_1 + \cdots + r_na_n, s_1a_1 + \cdots + s_na_n (r_i, s_i \in R)$,我们有
\begin{align*}
(r_1a_1 + \cdots + r_na_n) - (s_1a_1 + \cdots + s_na_n) = (r_1 - s_1)a_1 + \cdots + (r_n - s_n)a_n \in Ra_1 + \cdots + Ra_n 。
\end{align*}
因此 $Ra_1 + \cdots + Ra_n$ 对加法构成子群。

接下来,因为 $R$ 是交换环,我们只须证明 $R(Ra_1 + \cdots + Ra_n) \subset (Ra_1 + \cdots + Ra_n)$。而这是因为
\begin{align*}
R(Ra_1 + \cdots + Ra_n) = RRa_1 + \cdots + RRa_n = Ra_1 + \cdots + Ra_n 。
\end{align*}
这样,我们就证明了 $Ra_1 + \cdots + Ra_n$ 是个理想,而且显然包含 $\{a_1, \cdots, a_n\}$。

另一方面,若 $I$ 是一个包含了 $a_1, \cdots, a_n$ 的理想,那么根据加法的封闭性及乘法的 “吸收” 性,
\begin{align*}
I \supset Ra_1 + \cdots + Ra_n 。
\end{align*}

综上所述,这就证明了这个命题。 
\end{proof}





\end{document}
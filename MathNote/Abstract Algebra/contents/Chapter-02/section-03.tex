\documentclass[../../main.tex]{subfiles}
\graphicspath{{\subfix{../../image/}}} % 指定图片目录,后续可以直接使用图片文件名。

% 例如:
% \begin{figure}[H]
% \centering
% \includegraphics[scale=0.4]{图.png}
% \caption{}
% \label{figure:图}
% \end{figure}
% 注意:上述\label{}一定要放在\caption{}之后,否则引用图片序号会只会显示??.

\begin{document}

\section{对称多项式}

\begin{definition}
设 \( R \) 是一个交换幺环,\( R[x_1, x_2, \cdots, x_n] \) 是 \( R \) 上的 \( n \) 元多项式环. 如果\( R[x_1, x_2, \cdots, x_n] \)中的$n$元多项式\( f(x_1, x_2, \cdots, x_n) \) 中非零单项式都是 \( k \) 次的,那么称 \( f(x_1, x_2, \cdots, x_n) \) 为一个\textbf{\( \boldsymbol{k} \) 次齐次多项式}.
\end{definition}
\begin{remark}
显然任意两个齐次多项式的乘积仍是齐次多项式.
\end{remark}

\begin{theorem}[齐次多项式分解]\label{theorem:多项式的齐次分解}
设 \( R \) 是一个交换幺环,\( R[x_1, x_2, \cdots, x_n] \) 是 \( R \) 上的 \( n \) 元多项式环. \( f(x_1, x_2, \cdots, x_n) \)是\( R[x_1, x_2, \cdots, x_n] \)中的$n$元多项式且$\text{deg}\,f=m$,则存在$m$个齐次多项式$f_1,f_2,\cdots,f_m$,使得
\begin{align*}
f=f_0+f_1+\cdots+f_m.
\end{align*}
且上述分解(除所含零外)是唯一的.
\end{theorem}
\begin{proof}
若$f(x_1,x_2,\cdots,x_n)=0$,则$\text{deg}\,f=0$,取$f_0=f=0$即可.

若 \( f(x_1, x_2, \cdots, x_n) \neq 0 \),则有
\begin{align*}
f(x_1, x_2, \cdots, x_n) = \sum_{k_1k_2\cdots k_n} a_{k_1k_2\cdots k_n} x_1^{k_1} x_2^{k_2} \cdots x_n^{k_n} = \sum_{k=0}^m \left( \sum_{\substack{k_1 + k_2 + \cdots + k_n = k}} a_{k_1k_2\cdots k_n} x_1^{k_1} x_2^{k_2} \cdots x_n^{k_n} \right),
\end{align*}
其中,当 \( k \geqslant 1 \) 时,令
\begin{align*}
f_k(x_1, x_2, \cdots, x_n) &= \sum_{\substack{k_1 + k_2 + \cdots + k_n = k}} a_{k_1k_2\cdots k_n} x_1^{k_1} x_2^{k_2} \cdots x_n^{k_n},
\end{align*}
这是一个 \( k \) 次齐次多项式或零,\( f_0 \in R \). 于是
\begin{align*}
f = f_0 + f_1 + \cdots + f_m.
\end{align*}
设存在另外$m$个齐次多项式$f_0',f_1',\cdots,f_m'$,使得
\begin{align*}
f=f_0'+f_1'+\cdots+f_m'.
\end{align*}
令$h_k=f_k-f_k'(k=0,1,\cdots,m)$,则由$f_k,f_k'$的齐次性知$\deg h_k=k\text{或}0.$并且
\begin{align*}
h_0+h_1+\cdots+h_m=(f_0-f_0')+(f_1-f_1')+\cdots+(f_m-f_m')=0.
\end{align*}
因此$\deg (h_0+h_1+\cdots+h_m)=\underset{k=0,1,\cdots ,k}{\max}\left\{ \mathrm{deg}\,h_k \right\}=0.$故$\mathrm{deg}\,h_k=0\left( k=0,1,\cdots ,m \right),$即$f_k=f_k'\left( k=0,1,\cdots ,m \right).$
故上述分解(除所含零外)是唯一的.

\end{proof}

\begin{definition}[单项式的字典排序法]
设 \( R \) 是交换幺环,\( R[x_1, x_2, \cdots, x_n] \) 是 \( R \) 上的 \( n \) 元多项式环,\( ax_1^{k_1}x_2^{k_2}\cdots x_n^{k_n} \),\( bx_1^{l_1}x_2^{l_2}\cdots x_n^{l_n} \) 是两个非零单项式. 若有 \( s \),使得
\begin{align*}
k_i = l_i, \, i = 1, 2, \cdots, s, \quad k_{s+1} > l_{s+1},
\end{align*}
则称 \( ax_1^{k_1}x_2^{k_2}\cdots x_n^{k_n} \) 高于 \( bx_1^{l_1}x_2^{l_2}\cdots x_n^{l_n} \),记为
\begin{align*}
ax_1^{k_1}x_2^{k_2}\cdots x_n^{k_n} > bx_1^{l_1}x_2^{l_2}\cdots x_n^{l_n}.
\end{align*}
\end{definition}
\begin{note}
例如,当 \( n = 3 \) 时,有 \( x_1^3x_2x_3^5 > x_1^3x_3^6 > x_1x_3^5 \).
\end{note}

\begin{theorem}[单项式的字典排序法的基本性质]\label{theorem:单项式的字典排序法的基本性质}
设 \( R \) 是交换幺环,\( R[x_1, x_2, \cdots, x_n] \) 是 \( R \) 上的 \( n \) 元多项式环,\( ax_1^{k_1}x_2^{k_2}\cdots x_n^{k_n} \),\( bx_1^{l_1}x_2^{l_2}\cdots x_n^{l_n} \) 是两个非零单项式. 
\begin{enumerate}[(1)]
\item \textbf{传递性:}若
\begin{align*}
ax_1^{k_1}x_2^{k_2}\cdots x_n^{k_n} > bx_1^{l_1}x_2^{l_2}\cdots x_n^{l_n}, \quad bx_1^{l_1}x_2^{l_2}\cdots x_n^{l_n} > cx_1^{m_1}x_2^{m_2}\cdots x_n^{m_n},
\end{align*}
则
\begin{align*}
ax_1^{k_1}x_2^{k_2}\cdots x_n^{k_n} > cx_1^{m_1}x_2^{m_2}\cdots x_n^{m_n}.
\end{align*}

\item 若 \( ax_1^{k_1}x_2^{k_2}\cdots x_n^{k_n} > bx_1^{l_1}x_2^{l_2}\cdots x_n^{l_n} \),则有
\begin{align*}
ax_1^{k_1+m_1}x_2^{k_2+m_2}\cdots x_n^{k_n+m_n} > bx_1^{l_1+m_1}x_2^{l_2+m_2}\cdots x_n^{l_n+m_n}.
\end{align*}

\item\label{theorem:单项式的字典排序法的基本性质-3} 设 \( f, g \in R[x_1, x_2, \cdots, x_n] \) 且 \( f \neq 0 \),\( g \neq 0 \). 若 \( f \) 的最高项与 \( g \) 的最高项之积不为 \( 0 \),则此积为 \( f \cdot g \) 的最高项.

如果 \( f \) 与 \( g \) 的最高项系数之一为 \( R \) 中非零因子,则 \( fg \) 的最高项为 \( f \) 的最高项与 \( g \) 的最高项之积.

特别地,当 \( R \) 是交换整环且 \( f, g \) 为 \( R[x_1, x_2, \cdots, x_n] \) 中非零元素时,\( fg \) 的最高项为 \( f \) 的最高项与 \( g \) 的最高项的乘积.
\end{enumerate}
\end{theorem}
\begin{proof}
\begin{enumerate}[(1)]
\item 

\item 

\item 
\end{enumerate}

\end{proof}

\begin{theorem}\label{theorem:对称多项式和pi'}
设 \( R \) 是交换幺环,\( R[x_1, x_2, \cdots, x_n] \) 是 \( R \) 上的 \( n \) 元多项式环,对 \( n \) 个文字的对称群 \( S_n \) 中任一元素 \( \pi \),$\pi^{-1}$为$\pi$在$S_n$中的逆元,则存在$R[x_1, x_2, \cdots, x_n]$中的自同构满足
\begin{align*}
\pi'(a) = a, \forall a \in R, \quad \pi'(x_i) = x_{\pi(i)}, i = 1, 2, \cdots, n.
\end{align*}
且对任意
\begin{align*}
f = \sum\limits_{k_1k_2\cdots k_n} a_{k_1k_2\cdots k_n} x_1^{k_1} x_2^{k_2} \cdots x_n^{k_n}\in R[x_1,x_2,\cdots,x_n],
\end{align*}
有
\begin{align*}
(\pi' f)(x_1,x_2,\cdots ,x_n)=\sum_{k_1k_2\cdots k_n}{a_{k_{\pi (1)}k_{\pi (2)}\cdots k_{\pi (n)}}x_{1}^{k_1}x_{2}^{k_2}}\cdots x_{n}^{k_n}.
\end{align*}
并且\( \{\pi' \mid \pi \in S_n\} \) 是 \( R[x_1, x_2, \cdots, x_n] \) 的自同构群的子群.

若 \( f \in R[x_1, x_2, \cdots, x_n] \) 满足
\begin{align*}
\pi' f = f, \quad \forall \pi \in S_n,
\end{align*}
则称 \( f \) 为 \( x_1, x_2, \cdots, x_n \) 的一个\textbf{对称多项式}.
\end{theorem}
\begin{proof}
令$i$为\( R \) 到 \( R[y_1, y_2, \cdots, y_n] \) 的嵌入映射,满足\( i(a) = a (\forall a \in R) \).容易验证$i$是\( R \) 到 \( R[y_1, y_2, \cdots, y_n] \) 的环同态映射.
由\refthe{theorem:n元多项式环开拓定理}可将$i$开拓为 \( R[x_1, x_2, \cdots, x_n] \) 的一个自同态 \( \pi' \)(在\refthe{theorem:n元多项式环开拓定理}中取$u_i=x_{\pi(i)}$)满足
\begin{align*}
\pi'(a) &= a, \forall a \in R, \quad \pi'(x_i) = x_{\pi(i)}, i = 1, 2, \cdots, n.
\end{align*}
显然,对 \( f = \sum\limits_{k_1k_2\cdots k_n} a_{k_1k_2\cdots k_n} x_1^{k_1} x_2^{k_2} \cdots x_n^{k_n} \) 有
\begin{align*}
(\pi ' f)(x_1,x_2,\cdots ,x_n)&=\sum_{k_1k_2\cdots k_n}{a_{k_1k_2\cdots k_n}x_{\pi (1)}^{k_1}x_{\pi (2)}^{k_2}}\cdots x_{\pi (n)}^{k_n}
\\
&=\sum_{k_1k_2\cdots k_n}{a_{k_1k_2\cdots k_n}x_{1}^{k_{\pi ^{-1}(1)}}x_{2}^{k_{\pi ^{-1}(2)}}}\cdots x_{n}^{k_{\pi ^{-1}(n)}}
\\
&\xlongequal[k_i=k_{\pi ^{-1}\left( \pi \left( i \right) \right)}=t_{\pi \left( i \right)}]{t_i=k_{\pi ^{-1}\left( i \right)}}\sum_{t_1t_2\cdots t_n}{a_{t_{\pi (1)}t_{\pi (2)}\cdots t_{\pi (n)}}x_{1}^{t_1}x_{2}^{t_2}}\cdots x_{n}^{t_n}
\\
&=\sum_{k_1k_2\cdots k_n}{a_{k_{\pi (1)}k_{\pi (2)}\cdots k_{\pi (n)}}x_{1}^{k_1}x_{2}^{k_2}}\cdots x_{n}^{k_n}.
\end{align*}
同理,由\refthe{theorem:n元多项式环开拓定理}可将$i$开拓为 \( R[x_1, x_2, \cdots, x_n] \) 的一个自同态 \( \left( \pi ^{-1} \right)' \)(在\refthe{theorem:n元多项式环开拓定理}中取$u_i=x_{\pi^{-1}(i)}$)满足
\begin{align*}
\left( \pi ^{-1} \right)' (a)=a,\forall a\in R,\quad \left( \pi ^{-1} \right)' (x_i)=x_{\pi ^{-1}(i)},i=1,2,\cdots ,n.
\end{align*}
从而
\begin{gather*}
\left( \pi ^{-1} \right) ' \pi ' \left( a \right) =a,\forall a\in R,\quad \left( \pi ^{-1} \right) ' \pi ' (x_i)=\left( \pi ^{-1} \right) ' \left( x_{\pi (i)} \right) =x_{\pi ^{-1}\pi (i)}=x_i,i=1,2,\cdots ,n,
\\
\pi ' \left( \pi ^{-1} \right) ' \left( a \right) =a,\forall a\in R,\quad \pi ' \left( \pi ^{-1} \right) ' (x_i)=\pi ' \left( x_{\pi ^{-1}(i)} \right) =x_{\pi \pi ^{-1}(i)}=x_i,i=1,2,\cdots ,n,
\end{gather*}
即$\left( \pi ^{-1} \right)' \pi' =\pi' \left( \pi ^{-1} \right)' =\mathrm{id}_{R[x_1,x_2,\cdots ,x_n]}.$因此$\pi'$是双射,$(\pi^{-1})'$为其逆映射.故\( \pi' \) 是 \( R[x_1, x_2, \cdots, x_n] \) 的一个自同构.


由上述证明知$\forall \pi'\in \{\pi' \mid \pi \in S_n\}$,都存在逆元$(\pi^{-1})'$.

对 \(\forall \pi_1', \pi_2' \in \{\pi' \mid \pi \in S_n\} \),则由
\begin{gather*}
\pi_1'\pi_2'(a) = a=(\pi_1\pi_2)'(a), \quad \forall a \in R, \\
\pi_1'\pi_2'(x_i) = \pi_1'(x_{\pi_2(i)}) = x_{\pi_1\pi_2(i)}=(\pi_1\pi_2)'(x_i), \quad i = 1, 2, \cdots, n
\end{gather*}
知 \( \pi_1'\pi_2'=(\pi_1\pi_2)'\in \{\pi' \mid \pi \in S_n\} \).故$\{\pi' \mid \pi \in S_n\}$对乘法封闭.由映射的乘积必满足结合律知$\{\pi' \mid \pi \in S_n\}$对乘法也满足结合律.

对 \( S_n \) 中的幺元 \( \text{id} \) 有
$(\text{id})' = \text{id}_{R[x_1, x_2, \cdots, x_n]}$
也是$\{\pi' \mid \pi \in S_n\}$的幺元.

综上可知 \( \{\pi' \mid \pi \in S_n\} \) 是 \( R[x_1, x_2, \cdots, x_n] \) 的自同构群的子群.

\end{proof}

\begin{lemma}
设 \( R \) 是交换幺环,\( R[x_1, x_2, \cdots, x_n] \) 是 \( R \) 上的 \( n \) 元多项式环,\( R[x_1, x_2, \cdots, x_n] \) 中的多项式
\begin{align*}
s_1 &= x_1 + x_2 + \cdots + x_n, \\
s_2 &= x_1^2 + x_2^2 + \cdots + x_n^2, \\
&\cdots\cdots \\
s_m &= x_1^m + x_2^m + \cdots + x_n^m
\end{align*}
都是对称多项式,称为$\mathbb{Newton}$\textbf{对称幂和}或\textbf{等幂和}.
\end{lemma}
\begin{proof}


\end{proof}

\begin{lemma}\label{lemma:抽象代数--引理2.3.1}
设 \( R \) 是交换幺环,\( R[x_1, x_2, \cdots, x_n] \) 是 \( R \) 上的 \( n \) 元多项式环,\( R[x_1, x_2, \cdots, x_n] \) 中的多项式
\begin{align*}
p_1 &= s_1 = x_1 + x_2 + \cdots + x_n, \\
p_2 &= \sum_{\substack{1 \leqslant i < j \leqslant n}} x_i x_j = x_1 x_2 + \cdots + x_1 x_n + x_2 x_3 + \cdots + x_2 x_n + \cdots + x_{n-1} x_n, \\
&\cdots\cdots \\
p_{n-1} &= \sum_{\substack{1 \leqslant j_1 < j_2 < \cdots < j_{n-1} \leqslant n}} x_{j_1} x_{j_2} \cdots x_{j_{n-1}} , \\
p_n &= x_1 x_2 \cdots x_n.
\end{align*}
等 \( n \) 个齐次多项式都是对称多项式,称它们为 \( n \) \textbf{元初等对称多项式}.
\end{lemma}
\begin{proof}
令 \( p_0 = 1 \). 考虑 \( R[x_1, x_2, \cdots, x_n] = S \) 上的一元多项式环 \( S[x] \) 中的元素
\begin{align}\label{eq:::-:90ji34fi03g}
g(x) = \prod_{i=1}^n (x - x_i) = \sum_{k=0}^n (-1)^k p_k x^{n - k} = x^n - p_1 x^{n - 1} + \cdots + (-1)^{n - 1} p_{n - 1} x + (-1)^n p_n.
\end{align}
设 \( \pi \in S_n \),由\refthe{theorem:对称多项式和pi'}知存在$R[x_1,x_2,\cdots,x_n]$中的自同构$\pi'$满足
\begin{align*}
\pi'(a) = a, \forall a \in R, \quad \pi'(x_i) = x_{\pi(i)}, i = 1, 2, \cdots, n.
\end{align*}
于是
\begin{align}\label{eq:::-:90ji34fi03g-1}
g(x) = \prod_{i=1}^n (x - x_{\pi(i)}) = \sum_{k=0}^n (-1)^k (\pi' p_k) x^{n - k},
\end{align}
故比较\eqref{eq:::-:90ji34fi03g}式和\eqref{eq:::-:90ji34fi03g-1}式系数知\( \pi' p_k = p_k (0 \leqslant k \leqslant n) \),即 \( p_1, p_2, \cdots, p_n \) 是对称多项式.

\end{proof}

\begin{lemma}\label{lemma:抽象代数--引理2.3.2}
设 \( R \) 是交换幺环,\( R[x_1, x_2, \cdots, x_n] \) 是 \( R \) 上的 \( n \) 元多项式环,以 \( \Sigma \) 表示 \( R[x_1, x_2, \cdots, x_n] \) 中所有对称多项式的集合,则 \( \Sigma \) 是 \( R[x_1, x_2, \cdots, x_n] \) 的子环且 \( \Sigma \supseteq R \). 又若 \( f \in R[x_1, x_2, \cdots, x_n] \),且有齐次多项式分解
\begin{align*}
f = f_0 + f_1 + \cdots + f_k,
\end{align*}
则 \( f \in \Sigma \) 当且仅当 \( f_i \in \Sigma (0 \leqslant i \leqslant k) \).
\end{lemma}

\begin{proof}
显然 \( \Sigma \supseteq R \). 又若 \( f, g \in \Sigma \),\( \pi \in S_n \),则由\refthe{theorem:对称多项式和pi'}知存在$R[x_1,x_2,\cdots,x_n]$中的自同构$\pi'$满足
\begin{align*}
\pi'(a) = a, \forall a \in R, \quad \pi'(x_i) = x_{\pi(i)}, i = 1, 2, \cdots, n.
\end{align*}
于是
\begin{align*}
\pi'(f - g) &= \pi'(f) - \pi'(g) = f - g, \\
\pi'(fg) &= \pi'(f)\pi'(g) = fg,
\end{align*}
因此$f-g,fg\in \Sigma$,故 \( \Sigma \) 是一个子环.

又若 \( f \in R[x_1, x_2, \cdots, x_n] \),设 \( f = \sum_{i=0}^k f_i \) 为 \( f \) 的齐次多项式分解,\( \pi \in S_n \),则 \( \pi' f = \sum_{i=0}^k \pi' f_i \) 为 \( \pi' f \) 的齐次多项式分解. 因\hyperref[theorem:多项式的齐次分解]{齐次多项式分解唯一},故
\begin{align*}
\pi' f = f \iff \pi' f_i = f_i, \, 0 \leqslant i \leqslant k.
\end{align*}

\end{proof}

\begin{theorem}[对称多项式基本定理]\label{theorem:抽象代数--定理2.3.1--对称多项式基本定理}
设 \( R \) 是交换幺环,\( \Sigma \) 是 \( R \) 上 \( n \) 元多项式 \( R[x_1, x_2, \cdots, x_n] \) 中对称多项式构成的子环,\( p_1, p_2, \cdots, p_n \) 为初等对称多项式,则
\begin{enumerate}[(1)]
\item\label{theorem:抽象代数--定理2.3.1--对称多项式基本定理-1} \( \Sigma = R[p_1, p_2, \cdots, p_n] \);
\item \( p_1, p_2, \cdots, p_n \) 在 \( R \) 上是代数无关的,即$R[p_1, p_2, \cdots, p_n]$也是$R$上的$n$元多项式环.
\end{enumerate}
\end{theorem}
\begin{remark}
\begin{enumerate}
\item 这个定理的等价命题是\textbf{任一对称多项式可唯一地表示为初等对称多项式的多项式}.

\item 这个定理(1)的证明实际上给出了一个对称多项式如何表示为初等对称多项式的多项式的有效办法.
\end{enumerate}
\end{remark}
\begin{proof}
\begin{enumerate}[(1)]
\item 由$p_1,p_2,\cdots,p_n\in \Sigma$和$\Sigma$是环知$R[p_1,p_2,\cdots,p_n]\subseteq \Sigma.$下证$R[p_1,p_2,\cdots,p_n]\supseteq \Sigma.$
只需证明任一齐次对称多项式 \( f \in R[p_1, p_2, \cdots, p_n] \). 假设已经证明,则对任意$ f\in \Sigma$,由\reflem{lemma:抽象代数--引理2.3.2}知$f$有齐次多项式分解$f=f_0+f_1+\cdots+f_k$且$f_i\in \Sigma(0\leqslant i\leqslant k)$,即$f_i(0\leqslant i\leqslant k)$都是齐次对称多项式.于是有假设知$f_i\in R[p_1,p_2,\cdots,p_n](0\leqslant i\leqslant k)$,进而$f\in R[p_1,p_2,\cdots,p_n]$,故$R[p_1,p_2,\cdots,p_n]\supseteq \Sigma.$

设 \( f \) 是 \( m \) 次齐次对称多项式. 按字典序,\( f \) 的最高项为 \( a x_1^{k_1} x_2^{k_2} \cdots x_n^{k_n} \),则必有
\begin{align*}
k_1 \geqslant k_2 \geqslant \cdots \geqslant k_n, \quad k_1 + k_2 + \cdots + k_n = m.
\end{align*}
若不然,设有 \( i \),使得 \( k_i < k_{i+1} \). 于是有 \( \pi \in S_n \),使得
\begin{align*}
\pi(j) = \begin{cases}
j, & j \neq i, i+1, \\
i+1, & j = i, \\
i, & j = i+1.
\end{cases}
\end{align*}
由\refthe{theorem:对称多项式和pi'}知存在$R[x_1,x_2,\cdots,x_n]$中的自同构$\pi'$满足
\begin{align*}
\pi'(a) = a, \forall a \in R, \quad \pi'(x_i) = x_{\pi(i)}, i = 1, 2, \cdots, n.
\end{align*}
由$f$是对称多项式知\( \pi' f = f \).从而\( f \) 中有一项为 \( \pi'(a x_1^{k_1} x_2^{k_2} \cdots x_n^{k_n}) \),但
\begin{align*}
\pi'(ax_{1}^{k_1}x_{2}^{k_2}\cdots x_{n}^{k_n})=ax_{1}^{k_1}\cdots x_{i}^{k_{i+1}}x_{i+1}^{k_i}\cdots x_{n}^{k_n}>ax_{1}^{k_1}x_{2}^{k_2}\cdots x_{n}^{k_n}.
\end{align*}
这与 \( a x_1^{k_1} x_2^{k_2} \cdots x_n^{k_n} \) 为 \( f \) 的最高项矛盾.

令 \( d_i = k_i - k_{i+1} (1 \leqslant i \leqslant n - 1) \) 且 \( d_n = k_n \). 由 \( p_i \) 的最高项为 \( x_1 x_2 \cdots x_i \)及\rrefthe{theorem:单项式的字典排序法的基本性质}{theorem:单项式的字典排序法的基本性质-3}知 \( p_1^{d_1} p_2^{d_2} \cdots p_n^{d_n} \) 的最高项为
\begin{align*}
x_1^{d_1} (x_1 x_2)^{d_2} \cdots (x_1 x_2 \cdots x_n)^{d_n} 
= x_1^{d_1 + d_2 + \cdots + d_n} (x_2)^{d_2 + \cdots + d_n} \cdots (x_n)^{d_n} 
= x_1^{k_1} x_2^{k_2} \cdots x_n^{k_n}.
\end{align*}
由此知 \( m \) 次齐次对称多项式 \( f_1 = f - a p_1^{d_1} p_2^{d_2} \cdots p_n^{d_n} \) 的最高项 \( b x_1^{l_1} x_2^{l_2} \cdots x_n^{l_n} < a x_1^{k_1} x_2^{k_2} \cdots x_n^{k_n} \) 且
\begin{align*}
l_1 \geqslant l_2 \geqslant \cdots \geqslant l_n, \quad l_1 + l_2 + \cdots + l_n = m.
\end{align*}
否则同理可得矛盾!类似可知 \( m \) 次齐次对称多项式 \( f_2 = f_1 - b p_1^{l_1 - l_2} p_2^{l_2 - l_3} \cdots p_n^{l_n} \) 的最高项 \( c x_1^{m_1} x_2^{m_2} \cdots x_n^{m_n} < b x_1^{l_1} x_2^{l_2} \cdots x_n^{l_n} \) 且
\begin{align*}
m_1 \geqslant m_2 \geqslant \cdots \geqslant m_n, \quad m_1 + m_2 + \cdots + m_n = m.
\end{align*}
由于满足 \( k_1 \geqslant k_2 \geqslant \cdots \geqslant k_n \geqslant 0 \) 和 \( k_1 + k_2 + \cdots + k_n = m \) 的 \( n \) 重数组 \( (k_1, k_2, \cdots, k_n) \) 只有有限个,故有限步后可得
\begin{align*}
f = a p_1^{d_1} p_2^{d_2} \cdots p_n^{d_n} + b p_1^{l_1 - l_2} p_2^{l_2 - l_3} \cdots p_n^{l_n} + \cdots + c p_1^{t_1} p_2^{t_2} \cdots p_n^{t_n},
\end{align*}
即 \( f \in R[p_1, p_2, \cdots, p_n] \),所以 \( \Sigma = R[p_1, p_2, \cdots, p_n] \).

\item 由(1)可知\( p_1^{d_1} p_2^{d_2} \cdots p_n^{d_n} \) 的最高项为
\begin{align*}
x_1^{d_1 + d_2 + \cdots + d_n} (x_2)^{d_2 + \cdots + d_n} \cdots (x_n)^{d_n},
\end{align*}
因而由
\begin{align*}
d_i = c_i, \, 1 \leqslant i \leqslant n \iff \sum_{j=i}^n d_j = \sum_{j=i}^n c_j, \, 1 \leqslant i \leqslant n
\end{align*}
知 \( p_1^{d_1} p_2^{d_2} \cdots p_n^{d_n} = p_1^{c_1} p_2^{c_2} \cdots p_n^{c_n} \) 当且仅当它们的最高项相同. 假设有有限个\( a_{d_1 d_2 \cdots d_n} \neq 0 \) 而使
\begin{align}
\sum_{d_1 d_2 \cdots d_n} a_{d_1 d_2 \cdots d_n} p_1^{d_1} p_2^{d_2} \cdots p_n^{d_n} = 0. \label{eq::::90-j3t43ywat 53hu4e2.3.1}
\end{align}
因为上式中每一项都不相同,所以由之前的证明知,上式中每一项$p_{1}^{d_1}p_{2}^{d_2}\cdots p_{n}^{d_n}$的最高项都互不相同.
取所有系数不为0的$p_{1}^{d_1}p_{2}^{d_2}\cdots p_{n}^{d_n}$的最高项的所有$x_i$的幂之和最大数为
\begin{align*}
m=\max \left\{ \sum_{j=1}^n{jd_j}=\sum_{i=1}^n{\sum_{j=i}^n{d_j}}=\left( d_1+d_2+\cdots +d_n \right) +\left( d_2+\cdots +d_n \right) +\cdots +d_n\biggl| a_{d_1d_2\cdots d_n}\ne 0 \right\} .
\end{align*}
再取
\[
\left\{ x_1^{l_1} x_2^{l_2} \cdots x_n^{l_n} \bigg| l_i = \sum_{j=i}^n d_j, \sum_{i=1}^n l_i =\sum_{j=1}^n{jd_j}= m ,a_{d_1\cdots d_n}\ne 0\right\}
\]
中的最高项是\( x_1^{k_1} x_2^{k_2} \cdots x_n^{k_n} \),其中\( k_i = \sum_{j=i}^n c_j \).
由此知在 \(\eqref{eq::::90-j3t43ywat 53hu4e2.3.1}\) 式的左边含有一项 \( a_{c_1 c_2 \cdots c_n} x_1^{k_1} x_2^{k_2} \cdots x_n^{k_n}\ne 0 \),故\(\eqref{eq::::90-j3t43ywat 53hu4e2.3.1}\) 式左边不为零,而右边为零,矛盾!因而知 \( p_1, p_2, \cdots, p_n \) 在 \( R \) 上是代数无关的.
\end{enumerate}

\end{proof}

\begin{example}
将对称多项式
\begin{align*}
f(x_1, x_2, \cdots, x_n) = \sum_{\substack{1 \leqslant j_1 < j_2 < j_3 \leqslant n}} \left( x_{j_1}^2 x_{j_2}^2 x_{j_3} + x_{j_1}^2 x_{j_2} x_{j_3}^2 + x_{j_1} x_{j_2}^2 x_{j_3}^2 \right)
\end{align*}
表为初等对称多项式的多项式.
\end{example}
\begin{solution}
\( f \) 是一个5次齐次对称多项式,首项是 \( x_1^2 x_2^2 x_3 \),因而满足
\begin{align*}
k_1 \geqslant k_2 \geqslant \cdots \geqslant k_n, \quad \sum_{i=1}^n{k_i}=5
\end{align*}
且 \( x_1^{k_1} x_2^{k_2} \cdots x_n^{k_n} < x_1^2 x_2^2 x_3 \) 的项只有 \( x_1^2 x_2 x_3 x_4 \),$\,$\( x_1 x_2 x_3 x_4 x_5 \). 因而由\hyperref[theorem:抽象代数--定理2.3.1--对称多项式基本定理-1]{对称多项式基本定理(1)的证明}有
\begin{align*}
f(x_1, x_2, \cdots, x_n) &= p_1^{2-2} p_2^{2-1} p_3 + A p_1^{2-1} p_2^{1-1} p_3^{1-1} p_4^1 + B p_1^{1-1} p_2^{1-1} p_3^{1-1} p_4^{1-1} p_5^1 \\
&= p_2 p_3 + A p_1 p_4 + B p_5,
\end{align*}
其中,\( A, B \) 是待定系数.
取
\begin{align*}
x_i = \begin{cases}
1, & 1 \leqslant i \leqslant 4, \\
0, & i \geqslant 5,
\end{cases}
\end{align*}
则有 \( p_1 = 4 \),\( p_2 = \mathrm{C}_4^2 = 6 \),\( p_3 = \mathrm{C}_4^3 = 4 \),\( p_4 = 1 \),而 \( f = 3 \times \mathrm{C}_4^3 = 12 \),故有 \( 12 = 24 + 4A \),即 \( A = -3 \).
又取
\begin{align*}
x_i = \begin{cases}
1, & 1 \leqslant i \leqslant 5, \\
0, & i \geqslant 6,
\end{cases}
\end{align*}
则 \( p_1 = 5 \),\( p_2 = \mathrm{C}_5^2 \),\( p_3 = \mathrm{C}_5^3 \),\( p_4 = \mathrm{C}_5^4 \),\( p_5 = 1 \). 而 \( f = 3 \times \mathrm{C}_5^3 = 30 \),故有 \( 30 = 100 - 3 \times 25 + B \),即 \( B = 5 \). 最后得 \( f(x_1, x_2, \cdots, x_n) = p_2 p_3 - 3p_1 p_4 + 5p_5 \).

\end{solution}

\begin{example}\label{抽象代数--例题2.3.3}
对 \( j_1 \geqslant j_2 \geqslant \cdots \geqslant j_n \),记
\begin{align*}
s(x_1^{j_1} x_2^{j_2} \cdots x_n^{j_n}) = \sum_{\pi \in S_n} x_{\pi(1)}^{j_1} x_{\pi(2)}^{j_2} \cdots x_{\pi(n)}^{j_n},
\end{align*}
如
\begin{align*}
s(x_1^k) &= \sum_{\pi \in S_n} x_{\pi(1)}^k = x_1^k + x_2^k + \cdots + x_n^k, \\
s(x_1^2 x_2^2) &= \sum_{\pi \in S_n} x_{\pi(1)}^2 x_{\pi(2)}^2 = \sum_{\substack{1 \leqslant j_1 < j_2 \leqslant n}} x_{j_1}^2 x_{j_2}^2,
\end{align*}
则\( s(x_1^{j_1} x_2^{j_2} \cdots x_n^{j_n}) \) 是对称多项式.
\end{example}
\begin{proof}


\end{proof}

\begin{theorem}[Newton 公式]\label{theorem:抽象代数--Newton公式}
等幂和 \( s_k = \sum_{i=1}^n x_i^k \) 与初等对称多项式 \( p_i \) 有下列关系:
\begin{enumerate}[(1)]
\item 当 \( k \leqslant n \) 时,
\begin{align}
s_k - s_{k-1} p_1 + \cdots + (-1)^{k-1} s_1 p_{k-1} + (-1)^k k p_k = 0; \label{eq:::::::::9r32r43vby4y6u56mn34top897i530-j3t43ywat 53hu4e2.3.3}
\end{align}
\item 当 \( k > n \) 时,
\begin{align}
s_k - s_{k-1} p_1 + s_{k-2} p_2 - \cdots + (-1)^n s_{k-n} p_n = 0. \label{eq:::::::::9r32r43vby4y6u56mn34top897i530-j3t43ywat 53hu4e2.3.4}
\end{align}
\end{enumerate}
\end{theorem}
\begin{proof}
用\refexa{抽象代数--例题2.3.3}的符号,显然有
\begin{align}
\begin{cases}
s_{k-1} p_1 = s_k + s(x_1^{k-1} x_2), \\
-s_{k-2} p_2 = -s(x_1^{k-1} x_2) - s(x_1^{k-2} x_2 x_3), \\
\cdots\cdots \\
(-1)^{j-1} s_{k-j} p_j = (-1)^{j-1} s(x_1^{k-j+1} x_2 \cdots x_j) + (-1)^{j-1} s(x_1^{k-j} x_2 \cdots x_j x_{j+1}), \\
\cdots\cdots
\end{cases} \label{eq:::::::::9r32r43vby4y6u56mn34top897i530-j3t43ywat 53hu4e2.3.5}
\end{align}
且当 \( k \leqslant n \) 时,
\begin{align}
(-1)^{k-2} s_1 p_{k-1} = (-1)^{k-2} s(x_1^2 x_2 \cdots x_{k-1}) + (-1)^{k-2} k p_k. \label{eq:::::::::9r32r43vby4y6u56mn34top897i530-j3t43ywat 53hu4e2.3.6}
\end{align}
当 \( k > n \) 时,
\begin{align}
(-1)^{n-1} s_{k-n} p_n = (-1)^{n-1} s(x_1^{k-n+1} x_2 \cdots x_n). \label{eq:::::::::9r32r43vby4y6u56mn34top897i530-j3t43ywat 53hu4e2.3.7}
\end{align}
当 \( k \leqslant n \) 时, 将联立式 \(\eqref{eq:::::::::9r32r43vby4y6u56mn34top897i530-j3t43ywat 53hu4e2.3.5}\) 中各式及式 \(\eqref{eq:::::::::9r32r43vby4y6u56mn34top897i530-j3t43ywat 53hu4e2.3.6}\) 相加得
\begin{align*}
s_{k-1} p_1 - s_{k-2} p_2 + \cdots + (-1)^{k-2} s_1 p_{k-1} = s_k + (-1)^k k p_k,
\end{align*}
即式 \(\eqref{eq:::::::::9r32r43vby4y6u56mn34top897i530-j3t43ywat 53hu4e2.3.3}\) 成立.

当 \( k > n \) 时, 将联立式 \(\eqref{eq:::::::::9r32r43vby4y6u56mn34top897i530-j3t43ywat 53hu4e2.3.5}\) 中各式及式 \(\eqref{eq:::::::::9r32r43vby4y6u56mn34top897i530-j3t43ywat 53hu4e2.3.7}\) 相加得
\begin{align*}
s_{k-1} p_1 - s_{k-2} p_2 + \cdots + (-1)^{n-1} s_{k-n} p_n = s_k,
\end{align*}
即式 \(\eqref{eq:::::::::9r32r43vby4y6u56mn34top897i530-j3t43ywat 53hu4e2.3.4}\) 成立.

\end{proof}
















\end{document}
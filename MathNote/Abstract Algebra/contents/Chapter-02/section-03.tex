\documentclass[../../main.tex]{subfiles}
\graphicspath{{\subfix{../../image/}}} % 指定图片目录,后续可以直接使用图片文件名。

% 例如:
% \begin{figure}[H]
% \centering
% \includegraphics[scale=0.3]{image-01.01}
% \caption{图片标题}
% \label{figure:image-01.01}
% \end{figure}
% 注意:上述\label{}一定要放在\caption{}之后,否则引用图片序号会只会显示??.

\begin{document}

\section{理想}

\begin{definition}[由子集生成的理想]
设 $(R, +, \cdot)$ 是一个环,而 $A \subset R$。则 $(A)$,称为\textbf{由 $\boldsymbol{A}$ 生成的理想},定义为所有 $R$ 中包含 $A$ 的理想的交集,即
\begin{align*}
(A) = \bigcap \{I \subset R : I \supset A, I \lhd R\}.
\end{align*} 
\end{definition}
\begin{note}
因为$R\lhd R$且$A\subset R$,所以$R\subset (A)$.故$(A)\ne \varnothing.$
\end{note}

\begin{proposition}[生成的理想还是理想]\label{proposition:由子集生成的理想还是理想}
设 $(R, +, \cdot)$ 是一个环,而 $A \subset R$,则 $(A) \lhd R$。
\end{proposition}
\begin{proof}
 首先,取交集的集族非空,因为整个环 $R$ 是包含了 $A$ 的一个理想(对加法构成子群,且 “吸收” 了乘法)。

由于集族中每一个理想都是加法子群。因此根据\refpro{proposition:子群的任意交仍是子群}可知,它们的交还是加法子群。我们只须检验乘法的 “吸收” 性,即 $R(A) \subset (A)$,及 $(A)R \subset (A)$。根据对称性,我们证明第一个包含关系。假设 $r \in R$,$a \in (A)$,则对于任意集族中的理想 $I$,我们都有 $a \in I$。故 $ra \in I$。这对于任意这样的理想 $I$ 都是成立的,因此 $ra \in (A)$。这就证明了 $(A)$ 是 $R$ 的子环。 
\end{proof}

\begin{definition}
设 $(R, +, \cdot)$ 是一个环,而 $a \in R$,则我们定义
\begin{align*}
(a) = (\{a\}).
\end{align*}
称为\textbf{由 $\boldsymbol{a}$ 生成的主理想}。一般地,若一个理想能被一个元素生成,我们就称其为\textbf{主理想}。

对于 $a_1, \cdots, a_n \in R$,我们定义
\begin{align*}
(a_1, \cdots, a_n) = (\{a_1, \cdots, a_n\}).
\end{align*}
一般地,若一个理想能被有限个元素生成,我们就称其为\textbf{有限生成的理想}。 
\end{definition}

\begin{proposition}\label{proposition:生成的理想的集合表示}
设$(R, +, \cdot)$ 是一个交换环,而 $a \in R$,则
\begin{align*}
(a) = Ra = aR= \{ra : r \in R\}= \{ar : r \in R\}.
\end{align*}
一般地,若 $a_1, \cdots, a_n \in R$,则
\begin{align*}
(a_1, \cdots, a_n) = Ra_1 + \cdots + Ra_n = \{r_1a_1 + \cdots + r_na_n : r_1, \cdots, r_n \in R\} .
\end{align*}
\end{proposition}
\begin{remark}
若$(R, +, \cdot)$ 是环,但不是交换环,则上述结论仍成立.但是我们还可以同理得到,当$m=1,2,\cdots,n$时,都有
\begin{align*}
\left( a_1,\cdots ,a_n \right) =Ra_1+\cdots +Ra_m+a_{m+1}R+\cdots +a_nR.
\end{align*}
故此时与$(a_1,\cdots,a_n)$相等的集合就有$2^m$种不同的形式.

如果$(R, +, \cdot)$ 是一个交换环,那么当$m=1,2,\cdots,n$时,都有
\begin{align*}
\left( a_1,\cdots ,a_n \right) =Ra_1+\cdots +Ra_m+a_{m+1}R+\cdots +a_nR=Ra_1 + \cdots + Ra_n.
\end{align*}
这样在交换环下$(a_1,\cdots,a_n)$的形式就能够统一起来.
\end{remark}
\begin{proof}
显然有限生成的理想是主理想的特例,故我们只须证明第二个等式。

要证明 $(A) = I$,我们只须证明两点。一,$I$ 是包含 $A$ 的理想(即$(A)\subset I$);二,每一个包含 $A$ 的理想都会包含 $I$(即$\forall H\in (A)$,都有$I\subset H$.也即$I\subset (A)$)。

首先,要证明 $Ra_1 + \cdots + Ra_n$ 是个理想。对加法而言,$0 = 0a_1 + \cdots + 0a_n \in Ra_1 + \cdots + Ra_n$,而且对 $r_1a_1 + \cdots + r_na_n, s_1a_1 + \cdots + s_na_n (r_i, s_i \in R)$,我们有
\begin{align*}
(r_1a_1 + \cdots + r_na_n) - (s_1a_1 + \cdots + s_na_n) = (r_1 - s_1)a_1 + \cdots + (r_n - s_n)a_n \in Ra_1 + \cdots + Ra_n 。
\end{align*}
因此 $Ra_1 + \cdots + Ra_n$ 对加法构成子群。

接下来,根据对称性,我们只须证明 $R(Ra_1 + \cdots + Ra_n) \subset (Ra_1 + \cdots + Ra_n)$。而这是因为
\begin{align*}
R(Ra_1 + \cdots + Ra_n) = RRa_1 + \cdots + RRa_n = Ra_1 + \cdots + Ra_n 。
\end{align*}
这样,我们就证明了 $Ra_1 + \cdots + Ra_n$ 是个理想,而且显然包含 $\{a_1, \cdots, a_n\}$。

另一方面,设 $I$ 是一个包含了 $a_1, \cdots, a_n$ 的理想,那么根据加法的封闭性及乘法的 “吸收” 性,
\begin{align*}
I \supset Ra_1 + \cdots + Ra_n 。
\end{align*}

综上所述,这就证明了这个命题。 
\end{proof}

\begin{definition}[理想的加法]
设 $(R, +, \cdot)$ 是一个环,而 $I, J \lhd R$,则
\begin{align*}
I + J = \{a + b : a \in I, b \in J\}.
\end{align*}
\end{definition}

\begin{proposition}[理想的加法还是理想]\label{proposition:理想的加法还是理想}
设 $(R, +, \cdot)$ 是一个环,而 $I, J \lhd R$,则 $I + J$ 还是个理想,即
\begin{align*}
I + J \lhd R.
\end{align*}
\end{proposition}
\begin{proof}
由\reflem{lemma:HN<G的条件}可知$(I+J,+)<(R,+)$。因此我们只须证明乘法的 “吸收” 性。
\begin{align*}
R(I + J) &= RI + RJ \subseteq I + J,\\
(I + J)R &= IR + JR \subseteq I + J.
\end{align*}
这就证明了
\begin{align*}
I + J \lhd R.
\end{align*} 
\end{proof}

\begin{proposition}
设 $(R, +, \cdot)$ 是一个环,而 $I, J \lhd R$,则 $I + J$ 是由 $I \cup J$ 生成的理想,即
\begin{align*}
I + J = (I \cup J) .
\end{align*}
\end{proposition}
\begin{proof}
首先,由\refpro{proposition:理想的加法还是理想}可知 $I + J$ 是一个理想。而 $I + J \supset I + \{0\} = I$,同理 $I + J \supset J$,故 $I + J \supset I \cup J$。这就证明了 $I + J$ 是一个包含了 $I \cup J$ 的理想。

接着,如果 $K$ 是包含了 $I \cup J$ 的理想,则$K\supset I,K\supset J$,那么根据加法封闭性,我们当然有
\begin{align*}
K \supset I + J.
\end{align*}
综上所述,我们就证明了
\begin{align*}
I + J = (I \cup J).
\end{align*} 
\end{proof}

\begin{definition}[理想的乘法]
设$(R, +, \cdot)$ 是一个交换环,而 $I, J \lhd R$,则
\begin{align*}
IJ = (\{ab : a \in I, b \in J\})=(I\cdot J).
\end{align*}
上面的圆括号表示生成的理想。
\end{definition}
\begin{remark}
由\refpro{proposition:由子集生成的理想还是理想}可知,上述定义的$IJ$仍是$R$的一个理想.
\end{remark}

\begin{proposition}\label{proposition:理想的乘法表示形式}
设 $(R, +, \cdot)$ 是一个交换环,而 $I, J \lhd R$,则
\begin{align*}
IJ = \{a_1b_1 + \cdots + a_nb_n : a_1, \cdots, a_n \in I, b_1, \cdots, b_n \in J\} 。
\end{align*}
\end{proposition}
\begin{remark}
若$(R, +, \cdot)$ 是环,但不是交换环,则上述结论仍成立.但是我们还可以同理得到,当$m=1,2,\cdots,n$时,都有
\begin{align*}
IJ=\{\left( a_1b_1+\cdots +a_mb_m \right) +\left( b_{m+1}a_{m+1}+\cdots +b_na_n \right) :a_1,\cdots ,a_n\in I,b_1,\cdots ,b_n\in J\}.
\end{align*}
故此时与$IJ$相等的集合就有$2^m$种不同的形式.

如果$(R, +, \cdot)$ 是一个交换环,那么当$m=1,2,\cdots,n$时,都有
\begin{align*}
IJ&=\{a_1b_1+\cdots +a_nb_n:a_1,\cdots ,a_n\in I,b_1,\cdots ,b_n\in J\}
\\
&=\{\left( a_1b_1+\cdots +a_mb_m \right) +\left( b_{m+1}a_{m+1}+\cdots +b_na_n \right) :a_1,\cdots ,a_n\in I,b_1,\cdots ,b_n\in J\}.
\end{align*}
这样在交换环下$IJ$的形式就能够统一起来.
\end{remark}
\begin{proof}
首先,如果 $K$ 是交换环 $R$ 中包含了 $\{ab : a \in I, b \in J\}$ 的理想,则根据加法的封闭性,
\begin{align*}
K \supset \{a_1b_1 + \cdots + a_nb_n : a_1, \cdots, a_n \in I, b_1, \cdots, b_n \in J\} .
\end{align*}
故$\{a_1b_1 + \cdots + a_nb_n : a_1, \cdots, a_n \in I, b_1, \cdots, b_n \in J\} \subset IJ$.

接着,我们要证明 $A=\{a_1b_1 + \cdots + a_nb_n : a_1, \cdots, a_n \in I, b_1, \cdots, b_n \in J\}$ 确实是包含了 $\{ab : a \in I, b \in J\}$ 的一个$R$上的理想。包含关系是显然的,这就是有限和中只有一项的特例。

我们先证明加法是子群。$0 = 00 + \cdots + 00\in A$,而且对于 $a_1b_1 + \cdots + a_nb_n, c_1d_1 + \cdots + c_md_m \in A$,其中$a_i,c_i\in I,b_i,d_i\in J$.由$I,J\lhd R$可知$-c_i\in I,a_ib_i\in I,(-c_i)d_i\in J$.于是我们有
\begin{align*}
&(a_1b_1+\cdots +a_nb_n)-(c_1d_1+\cdots +c_md_m)=a_1b_1+\cdots +a_nb_n+(-c_1)d_1+\cdots +(-c_m)d_m
\\
&=\left( a_1b_1+\cdots +a_nb_n \right) \cdot 1+1\cdot \left( (-c_1)d_1+\cdots +(-c_m)d_m \right) +0+\cdots +0\in A.
\end{align*}
故$(A,+)$是$(R,+)$的子群.
我们再证明乘法的 “吸收性”。根据对称性,我们只证
“左吸收性”.令 $a_1b_1 + \cdots + a_nb_n \in A$,而 $\forall r \in R$,都有$ ra_i \in I$,不妨令 $a_i' = ra_i \in I$,则
\begin{align*}
r(a_1b_1 + \cdots + a_nb_n) = ra_1b_1 + \cdots + ra_nb_n = a_1'b_1 + \cdots + a_n'b_n \in A.
\end{align*}
综上所述,由交换环中的两个理想 $I$, $J$ 的乘积所生成的理想,就是它们元素乘积的有限和所构成的集合。
\end{proof}

\begin{proposition}[理想关于加法和乘法的运算律]
设$(R, +, \cdot)$ 是一个交换环,而 $I, J, K \lhd R$,则满足
\begin{enumerate}[(1)]
\item $I + J = J + I ;$
\item $I + (J + K) = (I + J) + K ;$
\item $I(J + K) = IJ + IK;$
\item $I(JK) = (IJ)K ;$
\item $I = RI = IR .$
\end{enumerate}
\end{proposition}
\begin{proof}
\begin{enumerate}[(1)]
\item 由$(R,+)$是一个Abel群可直接得到$I + J = J + I .$

\item 由$(R,+)$是一个Abel群也可直接得到$I + (J + K) = (I + J) + K .$

\item 一方面,$I(J + K) \supset I(J + \{0\}) = IJ$,同理 $I(J + K) \supset IK$.又$I(J + K)$是$R$上的理想,故根据$I(J+K)$对加法的封闭性可得 $I(J + K) \supset IJ + IK$。

另一方面,令 $\sum_i (a_i(b_i + c_i)) \in I(J + K)$,则
\begin{align*}
\sum_i (a_i(b_i + c_i)) = \sum_i (a_ib_i) + \sum_i (a_ic_i) \in IJ + IK .
\end{align*}
因此 $I(J + K) \subset IJ + IK$。

\item 根据对称性,我们只证明 $I(JK) \subset (IJ)K$。因为理想的乘积是由元素乘积的集合所生成的,故只须证明 $\{ad : a \in I, d \in JK\} \subset (IJ)K$。
令 $a \in I$,$d = \sum_i (b_ic_i) \in JK$。则
\begin{align*}
ad = a\sum_i (b_ic_i) = \sum_i ((ab_i)c_i) .
\end{align*}
其中 $ab_i \in IJ$,故 $ad \in (IJ)K$。因此 $I(JK) \subset (IJ)K$。

\item 我们只证明 $I = RI$。一方面,根据理想的定义,$I \supset RI$。另一方面,$I = 1I \subset RI$,因为 $1 \in R$。
\end{enumerate}
\end{proof}

\begin{lemma}\label{lemma:IJ、I与J的交、I+J的关系}
设$(R, +, \cdot)$ 是一个交换环,而 $I, J \lhd R$,则
\begin{align*}
IJ \subset I \cap J \subset I + J
\end{align*}
\end{lemma}
\begin{proof}
证明是简单的。因为 $R$ 是一个交换环,而 $I$ 是一个理想,故
\begin{align*}
IJ \subset IR = I.
\end{align*}
对 $J$ 是类似的,故
\begin{align*}
IJ \subset I \cap J.
\end{align*}
另外,$I \cap J \subset I$,而且 $I \cap J \subset J$,故
\begin{align*}
I \cap J \subset (I \cup J) = I + J.
\end{align*}
这就证明了这个引理。
\end{proof}

\begin{lemma}\label{lemma:I与J的交、I+J的乘积包含于IJ}
设$(R, +, \cdot)$ 是一个交换环,而 $I, J \lhd R$,则
\begin{align*}
(I \cap J)(I + J) \subset IJ .
\end{align*}
\end{lemma}
\begin{proof}
证明是不难的。由\refpro{proposition:理想的乘法表示形式}可知$(I \cap J)(I + J)=\{\sum_i (a_i(b_i + c_i)):a_i \in I \cap J, b_i \in I, c_i \in J\}.$于是任取$\sum_i (a_i(b_i + c_i)) \in (I \cap J)(I + J)$,则$a_i(b_i + c_i) \in (I \cap J)\cdot (I + J)$,其中 $a_i \in I \cap J, b_i \in I, c_i \in J$,从而
\begin{align*}
\sum_i (a_i(b_i + c_i)) = \sum_i (a_ib_i) + \sum_i (a_ic_i) \subset JI + IJ = IJ + IJ = IJ .
\end{align*}
第一个等号是因为$R$中的乘法对加法满足分配律,倒数第二个等号是根据交换环对乘法的交换律,最后一步是根据理想的乘积对加法的封闭性。
这就证明了这个命题.
\end{proof}

\begin{proposition}
设 $(R, +, \cdot)$ 是一个交换环,而 $I, J, K \lhd R$,则
\begin{align*}
I \cap (J + K) \supset I \cap J + I \cap K
\end{align*}
特别地,如果 $J \subset K$,则
\begin{align*}
I \cap (J + K) = I \cap J + I \cap K
\end{align*}
\end{proposition}
\begin{proof}
因为 $I \cap (J + K) \supset I \cap J$,且 $I \cap (J + K) \supset I \cap K$,又$I \cap (J + K)$构成$R$的加法子群,从而对加法封闭.所以
\begin{align*}
I \cap (J + K) \supset I \cap J + I \cap K.
\end{align*}
这就证明了第一点。

接下来,我们假设 $J \subset K$。我们只须证明
\begin{align*}
I \cap (J + K) \subset I \cap J + I \cap K.
\end{align*}
而这是因为
\begin{align*}
I \cap (J + K) \subset I \cap (K + K) = I \cap K \subset I \cap J + I \cap K.
\end{align*}
这就证明了这个命题。 
\end{proof}

\begin{definition}[理想的互素]
设 $(R, +, \cdot)$ 是一个交换环,而 $I, J \lhd R$。我们称 $I$,$J$ \textbf{互素},若其和为整个环,即
\begin{align*}
I + J = R.
\end{align*}
\end{definition}

\begin{proposition}[两个理想互素的充要条件]\label{proposition:两个理想互素的充要条件}
设 $(R, +, \cdot)$ 是一个交换环,而 $I, J \lhd R$。则 $I$,$J$ 互素,当且仅当
\begin{align*}
\exists a \in I, \exists b \in J, a + b = 1.
\end{align*}
\end{proposition}
\begin{proof}
一方面,若 $I + J = R$,则根据\reflem{lemma:理想是整个环的充要条件}可知 $1 \in R = I + J$,故存在 $a \in I$,$b \in J$,使得 $a + b = 1$。

另一方面,假设 $a + b = 1 (a \in I, b \in J)$,则对任何 $r \in R$,
\begin{align*}
r = r1 = r(a + b) = ra + rb \in RI + RJ = I + J
\end{align*}
这就证明了 $I + J \subset R$。而由$R$对加法封闭,显然有$R\subset I+J$。故$R=I+J.$
综上所述,两个理想互素当且仅当 $1$ 可以写成这两个理想中元素的和。
\end{proof}

\begin{proposition}\label{proposition:理想的乘积等于理想的交当且仅当它们互素}
设$(R, +, \cdot)$ 是一个交换环,而 $I, J \lhd R$ 互素,则
\begin{align*}
IJ = I \cap J.
\end{align*}
\end{proposition}
\begin{proof}
由\reflem{lemma:IJ、I与J的交、I+J的关系}可知
\begin{align*}
IJ \subset I \cap J.
\end{align*}
故只须证明
\begin{align*}
I \cap J \subset IJ.
\end{align*}
由$I,J$互素可知$I+J=R$.又由\refpro{proposition:理想的任意交还是理想}可知$I\cap J$仍是$R$的理想,从而$I \cap J = (I \cap J)R .$于是由\reflem{lemma:I与J的交、I+J的乘积包含于IJ}可得
\begin{align*}
I \cap J = (I \cap J)R = (I \cap J)(I + J) \subset IJ.
\end{align*}
这就证明了这个命题。
\end{proof}

\begin{proposition}\label{proposition:交换环中,理想在环同态下的原像还是理想}
设 $(R, +, \cdot)$ 和 $(R', +, \cdot)$ 是两个交换环, $f : (R, +, \cdot) \to (R', +, \cdot)$ 是一个环同态,而 $I' \lhd R'$,则 $f^{-1}(I') \lhd R$。
\end{proposition}
\begin{proof}
就加法子群而言,由\refpro{proposition:环同态的等价条件}可知 $0 = f^{-1}(0) \in R$,并且若 $a = f^{-1}(a'), b = f^{-1}(b') \in f^{-1}(I')$,则
\begin{align*}
&f\left( a-b \right) =f\left( a \right) +f\left( -b \right) =f\left( a \right) -f\left( b \right) =a' -b'
\\
&\Rightarrow a-b=f^{-1}(a'-b')\in f^{-1}(I' ).
\end{align*}
就乘法的 “吸收” 性来说。根据对称性,我们只须证明 $Rf^{-1}(I') \subset f^{-1}(I')$,对 $\forall r \in R, x \in f^{-1}(I')$,有 $f(x) \in I'$。由 $f$ 是环同态可知,$f(rx) = f(r)f(x)$。又由 $I'$ 是 $R'$ 的理想且 $f(r) \in R'$,因此 $f(rx) = f(r)f(x) \in I'$。于是 $rx \in f^{-1}(I')$。 
这样,我们就证明了这个命题,即交换环中,理想在环同态下的原像还是理想。 
\end{proof}





\end{document}
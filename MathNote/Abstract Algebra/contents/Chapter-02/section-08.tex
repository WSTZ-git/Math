\documentclass[../../main.tex]{subfiles}
\graphicspath{{\subfix{../../image/}}} % 指定图片目录,后续可以直接使用图片文件名。

% 例如:
% \begin{figure}[H]
% \centering
% \includegraphics[scale=0.4]{图.png}
% \caption{}
% \label{figure:图}
% \end{figure}
% 注意:上述\label{}一定要放在\caption{}之后,否则引用图片序号会只会显示??.

\begin{document}

\section{素理想与极大理想}

\begin{definition}[素理想]
若交换幺环$R$的理想$P$满足
\begin{enumerate}[(1)]
\item $P \neq R$;
\item 若$ab \in P$,则$a \in P$或$b \in P$,
\end{enumerate}
则称$P$为$R$的\textbf{素理想}.
\end{definition}

\begin{definition}[极大理想]
设环$R$中的理想$M \neq R$且不存在$R$的理想$A$,使$M \subset A \subset R$,则称$M$为$R$的\textbf{极大理想}.
\end{definition}
\begin{remark}
由定义可知若$A$为环$R$中的极大理想且$M\supset A$,则$M=R.$
\end{remark}

\begin{proposition}
\begin{enumerate}[(1)]
\item 设$p$为素数,则$\langle p \rangle=p\mathbf{Z}$为$\mathbf{Z}$的素理想,同时也是$\mathbf{Z}$的极大理想.

\item $\mathbf{Z}$中非平凡理想$I = m\mathbf{Z}$为素理想或极大理想当且仅当$m$为素数(或负素数).
\end{enumerate}
\end{proposition}
\begin{proof}
\begin{enumerate}[(1)]
\item 设$ab \in \langle p \rangle=p\mathbf{Z}$,即$p \mid ab$.由$p$为素数知$p \mid a$或$p \mid b$,亦即$a \in p\mathbf{Z}=\langle p \rangle$或$b \in p\mathbf{Z}= \langle p \rangle$,因而$\langle p \rangle$为素理想.

其次,设$\mathbf{Z}$的理想$A$满足$\langle p \rangle \subseteq A \subseteq \mathbf{Z}$.由\refpro{proposition:Z是主理想整环}知$\mathbf{Z}$为p.i.d.,故有$A = \langle n \rangle$.由$p \in \langle n \rangle$,因而$n \mid p$.因$p$为素数,故$n = \pm 1$或$n = \pm p$.若$n = \pm 1$,则$A = \mathbf{Z}$;若$n = \pm p$,则$A = \langle p \rangle$.由此知$\langle p \rangle$是$\mathbf{Z}$的极大理想.

\item 若$m = m_1m_2(m_i \neq \pm 1,i = 1,2)$,则$m_1m_2 = m \in I$.但$m_i \notin I(i = 1,2)$,故$I$不是素理想.又$m\mathbf{Z} \subset m_1\mathbf{Z} \subset \mathbf{Z}$,故$I$不是极大理想.\
\end{enumerate}

\end{proof}

\begin{lemma}\label{lemma:抽象代数--引理2.8.1}
设$R$为交换幺环,则有
\begin{enumerate}[(1)]
\item\label{lemma:抽象代数--引理2.8.1-1} $\{0\}$是$R$的素理想当且仅当$R$为整环;

\item\label{lemma:抽象代数--引理2.8.1-2} $\{0\}$是$R$的极大理想当且仅当$R$为域.
\end{enumerate}
\end{lemma}
\begin{proof}
\begin{enumerate}[(1)]
\item 设$R$为整环.若$a \neq 0$,$b \neq 0$,即$a \notin \{0\}$,$b \notin \{0\}$,则$ab \neq 0$,即$ab \notin \{0\}$,因而由素理想的逆否定义知$\{0\}$为素理想.

反之,设$\{0\}$为素理想.又$a,b \notin \{0\}$,故$ab \notin \{0\}$,即$a \neq 0$,$b \neq 0$得出$ab \neq 0$.又$R$是交换幺环,故$R$为整环.

\item 设$\{0\}$为极大理想.$\forall a \in R$且$a \neq 0$有$\langle a \rangle \supset \{0\}$,故$\langle a \rangle = R$.由$R$含有幺元$1$,故$1 \in \langle a \rangle = aR$.因而$\exists a^{-1} \in R$,使得$aa^{-1} = 1$.再由$R$可交换知$R$是一个域.

反之,设$R$是一个域,$A$为$R$的理想且$A \neq \{0\}$,即$\exists a \in A$,$a \neq 0$.又$R$为域,故$\exists a^{-1}$,使$1 = a^{-1}a$,再由$A$为$R$的理想知$=a^{-1}a \in A$.进而对$\forall b \in R$有$b = b \cdot 1 \in A$,因而$A = R$,故$\{0\}$为极大理想.
\end{enumerate}

\end{proof}

\begin{theorem}\label{theorem:抽象代数--定理2.8.1}
设$R$为交换幺环,$P$与$M$为$R$的理想,则
\begin{enumerate}[(1)]
\item 当且仅当$R/P$为整环时,$P$为素理想;

\item 当且仅当$R/M$为域时,$M$为极大理想.
\end{enumerate}
\end{theorem}
\begin{proof}
\begin{enumerate}[(1)]
\item 设$\pi$为$R$到$R/P$上的自然同态.若$P$为素理想,设$\pi(a) \neq 0$,$\pi(b) \neq 0$,亦即$a,b \notin P$,则$ab \notin P$,即$\pi(ab) = \pi(a)\pi(b) \neq 0$,因而$R/P$为整环.

反之,若$R/P$为整环且$ab \in P$,则有$\pi(ab) = \pi(a)\pi(b) = 0$,因而$\pi(a) = 0$或$\pi(b) = 0$,即$a \in P$或$b \in P$,所以$P$是素理想.

\item 设$R$的理想$A$满足$M \subseteq A \subseteq R$,于是由推论1.7.2知$A/M$是$R/M$的理想.当$M$为极大理想时有$M = A$或$R = A$.故$R/M$仅有的理想为$\{0\}$与$R/M$,即$\{0\}$为极大理想,故$R/M$为域(见引理2.8.1).

反之,$R/M$为域,则由\reflem{lemma:抽象代数--引理2.8.1}知$\{0\}$为极大理想,因而若$A \neq M$,则$A/M = R/M$.故$A = R$,即$M$为极大理想.
\end{enumerate}

\end{proof}

\begin{corollary}
交换幺环$R$的极大理想$M$必为素理想.
\end{corollary}
\begin{proof}
事实上,因$R/M$为域,故为整环,所以$M$为素理想.

\end{proof}

\begin{theorem}
设$R,R'$都是交换幺环,$\sigma$是$R$到$R'$上的同态,$N = \ker \sigma$.若$H$是$R$中包含$N$的素理想(或极大理想),则$\sigma(H)$是$R'$中的素理想(或极大理想).反之,若$H'$是$R'$的素理想(或极大理想),则$\sigma^{-1}(H') = \{x \in R|\sigma(x) \in H'\}$为$R$中包含$N$的素理想(或极大理想).
\end{theorem}
\begin{proof}
根据定理1.7.5有$R/H \cong R'/ \sigma(H)$,故由定理2.8.1知$H(H \supseteq N)$为素理想(或极大理想)当且仅当$\sigma(H)$为素理想(或极大理想).

\end{proof}

\begin{theorem}
若$R$是交换整环,$a \in R^*$,则由$a$生成的主理想$\langle a \rangle$为素理想的充分必要条件是$a$为素元素.
\end{theorem}
\begin{proof}
显然,当且仅当$a$为$R$的单位,即$a \in U$时,$\langle a \rangle = R$,故可设$a \in R^* \setminus U$.由$bc \in \langle a \rangle$ iff $a \mid bc$,因而$\langle a \rangle$为素理想当且仅当$a$为素元素.

\end{proof}

\begin{corollary}
设$R$为唯一析因环,$a \in R$,$a \neq 0$,则$\langle a \rangle$为素理想当且仅当$a$为不可约元素.
\end{corollary}

从2.4节知素元素一定为不可约元素.反之,在$R$为唯一析因环时,不可约元素也是素元素.故推论成立.

从这里可认为素理想的概念在一定意义下是素元素概念的推广.

\begin{example}
设$F$是一个域,$R = F[x_1,x_2,\cdots,x_n]$是$F$上$n$元多项式环.由定理2.8.3及推论2.8.2知对$f \in R$,$\langle f \rangle$为素理想当且仅当$f$为不可约多项式.因$R/\langle x_1,x_2 \rangle \cong F[x_3,\cdots,x_n]$,故由$x_1,x_2$生成的理想$\langle x_1,x_2 \rangle$也是素理想,而$\langle x_1,x_2 \rangle$不是主理想.当$n > 2$时,看到$\langle x_1,x_2 \rangle$也不是极大理想.
\end{example}

\begin{theorem}
设$R$为主理想整环,$a \in R^*$,则$\langle a \rangle$为极大理想的充分必要条件是$a$为素元素.
\end{theorem}
\begin{proof}
若$\langle a \rangle$为极大理想,由定理2.8.1的推论知$\langle a \rangle$为素理想,再由定理2.8.3知$a$为素元素.

反之,设$a$为素元素.若有$R$的理想$A$,使得$\langle a \rangle \subseteq A \subseteq R$.由于$R$为p.i.d.,故有$n \in R$,使得$A = \langle n \rangle$.于是$n \mid a$.由$a$为素元素即不可约元素知$n \sim 1$或$n \sim a$,亦即有$A = \langle n \rangle = R$或$A = \langle n \rangle = \langle a \rangle$,故$\langle a \rangle$为极大理想.

\end{proof}

\begin{theorem}
设$F$是一个域,$S$为交换整环且$F \subseteq S$,$F,S$有相同的幺元.
\begin{enumerate}[(1)]
\item 若$u \in S$是$F$上的代数元,则$F[u]$是一个域且存在不可约多项式$p(x) \in F[x]$,使得$F[u] \cong F[x]/\langle p(x) \rangle$,$\langle p(x) \rangle \cap F = \{0\}$;
\item 若$u \in S$是域$F$上的超越元,则$F[u] \cong F[x]$.
\end{enumerate}
\end{theorem}
\begin{proof}
从定理2.2.2及其推论知$u$是超越元时上述结论成立,故只需讨论$u$为代数元时的情形.此时$I = \{f(x)|f(x) \in F[x]|f(u) = 0\} \neq \{0\}$.因$F[x]$为Euclid环,故$I$为主理想环,于是$I = \langle p(x) \rangle$,$p(x) \neq 0$.由$S$为整环知$F[u]$也是整环,$F[u] \cong F[x]/\langle p(x) \rangle$.故$\langle p(x) \rangle$为素理想(见定理2.8.1).再由定理2.8.3,定理2.8.4及定理2.8.1知$p(x)$为不可约多项式,$\langle p(x) \rangle$为极大理想,$F[u] \cong F[x]/\langle p(x) \rangle$为域.

\end{proof}

反之,若$p(x)$是不可约多项式,则$F[x]/\langle p(x) \rangle$是$F$的扩域.








\end{document}
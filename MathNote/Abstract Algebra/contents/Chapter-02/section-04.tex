\documentclass[../../main.tex]{subfiles}
\graphicspath{{\subfix{../../image/}}} % 指定图片目录,后续可以直接使用图片文件名。

% 例如:
% \begin{figure}[H]
% \centering
% \includegraphics[scale=0.4]{图.png}
% \caption{}
% \label{figure:图}
% \end{figure}
% 注意:上述\label{}一定要放在\caption{}之后,否则引用图片序号会只会显示??.

\begin{document}

\section{素理想与极大理想}

\begin{definition}[整环]
设 $(R, +, \cdot)$ 是一个环,则我们称 $R$ 是个\textbf{整环},若它是个非零交换环,且没有零因子,即
\begin{gather*}
R \neq \{0\},
\\
R \text{是个交换环},
\\
\forall a, b \in R, (ab = 0 \implies a = 0 \text{ 或 } b = 0).
\end{gather*}
若 $a \neq 0$且$a\in R$ 满足 $\exists b \neq 0$且$b\in R$ 使得 $ab = 0$,我们就称$a$为$R$的一个\textbf{零因子}。
\end{definition}
\begin{note}
整环第三条性质的逆否命题就是:$\forall a, b \in R, (a \ne 0 \text{ 且 } b \ne 0 \implies ab \ne 0).$
\end{note}

\begin{lemma}\label{lemma:素数的基本性质}
若 $p$ 是一个素数,$a, b \in \mathbb{Z}$,则
\begin{align*}
p \mid ab \iff p \mid a \text{ 或 } p \mid b
\end{align*}
\end{lemma}
\begin{proof}
见初等数论.
\end{proof}

\begin{definition}[合数]
除了1和其本身外还有其他正因数的大于1的正整数就称为\textbf{合数}.此即大于1的不是素数的正整数.
\end{definition}

\begin{lemma}\label{lemma:合数的基本性质}
若 $n$ 是一个合数,则存在 $a, b \in \mathbb{Z}$,使得
\begin{align*}
n \mid ab ,
n \nmid a ,
n \nmid b .
\end{align*}
\end{lemma}
\begin{proof}
证明是简单的。若 $n$ 是一个合数,我们可以取一个非平凡分解 $n = ab$,其中 $a, b \neq \pm 1$。则 $n \mid ab$,可是 $|n| > |a|$,故 $n \nmid a$(因为若一个数整除另一个数,则这个数的绝对值必须小于等于另一个数)。同理 $n \nmid b$。这样,我们就证明了这个引理。 
\end{proof}

\begin{definition}[素理想]
设 $(R, +, \cdot)$ 是一个交换环,而 $\mathfrak{p} \lhd R$,则我们称 $\mathfrak{p}$ 是个\textbf{素理想},若
\begin{gather*}
\forall a, b \in \mathbb{Z}, (ab \in \mathfrak{p} \iff a \in \mathfrak{p} \text{ 或 } b \in \mathfrak{p}),
\\
\mathfrak{p} \neq R.
\end{gather*}
\end{definition}

\begin{proposition}\label{proposition:整数环中的素理想}
证明:$p\mathbb{Z}$是整数环$(\mathbb{Z},+,\cdot)$的素理想,而$m\mathbb{Z}$和$\mathbb{Z}$不是整数环$(\mathbb{Z},+,\cdot)$的素理想,其中$p$是素数,$m$是合数.
\end{proposition}
\begin{proof}
首先由\refpro{proposition:nZ是Z的理想}可知$p\mathbb{Z}$,$m\mathbb{Z}$,$\mathbb{Z}$都是$\mathbb{Z}$的理想.

由\reflem{lemma:素数的基本性质}可知,对$\forall a,b\in \mathbb{Z}$
\begin{align*}
ab\in p\mathbb{Z} \Leftrightarrow p\mid ab\Leftrightarrow p\mid a\text{或}p\mid b\Leftrightarrow a\in p\mathbb{Z} \text{或}b\in p\mathbb{Z} .
\end{align*}
故$p\mathbb{Z}$是整数环$(\mathbb{Z},+,\cdot)$的素理想.

而由\reflem{lemma:合数的基本性质}可知
\begin{align*}
ab\in m\mathbb{Z} \Leftrightarrow m\mid ab\Rightarrow m\nmid a\text{且}m\nmid b\Leftrightarrow a\notin m\mathbb{Z} \text{或}b\notin m\mathbb{Z} .
\end{align*}
故$m\mathbb{Z}$不是整数环$(\mathbb{Z},+,\cdot)$的素理想.

又因为素理想一定不是整个环,所以$\mathbb{Z}$也不是整数环$(\mathbb{Z},+,\cdot)$的素理想.
\end{proof}

\begin{proposition}
若$m,n\in \mathbb{N}_1$,由\refpro{proposition:nZ是Z的理想}可知$m\mathbb{Z},n\mathbb{Z}$一定是整数环的理想.则
\begin{align*}
m\mathbb{Z}\text{和}n\mathbb{Z}\text{互素} \iff m,n\text{互素}.
\end{align*}
\end{proposition}
\begin{proof}
由理想互素的定义和\refpro{lemma:理想是整个环的充要条件}可知
\begin{align*}
m\mathbb{Z} \text{和}n\mathbb{Z} \text{互素}
&\Longleftrightarrow m\mathbb{Z} +n\mathbb{Z} =\mathbb{Z} \\
&\Longleftrightarrow 1\in m\mathbb{Z} +n\mathbb{Z} \\
&\Longleftrightarrow \exists k,l\in \mathbb{Z},\,\mathrm{s}.\mathrm{t}.\,1=mk+nl
\end{align*}
又由\hyperref[lemma:Bézout定理]{Bézout定理}可知
\begin{align*}
\exists k,l\in \mathbb{Z},\,\mathrm{s}.\mathrm{t}.\,1=mk+nl
\Longleftrightarrow \mathrm{gcd}(m,n) = 1 
\Longleftrightarrow m,n\text{互素}
\end{align*}
故
\begin{align*}
m\mathbb{Z} \text{和}n\mathbb{Z} \text{互素} \Longleftrightarrow m,n\text{互素}.
\end{align*}
\end{proof}

\begin{proposition}[素理想的充要条件]\label{proposition:素理想的充要条件}
设 $(R, +, \cdot)$ 是一个交换环,而 $\mathfrak{p} \lhd R$。则 $\mathfrak{p}$ 是一个素理想,当且仅当商环 $R / \mathfrak{p}$ 是一个整环。
\end{proposition}
\begin{proof}
先证必要性。令 $\mathfrak{p}$ 是一个素理想。因为 $R$ 是交换环,则显然 $R / \mathfrak{p}$ 也是交换环。因为对 $a, b \in R$,我们有
\begin{align*}
(a + \mathfrak{p})(b + \mathfrak{p}) = ab + \mathfrak{p} = ba + \mathfrak{p} = (b + \mathfrak{p})(a + \mathfrak{p}).
\end{align*}
而且因为 $\mathfrak{p} \neq R$,所以任取$r\in R-\mathfrak{p}$,则$r+\mathfrak{p}\in R/\mathfrak{p}$.注意到$\mathfrak{p}\in R/\mathfrak{p}$,且$r\notin \mathfrak{p}$,因此$r+\mathfrak{p}\ne \mathfrak{p}$.故$R/\mathfrak{p}$中此时至少有两个互异的元素,即 $R / \mathfrak{p}$ 不是零环.

我们只须证明 $R / \mathfrak{p}$ 中没有零因子。假设
\begin{align*}
(a+\mathfrak{p} )(b+\mathfrak{p} )=0+\mathfrak{p} \Leftrightarrow ab+\mathfrak{p} =\mathfrak{p} \Leftrightarrow ab\in \mathfrak{p} .
\end{align*}
根据 $\mathfrak{p}$ 是素理想,不失一般性假设 $a \in \mathfrak{p}$。则
\begin{align*}
a + \mathfrak{p} = 0 + \mathfrak{p} .
\end{align*}
这就证明了 $R / \mathfrak{p}$ 是一个整环。

再证充分性。假设 $R / \mathfrak{p}$ 是一个整环。类似地,我们知道因为 $R / \mathfrak{p}$ 不是零环,所以 $\mathfrak{p} \neq R$。否则$R/\mathfrak{p}=R/R={0+R}$,只含一个元素,与$R/R$不是零环矛盾.

再令 $a, b \in R$,使得 $ab \in \mathfrak{p}$,则
\begin{align*}
ab + \mathfrak{p} = (a + \mathfrak{p})(b + \mathfrak{p}) = 0 + \mathfrak{p}.
\end{align*}
由于 $R / \mathfrak{p}$ 是一个整环,故不失一般性假设 $a + \mathfrak{p} = 0 + \mathfrak{p}$,这就证明了 $a \in \mathfrak{p}$,即 $\mathfrak{p}$ 是一个素理想。 
\end{proof}

\begin{definition}[极大理想]
设 $(R, +, \cdot)$ 是一个交换环,而 $\mathfrak{m} \lhd R$。则我们称 $\mathfrak{m}$ 是一个\textbf{极大理想},若 $\mathfrak{m} \neq R$,且它是个极大的理想,即对于任意 $I \lhd R$,如果 $I \supsetneq \mathfrak{m}$,则
\begin{align*}
I = R .
\end{align*}
这就是说,唯一严格大于 $\mathfrak{m}$ 的理想,是整个环。
\end{definition}

\begin{proposition}[极大理想的充要条件]\label{proposition:极大理想的充要条件}
设 $(R, +, \cdot)$ 是一个交换环,而 $\mathfrak{m} \lhd R$。则 $\mathfrak{m}$ 是一个极大理想,当且仅当商环 $R / \mathfrak{m}$ 是一个域。
\end{proposition}
\begin{proof}
先证必要性。令 $\mathfrak{m}$ 是一个极大理想。因为$R$是交换环,从而对 $a, b \in R$,我们有
\begin{align*}
(a + \mathfrak{m})(b + \mathfrak{m}) = ab + \mathfrak{m} = ba + \mathfrak{m} = (b + \mathfrak{m})(a + \mathfrak{m}).
\end{align*}
所以 $R / \mathfrak{m}$ 是交换环,因此我们只须证明每个非零元素都有逆元。令 $a + \mathfrak{m} \in R / \mathfrak{m}$且$a + \mathfrak{m} \neq 0 + \mathfrak{m}$,也就是说 $a \notin \mathfrak{m}$。只须证明存在 $b + \mathfrak{m} \in R / \mathfrak{m}(b \in R)$,使得 $ab + \mathfrak{m} = 1 + \mathfrak{m}$。
等价地,我们只须证明存在 $b \in R, m \in \mathfrak{m}$,使得
\begin{align*}
1 = ab + m .
\end{align*}
由\refpro{proposition:生成的理想的集合表示}可知 $\mathfrak{m}+(a)=\mathfrak{m} + Ra $,又因为 $a \notin \mathfrak{m}$,所以$\mathfrak{m}+(a)=\mathfrak{m} + Ra $是一个严格包含了 $\mathfrak{m}$ 的理想。因为 $\mathfrak{m}$ 是极大理想,所以 $\mathfrak{m} + Ra = R$。右边取 $1 \in R$,我们就得到了,存在 $b \in R,m\in \mathfrak{m}$,使得 $1 = ab + m$,这就证明了必要性。

再证充分性。如果 $R / \mathfrak{m}$ 是一个域,$\mathfrak{m}$是一个极大理想,那么对于任意理想 $I \supsetneq \mathfrak{m}$。由\refpro{proposition:域中加法单位元和乘法单位元一定不相等即0不等于1}可知$0\ne 1$,从而$0+\mathfrak{m}\ne 1+\mathfrak{m}$,于是$1\notin \mathfrak{m}$.故$\mathfrak{m}\ne R$,否则,由\refpro{lemma:理想是整个环的充要条件}可知$1\in \mathfrak{m}$矛盾!

再任取 $a \in I - \mathfrak{m}$。则 $a + \mathfrak{m} \neq 0 + \mathfrak{m}$.由于$R/\mathfrak{m}$是一个域,故$a+\mathbf{m}$有逆元,即存在 $b \in R$,使得 $(a+\mathfrak{m})(b+\mathfrak{m})=ab + \mathfrak{m} = 1 + \mathfrak{m}$。因此,也存在 $m \in \mathfrak{m}$,使 $1 = ab + m$。
因此,对任意 $r \in R$,由$I$和$\mathfrak{m}$都是$R$的理想可知
\begin{align*}
r = r(ab + m) = rab + rm \in Ib + \mathfrak{m} \subset I + I = I .
\end{align*}
这就证明了 $I\subset R$。又因为$I\subset R$,所以$I=R$.因此 $\mathfrak{m}$ 是一个极大理想。

综上所述,我们就证明了这个命题。 
\end{proof}

\begin{lemma}[域一定是整环]\label{lemma:域一定是整环}
设$(R, +, \cdot)$ 是一个域,则 $R$ 是一个整环。
\end{lemma}
\begin{remark}
但是整环不一定是域.
\end{remark}
\begin{proof}
由域的定义可知,一个域当然是一个交换环。又由\refpro{proposition:域中加法单位元和乘法单位元一定不相等即0不等于1}可知$0\ne 1$,故$0,1\in R$,因此$R\ne \{0\}.$令 $a, b \in R$,使 $ab = 0$。我们只须证明 $a = 0$ 或 $b = 0$。

假设 $a \neq 0, b \neq 0$,而 $ab = 0$。由$R$是域可知,存在 $c, d \in R$,使 $ac = bd = 1$。则
\begin{align*}
1 = 1 \cdot 1 = acbd = abcd = 0 \cdot cd = 0.
\end{align*}
而由\refpro{proposition:域中加法单位元和乘法单位元一定不相等即0不等于1}可知$0\ne 1$矛盾!因此每一个域都是整环。
\end{proof}

\begin{proposition}\label{proposition:在交换环中,每一个极大理想都是素理想}
设$(R, +, \cdot)$ 是一个交换环,则每一个极大理想都是素理想。
\end{proposition}
\begin{proof}
{\color{blue}证法一:}
令 $\mathfrak{m}$ 是一个极大理想,则 $R / \mathfrak{m}$ 是一个域。根据\reflem{lemma:域一定是整环}可知,$R / \mathfrak{m}$ 是一个整环,再利用\refpro{proposition:素理想的充要条件}可知 $\mathfrak{m}$ 是一个素理想。这就证明了这个命题。

{\color{blue}证法二:}
令 $\mathfrak{m}$ 是一个极大理想。假设 $a, b \in R$,使得 $ab \in \mathfrak{m}$,我们只须证明 $a \in \mathfrak{m}$ 或 $b \in \mathfrak{m}$。用反证法,假设 $a, b \notin \mathfrak{m}$。则 由\refpro{proposition:生成的理想的集合表示}可知 $\mathfrak{m}+(a)=\mathfrak{m} + Ra $,又因为 $a \notin \mathfrak{m}$,所以$\mathfrak{m}+(a)=\mathfrak{m} + Ra $是一个严格包含了 $\mathfrak{m}$ 的理想。因为 $\mathfrak{m}$ 是极大理想,这就迫使
\begin{align*}
R = \mathfrak{m} + Ra .
\end{align*}
从而由$1\in R$可知,存在 $m \in \mathfrak{m}$ 与 $r \in R$,使
\begin{align*}
1 = m + ra .
\end{align*}
则由于 $ab \in \mathfrak{m}$及$\mathfrak{m}$是一个理想,我们有
\begin{align*}
b = bm + r(ab) \in \mathfrak{m} + r\mathfrak{m} \subset \mathfrak{m} + \mathfrak{m} = \mathfrak{m} .
\end{align*}
可是这与 $b \notin \mathfrak{m}$ 相矛盾。
因此,$\mathfrak{m}$ 是一个素理想。 
\end{proof}

\begin{definition}[模理想同余]
设 $(R, +, \cdot)$ 是一个交换环,而 $I \lhd R$。令 $a, b \in R$,我们称 \textbf{$a, b$ 模 $I$ 同余},记作
\begin{align*}
a \equiv b \bmod I
\end{align*}
若它们的差在 $I$ 中,即
\begin{align*}
a - b \in I
\end{align*}
或等价地,
\begin{align*}
a + I = b + I
\end{align*}
\end{definition}

\begin{proposition}[模理想同余是一个等价关系]\label{proposition:模理想同余是一个等价关系}
设 $(R, +, \cdot)$ 是一个交换环,而 $I \lhd R$。令 $a, b ,c\in R$,则
\begin{enumerate}[(1)]
\item $a\equiv a \bmod I.$

\item 若$a\equiv b \bmod I,$则$b\equiv a \bmod I.$

\item 若$a\equiv b \bmod I,$ $b\equiv c \bmod I,$则$a\equiv c \bmod I.$
\end{enumerate}
\end{proposition}
\begin{proof}
\begin{enumerate}[(1)]
\item 因为$a-a=0\in I$,$(I,+)<(R,+)$,所以$a\equiv a \bmod I.$

\item 由$a\equiv b \bmod I$可知$a-b\in I.$于是由$(I,+)<(R,+)$可知$b-a=-(a-b)\in I$.故$b\equiv a \bmod I.$

\item 由$a\equiv b \bmod I,$ $b\equiv c \bmod I$可知$a-b,b-c\in I$.从而由$(I,+)<(R,+)$可知$a-c=(a-b)+(b-c)\in I.$故$a\equiv c \bmod I.$
\end{enumerate}
\end{proof}

\begin{definition}\label{definition:模理想同余等价类}
设 $(R, +, \cdot)$ 是一个交换环,而 $I \lhd R$。令 $a, b \in R$,令$a\in R$,我们定义$a$在模 $I$ 同余关系下的等价类为
\begin{align*}
\bar{a}=\{b\in R:b\equiv a \bmod I\}.
\end{align*}
\end{definition}

\begin{proposition}
设 $(R, +, \cdot)$ 是一个交换环,而 $I \lhd R$,$a\in R$,则
\begin{align*}
\bar{a}=\{b\in R:b\equiv a \bmod I\}=a+I.
\end{align*}
进而,$R/I={a+I:a\in R}$就是$R$在模 $I$ 同余关系下的一个分拆.
\end{proposition}
\begin{proof}
根据\refdef{definition:模理想同余等价类}可知
\begin{align*}
\bar{a}=\{b\in R:b\equiv a \bmod I\}=\left\{ b\in R:b-a\in I \right\} =\left\{ b\in R:b\in a+I \right\} =a+I.
\end{align*}
\end{proof}


\begin{proposition}[模理想同余的基本性质]\label{proposition:模理想同余的基本性质}
设 $(R, +, \cdot)$ 是一个交换环,而 $I \lhd R$.令$n \in \mathbb{N}_1$, $a, b, c, d \in R$.若
\begin{align*}
a &\equiv b \bmod I\\
c &\equiv d \bmod I
\end{align*}
则
\begin{align*}
a + c &\equiv b + d \bmod I\\
ac &\equiv bd \bmod I\\
a^n &\equiv b^n \bmod I
\end{align*}
进而,$f(a)\equiv f(b) \bmod I.$其中$f(x)$是关于$x$的多项式.
\end{proposition}
\begin{remark}
一个关系若对加法、乘法和幂次都成立,则它就一定对多项式也成立.
\end{remark}
\begin{proof}
由$a \equiv b \bmod I,c \equiv d \bmod I$可知$a-b,c-d\in I.$

第一条,因为$(I,+)<(R,+)$,$(R,+)$是Abel群,所以$(a+c)-(b+d)=(a-b)+(c-d)\in I.$故$a + c \equiv b + d \bmod I.$

第二条,由$a-b,c-d\in I$可知存在$r,s\in I,$使得$a=b+r,c=d+s$.从而由$I$是$R$的理想可得
\begin{align*}
ac-bd=(b+r)(d+s)-bd=bs+rd+rs\in I.
\end{align*}
故$ac \equiv bd \bmod I.$

第三条,结合数学归纳法,反复利用第二条结论即可得到$a^n \equiv b^n \bmod I$.
\end{proof}

\begin{theorem}[中国剩余定理]\label{theorem:中国剩余定理}
设$(R, +, \cdot)$ 是一个交换环,而 $(I_i)_{1 \leqslant i \leqslant n}$ 是一族两两互素的理想,即对任何 $i \neq j$ 都有 $I_i + I_j = R$。则对任何 $a_1, \cdots, a_n \in R$,都存在 $x \in R$,使
\begin{align*}
x &\equiv a_1 \bmod I_1 ,\\
&\cdots \\
x &\equiv a_n \bmod I_n .
\end{align*}
\end{theorem}
\begin{proof}
令 $a = (a_1, \cdots, a_n)$,则
\begin{align*}
a = a_1(1, 0, \cdots, 0) + \cdots + a_n(0, \cdots, 0, 1) .
\end{align*}
假如 $x_i (1 \leqslant i \leqslant n)$ 分别满足
\begin{gather*}
x_i \equiv 1 \bmod I_i.\\
\text{若 } j \neq i, \ x_i \equiv 0 \bmod I_j .
\end{gather*}
则根据\hyperref[proposition:模理想同余的基本性质]{模理想同余的基本性质}可知,$x = a_1 x_1 + \cdots + a_n x_n$ 就一定满足了同余方程组
\begin{gather*}
x \equiv a_1 \bmod I_1 ,\\
\cdots \\
x \equiv a_n \bmod I_n .
\end{gather*}
因此我们只须证明对任何 $1 \leqslant i \leqslant n$,我们能找到 $x_i \in R$,使得
\begin{gather*}
x_i \equiv 1 \bmod I_i,\\
\text{若 } j \neq i, \ x_i \equiv 0 \bmod I_j .
\end{gather*}
不失一般性,我们假设 $i = 1$。由于 $I_1$ 与 $I_j (j \neq 1)$ 都互素,特别地,$1\in I_1+I_j(j\ne 1)$.则存在 $b_j \in I_1, c_j \in I_j (j \neq 1)$,使得
\begin{gather*}
b_2 + c_2 = 1 ,\\
\cdots \\
b_n + c_n = 1 .
\end{gather*}
令 $x_1 = c_2 \cdots c_n \in R$。则对任何 $j \neq 1$,由$I_j\lhd R$,我们有
\begin{align*}
c_2 \cdots c_j \cdots c_n \in I_j.
\end{align*}
即
\begin{align*}
x_1 \equiv c_2 \cdots c_j \cdots c_n \equiv 0 \bmod I_j.
\end{align*}
并且
\begin{align*}
1 - c_2 \cdots c_n = (b_2 + c_2) \cdots (b_n + c_n) - (c_2 \cdots c_n) .
\end{align*}
根据分配律,将上式展开后,上面的每一项都包含至少某个 $b_i \in I_1$ 作为因子,因此
\begin{align*}
1 - c_2 \cdots c_n \in I_1 .
\end{align*}
于是
\begin{align*}
x_1 = c_2 \cdots c_n \equiv 1 \bmod I_1 .
\end{align*}
这就完成了 $x_1$ 的构造。类似地,我们可以构造出所有的 $x_i (1 \leqslant i \leqslant n)$,因此
\begin{align*}
x \equiv a_1 x_1 + \cdots + a_n x_n .
\end{align*}
给出了原命题所需的解。

综上所述,我们通过线性性对原同余方程组进行了化简,并不失一般性地证明了 $i = 1$ 的情形,这就完成了中国剩余定理的证明。
\end{proof}

\begin{proposition}[中国剩余定理推论]\label{proposition:中国剩余定理推论}
设 $(R, +, \cdot)$ 是一个交换环,而 $(I_i)_{1 \leqslant i \leqslant n}$ 是一族两两互素的理想,即对任何 $i \neq j$ 都有 $I_i + I_j = R$。则
\begin{align*}
\pi: R &\to \prod_{i = 1}^n (R / I_i) ,\\
\pi(a) &= (a + I_1, \cdots, a + I_n) .
\end{align*}
是个满同态。
特别地,
\begin{align*}
R \big/ \bigcap_{i = 1}^n I_i \simeq \prod_{i = 1}^n (R / I_i) .
\end{align*}
因此在以上条件下,$\pi$ 是个同构当且仅当
\begin{align*}
\bigcap_{i = 1}^n I_i = \{0\} .
\end{align*}
\end{proposition}
\begin{proof}
$\pi$ 的每一个坐标都是环同态,因此 $\pi$ 也是环同态。根据\hyperref[theorem:中国剩余定理]{中国剩余定理的证明}可知,对任意$(a_1+I_1,\cdots,a_n+I_n)\in \prod_{i = 1}^n (R / I_i)$,都存在$a\in R$,使得
\begin{align*}
&a\equiv a_i\,\,\mathrm{mod}\,I_i\left( i=1,2,\cdots ,n \right) \Longleftrightarrow a+I_i=a_i+I_i\left( i=1,2,\cdots ,n \right) 
\\
&\Longleftrightarrow \pi \left( a \right) =\left( a+I_1,\cdots ,a+I_n \right) =\left( a_1+I_1,\cdots ,a_n+I_n \right)  .
\end{align*}
故$\pi$ 是个满同态。
我们只须找到 $\pi$ 的核即可。根据 $\pi$ 的定义,
\begin{align*}
\pi(a) = 0 &\iff \forall i, a + I_i = 0 + I_i \\
&\iff \forall i, a \in I_i \\
&\iff a \in \bigcap_{i = 1}^n I_i .
\end{align*}
因此$\ker \pi=\bigcap_{i = 1}^n I_i.$
根据\hyperref[theorem:环同构第一定理]{环同构第一定理},这就证明了
\begin{align*}
R \big/ \bigcap_{i = 1}^n I_i \simeq \prod_{i = 1}^n (R / I_i).
\end{align*}
因此在以上的条件下,$\pi$ 是同构当且仅当 $\pi$ 是单的,当且仅当 $\ker(\pi) = \{0\}$,当且仅当
\begin{align*}
\bigcap_{i = 1}^n I_i = \{0\} .
\end{align*}
因此,最特殊的情况即 $R$ 中有有限多个两两互素且总的交集为 $\{0\}$ 的理想。在这种情况下,
\begin{align*}
R \simeq \prod_{i = 1}^n (R / I_i).
\end{align*}
综上所述,我们证明了这个命题。 
\end{proof}

\begin{corollary}[中国剩余定理]\label{corollary:中国剩余定理推论}
设$n\in \mathbb{N}_1$,由算术基本定理可知,$n$存在素幂因子分解,即存在$p_1,p_2,\cdots,p_m$两两互素,$\alpha_1,\alpha_2,\cdots,\alpha_m\in \mathbb{N}_1$,使得
\begin{align*}
n=p_{1}^{\alpha _1}p_{2}^{\alpha _2}\cdots p_{m}^{\alpha _m}.
\end{align*}
则
\begin{align*}
\mathbb{Z} /n\mathbb{Z} =\mathbb{Z} _n=\mathbb{Z} _{\prod_{i=1}^m{p_{i}^{\alpha _i}}}\cong \prod_{i=1}^m{\mathbb{Z} _{p_{i}^{\alpha _i}}}.
\end{align*}
\end{corollary}
\begin{proof}
由\refpro{proposition:理想的乘积等于理想的交当且仅当它们互素}可知
\begin{align*}
n\mathbb{Z} =\prod_{i=1}^m{p_{i}^{\alpha _i}}\mathbb{Z} =\bigcap_{i=1}^m{\left( p_{i}^{\alpha _i}\mathbb{Z} \right)}.
\end{align*}
从而由\hyperref[proposition:中国剩余定理推论]{中国剩余定理推论}可知
\begin{align*}
\mathbb{Z} _n=\mathbb{Z} /n\mathbb{Z} =\mathbb{Z} /\bigcap_{i=1}^m{\left( p_{i}^{\alpha _i}\mathbb{Z} \right)}\cong \prod_{i=1}^m{\left( \mathbb{Z} /p_{i}^{\alpha _i}\mathbb{Z} \right)}=\prod_{i=1}^m{\mathbb{Z} _{p_{i}^{\alpha _i}}}.
\end{align*}
\end{proof}






\end{document}
\documentclass[../../main.tex]{subfiles}
\graphicspath{{\subfix{../../image/}}} % 指定图片目录,后续可以直接使用图片文件名。

% 例如:
% \begin{figure}[H]
% \centering
% \includegraphics[scale=0.4]{图.png}
% \caption{}
% \label{figure:图}
% \end{figure}
% 注意:上述\label{}一定要放在\caption{}之后,否则引用图片序号会只会显示??.

\begin{document}

\section{唯一分解整环(UFD)}

\begin{theorem}
设 \( R \) 是交换整环,由\rrefpro{proposition:整环的一些性质}{proposition:整环的一些性质-2}知\( R^* = R \setminus \{0\} \) 对乘法构成交换幺半群且消去律成立. 以 \( U \) 表示 \( R^* \) 中加法可逆元素的集合,则 \( U \) 对加法构成一个 Abel 群,称为 \( R \) 的\textbf{单位群}. \( U \) 中元素称为 \( R \) 的\textbf{单位}.
\end{theorem}
\begin{proof}


\end{proof}

\begin{definition}[整数]
设 \( R \) 是交换整环,\( R^* = R \setminus \{0\} \), \( a, b \in R^* \),若 \( \exists c \in R^* \),使 \( b = ac \),则称 \( a \) 能\textbf{整除} \( b \),或 \( a \) 是 \( b \) 的\textbf{因子},或 \( b \) 是 \( a \) 的\textbf{倍式}. 记为 \( a|b \). \( a \) 不能整除 \( b \),记为 \( a \nmid b \).
\end{definition}

\begin{definition}[相伴]
设 \( R \) 是交换整环,\( R^* = R \setminus \{0\} \), \( a, b \in R^* \), 且 \( a|b, b|a \),则称 \( a \) 与 \( b \) \textbf{相伴},记为 \( a \sim b \).
\end{definition}

\begin{theorem}\label{theorem:单位群整除的基本性质}
设 \( R \) 是交换整环,\( R^* = R \setminus \{0\} \), \( a, b \in R^* \),则
\begin{enumerate}[(1)]
\item \( a|a, \forall a \in R^* \).

\item 若 \( a|b, b|c \),则 \( a|c \).

\item\label{theorem:单位群整除的基本性质-3} \( u \in U \),则 \( u|a, \forall a \in R^* \).

\item\label{theorem:单位群整除的基本性质-4} \( u \in U \iff u|1 \).

\item \( a \sim b \iff  \exists u \in U \),使 \( b = au \).

\item 相伴关系是幺半群 \( R^* \) 中的同余关系.

\item \( u \in U \iff u \sim 1 \).
\end{enumerate}
\end{theorem}
\begin{proof}
\begin{enumerate}[(1)]
\item 这是因为 \( a = 1 \cdot a \).

\item 由 \( b = ad, c = be \) 得 \( c = a(de) \).

\item 这是因为 \( a = u(u^{-1}a) \).

\item 由\hyperref[theorem:单位群整除的基本性质-3]{性质\ref{theorem:单位群整除的基本性质-3}}知 \( u \in U \) 时,\( u|1 \). 反之,若 \( u|1 \),即有 \( v \),使得 \( 1 = vu \),故 \( v = u^{-1}(u \in U) \).

由\hyperref[theorem:单位群整除的基本性质-3]{性质\ref{theorem:单位群整除的基本性质-3}}知 \( \forall u \in U, a \in R^* \),\( u \) 是 \( a \) 的因子,这种因子称为{\heiti 平凡因子}.

\item 事实上,若 \( b = au(u \in U) \),则 \( a = bu^{-1} \). 因而 \( a|b, b|a \),即 \( a \sim b \).

反之,若 \( a|b, b|a \),即有 \( c, d \in R^* \),使得 \( b = ac, a = bd \). 于是 \( b = b(dc) \). 故 \( dc = 1 \),因而 \( d, c \in U \).

\item 相伴关系显然是等价关系. 设 \( a \sim b, c \sim d \). 于是 \( \exists u_1, u_2 \in U \),使得 \( b = au_1 \),\( d = cu_2 \). 于是 \( bd = ac(u_1u_2) \). 由 \( u_1u_2 \in U \) 及性质1知 \( ac \sim bd \),即相伴关系是同余关系.

\item 这是整除的\hyperref[theorem:单位群整除的基本性质-3]{性质\ref{theorem:单位群整除的基本性质-3}}与\hyperref[theorem:单位群整除的基本性质-3]{性质\ref{theorem:单位群整除的基本性质-4}}的直接推论.
\end{enumerate}

\end{proof}

\begin{definition}
设 \( a, b \in R^* \). 若 \( b|a \),但 \( a \nmid b \),则称 \( b \) 为 \( a \) 的\textbf{真因子}.换言之,\( b \) 为 \( a \) 的真因子当且仅当 \( b \) 是 \( a \) 的因子且 \( b \) 与 \( a \) 不相伴.
\end{definition}

如果 \( u \in U \),则 \( u \) 无真因子. 事实上若 \( v \) 是 \( u \) 的因子,即 \( v|u \),又 \( u|1 \),故 \( v|1 \),因而 \( v \in U \),亦即 \( v \sim u \),由此知 \( u \) 无真因子.

\begin{definition}
设 \( a \in R^* \setminus U \). 若 \( a \) 无非平凡的真因子,则称 \( a \) 为\textbf{不可约元素}. 若 \( a \) 有非平凡的真因子,则称 \( a \) 为\textbf{可约元素}.
\end{definition}

\begin{example}
在整数环 \( \mathbb{Z} \) 中,\( U = \{1, -1\} \),于是 \( a \sim b \iff a = \pm b \),因而 \( a \) 为不可约元素当且仅当 \( a \) 为素数或负素数.
\end{example}

\begin{example}
设 \( \mathbf{P} \) 为数域,则 \( \mathbf{P} \) 上一元多项式环 \( \mathbf{P}[x] \) 为交换整环. 此时 \( U = \mathbf{P}^* = \mathbf{P} \setminus \{0\} \). \( f(x) \sim g(x) \) iff \( \exists c \in \mathbf{P}^* \),使得 \( f(x) = cg(x) \),因而 \( f(x) \) 为不可约元素当且仅当 \( f(x) \) 为不可约多项式.
\end{example}

\begin{definition}
若 \( p \in R^* \setminus U \) 且满足
\begin{align*}
p|ab \Rightarrow p|a \quad \text{或} \quad p|b,
\end{align*}
则称 \( p \) 为素元素.
\end{definition}

例2.4.1与例2.4.2中的不可约元素都是素元素.

\begin{lemma}
素元素一定是不可约元素.
\end{lemma}
\begin{proof}
若 \( a \) 是素元素 \( p \) 的一个因子,即有 \( b \in R^* \),使 \( p = ab \),因而 \( p|a \) 或 \( p|b \). 在 \( p|a \) 的情况,说明 \( a \) 不是 \( p \) 的真因子. 若 \( p|b \),即有 \( c \in R^* \),使 \( b = pc \),于是 \( p = pac \),故 \( ac = 1(a \in U) \),即 \( a \) 为平凡因子. 这说明 \( p \) 没有非平凡的真因子,故 \( p \) 是不可约元素.

\end{proof}














\end{document}
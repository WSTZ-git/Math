\documentclass[../../main.tex]{subfiles}
\graphicspath{{\subfix{../../image/}}} % 指定图片目录,后续可以直接使用图片文件名。

% 例如:
% \begin{figure}[H]
% \centering
% \includegraphics[scale=0.4]{图.png}
% \caption{}
% \label{figure:图}
% \end{figure}
% 注意:上述\label{}一定要放在\caption{}之后,否则引用图片序号会只会显示??.

\begin{document}

\section{唯一析因环(唯一分解整环)(UFD)}

\begin{theorem}
设 \( R \) 是交换整环,由\rrefpro{proposition:整环的一些性质}{proposition:整环的一些性质-2}知\( R^* = R \setminus \{0\} \) 对乘法构成交换幺半群且消去律成立. 以 \( U \) 表示 \( R^* \) 中乘法可逆元素的集合,则 \( U \) 对乘法构成一个 Abel 群,称为 \( R \) 的\textbf{单位群}. \( U \) 中元素称为 \( R \) 的\textbf{单位}.
\end{theorem}
\begin{proof}


\end{proof}

\begin{definition}[整数]
设 \( R \) 是交换整环,\( R^* = R \setminus \{0\} \), \( a, b \in R^* \),若 \( \exists c \in R^* \),使 \( b = ac \),则称 \( a \) 能\textbf{整除} \( b \),或 \( a \) 是 \( b \) 的\textbf{因子},或 \( b \) 是 \( a \) 的\textbf{倍式}. 记为 \( a|b \). \( a \) 不能整除 \( b \),记为 \( a \nmid b \).
\end{definition}

\begin{definition}[相伴]
设 \( R \) 是交换整环,\( R^* = R \setminus \{0\} \), \( a, b \in R^* \), 且 \( a|b, b|a \),则称 \( a \) 与 \( b \) \textbf{相伴},记为 \( a \sim b \).
\end{definition}

\begin{theorem}\label{theorem:单位群整除的基本性质}
设 \( R \) 是交换整环,\( R^* = R \setminus \{0\} \), \( a, b \in R^* \), \( U \) 表示 \( R^* \) 中乘法可逆元素的集合,则
\begin{enumerate}[(1)]
\item\label{theorem:单位群整除的基本性质-1} \( a|a, \forall a \in R^* \).

\item 若 \( a|b, b|c \),则 \( a|c \).

\item\label{theorem:单位群整除的基本性质-3} 若\( u \in U \),则 \( u|a, \forall a \in R^* \). 

\item\label{theorem:单位群整除的基本性质-4} \( u \in U \iff u|1 \).

\item\label{theorem:单位群整除的基本性质-5} \( a \sim b \iff  \exists u \in U \),使 \( b = au \).

\item 相伴关系是幺半群 \( R^* \) 中的同余关系.

\item\label{theorem:单位群整除的基本性质-7} \( u \in U \iff u \sim 1 \).
\end{enumerate}
\end{theorem}
\begin{proof}
\begin{enumerate}[(1)]
\item 这是因为 \( a = 1 \cdot a \).

\item 由 \( b = ad, c = be \) 得 \( c = a(de) \).

\item 这是因为 \( a = u(u^{-1}a) \).

\item 由\hyperref[theorem:单位群整除的基本性质-3]{性质\ref{theorem:单位群整除的基本性质-3}}知 \( u \in U \) 时,\( u|1 \). 反之,若 \( u|1 \),即有 \( v \),使得 \( 1 = vu \),故 \( v = u^{-1}(u \in U) \).

\item 事实上,若 \( b = au(u \in U) \),则 \( a = bu^{-1} \). 因而 \( a|b, b|a \),即 \( a \sim b \).

反之,若 \( a|b, b|a \),即有 \( c, d \in R^* \),使得 \( b = ac, a = bd \). 于是 \( b = b(dc) \). 由\rrefpro{proposition:整环的一些性质}{proposition:整环的一些性质-2}知$R^*$对乘法满足消去律,故 \( dc = 1 \),因而 \( d, c \in U \).

\item 相伴关系显然是等价关系. 设 \( a \sim b, c \sim d \). 于是 \( \exists u_1, u_2 \in U \),使得 \( b = au_1 \),\( d = cu_2 \). 于是 \( bd = ac(u_1u_2) \). 由 \( u_1u_2 \in U \) 及\hyperref[theorem:单位群整除的基本性质-5]{性质\ref{theorem:单位群整除的基本性质-5}}知 \( ac \sim bd \),即相伴关系是同余关系.

\item 注意到$1\in U$,故由\hyperref[theorem:单位群整除的基本性质-3]{性质\ref{theorem:单位群整除的基本性质-3}}知$1|u.$再由\hyperref[theorem:单位群整除的基本性质-3]{性质\ref{theorem:单位群整除的基本性质-4}}知 \( u \in U \iff u|1 \iff u\sim 1.\)
\end{enumerate}

\end{proof}

\begin{definition}
设 \( R \) 是交换整环,\( R^* = R \setminus \{0\} \), \( U \) 表示 \( R^* \) 中乘法可逆元素的集合,则\( \forall u \in U, a \in R^* \),由\rrefthe{theorem:单位群整除的基本性质}{theorem:单位群整除的基本性质-1}和\rrefthe{theorem:单位群整除的基本性质}{theorem:单位群整除的基本性质-3}知\( u\) 是 \( a \) 的因子,这种因子称为\textbf{平凡因子}.
\end{definition}

\begin{definition}
设 \( R \) 是交换整环,\( R^* = R \setminus \{0\} \),\( a, b \in R^* \). 若 \( b|a \),但 \( a \nmid b \),则称 \( b \) 为 \( a \) 的\textbf{真因子}.换言之,\( b \) 为 \( a \) 的真因子当且仅当 \( b \) 是 \( a \) 的因子且 \( b \) 与 \( a \) 不相伴.
\end{definition}

\begin{theorem}\label{theorem:单位群中元素都无真因子}
设 \( R \) 是交换整环,\( R^* = R \setminus \{0\} \), \( U \) 表示 \( R^* \) 中乘法可逆元素的集合,则如果 \( u \in U \),则 \( u \) 无真因子.
\end{theorem}
\begin{proof}
事实上若 \( v \) 是 \( u \) 的因子,即 \( v|u \),又由\rrefthe{theorem:单位群整除的基本性质}{theorem:单位群整除的基本性质-4}知\( u|1 \),故 \( v|1 \),因而再由\rrefthe{theorem:单位群整除的基本性质}{theorem:单位群整除的基本性质-4}知\( v \in U \subseteq R^*\),故由\rrefthe{theorem:单位群整除的基本性质}{theorem:单位群整除的基本性质-3}知$u|v$.因此 \( v \sim u \),由此知 \( u \) 无真因子.

\end{proof}

\begin{definition}
设 \( R \) 是交换整环,\( R^* = R \setminus \{0\} \),\( U \) 表示 \( R^* \) 中乘法可逆元素的集合, \( a \in R^* \setminus U \). 若 \( a \) 无非平凡的真因子,则称 \( a \) 为\textbf{不可约元素}. 若 \( a \) 有非平凡的真因子,则称 \( a \) 为\textbf{可约元素}.
\end{definition}

\begin{proposition}\label{proposition:不可约元素乘单位仍不可约}
设 \( R \) 是交换整环,\( R^* = R \setminus \{0\} \),\( U \) 表示 \( R^* \) 中乘法可逆元素的集合,$u\in U$,$a$为$R$的不可约元素,则$au$也是不可约的.进而,若$a\sim b$,则$b$也不可约.
\end{proposition}
\begin{proof}
反证,假设$au$可约,则存在$e\in R^*$为$au$的非平凡真因子.从而存在$x\in R^*$,使$au=ex$,进而$a=exu^{-1}$.于是$e$也是$a$的非平凡真因子,这与$a$不可约矛盾!故$au$不可约.由\rrefthe{theorem:单位群整除的基本性质}{theorem:单位群整除的基本性质-5}可知存在$u'\in U$,使$b=au'$.由之前证明知$b=au'$也不可约.

\end{proof}

\begin{definition}
设 \( R \) 是交换整环,\( R^* = R \setminus \{0\} \),\( U \) 表示 \( R^* \) 中乘法可逆元素的集合, 若 \( p \in R^* \setminus U \) 且满足
\begin{align*}
p|ab \Rightarrow p|a \text{或} p|b,
\end{align*}
则称 \( p \) 为\textbf{素元素}.
\end{definition}

\begin{example}\label{example:抽象代数--例2.4.2}
在整数环 \( \mathbf{Z} \) 中,\( U = \{1, -1\} \),于是 \( a \sim b \iff a = \pm b \),因而 \( a \) 为不可约元素当且仅当 \( a \) 为素数或负素数.并且整数环$\mathbf{Z}$的不可约元素都是素元素.
\end{example}
\begin{proof}
    

\end{proof}

\begin{example}\label{example:抽象代数--例2.4.3}
设 \( \mathbf{P} \) 为数域,则 \( \mathbf{P} \) 上一元多项式环 \( \mathbf{P}[x] \) 为交换整环. 此时 \( U = \mathbf{P}^* = \mathbf{P} \setminus \{0\} \). \( f(x) \sim g(x) \iff \exists c \in \mathbf{P}^* \),使得 \( f(x) = cg(x) \),因而 \( f(x) \) 为不可约元素当且仅当 \( f(x) \) 为不可约多项式.并且一元多项式环\( \mathbf{P}[x] \)的不可约元素都是素元素.
\end{example}
\begin{proof}


\end{proof}

\begin{lemma}
设 \( R \) 是交换整环,\( R^* = R \setminus \{0\} \),\( U \) 表示 \( R^* \) 中乘法可逆元素的集合, 则$R$中的素元素一定是不可约元素.
\end{lemma}
\begin{remark}
不可约元素不一定是素元素,反例见\refexa{example:不可约元素不是素元素的例子}.
\end{remark}
\begin{proof}
若 \( a \) 是素元素 \( p \) 的一个因子,即有 \( b \in R^* \),使 \( p = ab \),因而 \( p|a \) 或 \( p|b \). 在 \( p|a \) 的情况,说明 \( a \) 不是 \( p \) 的真因子. 若 \( p|b \),即有 \( c \in R^* \),使 \( b = pc \),于是 \( p = pac \),由\rrefpro{proposition:整环的一些性质}{proposition:整环的一些性质-2}知$R^*$对乘法满足消去律,故 \( ac = 1\),从而$a \in U$,即 \( a \) 为平凡因子. 这说明 \( p \) 没有非平凡的真因子,故 \( p \) 是不可约元素.

\end{proof}

\begin{example}\label{example:不可约元素不是素元素的例子}
令 \( R = \mathbf{Z}[\sqrt{-5}] = \{a + b\sqrt{-5} \mid a, b \in \mathbf{Z}\} \)。设 \( \alpha = a + b\sqrt{-5} \),称 \( \overline{\alpha} = a - b\sqrt{-5} \) 为 \( \alpha \) 的共轭,称 \( N(\alpha) = \alpha\overline{\alpha} = a^2 + 5b^2 \) 为 \( \alpha \) 的范数,显然 \( N(\alpha) \in \mathbf{Z} \) 且 \( N(\alpha) \geqslant 0 \),当且仅当 \( \alpha = 0 \) 时等号成立。则$\mathbf{Z}[\sqrt{-5}]$的单位群$U=\{1,-1\}$,且\( 3 \) 是 \( \mathbf{Z}[\sqrt{-5}] \) 的不可约元素,但不是 \( \mathbf{Z}[\sqrt{-5}] \) 的素元素。
\end{example}
\begin{proof}
注意到\( \forall \alpha, \beta \in R \) 有 \( N(\alpha\beta) = N(\alpha)N(\beta) \)。
先求 \( R \) 的单位群 \( U \)。\( \alpha \in U \),则有 \( \alpha\alpha^{-1} = 1 \),故 \( N(\alpha)N(\alpha^{-1}) = N(1) = 1 \),故 \( N(\alpha) = 1 \)。由此即得 \( U = \{1, -1\} \),因而 \( \alpha \sim \beta \iff \alpha = \pm\beta \)。

再证明 \( 3 \) 是 \( \mathbf{Z}[\sqrt{-5}] \) 的不可约元素,但不是 \( \mathbf{Z}[\sqrt{-5}] \) 的素元素。
设 \( \alpha = a + b\sqrt{-5} \) 是 \( 3 \) 的一个因子,故有 \( \beta \),使 \( 3 = \alpha\beta \),于是 \( N(3) = N(\alpha)N(\beta) \)。由 \( N(3) = 9 \) 知 \( N(\alpha) \) 有以下三种可能:
\begin{enumerate}[(1)]
\item \( N(\alpha) = 1 \),则 \( \alpha = \pm 1 \),即 \( \alpha \) 是 \( 3 \) 的平凡因子;
\item \( N(\alpha) = 3 \),于是 \( a^2 + 5b^2 = 3 \),但此方程无整数解,故这种情况不存在;
\item \( N(\alpha) = 9 \),于是 \( N(\beta) = 1 \),\( \beta = \pm 1 \),即有 \( \alpha = \pm 3 \),\( \alpha \sim 3 \),即 \( \alpha \) 不是 \( 3 \) 的真因子。
\end{enumerate}
由上知 \( 3 \) 是不可约元素。另一方面,\( 3 \mid 9 \),\( 9 = (2 + \sqrt{-5})(2 - \sqrt{-5}) \)。由于 \( N(2 + \sqrt{-5}) = N(2 - \sqrt{-5}) = N(3) = 9 \),而 \( 3 \) 与 \( 2 \pm \sqrt{-5} \) 不相伴,因而 \( 3 \nmid 2 \pm \sqrt{-5} \),即 \( 3 \) 不是素元素。
\end{proof}

\begin{definition}
若一个交换整环 \( R \) 的不可约元素是素元素,则称 \( R \) 满足\textbf{素性条件}。
\end{definition}

\begin{definition}
设 \( R \) 是交换整环,\( R^* = R \setminus \{0\} \),\( b, c \in R^* \)。若 \( d \in R^* \) 满足 \( d \mid b \),\( d \mid c \),则称 \( d \) 为 \( b, c \) 的\textbf{公因子}。若对 \( b, c \) 的任一公因子 \( d_1 \) 有 \( d_1 \mid d \),则称 \( d \) 是 \( b, c \) 的\textbf{最大公因子}。

对 \( R^* \) 中任意有限个元素也可类似地定义它们的最大公因子。
\end{definition}
\begin{remark}
一般来说,\( R^* \) 中任意两个元素的最大公因子不一定存在。
\end{remark}

\begin{definition}
设 \( R \) 是交换整环,\( R^* = R \setminus \{0\} \),如果 \( R^* \) 中任意两个元素的最大公因子存在,则称 \( R \) 满足\textbf{最大公因子条件}。
\end{definition}

\begin{lemma}\label{lemma:最大公因子条件的基本结论}
设交换整环 \( R \) 满足最大公因子条件,\( R^* = R \setminus \{0\} \),$a,b,c\in R^*$,则有下列结论:
\begin{enumerate}[(1)]
\item\label{lemma:最大公因子条件的基本结论-1} 设 \( d \) 是 \( a, b \) 的一个最大公因子,则 \( d_1 \) 为$a,b$的最大公因子当且仅当 \( d_1 \sim d \),即 \( a, b \) 的最大公因子在相伴意义下是唯一的,记为 \( (a, b) \).若 \( (a, b) \sim 1 \),则称 \( a \) 与 \( b \) 为\textbf{互素};
\item\label{lemma:最大公因子条件的基本结论-2} \( \forall a_1, a_2, \cdots, a_r \in R^* \) 均有最大公因子;

\item\label{lemma:最大公因子条件的基本结论-6} 若$b \sim c$,则$(a, b) \sim (a, c)$.

\item\label{lemma:最大公因子条件的基本结论-3} \( ((a, b), c) \sim (a, (b, c)) \);
\item\label{lemma:最大公因子条件的基本结论-4} \( c(a, b) = (ca, cb) \);

\item\label{lemma:最大公因子条件的基本结论-7} 若$a\in U$,则$(a,b)= 1.$ 

\item\label{lemma:最大公因子条件的基本结论-5} 若 \( (a, b) \sim 1 \),\( (a, c) \sim 1 \),则 \( (a, bc) \sim 1 \).

\item\label{lemma:最大公因子条件的基本结论-8} 若$p$是不可约元素,则$p\nmid a\iff (p,a)\sim 1.$
\end{enumerate}
\end{lemma}
\begin{proof}
\begin{enumerate}[(1)]
\item 由于\( d, d_1 \) 是 \( a, b \) 的最大公因子,故 \( d \mid d_1 \),\( d_1 \mid d \)。于是 \( d \sim d_1 \)。反之,\( d_1 \sim d \),故 \( d_1 \mid d \)。又$d\mid a,b$,于是 \( d_1 \mid a,b \),因而 \( d_1 \) 是 \( a, b \) 的公因子。又若 \( c \) 是 \( a,b \) 的公因子,则 \( c \mid d \),而 \( d \mid d_1 \),故有 \( c \mid d_1 \),因而 \( d_1 \) 是 \( a, b\) 的最大公因子。

\item 令 \( d_1 = (a_1, a_2) \),\( d_2 = (d_1, a_3) \),\( d_3 = (d_2, a_4) \),\( \cdots \),\( d = d_{r-1} = (d_{r-2}, a_r) \)。下面证明 \( d \) 是 \( a_1, a_2, \cdots, a_r \) 的最大公因子。显然有 \( d \mid d_k \ (1 \leqslant k \leqslant r-2) \),\( d \mid a_r \)。又 \( d_k \mid a_{k+1} \),故 \( d \mid a_i \ (1 \leqslant i \leqslant r) \),即 \( d \) 为公因子。又若 \( a \mid a_i \ (1 \leqslant i \leqslant r) \),则 \( a \mid d_1 \) 且依次 \( a \mid d_2 \),\( a \mid d_3 \),\( \cdots \),最后有 \( a \mid d_{r-1} \),即 \( a \mid d \),因而 \( d \) 是最大公因子。

\item 设$d = (a, b)$,则$d \mid a, b$. 由$b \sim c$知$b \mid c$,故$d \mid a, c$,即$d$是$a, c$的公因子. 又设$d_1$也是$a, c$的公因子,又$b \sim c$,故$c \mid b$,从而$d_1 \mid a, b$,即$d_1$是$a, b$的公因子. 故$d_1 \mid d$. 因此$d$是$a, c$的最大公因子. 由\hyperref[lemma:最大公因子条件的基本结论-1]{结论\ref{lemma:最大公因子条件的基本结论-1}}知$d \sim (a, c)$.

\item 由\hyperref[lemma:最大公因子条件的基本结论-2]{结论\ref{lemma:最大公因子条件的基本结论-2}}同理可知 \( ((a, b), c) \) 与 \( (a, (b, c)) \) 都是 \( a, b, c \) 的最大公因子。由\hyperref[lemma:最大公因子条件的基本结论-1]{结论\ref{lemma:最大公因子条件的基本结论-1}} 知它们相伴。
\item 设 \( d = (a, b) \),\( e = (ca, cb) \),则 \( cd \mid ca \),\( cd \mid cb \)。于是 \( cd \mid e \),因而 \( e = cdu(u\in R^*) \).又由$ca\mid e$知$ca=ex(x\in R^*)$.由此知 \( ca = ex = xucd \)。由\rrefpro{proposition:整环的一些性质}{proposition:整环的一些性质-2}知$R^*$对乘法满足消去律,故\( a = xud \),即 \( ud \mid a \),同样有 \( ud \mid b \),故 \( ud \mid d \),于是$d=udk(k\in R^*)$,同样由$R^*$对乘法满足消去律可得$uk=1$,因而 \( u \in U \).于是由\rrefthe{theorem:单位群整除的基本性质}{theorem:单位群整除的基本性质-5}知 \( e \) 与 \( cd \) 相伴。

\item 显然$1\mid a,b$. 设$d\mid a,b$,则存在$a_1\in R^*$,使$a=da_1$. 于是由$a\in U$知$1=aa^{-1}=d\left( a_1a^{-1} \right)$,故$d|1$. 因此$\left( a,b \right) =1$.

\item 因为\( (a, b) \sim 1 ,(1,c)\sim 1\),由\hyperref[lemma:最大公因子条件的基本结论-4]{结论\ref{lemma:最大公因子条件的基本结论-4}} 知 \( (ac, bc) \sim c \), \( (a, ac) \sim a \),故由\hyperref[lemma:最大公因子条件的基本结论-3]{结论\ref{lemma:最大公因子条件的基本结论-3}}及\hyperref[lemma:最大公因子条件的基本结论-6]{结论\ref{lemma:最大公因子条件的基本结论-6}}有 \( 1 \sim (a, c) \sim (a, (ac, bc)) \sim ((a, ac), bc) \sim (a, bc) \).

\item $\Longleftarrow:$ 假设$p \mid a$,则存在$a_1 \in R^*$,使$a = pa_1$。于是由结论(5)和结论(6)知
$1 \sim (p, a) = (p, pa_1) = p(1, a_1) = p.$
但由$p$不可约知$p \notin U$,由\rrefthe{theorem:单位群整除的基本性质}{theorem:单位群整除的基本性质-5}知$p \nmid 1$,矛盾!

$\Longrightarrow:$ 设$d = (p, a)$,则存在$p_1, a_1 \in R^*$,使$p = dp_1, a = da_1$。
假设$d \nmid 1$,则由\rrefthe{theorem:单位群整除的基本性质}{theorem:单位群整除的基本性质-5}知$d \notin U$,从而$d \in R^* \setminus U$。若$p \mid d$,则由$d \mid a$知$p \mid a$,这与$p \nmid a$矛盾!故$p \nmid d$。

若$d \neq p$,则由$p = dp_1, p \nmid d$及$d \in R^* \setminus U$知$d$是$p$的非平凡真因子,这与$p$不可约矛盾!

若$d = p$,则由$a = da_1$知$a = pa_1$,即$p \mid a$,这与$p \nmid a$矛盾!

因此$d \mid 1$,故$d \sim 1$。
\end{enumerate}

\end{proof}

\begin{definition}[唯一析因环]
设 \( R \) 是交换整环,\( R^* = R \setminus \{0\} \),\( U \) 表示 \( R^* \) 中乘法可逆元素的集合,如果\( R \) 满足下列条件:
\begin{enumerate}[(1)]
\item \textbf{有限析因条件}: \( \forall a \in R^* \setminus U \),可分解为有限个不可约元素的乘积,即有不可约元素 \( p_i(1 \leqslant i \leqslant r) \)及单位$u\in U$,使
\begin{align*}
a = p_1p_2\cdots p_r.
\end{align*}

\item 若 \( a \in R^* \setminus U \) 有两种不可约元素乘积的分解:
\begin{align*}
a = p_1p_2\cdots p_r = q_1q_2\cdots q_s,
\end{align*}
则有 \( r = s \) 且 \( \exists \pi \in S_n \),使 \( p_i \sim q_{\pi(i)}(1 \leqslant i \leqslant r) \). 

那么称 \( R \) 为\textbf{唯一析因环}(简记为 $\mathbf{UFD}$)或\textbf{唯一分解整环}或$\mathbf{Gauss}$\textbf{环}.称 \( |a| \triangleq r \) 为 \( a \) 的\textbf{长度}. 若 \( u \in U \),约定 \( |u| \triangleq 0 \).
\end{enumerate}
\end{definition}
\begin{remark}
所谓唯一析因环也就是使因式分解唯一性定理成立的交换整环,因而前面\refexa{example:抽象代数--例2.4.2}与\refexa{example:抽象代数--例2.4.3}中的环$\mathbf{Z}$与$\mathbf{P}[x]$都是UFD,而\refexa{example:不可约元素不是素元素的例子}中的环$\mathbf{Z}[\sqrt{-5}]$就不是。因为$9 = 3^2$与$9 = (2 + \sqrt{-5})(2 - \sqrt{-5})$是9的两种本质上不同的分解,即$\mathbf{Z}[\sqrt{-5}]$不满足唯一析因环定义中的条件(2)。

\end{remark}

\begin{theorem}\label{theorem:有限析因条件推出标准分解}
设 \( R \) 是唯一析因环(UFD),\( R^* = R \setminus \{0\} \),\( U \) 表示 \( R^* \) 中乘法可逆元素的集合,则
\begin{enumerate}[(1)]
\item\label{theorem:有限析因条件推出标准分解-1} 对$\forall a\in R^*\setminus U$,都存在$r\in \mathbf{N}$,单位$u\in U$以及互不相伴的不可约元素$p_1,p_2,\cdots,p_r$,使
\begin{align*}
a = up_1^{n_1}p_2^{n_2}\cdots p_r^{n_r}, \quad n_i \in \mathbf{N}.
\end{align*}
若 \( c \) 是 \( a \) 的一个非平凡因子,则存在$u_1\in U$以及$n_i'\leqslant n_i$且$n_i'\in \mathbf{N}(i=1,2,\cdots,r)$,使
\begin{align*}
c = u_1p_1^{n_1'}p_2^{n_2'}\cdots p_r^{n_r'}.
\end{align*}

\item\label{theorem:有限析因条件推出标准分解-2} 若$a,b\in R^*\setminus U$,则存在$r\in \mathbf{N}$,单位$u,v\in U$以及互不相伴的不可约元素$p_1,p_2,\cdots,p_r$,使
\begin{gather*}
a=up_{1}^{n_1}p_{2}^{n_2}\cdots p_{r}^{n_r},\quad u\in U,\,n_i\in \mathbf{N}\cup \{0\};
\\
b=vp_1^{m_1}p_2^{m_2}\cdots p_r^{m_r}, \quad v\in U,\,m_i \in \mathbf{N}\cup \{0\}.
\end{gather*}
若还有$d$是$a,b$的公因子,则存在$w\in U$以及$n_i'\leqslant \min \left\{ n_i,m_i \right\} $且$n_i'\in \mathbf{N}(i=1,2,\cdots,r)$,使
\begin{align*}
d = wp_1^{n_1'}p_2^{n_2'}\cdots p_r^{n_r'}.
\end{align*}
\end{enumerate}
\end{theorem}
\begin{proof}
\begin{enumerate}[(1)]
\item 由$a$满足有限析因条件知,存在不可约元素$q_1,q_2,\cdots,q_s$,使得
\begin{align*}
a=q_1q_2\cdots q_s.
\end{align*}
将$q_1,q_2,\cdots,q_s$按相伴关系分类,不妨设存在$r\in \mathbf{N}$和
\begin{align*}
0=i_0\leqslant i_1\leqslant \cdots\leqslant i_r=s,
\end{align*}
使$q_{i_1},q_{i_2},\cdots,q_{i_r}$互不相伴且
\begin{gather*}
q_{i_0+1}=q_1 \sim q_2 \sim \cdots \sim q_{i_1}; \\
q_{i_1+1} \sim q_{i_1+2}\sim \cdots \sim q_{i_2}; \\
\cdots \cdots \\
q_{i_{r-1}+1} \sim q_{i_{r-1}+2} \sim \cdots \sim q_{i_r} = q_s.
\end{gather*}
由\rrefthe{theorem:单位群整除的基本性质}{theorem:单位群整除的基本性质-5}知存在
\begin{align*}
&u_{11},u_{12},\cdots,u_{1,i_1-1},u_{21},u_{22},\cdots,u_{2,i_2-1},\cdots,u_{r1},u_{r2},\cdots,u_{r,i_r-1} \in U, \\
\end{align*}
使得
\begin{gather*}
q_1 = u_{11}q_{i_1}, \quad q_2 = u_{12}q_{i_1}, \cdots \cdots, q_{i_1-1} = u_{1,i_1-1}q_{i_1}; \\
q_{i_1+1} = u_{21}q_{i_2}, \quad q_{i_1+2} = u_{22}q_{i_2}, \cdots \cdots, q_{i_2-1} = u_{2,i_2-1}q_{i_2}; \\
\cdots \cdots \cdots \cdots \\
q_{i_{r-1}+1} = u_{r1}q_{i_r}, \quad q_{i_{r-1}+2} = u_{r2}q_{i_r}, \cdots \cdots, q_{i_r-1} = u_{r,i_r-1}q_{i_r}.
\end{gather*}
记$p_j=q_{i_j},n_j=i_j-i_{j-1}(j=1,2,\cdots,r)$,$u=\prod_{j=1}^r{\prod_{i=1}^{i_j-1}{u_{ji}}}\in U$,则$p_{1},p_{2},\cdots,p_{r}$互不相伴且
\begin{align*}
a &=q_1q_2\cdots q_s=q_{i_1}^{i_1-1}\prod_{i=1}^{i_1-1}{u_{1i}}\cdot q_{i_2}^{i_2-1}\prod_{i=1}^{i_2-1}{u_{2i}}\cdots q_{i_r}^{i_r-1}\prod_{i=1}^{i_r-1}{u_{ri}} \\
&= \prod_{j=1}^r{\prod_{i=1}^{i_j-1}{u_{ji}}}\cdot q_{i_1}^{i_1-1}q_{i_2}^{i_2-1}\cdots q_{i_r}^{i_r-1}=up_{1}^{n_1}p_{2}^{n_2}\cdots p_{r}^{n_r}.
\end{align*}
由$c$是$a$的非平凡因子知,存在$d\in R^*$,使$a=cd$.由$R$是唯一析因环(UFD)知$c,d$都满足有限析因条件,故存在不可约元素$c_1,c_2,\cdots,c_t$和$d_1,d_2,\cdots,d_m$使
\begin{gather*}
c=c_1c_2\cdots c_t,
\quad
d=d_1d_2\cdots d_m.
\end{gather*}
从而
\begin{align*}
q_1q_2\cdots q_s=a=cd=c_1c_2\cdots c_t\cdot d_1d_2\cdots d_m.
\end{align*}
由$R$是唯一析因环(UFD)知$a$的不可约分解在相伴意义下唯一,再记$f_i=\begin{cases}
c_i,&i=1,2,\cdots ,t\\
d_{i-t},&i=t+1,\cdots ,t+m\\
\end{cases}$,故$s=t+m$且存在$\pi \in S_{s}$,使$q_i\sim f_{\pi \left( i \right)}(i=1,2,\cdots,s)$,即$q_{\pi^{-1}(i)}\sim f_{i}(i=1,2,\cdots,s).$于是$c_i\sim q_{\pi^{-1}(i)}(i=1,2\cdots,t).$不妨设存在
\begin{align*}
0 = i_{0}' \leqslant i_{1}' \leqslant \cdots \leqslant i_{r}' = t,
\end{align*}
使
\begin{align*}
\pi^{-1}(i_{j-1}'+1), \cdots, \pi^{-1}(i_{j}') \in \{i_{j-1}+1, \cdots, i_j\}, \quad j = 1, 2, \cdots, r.
\end{align*}
记$n_{j}' = i_{j}' - i_{j-1}'$,则由$n_j = i_j - i_{j-1}$知$n_{j}' \leqslant n_j$. 又因为
\begin{align*}
q_k \sim q_{i_j} = p_j, \quad k = i_{j-1}+1, \cdots, i_j,\ \, j = 1, 2, \cdots, r.
\end{align*}
所以
\begin{align*}
q_{\pi^{-1}(i_{j-1}'+1)} \sim \cdots \sim q_{\pi^{-1}(i_{j}')} \sim p_j, \quad \, j = 1, 2, \cdots, r.
\end{align*}
因此
\begin{align*}
c_{i_{j-1}'+1} \sim \cdots \sim c_{i_{j}'} = c_{i_{j-1}'+n_{j}'} \sim p_j, \quad \, j = 1, 2, \cdots, r.
\end{align*}
由\rrefthe{theorem:单位群整除的基本性质}{theorem:单位群整除的基本性质-5}知存在
\begin{align*}
u_{j1}, u_{j2}, \cdots, u_{jn_{j}'}, \quad j = 1, 2, \cdots, r.
\end{align*}
使得
\begin{align*}
c_{i_{j-1}'+k} = u_{jk}p_j, \quad k = 1, 2, \cdots, n_{j}',\ \, j = 1, 2, \cdots, r.
\end{align*}
再记$u_1 = \prod\limits_{j=1}^r{\prod\limits_{k=1}^{n_{j}'}{u_{jk}}}$,于是
\begin{align*}
c &= c_1c_2\cdots c_t = \prod\limits_{j=1}^r{\prod\limits_{k=1}^{n_{j}'}{c_{i_{j-1}'+k}}} \\
&= \prod\limits_{j=1}^r{\left( \prod\limits_{k=1}^{n_{j}'}{u_{jk}p_j} \right)} = \prod\limits_{j=1}^r{p_{j}^{n_{j}'}\left( \prod\limits_{k=1}^{n_{j}'}{u_{jk}} \right)} \\
&= \prod\limits_{j=1}^r{p_{j}^{n_{j}'}\left( \prod\limits_{k=1}^{n_{j}'}{u_{jk}} \right)} = \left( \prod\limits_{j=1}^r{p_{j}^{n_{j}'}} \right) \left( \prod\limits_{j=1}^r{\prod\limits_{k=1}^{n_{j}'}{u_{jk}}} \right) \\
&= u_1p_{1}^{n_{1}'}p_{2}^{n_{2}'}\cdots p_{r}^{n_{r}'.}
\end{align*}

\item 由(1)知存在$t,s\in \mathbf{N}$,单位$u_1,v_1\in U$,互不相伴的不可约元素$p_1,p_2,\cdots,p_s$和互不相伴的不可约元素$q_1,q_2,\cdots,q_t$,使
\begin{align*}
a=u_1p_{1}^{n_1}p_{2}^{n_2}\cdots p_{s}^{n_s},\quad n_i\in \mathbf{N};
\end{align*}
\begin{align*}
b=v_1q_{1}^{m_1}q_{2}^{m_2}\cdots q_{t}^{m_t},\quad m_i\in \mathbf{N}.
\end{align*}
不妨设存在$k\leqslant \min\{s,t\}$,使
\begin{align*}
p_j\sim q_j,\quad j=1,2,\cdots,k.
\end{align*}
由\rrefthe{theorem:单位群整除的基本性质}{theorem:单位群整除的基本性质-5}知存在$w_j\in U(j=1,2,\cdots,k)$,使
\begin{align*}
q_j=w_jp_j,\quad j=1,2,\cdots,k.
\end{align*}
于是
\begin{align*}
b&=v_1(w_1p_1)^{m_1}(w_2p_2)^{m_2}\cdots(w_kp_k)^{m_k}\cdot q_{k+1}^{m_{k+1}}\cdots q_{t}^{m_t} \\
&=(v_1w_1w_2\cdots w_k)\left(p_{1}^{m_1}p_{2}^{m_2}\cdots p_{k}^{m_k}\cdot q_{k+1}^{m_{k+1}}\cdots q_{t}^{m_t}\right).
\end{align*}
再记$p_{s+j}=q_j(j=k+1,\cdots,t)$,$u=u_1$,$v=v_1w_1w_2\cdots w_k$,则
\begin{align*}
a&=up_{1}^{n_1}p_{2}^{n_2}\cdots p_{s}^{n_s}p_{s+1}^{0}\cdots p_{s+t}^{0}, \\
b&=vp_{1}^{m_1}p_{2}^{m_2}\cdots p_{k}^{m_k}p_{k+1}^{0}\cdots p_{s}^{0}p_{s+1}^{m_{k+1}}\cdots p_{s+t}^{m_t}.
\end{align*}
再取$r=s+t$,$n_j=m_l=0(j=s+1,\cdots,s+t;l=k+1,\cdots,s)$即得
\begin{align*}
a=up_{1}^{n_1}p_{2}^{n_2}\cdots p_{r}^{n_r},\quad b=vp_{1}^{m_1}p_{2}^{m_2}\cdots p_{r}^{m_r}.
\end{align*}
若$d\in U$,则取$n_{i}'=0(i=1,2,\cdots,r)$即可.

若$d\in R^*\setminus U$,则由$d$是$a,b$的公因子和(1)的结论可知,存在单位$u',u''\in U$,互不相伴的不可约元素$p_1,p_2,\cdots,p_r$以及$n_{i}'\leqslant n_i$,$n_{i}''\leqslant m_i(i=1,2\cdots,r)$使
\begin{align*}
d=u'p_{1}^{n_{1}'}p_{2}^{n_{2}'}\cdots p_{r}^{n_{r}'}=u''p_{1}^{n_{1}''}p_{2}^{n_{2}''}\cdots p_{r}^{n_{r}''}.
\end{align*}
若存在$j_1,j_2\cdots,j_k\in\{1,2\cdots,r\}$,使$n_{j_l}'\neq n_{j_l}''(l=1,2,\cdots,k)$.由\rrefpro{proposition:整环的一些性质}{proposition:整环的一些性质-2}知$R^*$对乘法满足消去律,故
\begin{align*}
u'(u'')^{-1}p_{j_1}^{n_{j_1}'-n_{j_1}''}p_{j_2}^{n_{j_2}'-n_{j_2}''}\cdots p_{j_k}^{n_{j_k}'-n_{j_k}''}=1.
\end{align*}
由此可知$p_{j_l}\in U(l=1,2,\cdots,k)$,这与$p_{j_l}$不可约矛盾!故$n_{i}'=n_{i}''(i=1,2,\cdots,r)$,从而$n_{i}'=n_{i}''\leqslant \min\{n_i,m_i\}(i=1,2\cdots,r)$,取$w=u'$,则
\begin{align*}
d=wp_{1}^{n_{1}'}p_{2}^{n_{2}'}\cdots p_{r}^{n_{r}'}.
\end{align*}

\end{enumerate}

\end{proof}

\begin{theorem}\label{theorem:UFD中元素的长度的基本性质}
设$R$是唯一析因环,,\( R^* = R \setminus \{0\} \),\( U \) 表示 \( R^* \) 中乘法可逆元素的集合,$a,b,c\in R^*$,则
\begin{enumerate}[(1)]
\item\label{theorem:UFD中元素的长度的基本性质-1} $|ab|=|a|+|b|;$

\item\label{theorem:UFD中元素的长度的基本性质-4} $a\mid b\Longrightarrow |a|\geqslant |b|;$

\item\label{theorem:UFD中元素的长度的基本性质-2} $a \in U \iff |a| = 0;$

\item\label{theorem:UFD中元素的长度的基本性质-3} $b \sim c \iff |b| = |c|,\ b|c.$
\end{enumerate}
\end{theorem}
\begin{proof}
\begin{enumerate}[(1)]
\item 

\item 根据定义显然成立.

\item 
\end{enumerate}

\end{proof}

\begin{definition}
设 \( R \) 是交换整环,\( R^* = R \setminus \{0\} \),\( R^* \) 中的一个序列 \( a_1, a_2, \cdots, a_n, a_{n+1}, \cdots \) 满足
\begin{align*}
a_{n+1} \mid a_n, \quad n = 1, 2, \cdots,
\end{align*}
则称为 \( R \) 的一个\textbf{因子链}.

若对 \( R^* \) 中任一因子链,存在自然数 \( m \),使
\begin{align*}
a_m \sim a_n, \quad \forall n \geqslant m,
\end{align*}
则称 \( R \) 满足\textbf{因子链条件}.
\end{definition}

\begin{lemma}\label{lemma:抽象代数--引理2.2.3}
设 \( R \) 是交换整环,\( R^* = R \setminus \{0\} \),\( U \) 表示 \( R^* \) 中乘法可逆元素的集合,若\( R \) 满足因子链条件,则必满足有限析因条件.
\end{lemma}
\begin{proof}
设 \( a \in R^* \setminus U \). 先证 \( a \) 有不可约因子. 不妨设 \( a \) 是可约的,则 \( a \) 有非平凡的真因子 \( a_1 \),即有 \( a = a_1b_1 \). 这时 \( b_1 \) 也是 \( a \) 的非平凡真因子,否则,$b_1\in U,$由\rrefthe{theorem:单位群整除的基本性质}{theorem:单位群整除的基本性质-5}知$a\sim a_1$,这与$a_1$为$a$真因子矛盾! 若有 \( a_1, b_1 \) 都可约,则 \( a_1 = a_2b_2 \),其中,\( a_2, b_2 \) 为 \( a_1 \) 的真因子. 如此继续,可得因子链
\begin{align*}
a, a_1, a_2, \cdots, a_n, a_{n+1}, \cdots
\end{align*}
且 \( a_{n+1} \mid a_n,a_n\mid a\). 
这个因子链是在假设$a_1,a_2,\cdots,a_n,\cdots$都可约且对$\forall n\in \mathbf{N}$有$a_{n+1}$是$a_n$的真因子的条件下得到的.
而由因子链条件有 \( m \),使得 \( a_m \sim a_{m+1} \),这与$a_{m+1}$是$a_m$的真因子矛盾!因而 \( a_m \) 是不可约的,即 \( a_m \) 是 \( a \) 的不可约因子.

再证 \( a \) 可分解为有限多个不可约因子的乘积. 设 \( p_1 \) 是 \( a \) 的一个不可约因子,于是 \( a = p_1a^{(1)} \). 若 \( a^{(1)} \in U \),则由\refpro{proposition:不可约元素乘单位仍不可约}知$a$不可约.此时$a$满足有限析因条件.

若 \( a^{(1)} \in R^* \setminus U \),则 \( a^{(1)} \) 有不可约因子 \( p_2 \),使$a^{(1)}=p_2a^{(2)}$,即 \( a = p_1p_2a^{(2)} \). 继续此过程,即得因子链
\begin{align*}
a, a^{(1)}, a^{(2)}, \cdots, a^{(n)}, a^{(n+1)}, \cdots
\end{align*}
且\( a^{(n+1)} \mid a^{(n)},a^{(n)}\mid a\),$p_{n+1}$都是$a^{(n)}$的不可约因子,$a^{(n)}=p_{n+1}a^{(n+1)}$.这个因子链是在假设$a^{(n)}\in R^*\setminus U(\forall n\in \mathbf{N})$的条件下得到的.
而由因子链条件有 \( s \),使 \( a^{(s-1)} \sim a^{(s)} \).于是存在$b\in R^*$,使$a^{(s)}=ba^{(s-1)},$从而
$a^{\left( s-1 \right)}=p_sa^{\left( s \right)}=p_sba^{\left( s-1 \right)}$.
由\rrefpro{proposition:整环的一些性质}{proposition:整环的一些性质-2}知$R^*$对乘法满足消去律,故$p_sb=1$,即$p_s\in U,$这与$p_s$不可约矛盾!故存在$m$,使得$a^{(m)}\in U.$于是记$q_m = p_ma^{(m)}$,则由\refpro{proposition:不可约元素乘单位仍不可约}知$q_m$不可约.故此时
\begin{align*}
a = p_1p_2\cdots p_m a^{(m)}=p_1p_2\cdots q_m.
\end{align*}
满足有限析因条件.这就证明了 \( R \) 满足有限析因条件.

\end{proof}

\begin{theorem}\label{theorem:UFD的充要条件}
设 \( R \) 是交换整环,\( R^* = R \setminus \{0\} \),\( U \) 表示 \( R^* \) 中乘法可逆元素的集合,则下列条件等价:
\begin{enumerate}[(1)]
\item\label{theorem:UFD的充要条件-1} \( R \) 是唯一析因环 (UFD);
\item\label{theorem:UFD的充要条件-2} \( R \) 满足因子链条件与素性条件;
\item\label{theorem:UFD的充要条件-3} \( R \) 满足因子链条件与最大公因子条件.
\end{enumerate}
\end{theorem}
\begin{proof}
\ref{theorem:UFD的充要条件-1} \( \Rightarrow \) \ref{theorem:UFD的充要条件-3}. 设 \( R \) 为唯一析因环. 先证 \( R \) 满足因子链条件. \( \forall a \in R^* \setminus U \),\( a \) 有不可约元素乘积分解 \( a = up_1p_2\cdots p_r \).
现设 \( a_1, a_2, \cdots, a_n, a_{n+1}, \cdots \) 是 \( R^* \) 的一个因子链. 于是由\rrefthe{theorem:UFD中元素的长度的基本性质}{theorem:UFD中元素的长度的基本性质-4}知必有 \( |a_i| \geqslant 0 \) 且
\begin{align*}
|a_1| \geqslant |a_2| \geqslant \cdots \geqslant |a_n| \geqslant |a_{n+1}| \geqslant \cdots,
\end{align*}
由于$|a_1|$是一个有限数,因而有 \( m \),使得当 \( n \geqslant m \) 时,\( |a_n| = |a_m| \),由\rrefthe{theorem:UFD中元素的长度的基本性质}{theorem:UFD中元素的长度的基本性质-3}知 \( a_n \sim a_m \),故 \( R \) 满足因子链条件.

现证 \( R \) 满足最大公因子条件. 设 \( a, b \in R^* \),若 \( a, b \) 中有一个是单位,则由\rreflem{lemma:最大公因子条件的基本结论}{lemma:最大公因子条件的基本结论-7}知$(a,b)=1$,故假定 \( a, b \in R^* \setminus U \). 这时由\rrefthe{theorem:有限析因条件推出标准分解}{theorem:有限析因条件推出标准分解-1},不妨设
\begin{gather*}
a = up_1^{n_1}p_2^{n_2}\cdots p_r^{n_r}, \quad
b = vp_1^{m_1}p_2^{m_2}\cdots p_r^{m_r}, 
\end{gather*}
其中,\( u, v \in U \),\( p_1, p_2, \cdots, p_r \) 是互不相伴的不可约元素,\( n_i \geqslant 0 \),\( m_j \geqslant 0 \),\( 1 \leqslant i, j \leqslant r \). 令 \( k_i = \min\{n_i, m_i\} \),记
\begin{align}
d = p_1^{k_1}p_2^{k_2}\cdots p_r^{k_r} \label{eq:2.4.6}
\end{align}
显然$d$是$a,b$的公因子.又设$d_1$也是\( a, b \) 的公因子,则由\rrefthe{theorem:有限析因条件推出标准分解}{theorem:有限析因条件推出标准分解-2}知存在$u_1\in U$以及$n_i'\leqslant k_i $且$n_i'\in \mathbf{N}(i=1,2,\cdots,r)$,使
\begin{align*}
d_1 = u_1p_1^{n_1'}p_2^{n_2'}\cdots p_r^{n_r'}.
\end{align*}
故$d_1\mid d$.因此$d$是$a,b$的最大公因子.

\ref{theorem:UFD的充要条件-3} \( \Rightarrow \) \ref{theorem:UFD的充要条件-2}. 为此只需证明素性条件成立. 设 \( p \) 是一个不可约元素且 \( p \nmid a \),\( p \nmid b \),由\rrefthe{lemma:最大公因子条件的基本结论}{lemma:最大公因子条件的基本结论-8}有 \( (p, a) \sim 1 \),\( (p, b) \sim 1 \). 由\rreflem{lemma:最大公因子条件的基本结论}{lemma:最大公因子条件的基本结论-5}知 \( (p, ab) \sim 1 \),因而再由\rrefthe{lemma:最大公因子条件的基本结论}{lemma:最大公因子条件的基本结论-8}知\( p \nmid ab \). 换言之,若 \( p|ab \),则有 \( p|a \) 或 \( p|b \),故 \( p \) 为素元素.

\ref{theorem:UFD的充要条件-2} \( \Rightarrow \) \ref{theorem:UFD的充要条件-1}. 由\reflem{lemma:抽象代数--引理2.2.3}知 \( R \) 满足有限析因环条件,故只需证因式分解的唯一性. 不妨设 \( a \in R^* \setminus U \) 且 \( a \) 有两个不可约元素乘积的分解
\begin{align}\label{eq:::90j3tj34g4ag}
a = p_1p_2\cdots p_s = q_1q_2\cdots q_t.
\end{align}
现对 \( s \) 用数学归纳法证. 若 \( s = 1 \),则 \( a \) 为不可约元素,由素性条件知$a$为素元素. 根据素元素的定义,可不妨设 \( a|q_1 \),则 \( a \sim q_1 \).从而由\rrefthe{theorem:单位群整除的基本性质}{theorem:单位群整除的基本性质-5}知存在$u\in U$,使$a=q_1u=q_1q_2\cdots q_t$,故 \( t = 1 \). 设 \( s - 1 \) 时已成立,现证 \( s \) 时成立. 因 \( p_s|a \),故 \( p_s|q_1q_2\cdots q_t \),由素性条件知$p_s$也是素元素,于是不妨设\( p_s|q_{t} \), 于是 \( q_{t} = u_sp_s(u_s \in U) \).由\rrefpro{proposition:整环的一些性质}{proposition:整环的一些性质-2}知$R^*$对乘法满足消去律,因而结合\eqref{eq:::90j3tj34g4ag}式有
\begin{align*}
p_1p_2\cdots p_{s-1}p_s=q_1q_2\cdots q_{t-1}q_t=u_sq_1q_2\cdots q_{t-1}p_s\Longrightarrow p_1p_2\cdots p_{s-1}=u_s\prod_{i=1}^{t-1}{q_i}.
\end{align*}
记$q_1'=u_sq_1,q_i'=q_i(2\leqslant i\leqslant t-1)$,由\refpro{proposition:不可约元素乘单位仍不可约}知$q_1'$也不可约,并且由\rrefthe{theorem:单位群整除的基本性质}{theorem:单位群整除的基本性质-5}知$q_i'\sim q_i(1\leqslant i\leqslant t-1)$,则
\begin{align*}
p_1p_2\cdots p_{s-1}=q_{1}^{\prime}q_2'\cdots q_{t-1}'.
\end{align*}
由归纳假设可知,\( s - 1 = t - 1 \) 且存在$\pi \in S_{t-1}$,使  \(p_i\sim q_{\pi \left( i \right)}^{\prime}\sim q_{\pi (i)}(1\leqslant i\leqslant t-1)\).
由\rrefpro{proposition:整环的一些性质}{proposition:整环的一些性质-2}知$R^*$对乘法满足消去律,再结合\eqref{eq:::90j3tj34g4ag}式及\rrefthe{theorem:单位群整除的基本性质}{theorem:单位群整除的基本性质-5}知
\begin{align*}
u_sp_s\prod_{i=1}^{t-1}{q_i}=p_1p_2\cdots p_s=q_1q_2\cdots q_t\Longrightarrow u_sp_s=q_t\Longrightarrow p_s\sim q_t.
\end{align*}
故 \( s = t \) 且有 \( \pi' \in S_t \),使得 \( p_i \sim q_{\pi'
(i)}(1 \leqslant i \leqslant t) \),即 \( R \) 是一个 UFD.

\end{proof}

因本节讨论并未用到 \( R \) 中的加法,因而可以认为 \( R^* \) 是满足消去律的幺半群. 因此,可定义\textbf{唯一析因幺半群} (或 $\mathbf{Gauss}$\textbf{幺半群}). \reflem{lemma:最大公因子条件的基本结论}, \reflem{lemma:抽象代数--引理2.2.3}与\refthe{theorem:UFD的充要条件}对 Gauss 幺半群也成立.

如果在 \( R \) 中约定 \( a|0, \forall a \in R \). 此时可得若 \( a|b, a|c \),则 \( a|(b + c) \),以及其他一些早已熟知的性质.














\end{document}
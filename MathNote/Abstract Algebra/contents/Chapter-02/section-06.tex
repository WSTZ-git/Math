\documentclass[../../main.tex]{subfiles}
\graphicspath{{\subfix{../../image/}}} % 指定图片目录,后续可以直接使用图片文件名。

% 例如:
% \begin{figure}[H]
% \centering
% \includegraphics[scale=0.4]{图.png}
% \caption{}
% \label{figure:图}
% \end{figure}
% 注意:上述\label{}一定要放在\caption{}之后,否则引用图片序号会只会显示??.

\begin{document}

\section{域上一元多项式}

\begin{definition}
设$R[x]$是交换整环$R$上的一元多项式环,若$f(x)\in R[x]$且$f(x) \neq 0$,有
\begin{align*}
f(x) = a_0 + a_1x + \cdots + a_nx^n,\quad a_n \neq 0.
\end{align*}
则$a_ix^i$称为$f(x)$的$\boldsymbol{i}$\textbf{次项},$a_i$称为$i$次项的\textbf{系数},$a_0$称为\textbf{常数项},$a_nx^n$称为\textbf{首项},$n$称为$f(x)$的\textbf{次数},记为$\deg f(x) = n$. 如果$f(x) \neq 0$且$f(x)$的首项系数为$1$,则称$f(x)$为\textbf{首一多项式}.今后以$(f(x),g(x))$表示$f(x),g(x)$的最大公因式中的首一多项式.

若$f(x) = a_0 \neq 0$,则记$\deg f(x) = 0$,一般对零元素$0$是不规定次数的,但规定$0$的次数为$-\infty$,即$\deg 0 = -\infty$且规定
\begin{align*}
-\infty + (-\infty) = -\infty,\quad -\infty + n = -\infty,\quad -\infty < n,\quad 2^{-\infty} = 0.
\end{align*}
\end{definition}

\begin{theorem}\label{theorem:域上一元多项式的基本性质}
设$R[x]$是交换整环$R$上的一元多项式环,$R^*=R\setminus \{0\}$,$f(x),g(x)\in R[x]$,则
\begin{enumerate}[(1)]
\item $\deg f(x) = 0\iff f(x) \in R^*$.

\item $\deg(f(x) + g(x)) \leqslant \max\{\deg f(x), \deg g(x)\}$.

\item $\deg(f(x)g(x)) = \deg f(x) + \deg g(x)$.

\item\label{theorem:域上一元多项式的基本性质-4} 令$\delta(f(x)) = 2^{\deg f(x)}$,则有
\begin{align*}
\deg f(x) < \deg g(x) \iff \delta(f(x)) < \delta(g(x)),
\end{align*}
\begin{align*}
\delta(f(x)g(x)) = \delta(f(x))\delta(g(x)).
\end{align*}

\item  首一多项式的乘积仍为首一多项式.

\item $R[x]$也是交换整环且$R[x]$的单位就是$R$的单位.

\item $R$上$n$元多项式环$R[x_1, x_2, \cdots, x_n]$也是交换整环且其单位就是$R$的单位.
\end{enumerate}
\end{theorem}
\begin{proof}
\begin{enumerate}[(1)]
\item 

\item 

\item 

\item 

\item 

\item 

\item 
\end{enumerate}

\end{proof}

\begin{theorem}\label{theorem:抽象代数--定理2.6.1}
设$F$是一个域,$F[x]$为$F$上一元多项式环,则
\begin{enumerate}[(1)]
\item\label{theorem:抽象代数--定理2.6.1-1} $\forall f(x), g(x) \in F[x],\ g(x) \neq 0$,存在唯一的一对多项式$q(x), r(x)$,使得
\begin{align*}
f(x) = q(x)g(x) + r(x),\quad \deg r(x) < \deg g(x);
\end{align*}
分别称$q(x), r(x)$为$f(x)$除以$g(x)$的\textbf{商}、\textbf{余式}.

\item\label{theorem:抽象代数--定理2.6.1-2} $F[x]$是Euclid环.
\end{enumerate}
\end{theorem}
\begin{remark}
由\rrefthe{theorem:抽象代数--定理2.6.1}{theorem:抽象代数--定理2.6.1-2}知$F[x]$是Euclid环,又由\refthe{theorem:Euclid环必是主理想整环}知$F[x]$是主理想整环,进而也是唯一析因环.
\end{remark}
\begin{proof}
\begin{enumerate}[(1)]
\item 首先证明$q(x), r(x)$的存在性. 对$\deg f(x)$作归纳. 设$\deg g(x) = m$. 由假设知$m \geqslant 0$,当$\deg f(x) < m$时可取$q(x) = 0$,即$r(x) = f(x)$. 现设$\deg f(x) < n$时,$q(x)$与$r(x)$已存在. 设$\deg f(x) = n$. 不妨设$n \geqslant m$. 又设
\begin{align*}
f(x) = a_nx^n + a_{n-1}x^{n-1} + \cdots + a_0,
\end{align*}
\begin{align*}
g(x) = b_mx^m + b_{m-1}x^{m-1} + \cdots + b_0.
\end{align*}
由$b_m \neq 0$,取$q_0(x) = a_nb_m^{-1}x^{n-m}$. 令
\begin{align*}
f_1(x) = f(x) - q_0(x)g(x) = (a_{n-1} - a_nb_m^{-1}b_{m-1})x^{n-1} + \cdots,
\end{align*}
故$\deg f_1(x) \leqslant n - 1 < n$. 由归纳假设有$q_1(x), r_1(x)$,使得
\begin{align*}
f_1(x) = q_1(x)g(x) + r_1(x),\quad \deg r_1(x) < \deg g(x),
\end{align*}
因而有
\begin{align*}
f(x) = q_0(x)g(x) + q_1(x)g(x) + r_1(x) = (q_0(x) + q_1(x))g(x) + r_1(x).
\end{align*}
取$q(x) = q_0(x) + q_1(x), r(x) = r_1(x)$,它们满足定理条件.

下面证明$q(x)$与$r(x)$的唯一性. 设有$q_2(x), r_2(x)$也满足
\begin{align*}
f(x) = q_2(x)g(x) + r_2(x),\quad \deg r_2(x) < \deg g(x),
\end{align*}
于是有$(q(x) - q_2(x))g(x) = r_2(x) - r(x)$. 若$q(x) - q_2(x) \neq 0$,则
\begin{align*}
\deg(r_2(x) - r(x)) \geqslant \deg g(x) > \max\{\deg r_2(x),\ \deg r(x)\}.
\end{align*}
另一方面有
\begin{align*}
\deg(r_2(x) - r(x)) \leqslant \max\{\deg r_2(x),\ \deg r(x)\},
\end{align*}
这就导出矛盾. 故$q(x) = q_2(x), r_2(x) = r(x)$. $q(x)$与$r(x)$的唯一性得证.

\item 令$\delta(f(x)) = 2^{\deg f(x)}$,注意到$\deg r(x) < \deg g(x)$,由\rrefthe{theorem:域上一元多项式的基本性质}{theorem:域上一元多项式的基本性质-4}得
\begin{align*}
\delta(r(x)) < \delta(g(x)),
\end{align*}
故$F[x]$为Euclid环.
\end{enumerate}
\end{proof}

\begin{definition}
设$F$是一个域,$F[x]$为$F$上一元多项式环,若$f_1(x)$与$f_2(x)$除以$g(x)$的余式相同,则称$f_1(x)$与$f_2(x)$模$g(x)$\textbf{同余}. 记为$f_1(x) \equiv f_2(x)\ (\mathrm{mod}\ g(x))$.
\end{definition}

\begin{corollary}\label{corollary:抽象代数--推论2.6.1}
设$F[x]$为域$F$上的一元多项式环,$f_1(x), f_2(x), g(x) \in F[x]$且$g(x) \neq 0$,则
\begin{align*}
f_1(x) \equiv f_2(x)\ (\mathrm{mod}\ g(x))\iff g(x) \mid (f_1(x) - f_2(x)),
\end{align*}
而且$f_1(x) \equiv f_2(x)\ (\mathrm{mod}\ g(x))$无论对$F[x]$的加法或乘法都是同余关系.
\end{corollary}
\begin{proof}
$f_1(x) \equiv f_2(x) \pmod{g(x)}$当且仅当存在$q_1(x), q_2(x), r(x) \in F[x]$,使
\begin{align*}
f_1(x) = q_1(x)g(x) + r(x),\quad f_2(x) = q_2(x)g(x) + r(x).
\end{align*}
这也当且仅当
\begin{align*}
f_1(x) - f_2(x) = (q_1(x) - q_2(x))g(x) \iff g(x) \mid (f_1(x) - f_2(x)).
\end{align*}

设$f_1(x) \equiv f_2(x) \pmod{g(x)}$,$f_3(x) \equiv f_4(x) \pmod{g(x)}$,则存在$q_1(x), q_2(x), q_3(x), q_4(x), r_1(x), r_2(x) \in F[x]$,使
\begin{align*}
f_1(x) = q_1(x)g(x) + r_1(x),\quad f_2(x) = q_2(x)g(x) + r_1(x),\\
f_3(x) = q_3(x)g(x) + r_2(x),\quad f_4(x) = q_4(x)g(x) + r_2(x).
\end{align*}
于是
\begin{gather*}
f_1(x) + f_3(x) = (q_1(x) + q_3(x))g(x) + r_1(x) + r_2(x),\\
f_2(x) + f_4(x) = (q_2(x) + q_4(x))g(x) + r_1(x) + r_2(x),\\
f_1(x)f_3(x) = (q_1(x)q_3(x)g(x) + q_1(x)r_2(x) + q_3(x)r_1(x))g(x) + r_1(x)r_2(x),\\
f_2(x)f_4(x) = (q_2(x)q_4(x)g(x) + q_2(x)r_2(x) + q_4(x)r_1(x))g(x) + r_1(x)r_2(x).
\end{gather*}
故
\begin{align*}
f_1(x) + f_3(x) \equiv f_2(x) + f_4(x) \pmod{g(x)},\\
f_1(x)f_3(x) \equiv f_2(x)f_4(x) \pmod{g(x)}.
\end{align*}
因此$f_1(x) \equiv f_2(x) \pmod{g(x)}$对$F[x]$的加法和乘法都是同余关系.

\end{proof}

\begin{definition}
设$F$是一个域,$F[x]$为$F$上一元多项式环,若$c \in F$且使$f(c) = 0$,则称$c$是$f(x)$的一个\textbf{根}.
\end{definition}

\begin{corollary}\label{corollary:抽象代数--推论2.6.2}
设$F[x]$为域$F$上的一元多项式环且$f(x) \in F[x],c \in F$,则
\begin{align}\label{eq:::--ioewf32.6.3}
f(x) \equiv f(c)\ (\mathrm{mod}\ (x - c))
\end{align}
且$(x - c) \mid f(x)\iff f(c) = 0\iff c\text{是}f(x)\text{的根}$.
\end{corollary}
\begin{proof}
事实上,由\refthe{theorem:抽象代数--定理2.2.2},$F$的恒等映射$\mathrm{id}_F$可开拓为$F[x]$到$F$的同态$\eta$,使得
\begin{align*}
\eta _F=\mathrm{id}_F,\quad \eta \left( x \right) =c.
\end{align*}
从而
\begin{align*}
\eta(f(x)) = f(c),\quad \forall f(x) \in F[x].
\end{align*}
现因$\deg(x - c) = 1$,故$\exists q(x) \in F[x],\ r \in F$,使得
\begin{align*}
f(x) = (x - c)q(x) + r.
\end{align*}
两边作用以$\eta$,则有
\begin{align*}
f(c) = (c - c)q(c) + r = r,
\end{align*}
因而式\eqref{eq:::--ioewf32.6.3}成立.
特别地,$(x - c) \mid f(x)\iff f(x) \equiv 0\ (\mathrm{mod}\ (x - c))\iff f(c) = 0$.

\end{proof}

\begin{corollary}\label{corollary:抽象代数--推论2.6.3}
设$F$是一个域,$F[x]$为$F$上一元多项式环,$f(x) \in F[x]$,$c_i \in F(i=1,2,\cdots,k)$,若$c_1, c_2, \cdots, c_k$是$f(x)$的互不相同的根,则有$\prod\limits_{i=1}^k (x - c_i) \mid f(x)$,从而$k \leqslant \deg f(x)$.
\end{corollary}
\begin{proof}
显然$x - c_i$是$F[x]$中不可约元素,由\rrefthe{theorem:抽象代数--定理2.6.1}{theorem:抽象代数--定理2.6.1-2}知$F[x]$是Euclid环,又由\refthe{theorem:Euclid环必是主理想整环}知$F[x]$是唯一析因环.再由\refthe{theorem:UFD的充要条件}知$F[x]$满足素性条件.因而$x - c_i$是素元素. 又由$c_i \neq c_j\ (i \neq j, 1 \leqslant i, j \leqslant k)$. 由
\begin{align*}
\frac{1}{c_i - c_j}(x - c_j) - \frac{1}{c_i - c_j}(x - c_i) = 1
\end{align*}
及\refthe{corollary:主理想整环中互素的充要条件}知$(x - c_i, x - c_j) = 1$.又由\refcor{corollary:抽象代数--推论2.6}知$(x-c_i)\mid f(x)$,故由\refthe{proposition:UFD中互素元素整除同一元素则乘积也整除这个元}知$\prod\limits_{i=1}^k (x - c_i) \mid f(x)$,从而$k \leqslant \deg f(x)$.

\end{proof}

\begin{proposition}\label{proposition:一元多项式的根不超过其次数}
设$S$为交换整环,$R$为$S$的子环且$1\in R$,则$f(x)\in R[x]$在$S$中不同根的个数不超过$\deg f(x)$。
\end{proposition}
\begin{remark}
设$R$为交换幺环,$S$为$R$的扩环,$f(x)\in R[x]$,$\deg f(x) > 0$,那么$f(x)$在$S$中不同根的数目是否超过$\deg f(x)$?如果$S$为交换整环,则回答是肯定的.若$S$非交换或有零因子则不然.
\end{remark}
\begin{proof}
事实上,设$F$为$S$的分式域。于是$R[x]\subseteq S[x]\subseteq F[x]$,即$f(x)\in F[x]$。由\refcor{corollary:抽象代数--推论2.6.3}知结论成立.

\end{proof}

\begin{example}
设$\mathbf{H}$为四元数体(见\refthe{theorem:抽象代数--例题1.4.8}),由\refpro{proposition:四元数体的基本结论----1}知
\begin{align*}
\mathbf{H}=\{a+bi+cj+dk\mid a,b,c,d\in\mathbf{R}\},
\end{align*}
因而$\{a+0\cdot i+0\cdot j+0\cdot k\mid a\in\mathbf{R}\}\cong\mathbf{R}$是$\mathbf{H}$的一个子环。由\refpro{proposition:四元数体的基本结论----1}知$i^=j^2=k^2=-1$,故$i,j,k$都是$\mathbf{R}$上的多项式$x^2+1$的根,从而$bi+cj+dk$都是$x^2+1$的根,因此$x^2+1$在$\mathbf{H}$中有无穷多个根.
\end{example}

\begin{example}
设$R=S=\mathbf{Z}_8$。不难看出$x^2-1\in R[x]$有4个不同的根$\overline{1},\overline{3},\overline{5},\overline{7}$,其中,$\overline{n}$表示$n+8\mathbf{Z}$。
\end{example}

\begin{proposition}\label{proposition:整除与不整除与素因子分解的相关定理}
设$a,b\in \mathbf{N}$,若$a\nmid b$,则存在素数$p$,使得
\begin{align*}
a=p^rl,\quad b=p^sk,\quad (p,l)=(p,k)=(p,lk)=1,\quad r>s.
\end{align*}
\end{proposition}
\begin{proof}
由算术基本定理知,存在$n\in \mathbf{N}$以及互不相同的素数$p_1,p_2,\cdots ,p_n$,使得
\begin{align*} a=\prod\limits_{i=1}^n{p_{i}^{k_i}},\quad b=\prod\limits_{i=1}^n{p_{i}^{k_{i}^{\prime}}}, \end{align*}
其中$k_i,k_{i}^{\prime}\in \mathbf{N}.$
因为$a\mid b$当且仅当$k_i\leqslant k_{i}^{\prime},i=1,2,\cdots ,n,$所以由$a\nmid b$可得,存在$i_0\in \left\{ 1,2,\cdots ,n \right\} ,$使得$k_{i_0}>k_{i_0}^{\prime}.$
于是记$p=p_{i_0},r=k_{i_0},s=k_{i_0}^{\prime},l=\prod\limits_{i\ne i_0}{p_{i}^{k_i}},k=\prod\limits_{i\ne i_0}{p_{i}^{k_{i}^{\prime}}},$则$r>s$且
\begin{align*} 
a=p_{i_0}^{k_{i_0}}\prod\limits_{i\ne i_0}{p_{i}^{k_i}}=p^rl,\quad b=p_{i_0}^{k_{i_0}^{\prime}}\prod\limits_{i\ne i_0}{p_{i}^{k_{i}^{\prime}}}=p^sk. 
\end{align*}
由$p_1,p_2,\cdots ,p_n$是互不相同的素数可知$(p,l)=(p,k)=1,$故$(p,lk)=1.$

\end{proof}

\begin{theorem}\label{theorem:抽象代数--定理2.6.2}
设$F$是一个域,$G$是$F^*=F\setminus\{0\}$的一个有限的乘法子群,则$G$为循环群。
\end{theorem}
\begin{proof}
设$|G|=n$,$g$是$G$中最大阶的元素且其阶为$m$,因而$\langle g\rangle=\{1,g,g^2,\cdots,g^{m-1}\}\subseteq G$,由\hyperref[theorem:抽象代数-Lagrange定理-定理 1.3.3]{Lagrange定理}知$m|n$.任取$h\in G$,设$h$的阶为$m_1$,如果$m_1\nmid m$,则由\refpro{proposition:整除与不整除与素因子分解的相关定理}知有素数$p$,使得
\begin{align*}
m_1=p^rl,\quad m=p^sk,\quad (p,l)=(p,k)=(p,lk)=1,\quad r>s.
\end{align*}
由$h$的阶为$m_1$,故$h^l$的阶为$p^r$,$g^{p^s}$的阶为$k$,由于$G$为Abel群,故$h^lg^{p^s}=g^{p^s}h^l$,$(p^r,k)=1$,由\refcor{corollary:抽象代数--推论1.8.3}知$h^lg^{p^s}$的阶为$p^rk$,但$p^rk>p^sk=m$。这与$m$的选取矛盾,故$m_1|m$。

由此得$\forall h\in G$,$h$都是$x^m-1$的根,即$G\subseteq \left\{ x\mid x^m-1=0 \right\} .$
由\refpro{proposition:一元多项式的根不超过其次数}知$x^m-1$至多有$m$个根,
又$\langle g\rangle$中$m$个元素都是$x^m-1$的根,故$\langle g\rangle\subseteq \left\{ x\mid x^m-1=0 \right\}$且$\left| \left\{ x\mid x^m-1=0 \right\} \right|=m=|\langle g\rangle|$,因此$\langle g\rangle= \left\{ x\mid x^m-1=0 \right\}.$于是$G\subseteq \langle g\rangle.$
故$G=\langle g\rangle$,这就证明了$G$是循环群.

\end{proof}

\begin{corollary}\label{corollary:抽象代数--推论2.6.4}
有限域$F$的非零元素集$F^*$对乘法为循环群。
\end{corollary}
\begin{remark}
这个推论对有限域理论是很重要的.
\end{remark}
\begin{proof}
由\refthe{theorem:抽象代数--定理2.6.2}立得.

\end{proof}

\begin{theorem}\label{theorem:抽象代数--定理2.6.3}
设$F$为域,$f(x),g(x)\in F[x]^*=F[x]\setminus \{0\}$,则$f(x),g(x)$非互素的充分必要条件为$\exists f_0(x),g_0(x)\in F[x]^*$,使得
\begin{align*}
g_0(x)f(x)=f_0(x)g(x),
\end{align*}
其中,
\begin{align*}
0\leqslant\deg f_0(x)<\deg f(x),\quad 0\leqslant\deg g_0(x)<\deg g(x).
\end{align*}
\end{theorem}
\begin{proof}
显然这样的最大公因式是唯一的。设$d(x)=(f(x),g(x))$,于是有$f(x)=d(x)f_0(x)$,$g(x)=d(x)g_0(x)$。由$f(x),g(x)$非互素,故$\deg d(x)>0$,因而
\begin{align*}
0\leqslant\deg f_0(x)<\deg f(x),\quad 0\leqslant\deg g_0(x)<\deg g(x)
\end{align*}
且有
\begin{align*}
g_0(x)f(x)=d(x)f_0(x)g_0(x)=f_0(x)g(x).
\end{align*}
由此知必要性成立。

反之,假设$(f(x),g(x))=1$,于是$\exists u(x),v(x)\in F[x]$,使$u(x)f(x)+v(x)g(x)=1$,因而有
\begin{align*}
f_0(x)&=f_0(x)\cdot1=f_0(x)u(x)f(x)+v(x)f_0(x)g(x) \\
&=f(x)(f_0(x)u(x)+v(x)g_0(x)),
\end{align*}
即得$\deg f_0(x)\geqslant\deg f(x)$。这与条件矛盾,故$(f(x),g(x))\neq1$。

\end{proof}

\begin{corollary}\label{corollary:抽象代数----定理2.6.3-推论}
设$F$为域,$f(x),g(x)\in F[x]^*=F[x]\setminus \{0\}$,记
\begin{align*}
f(x)=\sum_{k=0}^n a_kx^{n-k},\quad a_0\neq0,n\in\mathbf{N}, \\
g(x)=\sum_{k=0}^m b_kx^{m-k},\quad b_0\neq0,m\in\mathbf{N}.
\end{align*}
再记
\begin{align*}
f_0(x)=\sum_{k=1}^n x_kx^{n-k},\quad g_0(x)=\sum_{k=1}^m x_{n+k}x^{m-k},
\end{align*}
则$(f(x),g(x))\neq1$当且仅当存在$(x_1,x_2,\cdots,x_{m+n})\neq0$,使$f(x)g_0(x)=g(x)f_0(x)$,
\end{corollary}
\begin{proof}
由\refthe{theorem:抽象代数--定理2.6.3}立得.

\end{proof}

\begin{definition}[结式/Sylvester行列式]
设$F$是一个域,$f(x),g(x)\in F[x]^*=F[x]\setminus \{0\}$,则称
\begin{align*}
R(f,g)=\left| \begin{matrix}
a_0&		a_1&		\cdots&		\cdots&		a_n&		&		&		\\
&		a_0&		a_1&		\cdots&		\cdots&		a_n&		&		\\
&		&		\ddots&		\ddots&		&		&		\ddots&		\\
&		&		&		a_0&		a_1&		\cdots&		\cdots&		a_n\\
b_0&		b_1&		\cdots&		\cdots&		b_m&		&		&		\\
&		b_0&		b_1&		\cdots&		\cdots&		b_m&		&		\\
&		&		\ddots&		\ddots&		&		&		\ddots&		\\
&		&		&		b_0&		b_1&		\cdots&		\cdots&		b_m\\
\end{matrix} \right|,
\end{align*}
为$f(x)$与$g(x)$的\textbf{结式}或$\mathbf{Sylvester}$\textbf{行列式}.显然有
\begin{align*}
\begin{cases}
\mathrm{ent}_{i\ i+j}(R(f,g)) = a_j, & 1 \leqslant i \leqslant m,\ 0 \leqslant j \leqslant n, \\
\mathrm{ent}_{n+i\ i+j}(R(f,g)) = b_j, & 1 \leqslant i \leqslant n,\ 0 \leqslant j \leqslant m, \\
\mathrm{ent}_{i\ j}(R(f,g)) = 0, & \text{其他}.
\end{cases}
\end{align*}
\end{definition}

\begin{theorem}\label{theorem:抽象代数--定理2.6.4}
设$F$是一个域,$f(x),g(x)\in F[x]^*=F[x]\setminus \{0\}$非互素的充分必要条件是$f(x)$与$g(x)$的结式$R(f,g)=0$。
\end{theorem}
\begin{proof}
记
\begin{align*}
f(x)=\sum_{k=0}^n a_kx^{n-k},\quad a_0\neq0,n\in\mathbf{N}, \\
g(x)=\sum_{k=0}^m b_kx^{m-k},\quad b_0\neq0,m\in\mathbf{N}.
\end{align*}
再记
\begin{align*}
f_0(x)=\sum_{k=1}^n x_kx^{n-k},\quad g_0(x)=\sum_{k=1}^m x_{n+k}x^{m-k},
\end{align*}
由\refcor{corollary:抽象代数----定理2.6.3-推论}知$(f(x),g(x))\neq1$当且仅当存在$(x_1,x_2,\cdots,x_{m+n})\neq0$,使$f(x)g_0(x)=g(x)f_0(x)$,所以有
\begin{align*}
\sum_{l=0}^{n+m-1}\left( \sum_{\substack{j+k=l\\0\leqslant k\leqslant m-1}} a_jx_{k+n+1} \right)x^{n+m-1-l} 
=\sum_{r=0}^{n+m-1}\left( \sum_{\substack{p+q=r\\0\leqslant q\leqslant n-1}} b_px_{q+1} \right)x^{n+m-1-r}.
\end{align*}
由对应项系数相等,即得
\begin{align*}
\sum_{\substack{j+k=l\\0\leqslant k\leqslant m-1}} a_jx_{k+n+1}=\sum_{\substack{p+q=r\\0\leqslant q\leqslant n-1}} b_px_{q+1},\quad l=r=0,1,\cdots,n+m-1.
\end{align*}
这样得到一个齐次线性方程组
\begin{align*}
\boldsymbol{A}^T\mathbf{X}=\mathbf{0},
\end{align*}
其中,$\mathbf{X}=(x_1,x_2,\cdots,x_{m+n})'$,
\[
\begin{cases}
\mathrm{ent}_{i,i+j}(\boldsymbol{A})=b_j, & 1\leqslant i\leqslant n,\ 0\leqslant j\leqslant m, \\
\mathrm{ent}_{n+i,i+j}(\boldsymbol{A})=-a_j, & 1\leqslant i\leqslant m,\ 0\leqslant j\leqslant n, \\
\mathrm{ent}_{i,j}(\boldsymbol{A})=0, & \text{其他}.
\end{cases}
\]
显然当且仅当$\det\boldsymbol{A}=0$时才能找到\refthe{theorem:抽象代数--定理2.6.3}中的$f_0(x)$与$g_0(x)$。
又注意到$R(f,g)=\pm\det \boldsymbol{A}$,故$(f(x),g(x))\neq1$当且仅当$R(f,g)=0.$

\end{proof}

\begin{example}
设$f(x)=x^2+x+1$,$g(x)=x^3-1\in\mathbf{Q}[x]$。
证明$f(x),g(x)$不互素并求$f_0(x),g_0(x)$,使得$f(x)g_0(x)=g(x)f_0(x)$.
\end{example}
\begin{solution}
$$R(f,g)=\begin{vmatrix}
1 & 1 & 1 & 0 & 0 \\
0 & 1 & 1 & 1 & 0 \\
0 & 0 & 1 & 1 & 1 \\
1 & 0 & 0 & -1 & 0 \\
0 & 1 & 0 & 0 & -1
\end{vmatrix}=0.$$
故由\refthe{theorem:抽象代数--定理2.6.4}知$f(x),g(x)$不互素.令$f_0(x)=1$,$g_0(x)=x-1$,则$f(x)g_0(x)=g(x)f_0(x)$。

\end{solution}

\begin{proposition}\label{proposition:两个多项式的结式与根的关系}
设
\begin{align*}
f(x)&=(x-x_1)(x-x_2)\cdots(x-x_n)=x^n+a_1x^{n-1}+\cdots+a_n, \\
g(x)&=(x-y_1)(x-y_2)\cdots(x-y_m)=x^m+b_1x^{m-1}+\cdots+b_m,
\end{align*}
证明
\begin{align*}
R(f,g)=\prod\limits_{\substack{1\leqslant i\leqslant n \\ 1\leqslant j\leqslant m}}(x_i-y_j).
\end{align*}
\end{proposition}
\begin{proof}
注意$(-1)^ia_i$,$(-1)^jb_j$分别是$x_1,x_2,\cdots,x_n$的$i$次初等对称多项式,$y_1,y_2,\cdots,y_m$的$j$次初等对称多项式。

在行列式$R(f,g)$按组合求和定义的完全展开式中任一非零项
\begin{align*}
\prod\limits_{i=1}^{m+n}\mathrm{ent}_{i\ \sigma(i)}R(f,g),\quad \sigma\in S_{m+n}
\end{align*}
是$x_1,x_2,\cdots,x_n,y_1,y_2,\cdots,y_m$的齐次多项式,其次数为
\begin{align*}
\sum\limits_{i=1}^m(\sigma(i)-i)+\sum\limits_{i=m+1}^{m+n}(\sigma(i)-(i-m))=\sum\limits_{i=1}^{m+n}\sigma(i)-\sum\limits_{i=1}^{m+n}i+mn =mn.
\end{align*}
因此,$R(f,g)$是$x_1,x_2,\cdots,x_n,y_1,y_2,\cdots,y_m$的$mn$次齐次多项式。

当$x_i=y_j$时,此时$f(x),g(x)$有公共根,从而$f(x),g(x)$不互素,故由\refthe{theorem:抽象代数--定理2.6.4}知$R(f,g)=0$.
将$R(f,g)$视为关于$x_i$的一元多项式,则$y_j$为其根.于是$(x_i-y_j)\mid R(f,g)$,所以
\begin{align*}
\prod\limits_{\substack{1\leqslant i\leqslant n \\ 1\leqslant j\leqslant m}}(x_i-y_j)\mid R(f,g).
\end{align*}
又$\deg\prod\limits_{\substack{1\leqslant i\leqslant n \\ 1\leqslant j\leqslant m}}(x_i-y_j)=mn=\deg R(f,g)$,于是
\begin{align*}
\prod\limits_{\substack{1\leqslant i\leqslant n \\ 1\leqslant j\leqslant m}}(x_i-y_j)=\pm R(f,g).
\end{align*}
注意
\begin{align*}
\prod\limits_{i=1}^{m+n}\mathrm{ent}_{i\ i}R(f,g)=a_0^mb_m^n=(-1)^{mn}(y_1y_2\cdots y_m)^n,
\end{align*}
$\prod\limits_{\substack{1\leqslant i\leqslant n \\ 1\leqslant j\leqslant m}}(x_i-y_j)$中$(y_1y_2\cdots y_m)^n$的系数亦为$(-1)^{mn}$,因此有
\begin{align*}
R(f,g)=\prod\limits_{\substack{1\leqslant i\leqslant n \\ 1\leqslant j\leqslant m}}(x_i-y_j).
\end{align*}

\end{proof}




\end{document}
\documentclass[../../main.tex]{subfiles}
\graphicspath{{\subfix{../../image/}}} % 指定图片目录,后续可以直接使用图片文件名。

% 例如:
% \begin{figure}[H]
% \centering
% \includegraphics[scale=0.4]{图.png}
% \caption{}
% \label{figure:图}
% \end{figure}
% 注意:上述\label{}一定要放在\caption{}之后,否则引用图片序号会只会显示??.

\begin{document}

\section{唯一析因环的多项式环}

\begin{definition}[容度]
设$R$为唯一析因环,$f(x)=\sum\limits_{k=0}^n a_kx^k \in R[x]$. 若$f(x) \neq 0$,则称$a_0, a_1, \cdots, a_n$的最大公因子为$f(x)$的\textbf{容度},记为$c(f(x))$或$c(f)$.
\end{definition}
\begin{remark}
由\rreflem{lemma:最大公因子条件的基本结论}{lemma:最大公因子条件的基本结论-1}知$f(x)$的容度$c(f)$在相伴意义下是唯一的.由\rreflem{lemma:最大公因子条件的基本结论}{lemma:最大公因子条件的基本结论-4}易知$c(df(x))=d\cdot c(f(x)),\forall d\in R.$
\end{remark}

\begin{definition}[本原多项式]
若$f(x) \in R[x]$,$f(x) \neq 0$,$c(f) \sim 1$,则称$f(x)$为\textbf{本原多项式}.
\end{definition}
\begin{remark}
设$R[x]$的单位群也就是$R$的单位群$U$,于是若$f(x) \in U$,则$c(f) \sim 1$.
\end{remark}

\begin{theorem}\label{theorem:抽象代数--定理2.7.1}
设$R[x]$是唯一析因环$R$上的一元多项式环,则有下列结论:
\begin{enumerate}[(1)]
\item\label{theorem:抽象代数--定理2.7.1-1} $R[x]$中任一非零多项式$f(x)$是$c(f)$与一本原多项式$f_1(x)$的积,即
\begin{align}
f(x) = c(f)f_1(x) \label{eq:::--298fh232.7.3}
\end{align}
且这种分解在相伴意义下唯一;

\item\label{theorem:抽象代数--定理2.7.1-2} 次数大于零的不可约多项式是本原多项式;

\item\label{theorem:抽象代数--定理2.7.1-3} 本原多项式的积为本原多项式.
\end{enumerate}
\end{theorem}
\begin{proof}
\begin{enumerate}[(1)]
\item 设$c(f)=d$,$f(x)=\sum\limits_{k=0}^n a_kx^k$,于是$a_k = da_k'$,因而由\rreflem{lemma:最大公因子条件的基本结论}{lemma:最大公因子条件的基本结论-4}知$d(a_0', a_1', \cdots, a_n') \sim (a_0, a_1, \cdots, a_n) = d$,从而再由\rrefthe{theorem:单位群整除的基本性质}{theorem:单位群整除的基本性质-2}知$(a_0', a_1', \cdots, a_n') \sim 1$,故$f_1(x)=\sum\limits_{k=0}^n a_k'x^k$为本原多项式且式\eqref{eq:::--298fh232.7.3}成立.

若另有$f(x)=d_1f_2(x)$,$d_1 \in R$,$c(f_2) \sim 1$,则$d_1c(f_2) \sim c(d_1f_2(x)) \sim c(f)=d$,故由\rrefthe{theorem:单位群整除的基本性质}{theorem:单位群整除的基本性质-2}知$d_1 \sim d$,再由\rrefthe{theorem:单位群整除的基本性质}{theorem:单位群整除的基本性质-5}知$d_1 = du$($u \in U$),因而$f\left( x \right) =df_1\left( x \right) =duf_2\left( x \right) $,从而由\rrefpro{proposition:整环的一些性质}{proposition:整环的一些性质-2}得$f_1(x) = uf_2(x)$,由\rrefthe{theorem:单位群整除的基本性质}{theorem:单位群整除的基本性质-5}知$f_1(x) \sim f_2(x)$,亦即$f(x)$的上述分解在相伴意义下唯一.

\item 设$f(x)$不可约且$\deg f(x) > 0$,$d = c(f)$. 由结论(1)知$f(x) = df_1(x)$.由$\deg f(x)>0$知$\deg f_1(x)>0$,从而$f_1(x) \notin U$. 若$d \notin U$,则$f(x)$有非平凡的真因子$d$,与$f(x)$不可约矛盾. 故必有$d\in U$,即$c(f) = d \sim 1$,即$f(x)$是本原的.

\item 设$f(x)=\sum\limits_{k=0}^n a_kx^k$,$a_n \neq 0$;$g(x)=\sum\limits_{k=0}^m b_kx^k$,$b_m \neq 0$都是本原多项式. 又
\begin{align*}
h(x) = f(x)g(x) = \sum\limits_{k=0}^{m+n} c_kx^k,
\end{align*}
其中,
\begin{align*}
c_k = \sum\limits_{i+j=k} a_ib_j, \quad k=0,1,\cdots,m+n.
\end{align*}
假设$c(h) \notin U$,则由有限析因条件和\refthe{theorem:UFD的充要条件}知有$R$中素元素$p|c(h)$,即有$p|c_k$($k=0,1,\cdots,m+n$).

由$c(f) \sim c(g) \sim 1$及\rreflem{lemma:最大公因子条件的基本结论}{lemma:最大公因子条件的基本结论-6}知$(p,c(f))\sim (p,c(g))\sim (p,1)\sim 1$.因此$p$
于是由\rreflem{lemma:最大公因子条件的基本结论}{lemma:最大公因子条件的基本结论-9}知存在$r, s$,使得
\[
p|a_i, 0 \leqslant i \leqslant r-1, \quad p \nmid a_r; \quad p|b_j, 0 \leqslant i \leqslant s-1, \quad p \nmid b_s,
\]
再由
\[
c_{r+s} = \sum\limits_{i+j=r+s} a_ib_j = a_rb_s + \sum\limits_{\substack{i<r, \\ i+j=r+s}} a_ib_j + \sum\limits_{\substack{j<s, \\ i+j=r+s}} a_ib_j
\]
及$p\mid c_{r+s}$可知
\[
p \left| \sum\limits_{\substack{i<r, \\ i+j=r+s}} a_ib_j, \quad p \left| \sum\limits_{\substack{j<s, \\ i+j=r+s}} a_ib_j, \quad p|a_rb_s,
\right. \right.
\]
这与$p\nmid a_r,b_s$矛盾! 故$c(h) \sim 1$,即$f(x)g(x)$是本原多项式.
\end{enumerate}
\end{proof}

\begin{theorem}\label{theorem:抽象代数--定理2.7.2}
设$F$是唯一析因环$R$的分式域. 于是$F[x] \supseteq R[x]$. 又设$S$为$R[x]$中本原多项式的集合,$R[x]$中相伴关系记为$\stackrel{R}{\sim}$,$F[x]$中相伴关系记为$\stackrel{F}{\sim}$,则有下列结论:
\begin{enumerate}[(1)]
\item\label{theorem:抽象代数--定理2.7.2-1} $\forall f(x) \in F[x]$,$f(x) \neq 0$,$\exists g(x) \in S$,使$f(x) \stackrel{F}{\sim} g(x)$且$g(x)$在$\stackrel{R}{\sim}$意义下是唯一的;

\item\label{theorem:抽象代数--定理2.7.2-2} 设$f_1(x), f_2(x) \in F[x]$,$g(x), g_1(x), g_2(x) \in S$且
\[
f_1(x) \stackrel{F}{\sim} g_1(x), \quad f_2(x) \stackrel{F}{\sim} g_2(x), \quad f_1(x)f_2(x) \stackrel{F}{\sim} g(x),
\]
则有
\[
g_1(x)g_2(x) \stackrel{R}{\sim} g(x);
\]

\item\label{theorem:抽象代数--定理2.7.2-3} 设$f(x) \in R[x]$,$\deg f(x) \geqslant 1$,则$f(x)$在$R[x]$中不可约的充要条件是$f(x)$在$F[x]$中也不可约.
\end{enumerate}
\end{theorem}
\begin{proof}
\begin{enumerate}[(1)]
\item 设$f(x)=\sum\limits_{k=0}^n d_kx^k \in F[x]$,即$d_k \in F$. 于是由$F$是$R$的分式域知$d_k = \frac{a_k}{b_k}$,$a_k, b_k \in R$,$b_k \neq 0$,$0 \leqslant k \leqslant n$. 令$b = b_0b_1\cdots b_n$,则有
\[
d_kb = a_k \prod_{\substack{i \neq k}} b_i \in R, \quad 0 \leqslant k \leqslant n.
\]
再令$d = (d_0b, d_1b, \cdots, d_nb)\in R\setminus \{0\}$,则由$d\mid d_kb$知存在$c_k\in R\setminus \{0\}$,使$ dc_k= d_kb$.
于是由\rreflem{lemma:最大公因子条件的基本结论}{lemma:最大公因子条件的基本结论-4}知
\begin{align*}
d\left( c_0,c_1,\cdots ,c_n \right) \stackrel{R}{\sim} \left( dc_0,dc_1,\cdots ,dc_n \right) =\left( d_0b,d_1b,\cdots ,d_nb \right) =d.
\end{align*}
于是再由\rrefthe{theorem:单位群整除的基本性质}{theorem:单位群整除的基本性质-2}得
\begin{align*}
\left( c_0,c_1,\cdots ,c_n \right) \stackrel{R}{\sim} 1.
\end{align*}
而
\[
f(x) = \frac{d}{b} \sum\limits_{k=0}^n c_kx^k = \frac{d}{b}g(x),
\]
其中$\frac{d}{b} \in F$,$g(x) = \sum\limits_{k=0}^n c_kx^k \in R[x]$,$(c_0, c_1, \cdots, c_n) \stackrel{R}{\sim} 1$,故$g(x) \in S$且$g(x)=\frac{b}{d}f(x).$ 这样得到$f(x) \stackrel{F}{\sim} g(x)$.

现设$f(x) \stackrel{F}{\sim} g_1(x)$,$g_1(x) \in S$. 又$f(x) \stackrel{F}{\sim} g(x)$,所以$g_1(x) \stackrel{F}{\sim} g(x)$,即$\exists u \in F^*$,使$g_1(x) = ug(x)$. 又$u = \frac{d'}{d}$,$d', d \in R$,故有$dg_1(x) = d'g(x) \in R[x]$. 由\rrefthe{theorem:抽象代数--定理2.7.1}{theorem:抽象代数--定理2.7.1-1}知$d'g(x)$是其自身的一个分解,从而$g_1(x) \stackrel{R}{\sim} g(x)$. 唯一性得证.
\item 由于$f_1(x) \stackrel{F}{\sim} g_1(x)$,$f_2(x) \stackrel{F}{\sim} g_2(x)$,故由\hyperref[theorem:单位群整除的基本性质-8]{相伴关系对乘法构成同余关系}知
\[
f_1(x)f_2(x) \stackrel{F}{\sim} g_1(x)g_2(x) \stackrel{F}{\sim} g(x).
\]
由\rrefthe{theorem:抽象代数--定理2.7.1}{theorem:抽象代数--定理2.7.1-3}知$g_1(x)g_2(x) \in S$. 再由\hyperref[theorem:抽象代数--定理2.7.2-1]{本定理的结论\ref{theorem:抽象代数--定理2.7.2-1}}知$g_1(x)g_2(x) \stackrel{R}{\sim} g(x)$.
\item {\heiti 必要性:}用反证法证明. 假设$f(x)$作为$F[x]$中的多项式是可约的,由$F[x]$的单位群为$F^*$,故有$f_1(x), f_2(x) \in F[x]$且$\deg f_i(x) \geqslant 1$,使得$f(x) = f_1(x)f_2(x)$. 由\hyperref[theorem:抽象代数--定理2.7.2-1]{本定理的结论\ref{theorem:抽象代数--定理2.7.2-1}}知$\exists g_1(x), g_2(x) \in S$,使
\[
g_i(x) \stackrel{F}{\sim} f_i(x), \quad i=1,2.
\]
于是\hyperref[theorem:单位群整除的基本性质-8]{相伴关系对乘法构成同余关系}知
\[
f(x) \stackrel{F}{\sim} g_1(x)g_2(x).
\]
因为$f(x)$在$R[x]$中不可约,所以由\rrefthe{theorem:抽象代数--定理2.7.1}{theorem:抽象代数--定理2.7.1-2}有$f(x) \in S$. 又$f(x)\stackrel{F}{\sim} f(x)$,故再由\hyperref[theorem:抽象代数--定理2.7.2-2]{本定理的结论\ref{theorem:抽象代数--定理2.7.2-2}}知$f(x) \stackrel{R}{\sim} g_1(x)g_2(x)$. 这与已知的$f(x)$在$R[x]$中不可约矛盾,故$f(x)$在$F[x]$中也不可约.

{\heiti 充分性:}若$f(x)$在$R[x]$中可约,则存在$f_1(x),f_2(x)\in R[x]\subseteq F[x]$,使$f(x)=f_1(x)f_2(x)$.从而$f(x)$在$F[x]$中也可约,矛盾!
\end{enumerate}
\end{proof}

\begin{theorem}\label{theorem:抽象代数--定理2.7.3}
唯一析因环$R$上的一元多项式环$R[x]$也是唯一析因环.
\end{theorem}
\begin{proof}
设$U$为$R^*$中可逆元素的集合,$f(x) \in R[x]$,$f(x) \neq 0$. 于是由\rrefthe{theorem:抽象代数--定理2.7.1}{theorem:抽象代数--定理2.7.1-1}知$\exists d \in R$,$g(x) \in S$,使得
\[
f(x) = dg(x).
\]
因$d \in R$,则有$d = p_1p_2\cdots p_t$,$p_i$($1 \leqslant i \leqslant t$)为$R$中不可约元素,在$R[x]$中也不可约. 若$\deg f(x) = 0$,则$\deg g(x) = 0$,$g(x) \sim 1$,故$g(x)\in U$,即$f(x)$可分解为有限个不可约元素之积. 再设$\deg g(x) > 0$,于是$\deg g(x) = \deg f(x)$. 设$F$为$R$的分式域,则由\rrefthe{theorem:抽象代数--定理2.6.1}{theorem:抽象代数--定理2.6.1-2}知$F[x]$是Euclid环,进而也是唯一析因环.于是$g(x) \in F[x]$可分解为不可约多项式的积
\[
g(x) = g_1(x)g_2(x)\cdots g_r(x).
\]
根据\rrefthe{theorem:抽象代数--定理2.7.2}{theorem:抽象代数--定理2.7.2-1}有$p_i(x) \in S$(其中$S$为$R[x]$中本原多项式的集合)且满足$p_i(x) \stackrel{F}{\sim} g_i(x)$,由\refpro{proposition:不可约元素乘单位仍不可约}知$p_i(x)$是$F[x]$中不可约多项式.并且由\hyperref[theorem:单位群整除的基本性质-8]{相伴关系对乘法构成同余关系}知
\[
g(x) \stackrel{F}{\sim} p_1(x)p_2(x)\cdots p_r(x).
\]
由\rrefthe{theorem:抽象代数--定理2.7.1}{theorem:抽象代数--定理2.7.1-3}知$p_1(x)p_2(x)\cdots p_r(x)$为本原多项式,又$g(x)$也是本原多项式,故由\rrefthe{theorem:抽象代数--定理2.7.2}{theorem:抽象代数--定理2.7.2-2}得
\[
g(x) \stackrel{R}{\sim} p_1(x)p_2(x)\cdots p_r(x).
\]
因此可不妨设$g(x) = p_1(x)p_2(x)\cdots p_r(x)$,$p_i(x)$是$F[x]$中的不可约多项式,由\rrefthe{theorem:抽象代数--定理2.7.2}{theorem:抽象代数--定理2.7.2-3}知$p_i(x)$在$R[x]$中也不可约,故$f(x)$可分解为$R[x]$中有限个不可约元素之积
\[
f(x) = p_1p_2\cdots p_tp_1(x)p_2(x)\cdots p_r(x).
\]

下面证因式分解的唯一性. 设$f(x)$还有分解式
\[
f(x) = q_1q_2\cdots q_{t'}q_1(x)q_2(x)\cdots q_s(x),
\]
其中,$q_i$为$R$中不可约元素,$q_j(x)$为$R[x]$中不可约多项式且$\deg q_j(x) > 0$. 由\rrefthe{theorem:抽象代数--定理2.7.1}{theorem:抽象代数--定理2.7.1-2}知$q_j(x) \in S$,故由\rrefthe{theorem:抽象代数--定理2.7.1}{theorem:抽象代数--定理2.7.1-3}知$q_1(x)q_2(x)\cdots q_s(x) \in S$. 再由\rrefthe{theorem:抽象代数--定理2.7.1}{theorem:抽象代数--定理2.7.1-1}知有
\[
p_1p_2\cdots p_t \sim q_1q_2\cdots q_{t'},
\]
\[
p_1(x)p_2(x)\cdots p_r(x) \stackrel{R}{\sim} q_1(x)q_2(x)\cdots q_s(x).
\]
由$R$为唯一析因环知$t = t'$且$\exists \pi_1 \in S_t$,使得$p_i \sim q_{\pi_1(i)}$. 又由\rrefthe{theorem:抽象代数--定理2.7.2}{theorem:抽象代数--定理2.7.2-3}知$p_i(x)$,$q_j(x)$均为$F[x]$中不可约多项式,而$F[x]$为Euclid环,由\refthe{theorem:Euclid环必是主理想整环}知$F[x]$也是唯一析因环,故$r = s$且$\exists \pi_2 \in S_r$,使得$p_i(x) \stackrel{F}{\sim} q_{\pi_2(i)}(x)$.
又由\rrefthe{theorem:抽象代数--定理2.7.1}{theorem:抽象代数--定理2.7.1-2}知$p_i(x),q_{\pi_2(i)}(x)$都是$R[x]$中的本原多项式且$p_i(x)\stackrel{F}{\sim} p_i(x)$,故再由\rrefthe{theorem:抽象代数--定理2.7.2}{theorem:抽象代数--定理2.7.2-1}知$p_i(x) \stackrel{R}{\sim} q_{\pi_2(i)}(x)$. 因此,在$R[x]$中因式分解唯一性定理成立,即$R[x]$也是唯一析因环.

\end{proof}

\begin{corollary}\label{corollary:抽象代数--推论2.7.1}
唯一析因环$R$上的$n$元多项式环$R[x_1, x_2, \cdots, x_n]$也是唯一析因环.
\end{corollary}
\begin{proof}
对$n$用数学归纳法,再根据\refthe{theorem:抽象代数--定理2.7.3}同理可证.

\end{proof}

\begin{theorem}\label{theorem:抽象代数--定理2.7.5}
设$F$是唯一析因环$R$的分式域,又$f(x) = \sum\limits_{k=0}^n a_kx^k \in R[x]$,$a_n \neq 0$($n > 1$). 若有$R$中素元素$p$满足
\begin{enumerate}[(1)]
\item $p \nmid a_n$;
\item $p|a_k$,$0 \leqslant k \leqslant n-1$;
\item $p^2 \nmid a_0$,
\end{enumerate}
则$f(x)$是$F[x]$中不可约元素.
\end{theorem}
\begin{proof}
由\rrefthe{theorem:抽象代数--定理2.7.2}{theorem:抽象代数--定理2.7.2-3},只需证明$f(x)$在$R[x]$中不能分解为两个次数大于零的多项式的乘积即可. 若不然,则有$f(x) = g(x)h(x)$,其中
\[
g(x) = \sum\limits_{k=0}^r b_kx^k, \quad b_k \in R, \, b_r \neq 0, \, r \geqslant 1,
\]
\[
h(x) = \sum\limits_{k=0}^s c_kx^k, \quad c_k \in R, \, c_s \neq 0, \, s \geqslant 1
\]
且有
\[
r + s = n, \quad a_k = \sum\limits_{i+j=k} b_ic_j, \quad p|a_0, \quad p^2 \nmid a_0.
\]
从而$p\mid b_0c_0$.由素元素定义知$p\mid b_0$或$p\mid c_0$,不妨设$p|c_0$,$p \nmid b_0$. 又$p \nmid a_n$,故$p \nmid c_s$,$p \nmid b_r$,因而存在$t\in \{1,2,\cdots,s\}$,使得$p|c_i$($0 \leqslant i \leqslant t-1$),$p \nmid c_t$. 而
\[
a_t = \sum\limits_{i+j=t} c_ib_j = \sum\limits_{\substack{i<t, \\ i+j=t}}c_ib_j + c_tb_0.
\]
由$p$能整除上式右端第一项,而$p \nmid c_tb_0$,故$p \nmid a_t$. 这与定理中条件(2)矛盾,故$f(x)$在$F[x]$中不可约.

\end{proof}

\begin{example}
设$p$为素数,$f(x) = x^{p-1} + x^{p-2} + \cdots + 1 \in \mathbb{Q}[x]$是不可约多项式.
\end{example}
\begin{proof}
令$g(x) = f(x+1)$,
\begin{align*}
g(x) = \frac{(x+1)^p - 1}{x} = \sum\limits_{k=1}^p \mathrm{C}_p^k x^{k-1}.
\end{align*}
由$\mathrm{C}_p^p = 1$,$p|\mathrm{C}_p^k$($1 \leqslant k \leqslant p-1$),$p^2 \nmid \mathrm{C}_p^1 = p$知$g(x)$为$\mathbb{Q}[x]$中不可约多项式,从而$f(x)$为不可约多项式.

\end{proof}

\begin{example}
$f(x, y) = x^2y + x^2 + y^2 + 2y + 2 \in \mathbb{Q}[x, y]$是不可约多项式.
\end{example}
\begin{proof}
由\refthe{theorem:任意n元多项式环都可由n-1元的生成}知$\mathbb{Q}[x, y] = (\mathbb{Q}[x])[y]$,而$x^2y + x^2 + y^2 + 2y + 2 = y^2 + y(x^2 + 2) + (x^2 + 2)$. 又$x^2 + 2$是$\mathbb{Q}[x]$中不可约多项式,由\rrefthe{theorem:抽象代数--定理2.7.1}{theorem:抽象代数--定理2.7.1-3}知$x^2 + 2$也是$\mathbb{Q}[x]$中的素元素,故由Eisenstein判别法知$f(x, y)$为不可约多项式.

\end{proof}























\end{document}
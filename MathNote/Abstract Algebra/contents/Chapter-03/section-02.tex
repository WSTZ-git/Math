\documentclass[../../main.tex]{subfiles}% 注意这里的文件路径不能用 ./main.tex ,否则用latexmk编译子文件会报错
\graphicspath{{\subfix{./image/}}} % 指定图片目录,后续可以直接使用图片文件名
% 注意这里的文件路径不能用 ../../image/ ,否则用latexmk编译子文件会报错

% 例如:
% \begin{figure}[H]
% \centering
% \includegraphics[scale=0.3]{图.png}
% \caption{}
% \label{figure:图}
% \end{figure}
% 注意:上述\label{}一定要放在\caption{}之后,否则引用图片序号会只会显示??.

\begin{document}

\section{群集合上的作用}


\begin{definition}[群作用]\label{definition:群集合上的作用的定义}
设$G$是一个群,$X$是一个非空集合。若$G \times X$到$X$的映射$f$满足
\begin{enumerate}[(1)]
\item $f(e, x) = x$, $\forall x \in X$, $e$为$G$的幺元;
\item $f(g_1g_2, x) = f(g_1, f(g_2, x))$, $\forall g_1, g_2 \in G$, $x \in X$,
\end{enumerate}
则称$f$决定了\textbf{群$G$在$X$上的一个作用}.

群$G$可以多种方式作用在一个集合$X$上。在不需要特别指出映射$f$(即固定好一种作用方式)时,常记
\begin{align*}
f(g, x) = g(x), \quad \forall g \in G, x \in X.
\end{align*}
此时$f$满足的条件(1), (2)相应地变为
\begin{enumerate}[(1)]
\item $e(x) = x, \forall x \in X$,$e$为$G$的幺元;
\item $g_1g_2(x) = g_1(g_2(x)), \forall x \in X, g_1, g_2 \in G$.
\end{enumerate}
\end{definition}

\begin{theorem}\label{theorem:一些常见的群作用的良定义性}
\begin{enumerate}
\item 设$G$是一个群,取$X = G$,
\begin{enumerate}
\item 若定义$f$为
\begin{align*}
f(g, x) = L_g(x) = gx, \quad \forall g, x \in G.
\end{align*}
则$f$定义了$G$在$G$上的一个作用,这种作用称为\textbf{左平移作用}。

\item 若定义$f_1$为
\begin{align*}
f_1(g, x) = R_{g^{-1}}(x) = xg^{-1}, \quad \forall g, x \in G,
\end{align*}
则$f_1$也定义了$G$在$G$上的一个作用,这种作用称为\textbf{右平移作用}。

\item 若定义$f_2$为
\begin{align*}
f_2(g, x) = \text{ad}g(x) = gxg^{-1}, \quad \forall g, x \in G,
\end{align*}
则$f_2$也定义了$G$在$G$上的一个作用,称为\textbf{伴随作用}。
\end{enumerate}

\item 设$H$为群$G$的子群,取$X = G/H$($H$在$G$中全体左陪集的集合)。定义$f$为
\begin{align*}
f(g, xH) = gxH, \quad \forall g \in G, xH \in G/H,
\end{align*}
则$f$定义了$G$在$G/H$上的作用(\textbf{左平移作用})。特别地,当$H = \{e\}$时,$f$恰是$G$在$G$上的左平移作用。
\end{enumerate}

\end{theorem}
\begin{proof}


\end{proof}

\begin{definition}
设群$G$作用在集合$X$上。若$\forall x, y \in X$, $\exists g \in G$, 使$y = g(x)$,则称$G$在$X$上的作用是\textbf{可递的},$X$称为(对于$G$的)\textbf{齐性空间}。
\end{definition}

\begin{definition}\
设群$G$作用在集合$X$上。若$g(x) = x(\forall g \in G, \forall x \in X)$,则称$G$在$X$上的作用是\textbf{平凡的}。
\end{definition}
\begin{remark}
显然,对任意群$G$,任意非空集合$X$,总可定义$G$在$X$上的平凡作用。由上述定义知$G$在$G$上的伴随作用为平凡作用当且仅当$G$为Abel群。

\end{remark}

\begin{definition}
设群$G$作用在集合$X$上,$e$为$G$的幺元,若当且仅当$g = e$时,$g(x) = x(\forall x \in X)$成立,则称$G$在$X$上的作用是\textbf{有效的}。
\end{definition}

\begin{proposition}
群$G$在$G$上的左平移作用与右平移作用既是可递的又是有效的,而$G$在$G/H$上的左平移作用是可递的.
\end{proposition}
\begin{remark}
$G$在$G$上的伴随作用的可递性与有效性都不能肯定.

$G$在$G/H$上的左平移作用不一定是有效的。
\end{remark}
\begin{proof}


\end{proof}

\begin{definition}
设群$G$作用在集合$X$上,$x \in X$。称$X$中的子集
\begin{align*}
O_x = \{g(x)\in X \mid g \in G\}
\end{align*}
为$x$的\textbf{轨道}。
$G$中子集
\begin{align*}
S_x = \{g \in G\mid  g(x) = x\}
\end{align*}
称为$x$的\textbf{迷向子群}.也称为$x$在$G$中的\textbf{稳定子}或\textbf{稳定化子}.
\end{definition}
\begin{remark}
容易验证$x$的迷向子群是$G$的子群.
\end{remark}

\begin{proposition}\label{proposition:可递的群作用X等于其轨道}
群$G$在集合$X$上的作用是可递的充要条件是对$\forall x \in X$,有$X=O_x$.

进而对$\forall x\in X$,$G$在$O_x$上的作用都是可递的.
\end{proposition}
\begin{proof}
{\heiti 必要性:}对$\forall x\in X$,固定$x$.再对$\forall y\in X$,则由$G$在$X$上的作用是可递的知,存在$g\in G$,使得$y=g(x)\in O_x$.故由$y$的任意性知$X\subseteq O_x$.又显然$O_x\subseteq X$,故$X=O_x$.

{\heiti 充分性:}若对$\forall x\in X$,有$X=O_x$,则对$\forall y\in X$,都存在$g\in G$,使得$y=g(x)$.因此由$x,y$的任意性知$G$在$X$上的作用是可递的.

\end{proof}

\begin{example}
设$X = \mathbb{R}^n$为$n$维Euclid空间,$G = SO(n)$为$X$的特殊正交群,$G$以通常方式作用在$X$上。又设$X = (1, 0, \cdots, 0)'$,则易得
\begin{align*}
O_x = \{\boldsymbol{y} \mid \boldsymbol{y} \in X, |\boldsymbol{y}| = 1\} = S^{n-1}
\end{align*}
是$X$中$n-1$维单位球面,其中,$|\boldsymbol{y}|$为向量$\boldsymbol{y}$的长度,
\begin{align*}
S_x = \{\text{diag}(1, A) \mid A \in SO(n-1)\},
\end{align*}
故$S_x$与$n-1$维特殊正交群$SO(n-1)$同构。
\end{example}
\begin{proof}


\end{proof}

\begin{theorem}\label{theorem:抽象代数--定理4.2.2}
设群$G$作用在集合$X$上。则有
\begin{enumerate}[(1)]
\item\label{theorem:抽象代数--定理4.2.2-0} $O_x=O_y\iff O_x\cap O_y\neq \varnothing .$

\item\label{theorem:抽象代数--定理4.2.2-1} 在$X$中定义关系$R$:
\begin{align*}
xRy\iff \exists g \in G, \text{使}y = g(x).
\end{align*}
则$R$为等价关系且$x$所在的等价类为$x$的轨道$O_x$,进而$X$等价类(所有轨道)集合是$X$的一个分划.即$X$是所有不同轨道的并
\begin{align*}
X=\bigsqcup_x O_x,
\end{align*}
其中$x$取遍$X$的不同轨道的代表元素.

\item\label{theorem:抽象代数--定理4.2.2-2} 如果$X$为有限集,则
\begin{align*}
|X|=\sum_{x\in X}|O_{x}|,
\end{align*}
其中$x$取遍$X$的不同轨道的代表元素.
\end{enumerate}
\end{theorem}
\begin{proof}
\begin{enumerate}[(1)]
\item 设 $O_x \cap O_y \neq \varnothing$. 任取 $z \in O_x \cap O_y$, 则存在 $g_1,g_2 \in G$, 使
\begin{align*}
g_1x = z = g_2y.
\end{align*}
于是 $y = g_2^{-1}g_1x \in O_x$, 由此得 $O_y \subseteq O_x$. 同理可证 $O_x \subseteq O_y$. 所以 $O_x = O_y$.


\item 对$\forall x, y, z \in X$,由$e(x) = x$知$xRx$ ($\forall x \in X$),由$g(x) = y$得$g^{-1}(y) = g^{-1}(g(x)) =g^{-1}g(x)= x$,即$xRy \Rightarrow yRx$,再由$xRy, yRz$知$\exists g_1, g_2 \in G$,使得$y = g_1(x)$, $z = g_2(y)$,故$z = g_2g_1(x)$,即$xRz$。这就说明$R$是等价关系,由$R$的定义知$x$的等价类为$O_x$。

\item 因为$X$是有限集,所以至多只有有限多个不同的轨道.由\rrefthe{theorem:抽象代数--定理4.2.2}{theorem:抽象代数--定理4.2.2-1}知
\begin{align*}
X = \bigsqcup_{x\in X}O_{x_i}.
\end{align*}
其中$x$取遍$X$的不同轨道的代表元素.
又因为$O_{x_i}\cap O_{x_j}=\varnothing (i\neq j)$,所以
\begin{align*}
|X|=\sum_{x\in X}|O_{x}|.
\end{align*}
\end{enumerate}

\end{proof}

\begin{theorem}\label{theorem:抽象代数--定理4.2.1}
设群$G$作用在集合$X$上。
\begin{enumerate}[(1)]
\item\label{theorem:抽象代数--定理4.2.1-1} 对$\forall g \in G$,定义$X$到$X$的映射$\sigma$满足
\begin{align}
\sigma_g(x) = g(x), \quad \forall x \in X \label{eq::::--g89eht34y4yehygte4tegg4.2.1}
\end{align}
则定义的$\sigma_g$是$X$的可逆变换,即$\sigma_g \in S_X$。

\item\label{theorem:抽象代数--定理4.2.1-2} 定义的$G$到$S_X$的映射$\sigma$满足
\begin{align*}
\sigma \left( g \right) =\sigma_g,\forall g\in G.
\end{align*}
其中$\sigma_g$的定义如\eqref{eq::::--g89eht34y4yehygte4tegg4.2.1}式.则
\begin{enumerate}[(i)]
\item\label{theorem:抽象代数--定理4.2.1-2-1} $\sigma$是一个同态映射,并且$G$在$X$上的作用有效当且仅当$\sigma$是单同态.

\item\label{theorem:抽象代数--定理4.2.1-2-2} 
$\ker \sigma\lhd G$,
$G$在$O_x$上作用有效当且仅当$S_x$中所包含的$G$的正规子群仅为$\{e\}$.
\end{enumerate}

\item\label{theorem:抽象代数--定理4.2.1-3} 若$\sigma$是群$G$到$S_X$的同态,则由
\begin{align}
g(x) = \sigma(g)(x), \quad \forall g \in G, x \in X \label{eq::::--g89eht34y4yehygte4tegg4.2.2}
\end{align}
定义了$G$在$X$的作用.
\end{enumerate}
\end{theorem}
\begin{proof}
\begin{enumerate}[(1)]
\item 任取$g \in G$,由式\eqref{eq::::--g89eht34y4yehygte4tegg4.2.1}有
\begin{align*}
\sigma_{g^{-1}}\sigma_g(x) = g^{-1}(g(x)) = g^{-1}g(x) = e(x) = x,\quad \forall x \in X.
\end{align*}
同样有
\begin{align*}
\sigma_g\sigma_{g^{-1}}(x) = x,\quad \forall x \in X.
\end{align*}
故
\begin{align*}
\sigma_{g^{-1}}\sigma_g = \sigma_g\sigma_{g^{-1}} = \text{id}_X,
\end{align*}
因而$\sigma_g \in S_X$且$\sigma_{g^{-1}} = \sigma_g^{-1}$。

\item \begin{enumerate}[(i)]
\item 取$g_1, g_2 \in G$,对$\forall x \in X$有
\begin{align*}
\sigma (g_1g_2)(x)=\sigma _{g_1g_2}(x)=g_1g_2(x)=g_1(g_2(x))=\sigma _{g_1}(\sigma _{g_2}(x))=\sigma _{g_1}\sigma _{g_2}(x)=\sigma (g_1)\sigma (g_2)(x),
\end{align*}
即
\begin{align*}
\sigma(g_1g_2) = \sigma(g_1)\sigma(g_2), \quad \forall g_1, g_2 \in G,
\end{align*}
因而$\sigma$是$G$到$S_X$的同态。
注意到
\begin{align*}
g \in \ker\sigma \iff \sigma(g) = \sigma_g = \text{id}_X,
\end{align*}
即
\begin{align*}
g(x) = x, \quad \forall x \in X,
\end{align*}
故$G$在$X$上作用有效当且仅当$\ker\sigma = \{e\}$,即$\sigma$是单射。

\item 设$\sigma$为$G$到$S_{O_x}$的映射,满足$\sigma(g)y = g(y)(\forall y \in O_x)$。于是由\refthe{theorem:抽象代数--定理4.2.1}知$\sigma$是同态且$G$在$O_x$上作用有效当且仅当$\ker\sigma = \{e\}$。由\rrefpro{proposition:群同态的核是定义域的子群,像是陪域的子群-抽象代数}{proposition:群同态的核是定义域的子群,像是陪域的子群-抽象代数-1}知道$\ker\sigma \lhd G$.
注意到
\begin{align}\label{eq::oj3ty4f}
g \in \ker\sigma \iff \sigma(g) =  \text{id}_X \iff
g(x) = x(\forall x \in X) \iff g\in S_x.
\end{align}
故$\ker\sigma \subseteq S_x$,因而若$S_x$中所含$G$的正规子群仅为$\{e\}$,则必有$\ker\sigma = \{e\}$。从而$G$在$O_x$上作用有效。

设$N \lhd G$, $N \subseteq S_x$。任取$ h \in N,$ 对$\forall y\in O_x$,都存在$g \in G$,使得$y=g(x)$.由$N\lhd G$知$g^{-1}hg \in N \subseteq S_x$,因而
\begin{align*}
h(y)=h(g(x))=gg^{-1}hg\left( x \right) =g(g^{-1}hg(x))=g(x)=y,\quad \forall y\in O_x.
\end{align*}
由\eqref{eq::oj3ty4f}式知$h \in \ker\sigma$,即$N \subseteq \ker\sigma$。所以若$G$在$O_x$上作用有效,则$\ker\sigma = \{e\}$,由此知$N = \{e\}$,即$\{e\}$为$S_x$所包含的唯一的$G$的正规子群。
\end{enumerate}

\item 因$\sigma$是$G$到$S_X$的同态,由式\eqref{eq::::--g89eht34y4yehygte4tegg4.2.2}有
\begin{align*}
e(x) = \sigma(e)(x) = \text{id}_X(x) = x, \quad \forall x \in X,
\end{align*}
\begin{align*}
g_1(g_2(x)) = \sigma(g_1)(\sigma(g_2)(x)) = \sigma(g_1)\sigma(g_2)(x) = \sigma(g_1g_2)(x) = g_1g_2(x), \quad \forall x \in X, g_1, g_2 \in G,
\end{align*}
即$\sigma$定义了$G$在$X$上的作用。
\end{enumerate}

\end{proof}

\begin{definition}
设群$G$作用在集合$X$与$X'$上,若有$X$到$X'$上的一一对应$\phi$,使
\begin{align*}
g(\phi(x)) = \phi(g(x)), \quad \forall g \in G, x \in X,
\end{align*}
则称$G$在$X, X'$上的\textbf{作用等价}.
\end{definition}
\begin{remark}
如果将$g$引起的$X, X'$上的置换仍以$g$来表示,那么$G$在$X, X'$上的作用等价也就是对任何$g \in G$,\reffig{figure:交换图--xxx-1.1.1.1}是交换图.

如果在$G$作用的集合之间规定关系$R: XRX'$,若$G$在$X, X'$上作用等价.这显然是一个等价关系,因而从抽象的观点来看,等价作用可以看成是一样的.
\begin{figure}[H]
\centering
% https://q.uiver.app/#q=WzAsNCxbMCwwLCJYIl0sWzIsMCwiWCciXSxbMiwyLCJYJyJdLFswLDIsIlgiXSxbMCwxLCJcXHBoaSJdLFszLDIsIlxccGhpIiwyXSxbMSwyLCJnIl0sWzAsMywiZyIsMl1d
\begin{tikzcd}
X && {X'} \\
\\
X && {X'}
\arrow["\phi", from=1-1, to=1-3]
\arrow["g"', from=1-1, to=3-1]
\arrow["g", from=1-3, to=3-3]
\arrow["\phi"', from=3-1, to=3-3]
\end{tikzcd}
\caption{}
\label{figure:交换图--xxx-1.1.1.1}
\end{figure}
\end{remark}

\begin{theorem}\label{theorem:抽象代数--定理4.2.3}
设群$G$在$X$上的作用可递,$x \in X$,则$G$在$X$上的作用与$G$在$G/S_x$上的作用(左平移作用)等价.
\end{theorem}
\begin{remark}
这个定理表明$G$在每个轨道上的作用相当于$G$在某个左陪集空间上的作用.
\end{remark}
\begin{proof}
因$G$在$X$上的作用可递,于是由\refpro{proposition:可递的群作用X等于其轨道}有
\begin{align*}
X = O_x = \{g(x)\in X \mid g \in G\}.
\end{align*}
作$G/S_x$到$X$的映射$\phi$如下:
\begin{align*}
\phi(gS_x) = g(x), \quad \forall g \in G.
\end{align*}
显然$\phi$是满射.由于$g_1S_x = g_2S_x$当且仅当$g_1^{-1}g_2 \in S_x$,当且仅当$g_1^{-1}g_2(x) = x$,当且仅当$g_1(x) = g_2(x)$,因而$\phi$是单射.故$\phi$是$G/S_x$到$X$上的一一对应.又对$\forall h \in G$有
\begin{align*}
\phi(h(gS_x)) = \phi(hgS_x) = hg(x) = h(\phi(gS_x)),
\end{align*}
故$G$在$G/S_x$与$X$上的作用等价.

\end{proof}

\begin{corollary}\label{corollary:有限群轨道的阶与其商掉迷向子群的阶相同}
设有限群$G$作用在集合$X$上,$O_x$为$x \in X$的轨道,则$O_x$中元素个数$|O_x| = [G: S_x]$,因而
\begin{align*}
|G|=\left|O_x \right|\left| S_x \right|.
\end{align*}
如果$X$还是有限集,则
\begin{align}\label{eq::o2hir229jiofaijoddfijozsfijoszshfuw}
|X|=\sum_{x\in X}[G:S_{x_i}],
\end{align}
其中$x$取遍$X$的不同轨道的代表元素.公式\eqref{eq::o2hir229jiofaijoddfijozsfijoszshfuw}称为\textbf{轨道公式}.
\end{corollary}
\begin{proof}
由\rrefthe{theorem:抽象代数--定理4.2.2}{theorem:抽象代数--定理4.2.2-1}知$G$在$O_x$上作用可递,故由\refthe{theorem:抽象代数--定理4.2.3}知,$G$在$X$上的作用与$G$在$G/S_x$上的作用等价,即存在$X$到$G/S_x$的双射.因此$|O_x| = [G: S_x]$.再由\hyperref[theorem:抽象代数--Lagrange定理]{Lagrange定理}知
\begin{align*}
\left| G \right|=\left[ G:S_x \right] \left| S_x \right|=\left| O_x \right|\left| S_x \right|,
\end{align*}
再由\rrefthe{theorem:抽象代数--定理4.2.2}{theorem:抽象代数--定理4.2.2-2}可得
\begin{align*}
|X|=\sum_{x\in X}[G:S_{x_i}],
\end{align*}
其中$x$取遍$X$的不同轨道的代表元素.

\end{proof}

\begin{theorem}\label{theorem:伴随作用下ad的定义}
设$G$是一个群,在伴随作用下.
\begin{enumerate}[(1)]
\item\label{theorem:伴随作用下ad的定义-1} 对$\forall g\in G$,定义$G$上的变换$\text{ad}g$满足
\begin{align}\label{eq::--whiruhfeu8f2389jiosEAFgfeg}
\text{ad}g(x) = gxg^{-1}, \quad \forall x \in G.
\end{align}
则$\text{ad}g$是$G$的可逆变换,并且
\begin{align*}
S_{g(x)} = \text{ad}g(S_x) = gS_xg^{-1}.
\end{align*}

\item\label{theorem:伴随作用下ad的定义-2} 定义$G$到$S_G$的映射$\text{ad}$满足
\begin{align*}
\text{ad}:g \to \text{ad}g,\quad \forall g\in G.
\end{align*}
其中$\text{ad}g$的定义如\eqref{eq::--whiruhfeu8f2389jiosEAFgfeg}式.
则$\text{ad}$是$G$到$S_G$的同态.
\end{enumerate}
\end{theorem}
\begin{proof}
\begin{enumerate}[(1)]
\item 由\rrefthe{theorem:抽象代数--定理4.2.1}{theorem:抽象代数--定理4.2.1-1}知映射$\text{ad}g$是$G$的可逆变换,即$\text{ad}\in S_G$.

设$g(x) = y$且$g_1 \in S_y$,即有$y = g_1(y)$,则$g_1g(x) = g(x)$,因而$g_2 = g^{-1}g_1g \in S_x$,故$g_1 = gg_2g^{-1} \in \text{ad}g(S_x)$。反之,若$g_2 \in S_x$,则有
\begin{align*}
gg_2g^{-1}(y) = gg_2g^{-1}(g(x)) = g(x) = y,
\end{align*}
故$gg_2g^{-1} \in S_y$。这样就证明了$S_y = \text{ad}g(S_x)$。

\item 由\rrefthe{theorem:抽象代数--定理4.2.1}{theorem:抽象代数--定理4.2.1-2}知映射$\text{ad}$是$G$到$S_G$的同态.
\end{enumerate}

\end{proof}

\begin{definition}\label{definition:共轭和中心化子定义}
设$G$是一个群,$g \in G$, $g$在伴随作用下的轨道称为以$g$为代表的\textbf{共轭类},记为$C_g$.若$h \in C_g$,则称$h$与$g$\textbf{共轭}.

$g$在伴随作用下的迷向子群,称为$g$在$G$中的\textbf{中心化子},记作$C_G(g)$.在不混淆时,简称为$g$的\textbf{中心化子},记作$C(g)$.若$H\subseteq G$,则将$H$中每个元素在$G$中的中心化子之交
\begin{align*}
C_G\left( H \right) =\bigcap_{h\in H}{C_G\left( h \right)}=\bigcap_{h\in H}{F_h}
\end{align*}
称为$H$在$G$中的\textbf{中心化子}.

在伴随作用下,对$\forall g\in G$,定义$G$上的\hyperref[theorem:伴随作用下ad的定义]{可逆变换}$\text{ad}g$满足
\begin{align*}
\text{ad}g(x) = gxg^{-1}, \quad \forall x \in G.
\end{align*}
称$\text{ad}g$为群$G$的\textbf{共轭变换},$x$在共轭变换下的像$gxg^{-1}$称为$x$的\textbf{共轭元}.

再定义$G$到$S_G$的\hyperref[theorem:伴随作用下ad的定义]{同态}$\text{ad}$满足
\begin{align*}
\text{ad}:g \to \text{ad}g,\quad \forall g\in G.
\end{align*}
称$\ker \text{ad}$为$G$的\textbf{中心},记作$C(G)$.
\end{definition}

\begin{theorem}\label{theorem:共轭类,中心化子,中心的集合形式}
设$G$是一个群,$H\subseteq G$,$g\in G$,在伴随作用下,有
\begin{enumerate}[(1)]
\item\label{theorem:共轭类,中心化子,中心的集合形式-1} $C_g = \{kgk^{-1} \mid k \in G\}$.

\item\label{theorem:共轭类,中心化子,中心的集合形式-2} $g,h\in G\text{共轭}\Longleftrightarrow \exists k\in G,\text{使}h=kgk^{-1}$.

\item\label{theorem:共轭类,中心化子,中心的集合形式-3} $C_G(g) = C(g) = \left\{ k\in G\mid kgk^{-1}=g \right\} = \{k \in G \mid kg = gk\}$.

\item\label{theorem:共轭类,中心化子,中心的集合形式-5} $C_G(H) = \{g\in G\mid hgh^{-1}=g\} = \{g\in G\mid hg = gh,\forall \in H\}$.

\item\label{theorem:共轭类,中心化子,中心的集合形式-4} $C(G) =\left\{ k\in G\mid kgk^{-1}=g,\forall g\in G \right\}  = \{k \in G \mid kg = gk, \forall g \in G\}$.
\end{enumerate}
\end{theorem}
\begin{proof}
\begin{enumerate}[(1)]
\item 由定义知
\begin{align*}
C_g=\left\{ \mathrm{ad}k\left( g \right) \in G\mid k\in G \right\} =\left\{ kgk^{-1}\in G\mid k\in G \right\} .
\end{align*}

\item 由\rrefthe{theorem:共轭类,中心化子,中心的集合形式}{theorem:共轭类,中心化子,中心的集合形式-1}知
\begin{align*}
C_g =\left\{ kgk^{-1}\in G\mid k\in G \right\} ,
\end{align*}
则
\begin{align*}
g,h\in G\text{共轭}\Longleftrightarrow h\in C_g\Longleftrightarrow \exists k\in G,\text{使}h=kgk^{-1}.
\end{align*}

\item 由定义知
\begin{align*}
C_G\left( H \right) &=\bigcap_{h\in H}{C_G\left( h \right)}=\bigcap_{h\in H}{F_h}=\bigcap_{h\in H}{\left\{ g\in G\mid ghg^{-1}=h \right\}}\\
&=\left\{ g\in G\mid ghg^{-1}=h,\forall h\in H \right\} =\left\{ g\in G\mid gh=gh,\forall h\in H \right\} .
\end{align*}

\item 由定义知
\begin{align*}
C_G(g)=C(g)=\left\{ k\in G\mid \mathrm{ad}k\left( g \right) =g \right\} =\left\{ k\in G\mid kgk^{-1}=g \right\} =\{k\in G\mid kg=gk\}.
\end{align*}

\item 由定义知
\begin{align*}
C(G)&=\mathrm{ker}\,\mathrm{ad}=\left\{ k\in G\mid \mathrm{ad}\left( k \right) =\mathrm{id}_G \right\} =\left\{ k\in G\mid \mathrm{ad}k=\mathrm{id}_G \right\} 
\\
&=\left\{ k\in G\mid kgk^{-1}=g,\forall g\in G \right\} =\left\{ k\in G\mid kg=gk,\forall g\in G \right\} .
\end{align*}
\end{enumerate}

\end{proof}

\begin{theorem}\label{theorem:中心化子的基本性质}
设$G$是一个群,$g\in G$,则有
\begin{enumerate}[(1)]
\item $C(g)=C(g^{-1}).$

\item\label{theorem:中心化子的基本性质-1} $C(G)$是$G$的正规子群且$\text{ad}G\cong G/C(G)$.

\item\label{theorem:中心化子的基本性质-2} $G$中共轭关系为等价关系,因而$G$的共轭类的集合是$G$的一个分划.即$X$是所有不同轨道的并
\begin{align*}
X=\bigsqcup_x O_x,
\end{align*}
其中$x$取遍$X$的不同共轭类的代表元素.

\item\label{theorem:中心化子的基本性质-3} 若$G$是有限群,则$|C_g| = [G: C(g)]$,并且
\begin{align}\label{eq::--hj2i209jioss0j3jiojjapja}
|G| = |C(G)| + \sum_{x\in G}[G:C(x)],
\end{align}
其中$x$取遍非中心的元素的共轭类的代表元.公式\eqref{eq::--hj2i209jioss0j3jiojjapja}称为\textbf{群方程}.

\item\label{theorem:中心化子的基本性质-4} $h \in C(G)\iff |C_h| = 1 \iff h \in \bigcap\limits_{g \in G} C(g)$.进而$C(G)=\bigcap\limits_{g \in G} C(g).$
\end{enumerate}
\end{theorem}
\begin{proof}
\begin{enumerate}[(1)]
\item 只需注意到
\begin{align*}
C\left( g \right) =\left\{ k\in G\mid kgk^{-1}=g \right\} =\left\{ k\in G\mid \left( kgk^{-1} \right) ^{-1}=g^{-1} \right\} =\left\{ k\in G\mid kg^{-1}k^{-1}=g^{-1} \right\} =C\left( g^{-1} \right) .
\end{align*}

\item 由\rrrefthe{theorem:抽象代数--定理4.2.1}{theorem:抽象代数--定理4.2.1-2}{theorem:抽象代数--定理4.2.1-2-2}知$C(G)\lhd G$.再由\hyperref[theorem:群的同态基本定理]{群的同态基本定理}知$\text{ad}G$与$G/C(G)$同构.

\item 由\rrefthe{theorem:抽象代数--定理4.2.2}{theorem:抽象代数--定理4.2.2-1}即得.

\item 由\refcor{corollary:有限群轨道的阶与其商掉迷向子群的阶相同}得
\begin{align}\label{eq::--sejfij230fjjs0if90i90j2356564}
|G| = \sum_{x}[G:C(x)],
\end{align}
其中$x$取遍不同的共轭类的代表元素.又由于
\begin{align*}
[G:C(x)] = 1 \iff G = C(x) \iff x \in C(G),
\end{align*}
所以在\eqref{eq::--sejfij230fjjs0if90i90j2356564}式中把值为1的项加到求和号外面来即得
\begin{align*}
|G| = |C(G)| + \sum_{x}[G:C(x)],
\end{align*}
其中$x$取遍非中心的元素的共轭类的代表元.

\item 由\refthe{theorem:共轭类,中心化子,中心的集合形式}可得
\begin{align*}
&h\in C(G) \Longleftrightarrow h\in \left\{ k\in G\mid kg=gk,\forall g\in G \right\} \Longleftrightarrow hg=gh,\forall g\in G\\
&\Longleftrightarrow ghg^{-1}=h,\forall g\in G\Longleftrightarrow C_h=\left\{ ghg^{-1}\mid g\in G \right\} =\left\{ h \right\} \Longleftrightarrow |C_h|=1; 
\end{align*}
\begin{align*}
&h\in C(G) \Longleftrightarrow h\in \left\{ k\in G\mid kg=gk,\forall g\in G \right\} \Longleftrightarrow hg=gh,\forall g\in G \\
&\Longleftrightarrow h\in \left\{ k\in G\mid kg=gk \right\},\forall g\in G\Longleftrightarrow h\in C(g),\forall g\in G\Longleftrightarrow h\in \bigcap\limits_{g\in G}C(g).
\end{align*}
\end{enumerate}

\end{proof}

\begin{theorem}[Cauchy定理]\label{theorem:抽象代数--Cauchy定理}
设$G$为有限群,$|G|=n$,则对$n$的任一素因子$p$,$G$必有阶为$p$的元素.
\end{theorem}
\begin{proof}
对$n$应用数学归纳法.
当$n=2$时,结论显然成立.
假定结论对所有阶小于$n$的群成立.考察$n$阶群$G$的群方程:
\begin{align*}
|G| = |C(G)| + \sum_{i=1}^{t}[G:C(x_i)],
\end{align*}
其中$x_i$取遍非中心的元素的共轭类的代表元,$t$为$G$中所有不同轨道的个数.
\begin{enumerate}[(a)]
\item 如果$p\mid |C(G)|$,则由\refpro{proposition:有限交换群必有阶的任意素因子的阶的元素}知$C(G)$含有阶为$p$的元素,从而$G$含有阶为$p$的元素.

\item 如果$p\nmid |C(G)|$,则因为$p\mid |G|$,所以至少存在一个$x_i$,使$p\nmid [G:C(x_i)]$,即$p\nmid \frac{|G|}{|C(x_i)|}$.又$p\mid |G|$,故$p\mid |C(x_i)|$.因为$\frac{n}{|C(x_i)|}=\frac{|G|}{|C(x_i)|}=[G:C(x_i)]>1$,所以$|C(x_i)|<n$.由归纳假设知$C(x_i)$含有阶为$p$的元素,因此$G$含有阶为$p$的元素.
\end{enumerate}
从而由归纳法原理知结论成立.

\end{proof}

\begin{definition}
设群$G$作用在集合$X$上,$g\in G$,$x\in X$.
\begin{enumerate}[(1)]
\item 如果$g(x)=x$,则称$x$为$g$的一个\textbf{不动元素}(fixed element).$g$的全部不动元素的集合称为$g$的\textbf{不动元素集}(the set of fixed elements),记作$F_g$.

\item 如果对任意的$g\in G$,都有$g(x)=x$,则称$x$为$G$的一个\textbf{不动元素}.$G$的全部不动元素的集合称为$G$的\textbf{不动元素集},记作$F_G$.
\end{enumerate}
\end{definition}

\begin{theorem}[Burnside引理]\label{theorem:Burnside(伯恩赛德)引理}
设有限群$G$作用在有限集合$X$上,$n$表示$X$在$G$的作用下的不同轨道数,则
\begin{align}
n = \frac{1}{|G|}\sum_{g\in G}|F_g|,\label{eq:::--q8hro2h3r2.5.3}
\end{align}
其中$|F_g|$表示$g$的不动元素的个数.
\end{theorem}
\begin{proof}
对任意的$x\in X$,$g\in G$,定义
\begin{align*}
\delta(g,x) = \begin{cases}
1, & 如果\ g(x)=x, \\
0, & 如果\ g(x)\neq x.
\end{cases}
\end{align*}
由定义知
\begin{align*}
|F_g| = \sum_{x\in X}\delta(g,x),\quad |S_x| = \sum_{g\in G}\delta(g,x),
\end{align*}
则
\begin{align*}
\sum_{g\in G}|F_g| = \sum_{g\in G}\left(\sum_{x\in X}\delta(g,x)\right) = \sum_{x\in X}\left(\sum_{g\in G}\delta(g,x)\right) = \sum_{x\in X}|S_x|.
\end{align*}
如果$x\in O_{x_i}$,则$O_x = O_{x_i}$,从而由\refcor{corollary:有限群轨道的阶与其商掉迷向子群的阶相同}可得
\begin{align*}
|S_x| = \frac{|G|}{|O_x|} = \frac{|G|}{|O_{x_i}|} = |S_{x_i}|,
\end{align*}
所以,如果$x_1,x_2,\cdots,x_n$为$n$个不同轨道的代表元素,则由\rrefthe{theorem:抽象代数--定理4.2.2}{theorem:抽象代数--定理4.2.2-1}和\refcor{corollary:有限群轨道的阶与其商掉迷向子群的阶相同}可得
\begin{align*}
\sum_{g\in G}{|F_g|}&=\sum_{x\in X=\bigsqcup\limits_{i=1}^n{O_{x_i}}}{|S_x|}=\sum_{i=1}^n{\sum_{x\in O_{x_i}}{|S_x|}}
\\
&=\sum_{i=1}^n{\sum_{x\in O_{x_i}}{|S_{x_i}|}}=\sum_{i=1}^n{|O_{x_i}||S_{x_i}|}
\\
&=\sum_{i=1}^n{|G|}=n|G|.
\end{align*}
由此得
\begin{align*}
n = \frac{1}{|G|}\sum_{g\in G}|F_g|.
\end{align*}

\end{proof}

\begin{theorem}\label{theorem:共轭子群和正规化子}
设$H$是群$G$的子群,则
\begin{enumerate}[(1)]
\item $\forall g \in G, H_1 = gHg^{-1}$也是$G$的子群,并且$H_1\cong G$,称$H_1$为$H$的\textbf{共轭子群};

\item 群$G$的子集$N_G(H) = \{g \in G \mid gHg^{-1} = H\}$也是$G$的子群,而且$H \lhd N_G(H)$,$N_G(H)$称为$H$在$G$中的\textbf{正规化子},也简称为$H$的\textbf{正规化子}.
\end{enumerate}
\end{theorem}
\begin{proof}
\begin{enumerate}[(1)]
\item 以$e$表示$G$的幺元. 因为$e = geg^{-1} \in H_1$,故$H_1 \neq \varnothing$. 设$h_1, h_2 \in H$,于是
\begin{align*}
(gh_1g^{-1})(gh_2g^{-1}) = g(h_1h_2)g^{-1},\,\,(gh_1g^{-1})^{-1} = gh_1^{-1}g^{-1} \in H_1.
\end{align*}
由此知$H_1$是$G$的子群.

令
\begin{align*}
\phi: H &\to gHg^{-1} \\
x &\mapsto gxg^{-1}.
\end{align*}
显然, $\phi$为$H$到$gHg^{-1}$的一个良定义的映射.

设$x,y\in H$, 如果$\phi(x)=\phi(y)$, 即$gxg^{-1}=gyg^{-1}$, 则$x=y$, 所以$\phi$为$H$到$gHg^{-1}$的单映射.

对任意的$x\in gHg^{-1}$, 有$y=g^{-1}xg\in H$, 使
\begin{align*}
\phi(y) = gyg^{-1} = g(g^{-1}xg)g^{-1} = x,
\end{align*}
所以$\phi$为$H$到$gHg^{-1}$满映射.

对任意的$x,y\in H$, 有
\begin{align*}
\phi(xy) = g(xy)g^{-1} = (gxg^{-1})(gyg^{-1}) = \phi(x)\phi(y).
\end{align*}
所以$\phi$为$H$到$gHg^{-1}$的同构映射, 即
\begin{align*}
\phi: \ H \cong gHg^{-1}.
\end{align*}

\item $G$在$G$的伴随作用下,$H$的迷向子群恰为$N_G(H)$,故$N_G(H)$是$G$的子群. 因$H$是$G$的子群且$H \subseteq N_G(H)$,故$H$是$N_G(H)$的子群. 又$\forall g \in N_G(H), gHg^{-1} = H$,因此$H \lhd N_G(H)$.
\end{enumerate}

\end{proof}

\begin{theorem}\label{theorem:共轭子群的个数和正规化子的性质}
设$H$是群$G$的子群,则
\begin{enumerate}[(1)]
\item $G$中与$H$共轭的子群的个数为$[G : N_G(H)]$;

\item $H \lhd G\iff N_G(H) = G$.
\end{enumerate}
\end{theorem}
\begin{proof}
\begin{enumerate}[(1)]
\item 设$H$的共轭子群到$G/N_G(H)$的映射$f$,满足
\begin{align*}
f(gHg^{-1}) = gN_G(H),\quad \forall g\in G.
\end{align*}
显然$f$是满射.因为$gHg^{-1} = g_1Hg_1^{-1}$当且仅当$H=g^{-1}g_1H(g^{-1}g_1)^{-1}$当且仅当$g^{-1}g_1 \in N_G(H)$当且仅当$gN_G(H)=g_1N_G(H)$,所以$f$是单射.因此$f$是$H$的共轭子群到$G/N_G(H)$的双射.从而$G$中与$H$共轭的子群的个数与$|G/N|$相等,即$G$中与$H$共轭的子群的个数为$[G : N_G(H)]$.

\item 因为$H \lhd G$当且仅当$\forall g \in G, gHg^{-1} = H$,所以由$N_G(H)$的定义知$N_G(H) = G$.
\end{enumerate}

\end{proof}








\end{document}
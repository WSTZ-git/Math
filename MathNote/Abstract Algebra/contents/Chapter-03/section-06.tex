\documentclass[../../main.tex]{subfiles}
\graphicspath{{\subfix{./image/}}} % 指定图片目录,后续可以直接使用图片文件名
% 注意这里的文件路径不能用 ../../image/ ,否则用latexmk编译子文件会报错

% 例如:
% \begin{figure}[H]
% \centering
% \includegraphics[scale=0.4]{图.png}
% \caption{}
% \label{figure:图}
% \end{figure}
% 注意:上述\label{}一定要放在\caption{}之后,否则引用图片序号会只会显示??.

\begin{document}

\section{可解群和幂零群}

\begin{definition}[换位子]
设$g_1, g_2$是群$G$中的两个元素,称
\begin{align*}
[g_1, g_2] = g_1^{-1}g_2^{-1}g_1g_2
\end{align*}
为$g_1$与$g_2$的\textbf{换位子}.
\end{definition}
\begin{remark}
从换位子的定义即得
\begin{align*}
\alpha([g_1, g_2]) = [\alpha(g_1), \alpha(g_2)],\quad \forall \alpha \in \mathrm{Aut}G,\ g_1, g_2 \in G.
\end{align*}
\end{remark}

\begin{definition}[换位子群]
若$H, K$是群$G$的两个子群,称
\begin{align*}
[H, K] = \langle \{[h, k] \mid h \in H,\ k \in K\} \rangle
\end{align*}
为$H$与$K$的\textbf{换位子群}.
\end{definition}
\begin{remark}
从换位子群的定义即得
\begin{align*}
\alpha([H, K]) = [\alpha(H), \alpha(K)],\quad \forall \alpha \in \mathrm{Aut}G.
\end{align*}
\end{remark}

\begin{lemma}
设$H, K$是群$G$的子群,则有
\begin{enumerate}[(1)]
\item $[H, K] = \{1\}\iff H \subseteq C_G(K)$;

\item $[H, K] \subseteq K\iff H \subseteq N_G(K)$,
\\
$[H, K] \subseteq H\iff K \subseteq N_G(H)$;

\item 若$H \lhd G, K \lhd G$,则$[H, K] \lhd G$且$[H, K] \subseteq H \cap K$;

\item 当$H_1, K_1$分别为$H, K$的子群时有$[H_1, K_1] \subseteq [H, K]$.
\end{enumerate}
\end{lemma}
\begin{proof}
\begin{enumerate}[(1)]
\item $[H, K] = \{1\}$当且仅当对$\forall h \in H, k \in K$有
\begin{align*}
[h, k] = 1\iff h^{-1}k^{-1}hk=1\iff hk=kh\iff hkh^{-1}=k,
\end{align*}
即$h\in C_G(K),\forall h\in H$,即$H \subseteq C_G(K)$.

\item 先证$[H, K] \subseteq K\iff H \subseteq N_G(K)$.若$[H, K] \subseteq K$,则
\begin{align*}
[h, k] \in K,\quad [h^{-1}, k^{-1}] \in K,\quad \forall k \in K,\ h \in H.
\end{align*}
对$\forall h \in H$,设$k \in K$,则由$[h, k] \in K$知存在$k_1 \in K$,使
\begin{align*}
h^{-1}k^{-1}hk = k_1 \Longleftrightarrow hkk_1^{-1} = kh \Longleftrightarrow k = hkk_1^{-1}h^{-1} \in hKh^{-1},
\end{align*}
故$K \subseteq hKh^{-1}$. 再设$hkh^{-1} \in hKh^{-1}$,则由$[h^{-1}, k^{-1}] \in K$知存在$k_2 \in K$,使
\begin{align*}
hkh^{-1}k^{-1} = k_1 \Longleftrightarrow hkh^{-1} = kk_1 \in K,
\end{align*}
故$hKh^{-1} \subseteq K$. 因此$hKh^{-1} = K,,\forall h\in H.$即$h \in N_G(K),\forall h\in H$. 故$H \subseteq N_G(K)$.

反之,若$H \subseteq N_G(K)$,对$\forall h \in H,\ k \in K$,有$hKh^{-1} = K$,从而存在$h_1 \in H,\ k_1 \in K$,使
\begin{align*}
k = h_1k_1h_1^{-1} \Longleftrightarrow k^{-1} = h_1k_1^{-1}h_1^{-1}.
\end{align*}
于是
\begin{align*}
[h, k] = h^{-1}k^{-1}hk = h^{-1}h_1k_1^{-1}h_1^{-1}hk = (h^{-1}h_1)k_1^{-1}(h^{-1}h_1)^{-1}k.
\end{align*}
注意到$(h^{-1}h_1)k_1^{-1}(h^{-1}h_1)^{-1} \in hKh^{-1}$,所以存在$k_2 \in K$,使$(h^{-1}h_1)k_1^{-1}(h^{-1}h_1)^{-1} = k_2$. 从而
\begin{align*}
[h, k] = (h^{-1}h_1)k_1^{-1}(h^{-1}h_1)^{-1}k = k_2k \in K,\,\,\forall k\in K,h\in H.
\end{align*}
故$[H, K] \subseteq K$.

再证$[H, K] \subseteq H\iff K \subseteq N_G(H)$.
若$[H, K] \subseteq H$,则
\begin{align*}
[h, k] \in H,\quad [h^{-1}, k^{-1}] \in H,\quad \forall k \in K,\ h \in H.
\end{align*}
对$\forall k \in K$,设$h \in H$,则由$[h, k] \in H$知存在$h_1 \in H$,使
\begin{align*}
h^{-1}k^{-1}hk = h_1 \Longleftrightarrow hk = khh_1 \Longleftrightarrow h = khh_1k^{-1} \in kHk^{-1},
\end{align*}
故$H \subseteq kHk^{-1}$. 再设$khk^{-1} \in kHk^{-1}$,则由$[h^{-1}, k^{-1}] \in H$知存在$h_2 \in H$,使
\begin{align*}
hkh^{-1}k^{-1} = h_1 \Longleftrightarrow khk^{-1} = h^{-1}h_1 \in H,
\end{align*}
故$kHk^{-1} \subseteq H$. 因此$kHk^{-1} = H$,$\forall k \in K$. 即$k \in N_G(H)$,$\forall k \in K$. 故$K \subseteq N_G(H)$.

反之,若$K \subseteq N_G(H)$,对$\forall h \in H,\ k \in K$,有$kHk^{-1} = H$,从而存在$h_1 \in H,\ k_1 \in K$,使
\begin{align*}
h = k_1h_1k_1^{-1}.
\end{align*}
于是
\begin{align*}
[h, k] = h^{-1}k^{-1}hk = h^{-1}k^{-1}k_1h_1k_1^{-1}k = h^{-1}(k^{-1}k_1)h_1(k^{-1}k_1)^{-1}.
\end{align*}
注意到$(k^{-1}k_1)h_1(k^{-1}k_1)^{-1} \in kHk^{-1}$,所以存在$h_2 \in H$,使$(k^{-1}k_1)h_1(k^{-1}k_1)^{-1} = h_2$. 从而
\begin{align*}
[h, k] = h^{-1}(k^{-1}k_1)h_1(k^{-1}k_1)^{-1} = h^{-1}h_2 \in H.
\end{align*}
故$[H, K] \subseteq H$.

\item 设$H \lhd G, K \lhd G$,于是对$\forall \alpha=L_gR_{g^{-1}} \in \mathrm{Int}G$,有
\begin{align*}
g[H,K]g^{-1} = \alpha([H, K]) = [\alpha(H), \alpha(K)] = [gHg^{-1}, gKg^{-1}] = [H, K],
\end{align*}
即$[H, K] \lhd G$. 由$H \lhd G, K \lhd G$知
\begin{align*}
gHg^{-1} = H,\quad gKg^{-1} = K,\quad \forall g \in G.
\end{align*}
故
\begin{align*}
kHk^{-1} = H,\quad \forall k \in K;
\end{align*}
\begin{align*}
hKh^{-1} = K,\quad \forall h \in H.
\end{align*}
即$K \subseteq N_G(H),\ H \subseteq N_G(K)$. 再由结论(2)知$[H, K] \subseteq H \cap K$.

\item 此结论是显然的.
\end{enumerate}
\end{proof}

\begin{definition}
幺元为$1$的群$G$中的子群序列
\begin{align*}
G = G_1 \supset G_2 \supset \cdots \supset G_t \supset G_{t+1} = \{1\}.
\end{align*}
若满足$G_i \lhd G_{i-1}(2 \leqslant i \leqslant t+1)$,则称之为\textbf{次正规序列},$t$称为此序列的长度.$G_{i-1}/G_i(2 \leqslant i \leqslant t+1)$称为此序列的\textbf{因子}.

若在上述序列中有$G_i \lhd G(1 \leqslant i \leqslant t+1)$,则称此序列为\textbf{正规序列}.

若两个次正规序列(正规序列)
\begin{align*}
G = G_1' \supset G_2' \supset \cdots \supset G_r' \supset G_{r+1}' = \{1\},
\end{align*}
\begin{align*}
G = G_1 \supset G_2 \supset \cdots \supset G_t \supset G_{t+1} = \{1\}
\end{align*}
满足$\forall G_i'(1 \leqslant i \leqslant r+1), \exists G_{i_j} = G_i'(1 \leqslant i_j \leqslant t+1)$,则称此序列$\{G_j\}$是序列$\{G_i'\}$的\textbf{加细}.
\end{definition}

\begin{example}
$S_3 \supset A_3 \supset \{\mathrm{id}\}$是$S_3$的正规序列.
\end{example}

\begin{example}
$S_4 \supset A_4 \supset K_4 \supset \{\mathrm{id}\}$是$S_4$的正规序列,而$S_4 \supset A_4 \supset K_4 \supset \langle (12)(34) \rangle \supset \{\mathrm{id}\}$是$S_4$的次正规序列,后者是前者作为次正规序列的加细.
\end{example}

\begin{definition}
在群$G$中分别归纳地定义$\{G^{(k)}\}$,$\{\Gamma_k(G)$(或$G^k)\}$,$\{C_k(G)\}$为
\begin{align*}
G^{(0)} = G,\quad G^{(k)} = [G^{(k-1)}, G^{(k-1)}],\quad k > 0;
\end{align*}
\begin{align*}
\Gamma_1(G) = G,\quad \Gamma_k(G) = [G, \Gamma_{k-1}(G)],\quad k > 1;
\end{align*}
\begin{align*}
C_0(G) = \{1\},\quad C_k(G)/C_{k-1}(G) = C(G/C_{k-1}(G)),\quad k > 0.
\end{align*}
分别称群$G$中序列
\begin{align*}
G = G^{(0)} \supseteq G^{(1)} \supseteq G^{(2)} \supseteq \cdots,
\end{align*}
\begin{align*}
G = \Gamma_1(G) \supseteq \Gamma_2(G) \supseteq \cdots,
\end{align*}
\begin{align*}
C_0(G) \subseteq C_1(G) \subseteq C_2(G) \subseteq \cdots
\end{align*}
为$G$的导出列、降中心列和升中心列.
\end{definition}

这里要指出$C_k(G)$是存在的. 显然$C_1(G) = C(G)$. 设$\pi_1$是$G$到$G/C_1(G)$上的自然同态. 令
\begin{align*}
C_2(G) = \pi_1^{-1}(C(G/C_1(G))),
\end{align*}
则$C_2(G) \lhd G$且
\begin{align*}
C_1(G) \subseteq C_2(G),\quad C_2(G)/C_1(G) = C(G/C_1(G)).
\end{align*}
再令$\pi_2$是$G$到$G/C_2(G)$上的自然同态,则
\begin{align*}
C_3(G) = \pi_2^{-1}(C(G/C_2(G))),
\end{align*}
如此进行下去,即得所有$C_k(G)$.

\begin{definition}
设$G$是群,若有$k$,使$G^{(k)} = \{1\}$,则称$G$是可解群. 若有$k$,使$\Gamma_k(G) = \{1\}$,则称$G$是幂零群.

此定义也适用于无限群.
\end{definition}

\begin{example}
Abel群是幂零群,也是可解群.
\end{example}

\begin{example}
设$G = S_3$,于是$G^{(1)} = \Gamma_2(G) = A_3$,因而$G^{(2)} = \{1\}$,但$\Gamma_3(G) = A_3 = \Gamma_2(G)$,故当$k \geqslant 2$时均有$\Gamma_k(G) = A_3 \neq \{1\}$,故$S_3$是可解群但不是幂零群.
\end{example}

\begin{theorem}
设群$G$是群$B$过群$A$的扩张,则$G$可解的充分必要条件是$A, B$都是可解群.
\end{theorem}

\begin{proof}
由$G$是$B$过$A$的扩张,故可假定$A \lhd G, B = G/A$. 又设$\pi$是$G$到$B$的自然同态.

若$G$可解,则$A^{(1)} = [A, A] \subseteq [G, G] = G^{(1)}$. 一般有$A^{(k)} \subseteq G^{(k)}$,故$A$可解. 又$\forall a, b \in G$有
\begin{align*}
\pi(aba^{-1}b^{-1}) = \pi(a)\pi(b)\pi(a)^{-1}\pi(b)^{-1},
\end{align*}
故$\pi(G^{(1)}) = B^{(1)}$. 一般有$\pi(G^{(k)}) = B^{(k)}$,故$B$可解.

反之,若$A, B$可解,则存在$k_1, k_2$,使$A^{(k_1)} = B^{(k_2)} = \{1\}$,故$\pi(G^{(k_2)}) = \{1\}$,即$G^{(k_2)} \subseteq A$,因而$G^{(k_1 + k_2)} = \{1\}$,于是$G$可解.
\end{proof}

\begin{corollary}
可解群的子群,同态像必是可解群.
\end{corollary}

\begin{theorem}
设$G$是群,则下列条件等价:
\begin{enumerate}[(1)]
\item $G$是可解群;
\item 存在$G$的正规序列
\begin{align*}
G = G_1 \supset G_2 \supset \cdots \supset G_r = \{1\},
\end{align*}
使$G_i/G_{i+1}$为Abel群,$1 \leqslant i \leqslant r - 1$;
\item 存在$G$的次正规序列
\begin{align*}
G = G_1' \supset G_2' \supset \cdots \supset G_s' = \{1\},
\end{align*}
使$G_i'/G_{i+1}'$为Abel群,$1 \leqslant i \leqslant s - 1$;
\item 存在$G$的次正规序列
\begin{align*}
G = G_1'' \supset G_2'' \supset \cdots \supset G_t'' = \{1\},
\end{align*}
使$G_i''/G_{i+1}''$为素数阶群,$1 \leqslant i \leqslant t - 1$.
\end{enumerate}
\end{theorem}
\begin{proof}
(1) $\Rightarrow$ (2). 由于$G$可解,故有$k$,使$G^{(k)} = \{1\}$,因而$G$中有正规序列
\begin{align*}
G \supset G^{(1)} \supset G^{(2)} \supset \cdots \supset G^{(k)} = \{1\}.
\end{align*}
因$G^{(i)} = [G^{(i-1)}, G^{(i-1)}]$,又由4.1节的习题4知$G^{(i-1)}/G^{(i)}$是Abel群,故$G$的导出列满足2)的要求.

(2) $\Rightarrow$ (3). 由于正规序列必为次正规序列,故条件2)成立一定有条件3)成立.

(3) $\Rightarrow$ (4). 设$G = G_1' \supset G_2' \supset \cdots \supset G_s' = \{1\}$是$G$的次正规序列,并且$G_i'/G_{i+1}'$为Abel群. 由于$G$是有限群,故$G_i'/G_{i+1}'$也是有限群. 如果对某个$i$有
\begin{align*}
|G_i'/G_{i+1}'| = p_1^{a_1}p_2^{a_2} \cdots p_k^{a_k},
\end{align*}
其中,$k \geqslant 1$,$p_1, p_2, \cdots, p_k$是互不相等的素数.$\sum\limits_{j=1}^k a_j > 1$. 令$P_j$是$G_i'/G_{i+1}'$的Sylow $p_j$子群. 显然,$P_j$是$G_i'/G_{i+1}'$的正规子群. 由4.5节的习题11知有直积分解
\begin{align*}
G_i'/G_{i+1}' = P_1 \otimes P_2 \otimes \cdots \otimes P_k.
\end{align*}
又设$P_k'$为$P_k$的$p_k^{a_k - 1}$阶子群,则$G_i'/G_{i+1}'$中有正规子群$H' = P_1 \otimes P_2 \otimes \cdots \otimes P_{k-1} \otimes P_k'$且$[G_i'/G_{i+1}' : H'] = p_k$,设$\pi$为$G_i'$到$G_i'/G_{i+1}'$上的自然同态,令$H = \pi^{-1}(H')$,于是由群的同态基本定理知
\begin{align*}
G_i' \lhd H \lhd G_{i+1}'
\end{align*}
且$[G_i' : H] = p_k$,$[H : G_{i+1}'] = p_1^{a_1}p_2^{a_2} \cdots p_{k-1}^{a_{k-1}}p_k^{a_k - 1}$,于是$G_i'/H$是素数阶群,$H/G_{i+1}'$仍是Abel群.

(4) $\Rightarrow$ (1). 因$G_{t-1}''$为素数阶群,故为循环群,因而可解,而$G_{t-2}''$是循环群过循环群的扩张,由定理4.6.1知$G_{t-2}''$是可解群. 设已证$G_{i+1}''$为可解群,则$G_i''$是循环群过可解群的扩张,故由定理4.6.1知$G_i''$是可解群,故当$i = 1$时知$G$为可解群.
\end{proof}

\begin{example}
在$A_4$中有正规序列
\begin{align*}
A_4 \supset K_4 \supset \{\mathrm{id}\}.
\end{align*}
$A_4/K_4$是3阶群,$K_4$是Abel群,于是$A_4$是可解群.
\end{example}

\begin{theorem}
若群$G$是幂零群,则$G$的子群与同态像也是幂零群. 反之,幂零群的中心扩张或幂零群过幂零群的平凡扩张是幂零群.
\end{theorem}

\begin{proof}
设$A$是$G$的子群,由数学归纳法可证得$\Gamma_k(A) \subseteq \Gamma_k(G)(\forall k \in \mathbb{N})$. 由此知$G$为幂零群则$A$必为幂零群. 又若$f$是$G$到$G_1$上的同态. 用数学归纳法可证明$\Gamma_k(G_1) = f(\Gamma_k(G))$,于是$G$为幂零群可得$G_1$也是幂零群.

现设$G$是$B$过$A$的中心扩张,即$A \subseteq C(G), G/A = B$,又设$\pi$是$G$到$B$上的自然同态. 由$B$幂零有
\begin{align*}
\pi(\Gamma_{k_1}(G)) = \Gamma_{k_1}(B) = \{1\},
\end{align*}
因而$\Gamma_{k_1}(G) \subseteq A \subseteq C(G)$. 故$\forall a \in \Gamma_{k_1}(G), b \in G$有$[a, b] = 1$. 于是$\Gamma_{k_1 + 1}(G) = [G, \Gamma_{k_1}(G)] = \{1\}$,这就证明了$G$是幂零群.

最后,设$G = A \otimes B$为群的直积且$A, B$都是幂零群,由$\forall a_i \in A, b_i \in B(i = 1, 2)$,有$[a_i, b_i] = 1$. 于是$[a_1b_1, a_2b_2] = [a_1, a_2][b_1, b_2] \in [A, A][B, B]$,因而有$[G, G] = [A, A] \otimes [B, B]$. 由数学归纳法可证
\begin{align*}
\Gamma_k(G) = \Gamma_k(A) \otimes \Gamma_k(B).
\end{align*}
于是由$A, B$是幂零群可得$G$为幂零群.
\end{proof}

\begin{theorem}
设$G$是群,则下列条件等价:
\begin{enumerate}[(1)]
\item $G$是一个幂零群;
\item $G$中有正规序列
\begin{align*}
G = G_1 \supset G_2 \supset \cdots \supset G_r = \{1\},
\end{align*}
使$G_i/G_{i+1} \subseteq C(G/G_{i+1})$,$1 \leqslant i \leqslant r - 1$;
\item 存在$k$,使得$C_k(G) = G$.
\end{enumerate}
\end{theorem}
\begin{proof}
(1)$\Rightarrow$ (2). 因$G$是幂零群,故有$k$,使$\Gamma_k(G) = \{1\}$,于是$G$中有正规序列
\begin{align*}
G = \Gamma_1(G) \supset \Gamma_2(G) \supset \cdots \supset \Gamma_k(G) = \{1\},
\end{align*}
$\Gamma_{i+1}(G) \lhd G$,因而$\Gamma_i(G)/\Gamma_{i+1}(G) \subseteq G/\Gamma_{i+1}(G)$. 设$\pi$为$G$到$G/\Gamma_{i+1}(G)$的自然同态,由$[G, \Gamma_i(G)] = \Gamma_{i+1}(G)$知
\begin{align*}
[G/\Gamma_{i+1}(G), \Gamma_i(G)/\Gamma_{i+1}(G)] = [\pi(G), \pi(\Gamma_i(G))] = \pi(\Gamma_{i+1}(G)) = \{1\},
\end{align*}
因而$\Gamma_i(G)/\Gamma_{i+1}(G) \subseteq C(G/\Gamma_{i+1}(G))$,故$G$的降中心列满足条件2)的要求.

(2) $\Rightarrow$ (3). 用反序归纳法证明$G_i \subseteq C_{r - i}(G)$,其中,$C_0(G) = \{1\}$. 当$i = r$时,$G_r = \{1\} = C_0(G) = C_{r - r}(G)$. 设$i + 1$时已成立,因而$G_{i+1} \subseteq C_{r - (i+1)}(G)$,故$G_{i+1}C_{r - (i+1)}(G) = C_{r - (i+1)}(G)$. 又$G_i/G_{i+1} \subseteq C(G/G_{i+1})$,由此知$\forall a \in G_i, b \in G$有$aba^{-1}b^{-1} \in G_{i+1} \subseteq C_{r - (i+1)}(G)$. 因而$a \in C_{r - i}(G)$,故$G_i \subseteq C_{r - i}(G)$. 特别地,有$G_1 = G \subseteq C_{r - 1}(G)$,即取$k = r - 1$知条件3)成立.

(3) $\Rightarrow$ (1). 设有$k$,使$C_k(G) = G$,于是$G$中有正规序列
\begin{align*}
G = C_k(G) \supset C_{k-1}(G) \supset \cdots \supset C_1(G) \supset C_0(G) = \{1\}.
\end{align*}
用数学归纳法证明$\Gamma_i(G) \subseteq C_{k - i + 1}(G)$. 当$i = 1$时,显然成立. 由于$C_{k - i + 1}(G)/C_{k - i}(G) = C(G/C_{k - i}(G))$有$[G, C_{k - i + 1}(G)] \subseteq C_{k - i}(G)$,于是
\begin{align*}
\Gamma_{i+1}(G) = [G, \Gamma_i(G)] \subseteq [G, C_{k - i + 1}(G)] \subseteq C_{k - i}(G).
\end{align*}
特别地,有$\Gamma_{k+1}(G) \subseteq C_0(G) = \{1\}$,因而$G$是幂零群.
\end{proof}

\begin{theorem}
设$p$是一个素数,则有限$p$群$G$是幂零群.
\end{theorem}
\begin{proof}
从\refthe{theorem:抽象代数--定理4.3.1}知$C(G) \neq \{1\}$,故$G$是$G/C(G)$过$C(G)$的中心扩张,而$|G/C(G)| < |G|$,由数学归纳法可证得$G$为幂零群.
\end{proof}

\begin{example}
设$H$是四元数体,$G = \{\pm 1, \pm i, \pm j, \pm k\}$,其中,$1, i, j, k \in H$如1.5节的习题10所述,则$G$是8阶群. 这是一个非Abel幂零群的例子.
\end{example}












\end{document}
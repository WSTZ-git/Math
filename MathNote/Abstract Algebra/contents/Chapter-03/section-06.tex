\documentclass[../../main.tex]{subfiles}% 注意这里的文件路径不能用 ./main.tex ,否则用latexmk编译子文件会报错
\graphicspath{{\subfix{./image/}}} % 指定图片目录,后续可以直接使用图片文件名
% 注意这里的文件路径不能用 ../../image/ ,否则用latexmk编译子文件会报错

% 例如:
% \begin{figure}[H]
% \centering
% \includegraphics[scale=0.3]{图.png}
% \caption{}
% \label{figure:图}
% \end{figure}
% 注意:上述\label{}一定要放在\caption{}之后,否则引用图片序号会只会显示??.

\begin{document}

\section{可解群和幂零群}

\begin{definition}
幺元为$1$的群$G$中的子群序列
\begin{align*}
G = G_1 \supseteq G_2 \supseteq \cdots \supseteq G_t \supseteq G_{t+1} = \{1\}.
\end{align*}
若满足$G_i \lhd G_{i-1}(2 \leqslant i \leqslant t+1)$,则称之为\textbf{次正规序列},$t$称为此序列的\textbf{长度}.$G_{i-1}/G_i(2 \leqslant i \leqslant t+1)$称为此序列的\textbf{因子}.
若在上述序列中还有$G_i \lhd G(1 \leqslant i \leqslant t+1)$,则称此序列为\textbf{正规序列}.

若两个次正规序列(正规序列)
\begin{align*}
G = G_1' \supseteq G_2' \supseteq \cdots \supseteq G_r' \supseteq G_{r+1}' = \{1\},
\end{align*}
\begin{align*}
G = G_1 \supseteq G_2 \supseteq \cdots \supseteq G_t \supseteq G_{t+1} = \{1\}
\end{align*}
满足$\forall G_i'(1 \leqslant i \leqslant r+1), \exists G_{i_j} = G_i'(1 \leqslant i_j \leqslant t+1)$,则称此序列$\{G_j\}$是序列$\{G_i'\}$的\textbf{加细}.
\end{definition}

\begin{example}
$S_3 \supset A_3 \supset \{\mathrm{id}\}$是$S_3$的正规序列.
\end{example}
\begin{proof}


\end{proof}

\begin{example}
$S_4 \supset A_4 \supset K_4 \supset \{\mathrm{id}\}$是$S_4$的正规序列,而$S_4 \supset A_4 \supset K_4 \supset \langle (12)(34) \rangle \supset \{\mathrm{id}\}$是$S_4$的次正规序列,后者是前者作为次正规序列的加细.
\end{example}
\begin{proof}


\end{proof}

\begin{definition}
在群$G$中分别归纳地定义$G$的子群序列$\{G^{(k)}\}$,$\{\Gamma_k(G)\}$,$\{C_k(G)\}$为
\begin{align*}
G^{(0)} = G,\quad G^{(k)} = [G^{(k-1)}, G^{(k-1)}],\quad k > 0;
\end{align*}
\begin{align*}
\Gamma_1(G) = G,\quad \Gamma_k(G) = [G, \Gamma_{k-1}(G)],\quad k > 1;
\end{align*}
\begin{align*}
C_0(G) = \{1\},\quad C_k(G)/C_{k-1}(G) = C(G/C_{k-1}(G)),\quad k > 0.
\end{align*}
分别称群$G$中序列
\begin{align*}
G = G^{(0)} \supseteq G^{(1)} \supseteq G^{(2)} \supseteq \cdots,
\end{align*}
\begin{align*}
G = \Gamma_1(G) \supseteq \Gamma_2(G) \supseteq \cdots,
\end{align*}
\begin{align*}
C_0(G) \subseteq C_1(G) \subseteq C_2(G) \subseteq \cdots
\end{align*}
为$G$的\textbf{导出列}、\textbf{降中心列}和\textbf{升中心列}.

若有$k\in \mathbb{N}$,使$G^{(k)} = \{1\}$,则称$G$是\textbf{可解群}. 若有$k\in \mathbb{N}$,使$\Gamma_k(G) = \{1\}$,则称$G$是\textbf{幂零群}.
\end{definition}
\begin{remark}
显然有
\begin{align*}
G^{(0)} = G \supseteq [G, G] = G^{(1)}.
\end{align*}
假设$G^{(k-1)} \supseteq G^{(k)}$,则由\rreflem{lemma:换位子群的基本性质}{lemma:换位子群的基本性质-4}可得
\begin{align*}
G^{(k)} = [G^{(k-1)}, G^{(k-1)}] \supseteq [G^{(k)}, G^{(k)}] = G^{(k+1)}.
\end{align*}
故由数学归纳法知
\begin{align*}
G = G^{(0)} \supseteq G^{(1)} \supseteq G^{(2)} \supseteq \cdots.
\end{align*}
因此$G$的导出列是良定义的.

显然有
\begin{align*}
\Gamma_1(G) = G \supseteq [G, G] = \Gamma_2(G).
\end{align*}
假设$\Gamma_{k-1}(G) \supseteq \Gamma_k(G)$,则由\rreflem{lemma:换位子群的基本性质}{lemma:换位子群的基本性质-4}可得
\begin{align*}
\Gamma_k(G) = [G, \Gamma_{k-1}(G)] \supseteq [G, \Gamma_k(G)] = \Gamma_{k+1}(G).
\end{align*}
故由数学归纳法知
\begin{align*}
G = \Gamma_1(G) \supseteq \Gamma_2(G) \supseteq \cdots.
\end{align*}
因此$G$的降中心列是良定义的.

由\rrefthe{theorem:中心化子的基本性质}{theorem:中心化子的基本性质-1}知$C_1(G) = C(G)\lhd G$.设$\pi_1$是$G$到$G/C_1(G)$上的自然同态. 令
\begin{align*}
C_2(G) = \pi_1^{-1}(C(G/C_1(G))),
\end{align*}
则由\rrefthe{theorem:中心化子的基本性质}{theorem:中心化子的基本性质-1}知$\pi _1\left( C_2\left( G \right) \right) =C\left( G/C_1\left( G \right) \right) \lhd G/C_1\left( G \right),$
再由\rrefthe{theorem:抽象代数-定理1.7.2}{theorem:抽象代数-定理1.7.2-2}知$C_2(G)\lhd \pi _{1}^{-1}\left( G/C_1\left( G \right) \right) =G$.
因为$\pi_1(C_1(G)) = \{1\} \subseteq C(G/C_1(G))$,所以
\begin{align*}
C_1(G) \subseteq \pi_1^{-1}(C(G/C_1(G))) = C_2(G).
\end{align*}
从而由\rrefpro{proposition:正规子群的基本性质}{proposition:正规子群的基本性质-2}知$C_1(G)\lhd C_2(G)$.
将$\pi_1$看作限制在$C_2(G)$上的同态,则由\hyperref[theorem:群的同态基本定理]{群的同态基本定理}知
\begin{align*}
C_2(G)/C_1(G) = C_2(G)/\mathrm{ker}\pi_1|_{C_2(G)} \cong \pi_1|_{C_2(G)}(C_2(G)) = C(G/C_1(G)).
\end{align*}
因此
\begin{align*}
C_2(G)/C_1(G) = C(G/C_1(G)).
\end{align*}
再令$\pi_2$是$G$到$G/C_2(G)$上的自然同态,则
\begin{align*}
C_3(G) = \pi_2^{-1}(C(G/C_2(G))),
\end{align*}
同理可得
\begin{align*}
C_2(G)\lhd C_3(G),\quad C_3(G)/C_2(G) = C(G/C_2(G)).
\end{align*}
如此进行下去,即得所有$C_k(G)$.因此$G$的升中心列是良定义的.
\end{remark}

\begin{proposition}\label{proposition:导出列的基本性质--抽象代数}
\begin{enumerate}[(1)]
\item\label{proposition:导出列的基本性质--抽象代数-1} 若$G$是一个群,则$G^{(k_1+k_2)}=(G^{(k_1)})^{(k_2)},\Gamma _{k_1+k_2}\left( G \right) =\Gamma _{k_2}\left( \Gamma _{k_1}\left( G \right) \right) ,\,\,\forall k_1,k_2\in \mathbb{N}.$

\item\label{proposition:导出列的基本性质--抽象代数-2} 若$A$是群$B$的子群,则$A^{(k)}\subseteq B^{(k)},\Gamma _k\left( A \right) \subseteq \Gamma _k\left( B \right) ,\,\,\forall k\in \mathbb{N}.$

\item\label{proposition:导出列的基本性质--抽象代数-3} 若$f$是群$G$到群$B$的同态,则$(f(G))^{(k)}=f(G^{(k)}),\Gamma_k(f(G)) = f(\Gamma_k(G)),\,\,\forall k\in \mathbb{N}.$

\item\label{proposition:导出列的基本性质--抽象代数-4} 设$A,B$是两个群,则
\begin{gather*}
\left( nA\times B \right) ^{\left( k \right)}=A^{\left( k \right)}\times B^{\left( k \right)},\quad \forall k\in \mathbb{N}.
\\
\Gamma _k\left( A\times B \right) =\Gamma _k\left( A \right) \times \Gamma _k\left( B \right),\quad \forall k\in \mathbb{N}.
\end{gather*}

\item\label{proposition:导出列的基本性质--抽象代数-5} 设$\pi_k$群是$G$到$G/C_k(G)$($k\in \mathbb{N}$)上的自然同态,则
\begin{align*}
C_{k+1}(G) = \pi_k^{-1}\left(C\left(G/C_k(G)\right)\right),\quad \forall k\in \mathbb{N}.
\end{align*}

\item\label{proposition:导出列的基本性质--抽象代数-6} 设$G$是一个群,则对$\forall a\in G$,有
\begin{align*}
a\in C_{k+1}(G) \Longleftrightarrow [a,b] \in C_k(G),\forall b\in G.
\end{align*}

\item\label{proposition:导出列的基本性质--抽象代数-7} 设$G$是一个群,则
\begin{gather*}
G^{\left( k+1 \right)}\lhd G^{\left( k \right)},\quad G^{\left( k \right)}\lhd G,\,\quad k=0,1,\cdots .
\\
\Gamma _{k+1}\left( G \right) \lhd \Gamma _k\left( G \right) ,\quad \Gamma _k\left( G \right) \lhd G,\,\quad k=0,1,\cdots .
\\
C_k\left( G \right) \lhd C_{k+1}\left( G \right) ,\quad C_k\left( G \right) \lhd G,\,\quad k=0,1,\cdots .
\end{gather*}
\end{enumerate}
\end{proposition}
\begin{proof}
\begin{enumerate}[(1)]
\item 对$\forall k_1\in \mathbb{N}$,都有
\begin{align*}
G^{(k_1+1)} = [G^{(k_1)}, G^{(k_1)}] = (G^{(k_1)})^{(1)}.
\end{align*}
假设$G^{(k_1+k_2-1)} = (G^{(k_1)})^{(k_2-1)}$,则
\begin{align*}
G^{(k_1+k_2)} = [G^{(k_1+k_2-1)}, G^{(k_1+k_2-1)}] = [(G^{(k_1)})^{(k_2-1)}, (G^{(k_1)})^{(k_2-1)}] = (G^{(k_1)})^{(k_2)}.
\end{align*}
故由数学归纳法知
\begin{align*}
G^{(k_1+k_2)} = (G^{(k_1)})^{(k_2)},\quad \forall k_2 \in \mathbb{N}.
\end{align*}
再由$k_1$的任意性可得
\begin{align*}
G^{(k_1+k_2)} = (G^{(k_1)})^{(k_2)},\quad \forall k_1,k_2 \in \mathbb{N}.
\end{align*}

对$\forall k_1\in \mathbb{N}$,都有
\begin{align*}
\Gamma _{k_1+1}\left( G \right) =\left[ G,\Gamma _{k_1}\left( G \right) \right] =\Gamma _1\left( \Gamma _{k_1}\left( G \right) \right).
\end{align*}
假设$\Gamma _{k_1+k_2-1}\left( G \right) =\Gamma _{k_2-1}\left( \Gamma _{k_1}\left( G \right) \right)$,则
\begin{align*}
\Gamma _{k_1+k_2}\left( G \right) =\left[ G,\Gamma _{k_1+k_2-1}\left( G \right) \right] =\left[ G,\Gamma _{k_2-1}\left( \Gamma _{k_1}\left( G \right) \right) \right] =\Gamma _{k_2}\left( \Gamma _{k_1}\left( G \right) \right).
\end{align*}
故由数学归纳法知
\begin{align*}
\Gamma _{k_1+k_2}\left( G \right) =\Gamma _{k_2}\left( \Gamma _{k_1}\left( G \right) \right),\quad \forall k_2\in \mathbb{N}.
\end{align*}
再由$k_1$的任意性可得
\begin{align*}
\Gamma _{k_1+k_2}\left( G \right) =\Gamma _{k_2}\left( \Gamma _{k_1}\left( G \right) \right),\quad \forall k_1,k_2\in \mathbb{N}.
\end{align*}

\item 由条件知$A^{\left( 0 \right)}=A\subseteq B=B^{\left( 0 \right)}$。假设$A^{\left( k-1 \right)}\subseteq B^{\left( k-1 \right)}$,则由\rreflem{lemma:换位子群的基本性质}{lemma:换位子群的基本性质-4}知
\begin{align*}
A^{\left( k \right)}=\left[ A^{\left( k-1 \right)},A^{\left( k-1 \right)} \right] \subseteq \left[ B^{\left( k-1 \right)},B^{\left( k-1 \right)} \right] =B^{\left( k \right)}.
\end{align*}
故由数学归纳法知
\begin{align*}
A^{\left( k \right)}\subseteq B^{\left( k \right)},\quad \forall k\in \mathbb{N}.
\end{align*}

由条件知$\Gamma _1\left( A \right) =A\subseteq B=\Gamma _1\left( B \right)$。假设$\Gamma _{k-1}\left( A \right) \subseteq \Gamma _{k-1}\left( B \right)$,则由\rreflem{lemma:换位子群的基本性质}{lemma:换位子群的基本性质-4}知
\begin{align*}
\Gamma _k\left( A \right) =\left[ A,\Gamma _{k-1}\left( A \right) \right] \subseteq \left[ B,\Gamma _{k-1}\left( B \right) \right] =\Gamma _k\left( B \right).
\end{align*}
故由数学归纳法知
\begin{align*}
\Gamma _k\left( A \right) \subseteq \Gamma _k\left( B \right),\quad \forall k\in \mathbb{N}.
\end{align*}

\item 显然$f\left( G^{\left( 0 \right)} \right) =f\left( G \right) =\left( f\left( G \right) \right) ^{\left( 0 \right)}.$假设$f\left( G^{\left( k-1 \right)} \right) = \left( f\left( G \right) \right) ^{\left( k-1 \right)}.$任取$f\left( a^{-1}b^{-1}ab \right) \in f\left( \left[ G^{\left( k-1 \right)},G^{\left( k-1 \right)} \right] \right) =f\left( G^{\left( k \right)} \right) ,$则$a,b\in G^{\left( k-1 \right)},$进而
\begin{align*}
f\left( a \right) ,f\left( b \right) \in f\left( G^{\left( k-1 \right)} \right) = \left( f\left( G \right) \right) ^{\left( k-1 \right)}.
\end{align*}
从而
\begin{align*}
f\left( a^{-1}b^{-1}ab \right) =f\left( a \right) ^{-1}f\left( b \right) ^{-1}f\left( a \right) f\left( b \right) \in \left[ \left( f\left( G \right) \right) ^{\left( k-1 \right)},\left( f\left( G \right) \right) ^{\left( k-1 \right)} \right] =\left( f\left( G \right) \right) ^{\left( k \right)}.
\end{align*}
故$f\left( G^{\left( k \right)} \right) \subseteq \left( f\left( G \right) \right) ^{\left( k \right)}.$
再任取$f\left( a \right) ^{-1}f\left( b \right) ^{-1}f\left( a \right) f\left( b \right) \in \left[ \left( f\left( G \right) \right) ^{\left( k-1 \right)},\left( f\left( G \right) \right) ^{\left( k-1 \right)} \right] =\left( f\left( G \right) \right) ^{\left( k \right)},$则
\begin{align*}
f\left( a \right) ,f\left( b \right) \in \left( f\left( G \right) \right) ^{\left( k-1 \right)}= f\left( G^{\left( k-1 \right)} \right) ,
\end{align*}
进而由\rrefpro{Set Theory-proposition:映射的基本性质-1}{Set Theory-proposition:映射的基本性质-1}知$a,b\in G^{\left( k-1 \right)}.$从而
\begin{align*}
f\left( a \right) ^{-1}f\left( b \right) ^{-1}f\left( a \right) f\left( b \right) =f\left( a^{-1}b^{-1}ab \right) \in f\left( \left[ G^{\left( k-1 \right)},G^{\left( k-1 \right)} \right] \right) =f\left( G^{\left( k \right)} \right) .
\end{align*}
故$f\left( G^{\left( k \right)} \right) \supseteq \left( f\left( G \right) \right) ^{\left( k \right)}.$因此$f\left( G^{\left( k \right)} \right) =\left( f\left( G \right) \right) ^{\left( k \right)}.$
故由数学归纳法知$f\left( G^{\left( k \right)} \right) = \left( f\left( G \right) \right) ^{\left( k \right)},\forall k\in \mathbb{N} .$

显然$\Gamma _1\left( f\left( G \right) \right) =f\left( G \right) =f\left( \Gamma _1\left( G \right) \right)$。假设$\Gamma _{k-1}\left( f\left( G \right) \right) =f\left( \Gamma _{k-1}\left( G \right) \right)$。任取$f\left( a^{-1}b^{-1}ab \right) \in f\left( \left[ G,\Gamma _{k-1}\left( G \right) \right] \right) =f\left( \Gamma _k\left( G \right) \right)$,则$a\in G,b\in \Gamma _{k-1}\left( G \right)$,进而
\begin{align*}
f\left( a \right) \in f\left( G \right),\quad f\left( b \right) \in f\left( \Gamma _{k-1}\left( G \right) \right) =\Gamma _{k-1}\left( f\left( G \right) \right).
\end{align*}
从而
\begin{align*}
f\left( a^{-1}b^{-1}ab \right) =f\left( a \right) ^{-1}f\left( b \right) ^{-1}f\left( a \right) f\left( b \right) \in \left[ f\left( G \right) ,\Gamma _{k-1}\left( f\left( G \right) \right) \right] =\Gamma _k\left( f\left( G \right) \right).
\end{align*}
因此$f\left( \Gamma _k\left( G \right) \right) \subseteq \Gamma _k\left( f\left( G \right) \right)$。再任取$f\left( a \right) ^{-1}f\left( b \right) ^{-1}f\left( a \right) f\left( b \right) \in \left[ f\left( G \right) ,\Gamma _{k-1}\left( f\left( G \right) \right) \right] =\Gamma _k\left( f\left( G \right) \right)$,则
\begin{align*}
f\left( a \right) \in f\left( G \right),\quad f\left( b \right) \in \Gamma _{k-1}\left( f\left( G \right) \right) =f\left( \Gamma _{k-1}\left( G \right) \right).
\end{align*}
进而由\rrefpro{Set Theory-proposition:映射的基本性质-1}{Set Theory-proposition:映射的基本性质-1}知$a\in G,b\in \Gamma _{k-1}\left( G \right)$。从而
\begin{align*}
f\left( a \right) ^{-1}f\left( b \right) ^{-1}f\left( a \right) f\left( b \right) =f\left( a^{-1}b^{-1}ab \right) \in f\left( \left[ G,\Gamma _{k-1}\left( G \right) \right] \right) =f\left( \Gamma _k\left( G \right) \right).
\end{align*}
因此$\Gamma _k\left( f\left( G \right) \right) \subseteq f\left( \Gamma _k\left( G \right) \right)$。综上可知$\Gamma _k\left( f\left( G \right) \right) =f\left( \Gamma _k\left( G \right) \right)$。故由数学归纳法可知
\begin{align*}
\Gamma _k\left( f\left( G \right) \right) =f\left( \Gamma _k\left( G \right) \right),\quad \forall k\in \mathbb{N}.
\end{align*}

\item 显然$\left( A\times B \right) ^{\left( 0 \right)}=A\times B=A^{\left( 0 \right)}\times B^{\left( 0 \right)}$。假设$\left( A\times B \right) ^{\left( k-1 \right)}=A^{\left( k-1 \right)}\times B^{\left( k-1 \right)}$,则
\begin{align}
\left( A\times B \right) ^{\left( k \right)}=\left[ \left( A\times B \right) ^{\left( k-1 \right)},\left( A\times B \right) ^{\left( k-1 \right)} \right] =\left[ A^{\left( k-1 \right)}\times B^{\left( k-1 \right)},A^{\left( k-1 \right)}\times B^{\left( k-1 \right)} \right].\label{eq:::--q89h2dfwfzznmnghmghkhlyujf21.2}
\end{align}
下证$\left[ A^{\left( k-1 \right)}\times B^{\left( k-1 \right)},A^{\left( k-1 \right)}\times B^{\left( k-1 \right)} \right] =A^{\left( k \right)}\times B^{\left( k \right)}$。任取$\left[ \left( a_1,b_1 \right) ,\left( a_2,b_2 \right) \right] \in \left[ A^{\left( k-1 \right)}\times B^{\left( k-1 \right)},A^{\left( k-1 \right)}\times B^{\left( k-1 \right)} \right]$,则
\begin{align*}
a_i\in A^{\left( k-1 \right)},\quad b_i\in B^{\left( k-1 \right)},\quad i=1,2.
\end{align*}
从而
\begin{align*}
\left[ \left( a_1,b_1 \right) ,\left( a_2,b_2 \right) \right] &=\left( a_{1}^{-1},b_{1}^{-1} \right) \left( a_{2}^{-1},b_{2}^{-1} \right) \left( a_1,b_1 \right) \left( a_2,b_2 \right) =\left( a_{1}^{-1}a_{2}^{-1}a_1a_2,b_{1}^{-1}b_{2}^{-1}b_1b_2 \right) \\
&=\left( \left[ a_1,a_2 \right] ,\left[ b_1,b_2 \right] \right) \in \left[ A^{\left( k-1 \right)},A^{\left( k-1 \right)} \right] \times \left[ B^{\left( k-1 \right)},B^{\left( k-1 \right)} \right] =A^{\left( k \right)}\times B^{\left( k \right)}.
\end{align*}
故$\left[ A^{\left( k-1 \right)}\times B^{\left( k-1 \right)},A^{\left( k-1 \right)}\times B^{\left( k-1 \right)} \right] \subseteq A^{\left( k \right)}\times B^{\left( k \right)}$。再任取$$\left( \left[ a_1,a_2 \right] ,\left[ b_1,b_2 \right] \right) \in A^{\left( k \right)}\times B^{\left( k \right)}=\left[ A^{\left( k-1 \right)},A^{\left( k-1 \right)} \right] \times \left[ B^{\left( k-1 \right)},B^{\left( k-1 \right)} \right],$$则
\begin{align*}
a_i\in A^{\left( k-1 \right)},\quad b_i\in B^{\left( k-1 \right)},\quad i=1,2.
\end{align*}
从而
\begin{align*}
\left( \left[ a_1,a_2 \right] ,\left[ b_1,b_2 \right] \right) &=\left( a_{1}^{-1}a_{2}^{-1}a_1a_2,b_{1}^{-1}b_{2}^{-1}b_1b_2 \right) =\left( a_{1}^{-1},b_{1}^{-1} \right) \left( a_{2}^{-1},b_{2}^{-1} \right) \left( a_1,b_1 \right) \left( a_2,b_2 \right) \\
&=\left[ \left( a_1,b_1 \right) ,\left( a_2,b_2 \right) \right] \in \left[ A^{\left( k-1 \right)}\times B^{\left( k-1 \right)},A^{\left( k-1 \right)}\times B^{\left( k-1 \right)} \right].
\end{align*}
故$A^{\left( k \right)}\times B^{\left( k \right)}\subseteq \left[ A^{\left( k-1 \right)}\times B^{\left( k-1 \right)},A^{\left( k-1 \right)}\times B^{\left( k-1 \right)} \right]$。因此$\left[ A^{\left( k-1 \right)}\times B^{\left( k-1 \right)},A^{\left( k-1 \right)}\times B^{\left( k-1 \right)} \right] =A^{\left( k \right)}\times B^{\left( k \right)}$。再结合\eqref{eq:::--q89h2dfwfzznmnghmghkhlyujf21.2}式可得
\begin{align*}
\left( A\times B \right) ^{\left( k \right)}=A^{\left( k \right)}\times B^{\left( k \right)}.
\end{align*}
故由数学归纳法可知
\begin{align*}
\left( A\times B \right) ^{\left( k \right)}=A^{\left( k \right)}\times B^{\left( k \right)},\quad \forall k\in \mathbb{N}.
\end{align*}

显然$\Gamma _1\left( A\times B \right) =A\times B=\Gamma _1\left( A \right) \times \Gamma _1\left( B \right)$。假设$\Gamma _{k-1}\left( A\times B \right) =\Gamma _{k-1}\left( A \right) \times \Gamma _{k-1}\left( B \right)$,则
\begin{align}
\Gamma _k\left( A\times B \right) =\left[ A\times B,\Gamma _{k-1}\left( A\times B \right) \right] =\left[ A\times B,\Gamma _{k-1}\left( A \right) \times \Gamma _{k-1}\left( B \right) \right].\label{eq:::--q89h2dfwfzznmnghmghkhlyujf21.1}
\end{align}
下证$\left[ A\times B,\Gamma _{k-1}\left( A \right) \times \Gamma _{k-1}\left( B \right) \right] =\Gamma _k\left( A \right) \times \Gamma _k\left( B \right)$。任取$\left[ \left( a_1,b_1 \right) ,\left( a_2,b_2 \right) \right] \in \left[ A\times B,\Gamma _{k-1}\left( A \right) \times \Gamma _{k-1}\left( B \right) \right]$,则
\begin{align*}
a_1\in A,\quad b_1\in B,\quad a_2\in \Gamma _{k-1}\left( A \right),\quad b_2\in \Gamma _{k-1}\left( B \right).
\end{align*}
从而
\begin{align*}
\left[ \left( a_1,b_1 \right) ,\left( a_2,b_2 \right) \right] &=\left( a_{1}^{-1},b_{1}^{-1} \right) \left( a_{2}^{-1},b_{2}^{-1} \right) \left( a_1,b_1 \right) \left( a_2,b_2 \right) =\left( a_{1}^{-1}a_{2}^{-1}a_1a_2,b_{1}^{-1}b_{2}^{-1}b_1b_2 \right) \\
&=\left( \left[ a_1,a_2 \right] ,\left[ b_1,b_2 \right] \right) \in \left[ A,\Gamma _{k-1}\left( A \right) \right] \times \left[ B,\Gamma _{k-1}\left( B \right) \right] =\Gamma _k\left( A \right) \times \Gamma _k\left( B \right).
\end{align*}
因此$\left[ A\times B,\Gamma _{k-1}\left( A \right) \times \Gamma _{k-1}\left( B \right) \right] \subseteq \Gamma _k\left( A \right) \times \Gamma _k\left( B \right)$。再任取$$\left( \left[ a_1,a_2 \right] ,\left[ b_1,b_2 \right] \right) \in \Gamma _k\left( A \right) \times \Gamma _k\left( B \right)=\left[ A,\Gamma _{k-1}\left( A \right) \right] \times \left[ B,\Gamma _{k-1}\left( B \right) \right],$$
则
\begin{align*}
a_1\in A,\quad a_2\in \Gamma _{k-1}\left( A \right),\quad b_1\in B,\quad b_2\in \Gamma _{k-1}\left( B \right).
\end{align*}
从而
\begin{align*}
\left( \left[ a_1,a_2 \right] ,\left[ b_1,b_2 \right] \right) &=\left( a_{1}^{-1}a_{2}^{-1}a_1a_2,b_{1}^{-1}b_{2}^{-1}b_1b_2 \right) =\left( a_{1}^{-1},b_{1}^{-1} \right) \left( a_{2}^{-1},b_{2}^{-1} \right) \left( a_1,b_1 \right) \left( a_2,b_2 \right) \\
&=\left[ \left( a_1,b_1 \right) ,\left( a_2,b_2 \right) \right] \in \left[ A\times B,\Gamma _{k-1}\left( A \right) \times \Gamma _{k-1}\left( B \right) \right].
\end{align*}
因此$\Gamma _k\left( A \right) \times \Gamma _k\left( B \right) \subseteq \left[ A\times B,\Gamma _{k-1}\left( A \right) \times \Gamma _{k-1}\left( B \right) \right]$。故$\left[ A\times B,\Gamma _{k-1}\left( A \right) \times \Gamma _{k-1}\left( B \right) \right] =\Gamma _k\left( A \right) \times \Gamma _k\left( B \right)$。再由\eqref{eq:::--q89h2dfwfzznmnghmghkhlyujf21.1}式可得
\begin{align*}
\Gamma _k\left( A\times B \right) =\Gamma _k\left( A \right) \times \Gamma _k\left( B \right).
\end{align*}
故由数学归纳法可知
\begin{align*}
\Gamma _k\left( A\times B \right) =\Gamma _k\left( A \right) \times \Gamma _k\left( B \right),\quad \forall k\in \mathbb{N}.
\end{align*}

\item 由$G$的升中心列的定义可得
\begin{align*}
\pi_k\left(C_{k+1}(G)\right) = C_{k+1}(G)/C_k(G) = C\left(G/C_k(G)\right),\quad \forall k\in \mathbb{N}.
\end{align*}

\item 由\rrefpro{proposition:导出列的基本性质--抽象代数}{proposition:导出列的基本性质--抽象代数-5}可得
\begin{align*}
&a\in C_{k+1}(G) = \pi_k^{-1}\left(C\left(G/C_k(G)\right)\right)\Longleftrightarrow \pi_k(a) \in \pi_k\left(C_{k+1}(G)\right) \\
&\Longleftrightarrow aC_k(G) \in C_{k+1}(G)/C_k(G) = C\left(G/C_k(G)\right) \\
&\Longleftrightarrow \left(aC_k(G)\right)\left(bC_k(G)\right) = \left(bC_k(G)\right)\left(aC_k(G)\right),\forall b\in G \\
&\Longleftrightarrow [a,b]C_k(G) = a^{-1}b^{-1}abC_k(G) = C_k(G),\forall b\in G \\
&\Longleftrightarrow [a,b] \in C_k(G),\forall b\in G.
\end{align*}

\item 由\rreflem{lemma:换位子群的基本性质}{lemma:换位子群的基本性质-3}知
\begin{align*}
G^{(k)}=[G^{(k-1)},G^{(i-1)}]\lhd G^{(k-1)},k=1,2,\cdots.
\end{align*}
又显然$G^{\left( 0 \right)}=G\lhd G$.假设$G^{(k-1)}\lhd G$,则由\rreflem{lemma:换位子群的基本性质}{lemma:换位子群的基本性质-3}可得
\begin{align*}
G^{\left( k \right)}=\left[ G^{\left( k-1 \right)},G^{\left( k-1 \right)} \right] \lhd G.
\end{align*}
故由数学归纳法知$G^{\left( k \right)}\lhd G,k=1,2,\cdots.$

显然$\Gamma_1(G)=G\lhd G$.假设$\Gamma_{k-1}(G)\lhd G$,则由\rreflem{lemma:换位子群的基本性质}{lemma:换位子群的基本性质-3}可得
\begin{align*}
\Gamma_k(G)=[G,\Gamma_{k-1}(G)]\lhd  G.
\end{align*}
故由数学归纳法知$\Gamma_k(G)\lhd G,k=1,2,\cdots.$又$\Gamma_k(G)\subseteq \Gamma_{k-1}(G),k=1,2,\cdots$,再由\rrefpro{proposition:正规子群的基本性质}{proposition:正规子群的基本性质-2}知
\begin{align*}
\Gamma_k(G)\lhd \Gamma_{k-1}(G),\,\,k=1,2,\cdots.
\end{align*}

显然$C_0(G) = \{1\} \lhd G$. 假设$C_k(G) \lhd G$. 设$\pi_k$是$G$到$G/C_k(G)$的自然同态,则由\rrefpro{proposition:导出列的基本性质--抽象代数}{proposition:导出列的基本性质--抽象代数-5}知
\begin{align*}
C_{k+1}(G) = \pi_k^{-1}\left(C\left(G/C_k(G)\right)\right),\quad k=0,1,\cdots.
\end{align*}
由\rrefthe{theorem:中心化子的基本性质}{theorem:中心化子的基本性质-1}知$C(G/C_k) \lhd G/C_k$. 再由\rrefthe{theorem:抽象代数-定理1.7.2}{theorem:抽象代数-定理1.7.2-2}知
\begin{align*}
C_{k+1}(G) = \pi_k^{-1}\left(C\left(G/C_k(G)\right)\right) \lhd G,\quad k=0,1,\cdots.
\end{align*}
故由数学归纳法知
\begin{align*}
C_k(G) \lhd G,\quad k=0,1,\cdots.
\end{align*}
再由\rrefpro{proposition:正规子群的基本性质}{proposition:正规子群的基本性质-2}知
\begin{align*}
C_k(G) \lhd C_{k+1}(G),\quad k=0,1,\cdots.
\end{align*}

\end{enumerate}

\end{proof}

\begin{proposition}\label{proposition:Abel群都是幂零群也都是可解群}
Abel群是幂零群,也是可解群.特别地,循环群都是幂零群,也都是可解群.
\end{proposition}
\begin{proof}


因为循环群都是Abel群,所以循环群都是幂零群,也都是可解群.

\end{proof}

\begin{example}
设$G = S_3$,于是$G^{(1)} = \Gamma_2(G) = A_3$,因而$G^{(2)} = \{1\}$,但$\Gamma_3(G) = A_3 = \Gamma_2(G)$,故当$k \geqslant 2$时均有$\Gamma_k(G) = A_3 \neq \{1\}$,故$S_3$是可解群但不是幂零群.
\end{example}
\begin{proof}


\end{proof}

\begin{theorem}\label{theorem:可解群过可解群的扩张必是可解群}

\begin{enumerate}[(1)]
\item\label{theorem:可解群过可解群的扩张必是可解群-1} 设$G$是可解群,$A$是可解群$G$的子群,则$A$也是可解群.

\item\label{theorem:可解群过可解群的扩张必是可解群-2} 设$G$是可解群,$f$是群$G$到群$G_1$的同态,则$f(G)$也是可解群.

\item\label{theorem:可解群过可解群的扩张必是可解群-3} 设群$G$是群$B$过群$A$的扩张,则$G$可解的充分必要条件是$A, B$都是可解群.
\end{enumerate}
\end{theorem}
\begin{proof}
\begin{enumerate}[(1)]
\item 由$G$是可解群知,存在$k_0\in \mathbb{N}$,使得$G^{\left( k_0 \right)}=\left\{ 1 \right\}.$再由\rrefpro{proposition:导出列的基本性质--抽象代数}{proposition:导出列的基本性质--抽象代数-2}知$A^{\left( k_0 \right)}\subseteq G^{\left( k_0 \right)}=\left\{ 1 \right\},$故$A^{\left( k_0 \right)}=\left\{ 1 \right\},$即$A$是可解群.

\item 由$G$是可解群知,存在$k_0\in \mathbb{N}$,使得$G^{\left( k_0 \right)}=\left\{ 1 \right\}.$再由\rrefpro{proposition:导出列的基本性质--抽象代数}{proposition:导出列的基本性质--抽象代数-3}知$\left( f\left( G \right) \right) ^{\left( k_0 \right)}=f\left( G^{\left( k_0 \right)} \right) =\left\{ 1 \right\},$故$f\left( G \right)$是可解群.

\item 由$G$是$B$过$A$的扩张,故可假定$A \lhd G, B = G/A$. 又设$\pi$是$G$到$B$的自然同态.由\rrefpro{proposition:导出列的基本性质--抽象代数}{proposition:导出列的基本性质--抽象代数-2}知
\begin{align}\label{eq::ohefsio2j0392jr}
A^{(k)} \subseteq G^{(k)},\,\,k=0,1,\cdots.
\end{align}
由\rrefpro{proposition:导出列的基本性质--抽象代数}{proposition:导出列的基本性质--抽象代数-3}知
\begin{align}\label{eq::ohefsio2j0392jr--1}
B^{(k)}=\pi(G^{(k)}),\,\, k =0,1,\cdots.
\end{align}

若$G$可解,则存在$k_0\in \mathbb{N},$使得$G^{(k_0)}=\{1\}$.于是由\eqref{eq::ohefsio2j0392jr}式得$A^{(k_0)} \subseteq G^{(k_0)}=\{1\}$,故$A=\{1\}$,即$A$可解.
再由\eqref{eq::ohefsio2j0392jr--1}式得$B^{(k_0)}=\pi(G^{(k_0)})=\pi(\{1\})=\{1\}$,故$B$可解.

反之,若$A, B$可解,则存在$k_1, k_2$,使$A^{(k_1)} = B^{(k_2)} = \{1\}$,故由\eqref{eq::ohefsio2j0392jr--1}式得$\pi(G^{(k_2)}) =B^{k_2}= \{1\}$,即$G^{(k_2)} \subseteq \ker \pi=A$.由\rrefpro{proposition:导出列的基本性质--抽象代数}{proposition:导出列的基本性质--抽象代数-2}知$(G^{(k_1)})^{(k)}\subseteq A^{(k)},\forall k\in \mathbb{N}.$
因而再由\rrefpro{proposition:导出列的基本性质--抽象代数}{proposition:导出列的基本性质--抽象代数-1}知$G^{(k_1 + k_2)}=(G^{(k_1)})^{(k_2)}\subseteq A^{(k_2)}=\{1\}$,故$G^{(k_1 + k_2)} = \{1\}$,于是$G$可解.
\end{enumerate}

\end{proof}

\begin{theorem}
设$G$是有限群,则下列条件等价:
\begin{enumerate}[(1)]
\item $G$是可解群;

\item 存在$G$的正规序列
\begin{align*}
G = G_1 \supseteq G_2 \supseteq \cdots \supseteq G_r = \{1\},
\end{align*}
使$G_i/G_{i+1}$为Abel群,$1 \leqslant i \leqslant r - 1$;

\item 存在$G$的次正规序列
\begin{align*}
G = G_1' \supseteq G_2' \supseteq \cdots \supseteq G_s' = \{1\},
\end{align*}
使$G_i'/G_{i+1}'$为Abel群,$1 \leqslant i \leqslant s - 1$;
\item 存在$G$的次正规序列
\begin{align*}
G = G_1'' \supseteq G_2'' \supseteq \cdots \supseteq G_t'' = \{1\},
\end{align*}
使$G_i''/G_{i+1}''$为素数阶群,$1 \leqslant i \leqslant t - 1$.
\end{enumerate}
\end{theorem}
\begin{remark}
由这个定理的证明知:\textbf{群G是可解群的充要条件是G的导出列都是正规序列且因子都是Abel群.}
\end{remark}
\begin{proof}
(1) $\Rightarrow$ (2). 由于$G$可解,故有$k\in \mathbb{N}$,使$G^{(k)} = \{1\}$.再由\rrefpro{proposition:导出列的基本性质--抽象代数}{proposition:导出列的基本性质--抽象代数-7}知$G$中有正规序列
\begin{align*}
G =G^{0}\supseteq G^{(1)} \supseteq G^{(2)} \supseteq \cdots \supseteq G^{(k)} = \{1\}.
\end{align*}
因为$G^{\left( i \right)}=\left[ G^{\left( i-1 \right)},G^{\left( i-1 \right)} \right] ,G^{\left( i \right)}\lhd G^{\left( i-1 \right)}$,所以由\rrefpro{proposition:换位子(群)保持群自同构}{proposition:换位子(群)保持群自同构-3}知$G^{(i-1)}/G^{(i)}$是Abel群,故$G$的导出列满足(2)的要求.

(2) $\Rightarrow$ (3). 由于正规序列必为次正规序列,故条件(2)成立一定有条件(3)成立.

(3) $\Rightarrow$ (4). 设$G = G_1' \supseteq G_2' \supseteq \cdots \supseteq G_s' = \{1\}$是$G$的次正规序列,并且$G_i'/G_{i+1}'$为Abel群. 由于$G$是有限群,故$G_i'/G_{i+1}'$也是有限群. 如果对某个$i$有
\begin{align*}
|G_i'/G_{i+1}'| = p_1^{a_1}p_2^{a_2} \cdots p_k^{a_k},
\end{align*}
其中$k \geqslant 1$,$p_1, p_2, \cdots, p_k$是互不相等的素数,$\sum\limits_{j=1}^k a_j > 1$. 令$P_j$是Abel群$G_i'/G_{i+1}'$的Sylow $p_j$子群,则$|P_j|=p_j^{a_j}$. 由\rrefpro{proposition:正规子群的基本性质}{proposition:正规子群的基本性质-1}知$P_j$是$G_i'/G_{i+1}'$的正规子群. 由\refpro{proposition:抽象代数--群的内直积分解}知有内直积分解
\begin{align*}
G_i'/G_{i+1}' = P_1 \otimes P_2 \otimes \cdots \otimes P_k.
\end{align*}
又设$P_k'$为$P_k$的$p_k^{a_k - 1}$阶子群,则$P_k'$也为Abel群$G_i'/G_{i+1}'$的子群,由\rrefpro{proposition:正规子群的基本性质}{proposition:正规子群的基本性质-1}知$P_k'\lhd G_i'/G_{i+1}'$.由\rrefthe{theorem:群的内直积分解的基本性质}{theorem:群的内直积分解的基本性质-4}知
\begin{align*}
H' \triangleq P_1 \otimes P_2 \otimes \cdots \otimes P_{k-1} \otimes P_k'\lhd G_i'/G_{i+1}'.
\end{align*}
再利用\hyperref[theorem:Lagrange定理--抽象代数]{Lagrange定理}及\rrefthe{theorem:群的内直积分解的基本性质}{theorem:群的内直积分解的基本性质-2}可得
\begin{gather}
|H'|=p_{1}^{a_1}p_{2}^{a_2}\cdots p_{k}^{a_k-1},\nonumber
\\
[G_i' /G_{i+1}':H' ]=\frac{\left| G_i' /G_{i+1}' \right|}{\left| H' \right|}=\frac{p_{1}^{a_1}p_{2}^{a_2}\cdots p_{k}^{a_k}}{p_{1}^{a_1}p_{2}^{a_2}\cdots p_{k}^{a_k-1}}=p_k.\label{eq::fiohw3jfwiu30f9wk3}
\end{gather}
设$\pi$为$G_i'$到$G_i'/G_{i+1}'$上的自然同态,令$H = \pi^{-1}(H')$,则由$1\in H'$和\rrefcor{corollary:群同态第二定理推论}{corollary:群同态第二定理推论-2}知
\begin{align*}
G_{i+1}'=\ker\pi=\pi^{-1}(1) \subseteq \pi^{-1}(H')=H=\pi^{-1}(H')\lhd G_i'.
\end{align*}
于是由\rrefthe{theorem:抽象代数-定理1.7.2}{theorem:抽象代数-定理1.7.2-3}可得
\begin{align*}
G_{i+1}'\lhd H \lhd G_{i}',\quad G_{i}^{\prime}/H\cong \left( G_{i}^{\prime}/G_{i+1}^{\prime} \right) /\pi \left( H \right) =\left( G_{i}^{\prime}/G_{i+1}^{\prime} \right) /H' .
\end{align*}
将$\pi$限制在$H$上,则$\pi|_{H}$是$H$到$H'$的满同态.再利用\rrefthe{theorem:抽象代数-定理1.7.2}{theorem:抽象代数-定理1.7.2-3}可得
\begin{align*}
H/G_{i+1}^{\prime}\cong \pi \left( H \right) /\pi \left( G_{i+1}^{\prime} \right) =\pi \left( H \right) /\left\{ 1 \right\} =H'.
\end{align*}
再结合\eqref{eq::fiohw3jfwiu30f9wk3}式可得
\begin{gather*}
[G_i' : H]=[G_i' /G_{i+1}':H' ]= p_k,
\\
[H:G_{i+1}^{\prime}]=\left| H' \right|=p_{1}^{a_1}p_{2}^{a_2}\cdots p_{k-1}^{a_{k-1}}p_{k}^{a_k-1}.
\end{gather*}
故$G_i'/H$是素数阶群,$H/G_{i+1}'$仍是Abel群.同理可由Abel群$H/G_{i+1}'$得到$H_1$,使得$G_{i+1}'\lhd H_1\lhd H$,$H/H_1$是素数阶群,$H_1/G_{i+1}'$仍是Abel群且
\begin{align*}
\left[ H_1:G_{i+1}' \right] =p_{1}^{a_1}p_{2}^{a_2}\cdots p_{k-1}^{a_{k-1}}p_{k}^{a_k-2}.
\end{align*}
依次进行下去,因为$\left[ G_{i}':G_{i+1}' \right]$的阶有限,所以这个操作必会在有限步后终止,最终得到
\begin{align*}
G_{i+1}'\lhd H_m\lhd \cdots \lhd H_2\lhd H_1\lhd H,\quad m=\sum_{i=1}^k{a_i}-2,
\end{align*}
$H/H_1$,$H_k/H_{k+1}$,$k=1,2,\cdots m$都是素数阶群,并且$\left[ H_m:G_{i+1}' \right] =p_1$,故$H_m/G_{i+1}'$也是素数阶群.根据$i$的任意性,我们在每个$G_{i}'$,$G_{i+1}'$之间都可以像这样插入一列群,最终得到一个新的次正规序列,并且这个新的次正规序列的因子都是素数阶群.

(4) $\Rightarrow$ (1). 因$G_{t-1}'',G_{t-2}''/G_{t-1}''$都是素数阶群,故由\refpro{proposition:素数阶群必为循环群}知$G_{t-1}'',G_{t-2}''/G_{t-1}''$都是循环群,因而由\refpro{proposition:Abel群都是幂零群也都是可解群}知它们都可解.
注意到短正合序列
\begin{align*}
1\longrightarrow G_{t-1}''\xrightarrow{\lambda}G_{t-2}''\xrightarrow{\mu}G_{t-2}''/G_{t-1}''\longrightarrow 1,
\end{align*}
故$G_{t-2}''$是循环群$G_{t-2}^{''}/G_{t-1}^{''}$过可解群$G_{t-1}^{''}$的扩张,由\refthe{theorem:可解群过可解群的扩张必是可解群}知$G_{t-2}''$是可解群. 假设$G_{i+1}''$为可解群,则同理可得$G_i''$是循环群$G_{i}^{''}/G_{i+1}^{''}$过可解群$G_{i+1}^{''}$的扩张,故由\refthe{theorem:可解群过可解群的扩张必是可解群}知$G_i''$是可解群.因此由数学归纳法知当$i = 1$时知$G$为可解群.

\end{proof}

\begin{example}
在$A_4$中有正规序列
\begin{align*}
A_4 \supseteq K_4 \supseteq \{\mathrm{id}\}.
\end{align*}
$A_4/K_4$是3阶群,$K_4$是Abel群,于是$A_4$是可解群.
\end{example}
\begin{proof}


\end{proof}

\begin{theorem}\label{theorem:幂零群的基本结论--抽象代数}
\begin{enumerate}[(1)]
\item\label{theorem:幂零群的基本结论--抽象代数-1} 设$G$是幂零群,$A$是幂零群$G$的子群,则$A$也是幂零群.

\item\label{theorem:幂零群的基本结论--抽象代数-2} 设$G$是幂零群,$f$是群$G$到群$G_1$的同态,则$f(G)$也是幂零群.

\item\label{theorem:幂零群的基本结论--抽象代数-3} 设$G$是幂零群$B$过幂零群$A$的扩张,若$G$是中心扩张或平凡扩张,则$G$也是幂零群.
\end{enumerate}
\end{theorem}
\begin{proof}
\begin{enumerate}[(1)]
\item 由$G$是幂零群知,存在$k_0\in \mathbb{N}$,使得$\Gamma_{k_0}(G)=\left\{ 1 \right\}.$再由\rrefpro{proposition:导出列的基本性质--抽象代数}{proposition:导出列的基本性质--抽象代数-2}知$\Gamma_{k_0}(A) \subseteq \Gamma_{k_0}(G)=\{1\}$,故$\Gamma_{k_0}(A)=\{1\}$. 由此知$A$也为幂零群. 

\item 由$G$是幂零群知,存在$k_0\in \mathbb{N}$,使得$\Gamma_{k_0}(G)=\left\{ 1 \right\}.$再由\rrefpro{proposition:导出列的基本性质--抽象代数}{proposition:导出列的基本性质--抽象代数-2}知$\Gamma_{k_0}(f(G)) = f(\Gamma_{k_0}(G))=f(\{1\})=\{1\}$,故$\Gamma_{k_0}(f(G))=\{1\}$.由此知$G_1$也是幂零群.

\item 若$G$是中心扩张,则由条件知$A \subseteq C(G), G/A = B$,又设$\pi$是$G$到$B$上的自然同态. 由$B$幂零知,存在$k_1\in \mathbb{N}$,使得
\begin{align*}
\pi(\Gamma_{k_1}(G)) = \Gamma_{k_1}(B) = \{1\},
\end{align*}
因而$\Gamma_{k_1}(G) \subseteq \ker\pi = A \subseteq C(G)$. 故$\forall a \in \Gamma_{k_1}(G), b \in G$有
\begin{align*}
[a, b] =a^{-1}b^{-1}ab=a^{-1}a= 1.
\end{align*}
于是$\Gamma_{k_1 + 1}(G) = [G, \Gamma_{k_1}(G)] = \{1\}$,这就证明了$G$是幂零群.

若$G$是平凡扩张,则由\refthe{theorem:抽象代数--定理4.5.4}知$G \cong A \times B$且$A, B$都是幂零群.由\rrefpro{proposition:导出列的基本性质--抽象代数}{proposition:导出列的基本性质--抽象代数-4}知
\begin{align*}
\Gamma_k(G) = \Gamma_k(A) \times \Gamma_k(B),\,\,\forall k\in \mathbb{N}.
\end{align*}
又由$A, B$是幂零群知,存在$k_1,k_2\in \mathbb{N}$,使得$\Gamma_{k_1}(A)=\Gamma_{k_2}(B)=\{1\}$.又因为
\begin{align*}
\Gamma _{k_1}(A)\supseteq \Gamma _k(A),\,\,\forall k\geqslant k_1;\quad \Gamma _{k_2}(B)\supseteq \Gamma _k(B),\,\,\forall k\geqslant k_2.
\end{align*}
所以
\begin{align*}
\Gamma _k\left( A \right) =\left\{ 1 \right\} ,\,\,\forall k\geqslant k_1;\quad \Gamma _k\left( B \right) =\left\{ 1 \right\} ,\,\,\forall k\geqslant k_2.
\end{align*}
于是取$K=\max \left\{ k_1,k_2 \right\} $,则
\begin{align*}
\Gamma_K(G)=\Gamma_K(A)\times \Gamma_K(B)=\{1\}\times\{1\}=\{1\}.
\end{align*}
故$G$也为幂零群.
\end{enumerate}

\end{proof}

\begin{theorem}
设$G$是群,则下列条件等价:
\begin{enumerate}[(1)]
\item $G$是一个幂零群;

\item $G$中有正规序列
\begin{align*}
G = G_1 \supseteq G_2 \supseteq \cdots \supseteq G_r = \{1\},
\end{align*}
使$G_i/G_{i+1} \subseteq C(G/G_{i+1})$,$1 \leqslant i \leqslant r - 1$;

\item 存在$k\in \mathbb{N}$,使得$C_k(G) = G$.
\end{enumerate}
\end{theorem}
\begin{remark}
由这个定理的证明知:\textbf{群G是幂零群的充要条件是G的降中心列都是正规序列且$\mathbf{\Gamma }_{\boldsymbol{i}}/\mathbf{\Gamma }_{\boldsymbol{i}+\mathbf{1}}\subseteq \boldsymbol{C}\left( \boldsymbol{G}/\mathbf{\Gamma }_{\boldsymbol{i}+\mathbf{1}} \right) $.}
\end{remark}
\begin{proof}
(1)$\Rightarrow$ (2). 因$G$是幂零群,故有$k\in \mathbb{N}$,使$\Gamma_k(G) = \{1\}$.
再由\rrefpro{proposition:导出列的基本性质--抽象代数}{proposition:导出列的基本性质--抽象代数-7}知$G$中有正规序列
\begin{gather*}
G = \Gamma_1(G) \supseteq \Gamma_2(G) \supseteq \cdots \supseteq \Gamma_k(G) = \{1\},
\end{gather*}
因而由\rrefpro{proposition:正规子群的基本性质}{proposition:正规子群的基本性质-4}知$\Gamma_i(G)/\Gamma_{i+1}(G) \subseteq G/\Gamma_{i+1}(G)$. 设$\pi$为$G$到$G/\Gamma_{i+1}(G)$的自然同态,由\rrefpro{proposition:导出列的基本性质--抽象代数}{proposition:导出列的基本性质--抽象代数-3}及$[G, \Gamma_i(G)] = \Gamma_{i+1}(G)$知
\begin{align*}
[G/\Gamma _{i+1}(G),\Gamma _i(G)/\Gamma _{i+1}(G)]=[\pi (G),\pi (\Gamma _i(G))]=\pi \left( \left[ G,\Gamma _i(G) \right] \right) =\pi (\Gamma _{i+1}(G))=\left\{ 1 \right\},
\end{align*}
因而由\rreflem{lemma:换位子群的基本性质}{lemma:换位子群的基本性质-1}知$\Gamma_i(G)/\Gamma_{i+1}(G) \subseteq C(G/\Gamma_{i+1}(G))$,故$G$的降中心列满足条件(2)的要求.

(2) $\Rightarrow$ (3). 用反序归纳法证明$G_i \subseteq C_{r - i}(G)$,其中$C_0(G) = \{1\}$. 当$i = r$时,$G_r = \{1\} = C_0(G) = C_{r - r}(G)$. 设$i + 1$时已成立,因而$G_{i+1} \subseteq C_{r - (i+1)}(G)$. 又$G_i/G_{i+1} \subseteq C(G/G_{i+1})$,由\rrefcor{corollary:正规子群交为1则乘积可交换}{corollary:正规子群交为1则乘积可交换-1}知
\begin{align*}
\left[ G_i,G \right] \subseteq G_{i+1}.
\end{align*}
即对$\forall a \in G_i,$有
\begin{align*}
[a,b]\in G_{i+1} \subseteq C_{r - (i+1)}(G),\forall b\in G.
\end{align*}
因而由\rrefpro{proposition:导出列的基本性质--抽象代数}{proposition:导出列的基本性质--抽象代数-6}知$a \in C_{r - i}(G)$,故$G_i \subseteq C_{r - i}(G)$. 特别地,有$G_1 = G \subseteq C_{r - 1}(G)$,即取$k = r - 1$知条件(3)成立.

(3) $\Rightarrow$ (1). 设有$k\in \mathbb{N}$,使$C_k(G) = G$,再结合\rrefpro{proposition:导出列的基本性质--抽象代数}{proposition:导出列的基本性质--抽象代数-7}知$G$中有正规序列
\begin{align*}
G = C_k(G) \supseteq C_{k-1}(G) \supseteq \cdots \supseteq C_1(G) \supseteq C_0(G) = \{1\}.
\end{align*}
用数学归纳法证明$\Gamma_i(G) \subseteq C_{k - i + 1}(G),i=1,2,\cdots$. 当$i = 1$时,显然成立.假设结论对$i$成立,现在考虑$i+1$的情形.由于$C_{k - i + 1}(G)/C_{k - i}(G) = C(G/C_{k - i}(G))$,故由\rrefcor{corollary:正规子群交为1则乘积可交换}{corollary:正规子群交为1则乘积可交换-1}有$[G, C_{k - i + 1}(G)] \subseteq C_{k - i}(G)$,于是再结合\rreflem{lemma:换位子群的基本性质}{lemma:换位子群的基本性质-4}得
\begin{align*}
\Gamma_{i+1}(G) = [G, \Gamma_i(G)] \subseteq [G, C_{k - i + 1}(G)] \subseteq C_{k - i}(G).
\end{align*}
故由数学归纳法知$\Gamma_i(G) \subseteq C_{k - i + 1}(G),i=1,2,\cdots$.
特别地,有$\Gamma_{k+1}(G) \subseteq C_0(G) = \{1\},$故$\Gamma_{k+1}=\{1\}$.因而$G$是幂零群.

\end{proof}

\begin{theorem}
设$p$是一个素数,则有限$p$群都是幂零群.
\end{theorem}
\begin{proof}
由\refpro{proposition:Abel群都是幂零群也都是可解群}知阶数为$p$的有限$p$群都是幂零群.假设结论对阶数小于等于$p^{n-1}$的有限$p$群都成立,现设$|G|=p^n$,则由\rrefthe{theorem:抽象代数--定理4.3.1}{theorem:抽象代数--定理4.3.1-3}知$C(G) \neq \{1\}$,从而$|C(G)|>1.$若$|C(G)|=|G|$,则$G$是Abel群,由\refpro{proposition:Abel群都是幂零群也都是可解群}知$G$是幂零群.下设$|C(G)|=p^k(1\leqslant k<n),$则由\hyperref[theorem:Lagrange定理--抽象代数]{Lagrange定理}得
\begin{align*}
\left| G \right|=\left[ G:C\left( G \right) \right] \cdot \left| C\left( G \right) \right|\Longleftrightarrow \left[ G:C\left( G \right) \right] =p^{n-k}<p^n.
\end{align*}
于是由归纳假设知$G/C\left( G \right) ,C\left( G \right) $都是幂零群.又显然$G$是$G/C(G)$过$C(G)$的中心扩张,故由\rrefthe{theorem:幂零群的基本结论--抽象代数}{theorem:幂零群的基本结论--抽象代数-3}知$G$为幂零群.因此由数学归纳法知任何有限$p$群都是幂零群.

\end{proof}

\begin{example}
设$H$是四元数体,$G = \{\pm 1, \pm i, \pm j, \pm k\}$,其中$1, i, j, k \in H$如\refpro{proposition:四元数体的基本结论----1}所述,则$G$是8阶群. 这是一个非Abel幂零群的例子.
\end{example}
\begin{proof}


\end{proof}












\end{document}
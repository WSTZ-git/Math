\documentclass[../../main.tex]{subfiles}
\graphicspath{{\subfix{../../image/}}} % 指定图片目录,后续可以直接使用图片文件名。

% 例如:
% \begin{figure}[H]
% \centering
% \includegraphics[scale=0.4]{图.png}
% \caption{}
% \label{figure:图}
% \end{figure}
% 注意:上述\label{}一定要放在\caption{}之后,否则引用图片序号会只会显示??.

\begin{document}

\section{有限单群}

\begin{definition}[有限单群]
若有限群 \( G \) 无非平凡的正规子群,则称 \( G \) 为\textbf{有限单群}.
\end{definition}

\begin{theorem}\label{theorem:抽象代数--定理4.4.1}
设\( G \) 为Abel群且 \( G \neq \{e\} \),\( e \) 为 \( G \) 的幺元,则 \( G \) 为单群的充分必要条件是 \( G \) 的阶为素数. 这时\( G \) 必为循环群.
\end{theorem}
\begin{proof}
由\rrefpro{proposition:正规子群的基本性质}{proposition:正规子群的基本性质-1} Abel 群 \( G \) 的任何子群都是 \( G \) 的正规子群,故 Abel 群 \( G \) 为单群当且仅当 \( G \) 无非平凡子群. 

若 \( G \) 是有限阶的,当$G$的阶为素数时,由\refpro{proposition:素数阶群必为循环群}知$G$只有平凡子群.
当\( G \) 无非平凡子群时,若$G$的阶不是素数,又$G\neq \{e\}$,故$|G|$是不为1的合数.由因式分解定理知存在素数$p$以及正整数$m,l$,使$|G|=p^lm$且$(p,m)=1.$从而$m,l$不同时为1,否则与$|G|$不为素数矛盾!故$|G|>p$.
由\hyperref[theorem:Sylow 第一定理]{Sylow 第一定理}知$G$中一定有$p$阶子群$H$,而$1<p<|G|$,故$H$必是$G$的非平凡子群,矛盾!
因此\( G \) 无非平凡子群当且仅当 \( G \) 的阶为素数. 此时,由\refpro{proposition:素数阶群必为循环群}知\( \forall a \in G \) 且 \( a \neq e \) 有 \( G = \langle a \rangle \). 

若 \( G \) 是无限阶的,则 \( \langle a \rangle \lhd G (\forall a \in G, a \neq e) \). 若 \( \langle a \rangle \) 是有限阶的,则 \( \langle a \rangle \) 是非平凡的. 若\( \langle a \rangle \) 是无限阶的,则 \( \langle a^2 \rangle \) 是非平凡的,因为$a,-a\notin \langle a^2 \rangle$.即任何无限阶的 Abel 群都有非平凡的正规子群.故此时$G$必不是单群.

\end{proof}

\begin{proposition}\label{proposition:轮换的共轭性质}
对任意$r$轮换$(i_1i_2\cdots i_r)$和$\sigma \in S_n$,都有
\begin{align*}
\sigma \left( i_1i_2\cdots i_r \right) \sigma ^{-1}=\left( \sigma \left( i_1 \right) \sigma \left( i_2 \right) \cdots \sigma \left( i_r \right) \right) .
\end{align*}
\end{proposition}
\begin{proof}
对$\forall l \in \{1,2,\cdots,n\}$,若$l \notin \{i_1,i_2,\cdots,i_r\}$,则
\begin{align*}
\sigma(i_1i_2\cdots i_r)\sigma^{-1}(\sigma(l)) = \sigma(i_1i_2\cdots i_r)(l) = \sigma(l) = (\sigma(i_1)\sigma(i_2)\cdots\sigma(i_r))(\sigma(l)).
\end{align*}
若$l \in \{i_1,i_2,\cdots,i_r\}$,设$l = i_j$,$j \in \{1,2,\cdots,r\}$,则
\begin{align*}
\sigma(i_1i_2\cdots i_r)\sigma^{-1}(\sigma(i_j)) &= \sigma(i_1i_2\cdots i_r)(i_j) = \sigma(i_{j+1}) = (\sigma(i_1)\sigma(i_2)\cdots\sigma(i_r))(\sigma(i_j)),\ j=1,2,\cdots,r-1; \\
\sigma(i_1i_2\cdots i_r)\sigma^{-1}(\sigma(i_r)) &= \sigma(i_1i_2\cdots i_r)(i_r) = \sigma(i_1) = (\sigma(i_1)\sigma(i_2)\cdots\sigma(i_r))(\sigma(i_r)).
\end{align*}
故
\begin{align*}
\sigma(i_1i_2\cdots i_r)\sigma^{-1}(\sigma(l)) = (\sigma(i_1)\sigma(i_2)\cdots\sigma(i_r))(\sigma(l)),\ \forall l \in \{i_1,i_2,\cdots,i_r\}.
\end{align*}
即
\begin{align*}
\sigma(i_1i_2\cdots i_r)\sigma^{-1} = (\sigma(i_1)\sigma(i_2)\cdots\sigma(i_r)).
\end{align*}

\end{proof}

\begin{theorem}\label{theorem:抽象代数--定理4.4.2}
\begin{enumerate}[(1)]
\item\label{theorem:抽象代数--定理4.4.2-1} 当 \( n \geqslant 3 \) 时,\( A_n \) 由所有的 3 轮换生成,即 \( A_n = \langle \{(ijk)\} \rangle \);

\item\label{theorem:抽象代数--定理4.4.2-2} 当 \( n \geqslant 5 \) 时,任意3轮换\( (ijk) \) 在 \( A_n \) 中的共轭类由所有的 3 轮换构成,即 \( C_{(ijk)} = \{(i'j'k')\} \).
\end{enumerate}
\end{theorem}
\begin{proof}
\begin{enumerate}[(1)]
\item 由\refcor{corollary:奇置换与偶置换分别可表示成奇数和偶数个对换之积}知 \( a \in A_n \) 当且仅当 \( a \) 可表示为偶数个对换之积. 由\refcor{theorem:生成子群的元素的形式}知
\begin{align*}
\langle \{(ijk)\} \rangle = \{ (i_1j_1k_1)\cdots(i_mj_mk_m) \mid i_s,j_s,k_s \in \{1,2,\cdots,n\}, 1 \leqslant s \leqslant m, m \in \mathbf{N} \}.
\end{align*}
设 \( i, j, k, l \)且互不相等. 由
\begin{gather*}
(ij)(ij) = \mathrm{id}, \quad (ij)(ik) = (ikj), \\
(ik)(jl) = (ik)(ij)(ij)(jl) = (ijk)(jli)
\end{gather*}
知$A_n$中元素都可写成3轮换之积,因此$A_n\subseteq \langle \{(ijk)\} \rangle$.

设$(i_1j_1k_1)\cdots(i_mj_mk_m) \in \langle \{(ijk)\} \rangle$,则由\rrefthe{theorem:置换必可写成对换之积}{theorem:置换必可写成对换之积-1}知
\begin{align*}
(i_sj_sk_s) = (i_sk_s)(i_sj_s), \quad s=1,2,\cdots,m.
\end{align*}
故$(i_1j_1k_1)\cdots(i_mj_mk_m)$可写成偶数个对换之积,故$(i_1j_1k_1)\cdots(i_mj_mk_m) \in A_n$。因此$\langle \{(ijk)\} \rangle \subseteq A_n$。故\( A_n = \langle \{(ijk)\} \rangle \).

\item \(\forall \sigma \in S_n \),由\refpro{proposition:轮换的共轭性质}知
\begin{align}\label{eq::9083w89fjj348jv3egh83gjw4}
\sigma(ijk)\sigma^{-1} = (\sigma(i)\sigma(j)\sigma(k)).
\end{align}
于是 \( C_{(ijk)} \subseteq \{(i'j'k')\} \). 反之,对任意3轮换$(i'j'k')$,当 \( n \geqslant 5 \) 时,首先\( \exists \sigma \in S_n \),使 \( \sigma(i) = i', \sigma(j) = j', \sigma(k) = k' \). 若 \( \sigma \in A_n \),则由\eqref{eq::9083w89fjj348jv3egh83gjw4}式可得
\begin{align*}
(i'j'k')=(\sigma(i)\sigma(j)\sigma(k))=\sigma(ijk)\sigma^{-1} \in C_{(ijk)}.
\end{align*}
若 \( \sigma \notin A_n \),由 \( n \geqslant 5 \) 有 \( i_1, i_2 \notin \{i, j, k\} \). 故由\refcor{corollary:奇置换与偶置换分别可表示成奇数和偶数个对换之积}知\( \sigma(i_1i_2) \in A_n \).
再由\refpro{proposition:轮换的共轭性质}知
\begin{align*}
(i' j' k' )=\left( \sigma \left( i \right) \sigma \left( j \right) \sigma \left( k \right) \right) =\left( \left( \sigma \left( i_1i_2 \right) \left( i \right) \right) \left( \sigma \left( i_1i_2 \right) \left( j \right) \right) \left( \sigma \left( i_1i_2 \right) \left( k \right) \right) \right) =\sigma (i_1i_2)(ijk)(\sigma (i_1i_2))^{-1}\in C_{(ijk)}.
\end{align*}
即仍有 \( (i'j'k') \in C_{(ijk)} \).综上知 \( C_{(ijk)} = \{(i'j'k')\} \).
\end{enumerate}
\end{proof}

\begin{theorem}\label{theorem:A_n是非Abel有限单群}
当 \( n \geqslant 5 \) 时,\( A_n \) 是非Abel有限单群.
\end{theorem}
\begin{proof}
对 \( \alpha \in S_n \),令 \( \bar{F}_\alpha = \{j \mid \alpha(j) \neq j\} \). 显然有
\begin{enumerate}[(1)]
\item\label{theorem:A_n是非Abel有限单群-证明-情况1} \( \bar{F}_\alpha = \{i, j\} \) 当且仅当 \( \alpha = (ij) \);

\item \( \bar{F}_\alpha = \{i, j, k\} \) 当且仅当 \( \alpha = (ijk)\text{或}(ijk)^{-1} \);

\item 对 \( \forall \alpha \in A_n \) 且 \( \alpha \neq \mathrm{id} \),又由\refcor{corollary:奇置换与偶置换分别可表示成奇数和偶数个对换之积}知$\alpha$可写成偶数个对换之积,从而一定有 \( |\bar{F}_\alpha| \geqslant 3 \).
\end{enumerate}
设 \( H \lhd A_n \) 且 \( H \neq \{\mathrm{id}\} \). 取 \( \tau \in H \),使
\begin{align}\label{eq::infoj4fj3fminfo8ju43}
|\bar{F}_\tau| = \min\{|\bar{F}_\alpha| \mid \alpha \in H, \alpha \neq \mathrm{id}\}.
\end{align}
显然 \( |\bar{F}_\tau| \geqslant 3 \). 若 \( |\bar{F}_\tau| = 3 \),则 \( \tau \) 是 3 轮换. 
显然$\tau \in H$,由正规子群定义和\rrefthe{theorem:共轭类,中心化子,中心的集合形式}{theorem:共轭类,中心化子,中心的集合形式-1}知
\begin{align*}
\alpha \tau \alpha^{-1} \in H,\ \forall \alpha \in A_n \implies C_{\tau} = \{ \alpha \tau \alpha^{-1} \mid \alpha \in A_n \} \subseteq H.
\end{align*}
再由\refthe{theorem:抽象代数--定理4.4.2}知
\begin{align*}
A_n = \langle \{(ijk)\} \rangle =C_\tau \subseteq H,
\end{align*}
故 \( H = A_n \). 这就证明了 \( A_n \) 为有限单群,\( A_n \) 的阶不是素数,由\refthe{theorem:抽象代数--定理4.4.1}知 \( A_n \) 不是 Abel 群.

若\( |\bar{F}_\tau| > 3 \). 由\rrefthe{theorem:置换必可写成对换之积}{theorem:置换必可写成对换之积-2},可将 \( \tau \) 分解为不相交轮换之积,分两种情况讨论. 一种在分解中只出现对换,另一种在分解中有长度大于 2 的轮换.
\begin{enumerate}
\item 若$\tau$可分解为互不相交的对换之积,又由最开始得到的\hyperref[theorem:A_n是非Abel有限单群-证明-情况1]{情况(1)}知$\tau$不可能是对换,故可设\( \tau = (i_1i_2)(i_3i_4)\cdots \)且$i_1,i_2,i_3,i_4,\cdots$互不相同. 由 \( n \geqslant 5 \) 有 \( j \neq i_1, i_2, i_3, i_4 \),令 \( \phi = (i_3i_4j) \),则由\refcor{corollary:奇置换与偶置换分别可表示成奇数和偶数个对换之积}知$\phi \in A_n$. 由 \( H \lhd A_n \) 有 \( \tau_1 = \tau^{-1}(\phi\tau\phi^{-1}) \in H \). 于是由\refpro{proposition:轮换的共轭性质}可得
\begin{align*}
\tau_1 = (\tau^{-1}\phi\tau)\phi^{-1} = (\tau^{-1}(i_3)\tau^{-1}(i_4)\tau^{-1}(j))(i_4i_3j)=(i_4i_3\tau^{-1}(j))(i_4i_3j).
\end{align*}
若 \( j \notin \bar{F}_\tau \),即$\tau(j)=\tau^{-1}(j)=j$.则有
\begin{align*}
\tau_1 = (i_4i_3j)(i_4i_3j) = (i_3i_4j),
\end{align*}
这时 \( |\bar{F}_{\tau_1}| = |\{i_3, i_4, j\}| = 3 < |\bar{F}_\tau| \),这与\eqref{eq::infoj4fj3fminfo8ju43}式中$|\bar{F}_\tau|$的最小值定义矛盾! 

若 \( j \in \bar{F}_\tau \),则$\tau =(i_1i_2)\left( i_3i_4 \right) \cdots \left( j\tau ^{-1}\left( j \right) \right) \cdots $且$i_1,i_2,i_3,i_4,j,\tau ^{-1}\left( j \right) $互不相同.由 \( j \neq i_1, i_2, i_3, i_4 \)知$|\bar{F}_\tau| \geqslant 5$,又$\tau$可写成对换之积,故$|\bar{F}_\tau|$必为偶数,因此 \( |\bar{F}_\tau| \geqslant 6 \). 此时由\refpro{proposition:轮换的共轭性质}可得
\begin{align*}
\tau _1&=(i_4i_3\tau ^{-1}(j))(i_4i_3j)=(i_3j)(i_4\tau ^{-1}(j))
\\
&=\left( i_4\tau ^{-1}\left( j \right) \right) \left( i_4i_3 \right) \left( i_4j \right) \left( i_4i_3 \right) 
\\
&=\left( i_4\tau ^{-1}\left( j \right) \right) \left( \left( i_4i_3 \right) \left( i_4i_3 \right) ^{-1} \right) 
\\
&=\left( i_4\tau ^{-1}\left( j \right) \right) \left( \left( \left( i_4i_3 \right) \left( i_4 \right) \right) \left( \left( i_4i_3 \right) \left( j \right) \right) \right) 
\\
&=\left( i_4\tau ^{-1}\left( j \right) \right) \left( i_3j \right) .
\end{align*}
于是
\begin{align*}
|\bar{F}_{\tau_1}| = |\{i_3, i_4, j, \tau^{-1}(j)\}| = 4 < |\bar{F}_\tau|.
\end{align*}
这也\eqref{eq::infoj4fj3fminfo8ju43}式中$|\bar{F}_\tau|$的最小值定义矛盾! 因而 \( \tau = (i_1i_2)(i_3i_4)\cdots \) 是不可能的.
\item 设 \( \tau \) 的分解中有长度大于 2 的轮换,即
\begin{align*}
\tau = (i_1i_2i_3\cdots)\cdots.
\end{align*}
因为 \( (i_1i_2i_3i_4) \) 为奇置换,故 \( \tau \neq (i_1i_2i_3i_4) \),由此知 \( |\bar{F}_\tau| > 4 \),即$|\bar{F}_\tau|\geqslant 5$.因而有 \( j, k \in \bar{F}_\tau \). 令 \( \phi = (i_3jk) \),\( \tau_1 = \tau^{-1}\phi\tau\phi^{-1} \),由 \( H \lhd A_n \) 有 \( \tau_1 = \tau^{-1}(\phi\tau\phi^{-1}) \in H \).于是由\refpro{proposition:轮换的共轭性质}可得
\begin{align*}
\tau _1=(\tau ^{-1}\phi \tau )\phi ^{-1}=(\tau ^{-1}(i_3)\tau ^{-1}(j)\tau ^{-1}(k))(ji_3k)=(i_2\tau ^{-1}(j)\tau ^{-1}(k))(ji_3k).
\end{align*}
由 \( j, k \in \bar{F}_\tau \) 知
\begin{align*}
\bar{F}_{\tau _1} = \left\{ i_2,i_3,j,k,\tau ^{-1}(j),\tau ^{-1}(k) \right\} \subseteq \bar{F}_{\tau}.
\end{align*}
注意到
\begin{align*}
\tau _1(i_1)=\tau ^{-1}\phi \tau \phi ^{-1}(i_1)=\tau ^{-1}\phi \tau (i_1)=\tau ^{-1}\phi (i_2)=\tau ^{-1}(i_2)=i_1,
\end{align*}
即 \( i_1 \notin \bar{F}_{\tau_1} \),则有 \( \bar{F}_{\tau_1} \subset \bar{F}_\tau \),亦即 \( |\bar{F}_{\tau_1}| < |\bar{F}_\tau| \). 这也\eqref{eq::infoj4fj3fminfo8ju43}式中$|\bar{F}_\tau|$的最小值定义矛盾! 故 \( |\bar{F}_\tau| = 3 \).
\end{enumerate}
因而 \( H = A_n \),即 \( A_n \) 为单群.

\end{proof}

\begin{proposition}
对于 \( n \leqslant 4 \),\( A_n \) 的结构为
\begin{enumerate}[(1)]
\item $A_1 = A_2 = \{\mathrm{id}\};$

\item $A_3 = \langle (123) \rangle \text{ 为三阶循环群}; $

\item $A_4 \text{ 的阶为 12}, A_4 \text{ 含有一个非平凡的正规子群}$
\begin{align*}
\{\mathrm{id}, (12)(34), (13)(24), (14)(23)\}.
\end{align*}
此群与 $\mathbf{Klein}$ \textbf{四元数群}同构,也记为 \( K_4 \),而且 \( K_4 \) 也是 \( S_4 \) 的正规子群.
\end{enumerate}
\end{proposition}
\begin{proof}


\end{proof}






\end{document}
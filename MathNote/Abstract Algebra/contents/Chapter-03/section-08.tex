\documentclass[../../main.tex]{subfiles}% 注意这里的文件路径不能用 ./main.tex ,否则用latexmk编译子文件会报错
\graphicspath{{\subfix{./image/}}} % 指定图片目录,后续可以直接使用图片文件名
% 注意这里的文件路径不能用 ../../image/ ,否则用latexmk编译子文件会报错

% 例如:
% \begin{figure}[H]
% \centering
% \includegraphics[scale=0.3]{图.png}
% \caption{}
% \label{figure:图}
% \end{figure}
% 注意:上述\label{}一定要放在\caption{}之后,否则引用图片序号会只会显示??.

\begin{document}

\section{自由幺半群与自由群}

\begin{definition}[自由幺半群]
设$X$是一个非空集合,称$X$中任一有限长的序列
\begin{align*}
x_1x_2\cdots x_i,\quad x_1,x_2,\cdots,x_i\in X
\end{align*}
为一个\textbf{字}. 当$i=0$时,称为\textbf{空字},记为$\land$. 记所有字的集合为$\widetilde{X}$. 在$\widetilde{X}$上定义乘法为
\begin{align*}
(x_1x_2\cdots x_i)(y_1y_2\cdots y_j) = x_1x_2\cdots x_iy_1y_2\cdots y_j.
\end{align*}
显然$\widetilde{X}$对此乘法是以$\land$为幺元的幺半群,称为\textbf{由$\boldsymbol{X}$生成的自由幺半群}.
\end{definition}

\begin{theorem}\label{theorem:抽象代数--定理4.8.1}
设集合$X$非空,$S$是幺半群,$f$是$X$到$S$的映射,则存在唯一的$\widetilde{X}$到$S$的同态$\phi$,使
\begin{align*}
\phi(x) = f(x),\quad \forall x\in X.
\end{align*}
\end{theorem}
\begin{remark}
由这个定理知任何幺半群均可视为自由半群的同态像.
\end{remark}
\begin{proof}
记$e$为$S$的幺元,定义$\widetilde{X}$到$S$的映射$\phi$:
\begin{align*}
\phi(\land)=e,\quad \phi(x_1x_2\cdots x_i)=f(x_1)f(x_2)\cdots f(x_i),\,\,\forall \, x_1x_2\cdots x_i\in \widetilde{X}.
\end{align*}
则$\phi$显然为同态且$\phi(x) = f(x),\,\forall x\in X.$

若$\psi$为$\widetilde{X}$到$S$的同态且$\psi(x)=f(x)$,则对$\forall x_1x_2\cdots x_i\in \widetilde{X}$,有
\begin{align*}
\psi(x_1x_2\cdots x_i) = \psi(x_1)\psi(x_2)\cdots \psi(x_i) = f(x_1)f(x_2)\cdots f(x_i) = \phi(x_1x_2\cdots x_i),
\end{align*}
即$\phi$唯一.

\end{proof}

\begin{definition}
设非空集合$X$,令非空集合$X'$满足$X\cap X'=\varnothing$,并且存在$X$到$X'$上的一一对应$\varphi$,记$\varphi(a)=a',\forall a\in X.$
令$X^*=X\cup X'$,设$x\in X^*$,定义$x'$,
\begin{align}
x' = \begin{cases}
\varphi^{-1}(x)=a,\quad x=a'\in X', \\
\varphi(x)=a',\quad x=a\in X,
\end{cases}\label{eq::::--2r89hweio4sfklnaiomziw2q12344.8.1}
\end{align}
并且记$X^*$生成的自由幺半群为$\widetilde{X^*}$.
设$w_1,w_2\in \widetilde{X^*}$,若$\exists g,h\in \widetilde{X^*},x\in X^*$,使得
\begin{align*}
\begin{cases}
w_1=gh, \\
w_2=gxx'h
\end{cases}\text{或}\begin{cases}
w_1=gxx'h, \\
w_2=gh,
\end{cases}
\end{align*}
则称$w_1$与$w_2$是\textbf{相邻的}.
\end{definition}

\begin{theorem}\label{theorem:抽象代数--定理4.8.2}
设非空集合$X$,令非空集合$X'$满足$X\cap X'=\varnothing$,并且存在$X$到$X'$上的一一对应$\varphi$,记$\varphi(a)=a',\forall a\in X.$再设$X^*=X\cup X'$,$\widetilde{X^*}$为集合$X^*$生成的自由幺半群. 在$\widetilde{X^*}$中定义关系$\sim$如下: $w_1,w_2\in \widetilde{X^*}$,称$w_1\sim w_2$,如果存在$\widetilde{X^*}$中序列
\begin{align*}
w_1 = v_1, v_2, \cdots, v_l = w_2
\end{align*}
满足$v_i$与$v_{i+1}$相邻.则“$\sim$”是$\widetilde{X^*}$中同余关系,并且$\widetilde{X^*}$对于$\sim$的商幺半群$\widetilde{X^*}/\sim = F(X)$是群,称$ F(X)$为\textbf{由$\boldsymbol{X}$生成的自由群}.用$\overline{x}$表示$x$在$F(X)$中的同余类,则$\overline{\land}$是$F(X)$的幺元.并且
\begin{align*}
(\overline{x_1x_2\cdots x_m})^{-1}=\overline{x_m' x_{m-1}' \cdots x_1' },\,\,\forall \,\overline{x_1x_2\cdots x_m}\in F(X).
\end{align*}
特别地,$(\overline{x})^{-1}=\overline{x'},\,\forall x\in X$.因此也记$x'=x^{-1}$,$X'=X^{-1}.$
\end{theorem}
\begin{remark}
$X'$是根据$X$随便取一个形式逆集合,只是为了满足群中每个元素都有逆元而引入的符号.
\end{remark}
\begin{proof}
首先证$\sim$是等价关系.

$\forall w\in \widetilde{X^*}$,取$v_1=w=w\land$, $v_2=wa_1a_1'\land$, $v_3=w\land=w$. 于是$v_1$与$v_2$相邻,$v_2$与$v_3$相邻,故有$w\sim w$.

又设$w_1\sim w_2$,即有$w_1=v_1,v_2,\cdots,v_l=w_2$且$v_i$与$v_{i+1}$相邻. 令$u_i=v_{l-i+1}$,则$u_1=w_2,u_2,\cdots,u_l=w_1$且$u_i$与$u_{i+1}$相邻. 于是$w_2\sim w_1$.

再设$w_1\sim w_2$, $w_2\sim w_3$,于是存在以下序列:
\begin{align*}
w_1 = v_1, v_2, \cdots, v_l = w_2,\quad v_i\text{与}v_{i+1}\text{相邻}, \\
w_2 = u_1, u_2, \cdots, u_m = w_3,\quad u_j\text{与}u_{j+1}\text{相邻},
\end{align*}
因而序列$w_1 = v_1,v_2,\cdots,v_l=u_1,u_2,\cdots,u_m=w_3$的任意相邻两项是相邻的,故$w_1\sim w_3$.

其次证$\sim$为同余关系. 设$w_1\sim w_2,u_1\sim u_2$,则于是存在以下序列:
\begin{align*}
w_1 = v_1, v_2, \cdots, v_l = w_2,\quad v_i\text{与}v_{i+1}\text{相邻}, \\
u_1 = u_1, u_2, \cdots, u_m = u_2,\quad u_j\text{与}u_{j+1}\text{相邻}.
\end{align*}
注意到若$u_1$与$u_2$相邻,即有$u_1=gh,u_2=gxx'h$,因而对$\forall v\in \tilde{X^*}$,有$u_1v=ghv,u_2v=gxx'hv$,于是$u_1v$与$u_2v$相邻,同样$vu_1$与$vu_2$相邻.于是有
\begin{gather*}
w_1u_1 = v_1u_1, v_2u_1, \cdots, v_lu_1 = w_2u_1\quad v_iu_1\text{与}v_{i+1}u_1\text{相邻},
\\
w_2u_1 = v_lu_1, v_lu_2, \cdots, v_lu_m = w_2u_2\quad v_lu_i\text{与}v_lu_{i+1}\text{相邻}.
\end{gather*}
这说明$w_1u_1\sim w_2u_1$,$w_2u_1\sim w_2u_2$,故$w_1u_1\sim w_2u_2$,即$\sim$为同余关系.

最后,由\refthe{Set Theory-theorem:抽象代数--半群中的同余关系可导出商集合也是商半群}知$F(X)=\widetilde{X^*}/\sim$是商幺半群.再证明商幺半群$F(X)=\widetilde{X^*}/\sim$是群,只需证明$F(X)$中任一元素可逆. 对$\forall x\in \widetilde{X^*}$,$\land$为空字,$x'$如式\eqref{eq::::--2r89hweio4sfklnaiomziw2q12344.8.1},则有$\land xx'\land=xx'$,$\land=\land\land$,即$xx'$与$\land$相邻,因而
\begin{align}\label{eq::io3hrojeqqqq}
\land\sim xx',\,\,\forall x\in \widetilde{X^*}.
\end{align}
用$\overline{x}$表示$x$在$F(X)$中的同余类,则$\overline{\land}$是$F(X)$的幺元.对任意$\overline{x_1x_2\cdots x_m}\in F\left( X \right)$,
有$x_1x_2\cdots x_m\in \widetilde{X^*}$,从而$x_m' x_{m-1}' \cdots x_1'\in \widetilde{X^*}$,再结合“$\sim$”是同余关系和\eqref{eq::io3hrojeqqqq}式可得
\begin{align*}
(x_1x_2\cdots x_m)(x_m' x_{m-1}' \cdots x_1' )&=x_1x_2\cdots x_mx_m' \cdots x_1' \sim x_1x_2\cdots x_{m-1}\land x_{m-1}' \cdots x_1' \\
&=x_1x_2\cdots x_{m-1}x_{m-1}' \cdots x_1' \sim \cdots \sim x_1\land x_1' =x_1x_1' \sim \land.
\end{align*}
于是
\begin{align*}
&\left( \overline{x_1x_2\cdots x_m} \right) \left( \overline{x_m' x_{m-1}' \cdots x_1' } \right) =\overline{\left( x_1x_2\cdots x_m \right) \left( x_m' x_{m-1}' \cdots x_1' \right) }
\\
&=\left\{ x\in \widetilde{X^*}\mid x\sim \left( x_1x_2\cdots x_m \right) \left( x_m' x_{m-1}' \cdots x_1' \right) \right\} 
\\
&=\left\{ x\in \widetilde{X^*}\mid x\sim \land \right\} =\overline{\land }.
\end{align*}
故$\overline{x_m' x_{m-1}' \cdots x_1' }$就是$\overline{x_1x_2\cdots x_m}$的逆.
这就证明了$F(X)$中元素均可逆,故为群.


\end{proof}


\begin{proposition}\label{proposition:单元素集的自由群必是无限循环群}
设$X=\{a\}$,则$F(X)$为无限循环群.
\end{proposition}
\begin{proof}


\end{proof}

\begin{theorem}\label{theorem:抽象代数--定理4.8.3}
设非空集合$X$,令非空集合$X'$满足$X\cap X'=\varnothing$,并且存在$X$到$X'$上的一一对应$\varphi$,记$\varphi(a)=a',\forall a\in X.$再设$X^*=X\cup X'$,$\widetilde{X^*}$为集合$X^*$生成的自由幺半群. 在$\widetilde{X^*}$中定义关系$\sim$如下: $w_1,w_2\in \widetilde{X^*}$,称$w_1\sim w_2$,如果存在$\widetilde{X^*}$中序列
\begin{align*}
w_1 = v_1, v_2, \cdots, v_l = w_2
\end{align*}
满足$v_i$与$v_{i+1}$相邻.又$f$是$X$到$G$的映射,则存在唯一的自由群$F(X)$到群$G$的同态$\psi$,使得
\begin{align*}
\psi(\overline{x})=f(x)(\forall x\in X),
\end{align*}
其中$\overline{x}$表示$x$在$F(X)=\widetilde{X^*}/\sim$中的同余类.
\end{theorem}
\begin{proof}
首先将$f$扩充为$X^*$到$G$的映射,仍以$f$表示,满足$f(x')=f(x)^{-1}(\forall x'\in X')$.

由\refthe{theorem:抽象代数--定理4.8.1}知有唯一的幺半群$\widetilde{X^*}$到$G$的同态$\phi$,使得$\phi(x)=f(x)(\forall x\in X^*)$.若$\widetilde{X^*}$中元素$w_1$与$w_2$相邻,不妨设
\begin{align*}
w_1=gh,\,w_2=gxx'h,\quad g,h\in \widetilde{X^*},\,x,x'\in X^*.
\end{align*}
其中$x'$的定义如式\eqref{eq::::--2r89hweio4sfklnaiomziw2q12344.8.1},则有
\begin{align*}
\phi (w_2)=\phi (g)\phi (x)\phi (x' )\phi (h)=\phi (g)f(x)f(x)^{-1}\phi (h)=\phi (g)\phi (h)=\phi (w_1),
\end{align*}
即对$\forall w_1,w_2\in \widetilde{X^*}$,都有
\begin{align}\label{eq::iowhfiow3inofhiow368448iopefjw-4864}
w_1\sim w_2\Longrightarrow  \phi(w_1)=\phi(w_2).
\end{align}
定义$F(X)$到$G$的映射$\psi$,满足
\begin{align*}
\psi(\overline{w})=\phi(w),\forall \,\overline{w}\in F(X).
\end{align*}
若
$\overline{w_1}=\overline{w_2}$,则$w_1\sim w_2$,由\eqref{eq::iowhfiow3inofhiow368448iopefjw-4864}式知
\begin{align*}
\psi \left( \overline{w_1} \right) =\phi \left( w_1 \right) =\phi \left( w_2 \right) =\psi \left( \overline{w_2} \right) .
\end{align*}
故$\psi$是良定义的.对$\overline{w_1},\overline{w_2}\in F(x)$,有
\begin{align*}
\psi \left( \overline{w_1}\overline{w_2} \right) =\psi \left( \overline{w_1w_2} \right) =\phi \left( w_1w_2 \right) =\phi \left( w_1 \right) \phi \left( w_2 \right) =\psi \left( \overline{w_1} \right) \psi \left( \overline{w_2} \right) .
\end{align*}
故$\psi$为同态.显然$\psi(\overline{x})=\phi(x)=f(x)(\forall x\in X)$. 

最后证明$\psi$的唯一性.若$\psi'$为$F(X)$到$G$的同态且$\psi(\overline{x})=f(x)(\forall x\in X)$,则对$\forall x\in X$,有
\begin{align*}
\psi' \left( \overline{x} \right) =f\left( x \right) =\psi \left( \overline{x} \right) .
\end{align*}
对$\forall x'\in X'$,由\refthe{theorem:抽象代数--定理4.8.2}知$\left( \overline{x} \right) ^{-1}=\overline{x'}.$从而
\begin{align*}
\psi' \left( \overline{x'} \right) &=\psi' \left( \left( \overline{x} \right) ^{-1} \right) =\psi' \left( \overline{x} \right) ^{-1}=f\left( x \right) ^{-1}=f\left( x' \right) =\psi \left( \overline{x'} \right) .
\end{align*}
因此
\begin{align*}
\psi' \left( \overline{x} \right) =\psi \left( \overline{x} \right) ,\quad \forall x\in X\cup X' =X^*.
\end{align*}
对$\forall \overline{x_1x_2\cdots x_m}\in F\left( X \right) $,则$x_1,x_2,\cdots ,x_m\in X^*.$于是
\begin{align*}
&\psi' \left( \overline{x_1x_2\cdots x_m} \right) =\psi' \left( \overline{x_1}\overline{x_2}\cdots \overline{x_m} \right) =\psi' \left( \overline{x_1} \right) \psi' \left( \overline{x_2} \right) \cdots \psi' \left( \overline{x_m} \right) \\
&=\psi \left( \overline{x_1} \right) \psi \left( \overline{x_2} \right) \cdots \psi \left( \overline{x_m} \right) =\psi \left( \overline{x_1}\overline{x_2}\cdots \overline{x_m} \right) =\psi \left( \overline{x_1x_2\cdots x_m} \right) .
\end{align*}
因此$\psi$唯一.

\end{proof}


\begin{corollary}\label{corollary:抽象代数--推论4.8.1}
设非空集合$X=\{a_1,a_2,\cdots,a_n\}$,则$\alpha: x\to \overline{x},\,\forall x\in X$是$X$到$F(X)$中的单射.
\end{corollary}
\begin{remark}
由这个推论知,当$X$为非空有限集时,$X\cong \alpha(X)\subseteq F(X)$,从而$X$可视为$F(X)$的子集,此时,\refthe{theorem:抽象代数--定理4.8.3}中$\psi$的条件可改为
\begin{align*}
\psi(x) = f(x),\quad \forall x\in X.
\end{align*}
\end{remark}
\begin{proof}
在$\mathbb{Z}^n = \{(m_1,m_2,\cdots,m_n)\mid m_i\in \mathbb{Z},1\leqslant i\leqslant n\}$中定义加法运算为
\begin{align*}
(m_1,m_2,\cdots,m_n)+(l_1,l_2,\cdots,l_n) = (m_1+l_1,m_2+l_2,\cdots,m_n+l_n),
\end{align*}
则$\mathbb{Z}^n$是交换群,而$X$到$\mathbb{Z}^n$中映射
\begin{align*}
f: a_i\to (m_1,m_2,\cdots,m_n),\quad m_j=\delta _{ij}=\begin{cases}
1,&i=j\\
0,&i\ne j\\
\end{cases},\,1\leqslant i,j\leqslant n
\end{align*}
是单射.
如\reffig{figure:抽代--交换图4.8.1}所示,由\refthe{theorem:抽象代数--定理4.8.3}知有$F(X)$到$\mathbb{Z}^n$的同态$\psi$,使得
\begin{align*}
\psi(\overline{x}) = f(x),\quad \forall x\in X,
\end{align*}
即有$\psi\alpha = f$. 若$\alpha \left( a_i \right) =\alpha \left( a_j \right) \left( a_i,a_j\in X \right) $,即$\overline{a_i}=\overline{a_j}$,则两边同时作用$\psi$得
\begin{align*}
f\left( a_i \right) =\psi \left( \overline{a_i} \right) =\psi \left( \overline{a_j} \right) =f\left( a_j \right) .
\end{align*}
因为$f$是单射,所以$a_i=a_j.$
故$\alpha$也是单射.
\begin{figure}[H]
\centering
% https://q.uiver.app/#q=WzAsMyxbMCwwLCJYIl0sWzAsMiwiXFxtYXRoYmJ7Wn1ebiJdLFsyLDAsIkYoWCkiXSxbMCwyLCJhIl0sWzAsMSwiZiIsMl0sWzIsMSwiXFxwc2kgIl1d
\begin{tikzcd}
X && {F(X)} \\
\\
{\mathbb{Z}^n}
\arrow["a", from=1-1, to=1-3]
\arrow["f"', from=1-1, to=3-1]
\arrow["{\psi }", from=1-3, to=3-1]
\end{tikzcd}
\caption{}
\label{figure:抽代--交换图4.8.1}
\end{figure}
\end{proof}

\begin{corollary}\label{corollary:抽象代数--推论4.8.2}
设是有限生成群$G=\langle g_1,g_2,\cdots,g_n\rangle$,非空集合$X=\{a_1,a_2,\cdots,a_n\}$,$F(X)$是集合$X$的自由群.定义$X$到$G$的映射$f$,
\begin{align*}
f(a_i)=g_i,\quad 1\leqslant i\leqslant n.
\end{align*}
由\refthe{theorem:抽象代数--定理4.8.3}和\refcor{corollary:抽象代数--推论4.8.1}知,存在$F(X)$到$G$的满同态$\psi$满足
\begin{align*}
\psi(\overline{a_i})=\psi(a_i)=f(a_i)=g_i,\quad 1\leqslant i\leqslant n,
\end{align*}
则
\begin{align*}
G\cong F(X)/\ker \psi.
\end{align*}
称$\ker\psi$为$G$的生成元$g_1,g_2,\cdots,g_n$间的\textbf{关系集}. 若$\ker\psi$也是由有限个元素$w_1,w_2,\cdots,w_r$生成,即$\ker\psi = \langle w_1,w_2,\cdots, w_r\rangle$,则称$G$是\textbf{可有限表现的},而且
\begin{align*}
\psi(w_i)=e,\quad 1\leqslant i\leqslant r.
\end{align*}
称$w_1,w_2,\cdots,w_r$为$G$的生成元$g_1,g_2,\cdots,g_n$的一组\textbf{生成关系}.
\end{corollary}
\begin{proof}
由\refthe{theorem:生成子群的元素的形式}知
\begin{align*}
\langle G\rangle =\left\{ x_1x_2\cdots x_m\mid x_i\in G\cup G^{-1},1\leqslant i\leqslant m,m\in \mathbb{N} \right\}.
\end{align*}
于是对$\forall \,\overline{a_1a_2\cdots a_m}\in F\left( X \right) \left( 1\leqslant m\leqslant n \right)$,有
\begin{align*}
\psi \left( \overline{a_1a_2\cdots a_m} \right) =\psi \left( \overline{a_1}\overline{a_2}\cdots \overline{a_m} \right) =\psi \left( \overline{a_1} \right) \psi \left( \overline{a_2} \right) \cdots \psi \left( \overline{a_m} \right) =g_1g_2\cdots g_m\in \langle G\rangle.
\end{align*}
因此$\psi \left( F\left( X \right) \right) \subseteq \langle G\rangle$。对$\forall x_1x_2\cdots x_m\in \langle G\rangle \left( 1\leqslant m\leqslant n \right)$,不妨设
\begin{align*}
x_1x_2\cdots x_m=\left( g_{i_1}g_{i_2}\cdots g_{i_k} \right) \left( g_{i_{k+1}}^{-1}g_{i_{k+2}}^{-1}\cdots g_{i_m}^{-1} \right),\quad i_1,i_2,\cdots ,i_m\in \left\{ 1,2,\cdots ,m \right\}.
\end{align*}
从而
\begin{align*}
x_1x_2\cdots x_m&=\left( g_{i_1}g_{i_2}\cdots g_{i_k} \right) \left( g_{i_{k+1}}^{-1}g_{i_{k+2}}^{-1}\cdots g_{i_m}^{-1} \right) \\
&=\left( \psi \left( \overline{a_{i_1}} \right) \psi \left( \overline{a_{i_2}} \right) \cdots \psi \left( \overline{a_{i_k}} \right) \right) \left( \psi \left( \overline{a_{i_{k+1}}} \right) ^{-1}\psi \left( \overline{a_{i_{k+2}}} \right) ^{-1}\cdots \psi \left( \overline{a_{i_m}} \right) ^{-1} \right) \\
&=\psi \left( \overline{a_{i_1}a_{i_2}\cdots a_{i_k}} \right) \left( \psi \left( \left( \overline{a_{i_{k+1}}} \right) ^{-1} \right) \psi \left( \left( \overline{a_{i_{k+2}}} \right) ^{-1} \right) \cdots \psi \left( \left( \overline{a_{i_m}} \right) ^{-1} \right) \right) \\
&=\psi \left( \overline{a_{i_1}a_{i_2}\cdots a_{i_k}} \right) \left( \psi \left( \overline{a_{i_{k+1}}'} \right) \psi \left( \overline{a_{i_{k+2}}'} \right) \cdots \psi \left( \overline{a_{i_m}}' \right) \right) \\
&=\psi \left( \overline{a_{i_1}a_{i_2}\cdots a_{i_k}} \right) \psi \left( \overline{a_{i_{k+1}}'a_{i_{k+2}}'\cdots a_{i_m}'} \right) \\
&=\psi \left( \overline{a_{i_1}a_{i_2}\cdots a_{i_k}a_{i_{k+1}}'a_{i_{k+2}}'\cdots a_{i_m}'} \right) \in \psi \left( F\left( X \right) \right).
\end{align*}
因此$\psi \left( F\left( X \right) \right) \supseteq \langle G\rangle$。故$\psi \left( F\left( X \right) \right) =\langle G\rangle$,即$\psi$是$F\left( X \right)$到$G$的满同态。
故由\hyperref[theorem:群的同态基本定理]{群的同态基本定理}知$G\cong F(X)/\ker\psi$.

\end{proof}

\begin{proposition}
设$D_n$是保持正$n$边形不动的转动与反射(也叫对称)生成的群,通常称为\textbf{二面体群}.
以正$n$边形的中心为原点,并设$x$轴的正方向通过一个顶点. 设$a$是转动$\frac{2\pi}{n}$,而$b$是对$x$轴的反射. 容易看出$D_n$由$a$与$b$生成,即$D_n=\langle a,b\rangle.$

令$X=\{x_1,x_2\}$,于是由\refcor{corollary:抽象代数--推论4.8.2}知有$F(X)$到$D_n$的满同态$\psi$,使$\psi(x_1)=a,\psi(x_2)=b$. 则$x_1^n,x_2^2,x_1x_2x_1x_2$就是$D_n$的生成元$a,b$的一组生成关系. 
\end{proposition}
\begin{proof}
不难发现$D_n$中只有$n$个不同的转动和$n$个不同的反射对称,即
\begin{align*}
\mathrm{id},a,\cdots ,a^{n-1};\quad b,ab,\cdots ,a^{n-1}b.
\end{align*}
故$|D_n|=2n$且有
\begin{align*}
a^n = \text{id},\quad b^2 = \text{id},\quad abab = \text{id}.
\end{align*}
由上式知
\begin{align*}
\psi \left( x_{1}^{n} \right) =a^n=\mathrm{id},\quad \psi \left( x_{2}^{2} \right) =b^2=\mathrm{id},\quad \psi \left( x_1x_2x_1x_2 \right)=abab=\mathrm{id}.
\end{align*}
故$x_1^n,x_2^2,x_1x_2x_1x_2\in \ker\psi$.由\refcor{corollary:抽象代数--推论4.8.1}知可将$X$视为$F(X)$的子集,于是由$x_1^n,x_2^2,x_1x_2x_1x_2$生成的$F(X)$的子群$K$在$\ker \psi$中,故$|K|\leqslant |\ker\psi|.$

对$\forall x\in X^*$,用$\widetilde{x}\text{表示}x\text{在}F\left( X \right) \text{中的同余类}$.对$\forall \widetilde{x_{k_1}^{\varepsilon _1}x_{k_2}^{\varepsilon _2}\cdots x_{k_m}^{\varepsilon _m}}\in F\left( X \right) \left( k_i\in \left\{ 1,2 \right\}, \varepsilon _i\in \left\{ -1,1 \right\}, 1\leqslant i\leqslant m \right)$,有
\begin{align*}
\left( \widetilde{x_{k_1}^{\varepsilon _1}x_{k_2}^{\varepsilon _2}\cdots x_{k_m}^{\varepsilon _m}} \right) x_{1}^{n}\left( \widetilde{x_{k_1}^{\varepsilon _1}x_{k_2}^{\varepsilon _2}\cdots x_{k_m}^{\varepsilon _m}} \right) ^{-1}&=\left( \widetilde{x_{k_1}^{\varepsilon _1}x_{k_2}^{\varepsilon _2}\cdots x_{k_m}^{\varepsilon _m}} \right) \widetilde{x_{1}^{n}}\left( \widetilde{x_{k_m}^{-\varepsilon _m}x_{k_{m-1}}^{-\varepsilon _{m-1}}\cdots x_{k_1}^{-\varepsilon _1}} \right) 
\\
&=\widetilde{x_{k_1}^{\varepsilon _1}}\widetilde{x_{k_2}^{\varepsilon _2}}\cdots \widetilde{x_{k_m}^{\varepsilon _m}}\widetilde{x_{1}^{n}}\widetilde{x_{k_m}^{-\varepsilon _m}}\widetilde{x_{k_{m-1}}^{-\varepsilon _{m-1}}}\cdots \widetilde{x_{k_1}^{-\varepsilon _1}}\in F\left( X \right) ,
\end{align*}
\begin{align*}
\left( \widetilde{x_{k_1}^{\varepsilon _1}x_{k_2}^{\varepsilon _2}\cdots x_{k_m}^{\varepsilon _m}} \right) x_{2}^{2}\left( \widetilde{x_{k_1}^{\varepsilon _1}x_{k_2}^{\varepsilon _2}\cdots x_{k_m}^{\varepsilon _m}} \right) ^{-1}&=\left( \widetilde{x_{k_1}^{\varepsilon _1}x_{k_2}^{\varepsilon _2}\cdots x_{k_m}^{\varepsilon _m}} \right) \widetilde{x_{2}^{2}}\left( \widetilde{x_{k_m}^{-\varepsilon _m}x_{k_{m-1}}^{-\varepsilon _{m-1}}\cdots x_{k_1}^{-\varepsilon _1}} \right) 
\\
&=\widetilde{x_{k_1}^{\varepsilon _1}}\widetilde{x_{k_2}^{\varepsilon _2}}\cdots \widetilde{x_{k_m}^{\varepsilon _m}}\widetilde{x_{2}^{2}}\widetilde{x_{k_m}^{-\varepsilon _m}}\widetilde{x_{k_{m-1}}^{-\varepsilon _{m-1}}}\cdots \widetilde{x_{k_1}^{-\varepsilon _1}}\in F\left( X \right) ,
\end{align*}
\begin{align*}
\left( \widetilde{x_{k_1}^{\varepsilon _1}x_{k_2}^{\varepsilon _2}\cdots x_{k_m}^{\varepsilon _m}} \right) \left( x_1x_2x_1x_2 \right) \left( \widetilde{x_{k_1}^{\varepsilon _1}x_{k_2}^{\varepsilon _2}\cdots x_{k_m}^{\varepsilon _m}} \right) ^{-1}&=\left( \widetilde{x_{k_1}^{\varepsilon _1}x_{k_2}^{\varepsilon _2}\cdots x_{k_m}^{\varepsilon _m}} \right) \left( \widetilde{x_1x_2x_1x_2} \right) \left( \widetilde{x_{k_m}^{-\varepsilon _m}x_{k_{m-1}}^{-\varepsilon _{m-1}}\cdots x_{k_1}^{-\varepsilon _1}} \right) 
\\
&=\widetilde{x_{k_1}^{\varepsilon _1}}\widetilde{x_{k_2}^{\varepsilon _2}}\cdots \widetilde{x_{k_m}^{\varepsilon _m}}\widetilde{x_1}\widetilde{x_2}\widetilde{x_1}\widetilde{x_2}\widetilde{x_{k_m}^{-\varepsilon _m}}\widetilde{x_{k_{m-1}}^{-\varepsilon _{m-1}}}\cdots \widetilde{x_{k_1}^{-\varepsilon _1}}\in F\left( X \right) .
\end{align*}
因而$K$是$F(X)$的正规子群.

又$F(X)/\ker\psi$与$D_n$同构,从而$[F(X):\ker\psi]=|D_n|=2n$.因而只需证明$[F(X):K]\leqslant |D_n|=2n$,则由\refcor{corollary:抽象代数-推论 1.3.5}可得
\begin{align*}
\left[ F\left( X \right) :K \right] =\frac{\left| F\left( X \right) \right|}{\left| K \right|}\leqslant 2n\Longrightarrow \left| K \right|\geqslant \frac{\left| F\left( X \right) \right|}{2n}=\frac{\left| F\left( X \right) \right|}{\left[ F\left( X \right) :\mathrm{ker}\psi \right]}=\frac{\left| F\left( X \right) \right|}{\left| F\left( X \right) \right|}\cdot \left| \mathrm{ker}\psi \right|=\left| \mathrm{ker}\psi \right|.
\end{align*}
因此$|K|=|\ker\psi|$,故$\ker\psi=K$,即$x_1^n,x_2^2,x_1x_2x_1x_2$就是$D_n$的生成元$a,b$的一组生成关系. 

注意到
\begin{align*}
F\left( X \right) =\left\{ \widetilde{y_1y_2\cdots y_m}\mid y_i\in X^*=\left\{ x_1,x_2,x_{1}^{-1},x_{2}^{-1} \right\} ,1\leqslant i\leqslant m,m\in \mathbb{N} \right\} ,
\end{align*}
故$\overline{x_1}=x_1K,\overline{x_2}=x_2K$为$F(X)/K$的生成元,即$F(X)/K=\langle \overline{x_1},\overline{x_2}\rangle$.由$x_1^n,x_2^2,x_1x_2x_1x_2\in K$有
\begin{align*}
\overline{x_1^n}=\overline{x_2^2}=\overline{x_1}\overline{x_2}\overline{x_1}\overline{x_2}=\overline{e},
\end{align*}
故
\begin{align*}
\overline{x_1}\overline{x_2}\overline{x_1}\overline{x_2}=\overline{e}\Longrightarrow \overline{x_{1}^{-1}}\left( \overline{x_1}\overline{x_2}\overline{x_1}\overline{x_2} \right) \overline{x_2}=\overline{x_{1}^{-1}}\overline{x_2}\Longleftrightarrow \overline{x_2}\overline{x_1}=\overline{x_{1}^{-1}}\overline{x_2}.
\end{align*}
假设$\overline{x_2}\overline{x_{1}^{k}}=\overline{x_{1}^{-k}}\overline{x_2}$,则
\begin{align*}
\overline{x_2}\overline{x_{1}^{k+1}}&=\left( \overline{x_2}\overline{x_{1}^{k}} \right) \overline{x_1}=\left( \overline{x_{1}^{-k}}\overline{x_2} \right) \overline{x_1}=\overline{x_{1}^{-k}}\left( \overline{x_2}\overline{x_1} \right) =\overline{x_{1}^{-k}}\left( \overline{x_{1}^{-1}}\overline{x_2} \right) =\overline{x_{1}^{-\left( k+1 \right)}}\overline{x_2}.
\end{align*}
故由数学归纳法知
\begin{align*}
\overline{x_2}\overline{x_{1}^{k}}=\overline{x_{1}^{-k}}\overline{x_2},\quad 1\leqslant k\leqslant n.
\end{align*}
再结合$\left( \overline{x_{2}^{2}} \right) =\overline{x_2}$可得
\begin{align}
\overline{x_{1}^{k}}\overline{x_2}=\overline{x_2}\overline{x_{1}^{-k}},\quad 1\leqslant k\leqslant n.\label{eq::wfgjio3ji2fidmsogdfszzdf.2}
\end{align}
令$G_1=\{\overline{x_1^k},\overline{x_1^k}\overline{x_2}\mid 1\leqslant k\leqslant n\}$,则
对$\forall k,l\in \left\{ 1,2,\cdots ,n \right\}$,由$\overline{x_{1}^{n}}=\overline{e}$知
\begin{align*}
\overline{x_{1}^{k+l}}=\begin{cases}
\overline{x_{1}^{k+l-n}}\in G_1,&n<k+l\leqslant 2n,\\
\overline{x_{1}^{k+l}}\in G_1,&1\leqslant k+l\leqslant n.
\end{cases}
\end{align*}
于是再利用\eqref{eq::wfgjio3ji2fidmsogdfszzdf.2}式可得
\begin{enumerate}[(i)]
\item $\overline{x_{1}^{k}}\cdot \overline{x_{1}^{-l}}=\overline{x_{1}^{k-l}}\in G_1;$
\item $\left( \overline{x_{1}^{k}}\overline{x_2} \right) \cdot \left( \overline{x_{1}^{l}}\overline{x_2} \right) ^{-1}=\left( \overline{x_{1}^{k}}\overline{x_2} \right) \cdot \left( \overline{x_{2}^{-1}}\overline{x_{1}^{-l}} \right) =\left( \overline{x_{1}^{k}}\overline{x_2} \right) \cdot \left( \overline{x_2}\overline{x_{1}^{-l}} \right) =\overline{x_{1}^{k-l}}\in G_1;$
\item $\left( \overline{x_{1}^{l}}\overline{x_2} \right) \cdot \overline{x_{1}^{-k}}\xlongequal{\eqref{eq::wfgjio3ji2fidmsogdfszzdf.2}\text{式}}\left( \overline{x_2}\overline{x_{1}^{-l}} \right) \cdot \overline{x_{1}^{-k}}=\overline{x_2}\overline{x_{1}^{-k-l}}\xlongequal{\eqref{eq::wfgjio3ji2fidmsogdfszzdf.2}\text{式}}\overline{x_{1}^{k+l}}\overline{x_2}\in G_1;$
\item $\overline{x_{1}^{k}}\cdot \left( \overline{x_{1}^{l}}\overline{x_2} \right) ^{-1}=\overline{x_{1}^{k}}\cdot \left( \overline{x_{2}^{-1}}\overline{x_{1}^{-l}} \right) =\overline{x_{1}^{k}}\cdot \left( \overline{x_2}\overline{x_{1}^{-l}} \right) \xlongequal{\eqref{eq::wfgjio3ji2fidmsogdfszzdf.2}\text{式}}\overline{x_{1}^{k}}\cdot \left( \overline{x_{1}^{l}}\overline{x_2} \right) =\overline{x_{1}^{k+l}}\overline{x_2}\in G_1.$
\end{enumerate}
因而由\rrefthe{theorem:子群的充要条件--抽象代数}{theorem:子群的充要条件--抽象代数-4}知$F(X)/K=\langle \overline{x_1},\overline{x_2}\rangle\subseteq G_1$.
故$G_1=\{\overline{x_1^k},\overline{x_1^k}\overline{x_2}\mid 1\leqslant k\leqslant n\}$为$F(X)/K$的子群,显然$|G_1|\leqslant 2n$.但由$\overline{x_1},\overline{x_2}\in G_1$知$G_1=F(X)/K$,因而$[F(X):K]\leqslant 2n$.

\end{proof}




\end{document}
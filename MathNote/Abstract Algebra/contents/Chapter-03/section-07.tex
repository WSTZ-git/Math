\documentclass[../../main.tex]{subfiles}% 注意这里的文件路径不能用 ./main.tex ,否则用latexmk编译子文件会报错
\graphicspath{{\subfix{./image/}}} % 指定图片目录,后续可以直接使用图片文件名
% 注意这里的文件路径不能用 ../../image/ ,否则用latexmk编译子文件会报错

% 例如:
% \begin{figure}[H]
% \centering
% \includegraphics[scale=0.3]{图.png}
% \caption{}
% \label{figure:图}
% \end{figure}
% 注意:上述\label{}一定要放在\caption{}之后,否则引用图片序号会只会显示??.

\begin{document}

\section{Jordan-Hölder定理}

\begin{definition}
群$G$的两个次正规(正规)序列
\begin{align*}
G = G_1 \supseteq G_2 \supseteq \cdots \supseteq G_r = \{1\}, \\
G = H_1 \supset H_2 \supseteq \cdots \supseteq H_s = \{1\}
\end{align*}
称为\textbf{同构的},如果这两个序列的因子集$\{G_i/G_{i+1} \mid 1 \leqslant i \leqslant r-1\}$, $\{H_j/H_{j+1} \mid 1 \leqslant j \leqslant s-1\}$之间有一一对应关系,并且对应的因子是同构的.
\end{definition}
\begin{remark}
显然, 若两个次正规(正规)序列同构, 则它们有相同的长度.
\end{remark}

\begin{lemma}[Zassenhaus定理]\label{lemma:抽象代数-引理4.7.1-Zassenhaus定理}
设$H, K$是群$G$的子群, 又$H^*, K^*$分别是$H, K$的正规子群, 则
\begin{gather*}
H^*(H\cap K^*) \lhd H^*(H\cap K), 
\quad K^*(H^*\cap K) \lhd K^*(H\cap K), \\
H^*(H\cap K)/H^*(H\cap K^*) \cong (H\cap K)/(H\cap K^*)(H^*\cap K)\cong K^*(H\cap K)/K^*(H^*\cap K).
\end{gather*}
\begin{gather*}
(H\cap K^*)H^*\lhd (H\cap K)H^*,\quad (H^*\cap K)K^*\lhd (H\cap K)K^*,
\\
(H\cap K)H^*/(H\cap K^*)H^*\cong (H\cap K)/(H\cap K^*)(H^*\cap K)\cong (H\cap K)K^*/(H^*\cap K)K^*.
\end{gather*}
\end{lemma}
\begin{proof}
因为$H^*$是$H$的正规子群, $H\cap K, H\cap K^*$都是$H$的子群, 故由\rrefthe{theorem:抽象代数--定理1.7.3}{theorem:抽象代数--定理1.7.3-2}知$H^*(H\cap K), H^*(H\cap K^*)$都是$H$的子群. 同样$K^*(H\cap K), K^*(H^*\cap K)$都是$K$的子群, 而且$H^*(H\cap K^*)\subseteq H^*(H\cap K)$, $K^*(H^*\cap K)\subseteq K^*(H\cap K)$. 对$\forall a\in H^*\cap K,b\in H\cap K^*,x\in H\cap K,$有
\begin{align*}
xax^{-1}\in H^*\cap K,\quad xbx^{-1}\in H\cap K^*.
\end{align*}
故$H^*\cap K,H\cap K^*\lhd H\cap K.$再由\rrefcor{corollary:群同态第二定理推论}{corollary:群同态第二定理推论-2}知$(H\cap K^*)(H^*\cap K)=L$也是$H\cap K$的正规子群.

作$H^*(H\cap K)$到$(H\cap K)/L$上的映射$\phi$,
\begin{align*}
\phi(hx) = xL, \quad \forall h\in H^*, x\in H\cap K.
\end{align*}
若$h_1x_1 = h_2x_2(h_i\in H^*, x_i\in H\cap K, i=1, 2)$, 则
\begin{align*}
h_2^{-1}h_1 = x_2x_1^{-1}\in H^*\cap (H\cap K)=H^*\cap K\subseteq L\Longrightarrow x_2L=x_1L,
\end{align*}
因而$\phi(h_1x_1)=\phi(h_2x_2)$, 故$\phi$是良定义的.

再证明$\phi$是同态, 由于$H^*\lhd H$, $x_1h_2x_1^{-1}\in H^*$, 故
\begin{align*}
\phi((h_1x_1)(h_2x_2)) = \phi(h_1(x_1h_2x_1^{-1})x_1x_2) = x_1x_2L = (x_1L)\cdot (x_2L) = \phi(h_1x_1)\cdot \phi(h_2x_2),
\end{align*}
故$\phi$是$H^*(H\cap K)$到$H\cap K/L$上的同态,显然$\phi$还是满同态.注意到
\begin{align*}
hx\in \ker\phi\iff \phi(hx)=xL=L\iff x\in L\iff hx\in H^*L = H^*(H^*\cap K)(H\cap K^*) = H^*(H\cap K^*),
\end{align*}
故$\ker \phi = H^*(H\cap K^*)$.因而由\hyperref[theorem:群的同态基本定理]{群的同态基本定理}得
\begin{gather*}
H^*(H\cap K^*)\lhd H^*(H\cap K),
\\
H^*(H\cap K)/H^*(H\cap K^*) \cong (H\cap K)/(H\cap K^*)(H^*\cap K).
\end{gather*}
同理可得
\begin{gather*}
K^*(H^*\cap K)\lhd K^*(H\cap K),
\\
K^*(H\cap K)/K^*(H^*\cap K) \cong (H\cap K)/(H\cap K^*)(H^*\cap K).
\end{gather*}
故
\begin{align*}
H^*(H\cap K)/H^*(H\cap K^*) \cong (H\cap K)/(H\cap K^*)(H^*\cap K)\cong K^*(H\cap K)/K^*(H^*\cap K).
\end{align*}
综上,同理可得
\begin{gather*}
(H\cap K^*)H^*\lhd (H\cap K)H^*,\quad (H^*\cap K)K^*\lhd (H\cap K)K^*,
\\
(H\cap K)H^*/(H\cap K^*)H^*\cong (H\cap K)/(H\cap K^*)(H^*\cap K)\cong (H\cap K)K^*/(H^*\cap K)K^*.
\end{gather*}

\end{proof}

\begin{theorem}[Schreier定理]\label{theorem:抽象代数-定理4.7.1-Schreier定理}
群$G$的两个次正规(正规)序列有同构的加细.
\end{theorem}
\begin{proof}
设群$G$有次正规(正规)序列
\begin{align}
G = G_1 \supseteq G_2 \supseteq\cdots \supseteq G_r = \{1\}, \label{eq:::--289h2389fsdgsdvcxhrsd::--q89h2dfwfzznmnghmghkhlyujf21.1} \\
G = H_1 \supseteq H_2 \supseteq \cdots \supseteq H_s = \{1\}. \label{eq:::--289h2389fsdgsdvcxhrsd::--q89h2dfwfzznmnghmghkhlyujf21.2}
\end{align}
令
\begin{align*}
G_{i,k} &= (G_i\cap H_k)G_{i+1}, \quad 1\leqslant i\leqslant r-1, 1\leqslant k\leqslant s, \\
G_{r,s} &= \{1\}, \\
H_{i,k} &= (H_i\cap G_k)H_{i+1}, \quad 1\leqslant i\leqslant s-1, 1\leqslant k\leqslant r, \\
H_{s,r} &= \{1\}.
\end{align*}
由$G_{k+1}\lhd G_k(1\leqslant k\leqslant r-1),H_{k+1}\lhd H_k(1\leqslant k\leqslant s-1)$及\hyperref[lemma:抽象代数-引理4.7.1-Zassenhaus定理]{Zassenhaus定理}可得
\begin{align*}
G_{i,k+1}\lhd G_{i,k},  \quad 1\leqslant i\leqslant r-1,1\leqslant k\leqslant s-1, 
\\
H_{i,k+1}\lhd H_{i,k},\quad 1\leqslant i\leqslant s-1, 1\leqslant k\leqslant r-1.
\end{align*}
从而
\begin{gather*}
G_{i+1,1}=(G_{i+1}\cap G)G_{i+2}=G_{i+1}=(G_i\cap H_s)G_{i+1}=G_{i,s}\lhd G_{i,s-1},\,\,1\leqslant i\leqslant r-1,
\\
H_{i+1,1}=(H_{i+1}\cap G)H_{i+2}=H_{i+1}=(H_i\cap G_r)H_{i+1}=H_{i,r}\lhd H_{i,r-1},\,\,1\leqslant i\leqslant s-1.
\end{gather*}
又若序列\eqref{eq:::--289h2389fsdgsdvcxhrsd::--q89h2dfwfzznmnghmghkhlyujf21.1}, 序列\eqref{eq:::--289h2389fsdgsdvcxhrsd::--q89h2dfwfzznmnghmghkhlyujf21.2}都是正规序列,即$G_i, H_j$($1\leqslant i\leqslant s, 1\leqslant j\leqslant r$)均为$G$的正规子群,则由\hyperref[proposition:正规子群的基本性质]{命题\ref{proposition:正规子群的基本性质-5}\ref{proposition:正规子群的基本性质-3}}知$G_{i,k}, H_{j,k}$也是$G$的正规子群. 这样,即得序列\eqref{eq:::--289h2389fsdgsdvcxhrsd::--q89h2dfwfzznmnghmghkhlyujf21.1}, 序列\eqref{eq:::--289h2389fsdgsdvcxhrsd::--q89h2dfwfzznmnghmghkhlyujf21.2}的加细的次正规(正规)序列
\begin{align}
G&=G_{1,1} \supseteq G_{1,2} \supseteq \cdots \supseteq G_{1,s-1} \nonumber \\
&\supseteq G_{2,1} \supseteq G_{2,2} \supseteq \cdots \supseteq G_{2,s-1} \nonumber \\
&\supseteq \cdots \nonumber \\
&\supseteq G_{r-1,1} \supseteq G_{r-1,2} \supseteq \cdots \supseteq G_{r-1,s-1} \supseteq G_{r,s} = \{1\}, \label{eq:::--289h2389fsdgsdvcxhrsd::--q89h2dfwfzznmnghmghkhlyujf21.3}
\end{align}
\begin{align}
G&=H_{1,1} \supseteq H_{1,2} \supseteq \cdots \supseteq H_{1,r-1} \nonumber \\\
&\supseteq H_{2,1} \supseteq H_{2,2} \supseteq \cdots \supseteq H_{2,r-1} \nonumber \\
&\supseteq \cdots\nonumber \\
&\supseteq H_{s-1,1} \supseteq H_{s-1,2} \supseteq \cdots \supseteq H_{s-1,r-1} \supseteq H_{s,r} = \{1\}. \label{eq:::--289h2389fsdgsdvcxhrsd::--q89h2dfwfzznmnghmghkhlyujf21.4}
\end{align}
由\hyperref[lemma:抽象代数-引理4.7.1-Zassenhaus定理]{Zassenhaus定理}有
\begin{gather*}
\frac{G_{i,k}}{G_{i,k+1}} = \frac{(G_i\cap H_k)G_{i+1}}{(G_i\cap H_{k+1})G_{i+1}} \cong \frac{(G_i\cap H_k)H_{k+1}}{(G_{i+1}\cap H_k)H_{k+1}} = \frac{H_{k,i}}{H_{k,i+1}},
\\
H_{s-1,r-1}=H_{s-1}\cap G_{r-1}=G_{r-1,s-1}
\end{gather*}
故序列\eqref{eq:::--289h2389fsdgsdvcxhrsd::--q89h2dfwfzznmnghmghkhlyujf21.3}, \eqref{eq:::--289h2389fsdgsdvcxhrsd::--q89h2dfwfzznmnghmghkhlyujf21.4}是$G$的同构的次正规(正规)序列.

\end{proof}

\begin{definition}
\begin{enumerate}
\item 群$G$的次正规序列
\begin{align*}
G = G_1 \supseteq G_2 \supseteq \cdots \supseteq G_r = \{1\}
\end{align*}
的因子$G_i/G_{i+1}(1\leqslant i\leqslant r-1)$如果都是单群, 那么称此序列为$G$的\textbf{合成序列}.

\item 群$G$的次正规序列
\begin{align*}
G = G_1 \supseteq G_2 \supseteq \cdots \supseteq G_r = \{1\}
\end{align*}
的因子$G_i/G_{i+1}(1\leqslant i\leqslant r-1)$如果都是单群, 那么称此序列为$G$的\textbf{主序列}.
\end{enumerate}
\end{definition}
\begin{remark}
显然群$G$的主序列必是合成序列.
\end{remark}

\begin{proposition}
若$G = G_1 \supseteq G_2 \supseteq \cdots \supseteq G_r = \{1\}$为$G$的合成序列, 则$G_{i+1}$是$G_i$的极大正规子群.
\end{proposition}
\begin{proof}


\end{proof}

\begin{example}
设$G=\langle a \rangle$为15阶循环群,则
\begin{align*}
G \supseteq \langle a^3 \rangle \supseteq \{1\}, \\
G \supseteq \langle a^5 \rangle \supseteq \{1\}
\end{align*}
是两个同构的正规序列.并且这两个序列都是$G$的主序列,也是合成序列.
\end{example}
\begin{proof}


\end{proof}

\begin{example}
设$n\geqslant 5$, 则$S_n \supseteq A_n \supseteq \{1\}$是$S_n$的主序列, 也是$S_n$的合成序列.
\end{example}

\begin{example}
序列$S_4 \supseteq A_4 \supseteq K_4 \supseteq \langle (12)(34) \rangle \supseteq\{1\}$是$S_4$的合成序列, 但不是$S_4$的主序列.
\end{example}
\begin{proof}


\end{proof}

\begin{example}
无限循环群$G = \langle a \rangle$无合成序列.
\end{example}
\begin{proof}
事实上, 若$G$有合成序列
\begin{align*}
G = G_1 \supseteq G_2 \supseteq \cdots \supseteq G_{r-1} \supseteq G_r = \{1\},
\end{align*}
则$G_{r-1}/G_r = G_{r-1} = \langle a^k \rangle$为单群, 但$\langle a^k \rangle$是无限循环群, 而由\refpro{proposition:阶数大于2的循环群必非单群}知无限循环群非单群,矛盾!

\end{proof}

\begin{theorem}[Jordan-Hölder定理]\label{theorem:抽象代数--定理4.7.2-Jordan-Hölder定理}
若群$G$有合成(主)序列, 则$G$的任两个合成(主)序列是同构的.
\end{theorem}
\begin{remark}
这个定理表明:\textbf{任意群的合成(主)序列的因子在同构意义下唯一.}
\end{remark}
\begin{proof}
设$G$有两合成(主)序列
\begin{align}
G = G_1 \supseteq G_2 \supseteq \cdots \supseteq G_r = \{1\},\label{eq:::xulie---1} \\
G = H_1 \supseteq H_2 \supseteq \cdots \supseteq H_s = \{1\}.\label{eq:::xulie---2}
\end{align}
由于$G_i/G_{i+1}(1\leqslant i\leqslant r-1), H_j/H_{j+1}(1\leqslant i\leqslant s-1)$都是单群, 因而序列\eqref{eq:::xulie---1}和序列\eqref{eq:::xulie---2}都不可能再加细.否则,存在$k_1\in \{1,2,\cdots,r\},k_2\in \{1,2,\cdots,s\}$以及群$K_1,K_2$,使得
\begin{gather}
G_{k_1+1}\subset K_1\subset G_{k_1},
\quad  H_{k_2+1}\subset K_2\subset H_{k_2};\label{eq::ihfioj32j34092---jfioejf28j3rwfizxvnbnxm}
\\
G_{k_1+1}\lhd K_1\lhd G_{k_1},
\quad  H_{k_2+1}\lhd K_2\lhd H_{k_2}.\nonumber
\end{gather}
由\rrefcor{corollary:群同态第二定理推论}{corollary:群同态第二定理推论-2}知
\begin{align*}
K_1/G_{k_1+1}\lhd G_{k_1}/G_{k_1+1},\quad K_2/G_{k_2+1}\lhd H_{k_2}/H_{k_2+1}.
\end{align*}
又因为$G_{k_1}/G_{k_1+1},H_{k_2}/H_{k_2+1}$都是单群,所以
\begin{gather*}
K_1/G_{k_1+1}=G_{k_1+1}\text{或}G_{k_1}/G_{k_1+1},\quad K_2/H_{k_2+1}=H_{k_2+1}\text{或}H_{k_2}/H_{k_2+1}.
\end{gather*}
即
\begin{gather*}
K_1=G_{k_1+1}\text{或}G_{k_1},\quad K_2=H_{k_2+1}\text{或}H_{k_2}.
\end{gather*}
这与\eqref{eq::ihfioj32j34092---jfioejf28j3rwfizxvnbnxm}式矛盾!

另外, 由\hyperref[theorem:抽象代数-定理4.7.1-Schreier定理]{Schreier定理}知序列\eqref{eq:::xulie---1}和序列\eqref{eq:::xulie---2}可加细为同构的次正规(正规)序列,分别记为序列$S$和序列$T$,则序列$S$与序列$T$同构.而已经证明序列\eqref{eq:::xulie---1}和序列\eqref{eq:::xulie---2}不可加细,故序列$S$等于序列\eqref{eq:::xulie---1},序列$T$等于序列\eqref{eq:::xulie---2}.
因而序列\eqref{eq:::xulie---1}和序列\eqref{eq:::xulie---2}必然同构.

\end{proof}

\begin{definition}
设$G$是群, $\Omega$为一集合, 如果有$\Omega \times G$到$G$的映射$(\sigma, a) \to \sigma(a)(\forall \sigma \in \Omega, a \in G)$满足
\begin{align*}
\sigma(ab) = \sigma(a)\sigma(b),\ \forall \sigma \in \Omega,\ a, b \in G,
\end{align*}
那么称$G$为\textbf{带算子集}$\mathbf{\Omega}$\textbf{的群}, 简称$\mathbf{\Omega}$\textbf{群}. $\Omega$的元素称为$G$的\textbf{算子}.
\end{definition}
\begin{remark}
事实上, 由$\Omega$群的定义又知$\Omega$中每个元素$\sigma$都是群$G$的自同态.
\end{remark}

\begin{example}
设$G$是Abel群, 运算为加法, 则$G$是一个$\mathbb{Z}$群.
\end{example}
\begin{proof}
事实上, $(n, a) \to na(n \in \mathbb{Z}, a \in G)$为$\mathbb{Z} \times G$到$G$的映射.并且
\begin{align*}
n(a+b)=n\cdot(a+b)=na+nb=n(a)+n(b),\,\,\forall n\in \mathbb{Z},a,b\in G.
\end{align*}

\end{proof}

\begin{example}
设$M$是环$R$上的左模, 则$M$是$R$群.
特别地, 域$F$上的线性空间可看成$F$群.
\end{example}
\begin{proof}
事实上, $R \times M$到$M$的映射为$(a, x) \to ax(a \in R,\ x \in M)$.

\end{proof}

\begin{definition}
设$G$为$\Omega$群, $H$是$G$的子群,且满足
\begin{align*}
\sigma(H) \subseteq H,\ \forall \sigma \in \Omega,
\end{align*}
则称为$H$为$G$的$\mathbf{\Omega}$\textbf{子群}. 又若$H \lhd G$, 即$H \subseteq G$且$H \lhd G$, 则称$H$为$G$的$\mathbf{\Omega}$\textbf{正规子群}.
\end{definition}

\begin{proposition}
左$R$模$M$可看成$R$群, 其$R$子群$M_1$就是$M$的子模, 特别是一个环$R$, 即可看成左$R$模, 因而也可看成$R$模. 这时, $R$子群为$R$的左理想, 环$R$也可看成右$R$模、$R$群, 这时$R$子群为右理想, 环$R$也可看成$R$双模, 相应的$R$子群为$R$的理想.
\end{proposition}
\begin{remark}
这个命题表明:环与模都是特殊的$\Omega$群.
\end{remark}
\begin{proof}


\end{proof}

\begin{example}
设$G$是群,若$\Omega = \mathrm{Int}G$($G$的内自同构的集合), 则$G$的$\Omega$子群就是$G$的正规子群.
\end{example}
\begin{proof}


\end{proof}

\begin{definition}
设$G$是群.
\begin{enumerate}
\item 若$\Omega = \mathrm{Aut}G$($G$的自同构的集合), 则$G$的$\Omega$子群称为$G$的\textbf{特征子群};
\item 若$\Omega = \mathrm{End}G$($G$的自同态的集合), 则$G$的$\Omega$子群称为$G$的\textbf{完全不变子群}.
\end{enumerate}
\end{definition}

\begin{theorem}
设$H$是$\Omega$群$G$的$\Omega$正规子群, $\pi$为$G$到$G/H$的自然同态, 则$\Omega \times G/H$到$G/H$上的映射$(\sigma, \pi(a)) \to \pi(\sigma(a))(\sigma \in \Omega, \pi(a) \in G/H)$使得$G/H$也是$\Omega$群, 称为$G$对$H$的$\mathbf{\Omega}$\textbf{商群}.
\end{theorem}
\begin{proof}
若$(\sigma,\pi(a))=(\sigma',\pi(b))\in \Omega \times G/H$,则$\sigma=\sigma',\pi(a) = \pi(b)(a,b\in G)$,于是$ab^{-1} \in H$, 从而有$\sigma(a)\sigma'(b)^{-1}=\sigma(a)\sigma(b)^{-1} = \sigma(ab^{-1}) \in H$, 故$\pi(\sigma(a)) = \pi(\sigma'(b))$. 故$\Omega \times G/H$到$G/H$的映射$(\sigma, \pi(a)) \to \pi(\sigma(a))$是良定义的.

又$\forall a, b \in G, \sigma \in \Omega$有
\begin{align*}
\sigma(\pi(a)\pi(b)) &= \sigma(\pi(ab)) = \pi(\sigma(ab)) = \pi(\sigma(a)\sigma(b)) \\
&= \pi(\sigma(a))\pi(\sigma(b)) = \sigma(\pi(a))\sigma(\pi(b)),
\end{align*}
于是$G/H$是$\Omega$群.


\end{proof}

\begin{definition}
设$G, G_1$都是$\Omega$群, $\phi$是群$G$到群$G_1$的同态且满足$\phi\sigma = \sigma\phi(\forall \sigma \in \Omega)$, 则称$\phi$是$\mathbf{\Omega}$\textbf{同态}. 若$\phi$还是群的同构, 则称$\phi$是$\mathbf{\Omega}$\textbf{同构}.
\end{definition}
\begin{remark}
与通常的群的同态基本定理一样, 可证$\Omega$群的同态基本定理及其他一些定理, 如Zassenhaus引理. 同样, 也可以引入$\Omega$群的次正规(正规)序列、合成序列和主序列等概念, 并能证明Schreier定理与Jordan-Hölder定理. 由于环与模均可看成特殊的$\Omega$群, 因而就可将Jordan-Hölder定理用于环与模的研究.
\end{remark}

\begin{proposition}\label{proposition:Omega正规子群可诱导一个自然Omega同态}
若$H$为$\Omega$群$G$的$\Omega$正规子群, 则$G$到$G/H$的自然同态$\pi$是$\Omega$同态.
\end{proposition}
\begin{proof}


\end{proof}

\begin{definition}
$R$模$M$的有限子模序列
\begin{align*}
M = M_1 \supseteq M_2 \supseteq \cdots \supseteq M_r = 0
\end{align*}
满足$M_i/M_{i+1}(i = 1, 2, \cdots, r-1)$为单模, 则该序列称为$M$的一个\textbf{合成序列}, $M_i/M_{i+1}$称为\textbf{合成因子}.
\end{definition}

\begin{theorem}[模的Jordan-Hölder定理]\label{theorem:抽象代数--定理4.7.4-模的Jordan-Hölder定理}
若$R$模$M$有合成序列, 则$M$的任两合成序列是同构的, 即它们的合成因子在同构意义下唯一.
\end{theorem}
\begin{remark}
若将$M$看成带算子集$R$的$R$群, 则这个定理就是群的Jordan-Hölder定理的特殊情形.与群的Jordan-Hölder定理同样证明.
\end{remark}
\begin{proof}


\end{proof}









\end{document}
\documentclass[../../main.tex]{subfiles}% 注意这里的文件路径不能用 ./main.tex ,否则用latexmk编译子文件会报错
\graphicspath{{\subfix{./image/}}} % 指定图片目录,后续可以直接使用图片文件名
% 注意这里的文件路径不能用 ../../image/ ,否则用latexmk编译子文件会报错

% 例如:
% \begin{figure}[H]
% \centering
% \includegraphics[scale=0.3]{图.png}
% \caption{}
% \label{figure:图}
% \end{figure}
% 注意:上述\label{}一定要放在\caption{}之后,否则引用图片序号会只会显示??.

\begin{document}

\section{Jordan-Hölder定理}

\begin{definition}
群$G$的两个次正规(正规)序列
\begin{align*}
G = G_1 \supseteq G_2 \supseteq \cdots \supseteq G_r = \{1\}, \\
G = H_1 \supset H_2 \supseteq \cdots \supseteq H_s = \{1\}
\end{align*}
称为\textbf{同构的},如果这两个序列的因子集$\{G_i/G_{i+1} \mid 1 \leqslant i \leqslant r-1\}$, $\{H_j/H_{j+1} \mid 1 \leqslant j \leqslant s-1\}$之间有一一对应关系,并且对应的因子是同构的.
\end{definition}
\begin{remark}
显然, 若两个次正规(正规)序列同构, 则它们有相同的长度.
\end{remark}

\begin{example}
设$G=\langle a \rangle$为15阶循环群,则
\begin{align*}
G \supseteq \langle a^3 \rangle \supseteq \{1\}, \\
G \supseteq \langle a^5 \rangle \supseteq \{1\}
\end{align*}
是两个同构的正规序列.
\end{example}
\begin{proof}


\end{proof}

\begin{lemma}[Zassenhaus定理]\label{lemma:抽象代数-引理4.7.1-Zassenhaus定理}
设$H, K$是群$G$的子群, 又$H^*, K^*$分别是$H, K$的正规子群, 则
\begin{gather*}
H^*(H\cap K^*) \lhd H^*(H\cap K), 
\quad K^*(H^*\cap K) \lhd K^*(H\cap K), \\
H^*(H\cap K)/H^*(H\cap K^*) \cong (H\cap K)/(H\cap K^*)(H^*\cap K)\cong K^*(H\cap K)/K^*(H^*\cap K).
\end{gather*}
\begin{gather*}
(H\cap K^*)H^*\lhd (H\cap K)H^*,\quad (H^*\cap K)K^*\lhd (H\cap K)K^*,
\\
(H\cap K)H^*/(H\cap K^*)H^*\cong (H\cap K)/(H\cap K^*)(H^*\cap K)\cong (H\cap K)K^*/(H^*\cap K)K^*.
\end{gather*}
\end{lemma}
\begin{proof}
因为$H^*$是$H$的正规子群, $H\cap K, H\cap K^*$都是$H$的子群, 故由\rrefthe{theorem:抽象代数--定理1.7.3}{theorem:抽象代数--定理1.7.3-2}知$H^*(H\cap K), H^*(H\cap K^*)$都是$H$的子群. 同样$K^*(H\cap K), K^*(H^*\cap K)$都是$K$的子群, 而且$H^*(H\cap K^*)\subseteq H^*(H\cap K)$, $K^*(H^*\cap K)\subseteq K^*(H\cap K)$. 对$\forall a\in H^*\cap K,b\in H\cap K^*,x\in H\cap K,$有
\begin{align*}
xax^{-1}\in H^*\cap K,\quad xbx^{-1}\in H\cap K^*.
\end{align*}
故$H^*\cap K,H\cap K^*\lhd H\cap K.$再由\rrefpro{proposition:正规子群的基本性质}{proposition:正规子群的基本性质-4}知$(H\cap K^*)(H^*\cap K)=L$也是$H\cap K$的正规子群.

作$H^*(H\cap K)$到$(H\cap K)/L$上的映射$\phi$,
\begin{align*}
\phi(hx) = xL, \quad \forall h\in H^*, x\in H\cap K.
\end{align*}
若$h_1x_1 = h_2x_2(h_i\in H^*, x_i\in H\cap K, i=1, 2)$, 则
\begin{align*}
h_2^{-1}h_1 = x_2x_1^{-1}\in H^*\cap (H\cap K)=H^*\cap K\subseteq L\Longrightarrow x_2L=x_1L,
\end{align*}
因而$\phi(h_1x_1)=\phi(h_2x_2)$, 故$\phi$是良定义的.

再证明$\phi$是同态, 由于$H^*\lhd H$, $x_1h_2x_1^{-1}\in H^*$, 故
\begin{align*}
\phi((h_1x_1)(h_2x_2)) = \phi(h_1(x_1h_2x_1^{-1})x_1x_2) = x_1x_2L = (x_1L)\cdot (x_2L) = \phi(h_1x_1)\cdot \phi(h_2x_2),
\end{align*}
故$\phi$是$H^*(H\cap K)$到$H\cap K/L$上的同态,显然$\phi$还是满同态.注意到
\begin{align*}
hx\in \ker\phi\iff \phi(hx)=xL=L\iff x\in L\iff hx\in H^*L = H^*(H^*\cap K)(H\cap K^*) = H^*(H\cap K^*),
\end{align*}
故$\ker \phi = H^*(H\cap K^*)$.因而由\hyperref[theorem:群的同态基本定理-2]{群的同态基本定理\ref{theorem:群的同态基本定理-2}}得
\begin{gather*}
H^*(H\cap K^*)\lhd H^*(H\cap K),
\\
H^*(H\cap K)/H^*(H\cap K^*) \cong (H\cap K)/(H\cap K^*)(H^*\cap K).
\end{gather*}
同理可得
\begin{gather*}
K^*(H^*\cap K)\lhd K^*(H\cap K),
\\
K^*(H\cap K)/K^*(H^*\cap K) \cong (H\cap K)/(H\cap K^*)(H^*\cap K).
\end{gather*}
故
\begin{align*}
H^*(H\cap K)/H^*(H\cap K^*) \cong (H\cap K)/(H\cap K^*)(H^*\cap K)\cong K^*(H\cap K)/K^*(H^*\cap K).
\end{align*}
综上,同理可得
\begin{gather*}
(H\cap K^*)H^*\lhd (H\cap K)H^*,\quad (H^*\cap K)K^*\lhd (H\cap K)K^*,
\\
(H\cap K)H^*/(H\cap K^*)H^*\cong (H\cap K)/(H\cap K^*)(H^*\cap K)\cong (H\cap K)K^*/(H^*\cap K)K^*.
\end{gather*}

\end{proof}

\begin{theorem}[Schreier定理]\label{theorem:抽象代数-定理4.7.1-Schreier定理}
群$G$的两个次正规(正规)序列有同构的加细.
\end{theorem}
\begin{proof}
设群$G$有次正规(正规)序列
\begin{align}
G = G_1 \supseteq G_2 \supseteq\cdots \supseteq G_r = \{1\}, \label{eq:::--289h2389fsdgsdvcxhrsd::--q89h2dfwfzznmnghmghkhlyujf21.1} \\
G = H_1 \supseteq H_2 \supseteq \cdots \supseteq H_s = \{1\}. \label{eq:::--289h2389fsdgsdvcxhrsd::--q89h2dfwfzznmnghmghkhlyujf21.2}
\end{align}
令
\begin{align*}
G_{i,k} &= (G_i\cap H_k)G_{i+1}, \quad 1\leqslant i\leqslant r-1, 1\leqslant k\leqslant s, \\
G_{r,s} &= \{1\}, \\
H_{i,k} &= (H_i\cap G_k)H_{i+1}, \quad 1\leqslant i\leqslant s-1, 1\leqslant k\leqslant r, \\
H_{s,r} &= \{1\}.
\end{align*}
由$G_{k+1}\lhd G_k(1\leqslant k\leqslant r-1),H_{k+1}\lhd H_k(1\leqslant k\leqslant s-1)$及\hyperref[lemma:抽象代数-引理4.7.1-Zassenhaus定理]{Zassenhaus定理}可得
\begin{align*}
G_{i,k+1}\lhd G_{i,k},  \quad 1\leqslant i\leqslant r-1,1\leqslant k\leqslant s-1, 
\\
H_{i,k+1}\lhd H_{i,k},\quad 1\leqslant i\leqslant s-1, 1\leqslant k\leqslant r-1.
\end{align*}
从而
\begin{gather*}
G_{i+1,1}=(G_{i+1}\cap G)G_{i+2}=G_{i+1}=(G_i\cap H_s)G_{i+1}=G_{i,s}\lhd G_{i,s-1},\,\,1\leqslant i\leqslant r-1,
\\
H_{i+1,1}=(H_{i+1}\cap G)H_{i+2}=H_{i+1}=(H_i\cap G_r)H_{i+1}=H_{i,r}\lhd H_{i,r-1},\,\,1\leqslant i\leqslant s-1.
\end{gather*}
又若序列\eqref{eq:::--289h2389fsdgsdvcxhrsd::--q89h2dfwfzznmnghmghkhlyujf21.1}, 序列\eqref{eq:::--289h2389fsdgsdvcxhrsd::--q89h2dfwfzznmnghmghkhlyujf21.2}都是正规序列,即$G_i, H_j$($1\leqslant i\leqslant s, 1\leqslant j\leqslant r$)均为$G$的正规子群,则由\hyperref[proposition:正规子群的基本性质]{命题\ref{proposition:正规子群的基本性质-5}\ref{proposition:正规子群的基本性质-3}}知$G_{i,k}, H_{j,k}$也是$G$的正规子群. 这样,即得序列\eqref{eq:::--289h2389fsdgsdvcxhrsd::--q89h2dfwfzznmnghmghkhlyujf21.1}, 序列\eqref{eq:::--289h2389fsdgsdvcxhrsd::--q89h2dfwfzznmnghmghkhlyujf21.2}的加细的次正规(正规)序列
\begin{align}
G&=G_{1,1} \supseteq G_{1,2} \supseteq \cdots \supseteq G_{1,s-1} \nonumber \\
&\supseteq G_{2,1} \supseteq G_{2,2} \supseteq \cdots \supseteq G_{2,s-1} \nonumber \\
&\supseteq \cdots \nonumber \\
&\supseteq G_{r-1,1} \supseteq G_{r-1,2} \supseteq \cdots \supseteq G_{r-1,s-1} \supseteq G_{r,s} = \{1\}, \label{eq:::--289h2389fsdgsdvcxhrsd::--q89h2dfwfzznmnghmghkhlyujf21.3}
\end{align}
\begin{align}
G&=H_{1,1} \supseteq H_{1,2} \supseteq \cdots \supseteq H_{1,r-1} \nonumber \\\
&\supseteq H_{2,1} \supseteq H_{2,2} \supseteq \cdots \supseteq H_{2,r-1} \nonumber \\
&\supseteq \cdots\nonumber \\
&\supseteq H_{s-1,1} \supseteq H_{s-1,2} \supseteq \cdots \supseteq H_{s-1,r-1} \supseteq H_{s,r} = \{1\}. \label{eq:::--289h2389fsdgsdvcxhrsd::--q89h2dfwfzznmnghmghkhlyujf21.4}
\end{align}
由\hyperref[lemma:抽象代数-引理4.7.1-Zassenhaus定理]{Zassenhaus定理}有
\begin{gather*}
\frac{G_{i,k}}{G_{i,k+1}} = \frac{(G_i\cap H_k)G_{i+1}}{(G_i\cap H_{k+1})G_{i+1}} \cong \frac{(G_i\cap H_k)H_{k+1}}{(G_{i+1}\cap H_k)H_{k+1}} = \frac{H_{k,i}}{H_{k,i+1}},
\\
H_{s-1,r-1}=H_{s-1}\cap G_{r-1}=G_{r-1,s-1}
\end{gather*}
故序列\eqref{eq:::--289h2389fsdgsdvcxhrsd::--q89h2dfwfzznmnghmghkhlyujf21.3}, \eqref{eq:::--289h2389fsdgsdvcxhrsd::--q89h2dfwfzznmnghmghkhlyujf21.4}是$G$的同构的次正规(正规)序列.

\end{proof}














\end{document}
\documentclass[../../main.tex]{subfiles}
\graphicspath{{\subfix{./image/}}} % 指定图片目录,后续可以直接使用图片文件名
% 注意这里的文件路径不能用 ../../image/ ,否则用latexmk编译子文件会报错

% 例如:
% \begin{figure}[H]
% \centering
% \includegraphics[scale=0.4]{图.png}
% \caption{}
% \label{figure:图}
% \end{figure}
% 注意:上述\label{}一定要放在\caption{}之后,否则引用图片序号会只会显示??.

\begin{document}

\section{群的直积}

\begin{definition}[外直积]
设$G_1, G_2$是两个群,构造集合$G_1$与$G_2$的笛卡尔积
\begin{align*}
G = \{(a_1,a_2) \mid a_1 \in G_1, a_2 \in G_2\},
\end{align*}
并在$G$中定义乘法运算
\begin{align*}
(a_1,a_2) \cdot (b_1,b_2) = (a_1b_1, a_2b_2), \quad (a_1,a_2), (b_1,b_2) \in G,
\end{align*}
则$G$关于上述定义的乘法构成群,称为群$G_1$与$G_2$的\textbf{外直积},记作$G = G_1 \times G_2$.
\end{definition}
\begin{remark}
\begin{enumerate}[(1)]
\item 如果$e_1, e_2$分别是群$G_1$和$G_2$的单位元,则$(e_1,e_2)$是$G_1 \times G_2$的单位元;
\item 设$(a_1,a_2) \in G$,则$(a_1,a_2)^{-1} = (a_1^{-1}, a_2^{-1})$;
\item 当$G_1$和$G_2$都是加群时,$G_1$与$G_2$的外直积也可记作$G_1 \oplus G_2$.
\end{enumerate}
\end{remark}

\begin{theorem}
设$G = G_1 \times G_2$是群$G_1$与$G_2$的外直积,则
\begin{enumerate}[(1)]
\item $G$是有限群的充分必要条件是$G_1$与$G_2$都是有限群. 并且,当$G$是有限群时,有
\begin{align*}
|G| = |G_1| \cdot |G_2|;
\end{align*}
\item $G$是交换群的充分必要条件是$G_1$与$G_2$都是交换群;
\item $G_1 \times G_2 \cong G_2 \times G_1$.
\end{enumerate}
\end{theorem}
\begin{proof}
\begin{enumerate}[(1)]
\item 由笛卡尔积的定义易得.

\item 如果$G_1$与$G_2$都是交换群,则对任意的$(a_1,a_2), (b_1,b_2) \in G$,有
\begin{align*}
(a_1,a_2) \cdot (b_1,b_2) = (a_1b_1, a_2b_2) = (b_1a_1, b_2a_2) = (b_1,b_2) \cdot (a_1,a_2),
\end{align*}
所以$G$是交换群.

反之,如果$G$是交换群,那么对任意的$a_1,b_1 \in G_1, a_2,b_2 \in G_2$,有
\begin{align*}
(a_1,a_2) \cdot (b_1,b_2) = (b_1,b_2) \cdot (a_1,a_2),
\end{align*}
即
\begin{align*}
(a_1b_1, a_2b_2) = (b_1a_1, b_2a_2).
\end{align*}
因此$a_1b_1 = b_1a_1, a_2b_2 = b_2a_2$,从而$G_1, G_2$都是交换群.

\item 构造映射
\begin{align*}
\phi: G_1 \times G_2 \longrightarrow G_2 \times G_1,
\end{align*}
\begin{align*}
(a_1,a_2) \longmapsto (a_2,a_1), \quad \forall (a_1,a_2) \in G_1 \times G_2,
\end{align*}
则显然$\phi$是双射,且
\begin{align*}
\phi((a_1,a_2)(b_1,b_2)) = \phi(a_1b_1, a_2b_2) = (a_2b_2, a_1b_1)
\end{align*}
\begin{align*}
= (a_2,a_1)(b_2,b_1) = \phi(a_1,a_2) \cdot \phi(b_1,b_2).
\end{align*}
因此,$\phi$是$G_1 \times G_2$到$G_2 \times G_1$的同构映射,即
\begin{align*}
G_1 \times G_2 \cong G_2 \times G_1.
\end{align*}
\end{enumerate}

\end{proof}

\begin{theorem}
设$G_1, G_2$是两个群,$a$和$b$分别是$G_1$和$G_2$中的有限阶元素,则对于$(a,b) \in G_1 \times G_2$,有
\begin{align*}
\mathrm{ord}(a,b) = [\mathrm{ord}\,a, \mathrm{ord}\,b].
\end{align*}
\end{theorem}
\begin{proof}
设$\mathrm{ord}\,a = m$, $\mathrm{ord}\,b = n$, $s = [m,n]$,则
\begin{align}
(a,b)^s = (a^s, b^s) = (e_1, e_2). \label{eq:::---g90fh43f3g42.4.1}
\end{align}
从而$(a,b)$的阶有限,设其为$t$,则要证明$t = s$. 由\eqref{eq:::---g90fh43f3g42.4.1}式得$t \mid s$.

又因为
\begin{align*}
(e_1,e_2) = (a,b)^t = (a^t, b^t),
\end{align*}
所以$a^t = e_1$, $b^t = e_2$. 于是$m \mid t$,且$n \mid t$,从而$t$是$m$和$n$的公倍数. 而$s$是$m$和$n$的最小公倍数,因此$s \mid t$. 结合以上讨论得$s = t$.

\end{proof}

\begin{theorem}
设$G_1$和$G_2$分别是$m$阶及$n$阶的循环群,则$G_1 \times G_2$是循环群的充要条件是$(m,n) = 1$.
\end{theorem}
\begin{proof}
设$G_1 = \langle a \rangle$, $G_2 = \langle b \rangle$.

假设$G_1 \times G_2$是循环群. 若$(m,n) = t \neq 1$,则由于$\mathrm{ord}\,a = m$, $\mathrm{ord}\,b = n$,而$a^{m/t}$和$b^{n/t}$的阶都是$t$,因此由\refcor{corollary:抽象代数-推论 1.3.4}知$\langle (a^{m/t}, e_2) \rangle$和$\langle (e_1, b^{n/t}) \rangle$是循环群$G_1 \times G_2$中的两个不同的$t$阶子群. 而这与\refcor{corollary:循环群的所有子群的具体形式}相矛盾,所以$(m,n) = 1$.

反之,假设$(m,n) = 1$,则
\begin{align*}
\left| \langle \left( a,b \right) \rangle \right| = \mathrm{ord}(a,b) = [m,n] = mn
= |G_1| \cdot |G_2| = |G_1 \times G_2|,
\end{align*}
又$\langle \left( a,b \right) \rangle \subseteq G_1 \times G_2$,故$\langle \left( a,b \right) \rangle=G_1 \times G_2$,因此$G_1 \times G_2$是循环群.

\end{proof}

\begin{definition}
设$A$, $B$, $G$都是群, 若有$G$的正规子群$N$与$A$同构, 而商群$G/N$与$B$同构, 则称$G$是\textbf{$B$过$A$的扩张}, $N$称为该扩张的\textbf{核},简称\textbf{扩张核}.
\end{definition}
\begin{remark}
显然,若$N$是$G$的正规子群, 则$G$是$G/N$过$N$的扩张, 扩张核为$N$.
\end{remark}

\begin{definition}
设$G$是$B$过$A$的扩张, $N$为扩张核, $\lambda$是$A$到$N$上的同构,$\mu$是$G$到$B$上的同态且$\mu$满足$\ker\mu = N$. $1$为$A$的幺元, $1'$为$B$的幺元, $i$是$\{1\}$到$A$的映射, $i(1)=1$. $0'$是$B$到$\{1'\}$的映射, $0'(b)=1'$($\forall b\in B$). 于是有群及其映射的序列(以$1$, $1'$代替$\{1\}$, $\{1'\}$)
\begin{align*}
1 \stackrel{i}{\longrightarrow} A \stackrel{\lambda}{\longrightarrow} G \stackrel{\mu}{\longrightarrow} B \stackrel{0'}{\longrightarrow} 1',
\end{align*}
每个映射都是群的同态映射, 并且前一映射的像恰是后一映射的核, 即
\begin{align*}
i(1) = \ker\lambda, \quad\lambda(A) = \ker \mu  , \quad \mu(G) = \ker 0'.
\end{align*}
这样的序列称为\textbf{(短)正合序列}.
以后记(短)正合序列时, $i$与$0'$省略不写, 同时也将$1'$记为$1$.
\end{definition}
\begin{remark}
由\refthe{theorem:抽象代数--1.5.}知存在$G$到$G/N$的自然群同态$\mu_1$.由$G/N\cong B$可设$G/N$到$B$的同构$f$,则$\mu = f\mu_1$就是$G$到$B$的同态.
由\refpro{proposition:自然同态的ker等于商掉的正规子群}知$ \ker \mu_1 = N $,从而
\begin{align*}
\mu(N) = f\mu_1(N) = f(N) = 1'.
\end{align*}
故$ N \subseteq \ker \mu $。再设$ x \in \ker \mu $,则
\begin{align*}
\mu(x) = f\mu_1(x) = 1' \implies \mu_1(x) = f^{-1}(1') = N \implies x \in \ker \mu_1 = N.
\end{align*}
故$ \ker \mu \subseteq N $。综上可得$ \lambda(A) = N = \ker \mu $。故上述定义中的同态$\mu$是良定义的.

\end{remark}

\begin{proposition}\label{proposition:正合序列的基本性质}
若群$A$, $B$, $G$之间有(短)正合序列
\begin{align*}
1\longrightarrow A \stackrel{\lambda}{\longrightarrow} G \stackrel{\mu}{\longrightarrow} B\longrightarrow 1,
\end{align*}
即存在$G$的正规子群$N$,还存在$\lambda$是$A$到$N$上的同构,以及$\mu$是$G$到$B$上的同态且$\mu$满足$\ker\mu = N$.

则$\lambda$是$A$到$G$的单同态,$\mu$是$G$到$B$的满同态,并且$G$是$B$过$A$的扩张.
\end{proposition}
\begin{proof}


\end{proof}

\begin{theorem}\label{theorem:抽象代数--定理4.5.1}
设$A$, $B$, $G$, $G'$是群.
\begin{enumerate}[(1)]
\item 若$G$是$B$过$A$的扩张, $G$与$G'$同构, 则$G'$也是$B$过$A$的扩张;
\item 若$G$, $G'$都是$B$过$A$的扩张且有$G$到$G'$的同态$f$, 使\reffig{figure:图4.5.12142f}为交换图, 则$f$是$G$到$G'$上的同构, 这时称$G$与$G'$是$B$过$A$的\textbf{等价扩张}.
\begin{figure}[H]
\centering
% https://q.uiver.app/#q=WzAsMTAsWzAsMCwiMSJdLFsyLDAsIkciXSxbMSwwLCJBIl0sWzAsMSwiMSJdLFsxLDEsIkEiXSxbMiwxLCJHJyJdLFszLDAsIkIiXSxbMywxLCJCIl0sWzQsMCwiMSJdLFs0LDEsIjEiXSxbMCwyXSxbMiwxLCJcXGxhbWJkYSAiXSxbMiw0LCJcXG1hdGhybXtpZH1fQSJdLFszLDRdLFs0LDUsIlxcbGFtYmRhJyIsMl0sWzEsNSwiZiJdLFsxLDYsIlxcbXUiXSxbNSw3LCJcXG11JyIsMl0sWzYsNywiXFxtYXRocm17aWR9X0IiXSxbNiw4XSxbNyw5XV0=
\begin{tikzcd}
1 & A & G & B & 1 \\
1 & A & {G'} & B & 1
\arrow[from=1-1, to=1-2]
\arrow["{\lambda }", from=1-2, to=1-3]
\arrow["{\mathrm{id}_A}", from=1-2, to=2-2]
\arrow["\mu", from=1-3, to=1-4]
\arrow["f", from=1-3, to=2-3]
\arrow[from=1-4, to=1-5]
\arrow["{\mathrm{id}_B}", from=1-4, to=2-4]
\arrow[from=2-1, to=2-2]
\arrow["{\lambda'}"', from=2-2, to=2-3]
\arrow["{\mu'}"', from=2-3, to=2-4]
\arrow[from=2-4, to=2-5]
\end{tikzcd}
\caption{}
\label{figure:图4.5.12142f}
\end{figure}
\end{enumerate}
\end{theorem}
\begin{proof}
\begin{enumerate}[(1)]
\item 设$A$, $B$, $G$对应的(短)正合序列为
\begin{align*}
1\longrightarrow A \stackrel{\lambda}{\longrightarrow} G \stackrel{\mu}{\longrightarrow} B\longrightarrow 1,
\end{align*}
$f$是$G$到$G'$上的同构. 令$\lambda' = f\lambda$, $\mu' = \mu f^{-1}$.
由\refpro{proposition:正合序列的基本性质}知$\lambda$是$A$到$G$的单同态且$\lambda(A)=N$.从而$\lambda'$是单同态且$\lambda'(A) = f(\lambda(A))$与$A$同构. $\mu' = \mu f^{-1}$是$G'$到$B$上的同态,又注意到
\begin{align*}
\mu' \left( \mathrm{ker}\mu ' \right) =1' \iff \mu \left( f^{-1}\left( \mathrm{ker}\mu ' \right) \right) =1' \Longleftrightarrow f^{-1}\left( \mathrm{ker}\mu' \right) =\mathrm{ker}\mu \Longleftrightarrow \mathrm{ker}\mu' =f\left( \mathrm{ker}\mu \right) ,
\end{align*}
故
\begin{align*}
\ker\mu' = \ker(\mu f^{-1}) = f(\ker\mu) = f(\lambda(A)) = \lambda'(A).
\end{align*}
因而$G'$是$B$过$A$的扩张.

\item 先证$\ker f = \{1\}$, 即$f$是单射. 若$x\in\ker f$, 则$\mu(x) = \mu' f(x) = \mu'(1) = 1$知$x\in\ker\mu = \lambda(A)$, 因而$\exists y\in A$, 使得$x = \lambda(y)$. 于是$\lambda'(y) = f(\lambda(y)) = f(x) = 1$. 由(1)的证明知$\lambda'$是单射, 故$y = 1$,于是$x=\lambda(1)=1$,即$\ker f = \{1\}$.

下面证$f(G) = G'$, 即$f$是满映射. 设$x'\in G'$, 由\refpro{proposition:正合序列的基本性质}知$\mu$是$G$到$B$的满同态,即$\mu(G) = B$,从而$\exists x\in G$, 使$\mu(x) = \mu'(x')$, 但$\mu = \mu' f$, 故
\begin{align*}
\mu ' (f(x))=\mu (x)=\mu ' (x' )\Longleftrightarrow 1=\left( \mu ' (x' ) \right) ^{-1}\mu ' (f(x))=\left( \mu ' (x' )^{-1} \right) \mu ' (f(x))=\mu ' ((x' )^{-1}f(x)).
\end{align*}
因而$(x')^{-1}f(x)\in\ker\mu' = \lambda'(A) = f\lambda(A)$. 故$\exists a\in A$, 使$(x')^{-1}f(x) = f(\lambda(a))\in f(G)$, 于是$x'\in f(G)$, 即$f$是满映射.
\end{enumerate}
\end{proof}

\begin{theorem}\label{theorem:抽象代数--定理4.5.2}
设群$G$是群$B$过群$A$的扩张, 对应的(短)正合序列为
\begin{align*}
1\longrightarrow A \stackrel{\lambda}{\longrightarrow} G \stackrel{\mu}{\longrightarrow} B\longrightarrow 1,
\end{align*}
扩张核为$N = \lambda(A) = \ker\mu$.
\begin{enumerate}[(1)]
\item 若有$G$的子群$H$满足$G = HN$, $H\cap N = \{1\}$, 则$\mu|_H$是$H$到$B$上的同构, 此时$(\mu|_H)^{-1} = \nu$是$B$到$G$中的同态且$\mu\nu = \text{id}_B$;
\item 若存在$B$到$G$的同态$\nu$, 使得$\mu\nu = \text{id}_B$, 则$\nu(B) = H$是$G$的子群,$\nu$是$B$到$H = \nu(B)$上的同构且$G = HN$, $H\cap N = \{1\}$.
\end{enumerate}
\end{theorem}
\begin{proof}
\begin{enumerate}[(1)]
\item 由$\ker(\mu|_H) = H\cap\ker\mu = H\cap N = \{1\}$知$\mu|_H$是$H$到$B$的单射, 又$\forall b\in B$, $\exists x\in G$, 使$\mu(x) = b$, 而$G = HN$, 故$\exists y\in H$, $z\in N$, 使$x = yz$. 于是$b = \mu(x) = \mu(y)\mu(z) = \mu(y)$, 故$\mu|_H$是$H$到$B$上的满映射, 于是$\mu|_H$是$H$到$B$上的同构. 从而$v$是$B$到$H$中的同构,故$v$是$B$到$G$中的同态.又$\mu\nu(b) = \mu(y) = b$, 故$\mu\nu = \text{id}_B$.
\item 由\rrefpro{proposition:群同态的核是定义域的子群,像是陪域的子群-抽象代数}{proposition:群同态的核是定义域的子群,像是陪域的子群-抽象代数-1}知$\nu(B)=H$是$G$的子群.由$\mu\nu = \text{id}_B$知$x=\mu\nu(x)=\mu(1)=1,\forall x\in \ker \nu$,即$\ker \nu\subseteq \{1\}$,因此$\ker\nu = \{1\}$,故$\nu$是$B$到$H = \nu(B)$上的同构, 若$x\in N\cap H$, 则由$x\in N=\ker \mu$知$\mu(x) = 1$,由$x\in H=\nu(B)$知存在$b\in B$,使$x=\nu(b)$.从而
$$
1=\mu \left( x \right) =\mu \nu \left( b \right) =\mathrm{id}_B\left( b \right) =b.
$$
故$x=\nu \left( b \right) =\nu \left( 1 \right) =1$,即$H\cap N = \{1\}$. 

对$\forall b \in B$,由$\mu\nu = \mathrm{id}_B$知$\mu(\nu(b)) = \mathrm{id}_B(b) = b$且$\nu(b) \in H$,故$\mu|_H$是$H$到$B$上的满同态.
设$x\in G$,则$\mu (x)\in B$. 于是$\exists y\in H$, 使$\mu(y) = \mu(x)$, 因而$\mu(y^{-1}x) = 1$, 即$z = y^{-1}x\in \ker\mu =N$有$x = yz\in HN$, 故$G = HN$.
\end{enumerate}
\end{proof}

\begin{definition}\label{definition:内直积和中心扩张定义}
设$G$是群$B$过群$A$的扩张, $N$是扩张的核. 如果存在$G$的子群$H$, 使$H\cap N = \{1\}$, $G = HN$, 那么称此扩张为\textbf{非本质扩张}, 并称$G$是$N$与$H$的\textbf{半直积}, 记为$G = H\ltimes N$.

如果$H$还是$G$的正规子群, 则称此扩张为\textbf{平凡扩张}. $G$是$N$与$H$的\textbf{(内)直积}, 记为$G = H\otimes N$.

如果$N\subseteq C(G)$, 那么称此扩张为\textbf{中心扩张}.
\end{definition}

\begin{theorem}\label{theorem:两个正规子群构成的内直积}
设$G$是一个群,则$G=A\otimes B$的充要条件是$A,B\lhd G$且$G=AB,A\cap B=\{1\}$.
\end{theorem}
\begin{proof}
{\heiti 充分性:}由$B \lhd G$知$G$是$G/B$过$B$的扩张, 扩张核为$B$. 又$G=AB$, $A \cap B=\{1\}$, $A \lhd G$, 故由\refdef{definition:内直积和中心扩张定义}知$G=A \otimes B$.

{\heiti 必要性:}根据\hyperref[definition:内直积和中心扩张定义]{内直积的定义}是显然的.

\end{proof}

\begin{example}
\begin{enumerate}
\item 对整数加群$\mathbb{Z}$, 它的正规子群$2\mathbb{Z}$与$\mathbb{Z}$同构, 而$\mathbb{Z}/2\mathbb{Z} = \mathbb{Z}_2$是$2$阶循环群, 因而$\mathbb{Z}$是$\mathbb{Z}_2$过$2\mathbb{Z}$的扩张. 由于$\mathbb{Z}$的任何子群都不同构于$\mathbb{Z}_2$, 因而这个扩张不是非本质扩张.
\item 设$n\geqslant 3$. $A_n$是$S_n$的正规子群, $\langle(12)\rangle$是$S_n$的$2$阶子群, $\langle(12)\rangle\cap A_n = \{\text{id}\}$, $S_n = \langle(12)\rangle A_n$, 但$\langle(12)\rangle$不是$S_n$的正规子群, 故$S_n = \langle(12)\rangle\ltimes A_n$.
\item $3$阶循环群过$5$阶循环群的扩张$G$是$15$阶群. 由4.3节的习题8知这种扩张必然是平凡扩张, 即$G = \langle a\rangle\otimes\langle b\rangle$, 其中, $a$, $b$分别为$G$的$3$阶元素与$5$阶元素.
\end{enumerate}
\end{example}
\begin{proof}


\end{proof}

\begin{theorem}\label{theorem:抽象代数--定理4.5.3}
设$A$, $B$是$G$的子群.
\begin{enumerate}[(1)]
\item\label{theorem:抽象代数--定理4.5.3-1} $G = AB$, $A\cap B = \{1\}$当且仅当$\forall g\in G$, $\exists a\in A$, $b\in B$, 使得$g = ab$且这种表示唯一.

\item\label{theorem:抽象代数--定理4.5.3-2} 若$G = AB$, $A\cap B = \{1\}$, 则$A$, $B$都是$G$的正规子群的充分必要条件是$ab = ba$($\forall a\in A$, $b\in B$), 此时$G = A\otimes B$.
\end{enumerate}
\end{theorem}
\begin{proof}
\begin{enumerate}[(1)]
\item 由$G = AB$, $A\cap B = \{1\}$知$\forall g\in G$, $\exists a\in A$, $b\in B$, 使$g = ab$. 若另有$g = a'b'$, $a'\in A$, $b'\in B$, 则$a^{-1}a' = bb'^{-1}\in A\cap B=\{1\}$, 于是$a = a'$, $b = b'$.

反之, 若$\forall g\in G$, $\exists a\in A$, $b\in B$, 使$g = ab$, 则$G = AB$. 又若$c\in A\cap B$, 由$c = 1\cdot c = c\cdot 1$的表示唯一可知$c = 1$, 故$A\cap B = \{1\}$.

\item 设$A\lhd G$, $B\lhd G$, 于是$a\left( ba^{-1}b^{-1} \right) =\left( aba^{-1} \right) b^{-1}\in A\cap B=\left\{ 1 \right\} $, 故$ab = ba$($\forall a\in A$, $b\in B$).

反之, 由于$G = AB$, $\forall g\in G$, $\exists a\in A$, $b\in B$, 使$g = ab$. 又若$a_0\in A$, 则由$ab = ba$($\forall a\in A$, $b\in B$)有
\begin{align*}
ga_0g^{-1} = (ab)a_0(ab)^{-1} = aa_0a^{-1}\in A,
\end{align*}
故$A\lhd G$, 同样$B\lhd G$, 由\refpro{theorem:两个正规子群构成的内直积}知$G=A \otimes B$.
\end{enumerate}
\end{proof}

\begin{definition}
设$N_1$, $N_2$, $\cdots$, $N_k$都是群$G$的正规子群, 并且$G = N_1N_2\cdots N_k$, $G$中任一元素可分解为$N_i$($1\leqslant i\leqslant k$)中元素的积且这种分解是唯一的, 则称$G$是$N_1$, $N_2$, $\cdots$, $N_k$的\textbf{(内)直积}, 记为
\begin{align*}
G = N_1\otimes N_2\otimes\cdots\otimes N_k.
\end{align*}
\end{definition}

\begin{theorem}\label{theorem:抽象代数--定理4.5.4}
设$A$, $B$是两个群, 则一定存在$B$过$A$的平凡扩张$G=A\times B$, 并且$G$在同构意义下唯一.
\end{theorem}
\begin{proof}
在$G = A\times B = \{(a, b)\mid a\in A, b\in B\}$中定义乘法
\begin{align*}
(a_1, b_1)(a_2, b_2) = (a_1a_2, b_1b_2),\quad \forall (a_i, b_i)\in G, i = 1, 2.
\end{align*}
容易验证$G$是群, 幺元为$(1, 1')$, 其中, $1$, $1'$分别为$A$, $B$的幺元. $\forall (a, b)\in G$, $(a, b)^{-1} = (a^{-1}, b^{-1})$, 而且
$$
A' = \{(a, 1')\mid a\in A\},\quad B' = \{(1, b)\mid b\in B\}
$$
都是$G$的正规子群. 又$G = A'B'$, $A'\cap B' = \{(1, 1')\}$, 于是由\refpro{theorem:两个正规子群构成的内直积}知$G = A'\otimes B'$. 

又映射$\lambda$: $\lambda(a) = (a, 1')$($\forall a\in A$)是$A$到$G$的单同态, $\lambda(A) = A'$,故$\lambda$是$A$到$A'$的同构. 而映射$\mu$: $\mu((a, b)) = b$则是$G$到$B$上的同态, 并且$\ker\mu = A'=\lambda(A)$, $\mu|_{B'}$是$B'$到$B$上的同构.即有(短)正合序列
\begin{align*}
1\longrightarrow A \stackrel{\lambda}{\longrightarrow} G \stackrel{\mu}{\longrightarrow} B\longrightarrow 1,
\end{align*}
故由\refpro{proposition:正合序列的基本性质}知$G$是$B$过$A$的扩张,扩张核为$A'$.
由$G = A'\otimes B'$及$A\cong A' ,B\cong B'$知
\begin{align*}
G=A' B' \cong AB,\quad A\cap B\cong A' \cap B' =\left\{ 1 \right\}.
\end{align*}
又$A' \lhd G$,故由\refpro{proposition:正合序列的基本性质}知$G$是$B$过$A$的平凡扩张.

设$G_1$也是$B$过$A$的平凡扩张, 于是$G_1 = A_1\otimes B_1$. 设$\lambda_1$为$A$到$A_1$的同构, $\gamma_1$是$B$到$B_1$的同构, 令
\begin{align*}
f((a, b)) = \lambda_1(a)\gamma_1(b),\quad \forall a\in A, b\in B.
\end{align*}
由$G_1=A_1\otimes B_1$知$G_1=A_1B_1$,$A_1\cap B_1=\{1\}$且$A_1,B_1\lhd G_1$。从而由\rrefthe{theorem:抽象代数--定理4.5.3}{theorem:抽象代数--定理4.5.3-2}知
\begin{align*}
a_1b_1=b_1a_1,\quad \forall a_1\in A_1,b_1\in B_1.
\end{align*}
于是对$\forall (a,b),(a',b')\in G$,有
\begin{align*}
f((a,b)(a',b'))&=f((aa',bb'))=\lambda_1(aa')\gamma_1(bb') \\
&=\lambda_1(a)\lambda_1(a')\gamma_1(b)\gamma_1(b')=\lambda_1(a)\gamma_1(b)\lambda_1(a')\gamma_1(b') \\
&=f((a,b))f((a',b')).
\end{align*}
因此$f$是$G$到$G_1$的同态。

设$a_1b_1\in A_1B_1=G_1$,则由$\lambda_1$是$A$到$A_1$的同构,$\gamma_1$是$B$到$B_1$的同构可知,存在$a\in A$,$b\in B$,使
\begin{align*}
\lambda_1(a)=a_1,\gamma_1(b)=b_1\Longrightarrow f((a,b))=\lambda_1(a)\gamma_1(b)=a_1b_1.
\end{align*}
故$f$是满同态。

设$f((a,b))=f((a',b'))\in G_1$,则
\begin{align*}
\lambda_1(a)\gamma_1(b)=f((a,b))=f((a',b'))=\lambda_1(a')\gamma_1(b').
\end{align*}
由\rrefthe{theorem:抽象代数--定理4.5.3}{theorem:抽象代数--定理4.5.3-1}知$\lambda_1(a)=\lambda_1(a')$,$\gamma_1(b)=\gamma_1(b')$。又$\lambda_1$是$A$到$A_1$的同构,$\gamma_1$是$B$到$B_1$的同构,故
\begin{align*}
a=a',b=b'\Longrightarrow (a,b)=(a',b').
\end{align*}
因此$f$是单同态。
综上可知$f$是$G$到$G_1$的同构。

\end{proof}





\end{document}
\documentclass[../../main.tex]{subfiles}% 注意这里的文件路径不能用 ./main.tex ,否则用latexmk编译子文件会报错
\graphicspath{{\subfix{./image/}}} % 指定图片目录,后续可以直接使用图片文件名
% 注意这里的文件路径不能用 ../../image/ ,否则用latexmk编译子文件会报错

% 例如:
% \begin{figure}[H]
% \centering
% \includegraphics[scale=0.3]{图.png}
% \caption{}
% \label{figure:图}
% \end{figure}
% 注意:上述\label{}一定要放在\caption{}之后,否则引用图片序号会只会显示??.

\begin{document}

\section{群的直积}

\begin{definition}[外直积]
设$G_1, G_2,\cdots,G_n$是$n$个群,构造集合$G_1,G_2,\cdots,G_n$的笛卡尔积
\begin{align*}
G=\{(a_1,a_2,\cdots ,a_n)\mid a_i\in G_i,\,i=1,2,\cdots ,n\},
\end{align*}
并在$G$中定义乘法运算
\begin{align*}
(a_1,a_2,\cdots ,a_n)\cdot (b_1,b_2,\cdots ,b_n)=(a_1b_1,a_2b_2,\cdots ,a_nb_n),\,\,\forall (a_1,a_2,\cdots ,a_n),(b_1,b_2,\cdots ,b_n)\in G,
\end{align*}
则$G$关于上述定义的乘法构成群,称为群$G_1,G_2,\cdots,G_n$的\textbf{外直积},记作$G = G_1 \times G_2\times \cdots \times G_n$.
\end{definition}
\begin{remark}
\begin{enumerate}[(1)]
\item 如果$e_1, e_2,\cdots,e_n$分别是群$G_1,G_2,\cdots,G_n$的单位元,则$(e_1,e_2,\cdots,e_n)$是$G_1 \times G_2\times \cdots\times G_n$的单位元;

\item 设$(a_1,a_2,\cdots,a_n) \in G$,则$(a_1,a_2,\cdots,a_n)^{-1} = (a_1^{-1}, a_2^{-1},\cdots,a_n^{-1})$;

\item 当$G_1,G_2,\cdots,G_n$都是加群时,$G_1$与$G_2$的外直积也可记作$G_1 \oplus G_2 \oplus\cdots \oplus G_n$.
\end{enumerate}
\end{remark}

\begin{theorem}\label{theorem:群的外直积的基本性质}
设$G =G_1 \times G_2\times \cdots \times G_n$是$n$个群$G_1,G_2,\cdots,G_n$的外直积,则
\begin{enumerate}[(1)]
\item\label{theorem:群的外直积的基本性质-1} $G$是有限群的充分必要条件是$G_1,G_2,\cdots,G_n$都是有限群. 并且,当$G$是有限群时,有
\begin{align*}
|G| = |G_1| \cdot |G_2|\cdots|G_n|;
\end{align*}

\item\label{theorem:群的外直积的基本性质-2} $G$是交换群的充分必要条件是$G_1,G_2,\cdots,G_n$都是交换群;

\item\label{theorem:群的外直积的基本性质-3} $G_1\times G_2\times \cdots \times G_n\cong G_{\sigma \left( 1 \right)}\times G_{\sigma \left( 2 \right)}\times \cdots \times G_{\sigma \left( n \right)},\,\,\forall \sigma \in S_n.$

\item\label{theorem:群的外直积的基本性质-4} 若$a_1,a_2,\cdots,a_n$分别是$G_1,G_2,\cdots,G_n$中的有限阶元素,则对$(a_1,a_2,\cdots,a_n) \in G_1 \times G_2\times \cdots \times G_n$,有
\begin{align*}
\mathrm{ord}(a_1,a_2,\cdots ,a_n)=[\mathrm{ord}\,a_1,\mathrm{ord}\,a_2,\cdots ,\mathrm{ord}a_n].
\end{align*}

\item\label{theorem:群的外直积的基本性质-5} $C(G) = C(G_1) \times C(G_2) \times \cdots \times C(G_n)$.

\item 若$G_1,G_2,\cdots,G_n$分别是$m_1,m_2,\cdots,m_n$阶的循环群,则$G$是循环群的充要条件是$(m_1,m_2,\cdots,m_n) = 1$.
\end{enumerate}
\end{theorem}
\begin{proof}
\begin{enumerate}[(1)]
\item 由笛卡尔积的定义易得.

\item 如果$G_1$与$G_2$都是交换群,则对任意的$(a_1,a_2,\cdots ,a_n),(b_1,b_2,\cdots ,b_n)\in G$,有
\begin{align*}
&\,\,(a_1,a_2,\cdots ,a_n)\cdot (b_1,b_2,\cdots ,b_n)
\\
&=(a_1b_1,a_2b_2,\cdots ,a_nb_n)
\\
&=(b_1a_1,b_2a_2,\cdots ,b_na_n)
\\
&=(b_1,b_2,\cdots ,b_n)\cdot (a_1,a_2,\cdots ,a_n).
\end{align*}
所以$G$是交换群.

反之,如果$G$是交换群,那么对任意的$a_1,b_1 \in G_1, a_2,b_2 \in G_2$,有
\begin{align*}
(a_1,a_2,\cdots ,a_n)\cdot (b_1,b_2,\cdots ,b_n)=(b_1,b_2,\cdots ,b_n)\cdot (a_1,a_2,\cdots ,a_n),
\end{align*}
即
\begin{align*}
(a_1b_1,a_2b_2,\cdots ,a_nb_n)=(b_1a_1,b_2a_2,\cdots ,b_na_n).
\end{align*}
因此$a_ib_i=b_ia_i,\,i=1,2,\cdots,n.$从而$G_i(i=1,2,\cdots,n)$都是交换群.

\item 对$\forall \sigma \in S_n$,构造映射
\begin{align*}
\phi: G_1\times G_2\times \cdots \times G_n\longrightarrow G_{\sigma \left( 1 \right)}\times G_{\sigma \left( 2 \right)}\times \cdots \times G_{\sigma \left( n \right)},
\end{align*}
\begin{align*}
(a_1,a_2,\cdots ,a_n)\longmapsto (a_{\sigma \left( 1 \right)},a_{\sigma \left( 2 \right)},\cdots ,a_{\sigma \left( n \right)}),\quad \forall (a_1,a_2,\cdots ,a_n)\in G_1\times G_2\times \cdots \times G_n,
\end{align*}
因为$\sigma$是双射,所以$\phi$也是双射,且
\begin{align*}
&\,\,\phi ((a_1,a_2,\cdots ,a_n)(b_1,b_2,\cdots ,b_n))
\\
&=\phi (a_1b_1,a_2b_2,\cdots ,a_nb_n)
\\
&=(a_{\sigma \left( 1 \right)}b_{\sigma \left( 1 \right)},a_{\sigma \left( 1 \right)}b_{\sigma \left( 2 \right)},\cdots ,a_{\sigma \left( 1 \right)}b_{\sigma \left( n \right)})
\end{align*}
\begin{align*}
= (a_2,a_1)(b_2,b_1) = \phi(a_1,a_2) \cdot \phi(b_1,b_2).
\end{align*}
因此$\phi$是$G_1\times G_2\times \cdots \times G_n$到$G_{\sigma \left( 1 \right)}\times G_{\sigma \left( 2 \right)}\times \cdots \times G_{\sigma \left( n \right)}$的同构映射,即
\begin{align*}
G_1\times G_2\times \cdots \times G_n\cong G_{\sigma \left( 1 \right)}\times G_{\sigma \left( 2 \right)}\times \cdots \times G_{\sigma \left( n \right)}.
\end{align*}

\item 当$n=1$时,结论显然成立.假设结论对$n-1$成立,现在考虑$n$的情况.

设$a_i\in G_i,\,i=1,2,\cdots,n$,记$b=(a_2,\cdots,a_n)\in G_2\times \cdots\times G_n$,$\mathrm{ord}\,a_1 = m$,则由归纳假设知
\begin{align*}
\mathrm{ord}\,b=\mathrm{ord}(a_2,\cdots ,a_n)=\left[ \mathrm{ord}\,a_2,\cdots ,\mathrm{ord}a_n \right] .
\end{align*}
再记$\mathrm{ord}\,b = n$, $s = [m,n]$,$e_1$为$G_1$的幺元,$e_2$为$G_2\times \cdots\times G_n$的幺元.则
\begin{align*}
(a_1,b)^s = (a_1^s, b^s) = (e_1, e_2). 
\end{align*}
从而$(a_1,b)$的阶有限,设其为$t$,则由上式得$t \mid s$.

又因为
\begin{align*}
(e_1,e_2) = (a_1,b)^t = (a_1^t, b^t),
\end{align*}
所以$a_1^t = e_1$, $b^t = e_2$. 于是$m \mid t$,且$n \mid t$,从而$t$是$m$和$n$的公倍数. 而$s$是$m$和$n$的最小公倍数,因此$s \mid t$. 结合以上讨论得$s = t$,即
\begin{align*}
\mathrm{ord}\left( a_1,a_2,\cdots ,a_n \right) &=\mathrm{ord}\left( a_1,b \right) =\left[ \mathrm{ord}\,a_1,\mathrm{ord}\,b \right] 
\\
&=\left[ \mathrm{ord}\,a_1,\left[ \mathrm{ord}\,a_2,\cdots ,\mathrm{ord}\,a_n \right] \right] 
\\
&=\left[ \mathrm{ord}\,a_1,\mathrm{ord}\,a_2,\cdots ,\mathrm{ord}\,a_n \right] .
\end{align*}
故由数学归纳法知结论成立.

\item 设$a = (a_1,a_2,\cdots,a_n) \in C(G)$,则对任意的$x = (x_1,x_2,\cdots,x_n) \in G$,
由$ax = xa$得$a_i \in C(G_i),i = 1,2,\cdots,n$,因此
\begin{align*}
a = (a_1,a_2,\cdots,a_n) \in C(G_1) \times C(G_2) \times \cdots \times C(G_n).
\end{align*}

另一方面,设$a = (a_1,a_2,\cdots,a_n) \in C(G_1) \times C(G_2) \times \cdots \times C(G_n)$,则对任意的
$x = (x_1,x_2,\cdots,x_n) \in G$,有
\begin{align*}
ax = (a_1x_1,a_2x_2,\cdots,a_nx_n) 
= (x_1a_1,x_2a_2,\cdots,x_na_n) = xa.
\end{align*}
所以$a = (a_1,a_2,\cdots,a_n) \in C(G)$。因此
\begin{align*}
C(G) = C(G_1) \times C(G_2) \times \cdots \times C(G_n).
\end{align*}

\item 设$G_1 = \langle a_1 \rangle,G_2 = \langle a_2 \rangle,\cdots ,G_n=\langle a_n \rangle$,$e_1,e_2,\cdots,e_n$分别为$G_1,G_2,\cdots,G_n$的幺元.

{\heiti 必要性:}设$G_1\times G_2\times\cdots\times G_n$是循环群.若$(m_1,m_2,\cdots,m_n)=t\ne1$,则由于$\text{ord}\,a_i=m_i,\,i=1,2,\cdots,n$.而$a_{i}^{\frac{m_i}{t}}$的阶都是$t$,
因此由\refthe{theorem:抽象代数-推论 1.3.4}知
\begin{align*}
\langle (e_1,\cdots ,\underset{\text{第}i\text{个位置}}{\underbrace{a_{i}^{\frac{m_i}{t}}}},\cdots ,e_n)\rangle ,\,\,i=1,2,\cdots ,n
\end{align*}
都是循环群$G_1\times G_2\times\cdots\times G_n$中的$n$个不同的$t$阶子群.而这与\hyperref[theorem:有限和无限循环群的常用结论-2-2]{定理\ref{theorem:有限和无限循环群的常用结论}\ref{theorem:有限和无限循环群的常用结论-2}\ref{theorem:有限和无限循环群的常用结论-2-2}}矛盾!故$(m_1,m_2,\cdots,m_n)=1$.

{\heiti 充分性:}设$(m_1,m_2,\cdots,m_n)=1$,则由\rrefthe{theorem:群的外直积的基本性质}{theorem:群的外直积的基本性质-4}
和\rrefthe{theorem:群的外直积的基本性质}{theorem:群的外直积的基本性质-1}可得
\begin{align*}
|\langle (a_1,a_2,\cdots ,a_n)\rangle |&=\mathrm{ord}(a_1,a_2,\cdots ,a_n)=[m_1,m_2,\cdots ,m_n]
\\
&=m_1m_2\cdots m_n=|G_1|\cdot |G_2|\cdots |G_n|
\\
&=|G_1\times G_2\times \cdots \times G_n|.
\end{align*}
又$\langle(a_1,a_2,\cdots,a_n)\rangle\subseteq G_1\times G_2\times\cdots\times G_n$,故$\langle(a_1,a_2,\cdots,a_n)\rangle=G_1\times G_2\times\cdots\times G_n$.因此$G_1\times G_2\times\cdots\times G_n$是循环群.
\end{enumerate}

\end{proof}

\begin{definition}
设$A$, $B$, $G$都是群, 若有$G$的正规子群$N$与$A$同构, 而商群$G/N$与$B$同构, 则称$G$是\textbf{$B$过$A$的扩张}, $N$称为该扩张的\textbf{核},简称\textbf{扩张核}.
\end{definition}
\begin{remark}
显然,若$N$是$G$的正规子群, 则$G$是$G/N$过$N$的扩张, 扩张核为$N$.
\end{remark}

\begin{definition}
设$G$是$B$过$A$的扩张, $N$为扩张核, $\lambda$是$A$到$N$上的同构,$\mu$是$G$到$B$上的同态且$\mu$满足$\ker\mu = N$. $1$为$A$的幺元, $1'$为$B$的幺元, $i$是$\{1\}$到$A$的映射, $i(1)=1$. $0'$是$B$到$\{1'\}$的映射, $0'(b)=1'$($\forall b\in B$). 于是有群及其映射的序列(以$1$, $1'$代替$\{1\}$, $\{1'\}$)
\begin{align*}
1 \stackrel{i}{\longrightarrow} A \stackrel{\lambda}{\longrightarrow} G \stackrel{\mu}{\longrightarrow} B \stackrel{0'}{\longrightarrow} 1',
\end{align*}
每个映射都是群的同态映射, 并且前一映射的像恰是后一映射的核, 即
\begin{align*}
i(1) = \ker\lambda, \quad\lambda(A) = \ker \mu  , \quad \mu(G) = \ker 0'.
\end{align*}
这样的序列称为\textbf{(短)正合序列}.
以后记(短)正合序列时, $i$与$0'$省略不写, 同时也将$1'$记为$1$.
\end{definition}
\begin{remark}
由\refthe{theorem:抽象代数--1.5.}知存在$G$到$G/N$的自然群同态$\mu_1$.由$G/N\cong B$可设$G/N$到$B$的同构$f$,则$\mu = f\mu_1$就是$G$到$B$的同态.
由\refpro{proposition:自然同态的ker等于商掉的正规子群}知$ \ker \mu_1 = N $,从而
\begin{align*}
\mu(N) = f\mu_1(N) = f(N) = 1'.
\end{align*}
故$ N \subseteq \ker \mu $。再设$ x \in \ker \mu $,则
\begin{align*}
\mu(x) = f\mu_1(x) = 1' \implies \mu_1(x) = f^{-1}(1') = N \implies x \in \ker \mu_1 = N.
\end{align*}
故$ \ker \mu \subseteq N $。综上可得$ \lambda(A) = N = \ker \mu $。故上述定义中的同态$\mu$是良定义的.

\end{remark}

\begin{proposition}\label{proposition:正合序列的基本性质}
若群$A$, $B$, $G$之间有(短)正合序列
\begin{align*}
1\longrightarrow A \stackrel{\lambda}{\longrightarrow} G \stackrel{\mu}{\longrightarrow} B\longrightarrow 1,
\end{align*}
即存在$G$的正规子群$N$,还存在$\lambda$是$A$到$N$上的同构,以及$\mu$是$G$到$B$上的同态且$\mu$满足$\ker\mu = N$.

则$\lambda$是$A$到$G$的单同态,$\mu$是$G$到$B$的满同态,并且$G$是$B$过$A$的扩张.
\end{proposition}
\begin{proof}


\end{proof}

\begin{theorem}\label{theorem:抽象代数--定理4.5.1}
设$A$, $B$, $G$, $G'$是群.
\begin{enumerate}[(1)]
\item 若$G$是$B$过$A$的扩张, $G$与$G'$同构, 则$G'$也是$B$过$A$的扩张;
\item 若$G$, $G'$都是$B$过$A$的扩张且有$G$到$G'$的同态$f$, 使\reffig{figure:图4.5.12142f}为交换图, 则$f$是$G$到$G'$上的同构, 这时称$G$与$G'$是$B$过$A$的\textbf{等价扩张}.
\begin{figure}[H]
\centering
% https://q.uiver.app/#q=WzAsMTAsWzAsMCwiMSJdLFsyLDAsIkciXSxbMSwwLCJBIl0sWzAsMSwiMSJdLFsxLDEsIkEiXSxbMiwxLCJHJyJdLFszLDAsIkIiXSxbMywxLCJCIl0sWzQsMCwiMSJdLFs0LDEsIjEiXSxbMCwyXSxbMiwxLCJcXGxhbWJkYSAiXSxbMiw0LCJcXG1hdGhybXtpZH1fQSJdLFszLDRdLFs0LDUsIlxcbGFtYmRhJyIsMl0sWzEsNSwiZiJdLFsxLDYsIlxcbXUiXSxbNSw3LCJcXG11JyIsMl0sWzYsNywiXFxtYXRocm17aWR9X0IiXSxbNiw4XSxbNyw5XV0=
\begin{tikzcd}
1 & A & G & B & 1 \\
1 & A & {G'} & B & 1
\arrow[from=1-1, to=1-2]
\arrow["{\lambda }", from=1-2, to=1-3]
\arrow["{\mathrm{id}_A}", from=1-2, to=2-2]
\arrow["\mu", from=1-3, to=1-4]
\arrow["f", from=1-3, to=2-3]
\arrow[from=1-4, to=1-5]
\arrow["{\mathrm{id}_B}", from=1-4, to=2-4]
\arrow[from=2-1, to=2-2]
\arrow["{\lambda'}"', from=2-2, to=2-3]
\arrow["{\mu'}"', from=2-3, to=2-4]
\arrow[from=2-4, to=2-5]
\end{tikzcd}
\caption{}
\label{figure:图4.5.12142f}
\end{figure}
\end{enumerate}
\end{theorem}
\begin{proof}
\begin{enumerate}[(1)]
\item 设$A$, $B$, $G$对应的(短)正合序列为
\begin{align*}
1\longrightarrow A \stackrel{\lambda}{\longrightarrow} G \stackrel{\mu}{\longrightarrow} B\longrightarrow 1,
\end{align*}
$f$是$G$到$G'$上的同构. 令$\lambda' = f\lambda$, $\mu' = \mu f^{-1}$.
由\refpro{proposition:正合序列的基本性质}知$\lambda$是$A$到$G$的单同态且$\lambda(A)=N$.从而$\lambda'$是单同态且$\lambda'(A) = f(\lambda(A))$与$A$同构. $\mu' = \mu f^{-1}$是$G'$到$B$上的同态,又注意到
\begin{align*}
\mu' \left( \mathrm{ker}\mu ' \right) =1' \iff \mu \left( f^{-1}\left( \mathrm{ker}\mu ' \right) \right) =1' \Longleftrightarrow f^{-1}\left( \mathrm{ker}\mu' \right) =\mathrm{ker}\mu \Longleftrightarrow \mathrm{ker}\mu' =f\left( \mathrm{ker}\mu \right) ,
\end{align*}
故
\begin{align*}
\ker\mu' = \ker(\mu f^{-1}) = f(\ker\mu) = f(\lambda(A)) = \lambda'(A).
\end{align*}
因而$G'$是$B$过$A$的扩张.

\item 先证$\ker f = \{1\}$, 即$f$是单射. 若$x\in\ker f$, 则$\mu(x) = \mu' f(x) = \mu'(1) = 1$知$x\in\ker\mu = \lambda(A)$, 因而$\exists y\in A$, 使得$x = \lambda(y)$. 于是$\lambda'(y) = f(\lambda(y)) = f(x) = 1$. 由(1)的证明知$\lambda'$是单射, 故$y = 1$,于是$x=\lambda(1)=1$,即$\ker f = \{1\}$.

下面证$f(G) = G'$, 即$f$是满映射. 设$x'\in G'$, 由\refpro{proposition:正合序列的基本性质}知$\mu$是$G$到$B$的满同态,即$\mu(G) = B$,从而$\exists x\in G$, 使$\mu(x) = \mu'(x')$, 但$\mu = \mu' f$, 故
\begin{align*}
\mu ' (f(x))=\mu (x)=\mu ' (x' )\Longleftrightarrow 1=\left( \mu ' (x' ) \right) ^{-1}\mu ' (f(x))=\left( \mu ' (x' )^{-1} \right) \mu ' (f(x))=\mu ' ((x' )^{-1}f(x)).
\end{align*}
因而$(x')^{-1}f(x)\in\ker\mu' = \lambda'(A) = f\lambda(A)$. 故$\exists a\in A$, 使$(x')^{-1}f(x) = f(\lambda(a))\in f(G)$, 于是$x'\in f(G)$, 即$f$是满映射.
\end{enumerate}
\end{proof}

\begin{theorem}\label{theorem:抽象代数--定理4.5.2}
设群$G$是群$B$过群$A$的扩张, 对应的(短)正合序列为
\begin{align*}
1\longrightarrow A \stackrel{\lambda}{\longrightarrow} G \stackrel{\mu}{\longrightarrow} B\longrightarrow 1,
\end{align*}
扩张核为$N = \lambda(A) = \ker\mu$.
\begin{enumerate}[(1)]
\item 若有$G$的子群$H$满足$G = HN$, $H\cap N = \{1\}$, 则$\mu|_H$是$H$到$B$上的同构, 此时$(\mu|_H)^{-1} = \nu$是$B$到$G$中的同态且$\mu\nu = \text{id}_B$;
\item 若存在$B$到$G$的同态$\nu$, 使得$\mu\nu = \text{id}_B$, 则$\nu(B) = H$是$G$的子群,$\nu$是$B$到$H = \nu(B)$上的同构且$G = HN$, $H\cap N = \{1\}$.
\end{enumerate}
\end{theorem}
\begin{proof}
\begin{enumerate}[(1)]
\item 由$\ker(\mu|_H) = H\cap\ker\mu = H\cap N = \{1\}$知$\mu|_H$是$H$到$B$的单射, 又$\forall b\in B$, $\exists x\in G$, 使$\mu(x) = b$, 而$G = HN$, 故$\exists y\in H$, $z\in N$, 使$x = yz$. 于是$b = \mu(x) = \mu(y)\mu(z) = \mu(y)$, 故$\mu|_H$是$H$到$B$上的满映射, 于是$\mu|_H$是$H$到$B$上的同构. 从而$v$是$B$到$H$中的同构,故$v$是$B$到$G$中的同态.又$\mu\nu(b) = \mu(y) = b$, 故$\mu\nu = \text{id}_B$.
\item 由\rrefpro{proposition:群同态的核是定义域的子群,像是陪域的子群-抽象代数}{proposition:群同态的核是定义域的子群,像是陪域的子群-抽象代数-1}知$\nu(B)=H$是$G$的子群.由$\mu\nu = \text{id}_B$知$x=\mu\nu(x)=\mu(1)=1,\forall x\in \ker \nu$,即$\ker \nu\subseteq \{1\}$,因此$\ker\nu = \{1\}$,故$\nu$是$B$到$H = \nu(B)$上的同构, 若$x\in N\cap H$, 则由$x\in N=\ker \mu$知$\mu(x) = 1$,由$x\in H=\nu(B)$知存在$b\in B$,使$x=\nu(b)$.从而
$$
1=\mu \left( x \right) =\mu \nu \left( b \right) =\mathrm{id}_B\left( b \right) =b.
$$
故$x=\nu \left( b \right) =\nu \left( 1 \right) =1$,即$H\cap N = \{1\}$. 

对$\forall b \in B$,由$\mu\nu = \mathrm{id}_B$知$\mu(\nu(b)) = \mathrm{id}_B(b) = b$且$\nu(b) \in H$,故$\mu|_H$是$H$到$B$上的满同态.
设$x\in G$,则$\mu (x)\in B$. 于是$\exists y\in H$, 使$\mu(y) = \mu(x)$, 因而$\mu(y^{-1}x) = 1$, 即$z = y^{-1}x\in \ker\mu =N$有$x = yz\in HN$, 故$G = HN$.
\end{enumerate}
\end{proof}

\begin{definition}
设$G$是一个群,$N\lhd G$,$H$是$G$的子群,且$H\cap N = \{1\}$, $G = HN$,则称$G$是$N$与$H$的\textbf{半直积}, 记为$G = H\ltimes N$.如果$H$还是$G$的正规子群,则称$G$是$N$与$H$的\textbf{内直积}, 记为$G = H\otimes N$.
\end{definition}

\begin{definition}\label{definition:内直积和中心扩张定义}
设$G$是群$B$过群$A$的扩张, $N$是扩张的核. 

(1)如果存在$G$的子群$H$, 使$H\cap N = \{1\}$, $G = HN$, 那么称此扩张为\textbf{非本质扩张}.若$H$还是$G$的正规子群,则称此扩张为\textbf{平凡扩张}.

(2)如果$N\subseteq C(G)$, 那么称此扩张为\textbf{中心扩张}.
\end{definition}

\begin{example}
\begin{enumerate}
\item 对整数加群$\mathbb{Z}$, 它的正规子群$2\mathbb{Z}$与$\mathbb{Z}$同构, 而$\mathbb{Z}/2\mathbb{Z} = \mathbb{Z}_2$是$2$阶循环群, 因而$\mathbb{Z}$是$\mathbb{Z}_2$过$2\mathbb{Z}$的扩张. 由于$\mathbb{Z}$的任何子群都不同构于$\mathbb{Z}_2$, 因而这个扩张不是非本质扩张.
\item 设$n\geqslant 3$. $A_n$是$S_n$的正规子群, $\langle(12)\rangle$是$S_n$的$2$阶子群, $\langle(12)\rangle\cap A_n = \{\text{id}\}$, $S_n = \langle(12)\rangle A_n$, 但$\langle(12)\rangle$不是$S_n$的正规子群, 故$S_n = \langle(12)\rangle\ltimes A_n$.
\item $3$阶循环群过$5$阶循环群的扩张$G$是$15$阶群. 由\refpro{proposition:阶为特定素数乘积的群必为循环群}知这种扩张必然是平凡扩张, 即$G = \langle a\rangle\otimes\langle b\rangle$, 其中, $a$, $b$分别为$G$的$3$阶元素与$5$阶元素.
\end{enumerate}
\end{example}
\begin{proof}


\end{proof}

\begin{theorem}\label{theorem:抽象代数--定理4.5.3}
设$A$, $B$是$G$的子群.
\begin{enumerate}[(1)]
\item\label{theorem:抽象代数--定理4.5.3-1} $G = AB$, $A\cap B = \{1\}$当且仅当$\forall g\in G$, $\exists a\in A$, $b\in B$, 使得$g = ab$且这种表示唯一.

\item\label{theorem:抽象代数--定理4.5.3-2} $G$是$A$和$B$的内直积的充分必要条件是$G$满足如下两个条件:
\begin{enumerate}[(i)]
\item $G$中每个元素可唯一地表为$ab$的形式,其中$a\in A,b\in B$;
\item $A$中每个元素与$B$中任意元素可交换,即:对任意$a\in A,b\in B$,有$ab=ba$.
\end{enumerate}
\end{enumerate}
\end{theorem}
\begin{proof}
\begin{enumerate}[(1)]
\item 由$G = AB$, $A\cap B = \{1\}$知$\forall g\in G$, $\exists a\in A$, $b\in B$, 使$g = ab$. 若另有$g = a'b'$, $a'\in A$, $b'\in B$, 则$a^{-1}a' = bb'^{-1}\in A\cap B=\{1\}$, 于是$a = a'$, $b = b'$.

反之, 若$\forall g\in G$, $\exists a\in A$, $b\in B$, 使$g = ab$, 则$G = AB$. 又若$c\in A\cap B$, 由$c = 1\cdot c = c\cdot 1$的表示唯一可知$c = 1$, 故$A\cap B = \{1\}$.

\item 由\rrefthe{theorem:抽象代数-定理 1.3.5}{theorem:抽象代数-定理 1.3.5-3}知条件(i)成立当且仅当$A,B\lhd G$.又由\rrefthe{theorem:抽象代数--定理4.5.3}{theorem:抽象代数--定理4.5.3-1}知条件(ii)成立当且仅当$G=AB,A\cap B=\{1\}$.故$G=A\otimes B$当且仅当$G$同时满足条件(i)(ii).
\end{enumerate}
\end{proof}

\begin{theorem}\label{theorem:两个群的内外直积同构}
\begin{enumerate}[(1)]
\item\label{theorem:两个群的内外直积同构-1} 如果群$G$是正规子群$H$和$K$的内直积,则$H \times K \cong G$.

\item\label{theorem:两个群的内外直积同构-2} 如果群$G = G_1 \times G_2$,则存在$G$的正规子群$G_1'$和$G_2'$,且$G_i'$与$G_i$同构$(i = 1,2)$,使得$G=G_1'\otimes G_2'$.
\end{enumerate}
\end{theorem}
\begin{remark}
从这个定理可以看到,群的内外直积的概念在同构意义下是一致的,所以有时可不对内外直积加以区分,而统称为\textbf{群的直积}.
\end{remark}
\begin{proof}
\begin{enumerate}[(1)]
\item 如果群$G$是正规子群$H$和$K$的内直积.定义映射
\begin{align*}
\phi: \ H \times K \ &\longrightarrow \ G, \\
(h,k) \ &\longmapsto \ hk, \ \forall (h,k) \in H \times K,
\end{align*}
则由于$G = HK$,故$\phi$是满射.又由\rrefthe{theorem:抽象代数--定理4.5.3}{theorem:抽象代数--定理4.5.3-2}知$G$中元素为$hk$形式时表法唯一,故$\phi$是单射.又对任意的$(h_1,k_1), (h_2,k_2) \in H \times K$,由于$H$中的元素与$K$中的元素可交换,故
\begin{align*}
\phi((h_1,k_1) \cdot (h_2,k_2)) &= \phi(h_1h_2,k_1k_2) = (h_1h_2)(k_1k_2) \\
&= (h_1k_1)(h_2k_2) = \phi(h_1,k_1) \cdot \phi(h_2,k_2),
\end{align*}
所以$\phi$是同构映射,从而$H \times K \cong G$.

\item 如果$G = G_1 \times G_2$.令
\begin{align*}
G_1' = \{(a_1,e_2) \mid a_1 \in G_1\}, \ G_2' = \{(e_1,a_2) \mid a_2 \in G_2\},
\end{align*}
则容易验证$G_1', G_2'$都是$G$的子群,且对任意的$(a_1,a_2) \in G$,
\begin{align*}
(a_1,a_2) = (a_1,e_2)(e_1,a_2) \in G_1'G_2'.
\end{align*}
这一表法是唯一的,且对任意的$(a_1,e_2) \in G_1', (e_1,a_2) \in G_2'$,有
\begin{align*}
(a_1,e_2) \cdot (e_1,a_2) = (a_1,a_2) = (e_1,a_2) \cdot (a_1,e_2),
\end{align*}
所以由\rrefthe{theorem:抽象代数--定理4.5.3}{theorem:抽象代数--定理4.5.3-2}知$G$是$G_1'$与$G_2'$的内直积.而
\begin{align*}
\phi_1: \ a_1 \longmapsto (a_1,e_2)
\end{align*}
以及
\begin{align*}
\phi_2: \ a_2 \longmapsto (e_1,a_2)
\end{align*}
分别为$G_1$到$G_1'$和$G_2$到$G_2'$的同构映射.
\end{enumerate}

\end{proof}

\begin{theorem}\label{theorem:抽象代数--定理4.5.4}
设$A$, $B$是两个群, 则一定存在$B$过$A$的平凡扩张$G=A\times B$, 并且$G$在同构意义下唯一.
\end{theorem}
\begin{proof}
在$G = A\times B = \{(a, b)\mid a\in A, b\in B\}$中定义乘法
\begin{align*}
(a_1, b_1)(a_2, b_2) = (a_1a_2, b_1b_2),\quad \forall (a_i, b_i)\in G, i = 1, 2.
\end{align*}
容易验证$G$是群, 幺元为$(1, 1')$, 其中, $1$, $1'$分别为$A$, $B$的幺元. $\forall (a, b)\in G$, $(a, b)^{-1} = (a^{-1}, b^{-1})$, 而且
$$
A' = \{(a, 1')\mid a\in A\},\quad B' = \{(1, b)\mid b\in B\}
$$
都是$G$的正规子群. 又$G = A'B'$, $A'\cap B' = \{(1, 1')\}$, 于是$G = A'\otimes B'$. 

又映射$\lambda$: $\lambda(a) = (a, 1')$($\forall a\in A$)是$A$到$G$的单同态, $\lambda(A) = A'$,故$\lambda$是$A$到$A'$的同构. 而映射$\mu$: $\mu((a, b)) = b$则是$G$到$B$上的同态, 并且$\ker\mu = A'=\lambda(A)$, $\mu|_{B'}$是$B'$到$B$上的同构.即有(短)正合序列
\begin{align*}
1\longrightarrow A \stackrel{\lambda}{\longrightarrow} G \stackrel{\mu}{\longrightarrow} B\longrightarrow 1,
\end{align*}
故由\refpro{proposition:正合序列的基本性质}知$G$是$B$过$A$的扩张,扩张核为$A'$.
由$G = A'\otimes B'$及$A\cong A' ,B\cong B'$知
\begin{align*}
G=A'B',\quad A' \cap B' =\left\{ 1 \right\},\quad B'\lhd G.
\end{align*}
故$G$是$B$过$A$的平凡扩张.

设$G_1$也是$B$过$A$的平凡扩张, 于是$G_1 = A_1\otimes B_1$. 设$\lambda_1$为$A$到$A_1$的同构, $\gamma_1$是$B$到$B_1$的同构, 令
\begin{align*}
f((a, b)) = \lambda_1(a)\gamma_1(b),\quad \forall a\in A, b\in B.
\end{align*}
由$G_1=A_1\otimes B_1$知$G_1=A_1B_1$,$A_1\cap B_1=\{1\}$且$A_1,B_1\lhd G_1$。从而由\rrefthe{theorem:抽象代数-定理 1.3.5}{theorem:抽象代数-定理 1.3.5-3}知
\begin{align*}
a_1b_1=b_1a_1,\quad \forall a_1\in A_1,b_1\in B_1.
\end{align*}
于是对$\forall (a,b),(a',b')\in G$,有
\begin{align*}
f((a,b)(a',b'))&=f((aa',bb'))=\lambda_1(aa')\gamma_1(bb') \\
&=\lambda_1(a)\lambda_1(a')\gamma_1(b)\gamma_1(b')=\lambda_1(a)\gamma_1(b)\lambda_1(a')\gamma_1(b') \\
&=f((a,b))f((a',b')).
\end{align*}
因此$f$是$G$到$G_1$的同态。

设$a_1b_1\in A_1B_1=G_1$,则由$\lambda_1$是$A$到$A_1$的同构,$\gamma_1$是$B$到$B_1$的同构可知,存在$a\in A$,$b\in B$,使
\begin{align*}
\lambda_1(a)=a_1,\gamma_1(b)=b_1\Longrightarrow f((a,b))=\lambda_1(a)\gamma_1(b)=a_1b_1.
\end{align*}
故$f$是满同态。

设$f((a,b))=f((a',b'))\in G_1$,则
\begin{align*}
\lambda_1(a)\gamma_1(b)=f((a,b))=f((a',b'))=\lambda_1(a')\gamma_1(b').
\end{align*}
由\rrefthe{theorem:抽象代数--定理4.5.3}{theorem:抽象代数--定理4.5.3-1}知$\lambda_1(a)=\lambda_1(a')$,$\gamma_1(b)=\gamma_1(b')$。又$\lambda_1$是$A$到$A_1$的同构,$\gamma_1$是$B$到$B_1$的同构,故
\begin{align*}
a=a',b=b'\Longrightarrow (a,b)=(a',b').
\end{align*}
因此$f$是单同态。
综上可知$f$是$G$到$G_1$的同构。

\end{proof}

\begin{definition}
若$N_1$, $N_2$, $\cdots$, $N_k$都是群$G$的正规子群,并且
\begin{align*}
G = N_1N_2\cdots N_k,\,\,\text{其中}N_i\cap \prod_{j=1}^{i-1}{N_j}=\left\{ 1 \right\} ,\,\,i=1,2,\cdots ,k.
\end{align*}
则称$G$是$N_1$, $N_2$, $\cdots$, $N_k$的\textbf{内直积}, 记为
\begin{align*}
G = N_1\otimes N_2\otimes\cdots\otimes N_k.
\end{align*}
\end{definition}
\begin{remark}
由$N_i\cap \prod_{j=1}^{i-1}{N_j}=\left\{ 1 \right\} ,\,\,i=1,2,\cdots ,k$可推出$N_i\cap \prod_{j\neq i}{N_j}=\left\{ 1 \right\} ,\,\,i=1,2,\cdots ,k.$
\end{remark}

\begin{theorem}
如果群$G$是有限多个子群$H_1,H_2,\cdots,H_n$的内直积,则$G$同构于$H_1,H_2,\cdots,H_n$的外直积.
\end{theorem}
\begin{proof}
对$n$用数学归纳法.
当$n=2$时,由\rrefthe{theorem:两个群的内外直积同构}{theorem:两个群的内外直积同构-1}知结论成立.

假定结论对$n-1$成立.考察$G$是$n$个正规子群$H_1,H_2,\cdots,H_n$的内直积的情形.令$K=H_1H_2\cdots H_{n-1}$,则由\rrefpro{proposition:正规子群的基本性质}{proposition:正规子群的基本性质-3}知$K$为$G$的正规子群,由\rrefpro{proposition:正规子群的基本性质}{proposition:正规子群的基本性质-2}知$H_i(i=1,2,\cdots,n-1)$为$K$的正规子群.从而由内直积的定义可得,$G$为$K$与$H_n$的内直积,$K$为正规子群$H_1,H_2,\cdots,H_{n-1}$的内直积.于是由\rrefthe{theorem:两个群的内外直积同构}{theorem:两个群的内外直积同构-1}及归纳假设得
\begin{align*}
G \cong K \times H_n, \quad
K \cong H_1 \times H_2 \times \cdots \times H_{n-1}.
\end{align*}
因此
\begin{align*}
G \cong H_1 \times H_2 \times \cdots \times H_n.
\end{align*}

\end{proof}

\begin{theorem}\label{theorem:群的内直积分解的基本性质}
\begin{enumerate}[(1)]
\item\label{theorem:群的内直积分解的基本性质-1} 设群$N_1,N_2,\cdots,N_k$的内直积为$G$,即$G = N_1\otimes N_2\otimes\cdots\otimes N_k$,则对$\forall i,j\in \{1,2,\cdots,k\}$,有
\begin{align*}
n_in_j=n_jn_i,\,\,\forall n_i\in N_i,n_j\in N_j.
\end{align*}

\item\label{theorem:群的内直积分解的基本性质-2} 
设有限群$N_1,N_2,\cdots,N_k$的内直积为$G$,即$G = N_1\otimes N_2\otimes\cdots\otimes N_k$,则
\begin{align*}
|G|=|N_1||N_2|\cdots |N_k|.
\end{align*}

\item\label{theorem:群的内直积分解的基本性质-4} 设$G$是一个群,群$N_1,N_2,\cdots,N_k\lhd G$,则
\begin{align*}
N_1\otimes N_2\otimes\cdots\otimes N_k\lhd G.
\end{align*}

\item\label{theorem:群的内直积分解的基本性质-3} 设$N_1$, $N_2$, $\cdots$, $N_k$都是群$G$的正规子群,则群$G$满足
\begin{align*}
G = N_1\otimes N_2\otimes\cdots\otimes N_k.
\end{align*}
的充要条件是$G$中任一元素可分解为$N_i$($1\leqslant i\leqslant k$)中元素的积且这种分解是唯一的.
\end{enumerate}
\end{theorem}
\begin{proof}
\begin{enumerate}[(1)]
\item 由内直积的定义知当$i\neq j$时,有$N_i\cap N_j\subseteq N_i\cap \prod_{j\ne i}{N_j}=\left\{ 1 \right\}$,故利用\rrefpro{proposition:正规子群的基本性质}{proposition:正规子群的基本性质-7}得,对$\forall i,j\in \{1,2,\cdots,k\}$,有
\begin{align*}
n_in_j=n_jn_i,\,\,\forall n_i\in N_i,n_j\in N_j.
\end{align*}

\item 由内直积的定义知当$i\neq j$时,有$N_i\cap N_j\subseteq N_i\cap \prod_{j\ne i}{N_j}=\left\{ 1 \right\}$,故利用\refpro{proposition:有限群乘积的阶}得
\begin{align*}
|G|=|N_1N_2\cdots N_k|=|N_1||N_2|\cdots |N_k|.
\end{align*}

\item 由内直积的定义和\rrefpro{proposition:正规子群的基本性质}{proposition:正规子群的基本性质-3}立得.

\item {\heiti 必要性:}由内直积的定义知$G=N_1N_2\cdots N_k$且$N_i\cap N_j=\{1\}(i\ne j)$,则显然$G$中任一元素可分解为$N_i(1\leqslant i\leqslant k)$中元素的积.
由\rrefthe{theorem:群的内直积分解的基本性质}{theorem:群的内直积分解的基本性质-1}知,对$\forall i,j\in \{1,2,\cdots,k\}$,有
\begin{align}
n_in_j=n_jn_i,\,\,\forall n_i\in N_i,n_j\in N_j.\label{eq:::asfj8wj3fj20321.1}
\end{align}
设$g\in G$且
\begin{align*}
g=x_1x_2\cdots x_k=y_1y_2\cdots y_k,\quad x_i,y_i\in N_i(i=1,2,\cdots,k).
\end{align*}
则由\eqref{eq:::asfj8wj3fj20321.1}式可得
\begin{align*}
1=x_1x_2\cdots x_ky_{k}^{-1}y_{k-1}^{-1}\cdots y_{1}^{-1}=(x_1y_{1}^{-1})(x_2y_{2}^{-1})\cdots(x_ky_{k}^{-1}).
\end{align*}
记$n_i=x_iy_{i}^{-1}\in N_i(i=1,2,\cdots,k)$,则
\begin{align*}
1=n_1n_2\cdots n_k.
\end{align*}
再结合\eqref{eq:::asfj8wj3fj20321.1}式可得,对$\forall i\in\{1,2,\cdots,k\}$,都有
\begin{align*}
n_i=\prod\limits_{j\ne i}n_{j}^{-1}\in N_i\cap\prod\limits_{j\ne i}N_j=\{1\}.
\end{align*}
故$n_i=1$,即$x_i=y_i,\,i=1,2,\cdots,k$.因此$g$的分解唯一.

{\heiti 充分性:}由$G$中任一元素可分解为$N_i(1\leqslant i\leqslant k)$中元素的积知$G\subseteq N_1N_2\cdots N_k$.又因为$N_i\lhd G(i=1,2,\cdots,k)$,所以$N_1N_2\cdots N_k\subseteq G$.故$G=N_1N_2\cdots N_k$.
若存在$i,j\in\{1,2,\cdots,k\}$,使得$N_i\cap N_j\ne\{1\}$.取$n_i\in N_i\cap N_j$,则
\begin{align*}
n_i=1\cdots\underset{\text{第}i\text{个位置}}{\underbrace{n_i}}\cdots1=1\cdots\underset{\text{第}j\text{个位置}}{\underbrace{n_i}}\cdots1\in N_1N_2\cdots N_k.
\end{align*}
这与$n_i$的分解唯一矛盾!故$N_i\cap N_j=\{1\}(i\ne j)$.因此
\begin{align*}
G=N_1\otimes N_2\otimes\cdots\otimes N_k.
\end{align*}
\end{enumerate}

\end{proof}

\begin{proposition}\label{proposition:抽象代数--群的内直积分解}
设$G$为有限群,$|G|=p_1^{a_1}p_2^{a_2}\cdots p_k^{a_k}$,$p_1,p_2,\cdots,p_k$为互不相等的素数.又每个Sylow $p_i$子群$P_i\lhd G$.则
\begin{align*}
G=P_1\otimes P_2\otimes\cdots\otimes P_k.
\end{align*}
\end{proposition}
\begin{proof}
由\hyperref[theorem:Sylow第三定理-1]{Sylow第三定理\ref{theorem:Sylow第三定理-1}}知$P_i$是$G$中唯一的Sylow $p_i$子群$(i=1,2,\cdots,k)$.由条件知$|P_i|=p_{i}^{a_i},i=1,2,\cdots,k$.再由\rrefthe{theorem:群的内直积分解的基本性质}{theorem:群的内直积分解的基本性质-2}知
\begin{align*}
\left| \prod\limits_{i=1}^k{P_i} \right|=|P_1||P_2|\cdots|P_k|=p_{1}^{a_1}p_{2}^{a_2}\cdots p_{k}^{a_k}=|G|.
\end{align*}
显然$\prod\limits_{i=1}^k{P_i}\subseteq G$,故$G=\prod\limits_{i=1}^k{P_i}$.

由\refpro{theorem:Sylow p群中元素的阶}知对$\forall x_i\in P_i\backslash\{1\},i=1,2,\cdots,k$,都存在$k_i\in\mathbb{N}\backslash\{0\}$,使
\begin{align*}
\mathrm{ord}\,x_i=p_{i}^{k_i},\quad i=1,2,\cdots,k.
\end{align*}
又因为$p_1,\cdots,p_k$是互不相同的素数,所以
\begin{align*}
\mathrm{ord}\prod\limits_{j\ne i}{x_j}=\prod\limits_{j\ne i}{p_{j}^{k_j}}\ne p_{i}^{k_i}=\mathrm{ord}\,x_i,\quad i=1,2,\cdots,k.
\end{align*}
因此$P_i\cap\prod\limits_{j\ne i}{P_j}=\{1\},i=1,2,\cdots,k$.从而$P_i\cap P_j=\{1\}(i\ne j)$.

综上可知
\begin{gather*}
G=\prod_{i=1}^k{P_i},\quad P_i\lhd G,\quad 1\leqslant i\leqslant k,
\\
P_i\cap \prod_{j\ne i}{P_j=\{1\},\quad 1}\leqslant i\leqslant k,
\end{gather*}
故$G=P_1\otimes P_2\otimes\cdots\otimes P_k$.




\end{proof}





\end{document}
\documentclass[../../main.tex]{subfiles}% 注意这里的文件路径不能用 ./main.tex ,否则用latexmk编译子文件会报错
\graphicspath{{\subfix{./image/}}} % 指定图片目录,后续可以直接使用图片文件名
% 注意这里的文件路径不能用 ../../image/ ,否则用latexmk编译子文件会报错

% 例如:
% \begin{figure}[H]
% \centering
% \includegraphics[scale=0.3]{图.png}
% \caption{}
% \label{figure:图}
% \end{figure}
% 注意:上述\label{}一定要放在\caption{}之后,否则引用图片序号会只会显示??.

\begin{document}

\section{群的生成组}

\begin{definition}
设$S$是群$G$的非空子集,以$\langle S\rangle$表示$G$的包含$S$的最小子群,即$S$\textbf{生成的子群}.
显然,$\langle S\rangle$是$G$中所有包含$S$的子群之交,即$\langle S\rangle=\bigcap_{S<H}{H}.$
\end{definition}
\begin{note}
由\rrefpro{proposition:抽象代数---子群的基本性质}{proposition:抽象代数---子群的基本性质-2}知$\langle S\rangle=\bigcap_{S<H}{H}$是一个群,故上述定义是良定义的.
\end{note}

\begin{theorem}\label{theorem:生成子群的元素的形式}
设$S$是群$G$的非空子集,$S^{-1}$是$S$中所有元素的逆元构成的集合,则
\begin{align*}
\langle S\rangle = \{x_1x_2\cdots x_m \mid x_i \in S \cup S^{-1}, 1 \leqslant i \leqslant m, m \in \mathbb{N}\}.
\end{align*}
进而若$S$在群$H$中,则$S\subseteq H$.
\end{theorem}
\begin{proof}
令$\bar{S} = \{x_1x_2\cdots x_m \mid x_i \in S \cup S^{-1}, 1 \leqslant i \leqslant m, m \in \mathbb{N}\}$. 由$\langle S\rangle$为子群且$S \subseteq \langle S\rangle$知$S^{-1} \subseteq \langle S\rangle$,因而$S \subseteq \bar{S} \subseteq \langle S\rangle$. 又$\langle S\rangle$是含$S$的最小子群,故只需证明$\bar{S}$为子群,则$\bar{S} \supseteq \langle S\rangle$.

设$x_1x_2\cdots x_m \in \bar{S}$,$y_1y_2\cdots y_n \in \bar{S}$,于是$y_i^{-1} \in S \cup S^{-1}(1 \leqslant i \leqslant m)$,则有
\begin{align*}
(x_1x_2\cdots x_m)(y_1y_2\cdots y_n)^{-1} = x_1x_2\cdots x_m y_n^{-1} y_{n-1}^{-1} \cdots y_2^{-1} y_1^{-1} \in \bar{S},
\end{align*}
因而$\bar{S}$为$G$的子群,故$\bar{S} = \langle S\rangle$.

\end{proof}

\begin{definition}
若$S$为群$G$的子集且$G = \langle S\rangle$,则称$S$为$G$的\textbf{生成组}. 若$G$有一个含有限个元素的生成组,则称$G$是\textbf{有限生成的}.

若$G = \langle a\rangle$为循环群,则$a$本身就是生成组,这时称$a$为$G$的\textbf{生成元}.
\end{definition}

\begin{definition}[全变换群/置换群]
设 \( X \) 是非空集合. 以 \( S_X \) 表示 \( X \) 的所有可逆变换 (即 \( X \) 到 \( X \) 的一一对应) 的集合, 则 \( S_X \) 对变换的乘法构成一个群, \( \text{id}_X \) 为左幺元, \( f^{-1} \) 为 \( f \) 的左逆元. \( S_X \) 称 \( X \) 的\textbf{全变换群}.$S_X$的子群称为$X$上的\textbf{变换群}.

如果集合$X$所含元素的个数\( |X| = n < +\infty \). 此时 \( S_X \) 记为 \( S_n \), 称为 \( n \) 个文字的\textbf{对称群}或 \( n \) 个文字的\textbf{置换群},也称为\textbf{$n$阶置换群}, 其元素称为\textbf{置换}.
\end{definition}

\begin{definition}
假定集合\( X = \{1, 2, \cdots, n\} \),记$S_n$为$X$的对称群,设 \( \sigma \in S_n \), 则 \( \sigma(1), \sigma(2), \cdots, \sigma(n) \) 是 \( 1, 2, \cdots, n \) 的一个排列. 常用下面记法:
\[
\sigma = \begin{pmatrix} 1 & 2 & & n \\ \sigma(1) & \sigma(2) & \cdots & \sigma(n) \end{pmatrix}
\]
更一般地, 若 \( i_1, i_2, \cdots, i_n \) 是 \( 1, 2, \cdots, n \) 的一个排列, 则可记
\[
\sigma = \begin{pmatrix} i_1 & i_2 & \cdots & i_n \\ \sigma(i_1) & \sigma(i_2) & \cdots & \sigma(i_n) \end{pmatrix}
\]
易知 \( S_n \) 中有 \( n! \) 个元素, \( S_n \) 中一个元素可以有 \( n! \) 种表示法.

例如, \( \sigma \in S_3 \), 满足 \( \sigma(1) = 2, \sigma(2) = 3, \sigma(3) = 1 \), 则可记
\[
\sigma = \begin{pmatrix} 1 & 2 & 3 \\ 2 & 3 & 1 \end{pmatrix} = \begin{pmatrix} 1 & 3 & 2 \\ 2 & 1 & 3 \end{pmatrix} = \begin{pmatrix} 2 & 1 & 3 \\ 3 & 2 & 1 \end{pmatrix} = \cdots
\]
\end{definition} 

\begin{theorem}\label{theorem:奇置换与偶置换}
设\( n \) 个不定元 \( x_1, x_2, \cdots, x_n \) 的多项式
\[
A = \prod_{1 \leqslant i < j \leqslant n} (x_i - x_j) \in \mathbb{C}[x_1, x_2, \cdots, x_n].
\]
记$S_n$为$\{1,2,\cdots,n\}$的对称群,对于 \( \sigma \in S_n \),令
\[
A_\sigma = \prod_{1 \leqslant i < j \leqslant n} (x_{\sigma(i)} - x_{\sigma(j)}),
\]
则$A_{\sigma}=\pm A$.若 \( A_\sigma = A \), 则称 \( \sigma \) 为\textbf{偶置换}, 并记 \( \text{sgn}\sigma = 1 \); 若 \( A_\sigma = -A \), 则称 \( \sigma \) 为\textbf{奇置换}, 并记 \( \text{sgn}\sigma = -1 \), \( \text{sgn}\sigma \) 称为 \( \sigma \) 的\textbf{符号}. 故有$\text{sgn}$是$S_n$到$\{-1,1\}$的同态且
\[
A_\sigma = \text{sgn}\sigma A.
\]
令 \( A_n \) 为 \( S_n \) 中偶置换集合, 即
\[
A_n \triangleq \{\sigma \in S_n|\text{sgn}\sigma = 1\},
\]
则 \( A_n \) 为 \( S_n \) 的子群.\( A_n \) 称为 \( n \) 个文字的\textbf{交错群}或\textbf{交代群},也称为\textbf{$n$阶交错群}或\textbf{交代群}.
\end{theorem}
\begin{proof}
先证明 \( A_\sigma = \pm A \). 注意到 \( A \) 中没有 \( x_i - x_j \) 的重因式, 因而只需说明 \( A_\sigma \) 中没有重因式即可. 设有 \( \{\sigma(i), \sigma(j)\} = \{\sigma(k), \sigma(l)\} \), 则有如下两种可能:

(1) \( \sigma(i) = \sigma(k), \sigma(j) = \sigma(l) \), 则有 \( i = k, j = l \);

(2) \( \sigma(i) = \sigma(l), \sigma(j) = \sigma(k) \), 则有 \( i = l, j = k \),

因而都有 \( \{i, j\} = \{k, l\} \), 由此知 \( A_\sigma = \pm A \).

事实上, 若 \( \tau, \sigma \in S_n \), 则有
\[
A_{\sigma\tau} = \prod_{1 \leqslant i < j \leqslant n} (x_{\sigma\tau(i)} - x_{\sigma\tau(j)}).
\]
将 \( A_{\sigma\tau} \) 与 \( A_\sigma \) 进行比较. 若 \( \tau(i) < \tau(j) \), 则 \( x_{\sigma\tau(i)} - x_{\sigma\tau(j)} \) 仍是 \( A_\sigma \) 中一个因子; 若 \( \tau(i) > \tau(j) \), 则 \( x_{\sigma\tau(j)} - x_{\sigma\tau(i)} = -(x_{\sigma\tau(i)} - x_{\sigma\tau(j)}) \) 为 \( A_\sigma \) 中一因子, 因而将 \( A_\sigma \) 变成 \( A_{\sigma\tau} \) 时改变因子符号的次数与将 \( A \) 变成 \( A_\tau \) 时改变因子符号的次数相同, 因而有
\[
A_{\sigma\tau} = \text{sgn}\tau \cdot \prod_{1 \leqslant i < j \leqslant n} (x_{\sigma(i)} - x_{\sigma(j)}) = \text{sgn}\sigma\text{sgn}\tau A.
\]
于是
\begin{align*}
\text{sgn}(\sigma\tau) = \text{sgn}\sigma\text{sgn}\tau, \quad \forall \sigma, \tau \in S_n.
\end{align*}
故$\text{sgn}$是$S_n$到$\{-1,1\}$的同态.
又注意到$\text{sgn}\tau^{-1}=\text{sgn}\tau,  \forall \tau \in S_n$,故
\begin{align*}
\text{sgn}(\sigma\tau^{-1}) = \text{sgn}\sigma\text{sgn}\tau^{-1}=\text{sgn}\sigma\text{sgn}\tau=1\Longrightarrow \sigma\tau^{-1}\in A_n, \quad \forall \sigma, \tau \in A_n.
\end{align*}
由此知 \( A_n \) 为 \( S_n \) 的子群.

\end{proof}

\begin{example}
设 \( \sigma \) 是 \( S_n \) 中任一奇置换, 则有 \( S_n = A_n \cup \sigma A_n \), 故 \( [S_n: A_n] = 2 \).
\end{example}
\begin{proof}


\end{proof}

\begin{example}
设$G = S_3$,又$a = \begin{pmatrix} 1 & 2 & 3 \\ 2 & 1 & 3 \end{pmatrix}$,$b = \begin{pmatrix} 1 & 2 & 3 \\ 3 & 2 & 1 \end{pmatrix}$,则$S_3 = \langle \{a, b\} \rangle$.
\end{example}
\begin{proof}
事实上,设$G_1 = \langle a\rangle$,注意到
\begin{align*}
a^{-1}=\begin{pmatrix}1&2&3\\1&2&3\end{pmatrix},a^2=(a^{-1})^2=\begin{pmatrix}1&2&3\\1&2&3\end{pmatrix},
\end{align*}
故由\refthe{theorem:生成子群的元素的形式}知$G_1=\{a,a^{-1}\}$。从而$G_1$为$S_3$的2阶子群且$b \notin G_1$,于是$G_1 \subset \langle \{a, b\} \rangle$. 设$\langle \{a, b\} \rangle$的阶为$n$,则由\hyperref[theorem:抽象代数--Lagrange定理]{Lagrange定理}知$2\mid n$且$2 < n$.又因为$\langle \{a, b\} \rangle$是$G$的子群,所以
由\hyperref[theorem:抽象代数--Lagrange定理]{Lagrange定理}知$n\mid 6$.因而有$n = 6$,由此知$S_3 = \langle \{a, b\} \rangle$.

\end{proof}

\begin{definition}
设集合$\{i_1, i_2, \cdots, i_r\}$为集合$\{1, 2, \cdots, n\}$的子集. 若$\sigma \in S_n$满足
\begin{align*}
\sigma(i_j) &= i_{j+1}, \quad 1 \leqslant j \leqslant r-1, \\
\sigma(i_r) &= i_1, \\
\sigma(k) &= k, \quad k \notin \{i_1, i_2, \cdots, i_r\},
\end{align*}
则称$\sigma$为一个\textbf{长为$r$的轮换}或\textbf{$r$轮换},这时记$\sigma = (i_1 i_2 \cdots i_r)$.
特别地,将2轮换$(ij)$称为\textbf{对换}.将$S_n$中的幺元$\text{id}$记为长为1的轮换,即$\text{id} = (i),i=1,2,\cdots,n$

若$\sigma = (i_1 i_2 \cdots i_r)$与$\sigma = (j_1 j_2 \cdots j_s)$是两个轮换且
$$\{i_1, i_2, \cdots, i_r\} \bigcap \{j_1, j_2, \cdots, j_s\} = \varnothing,$$
则称$\sigma$与$\sigma$为\textbf{不相交的轮换}.

显然,一个$r$轮换$(i_1 i_2 \cdots i_r)$有$r$种不同的表示,
\begin{align*}
(i_1 i_2 \cdots i_r) = (i_2 i_3 \cdots i_r i_1) = \cdots = (i_r i_1 \cdots i_{r-1}).
\end{align*}
\end{definition}

\begin{proposition}\label{proposition:对换与轮换的基本性质}
\begin{enumerate}[(1)]
\item\label{proposition:对换与轮换的基本性质-1} $[(i_1i_1')(i_2i_2')\cdots (i_ri_r')]^{-1}=(i_ri_r')(i_{r-1}i_{r-1}')\cdots (i_1i_1')$.

\item\label{proposition:对换与轮换的基本性质-2} $(i_1i_2\cdots i_r)^{-1}=(i_ri_{r-1}\cdots i_1).$特别地,$(i_1i_2)^{-1}=(i_2i_1)=(i_1i_2).$

\item\label{proposition:对换与轮换的基本性质-7} 任何两个不相交的轮换的乘积是可以交换的.

\item\label{proposition:对换与轮换的基本性质-4} $(kl)(ka\cdots b)(lc\cdots d) = (ka\cdots blc\cdots d),$其中$a,\cdots,b,c,\cdots,d,k,l$为互不相同的正整数.

\item\label{proposition:对换与轮换的基本性质-5} $(kl)(ka\cdots blc\cdots d) = (ka\cdots b)(lc\cdots d),$
其中$a,\cdots,b,c,\cdots,d,k,l$为互不相同的正整数.

\item\label{proposition:对换与轮换的基本性质-3} 对任意$r$轮换$(i_1i_2\cdots i_r)$和$\sigma \in S_n$,都有
\begin{align*}
\sigma \left( i_1i_2\cdots i_r \right) \sigma ^{-1}=\left( \sigma \left( i_1 \right) \sigma \left( i_2 \right) \cdots \sigma \left( i_r \right) \right) .
\end{align*}

\item\label{proposition:对换与轮换的基本性质-6} 任何轮换$(i_1 i_2 \cdots i_r)$可写成如下对换之积
\begin{align*}
(i_1i_2\cdots i_r)=(i_1i_2)(i_2i_3)\cdots (i_{r-1}i_r)=(i_1i_r)(i_1i_{r-1})\cdots (i_1i_2).
\end{align*}
\end{enumerate}
\end{proposition}
\begin{proof}
\begin{enumerate}[(1)]
\item 

\item 

\item 设$\sigma = (i_1\ i_2\ \cdots\ i_r)$与$\tau = (j_1\ j_2\ \cdots\ j_s)$是两个不相交的轮换,$a$是$X$中的任意一个数.

\begin{enumerate}[(a)]
\item 如果$a \neq i_k,j_l(k = 1,2,\cdots,r;l = 1,2,\cdots,s)$,则
\begin{align*}
\sigma\tau(a) = \sigma(a) = a, \\
\tau\sigma(a) = \tau(a) = a,
\end{align*}
所以$\sigma\tau(a) = \tau\sigma(a)$.

\item 如果$a = i_k(1 \leqslant k \leqslant r)$,则$a,\sigma(a) \neq j_l(l = 1,2,\cdots,s)$.从而
\begin{align*}
\sigma\tau(a) = \sigma(a), \\
\tau\sigma(a) = \tau(\sigma(a)) = \sigma(a),
\end{align*}
所以$\sigma\tau(a) = \tau\sigma(a)$.

\item 同理可证,如果$a = j_l(1 \leqslant l \leqslant s)$,也有$\sigma\tau(a) = \tau\sigma(a)$.
\end{enumerate}
这就证明了结论.

\item 

\item 

\item 对$\forall l \in \{1,2,\cdots,n\}$,若$l \notin \{i_1,i_2,\cdots,i_r\}$,则
\begin{align*}
\sigma(i_1i_2\cdots i_r)\sigma^{-1}(\sigma(l)) = \sigma(i_1i_2\cdots i_r)(l) = \sigma(l) = (\sigma(i_1)\sigma(i_2)\cdots\sigma(i_r))(\sigma(l)).
\end{align*}
若$l \in \{i_1,i_2,\cdots,i_r\}$,设$l = i_j$,$j \in \{1,2,\cdots,r\}$,则
\begin{align*}
\sigma(i_1i_2\cdots i_r)\sigma^{-1}(\sigma(i_j)) &= \sigma(i_1i_2\cdots i_r)(i_j) = \sigma(i_{j+1}) = (\sigma(i_1)\sigma(i_2)\cdots\sigma(i_r))(\sigma(i_j)),\ j=1,2,\cdots,r-1; \\
\sigma(i_1i_2\cdots i_r)\sigma^{-1}(\sigma(i_r)) &= \sigma(i_1i_2\cdots i_r)(i_r) = \sigma(i_1) = (\sigma(i_1)\sigma(i_2)\cdots\sigma(i_r))(\sigma(i_r)).
\end{align*}
故
\begin{align*}
\sigma(i_1i_2\cdots i_r)\sigma^{-1}(\sigma(l)) = (\sigma(i_1)\sigma(i_2)\cdots\sigma(i_r))(\sigma(l)),\ \forall l \in \{i_1,i_2,\cdots,i_r\}.
\end{align*}
即
\begin{align*}
\sigma(i_1i_2\cdots i_r)\sigma^{-1} = (\sigma(i_1)\sigma(i_2)\cdots\sigma(i_r)).
\end{align*}

\item 由对换的定义可得
\begin{gather*}
(i_1i_2)(i_2i_3)\cdots (i_{r-1}i_r)\left( i_r \right) =i_1,
\\
(i_1i_2)(i_2i_3)\cdots (i_{r-1}i_r)\left( k \right) =k,\quad k\notin \left\{ i_1,i_2,\cdots ,i_r \right\} .
\end{gather*}
故$(i_1i_2\cdots i_r)=(i_1i_2)(i_2i_3)\cdots (i_{r-1}i_r).$

再利用数学归纳法证明任何轮换$(i_1 i_2 \cdots i_r)$可写成如下对换之积
\begin{align}
(i_1 i_2 \cdots i_r) = (i_1 i_r)(i_1 i_{r-1}) \cdots (i_1 i_2). \label{eq:::09gjhy45he4t2fbety4swygew3tygwef23g2.5.2-1}
\end{align}
当$r = 2$时,\eqref{eq:::09gjhy45he4t2fbety4swygew3tygwef23g2.5.2-1}式显然成立. 假设定理对$r - 1(r \geqslant 3)$成立,并记$a = (i_1 i_2 \cdots i_r)$,于是有
\begin{align*}
(i_1 i_r)(i_1 i_{r-1}) \cdots (i_1 i_3)(i_1 i_2) = (i_1 i_3 \cdots i_r)(i_1 i_2) = a'.
\end{align*}

当$j \neq i_k$时,$a'(j) = j = a(j)$;

当$j = i_k(k \geqslant 3)$时,$a'(j) = (i_1 i_3 \cdots i_r)(j) = a(j)$;

当$j = i_1$时,$a'(i_1) = (i_1 i_3 \cdots i_r)(i_2) = i_2 = a(i_1)$;

当$j = i_2$时,$a'(i_2) = (i_1 i_3 \cdots i_r)(i_1) = i_3 = a(i_2)$.

综上知$a=a'.$
故知式\eqref{eq:::09gjhy45he4t2fbety4swygew3tygwef23g2.5.2-1}成立.
\end{enumerate}

\end{proof}

\begin{theorem}[Ruffini定理]\label{theorem:抽象代数--Ruffini(鲁菲尼)定理}
设$a \in S_n$且$a = \sigma_1 \sigma_2 \cdots \sigma_k$,其中,$\sigma_i$为$r_i$轮换且当$i \neq j$时,$\sigma_i$与$\sigma_j$不相交,$1 \leqslant i, j \leqslant k$,则$a$的阶为$r_1, r_2, \cdots, r_k$的最小公倍数$[r_1, r_2, \cdots, r_k]$.进而$\sigma_i$的阶为$r_i$.
\end{theorem}
\begin{proof}
{\color{blue}证法一:}设$m = [r_1, r_2, \cdots, r_s]$. 由\rrefpro{proposition:对换与轮换的基本性质}{proposition:对换与轮换的基本性质-7}知不相交轮换的乘积是可以互相交换的, 因此
\begin{align*}
\sigma ^m=\left( \sigma _1\sigma _2\cdots \sigma _s \right) ^m=\sigma _{1}^{m}\sigma _{2}^{m}\cdots \sigma _{s}^{m}=(1),
\end{align*}
从而$\text{ord}\sigma \mid m$.

另一方面, 设$\sigma_1 = (i_1\ i_2\ \cdots\ i_{r_1})$, 则对任意的$i_j \in \{i_1, i_2, \cdots, i_{r_1}\}$, 由于$\sigma_1, \sigma_2, \cdots, \sigma_s$为互不相交的轮换, 因此
\begin{align*}
\sigma_1^{\text{ord}\sigma}(i_j) = \sigma_1^{\text{ord}\sigma}\sigma_2^{\text{ord}\sigma}\cdots\sigma_s^{\text{ord}\sigma}(i_j) 
= \sigma^{\text{ord}\sigma}(i_j) = i_j.
\end{align*}
由此推出$\sigma_1^{\text{ord}\sigma} = (1)$, 从而$r_1 \mid \text{ord}\sigma$. 同理可证$r_i \mid \text{ord}\sigma(i = 1, 2, \cdots, s)$. 于是
\begin{align*}
m = [r_1, r_2, \cdots, r_s] \mid \text{ord}\sigma.
\end{align*}
所以
\begin{align*}
\text{ord}\sigma = [r_1, r_2, \cdots, r_s].
\end{align*}

{\color{blue}证法二:}
对因子个数$k$用数学归纳法证明. 当$k = 1$时,$a = (i_1 i_2 \cdots i_{r_1})$是一个轮换. 对任何$s(1 \leqslant s \leqslant r_1)$有
\begin{align*}
a^s(j) &= j, \quad j \neq i_1, i_2, \cdots, i_{r_1},
\end{align*}
而
\begin{align*}
a^s(i_j) = \begin{cases}
i_{s+j}, & j + s \leqslant r_1, \\
i_{s+j - r_1}, & j + s > r_1,
\end{cases}
\end{align*}
于是当$s < r_1$时,$a^s \neq \text{id}$,而当$s = r_1$时,$a^{r_1} = \text{id}$,故$a$的阶为$r_1$.由此可知$\sigma_i$的阶为$r_i.$

设$k - 1(k \geqslant 2)$时定理成立. 设$a = \sigma_1 \sigma_2 \cdots \sigma_k$,令
\begin{align*}
a_1 = \sigma_2 \sigma_3 \cdots \sigma_k,
\end{align*}
于是由归纳假设知$a_1$的阶为$[r_2, r_3, \cdots, r_k]$. 因为$\sigma_1$与$\sigma_j(j=2,\cdots,n)$不相交,所以可设$\sigma_2, \sigma_3, \cdots, \sigma_k$中包含的文字(作用的对象)为$\{i_{r_1+1}, i_{r_1+2}, \cdots, i_t\}$,$\sigma_1$中的文字(作用的对象)为$\{i_1, i_2, \cdots, i_{r_1}\}$. 

若$j \neq i_l(1 \leqslant l \leqslant t)$,则$\sigma_1(j) = a_1(j) = j$,故$\sigma_1 a_1(j) = a_1 \sigma_1(j) = j$. 

若$j = i_l$且$1 \leqslant l \leqslant r_1$,则$a_1(j) = j$,$\sigma_1(j) = i_{l'}$,$l' \leqslant r_1$,因而$a_1 \sigma_1(j) = i_{l'} = \sigma_1 a_1(j)$.

若$j = i_l$且$t \geqslant l \geqslant r_1 + 1$,则$\sigma_1(i_l) = i_l$,$a_1(i_l) = i_{l'}(t \geqslant l' \geqslant r_1 + 1)$,故有$a_1 \sigma_1(j) = i_{l'} = \sigma_1 a_1(j)$. 

总之有$a_1 \sigma_1 = \sigma_1 a_1$. 

又设$\beta \in \langle \sigma_1 \rangle \cap \langle a_1 \rangle$.由\refthe{theorem:生成子群的元素的形式}知$\beta= f_1f_2\cdots f_m,\text{其中}f_i\in \left\{ \sigma _1,\sigma _{1}^{-1} \right\} \cap \left\{ a_1,a_{1}^{-1} \right\} ,m\in \mathbb{N}$. 

若$j \neq i_l(1 \leqslant l \leqslant t)$,则$\beta(j) = j$. 

若$j = i_l(1 \leqslant l \leqslant r_1)$,由$\beta \in \langle a_1 \rangle$,则$\beta(j) = j$. 若$j = i_l(t \geqslant l \geqslant r_1 + 1)$,由$\beta \in \langle \sigma_1 \rangle$,则$\beta(j) = j$.

故$\beta = \text{id}$,即有$\langle \sigma_1 \rangle \cap \langle a_1 \rangle = \{\text{id}\}$.

设$m$为$a=a_1\sigma_1$的阶,则再由$a_1\sigma_1=\sigma_1a_1$可得
\begin{align*}
a^m=a_1^m\sigma_1^m=\sigma_1^m a_1^m=\mathrm{id}.
\end{align*}
因此$\sigma_1^m=a_1^{-m}\in \langle \sigma_1\rangle \cap \langle a_1\rangle$。又由$\langle \sigma_1\rangle \cap \langle a_1\rangle =\{\mathrm{id}\}$知$\sigma_1^m=a_1^{-m}=\mathrm{id}$,从而$m$是$\sigma_1,a_1$的阶的公倍数,即$m\mid r_1$,$m\mid [r_2,\cdots,r_k]$。
再设$n$也是$r_1$,$[r_2,\cdots,r_k]$的公倍数,则
\begin{align*}
\sigma_1^n=a_1^n=\mathrm{id}\Longrightarrow a^n=\sigma_1^n a_1^n=\mathrm{id}.
\end{align*}
故$m\mid n$。
因而$a = \sigma_1 a_1$的阶为$[r_1, [r_2, \cdots, r_k]] = [r_1, r_2, \cdots, r_k]$.

\end{proof}

\begin{theorem}\label{theorem:置换必可写成对换之积}
\begin{enumerate}[(1)]
\item\label{theorem:置换必可写成对换之积-2} 任意$n$阶置换$a \in S_n$一定可写成互不相交的轮换之积.即存在互不相交的轮换$\sigma_1,\sigma_2,\cdots,\sigma_r$,使得
\begin{align}\label{eq:::adhjounweiunfiosogueaiojiesfj}
a = \sigma_1\sigma_2\cdots\sigma_r.
\end{align}

\item\label{theorem:置换必可写成对换之积--1} 设$\sigma$为一个$n$阶置换,由\rrefthe{theorem:置换必可写成对换之积}{theorem:置换必可写成对换之积-2}可设$\sigma$可表为不相交轮换(包括1轮换)的乘积
\begin{align*}
\sigma = \sigma_1\sigma_2\cdots\sigma_s,
\end{align*}
在集合$X = \{1, 2, \cdots, n\}$中,规定关系“$\sim$”:
\begin{align*}
k \sim l \iff 存在\ r \in \mathbb{Z}, 使\ \sigma^r(k) = l.
\end{align*}
\begin{enumerate}[(i)]
\item\label{theorem:置换必可写成对换之积--1-1} 证明: $\sim$ 是$X$的一个等价关系;

\item\label{theorem:置换必可写成对换之积--1-2} 证明: $k \sim l$的充分必要条件是$k$与$l$属于$\sigma$的同一个轮换.进而$X$的所有等价类为
\begin{align*}
\{k\in \mathbb{N}\mid k\in \sigma_i\},\quad i=1,2,\cdots,s.
\end{align*}

\end{enumerate}

\item\label{theorem:置换必可写成对换之积-0} 如果不考虑因子的次序和乘积中1轮换的个数,则分解式\eqref{eq:::adhjounweiunfiosogueaiojiesfj}是唯一的.
\end{enumerate}
\end{theorem}
\begin{remark}
在\ref{theorem:置换必可写成对换之积--1}定义的等价关系下,对于置换
\begin{align*}
\sigma = \begin{pmatrix} 1 & 2 & 3 & 4 & 5 & 6 & 7 & 8 & 9 & 10 \\ 3 & 2 & 6 & 8 & 9 & 1 & 7 & 10 & 4 & 5 \end{pmatrix},
\end{align*}
由于
\begin{align*}
\sigma = (1\ 3\ 6)(4\ 8\ 10\ 5\ 9)(2)(7),
\end{align*}
所以集合$X$的所有等价类为
\begin{align*}
[2] = \{2\},\quad [7] = \{7\},\quad [1] = \{1, 3, 6\},\quad [4] = \{4, 5, 8, 9, 10\}.
\end{align*}
\end{remark}
\begin{proof}
\begin{enumerate}[(1)]
\item {\color{blue}证法一:}对$X$的元素个数$n$用数学归纳法.
当$n=1$时,1阶置换只有$a=(1)$,已经是轮换,因此结论对$n=1$成立.
假定结论对$n-1$成立,考察$n$阶置换
\begin{align*}
a = \begin{pmatrix} 1 & 2 & \cdots & n-1 & n \\ i_1 & i_2 & \cdots & i_{n-1} & i_n \end{pmatrix}.
\end{align*}

\begin{enumerate}[(a)]
\item 如果$i_n = n$,即
\begin{align*}
a = \begin{pmatrix} 1 & 2 & \cdots & n-1 & n \\ i_1 & i_2 & \cdots & i_{n-1} & n \end{pmatrix}.
\end{align*}
令
\begin{align*}
a_1 = \begin{pmatrix} 1 & 2 & \cdots & n-1 \\ i_1 & i_2 & \cdots & i_{n-1} \end{pmatrix},
\end{align*}
则$a_1$是一个$n-1$阶置换.由归纳假设,$a_1$可表为一些不相交轮换的乘积
\begin{align*}
a_1 = \sigma_1\sigma_2\cdots\sigma_s.
\end{align*}
将$\sigma_i$看作$n$阶置换,即得
\begin{align*}
a = \sigma_1\sigma_2\cdots\sigma_s \cdot (n) = \sigma_1\sigma_2\cdots\sigma_s.
\end{align*}

\item 如果$i_n \neq n$,则有某个$k(1\leqslant k \leqslant n-1)$,使得$i_k = n$.令
\begin{align*}
\beta = (i_k\ i_n)a = \begin{pmatrix} 1 & 2 & \cdots & k-1 & k & k+1 & \cdots & n-1 & n \\ i_1 & i_2 & \cdots & i_{k-1} & i_n & i_{k+1} & \cdots & i_{n-1} & n \end{pmatrix}.
\end{align*}
由(a)所证,$\beta$可表为一些不相交轮换的乘积.设
\begin{align*}
\beta = a_1a_2\cdots a_s,
\end{align*}
其中,$a_1,a_2,\cdots,a_s$为互不相交的轮换,则
\begin{align*}
a = (i_k\ i_n)a_1a_2\cdots a_s.
\end{align*}
如果每个$a_i$都不与$(i_k\ i_n)$相交,则
\begin{align*}
a = (i_k\ i_n)a_1a_2\cdots a_s
\end{align*}
为不相交轮换的乘积.如果有某个$a_i$与$(i_k\ i_n)$相交,则至多有一个$a_i$与$(i_k\ i_n)$相交.不妨设$a_1=(i_n\ a\ \cdots\ b)$,则
\begin{align*}
a &= (i_k\ i_n)(i_n\ a\ \cdots\ b)a_2a_3\cdots a_s \\
&= (i_k\ i_n\ a\ \cdots\ b)a_2a_3\cdots a_s
\end{align*}
为不相交轮换的乘积.从而由归纳法知结论成立.
\end{enumerate}

{\color{blue}证法二:}设$a \in S_n$,令$\bar{F}_a = \{j \mid a(j) \neq j\}$. 显然有
\begin{align}
\bar{F}_{\text{id}} = \varnothing. \label{eq:::09gjhy45he4t2fbety4swygew3tygwef23g2.5.3}
\end{align}
当$a \neq \text{id}$时,
\begin{align}
|\bar{F}_a| \geqslant 2 \label{eq:::09gjhy45he4t2fbety4swygew3tygwef23g2.5.4}
\end{align}
当且仅当$a$为对换时,式\eqref{eq:::09gjhy45he4t2fbety4swygew3tygwef23g2.5.4}中等号成立. 下面不妨设$a \neq \text{id}$. 证明存在轮换$\sigma_1$满足
\begin{align}
\begin{cases}
\bar{F}_a = \bar{F}_{\sigma_1} \cup \bar{F}_{\sigma_1^{-1} a}, \\
\bar{F}_{\sigma_1} \cap \bar{F}_{\sigma_1^{-1} a} = \varnothing.
\end{cases} \label{eq:::09gjhy45he4t2fbety4swygew3tygwef23g2.5.5}
\end{align}
因$a \neq \text{id}$,故由式\eqref{eq:::09gjhy45he4t2fbety4swygew3tygwef23g2.5.4}知有$i_1 \in \bar{F}_a$. 令
$$i_2 = a(i_1), \quad i_3 = a(i_2), \quad \cdots, \quad i_k = a(i_{k-1}),$$
则$i_1\neq i_2.$
由于$\bar{F}_a$是有限集,故存在$r\geqslant 3$,使得$i_1, i_2, \cdots, i_{r-1}$互不相同,而$i_r = i_t(1 \leqslant t \leqslant r - 1)$. 现证$t = 1$. 若不然,则有
\begin{align*}
a(i_{t-1}) = i_t = i_r = a(i_{r-1}).
\end{align*}
于是
\begin{align*}
i_{t-1} = i_{r-1},
\end{align*}
即有$t = r$,矛盾,故$t = 1$. 令$\sigma_1 = (i_1 i_2 \cdots i_{r-1})$,显然
\begin{align*}
\sigma_1(i_k) = a(i_k),\ 1 \leqslant k \leqslant r-1, \quad \bar{F}_{\sigma_1} = \{i_1, i_2, \cdots, i_{r-1}\} \subseteq \bar{F}_a.
\end{align*}
再令$a_1 = \sigma_1^{-1} a$,若$l \notin \bar{F}_a$,则$l \notin \bar{F}_{\sigma_1^{-1}}$,故$a_1(l) = l(l \notin \bar{F}_{a_1})$,因而$\bar{F}_{a_1} \subseteq \bar{F}_a$. 于是$\bar{F}_{a_1} \cup \bar{F}_{\sigma_1} \subseteq \bar{F}_a$. 反之,若$l \notin \bar{F}_{a_1} \cup \bar{F}_{\sigma_1}$,则有$a_1(l) = \sigma_1(l) = l$,故$a(l) =a_1\sigma_1^{-1}(l)= l$,即$l \notin \bar{F}_a$. 于是式\eqref{eq:::09gjhy45he4t2fbety4swygew3tygwef23g2.5.5}中第一个等式成立.

设$i_k \in \bar{F}_{\sigma_1}$,则有$a_1(i_k) = \sigma_1^{-1} a(i_k) = \sigma_1^{-1} \sigma_1(i_k) = i_k$,即$i_k \notin \bar{F}_{a_1}= \bar{F}_{\sigma_1^{-1}a}$. 故\eqref{eq:::09gjhy45he4t2fbety4swygew3tygwef23g2.5.5}式中第二个等式也成立. 

若$a \neq \sigma_1$,则$\bar{F}_{\sigma_1^{-1} a} \neq \bar{F}_{\text{id}}= \varnothing$. 从而$\bar{F}_{\sigma_1^{-1} a} \neq \bar{F}_{a}$,否则由\eqref{eq:::09gjhy45he4t2fbety4swygew3tygwef23g2.5.5}式知$\bar{F}_{\sigma _1}=\varnothing $,即$\sigma_1=\text{id}$,这与$i_1,i_2,\cdots,i_{r-1}$互不相同矛盾!
再对$\sigma_1^{-1} a$用上述方法同理可得另一轮换$\sigma_2=(j_1j_2\cdots j_{l-1})$,使得
\begin{align}
\bar{F}_{\sigma _2}=\{j_1,j_2,\cdots, j_{l-1}\}\subseteq \bar{F}_{\sigma _{1}^{-1}a},\label{eq::9h2344ttf3g4w5yehbyjkym}
\end{align}
并且
\begin{align*}
\begin{cases}
\bar{F}_{\sigma _{1}^{-1}a}=\bar{F}_{\sigma _2}\cup \bar{F}_{\sigma _{2}^{-1}\sigma _{1}^{-1}a},\\
\bar{F}_{\sigma _2}\cap \bar{F}_{\sigma _{2}^{-1}\sigma _{1}^{-1}a}=\varnothing .\\
\end{cases}
\end{align*}
若$a\ne \sigma_1\sigma_2$,则同理有$\bar{F}_{\sigma_2^{-1} \sigma_1^{-1} a} \neq \bar{F}_{\sigma_1^{-1} a}$.
从而$\bar{F}_{\sigma_2^{-1} \sigma_1^{-1} a} \subset \bar{F}_{\sigma_1^{-1} a} \subset \bar{F}_a$. 由\eqref{eq:::09gjhy45he4t2fbety4swygew3tygwef23g2.5.5}式和\eqref{eq::9h2344ttf3g4w5yehbyjkym}式知$$\{i_1, i_2, \cdots, i_{r-1}\}\cap \{j_1,j_2,\cdots, j_{l-1}\}=\bar{F}_{\sigma_1}\cap \bar{F}_{\sigma _2}=\varnothing,$$
故$\sigma_1$与$\sigma_2$为不相交的轮换.
继续做下去.由于$\bar{F}_a$是有限的,最后有$s\in \mathbb{N}$,使得互不相交的轮换$\sigma_1,\sigma_2,\cdots,\sigma_s$满足
\begin{align*}
\bar{F}_{\sigma_s^{-1} \sigma_{s-1}^{-1} \cdots \sigma_1^{-1} a} = \varnothing,
\end{align*}
即$\sigma_s^{-1} \sigma_{s-1}^{-1} \cdots \sigma_1^{-1} a = \text{id}$,因而
\begin{align*}
a = \sigma_1 \sigma_2 \cdots \sigma_s,
\end{align*}
即$S_n$中任何元素可表为互不相交的轮换之积.

\item \begin{enumerate}[(i)]
\item 设$\text{ord}\,\sigma = m$,对$\forall j,k,l \in X$.

因为$\sigma^m(j) = (1)j = j$,所以$j \sim j$,于是$\sim$具有反身性;

如果$k \sim l$,则存在$r \in \mathbb{Z}$,使$\sigma^r(k) = l$,于是$\sigma^{-r}(l) = k$,从而$l \sim k$,这说明$\sim$具有对称性;

如果$j \sim k$, $k \sim l$,则存在$r_1,r_2 \in \mathbb{Z}$,使$\sigma^{r_1}(j) = k$, $\sigma^{r_2}(k) = l$,于是$\sigma^{r_1+r_2}(j) = l$,从而$j \sim l$,这说明$\sim$具有传递性.

这就证明了$\sim$是$X$的一个等价关系.

\item {\heiti 必要性:} 设$k \sim l$,则存在$r \in \mathbb{Z}$,使$\sigma^r(k) = l$.设$k$属于轮换$\sigma_i$,则
\begin{align*}
l = \sigma^r(k) = (\sigma_1\sigma_2\cdots\sigma_s)^r(k) 
= \sigma_1^r\sigma_2^r\cdots\sigma_s^r(k) 
= \sigma_i^r(k),
\end{align*}
即$l$也属于轮换$\sigma_i$,从而$k$与$l$属于$\sigma$的同一个轮换.

{\heiti 充分性:} 如果$k$与$l$属于$\sigma$的同一个轮换$\sigma_i$,则必有$r \in \mathbb{Z}$,使$\sigma_i^r(k) = l$,从而
\begin{align*}
l = \sigma_i^r(k) = \sigma_1^r\sigma_2^r\cdots\sigma_s^r(k) 
= (\sigma_1\sigma_2\cdots\sigma_s)^r(k) 
= \sigma^r(k),
\end{align*}
所以$k \sim l$.
\end{enumerate}

\item 设$\sigma$为任一$n$阶置换,
\begin{align*}
\sigma = \sigma_1\sigma_2\cdots\sigma_s 
= \delta_1\delta_2\cdots\delta_t.
\end{align*}
是$\sigma$的两个表为不相交轮换(包括1轮换)的乘积的分解式,则$\sigma$在集合$X = \{1,2,\cdots,n\}$上规定关系“$\sim$”
\begin{align*}
k \sim l \iff 存在\ r \in \mathbb{Z}, 使\ \sigma^r(k) = l.
\end{align*}
由\rrrefthe{theorem:置换必可写成对换之积}{theorem:置换必可写成对换之积--1}{theorem:置换必可写成对换之积--1-2}知,在此等价关系之下,集合$X$的等价类的个数等于$\sigma$分解为不相交轮换(包括1轮换)的乘积的因子数,所以$s = t$.又由\rrrefthe{theorem:置换必可写成对换之积}{theorem:置换必可写成对换之积--1}{theorem:置换必可写成对换之积--1-2}知,$X$的一个等价类由属于$\sigma$的同一个轮换中的元素组成.因此,适当交换因子的次序,可使$\sigma_i$与$\delta_i$含有$X$的同一个等价类$X_i$中的元素.从而,对任意的$x \in X$,如果$x \notin X_i$,则有
\begin{align*}
\sigma_i(x) = x = \delta_i(x).
\end{align*}
如果$x \in X_i$,也有
\begin{align*}
\sigma_i(x) = \sigma_1\sigma_2\cdots\sigma_s(x) 
= \sigma(x) 
= \delta_1\delta_2\cdots\delta_s(x) 
= \delta_i(x).
\end{align*}
因此$\sigma_i = \delta_i(i = 1,2,\cdots,s)$.这就证明了分解的唯一性.
\end{enumerate}

\end{proof}

\begin{corollary}\label{theorem:抽象代数--置换必可写成对换之积}
\begin{enumerate}[(1)]
\item 任何置换都可写成对换之积.并且令$S = \{(1i) \mid 2 \leqslant i \leqslant n\}$,则有$S_n = \langle S \rangle$.

\item 将一个置换写成对换的乘积,所用对换个数的奇偶性是唯一的.
\end{enumerate}
\end{corollary}
\begin{remark}
由\rrefthe{theorem:置换必可写成对换之积}{theorem:置换必可写成对换之积-2}和\rrefpro{proposition:对换与轮换的基本性质}{proposition:对换与轮换的基本性质-6}知,任何置换至少可分解成两个不同的对换之积(这里(1)中只证明了其中一种).
\end{remark}
\begin{proof}
\begin{enumerate}[(1)]
\item 由\rrefthe{theorem:置换必可写成对换之积}{theorem:置换必可写成对换之积-2}知任何置换可分解为不相交的轮换之积,又由\rrefpro{proposition:对换与轮换的基本性质}{proposition:对换与轮换的基本性质-6}知任何轮换可分解为对换之积,故任何置换都可分解为对换之积.

事实上,
\begin{align}
(ij) = (1i)(1j)(1i). \label{eq:::09gjhy45he4t2fbety4swygew3tygwef23g2.5.1}
\end{align}
由\rrefthe{theorem:置换必可写成对换之积}{theorem:置换必可写成对换之积-2}知$\forall a \in S_n$一定可写成轮换之积,从而由\rrefpro{proposition:对换与轮换的基本性质}{proposition:对换与轮换的基本性质-6}知$a$可写成对换之积.再利用\eqref{eq:::09gjhy45he4t2fbety4swygew3tygwef23g2.5.1}式知$a$可写成$S$中元素之积,再由\refthe{theorem:生成子群的元素的形式}可知$a\in \langle S\rangle$,即$\langle S\rangle\supseteq S_n$.又显然有$\langle S\rangle\subseteq S_n,$故$\langle S\rangle=S_n.$

\item 设$\sigma$为任一$n$阶置换,并设$\sigma$已表为$s$个不相交轮换(包括1轮换)之积:$\sigma = \tau_1\tau_2\cdots\tau_s$.
定义
\begin{align*}
\mathcal{N}(\sigma) = (-1)^{n-s}.
\end{align*}
显然$\mathcal{N}(\sigma)$由$\sigma$唯一确定.
设$(ab)$为任一对换,考察乘积$(ab)\sigma.$下证
\begin{align}\label{eq:i89h382jf90w3023234afds}
\mathcal{N}((ab)\sigma)=-\mathcal{N}(\sigma).
\end{align}
如果$a,b$处于$\sigma$的同一个轮换
\begin{align*}
\tau_1 = (a c_1 c_2 \cdots c_k b d_1 d_2 \cdots d_h)
\end{align*}
中,则由\rrefpro{proposition:对换与轮换的基本性质}{proposition:对换与轮换的基本性质-5}知
\begin{align*}
(ab)\sigma = (a c_1 c_2 \cdots c_k)(b d_1 d_2 \cdots d_h)\tau_2\tau_3 \cdots \tau_s.
\end{align*}
从而
\begin{align*}
\mathcal{N}((ab)\sigma) = (-1)^{n-s-1} = -\mathcal{N}(\sigma).
\end{align*}
如果$a,b$分别处于$\sigma$的两个不同轮换
\begin{align*}
\tau_1 = (a c_1 c_2 \cdots c_k),\quad \tau_2 = (b d_1 d_2 \cdots d_h)
\end{align*}
中,则由\rrefpro{proposition:对换与轮换的基本性质}{proposition:对换与轮换的基本性质-4}知
\begin{align*}
(ab)\sigma = (a c_1 c_2 \cdots c_k b d_1 d_2 \cdots d_h)\tau_3\tau_4 \cdots \tau_s.
\end{align*}
从而
\begin{align*}
\mathcal{N}((ab)\sigma) = (-1)^{n-s+1} = -\mathcal{N}(\sigma).
\end{align*}
故\eqref{eq:i89h382jf90w3023234afds}式成立.

设$\sigma$可分别表示为$h$个对换和$k$个对换的乘积
\begin{align*}
\sigma = (a_1 b_1)(a_2 b_2)\cdots(a_h b_h) 
= (c_1 d_1)(c_2 d_2)\cdots(c_k d_k),
\end{align*}
则由\eqref{eq:i89h382jf90w3023234afds}式可得
\begin{align*}
\mathcal{N}(\sigma) = \mathcal{N}(\sigma \cdot (1)) 
= \mathcal{N}((a_1 b_1)(a_2 b_2)\cdots(a_h b_h) \cdot (1)) 
= (-1)^h \mathcal{N}((1)) = (-1)^h.
\end{align*}
由\eqref{eq:i89h382jf90w3023234afds}式同理可得
\begin{align*}
\mathcal{N}(\sigma) = (-1)^k.
\end{align*}
因此$(-1)^h = (-1)^k$,所以$h$与$k$有相同的奇偶性.
\end{enumerate}

\end{proof}

\begin{corollary}\label{corollary:奇置换与偶置换分别可表示成奇数和偶数个对换之积}
\begin{enumerate}[(1)]
\item\label{corollary:奇置换与偶置换分别可表示成奇数和偶数个对换之积-1} 对换都是奇置换.

\item\label{corollary:奇置换与偶置换分别可表示成奇数和偶数个对换之积-2} 置换是偶(奇)置换当且仅当其可表示为偶(奇)数个对换之积.

\item\label{corollary:奇置换与偶置换分别可表示成奇数和偶数个对换之积-3} 轮换是奇(偶)置换当且仅当其长度为偶(奇)数.

\item\label{corollary:奇置换与偶置换分别可表示成奇数和偶数个对换之积-4} 任何两个偶(奇)置换之积是偶置换.

\item\label{corollary:奇置换与偶置换分别可表示成奇数和偶数个对换之积-5} 一个偶置换与一个奇置换之积是奇置换.

\item\label{corollary:奇置换与偶置换分别可表示成奇数和偶数个对换之积-6} 一个偶(奇)置换的逆置换仍是一个偶(奇)置换.

\item\label{corollary:奇置换与偶置换分别可表示成奇数和偶数个对换之积-7} 置换$\sigma$与$\sigma^{-1}$具有相同的奇偶性.
\end{enumerate}
\end{corollary}
\begin{proof}
\begin{enumerate}[(1)]
\item 由\refthe{theorem:奇置换与偶置换}中奇置换定义知对换显然都是奇置换.

\item 设$\sigma \in S_n$,则由\refcor{theorem:抽象代数--置换必可写成对换之积}知$\sigma = \sigma_1\sigma_2\cdots\sigma_k$,其中$\sigma_i$都是对换。又注意到对换$\sigma_i = (ij)$都是奇置换,故$\text{sgn} \sigma_i = -1$。由\refthe{theorem:奇置换与偶置换}知$\text{sgn}$是$S_n$到$\{-1,1\}$的同态,因此
\begin{align*}
\text{sgn} \sigma = \text{sgn}(\sigma_1\sigma_2\cdots\sigma_k) = (\text{sgn} \sigma_1)(\text{sgn} \sigma_2)\cdots(\text{sgn} \sigma_k) = (-1)^k.
\end{align*}

故$\sigma$是奇置换当且仅当$\text{sgn} \sigma = (-1)^k = -1$当且仅当$k$为奇数;

$\sigma$是偶置换当且仅当$\text{sgn} \sigma = (-1)^k = 1$当且仅当$k$为偶数。

\item 设$r$轮换$(i_1i_2,\cdots i_r)$,则由\rrefpro{proposition:对换与轮换的基本性质}{proposition:对换与轮换的基本性质-6}知
\begin{align*}
(i_1i_2\cdots i_r)=(i_1i_r)(i_1i_{r-1})\cdots (i_1i_2).
\end{align*}
由\refthe{theorem:奇置换与偶置换}知$\text{sgn}$是$S_n$到$\{-1,1\}$的同态,因此
\begin{align*}
\text{sgn}\mathrm{(}i_1i_2\cdots i_r)=\text{sgn}\mathrm{(}i_1i_r)\cdot \text{sgn}\mathrm{(}i_1i_{r-1})\cdots \text{sgn}\mathrm{(}i_1i_2)=\left( -1 \right) ^{r-1}.
\end{align*}

故$(i_1i_2,\cdots i_r)$为偶置换当且仅当$\text{sgn}\mathrm{(}i_1i_2\cdots i_r) = (-1)^{r-1} = 1$当且仅当$r$是奇数;

若$(i_1i_2,\cdots i_r)$为奇置换当且仅当$\text{sgn}\mathrm{(}i_1i_2\cdots i_r) = (-1)^{r-1} = -1$当且仅当$r$是偶数;

\item 

\item 

\item 

\item 如果$\sigma$可表示为$k$个对换的乘积
\begin{align*}
\sigma = (i_1\ j_1)(i_2\ j_2)\cdots(i_k\ j_k),
\end{align*}
则
\begin{align*}
\sigma^{-1} = (i_k\ j_k)(i_{k-1}\ j_{k-1})\cdots(i_1\ j_1)
\end{align*}
也可表示为$k$个对换的乘积. 所以$\sigma$与$\sigma^{-1}$具有相同的奇偶性.
\end{enumerate}

\end{proof}

\begin{theorem}
设$S_X$是置换群,则有以下结论
\begin{enumerate}[(1)]
\item 若$S_X$中存在奇置换, 则$S_X$中奇置换的个数与偶置换的个数相同.特别地,在全体$n$阶置换$S_n$中, 奇置换与偶置换各有$\frac{n!}{2}$个,进而$|A_n|=\frac{n!}{2}.$

\item $S_X$中所有偶置换的集合$H$是$S_X$的子群.
\end{enumerate}
\end{theorem}
\begin{proof}
\begin{enumerate}[(1)]
\item 设$S_X$中有奇置换. 由于$S_X$是置换群, 所以$(1) \in S_X$, 而$(1)$为偶置换. 所以$S_X$中既有奇置换又有偶置换. 以$O$与$E$分别表示$S_X$中奇置换与偶置换的集合. 设$\sigma$为$S_X$的任一奇置换, 则由\rrefcor{corollary:奇置换与偶置换分别可表示成奇数和偶数个对换之积}{corollary:奇置换与偶置换分别可表示成奇数和偶数个对换之积-4}和\rrefcor{corollary:奇置换与偶置换分别可表示成奇数和偶数个对换之积}{corollary:奇置换与偶置换分别可表示成奇数和偶数个对换之积-5}可得
\begin{align*}
\sigma O = \{\sigma\delta \mid \delta \in O\} \subseteq E, \\
\sigma E = \{\sigma\tau \mid \tau \in E\} \subseteq O.
\end{align*}
因此
\begin{align*}
|O| = |\sigma O| \leqslant |E|, \quad |E| = |\sigma E| \leqslant |O|,
\end{align*}
由此得$|O| = |E|$. 这就证明了结论.

\item 因$(1) \in G$为偶置换, 所以$(1) \in H$, 从而$H$非空. 又由\rrefcor{corollary:奇置换与偶置换分别可表示成奇数和偶数个对换之积}{corollary:奇置换与偶置换分别可表示成奇数和偶数个对换之积-4}知两个偶置换的乘积仍是偶置换, 所以$H$关于置换的乘积封闭. 从而由\rrefpro{proposition:抽象代数---子群的基本性质}{proposition:抽象代数---子群的基本性质-1}知$H$为$G$的子群.
\end{enumerate}

\end{proof}

\begin{example}
把下列置换分别写成不相交的轮换的乘积和对换的乘积,并计算置换的奇偶性:
\begin{align*}
\begin{pmatrix}
1&2&3&4&5&6\\
1&3&6&5&2&4
\end{pmatrix}
\begin{pmatrix}
1&2&3&4&5&6\\
6&2&4&1&5&3
\end{pmatrix}.
\end{align*}
\end{example}
\begin{solution}
注意到
\begin{align*}
\begin{pmatrix}
1&2&3&4&5&6\\
1&3&6&5&2&4
\end{pmatrix}
\begin{pmatrix}
1&2&3&4&5&6\\
6&2&4&1&5&3
\end{pmatrix}
=
\begin{pmatrix}
1&2&3&4&5&6\\
6&4&3&5&2&1
\end{pmatrix}
=
(1\ 6)(2\ 4\ 5).
\end{align*}
再由\rrefpro{proposition:对换与轮换的基本性质}{proposition:对换与轮换的基本性质-6}知
\begin{align*}
(1\ 6)(2\ 4\ 5)=(1\ 6)(2\ 4)(4\ 5)\quad \text{or}\quad (1\ 6)(2\ 5)(2\ 4).
\end{align*}
故由\rrefcor{corollary:奇置换与偶置换分别可表示成奇数和偶数个对换之积}{corollary:奇置换与偶置换分别可表示成奇数和偶数个对换之积-2}知这个置换是奇置换.

\end{solution}





\end{document}
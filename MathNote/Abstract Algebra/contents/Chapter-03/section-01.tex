\documentclass[../../main.tex]{subfiles}
\graphicspath{{\subfix{../../image/}}} % 指定图片目录,后续可以直接使用图片文件名。

% 例如:
% \begin{figure}[H]
% \centering
% \includegraphics[scale=0.4]{图.png}
% \caption{}
% \label{figure:图}
% \end{figure}
% 注意:上述\label{}一定要放在\caption{}之后,否则引用图片序号会只会显示??.

\begin{document}

\section{群的生成组}

\begin{definition}
设$S$是群$G$的非空子集,以$\langle S\rangle$表示$G$的包含$S$的最小子群,即$S$\textbf{生成的子群}.
显然,$\langle S\rangle$是$G$中所有包含$S$的子群之交,即$S=\bigcap_{S<H}{H}.$
\end{definition}
\begin{note}
由\rrefpro{proposition:抽象代数---子群的基本性质}{proposition:抽象代数---子群的基本性质-2}知$S=\bigcap_{S<H}{H}$是一个群,故上述定义是良定义的.
\end{note}

\begin{theorem}\label{theorem:生成子群的元素的形式}
设$S$是群$G$的非空子集,则
\begin{align*}
\langle S\rangle = \{x_1x_2\cdots x_m \mid x_i \in S \cup S^{-1}, 1 \leqslant i \leqslant m, m \in \mathbf{N}\}.
\end{align*}
\end{theorem}
\begin{proof}
令$\bar{S} = \{x_1x_2\cdots x_m \mid x_i \in S \cup S^{-1}, 1 \leqslant i \leqslant m, m \in \mathbf{N}\}$. 由$\langle S\rangle$为子群且$S \subseteq \langle S\rangle$知$S^{-1} \subseteq \langle S\rangle$,因而$S \subseteq \bar{S} \subseteq \langle S\rangle$. 又$\langle S\rangle$是含$S$的最小子群,故只需证明$\bar{S}$为子群,则$\bar{S} \supseteq \langle S\rangle$.

设$x_1x_2\cdots x_m \in \bar{S}$,$y_1y_2\cdots y_n \in \bar{S}$,于是$y_i^{-1} \in S \cup S^{-1}(1 \leqslant i \leqslant m)$,则有
\begin{align*}
(x_1x_2\cdots x_m)(y_1y_2\cdots y_n)^{-1} = x_1x_2\cdots x_m y_n^{-1} y_{n-1}^{-1} \cdots y_2^{-1} y_1^{-1} \in \bar{S},
\end{align*}
因而$\bar{S}$为$G$的子群,故$\bar{S} = \langle S\rangle$.

\end{proof}

\begin{definition}
若$S$为群$G$的子集且$G = \langle S\rangle$,则称$S$为$G$的\textbf{生成组}. 若$G$有一个含有限个元素的生成组,则称$G$是\textbf{有限生成的}.

若$G = \langle a\rangle$为循环群,则$a$本身就是生成组,这时称$a$为$G$的\textbf{生成元}.
\end{definition}

\begin{example}
设$G = S_3$,又$a = \begin{pmatrix} 1 & 2 & 3 \\ 2 & 1 & 3 \end{pmatrix}$,$b = \begin{pmatrix} 1 & 2 & 3 \\ 3 & 2 & 1 \end{pmatrix}$,则$S_3 = \langle \{a, b\} \rangle$.
\end{example}
\begin{proof}
事实上,设$G_1 = \langle a\rangle$,注意到
\begin{align*}
a^{-1}=\begin{pmatrix}1&2&3\\1&2&3\end{pmatrix},a^2=(a^{-1})^2=\begin{pmatrix}1&2&3\\1&2&3\end{pmatrix},
\end{align*}
故由\refthe{theorem:生成子群的元素的形式}知$G_1=\{a,a^{-1}\}$。从而$G_1$为$S_3$的2阶子群且$b \notin G_1$,于是$G_1 \subset \langle \{a, b\} \rangle$. 设$\langle \{a, b\} \rangle$的阶为$n$,则由\hyperref[theorem:抽象代数-Lagrange定理-定理 1.3.3]{Lagrange定理}知$2\mid n$且$2 < n$.又因为$\langle \{a, b\} \rangle$是$G$的子群,所以
由\hyperref[theorem:抽象代数-Lagrange定理-定理 1.3.3]{Lagrange定理}知$n\mid 6$.因而有$n = 6$,由此知$S_3 = \langle \{a, b\} \rangle$.

\end{proof}

\begin{definition}
设集合$\{i_1, i_2, \cdots, i_r\}$为集合$\{1, 2, \cdots, n\}$的子集. 若$\sigma \in S_n$满足
\begin{align*}
\sigma(i_j) &= i_{j+1}, \quad 1 \leqslant j \leqslant r-1, \\
\sigma(i_r) &= i_1, \\
\sigma(k) &= k, \quad k \notin \{i_1, i_2, \cdots, i_r\},
\end{align*}
则称$\sigma$为一个\textbf{长为$r$的轮换}或\textbf{$r$轮换},这时记$\sigma = (i_1 i_2 \cdots i_r)$.
特别地,将2轮换$(ij)$称为\textbf{对换}.

若$\sigma = (i_1 i_2 \cdots i_r)$与$\tau = (j_1 j_2 \cdots j_s)$是两个轮换且
$$\{i_1, i_2, \cdots, i_r\} \bigcap \{j_1, j_2, \cdots, j_s\} = \varnothing,$$
则称$\sigma$与$\tau$为\textbf{不相交的轮换}.

显然,一个$r$轮换$(i_1 i_2 \cdots i_r)$有$r$种不同的表示,
\begin{align*}
(i_1 i_2 \cdots i_r) = (i_2 i_3 \cdots i_r i_1) = \cdots = (i_r i_1 \cdots i_{r-1}).
\end{align*}
\end{definition}

\begin{proposition}\label{proposition:对换乘积的逆}
设$\sigma \in S_n$且$\sigma = (i_1i_1')(i_2i_2')\cdots (i_ri_r')$,则$\sigma^{-1}=(i_ri_r')(i_{r-1}i_{r-1}')\cdots (i_1i_1')$.
\end{proposition}
\begin{proof}


\end{proof}

\begin{theorem}
设$a \in S_n$且$a = \sigma_1 \sigma_2 \cdots \sigma_k$,其中,$\sigma_i$为$r_i$轮换:当$i \neq j$时,$\sigma_i$与$\sigma_j$不相交,$1 \leqslant i, j \leqslant k$,则$a$的阶为$r_1, r_2, \cdots, r_k$的最小公倍数$[r_1, r_2, \cdots, r_k]$.进而$\sigma_i$的阶为$r_i$.
\end{theorem}
\begin{proof}
对因子个数$k$用数学归纳法证明. 当$k = 1$时,$a = (i_1 i_2 \cdots i_{r_1})$是一个轮换. 对任何$s(1 \leqslant s \leqslant r_1)$有
\begin{align*}
a^s(j) &= j, \quad j \neq i_1, i_2, \cdots, i_{r_1},
\end{align*}
而
\begin{align*}
a^s(i_j) = \begin{cases}
i_{s+j}, & j + s \leqslant r_1, \\
i_{s+j - r_1}, & j + s > r_1,
\end{cases}
\end{align*}
于是当$s < r_1$时,$a^s \neq \text{id}$,而当$s = r_1$时,$a^{r_1} = \text{id}$,故$a$的阶为$r_1$.由此可知$\sigma_i$的阶为$r_i.$

设$k - 1(k \geqslant 2)$时定理成立. 设$a = \sigma_1 \sigma_2 \cdots \sigma_k$,令
\begin{align*}
a_1 = \sigma_2 \sigma_3 \cdots \sigma_k,
\end{align*}
于是由归纳假设知$a_1$的阶为$[r_2, r_3, \cdots, r_k]$. 因为$\sigma_1$与$\sigma_j(j=2,\cdots,n)$不相交,所以可设$\sigma_2, \sigma_3, \cdots, \sigma_k$中包含的文字(作用的对象)为$\{i_{r_1+1}, i_{r_1+2}, \cdots, i_t\}$,$\sigma_1$中的文字(作用的对象)为$\{i_1, i_2, \cdots, i_{r_1}\}$. 

若$j \neq i_l(1 \leqslant l \leqslant t)$,则$\sigma_1(j) = a_1(j) = j$,故$\sigma_1 a_1(j) = a_1 \sigma_1(j) = j$. 

若$j = i_l$且$1 \leqslant l \leqslant r_1$,则$a_1(j) = j$,$\sigma_1(j) = i_{l'}$,$l' \leqslant r_1$,因而$a_1 \sigma_1(j) = i_{l'} = \sigma_1 a_1(j)$.

若$j = i_l$且$t \geqslant l \geqslant r_1 + 1$,则$\sigma_1(i_l) = i_l$,$a_1(i_l) = i_{l'}(t \geqslant l' \geqslant r_1 + 1)$,故有$a_1 \sigma_1(j) = i_{l'} = \sigma_1 a_1(j)$. 

总之有$a_1 \sigma_1 = \sigma_1 a_1$. 

又设$\beta \in \langle \sigma_1 \rangle \cap \langle a_1 \rangle$.由\refthe{theorem:生成子群的元素的形式}知$\beta= f_1f_2\cdots f_m,\text{其中}f_i\in \left\{ \sigma _1,\sigma _{1}^{-1} \right\} \cap \left\{ a_1,a_{1}^{-1} \right\} ,m\in \mathbf{N}$. 

若$j \neq i_l(1 \leqslant l \leqslant t)$,则$\beta(j) = j$. 

若$j = i_l(1 \leqslant l \leqslant r_1)$,由$\beta \in \langle a_1 \rangle$,则$\beta(j) = j$. 若$j = i_l(t \geqslant l \geqslant r_1 + 1)$,由$\beta \in \langle \sigma_1 \rangle$,则$\beta(j) = j$.

故$\beta = \text{id}$,即有$\langle \sigma_1 \rangle \cap \langle a_1 \rangle = \{\text{id}\}$.

设$m$为$a=a_1\sigma_1$的阶,则再由$a_1\sigma_1=\sigma_1a_1$可得
\begin{align*}
a^m=a_1^m\sigma_1^m=\sigma_1^m a_1^m=\mathrm{id}.
\end{align*}
因此$\sigma_1^m=a_1^{-m}\in \langle \sigma_1\rangle \cap \langle a_1\rangle$。又由$\langle \sigma_1\rangle \cap \langle a_1\rangle =\{\mathrm{id}\}$知$\sigma_1^m=a_1^{-m}=\mathrm{id}$,从而$m$是$\sigma_1,a_1$的阶的公倍数,即$m\mid r_1$,$m\mid [r_2,\cdots,r_k]$。
再设$n$也是$r_1$,$[r_2,\cdots,r_k]$的公倍数,则
\begin{align*}
\sigma_1^n=a_1^n=\mathrm{id}\Longrightarrow a^n=\sigma_1^n a_1^n=\mathrm{id}.
\end{align*}
故$m\mid n$。
因而$a = \sigma_1 a_1$的阶为$[r_1, [r_2, \cdots, r_k]] = [r_1, r_2, \cdots, r_k]$.

\end{proof}

\begin{theorem}\label{theorem:置换必可写成对换之积}
\begin{enumerate}[(1)]
\item\label{theorem:置换必可写成对换之积-1} 任何轮换$(i_1 i_2 \cdots i_r)$可写成如下对换之积
\begin{align*}
(i_1 i_2 \cdots i_r) = (i_1 i_r)(i_1 i_{r-1}) \cdots (i_1 i_2).
\end{align*}

\item\label{theorem:置换必可写成对换之积-2} 若把$S_n$中的幺元$\text{id}$记为长为1的轮换,即$\text{id} = (i)$,则$\forall a \in S_n$,一定可写成互不相交的轮换之积.

\item\label{theorem:置换必可写成对换之积-3} 令$S = \{(1i) \mid 2 \leqslant i \leqslant n\}$,则$S_n = \langle S \rangle$.即任何置换都可写成对换之积.
\end{enumerate}
\end{theorem}
\begin{proof}
\begin{enumerate}[(1)]
\item 利用数学归纳法证明任何轮换$(i_1 i_2 \cdots i_r)$可写成如下对换之积
\begin{align}
(i_1 i_2 \cdots i_r) = (i_1 i_r)(i_1 i_{r-1}) \cdots (i_1 i_2). \label{eq:::09gjhy45he4t2fbety4swygew3tygwef23g2.5.2}
\end{align}
当$r = 2$时,\eqref{eq:::09gjhy45he4t2fbety4swygew3tygwef23g2.5.2}式显然成立. 假设定理对$r - 1(r \geqslant 3)$成立,并记$a = (i_1 i_2 \cdots i_r)$,于是有
\begin{align*}
(i_1 i_r)(i_1 i_{r-1}) \cdots (i_1 i_3)(i_1 i_2) = (i_1 i_3 \cdots i_r)(i_1 i_2) = a'.
\end{align*}

当$j \neq i_k$时,$a'(j) = j = a(j)$;

当$j = i_k(k \geqslant 3)$时,$a'(j) = (i_1 i_3 \cdots i_r)(j) = a(j)$;

当$j = i_1$时,$a'(i_1) = (i_1 i_3 \cdots i_r)(i_2) = i_2 = a(i_1)$;

当$j = i_2$时,$a'(i_2) = (i_1 i_3 \cdots i_r)(i_1) = i_3 = a(i_2)$.

综上知$a=a'.$
故知式\eqref{eq:::09gjhy45he4t2fbety4swygew3tygwef23g2.5.2}成立,故任何轮换可写成$S$中元素之积.

\item 设$a \in S_n$,令$\bar{F}_a = \{j \mid a(j) \neq j\}$. 显然有
\begin{align}
\bar{F}_{\text{id}} = \varnothing. \label{eq:::09gjhy45he4t2fbety4swygew3tygwef23g2.5.3}
\end{align}
当$a \neq \text{id}$时,
\begin{align}
|\bar{F}_a| \geqslant 2 \label{eq:::09gjhy45he4t2fbety4swygew3tygwef23g2.5.4}
\end{align}
当且仅当$a$为对换时,式\eqref{eq:::09gjhy45he4t2fbety4swygew3tygwef23g2.5.4}中等号成立. 下面不妨设$a \neq \text{id}$. 证明存在轮换$\sigma_1$满足
\begin{align}
\begin{cases}
\bar{F}_a = \bar{F}_{\sigma_1} \cup \bar{F}_{\sigma_1^{-1} a}, \\
\bar{F}_{\sigma_1} \cap \bar{F}_{\sigma_1^{-1} a} = \varnothing.
\end{cases} \label{eq:::09gjhy45he4t2fbety4swygew3tygwef23g2.5.5}
\end{align}
因$a \neq \text{id}$,故由式\eqref{eq:::09gjhy45he4t2fbety4swygew3tygwef23g2.5.4}知有$i_1 \in \bar{F}_a$. 令
$$i_2 = a(i_1), \quad i_3 = a(i_2), \quad \cdots, \quad i_k = a(i_{k-1}),$$
则$i_1\neq i_2.$
由于$\bar{F}_a$是有限集,故存在$r\geqslant 3$,使得$i_1, i_2, \cdots, i_{r-1}$互不相同,而$i_r = i_t(1 \leqslant t \leqslant r - 1)$. 现证$t = 1$. 若不然,则有
\begin{align*}
a(i_{t-1}) = i_t = i_r = a(i_{r-1}).
\end{align*}
于是
\begin{align*}
i_{t-1} = i_{r-1},
\end{align*}
即有$t = r$,矛盾,故$t = 1$. 令$\sigma_1 = (i_1 i_2 \cdots i_{r-1})$,显然
\begin{align*}
\sigma_1(i_k) = a(i_k),\ 1 \leqslant k \leqslant r-1, \quad \bar{F}_{\sigma_1} = \{i_1, i_2, \cdots, i_{r-1}\} \subseteq \bar{F}_a.
\end{align*}
再令$a_1 = \sigma_1^{-1} a$,若$l \notin \bar{F}_a$,则$l \notin \bar{F}_{\sigma_1^{-1}}$,故$a_1(l) = l(l \notin \bar{F}_{a_1})$,因而$\bar{F}_{a_1} \subseteq \bar{F}_a$. 于是$\bar{F}_{a_1} \cup \bar{F}_{\sigma_1} \subseteq \bar{F}_a$. 反之,若$l \notin \bar{F}_{a_1} \cup \bar{F}_{\sigma_1}$,则有$a_1(l) = \sigma_1(l) = l$,故$a(l) =a_1\sigma_1^{-1}(l)= l$,即$l \notin \bar{F}_a$. 于是式\eqref{eq:::09gjhy45he4t2fbety4swygew3tygwef23g2.5.5}中第一个等式成立.

设$i_k \in \bar{F}_{\sigma_1}$,则有$a_1(i_k) = \sigma_1^{-1} a(i_k) = \sigma_1^{-1} \sigma_1(i_k) = i_k$,即$i_k \notin \bar{F}_{a_1}= \bar{F}_{\sigma_1^{-1}a}$. 故\eqref{eq:::09gjhy45he4t2fbety4swygew3tygwef23g2.5.5}式中第二个等式也成立. 

若$a \neq \sigma_1$,则$\bar{F}_{\sigma_1^{-1} a} \neq \bar{F}_{\text{id}}= \varnothing$. 从而$\bar{F}_{\sigma_1^{-1} a} \neq \bar{F}_{a}$,否则由\eqref{eq:::09gjhy45he4t2fbety4swygew3tygwef23g2.5.5}式知$\bar{F}_{\sigma _1}=\varnothing $,即$\sigma_1=\text{id}$,这与$i_1,i_2,\cdots,i_{r-1}$互不相同矛盾!
再对$\sigma_1^{-1} a$用上述方法同理可得另一轮换$\sigma_2=(j_1j_2\cdots j_{l-1})$,使得
\begin{align}
\bar{F}_{\sigma _2}=\{j_1,j_2,\cdots, j_{l-1}\}\subseteq \bar{F}_{\sigma _{1}^{-1}a},\label{eq::9h2344ttf3g4w5yehbyjkym}
\end{align}
并且
\begin{align*}
\begin{cases}
\bar{F}_{\sigma _{1}^{-1}a}=\bar{F}_{\sigma _2}\cup \bar{F}_{\sigma _{2}^{-1}\sigma _{1}^{-1}a},\\
\bar{F}_{\sigma _2}\cap \bar{F}_{\sigma _{2}^{-1}\sigma _{1}^{-1}a}=\varnothing .\\
\end{cases}
\end{align*}
若$a\ne \sigma_1\sigma_2$,则同理有$\bar{F}_{\sigma_2^{-1} \sigma_1^{-1} a} \neq \bar{F}_{\sigma_1^{-1} a}$.
从而$\bar{F}_{\sigma_2^{-1} \sigma_1^{-1} a} \subset \bar{F}_{\sigma_1^{-1} a} \subset \bar{F}_a$. 由\eqref{eq:::09gjhy45he4t2fbety4swygew3tygwef23g2.5.5}式和\eqref{eq::9h2344ttf3g4w5yehbyjkym}式知$$\{i_1, i_2, \cdots, i_{r-1}\}\cap \{j_1,j_2,\cdots, j_{l-1}\}=\bar{F}_{\sigma_1}\cap \bar{F}_{\sigma _2}=\varnothing,$$
故$\sigma_1$与$\sigma_2$为不相交的轮换.
继续做下去.由于$\bar{F}_a$是有限的,最后有$n$,使得互不相交的轮换$\sigma_1,\sigma_2,\cdots,\sigma_n$满足
\begin{align*}
\bar{F}_{\sigma_n^{-1} \sigma_{n-1}^{-1} \cdots \sigma_1^{-1} a} = \varnothing,
\end{align*}
即$\sigma_n^{-1} \sigma_{n-1}^{-1} \cdots \sigma_1^{-1} a = \text{id}$,因而
\begin{align*}
a = \sigma_1 \sigma_2 \cdots \sigma_n,
\end{align*}
即$S_n$中任何元素可表为互不相交的轮换之积,故定理成立.

\item 事实上,
\begin{align}
(ij) = (1i)(1j)(1i). \label{eq:::09gjhy45he4t2fbety4swygew3tygwef23g2.5.1}
\end{align}
由结论(2)知$\forall a \in S_n$一定可写成轮换之积,从而由结论(1)知$a$可写成对换之积.再利用\eqref{eq:::09gjhy45he4t2fbety4swygew3tygwef23g2.5.1}式知$a$可写成$S$中元素之积,再由\refthe{theorem:生成子群的元素的形式}可知$a\in \langle S\rangle$,即$\langle S\rangle\supseteq S_n$.又显然有$\langle S\rangle\subseteq S_n,$故$\langle S\rangle=S_n.$
\end{enumerate}

\end{proof}

\begin{corollary}\label{corollary:奇置换与偶置换分别可表示成奇数和偶数个对换之积}
对换都是奇置换,并且奇置换可表示为奇数个对换之积,偶置换可表示为偶数个对换之积.

进而长度为奇数的轮换都是奇置换,长度为偶数的轮换都是偶置换.
\end{corollary}
\begin{proof}
由\refthe{theorem:奇置换与偶置换}中奇置换定义知对换显然都是奇置换.设$\sigma \in S_n$,则由\rrefthe{theorem:置换必可写成对换之积}{theorem:置换必可写成对换之积-3}知$\sigma = \tau_1\tau_2\cdots\tau_k$,其中$\tau_i$都是对换。又注意到对换$\tau_i = (ij)$都是奇置换,故$\text{sgn} \tau_i = -1$。由\refthe{theorem:奇置换与偶置换}知$\text{sgn}$是$S_n$到$\{-1,1\}$的同态,因此
\begin{align*}
\text{sgn} \sigma = \text{sgn}(\tau_1\tau_2\cdots\tau_k) = (\text{sgn} \tau_1)(\text{sgn} \tau_2)\cdots(\text{sgn} \tau_k) = (-1)^k.
\end{align*}

若$\sigma$是奇置换,则$\text{sgn} \sigma = (-1)^k = -1$,即$k$为奇数。

若$\sigma$是偶置换,则$\text{sgn} \sigma = (-1)^k = 1$,即$k$为偶数。

设$r$轮换$(i_1i_2,\cdots i_r)$,则由\rrefthe{theorem:置换必可写成对换之积}{theorem:置换必可写成对换之积-1}知
\begin{align*}
(i_1i_2\cdots i_r)=(i_1i_r)(i_1i_{r-1})\cdots (i_1i_2).
\end{align*}
由\refthe{theorem:奇置换与偶置换}知$\text{sgn}$是$S_n$到$\{-1,1\}$的同态,因此
\begin{align*}
\text{sgn}\mathrm{(}i_1i_2\cdots i_r)=\text{sgn}\mathrm{(}i_1i_r)\cdot \text{sgn}\mathrm{(}i_1i_{r-1})\cdots \text{sgn}\mathrm{(}i_1i_2)=\left( -1 \right) ^r.
\end{align*}
若$r$是奇数,则$\text{sgn}\mathrm{(}i_1i_2\cdots i_r) = (-1)^r = -1$,即$(i_1i_2,\cdots i_r)$为奇置换.

若$r$是偶数,则$\text{sgn}\mathrm{(}i_1i_2\cdots i_r) = (-1)^r = 1$,即$(i_1i_2,\cdots i_r)$为偶置换。

\end{proof}







\end{document}
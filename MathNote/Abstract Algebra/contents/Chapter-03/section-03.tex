\documentclass[../../main.tex]{subfiles}% 注意这里的文件路径不能用 ./main.tex ,否则用latexmk编译子文件会报错
\graphicspath{{\subfix{./image/}}} % 指定图片目录,后续可以直接使用图片文件名
% 注意这里的文件路径不能用 ../../image/ ,否则用latexmk编译子文件会报错

% 例如:
% \begin{figure}[H]
% \centering
% \includegraphics[scale=0.3]{图.png}
% \caption{}
% \label{figure:图}
% \end{figure}
% 注意:上述\label{}一定要放在\caption{}之后,否则引用图片序号会只会显示??.

\begin{document}

\section{Sylow子群}

\begin{definition}[$p$群]
设$p$是素数. 若群$G$的阶是$p$的方幂, 即$|G|=[G:e]=p^k(k\in\mathbb{N})$, $e$为$G$的幺元, 则称$G$是一个$\boldsymbol{p}$\textbf{群}.
\end{definition}

\begin{theorem}\label{theorem:抽象代数--定理4.3.1}
设$p$群$G$作用在集合$X$上, $|X|=n$, $t=|\{x\in X\mid g(x)=x, \forall g\in G\}|$,则有下列结论:
\begin{enumerate}[(1)]
\item\label{theorem:抽象代数--定理4.3.1-1} $t\equiv n\pmod{p}$.也即$n\equiv t\pmod{p}$. \label{eq:::09jf0r2g4h4hj4.3.1}

\item\label{theorem:抽象代数--定理4.3.1-2} 当$(n,p)=1$时, $t\geqslant 1$. 即$\exists x\in X$, 使$g(x)=x(\forall g\in G)$.也即$\exists x\in X$,使$O_x=\{x\}$.亦即$X$中必有不动元素.

\item\label{theorem:抽象代数--定理4.3.1-3} $G$的中心$C(G)\neq \{e\}$.
\end{enumerate}
\end{theorem}
\begin{remark}
由\eqref{eq::98h20jf30tuj3pg34}式知$\{x\in X\mid g(x)=x, \forall g\in G\}$中的元素$x$的轨道都只包含其自身一个元素即$O_x=\{x\},|O_x|=1.$故$t$就是只包含一个元素的$X$的轨道的个数.
\end{remark}
\begin{proof}
\begin{enumerate}[(1)]
\item 由\rrefthe{theorem:抽象代数--定理4.2.2}{theorem:抽象代数--定理4.2.2-1}及$|X|=n$,可设$X$的轨道分解为
\begin{align*}
X=O_{x_1}\bigsqcup O_{x_2}\bigsqcup\cdots\bigsqcup O_{x_m},
\end{align*}
其中$O_{x_1},O_{x_2},\cdots,O_{x_m}(m\leqslant n)$为$X$中所有不同的轨道.
注意到
\begin{align*}
&x\in \{x\in X\mid g(x)=x,\forall g\in G\}\Longleftrightarrow g(x)=x(\forall g\in G)
\\
&\Longleftrightarrow O_x=\left\{ g\left( x \right) \in X\mid g\in G \right\} =\left\{ x \right\} \Longleftrightarrow |O_x|=1,
\end{align*}
故
\begin{align}\label{eq::98h20jf30tuj3pg34}
\{x\in X\mid g(x)=x,\forall g\in G\}=\left\{ x\in X\mid O_x=\left\{ x \right\} \right\} =\left\{ x\in X\mid |O_x|=1 \right\}.
\end{align}
从而对$\forall x,y\in \{x\in X\mid g(x)=x,\forall g\in G\}$且$x\ne y$,有$O_x=\{x\}\ne \{y\}=O_y$.因此$x,y\in \{x_1,x_2,\cdots ,x_m\}$.故$\{x\in X\mid g(x)=x,\forall g\in G\}\subseteq \{x_1,x_2,\cdots ,x_m\}$.
于是
\begin{align*}
n&=\left| O_{x_1} \right|+\cdots +\left| O_{x_m} \right|=\sum_{\substack{|O_{x_i}|= 1}} |O_{x_i}|+\sum_{\substack{|O_{x_i}|\neq 1}} |O_{x_i}|
\\
&=\sum_{\substack{x_i\in \{x\in X\mid g(x)=x,\forall g\in G\}}} 1+\sum_{\substack{|O_{x_i}|\neq 1}} |O_{x_i}|=t+\sum_{\substack{|O_{x_i}|\neq 1}} |O_{x_i}|.
\end{align*}
由\refcor{corollary:有限群轨道的阶与其商掉迷向子群的阶相同}知$|O_{x_i}|\mid |G|$. 由$G$为$p$群, $|O_{x_i}|>1$, 故$p\mid |O_{x_i}|$, 因而结论(1)成立.
\item $(n,p)=1$, 由结论(1)知$t\neq 0$, 故结论(2)成立.
\item 考虑$G$在$G$上的伴随作用. 由\rrefthe{theorem:共轭类,中心化子,中心的集合形式}{theorem:共轭类,中心化子,中心的集合形式-4}知
$$
C(G)=\{x\in G\mid \mathrm{ad}x\left( g \right) =\mathrm{id}_G\left( g \right)=g ,\forall g\in G\}.
$$
自然$e\in C(G)$, 故$|C(G)|\geqslant 1$. 又$p\mid |G|$, 由结论(1)(取$X=C(G)$)知$|G|\equiv |C(G)|\pmod{p}$, 故$|C(G)|>1$, 即$C(G)\neq \{e\}$.
\end{enumerate}

\end{proof}

\begin{lemma}\label{lemma:Sylow定理证明引理1}
设$p$是素数, $n=p^lm$, $(m,p)=1$. 若$k\in\mathbb{N}, k\leqslant l$, 则
\begin{align*}
p^{l-k}||\mathrm{C}_n^{p^k}, 
\end{align*}
其中 $||$表示恰能整除,即$p^{l-k}\mid \mathrm{C}_n^{p^k}$但$p^{l-k+1}\nmid \mathrm{C}_n^{p^k}$, $\mathrm{C}_n^{p^k}$是组合数.
\end{lemma}
\begin{proof}
当$1\leqslant i\leqslant p^k-1$时,$i$都有分解$i=j_ip^t$, 其中, $(j_i,p)=1$, 于是有$t<k\leqslant l$, 而此时
\begin{align*}
n-i&=p^lm-p^tj=p^t(p^{l-t}m-j_i), \\
p^k-i&=p^t(p^{k-t}-j_i),
\end{align*}
因而$p^t||(n-i)$, $p^t||(p^k-i)$. 又
\begin{align*}
\mathrm{C}_{n}^{p^k}&=\frac{n}{p^k}\frac{n-1}{p^k-1}\cdots \frac{n-(p^k-1)}{p^k-(p^k-1)}=\frac{n}{p^k}\cdot \prod_{i=1}^{p^k-1}{\frac{n-i}{p^k-i}}
\\
&=\frac{p^lm}{p^k}\cdot \prod_{i=1}^{p^k-1}{\frac{p^t\left( p^{l-t}m-j_i \right)}{p^t\left( p^{k-t}-j_i \right)}}=p^{l-k}\cdot m\prod_{i=1}^{p^k-1}{\frac{p^{l-t}m-j_i}{p^{k-t}-j_i}}.
\end{align*}
注意到$(m\prod_{i=1}^{p^k-1}{\frac{p^{l-t}m-j_i}{p^{k-t}-j_i}},p)=1$,故由此知$p^{l-k}||\mathrm{C}_n^{p^k}$.

\end{proof}

\begin{theorem}[Sylow 第一定理]\label{theorem:Sylow 第一定理}
设$G$是一个阶为$p^lm$的群, 其中, $p$为素数, $l\geqslant 1$, $(p,m)=1$, 则对任何$1\leqslant k\leqslant l$, $G$中一定有$p^k$阶子群.
\end{theorem}
\begin{proof}
令$X$是$G$中所有含$p^k$个元素的子集的集合, 即
\begin{align*}
X=\{A\subseteq G\mid |A|=p^k\}.
\end{align*}
显然$|X|=\mathrm{C}_n^{p^k}$, 其中$n=p^lm$.

$G\times X$到$X$上的映射
\begin{align*}
f(g,A)=gA=\{ga|a\in A\}
\end{align*}
定义了$G$在$X$上的作用. 于是由\rrefthe{theorem:抽象代数--定理4.2.2}{theorem:抽象代数--定理4.2.2-1}知$X$有轨道分解
\begin{align*}
X=\bigsqcup O_A, \quad |X|=\sum |O_A|.
\end{align*}
由\reflem{lemma:Sylow定理证明引理1}知$p^{l-k}||\mathrm{C}_n^{p^k}$,即$p^{l-k}|||X|$ .因而$\exists A\in X$, 使$p^{l-k}\mid |O_A|,p^{l-k+1}\nmid |O_A|$. 从而存在$t$,使$(p,t)=1$且$|O_A|=p^{l-k}t$.
设$F_A$是$A$的迷向子群, 于是由\refcor{corollary:有限群轨道的阶与其商掉迷向子群的阶相同}及\hyperref[theorem:抽象代数--Lagrange定理]{Lagrange定理}可得
\begin{align*}
|O_A|&=[G:F_A]=\frac{p^lm}{[F_A:e]}=\frac{p^lm}{|F_A|}\Longrightarrow |O_A|\cdot |F_A|=p^lm
\\
&\Longrightarrow p^{l-k}t\cdot |F_A|=p^lm\Longrightarrow |F_A|t=p^k.
\end{align*}
又$(p,t)=1$,故$p^k\mid |F_A|$。若$p^{k+1}\mid |F_A|$,则存在$c$,使$|F_A|=p^{k+1}c$,从而由上式知
\begin{align*}
p^lm=|O_A|\cdot |F_A|=p^{l-k}t\cdot p^{k+1}c=p^{l+1}tc\implies m=ptc,
\end{align*}
这与$(p,m)=1$矛盾!故$p^{k+1}\nmid |F_A|$,因此$p^k\parallel |F_A|$。

另一方面, 对$g\in F_A$有$gA=A$,即$g(a)=ga\in A(\forall a\in A)$. 于是$F_A\cdot a\subseteq A$, 故再由\refpro{proposition:有限群乘积的阶}知
$$
|F_A\cdot a|=|F_A|\leqslant |A|=p^k.
$$
由此知$|F_A|=p^k$, 即$F_A$是一个$p^k$阶子群.

\end{proof}

\begin{definition}[Sylow $p$子群]
设群$G$的阶为$p^lm$, $p$为素数且$(p,m)=1$, 则$G$的$p^l$阶子群称为$G$的\textbf{Sylow $\boldsymbol{p}$子群}.
\end{definition}
\begin{remark}
\hyperref[theorem:Sylow 第一定理]{Sylow 第一定理}肯定了 Sylow $p$ 子群的存在性,故上述定义是良定义的.
\end{remark}

\begin{proposition}\label{theorem:Sylow p群中元素的阶}
设$P$为群$G$的一个Sylow $p$子群,则对$\forall a\in P$,都存在$k\in \mathbb{N}$,使得$\mathrm{ord}\,a = p^k.$
\end{proposition}
\begin{remark}
这个命题表明:Sylow $p$子群的元素的阶都是$p$的方幂.
\end{remark}
\begin{proof}
设$|P|=p^l$,则由\refthe{theorem:抽象代数-推论 1.3.4}知$\mathrm{ord}\,a \mid |P|$,故存在$k\in \mathbb{N}$,使得$\mathrm{ord}\,a = p^k.$

\end{proof}

\begin{theorem}[Sylow 第二定理]\label{theorem:Sylow 第二定理}
设群$G$的阶为$p^lm$, $p$为素数, $(p,m)=1$. 又$P$是$G$的一个Sylow $p$子群, $H$是$G$的一个$p^k$阶子群, 则$\exists g\in G$, 使$H\subseteq gPg^{-1}$.
特别地, $G$的Sylow $p$子群是相互共轭的.
\end{theorem}
\begin{proof}
将$G$在$G/P$上的左平移作用限制在$H$上, 于是得到$H$在$G/P$上的左平移作用
\begin{align*}
h(gP)=hgP, \quad \forall h\in H, g\in G.
\end{align*}
由\hyperref[theorem:抽象代数--Lagrange定理]{Lagrange定理}知
\begin{align*}
\left[ G:e \right] =\left[ G:P \right] \left[ P:e \right] \Longleftrightarrow \left| G \right|=\left| G/P \right|\left| P \right|\Longleftrightarrow p^lm=\left| G/P \right|p^l\Longleftrightarrow \left| G/P \right|=m.
\end{align*}
又$|H|=p^k$, $(p,m)=1$, 故由\rrefthe{theorem:抽象代数--定理4.3.1}{theorem:抽象代数--定理4.3.1-2}知$G/P$中含有元素$gP$, 其轨道仅含$gP$, 即$hgP=gP(\forall h\in H)$, 故存在$p_1,p_2$,使$hgp_1=gp_2$,从而$h=gp_2p_1^{-1}g^{-1}\in gpg^{-1}$.因此$H\subseteq gPg^{-1}$.

特别地, 若$H$也是$G$的一个Sylow $p$子群,则$|H|=p^l$,再由\refpro{proposition:有限群乘积的阶}知$|H|=p^l=|P|=|gPg^{-1}|$. 又由之前证明知$H\subseteq gPg^{-1}$,从而$H=gPg^{-1}$.由\refthe{theorem:共轭子群和正规化子}知$H,P$相互共轭.

\end{proof}

\begin{corollary}\label{corollary:p群中的所有Sylow p子群}
设群$G$的阶为$p^lm$, $p$为素数, $(p,m)=1$. 又$P$是$G$的一个Sylow $p$子群,则
\begin{enumerate}[(1)]
\item\label{corollary:p群中的所有Sylow p子群-1} 群$G$中Sylow $p$子群的集合是$X=\{gPg^{-1}|g\in G\}$,即$P$的共轭子群构成的集合.

\item\label{corollary:p群中的所有Sylow p子群-2} $G\times X$到$X$的映射
\begin{align*}
f(g,P_1)=g(P_1)=gP_1g^{-1}, \quad \forall g\in G, P_1\in X.
\end{align*}
是群$G$在$X$上的作用.并且$G$在$X$上的作用$f$是可递的.
\end{enumerate}
\end{corollary}
\begin{proof}
\begin{enumerate}[(1)]
\item 任取$g\in G$,对$\forall p_1,p_2\in P$,有$(gp_1g^{-1})(gp_2g^{-1})^{-1}=gp_1p_2^{-1}g^{-1}\in gPg^{-1},$因此$gPg^{-1}$是$G$的子群.
又由\refpro{proposition:有限群乘积的阶}知$|P|=|gPg^{-1}|=p^l$,故$gPg^{-1}$也是$G$的Sylow $p$子群.

又若$P_1$是$G$的另一Sylow $p$子群. 由\hyperref[theorem:Sylow 第二定理]{Sylow 第二定理}知$\exists g_1\in G$, 使得$g_1Pg_1^{-1}=P_1$, 因而$X=\{gPg^{-1}|g\in G\}$是$G$中Sylow $p$子群的集合.

\item 根据群作用的定义容易验证$f$是群$G$在$X$上的一个作用.对$\forall P_1,P_2\in \{X\}$,由\hyperref[theorem:Sylow 第二定理]{Sylow 第二定理}知$P_1,P_2$共轭,即存在$g\in G$,使
\begin{align*}
P_1=gP_2g^{-1}=g(P_1)=f(g,P_1).
\end{align*}
故$G$在$X$上的作用$f$是可递的.
\end{enumerate}

\end{proof}

\begin{theorem}[Sylow第三定理]\label{theorem:Sylow第三定理}
设群$G$的阶为$p^lm$, $p$为素数, $(p,m)=1$. 又设$G$中Sylow $p$子群的个数为$k$, 则$k$也是$G$中任意一个Sylow $p$子群的共轭子群的个数.
并且有
\begin{enumerate}[(1)]
\item\label{theorem:Sylow第三定理-1} 当且仅当$k=1$时, $G$的Sylow $p$子群$P\lhd G$;

\item\label{theorem:Sylow第三定理-2} $k|m$, $k\equiv 1\pmod{p}$. \label{eq:::09jf0r2g4h4hj4.3.3}
\end{enumerate}
\end{theorem}
\begin{proof}
设$P$是$G$的一Sylow $p$子群.则由\rrefcor{corollary:p群中的所有Sylow p子群}{corollary:p群中的所有Sylow p子群-1}知$X=\{gPg^{-1}|g\in G\}$是$G$中Sylow $p$子群的集合.从而$k$也是$G$中任意一个Sylow $p$子群的共轭子群的个数.
\begin{enumerate}[(1)]
\item 若$|X|=1$, 即$gPg^{-1}=P(\forall g\in G)$, 故由正规子群定义知$P\lhd G$. 反之, 若$P\lhd G$, 则$gPg^{-1}=P(\forall g\in G)$, 故$|X|=1$. 这样就证明了结论(1).

\item 由\rrefcor{corollary:p群中的所有Sylow p子群}{corollary:p群中的所有Sylow p子群-1}知$X=\{gPg^{-1}|g\in G\}$是$G$中Sylow $p$子群的集合.现设$|X|=k$, 由\rrefcor{corollary:p群中的所有Sylow p子群}{corollary:p群中的所有Sylow p子群-2}知$G\times X$到$X$的映射
\begin{align*}
f(g,P_1)=gP_1g^{-1}, \quad \forall g\in G, P_1\in X.
\end{align*}
定义了$G$在$X$上的作用. 设$F_P$为$P$的迷向子群, 即
\begin{align*}
F_P=\{g\in G, |gPg^{-1}=P\}.
\end{align*}
显然, $P\lhd F_P$, 故由\hyperref[theorem:抽象代数--Lagrange定理]{Lagrange定理}知$|P|\mid |F_P|$,即$p^l\mid |F_P|$, 因而存在$t$,使得
\begin{align}\label{eq::90j243gy5yhserfgdjirzz}
|F_P|=p^lt.
\end{align}
于是由\hyperref[theorem:抽象代数--Lagrange定理]{Lagrange定理}知
\begin{align}\label{eq::iogjio30uv33}
|G|=[G:F_P]|F_P| \iff p^lm=[G:F_P]p^lt \iff m=[G:F_P]t\Longrightarrow [G:F_P] \mid m,\, t\mid m.
\end{align}
又注意到$G$在$X$上的作用下$P$的轨道为$O_P=X$,故由\refcor{corollary:有限群轨道的阶与其商掉迷向子群的阶相同}知
\begin{align*}
k=|X|=[G:F_P].
\end{align*}
因此再结合\eqref{eq::iogjio30uv33}式得$k\mid m$.

将上面$G$在$X$上的作用限制为$P$在$X$上的作用, 显然$P\in X$, $P$在$X$上的作用下$P$的轨道$O_P'=\{P\}$. 若另有$P_1\in X$, 在$P$作用下的轨道$O_{P_1}'=\{P_1\}$, 即有$gP_1g^{-1}=P_1(\forall g\in P)$. 由\hyperref[theorem:Sylow 第二定理]{Sylow 第二定理}, $\exists h\in G$, 使得
$P_1=hPh^{-1}$, 因而
$$
g(hPh^{-1})g^{-1}=hPh^{-1}(\forall g\in P) \iff (h^{-1}gh)P(h^{-1}gh)^{-1}=P(\forall g\in P).
$$
故$h^{-1}gh\in F_P(\forall g\in P)$,从而$hPh^{-1}\subseteq F_P$. 因此$h^{-1}Ph$, $P$均为$F_P$的子群.由\eqref{eq::iogjio30uv33}式知$t\mid m$,又因为$(p,m)=1$,所以$(p,t)=1.$而由\eqref{eq::90j243gy5yhserfgdjirzz}知$|F_p|=p^lt,|P|=p^l$,再由\refpro{proposition:有限群乘积的阶}知$|h^{-1}Ph|=|P|=p^l$,故$h^{-1}Ph$, $P$均为$F_P$的Sylow $p$子群. 又$P\lhd F_P$, 故由结论(1)知$h^{-1}Ph=P$, 故$P=P_1$. 这就说明包含一个元素的$X$的轨道仅有一个. 注意到
\begin{align*}
P' \in \{ P' \in X \mid g(P') = gP'g^{-1} = P', \forall g \in P \} \iff O_{P'} = \{ g(P') = gP'g^{-1} = P' \mid g \in P \} = \{ P' \},
\end{align*}
故
\begin{align*}
\{ P' \in X \mid g(P') = gP'g^{-1} = P', \forall g \in P \} = \{ P' \in X \mid O_{P'} = \{ P' \} \} = \{ P \}.
\end{align*}
即$| \{ P' \in X \mid g(P') = gP'g^{-1} = P', \forall g \in P \} | = 1.$
故由\rrefthe{theorem:抽象代数--定理4.3.1}{theorem:抽象代数--定理4.3.1-1}知$k\equiv 1\pmod{p}$.
\end{enumerate}

\end{proof}

\begin{definition}[单群]\label{definecolor:单群定义--抽象代数}
一个群如果没有非平凡的正规子群就称为\textbf{单群}.
\end{definition}

\begin{proposition}\label{proposition:阶数大于2的循环群必非单群}
阶数大于2的循环群都不是单群
\end{proposition}
\begin{proof}
因为循环群都是Abel群,所以其子群都是正规子群.又有阶数大于2,故一定有非平凡的正规子群.

\end{proof}

\begin{example}
设群$G$的阶为72, 则$G$不是单群.
\end{example}
\begin{remark}
Sylow 定理在群论中有许多应用, 其一就是判断某些有限群不是单群.
\end{remark}
\begin{solution}
$72=2^3\cdot 3^2$. 设$G$中Sylow 3子群的个数为$k$, 于是由\hyperref[theorem:Sylow第三定理]{Sylow第三定理}知有$t$, 使得$k=3t+1$, $k|8$, 因而$t=0$或$t=1$.

若$t=0$, 则$k=1$. 此时再由\hyperref[theorem:Sylow第三定理]{Sylow第三定理}知Sylow 3子群为$G$的正规子群, 故$G$不是单群.

若$t=1$, 则$k=4$. 设$X=\{P_1,P_2,P_3,P_4\}$为$G$的Sylow 3子群的集合, 由\rrefcor{corollary:p群中的所有Sylow p子群}{corollary:p群中的所有Sylow p子群-2}知有$G$在$X$上的作用
\begin{align*}
g(P_1)=gP_1g^{-1}, \quad \forall g\in G, P_1\in X.
\end{align*}
并且这个$G$在$X$上的作用是可递的.
由\refthe{theorem:抽象代数--定理4.2.1}知有$G$到$S_X=S_4$中的同态$\sigma$满足
\begin{align*}
\sigma \left( g \right) =\sigma _g,\forall g\in G,
\end{align*}
其中$\sigma _g:X\rightarrow X,P\longmapsto g\left( P \right) =gPg^{-1}.$
于是由\hyperref[theorem:群的同态基本定理]{群的同态基本定理}知$\ker \sigma \lhd G$且$G/\ker\sigma$与$S_4$的一个子群$\sigma(G)$同构, 而$|S_4|=24<72$,于是
\begin{align*}
[G:\ker\sigma] = |\sigma(G)| \leqslant |S_4| = 24 < 72 = |G|.
\end{align*}
再利用\hyperref[theorem:抽象代数--Lagrange定理]{Lagrange定理}可得
\begin{align*}
|G| = [G:\ker\sigma]|\ker\sigma| \implies |\ker\sigma| = \frac{|G|}{[G:\ker\sigma]} \geqslant \frac{72}{24} > 1.
\end{align*}
因此$\ker\sigma \neq \{e\}$.

注意到
\begin{align*}
\ker\sigma = \{ g \in G \mid \sigma(g) = \mathrm{id}_G \} = \{ g \in G \mid gPg^{-1} = P, \forall P \in X \},
\end{align*}
又由$G$在$X$上的作用可递知对$P_1,P_2 \in X$,存在$g_1 \in G$,使$P_2 = g(P_1) = g_1P_1g_1^{-1}$。若$\ker\sigma = G$,则$g_1 \in \ker\sigma$,从而
\begin{align*}
P_2 = g_1P_1g_1^{-1} = P_1,
\end{align*}
这与$P_1 \neq P_2$矛盾!因此$\ker \sigma\neq G$. 故$\ker\sigma$是$G$的非平凡正规子群, 因而$G$不是单群.

\end{solution}

\begin{proposition}\label{proposition:阶为特定素数乘积的群必为循环群}
设 $p,q$ 都是素数, $p<q$, $p\nmid (q-1)$. 证明 $pq$ 阶群一定是循环群.
\end{proposition}
\begin{proof}
设 $P,Q$ 分别为 $pq$ 阶群 $G$ 的 $p,q$ 阶子群. 于是由\hyperref[theorem:Sylow第三定理-2]{Sylow第三定理\ref{theorem:Sylow第三定理-2}},可设$P,Q$ 共轭子群的个数分别为$sp+1,tq+1(s,t\in \mathbb{N})$.由\rrefcor{corollary:p群中的所有Sylow p子群}{corollary:p群中的所有Sylow p子群-1}知,$P$的共轭子群都是$G$的Sylow $p$子群,$Q$的共轭子群都是$G$的Sylow $p$子群.因此
\begin{align*}
sp+1=q,\quad tq+1=p.
\end{align*}
由 $p<q$, 于是 $t=0$, 因而 $Q$ 是 $G$ 的唯一的 $q$ 阶子群. 若 $s\neq 0$, 则 $sp+1=q$, 这与 $p\nmid (q-1)$ 矛盾. 于是 $s=0$, 因而 $P$ 是 $G$ 的唯一的 $p$ 阶子群.

因为$p,q$都是素数且$p<q$,所以$|P\cup Q|\leqslant p+q<pq=|G|$,因此$G\neq Q\cup P$. 对$\forall a \in G\setminus (P\cup Q)$,则$\langle a\rangle \neq P,Q$,从而由$P,Q$分别是$G$的唯一$p,q$阶子群知$\langle a\rangle$ 的阶既不是 $p$ 也不是 $q$.又由\hyperref[theorem:抽象代数--Lagrange定理]{Lagrange定理}知$|\langle a\rangle|=\mid pq$,于是$|\langle a\rangle|=pq$, 因此$G=\langle a\rangle$为循环群.

\end{proof}

\begin{proposition}
设$G$是有限Abel群,则$g$对应到$g^k$是$G$的一个自同构当且仅当$k$和$|G|$互素.
\end{proposition}
\begin{proof}
{\heiti 充分性:}设$|G|$与$k$互素,令
\begin{align*}
\alpha: G\to G,\ \alpha(g)=g^k,\ \forall g\in G.
\end{align*}
因$G$是Abel群,故
\begin{align*}
\alpha(g_1g_2)=(g_1g_2)^k=g_1^kg_2^k=\alpha(g_1)\alpha(g_2).
\end{align*}
因此$\alpha$是群同态.设$\alpha(g)=\alpha(h)$,即$(h^{-1}g)^k=1$.又由\refthe{theorem:抽象代数-推论 1.3.4}知$h^{-1}g$的阶是$|G|$的因子,由题设知$k$和$h^{-1}g$的阶互素.由\rrefthe{theorem:群元素的阶的基本性质}{theorem:群元素的阶的基本性质-4}可推出$1=\text{ord}\,1=\text{ord}(h^{-1}g)^k=\text{ord}(h^{-1}g)$,故$h=g$,即$\alpha$是单射.设$l|G|+mk=1$,则对任一$g\in G$,由\refthe{theorem:抽象代数-推论 1.3.4}有
\begin{align*}
g=g^{l|G|+mk}=(g^m)^k.
\end{align*}
这表明$\alpha$是满射,即$\alpha$是群$G$的自同构.

{\heiti 必要性:}设$\alpha: G\to G,\ \alpha(g)=g^k,\ \forall g\in G$是群自同构.则对任一$g\in G$, $g$与$g^k=\alpha(g)$的阶相同.另一方面,根据\rrefthe{theorem:群元素的阶的基本性质}{theorem:群元素的阶的基本性质-4}知
\begin{align*}
\text{ord}\,g^k=\frac{\text{ord}\,g}{(k,\text{ord}\,g)},
\end{align*}
即知$k$和$\text{ord}\,g$互素,对$\forall g\in G$成立.

{\color{blue}证法一:}设$|G|=p_1^{k_1}p_2^{k_2}\cdots p_s^{k_s}$,其中$p_1,p_2,\cdots,p_s$两两互素.
由\hyperref[theorem:抽象代数--Cauchy定理]{Cauchy定理}知,
对$|G|$的任一素因子$p_i$,都存在$p$阶元$g\in G$,故$k$与$p_i$互素,从而$k$与$|G|$互素.

{\color{blue}证法二:}对$|G|$用数学归纳法证明$k$与$|G|$互素.假设结论对阶小于$|G|$的有限Abel群都成立.

若$\exists g\in G,\,\text{s.t.}\,|G|=\text{ord}\,g$,则已证.设$\text{ord}\,g<|G|$, $\forall g\in G$.
令$1\neq a\in G,\text{ord}\,a=n<|G|$,则由\refthe{theorem:抽象代数-推论 1.3.4}知$n\mid |G|$,从而$(k,n)=1$.考虑$G$的商群$\overline{G}=G/\langle a\rangle$,则$|\overline{G}|=\frac{|G|}{n}<|G|$.记$g\langle a\rangle= \overline{g},\forall g\in G$,再考虑由$\alpha$诱导的映射$\overline{\alpha}$:
\begin{align*}
\overline{\alpha}:\overline{G}\longrightarrow \overline{G},\,\,\overline{g} \longmapsto \overline{g^k}.
\end{align*}
因$G$是Abel群,故显然$\overline{\alpha}$是$\overline{G}$的自同态.由$(|G|,k)=1$知存在$l,m\in \mathbb{N}$,使得$l_1|G|+m_1k=1$.从而对$\forall \overline{g} \in \overline{G}$,由\refthe{theorem:抽象代数-推论 1.3.4}有
\begin{align*}
g=g^{l_1|G|+m_1k}=(g^{m_1})^k.
\end{align*}
因此$\overline{\alpha}(\overline{g^{m_1}})=\overline{g}$,故$\overline{\alpha}$是满自同态.
下证$\overline{\alpha}$作为$\overline{G}$的自同态也是单射.若$\overline{\alpha}(\overline{g})=\overline{\alpha}(\overline{h})$,即$(h^{-1}g)^k\in\langle a\rangle$,则
\begin{align*}
(h^{-1}g)^{kn}=1\Longrightarrow (h^{-1}g)^n=1.
\end{align*}
由$(k,n)=1$,可设$ln+mk=1$,于是
\begin{align*}
h^{-1}g=\left( h^{-1}g \right) ^{ln+mk}=((h^{-1}g)^{k})^{m}\in \langle a\rangle .
\end{align*}
即$\overline{h}=\overline{g}$,故$\overline{\alpha}$是单射.因此$\overline{\alpha}$是$\overline{G}$上自同构.于是由归纳假设知$(k,\frac{|G|}{n})=1$.而已知$(k,n)=1$且$n\mid |G|$,从而$(k,|G|)=(k,\frac{|G|}{n}\cdot n)=1$.

\end{proof}





\end{document}
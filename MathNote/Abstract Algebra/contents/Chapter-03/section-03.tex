\documentclass[../../main.tex]{subfiles}
\graphicspath{{\subfix{../../image/}}} % 指定图片目录,后续可以直接使用图片文件名。

% 例如:
% \begin{figure}[H]
% \centering
% \includegraphics[scale=0.4]{图.png}
% \caption{}
% \label{figure:图}
% \end{figure}
% 注意:上述\label{}一定要放在\caption{}之后,否则引用图片序号会只会显示??.

\begin{document}

\section{Sylow子群}

\begin{definition}[$p$群]
设$p$是素数. 若群$G$的阶是$p$的方幂, 即$|G|=[G:e]=p^k(k\in\mathbf{N})$, $e$为$G$的幺元, 则称$G$是一个$\boldsymbol{p}$\textbf{群}.
\end{definition}

\begin{theorem}
设$p$群$G$作用在集合$X$上, $|X|=n$, $t=|\{x\in X\mid g(x)=x, \forall g\in G\}|$,则有下列结论:
\begin{enumerate}[(1)]
\item $t\equiv n\pmod{p}$; \label{eq:::09jf0r2g4h4hj4.3.1}
\item 当$(n,p)=1$时, $t\geqslant 1$, 即$\exists x\in X$, 使$g(x)=x(\forall g\in G)$;
\item $G$的中心$C(G)\neq \{e\}$.
\end{enumerate}
\end{theorem}
\begin{proof}
\begin{enumerate}[(1)]
\item 由\rrefthe{theorem:抽象代数--定理4.2.2}{theorem:抽象代数--定理4.2.2-1}及$|X|=n$,可设$X$的轨道分解为
\begin{align*}
X=O_{x_1}\bigcup O_{x_2}\bigcup\cdots\bigcup O_{x_m},
\end{align*}
其中$O_{x_1},O_{x_2},\cdots,O_{x_m}(m\leqslant n)$为$X$中所有不同的轨道.
注意到
\begin{align*}
&x\in \{x\in X\mid g(x)=x,\forall g\in G\}\Longleftrightarrow g(x)=x(\forall g\in G)
\\
&\Longleftrightarrow O_x=\left\{ g\left( x \right) \in X\mid g\in G \right\} =\left\{ x \right\} \Longleftrightarrow |O_x|=1,
\end{align*}
从而对$\forall x,y\in \{x\in X\mid g(x)=x,\forall g\in G\}$且$x\ne y$,有$O_x=\{x\}\ne \{y\}=O_y$.因此$x,y\in \{x_1,x_2,\cdots ,x_m\}$.故$\{x\in X\mid g(x)=x,\forall g\in G\}\subseteq \{x_1,x_2,\cdots ,x_m\}$.
于是
\begin{align*}
n&=\left| O_{x_1} \right|+\cdots +\left| O_{x_m} \right|=\sum_{\substack{|O_{x_i}|= 1}} |O_{x_i}|+\sum_{\substack{|O_{x_i}|\neq 1}} |O_{x_i}|
\\
&=\sum_{\substack{x_i\in \{x\in X\mid g(x)=x,\forall g\in G\}}} 1+\sum_{\substack{|O_{x_i}|\neq 1}} |O_{x_i}|=t+\sum_{\substack{|O_{x_i}|\neq 1}} |O_{x_i}|.
\end{align*}
由\refcor{corollary:有限群轨道的阶与其商掉迷向子群的阶相同}知$|O_{x_i}|\mid |G|$. 由$G$为$p$群, $|O_{x_i}|>1$, 故$p\mid |O_{x_i}|$, 因而结论(1)成立.
\item $(n,p)=1$, 由结论(1)知$t\neq 0$, 故结论(2)成立.
\item 考虑$G$在$G$上的伴随作用. 由\rrefthe{theorem:共轭类,中心化子,中心的集合形式}{theorem:共轭类,中心化子,中心的集合形式-4}知
$$
C(G)=\{x\in G\mid \mathrm{ad}x\left( g \right) =\mathrm{id}_G\left( g \right)=g ,\forall g\in G\}.
$$
自然$e\in C(G)$, 故$|C(G)|\geqslant 1$. 又$p\mid |G|$, 由结论(1)(取$X=C(G)$)知$|G|\equiv |C(G)|\pmod{p}$, 故$|C(G)|>1$, 即$C(G)\neq \{e\}$.
\end{enumerate}

\end{proof}

\begin{lemma}\label{lemma:Sylow定理证明引理1}
设$p$是素数, $n=p^lm$, $(m,p)=1$. 若$k\in\mathbf{N}, k\leqslant l$, 则
\begin{align*}
p^{l-k}||\mathrm{C}_n^{p^k}, 
\end{align*}
其中 $||$表示恰能整除,即$p^{l-k}\mid \mathrm{C}_n^{p^k}$但$p^{l-k+1}\nmid \mathrm{C}_n^{p^k}$, $\mathrm{C}_n^{p^k}$是组合数.
\end{lemma}
\begin{proof}
当$1\leqslant i\leqslant p^k-1$时,$i$都有分解$i=j_ip^t$, 其中, $(j_i,p)=1$, 于是有$t<k\leqslant l$, 而此时
\begin{align*}
n-i&=p^lm-p^tj=p^t(p^{l-t}m-j_i), \\
p^k-i&=p^t(p^{k-t}-j_i),
\end{align*}
因而$p^t||(n-i)$, $p^t||(p^k-i)$. 又
\begin{align*}
\mathrm{C}_{n}^{p^k}&=\frac{n}{p^k}\frac{n-1}{p^k-1}\cdots \frac{n-(p^k-1)}{p^k-(p^k-1)}=\frac{n}{p^k}\cdot \prod_{i=1}^{p^k-1}{\frac{n-i}{p^k-i}}
\\
&=\frac{p^lm}{p^k}\cdot \prod_{i=1}^{p^k-1}{\frac{p^t\left( p^{l-t}m-j_i \right)}{p^t\left( p^{k-t}-j_i \right)}}=p^{l-k}\cdot m\prod_{i=1}^{p^k-1}{\frac{p^{l-t}m-j_i}{p^{k-t}-j_i}}.
\end{align*}
注意到$(m\prod_{i=1}^{p^k-1}{\frac{p^{l-t}m-j_i}{p^{k-t}-j_i}},p)=1$,故由此知$p^{l-k}||\mathrm{C}_n^{p^k}$.

\end{proof}

\begin{theorem}[Sylow 第一定理]\label{theorem:Sylow 第一定理}
设$G$是一个阶为$p^lm$的群, 其中, $p$为素数, $l\geqslant 1$, $(p,m)=1$, 则对任何$1\leqslant k\leqslant l$, $G$中一定有$p^k$阶子群.
\end{theorem}
\begin{proof}
令$X$是$G$中所有含$p^k$个元素的子集的集合, 即
\begin{align*}
X=\{A|A\subseteq G, |A|=p^k\}.
\end{align*}
显然$|X|=\mathrm{C}_n^{p^k}$, 其中, $n=p^lm$.

$G\times X$到$X$上的映射
\begin{align*}
f(g,A)=gA=\{ga|a\in A\}
\end{align*}
定义了$G$在$X$上的作用. 于是$X$有轨道分解
\begin{align*}
X=\bigcup O_A, \quad |X|=\sum |O_A|.
\end{align*}
由\reflem{lemma:Sylow定理证明引理1}知$p^{l-k}||\mathrm{C}_n^{p^k}$, 因而$\exists A\in X$, 使$p^{l-k+1}\nmid |O_A|$. 设$F_A$是$A$的迷向子群, 于是
\begin{align*}
|O_A|=[G:F_A]=\frac{p^lm}{[F_A:e]}=\frac{p^lm}{|F_A|},
\end{align*}
因而$p^k||[F_A:e]$.

另一方面, 对$\forall a\in A$, $g\in F_A$有$g(a)=ga\in A$. 于是$F_A\cdot a\subseteq A$, 故$|F_A\cdot a|=|F_A|\leqslant |A|=p^k$. 由此知$[F_A:e]=p^k$, 即$F_A$是一个$p^k$阶子群.

\end{proof}

\begin{definition}[Sylow $p$子群]
设群$G$的阶为$p^lm$, $p$为素数且$(p,m)=1$, 则$G$的$p^l$阶子群称为$G$的\textbf{Sylow $\boldsymbol{p}$子群}.
\end{definition}
\begin{remark}
\hyperref[theorem:Sylow 第一定理]{Sylow 第一定理}肯定了 Sylow $p$ 子群的存在性,故上述定义是良定义的.
\end{remark}


\begin{theorem}[Sylow 第二定理]\label{theorem:Sylow 第二定理}
设群$G$的阶为$p^lm$, $p$为素数, $(p,m)=1$. 又$P$是$G$的一个Sylow $p$子群, $H$是$G$的一个$p^k$阶子群, 则$\exists g\in G$, 使$H\subseteq gPg^{-1}$.

特别地, $G$的Sylow $p$子群是相互共轭的.
\end{theorem}

\begin{proof}
将$G$在$G/P$上的左平移作用限制在$H$上, 于是得到$H$在$G/P$上的左平移作用
\begin{align*}
h(gP)=hgP, \quad \forall h\in H, g\in G.
\end{align*}
因$|H|=p^k$, $|G/P|=m$, $(p,m)=1$, 故由定理4.3.1的结论(2)知$G/P$中含有元素$gP$, 其轨道仅含$gP$, 即$hgP=gP(\forall h\in H)$, 故$hg\in gP$, $H\subseteq gPg^{-1}$.

特别地, 若$|H|=p^l$, 则$H=gPg^{-1}$.

\end{proof}

\begin{theorem}[Sylow 第三定理]\label{theorem:Sylow 第三定理}
设群$G$的阶为$p^lm$, $p$为素数, $(p,m)=1$. 又设$G$中Sylow $p$子群的个数为$k$, 则有
\begin{enumerate}[(1)]
\item 当且仅当$k=1$时, $G$的Sylow $p$子群$P\lhd G$;
\item $k|m$, $k\equiv 1\pmod{p}$. \label{eq:::09jf0r2g4h4hj4.3.3}
\end{enumerate}
\end{theorem}

\begin{proof}
\begin{enumerate}[(1)]
\item 设$P$是$G$的一Sylow $p$子群. 显然$\forall g\in G$, $gPg^{-1}$也是$G$的Sylow $p$子群. 又若$P_1$是$G$的另一Sylow $p$子群. 由定理4.3.3知$\exists g_1\in G$, 使得$g_1Pg_1^{-1}=P_1$, 因而$X=\{gPg^{-1}|g\in G\}$是$G$中Sylow $p$子群的集合.

若$|X|=1$, 即$gPg^{-1}=P(\forall g\in G)$, 故$P\lhd G$. 反之, 若$P\lhd G$, 则$gPg^{-1}=P$, 故$|X|=1$. 这样就证明了结论(1).
\item 现设$|X|=k$, 则$G\times X$到$X$的映射
\begin{align*}
f(g,P_1)=gP_1g^{-1}, \quad \forall g\in G, P_1\in X
\end{align*}
定义了$G$在$X$上的作用, 由定理4.3.3知这个作用可递. 设$F_P$为$P$的迷向子群, 即
\begin{align*}
F_P=\{g\in G, |gPg^{-1}=P\}=N_G(P).
\end{align*}
显然, $P\lhd F_P$, 故$p^l||F_P|$, 因而
\begin{align*}
k=|X|=[G:F_P], \quad [G:F_P]|m.
\end{align*}

将上面$G$在$X$上的作用限制为$P$在$X$上的作用, 显然$P\in X$, $P$在$P$作用下的轨道$O_P'=\{P\}$. 若另有$P_1\in X$, 在$P$作用下的轨道$O_{P_1}'=\{P_1\}$, 即有$gP_1g^{-1}=P_1(\forall g\in P)$. 由定理4.3.3, $\exists h\in G$, 使得$P_1=hPh^{-1}$, 因而$g(hPh^{-1})g^{-1}=hPh^{-1}$, 故$h^{-1}gh\in F_P$. 故$h^{-1}Ph$, $P$均为$F_P$的Sylow $p$子群, 又$P\lhd F_P$, 故由结论(1)知$h^{-1}Ph=P$, 故$P=P_1$. 这就说明包含一个元素的$X$的轨道仅有一个. 由定理4.3.1的结论(1)知$k\equiv 1\pmod{p}$.
\end{enumerate}

\end{proof}

Sylow 定理在群论中有许多应用, 其一就是判断某些有限群不是单群(一个群如果没有非平凡的正规子群就称为\textbf{单群}).

\begin{example}
设群$G$的阶为72, 则$G$不是单群.
\end{example}
\begin{solution}
$72=2^3\cdot 3^2$. 设$G$中Sylow 3子群的个数为$k$, 于是由定理4.3.4知有$t$, 使得$k=3t+1$, $k|8$, 因而$t=0$或$t=1$.

若$t=0$, 则$k=1$. 此时Sylow 3子群为$G$的正规子群, 故$G$不是单群.

若$t=1$, 则$k=4$. 设$X=\{P_1,P_2,P_3,P_4\}$为$G$的Sylow 3子群的集合, $G$在$X$上的作用可递, 由定理4.2.1知有$G$到$S_X=S_4$中的同态$\sigma$. 于是$G/\ker\sigma$与$S_4$的一个子群同构, 而由$|S_4|=24<72$知$\ker\sigma\neq \{e\}$. 又由$G$在$X$上作用可递知$\ker\sigma\neq G$, 故$\ker\sigma$是$G$的非平凡正规子群, 因而$G$不是单群.

\end{solution}



\end{document}
\documentclass[../../main.tex]{subfiles}% 注意这里的文件路径不能用 ./main.tex ,否则用latexmk编译子文件会报错
\graphicspath{{\subfix{./image/}}} % 指定图片目录,后续可以直接使用图片文件名
% 注意这里的文件路径不能用 ../../image/ ,否则用latexmk编译子文件会报错

% 例如:
% \begin{figure}[H]
% \centering
% \includegraphics[scale=0.3]{图.png}
% \caption{}
% \label{figure:图}
% \end{figure}
% 注意:上述\label{}一定要放在\caption{}之后,否则引用图片序号会只会显示??.

\begin{document}

\section{同态基本定理}

\begin{definition}[同态核]
\begin{enumerate}
\item 设$f$是群$G_1$到群$G_2$的同态,$G_2$的幺元$e_2$的原像集合
\begin{align*}
\ker f = f^{-1}(e_2) = \{x \in G_1 | f(x) = e_2\}
\end{align*}
称为$f$的\textbf{核}或\textbf{同态核}.

$G_1$中所有元素的像集合
\begin{align*}
\mathrm{im}(f) =f(G_1)= \{y \in G_2 : \exists x \in G_1, y = f(x)\} = \{f(x) : x \in G_1\} \subseteq G_2.
\end{align*}
称为$f$的\textbf{像}.

\item 设$f$是环$R_1$到环$R_2$的同态,$R_2$的零元素$0$的原像集合
\begin{align*}
\ker f = f^{-1}(0) = \{x \in R_1 | f(x) = 0\}
\end{align*}
称为$f$的\textbf{核}或\textbf{同态核}.

$G_1$中所有元素的像集合
\begin{align*}
\mathrm{im}(f)=f(G_1) = \{y \in R_2 : \exists x \in R_1, y = f(x)\} = \{f(x) : x \in R_1\} \subseteq R_2.
\end{align*}
称为$f$的\textbf{像}.

\item 设$R$是一个环,$M_1, M_2$都是$R$模,$f$是$M_1$到$M_2$的模同态. $M_2$的零元素$0$的原像集合
\begin{align*}
\ker f = f^{-1}(0) = \{x \in M_1 | f(x) = 0\}
\end{align*}
称为$f$的\textbf{核}或\textbf{同态核}.

$G_1$中所有元素的像集合
\begin{align*}
\mathrm{im}(f) =f(G_1)= \{y \in M_2 : \exists x \in M_1, y = f(x)\} = \{f(x) : x \in M_1\} \subseteq M_2.
\end{align*}
称为$f$的\textbf{像}.
\end{enumerate}
\end{definition}


\begin{proposition}\label{proposition:群同态的核是定义域的子群,像是陪域的子群-抽象代数}
\begin{enumerate}[(1)]
\item\label{proposition:群同态的核是定义域的子群,像是陪域的子群-抽象代数-1} 设$f$是群$G$到群$G'$的同态,则$\ker f$是$G$的正规子群,$f(G)$是$G'$的子群.

\item\label{proposition:群同态的核是定义域的子群,像是陪域的子群-抽象代数-2} 设$f$是环$R$到环$R'$的同态,则$\ker f$是$R$的理想,$f(R)$是$R'$的子环.

\item\label{proposition:群同态的核是定义域的子群,像是陪域的子群-抽象代数-3} 设$R$是一个环,$M, M'$都是$R$模,$f$是$M$到$M'$的模同态,则$\ker f$是$M$的子模,$f(M)$是$M'$的子模.
\end{enumerate}
\end{proposition}
\begin{proof}
\begin{enumerate}[(1)]
\item 设$e, e'$分别为$G', H$的幺元,于是$f(e) = e'$,又设$x, y \in \ker f, z \in G'$,则
\begin{align*}
f(xy^{-1}) = f(x)f(y^{-1}) = f(x)f(y)^{-1} = e',
\end{align*}
因此$xy^{-1}\in \ker f$,故知$\ker f$是$G'$的子群,而且有
\begin{align*}
f(zxz^{-1}) = f(z)f(x)f(z)^{-1} = e',
\end{align*}
即$zxz^{-1} \in \ker f$,由此知$\ker f \lhd G'$.

同样由\hyperref[theorem:群同态与同构的基本性质]{群同态与同构的基本性质}知 $f(e)=e'$,我们有 $e'\in\mathrm{im}(f)$。设 $y = f(x),y' = f(x')\in\mathrm{im}(f)$,同样利用同态的性质,$yy'^{-1}=f(x)f(x')^{-1}=f(xx'^{-1})\in\mathrm{im}(f)$。故$f(G)$是$G'$的子群.

\item 设$x,y \in \ker f$,则有$f(x - y) = 0$,故$x - y \in \ker f$.又显然有$\ker f$对乘法封闭,因此$\ker f$是$R$的子环.又设$a \in R$,则$f(ax) = f(a)f(x) = 0$,$f(xa) = f(x)f(a) = 0$,即$ax, xa \in \ker f$,故$\ker f$为$R$的理想.

由\rrefpro{proposition:群同态的核是定义域的子群,像是陪域的子群-抽象代数}{proposition:群同态的核是定义域的子群,像是陪域的子群-抽象代数-1}知$f(R)$构成$R'$的加法子群.由$R$对加法构成Abel群知$f(R)$对加法也构成Abel群.由同态的性质易知$f$对乘法构成半群,故$f(R)$是$R'$的子环.

\item 对$\forall x,y\in \ker f,$由$f$是摸同态知$f(x-y)=f(x)-f(y)=0$.从而$x-y\in \ker f$,于是$\ker f = N$是加法群$M$的子群,设$a \in R$,$x \in N$,则$f(ax) = af(x) = 0$,因而$ax \in N$,故$N$是$M$的子模. 


\end{enumerate}

\end{proof}

\begin{proposition}\label{proposition:自然同态的ker等于商掉的正规子群}
\begin{enumerate}[(1)]
\item 设$H$是群$G$的正规子群. $\pi$是$G$到商群$G/H$的自然同态(见\refthe{theorem:抽象代数--1.5.}),则有$\ker\pi = H$.

\item 设$I$是环$R$的理想,$\pi$是$R$到商环$R/I$的自然同态(见\refthe{theorem:抽象代数--1.5.7}),则有$\ker\pi = I$.

\item 设$N$是$R$模$M$的子模,$\pi$是$M$到商模$M/N$的自然同态(见\refthe{theorem:自然模同态的良定义}),则有$\ker\pi = N$.
\end{enumerate}
\end{proposition}
\begin{proof}
\begin{enumerate}[(1)]
\item 

\item 

\item 
\end{enumerate}

\end{proof}

\begin{proposition}\label{proposition:单同态的充要条件是核为平凡零子空间}
\begin{enumerate}[(1)]
\item 设$f$是群$G_1$到群$G_2$的同态,$G_1$的幺元是$e_1$,则$f$是单同态的充要条件是$\ker f=\{e_1\}$.

\item 设$f$是环$R_1$到环$R_2$的同态,则$f$是单同态的充要条件是$\ker f=\{0\}$.

\item  设$R$是一个环,$M_1, M_2$都是$R$模,$f$是$M_1$到$M_2$的模同态,则$f$是单同态的充要条件是$\ker f=\{0\}$.
\end{enumerate}
\end{proposition}
\begin{proof}
\begin{enumerate}[(1)]
\item 设$G_2$的幺元为$e_2$.

{\heiti 必要性:}设$f$为单同态,则对任意的$x\in\ker f$,有$f(x)=e_2=f(e_1)$。
因为$f$为单同态,所以$x=e$,于是$\ker f=\{e\}$。

{\heiti 充分性:}设$x,y\in G_1$,如果$f(x)=f(y)$,则$f(xy^{-1})=f(x)(f(y))^{-1}=e_2$,从而
$xy^{-1}\in\ker f$。而$\ker f=\{e\}$,所以$xy^{-1}=e$,因此$x=y$,从而$f$为单同态。

\item 

\item 
\end{enumerate}

\end{proof}

\begin{theorem}[群的同态基本定理]\label{theorem:群的同态基本定理}
设$f$是群$G$到群$H$上的同态,$\pi$为$G$到商群$G/\ker f$上的自然同态,则有$G/\ker f$到$f(G)$上的群同构映射$\overline{f}$,使得
\begin{align}
f = \overline{f} \cdot \pi, \label{eq:1.gdgp9343f43dc3237.123r3te4g45y4h5h56}
\end{align}
进而
\begin{align*}
G\backslash \mathrm{ker}\,f\cong  f(G).
\end{align*}
如\reffig{figure:图1.7}所示.
\end{theorem}
\begin{note}
\begin{figure}[H]
\centering
% https://q.uiver.app/#q=WzAsMyxbMCwwLCJHIl0sWzIsMCwiR1xcYmFja3NsYXNoIFxcbWF0aHJte2tlcn1mIl0sWzAsMiwiZihHKSJdLFswLDEsIlxccGkiXSxbMCwyLCJmIiwyXSxbMSwyLCJ7XFxvdmVybGluZXtmfX0iXV0=
\begin{tikzcd}
G && {G\backslash \mathrm{ker}f} \\
\\
{f(G)}
\arrow["\pi", from=1-1, to=1-3]
\arrow["f"', from=1-1, to=3-1]
\arrow["{{\overline{f}}}", from=1-3, to=3-1]
\end{tikzcd}
\caption{}
\label{figure:图1.7}
\end{figure}
\end{note}
\begin{proof}
由\refpro{proposition:群同态的核是定义域的子群,像是陪域的子群-抽象代数}知$f(G)$是$G$的子群.注意到$f$是$G$到$f(G)$上的满映射,故由\refthe{Set Theory-theorem:抽象代数--定理1.1.2}知$f$在$G$中诱导一个等价关系
\begin{align*}
R: xRy, \quad x, y \in G,
\end{align*}
当且仅当$f(x) = f(y)$,即
\begin{align*}
f(x)=f(y) \iff f(x)^{-1}f(y) = f(x^{-1}y) = e'\iff x^{-1}y \in \ker f .
\end{align*}
因而$f$诱导的等价关系恰好是$G$的正规子群$\ker f$诱导的同余关系,即有商群$G/R = G/\ker f$且
\begin{align*}
\pi(x) = \pi(y) \text{ 当且仅当 } f(x) = f(y).
\end{align*}
又由\refthe{Set Theory-theorem:抽象代数--定理1.1.2}知有$G/\ker f$到$f(G)$的一一对应$\overline{f}$,使得$\overline{f} \cdot \pi = f$,又$\forall x, y \in G$有
\begin{align*}
\overline{f}(\pi(x)\pi(y)) = \overline{f}(\pi(xy)) = f(xy) = f(x)f(y) = \overline{f}(\pi(x)) \cdot \overline{f}(\pi(y)).
\end{align*}
由此知$\overline{f}$是$G/\ker f$到$f(G)$上的群同构.


\end{proof}

\begin{theorem}\label{theorem:抽象代数-定理1.7.2}
设$f$是群$G$到群$H$上的满同态,$f$的核为$K$,即$K = \ker f$,$G$中包含$K$的子群的集合为$\Sigma$,$H$的子群的集合为$\Gamma$,则有下列结论:
\begin{enumerate}[(1)]
\item\label{theorem:抽象代数-定理1.7.2-1} $f$是$\Sigma \to \Gamma$的一一对应;

\item\label{theorem:抽象代数-定理1.7.2-2} 若$G_1 \lhd G$,$G_1 \supseteq K$,则$$f(G_1)\lhd H.$$
若$H_1 \lhd H$,则$$f^{-1}(H_1)\lhd G.$$

\item\label{theorem:抽象代数-定理1.7.2-3} 若$G_1 \lhd G$,$G_1 \supseteq K$,则
\begin{align}
K\lhd G_1\lhd G,\quad G/G_1 \cong H/f(G_1).\label{eq:1.gdgp9343f43dc3237.2}
\end{align}
\end{enumerate}
\end{theorem}
\begin{proof}
\begin{enumerate}[(1)]
\item 对$\forall G_1 \in \Sigma$,由$f(G_1)$是$G_1$在$f|_{G_1}$下的像,又$f$是群同态,故$f(G_1)$为$H$的子群,即$f(G_1) \in \Gamma$. 由此知$f$是$\Sigma$到$\Gamma$的良定义的映射.
设$H_1 \in \Gamma$,$H_1$在$f$下原像的集合
\begin{align*}
G_1 = f^{-1}(H_1) = \{x \in G|f(x) \in H_1\}\supseteq \{x \in G|f(x) = e',\ e'\text{ 为 }H\text{ 的幺元}\} = K,
\end{align*}
而且对$\forall x,y \in G_1$,$f(xy^{-1}) = f(x)f(y)^{-1} \in H_1$,故$xy^{-1} \in G_1$,因而$G_1$为$G$的子群,故$G_1 \in \Sigma$,因此$f^{-1}$可视为$\Gamma$到$\Sigma$的良定义的映射.

由$f$是$G\to H$上的满同态知$f(G_1)=f(f^{-1}(H_1))=H_1$,由$H_1$的任意性知$ff^{-1}=\mathrm{id}_{\Gamma}$.
反之,设$G_1 \in \Sigma$,显然有$G_1 \subseteq f^{-1}(f(G_1))$. 若$u \in f^{-1}(f(G_1))$,即有$v \in G_1$,使得$f(u) = f(v)$,从而
$$
f(uv^{-1})=f(u)f(v)^{-1}=e'.
$$
因而$uv^{-1} \in K \subseteq G_1$,故$u \in G_1$,即有$f^{-1}(f(G_1)) = G_1$,由$G_1$的任意性知$f^{-1}f = \text{id}_\Sigma$.

综上所述知$f$是$\Sigma\to \Gamma$的一一对应,$f^{-1}$是其逆映射.故结论(1)成立.

\item 设$G_1 \supset K$且$G_1 \lhd G$,即$G_1\in \Sigma$且$G_1\lhd G$,则由(1)可知$f(G_1)$是$H$的子群.对$\forall g\in f(G_1),y\in H$,因为$f$是满同态,所以存在$a \in G_1$,$x \in G$,使得$f(a)=g,f(x)=y$.从而
\begin{align*}
ygy^{-1}=f(x)f(a)f(x)^{-1} = f(xax^{-1}) \in f(G_1).
\end{align*}
故知$f(G_1) \lhd H$.

反之,若$H_1 \lhd H$且对$\forall b \in f^{-1}(H_1)$,$y \in G$,由
\begin{align*}
f(yby^{-1}) = f(y)f(b)f(y)^{-1} \in H_1
\end{align*}
知$yby^{-1} \in f^{-1}(H_1)$,故知$f^{-1}(H_1) \lhd G$,即结论(2)成立.

\item 由\rrefpro{proposition:群同态的核是定义域的子群,像是陪域的子群-抽象代数}{proposition:群同态的核是定义域的子群,像是陪域的子群-抽象代数-1}知$\ker f\lhd G_1$,再由\rrefpro{proposition:正规子群的基本性质}{proposition:正规子群的基本性质-2}知$K\lhd G_1$,再结合条件可得
\begin{align*}
K\lhd G_1\lhd G.
\end{align*}
由结论(2)的证明知$f(G_1) \lhd H$. 令$\pi'$是$H$到商群$H/f(G_1)$的自然同态,由此可知有$G$到$H/f(G_1)$上的同态映射$\pi' \cdot f$,注意到$H/f(G_1)$的幺元为$f(G_1)$,则知
\begin{align*}
\ker(\pi'f) &= \{x \in G|\pi' f(x) = f(G_1)\}\\
&= \{x \in G|f(x) \in f(G_1)\}\\
&= f^{-1}(f(G_1)) = G_1.
\end{align*}
最后一个等号是因为由(1)知$f$是$\Sigma \to \Gamma$的一一对应.设$\pi$为$G$到$G/G_1$的自然同态,又因为自然同态$\pi'$是满同态且$f$也是满同态,所以由\hyperref[theorem:群的同态基本定理]{群的同态基本定理}知有$G/G_1$到$H/f(G_1)$的群同构$\overline{f}$,使得$\pi'f=\overline{f}\cdot \pi$,亦使\reffig{figure:抽象代数--图1.7.2}为交换图,即式\eqref{eq:1.gdgp9343f43dc3237.2}成立.
\begin{figure}[H]
\centering
\begin{tikzcd}
G && H \\
\\
{G/G_1} && {H/f(G_1)}
\arrow["f", from=1-1, to=1-3]
\arrow["\pi"', from=1-1, to=3-1]
\arrow["{\pi'f}", from=1-1, to=3-3]
\arrow["{\pi'}", from=1-3, to=3-3]
\arrow["{\overline{f}}"', from=3-1, to=3-3]
\end{tikzcd}
\caption{}
\label{figure:抽象代数--图1.7.2}
\end{figure}
\end{enumerate}
\end{proof}

\begin{corollary}\label{corollary:群同态第二定理推论}
设$N$为群$G$的正规子群,$\pi$为$G$到商群$G/N$上的自然同态,$G$中包含$N$的子群的集合为$\Sigma$,$G/N$的子群的集合为$\Gamma$,则
\begin{enumerate}[(1)]
\item\label{corollary:群同态第二定理推论-1} $\pi$是$\Sigma \to \Gamma$的一一对应;

\item\label{corollary:群同态第二定理推论-2} 若$H \lhd G$,$H \supseteq N$,则$$
\pi(H)=H/N\lhd G/N.
$$
若$H' \lhd G/N$,则
$$
\pi^{-1}(H')\lhd G.
$$

\item\label{corollary:群同态第二定理推论-3} 若$H \lhd G$,$H \supseteq N$,则
\begin{align*}
N\lhd H\lhd G,\quad G/H \cong (G/N)/(H/N). 
\end{align*}
\end{enumerate}
\end{corollary}
\begin{proof}
事实上,由于自然同态必是满同态,故只要在\refthe{theorem:抽象代数-定理1.7.2}中将$H$换成$G/N$,$f$换成$\pi$,即得本推论。对于(3),由\rrefpro{proposition:群的自然映射的基本性质}{proposition:群的自然映射的基本性质-1}知$\pi (H)=H/N$,故由\refthe{theorem:抽象代数-定理1.7.2}可得
\begin{align*}
N\lhd H\lhd G,
\end{align*}
以及如下交换图.
\begin{figure}[H]
\centering
\begin{tikzcd}
G && {G/N} \\
\\
{G/H} && {(G/N)/(H/N)}
\arrow["\pi", from=1-1, to=1-3]
\arrow["{\pi''}"', from=1-1, to=3-1]
\arrow["{\pi'\pi}", from=1-1, to=3-3]
\arrow["{\pi'}", from=1-3, to=3-3]
\arrow["{\overline{\pi}}"', from=3-1, to=3-3]
\end{tikzcd}
\caption{}
\label{figure:抽象代数--图1.7.gq34gaewe2}
\end{figure}
\end{proof}

\begin{theorem}\label{theorem:抽象代数--定理1.7.3}
设$N$是群$G$的正规子群,$\pi$是$G$到商群$G/N$上的自然同态,$H$是$G$的一个子群,则有下列结论:
\begin{enumerate}[(1)]
\item\label{theorem:抽象代数--定理1.7.3-1} $HN$是$G$中包含$N$的子群且
\begin{align}
N\lhd HN = \pi^{-1}(\pi(H)). \label{eq:::fi0ejw89f38j843g2fw2f4}
\end{align}

\item\label{theorem:抽象代数--定理1.7.3-2} $H \cap N \lhd H$且$H \cap N = \ker(\pi|_H)$,$\pi|_H$表示$\pi$在$H$上的限制;

\item\label{theorem:抽象代数--定理1.7.3-3}
\begin{align*}
HN/N \cong H/(H \cap N ).
\end{align*}
\end{enumerate}
\end{theorem}
\begin{proof}
\begin{enumerate}[(1)]
\item 显然,$HN \supseteq N$.设$h_in_i \in HN$($i=1,2$),则由$N\lhd G$有
\begin{align*}
h_1n_1(h_2n_2)^{-1} = h_1h_2^{-1}(h_2(n_1n_2^{-1})h_2^{-1}) \in HN.
\end{align*}
故$HN$是$G$中含$N$的子群且$\pi(h_1n_1) = \pi(h_1)\pi(n_1) = \pi(h_1)\in \pi(H)$,故$HN \subseteq \pi^{-1}(\pi(H))$. 

又设$x\in \pi^{-1}(\pi(H))$,则$\pi(x)\in \pi(H),$从而存在$h\in H$,使得
$$
\pi(x)=\pi(h)\iff xN=hN\iff x^{-1}h\in N.
$$
于是存在$n\in N$,使得$x^{-1}h=n$.故$x=hn^{-1}\in HN.$因此$\pi^{-1}(\pi(H))\subseteq HN.$
综上可知$HN = \pi^{-1}(\pi(H))$.因为$H$是$G$的包含$N$的子群且$N\lhd G$,所以由\rrefpro{proposition:正规子群的基本性质}{proposition:正规子群的基本性质-2}知$N\lhd HN$.

\item 由于$N \lhd G$,对$\forall h \in H$,$a \in N \cap H$有$hah^{-1} \in N \cap H$,故$N \cap H \lhd H$. 又$\pi|_H(h) = \pi(h)$且$\ker \pi=N$,于是$\ker(\pi|_H) = H \cap N$.

\item 由(1)的结论知$HN=\pi^{-1}(\pi(H))$,再由自然同态是满同态知
\begin{align*}
\pi(HN) = \pi(\pi^{-1}(\pi(H)))=\pi(H).
\end{align*}
由\hyperref[theorem:群的同态基本定理]{群的同态基本定理}知
\begin{align*}
HN/\mathrm{ker}\pi |_{HN}\cong \pi (HN)=\pi \left( H \right) \cong H/\mathrm{ker}\pi |_H.
\end{align*}
又注意到$\ker(\pi|_{HN}) = HN \cap N = N$,$\ker \pi|_{H}=H\cap N$,故
\begin{align*}
HN/N \cong H/(H \cap N ).
\end{align*}
\end{enumerate}

\end{proof}

\begin{theorem}[环的同态基本定理]\label{theorem:环的同态基本定理}
设$f$是环$R$到环$R'$上的同态,$\pi$是$R$到商环$R/\ker f$上的自然同态,则有$R/\ker f$到$f(R)$上的环同构映射$\overline{f}$,使得
\begin{align}
f = \overline{f} \cdot \pi. \label{eq:1ger4wg34wgg54.7.g890u34349n8v4b6}
\end{align}
即
\begin{align*}
R/\ker f \cong f(R).
\end{align*}
\end{theorem}
\begin{proof}
由\refpro{proposition:群同态的核是定义域的子群,像是陪域的子群-抽象代数}知$f(R)$是$R'$的子环.又$f$为环同态,故也是加法群$R$到加法群$f(R)$上的同态,$\pi$也是加法群$R$到商群$R/\ker f$上的自然同态,于是由\hyperref[theorem:群的同态基本定理]{群的同态基本定理}知有加法群$R/\ker f$到加法群$f(R)$上的同构$\overline{f}$,使$f = \overline{f} \cdot \pi$.

另外,$\forall a,b \in R$有
\begin{align*}
\overline{f}(\pi(a)\pi(b)) &= \overline{f}(\pi(ab)) = f(ab) = f(a)f(b) \\
&= \overline{f}(\pi(a))\overline{f}(\pi(b)),
\end{align*}
因而$\overline{f}$也是环$R/\ker f$到环$f(R)$上的环同构.


\end{proof}

\begin{theorem}\label{theorem:抽象代数--定理1.7.5}
设$f$是环$R$到环$R'$上的满同态,又$K = \ker f$,$R$中包含$K$的子环集合为$\Sigma$,$R'$的子环集合为$\Gamma$,则有下列结论:
\begin{enumerate}[(1)]
\item\label{theorem:抽象代数--定理1.7.5-1} $f$是$\Sigma\to \Gamma$的一一对应;

\item\label{theorem:抽象代数--定理1.7.5-2} 若$H$为$R$的理想且$H\supseteq K$,则$f(H)$为$R'$的理想;

若$H'$为$R'$的理想,则$f^{-1}(H')$为$R$的理想;

\item\label{theorem:抽象代数--定理1.7.5-3} 若$I$是$R$的理想且$I \supseteq K$,则
\begin{align}
R/I \cong R'/f(I) .\label{eq:1ger4wg34wgg54.7.g890u34349n8v4b7}
\end{align}
\end{enumerate}
\end{theorem}
\begin{proof}
\begin{enumerate}[(1)]
\item 设$H$为$R$的子环且$H \supseteq K$,由\hyperref[proposition:环同态的基本性质]{环同态的基本性质\ref{proposition:环同态的基本性质-1}}知$f(H)$为$R'$的子环. 故$f$是$\Sigma\to \Gamma$上的良定义的映射.
反之,若$H'$为$R'$的子环,则$H'$也是$R'$的加法子群,由\rrefthe{theorem:抽象代数-定理1.7.2}{theorem:抽象代数-定理1.7.2-1}知$f$建立了加法群$R$中包含$K$的子群与加法群$R'$的子群间的一一对应,故$f^{-1}(H')$是$R$中唯一包含$K$的加法子群. 又若$a,b \in f^{-1}(H')$,则有$f(ab) = f(a)f(b) \in H'$,即$ab \in f^{-1}(H')$,故$f^{-1}(H')$对乘法构成半群.再设$c\in f^{-1}(H'),$则
\begin{align*}
f((a+b)c)=f(a+b)f(c)=f(a)f(c)+f(b)f(c)\in H',
\\
f(c(a+b))=f(c)f(a+b)=f(c)f(a)+f(c)f(b)\in H'.
\end{align*}
因而$f^{-1}(H')$是$R$中包含$K$的子环,故$f^{-1}$可视为$\Gamma\to \Sigma$上的良定义的映射.

对$\forall H\in \Sigma,H'\in \Gamma$,注意到$H$也是$R$中包含$K$的加法子群,$H'$也是$R'$的加法子群,由\rrefthe{theorem:抽象代数-定理1.7.2}{theorem:抽象代数-定理1.7.2-1}知$f^{-1}f(H)=H,ff^{-1}(H')=H'.$由$H$的任意性知$f^{-1}f=\mathrm{id}_{\Sigma},ff^{-1}=\mathrm{id}_{\Gamma}.$故$f$是$\Sigma\to \Gamma$的一一对应,$f^{-1}$是其逆映射.即结论(1)成立.

\item 对$\forall a',b' \in R',h \in H$,由环同态都是满同态知存在$a,b\in R$,使得$f(a)=a',f(b)=b'$.于是再由$H$是$R$的理想知
\begin{align*}
a'f(a)b'=f(a)f(h)f(b) = f(ahb) \in f(H).
\end{align*}
故$f(H)$为$R'$的理想. 

反之,设$H'$为$R'$的理想. 对$\forall b \in R$,$x \in f^{-1}(H')$,由$H'$是$R'$的理想知
\begin{align*}
f(bx) = f(b)f(x) \in H',f(xb) = f(x)f(b) \in H'.
\end{align*}
即$bx, xb \in f^{-1}(H')$,故$f^{-1}(H')$为$R$的理想. 由此知结论(2)成立.

\item 设$\pi$是$R$到$R/I$的自然同态,$\pi'$是$R'$到$R'/f(I)$的自然同态. 由\rrefpro{proposition:环同态的基本性质}{proposition:环同态的基本性质-2}知$\pi'f$是$R$到$R'/f(I)$上的环同态.注意到
\begin{align*}
\ker(\pi' f) &= \{ x \in R : \pi' f(x) = f(I) \} \\
&= \{ x \in R : f(x) \in f(I) \} \\
&= f^{-1}(f(I)) = I.
\end{align*}
最后一个等号是因为由(1)知$f$是$\Sigma \rightarrow \Gamma$的一一对应.
于是由\hyperref[theorem:环的同态基本定理]{环的同态基本定理}得式\eqref{eq:1ger4wg34wgg54.7.g890u34349n8v4b7}成立.
\end{enumerate}
\end{proof}

\begin{corollary}\label{corollary:抽象代数--推论1.7.2}
设$A,B$均为环$R$的理想且$A \subseteq B$,则有$B/A$是$R/A$的理想且
\begin{align*}
R/B \cong (R/A)/(B/A). 
\end{align*}
\end{corollary}
\begin{proof}
事实上,只要在\refthe{theorem:抽象代数--定理1.7.5}中取$R' = R/A$,$f$为$R$到$R/A$的自然同态,并且由\rrefthe{proposition:环的自然同态的基本性质}{proposition:环的自然同态的基本性质-1}知$f(B)=B/A$,再由\rrefthe{theorem:抽象代数--定理1.7.5}{theorem:抽象代数--定理1.7.5-2}知$f(B)=B/A$是$R'=R/A$的理想.因此即得本推论.

\end{proof}

\begin{theorem}\label{theorem:抽象代数--定理1.7.6}
设$H$为环$R$的子环,$K$为$R$的理想,$\pi$是环$R$到商环$R/K$上的自然同态,则有
\begin{enumerate}[(1)]
\item\label{theorem:抽象代数--定理1.7.6-1} $H + K$为$R$中包含$K$的子环,$K$是$H+K$的理想,并且
\begin{align*}
H+K=\pi^{-1}(\pi(H)).
\end{align*}

\item\label{theorem:抽象代数--定理1.7.6-2} $H \cap K$为$H$的理想且$H\cap K=\ker \pi|_{H}.$

\item\label{theorem:抽象代数--定理1.7.6-3} \begin{align}
(H + K)/K \cong H/(H \cap K). \label{eq:1ger4wg34wgg54.7.g890u34349n8v4b9}
\end{align}
\end{enumerate}
\end{theorem}
\begin{proof}
\begin{enumerate}[(1)]
\item 显然$H+K \supseteq K$. 设$h_i + k_i \in H+K$($i=1,2$),$r \in R$,则$(h_1 + k_1) - (h_2 + k_2) = h_1 - h_2 + k_1 - k_2 \in H+K$.
于是$H+K$是$R$的加法子群. 由$H+K \subseteq R$知$H+K$对乘法满足结合律且加法与乘法间满足左右分配律. 故$H+K$是$R$中含$K$的子环. 又注意到
$\pi(h_1 + k_1) = h_1 + k_1 + K = h_1 + K \in \pi(H)$.
故$h_1 + k_1 \in \pi^{-1}(\pi(H))$,因此$H+K \subseteq \pi^{-1}(\pi(H))$.

反之,设$x \in \pi^{-1}(\pi(H))$,则$\pi(x) \in \pi(H)$. 从而存在$h' \in H$,使得
$\pi(x) = \pi(h') \Longleftrightarrow x + K = h' + K \Longleftrightarrow -x + h' \in K$.
于是存在$k' \in K$,使得$-x + h' = k'$,从而$x = h' - k' \in H+K$. 故$\pi^{-1}(\pi(H)) \subseteq H+K$. 综上可知$H+K = \pi^{-1}(\pi(H))$.

因为$H$为环$R$的子环,$K$为$R$的理想且$H+K\supseteq K$,所以由\rrefthe{theorem:子环和理想的基本性质}{theorem:子环和理想的基本性质-2}知$K$是$H+K$的理想.

\item 由$H,K$都是$R$的子环知$H\cap K$是$R$的子环. 又因为$H\supseteq H\cap K$, 所以$H\cap K$也是$H$的子环. 对$\forall x\in H\cap K,h\in H$, 由$K$是$R$的理想知$hx,xh\in H\cap K$. 故$H\cap K$是$H$的理想.
又$\pi|_H(h) = \pi(h)$且$\ker\pi = K$, 故$\ker\pi|_H = H\cap K.$

\item 由结论(1)知$H+K=\pi^{-1}(\pi(H))$,再由自然同态都是满同态知
\begin{align*}
\pi(H+K) = \pi(\pi^{-1}(\pi(H))) = \pi(H).
\end{align*}
于是由\hyperref[theorem:环的同态基本定理]{环的同态基本定理}知
\begin{align*}
(H+K)/\ker\pi|_{H+K} \cong \pi(H+K) = \pi(H) \cong H/\ker\pi|_H.
\end{align*}
注意到$\ker\pi|_{H+K} = (H+K)\cap K = K$,$\ker\pi|_H = H\cap K$,故
\begin{align*}
(H+K)/K \cong H/(H\cap K).
\end{align*}
\end{enumerate}

\end{proof}

\begin{theorem}[模同态的基本定理]\label{theorem:模同态的基本定理}
设$M, M'$都是幺环$R$上的模,$f$是模$M$到模$M'$上的同态,$M$中包含$N$的子模集合为$\Sigma$,$M'$中子模集合为$\Gamma$,$\pi$是$M$到$M/N$上的自然模同态,则有$M/N$到$f(M)$的模同构$\overline{f}$,使得
\begin{align}
\overline{f} \cdot \pi = f \label{eq::::--928fh3489h4h3t4g1.7.10}
\end{align}
即
\begin{align*}
M/N \cong f(M).
\end{align*}
\end{theorem}
\begin{proof}
由\refpro{proposition:群同态的核是定义域的子群,像是陪域的子群-抽象代数}知$f(M)$是$M'$的子模.由\hyperref[theorem:群的同态基本定理]{群的同态基本定理}知有加法群$M/N$到加法群$f(M)$上的同构$\overline{f}$,使$\overline{f} \cdot \pi = f$. 现只需证$\overline{f}$是模同构. 又设$a \in R$,$x \in M$,于是有
\begin{align*}
\overline{f}(a\pi(x)) = \overline{f}(\pi(ax)) = f(ax) = af(x) = a\overline{f}(\pi(x)),
\end{align*}
即$\overline{f}$为模同构. 

\end{proof}

\begin{theorem}\label{theorem:模同态的基本定理推论}
设$M, M'$都是幺环$R$上的模,$f$是模$M$到模$M'$上的满同态,$M$中包含$N$的子模集合为$\Sigma$,$M'$中子模集合为$\Gamma$,则有下面结论:
\begin{enumerate}[(1)]
\item\label{theorem:模同态的基本定理推论-1} $f$是$\Sigma\to \Gamma$的一一对应. 

\item\label{theorem:模同态的基本定理-3} 若$M_1$是$M$的子模且$M_1 \supseteq N$,则
\begin{align}
M/M_1 \cong M'/f(M_1) \label{eq::::--928fh3489h4h3t4g1.7.11}
\end{align}
\end{enumerate}
\end{theorem}
\begin{proof}
\begin{enumerate}[(1)]
\item 若$M_1$为$M$的子模,则由\rrefthe{theorem:摸同态的基本性质}{theorem:摸同态的基本性质-1}知$f(M_1)$为$M'$的子模.故$f$是$\Sigma\to \Gamma$上的良定义的映射.

反之,若$M_1'$为$M'$的子模,则$M_1'$也是$M'$的加法子群.从而由\rrefthe{theorem:抽象代数-定理1.7.2}{theorem:抽象代数-定理1.7.2-1}知$f^{-1}(M_1')$是$M$中唯一包含$N$的加法子群. 又设$a \in R$,$x \in f^{-1}(M_1')$. 由$f(ax) = af(x) \in M_1'$知$ax \in f^{-1}(M_1')$,即$f^{-1}(M')$是$M$的子模.故$f^{-1}$可视为$\Gamma\to \Sigma$上的良定义的映射. 

对$\forall H\in \Sigma,H'\in \Gamma$,注意到$H$也是$R$中包含$K$的加法子群,$H'$也是$R'$的加法子群,由\rrefthe{theorem:抽象代数-定理1.7.2}{theorem:抽象代数-定理1.7.2-1}知$f^{-1}f(H)=H,ff^{-1}(H')=H'.$由$H$的任意性知$f^{-1}f=\mathrm{id}_{\Sigma},ff^{-1}=\mathrm{id}_{\Gamma}.$故$f$是$\Sigma\to \Gamma$的一一对应,$f^{-1}$是其逆映射,即结论(1)成立.

\item 设$M_1$为$M$的子模且$M_1 \supseteq N$. 又设$\pi_1$是$M$到$M/M_1$的自然同态,$\pi'$是$M'$到$M'/f(M_1)$的自然同态. 于是$\pi' f$是$M$到$M'/f(M_1)$上的同态,而且
\begin{align*}
\ker(\pi' f) &= \{ x \in R : \pi' f(x) = f(M_1) \} \\
&= \{ x \in R : f(x) \in f(M_1) \} \\
&= f^{-1}(f(M_1)) =  M_1.
\end{align*}
最后一个等号是因为由结论(1)知$f$是$\Sigma \rightarrow \Gamma$的一一对应.故由\hyperref[theorem:模同态的基本定理]{模同态的基本定理}可知式\eqref{eq::::--928fh3489h4h3t4g1.7.11}成立.
\end{enumerate}

\end{proof}

\begin{corollary}
设$M_1, N$都是$R$模$M$的子模,而且$M_1 \supseteq N$,则有模同构\begin{align*}
M/M_1\cong (M/N)/(M_1/N).
\end{align*}
\end{corollary}
\begin{proof}
事实上,只要在\hyperref[theorem:模同态的基本定理推论]{定理\ref{theorem:模同态的基本定理推论}\ref{theorem:模同态的基本定理-3}}中取$M' = M/N$,$f$为$M$到$M' = M/N$的自然同态,再由\rrefpro{proposition:自然模同态的基本性质}{proposition:自然模同态的基本性质-1}知$f(M_1)=M_1/N$,即得本推论.

\end{proof}

\begin{theorem}
设$H, N$为$R$模$M$的子模,则有模同构
\begin{align}
(H + N)/N \cong H/(H \cap N) \label{eq::::--928fh3489h4h3t4g1.7.12}
\end{align}
\end{theorem}

\begin{proof}
设$\pi$为模$M$到商模$M/N$的自然模同态,由于$N$为商群$M/N$中的加法幺元,即商模$M/N$中的零元,于是有$\pi(H + N) =\pi(H)+N= \pi(H)$,因而由\rrefpro{proposition:群同态的核是定义域的子群,像是陪域的子群-抽象代数}{proposition:群同态的核是定义域的子群,像是陪域的子群-抽象代数-3}知
\begin{align*}
H+N/\mathrm{ker(}\pi |_{H+N})\cong \pi \left( H+N \right) =\pi \left( H \right) \cong H/\mathrm{ker(}\pi |_H).
\end{align*}
由$\ker(\pi|_{H + N}) = (H + N)\cap N =N$,$\ker(\pi|_H) = H \cap N$,即得式\eqref{eq::::--928fh3489h4h3t4g1.7.12}成立.

\end{proof}













\end{document}
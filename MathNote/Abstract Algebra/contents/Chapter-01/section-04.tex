\documentclass[../../main.tex]{subfiles}
\graphicspath{{\subfix{../../image/}}} % 指定图片目录,后续可以直接使用图片文件名。

% 例如:
% \begin{figure}[H]
% \centering
% \includegraphics[scale=0.4]{图.png}
% \caption{}
% \label{figure:图}
% \end{figure}
% 注意:上述\label{}一定要放在\caption{}之后,否则引用图片序号会只会显示??.

\begin{document}

\section{环与域}

\begin{definition}[环]
若在非空集合 \( R \) 中定义了加法和乘法两种二元运算, 并满足下列条件:
\begin{enumerate}[(1)]
\item \( R \) 对加法为 Abel 群;
\item \( R \) 对乘法为半群;
\item 加法与乘法间有分配律, 即 \( \forall a,b,c \in R \),
\[
a(b + c) = ab + ac, \quad (b + c)a = ba + ca,
\]
\end{enumerate}
则称 \( R \) 是一个\textbf{环}.
\end{definition}

\begin{proposition}
一切数域都是环.
\end{proposition}
\begin{proof}


\end{proof}

\begin{example}
\begin{enumerate}[(1)]
\item \( \mathbf{Z} \) 对加法与乘法是环, 称为\textbf{整数环}.

\item 数域 \( P \) 上的 \( n \) 元多项式集合 \( P[x_1,x_2,\cdots,x_n] \) 对多项式的加法和乘法是环, 称为 \( P \) 上的 \( \boldsymbol{n} \) \textbf{元多项式环}. 

\item \( R^{n \times n} \) 表示以环 \( R \) 中元素为矩阵元的 \( n \) 阶方阵的集合, 即 \( \alpha \in R^{n \times n} \) 可写成
\[
\alpha = \begin{pmatrix}
a_{11} & a_{12} & \cdots & a_{1n} \\
a_{21} & a_{22} & \cdots & a_{2n} \\
\vdots & \vdots & & \vdots \\
a_{n1} & a_{n2} & \cdots & a_{nn}
\end{pmatrix}, \quad a_{ij} \in R.
\]
记 \( a_{ij} = \text{ent}_{ij}(\alpha) \). 由下面的两个关系:
\begin{enumerate}[(i)]
\item \( \text{ent}_{ij}(\alpha + \beta) = \text{ent}_{ij}(\alpha) + \text{ent}_{ij}(\beta) \);
\item \( \text{ent}_{ij}(\alpha\beta) = \sum_{k=1}^n \text{ent}_{ik}(\alpha)\text{ent}_{kj}(\beta) \)
\end{enumerate}
定义的 \( R^{n \times n} \) 加法与乘法使其成为一个环, 称为 \( R \) 上的 \(\boldsymbol{n}\)\textbf{阶方阵环}.

\item 设 \( C([a,b]) \) 是闭区间 \([a,b]\) 上的连续函数的集合, 它对函数的加法与乘法是一个环, 称为 \([a,b]\) 上的\textbf{连续函数环}.

\item 设 \( A \) 是一个 Abel 群, \( A \) 的运算是加法. 在 \( A \) 中定义乘法运算为 \( ab = 0 (\forall a,b \in A) \), 则 \( A \) 为一环, 这种环称为\textbf{零环}.
\end{enumerate}
\end{example}
\begin{remark}
(5)说明, 任何 Abel 群均可作为零环的加法群, 但是并非所有 Abel 群都可成为非零环的加法群.
\end{remark}
\begin{proof}


\end{proof}

\begin{theorem}[环的基本性质]\label{theorem:环的基本性质}
\begin{enumerate}[(1)]
\item 在环 \( R \) 中可定义\hyperref[definition:倍数的定义]{任何整数的倍数}及\hyperref[definition:n次乘幂]{正整数次乘幂}, 并且满足
\begin{enumerate}[(i)]
\item \( \forall m,n \in \mathbf{Z}, a,b \in R \),
\[
(m + n)a = ma + na,
\]
\[
(mn)a = m(na),
\]
\[
m(a + b) = ma + mb;
\]
\item \( a^m \cdot a^n = a^{m+n}, (a^m)^n = a^{mn}, \forall m,n \in \mathbf{N}, a \in R \);
\item 若 \( a,b \in R \) 且 \( ab = ba \), 则 \( (ab)^m = a^m b^m, \forall m \in \mathbf{N} \).
\end{enumerate}

\item 由分配律成立有
\[
\sum_{i=1}^m \sum_{j=1}^n a_i b_j = \sum_{j=1}^n \sum_{i=1}^m a_i b_j.
\]

\item \( \forall a,b \in R \) 有 \( a0 = 0a = 0, (-a)b = a(-b) = -ab, (-a)(-b) = ab \).
\end{enumerate}
\end{theorem}
\begin{proof}
\begin{enumerate}[(1)]
\item 

\item 

\item 事实上, 由 \( a \cdot 0 + ab = a(0 + b) = ab \) 知 \( a \cdot 0 = 0 \). 同样 \( 0 \cdot a = 0, a(-b) = a(-b) + ab + (-ab) = -ab \). 最后 \( (-a)(-b) = -(a(-b)) = -(-ab) = ab \).
\end{enumerate}

\end{proof}

\begin{definition}
\begin{enumerate}
\item \textbf{交换环:}乘法是交换半群的环.

\item \textbf{幺环:}乘法是幺半群的环, 通常记幺元为 \( 1 \).

\item \textbf{交换幺环:}乘法是交换幺半群的环.

\item \textbf{无零因子环:}任意两个非零元的积不为零的环.

\item 设 \( R \) 是环. \( a,b \in R \) 且 \( a \neq 0, b \neq 0 \). 若 \( ab = 0 \), 则称 \( a \) 是 \( R \) 的一个\textbf{左零因子}, \( b \) 是 \( R \) 的一个\textbf{右零因子}, 都简称为\textbf{零因子}. 有时为方便也将 \( 0 \) 称为零因子. 
\item \textbf{整环:}无零因子的幺环.

\item \textbf{体:}非零元素集合对乘法构成群的环.

\item \textbf{域:}交换的体, 即非零元素集合对乘法为 Abel 群的环.
\end{enumerate}
\end{definition}
\begin{remark}
当 \( n > 1 \) 时, \( R \) 上的 \( n \) 阶方阵环 \( R^{n \times n} \) 就不是无零因子环.

显然,一切数域 \( P \) 都是域, 因而也是体.
\end{remark}

\begin{proposition}\label{proposition:整环的一些性质}
\begin{enumerate}[(1)]
\item\label{proposition:整环的一些性质-1} 环 \( R \) 为整环的充要条件是 \( R \) 的非零元素集合 \( R^* = R \setminus \{0\} \) 是乘法幺半群 \( R \) 的子幺半群.

\item\label{proposition:整环的一些性质-2}  若\( R \) 是交换整环,则\( R^* = R \setminus \{0\} \) 对称乘法构成交换幺半群且消去律成立,即
\begin{align*}
ax = bx \ (\text{或 } xa = xb), \text{ 则 } a = b, \ \forall a, b, x \in R.
\end{align*}
\end{enumerate}
\end{proposition}
\begin{proof}
\begin{enumerate}[(1)]
\item 

\item 
\end{enumerate}

\end{proof}

\begin{example}
设 \( p \) 是一个素数. 于是 \( \mathbf{Z} \) 中关系 \( a \equiv b(\text{mod}\ p) \) 对加法及乘法都是同余关系, 因而在 \( \mathbf{Z}_p = \mathbf{Z}/p\mathbf{Z} \) 中有加法运算, 使 \( \mathbf{Z}_p \) 为 Abel 群, 而且在 \( \mathbf{Z}_p \) 中有乘法运算, 使 \( \mathbf{Z}_p \) 为交换幺半群. \( \mathbf{Z}_p = \{0,1,\cdots,\overline{p-1}\} \). 又 \( \forall \overline{a},\overline{b},\overline{c} \in \mathbf{Z}_p \) 有
\[
\overline{a}(\overline{b} + \overline{c}) = \overline{a(b + c)} = \overline{ab + ac} = \overline{ab} + \overline{ac} = \overline{a}\overline{b} + \overline{a}\overline{c},
\]
即分配律成立. 故 \( \mathbf{Z}_p \) 是交换幺环. 又对 \( a \in \mathbf{N}, a < p \), 由 \( p \) 为素数知有 \( m,n \in \mathbf{Z} \), 使 \( ma + np = 1 \), 因而 \( \overline{m} \cdot \overline{a} = \overline{1} \), 即 \( \mathbf{Z}_p \) 中每个非零元素可逆, 因而 \( \mathbf{Z}_p \) 是只含 \( p \) 个元素的域且非数域.
\end{example}
\begin{proof}


\end{proof}

\begin{example}
设 \( \mathbf{C} \) 为复数域. 考虑 \( \mathbf{C}^{2 \times 2} \) 中子集
\[
H = \left\{ \begin{pmatrix} \alpha & \beta \\ -\overline{\beta} & \overline{\alpha} \end{pmatrix} \bigg| \alpha,\beta \in \mathbf{C} \right\}.
\]
证明$H$是体,称 \( H \) 为 \( \mathbf{R} \) 上的\textbf{四元数体}.
\end{example}
\begin{proof}
容易验证 \( H \) 对矩阵的加法为 Abel 群. 又对 \( \forall \alpha,\beta,\gamma,\delta \in \mathbf{C} \) 有
\[
\begin{pmatrix} \alpha & \beta \\ -\overline{\beta} & \overline{\alpha} \end{pmatrix} \begin{pmatrix} \gamma & \delta \\ -\overline{\delta} & \overline{\gamma} \end{pmatrix} = \begin{pmatrix} \alpha\gamma - \beta\overline{\delta} & \alpha\delta + \beta\overline{\gamma} \\ -\overline{\alpha}\overline{\delta} - \overline{\beta}\gamma & \overline{\alpha}\overline{\gamma} - \overline{\beta}\delta \end{pmatrix} \in H,
\]
故 \( H \) 对矩阵乘法为幺半群. 显然加法与乘法间有分配律, 故 \( H \) 为幺环. 又若
\[
\begin{pmatrix} \alpha & \beta \\ -\overline{\beta} & \overline{\alpha} \end{pmatrix} \neq 0,
\]
则
\[
\left| \begin{array}{cc} \alpha & \beta \\ -\overline{\beta} & \overline{\alpha} \end{array} \right| = \alpha\overline{\alpha} + \beta\overline{\beta} > 0.
\]
此时有
\[
\begin{pmatrix} \alpha & \beta \\ -\overline{\beta} & \overline{\alpha} \end{pmatrix}^{-1} = (\alpha\overline{\alpha} + \beta\overline{\beta})^{-1} \begin{pmatrix} \overline{\alpha} & -\beta \\ \overline{\beta} & \alpha \end{pmatrix} \in H,
\]
即 \( H^* = H \setminus \{0\} \) 为群, 因而 \( H \) 是体. 又 \( H \) 中有元素
\[
\boldsymbol{A} = \begin{pmatrix} \sqrt{-1} & 0 \\ 0 & -\sqrt{-1} \end{pmatrix}, \quad \boldsymbol{B} = \begin{pmatrix} 0 & 1 \\ -1 & 0 \end{pmatrix}.
\]
由 \( \boldsymbol{AB} \neq \boldsymbol{BA} \), 故 \( H \) 不是域.

\end{proof}

\begin{definition}
若环 \( R \) 的非空子集 \( R_1 \) 对 \( R \) 的加法与乘法也构成环, 则称 \( R_1 \) 为 \( R \) 的\textbf{子环}. 若 \( R_1 \) 还满足 \( RR_1 \subseteq R_1 \) (或 \( R_1R \subseteq R_1 \)), 则称 \( R_1 \) 为 \( R \) 的\textbf{左理想} (或\textbf{右理想}). 若环 \( R \) 的非空子集 \( I \) 既是左理想又是右理想,也即$RR_1R\subseteq R_1$,则称 \( I \) 为 \( R \) 的\textbf{双边理想}, 简称\textbf{理想}.
\end{definition}
\begin{remark}
\( \{0\} \) 与 \( R \) 都是 \( R \) 的理想, 称为\textbf{平凡理想}. 在交换环中, 左理想、右理想与理想这三个概念是一致的.
\end{remark}

\begin{theorem}\label{theorem:子环和理想的基本性质}
\begin{enumerate}[(1)]
\item 一个环中任意多个理想之交还是理想.

\item\label{theorem:子环和理想的基本性质-2} 若$A$是环$R$的理想,$B$是环$R$的子环且$B\supseteq A$,则$A$也是环$B$的理想.

\item 若 \( A \) 是环 \( R \) 的非空子集, 则所有包含 \( A \) 的理想之交仍是一个包含 \( A \) 的理想, 称为\textbf{由\(\boldsymbol{A}\)生成的理想}, 记为 \( \langle A \rangle \).
\end{enumerate}
\end{theorem}
\begin{proof}
\begin{enumerate}[(1)]
\item 

\item 

\item 
\end{enumerate}

\end{proof}

\begin{theorem}\label{theorem:抽象代数-定理 1.4.1}
设 \( I \) 为环 \( R \) 的子环. 在 \( R \) 中定义关系 “\(\sim\)”,
\[
a \sim b, \ 若\ a + (-b) = a - b \in I,
\]
则关系 “\(\sim\)” 对加法为同余关系.\( a \) 所在的等价类为 \( a + I \).
关系 “\(\sim\)” 对乘法也为同余关系的充分必要条件是 \( I \) 为 \( R \) 的理想.

若 \( I \) 为理想, 则将$R$对等价关系$I$的商集合记为\( R/\sim = R/I \), 并且\( R/\sim = R/I \)中可定义加法、乘法为
\begin{gather}
(a + I) + (b + I) = (a + b) + I, \quad \forall a,b \in R, \label{eq:gq34t34fg34t13q.4.1}
\\
(a + I) \cdot (b + I) = ab + I, \quad \forall a,b \in R. \label{eq:gq34t34fg34t13q.4.2}
\end{gather}
\( R/I \) 对这种加法与乘法也构成环, 称为 \( R \) 对 \( I \) 的\textbf{商环}.
\end{theorem}
\begin{proof}
因 \( R \) 对加法为 Abel 群, 故 \( R \) 的加法子群 \( I \) 为正规子群. 由\refthe{theorem:抽象代数-定理 1.3.4} 知 “\(\sim\)” 对 \( R \) 的加法为同余关系, 再由\refpro{proposition:正规子群的基本性质}知在 \( R/I \) 中有加法运算 \(\eqref{eq:gq34t34fg34t13q.4.1}\) 且为 Abel 群.

现设 “\(\sim\)” 对乘法也是同余关系. 对 \( \forall a \in I, b \in R \) 有 \( a \sim 0, b \sim b \), 因而 \( ab \sim 0, ba \sim 0 \), 故 \( ab, ba \in I \), 因而 \( I \) 为 \( R \) 的理想.

反之, 设 \( I \) 是 \( R \) 的理想, \( a,b,c,d \in R \) 且 \( a \sim b, c \sim d \), 即 \( a - b, c - d \in I \). 此时有 \( ac - bd = ac - ad + ad - bd = a(c - d) + (a - b)d \in I \), 即 \( ac \sim bd \), 故 “\(\sim\)” 对乘法也是同余关系.

当 \( I \) 为理想时, 在 \( R/I \) 中可定义乘法如式 \(\eqref{eq:gq34t34fg34t13q.4.2}\) 且对 \( \forall a,b,c \in R \) 有
\begin{align*}
((a + I)(b + I))(c + I) &= (ab + I)(c + I) = (ab)c + I = a(bc) + I \\
&= (a + I)((b + I)(c + I)),
\end{align*}
并且
\begin{align*}
((a + I) + (b + I))(c + I) &= ((a + b) + I)(c + I) \\
&= (a + b)c + I = (ac + bc) + I = (ac + I) + (bc + I) \\
&= (a + I)(c + I) + (b + I)(c + I).
\end{align*}
类似有
\[
(a + I)((b + I) + (c + I)) = (a + I)(b + I) + (a + I)(c + I),
\]
即 \( R/I \) 为半群, 且对加法乘法的分配律成立. 故 \( R/I \) 是一个环.

\end{proof}

\begin{corollary}\label{corollary:抽象代数-推论 1.4.1}
若 \( R \) 为交换环, 则 \( R/I \) 也是交换环.
\end{corollary}
\begin{proof}


\end{proof}

\begin{corollary}\label{corollary:抽象代数-推论 1.4.2}
若 \( R \) 为幺环, 则 \( R/I \) 也是幺环且 \( 1 + I \) 为幺元.
\end{corollary}
\begin{proof}


\end{proof}

\begin{example}
从\refthe{theorem:抽象代数-定理 1.4.1}知 \( m\mathbf{Z} \) 为 \( \mathbf{Z} \) 的理想, 故 \( \mathbf{Z}_m = \mathbf{Z}/m\mathbf{Z} \) 对剩余类 \( (\text{mod}\ m) \) 的加法与乘法是一个环.

当 \( p \) 为素数时, \( \mathbf{Z}_p \) 为域.

若 \( m \) 是合数, 即 \( m = m_1m_2 (m_i \in \mathbf{Z}, |m_i| > 1, i = 1,2) \), 则 \( \mathbf{Z}_m \) 有零因子 \( \overline{m_1}, \overline{m_2} \).
\end{example}

\begin{example}
设 \( R \) 是一个环. 考虑 \( R^{n \times n} \) 中子集
\[
A = \{\alpha \mid \alpha \in R^{n \times n}, j \neq 1 \text{ 时}, \text{col}_j\alpha = 0\},
\]
\[
B = \{\alpha \mid \alpha \in R^{n \times n}, i \neq 1 \text{ 时}, \text{row}_i\alpha = 0\},
\]
则 \( A, B \) 分别为 \( R^{n \times n} \) 的左理想与右理想. 当 \( n \geqslant 2 \) 时, 一般来说, \( A, B \) 都不是双边理想.
\end{example}
































\end{document}
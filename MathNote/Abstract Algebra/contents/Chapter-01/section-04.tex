\documentclass[../../main.tex]{subfiles}
\graphicspath{{\subfix{../../image/}}} % 指定图片目录,后续可以直接使用图片文件名。

% 例如:
% \begin{figure}[h]
% \centering
% \includegraphics{image-01.01}
% \label{fig:image-01.01}
% \caption{图片标题}
% \end{figure}

\begin{document}

\section{正规子群}

\begin{definition}[正规子群]
令 $(G,\cdot)$ 是一个群,且 $N\subset G$。我们称 $N$ 是个\textbf{正规子群},记作 $N\lhd G$,若
\begin{gather*}
N\text{ 是个子群},\\
\forall a\in G, aN = Na.
\end{gather*}
\end{definition}
\begin{remark}
注意$aN=Na\nLeftrightarrow an=na,\forall n\in N.$虽然$an=na,\forall n\in N\Rightarrow aN=Na$,但是$aN=Na\nRightarrow an=na,\forall n\in N.$实际上,$aN=Na\Leftrightarrow \exists n,n' \in N\,\mathrm{s}.\mathrm{t}.\,an=n'a.$
\end{remark}

\begin{lemma}\label{lemma:子群或子幺半群与自身的乘积还等于其本身}
设 $G$ 是一个群或幺半群,若 $H < G$,则 $HH = H$。
\end{lemma}
\begin{proof}
一方面,对 $\forall h_1,h_2\in H$,根据乘法封闭性,都有 $h_1h_2\in H$。故 $HH\subset H$。
另一方面,设 $h\in H$,则 $h = he\in HH$。故 $H\subset HH$。
因此 $HH = H$。 
\end{proof}

\begin{proposition}\label{proposition:陪集乘法的良定义性}
令 $(G,\cdot)$ 是一个群,且 $N\lhd G$,$a,b\in G$,则
\begin{align*}
(aN)\cdot(bN)=(ab)N.
\end{align*}
是良定义的。
\end{proposition}
\begin{conclusion}
元素与群(其实只要满足结合律的半群就足够了)的乘积满足广义结合律. 例如:
设$G$是一个群,若$H,K<G$,$a,b\in G$,则
\begin{gather*}
aHbK=(aH)(bK)=a((Hb)K)=a(H(bK))=(a(Hb))K=((aH)b)K.
\\
abHK=(ab)(HK)=a((bH)K)=a(b(HK))=((ab)H)K.
\\
\cdots \cdots \cdots \cdots
\end{gather*}
即两个陪集相乘可以看作一个陪集或两个陪集的乘积的陪集等.
\end{conclusion}
\begin{proof}
{\color{blue}证法一:}
设 $aN = a'N, bN = b'N$,则由\hyperref[lemma:关于两个陪集相等的充要条件]{引理\ref{lemma:关于两个陪集相等的充要条件}}可知$a^{-1}a',b^{-1}b'\in N$,我们只须证明 $abN = a'b'N$,即 $(ab)^{-1}a'b' = b^{-1}a^{-1}a'b'\in N$。首先中间这个部分,即 $a^{-1}a'$,是在 $N$ 中的。接着,利用 $N$ 是个正规子群,再结合\hyperref[lemma:两个相等的陪集同时左乘相同元素保持等号]{引理\ref{lemma:两个相等的陪集同时左乘相同元素保持等号}},我们可以得到 $b^{-1}Nb = N$,因此,$b^{-1}a^{-1}a'b'\in b^{-1}Nb' = N$。进一步地,由\hyperref[lemma:关于两个陪集相等的充要条件]{引理\ref{lemma:关于两个陪集相等的充要条件}}可得$abN = a'b'N$。这就证明了良定义性。

{\color{blue}证法二:}事实上,这个乘法可以简单地理解成子集乘法,即 $(aN)(bN)=\{xy:x\in aN, y\in bN\}$。我们只须说明,这从集合意义上,等于 $abN$。而这几乎是显然的。由于 $Nb = bN$及\hyperref[lemma:子群或子幺半群与自身的乘积还等于其本身]{引理\ref{lemma:子群或子幺半群与自身的乘积还等于其本身}},我们有 $aNbN = abNN = abN$.这样,既然从集合意义上相等,那么自然就是良定义的(因为我们不必选取单位元)。 
\end{proof}

\begin{proposition}[商群]\label{proposition:商群}
令 $(G,\cdot)$ 是一个群,且 $N\lhd G$,则 $(G/N,\cdot)$ 构成一个群,称为($G$ 在 $N$ 上的)\textbf{商群},其中的单位元是 $eN = N$,每个陪集 $aN$ 的逆元是 $a^{-1}N$。
\end{proposition}
\begin{proof}
由\hyperref[proposition:陪集乘法的良定义性]{命题\ref{proposition:陪集乘法的良定义性}}可知商群$(G/N,\cdot)$的乘法是良定义的.

封闭性:对$\forall aN,bN\in (G/N,\cdot)$,其中$a,b\in G$,根据$G$对乘法的封闭性可得$ab\in G$,从而$(aN)(bN)=abN\in (G/N,\cdot)$.

结合律:令 $a,b,c\in G$,则利用乘法的定义,$(aNbN)cN=(abN)(cN)=((ab)c)N$。利用 $G$ 对乘法的结合律,得到这是等于 $(a(bc))N$ 的。类似地,这最终等于 $aN(bNcN)$。

单位元:令 $a\in G$,则 $aNeN=(ae)N = aN$,类似地 $eNaN = aN$。

逆元:令 $a\in G$,则 $aNa^{-1}N=(aa^{-1})N = eN$,类似地 $a^{-1}NaN = eN$。

综上,若 $N\lhd G$,则 $G/N$ 在这个自然的乘法下构成群,称为一个商群。 
\end{proof}

\begin{lemma}[正规子群的等价条件]\label{lemma:正规子群的等价条件}
令 $(G,\cdot)$ 是一个群,且 $N < G$,则下列命题等价
\begin{enumerate}[(1)]
\item $N$ 是 $G$ 的正规子群,即
$\forall a\in G, aN = Na .$

\item
$\forall a\in G, aNa^{-1}\subset N .$


\item 
$\forall a\in G, \forall n\in N, ana^{-1}\in N .$
\end{enumerate}
\end{lemma}
\begin{proof}
显然第二个条件和第三个条件等价。我们只要证明第一个条件与第二个条件等价即可。

一方面,设 $N$ 是 $G$ 的正规子群。令 $a\in G$,则 $aN = Na$。同时右乘 $a^{-1}$ 并取一半的包含关系,我们得到了 $aNa^{-1}\subset N$。

另一方面,设第二个条件成立。令 $a\in G$,则由 $aNa^{-1}\subset N$ 及\hyperref[lemma:两个相等的陪集同时左乘相同元素保持等号]{引理\ref{lemma:两个相等的陪集同时左乘相同元素保持等号}}得到 $aN\subset Na$,由 $a^{-1}N (a^{-1})^{-1}\subset N$及\hyperref[lemma:两个相等的陪集同时左乘相同元素保持等号]{引理\ref{lemma:两个相等的陪集同时左乘相同元素保持等号}}得到得到 $Na\subset aN$。因此,$aN = Na$。 
\end{proof}

\begin{proposition}[正规子群的任意交还是正规子群]\label{proposition:一族正规子群的任意交还是正规子群}
令 $(N_i)_{i\in I}$ 是一族 $G$ 的正规子群,则它们的交集仍然是 $G$ 的正规子群,即
\begin{align*}
\bigcap_{i\in I}N_i\lhd G .
\end{align*}
\end{proposition}
\begin{proof}
首先,由\hyperref[proposition:子群的任意交仍是子群]{子群的任意交仍是子群}可知$\bigcap_{i\in I}N_i< G .$
因此我们只需证明正规性。利用\hyperref[lemma:正规子群的等价条件]{正规子群的等价条件(3)}可知,对$\forall a\in G,\forall n\in \bigcap_{i\in I}N_i$,我们只须证明$ana^{-1}\in \bigcap_{i\in I}N_i$即可.任取 $i\in I$,则 $n\in N_i$。由于 $N_i\lhd G$,我们有 $ana^{-1}\in N_i$。因此,由$i$的任意性可知$ana^{-1}\in \bigcap_{i\in I}N_i$。这就证明了 $\bigcap_{i\in I}N_i\lhd G$。
\end{proof}

\begin{proposition}\label{proposition:平凡群和整个群都是正规子群}
令 $(G,\cdot)$ 是一个群,则
\begin{align*}
\{e\}&\lhd G ,\\
G&\lhd G .
\end{align*}
\end{proposition}
\begin{proof}
平凡群:怎么乘都是单位元,所以对乘法封闭;包含单位元;唯一的元素的逆元还是单位元;在这个群中,$a$ 的左右陪集都是 $a\{e\}=\{e\}a = \{a\}$。因此,$\{e\}\lhd G$。

整个群:子群是显然的;在整个群 $G$ 中,每个元素的左右陪集都是全集,即 $aG = Ga = G$,这是因为 $a\in G$。因此,$G\lhd G$(\hyperref[corollary:关于两个陪集相等的充要条件推论]{推论\ref{corollary:关于两个陪集相等的充要条件推论}}).
\end{proof}

\begin{corollary}
\begin{enumerate}
\item 若 $G$ 是一个群,$e$ 是其单位元,则 $G/\{e\}$ 同构于 $G$,即 $G/\{e\}\simeq G$。

\item 若 $G$ 是一个群,则 $G/G$ 是平凡群,即 $G/G = \{e\}$。
\end{enumerate}
\end{corollary}
\begin{proof}
\begin{enumerate}
\item 令 
\[f:G\rightarrow G/\{e\}, a\mapsto a\{e\}=\{a\}.\]
显然 $f$ 是双射。对 $\forall a,b\in G$,我们都有
\begin{align*}
f(ab)&=\{ab\}=ab\{e\}=(a\{e\})(b\{e\})=\{a\}\{b\}=f(a)f(b).
\end{align*}
因此 $f$ 也是同态映射。于是 $f$ 是同构映射。故 $G/\{e\}\simeq G$。

\item 由\hyperref[proposition:商群]{命题\ref{proposition:商群}}及\hyperref[proposition:平凡群都是正规子群]{命题\ref{proposition:平凡群和整个群都是正规子群}}可知 $G/G$ 是一个群。注意到 $\forall a\in G$,都有 $aG = G$。因此 $G/G = G$。于是 $|G/G| = 1$。故 $G/G = \{e\}$。 
\end{enumerate}
\end{proof}

\begin{proposition}
令 $(G,\cdot)$ 是个阿贝尔群,则子群就是正规子群,正规子群也就是子群,即
\begin{align*}
H < G\iff H\lhd G
\end{align*}
\end{proposition}
\begin{proof}
根据阿贝尔群满足交换律可知,$aH = \{ah:h\in H\} = \{ha:h\in H\} = Ha$。
\end{proof}

\begin{theorem}

\end{theorem}
\begin{proof}

\end{proof}

\begin{definition}

\end{definition}

\begin{proposition}

\end{proposition}
\begin{proof}

\end{proof}

\begin{theorem}

\end{theorem}
\begin{proof}

\end{proof}

\begin{definition}

\end{definition}

\begin{proposition}

\end{proposition}
\begin{proof}

\end{proof}

\begin{theorem}

\end{theorem}
\begin{proof}

\end{proof}

\begin{definition}

\end{definition}

\begin{proposition}

\end{proposition}
\begin{proof}

\end{proof}

\begin{theorem}

\end{theorem}
\begin{proof}

\end{proof}
















\end{document}
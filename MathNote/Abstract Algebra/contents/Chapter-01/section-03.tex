\documentclass[../../main.tex]{subfiles}% 注意这里的文件路径不能用 ./main.tex ,否则用latexmk编译子文件会报错
\graphicspath{{\subfix{./image/}}} % 指定图片目录,后续可以直接使用图片文件名
% 注意这里的文件路径不能用 ../../image/ ,否则用latexmk编译子文件会报错

% 例如:
% \begin{figure}[H]
% \centering
% \includegraphics[scale=0.3]{图.png}
% \caption{}
% \label{figure:图}
% \end{figure}
% 注意:上述\label{}一定要放在\caption{}之后,否则引用图片序号会只会显示??.

\begin{document}

\section{子群与商群}

\begin{definition}
设 \( A, B \) 是群 \( G \) 的两个子集, 约定
\begin{align*}
AB = \{ab|a \in A, b \in B\} , A^{-1} = \{a^{-1}|a \in A\}.
\end{align*}
特别地, 当 \( A = \{a\} \) 为单点集时, 记 \( AB = aB \), \( BA = Ba \). 当然这些符号对半群与幺半群可同样使用.
\end{definition}

\begin{proposition}\label{proposition:有限群乘积的阶}
设有限群$N_1,N_2,\cdots,N_k$满足
\begin{align*}
N_i\cap N_j=\{1\},\,\,i\neq j.
\end{align*}
则
\begin{align*}
|N_1N_2\cdots N_k|=|N_1||N_2|\cdots |N_k|.
\end{align*}
\end{proposition}
\begin{proof}
因为$N_i$都是有限群,所以设
\begin{align*}
N_i=\{n_{1}^{i},n_{2}^{i},\cdots,n_{|N_i|}^{i}\},\quad i=1,2,\cdots,k.
\end{align*}
其中$n_{1}^{i}=1,\,i=1,2,\cdots,k$.由$N_i\cap N_j=\{1\}(i\ne j)$知当$i\ne j$时,有
\begin{align*}
n_{s}^{i}\ne n_{t}^{j},\quad \forall s,t\in\{1,2,\cdots,k\}.
\end{align*}
于是
\begin{align*}
N_1N_2\cdots N_k=\{n_{j_1}^{1}n_{j_2}^{2}\cdots n_{j_k}^{k}\mid j_i\in\{1,2,\cdots,|N_i|\},i=1,2,\cdots,k\}.
\end{align*}
因此直接计算$N_1N_2\cdots N_k$的元素个数可得
\begin{align*}
|N_1N_2\cdots N_k|=|N_1||N_2|\cdots|N_k|.
\end{align*}
若$G = N_1\otimes N_2\otimes\cdots\otimes N_k$,则当$i\neq j$时,有$N_i\cap N_j\subseteq N_i\cap \prod_{j\ne i}{N_j}=\left\{ 1 \right\}$,故此时有
\begin{align*}
|G|=|N_1||N_2|\cdots |N_k|.
\end{align*}

\end{proof}

\begin{definition}
群 \( G \) 的非空子集 \( H \) 若对 \( G \) 的运算也构成一个群, 则称为 \( G \) 的\textbf{子群},记作$H<G$.
\end{definition}
\begin{remark}
显然, \( H = \{1\} \)(1 为 \( G \) 的幺元) 与 \( H = G \) 均为 \( G \) 的子群, 称为 \( G \) 的\textbf{平凡子群}, 其他的子群称为\textbf{非平凡子群}.
\end{remark}

\begin{theorem}\label{theorem:子群的充要条件--抽象代数}
设 \( H \) 是群 \( G \) 的非空子集, 则下列条件等价:
\begin{enumerate}[(1)]
\item \( H \) 是 \( G \) 的子群;

\item \( 1 \in H \); 若\( a \in H \), 则 \( a^{-1} \in H \); 若\( a, b \in H \), 则 \( ab \in H \);

\item 若\( a, b \in H \), 则 \( ab \in H \), \( a^{-1} \in H \);

\item\label{theorem:子群的充要条件--抽象代数-4} 若\( a, b \in H \), 则 \( ab^{-1} \in H \).
\end{enumerate}
\end{theorem}
\begin{proof}
\( (1) \Rightarrow (2) \). 由 \( H \) 对 \( G \) 的乘法构成群知 \( a, b \in H \), 则 \( ab \in H \). 又 \( H \) 有幺元 \( 1' \), 即有 \( 1' \cdot 1' = 1' \). 设 \( 1' \) 在 \( G \) 中的逆元为 \( 1'^{-1} \), 则有
\[
1 = 1' \cdot 1'^{-1} = (1' \cdot 1') \cdot 1'^{-1} = 1',
\]
故 \( 1 \in H \). 设 \( a \) 在 \( H \) 中的逆元为 \( a' \), 于是 \( aa' = 1' = 1 \), 即 \( a' = a^{-1} \), 故 \( a^{-1} \in H \). 由此知 \( (2) \) 成立, 而且 \( H \) 的幺元是 \( G \) 的幺元. \( a \in H \), \( a \) 在 \( H \) 中的逆元与在 \( G \) 中的逆元一致.

\( (2) \Rightarrow (3) \). 这是显然的.

\( (3) \Rightarrow (4) \). 若 \( a, b \in H \), 故 \( a, b^{-1} \in H \), 故 \( ab^{-1} \in H \).

\( (4) \Rightarrow (1) \). 由 \( H \neq \varnothing \) 知 \( \exists a \in H \), 因而 \( 1 = aa^{-1} \in H \). 又由 \( 1, a \in H \) 知 \( a^{-1} = 1 \cdot a^{-1} \in H \). 又若 \( a, b \in H \), 由 \( b^{-1} \in H \) 得 \( ab = a(b^{-1})^{-1} \in H \). 由此可知 \( G \) 的乘法也是 \( H \) 的乘法. 对 \( H \) 而言有幺元 \( 1 \); 对 \( a \in H \) 有逆元 \( a^{-1} \); 结合律显然成立. 故 \( H \) 是 \( G \) 的子群.

\end{proof}

\begin{theorem}\label{theorem:子群陪集的基本性质}
设$A,B,C,H$是群$G$的非空子集,$g$是群$G$的一个元素,则
\begin{enumerate}[(1)]
\item\label{theorem:子群陪集的基本性质-1} $A(BC)=(AB)C$;

\item\label{theorem:子群陪集的基本性质-2} $gA=gB\text{或}Ag=Bg\iff A=B$;

\item\label{theorem:子群陪集的基本性质-3} \( H \) 是 \( G \) 的子群$\iff HH = H , H^{-1} = H\iff H^{-1}H = H$;

\item\label{theorem:子群陪集的基本性质-4} 如果$A,B$是群$G$的两个子群,则$AB$也是群$G$的子群的充分必要条件是$AB=BA$.
\end{enumerate}
\end{theorem}
\begin{proof}
\begin{enumerate}[(1)]
\item 

\item 

\item 

\item {\heiti 必要性:}设$AB$为$G$的子群.
对任意的$ab\in AB$,其中$a\in A,b\in B$,有$(ab)^{-1}\in AB$.因而存在$a_1\in A$,
$b_1\in B$,使$a_1b_1=(ab)^{-1}$.从而
\begin{align*}
ab=(a_1b_1)^{-1}=b_1^{-1}a_1^{-1}\in BA,
\end{align*}
所以
$$AB\subseteq BA.$$

反之,对任意的$ba\in BA$,其中$b\in B,a\in A$,有$(ba)^{-1}=a^{-1}b^{-1}\in AB$.于是
\begin{align*}
ba=(a^{-1}b^{-1})^{-1}\in AB,
\end{align*}
所以
$$BA\subseteq AB.$$

这就证明了$AB=BA$.

{\heiti 充分性:}对任意的$a_1b_1,a_2b_2\in AB$,其中$a_i\in A,b_i\in B\ (i=1,2)$,由于$AB=BA$,因此由\rrefthe{theorem:子群陪集的基本性质}{theorem:子群陪集的基本性质-1}和\rrefthe{theorem:子群陪集的基本性质}{theorem:子群陪集的基本性质-3}有
\begin{align*}
a_1b_1(a_2b_2)^{-1}=a_1b_1b_2^{-1}a_2^{-1}=a_1(b_1b_2^{-1})a_2^{-1}\in ABA=A(BA)=A(AB)=(AA)B=AB,
\end{align*}
由此知$AB$是$G$的子群.
\end{enumerate}

\end{proof}

\begin{proposition}\label{proposition:抽象代数---子群的基本性质}
\begin{enumerate}[(1)]
\item\label{proposition:抽象代数---子群的基本性质-1} 设$H$是群$G$的非空有限子集. 证明:$H$是$G$的子群的充分必要条件是$H$关于$G$的运算封闭.

\item\label{proposition:抽象代数---子群的基本性质-2} 若$G$是一个群,则$G$的任意子群的交$\bigcap_{H<G}{H}$也是$G$的子群.

\item\label{proposition:抽象代数---子群的基本性质-3} 若$H_1,H_2$都是群$G$的子群且$H_2\subseteq H_1$,则$H_2$也是$H_1$的子群.

\item\label{proposition:抽象代数---子群的基本性质-4} 设$H,K$是群$G$的两个子群. 则$H\cup K$是$G$的子群的充要条件是$H\subseteq K$或$K\subseteq H$. 并且群$G$不能被它的两个真子群所覆盖.
\end{enumerate}
\end{proposition}
\begin{remark}
在这个\rrefpro{proposition:抽象代数---子群的基本性质}{proposition:抽象代数---子群的基本性质-4}中,群$G$可能被它的三个真子群所覆盖. 例如, 群
\begin{align*}
U(8)=\{\overline{1},\overline{3},\overline{5},\overline{7}\}.
\end{align*}
易知
\begin{align*}
H=\{\overline{1},\overline{3}\}, \quad J=\{\overline{1},\overline{5}\}, \quad K=\{\overline{1},\overline{7}\}
\end{align*}
都是$U(8)$的真子群, 且$U(8)=H\cup J\cup K$.
\end{remark}
\begin{proof}
\begin{enumerate}[(1)]
\item 必要性显然, 下证充分性.
因为$H$关于$G$的运算封闭, 所以$G$的运算是$H$的代数运算. 又因为$G$的运算满足结合律, 所以$H$的运算也满足结合律.

设$H=\{a_1,a_2,\cdots,a_n\}$. 对任意的$a\in H$, 记$Ha=\{a_1a,a_2a,\cdots,a_na\}$, 则$Ha\subseteq H$.
于是$a_ia\neq a_ja(i\neq j)$.否则,由$a_ia = a_ja(i\neq j)$可得
\begin{align*}
a_i=a_i(aa^{-1})=(a_ia)a^{-1}=(a_ja)a^{-1}=a_j(aa^{-1})=a_j,
\end{align*}
显然矛盾!
由此推出$|Ha|=n=|H|$, 于是$Ha=H$. 这样, 对任意的$a,b\in H$, 因为$Ha=H$, 所以必有$a_i\in H$, 使$a_ia=b$. 这说明, 对任意的$a,b\in H$, 方程$xa=b$在$H$中必有解. 同理可证, 方程$ay=b$在$H$中也有解. 从而, 由\refthe{theorem:群的充要条件--方程有解}知$H$为群.

\item 设$I$为任一(有限或无限的)指标集, $\{H_i\mid i\in I\}$为群$G$的一些子群的集合, 令
\begin{align*}
J = \bigcap\limits_{i\in I} H_i,
\end{align*}
因为$e\in H_i(\forall i\in I)$, 所以$e\in \bigcap\limits_{i\in I} H_i$, 从而$J$非空;
对$\forall a,b\in J$, 有$a,b\in H_i(\forall i\in I)$. 由于$H_i < G$, 因此$ab^{-1}\in H_i(\forall i\in I)$, 于是$ab^{-1}\in J$.
这就证明了$\bigcap\limits_{i\in I} H_i$为$G$的子群.

\item 由$H_2$是$G$的子群知$ab^{-1} \in H_2$,$\forall a,b \in H_2$.又$H_2 \subseteq H_1$,故$H_2$也是$H_1$的子群.

\item 充分性显然, 下证必要性.
设$H\cup K$是$G$的子群. 如果$H\subseteq K$, 则结论成立. 如果$H\nsubseteq K$, 则存在$h\in H$, 使$h\notin K$. 由于$H\cup K$为$G$的子群, 因此对任意的$k\in K$, 有$hk\in H\cup K$. 从而必有$hk\in H$或$hk\in K$. 如果$hk\in K$, 则$h=hk\cdot k^{-1}\in K$, 这与$h$的选取矛盾. 从而必有$hk\in H$, 由此推出$k=h^{-1}\cdot hk\in H$. 由$k$的任意性知$K\subseteq H$. 这就证明了必要性.

设$H,K$是群$G$的两个子群. 如果$G=H\cup K$, 由前面所证, 应有$H\subseteq K$或$K\subseteq H$, 于是有$G=K$或$G=H$. 这说明$H,K$不可能都是$G$的真子群. 因此群$G$不能被它的两个真子群所覆盖.
\end{enumerate}

\end{proof}

\begin{theorem}
\begin{enumerate}
\item 设 \( V \) 是数域 \( \mathbb{P} \) 上的 \( n \) 维线性空间.\( S_V \) 为 \( V \) 上的全变换群, \( GL(V) \) 表示 \( V \) 上所有可逆线性变换的集合, 则 \( GL(V) \) 为 \( S_V \) 的子群, 称为线性空间 \( V \) 的\textbf{一般线性群}. 

又设 \( SL(V) \) 为 \( V \) 上所有行列式等于 1 的线性变换的集合, 则 \( SL(V) \) 是 \( GL(V) \)(同时也是 \( S_V \)) 的子群, 称为\textbf{特殊线性群}.

\item 设 \( V \) 是 \( n \) 维 Euclid 空间. 以 \( O(V) \) 表示 \( V \) 上所有正交变换的集合, \( SO(V) \) 表示所有行列式等于 1 的正交变换的集合, 则 \( O(V) \) 是 \( GL(V) \) 的子群, \( SO(V) \) 是 \( O(V) \) 的子群. \( O(V) \) 称为 \( V \) 的\textbf{正交变换群}, 简称\textbf{正交群}, \( SO(V) \) 称为\textbf{转动群}(或\textbf{特殊正交变换群}、\textbf{特殊正交群}).
\end{enumerate}
\end{theorem}
\begin{remark}
将上述$S_V$换成数域$\mathbb{P}$上的全体方阵构成的乘法群,线性变换换成方阵,结论也成立.
\end{remark}
\begin{proof}


\end{proof}

\begin{definition}
设 \( H \) 是群 \( G \) 的子群, 又 \( a \in G \). 集合 \( aH \) 与 \( Ha \) 分别称为以 \( a \) 为代表的 \( H \) 的\textbf{左陪集}与\textbf{右陪集}.
\end{definition}

\begin{theorem}\label{theorem:抽象代数-定理 1.3.2}
设 \( H \) 是群 \( G \) 的子群, 则由
\[
aRb, \ \text{若} \ a^{-1}b \in H
\]
所确定的 \( G \) 中的关系 \( R \) 是一个等价关系, 并且 \( a \) 所在的等价类为 \( \{aH:a\in G\} \), 故 \( H \)的左陪集族 \( \{aH:a\in G\} \)(集合无相同元素) 是 \( G \) 的一个分划.即
\begin{align*}
G = \bigcup_{a\in G}aH,
\end{align*}
其中$a$取遍不同$H$的左陪集的代表元.
\end{theorem}
\begin{proof}
由 \( a^{-1}a \in H \) 知 \( aRa(\forall a \in G) \). 又设 \( aRb \), 即 \( a^{-1}b \in H \), 故 \( (a^{-1}b)^{-1} = b^{-1}a \in H \), 即 \( bRa \). 再设 \( aRb, cRb \), 即 \( a^{-1}b, b^{-1}c \in H \), 故 \( a^{-1}c = (a^{-1}b)(b^{-1}c) \in H \), 即 \( aRc \). 这样知 \( R \) 是等价关系. 又由 \( b = a(a^{-1}b) \) 知
\[
aRb \iff \ a^{-1}b \in H \iff \ b \in aH,
\]
故 \( a \) 所在的等价类为 \( aH \). 由\refthe{Set Theory-theorem:等价类就和集合的分划对应-定理1.1.1}知 \( \{aH:a\in G\} \) 为 \( G \) 的一个分划.

\end{proof}

\begin{theorem}\label{theorem:子群陪集相关性质}
设$H$是群$G$的子群,$a,b\in G$,则
\begin{enumerate}[(1)]
\item $a\in aH$;

\item $aH=H$的充分必要条件是$a\in H$;

\item $aH$为子群的充分必要条件是$a\in H$;

\item\label{theorem:子群陪集相关性质-4} $aH=bH\iff a^{-1}b\in H\text{或}b^{-1}a\in H\iff aH\cap bH\neq \varnothing$;

\item\label{theorem:子群陪集相关性质-5} $|aH|=|bH|=|Ha|=|Hb|=|aHb|$.
\end{enumerate}
\end{theorem}
\begin{proof}
\begin{enumerate}[(1)]
\item $a=ae\in aH$.

\item 如果$aH=H$,因$a\in aH$,所以$a\in H$.

反之,如果$a\in H$,则$a^{-1}\in H$.从而
\begin{align*}
aH\subseteq H\cdot H=H, \quad
H=(aa^{-1})H=a(a^{-1}H)\subseteq aH,
\end{align*}
所以$aH=H$.

\item 设$aH$为子群.因为$a\in aH$,所以$a^{-1}\in aH$,于是$e=aa^{-1}\in aH$.从而存在$h\in H$,使$e=ah$.所以$a=eh^{-1}=h^{-1}\in H$.

另一方面,如果$a\in H$,则$aH=H$为子群.

\item 如果$aH=bH$,则
\begin{align*}
a^{-1}bH=a^{-1}aH=H,
\end{align*}
从而由(2)知,$a^{-1}b\in H$.

反之,如果$a^{-1}b\in H$,则又由(2)得$a^{-1}bH=H$,于是
\begin{align*}
aH=a(a^{-1}bH)=(aa^{-1})bH=ebH=bH.
\end{align*}
故$aH=bH\iff a^{-1}b\in H$.交换$a,b$位置即得$bH=aH\iff b^{-1}a\in H$.

若$aH=bH$,则显然有$aH\cap bH\neq \varnothing$.假设$aH\cap bH\neq\varnothing$.任取$g\in aH\cap bH$,则存在$h_1,h_2\in H$,使
\begin{align*}
ah_1=g=bh_2.
\end{align*}
从而
\begin{align*}
aH=a(h_1H)=(ah_1)H=(bh_2)H=b(h_2H)=bH.
\end{align*}
故$aH=bH\iff aH\cap bH\neq \varnothing$.

\item 考察映射
\begin{align*}
\sigma: aH\to bH, \\
ah\mapsto bh,
\end{align*}
易知$\sigma$为一一对应,所以$|aH|=|bH|$.其他同理可证.

\end{enumerate}

\end{proof}

\begin{definition}
设$H$是群$G$的子群,由\refthe{theorem:抽象代数-定理 1.3.2}定义$G$中的等价关系$R$为
\[
aRb, \ \text{若} \ a^{-1}b \in H.
\]
将 \( G \) 对等价关系$R$的商集合, 即以左陪集$aH,\,a\in G$为元素的集合记为 \( G/H=\{aH:a\in G\} \), 称为 \( G \) 对 \( H \) 的\textbf{左陪集空间}. \( G/H \) 中元素个数 \( |G/H| \) 称为 \( H \) 在 \( G \) 中的\textbf{指数}, 记为 \( [G:H] \). 相应可定义\textbf{右陪集空间}.
\end{definition}
\begin{remark}
\(\{1\}\) 作为 \( G \) 的子群, 在 \( G \) 中指数显然为 \( |G| \). 故也记 \( |G| = [G: 1] \).
\end{remark}

\begin{example}
设 \( V \) 是数域 \( \mathbb{P} \) 上的 \( n \) 维线性空间, \( GL(V) \) 有子群 \( SL(V) \). 在 \( V \) 中取定一组基, 任何一个线性变换由它在这组基下的矩阵完全确定, 可把它们等同起来.\( \forall \lambda \in \mathbb{P}, \lambda \neq 0, \) 令 \( D(\lambda) = \text{diag}(\lambda, 1, \cdots, 1) \), 于是 \( D(\lambda) \in GL(V) \), 对于 \( A \in GL(V) \) 有
\[
A SL(V) = D(\lambda) SL(V)\iff \ \det A = \lambda.
\]
于是
\[
GL(V) = \bigcup_{\lambda \neq 0} D(\lambda) SL(V),
\]
因而
\[
[GL(V): SL(V)] = +\infty.
\]
\end{example}
\begin{proof}


\end{proof}

\begin{example}
设 \( V \) 是 \( n \) 维 Euclid 空间. 由 \( A \in O(V) \) 有 \( \det A = \pm 1 \),令 \( D(\lambda) = \text{diag}(\lambda, 1, \cdots, 1) \),  于是
\[
O(V) = SO(V) \bigcup D(-1) SO(V),
\quad
[O(V): SO(V)] = 2.
\]
\end{example}
\begin{proof}


\end{proof}

\begin{theorem}[Lagrange定理]\label{theorem:抽象代数--Lagrange定理}
设 \( H \) 是有限群 \( G \) 的子群, 记1为$G,H$的幺元,则有
\begin{align}
[G:1] = [G:H][H:1] \label{eq:1.3.229034890--1}
\end{align}
因而子群 \( H \) 的阶是群 \( G \) 的阶的因子.
\end{theorem}
\begin{remark}
这个结论对无限群 \( G \) 也正确, 此时等式两边都是 \( +\infty \).
\end{remark}
\begin{proof}
设 \( a \in G \). 显然, 映射 \( h \to ah \) 是 \( H \) 到 \( aH \) 上的一一对应, 因而 \( |aH| = |H| = [H:1] \). 又由\refthe{theorem:抽象代数-定理 1.3.2}知 \( G = \bigcup\limits_{a \in G} aH \) 为不相交的并, \( \{aH:a\in G\} \) 的不同左陪集个数为 \( [G:H] \), 故式 \(\eqref{eq:1.3.229034890--1}\) 成立.

\end{proof}

\begin{theorem}\label{theorem:抽象代数-定理 1.3.4}
设 \( H \) 是群 \( G \) 的子群, 则 \( G \) 中由
\[
a R b, \ \text{若} \ a^{-1}b \in H
\]
所定义的关系 \( R \) 为同余关系的充分必要条件是
\[
g h g^{-1} \in H, \quad \forall g \in G, h \in H.
\]
此时称 \( H \) 为 \( G \) 的\textbf{正规子群},记为 \( H \lhd  G \). 
同时, 商集合 \( G/H \) 对同余关系 \( R \) 导出的运算
\begin{align*}
aH \cdot bH = abH, \quad \forall a, b \in G
\end{align*}
也构成一个群, 称为 \( G \) 对 \( H \) 的\textbf{商群}.商群 \( G/H \) 的幺元为 \( 1 \cdot H = H \).为方便计, 常将商群 \( G/H \) 中元素记为 \( \overline{g} = gH \).有时也将商群$G/H$记作$\frac{G}{H}$.并且$H$为$G/H$的幺元,$aH$的逆元为$a^{-1}H$.
\end{theorem}
\begin{proof}
设 \( R \) 为同余关系. 又 \( g \in G, h \in H \), 于是有
\[
gRgh, \quad g^{-1}Rg^{-1},
\]
因而 \( gg^{-1}R(ghg^{-1}) \), 即 \( 1  R  ghg^{-1} \), 亦即 \( ghg^{-1} \in H \).

反之, 设 \( \forall g \in G, h \in H \) 有 \( ghg^{-1} \in H \). 设 \( a R b, c R d \), 则$a^{-1}b,c^{-1}d\in H$,即 \( \exists h_1, h_2 \in H \), 使 \( b = a h_1 \), \( d = c h_2 \), 从而$c^{-1}=h_2d^{-1}$.因而
\( (ac)^{-1}(bd)=c^{-1}a^{-1}ah_1d=h_2\left( d^{-1}h_1d \right) \in H\), 则有 \( (ac)  R  (bd) \), 即 \( R \) 为同余关系.

设 \( R \) 为同余关系. 因 \( a \) 所在等价类为 \( aH \), 由\refthe{Set Theory-theorem:同余关系诱导商集中的乘法-定理1.1.3} 知 \( G/H \) 中的乘法为
\begin{align}
aH \cdot bH = abH, \quad \forall a, b \in G. \label{eq:1.3.3}
\end{align}
显然有 \( (aH \cdot bH)cH = abcH = aH(bH \cdot cH) \), \( 1H \cdot aH = aH \), \( a^{-1}H \cdot aH = 1 \cdot H \), 故 \( G/H \) 为群.

\end{proof}

\begin{corollary}\label{corollary:抽象代数-推论 1.3.5}
\begin{enumerate}[(1)]
\item\label{corollary:抽象代数-推论 1.3.5-1} 若 \( G \) 为有限群, \( H \lhd  G \), 商群 \( G/H \) 的阶 \( [G/H : H] = [G : H] = \frac{[G:1]}{[H:1]} \).

\item\label{corollary:抽象代数-推论 1.3.5-2} 若 \( G \) 为无限群, \( H \lhd  G \), 商群 \( G/H \) 的阶 \( [G/H : H] = [G : H] \).
\end{enumerate}
\end{corollary}
\begin{proof}
这是\hyperref[theorem:抽象代数--Lagrange定理]{Lagrange定理}的直接推论. 

\end{proof}

\begin{theorem}\label{theorem:抽象代数-定理 1.3.5}
设 \( H \) 是群 \( G \) 的子群, 则下列条件等价:
\begin{enumerate}[(1)]
\item\label{theorem:抽象代数-定理 1.3.5-1} \( H \lhd  G \);

\item\label{theorem:抽象代数-定理 1.3.5-3} 对$\forall a,b\in G$,如果$ab\in H$,则$ba\in H$;

\item\label{theorem:抽象代数-定理 1.3.5-2} $g h g^{-1} \in H,\forall g \in G, h \in H\iff gHg^{-1}\subseteq H,\forall g\in G$;

\item\label{theorem:抽象代数-定理 1.3.5-4} \( gHg^{-1} = H, \forall g \in G \);

\item\label{theorem:抽象代数-定理 1.3.5-5} \( gH = Hg, \forall g \in G \iff GH=HG\);

\item\label{theorem:抽象代数-定理 1.3.5-6} \( g_1Hg_2H = g_1g_2H, \forall g_1, g_2 \in G \).
\end{enumerate}
\end{theorem}
\begin{remark}
由这个定理可知\textbf{一个群的任意正规子群都是Abel群.}
\end{remark}
\begin{proof}
\ref{theorem:抽象代数-定理 1.3.5-1} $\Rightarrow$ \ref{theorem:抽象代数-定理 1.3.5-3}. 设$H$是$G$的正规子群,则对任意的$a,b\in G$,如果$ab\in H$,则
\begin{align*}
ba = b(ab)b^{-1}\in H.
\end{align*}

\ref{theorem:抽象代数-定理 1.3.5-3} $\Rightarrow$ \ref{theorem:抽象代数-定理 1.3.5-2}.对$\forall a,b\in G$,如果由$ab\in H$,可推出$ba\in H$,则对任意的$a\in G$,
$h\in H$,由于$a^{-1}(ah)=h\in H$,因此
\begin{align*}
aha^{-1} = (ah)a^{-1}\in H,
\end{align*}

\ref{theorem:抽象代数-定理 1.3.5-2} $\Rightarrow$ \ref{theorem:抽象代数-定理 1.3.5-4}.对$\forall g\in G$,由$gHg^{-1}\subseteq H$知,对$\forall h\in H$,有$g^{-1}hg=\left( g^{-1} \right) ^{-1}hg^{-1}\in H.$
从而$h=g\left( g^{-1}hg \right) g^{-1}\in gHg^{-1}.$
故由$h$的任意性知$H\subseteq gHg^{-1}$.因此$gHg^{-1}=H$,$\forall g\in G$.

\ref{theorem:抽象代数-定理 1.3.5-4} $\Rightarrow$ \ref{theorem:抽象代数-定理 1.3.5-5}.  \( \forall g \in G, h \in H \) 有 \( gh = ghg^{-1}g \in Hg, hg = gg^{-1}hg \in gH \), 故 \( gH = Hg \).

\ref{theorem:抽象代数-定理 1.3.5-5} $\Rightarrow$ \ref{theorem:抽象代数-定理 1.3.5-6}. 设 \( g_1, g_2 \in G, h_1, h_2, h \in H \). 由$gH=Hg$知 \( \exists h_1', h' \in H \), 使 \( h_1g_2 = g_2h_1' \), \( g_2h = h'g_2 \). 于是 \( g_1h_1g_2h_2 = g_1g_2h_1'h_2 \in g_1g_2H \), \( g_1g_2h = g_1h'g_2 \cdot 1 \in g_1H \cdot g_2H \), 故 \( g_1H \cdot g_2H = g_1g_2H \).

\ref{theorem:抽象代数-定理 1.3.5-6} $\Rightarrow$ \ref{theorem:抽象代数-定理 1.3.5-1}. 设 \( g \in G, h \in H \), 故有 \( ghg^{-1} \in gHg^{-1}H = gg^{-1}H = H \), 则 \( H \lhd  G \).

\end{proof}

\begin{proposition}\label{proposition:正规子群的基本性质}
\begin{enumerate}[(1)]
\item\label{proposition:正规子群的基本性质-1} Abel 群 \( G \) 的任一子群 \( H \) 都是正规子群, 商群 \( G/H \) 也是 Abel 群.

\item\label{proposition:正规子群的基本性质-2} 若$H$是群$G$的子群且$H\supseteq N,\,N\lhd G$,则$N\lhd H.$

\item\label{proposition:正规子群的基本性质-6} 设$H$和$N$分别是群$G$的子群和正规子群. 证明:$HN$是$G$的子群.

\item\label{proposition:正规子群的基本性质-5} 若$G$是一个群,则$G$的任意正规子群的交$\bigcap_{H\lhd G}{H}\lhd G$.

\item\label{proposition:正规子群的基本性质-3} 设$G$是一个群,且$N_1$, $N_2$, $\cdots$, $N_k\lhd G$,则$N_1 N_2\cdots N_k\lhd G.$

\item 设$H$是群$G$的子群.证明:如果$H$的任一个左陪集也是它的一个右陪集,则$H\lhd G$.
\end{enumerate}
\end{proposition}
\begin{proof}
\begin{enumerate}[(1)]
\item 因为
\begin{align*}
aH = \{ah \mid h \in H\} = \{ha \mid h \in H\} = Ha, \quad \forall a \in G,
\end{align*}
所以$H$是$G$的正规子群.

对$\forall g_1H,g_2H\in G/H$,由\rrefthe{theorem:抽象代数-定理 1.3.5}{theorem:抽象代数-定理 1.3.5-6}有
\begin{align*}
g_1Hg_2H=g_1g_2H=g_2g_1H=g_2Hg_1H.
\end{align*}
故$G/H$也是Abel群.

\item 由\rrefpro{proposition:抽象代数---子群的基本性质}{proposition:抽象代数---子群的基本性质-3}知$N$是$H$的子群.
又由$N \lhd G$知
\begin{align*}
gng^{-1} \in N\subseteq H,\ \forall n \in N,\ g \in H.
\end{align*}
故$N\lhd H.$

\item 由于$N$是$G$的正规子群,因此对任意的$h\in H$,有$hN=Nh$.由此推出,$HN=NH$,从而由\rrefthe{theorem:子群陪集的基本性质}{theorem:子群陪集的基本性质-4}知,$HN$为$G$的子群.

\item 设$I$为任意指标集,$H_i\lhd G$,$\forall i\in I$.则对$\forall g\in G$,有
\begin{align*}
gH_ig^{-1}\subseteq H_i,\quad \forall i\in I.
\end{align*}
对$\forall h\in \bigcap\limits_{i\in I}{H_i}$,有$h\in H_i$,$\forall i\in I$.从而由上式可得
\begin{align*}
ghg^{-1}\in H_i,\quad \forall i\in I\Longrightarrow ghg^{-1}\in \bigcap\limits_{i\in I}{H_i}.
\end{align*}
进而$g\bigcap\limits_{i\in I}{H_i}g^{-1}\subseteq \bigcap\limits_{i\in I}{H_i}$.故由\rrefthe{theorem:抽象代数-定理 1.3.5}{theorem:抽象代数-定理 1.3.5-3}知$\bigcap\limits_{i\in I}{H_i}\lhd G$.

\item 由$N_i\lhd G,i=1,2,\cdots,k$知
\begin{align*}
gn_ig^{-1}\in N_i,\quad \forall n_i\in N_i,g\in G.
\end{align*}
于是对$\forall n_1n_2\cdots n_k\in N_1N_2\cdots N_k,g\in G$,有
\begin{align*}
g\left(n_1n_2\cdots n_k\right)g^{-1}=\left(gn_1g^{-1}\right)\left(gn_2g^{-1}\right)\cdots\left(gn_kg^{-1}\right)\in N_1N_2\cdots N_k.
\end{align*}
故$N_1N_2\cdots N_k\lhd G.$
\end{enumerate}

\end{proof}

\begin{example}
将商群 \( G/H \) 中元素记为 \( \overline{g} = gH \), 则
\begin{enumerate}[(1)]
\item \( SL(V) \lhd  GL(V) \), \( GL(V)/SL(V) = \{\overline{D(\lambda)}|\lambda \neq 0\} \) 且 \( \overline{D(\lambda)}\overline{D(\mu)} = \overline{D(\lambda\mu)} \);

\item \( SO(V) \lhd  O(V) \), \( O(V)/SO(V) = \{\overline{D(1)}, \overline{D(-1)}\} \);

\item \( A_n \lhd  S_n \), \( S_n/A_n = \{\overline{1}, \overline{\sigma}|\sigma \text{ 奇置换} \} \) 且
\[
\overline{1} \cdot \overline{\sigma} = \overline{\sigma} \cdot \overline{1} = \overline{\sigma}, \quad \overline{\sigma} \cdot \overline{\sigma} = \overline{1} \cdot \overline{1} = \overline{1}.
\]

\item 对任意的$a\in G$,由已知条件知,存在$b\in G$,使$aH = Hb$,则$a\in aH = Hb$,从而$a\in Ha\cap Hb$,即$Ha\cap Hb$非空,因此$Ha = Hb$,于是$aH = Ha$.所以由\refthe{theorem:抽象代数-定理 1.3.5}知$H$是$G$的正规子群.
\end{enumerate}
\end{example}
\begin{proof}


\end{proof}

\begin{definition}[极大正规子群]
设$G$是一个群,$H\lhd G$,如果$H$满足以下两个条件:
\begin{enumerate}[(1)]
\item $H\subset G$,即$H\neq G$.

\item 若$K\lhd G$且$H\subseteq K\subseteq G$,则必有$K=H$或$K=G$.
\end{enumerate}
则称$H$是$G$的\textbf{极大正规子群}.
\end{definition}

\begin{definition}
若半群 \( S \) 的非空子集 \( S_1 \) 对 \( S \) 的运算也是半群, 则称 \( S_1 \) 为 \( S \) 的\textbf{子半群}.

若幺半群 \( M \) 的子集 \( Q \) 对 \( M \) 的运算也是幺半群且 \( M \) 的幺元 \( 1 \in Q \), 则称 \( Q \) 为 \( M \) 的\textbf{子幺半群}.
\end{definition}

\begin{theorem}\label{theorem:抽象代数--半群中的同余关系可导出商集合也是商半群}
如果关系 \( \sim \) 是幺半群 (或半群)$G$中的同余关系, 那么商集合$G/\sim$对导出的运算(见\refthe{Set Theory-theorem:同余关系诱导商集中的乘法-定理1.1.3})也是幺半群 (或半群), 称之为\textbf{商幺半群}(或\textbf{商半群}). 

若$G$是交换幺半群(或交换半群),则商集合$G/\sim$对导出的运算也是交换幺半群(或交换半群).
\end{theorem}
\begin{proof}


\end{proof}



























\end{document}
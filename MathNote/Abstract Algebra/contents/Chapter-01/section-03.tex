\documentclass[../../main.tex]{subfiles}
\graphicspath{{\subfix{../../image/}}} % 指定图片目录,后续可以直接使用图片文件名。

% 例如:
% \begin{figure}[H]
% \centering
% \includegraphics[scale=0.4]{图.png}
% \caption{}
% \label{figure:图}
% \end{figure}
% 注意:上述\label{}一定要放在\caption{}之后,否则引用图片序号会只会显示??.

\begin{document}

\section{子群与商群}

\begin{definition}
设 \( A, B \) 是群 \( G \) 的两个子集, 约定
\begin{align*}
AB = \{ab|a \in A, b \in B\} , A^{-1} = \{a^{-1}|a \in A\}.
\end{align*}
特别地, 当 \( A = \{a\} \) 为单点集时, 记 \( AB = aB \), \( BA = Ba \). 当然这些符号对半群与幺半群可同样使用.
\end{definition}

\begin{definition}
群 \( G \) 的非空子集 \( H \) 若对 \( G \) 的运算也构成一个群, 则称为 \( G \) 的\textbf{子群},记作$H<G$.
\end{definition}
\begin{remark}
显然, \( H = \{1\} \)(1 为 \( G \) 的幺元) 与 \( H = G \) 均为 \( G \) 的子群, 称为 \( G \) 的\textbf{平凡子群}, 其他的子群称为\textbf{非平凡子群}.
\end{remark}

\begin{theorem}
设 \( H \) 是群 \( G \) 的非空子集, 则下列条件等价:
\begin{enumerate}[(1)]
\item \( H \) 是 \( G \) 的子群;

\item \( 1 \in H \); \( a \in H \), 则 \( a^{-1} \in H \); \( a, b \in H \), 则 \( ab \in H \);

\item \( a, b \in H \), 则 \( ab \in H \), \( a^{-1} \in H \);

\item \( a, b \in H \), 则 \( ab^{-1} \in H \).
\end{enumerate}
\end{theorem}
\begin{proof}
\( (1) \Rightarrow (2) \). 由 \( H \) 对 \( G \) 的乘法构成群知 \( a, b \in H \), 则 \( ab \in H \). 又 \( H \) 有幺元 \( 1' \), 即有 \( 1' \cdot 1' = 1' \). 设 \( 1' \) 在 \( G \) 中的逆元为 \( 1'^{-1} \), 则有
\[
1 = 1' \cdot 1'^{-1} = (1' \cdot 1') \cdot 1'^{-1} = 1',
\]
故 \( 1 \in H \). 设 \( a \) 在 \( H \) 中的逆元为 \( a' \), 于是 \( aa' = 1' = 1 \), 即 \( a' = a^{-1} \), 故 \( a^{-1} \in H \). 由此知 \( (2) \) 成立, 而且 \( H \) 的幺元是 \( G \) 的幺元. \( a \in H \), \( a \) 在 \( H \) 中的逆元与在 \( G \) 中的逆元一致.

\( (2) \Rightarrow (3) \). 这是显然的.

\( (3) \Rightarrow (4) \). 若 \( a, b \in H \), 故 \( a, b^{-1} \in H \), 故 \( ab^{-1} \in H \).

\( (4) \Rightarrow (1) \). 由 \( H \neq \varnothing \) 知 \( \exists a \in H \), 因而 \( 1 = aa^{-1} \in H \). 又由 \( 1, a \in H \) 知 \( a^{-1} = 1 \cdot a^{-1} \in H \). 又若 \( a, b \in H \), 由 \( b^{-1} \in H \) 得 \( ab = a(b^{-1})^{-1} \in H \). 由此可知 \( G \) 的乘法也是 \( H \) 的乘法. 对 \( H \) 而言有幺元 \( 1 \); 对 \( a \in H \) 有逆元 \( a^{-1} \); 结合律显然成立. 故 \( H \) 是 \( G \) 的子群.
\end{proof}

\begin{corollary}
设 \( H \) 是群 \( G \) 的非空子集, 则下列条件等价:
\begin{enumerate}[(1)]
\item \( H \) 是 \( G \) 的子群;

\item \( HH = H \), \( H^{-1} = H \);

\item \( H^{-1}H = H \).
\end{enumerate}
\end{corollary}
\begin{proof}

\end{proof}

\begin{corollary}
\begin{enumerate}
\item 若 \( H_1, H_2 \) 是群 \( G \) 的子群, 则 \( H_1 \cap H_2 \) 也是 \( G \) 的子群.

\item 若$G$是一个群,则$G$的任意子群的交$\bigcap_{H<G}{H}$也是$G$的子群.
\end{enumerate}
\end{corollary}
\begin{proof}

\end{proof}

\begin{example}
\begin{enumerate}
\item 设 \( V \) 是数域 \( \mathbf{P} \) 上的 \( n \) 维线性空间.\( S_V \) 为 \( V \) 上的全变换群, \( GL(V) \) 表示 \( V \) 上所有可逆线性变换的集合, 则 \( GL(V) \) 为 \( S_V \) 的子群, 称为线性空间 \( V \) 的\textbf{一般线性群}. 

又设 \( SL(V) \) 为 \( V \) 上所有行列式等于 1 的线性变换的集合, 则 \( SL(V) \) 是 \( GL(V) \)(同时也是 \( S_V \)) 的子群, 称为\textbf{特殊线性群}.

\item 设 \( V \) 是 \( n \) 维 Euclid 空间. 以 \( O(V) \) 表示 \( V \) 上所有正交变换的集合, \( SO(V) \) 表示所有行列式等于 1 的正交变换的集合, 则 \( O(V) \) 是 \( GL(V) \) 的子群, \( SO(V) \) 是 \( O(V) \) 的子群. \( O(V) \) 称为 \( V \) 的\textbf{正交变换群}, 简称\textbf{正交群}, \( SO(V) \) 称为\textbf{转动群}(或\textbf{特殊正交变换群}、\textbf{特殊正交群}).
\end{enumerate}
\end{example}
\begin{remark}
将上述$S_V$换成数域$\mathbf{P}$上的全体方阵构成的乘法群,线性变换换成方阵,结论也成立.
\end{remark}
\begin{proof}

\end{proof}

\begin{example}
设 \( m \in \mathbf{Z} \), 则 \( m\mathbf{Z} = \{mx|x \in \mathbf{Z}\} \) 是整数加法群 \( \mathbf{Z} \) 的子群. 并且\( \mathbf{Z} \) 的任何子群都是这样的子群.
\end{example}
\begin{proof}

\end{proof}

\begin{example}
先考虑 \( n \) 个不定元 \( x_1, x_2, \cdots, x_n \) 的多项式
\[
A = \prod_{1 \leqslant i < j \leqslant n} (x_i - x_j) \in \mathbf{C}[x_1, x_2, \cdots, x_n].
\]
对于 \( \sigma \in S_n \), 令
\[
A_\sigma = \prod_{1 \leqslant i < j \leqslant n} (x_{\sigma(i)} - x_{\sigma(j)}),
\]
则$A_{\sigma}=\pm A$.若 \( A_\sigma = A \), 则称 \( \sigma \) 为\textbf{偶置换}, 并记 \( \text{sgn}\sigma = 1 \); 若 \( A_\sigma = -A \), 则称 \( \sigma \) 为\textbf{奇置换}, 并记 \( \text{sgn}\sigma = -1 \), \( \text{sgn}\sigma \) 称为 \( \sigma \) 的\textbf{符号}. 故有
\[
A_\sigma = \text{sgn}\sigma A.
\]
令 \( A_n \) 为 \( S_n \) 中偶置换集合, 即
\[
A_n = \{\sigma \in S_n|\text{sgn}\sigma = 1\},
\]
则 \( A_n \) 为 \( S_n \) 的子群.\( A_n \) 称为 \( n \) 个文字的\textbf{交错群}.
\end{example}
\begin{proof}
先证明 \( A_\sigma = \pm A \). 注意到 \( A \) 中没有 \( x_i - x_j \) 的重因式, 因而只需说明 \( A_\sigma \) 中没有重因式即可. 设有 \( \{\sigma(i), \sigma(j)\} = \{\sigma(k), \sigma(l)\} \), 则有如下两种可能:

(1) \( \sigma(i) = \sigma(k), \sigma(j) = \sigma(l) \), 则有 \( i = k, j = l \);

(2) \( \sigma(i) = \sigma(l), \sigma(j) = \sigma(k) \), 则有 \( i = l, j = k \),

因而都有 \( \{i, j\} = \{k, l\} \), 由此知 \( A_\sigma = \pm A \).

事实上, 若 \( \tau, \sigma \in S_n \), 则有
\[
A_{\sigma\tau} = \prod_{1 \leqslant i < j \leqslant n} (x_{\sigma\tau(i)} - x_{\sigma\tau(j)}).
\]
将 \( A_{\sigma\tau} \) 与 \( A_\sigma \) 进行比较. 若 \( \tau(i) < \tau(j) \), 则 \( x_{\sigma\tau(i)} - x_{\sigma\tau(j)} \) 仍是 \( A_\sigma \) 中一个因子; 若 \( \tau(i) > \tau(j) \), 则 \( x_{\sigma\tau(j)} - x_{\sigma\tau(i)} = -(x_{\sigma\tau(i)} - x_{\sigma\tau(j)}) \) 为 \( A_\sigma \) 中一因子, 因而将 \( A_\sigma \) 变成 \( A_{\sigma\tau} \) 时改变因子符号的次数与将 \( A \) 变成 \( A_\tau \) 时改变因子符号的次数相同, 因而有
\[
A_{\sigma\tau} = \text{sgn}\tau \cdot \prod_{1 \leqslant i < j \leqslant n} (x_{\sigma(i)} - x_{\sigma(j)}) = \text{sgn}\sigma\text{sgn}\tau A.
\]
于是
\begin{align*}
\text{sgn}(\sigma\tau) = \text{sgn}\sigma\text{sgn}\tau, \quad \forall \sigma, \tau \in S_n.
\end{align*}
又注意到$\text{sgn}\tau^{-1}=\text{sgn}\tau,  \forall \tau \in S_n$,故
\begin{align*}
\text{sgn}(\sigma\tau^{-1}) = \text{sgn}\sigma\text{sgn}\tau^{-1}=\text{sgn}\sigma\text{sgn}\tau=1\Longrightarrow \sigma\tau^{-1}\in A_n, \quad \forall \sigma, \tau \in A_n.
\end{align*}
由此知 \( A_n \) 为 \( S_n \) 的子群.
\end{proof}

\begin{definition}
设 \( H \) 是群 \( G \) 的子群, 又 \( a \in G \). 集合 \( aH \) 与 \( Ha \) 分别称为以 \( a \) 为代表的 \( H \) 的\textbf{左陪集}与\textbf{右陪集}.
\end{definition}

\begin{theorem}\label{theorem:抽象代数-定理 1.3.2}
设 \( H \) 是群 \( G \) 的子群, 则由
\[
aRb, \ \text{若} \ a^{-1}b \in H
\]
所确定的 \( G \) 中的关系 \( R \) 是一个等价关系, 并且 \( a \) 所在的等价类为 \( aH:a\in G \), 故 \( H \)的左陪集族 \( \{aH:a\in G\} \)(集合无相同元素) 是 \( G \) 的一个分划.
\end{theorem}
\begin{proof}
由 \( a^{-1}a \in H \) 知 \( aRa(\forall a \in G) \). 又设 \( aRb \), 即 \( a^{-1}b \in H \), 故 \( (a^{-1}b)^{-1} = b^{-1}a \in H \), 即 \( bRa \). 再设 \( aRb, cRb \), 即 \( a^{-1}b, b^{-1}c \in H \), 故 \( a^{-1}c = (a^{-1}b)(b^{-1}c) \in H \), 即 \( aRc \). 这样知 \( R \) 是等价关系. 又由 \( b = a(a^{-1}b) \) 知
\[
aRb \iff \ a^{-1}b \in H \iff \ b \in aH,
\]
故 \( a \) 所在的等价类为 \( aH \). 由\refthe{theorem:等价类就和集合的分划对应-定理1.1.1}知 \( \{aH:a\in G\} \) 为 \( G \) 的一个分划.
\end{proof}

\begin{corollary}\label{corollary:抽象代数-推论 1.3.3}
设 \( H \) 是群 \( G \) 的子群, 则下列条件等价:
\begin{enumerate}[(1)]
\item \( aH \cap bH \neq \varnothing \);

\item \( aH = bH \);

\item \( a^{-1}b \in H \),
\end{enumerate}
而且 \( G = \bigcup\limits_{a \in G} aH \) 为不相交的并.
\end{corollary}
\begin{proof}

\end{proof}

\begin{definition}
设$H$是群$G$的子群,由\refthe{theorem:抽象代数-定理 1.3.2}定义$G$中的等价关系$R$为
\[
aRb, \ \text{若} \ a^{-1}b \in H.
\]
将 \( G \) 对等价关系$R$的商集合, 即以左陪集$aH,\,a\in G$为元素的集合记为 \( G/H=\{aH:a\in G\} \), 称为 \( G \) 对 \( H \) 的\textbf{左陪集空间}. \( G/H \) 中元素个数 \( |G/H| \) 称为 \( H \) 在 \( G \) 中的\textbf{指数}, 记为 \( [G:H] \). 相应可定义\textbf{右陪集空间}.
\end{definition}
\begin{remark}
\(\{1\}\) 作为 \( G \) 的子群, 在 \( G \) 中指数显然为 \( |G| \). 故也记 \( |G| = [G: 1] \).
\end{remark}

\begin{example}
设 \( V \) 是数域 \( \mathbf{P} \) 上的 \( n \) 维线性空间, \( GL(V) \) 有子群 \( SL(V) \). 在 \( V \) 中取定一组基, 任何一个线性变换由它在这组基下的矩阵完全确定, 可把它们等同起来.\( \forall \lambda \in \mathbf{P}, \lambda \neq 0, \) 令 \( D(\lambda) = \text{diag}(\lambda, 1, \cdots, 1) \), 于是 \( D(\lambda) \in GL(V) \), 对于 \( A \in GL(V) \) 有
\[
A SL(V) = D(\lambda) SL(V)\iff \ \det A = \lambda.
\]
于是
\[
GL(V) = \bigcup_{\lambda \neq 0} D(\lambda) SL(V),
\]
因而
\[
[GL(V): SL(V)] = +\infty.
\]
\end{example}
\begin{proof}

\end{proof}

\begin{example}
设 \( V \) 是 \( n \) 维 Euclid 空间. 由 \( A \in O(V) \) 有 \( \det A = \pm 1 \),令 \( D(\lambda) = \text{diag}(\lambda, 1, \cdots, 1) \),  于是
\[
O(V) = SO(V) \bigcup D(-1) SO(V),
\quad
[O(V): SO(V)] = 2.
\]
\end{example}
\begin{proof}

\end{proof}

\begin{example}
设 \( m > 0, m\mathbf{Z} \) 为 \( \mathbf{Z} \) 的子群, 则有
\[
\mathbf{Z} = \bigcup_{k=0}^{m-1} (k + m\mathbf{Z}), \quad [\mathbf{Z}: m\mathbf{Z}] = m.
\]
\end{example}
\begin{proof}

\end{proof}

\begin{example}
设 \( \sigma \) 是 \( S_n \) 中任一奇置换, 则有 \( S_n = A_n \cup \sigma A_n \), 故 \( [S_n: A_n] = 2 \).
\end{example}
\begin{proof}

\end{proof}

\begin{theorem}[Lagrange定理]\label{theorem:抽象代数-Lagrange定理-定理 1.3.3}
设 \( H \) 是有限群 \( G \) 的子群, 则有
\begin{align}
[G:1] = [G:H][H:1] \label{eq:1.3.229034890--1}
\end{align}
因而子群 \( H \) 的阶是群 \( G \) 的阶的因子.
\end{theorem}
\begin{remark}
这个结论对无限群 \( G \) 也正确, 此时等式两边都是 \( +\infty \).
\end{remark}
\begin{proof}
设 \( a \in G \). 显然, 映射 \( h \to ah \) 是 \( H \) 到 \( aH \) 上的一一对应, 因而 \( |aH| = |H| = [H:1] \). 又由\refcor{corollary:抽象代数-推论 1.3.3}知 \( G = \bigcup\limits_{a \in G} aH \) 为不相交的并, \( \{aH\} \) 的不同左陪集个数为 \( [G:H] \), 故式 \(\eqref{eq:1.3.229034890--1}\) 成立.
\end{proof}

\begin{corollary}\label{corollary:抽象代数-推论 1.3.4}
有限群 \( G \) 的任一元素 \( a \) 的阶是 \( G \) 的阶的因子.
\end{corollary}
\begin{proof}
令 \( \langle a \rangle = \{a^n|n \in \mathbf{Z}\} \), 容易验证这是 \( G \) 的一个子群. 又由于 \( G \) 有限, 故 \( \langle a \rangle \) 有限, 因而 \( a \) 是有限阶的, 设为 \( d \). 对 \( n \in \mathbf{Z} \) 有 \( t_n \) 与 \( r_n \) (\( 0 \leqslant r_n < d \)), 使 \( n = t_nd + r_n \), 于是 \( a^n = a^{r_n} \).因此$\langle a \rangle$中至多只有$d$个元素$1,a,\cdots,a^{d-1}$.

又对 \( \forall r_1, r_2 \in \mathbf{N} \), 且 \( r_1 \neq r_2 \), \( 0 \leqslant r_1, r_2 < d \), 则 \( |r_1 - r_2| < d \), 从而 \( a^{r_1 - r_2} \neq 1 \), 进而 \( a^{r_1} \neq a^{r_2} \). 故 \( 1, a, \cdots, a^{d-1} \) 互不相同.
由此知 \( \langle a \rangle = \{1, a, \cdots, a^{d-1}\} \), 即 \( \langle a \rangle \) 是 \( d \) 阶群. 故由\hyperref[theorem:抽象代数-Lagrange定理-定理 1.3.3]{Lagrange定理}知 \( d \) 为 \( [G:1] \) 的因子.
\end{proof}

\begin{definition}[循环群]
我们称
\begin{align*}
\langle a \rangle = \{a^n|n \in \mathbf{Z}\}
\end{align*}
是由 \( a \) 生成的 \( G \) 的子群, 如果在一个群 \( G \) 中存在一个元素 \( a \), 使得 \( G = \langle a \rangle \), 即 \( G \) 由 \( a \) 生成, 则称 \( G \) 是\textbf{循环群}, \( a \) 为 \( G \) 的一个\textbf{生成元}.
\end{definition}

\begin{theorem}\label{theorem:抽象代数-定理 1.3.4}
设 \( H \) 是群 \( G \) 的子群, 则 \( G \) 中由
\[
a R b, \ \text{当} \ a^{-1}b \in H
\]
所定义的关系 \( R \) 为同余关系的充分必要条件是
\[
g h g^{-1} \in H, \quad \forall g \in G, h \in H.
\]
此时称 \( H \) 为 \( G \) 的\textbf{正规子群},记为 \( H \lhd  G \). 
同时, 商集合 \( G/H \) 对同余关系 \( R \) 导出的运算
\begin{align*}
aH \cdot bH = abH, \quad \forall a, b \in G
\end{align*}
也构成一个群, 称为 \( G \) 对 \( H \) 的\textbf{商群}.商群 \( G/H \) 的幺元为 \( 1 \cdot H = H \).
\end{theorem}
\begin{proof}
设 \( R \) 为同余关系. 又 \( g \in G, h \in H \), 于是有
\[
gRgh, \quad g^{-1}Rg^{-1},
\]
因而 \( gg^{-1}R(ghg^{-1}) \), 即 \( 1  R  ghg^{-1} \), 亦即 \( ghg^{-1} \in H \).

反之, 设 \( \forall g \in G, h \in H \) 有 \( ghg^{-1} \in H \). 设 \( a R b, c R d \), 则$a^{-1}b,c^{-1}d\in H$,即 \( \exists h_1, h_2 \in H \), 使 \( b = a h_1 \), \( d = c h_2 \), 从而$c^{-1}=h_2d^{-1}$.因而
\( (ac)^{-1}(bd)=c^{-1}a^{-1}ah_1d=h_2\left( d^{-1}h_1d \right) \in H\), 则有 \( (ac)  R  (bd) \), 即 \( R \) 为同余关系.

设 \( R \) 为同余关系. 因 \( a \) 所在等价类为 \( aH \), 由\refthe{theorem:同余关系诱导商集中的乘法-定理1.1.3} 知 \( G/H \) 中的乘法为
\begin{align}
aH \cdot bH = abH, \quad \forall a, b \in G. \label{eq:1.3.3}
\end{align}
显然有 \( (aH \cdot bH)cH = abcH = aH(bH \cdot cH) \), \( 1H \cdot aH = aH \), \( a^{-1}H \cdot aH = 1 \cdot H \), 故 \( G/H \) 为群.
\end{proof}

\begin{corollary}\label{corollary:抽象代数-推论 1.3.5}
若 \( G \) 为有限群, \( H \lhd  G \), 商群 \( G/H \) 的阶 \( [G/H : H] = [G : H] = \frac{[G:1]}{[H:1]} \).
\end{corollary}
\begin{proof}
这是\hyperref[theorem:抽象代数-Lagrange定理-定理 1.3.3]{Lagrange定理}的直接推论. 当 \( G \) 为无限群时, \( [G/H : H] = [G : H] \) 仍然成立.
\end{proof}
\textbf{} 

\begin{theorem}\label{theorem:抽象代数-定理 1.3.5}
设 \( H \) 是群 \( G \) 的子群, 则下列条件等价:
\begin{enumerate}[(1)]
\item \( H \lhd  G \);

\item \( gHg^{-1} = H, \forall g \in G \);

\item \( gH = Hg, \forall g \in G \);

\item \( g_1Hg_2H = g_1g_2H, \forall g_1, g_2 \in G \).
\end{enumerate}
\end{theorem}
\begin{proof}
\( (1) \Rightarrow (2) \). \( g \in G, h \in H \), 则由 \( H \lhd  G \) 有 \( ghg^{-1} \in H \), 又 \( h = g(g^{-1}hg)g^{-1} \in gHg^{-1} \), 故有 \( gHg^{-1} = H \).

\( (2) \Rightarrow (3) \). \( \forall g \in G, h \in H \) 有 \( gh = ghg^{-1}g \in Hg, hg = gg^{-1}hg \in gH \), 故 \( gH = Hg \).

\( (3) \Rightarrow (4) \). 设 \( g_1, g_2 \in G, h_1, h_2, h \in H \). 由条件 (3) 成立知 \( \exists h_1', h' \in H \), 使 \( h_1g_2 = g_2h_1' \), \( g_2h = h'g_2 \). 于是 \( g_1h_1g_2h_2 = g_1g_2h_1'h_2 \in g_1g_2H \), \( g_1g_2h = g_1h'g_2 \cdot 1 \in g_1H \cdot g_2H \), 故 \( g_1H \cdot g_2H = g_1g_2H \).

\( (4) \Rightarrow (1) \). 设 \( g \in G, h \in H \), 故有 \( ghg^{-1} \in gHg^{-1}H = gg^{-1}H = H \), 则 \( H \lhd  G \).
\end{proof}

\begin{proposition}
Abel 群 \( G \) 的任一子群 \( H \) 都是正规子群, 商群 \( G/H \) 也是 Abel 群.
\end{proposition}
\begin{proof}

\end{proof}

\begin{example}
为方便计, 将商群 \( G/H \) 中元素记为 \( \bar{g} = gH \), 则
\begin{enumerate}[(1)]
\item \( SL(V) \lhd  GL(V) \), \( GL(V)/SL(V) = \{\overline{D(\lambda)}|\lambda \neq 0\} \) 且 \( \overline{D(\lambda)}\overline{D(\mu)} = \overline{D(\lambda\mu)} \);

\item \( SO(V) \lhd  O(V) \), \( O(V)/SO(V) = \{\overline{D(1)}, \overline{D(-1)}\} \);

\item \( m\mathbf{Z} \lhd  \mathbf{Z} \), \( \mathbf{Z}/m\mathbf{Z} = \mathbf{Z}_m = \{\overline{0}, \overline{1}, \cdots, \overline{m-1}\} \);

\item \( A_n \lhd  S_n \), \( S_n/A_n = \{\overline{1}, \overline{\sigma}|\sigma \text{ 奇置换} \} \) 且
\[
\overline{1} \cdot \overline{\sigma} = \overline{\sigma} \cdot \overline{1} = \overline{\sigma}, \quad \overline{\sigma} \cdot \overline{\sigma} = \overline{1} \cdot \overline{1} = \overline{1}.
\]
\end{enumerate}
\end{example}

\begin{definition}
若半群 \( S \) 的非空子集 \( S_1 \) 对 \( S \) 的运算也是半群, 则称 \( S_1 \) 为 \( S \) 的\textbf{子半群}.

若幺半群 \( M \) 的子集 \( Q \) 对 \( M \) 的运算也是幺半群且 \( M \) 的幺元 \( 1 \in Q \), 则称 \( Q \) 为 \( M \) 的\textbf{子幺半群}.

如果关系 \( \sim \) 是幺半群 (或半群) 中的同余关系, 那么商集合对导出的运算也是幺半群 (或半群), 称之为\textbf{商幺半群}(或\textbf{商半群}). 
\end{definition}





























\end{document}
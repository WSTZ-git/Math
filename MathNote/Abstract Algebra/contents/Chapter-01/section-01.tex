\documentclass[../../main.tex]{subfiles}
\graphicspath{{\subfix{./image/}}} % 指定图片目录,后续可以直接使用图片文件名
% 注意这里的文件路径不能用 ../../image/ ,否则用latexmk编译子文件会报错

% 例如:
% \begin{figure}[H]
% \centering
% \includegraphics[scale=0.4]{图.png}
% \caption{}
% \label{figure:图}
% \end{figure}
% 注意:上述\label{}一定要放在\caption{}之后,否则引用图片序号会只会显示??.

\begin{document}

\section{二元运算}

\begin{definition}[数域]
设$P$是由一些复数组成的集合,其中包括$0$和$1$。如果$P$中任意两个数的和、差、积、商(除数不为$0$)仍然是$P$中的数,那么$P$就称为一个\textbf{数域}。
\end{definition}

\begin{definition}
设 \( A \) 是一个集合. \( A \times A \) 到 \( A \) 的一个映射 \( \varphi \), 称为 \( A \) 的一个\textbf{二元运算}.

若记 \( \varphi(a,b) = ab \), 则称 \( ab \) 为 \( a \) 与 \( b \) 的\textbf{积}. 若记 \( \varphi(a,b) = a + b \), 则称 \( a + b \) 为 \( a \) 与 \( b \) 的\textbf{和}.

若 \( A \) 上的二元运算 \( \varphi(a,b) = ab \) 满足结合律
\[
(ab)c = a(bc),\quad \forall a,b,c \in A,
\]
则此二元运算称为\textbf{结合的}.

若 \( A \) 上的二元运算 \( \varphi(a,b) = ab \) 满足交换律
\[
ab = ba,\quad \forall a,b \in A,
\]
则此二元运算称为\textbf{交换的}. 一般地, 若 \( c,d \in A \) 有 \( cd = dc \), 则称 \( c \) 与 \( d \) 是\textbf{交换的}.
\end{definition}

\begin{definition}\label{definition:n次乘幂}
设集合 \( A \) 有二元运算 \( (a,b) \to ab \) 且满足结合律, 则对 \( \forall n \in \mathbb{N} \) (\( \mathbb{N} \) 表示自然数, 即正整数的集合), 定义
\[
a^1 = a,\quad a^{n + 1} = a^n \cdot a,\quad \forall a \in A,
\]
\( a^n \) 称为 \( a \) 的  \( \boldsymbol{n} \) \textbf{次乘幂}, 也简称 \( \boldsymbol{n} \) \textbf{次幂}.

在 \( A \) 中也可以定义\textbf{连乘积}
\[
\prod_{i = 1}^n a_i = \left( \prod_{i = 1}^{n - 1} a_i \right) a_n,\quad a_i \in A, i = 1,2,\cdots, n.
\]
\end{definition}

\begin{proposition}\label{proposition:乘幂和连乘积的性质}
\begin{enumerate}
\item \( a^n a^m = a^{n + m}, (a^m)^n = a^{nm} (\forall a \in A, m,n \in \mathbb{N}) \). 

\item 若 \( a,b \in A \) 且 \( ab = ba \), 则 \( (ab)^n = a^n b^n (\forall n \in \mathbb{N}) \).

\item 若有
\[
0 = n_0 < n_1 < \cdots < n_r = n,
\]
则
\[
\prod_{j = 1}^r \left( \prod_{k = n_{j - 1} + 1}^{n_j} a_k \right) = \prod_{i = 1}^n a_i.
\]
\end{enumerate}
\end{proposition}
\begin{proof}
证明是显然的.

\end{proof}

\begin{definition}\label{definition:倍数的定义}
如果将二元运算记为加法且满足结合律, 于是可定义\textbf{倍数}与\textbf{连加}如下:
\[
1 \cdot a = a,\quad (n + 1)a = na + a,
\]
\[
\sum_{i = 1}^n a_i = \left( \sum_{i = 1}^{n - 1} a_i \right) + a_n.
\]
\end{definition}

\begin{proposition}\label{proposition:倍数和连加的性质}
\begin{enumerate}
\item $na + ma = (n + m)a,\quad n(ma) = (nm)a,\quad \forall a \in A, m,n \in \mathbb{N}.$

\item 若 \( a + b = b + a \), 则
\[
n(a + b) = na + nb,\quad \forall n \in \mathbb{N},
\]

\item 若有
\[
0 = n_0 < n_1 < \cdots < n_r = n,
\]
则
\[
\sum_{j = 1}^r \left( \sum_{k = n_{j - 1} + 1}^{n_j} a_k \right) = \sum_{i = 1}^n a_i.
\]
\end{enumerate}
\end{proposition}
\begin{proof}
证明是显然的.

\end{proof}







\end{document}
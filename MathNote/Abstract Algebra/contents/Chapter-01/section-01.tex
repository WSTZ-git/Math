\documentclass[../../main.tex]{subfiles}
\graphicspath{{\subfix{../../image/}}} % 指定图片目录,后续可以直接使用图片文件名。

% 例如:
% \begin{figure}[H]
% \centering
% \includegraphics[scale=0.4]{图.png}
% \caption{}
% \label{figure:图}
% \end{figure}
% 注意:上述\label{}一定要放在\caption{}之后,否则引用图片序号会只会显示??.

\begin{document}

\section{半群、幺半群和群}

\begin{definition}[二元运算]\label{definition:二元运算定义}
如果$G$是一个非空集合,每个函数$G \times G \to G$叫作$G$上的一个\textbf{二元运算}。对$(a, b)$在一个二元运算之下的象有许多常用的记号:$ab$(乘法记号),$a + b$(加法记号),$a \cdot b$,$a \times b$等等。
\end{definition}
\begin{remark}
为方便起见,我们在本章中一般采用乘法记号,并且把$ab$叫作$a$和$b$的\textbf{积}。一个集合上可以有许多个不同的二元运算(例如在$\mathbb{Z}$上由$(a, b) \longmapsto a + b$和$(a, b) \longmapsto ab$分别给出\textbf{通常的加法}和\textbf{乘法运算})。
\end{remark}

\begin{definition}[半群和交换半群]
设$G$是一个非空集合,如果$G$上满足

(i) 结合律:$a(bc) = (ab)c$(对所有$a, b, c \in G$)的一个二元运算,便称$G$是一个\textbf{半群}。

如果一个半群$G$包含有一个

(ii)\textbf{(双侧)幺元素}$e \in G$,使得$ae = ea = a$(对所有$a \in G$),便称$G$是一个\textbf{幺半群}。

如果幺半群$G$满足

(iii) 对于每个$a \in G$均存在\textbf{(双侧)逆元素}$a^{-1} \in G$,使得$a^{-1}a = aa^{-1} = e$,便称$G$是一个\textbf{群}。

如果半群$G$的二元运算满足

(iv) 交换律:$ab = ba$(对所有$a, b \in G$),便称$G$为\textbf{交换半群}或者 \textbf{Abel 半群}。
\end{definition}
\begin{remark}
如果$G$是幺半群而其上的二元运算写成乘法,则$G$的幺元素永远写成$e$。如果二元运算写成加法,则$a + b$($a, b \in G$)叫做$a$与$b$的和,并且幺元素写成$0$。这时又如果$G$是群,则$a \in G$的逆元素表示成$-a$。我们以$a - b$表示$a + (-b)$。Abel 群常常写成加法形式。
\end{remark}

\begin{example}
$\left( M_n\left( \mathbb{R} \right) ,\cdot \right)$是一个含幺(乘法)半群.
\end{example}
\begin{proof}
\(\forall A,B,C\in (M_n(\mathbb{R}),\cdot)\),则不妨设\(A=(a_{ij})_{n\times n}\),\(B=(b_{ij})_{n\times n}\),\(C=(c_{ij})_{n\times n}\).再设\(A\cdot B=(d_{ij})_{n\times n}\),\(B\cdot C=(e_{ij})_{n\times n}\),\((A\cdot B)\cdot C=(f_{ij})_{n\times n}\),\(A\cdot (B\cdot C)=(g_{ij})_{n\times n}\).于是
\[
d_{ij}=\sum_{k = 1}^n{a_{ik}b_{kl}},e_{ij}=\sum_{k = 1}^n{b_{ik}c_{kl}}.
\]
其中\(i,j = 1,2,\cdots,n\).

从而
\[
f_{ij}=\sum_{l = 1}^n{d_{il}c_{lj}}=\sum_{l = 1}^n{\left(\sum_{k = 1}^n{a_{ik}b_{kl}}\right)\cdot c_{lj}}=\sum_{l = 1}^n{\sum_{k = 1}^n{a_{ik}b_{kl}c_{lj}}},
\]
\[
g_{ij}=\sum_{k = 1}^n{a_{ik}e_{kj}}=\sum_{k = 1}^n{a_{ik}\cdot\left(\sum_{l = 1}^n{b_{kl}c_{lj}}\right)}=\sum_{k = 1}^n{\sum_{l = 1}^n{a_{ik}b_{kl}c_{lj}}}.
\]
由二重求和号的可交换性,可知\(f_{ij}=g_{ij}\),\(\forall i,j\in \{1,2,\cdots,n\}\).故\((A\cdot B)\cdot C = A\cdot (B\cdot C)\).

记\(I_n=\begin{pmatrix}
1 & & & \\
& 1 & & \\
& & \ddots & \\
& & & 1\\
\end{pmatrix}\in M_n(\mathbb{R})\),于是\(\forall X\in M_n(\mathbb{R})\),则不妨设\(X=(x_{ij})_{n\times n}\),\(I_n = (\delta_{ij})_{n\times n}\).其中\(\delta_{ij}=\begin{cases}
1,\text{当 }i = j\text{ 时},\\
0,\text{当 }i\neq j\text{ 时},
\end{cases}\).
再设\(I_n\cdot X=(x_{ij}')_{n\times n}\),\(X\cdot I_n=(x_{ij}'')_{n\times n}\),于是由矩阵乘法的定义可知
\[
x_{ij}'=\sum_{k = 1}^n{x_{ik}\delta_{kj}}=x_{ij}\delta_{jj}=x_{ij}.
\]
\[
x_{ij}''=\sum_{k = 1}^n{\delta_{ik}x_{kj}}=\delta_{ii}x_{ij}=x_{ij}.
\]
故\(x_{ij}'=x_{ij}''=x_{ij}\),\(\forall i,j\in \{1,2,\cdots,n\}\).从而\(X = I_n\cdot X = X\cdot I_n\).因此\(I_n\)是\((M_n(\mathbb{R}),\cdot)\)的单位元.综上所述,$\left( M_n\left( \mathbb{R} \right) ,\cdot \right)$是一个含幺(乘法)半群.
\end{proof}

\begin{example}[$\,\,$常见的群]
\begin{enumerate}
\item 我们称只有一个元素的群为\textbf{平凡群},记作${e}$.其中的二元运算是$e\cdot e=e$.
\item 常见的加法群有$(\mathbb{Z},+),$ $(\mathbb{Q},+),$ $(\mathbb{R},+),$ $(\mathbb{C},+)$等.这些加法群分别称为整数加群、有理数加群、实数加群、复数加群.
\item 常见的乘法群有$(\mathbb{Q}^\times,+),$ $(\mathbb{R}^\times,+),$ $(\mathbb{C}^\times,+)$等, 其中$\mathbb{Q}^\times=\mathbb{Q}\backslash {0}$,类似地定义其余两个集合.这些乘法群分别称为有理数乘群、实数乘群、复数称群.
\item 在向量空间中,$n$维欧式空间对加法构成群即$(\mathbb{R}^n,+)$.类似地$(\mathbb{C}^n,+),$ $(\mathbb{Q}^n,+),$ $(\mathbb{Z}^n,+)$也是群.对于这些群,单位元都是零向量,加法逆元则是对每个坐标取相反数,如$(x_1,\cdots,x_n)$的加法逆元是$(-x_1,\cdots,-x_n)$.
\item 所有的$m\times n$矩阵也对加法构成群,单位元都是零矩阵,加法逆元则是对每一项取相反数.对于$n\times n$的实矩阵加法群,我们记作$(M(n,\mathbb{R}),+)$,类似地我们将$n\times n$的复矩阵加法群记作$(M(n,\mathbb{C}),+)$.
\end{enumerate}
\end{example}
\begin{proof}
证明都是显然的.
\end{proof}

\begin{definition}[阶]
势$|G|$叫作群$G$的\textbf{阶}。如果$|G|$是有限的或者是无限的,则群$G$也分别叫做\textbf{有限的}或者\textbf{无限的},也分别叫做\textbf{有限群}或者\textbf{无限群}. 
\end{definition}

\begin{theorem}\label{theorem:(幺半)群的基本性质}
(1)如果$G$是幺半群,则幺元素$e$是唯一的。

(2)如果$G$是群,则
\begin{enumerate}[(i)]
\item $c \in G$并且$cc = c \Rightarrow c = e$;

\item 对于所有的$a, b, c \in G$,$ab = ac \Rightarrow b = c$,同样地$ba = ca \Rightarrow b = c$(左消去律和右消去律);

\item 对于每个$a \in G$,逆元素$a^{-1}$是唯一的;

\item 对于每个$a \in G$,$(a^{-1})^{-1} = a$;

\item 对于$a, b \in G$,$(ab)^{-1} = b^{-1}a^{-1}$;

\item 对于$a, b \in G$,方程$ax = b$和$ya = b$均在$G$中有唯一解:$x = a^{-1}b$,$y = ba^{-1}$。
\end{enumerate}
\end{theorem}
\begin{proof}
(1)若$e'$也是(双侧)幺元素,则$e'=e'e=e$.下证(2).

\begin{enumerate}[(i)]
\item $cc = c \Rightarrow c^{-1}(cc) = c^{-1}c \Rightarrow (c^{-1}c)c = c^{-1}c \Rightarrow ec = e \Rightarrow c = e$。

\item $ab=ac\Rightarrow a^{-1}(ab)=a^{-1}(ac)\Rightarrow (a^{-1}a)b=(a^{-1}a)c\Rightarrow eb=ec\Rightarrow b=c.$

$ba=ca\Rightarrow (ba)a^{-1}=(ca)a^{-1}\Rightarrow b(a^{-1}a)=c(a^{-1}a)\Rightarrow be=ce\Rightarrow b=c.$

\item 若$a'$也为$a$的逆元素,则$a^{-1}a=e=a'a$,于是由(ii)可得$a^{-1}=a'$。

\item 由(双侧)逆元素的定义立得.

\item $(ab)(b^{-1}a^{-1}) = a(bb^{-1})a^{-1} = (ae)a^{-1} = aa^{-1} = e $.同理$(b^{-1}a^{-1})(ab) = e$.再根据(iii)可知$(ab)^{-1} = b^{-1}a^{-1}$.

\item 将$x=a^{-1}b$代入方程$ax=b$可得
$ax=a\left( a^{-1}b \right) =\left( aa^{-1} \right) b=eb=b$,
故$x=a^{-1}b$是方程$ax=b$的解。若$x=c$也是$ax=b$的解,则
$ac=b\Rightarrow a^{-1}\left( ac \right) =a^{-1}b=\left( aa^{-1} \right) c=a^{-1}b\Rightarrow ec=a^{-1}b\Rightarrow c=a^{-1}b$。
故$x=a^{-1}b$是方程$ax=b$的唯一解。类似可证$y=ba^{-1}$是方程$ya=b$的唯一解。
\end{enumerate}
\end{proof}

\begin{proposition}\label{proposition:半群是群的充要条件}
设$G$是半群,则$G$是群的充要条件是下面两条件成立:

(i) 存在一个元素$e \in G$,使得对所有$a \in G$均有$ea = a$(左幺元素);

(ii) 对于每个$a \in G$,均存在一个元素$a^{-1} \in G$,使得$a^{-1}a = e$(左逆)。
\end{proposition}
\begin{remark}
如果改成“右幺元素”和“右逆”,则类似的结果也成立。
\end{remark}
\begin{proof}
$(\Rightarrow)$:显然。

$(\Leftarrow)$:先证若$c\in G$且$cc=c$,则$c=e$.由(i)(ii)可知$c^{-1}(cc)=c^{-1}c=e\Rightarrow (c^{-1}c)c=e\Rightarrow e=ec$,从而再由(i)可得$c=e$.

由于$e \in G$,从而$G \neq \varnothing$。如果$a \in G$,由(ii)可知$(aa^{-1})(aa^{-1}) = a (a^{-1}a) a^{-1} = a(ea^{-1}) = aa^{-1}$,从而由上述结论可知$aa^{-1} = e$。因此$a^{-1}$是$a$的双侧逆。由于对每个$a \in G$均有$ae = (aa^{-1}a) = (aa^{-1}) a = ea = a$,从而$e$为双侧幺元素。因此$G$是群。
\end{proof}

\begin{proposition}
设$G$是半群,则$G$是群的充要条件是对于所有$a, b \in G$,方程$ax = b$和$ya = b$在$G$中均可解。
\end{proposition}
\begin{proof}
$(\Rightarrow)$:由\nrefthe{theorem:(幺半)群的基本性质}{(iv)}立得.

$(\Leftarrow)$:对$\forall a\in G$,取$b=a$,由$ya=b$可解,故存在$e\in G$,使得对$\forall a\in G$均有$ea=a$.
对$\forall a\in G$,取$b=e$,由$ya=b$可解,故对每个$a\in G$,都存在一个元素$a^{-1}\in G$,使得$a^{-1}a=e$.因此由\refpro{proposition:半群是群的充要条件}可知$G$是群.
\end{proof}

\begin{example}
整数集合$\mathbb{Z}$,有理数集合$\mathbb{Q}$和实数集合$\mathbb{R}$对于通常加法都是无限Abel群。对于通常的乘法都是幺半群但不是群($0$没有逆)。但是$\mathbb{Q}$和$\mathbb{R}$的非零元素集合对于乘法分别形成无限Abel群。偶整数集合对于乘法形成半群但不是幺半群。
\end{example}

\begin{theorem}\label{theorem:同余关系}
假设$R(\sim)$是幺半群$G$上的一个等价关系,并且对所有$a_i, b_i \in G$,由$a_1 \sim a_2$,$b_1 \sim b_2$可以导出$a_1 b_1 \sim a_2 b_2$。则$G$的所有$R$等价类组成的集合$G/R$对于二元运算$(\bar{a})(\bar{b}) = \overline{ab}$是幺半群。其中$\bar{x}$表示$x \in G$的等价类。

幺半群$G$上满足此定理中条件的等价关系称作$G$上的一个\textbf{同余关系}。

如果$G$为Abel群,则$G/R$也为Abel群。
\end{theorem}
\begin{proof}
如果$\bar{a}_1 = \bar{a}_2$并且$\bar{b}_1 = \bar{b}_2$($a_i, b_i \in G$),由引论中第4节的(20)式有$a_1 \sim a_2$和$b_1 \sim b_2$。由假设有$a_1 b_1 \sim a_2 b_2$,从而再由(20)式$\overline{a_1 b_1} = \overline{a_2 b_2}$。因此可以定义$G/R$中的二元运算(即与等价类代表元的选取无关)。这个二元运算是满足结合律的,因为$\bar{a} (\bar{b} \bar{c}) = \bar{a}(\overline{bc}) = \overline{a(bc)} = \overline{(ab)c} = \overline{(ab)} \bar{c} = (\bar{a} \bar{b})\bar{c}$。又由于$(\bar{a})(\bar{e}) = \overline{ae} = \bar{a} = \overline{ea} = (\bar{e})(\bar{a})$,从而$\bar{e}$是幺元素,于是$G/R$为幺半群。如果$G$为群,则$\bar{a} \in G/R$显然有逆元素$\overline{a^{-1}}$,因此$G/R$也是群。类似地,$G$的交换性导致$G/R$的交换性。
\end{proof}

\begin{example}
假设$m$是固定的整数。根据引论的定理6.8可知模$m$同余是加法群$\boldsymbol{Z}$上的同余关系。以$\boldsymbol{Z}_m$表示$\boldsymbol{Z}$在模$m$同余之下的等价类集合。由\refthe{theorem:同余关系}(采用加法记号)知$\boldsymbol{Z}_m$是Abel群,其加法由$\bar{a} + \bar{b} = \overline{a + b}$给出($a, b \in \boldsymbol{Z}$)。引论中定理6.8的证明表明$\boldsymbol{Z}_m = \{ \bar{0}, \bar{1}, \dots, \overline{m - 1} \}$,从而$\boldsymbol{Z}_m$对于加法是$m$阶有限群,叫作是\textbf{模$m$整数(加法)群}。类似地,由于$\boldsymbol{Z}$对于乘法是交换幺半群,而模$m$同余对于乘法也是同余关系(引论的定理6.8),从而$\boldsymbol{Z}_m$对于由$(\bar{a})(\bar{b}) = \overline{ab}$($a, b \in \boldsymbol{Z}$)给出的乘法是交换幺半群。验证对于所有$\bar{a}, \bar{b}, \bar{c} \in \boldsymbol{Z}_m$:
\[\bar{a}(\bar{b} + \bar{c}) = \bar{a}\bar{b} + \bar{a}\bar{c}, (\bar{a} + \bar{b})\bar{c} = \bar{a}\bar{c} + \bar{b}\bar{c} \text{(分配律)}\]
进而,如果$p$为素数,则$\boldsymbol{Z}_p$的非零元素形成$p - 1$阶乘法群。
\end{example}
\begin{remark}
习惯上我们仍把$\boldsymbol{Z}_m$中元素表示成$0, 1, \dots, m - 1$,而不写成$\bar{0}, \bar{1}, \dots, \overline{m - 1}$。这种有些混淆的记号在课文中不会引起困难,所以在方便的时候我们就使用它。
\end{remark}
\begin{proof}

\end{proof}


\begin{example}
有理数加法群$\boldsymbol{Q}$上的下列关系是同余关系:
\[a \sim b \Longleftrightarrow a - b \in \boldsymbol{Z}\]
由\refthe{theorem:同余关系},等价类集合(表示成$\boldsymbol{Q}/\boldsymbol{Z}$)对于由$\bar{a} + \bar{b} = \overline{a + b}$给出的加法是(无限)Abel群。$\boldsymbol{Q}/\boldsymbol{Z}$叫作\textbf{模1有理数群}。
\end{example}
\begin{proof}

\end{proof}

\begin{definition}[有意义乘积和标准n元乘积]\label{definition:有意义乘积和标准n元乘积}
设$G$是一个半群, $\{a_1,a_2, \cdots\}$是$G$中任意一个元素序列,

(i)我们归纳地定义$a_1, \dots, a_n$(以这种排列次序)的一个\textbf{有意义乘积}\footnote{为了证明这个定义的良定义性,需要\hyperref[Set Theory-theorem:逆归定理]{逆归定理}的更强的形式见Burrill, C. ; W, Foundations of Real Numbers. New York: McGraw-Hill, Inc, 1967.,57页}:如果$n = 1$,则唯一的有意义乘积为$a_1$。如果$n > 1$,则有意义乘积定义为形如$(a_1 \cdots a_m)(a_{m + 1} \cdots a_n)$的任何一个乘积,其中$m < n$,并且$(a_1 \cdots a_m)$和$(a_{m + 1} \cdots a_n)$分别是$m$元和$n - m$元的有意义乘积。

(ii)我们如下归纳定义$a_1,\cdots,a_n$的\textbf{标准n元乘积}\footnote{由\hyperref[Set Theory-theorem:逆归定理]{逆归定理}可以推出对每个$n \in \boldsymbol{N}^*$,$G$中任意$n$个元素的标准$n$元乘积对应$G$中的唯一元素(它显然是一个有意义乘积),因此这个定义是良定义的}$\prod_{i=1}^n{a_i}$:
\begin{align*}
\prod_{i=1}^1{a_i}=a_1,\text{而当}n>1\text{时},\prod_{i=1}^n{a_i}=\left( \prod_{i=1}^{n-1}{a_i} \right) a_n.
\end{align*}
\end{definition}
\begin{remark}
注意当$n \geqslant 3$时,可能存在$a_1, \dots, a_n$的许多个有意义乘积。
\end{remark}

\begin{theorem}[广义结合律]\label{theorem:广义结合律}
如果$G$是半群而$a_1, \dots, a_n \in G$,则$a_1, \dots, a_n$以此排列次序的任意两个有意义乘积均彼此相等。
\end{theorem}
\begin{remark}
根据这个定理,我们可以将$a_1, \dots, a_n \in G$($G$为半群)的任何有意义乘积写成$a_1 a_2 \dots a_n$,即不加任何括号也不会有任何混淆。
\end{remark}
\begin{proof}
我们归纳证明:对于每个$n$,任意一个有意义乘积$a_1 \dots a_n$均等于标准$n$元乘积$\prod_{i = 1}^n a_i$。对于$n = 1, 2$这显然是对的。如果$n > 2$,由定义$(a_1 \dots a_n) = (a_1 \dots a_m)(a_{m + 1} \dots a_n)$,其中$m < n$。从而根据归纳假设和结合性便有:
\begin{align*}
(a_1 \dots a_n) &= (a_1 \dots a_m)(a_{m + 1} \dots a_n) = \left( \prod_{i = 1}^m a_i \right) \left( \prod_{i = 1}^{n - m} a_{m + i} \right) \\
&= \left( \prod_{i = 1}^m a_i \right) \left( \left( \prod_{i = 1}^{n - m - 1} a_{m + i} \right) a_n \right) = \left( \left( \prod_{i = 1}^m a_i \right) \left( \prod_{i = 1}^{n - m - 1} a_{m + i} \right) \right) a_n \\
&= \left( \prod_{i = 1}^{n - 1} a_i \right) a_n = \prod_{i = 1}^n a_i.
\end{align*}
\end{proof}

\begin{theorem}[广义交换律]\label{theorem:广义交换律}
如果$G$为交换半群而$a_1, \dots, a_n \in G$,则对于$1, 2, \dots, n$的任意一个置换$i_1, \dots i_n$,均有
\[a_{1}a_{2}\cdots a_{n} = a_{i_1}a_{i_2}\cdots a_{i_n}.\]
\end{theorem}
\begin{proof}

\end{proof}

\begin{definition}[方幂]
假设$G$为半群,$a \in G$,$n \in \boldsymbol{N}^*$。元素$a^n \in G$定义为标准$n$元乘积$\prod_{i = 1}^n a_i$,其中$a_i = a(1 \leqslant i \leqslant n)$。如果$G$是幺半群,则$a^0$定义为幺元素$e$。如果$G$是群,则对于每个$n \in \boldsymbol{N}^*$,$a^{-n}$定义为$(a^{-1})^n \in G$。
\end{definition}
\begin{remark}
根据定义和\hyperref[theorem:广义结合律]{广义结合律},$a^1 = a$,$a^2 = aa$,$a^3 = (aa)a = aaa$,$\dots$,$a^n = a^{n - 1}a = aa \dots a$($n$个因子)。

注意当$m \neq n$时可能会有$a^m = a^n$(例如在$\boldsymbol{C}$中,$-1 = i^2 = i^6$)。

\textbf{加法记号}$\,\,$如果$G$中的二元运算写成加法,我们便用$na$代替$a^n$。因此$0a = 0$,$1a = a$,$na = (n - 1)a + a$,如此等等。
\end{remark}

\begin{theorem}
如果$G$是群[半群,幺半群],而$a \in G$,则对所有$m, n \in \boldsymbol{Z}[\boldsymbol{N}^*, \boldsymbol{N}]$,均有

\text{(i)} $ a^m a^n = a^{m + n} \text{(加法记号:} ma + na = (m + n)a\text{)}; $

\text{(ii)} $ (a^m)^n = a^{mn} \text{(加法记号:} n(ma) = nma\text{)}.$
\end{theorem}
\begin{proof}
显然对每个$n \in \boldsymbol{N}$均有$(a^n)^{-1} = (a^{-1})^n$,并且对每个$n \in \boldsymbol{Z}$均有$(a^{-1})^n=((a^{-1})^{-1})^{-n}=a^{-n}$.

(i) 对于$m > 0$和$n > 0$ 是对的,因为标准$n$元乘积和标准$m$元乘积相乘是一个有意义乘积,根据\hyperref[theorem:广义结合律]{广义结合律},它等于标准$(m + n)$元乘积。将$a, m, n$改为$a^{-1}, -m, -n$并且利用上述推理就可得到$m < 0$和$n < 0$的情形。情形$m = 0$或者$n = 0$是平凡的。而情形$m \geqslant 0$,$n < 0$和$m < 0$,$n \geqslant 0$可以分别对$m$和$n$作归纳法。

(ii) 对于$m = 0$是显然的。当$m > 0$,$n \in \boldsymbol{Z}$时,可以对$m$归纳证得。然后用此结果证明$m < 0$和$n \in \boldsymbol{Z}$的情形。
\end{proof}













\end{document}
\documentclass[../../main.tex]{subfiles}
\graphicspath{{\subfix{../../image/}}} % 指定图片目录,后续可以直接使用图片文件名。

% 例如:
% \begin{figure}[h]
% \centering
% \includegraphics{image-01.01}
% \caption{图片标题}
% \label{fig:image-01.01}
% \end{figure}
% 注意:上述\label{}一定要放在\caption{}之后,否则引用图片序号会只会显示??.

\begin{document}

\section{幺半群}

\begin{definition}[代数运算/二元运算]\label{definition:代数运算/二元运算定义}
设\(A\)是一个非空集合,若对\(A\)中任意两个元素\(a,b\),通过某个法则“\(\cdot\)”,有\(A\)中唯一确定的元素\(c\)与之对应,则称法则“\(\cdot\)”为集合\(A\)上的一个\textbf{代数运算(algebraic operation)或二元运算}.元素\(c\)是\(a,b\)通过运算“\(\cdot\)”作用的结果,将此结果记为\(a \cdot b = c\).
\end{definition}

\begin{definition}[半群和交换半群]
非空集合\(S\)和\(S\)上满足结合律的二元运算\(\cdot\)所形成的代数结构叫做\textbf{半群}.此即
\begin{align*}
\forall x, y, z \in S, x \cdot (y \cdot z) = (x \cdot y) \cdot z.
\end{align*}
这个半群记成\((S,\cdot)\)或者简记成\(S\),运算\(x\cdot y\)也常常简写成\(xy\).此外,如果半群\((S,\cdot)\)中的运算“$\cdot$”又满足交换律,则\((S,\cdot)\)叫做\textbf{交换半群}.此即
\begin{align*}
&\forall x, y, z \in S, x \cdot (y \cdot z) = (x \cdot y) \cdot z,\\
&\forall x,y\in S,x\cdot y=y\cdot x.
\end{align*}
\end{definition}
\begin{remark}
像通常那样令\(x^2 = x\cdot x\),\(x^{n + 1} = x^n\cdot x( = x\cdot x^n, n\geq1)\).
\end{remark}

\begin{definition}[幺元素]\label{definition:幺元素定义}
设\(S\)是半群,元素\(e\in S\)叫做半群\(S\)的\textbf{幺元素(也叫单位元(unit element)或恒等元(identity))},是指对每个\(x\in S\),\(xe = ex = x\).
\end{definition}

\begin{proposition}[幺元素存在必唯一]
如果半群$(S,\cdot)$中有幺元素,则幺元素一定唯一.我们将半群$(S,\cdot)$中这个唯一的幺元素(如果存在的话)通常记作$\boldsymbol{1_S}$\textbf{或者1}.
\end{proposition}
\begin{proof}
因若$e'$也是幺元素,则$e'=e'e=e$.
\end{proof}

\begin{definition}[含幺半群和交换含幺半群]
如果半群\((S,\cdot)\)含有幺元素,则\((S,\cdot)\)称为\textbf{(含)幺半群}.此即
\begin{align*}
&\forall x, y, z \in S, x \cdot (y \cdot z) = (x \cdot y) \cdot z,\\
&\exists e \in S, \forall x \in S, e \cdot x = x \cdot e = x.
\end{align*} 
此外,如果幺半群\((S,\cdot)\)中的运算“$\cdot$”又满足交换律,则\((S,\cdot)\)叫做\textbf{交换幺半群}.此即
\begin{align*}
&\forall x, y, z \in S, x \cdot (y \cdot z) = (x \cdot y) \cdot z,\\
&\exists e \in S, \forall x \in S, e \cdot x = x \cdot e = x,\\
&\forall x,y\in S,x\cdot y=y\cdot x.
\end{align*} 
\end{definition}

\begin{example}
$\left( M_n\left( \mathbb{R} \right) ,\cdot \right)$是一个含幺(乘法)半群.
\end{example}
\begin{proof}
\(\forall A,B,C\in (M_n(\mathbb{R}),\cdot)\),则不妨设\(A=(a_{ij})_{n\times n}\),\(B=(b_{ij})_{n\times n}\),\(C=(c_{ij})_{n\times n}\).再设\(A\cdot B=(d_{ij})_{n\times n}\),\(B\cdot C=(e_{ij})_{n\times n}\),\((A\cdot B)\cdot C=(f_{ij})_{n\times n}\),\(A\cdot (B\cdot C)=(g_{ij})_{n\times n}\).于是
\[
d_{ij}=\sum_{k = 1}^n{a_{ik}b_{kl}},e_{ij}=\sum_{k = 1}^n{b_{ik}c_{kl}}.
\]
其中\(i,j = 1,2,\cdots,n\).

从而
\[
f_{ij}=\sum_{l = 1}^n{d_{il}c_{lj}}=\sum_{l = 1}^n{\left(\sum_{k = 1}^n{a_{ik}b_{kl}}\right)\cdot c_{lj}}=\sum_{l = 1}^n{\sum_{k = 1}^n{a_{ik}b_{kl}c_{lj}}},
\]
\[
g_{ij}=\sum_{k = 1}^n{a_{ik}e_{kj}}=\sum_{k = 1}^n{a_{ik}\cdot\left(\sum_{l = 1}^n{b_{kl}c_{lj}}\right)}=\sum_{k = 1}^n{\sum_{l = 1}^n{a_{ik}b_{kl}c_{lj}}}.
\]
由二重求和号的可交换性,可知\(f_{ij}=g_{ij}\),\(\forall i,j\in \{1,2,\cdots,n\}\).故\((A\cdot B)\cdot C = A\cdot (B\cdot C)\).

记\(I_n=\begin{pmatrix}
1 & & & \\
& 1 & & \\
& & \ddots & \\
& & & 1\\
\end{pmatrix}\in M_n(\mathbb{R})\),于是\(\forall X\in M_n(\mathbb{R})\),则不妨设\(X=(x_{ij})_{n\times n}\),\(I_n = (\delta_{ij})_{n\times n}\).其中\(\delta_{ij}=\begin{cases}
1,\text{当 }i = j\text{ 时},\\
0,\text{当 }i\neq j\text{ 时}
\end{cases}\).
再设\(I_n\cdot X=(x_{ij}')_{n\times n}\),\(X\cdot I_n=(x_{ij}'')_{n\times n}\),于是由矩阵乘法的定义可知
\[
x_{ij}'=\sum_{k = 1}^n{x_{ik}\delta_{kj}}=x_{ij}\delta_{jj}=x_{ij}.
\]
\[
x_{ij}''=\sum_{k = 1}^n{\delta_{ik}x_{kj}}=\delta_{ii}x_{ij}=x_{ij}.
\]
故\(x_{ij}'=x_{ij}''=x_{ij}\),\(\forall i,j\in \{1,2,\cdots,n\}\).从而\(X = I_n\cdot X = X\cdot I_n\).因此\(I_n\)是\((M_n(\mathbb{R}),\cdot)\)的单位元.综上所述,$\left( M_n\left( \mathbb{R} \right) ,\cdot \right)$是一个含幺(乘法)半群.
\end{proof}

\begin{definition}[幺半群中多个元素的乘积]\label{definition:幺半群中多个元素的乘积}
设$(S,\cdot)$是一个幺半群,令 $x_1, \cdots, x_n \in S$,我们递归地定义
\begin{align*}
x_1 \cdot x_2 \cdots x_n &= (x_1 \cdot x_2 \cdots x_{n - 1}) \cdot x_n
\end{align*}
令 $x \in S, n \in \mathbb{N}$。若 $n > 0$,我们定义 $x^n = x \cdots x$,而 $x^0 = e$。 
\end{definition}

\begin{definition}[广义结合律]
设$S$是一个非空集合,“$\cdot$”是一个二元运算,若对于任意有限多个元素 $x_1, x_2, \dots, x_n \in S$,乘积 $x_1\cdot x_2 \cdots x_n$ 的任何一种 “有意义的加括号方式”(即给定的乘积的顺序)都得出相同的值。 
\end{definition}

\begin{proposition}\label{proposition:半群一定满足广义结合律}
设$S$是一个非空集合,“$\cdot$”是一个满足结合律的二元运算,令$x_1, \cdots, x_n, y_1, \cdots, y_m \in S$,则
\begin{align*}
x_1 \cdot x_2 \cdots x_n \cdot y_1 \cdot y_2 \cdots y_m &= (x_1 \cdot x_2 \cdots x_n) \cdot (y_1 \cdot y_2 \cdots y_m) \tag{1.6}
\end{align*}
\end{proposition}
\begin{note}
根据这个命题,我们就可以得到一个半群$(S,\cdot)$一定满足广义结合律,只要 $x_1, \cdots, x_n\in S$ 的$\cdot$运算顺序是固定的,无论怎么添加括号,我们都可以利用这个命题的结论,将括号重排至从前往后依次乘的顺序而保持结果不变。所以,如果一个集合上的二元运算有结合律,我们就可以在连续元素的乘积中不加括号,也可以按照我们的需要随意加括号。
\end{note}
\begin{proof}
对 $m$ 做数学归纳。当 $m = 1$ 时,由\hyperref[definition:幺半群中多个元素的乘积]{定义\ref{definition:幺半群中多个元素的乘积}}直接得到。
接下来,假设
\begin{align*}
x_1 \cdot x_2 \cdots x_n \cdot y_1 \cdot y_2 \cdots y_k &= (x_1 \cdot x_2 \cdots x_n) \cdot (y_1 \cdot y_2 \cdots y_k)
\end{align*}
则由“$\cdot$”满足结合律,我们有
\begin{align*}
&x_1 \cdot x_2 \cdots x_n \cdot y_1 \cdot y_2 \cdots y_{k + 1} \\
&= ((x_1 \cdot x_2 \cdots x_n) \cdot (y_1 \cdot y_2 \cdots y_k)) \cdot y_{k + 1} \\
&= (x_1 \cdot x_2 \cdots x_n) \cdot ((y_1 \cdot y_2 \cdots y_k) \cdot y_{k + 1}) \\
&= (x_1 \cdot x_2 \cdots x_n) \cdot (y_1 \cdot y_2 \cdots y_{k + 1}) 
\end{align*}
\end{proof}

\begin{corollary}\label{corollary:满足结合律一定满足广义结合律的推论}
令 $x \in S, m, n \in \mathbb{N}$,则
\begin{align*}
x^{m + n} &= x^m \cdot x^n
\end{align*} 
\end{corollary}
\begin{proof}
令\hyperref[proposition:半群一定满足广义结合律]{命题\ref{proposition:半群一定满足广义结合律}}中的所有$x_i$和$y_j$都等于$x$即可得到.
\end{proof}

\begin{definition}[子幺半群]
令 $(S, \cdot)$ 是一个幺半群,若 $T \subset S$, $e \in T$,且 $T$ 在乘法下封闭,即
\begin{gather*}
e \in T ,\\
\forall x, y \in T, x \cdot y\in T .
\end{gather*}
则我们称$(T, \cdot)$ 是 $(S, \cdot)$ 的一个\textbf{子幺半群}
\end{definition}

\begin{proposition}[子幺半群也是幺半群]
若 $(T, \cdot)$ 是 $(S, \cdot)$ 的一个子幺半群,则 $(T, \cdot)$ 是个幺半群.
\end{proposition}
\begin{proof}
就二元运算的定义而言,子群第一个条件(封闭性)就满足了,这使得我们后面的谈论是有意义的。首先,结合律对于 $S$ 中元素都满足,当然对 $T$ 中元素也满足($T$ 是子集)。接下来,类似地,$e$ 对于所有 $S$ 中元素都是单位元,固然对于 $T$ 中元素亦是单位元。 
\end{proof}

\begin{definition}[幺半群同态]
假设 $(S, \cdot), (T, *)$ 是两个幺半群,且 $f : S \to T$ 是一个映射,我们称 $f$ 是一个\textbf{幺半群同态},当 $f$ 保持了乘法运算,且把单位元映到了单位元。此即
\begin{align*}
\forall x, y \in S, f(x \cdot y) &= f(x) * f(y) ,\\
f(e) &= e'.
\end{align*}
其中,$e$ 和 $e'$ 分别是 $(S, \cdot)$ 和 $(T, *)$ 的单位元。 
\end{definition}

\begin{definition}[由$A$生成的子幺半群]
假设 $(S, \cdot)$ 是一个幺半群,而 $A \subset S$ 是一个子集。我们称$S$ 中所有包含了 $A$ 的子幺半群的交集为\textbf{由$\boldsymbol{A}$ 生成的子幺半群},记作 $\langle A \rangle$.此即
\begin{align*}
\langle A \rangle = \bigcap \{T \subset S : T \supset A, T \text{ 是子幺半群}\}.
\end{align*} 
\end{definition}

\begin{proposition}[$\langle A \rangle$ 包含了 $A$ 的最小的子幺半群]
假设 $(S, \cdot)$ 是一个幺半群,而 $A \subset S$ 是一个子集。则 $\langle A \rangle$ 也是一个子幺半群。因此,这是包含了 $A$ 的最小的子幺半群。
\end{proposition}
\begin{remark}
这里说的 “最小”,指的是在包含关系下最小的,也就是,它包含于所有包含 $A$ 的子幺半群。
\end{remark}
\begin{proof}
要证明 $\langle A \rangle$ 是子幺半群,只需要证明它包含了 $e$,并在乘法运算下封闭。首先,因为集族中每一个 $T$,作为子幺半群,都会包含 $e$;因此 $\langle A \rangle$ 作为这些集合的交集也会包含 $e$,这就证明了第一点。而对于第二点,我们首先假设 $x, y \in \langle A \rangle$,而想要证明 $x \cdot y \in \langle A \rangle$。注意到,因为 $x, y \in \langle A \rangle$,任取一个包含了 $A$ 的子幺半群 $T$(集族中的集合),我们都有 $x, y \in T$,于是有 $x \cdot y \in T$。而 $x \cdot y \in T$ 对于所有这样的 $T$ 都成立,我们就有 $x \cdot y$ 属于它们的交集,也就是 $\langle A \rangle$。这样,我们就证明了第二点。综上,由一个幺半群 $S$ 的任意子集 $A$ 生成的子幺半群都确实是一个子幺半群。 
\end{proof}

\begin{definition}[幺半群同构]
假设 $(S, \cdot), (T, *)$ 是两个幺半群,且 $f : S \to T$ 是一个映射,我们称 $f$ 是一个\textbf{幺半群同构},当 $f$ 是一个双射,且是一个同态。
\begin{gather*}
f \text{ 是双射} ,\\
\forall x, y \in S, f(x \cdot y) = f(x) * f(y) ,\\
f(e) = e' .
\end{gather*}
其中,$e$ 和 $e'$ 分别是 $(S, \cdot)$ 和 $(T, *)$ 的单位元。 
\end{definition}
\begin{remark}
容易验证同构是一个等价关系.
\end{remark}

\begin{proposition}[幺半群同构的逆映射一定是幺半群同态]\label{proposition:幺半群同构的逆映射一定是幺半群同态}
若 $f : (S, \cdot) \to (T, *)$ 是一个幺半群同构,则 $f^{-1} : T \to S$ 是一个幺半群同态。因此,$f^{-1}$ 也是个幺半群同构。
\end{proposition}
\begin{proof}
令 $x', y' \in T$,我们只需证明 $f^{-1}(x' * y') = f^{-1}(x') \cdot f^{-1}(y')$。为了方便起见,根据 $f$ 是一个双射,从而存在$x,y\in S$,使得$x = f^{-1}(x'), y = f^{-1}(y')$,并且$f(x)=x',f(y)=y'$.我们只需证明 $f^{-1}(x' * y') = x \cdot y$。而由于 $f$ 是幺半群同态,所以 $f(x \cdot y) = f(x) * f(y) = x' * y'$。反过来说,$f^{-1}(x' * y') = x \cdot y = f^{-1}(x') \cdot f^{-1}(y')$。这就证明了这个命题。 
\end{proof}














\end{document}
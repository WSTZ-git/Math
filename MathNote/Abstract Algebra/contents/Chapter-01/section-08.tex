\documentclass[../../main.tex]{subfiles}
\graphicspath{{\subfix{../../image/}}} % 指定图片目录,后续可以直接使用图片文件名。

% 例如:
% \begin{figure}[H]
% \centering
% \includegraphics[scale=0.4]{图.png}
% \caption{}
% \label{figure:图}
% \end{figure}
% 注意:上述\label{}一定要放在\caption{}之后,否则引用图片序号会只会显示??.

\begin{document}

\section{循环群}

\begin{definition}[循环群]
设$G$是一个群且$a\in G$,我们称
\begin{align*}
\langle a \rangle = \{a^n|n \in \mathbb{Z}\}
\end{align*}
是由 \( a \) 生成的 \( G \) 的子群, 如果在一个群 \( G \) 中存在一个元素 \( a \), 使得 \( G = \langle a \rangle \), 即 \( G \) 由 \( a \) 生成, 则称 \( G \) 是\textbf{循环群}, \( a \) 为 \( G \) 的一个\textbf{生成元}.
\end{definition}
\begin{remark}
对$\forall n_1,n_2\in \mathbb{Z}$,有$a^{n_1}a^{-n_2}\in G$.因此$\langle a \rangle$是$G$的子群.故由$a$生成的$G$的子群是良定义的.
\end{remark}

\begin{corollary}\label{corollary:抽象代数-推论 1.3.4}
有限群 \( G \) 的任一元素 \( a \) 的阶是 \( G \) 的阶的因子,即$\text{ord}\,a\mid [G:1].$
进一步,若$G=\langle a \rangle,$则$\text{ord}\,a=[G:1],$并且$G=\langle a \rangle=\{1,a,\cdots,a^{\text{ord}\,a-1}\}$.
\end{corollary}
\begin{proof}
令 \( \langle a \rangle = \{a^n|n \in \mathbb{Z}\} \), 容易验证这是 \( G \) 的一个子群. 又由于 \( G \) 有限, 故 \( \langle a \rangle \) 有限, 因而 \( a \) 是有限阶的, 设为 \( d \). 对 \( n \in \mathbb{Z} \) 有 \( t_n \) 与 \( r_n \) (\( 0 \leqslant r_n < d \)), 使 \( n = t_nd + r_n \), 于是 \( a^n = a^{r_n} \).因此$\langle a \rangle$中至多只有$d$个元素$1,a,\cdots,a^{d-1}$.

又对 \( \forall r_1, r_2 \in \mathbb{N} \), 且 \( r_1 \neq r_2 \), \( 0 \leqslant r_1, r_2 < d \), 则 \( |r_1 - r_2| < d \), 从而 \( a^{r_1 - r_2} \neq 1 \), 进而 \( a^{r_1} \neq a^{r_2} \). 故 \( 1, a, \cdots, a^{d-1} \) 互不相同.
由此知 \( \langle a \rangle = \{1, a, \cdots, a^{d-1}\} \), 即 \( \langle a \rangle \) 是 \( d \) 阶群. 故由\hyperref[theorem:抽象代数-Lagrange定理-定理 1.3.3]{Lagrange定理}知 \( d \) 为 \( [G:1] \) 的因子.

若$G=\langle a \rangle,\text{ord}\,a=d$,则由上述证明知$G=\langle a \rangle = \{1, a, \cdots, a^{d-1}\}$是$d$阶群,故$d=[G:1].$

\end{proof}

\begin{theorem}\label{theorem:抽象代数--定理1.8.1}
循环群的任何子群也是循环群.
\end{theorem}
\begin{proof}
设$G_1$是循环群$G=\langle a\rangle$的一个非平凡子群. 令
\[
k=\min\{m'\in\mathbb{N}|a^{m'}\in G_1\},
\]
于是$G$中由$a^k$生成的子群$\langle a^k\rangle\subseteq G_1$, 又若有$a^{m'}\in G_1$, 则有整数$q,r$满足
\[
m'=kq+r,\quad 0\leqslant r<k,
\]
因而$a^r=a^{m'}(a^k)^{-q}\in G_1$, 由$k$的取法知$r=0$, 否则与$k$的最小值取法矛盾!因而$a^{m'}=(a^k)^q\in\langle a^k\rangle$, 故$G_1\subseteq \langle a^k \rangle$,所以$G_1=\langle a^k\rangle$为循环群.

\end{proof}

\begin{corollary}\label{corollary:抽象代数--循环群的相关结论--1}
设$\mathrm{ord}\,a = n$, $r$是任一整数. 如果$(n,r) = d$, 则
$\langle a^r \rangle = \langle a^d \rangle.$
\end{corollary}
\begin{proof}
因为$(n,r)=d$, 所以存在$u,v\in\mathbb{Z}$, 使
\begin{align*}
d=nu+rv.
\end{align*}
于是$a^d=a^{nu+rv}=a^{rv}\in\langle a^r\rangle$. 另一方面, 同样由于$(n,r)=d$, 所以$d\mid r$, 从而又有$a^r\in\langle a^d\rangle$,于是$\langle a^r\rangle\subseteq \langle a^d\rangle$. 由此得$\langle a^r\rangle=\langle a^d\rangle$.

\end{proof}

\begin{corollary}\label{corollary:循环群的所有子群的具体形式}
设$G = \langle a \rangle$为循环群,
\begin{enumerate}[(1)]
\item 如果$|G| = \infty$, 则$G$的全部子群为
$\{\langle a^d \rangle \mid d = 0,1,2,\cdots\}$,并且$\langle a^{d_1} \rangle \neq \langle a^{d_2} \rangle,\forall d_1,d_2\in \mathbb{N}_0\text{且}d_1\neq d_2$;

\item 如果$|G| = n$, 则$G$的全部子群为
$\{\langle a^d \rangle \mid d$为$n$的正因子$\}$,并且$\langle a^{d_1} \rangle \neq \langle a^{d_2} \rangle,\forall d_1,d_2$为$n$的正因子.
\end{enumerate}
\end{corollary}
\begin{proof}
设$e$为$G$的幺元,$G$的所有子群构成的集合为$S$.由\refthe{theorem:抽象代数--定理1.8.1}知, 循环群的任一子群必形如$\langle a^r\rangle(r\in\mathbb{Z})$. 显然, 有
\begin{align*}
\langle a^r\rangle=\langle a^{-r}\rangle.
\end{align*}
因此, 循环群的任一子群必形如$\langle a^r\rangle(r\in\mathbb{Z}, 且r\geqslant0)$.此即
\begin{align}\label{eq:::--0fj2jj81f}
S=\{\langle a^r\rangle \mid r\in \mathbb{Z},r\geqslant 0\}=\{\langle a^r\rangle \mid r=0,1,2,\cdots \}.
\end{align}
\begin{enumerate}[(1)]
\item 如果$|G|=\infty$, 由\eqref{eq:::--0fj2jj81f}式知
\begin{align*}
S=\{\langle a^r\rangle \mid r=0,1,2,\cdots \}.
\end{align*}
只需证这个集合中的元素两两不同即可.
因为对任意的$r_1>r_2>0$, 有$r_1\nmid r_2$, 所以$a^{r_2}\notin\langle a^{r_1}\rangle$, 于是
\begin{align*}
\langle a^{r_1}\rangle\neq\langle a^{r_2}\rangle.
\end{align*}
另一方面, 对任意的$r>0$, 显然$a^r\notin\langle a^0\rangle=\langle e\rangle=\{e\}$, 所以有
\begin{align*}
\langle a^r\rangle\neq\langle e\rangle.
\end{align*}
故$\langle a^{d_1} \rangle \neq \langle a^{d_2} \rangle,\forall d_1,d_2\in \mathbb{N}_0\text{且}d_1\neq d_2$.

\item 如果$|G|=n$,对任意的正整数$r$, 存在$n$的正因子$d=(n,r)$, 由\refcor{corollary:抽象代数--循环群的相关结论--1}可知
\begin{align*}
\langle a^r\rangle=\langle a^d\rangle\in \{\langle a^d\rangle\mid d\text{为}n\text{的正因子}\}.
\end{align*}
故再由\eqref{eq:::--0fj2jj81f}式知$S\subseteq \{\langle a^d\rangle\mid d\text{为}n\text{的正因子}\}$.又显然有$S\supseteq \{\langle a^d\rangle\mid d\text{为}n\text{的正因子}\}$,故
\begin{align*}
S=\{\langle a^d\rangle\mid d\text{为}n\text{的正因子}\}.
\end{align*}
若$d_1>d_2$为$n$的两个不同的正因子, 则$d_1\nmid d_2$, 于是$a^{d_2}\notin\langle a^{d_1}\rangle$, 从而
\begin{align*}
\langle a^{d_1}\rangle\neq\langle a^{d_2}\rangle.
\end{align*}
故$\langle a^{d_1} \rangle \neq \langle a^{d_2} \rangle,\forall d_1,d_2$为$n$的正因子.
\end{enumerate}

\end{proof}

\begin{corollary}\label{corollary:抽象代数--定理1.8.1推论}
\begin{enumerate}[(1)]
\item\label{corollary:抽象代数--定理1.8.1推论-1} 设 \( m \in \mathbb{Z} \), 则 \( m\mathbb{Z} \triangleq  \{mx|x \in \mathbb{Z}\} \) 是整数加法群 \( \mathbb{Z} \) 的子群. 

\item\label{corollary:抽象代数--定理1.8.1推论-2} 整数加法群$\mathbb{Z}$的任何子群必为$m\mathbb{Z}(m\geqslant 0\text{且}m\in \mathbb{Z})$.
\end{enumerate}
\end{corollary}
\begin{proof}
\begin{enumerate}[(1)]
\item 对$\forall x_1,x_2\in \mathbb{Z},$有
\begin{align*}
mx_1-mx_2=m(x_1-x_2)\in m\mathbb{Z}.
\end{align*}
故$m\mathbb{Z}$是整数加法群 \( \mathbb{Z} \) 的子群. 

\item 事实上, $\mathbb{Z}=\langle 1\rangle$. 设$G_1$为$\mathbb{Z}$的子群. 于是由\refthe{theorem:抽象代数--定理1.8.1}有$m\geqslant 0\text{且}m\in \mathbb{Z}$, 使得$G_1=\langle m\rangle=m\mathbb{Z}$.
\end{enumerate}

\end{proof}

\begin{proposition}[素数阶群必为循环群]\label{proposition:素数阶群必为循环群}
设$G$是一个群,且$|G|=p$为一个素数,则
\begin{enumerate}[(1)]
\item $G$必是循环群,并且\( \forall a \in G \) 且 \( a \neq e \) 有 \( G = \langle a \rangle \). 

\item $G$只有平凡子群.
\end{enumerate}
\end{proposition}
\begin{proof}
\begin{enumerate}[(1)]
\item 由$p>1$知$G$中至少存在一个非幺元$a\ne e$,则对\( \forall a \in G \) 且 \( a \neq e \) ,有$\langle a\rangle$是$G$的子群。由Lagrange定理知$\langle a\rangle$的阶是$|G|=p$的因数,而$p$为素数,故$\langle a\rangle$的阶为$1$或$p$。由$a,e\in\langle a\rangle$知$\langle a\rangle$的阶必大于$1$,因此$\langle a\rangle$的阶为$p$。又因为$\langle a\rangle\subseteq G$,所以$G=\langle a\rangle$。故$G$为循环群。

\item 由结论(1)知$G$是循环群.又\hyperref[theorem:抽象代数--定理1.8.1]{循环群的任何子群也是循环群},故$G$的任意子群$H$也是循环群,若$H\neq \{e\},G$,则可设$H=\langle a \rangle,a\in G\setminus\{e\}$,再由结论(1)知$G=\langle a \rangle=H$矛盾!故$H=\{e\}\text{或}G$.
\end{enumerate}

\end{proof}

\begin{proposition}\label{proposition:整数加群的商群}
设 \( m > 0\), 则有
\begin{align*}
m\mathbb{Z} \lhd  \mathbb{Z},\quad \mathbb{Z} = \bigcup_{k=0}^{m-1} (k + m\mathbb{Z}), \quad \mathbb{Z}_m \triangleq  \mathbb{Z}/m\mathbb{Z}  = \{\overline{0}, \overline{1}, \cdots, \overline{m-1}\} ,\quad [\mathbb{Z}: m\mathbb{Z}] = m.
\end{align*}
\end{proposition}
\begin{proof}
由\rrefcor{corollary:抽象代数--定理1.8.1推论}{corollary:抽象代数--定理1.8.1推论-1}知\(m\mathbb{Z} \) 为 \( \mathbb{Z} \) 的子群.

\end{proof}

\begin{theorem}\label{theorem:循环群都同构于整数加群}
设$G=\langle a\rangle$是一个循环群.
\begin{enumerate}[(1)]
\item 若$G$是无限阶的, 则$G$与整数加法群$\mathbb{Z}$同构.

\item 若$G$的阶$m$有限, 则$G$与加法群$\mathbb{Z}_m$同构.
\end{enumerate}
进而两个循环群同构当且仅当它们的阶相同.
\end{theorem}
\begin{proof}
作$\mathbb{Z}$到$G$上的映射$\varphi:\varphi(n)=a^n(n\in\mathbb{Z})$. 于是有
\[
\varphi(n_1+n_2)=a^{n_1+n_2}=a^{n_1}\cdot a^{n_2}=\varphi(n_1)\varphi(n_2),
\]
因而$\varphi$是$\mathbb{Z}$到$G$上的同态映射, 故由\hyperref[theorem:群的同态基本定理]{群的同态基本定理}知$G\cong \mathbb{Z}/\ker\varphi$且$\ker \varphi \lhd \mathbb{Z}$. 由\rrefcor{corollary:抽象代数--定理1.8.1推论}{corollary:抽象代数--定理1.8.1推论-2}知存在$m\geqslant 0\text{且}m\in \mathbb{Z},$使得$\ker\varphi=m\mathbb{Z}$.

若$m>0$, 则由\refpro{proposition:整数加群的商群}知,此时$G\cong \mathbb{Z}/\ker\varphi=\mathbb{Z}/m\mathbb{Z}=\mathbb{Z}_m$且$|G|=|\mathbb{Z}_m|=m$.

若$m=0$, 则$G\cong \mathbb{Z}$同构, 此时$G$的阶为无限.

\end{proof}

\begin{corollary}
无限循环群的非平凡子群仍为无限循环群.
\end{corollary}
\begin{proof}
设$G$为无限循环群,则由\refthe{theorem:循环群都同构于整数加群}知$G\cong \mathbb{Z}.$
又由\rrefcor{corollary:抽象代数--定理1.8.1推论}{corollary:抽象代数--定理1.8.1推论-2}知$\mathbb{Z}$的非平凡子群为$m\mathbb{Z}(m\neq 0,1)$为无限循环群.故$G$的非平凡子群也为无限循环群.

\end{proof}

\begin{theorem}\label{theorem:抽象代数--定理1.8.3}
设$G$是$m$阶循环群,$m_1$是$m$的一个因数,则存在唯一的$m_1$阶子群.
\end{theorem}
\begin{proof}
设$G=\langle a\rangle$. 从\refcor{corollary:抽象代数-推论 1.3.4}知$G$的阶$m$也就是元素$a$的阶. 由$m_1|m$知当$0<k<m_1$时有$0<km/m_1<m$,因而$(a^{m/m_1})^k\neq 1$,但$(a^{m/m_1})^{m_1}=1$,故$\langle a^{m/m_1}\rangle$是$G$的$m_1$阶子群.

下面证$m_1$阶子群的唯一性. 设$G_1$是$G$中的$m_1$阶子群,由\refthe{theorem:抽象代数--定理1.8.1}知$G_1=\langle a^k\rangle$,其中,$k\geqslant 0$,并且当$a^{m'}\in G_1$时,$k|m'$. 由$a^m=1\in G_1$知$k|m$,若$0<n<m/k$,则$0<kn<m$,从而$(a^k)^n=a^{kn}\neq 1$. 另外$(a^k)^{m/k}=1$,故$G_1$的阶为$m/k=m_1$,因而$k=m/m_1$,即$G_1=\langle a^{m/m_1}\rangle$.

\end{proof}

\begin{proposition}\label{proposition:有限群不同子群有不同阶的性质}
设$G$是$n$阶群且其不同的子群有不同的阶. 试证:
\begin{enumerate}[(1)]
\item $G$的任何子群都是正规子群;
\item $G$的子群与商群的不同子群也有不同的阶;
\item\label{proposition:有限群不同子群有不同阶的性质-3} $G$是循环群.
\end{enumerate}
\end{proposition}
\begin{proof}
\begin{enumerate}[(1)]
\item 设$H$为$G$的子群, $g\in G$.对$\forall h_1,h_2\in H$,有
\begin{align*}
(gh_1g^{-1})(gh_2g^{-1})^{-1}=(gh_1g^{-1})(gh_2^{-1}g^{-1})=gh_1h_2^{-1}g^{-1}\in gHg^{-1}.
\end{align*}
故$gHg^{-1}$是$G$的子群.又由\refpro{proposition:陪集aH的阶与H相同}知$gHg^{-1}$与$H$有相同的阶. 因此由条件知$gHg^{-1}=H$, 故$H$是正规子群.

\item 设$H_1, H_2$是$G$的子群$H$的子群, 自然也是$G$的子群, 于是由条件知$H_1=H_2$当且仅当$|H_1|=|H_2|$.

设$\overline{H_1}, \overline{H_2}$是商群$G/H$的子群.记$\pi$为$G$到商群$G/H$上的自然同态,$G$中包含$H$的子群的集合为$\Sigma$,$G/H$的子群的集合为$\Gamma$, 由\rrefcor{corollary:群同态第二定理推论}{corollary:群同态第二定理推论-1}知有$G$的子群$H_1\supseteq H$, $H_2\supseteq H$使得
\begin{align*}
\overline{H_1}= \pi(H_1)=H_1/H,\quad \overline{H_2}=\pi(H_2)= H_2/H.
\end{align*}
因为$\pi$是$\Sigma\to \Gamma$的双射,所以$\overline{H_1}=\overline{H_2}$当且仅当$H_1=H_2$. 而$H_1=H_2$当且仅当$|H_1|=|H_2|$. 注意
\begin{align*}
|H_i|=[H_i:H]|H|=|\overline{H_i}||H|,\quad i=1,2.
\end{align*}
于是$\overline{H_1}=\overline{H_2}$当且仅当$|\overline{H_1}|=|\overline{H_2}|$.
\item 设$|G|=p_1p_2\cdots p_s$, 其中$p_i(1\leqslant i\leqslant s)$是素数.

对$s$作归纳证明$G$是循环群. 若$s=0$, 则$|G|=1$, 显然$G$是循环群. 若$s=1$, $|G|=p_1$是素数, 由\refpro{proposition:素数阶群必为循环群}知$G$是循环群. 假定$s-1$时结论成立. 以$e$表示$G$的幺元, 取$a_1\in G$, $a_1\neq e$. 若$a_1$的阶为$n$, 则$G$是循环群. 不妨设$a_1$的阶为$p_sp_{s-1}\cdots p_k\neq n$, 于是$a=a_1^{p_{s-1}\cdots p_k}$的阶为$p_s$. 由结论(1), $\langle a\rangle$是$G$的正规子群. 由结论(2), 商群$G/\langle a\rangle$的不同子群有不同的阶, 由\refcor{corollary:抽象代数-推论 1.3.5}知$G/\langle a\rangle$的阶为$n_1=p_1p_2\cdots p_{s-1}$. 由归纳假设, $G/\langle a\rangle$是循环群. 于是存在$b\in G$使得$G/\langle a\rangle$的元素为$\langle a\rangle$, $b\langle a\rangle$, $\cdots$, $b^{n_1-1}\langle a\rangle$. 从而由$(b\langle a\rangle)^{n_1}=\langle a \rangle$知对$0\leqslant k<p_s$, 有$k_0(0\leqslant k_0<p_s)$使得
\begin{align*}
(ba^k)^{n_1}=a^{k_0}.
\end{align*}

下面证明$b\langle a\rangle$中有元素$c$使得$c^{n_1}\neq e$. 若$b^{n_1}\neq e$, 则可取$c=b$. 故设$b^{n_1}=e$. 注意$G/\langle a\rangle$的阶为$n_1$, 于是当$0<r<n_1$时, $b^r\neq e$, $(ba)^r\neq e$.
如果$(ba)^{n_1}=e$, 则$\langle b\rangle$与$\langle ba\rangle$均为$n_1$阶群, 因而由条件知$\langle b\rangle=\langle ba\rangle$, 于是有$ba=b^m$, $0<m<n_1$. 由于$ba\in b\langle a\rangle$, $b^m\in b^m\langle a\rangle$, 而$m\neq 1$时, 由\refcor{corollary:抽象代数-推论 1.3.3}知$b\langle a\rangle\cap b^m\langle a\rangle=\varnothing$, 于是$m=1$, 即$ba=b$, 从而$a=e$, 这就得到矛盾. 由此可知$(ba)^{n_1}\neq e$. 取$c=ba$.
由$c\in b\langle a\rangle$, 知$b\langle a\rangle=c\langle a\rangle$, 于是$G/\langle a\rangle=\langle c\langle a\rangle\rangle$. 因为$G/\langle a \rangle$的阶为$n_1$,所以$(c\langle a \rangle)^{n_1}=c^{n_1}\langle a \rangle=\langle a \rangle$.因而$c^{n_1}\in\langle a\rangle$. 注意$c^{n_1}\neq e$, 于是
\begin{align*}
c^{n_1}=a^m\neq e,\quad 1\leqslant m<p_s.
\end{align*}
因为$p_s$是素数, 所以有$(m,p_s)=1$. 进而$a\in\langle c\rangle$, $\langle a\rangle\subset\langle c\rangle$. 于是有
\begin{align*}
\langle c\rangle/\langle a\rangle=G/\langle a\rangle.
\end{align*}
因此$G=\langle c\rangle$为循环群.
\end{enumerate}
\end{proof}

\begin{theorem}
一个$m$阶群$G$对$m$的每个因数$m_1$存在唯一的$m_1$阶子群,则群$G$必是循环群.
\end{theorem}
\begin{proof}
设$G_1,G_2$是$G$的两个不同子群,则由\hyperref[theorem:抽象代数-Lagrange定理-定理 1.3.3]{Lagrange定理}知$[G_1:1],[G_2:1]$都是$m$的因数.
若$[G_1:1]=[G_2:1]$,则由条件知$G_1=G_2$矛盾!故$[G_1:1]\ne [G_2:1]$.因此$G$的不同的子群有不同的阶.于是由\rrefpro{proposition:有限群不同子群有不同阶的性质}{proposition:有限群不同子群有不同阶的性质-3}知$G$必是循环群.

\end{proof}

\begin{theorem}\label{theorem:抽象代数--定理1.8.4}
设$G$是一个群,$a,b\in G$. 它们的阶分别为$m,n$,则有下列结论:
\begin{enumerate}[(1)]
\item $a^k$的阶为$\dfrac{m}{(m,k)}$,$(m,k)$是$m$与$k$的最大公因数;
\item 若$\langle a\rangle\cap\langle b\rangle=\{1\}$,$ab=ba$,则$ab$的阶为$m,n$的最小公倍数$[m,n]$.
\end{enumerate}
\end{theorem}
\begin{proof}
\begin{enumerate}[(1)]
\item 设$a^k$的阶为$q$,即$a^{kq}=1$,因而有$m|kq$,故由数论相关结论知$\dfrac{m}{(m,k)}|q$. 又$(a^k)^{m/(m,k)}=(a^m)^{k/(m,k)}=1$,即得$q|(\dfrac{m}{(m,k)})$,因而
\begin{align*}
q=\dfrac{m}{(m,k)}.
\end{align*}
\item 设$ab$的阶为$m_1$,则有$(ab)^{m_1}=1$. 由$ab=ba$知$a^{m_1}b^{m_1}=(ab)^{m_1}=1$,即$a^{m_1}=b^{-m_1}\in\langle a\rangle\cap\langle b\rangle=\{1\}$,因而$a^{m_1}=b^{m_1}=1$,故$m|m_1$,$n|m_1$,因而$[m,n]|m_1$. 另有$(ab)^{[m,n]}=a^{[m,n]}b^{[m,n]}=1$,故$m_1|[m,n]$,即$m_1=[m,n]$.
\end{enumerate}
\end{proof}

\begin{corollary}\label{corollary:抽象代数--推论1.8.3}
\begin{enumerate}[(1)]
\item\label{corollary:抽象代数--推论1.8.3-1} 若$a$为$m$阶元素,则$a^k$为$m$阶元素的充要条件是$(m,k)=1$;

\item\label{corollary:抽象代数--推论1.8.3-2} 若$a,b$的阶分别为$m,n$且$ab=ba$,$(m,n)=1$,则$ab$的阶为$mn$.
\end{enumerate}
\end{corollary}
\begin{proof}
\begin{enumerate}[(1)]
\item 这是\refthe{theorem:抽象代数--定理1.8.4}的自然推论.

\item 设$m_1$是$\langle a\rangle\cap\langle b\rangle$的阶,由\refcor{corollary:抽象代数-推论 1.3.4}知$\langle a\rangle,\langle b\rangle$的阶分别为$m,n$.由于$\langle a\rangle\cap\langle b\rangle$是$\langle a\rangle,\langle b\rangle$的子群,故由\hyperref[theorem:抽象代数-Lagrange定理-定理 1.3.3]{Lagrange定理}知$m_1|m$,$m_1|n$. 但$(m,n)=1$,故$m_1=1$,因而$\langle a\rangle\cap\langle b\rangle=\{1\}$,于是由\refthe{theorem:抽象代数--定理1.8.4}知$ab$的阶为$[m,n]=mn$.
\end{enumerate}
\end{proof}












\end{document}
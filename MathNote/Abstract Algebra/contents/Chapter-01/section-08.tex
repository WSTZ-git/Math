\documentclass[../../main.tex]{subfiles}% 注意这里的文件路径不能用 ./main.tex ,否则用latexmk编译子文件会报错
\graphicspath{{\subfix{./image/}}} % 指定图片目录,后续可以直接使用图片文件名
% 注意这里的文件路径不能用 ../../image/ ,否则用latexmk编译子文件会报错

% 例如:
% \begin{figure}[H]
% \centering
% \includegraphics[scale=0.3]{图.png}
% \caption{}
% \label{figure:图}
% \end{figure}
% 注意:上述\label{}一定要放在\caption{}之后,否则引用图片序号会只会显示??.

\begin{document}

\section{循环群}

\begin{definition}[循环群]
设$G$是一个群且$a\in G$,我们称
\begin{align*}
\langle a \rangle = \{a^n|n \in \mathbb{Z}\}
\end{align*}
是\textbf{由 \( a \) 生成的 \( G \) 的子群}, 如果在一个群 \( G \) 中存在一个元素 \( a \), 使得 \( G = \langle a \rangle \), 即 \( G \) 由 \( a \) 生成, 则称 \( G \) 是\textbf{循环群}, \( a \) 为 \( G \) 的一个\textbf{生成元}.
\end{definition}
\begin{remark}
对$\forall n_1,n_2\in \mathbb{Z}$,有$a^{n_1}a^{-n_2}=a^{n_1-n_2}\in G$.因此$\langle a \rangle$是$G$的子群.故由$a$生成的$G$的子群是良定义的.
\end{remark}
\begin{remark}
显然若$a$在群$G$中,则$\langle a \rangle\subseteq G.$
\end{remark}

\begin{proposition}[循环群都是Abel群]
循环群都是Abel群.
\end{proposition}
\begin{proof}
设$G=\langle a \rangle$为循环群,则对任意的$x,y \in G$,存在$k,l$,使$x=a^k$,$y=a^l$,于是
\begin{align*}
xy = a^k a^l = a^{k+l} = a^l a^k = yx.
\end{align*}
所以$G$为Abel群.

\end{proof}

\begin{proposition}[素数阶群必为循环群]\label{proposition:素数阶群必为循环群}
设$G$是一个群,且$|G|=p$为一个素数,则
$G$必是循环群,并且\( \forall a \in G \) 且 \( a \neq e \) 有 \( G = \langle a \rangle \). 
\end{proposition}
\begin{proof}
由$p>1$知$G$中至少存在一个非幺元$a\ne e$,则对\( \forall a \in G \) 且 \( a \neq e \) ,有$\langle a\rangle$是$G$的子群。由\hyperref[theorem:抽象代数--Lagrange定理]{Lagrange定理}知$\langle a\rangle$的阶是$|G|=p$的因数,而$p$为素数,故$\langle a\rangle$的阶为$1$或$p$。由$a,e\in\langle a\rangle$知$\langle a\rangle$的阶必大于$1$,因此$\langle a\rangle$的阶为$p$。又因为$\langle a\rangle\subseteq G$,所以$G=\langle a\rangle$。故$G$为循环群。

\end{proof}

\begin{definition}
设 \( a \) 是群 \( G \) 的元素. 若 \( \forall k \in \mathbb{N}, a^k \neq 1 \), 则称 \( a \) 的\textbf{阶为无穷},记作$\mathrm{ord}\,a=\infty $. 若 \( \exists k \in \mathbb{N} \), 使得 \( a^k = 1 \), 则 \( r=\min\{k|k \in \mathbb{N}, a^k = 1\} \) 称为 \( a \) 的\textbf{阶},记作$\mathrm{ord}\,a=r$.
\end{definition}

\begin{theorem}\label{theorem:抽象代数-推论 1.3.4}
有限群 \( G \) 的任一元素 \( a \) 的阶是 \( G \) 的阶的因子,即$\text{ord}\,a\mid |G|.$
特别地,$\,G=\langle a \rangle\iff \text{ord}\,a=|G|.$
\end{theorem}
\begin{proof}
令 \( \langle a \rangle = \{a^n|n \in \mathbb{Z}\} \), 容易验证这是 \( G \) 的一个子群. 又由于 \( G \) 有限, 故 \( \langle a \rangle \) 有限, 因而 \( a \) 是有限阶的, 设为 \( d \). 对 \( n \in \mathbb{Z} \) 有 \( t_n \) 与 \( r_n \) (\( 0 \leqslant r_n < d \)), 使 \( n = t_nd + r_n \), 于是 \( a^n = a^{r_n} \).因此$\langle a \rangle$中至多只有$d$个元素$1,a,\cdots,a^{d-1}$.

又对 \( \forall r_1, r_2 \in \mathbb{N} \), 且 \( r_1 \neq r_2 \), \( 0 \leqslant r_1, r_2 < d \), 则 \( |r_1 - r_2| < d \), 从而 \( a^{r_1 - r_2} \neq 1 \), 进而 \( a^{r_1} \neq a^{r_2} \). 故 \( 1, a, \cdots, a^{d-1} \) 互不相同.
由此知 \( \langle a \rangle = \{1, a, \cdots, a^{d-1}\} \), 即 \( \langle a \rangle \) 是 \( d \) 阶群. 故由\hyperref[theorem:抽象代数--Lagrange定理]{Lagrange定理}知 \( d \) 为 \( [G:1] \) 的因子.

设$\text{ord}\,a=d$.
若$G=\langle a \rangle$,则由上述证明知$G=\langle a \rangle = \{1, a, \cdots, a^{d-1}\}$是$d$阶群,故$d=|G|.$
又若$d=|G|$,则由上述证明知$\langle a \rangle=d=|G|$.又显然有$\langle a \rangle\subseteq G$,故$\langle a \rangle = G$.

\end{proof}

\begin{theorem}[群元素的阶的基本性质]\label{theorem:群元素的阶的基本性质}
设$(G,\cdot)$是一个群,$a,b\in G$,则
\begin{enumerate}[(1)]
\item\label{theorem:群元素的阶的基本性质-1} \( a \) 的阶为无穷当且仅当$\forall m,n\in \mathbb{Z}$且\( m \neq n \) 时, \( a^m \neq a^n \).

\item\label{theorem:群元素的阶的基本性质-2}  设 \( a \) 的阶为 \( d \), 则
\begin{align}
a^m = a^n \iff \ m \equiv n \ (\text{mod}\ d). \label{eq:1.2.4-1341294782947}
\end{align}

\item\label{theorem:群元素的阶的基本性质-3}  $\text{ord}\,a = \text{ord}\,a^{-1} = \text{ord}(gag^{-1}),\forall g\in G.$进而$\text{ord}(ab)=\text{ord}(ba).$

\item\label{theorem:群元素的阶的基本性质-4} 若$a$为$m$阶元素,则对$\forall k\in \mathbb{N}$,$a^k$的阶为$\dfrac{m}{(m,k)}$,其中$(m,k)$是$m$与$k$的最大公因数.

进而,$\,a^k$为$m$阶元素的充要条件是$(m,k)=1$.

\item\label{theorem:群元素的阶的基本性质-5} 若$a,b$的阶分别为$m,n$且$\langle a\rangle\cap\langle b\rangle=\{1\}$,$ab=ba$,则$ab$的阶为$m,n$的最小公倍数$[m,n]$.

\item\label{theorem:群元素的阶的基本性质-6} 若$a,b$的阶分别为$m,n$且$ab=ba$,$(m,n)=1$,则$ab$的阶为$mn$.

\item\label{theorem:群元素的阶的基本性质-7} 设$\mathrm{ord}\,a = n$, $r$是任一整数. 如果$(n,r) = d$, 则
$\langle a^r \rangle = \langle a^d \rangle.$
\end{enumerate}
\end{theorem}
\begin{proof}
\begin{enumerate}[(1)]
\item 事实上, 若 \( a \) 的阶为无穷, 而有 \( m \neq n \), 使 \( a^m = a^n \). 设 \( m > n \), 于是 \( a^m(a^n)^{-1} = 1 \), 而 \( a^m(a^n)^{-1} = a^{m-n} = 1 \), 自然 \( m - n \in \mathbb{N} \). 矛盾.

反之, \( \forall m, n \in \mathbb{Z} \) 且 \( m \neq n \),有\( a^m \neq a^n \),则\( a^{m-n} = a^m(a^n)^{-1} = 1 \), 即 \( \forall k \in \mathbb{N} \) 有 \( a^k \neq 1 \), 故 \( a \) 的阶为无穷.

\item 设 \( a \) 的阶为 \( d \), \( m, n \in \mathbb{N} \), 由带余除法知,一定能找到整数 \( t_1, t_2, r_1, r_2 \), 使 \( m = dt_1 + r_1 (0 \leqslant r_1 < d) \), \( n = dt_2 + r_2 (0 \leqslant r_2 < d) \). 于是 \( a^m = (a^d)^{t_1} a^{r_1} = a^{r_1} \), \( a^n = (a^d)^{t_2} a^{r_2} = a^{r_2} \), 因而
\[
a^m = a^n \iff \ a^{r_1} = a^{r_2} \iff  \ a^{r_1 - r_2} = a^{r_2 - r_1} = 1.
\]
又 \( |r_1 - r_2| < d \), 故上式也等价于 \( r_1 - r_2 = 0 \), 即式 \(\eqref{eq:1.2.4-1341294782947}\) 成立.

\item 如果$\text{ord}\,a$或$\text{ord}\,a^{-1}$有一个为有限的,记为$n$.则由 \( (a^n)^{-1} = (a^{-1})^n \) 知另一个也必是有限的,这说明$\text{ord}\,a$与$\text{ord}\,a^{-1}$必同时有限或同时无限,故仅需对$\text{ord}\,a$与$\text{ord}\,a^{-1}$同时有限的情形加以证明.再由 \( (a^n)^{-1} = (a^{-1})^n \) 知\( a^k = 1 \) 当且仅当 \( (a^{-1})^k = 1 \), 故 \( a^{-1} \) 与 \( a \) 同阶.

如果$\text{ord}\,a$或$\text{ord}(gag^{-1})$有一个为有限的,当$\text{ord}\,a = n<\infty$时,则
\begin{align*}
(gag^{-1})^n=ga^ng^{-1}=1,
\end{align*}
故$\text{ord}(gag^{-1})\leqslant n<\infty.$当$\text{ord}\,(gag^{-1})<\infty$时,同理可证$\text{ord}\,a <\infty.$
这说明$\text{ord}\,a$与$\text{ord}(gag^{-1})$必同时有限或同时无限,故仅需对$\text{ord}\,a$与$\text{ord}(gag^{-1})$同时有限的情形加以证明.
现设$\text{ord}\,a = r_1$,$\text{ord}(gag^{-1}) = r_2$,则
\begin{align*}
a^{r_2} = g^{-1}(gag^{-1})^{r_2}g = 1, \\
(gag^{-1})^{r_1} = ga^{r_1}g^{-1} = 1.
\end{align*}
由此得$r_2 \geqslant r_1$,$r_1 \geqslant r_2$,从而$r_1 = r_2$.所以$gag^{-1}$与$a$有相同的阶.

注意到$ba = b(ab)b^{-1}$,故只需取$g=b$,则由上述证明知$ab$与$ba$有相同的阶.

\item 设$a^k$的阶为$q$,即$a^{kq}=1$,因而有$m|kq$,故由数论相关结论知$\frac{m}{(m,k)}|q$. 又$(a^k)^{\frac{m}{(m,k)}}=(a^m)^{\frac{k}{(m,k)}}=1$,即得$q|\frac{m}{(m,k)}$,因而$q=\frac{m}{(m,k)}.$

\item 设$ab$的阶为$m_1$,则有$(ab)^{m_1}=1$. 由$ab=ba$知$a^{m_1}b^{m_1}=(ab)^{m_1}=1$,即$a^{m_1}=b^{-m_1}\in\langle a\rangle\cap\langle b\rangle=\{1\}$,因而$a^{m_1}=b^{m_1}=1$,故$m|m_1$,$n|m_1$,因而$[m,n]|m_1$. 另有$(ab)^{[m,n]}=a^{[m,n]}b^{[m,n]}=1$,故$m_1|[m,n]$,即$m_1=[m,n]$.

\item 设$m_1$是$\langle a\rangle\cap\langle b\rangle$的阶,由\refthe{theorem:抽象代数-推论 1.3.4}知$\langle a\rangle,\langle b\rangle$的阶分别为$m,n$.由于$\langle a\rangle\cap\langle b\rangle$是$\langle a\rangle,\langle b\rangle$的子群,故由\hyperref[theorem:抽象代数--Lagrange定理]{Lagrange定理}知$m_1|m$,$m_1|n$. 但$(m,n)=1$,故$m_1=1$,因而$\langle a\rangle\cap\langle b\rangle=\{1\}$,于是由\rrefthe{theorem:群元素的阶的基本性质}{theorem:群元素的阶的基本性质-5}知$ab$的阶为$[m,n]=mn$.

\item 因为$(n,r)=d$, 所以存在$u,v\in\mathbb{Z}$, 使
\begin{align*}
d=nu+rv.
\end{align*}
于是$a^d=a^{nu+rv}=a^{rv}\in\langle a^r\rangle$,故$\langle a^d \rangle\subseteq \langle a^r \rangle$.另一方面, 同样由于$(n,r)=d$, 所以$d\mid r$, 从而又有$a^r\in\langle a^d\rangle$,于是$\langle a^r\rangle\subseteq \langle a^d\rangle$. 由此得$\langle a^r\rangle=\langle a^d\rangle$.
\end{enumerate}

\end{proof}

\begin{theorem}\label{theorem:抽象代数--定理1.8.1}
循环群的任何子群也是循环群.进而循环群$\langle a \rangle$的所有子群构成的集合为
\begin{align*}
\{\langle a^r \rangle\mid r=0,1,2\cdots\}.
\end{align*}
\end{theorem}
\begin{proof}
设$G_1$是循环群$G=\langle a\rangle$的一个非平凡子群. 令
\[
k=\min\{m'\in\mathbb{N}|a^{m'}\in G_1\},
\]
于是$G$中由$a^k$生成的子群$\langle a^k\rangle\subseteq G_1$. 又若有$a^{m'}\in G_1$, 则有整数$q,r$满足
\[
m'=kq+r,\quad 0\leqslant r<k,
\]
因而$a^r=a^{m'}(a^k)^{-q}\in G_1$, 由$k$的取法知$r=0$, 否则与$k$的最小值取法矛盾!因而$a^{m'}=(a^k)^q\in\langle a^k\rangle$, 故$G_1\subseteq \langle a^k \rangle$,所以$G_1=\langle a^k\rangle$为循环群.于是$G_1\subseteq \{\langle a^r \rangle\mid r=0,1,2\cdots\}.$因此记$\langle a\rangle$的所有子群构成的集合为$S$,则$S\subseteq \{\langle a^r \rangle\mid r=0,1,2\cdots\}.$又显然有$S\supseteq \{\langle a^r \rangle\mid r=0,1,2\cdots\},$故
\begin{align*}
S= \{\langle a^r \rangle\mid r=0,1,2\cdots\}.
\end{align*}

\end{proof}

\begin{corollary}\label{corollary:抽象代数--定理1.8.1推论}
\begin{enumerate}[(1)]
\item\label{corollary:抽象代数--定理1.8.1推论-1} 设 \( m \in \mathbb{Z} \), 则 \( m\mathbb{Z} \triangleq  \{mx|x \in \mathbb{Z}\} \) 是整数加法群 \( \mathbb{Z} \) 的子群. 

\item\label{corollary:抽象代数--定理1.8.1推论-2} 整数加法群$\mathbb{Z}$的任何子群必形如$m\mathbb{Z}(m\in \mathbb{N}_0)$.
\end{enumerate}
\end{corollary}
\begin{proof}
\begin{enumerate}[(1)]
\item 对$\forall x_1,x_2\in \mathbb{Z},$有
\begin{align*}
mx_1-mx_2=m(x_1-x_2)\in m\mathbb{Z}.
\end{align*}
故$m\mathbb{Z}$是整数加法群 \( \mathbb{Z} \) 的子群. 

\item 事实上, $\mathbb{Z}=\langle 1\rangle$. 设$G_1$为$\mathbb{Z}$的子群. 于是由\refthe{theorem:抽象代数--定理1.8.1}有$m\geqslant 0\text{且}m\in \mathbb{Z}$, 使得$G_1=\langle m\rangle=m\mathbb{Z}$.
\end{enumerate}

\end{proof}

\begin{proposition}\label{proposition:整数加群的商群}
设 \( m > 0\), 则有
\begin{align*}
m\mathbb{Z} \lhd  \mathbb{Z},\quad \mathbb{Z} = \bigcup_{k=0}^{m-1} (k + m\mathbb{Z}), \quad \mathbb{Z}_m \triangleq  \mathbb{Z}/m\mathbb{Z}  = \{\overline{0}, \overline{1}, \cdots, \overline{m-1}\} ,\quad |\mathbb{Z}_m|=[\mathbb{Z}: m\mathbb{Z}] = m.
\end{align*}
\end{proposition}
\begin{proof}
由\rrefcor{corollary:抽象代数--定理1.8.1推论}{corollary:抽象代数--定理1.8.1推论-1}知\(m\mathbb{Z} \) 为 \( \mathbb{Z} \) 的子群.

\end{proof}

\begin{theorem}\label{theorem:有限和无限循环群的常用结论}
设$G = \langle a \rangle$为循环群,则有以下结论.
\begin{enumerate}[(1)]
\item\label{theorem:有限和无限循环群的常用结论-0} $\langle a^{-1} \rangle = \langle a \rangle$.

\item\label{theorem:有限和无限循环群的常用结论-1} 如果$|G| = \infty$, 则
\begin{enumerate}[(i)]
\item\label{theorem:有限和无限循环群的常用结论-1-0} $G = \{1, a, a^{-1}, a^2, a^{-2}, a^3, a^{-3}, \cdots\},$
且对$\forall k,l \in \mathbb{Z}$,有$a^k = a^l\iff k = l$.

\item\label{theorem:有限和无限循环群的常用结论-1-1} $a$与$a^{-1}$是$G$的两个仅有的生成元.

\item\label{theorem:有限和无限循环群的常用结论-1-2} $G$的全部子群为
$\{\langle a^d \rangle \mid d = 0,1,2,\cdots\}$,并且对$\forall d_1,d_2\in \mathbb{N}_0\text{且}d_1\neq d_2$,有$\langle a^{d_1} \rangle \neq \langle a^{d_2} \rangle$.

\item\label{theorem:有限和无限循环群的常用结论-1-3} $G\cong (\mathbb{Z},+)$.
\end{enumerate}

\item\label{theorem:有限和无限循环群的常用结论-2} 如果$|G| = n<\infty$, 则
\begin{enumerate}[(i)]
\item\label{theorem:有限和无限循环群的常用结论-2-0} $G = \{1, a, a^2, a^3, \cdots, a^{n-1}\},$
且对$\forall k,l \in \mathbb{Z}$,有$a^k = a^l \iff n \mid k - l.$

\item\label{theorem:有限和无限循环群的常用结论-2-1} $G$恰有$\phi(n)$个生成元,且$a^r$是$G$的生成元的充分必要条件是$(n,r)=1$,其中,$\phi(n)$是欧拉函数.

\item\label{theorem:有限和无限循环群的常用结论-2-2} $G$的全部子群为
$\{\langle a^d \rangle \mid d$为$n$的正因子$\}$,并且对$\forall d_1,d_2$为$n$的不同正因子,有$\langle a^{d_1} \rangle \neq \langle a^{d_2} \rangle$.

若还有正整数$n$的标准分解式为
\begin{align*}
n = p_1^{r_1}p_2^{r_2}\cdots p_s^{r_s},
\end{align*}
其中$p_1,p_2,\cdots,p_s$是$n$的不同素因子. 则$n$阶循环群$G$的子群的个数为
\begin{align*}
r = (r_1 + 1)(r_2 + 1)\cdots(r_s + 1).
\end{align*}

\item\label{theorem:有限和无限循环群的常用结论-2-3} $G\cong \mathbb{Z}_n$.
\end{enumerate}
\end{enumerate}
\end{theorem}
\begin{proof}
\begin{enumerate}[(1)]
\item 由循环群的定义易得.

\item \begin{enumerate}[(i)]
\item 由循环群的定义知$G = \{1, a, a^{-1}, a^2, a^{-2}, a^3, a^{-3}, \cdots\}$.因为这个集合中的元素互不相同,所以
\begin{align*}
a^k=a^l\iff a^{k-l}=1\iff k-l = 0\iff k=l.
\end{align*}

\item 由\rrefthe{theorem:有限和无限循环群的常用结论}{theorem:有限和无限循环群的常用结论-0}知$a$与$a^{-1}$都是$G$的生成元.又如,$a^k$是$G$的任一生成元,则存在$n \in \mathbb{Z}$,使
\begin{align*}
(a^k)^n = a^{kn} = a.
\end{align*}
由\rrrefthe{theorem:有限和无限循环群的常用结论}{theorem:有限和无限循环群的常用结论-1}{theorem:有限和无限循环群的常用结论-1-0}得$kn=1$,从而$k=\pm1$.

\item 如果$|G|=\infty$, 由\refthe{theorem:抽象代数--定理1.8.1}知$G$的所有子群构成的集合为
\begin{align*}
S=\{\langle a^r\rangle \mid r=0,1,2,\cdots \}.
\end{align*}
只需证这个集合中的元素两两不同即可.
因为对任意的$r_1>r_2>0$, 有$r_1\nmid r_2$, 所以$a^{r_2}\notin\langle a^{r_1}\rangle$, 于是
\begin{align*}
\langle a^{r_1}\rangle\neq\langle a^{r_2}\rangle.
\end{align*}
另一方面, 对任意的$r>0$, 显然$a^r\notin\langle a^0\rangle=\langle 1\rangle=\{1\}$, 所以有
\begin{align*}
\langle a^r\rangle\neq\langle 1\rangle.
\end{align*}
故$\langle a^{d_1} \rangle \neq \langle a^{d_2} \rangle,\forall d_1,d_2\in \mathbb{N}_0\text{且}d_1\neq d_2$.

\item {\color{blue}证法一:}令
\begin{align*}
\phi: \mathbb{Z} &\to G, \\
k &\mapsto a^k,\quad \forall k \in \mathbb{Z}.
\end{align*}
显然$\phi$是$\mathbb{Z}$到$G$的良定义的映射;

设$k,l \in \mathbb{Z}$,如果$a^k = a^l$,则由\rrefthe{theorem:群元素的阶的基本性质}{theorem:群元素的阶的基本性质-4}得,$k = l$,所以$\phi$为$\mathbb{Z}$到$G$的单映射;

对任意的$a^k \in G$,有$k \in \mathbb{Z}$,使$\phi(k) = a^k$,所以$\phi$是$\mathbb{Z}$到$G$的满映射;

对任意的$k,l \in \mathbb{Z}$,
\begin{align*}
\phi(k + l) = a^{k+l} = a^k \cdot a^l = \phi(k) \cdot \phi(l),
\end{align*}
所以$\phi$是$\mathbb{Z}$到$G$的同构映射.因此,$G \cong (\mathbb{Z},+)$.

{\color{blue}证法二:}作$\mathbb{Z}$到$G$上的映射$\varphi:\varphi(n)=a^n(n\in\mathbb{Z})$. 于是有
\[
\varphi(n_1+n_2)=a^{n_1+n_2}=a^{n_1}\cdot a^{n_2}=\varphi(n_1)\varphi(n_2),
\]
因而$\varphi$是$\mathbb{Z}$到$G$上的同态映射, 故由\hyperref[theorem:群的同态基本定理]{群的同态基本定理}知$G\cong \mathbb{Z}/\ker\varphi$且$\ker \varphi \lhd \mathbb{Z}$. 由\rrefcor{corollary:抽象代数--定理1.8.1推论}{corollary:抽象代数--定理1.8.1推论-2}知存在$m\in \mathbb{N}_0,$使得$\ker\varphi=m\mathbb{Z}$.因此$G\cong \mathbb{Z}/m\mathbb{Z}$.

若$m\neq 0$,则由\refpro{proposition:整数加群的商群}知
\begin{align*}
G\cong\mathbb{Z}/m\mathbb{Z}=\mathbb{Z}_m.
\end{align*}
从而$|G|=|\mathbb{Z}_m|=m<\infty$,这与$|G|=\infty$矛盾!故此时$m=0,$因此$G\cong \mathbb{Z}.$
\end{enumerate}

\item \begin{enumerate}[(i)]
\item 由\refthe{theorem:抽象代数-推论 1.3.4}知$\text{ord}\,a=n$,故$a^n=1$。注意到对$\forall m\in \mathbb{Z}$,有
\begin{align*}
m\equiv 0\ \mathrm{or}\ 1\ \mathrm{or}\ \cdots\ \mathrm{or}\ n-1\ (\mathrm{mod}\;n).
\end{align*}
故由\rrefthe{theorem:群元素的阶的基本性质}{theorem:群元素的阶的基本性质-2}知
\begin{align*}
a^m=1\ \mathrm{or}\ a\ \mathrm{or}\ \cdots\ \mathrm{or}\ a^{n-1}.
\end{align*}
因此$\langle a\rangle \subseteq \{1,a,\cdots,a^{n-1}\}$。又显然有$\langle a\rangle \supseteq \{1,a,\cdots,a^{n-1}\}$,故$\langle a\rangle = \{1,a,\cdots,a^{n-1}\}$。

对$\forall k,l\in \mathbb{Z}$,由\rrefthe{theorem:群元素的阶的基本性质}{theorem:群元素的阶的基本性质-2}知
\begin{align*}
a^k=a^l \iff k\equiv l\ (\mathrm{mod}\;n) \iff n\mid k-l.
\end{align*}

\item 由\rrefthe{theorem:群元素的阶的基本性质}{theorem:群元素的阶的基本性质-4}可得$\text{ord}\,a^r = \frac{n}{(n,r)}$.再由\refthe{theorem:抽象代数-推论 1.3.4}可得
\begin{align*}
a^r \text{为} \, G \text{的生成元} \iff \text{ord}\,a^r = n \iff \frac{n}{(n,r)} = n \iff (n,r) = 1,
\end{align*}
故由欧拉函数的定义知$G$的生成元的个数为$\phi(n)$.

\item 如果$|G|=n$,对任意的正整数$r$, 存在$n$的正因子$d=(n,r)$, 由\rrefthe{theorem:群元素的阶的基本性质}{theorem:群元素的阶的基本性质-7}可知
\begin{align*}
\langle a^r\rangle=\langle a^d\rangle\in \{\langle a^d\rangle\mid d\text{为}n\text{的正因子}\}.
\end{align*}
记$G$的所有子群构成的集合为$S$,则由\refthe{theorem:抽象代数--定理1.8.1}知$S\subseteq \{\langle a^d\rangle\mid d\text{为}n\text{的正因子}\}$.又显然有$S\supseteq \{\langle a^d\rangle\mid d\text{为}n\text{的正因子}\}$,故
\begin{align*}
S=\{\langle a^d\rangle\mid d\text{为}n\text{的正因子}\}.
\end{align*}
若$d_1>d_2$为$n$的两个不同的正因子, 则$d_1\nmid d_2$, 于是$a^{d_2}\notin\langle a^{d_1}\rangle$, 从而
\begin{align*}
\langle a^{d_1}\rangle\neq\langle a^{d_2}\rangle.
\end{align*}
故$\langle a^{d_1} \rangle \neq \langle a^{d_2} \rangle,\forall d_1,d_2$为$n$的不同正因子.

由上述证明知,$n$阶循环群$G$的子群的个数恰为$n$的不同正因子的个数. 而$n$的不同正因子的个数等于
\begin{align*}
(r_1 + 1)(r_2 + 1)\cdots(r_s + 1),
\end{align*}
即得所证.

\item {\color{blue}证法一:}作$\mathbb{Z}$到$G$上的映射$\varphi:\varphi(n)=a^n(n\in\mathbb{Z})$. 于是有
\[
\varphi(n_1+n_2)=a^{n_1+n_2}=a^{n_1}\cdot a^{n_2}=\varphi(n_1)\varphi(n_2),
\]
因而$\varphi$是$\mathbb{Z}$到$G$上的同态映射, 故由\hyperref[theorem:群的同态基本定理]{群的同态基本定理}知$G\cong \mathbb{Z}/\ker\varphi$且$\ker \varphi \lhd \mathbb{Z}$. 由\rrefcor{corollary:抽象代数--定理1.8.1推论}{corollary:抽象代数--定理1.8.1推论-2}知存在$m\in \mathbb{N}_0,$使得$\ker\varphi=m\mathbb{Z}$.因此$G\cong \mathbb{Z}/m\mathbb{Z}$.

若$m\neq n$,则当$m=0$时,有$G\cong \mathbb{Z}$,从而$|G|=\infty$矛盾!当$m\neq 0,n$时,有
\begin{align*}
G\cong\mathbb{Z}/m\mathbb{Z}=\mathbb{Z}_m.
\end{align*}
从而$|G|=|\mathbb{Z}_m|=m\neq n$矛盾!故$m=n$.因此
\begin{align*}
G\cong \mathbb{Z}/n\mathbb{Z}=\mathbb{Z}_n.
\end{align*}

{\color{blue}证法二:}令
\begin{align*}
\phi: \mathbb{Z}_n &\to G, \\
\bar{k} &\mapsto a^k,\quad \forall \bar{k} \in \mathbb{Z}_n.
\end{align*}
设$\bar{k} = \bar{l}$,则$n \mid k - l$,于是$a^{k-l} = e$,从而$a^k = a^l$,所以$\phi$是$\mathbb{Z}_n$到$G$的良定义的映射;

设$\bar{k},\bar{l} \in \mathbb{Z}_n$,如果$\phi(\bar{k}) = \phi(\bar{l})$,即$a^k = a^l$,则$n \mid k - l$,从而$\bar{k} = \bar{l}$,所以$\phi$是$\mathbb{Z}_n$到$G$的单映射;

对任意的$a^k \in G$,有$\bar{k} \in \mathbb{Z}_n$,使$\phi(\bar{k}) = a^k$,所以$\phi$是$\mathbb{Z}_n$到$G$的满映射;

对任意的$\bar{k},\bar{l} \in \mathbb{Z}_n$,有
\begin{align*}
\phi(\bar{k} + \bar{l}) = \phi(\overline{k + l}) = a^{k+l} = a^k \cdot a^l = \phi(\bar{k}) \cdot \phi(\bar{l}),
\end{align*}
所以$\phi$是$\mathbb{Z}_n$到$G$的同构映射.因此$G \cong (\mathbb{Z}_n,+)$.
\end{enumerate}
\end{enumerate}

\end{proof}

\begin{corollary}
两个循环群同构当且仅当它们的阶相同.
\end{corollary}
\begin{proof}
设$G_1,G_2$为两个循环群,则由\rrrefthe{theorem:有限和无限循环群的常用结论}{theorem:有限和无限循环群的常用结论-1}{theorem:有限和无限循环群的常用结论-1-3}和\rrrefthe{theorem:有限和无限循环群的常用结论}{theorem:有限和无限循环群的常用结论-2}{theorem:有限和无限循环群的常用结论-2-3}以及\refpro{proposition:整数加群的商群}知
\begin{align*}
&\,\,G_1\cong G_2\iff G_1\cong G_2\cong \mathbb{Z}\text{或}G_1\cong G_2\cong \mathbb{Z}_m(m\in \mathbb{N})\\
&\iff |G_1|=|G_2|=\infty\text{或}|G_1|=|G_2|=|\mathbb{Z}_m|=m(m\in \mathbb{N}).
\end{align*}

\end{proof}

\begin{corollary}\label{corollary:群仅有平凡子群的充要条件}
\begin{enumerate}[(1)]
\item\label{corollary:群仅有平凡子群的充要条件-1} 群$G$仅有平凡子群的充分必要条件是$G = \{1\}$或$G$是素数阶循环群.

\item\label{corollary:群仅有平凡子群的充要条件-2} 无限循环群的非平凡子群仍为无限循环群.
\end{enumerate}
\end{corollary}
\begin{proof}
\begin{enumerate}[(1)]
\item {\heiti 必要性:} 设$G$仅有平凡子群. 如果$G = \{1\}$,则结论成立. 如果$G \neq \{1\}$,则存在$a \in G$,使$a\neq 1$,从而$\langle a \rangle \neq \{1\}$,于是由$G$仅有平凡子群知$\langle a \rangle = G$. 由\rrrefthe{theorem:有限和无限循环群的常用结论}{theorem:有限和无限循环群的常用结论-1}{theorem:有限和无限循环群的常用结论-1-2}可知,$G$不可能是无限循环群.否则,由\rrrefthe{theorem:有限和无限循环群的常用结论}{theorem:有限和无限循环群的常用结论-1}{theorem:有限和无限循环群的常用结论-1-2}知$G$有无穷多个非平凡子群矛盾! 设$|G|=n$,由于$G$仅有平凡子群,所以再由\rrrefthe{theorem:有限和无限循环群的常用结论}{theorem:有限和无限循环群的常用结论-2}{theorem:有限和无限循环群的常用结论-2-2}知$n$无真因子,因此$n$为素数.

{\heiti 充分性:} 如果$G = \{1\}$,则$G$显然只有平凡子群. 如果$G$是素数阶循环群,则$|G|$的仅有的正因子为$1$及$|G|$,由这两个因子得到的都是$G$的平凡子群,所以再由\rrrefthe{theorem:有限和无限循环群的常用结论}{theorem:有限和无限循环群的常用结论-2}{theorem:有限和无限循环群的常用结论-2-2}知$G$仅有平凡子群.

\item 设$G$为无限循环群,则由\refthe{theorem:循环群都同构于整数加群}知$G\cong \mathbb{Z}.$
又由\rrefcor{corollary:抽象代数--定理1.8.1推论}{corollary:抽象代数--定理1.8.1推论-2}知$\mathbb{Z}$的非平凡子群为$m\mathbb{Z}(m\in \mathbb{Z}\text{且}m\neq 0,1)$为无限循环群.故$G$的非平凡子群也为无限循环群.
\end{enumerate}

\end{proof}

\begin{example}
任一偶数阶群必含有阶为2的元素.
\end{example}
\begin{proof}
设$S$为$G$的所有阶大于2的元素的集合,$T$为$G$的所有阶小于等于2的元素的集合. 如果$S$为空集,则$|S| = 0$;如果$S$非空,则对任意的$x \in S$,有$\text{ord}\,x^{-1} = \text{ord}\,x > 2$,所以$x^{-1} \in S$且$x^{-1} \neq x$,由此得$|S|$必为偶数. 因为已知$G$的阶为偶数,所以$G$中阶小于等于2的元素的个数$|T|$为偶数. 由于$G$中有且仅有一个阶为1的元素,即$1$,所以$|T| \neq 0$,从而$|T| \geqslant 2$且$T$中除$1$外其余元素的阶都是2. 因此$G$必含有阶为2的元素.

\end{proof}

\begin{proposition}\label{proposition:有限交换群必有阶的任意素因子的阶的元素}
设$G$为有限交换群,$|G|=n$. 证明:对$n$的任一素因子$p$,$G$必有阶为$p$的元素.
\end{proposition}
\begin{proof}
对$n$应用数学归纳法.
首先,当$n=2$时,结论显然成立.
假设结论对所有阶小于$n$的交换群成立.考察阶为$n$的交换群$G$,设$p$为$n$的任一素因子.
任取$a\in G,a\neq e$,设$\text{ord}a=r$.
\begin{enumerate}[(a)]
\item 如果$r=pk$,则由\rrefthe{theorem:群元素的阶的基本性质}{theorem:群元素的阶的基本性质-4}知$\text{ord}a^k=\frac{r}{(r,k)}=p$,结论成立.
\item 如果$p\nmid r$,令$H=\langle a\rangle$,则由\rrefpro{proposition:正规子群的基本性质}{proposition:正规子群的基本性质-1}知$H$为$G$的正规子群,且商群$G/H$为交换群.而$|G/H|=\frac{n}{r}<n$,且因$p\nmid r$,所以$p\mid\left(\frac{n}{r}\right)$.从而由归纳假设知,存在$bH\in G/H$,使$\text{ord}bH=p$,则$b^p\in H$.于是$b^{pr}=e$.由于$p\nmid r$,所以$(bH)^r\neq H$,即$b^r\notin H$,于是$b^r\neq e$.而$(b^r)^p=e$,所以$\text{ord}b^r=p$.
\end{enumerate}
从而由归纳法原理知结论成立.

\end{proof}

\begin{proposition}\label{proposition:有限群不同子群有不同阶的性质}
设$G$是$n$阶群且其不同的子群有不同的阶. 试证:
\begin{enumerate}[(1)]
\item $G$的任何子群都是正规子群;
\item $G$的子群与商群的不同子群也有不同的阶;
\item\label{proposition:有限群不同子群有不同阶的性质-3} $G$是循环群.
\end{enumerate}
\end{proposition}
\begin{proof}
\begin{enumerate}[(1)]
\item 设$H$为$G$的子群, $g\in G$.对$\forall h_1,h_2\in H$,有
\begin{align*}
(gh_1g^{-1})(gh_2g^{-1})^{-1}=(gh_1g^{-1})(gh_2^{-1}g^{-1})=gh_1h_2^{-1}g^{-1}\in gHg^{-1}.
\end{align*}
故$gHg^{-1}$是$G$的子群.又由\rrefthe{theorem:子群陪集相关性质}{theorem:子群陪集相关性质-5}知$gHg^{-1}$与$H$有相同的阶. 因此由条件知$gHg^{-1}=H$, 故$H$是正规子群.

\item 设$H_1, H_2$是$G$的子群$H$的子群, 自然也是$G$的子群, 于是由条件知$H_1=H_2$当且仅当$|H_1|=|H_2|$.

设$\overline{H_1}, \overline{H_2}$是商群$G/H$的子群.记$\pi$为$G$到商群$G/H$上的自然同态,$G$中包含$H$的子群的集合为$\Sigma$,$G/H$的子群的集合为$\Gamma$, 由\rrefcor{corollary:群同态第二定理推论}{corollary:群同态第二定理推论-1}知有$G$的子群$H_1\supseteq H$, $H_2\supseteq H$使得
\begin{align*}
\overline{H_1}= \pi(H_1)=H_1/H,\quad \overline{H_2}=\pi(H_2)= H_2/H.
\end{align*}
因为$\pi$是$\Sigma\to \Gamma$的双射,所以$\overline{H_1}=\overline{H_2}$当且仅当$H_1=H_2$. 而$H_1=H_2$当且仅当$|H_1|=|H_2|$. 注意
\begin{align*}
|H_i|=[H_i:H]|H|=|\overline{H_i}||H|,\quad i=1,2.
\end{align*}
于是$\overline{H_1}=\overline{H_2}$当且仅当$|\overline{H_1}|=|\overline{H_2}|$.
\item 设$|G|=p_1p_2\cdots p_s$, 其中$p_i(1\leqslant i\leqslant s)$是素数.

对$s$作归纳证明$G$是循环群. 若$s=0$, 则$|G|=1$, 显然$G$是循环群. 若$s=1$, $|G|=p_1$是素数, 由\refpro{proposition:素数阶群必为循环群}知$G$是循环群. 假定$s-1$时结论成立. 以$e$表示$G$的幺元, 取$a_1\in G$, $a_1\neq e$. 若$a_1$的阶为$n$, 则$G$是循环群. 不妨设$a_1$的阶为$p_sp_{s-1}\cdots p_k\neq n$, 于是$a=a_1^{p_{s-1}\cdots p_k}$的阶为$p_s$. 由结论(1), $\langle a\rangle$是$G$的正规子群. 由结论(2), 商群$G/\langle a\rangle$的不同子群有不同的阶, 由\refcor{corollary:抽象代数-推论 1.3.5}知$G/\langle a\rangle$的阶为$n_1=p_1p_2\cdots p_{s-1}$. 由归纳假设, $G/\langle a\rangle$是循环群. 于是存在$b\in G$使得$G/\langle a\rangle$的元素为$\langle a\rangle$, $b\langle a\rangle$, $\cdots$, $b^{n_1-1}\langle a\rangle$. 从而由$(b\langle a\rangle)^{n_1}=\langle a \rangle$知对$0\leqslant k<p_s$, 有$k_0(0\leqslant k_0<p_s)$使得
\begin{align*}
(ba^k)^{n_1}=a^{k_0}.
\end{align*}

下面证明$b\langle a\rangle$中有元素$c$使得$c^{n_1}\neq e$. 若$b^{n_1}\neq e$, 则可取$c=b$. 故设$b^{n_1}=e$. 注意$G/\langle a\rangle$的阶为$n_1$, 于是当$0<r<n_1$时, $b^r\neq e$, $(ba)^r\neq e$.
如果$(ba)^{n_1}=e$, 则$\langle b\rangle$与$\langle ba\rangle$均为$n_1$阶群, 因而由条件知$\langle b\rangle=\langle ba\rangle$, 于是有$ba=b^m$, $0<m<n_1$. 由于$ba\in b\langle a\rangle$, $b^m\in b^m\langle a\rangle$, 而$m\neq 1$时, 由\rrefthe{theorem:子群陪集相关性质}{theorem:子群陪集相关性质-4}知$b\langle a\rangle\cap b^m\langle a\rangle=\varnothing$, 于是$m=1$, 即$ba=b$, 从而$a=e$, 这就得到矛盾. 由此可知$(ba)^{n_1}\neq e$. 取$c=ba$.
由$c\in b\langle a\rangle$, 知$b\langle a\rangle=c\langle a\rangle$, 于是$G/\langle a\rangle=\langle c\langle a\rangle\rangle$. 因为$G/\langle a \rangle$的阶为$n_1$,所以$(c\langle a \rangle)^{n_1}=c^{n_1}\langle a \rangle=\langle a \rangle$.因而$c^{n_1}\in\langle a\rangle$. 注意$c^{n_1}\neq e$, 于是
\begin{align*}
c^{n_1}=a^m\neq e,\quad 1\leqslant m<p_s.
\end{align*}
因为$p_s$是素数, 所以有$(m,p_s)=1$. 进而$a\in\langle c\rangle$, $\langle a\rangle\subset\langle c\rangle$. 于是有
\begin{align*}
\langle c\rangle/\langle a\rangle=G/\langle a\rangle.
\end{align*}
因此$G=\langle c\rangle$为循环群.
\end{enumerate}
\end{proof}

\begin{theorem}
设$G$是一个$m$阶群,则$G$是循环群的充要条件是对$m$的每个因数$m_1$存在唯一的$m_1$阶子群.
\end{theorem}
\begin{proof}
{\heiti 必要性:}设$G=\langle a\rangle$. 从\refthe{theorem:抽象代数-推论 1.3.4}知$G$的阶$m$也就是元素$a$的阶. 由$m_1|m$知当$0<k<m_1$时有$0<\frac{km}{m_1}<m$,因而$(a^{\frac{m}{m_1}})^k\neq 1$,但$(a^{\frac{m}{m_1}})^{m_1}=1$,故$\langle a^{\frac{m}{m_1}}\rangle$是$G$的$m_1$阶子群.

下面证$m_1$阶子群的唯一性. 设$G_1$是$G$中的$m_1$阶子群,由\refthe{theorem:抽象代数--定理1.8.1}知$G_1=\langle a^k\rangle$,其中,$k\geqslant 0$,并且当$a^{m'}\in G_1$时,$k|m'$. 由$a^m=1\in G_1$知$k|m$,若$0<n<\frac{m}{k}$,则$0<kn<m$,从而$(a^k)^n=a^{kn}\neq 1$. 另外$(a^k)^{\frac{m}{k}}=1$,故$G_1$的阶为$\frac{m}{k}=m_1$,因而$k=\frac{m}{m_1}$,即$G_1=\langle a^{\frac{m}{m_1}}\rangle$.

{\heiti 充分性:}设$G_1,G_2$是$G$的两个不同子群,则由\hyperref[theorem:抽象代数--Lagrange定理]{Lagrange定理}知$[G_1:1],[G_2:1]$都是$m$的因数.
若$[G_1:1]=[G_2:1]$,则由条件知$G_1=G_2$矛盾!故$[G_1:1]\ne [G_2:1]$.因此$G$的不同的子群有不同的阶.于是由\rrefpro{proposition:有限群不同子群有不同阶的性质}{proposition:有限群不同子群有不同阶的性质-3}知$G$必是循环群.

\end{proof}













\end{document}
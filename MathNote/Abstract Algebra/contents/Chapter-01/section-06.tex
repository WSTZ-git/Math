\documentclass[../../main.tex]{subfiles}
\graphicspath{{\subfix{../../image/}}} % 指定图片目录,后续可以直接使用图片文件名。

% 例如:
% \begin{figure}[H]
% \centering
% \includegraphics[scale=0.4]{图.png}
% \caption{}
% \label{figure:图}
% \end{figure}
% 注意:上述\label{}一定要放在\caption{}之后,否则引用图片序号会只会显示??.

\begin{document}

\section{群论与数论}

\begin{definition}[整除]
令 $n\in \mathbb{Z}\setminus\{0\}$,而 $m\in \mathbb{Z}$。我们说 $n$ 整除 $m$,记作 $n\mid m$,若
\begin{align*}
m\in n\mathbb{Z}=\{kn:k\in\mathbb{Z}\}
\end{align*} 
\end{definition}

\begin{proposition}\label{proposition:nZ是Z的正规子群}
若 $n\in\mathbb{Z}$,则 $(n\mathbb{Z},+)\lhd(\mathbb{Z},+)$。
\end{proposition}
\begin{remark}
这里的加法和乘法都是通常意义下的整数加法和整数乘法.
\end{remark}
\begin{proof}
令 $f:\mathbb{Z}\to \mathbb{Z}$,对 $m\in\mathbb{Z}$,定义为
\begin{align*}
f(m)=mn.
\end{align*}
则对$\forall m_1,m_2\in (\mathbb{Z},+)$,都有 
\begin{align*}
f\left( m_1+m_2 \right) =\left( m_1+m_2 \right) n=m_1n+m_2n=f\left( m_1 \right) +f\left( m_2 \right) .
\end{align*}
故$f$是$(\mathbb{Z},+)$到$(\mathbb{Z},+)$的群同态。因此由\refpro{proposition:群同态的核是定义域的子群,像是陪域的子群}可知$n\mathbb{Z}=\operatorname{im}(f)<\mathbb{Z}$。又因为 $(\mathbb{Z},+)$ 是阿贝尔群,因此由\refpro{proposition:阿贝尔群的子群与正规子群等价}可知$(n\mathbb{Z},+)\lhd(\mathbb{Z},+)$.
\end{proof}

\begin{proposition}\label{proposition:Z的任意子群都形如nZ}
若 $(A, +)<(\mathbb{Z}, +)$,则存在 $n\in\mathbb{N}_0$,使得 $A = n\mathbb{Z}$。
\end{proposition}
\begin{proof}
(i) 若 $A = \{0\}$,则 $A = 0\mathbb{Z}$。

(ii) 若 $A\neq\{0\}$,则由 $(A, +)<(\mathbb{Z}, +)$ 可知,$A$ 在加法逆元下封闭。
从而 $A\cap\mathbb{N}_1\neq\varnothing$,否则 $A\subset\mathbb{Z}-\mathbb{N}_1$ 且 $A\neq\{0\}$,于是任取 $x\in A\subset\mathbb{Z}-\mathbb{N}_1$ 且 $x\neq 0$,则其加法逆元 $-x\in A$,但 $-x\in\mathbb{N}_1$,这与 $A\subset\mathbb{Z}-\mathbb{N}_1$ 矛盾!

令 $n = \min(A\cap\mathbb{N}_1)$($n$ 的良定义是因为良序公理),则 $n\in A$。我们断言 $A = n\mathbb{Z}$。

注意到 $n\mathbb{Z}=\{nm:m\in\mathbb{Z}\}=\langle n\rangle$,故我们只需证 $A = \langle n\rangle$。

任取 $m\in\mathbb{Z}$,则由 $n\in A$ 及 $A$ 在加法下封闭可知,$nm=\underset{m\text{个}}{\underbrace{n + n+\cdots + n}}\in A$。故 $\langle n\rangle\subset A$。

任取 $a\in A$,假设 $a\notin n\mathbb{Z}$,则由带余除法可知,存在 $q,r\in\mathbb{Z}$,使得 $a = qn + r$,其中 $0\leqslant r\leqslant n - 1$。因为 $a\notin n\mathbb{Z}$,所以 $r\neq 0$。又 $qn\in\langle n\rangle\subset A$,$a\in A$。故由 $A$ 对加法和加法逆元封闭可知,$r = a - qn\in A$。而 $1\leqslant r\leqslant n - 1 < n$,这与 $n = \min(A\cap\mathbb{N}_1)$ 矛盾!故 $a\in n\mathbb{Z}$。 
\end{proof}

\begin{corollary}
任意的无限循环群 $\langle x\rangle$ $(|x|=\infty)$的子群都是形如 $\langle x^n\rangle=\{x^{nm}:m\in\mathbb{Z}\}$ 的形式,进而都是正规子群。
 
即对任意的无限循环群 $\langle x\rangle$ $(|x|=\infty)$,任取$A<\langle x\rangle$,则一定存在$n\in \mathbb{Z}$,使得$A=\langle x^n\rangle=\{x^{nm}:m\in\mathbb{Z}\}$,并且$A\lhd \langle x\rangle$.
\end{corollary}
\begin{proof}
由\refpro{proposition:所有的无限循环群是彼此同构的}可知,任意无限循环群$\langle x\rangle(|x|=\infty)$都同构于整数加群$(\mathbb{Z},+)$,故$A$一定同构于$\mathbb{Z}$的某一子群.于是由\refpro{proposition:Z的任意子群都形如nZ}可知,存在$n\in \mathbb{Z}$,使得$A$同构于$n \mathbb{Z}$.因此$A=\langle x^n\rangle=\{x^{nm}:m\in\mathbb{Z}\}$.
又由\refpro{proposition:nZ是Z的正规子群}可知$n \mathbb{Z}\lhd \mathbb{Z}$.故$A\lhd \langle x\rangle$.
\end{proof}

\begin{definition}[同余(模$n$)]
设 $n\in \mathbb{N}_1$,而 $a,b\in \mathbb{Z}$。我们说 $a$ 同余 $b$(模 $n$),记作 $a\equiv b\bmod n$,若
\begin{align*}
a + n\mathbb{Z}=b + n\mathbb{Z},
\end{align*}
或
\begin{align*}
a - b\in n\mathbb{Z}.
\end{align*}
或
\begin{align*}
n\mid (a-b).
\end{align*}
或
\begin{align*}
    a\text{和}b (\mathrm{mod}\,\,n)\text{的余数相同}.
\end{align*}
\end{definition}
\begin{proof}
$n\mid (a-b)\Leftrightarrow a - b\in n\mathbb{Z}$是显然的.
由\reflem{lemma:关于两个陪集相等的充要条件}可知$a+n\mathbb{Z} =b+n\mathbb{Z} \Leftrightarrow a-b\in n\mathbb{Z} .$下证$a - b \in n\mathbb{Z} \Leftrightarrow a$和$b (\mathrm{mod}\,\,n)$的余数相同.

$\Rightarrow$: 由$a - b \in n\mathbb{Z}$可知, 存在$m \in \mathbb{Z}$, 使得$a - b = nm$. 从而$a = b + nm$. 由带余除法可知, 存在$q, r \in \mathbb{Z}$, 使得$b = qn + r$, 其中$0 \leqslant r \leqslant n - 1$. 于是
\begin{align*}
a = b + nm 
= (q + m)n + r.
\end{align*}
故$a$和$b (\mathrm{mod}\,\,n)$的余数都是$r$.

$\Leftarrow$: 由$a$和$b (\mathrm{mod}\,\,n)$的余数相同可知, 存在$q, p, r \in \mathbb{Z}$, 使得
\begin{align*}
a &= qn + r, \quad b = pn + r.
\end{align*}
其中$0 \leqslant r \leqslant n - 1$. 于是$a - b = (q - p)n \in n\mathbb{Z}$.

综上所述, $a$ 同余 $b$(模 $n$)是良定义的.
\end{proof}

\begin{proposition}[同余(模$n$)是($\mathbb{Z}$上的)等价关系]
设 $n\in \mathbb{N}_1$,对$\forall a, b, c \in \mathbb{Z}$, 都满足

自反性: $a \equiv a (\mathrm{mod}\,\,n)$.

对称性: 若$a \equiv b (\mathrm{mod}\,\,n)$, 则$b \equiv a (\mathrm{mod}\,\,n)$.

传递性: 若$a \equiv b (\mathrm{mod}\,\,n), b \equiv c (\mathrm{mod}\,\,n)$, 则$a \equiv c (\mathrm{mod}\,\,n)$.
\end{proposition}
\begin{proof}
自反性: 由$a + n\mathbb{Z} = a + n\mathbb{Z}$可知$a \equiv a (\mathrm{mod}\,\,n)$.

对称性: 由$a \equiv b (\mathrm{mod}\,\,n)$可知$a + n\mathbb{Z} = b + n\mathbb{Z}$, 从而$b + n\mathbb{Z} = a + n\mathbb{Z}$, 故$b \equiv a (\mathrm{mod}\,\,n)$.

传递性: 由$a \equiv b (\mathrm{mod}\,\,n), b \equiv c (\mathrm{mod}\,\,n)$可知$a + n\mathbb{Z} = b + n\mathbb{Z}, b + n\mathbb{Z} = c + n\mathbb{Z}$. 从而$a + n\mathbb{Z} = c + n\mathbb{Z}$. 故$a \equiv c (\mathrm{mod}\,\,n)$. 
\end{proof}

\begin{proposition}\label{proposition:模n同余等价类就是nZ}
设$n \in \mathbb{N}_1, a \in \mathbb{Z}$, 记在同余$(\mathrm{mod}\,\,n)$的等价关系下以$a$为代表元的等价类为$\overline{a} = [a]$, 则
\begin{align*}
\overline{a} = [a] = a + n\mathbb{Z}.
\end{align*}
\end{proposition}
\begin{proof}
若$b \in \overline{a}$, 则$a \equiv b (\mathrm{mod}\,\,n)$. 从而$a + n\mathbb{Z} = b + n\mathbb{Z}$. 于是$b = b + 0 \in b + n\mathbb{Z} = a + n\mathbb{Z}$. 故$\overline{a} \subset a + n\mathbb{Z}$.

若$b \in a + n\mathbb{Z}$, 则存在$m \in \mathbb{Z}$, 使得$b = a + nm$. 从而$a - b = nm \in n\mathbb{Z}$. 故$a \equiv b (\mathrm{mod}\,\,n)$. 因此$b \in \overline{a}$.故$ a + n\mathbb{Z}\subset \overline{a}$.

综上,$\overline{a} = a + n\mathbb{Z}$.
\end{proof}

\begin{definition}[模 $n$ 的同余类]
令 $n\in \mathbb{N}_1$,则 $\mathbb{Z}_n$ 定义为
\begin{align*}
\mathbb{Z}_n=\mathbb{Z}/n\mathbb{Z}.
\end{align*}
$\mathbb{Z}_n$ 中的每个元素,被称为一个\textbf{模 $n$ 的同余类}。
\end{definition}
\begin{note}
不难发现,$0,\cdots,n - 1$ 分别代表了 $n$ 个同余类。并且由\refpro{proposition:nZ是Z的正规子群}及\hyperref[proposition:商群]{商群的定义}可知$\mathbb{Z}_n$是一个商群.
\end{note}

\begin{proposition}\label{proposition:Z_n是一个Abel群}
$(\mathbb{Z}_n,+)$是一个Abel群.
\end{proposition}
\begin{proof}
设 $a+n\mathbb{Z} ,b+n\mathbb{Z} \in \mathbb{Z} _n$, 由\refpro{proposition:nZ是Z的正规子群}可知 $n\mathbb{Z} \lhd \mathbb{Z}$. 从而
\begin{align*}
a+n\mathbb{Z} +b+n\mathbb{Z} &=a+b+n\mathbb{Z} +n\mathbb{Z} \\
&=b+a+n\mathbb{Z} +n\mathbb{Z} \\
&=b+n\mathbb{Z} +a+n\mathbb{Z} .
\end{align*}
故 $(\mathbb{Z} _n,+)$ 是一个 Abel 群.
\end{proof}


\begin{proposition}\label{proposition:枚举Z_n的所有元素}
\[
\mathbb{Z}_n = \{k + n\mathbb{Z} : 0\leqslant k\leqslant n - 1\}
\]
其中枚举法(上述集合)中的这些陪集是两两不同的。
\end{proposition}
\begin{note}
这个命题和\refpro{proposition:模n同余等价类就是nZ}表明:
\begin{align*}
\mathbb{Z} _n=\mathbb{Z} /n\mathbb{Z} =\left\{ n\mathbb{Z} ,1+n\mathbb{Z} ,\cdots ,n-1+n\mathbb{Z} \right\} =\left\{ \overline{0},\overline{1},\cdots ,\overline{n-1} \right\} .
\end{align*}
\end{note}
\begin{proof}
首先证明这里列完了所有的陪集。令 $m\in\mathbb{Z}$,根据带余除法,我们可以找到 $q\in\mathbb{Z}$,以及 $0\leqslant r\leqslant n - 1$,使得
\begin{align*}
m = qn + r .
\end{align*}
由于
\begin{align*}
qn\in n\mathbb{Z},
\end{align*}
因此 $m + n\mathbb{Z}=r + n\mathbb{Z}\in\{k + n\mathbb{Z} : 0\leqslant k\leqslant n - 1\}$。这就证明了最多只有这 $n$ 个同余类。

接下来证明这 $n$ 个同余类是互异的。假如 $k + n\mathbb{Z}=k' + n\mathbb{Z}$,其中 $0\leqslant k,k'\leqslant n - 1$,则 $k - k'\in n\mathbb{Z}$。但是 $-(n - 1)\leqslant k - k'\leqslant (n - 1)$。而在这个范围内唯一 $n$ 的倍数就是 $0$,于是 $k - k' = 0$,或 $k = k'$。这就证明了这 $n$ 个同余类是互异的。

综上所述,
\begin{align*}
\mathbb{Z}_n = \{k + n\mathbb{Z} : 0\leqslant k\leqslant n - 1\}.
\end{align*} 
\end{proof}

\begin{proposition}
令$n \in \mathbb{N}_1$,则$\mathbb{Z}_n$是个$n$阶循环群。  
\end{proposition}
\begin{note}
由\refpro{proposition:所有有限阶循环群彼此同构}可知,给定$n$,所有$n$阶循环群都是同构的。因此我们只要研究了$\mathbb{Z}_n$,就研究了所有的有限循环群。 
\end{note}
\begin{proof}
我们只须证明$\mathbb{Z}_n $是一个循环群即可,也即$\mathbb{Z}_n = \langle 1 + n\mathbb{Z} \rangle$。任取$A\in \mathbb{Z}_n$,则由\refpro{proposition:枚举Z_n的所有元素}可知,$A=k+n\mathbb{Z}$,其中$0\leq k\leq n-1$.又由\refpro{proposition:nZ是Z的正规子群}可知$(n\mathbb{Z},+)\lhd (\mathbb{Z},+)$.从而
\begin{align*}
\underset{k\text{个}}{\underbrace{\left( 1+n\mathbb{Z} \right) +\cdots +\left( 1+n\mathbb{Z} \right) }}=k+n\mathbb{Z} =A.
\end{align*}
(注意$0$个$1 + n\mathbb{Z}$相加规定为$0 + n\mathbb{Z} = n\mathbb{Z}$).因此$\mathbb{Z}_n = \langle 1 + n\mathbb{Z} \rangle$.而由\refpro{proposition:枚举Z_n的所有元素}可知,这个群又是$n$阶的,因此是$n$阶循环群.
\end{proof}

\begin{definition}
定义乘法$\cdot: \mathbb{Z}_n \times \mathbb{Z}_n \to \mathbb{Z}_n$,$\, (a + n\mathbb{Z}) \cdot (b + n\mathbb{Z}) \mapsto ab + n\mathbb{Z}.$也即$\overline{a}\cdot \overline{b}\mapsto \overline{ab}.$
\end{definition}
\begin{proof}
设$\overline{a} = \overline{a^\prime} \in \mathbb{Z}_n, \overline{b} = \overline{b^\prime} \in \mathbb{Z}_n$,则
\begin{align*}
a + n\mathbb{Z} = a^\prime + n\mathbb{Z}, \quad b + n\mathbb{Z} = b^\prime + n\mathbb{Z}.
\end{align*}
从而$(a - a^\prime), (b - b^\prime) \in n\mathbb{Z}$。于是存在$k, l \in \mathbb{Z}$,使得
\begin{align*}
a^\prime - a = kn, \quad b^\prime - b = ln.
\end{align*}
因此
\begin{align*}
a^\prime b^\prime - ab = (a + kn)(b + ln) - ab 
= aln + bkn + kln^2 
= n(al + bk + ln) \in n\mathbb{Z}.
\end{align*}
故$a^\prime b^\prime + n\mathbb{Z} = ab + n\mathbb{Z}$,即$\overline{a^\prime b^\prime} = \overline{ab}$。故上述定义的乘法是良定义的.
\end{proof}

\begin{proposition}\label{proposition:Z_n是一个交换幺半群}
$(\mathbb{Z}_n, \cdot)$是个交换幺半群。
\end{proposition}
\begin{proof}
我们先证明乘法是良定义的。假设$a' + n\mathbb{Z} = a + n\mathbb{Z}, b' + n\mathbb{Z} = b + n\mathbb{Z}$。故$a' = a + nk, b' = b + nl$,其中$k, l \in \mathbb{Z}$。我们只须证明$a'b' - ab \in n\mathbb{Z}$。而这是因为
\begin{align*}
a'b' - ab = (a + nk)(b + nl) - ab 
= anl + bnk + n^2kl 
= n(al + bk + nkl) \in n\mathbb{Z} .
\end{align*}
单位元显然是$1 + n\mathbb{Z}$。这是因为$(a + n\mathbb{Z})(1 + n\mathbb{Z}) = a + n\mathbb{Z}$。

结合律也是显然的,因为$(\mathbb{Z}, \cdot)$是幺半群,所以设$\overline{a},\overline{b},\overline{c}\in \mathbb{Z} _n$,都有
\[
\left( \overline{a}\cdot \overline{b} \right) \cdot \overline{c}=\overline{ab}\cdot \overline{c}=\overline{abc}=abc+n\mathbb{Z} =\overline{a}\cdot \overline{bc}=\overline{a}\cdot \left( \overline{b}\cdot \overline{c} \right) .
\]

交换律,设$\overline{a},\overline{b}\in \mathbb{Z} _n$,则$\overline{a}\cdot \overline{b}=\overline{ab}=ab+n\mathbb{Z} =ba+n\mathbb{Z} =\overline{ba}.$

这样,我们就证明了$(\mathbb{Z}_n, \cdot)$是个幺半群。
\end{proof}

\begin{definition}
令$n \in \mathbb{N}_2$,则$\mathbb{Z}_n^\times$,定义为由$(\mathbb{Z}_n, \cdot)$中所有可逆元素构成的群。即
\begin{align*}
\mathbb{Z}_n^\times = \{k + n\mathbb{Z} : 0 \leqslant k \leqslant n - 1, \exists l \in \mathbb{Z}, kl \equiv 1 \pmod{n}\}
\end{align*}
也即
\begin{align*}
\mathbb{Z} _{n}^{\times}=\{\overline{k}:0\leqslant k\leqslant n-1,\exists \overline{l}\in \mathbb{Z} _n,\overline{k}\cdot \overline{l}\equiv \overline{1} \pmod{n} \}.
\end{align*}
\end{definition}
\begin{remark}
由\reflem{lemma:幺半群中所有可逆元构成了群}可知上述定义的$\mathbb{Z}_n^\times$确实是一个群.故上述定义是良定义的.
\end{remark}

\begin{lemma}[Bézout定理]\label{lemma:Bézout定理}
若$a, b, c \in \mathbb{N}_1$,则$ax + by = c$有整数解$x, y$当且仅当$\gcd(a, b) \mid c$。

特别地,对任意$a, b \in \mathbb{N}_1$,我们可以找到$x, y \in \mathbb{Z}$,使得$\gcd(a, b) = ax + by$。
\end{lemma}
\begin{proof}

\end{proof}

\begin{proposition}
设$n \in \mathbb{N}_2$,则
\begin{align*}
\mathbb{Z} _{n}^{\times}=\{k+n\mathbb{Z} :1\leqslant k\leqslant n-1,\mathrm{gcd(}k,n)=1\}=\{\overline{k}:1\leqslant k\leqslant n-1,\mathrm{gcd(}k,n)=1\}.
\end{align*}
因此 
\begin{align*}
|\mathbb{Z}_n^\times| = \phi(n) .
\end{align*}
特别地,若$p$是一个素数,则
\begin{align*}
\mathbb{Z}_p^\times = \{1 + p\mathbb{Z}, 2 + p\mathbb{Z}, \cdots, (p - 1) + p\mathbb{Z}\}=\{\overline{k}:1\leqslant k\leqslant p-1\}.
\end{align*}
因此
\begin{align*}
|\mathbb{Z}_p^\times| = p - 1 .
\end{align*}
\end{proposition}
\begin{proof}
我们只须证明,若$0 \leqslant k \leqslant n - 1$,则
\begin{align*}
(\exists l \in \mathbb{Z}, kl \equiv 1 \pmod{n}) \iff \gcd(k, n) = 1 .
\end{align*}
分两类情况。若$k = 0$,则显然左边是错的,而右边甚至是没有定义的,当然也是错的。即便你考虑$k$是$n$的倍数,那么$\gcd(k, n) = n$,也是错的。
若$1 \leqslant k \leqslant n - 1$,则
\begin{align*}
&\quad \quad \quad \exists l \in \mathbb{Z}, kl \equiv 1 \pmod{n} .
\\
&\iff \exists l \in \mathbb{Z}, \exists m \in \mathbb{Z}, kl + mn = 1 .
\\
&\iff \gcd(k, n) = 1 .
\end{align*}
其中第一个充要条件是因为同余的定义,第二个充要条件是因为\reflem{lemma:Bézout定理}.
这样我们就证明了$\mathbb{Z}_n^\times$是由那些$n$互素的数所在的陪集所构成的。特别地,这样的陪集的数量就是由欧拉$\phi$函数给出的,即
\begin{align*}
\phi(n) = |\{1 \leqslant k \leqslant n - 1 : \gcd(k, n) = 1\}| .
\end{align*}
接下来,若$p$是一个素数,则
\begin{align*}
\gcd(k, p) = 1 \iff p \nmid k.
\end{align*}
当然,从$1$到$p - 1$的这些数,都和$p$互素.因此
\begin{align*}
\mathbb{Z}_p^\times = \{1 + p\mathbb{Z}, 2 + p\mathbb{Z}, \cdots, (p - 1) + p\mathbb{Z}\} .
\end{align*}
故
\begin{align*}
|\mathbb{Z}_p^\times| = p - 1 .
\end{align*}
这就证明了这个命题。
\end{proof}

\begin{lemma}\label{lemma:有限群关于元素阶次方的引理}
令$(G, \cdot)$是个有限群,则对任意$a \in G$,$a^{|G|} = e$。
\end{lemma}
\begin{proof}
令$\langle a \rangle$是由$a$生成的循环子群。则由\hyperref[theorem:Lagrange定理]{Lagrange定理}可知,
\begin{align*}
|\langle a \rangle| \Big| |G|
\end{align*}
而由\refpro{proposition:有限循环群的元素可用枚举法全部列举出来}我们知道
\begin{align*}
|a| = |\langle a \rangle|
\end{align*}
因此,
\begin{align*}
a^{|G|} = \left(a^{|a|}\right)^{|G|/|a|} = e^{|G|/|a|} = e
\end{align*}
这就证明了这个引理。
\end{proof}

\begin{theorem}[Fermat小定理]\label{theorem:Fermat小定理}
令$p$是一个素数,而$p \nmid a$,则
\begin{align*}
a^{p - 1} \equiv 1 \pmod{p}
\end{align*}
同时左乘$a$,也可以得到
\begin{align*}
a^{p} \equiv a \pmod{p}
\end{align*}
\end{theorem}
\begin{note}
不妨设$1 \leqslant a \leqslant p - 1$的原因:

假设结论对$1 \leqslant a \leqslant p - 1$已经成立,则当$a \in \mathbb{Z}$时,由带余除法可知,存在$m, r \in \mathbb{Z}$且$1 \leqslant r \leqslant p - 1$,使得
\begin{align*}
a = mp + r.
\end{align*}
于是$1 \leqslant r = a - mp \leqslant p - 1$且$p \nmid a$. 从而由假设可知
\begin{align*}
(a - mp)^{p - 1} = r^{p - 1} \equiv 1 \pmod{p}.
\end{align*}
即
\begin{align*}
(a - mp)^{p - 1} - 1 \in p\mathbb{Z}.
\end{align*}
因此存在$s \in \mathbb{Z}$,使得
\begin{align*}
(a - mp)^{p - 1} - 1 = ps \iff a^{p - 1} + Q(p) - 1 = ps.
\end{align*}
其中$Q(p) = (a - mp)^{p - 1} - a^{p - 1}$. 注意到$Q(p)$的每一项$p$的次数都至少为$1$,故$p \mid Q(p)$. 进而
\begin{align*}
a^{p - 1} - 1 = ps - Q(p) \in p\mathbb{Z}.
\end{align*}
因此$a^{p - 1} \equiv 1 \pmod{p}$. 
\end{note}
\begin{proof}
根据$(\mathbb{Z}_p, \cdot)$中乘法的良定义性和$p \nmid a$,我们不失一般性,假设
\begin{align*}
1 \leqslant a \leqslant p - 1.
\end{align*}
从而$\overline{a} \in \mathbb{Z}_p^\times$(实际上,由$p \nmid a$就直接可以得到$\overline{a} \in \mathbb{Z}_p^\times$)。根据\reflem{lemma:有限群关于元素阶次方的引理},可得
\begin{align*}
\overline{a^{p-1}}=\overline{a}^{p-1}=\overline{a}^{|\mathbb{Z}_p^\times|} = \overline{1}.
\end{align*}
此即
\begin{align*}
a^{p - 1} \equiv 1 \pmod{p}.
\end{align*}
同时左乘后的结论是显然的。综上所述,我们用群论证明了费马小定理。 
\end{proof}

\begin{theorem}[Euler定理]\label{theorem:Euler定理}
令$n \in \mathbb{N}_2$,而$\gcd(a, n) = 1$,则
\begin{align*}
a^{\phi(n)} \equiv 1 \pmod{n}.
\end{align*}
\end{theorem}
\begin{remark}
注意,当$n = p$的时候,欧拉定理就退化为\hyperref[theorem:Fermat小定理]{费马小定理}。 
\end{remark}
\begin{note}
这个定理叫欧拉定理,这也在一定程度上解释了为什么$\phi$函数被称为欧拉函数。欧拉定理显然是费马小定理的推广(当$p$为素数时,就有$\phi (p)=p-1.$)。通过群论来证明的思路是一致的。
\end{note}
\begin{remark}
这里不妨设$1 \leqslant a \leqslant n - 1, \gcd(a, n) = 1 $的原因与\hyperref[theorem:Fermat小定理]{费马小定理的证明}类似.
\end{remark}
\begin{proof}
首先,根据$(\mathbb{Z}_n, \cdot)$中乘法的良定义性和$\gcd(a, n) = 1$,我们不失一般性,假设
\begin{align*}
1 \leqslant a \leqslant n - 1, \gcd(a, n) = 1 .
\end{align*}
从而$\overline{a} \in \mathbb{Z}_n^\times$(实际上,由$n \nmid a$就直接可以得到$\overline{a} \in \mathbb{Z}_n^\times$)。利用\reflem{lemma:有限群关于元素阶次方的引理},可得
\begin{align*}
\overline{a^{\phi(n)}}=\overline{a}^{\phi(n)}=\overline{a}^{|\mathbb{Z}_n^\times|} = \overline{1} .
\end{align*}
此即
\begin{align*}
a^{\phi(n)} \equiv 1 \pmod{n} .
\end{align*}
这就证明了欧拉定理。
\end{proof}

\begin{theorem}[Wilson定理]\label{theorem:Wilson}
若$p$是一个奇素数(即除了$2$以外的素数),则
\begin{align*}
(p - 1)! \equiv -1 \pmod{p} .
\end{align*}
其中$!$表示阶乘。
\end{theorem}
\begin{proof}
我们令$p$是一个奇素数,故$\mathbb{Z}_p^\times$包含$p - 1$(偶数)个元素。
我们希望将逆元进行配对。注意到每一个元素都对应了一个逆元。而元素和逆元相等当且仅当这个元素的平方是单位元,即
\begin{align*}
\overline{a} = \overline{a}^{-1} \iff \overline{a}^2=\overline{1} \iff a^2 \equiv 1 \pmod{p} .
\end{align*}
而这就是
\begin{align*}
p \mid (a^2 - 1) = (a - 1)(a + 1) .
\end{align*}
所以要么$p \mid (a - 1)$,要么$p \mid (a + 1)$。即$a \equiv \pm 1 \pmod{p}$.这就等价于$a \equiv 1\text{或}p-1 \pmod{p} .$
这就说明了所有逆元是自己的元素恰好是$\overline{1}$和$\overline{p-1}$这两个。我们去掉这两个元素,剩下$p - 3$(偶数)个元素一定是两两配对的。因此剩下所有元素的乘积是$1$。
因此
\begin{align*}
\overline{(p-1)!}=\overline{1}\cdot \overline{(p-1)}\cdot \underset{\frac{p-3}{2}\text{个}}{\underbrace{\overline{1}\cdots \overline{1}}}=\overline{p-1}=\overline{-1}.
\end{align*}
即$(p-1)!\equiv -1 \pmod{p}.$
这就证明了威尔逊定理。 
\end{proof}










\end{document}
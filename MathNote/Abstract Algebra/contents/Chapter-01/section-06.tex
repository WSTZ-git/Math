\documentclass[../../main.tex]{subfiles}
\graphicspath{{\subfix{../../image/}}} % 指定图片目录,后续可以直接使用图片文件名。

% 例如:
% \begin{figure}[h]
% \centering
% \includegraphics{image-01.01}
% \caption{图片标题}
% \label{fig:image-01.01}
% \end{figure}
% 注意:上述\label{}一定要放在\caption{}之后,否则引用图片序号会只会显示??.

\begin{document}

\section{群论与数论}

\begin{definition}[整除]
令 $n\in \mathbb{Z}\setminus\{0\}$,而 $m\in \mathbb{Z}$。我们说 $n$ 整除 $m$,记作 $n\mid m$,若
\begin{align*}
m\in n\mathbb{Z}=\{kn:k\in\mathbb{Z}\}
\end{align*} 
\end{definition}

\begin{proposition}\label{proposition:nZ是Z的正规子群}
若 $n\in\mathbb{Z}$,则 $n\mathbb{Z}\lhd\mathbb{Z}$。
\end{proposition}
\begin{remark}
这里的加法和乘法都是通常意义下的整数加法和整数乘法.
\end{remark}
\begin{proof}
令 $f:\mathbb{Z}\to \mathbb{Z}$,对 $m\in\mathbb{Z}$,定义为
\begin{align*}
f(m)=mn.
\end{align*}
则对$\forall m_1,m_2\in (\mathbb{Z},+)$,都有 
\begin{align*}
f\left( m_1+m_2 \right) =\left( m_1+m_2 \right) n=m_1n+m_2n=f\left( m_1 \right) +f\left( m_2 \right) .
\end{align*}
故$f$是$(\mathbb{Z},+)$到$(\mathbb{Z},+)$的群同态。因此由\refproposition{proposition:群同态的核是定义域的子群,像是陪域的子群}可知$n\mathbb{Z}=\operatorname{im}(f)<\mathbb{Z}$。又因为 $\mathbb{Z}$ 是阿贝尔群,因此由\refproposition{proposition:阿贝尔群的子群与正规子群等价}可知$n\mathbb{Z}\lhd\mathbb{Z}$.
\end{proof}

\begin{proposition}
若 $(A, +)<(\mathbb{Z}, +)$,则存在 $n\in\mathbb{N}_0$,使得 $A = n\mathbb{Z}$。
\end{proposition}
\begin{proof}
(i) 若 $A = \{0\}$,则 $A = 0\mathbb{Z}$。

(ii) 若 $A\neq\{0\}$,则由 $(A, +)<(\mathbb{Z}, +)$ 可知,$A$ 在加法逆元下封闭。
从而 $A\cap\mathbb{N}_1\neq\varnothing$,否则 $A\subset\mathbb{Z}-\mathbb{N}_1$ 且 $A\neq\{0\}$,于是任取 $x\in A\subset\mathbb{Z}-\mathbb{N}_1$ 且 $x\neq 0$,则其加法逆元 $-x\in A$,但 $-x\in\mathbb{N}_1$,这与 $A\subset\mathbb{Z}-\mathbb{N}_1$ 矛盾!

令 $n = \min(A\cap\mathbb{N}_1)$($n$ 的良定义是因为良序公理),则 $n\in A$。我们断言 $A = n\mathbb{Z}$。

注意到 $n\mathbb{Z}=\{nm:m\in\mathbb{Z}\}=\langle n\rangle$,故我们只需证 $A = \langle n\rangle$。

任取 $m\in\mathbb{Z}$,则由 $n\in A$ 及 $A$ 在加法下封闭可知,$nm=\underset{m\text{个}}{\underbrace{n + n+\cdots + n}}\in A$。故 $\langle n\rangle\subset A$。

任取 $a\in A$,假设 $a\notin n\mathbb{Z}$,则由带余除法可知,存在 $q,r\in\mathbb{Z}$,使得 $a = qn + r$,其中 $0\leqslant r\leqslant n - 1$。因为 $a\notin n\mathbb{Z}$,所以 $r\neq 0$。又 $qn\in\langle n\rangle\subset A$,$a\in A$。故由 $A$ 对加法和加法逆元封闭可知,$r = a - qn\in A$。而 $1\leqslant r\leqslant n - 1 < n$,这与 $n = \min(A\cap\mathbb{N}_1)$ 矛盾!故 $a\in n\mathbb{Z}$。 
\end{proof}

\begin{corollary}
任意的无限循环群 $\langle x\rangle$ $(|x|=\infty)$的子群都是形如 $\langle x^n\rangle=\{x^{nm}:m\in\mathbb{Z}\}$ 的形式,进而都是正规子群。
 
即对任意的无限循环群 $\langle x\rangle$ $(|x|=\infty)$,任取$A<\langle x\rangle$,则一定存在$n\in \mathbb{Z}$,使得$A=\langle x^n\rangle$,并且$A\lhd \langle x\rangle$.
\end{corollary}
\begin{proof}
由\refproposition{proposition:所有的无限循环群是彼此同构的}可知,任意无限循环群$\langle x\rangle(|x|=\infty)$都同构于整数加群$(\mathbb{Z},+)$.
\end{proof}

\begin{definition}[同余(模$n$)]
令 $n\in \mathbb{N}_1$,而 $a,b\in \mathbb{Z}$。我们说 $a$ 同余 $b$(模 $n$),记作 $a\equiv b\bmod n$,若
\begin{align*}
&a + n\mathbb{Z}=b + n\mathbb{Z}\\
&a - b\in n\mathbb{Z}
\end{align*}
\end{definition}

\begin{definition}[模 $n$ 的同余类]
令 $n\in \mathbb{N}_1$,则 $\mathbb{Z}_n$ 定义为
\begin{align*}
\mathbb{Z}_n=\mathbb{Z}/n\mathbb{Z}
\end{align*}
$\mathbb{Z}_n$ 中的每个元素,被称为一个\textbf{模 $n$ 的同余类}。
\end{definition}
\begin{note}
不难发现,$0,\cdots,n - 1$ 分别代表了 $n$ 个同余类。
\end{note}

\begin{proposition}
\[
\mathbb{Z}_n = \{k + n\mathbb{Z} : 0\leqslant k\leqslant n - 1\}
\]
其中枚举法中的这些陪集是两两不同的。
\end{proposition}
\begin{proof}
首先证明这里列完了所有的陪集。令 $m\in\mathbb{Z}$,根据带余除法,我们可以找到 $q\in\mathbb{Z}$,以及 $0\leqslant r\leqslant n - 1$,使得
\begin{align*}
m = qn + r .
\end{align*}
由于
\begin{align*}
qn\in n\mathbb{Z},
\end{align*}
因此 $m + n\mathbb{Z}=r + n\mathbb{Z}\in\{k + n\mathbb{Z} : 0\leqslant k\leqslant n - 1\}$。这就证明了最多只有这 $n$ 个同余类。

接下来证明这 $n$ 个同余类是互异的。假如 $k + n\mathbb{Z}=k' + n\mathbb{Z}$,其中 $0\leqslant k,k'\leqslant n - 1$,则 $k - k'\in n\mathbb{Z}$。但是 $-(n - 1)\leqslant k - k'\leqslant (n - 1)$。而在这个范围内唯一 $n$ 的倍数就是 $0$,于是 $k - k' = 0$,或 $k = k'$。这就证明了这 $n$ 个同余类是互异的。

综上所述,
\begin{align*}
\mathbb{Z}_n = \{k + n\mathbb{Z} : 0\leqslant k\leqslant n - 1\}.
\end{align*} 
\end{proof}










\end{document}
\documentclass[../../main.tex]{subfiles}
\graphicspath{{\subfix{../../image/}}} % 指定图片目录,后续可以直接使用图片文件名。

% 例如:
% \begin{figure}[H]
% \centering
% \includegraphics[scale=0.4]{图.png}
% \caption{}
% \label{figure:图}
% \end{figure}
% 注意:上述\label{}一定要放在\caption{}之后,否则引用图片序号会只会显示??.

\begin{document}

\section{幺半群$\,\,$群}

\begin{definition}[(幺)半群]
设 \( S \) 是非空集合. 在 \( S \) 中定义了二元运算称为乘法, 满足结合律, 即
\[
(ab)c = a(bc),\quad \forall a,b,c \in S ,
\]
则称 \( S \) 为\textbf{半群}.

如果在半群 \( M \) 中存在元素 \( 1 \), 使得
\begin{align}
1a = a1 = a,\quad \forall a \in M,\label{eq:::--1.1.2...234235}
\end{align}
则称 \( M \) 为幺半群, \( 1 \) 称为\textbf{幺元素}或\textbf{幺元}.

如果一个幺半群 \( M \)(或半群 \( S \)) 的乘法还满足交换律, 即
\[
ab = ba,\quad \forall a,b \in M \, (\text{或 } S) ,
\]
则称 \( M \)(或 \( S \)) 为\textbf{交换幺半群}(或\textbf{交换半群}), 也简单地称 \( M \)(或 \( S \)) 为\textbf{可换的}.

对于交换幺半群, 有时把二元运算记为加法, 此时幺元素记为 \( 0 \), 改称\textbf{零元素}或\textbf{零}.
\end{definition}

\begin{example}
\begin{enumerate}
\item \( \mathbf{N} \) 对乘法是幺半群, 对加法是半群而不是幺半群. 非负整数集对加法与乘法均为幺半群.

\item 令 \( M(X) \) 为非空集 \( X \) 的所有变换 (即 \( X \) 到 \( X \) 的映射) 的集合, 则对于变换的乘法, \( M(X) \) 是一个幺半群, \( \text{id}_X \) 是一个幺元素. 当 \( |X| \geq 2 \) 时, \( M(X) \) 不是可换的.

\item 设 \( P(X) \) 为非空集合 \( X \) 的所有子集的集合. 空集 \( \varnothing \) 也是 \( X \) 的一个子集, 则 \( P(X) \) 对集合的并的运算是一个幺半群, \( \varnothing \) 为幺元素. 同样, \( P(X) \) 对集合的交的运算是一个幺半群, \( X \) 为幺元素, 这两种幺半群都是可换的.
\end{enumerate}
\end{example}

\begin{proposition}
幺半群中的幺元素是唯一的.
\end{proposition}
\begin{proof}
如果 \( 1 \) 与 \( 1' \) 都是幺半群 \( M \) 的幺元素, 则由条件 \eqref{eq:::--1.1.2...234235}可知 \( 1 = 1' \). 
\end{proof}

\begin{definition}[群]
在非空集合 \( G \) 中定义了二元运算, 称为乘法. 若满足下列条件:
\begin{enumerate}[(1)]
\item 结合律成立, 即 \( (ab)c = a(bc) (\forall a,b,c \in G) \);

\item 存在\textbf{左幺元}, 即 \( \exists e \in G \), 使 \( ea = a (\forall a \in G) \);

\item 对 \( \forall a \in G \) 有\textbf{左逆元}, 即有 \( b \in G \), 使 \( ba = e \),
\end{enumerate}
则称 \( G \) 是一个\textbf{群}. 若 \( G \) 的乘法还满足交换律, 则称 \( G \) 为\textbf{交换群}或$\mathbf{Abel}$\textbf{群}.
\end{definition}

\begin{definition}[全变换群/置换群]
设 \( X \) 是非空集合. 以 \( S_X \) 表示 \( X \) 的所有可逆变换 (即 \( X \) 到 \( X \) 的一一对应) 的集合, 则 \( S_X \) 对变换的乘法构成一个群, \( \text{id}_X \) 为左幺元, \( f^{-1} \) 为 \( f \) 的左逆元. \( S_X \) 称 \( X \) 的\textbf{全变换群}.

如果集合$X$所含元素的个数\( |X| = n < +\infty \). 此时 \( S_X \) 记为 \( S_n \), 称为 \( n \) 个文字的\textbf{对称群}或 \( n \) 个文字的\textbf{置换群}, 其元素称为\textbf{置换}.
\end{definition}
\begin{remark}
假定集合\( X = \{1, 2, \cdots, n\} \),记$S_n$为$X$的对称群,设 \( \sigma \in S_n \), 则 \( \sigma(1), \sigma(2), \cdots, \sigma(n) \) 是 \( 1, 2, \cdots, n \) 的一个排列. 常用下面记法:
\[
\sigma = \begin{pmatrix} 1 & 2 & & n \\ \sigma(1) & \sigma(2) & \cdots & \sigma(n) \end{pmatrix}
\]
更一般地, 若 \( i_1, i_2, \cdots, i_n \) 是 \( 1, 2, \cdots, n \) 的一个排列, 则可记
\[
\sigma = \begin{pmatrix} i_1 & i_2 & \cdots & i_n \\ \sigma(i_1) & \sigma(i_2) & \cdots & \sigma(i_n) \end{pmatrix}
\]
易知 \( S_n \) 中有 \( n! \) 个元素, \( S_n \) 中一个元素可以有 \( n! \) 种表示法.

例如, \( \sigma \in S_3 \), 满足 \( \sigma(1) = 2, \sigma(2) = 3, \sigma(3) = 1 \), 则可记
\[
\sigma = \begin{pmatrix} 1 & 2 & 3 \\ 2 & 3 & 1 \end{pmatrix} = \begin{pmatrix} 1 & 3 & 2 \\ 2 & 1 & 3 \end{pmatrix} = \begin{pmatrix} 2 & 1 & 3 \\ 3 & 2 & 1 \end{pmatrix} = \cdots
\]
\end{remark}

\begin{definition}

\end{definition}

\begin{theorem}

\end{theorem}
\begin{proof}

\end{proof}


\end{document}
\documentclass[../../main.tex]{subfiles}% 注意这里的文件路径不能用 ./main.tex ,否则用latexmk编译子文件会报错
\graphicspath{{\subfix{./image/}}} % 指定图片目录,后续可以直接使用图片文件名
% 注意这里的文件路径不能用 ../../image/ ,否则用latexmk编译子文件会报错

% 例如:
% \begin{figure}[H]
% \centering
% \includegraphics[scale=0.3]{图.png}
% \caption{}
% \label{figure:图}
% \end{figure}
% 注意:上述\label{}一定要放在\caption{}之后,否则引用图片序号会只会显示??.

\begin{document}

\section{幺半群和群}

\begin{definition}[(幺)半群]
设 \( S \) 是非空集合. 在 \( S \) 中定义了二元运算称为乘法, 满足结合律, 即
\[
(ab)c = a(bc),\quad \forall a,b,c \in S ,
\]
则称 \( S \) 为\textbf{半群}.

如果在半群 \( M \) 中存在元素 \( 1 \), 使得
\begin{align}
1a = a1 = a,\quad \forall a \in M,\label{eq:::--1.1.2...234235}
\end{align}
则称 \( M \) 为幺半群, \( 1 \) 称为\textbf{幺元素}或\textbf{幺元}或\textbf{单位元}.

如果一个幺半群 \( M \)(或半群 \( S \)) 的乘法还满足交换律, 即
\[
ab = ba,\quad \forall a,b \in M \, (\text{或 } S) ,
\]
则称 \( M \)(或 \( S \)) 为\textbf{交换幺半群}(或\textbf{交换半群}), 也简单地称 \( M \)(或 \( S \)) 为\textbf{可换的}.

对于交换幺半群, 有时把二元运算记为加法, 此时幺元素记为 \( 0 \), 改称\textbf{零元素}或\textbf{零}.
\end{definition}

\begin{example}\label{example:抽象代数-例题1.2.2}
\begin{enumerate}[(1)]
\item \( \mathbb{N} \) 对乘法是幺半群, 对加法是半群而不是幺半群. 非负整数集对加法与乘法均为幺半群.

\item 令 \( M(X) \) 为非空集 \( X \) 的所有变换 (即 \( X \) 到 \( X \) 的映射) 的集合, 则对于变换的乘法, \( M(X) \) 是一个幺半群, \( \text{id}_X \) 是一个幺元素. 当 \( |X| \geqslant 2 \) 时, \( M(X) \) 不是可换的.

\item 设 \( P(X) \) 为非空集合 \( X \) 的所有子集的集合. 空集 \( \varnothing \) 也是 \( X \) 的一个子集, 则 \( P(X) \) 对集合的并的运算是一个幺半群, \( \varnothing \) 为幺元素. 同样, \( P(X) \) 对集合的交的运算是一个幺半群, \( X \) 为幺元素, 这两种幺半群都是可换的.
\end{enumerate}
\end{example}

\begin{proposition}
幺半群中的幺元素是唯一的.
\end{proposition}
\begin{proof}
如果 \( 1 \) 与 \( 1' \) 都是幺半群 \( M \) 的幺元素, 则由条件 \eqref{eq:::--1.1.2...234235}可知 \( 1 = 1' \). 

\end{proof}

\begin{definition}[群]
在非空集合 \( G \) 中定义了二元运算, 称为乘法. 若满足下列条件:
\begin{enumerate}[(1)]
\item 结合律成立, 即 \( (ab)c = a(bc) (\forall a,b,c \in G) \),也即$G$是半群;

\item 存在\textbf{左幺元}, 即 \( \exists e \in G \), 使 \( ea = a (\forall a \in G) \);

\item 对 \( \forall a \in G \) 有\textbf{左逆元}, 即有 \( b \in G \), 使 \( ba = e \),
\end{enumerate}
则称 $(G,\cdot)$或\( G \) 是一个\textbf{群}. 若 \( G \) 的乘法还满足交换律, 则称 \( G \) 为\textbf{交换群}或$\mathbf{Abel}$\textbf{群}.

有时将 Abel 群的运算记作加法. 这时左幺元改称\textbf{零元}, 以 $0$ 表示; $a$ 的左逆元改称 $a$ 的\textbf{负元}, 记为 $-a$.
\end{definition}
\begin{remark}
数域 $\mathbb{P}$ 对加法构成一个群, 左幺元为 $0$, $a$ 的左逆元为 $-a$. $\mathbb{P}$ 对乘法是幺半群, 不是群. 但是 $\mathbb{P}$ 中非零元素的集合 $\mathbb{P}^*$ 对乘法是群, $1$ 为左幺元, $1/a$ 为 $a$ 的左逆元.
\end{remark}

\begin{example}
令$\Omega$是任意一个集合,$G$是一个群,$G^\Omega$是$\Omega$到$G$的所有映射的集合.对任意两个映射$f,g \in G^\Omega$,定义乘积$fg$是这样的映射: 对任意$a \in \Omega$, $(fg)(a) = f(a)g(a)$.则$G^\Omega$是群,并且这个群$G$为Abel群当且仅当$G$是Abel群.
\end{example}
\begin{proof}
易知上述乘法有结合律,$1 \in G^\Omega$是其单位元,其中$1(a) = 1_G,\ \forall\ a \in \Omega$($1_G$表示群$G$的单位元);任一元$f \in G^\Omega$有逆元$f^{-1}$,其中$f^{-1}(a) := (f(a))^{-1},\ \forall\ a \in \Omega$.故$G^\Omega$成为群.

显然群$G$为Abel群当且仅当$G$是Abel群.

\end{proof}

\begin{example}
$b$ 是含幺半群 $G$ 中的元 $a$ 的逆元当且仅当成立 $aba = a$, $ab^2a = 1$.
\end{example}
\begin{proof}
设 $b$ 是 $a$ 的逆元, 显然有 $aba = a$ 和 $ab^2a = 1$. 反之, 若 $aba = a, ab^2a = 1$, 则
\begin{align*}
ab = ab(ab^2a) = ab^2a = 1, \quad ba = (ab^2a)ba = ab^2a = 1.
\end{align*}
即 $b$ 是 $a$ 的逆元.

\end{proof}

\begin{example}
\begin{enumerate}[(1)]
\item 令$G$是所有秩不大于$r$的$n\times n$复方阵的集合,则$G$在矩阵乘法下构成一个半群,并且当$r < n$时,它不是一个含幺半群;当$r = n$时,它是一个含幺半群,但不是群.

\item 整数集$\mathbb{Z}$对于乘法成为一个含幺半群,但不是群.
\end{enumerate}
\end{example}

\begin{theorem}
设$m$是大于1的正整数,记
\begin{align*}
U(m) = \{ \overline{a} \in \mathbb{Z}_m \mid (a,m) = 1 \},
\end{align*}
则$U(m)$关于剩余类的乘法构成群.群$(U(m),\cdot)$称为\textbf{$\mathbb{Z}$的模$\boldsymbol{m}$单位群},显然这是一个交换群.当$p$为素数时,$U(p)$常记作$\mathbb{Z}_p^*$.易知
\begin{align*}
\mathbb{Z}_p^* = \{ \overline{1},\overline{2},\dots,\overline{p-1} \}.
\end{align*}
\end{theorem}
\begin{remark}
由初等数论可知,$U(m)$的阶等于$\phi(m)$,这里$\phi(m)$是欧拉函数,如果
\begin{align*}
m = p_1^{r_1}p_2^{r_2}\dots p_s^{r_s},
\end{align*}
其中$p_1,p_2,\dots,p_s$为$m$的不同素因子,那么
\begin{align*}
\phi(m) = (p_1^{r_1} - p_1^{r_1 - 1})(p_2^{r_2} - p_2^{r_2 - 1})\dots (p_s^{r_s} - p_s^{r_s - 1}) = m \prod\limits_{i=1}^s \left( 1 - \frac{1}{p_i} \right).
\end{align*}
\end{remark}
\begin{proof}
对任意的$\overline{a},\overline{b} \in U(m)$,有$(a,m)=1$,$(b,m)=1$,于是$(ab,m)=1$,从而$\overline{ab} \in U(m)$.所以剩余类的乘法“$\cdot$”是$U(m)$的代数运算.

对任意的$\overline{a},\overline{b},\overline{c} \in U(m)$,
\begin{align*}
(\overline{a} \cdot \overline{b}) \cdot \overline{c} = \overline{ab} \cdot \overline{c} = \overline{(ab)c} = \overline{a(bc)} = \overline{a} \cdot \overline{bc} = \overline{a} \cdot (\overline{b} \cdot \overline{c}).
\end{align*}
所以结合律成立.

因为$(1,m)=1$,从而$\overline{1} \in \mathbb{Z}_m$,且对任意的$\overline{a} \in U(m)$,
\begin{align*}
\overline{a} \cdot \overline{1} = \overline{a \cdot 1} = \overline{a}, \\
\overline{1} \cdot \overline{a} = \overline{1 \cdot a} = \overline{a},
\end{align*}
所以$\overline{1}$为$U(m)$的单位元.

对任意的$\overline{a} \in U(m)$,有$(a,m)=1$.由整数的性质可知,存在$u,v \in \mathbb{Z}$,使
\begin{align*}
au + mv = 1.
\end{align*}
显然$(u,m)=1$,所以$\overline{u} \in U(m)$,且
\begin{gather*}
\overline{a}\cdot \overline{u}=\overline{au}=\overline{au+mv}=\overline{1},
\\
\overline{u}\cdot \overline{a}=\overline{ua}=\overline{au}=\overline{1}.
\end{gather*}
所以$\overline{u}$为$\overline{a}$的逆元.从而知,$U(m)$的每个元素在$U(m)$中都可逆.

这就证明了,$U(m)$关于剩余类的乘法构成群.

\end{proof}

\begin{theorem}[群的基本性质]\label{theorem:抽象代数--群的基本性质}
设$(G,\cdot)$是一个群,$a\in G$,1是$G$的左幺元,则
\begin{enumerate}[(1)]
\item 若 \( b \) 为 \( a \) 的左逆元, 则 \( b \) 也是 \( a \) 的\textbf{右逆元}, 即有 \( ab = 1 \), 故称 \( b \) 为 \( a \) 的\textbf{逆元}.

\item 任一元素 \( a \) 的逆元唯一, 记为 \( a^{-1} \), 并且 \( 1^{-1} = 1 \), \( (a^{-1})^{-1} = a \), \( (ab)^{-1} = b^{-1}a^{-1} \), \( (a^n)^{-1} = (a^{-1})^n \).

\item 若$a_1,a_2,\cdots,a_r\in G$,则
\begin{align*}
(a_1a_2\cdots a_r)^{-1}=a_r^{-1}a_{r-1}^{-1}\cdots a_1^{-1}.
\end{align*}

\item \( 1 \) 也是 \( G \) 的\textbf{右幺元}, 即 \( a \cdot 1 = a \ (\forall a \in G) \), 故 \( 1 \) 为 \( G \) 的\textbf{幺元}. 故 \( G \) 为幺半群, 幺元唯一.

\item\label{theorem:抽象代数--群的基本性质-4} 群运算满足\textbf{消去律}, 即
\[
ax = bx \ \text{或}\ xa = xb, \text{ 则 } a = b, \ \forall a, b, x \in G.
\]
\end{enumerate}
\end{theorem}
\begin{proof}
\begin{enumerate}[(1)]
\item 事实上, 设 \( c \) 是 \( b \) 的左逆元, 则有
\[
ab = 1 \cdot (ab) = (cb)(ab) = c(ba)b = c(1 \cdot b) = 1.
\]

\item 设 \( b_1, b_2 \) 均为 \( a \) 的逆元, 则有
\[
b_1 = b_1 \cdot 1 = b_1(ab_2) = (b_1a)b_2 = 1 \cdot b_2 = b_2.
\]
其余各式显然.

\item 只需注意到$(a_1a_2\cdots a_r)(a_r^{-1}a_{r-1}^{-1}\cdots a_1^{-1})=1$即可.

\item 设 \( b \) 为 \( a \) 的逆元, 则有
\[
a \cdot 1 = a(ba) = (ab)a = 1 \cdot a = a.
\]

\item 两边同乘$x^{-1}$即得.
\end{enumerate}

\end{proof}

\begin{theorem}\label{theorem:群的充要条件--方程有解}
设$(G,\cdot)$是半群,则$G$构成群的充分必要条件是对任意的$a,b \in G$,方程
\begin{align*}
ax = b \quad \text{与} \quad ya = b
\end{align*}
在$G$中都有解.并且当$G$为群时,上述方程的解存在且唯一,即对 \(\forall a, b \in G \),群中方程 \( ax = b \)与 \( xa = b \)的解都存在且唯一.
\end{theorem}
\begin{proof}
{\heiti 必要性:}事实上, \( x = a^{-1}b \)和\( x = ba^{-1} \)分别为两个方程的解, 由\rrefthe{theorem:抽象代数--群的基本性质}{theorem:抽象代数--群的基本性质-4}知解唯一.

{\heiti 充分性:} 任取$b \in G$,由条件知,$yb = b$有解,设为$e$,则$eb = b$.又对任意的$a \in G$,$bx = a$有解,设为$c$.于是
\begin{align*}
ea = e(bc) = (eb)c = bc = a,
\end{align*}
从而知$e$是$G$的左单位元.

其次,对每个$a \in G$,$ya = e$有解,设为$a'$.于是
\begin{align*}
a'a = e,
\end{align*}
从而知$a$有左逆元.故$G$构成群.

\end{proof}

\begin{definition}
设 \( a \) 是群 \( G \) 的元素,可定义 \( a \) 的\textbf{非正整数次乘幂}如下:
\[
a^0 = 1, \quad a^{-n} = (a^{-1})^n, \quad \forall n \in \mathbb{N}.
\]
\end{definition}

\begin{theorem}
设$G$是一个群,则对 \( \forall m, n \in \mathbb{Z}, a, b \in G \) 有
\[
a^m \cdot a^n = a^{m+n}, \quad (a^m)^n = a^{mn}, \quad 1^m = 1.
\]
又若 \( ab = ba \), 则有 \( (ab)^m = a^m b^m \).
\end{theorem}
\begin{proof}


\end{proof}

\begin{definition}
群 \( G \) 中所含元素个数 \( |G| \) 称为 \( G \) 的\textbf{阶}. 若 \( |G| \) 有限, 则称 \( G \) 为\textbf{有限群}; 若 \( |G| \) 无限, 则称 \( G \) 为\textbf{无限群}.

有限群 \( G \) 的乘法可列表给出, 此表称为 \( G \) 的\textbf{群表}. 设 \( G = \{1, a_1, a_2, \cdots, a_{n-1}\} \) 为 \( n \) 阶群, 则 \( G \) 的群表为
\begin{align*}
\begin{array}{c|ccccc}
& 1 & a_1 & a_2 & \cdots & a_{n-1} \\
\hline
1 & 1 & a_1 & a_2 & \cdots & a_{n-1} \\
a_1 & a_1 & a_1^2 & a_1a_2 &  & a_1a_{n-1} \\
a_2 & a_2 & a_2a_1 & a_2^2 &  & a_2a_{n-1} \\
\vdots & \vdots & \vdots & \vdots & \ddots & \vdots \\
a_{n-1} & a_{n-1} & a_{n-1}a_1 & a_{n-1}a_2 & \cdots & a_{n-1}^2 \\
\end{array}
\end{align*}
同样, 可定义半群与幺半群的阶, 对于有限半群与幺半群, 其运算也可列表给出.
\end{definition}

\begin{proposition}
设$G$是一个群且$|G|=n$,若$A\subseteq G$且$|A|=n$,则$A=G.$
\end{proposition}
\begin{proof}
证明是显然的.

\end{proof}

\begin{theorem}
\begin{enumerate}[(1)]
\item 设 $G$ 是一个有限半群, 则 $G$ 是群的充要条件是在$G$ 内左右消去律均成立, 即由 $ax=ay$ 或 $xa=ya$ 可推出 $x=y$.

\item 设$G$是一个具有乘法运算(对乘法封闭)的非空有限集合,则$G$是一个群的充要条件是$G$满足结合律,有左单位元,且右消去律成立.

\item 一个具有乘法运算(对乘法封闭)的非空集合$G$,则$G$是一个群的充要条件是$G$满足结合律,有右单位元(即有$e \in G$,使对任意的$a \in G$,有$ae = a$),且$G$中每个元素有右逆元(即对每个$a \in G$,有$a' \in G$,使$aa' = e$).
\end{enumerate}
\end{theorem}
\begin{proof}
\begin{enumerate}[(1)]
\item 必要性由\rrefthe{theorem:抽象代数--群的基本性质}{theorem:抽象代数--群的基本性质-4}立得,下证充分性.设 $G=\{a_1,\cdots,a_n\}$, 由消去律知
\begin{align*}
\{a_1a_i,\cdots,a_na_i\}=G=\{a_ia_1,\cdots,a_ia_n\},
\end{align*}
$\forall a_i\in G$. 故存在 $e\in G$, 使得 $a_i=ea_i$.

于是对于任一 $a_j\in G$, 有 $a_k\in G$ 使得 $a_j=a_i a_k$. 从而
\begin{align*}
ea_j=ea_i a_k=a_i a_k=a_j.
\end{align*}
这表明 $e$ 是左单位元, 又因为 $e\in G=G a_j$, 故 $a_j$ 有左逆元. 由此即知 $G$ 是群.

\item 必要性由\refthe{theorem:抽象代数--群的基本性质}立得,下证充分性.只需证$G$中每个元素有左逆即可.
设$G = \{a_1,a_2,\cdots,a_n\}$,则对任意的$a \in G$,
\begin{align*}
Ga = \{a_1a,a_2a,\cdots,a_na\} \subseteq G.
\end{align*}
当$i \neq j$时,有$a_i a \neq a_j a$.否则,由右消去律得$a_i=a_j$矛盾!从而$|Ga| = |G|$,所以$Ga = G$. 于是,对$G$中任一元素$a$及$G$的左单位元$e$,因$e \in G = Ga$,所以必存在$a_i \in G$,使$a_i a = e$. 于是$a$有左逆元$a_i$. 故由群的定义知$G$为群.

\item 必要性由\refthe{theorem:抽象代数--群的基本性质}立得,下证充分性.只需证$e$是$G$的单位元,$a \in G$的右逆元$a'$是$a$的逆元即可.
由已知,$a' \in G$,因此$a'$也有右逆元,设为$a''$,则
\begin{align*}
a' a'' = e.
\end{align*}
于是
\begin{align*}
a' a &= (a' a)e = (a' a)(a' a'') = a'(aa')a'' = (a' e)a'' = a' a'' = e,
\end{align*}
且
\begin{align*}
ea &= (aa')a = a(a' a) = ae = a.
\end{align*}
于是$e$是$G$的单位元,$a'$是$a$的逆元. 从而,由群的定义知$G$为群.
\end{enumerate}

\end{proof}

\begin{proposition}
设$G$是群. 
\begin{enumerate}[(1)]
\item 如果对任意的$x \in G$,都有$x^2 = e$,则$G$是一个Abel群.

\item $G$是Abel群的充分必要条件是对任意的$a,b \in G$, $(ab)^2 = a^2 b^2$.
\end{enumerate}
\end{proposition}
\begin{proof}
\begin{enumerate}[(1)]
\item 对任意的$x,y \in G$,有
\begin{align*}
yx = eyx = (xy)^2 yx = xyxyyx = xyexx = xyxx = xy.
\end{align*}
所以$G$是一个交换群.

\item {\heiti 必要性:} 如果$G$为交换群,则对任意的$a,b \in G$,有
\begin{align*}
(ab)^2 &= abab = aabb = a^2 b^2.
\end{align*}

{\heiti 充分性:} 如果对任意的$a,b \in G$,有$(ab)^2 = a^2 b^2$,则
\begin{align*}
ba &= (a^{-1}a)ba(bb^{-1}) = a^{-1}(abab)b^{-1} = a^{-1}(ab)^2 b^{-1} = a^{-1}a^2 b^2 b^{-1} = ab.
\end{align*}
所以$G$为交换群.
\end{enumerate}

\end{proof}

\begin{example}
设G是有限群. 证明: $G$中使$x^3 = e$的元素$x$的个数是奇数.
\end{example}
\begin{proof}
令$S = \{x \in G \mid x^3 = e\}$. 由于G是有限群,所以S为有限集. 又因为$e^3 = e$,所以$e \in S$,从而S不是空集. 如果另有$x \neq e$,使$x^3 = e$,则$(x^{-1})^3 = e$. 因为$x \neq e$,所以$x \neq x^{-1}$. 这说明S中的非单位元(如果有的话)总是成对出现,又因为$e^{-1} = e$,所以$G$中使$x^3 = e$的元素$x$的个数是奇数.

\end{proof}

\begin{example}
在偶数阶群 $G$ 中, 方程 $x^2=1$ 总有偶数个解.
\end{example}
\begin{proof}
注意到若 $g^2 \neq 1$, 则 $(g^{-1})^2 \neq 1$ 且 $g \neq g^{-1}$. 因此 $G$ 中满足 $x^2 \neq 1$ 的元 $x$ 是成对出现的. 从而 $G$ 中满足 $x^2 \neq 1$ 的元 $x$ 有偶数个. 因此在偶数阶群 $G$ 中, 方程 $x^2=1$ 总有偶数个解.

\end{proof}

\begin{example}
令 $G$ 是 $n$ 阶有限群, $a_1,a_2,\cdots,a_n$ 是群 $G$ 的任意 $n$ 个元, 不一定两两不同. 试证存在整数 $p$ 和 $q$, $1 \leqslant p \leqslant q \leqslant n$, 使得 $a_pa_{p+1}\cdots a_q = 1$.
\end{example}
\begin{proof}
令 $S = \{a_1,a_1a_2,\cdots,a_1a_2\cdots a_n\}$. 若 $1 \in S$, 则结论已得证. 若 $1 \notin S$, 因 $|G|=n$, 故 $S$ 中至少有两个元是相等的. 设 $a_1\cdots a_i = a_1\cdots a_i a_{i+1}\cdots a_j$, 则
\begin{align*}
a_{i+1}\cdots a_j = 1.
\end{align*}

\end{proof}

\begin{example}
设 $a,b$ 是群 $G$ 的两个元, 满足 $aba = ba^2b$, $a^3 = 1$, $b^{2n-1} = 1$. 试证 $b = 1$.
\end{example}
\begin{note}
注意到$b=1$等价于$a,b$可交换,故只需证明$a,b$可交换.再利用条件和数学归纳法证明即可.
\end{note}
\begin{proof}
由题设条件得
\begin{align*}
ab^2a = aba^3ba = (aba)a^2ba = (ba^2b)a^2ba = ba^2(ba^2b)a = ba^2(aba)a = b^2a^2.
\end{align*}
故 $ab^2 = b^2a$. 设 $ab^{2^r} = b^{2^r}a$, 则 $ab^{2^{r+1}} = ab^{2^r}b^2 = b^{2^r}ab^2 = b^{2^r}b^2a = b^{2^{r+1}}a$. 因此对任一正整数 $k$ 有 $ab^{2^k} = b^{2^k}a$. 特别地, 取 $k = n$ 得到 $ab^{2^n} = b^{2^n}a$. 因为 $b^{2n-1} = 1$, 所以 $b^{2n} = b$, 从而 $ab = ba$. 于是 $ba^2 = aba = ba^2b$. 故 $b = 1$.

\end{proof}

\begin{example}
设群$G$的元$a_1,a_2,b_1,b_2$满足
\begin{align*}
a_1b_1 = a_2b_2 = b_1a_1 = b_2a_2,\quad a_1^m = a_2^m = b_1^n = b_2^n = 1,
\end{align*}
其中$m$和$n$是互素的正整数.则$a_1 = a_2$, $b_1 = b_2$.
\end{example}
\begin{proof}
设$k,l$是整数使得$km + ln = 1$.由$a_1b_1 = a_2b_2$知
\begin{align*}
(a_1b_1)^{km} = (a_2b_2)^{km}.
\end{align*}
因$a_1b_1 = b_1a_1$, $a_2b_2 = b_2a_2$, $a_1^m = 1 = a_2^m$,故
\begin{align*}
(a_1b_1)^{km} = a_1^{km}b_1^{km} = b_1^{km},\quad (a_2b_2)^{km} = a_2^{km}b_2^{km} = b_2^{km}.
\end{align*}
从而有$b_1^{km} = b_2^{km}$.而$b_1^{ln} = b_2^{ln} = 1$,故$b_1 = b_1^{km+ln} = b_2^{km+ln} = b_2$.
同理可证$a_1 = a_2$.

\end{proof}



\end{document}
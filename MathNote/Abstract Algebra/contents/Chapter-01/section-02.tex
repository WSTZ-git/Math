\documentclass[../../main.tex]{subfiles}
\graphicspath{{\subfix{../../image/}}} % 指定图片目录,后续可以直接使用图片文件名。

% 例如:
% \begin{figure}[H]
% \centering
% \includegraphics[scale=0.4]{图.png}
% \caption{}
% \label{figure:图}
% \end{figure}
% 注意:上述\label{}一定要放在\caption{}之后,否则引用图片序号会只会显示??.

\begin{document}

\section{幺半群$\,\,$群}

\begin{definition}[(幺)半群]
设 \( S \) 是非空集合. 在 \( S \) 中定义了二元运算称为乘法, 满足结合律, 即
\[
(ab)c = a(bc),\quad \forall a,b,c \in S ,
\]
则称 \( S \) 为\textbf{半群}.

如果在半群 \( M \) 中存在元素 \( 1 \), 使得
\begin{align}
1a = a1 = a,\quad \forall a \in M,\label{eq:::--1.1.2...234235}
\end{align}
则称 \( M \) 为幺半群, \( 1 \) 称为\textbf{幺元素}或\textbf{幺元}.

如果一个幺半群 \( M \)(或半群 \( S \)) 的乘法还满足交换律, 即
\[
ab = ba,\quad \forall a,b \in M \, (\text{或 } S) ,
\]
则称 \( M \)(或 \( S \)) 为\textbf{交换幺半群}(或\textbf{交换半群}), 也简单地称 \( M \)(或 \( S \)) 为\textbf{可换的}.

对于交换幺半群, 有时把二元运算记为加法, 此时幺元素记为 \( 0 \), 改称\textbf{零元素}或\textbf{零}.
\end{definition}

\begin{example}\label{example:抽象代数-例题1.2.2}
\begin{enumerate}[(1)]
\item \( \mathbf{N} \) 对乘法是幺半群, 对加法是半群而不是幺半群. 非负整数集对加法与乘法均为幺半群.

\item 令 \( M(X) \) 为非空集 \( X \) 的所有变换 (即 \( X \) 到 \( X \) 的映射) 的集合, 则对于变换的乘法, \( M(X) \) 是一个幺半群, \( \text{id}_X \) 是一个幺元素. 当 \( |X| \geq 2 \) 时, \( M(X) \) 不是可换的.

\item 设 \( P(X) \) 为非空集合 \( X \) 的所有子集的集合. 空集 \( \varnothing \) 也是 \( X \) 的一个子集, 则 \( P(X) \) 对集合的并的运算是一个幺半群, \( \varnothing \) 为幺元素. 同样, \( P(X) \) 对集合的交的运算是一个幺半群, \( X \) 为幺元素, 这两种幺半群都是可换的.
\end{enumerate}
\end{example}

\begin{proposition}
幺半群中的幺元素是唯一的.
\end{proposition}
\begin{proof}
如果 \( 1 \) 与 \( 1' \) 都是幺半群 \( M \) 的幺元素, 则由条件 \eqref{eq:::--1.1.2...234235}可知 \( 1 = 1' \). 
\end{proof}

\begin{definition}[群]
在非空集合 \( G \) 中定义了二元运算, 称为乘法. 若满足下列条件:
\begin{enumerate}[(1)]
\item 结合律成立, 即 \( (ab)c = a(bc) (\forall a,b,c \in G) \);

\item 存在\textbf{左幺元}, 即 \( \exists e \in G \), 使 \( ea = a (\forall a \in G) \);

\item 对 \( \forall a \in G \) 有\textbf{左逆元}, 即有 \( b \in G \), 使 \( ba = e \),
\end{enumerate}
则称 $(G,\cdot)$或\( G \) 是一个\textbf{群}. 若 \( G \) 的乘法还满足交换律, 则称 \( G \) 为\textbf{交换群}或$\mathbf{Abel}$\textbf{群}.
\end{definition}
\begin{remark}
数域 $\mathbf{P}$ 对加法构成一个群, 左幺元为 $0$, $a$ 的左逆元为 $-a$. $\mathbf{P}$ 对乘法是幺半群, 不是群. 但是 $\mathbf{P}$ 中非零元素的集合 $\mathbf{P}^*$ 对乘法是群, $1$ 为左幺元, $1/a$ 为 $a$ 的左逆元.

有时将 Abel 群的运算记作加法. 这时左幺元改称\textbf{零元}, 以 $0$ 表示; $a$ 的左逆元改称 $a$ 的\textbf{负元}, 记为 $-a$.
\end{remark}

\begin{definition}[全变换群/置换群]
设 \( X \) 是非空集合. 以 \( S_X \) 表示 \( X \) 的所有可逆变换 (即 \( X \) 到 \( X \) 的一一对应) 的集合, 则 \( S_X \) 对变换的乘法构成一个群, \( \text{id}_X \) 为左幺元, \( f^{-1} \) 为 \( f \) 的左逆元. \( S_X \) 称 \( X \) 的\textbf{全变换群}.

如果集合$X$所含元素的个数\( |X| = n < +\infty \). 此时 \( S_X \) 记为 \( S_n \), 称为 \( n \) 个文字的\textbf{对称群}或 \( n \) 个文字的\textbf{置换群}, 其元素称为\textbf{置换}.
\end{definition}
\begin{remark}
$S_X$的子群称为$X$上的\textbf{变换群}.
\end{remark}

\begin{example}
假定集合\( X = \{1, 2, \cdots, n\} \),记$S_n$为$X$的对称群,设 \( \sigma \in S_n \), 则 \( \sigma(1), \sigma(2), \cdots, \sigma(n) \) 是 \( 1, 2, \cdots, n \) 的一个排列. 常用下面记法:
\[
\sigma = \begin{pmatrix} 1 & 2 & & n \\ \sigma(1) & \sigma(2) & \cdots & \sigma(n) \end{pmatrix}
\]
更一般地, 若 \( i_1, i_2, \cdots, i_n \) 是 \( 1, 2, \cdots, n \) 的一个排列, 则可记
\[
\sigma = \begin{pmatrix} i_1 & i_2 & \cdots & i_n \\ \sigma(i_1) & \sigma(i_2) & \cdots & \sigma(i_n) \end{pmatrix}
\]
易知 \( S_n \) 中有 \( n! \) 个元素, \( S_n \) 中一个元素可以有 \( n! \) 种表示法.

例如, \( \sigma \in S_3 \), 满足 \( \sigma(1) = 2, \sigma(2) = 3, \sigma(3) = 1 \), 则可记
\[
\sigma = \begin{pmatrix} 1 & 2 & 3 \\ 2 & 3 & 1 \end{pmatrix} = \begin{pmatrix} 1 & 3 & 2 \\ 2 & 1 & 3 \end{pmatrix} = \begin{pmatrix} 2 & 1 & 3 \\ 3 & 2 & 1 \end{pmatrix} = \cdots
\]
\end{example} 

\begin{theorem}[群的基本性质]
设$(G,\cdot)$是一个群,$a\in G$,1是$G$的左幺元,则
\begin{enumerate}
\item 若 \( b \) 为 \( a \) 的左逆元, 则 \( b \) 也是 \( a \) 的\textbf{右逆元}, 即有 \( ab = 1 \), 故称 \( b \) 为 \( a \) 的\textbf{逆元}.

\item \( 1 \) 也是 \( G \) 的\textbf{右幺元}, 即 \( a \cdot 1 = a \ (\forall a \in G) \), 故 \( 1 \) 为 \( G \) 的\textbf{幺元}. 故 \( G \) 为幺半群, 幺元唯一.

\item 任一元素 \( a \) 的逆元唯一, 记为 \( a^{-1} \), 并且 \( 1^{-1} = 1 \), \( (a^{-1})^{-1} = a \), \( (ab)^{-1} = b^{-1}a^{-1} \), \( (a^n)^{-1} = (a^{-1})^n \).

\item 群运算满足\textbf{消去律}, 即
\[
ax = bx \ (\text{或 } xa = xb), \text{ 则 } a = b, \ \forall a, b, x \in G.
\]

\item 若 \( a, b \in G \), 则群中方程 \( ax = b \)(或 \( xa = b \)) 的解存在且唯一.
\end{enumerate}
\end{theorem}
\begin{proof}
\begin{enumerate}
\item 事实上, 设 \( c \) 是 \( b \) 的左逆元, 则有
\[
ab = 1 \cdot (ab) = (cb)(ab) = c(ba)b = c(1 \cdot b) = 1.
\]

\item 设 \( b \) 为 \( a \) 的逆元, 则有
\[
a \cdot 1 = a(ba) = (ab)a = 1 \cdot a = a.
\]

\item 设 \( b_1, b_2 \) 均为 \( a \) 的逆元, 则有
\[
b_1 = b_1 \cdot 1 = b_1(ab_2) = (b_1a)b_2 = 1 \cdot b_2 = b_2.
\]
其余各式显然.

\item 两边同乘$x^{-1}$即得.

\item 事实上, \( x = a^{-1}b \)(或 \( x = ba^{-1} \)) 为解, 由性质4知解唯一.
\end{enumerate}
\end{proof}

\begin{definition}
群 \( G \) 中所含元素个数 \( |G| \) 称为 \( G \) 的\textbf{阶}. 若 \( |G| \) 有限, 则称 \( G \) 为\textbf{有限群}; 若 \( |G| \) 无限, 则称 \( G \) 为\textbf{无限群}.
\end{definition}
\begin{remark}
有限群 \( G \) 的乘法可列表给出, 此表称为 \( G \) 的群表. 设 \( G = \{1, a_1, a_2, \cdots, a_{n-1}\} \) 为 \( n \) 阶群, 则 \( G \) 的群表为
\begin{align*}
\begin{array}{c|ccccc}
& 1 & a_1 & a_2 & \cdots & a_{n-1} \\
\hline
1 & 1 & a_1 & a_2 & \cdots & a_{n-1} \\
a_1 & a_1 & a_1^2 & a_1a_2 &  & a_1a_{n-1} \\
a_2 & a_2 & a_2a_1 & a_2^2 &  & a_2a_{n-1} \\
\vdots & \vdots & \vdots & \vdots & \ddots & \vdots \\
a_{n-1} & a_{n-1} & a_{n-1}a_1 & a_{n-1}a_2 & \cdots & a_{n-1}^2 \\
\end{array}
\end{align*}
同样, 可定义半群与幺半群的阶, 对于有限半群与幺半群, 其运算也可列表给出.
\end{remark}

\begin{definition}
设 \( a \) 是群 \( G \) 的元素. 若 \( \forall k \in \mathbf{N}, a^k \neq 1 \), 则称 \( a \) 的\textbf{阶为无穷},记作$\mathrm{ord}\,a=\infty $. 若 \( \exists k \in \mathbf{N} \), 使得 \( a^k = 1 \), 则 \( r=\min\{k|k \in \mathbf{N}, a^k = 1\} \) 称为 \( a \) 的\textbf{阶},记作$\mathrm{ord}\,a=r$.
\end{definition}

\begin{definition}
设 \( a \) 是群 \( G \) 的元素,可定义 \( a \) 的\textbf{非正整数次乘幂}如下:
\[
a^0 = 1, \quad a^{-n} = (a^{-1})^n, \quad \forall n \in \mathbf{N}.
\]
\end{definition}

\begin{theorem}
设$G$是一个群,则对 \( \forall m, n \in \mathbf{Z}, a, b \in G \) 有
\[
a^m \cdot a^n = a^{m+n}, \quad (a^m)^n = a^{mn}, \quad 1^m = 1.
\]
又若 \( ab = ba \), 则有 \( (ab)^m = a^m b^m \).
\end{theorem}
\begin{proof}

\end{proof}

\begin{theorem}[群的阶的基本性质]
设$(G,\cdot)$是一个群,$a\in G$,则
\begin{enumerate}
\item \( a \) 的阶为无穷当且仅当$\forall m,n\in \mathbb{Z}$且\( m \neq n \) 时, \( a^m \neq a^n \).

\item  设 \( a \) 的阶为 \( d \), 则
\begin{align}
a^m = a^n \iff \ m \equiv n \ (\text{mod}\ d). \label{eq:1.2.4-1341294782947}
\end{align}

\item  \( a \) 与 \( a^{-1} \) 阶相同.
\end{enumerate}
\end{theorem}
\begin{proof}
\begin{enumerate}
\item 事实上, 若 \( a \) 的阶为无穷, 而有 \( m \neq n \), 使 \( a^m = a^n \). 设 \( m > n \), 于是 \( a^m(a^n)^{-1} = 1 \), 而 \( a^m(a^n)^{-1} = a^{m-n} = 1 \), 自然 \( m - n \in \mathbf{N} \). 矛盾.

反之, \( \forall m, n \in \mathbf{Z} \) 且 \( m \neq n \),有\( a^m \neq a^n \),则\( a^{m-n} = a^m(a^n)^{-1} = 1 \), 即 \( \forall k \in \mathbf{N} \) 有 \( a^k \neq 1 \), 故 \( a \) 的阶为无穷.

\item 设 \( a \) 的阶为 \( d \), \( m, n \in \mathbf{N} \), 由带余除法知,一定能找到整数 \( t_1, t_2, r_1, r_2 \), 使 \( m = dt_1 + r_1 (0 \leqslant r_1 < d) \), \( n = dt_2 + r_2 (0 \leqslant r_2 < d) \). 于是 \( a^m = (a^d)^{t_1} a^{r_1} = a^{r_1} \), \( a^n = (a^d)^{t_2} a^{r_2} = a^{r_2} \), 因而
\[
a^m = a^n \iff \ a^{r_1} = a^{r_2} \iff  \ a^{r_1 - r_2} = a^{r_2 - r_1} = 1.
\]
又 \( |r_1 - r_2| < d \), 故上式也等价于 \( r_1 - r_2 = 0 \), 即式 \(\eqref{eq:1.2.4-1341294782947}\) 成立.

\item 由 \( (a^n)^{-1} = (a^{-1})^n \) 知 \( a^k = 1 \) 当且仅当 \( (a^{-1})^k = 1 \), 故 \( a^{-1} \) 与 \( a \) 同阶.
\end{enumerate}
\end{proof}













































































































\end{document}
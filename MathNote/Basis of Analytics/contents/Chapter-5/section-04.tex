\documentclass[../../main.tex]{subfiles}
\graphicspath{{\subfix{../../image/}}} % 指定图片目录,后续可以直接使用图片文件名。

% 例如:
% \begin{figure}[H]
% \centering
% \includegraphics[scale=0.4]{图.png}
% \caption{}
% \label{figure:图}
% \end{figure}
% 注意:上述\label{}一定要放在\caption{}之后,否则引用图片序号会只会显示??.

\begin{document}

\section{中值极限问题}

此类问题有一个固定操作,即对中值点再套一次中值定理,使得中值参数可以暴露出来,从而解出参数求极限得到证明。

\begin{example}
设$f\in C^2[0,1],f'(0)=0,f''(0)\neq 0$,证明对任何$x\in(0,1)$,存在$\xi(x)\in(0,1)$,使得
\begin{align*}
\int_0^x f(t)\mathrm{d}t = f(\xi(x))x,
\end{align*}
且
\begin{align*}
\lim_{x\to 0^+}\frac{\xi(x)}{x}=\frac{1}{\sqrt{3}}.
\end{align*} 
\end{example}
\begin{proof}
对$\forall x\in(0,1)$,由积分中值定理可知,存在$\xi(x)\in(0,1)$,使得
\begin{align*}
\int_0^x f(t) \mathrm{d}t = f(\xi(x))x.
\end{align*}
从而对$\forall x\in(0,1)$,由Taylor定理可知,存在$\theta(x)\in(0,\xi(x))$,使得
\begin{align*}
f(\xi(x)) = f(0) + f'(0)\xi(x) + \frac{1}{2}f''(\theta(x))\xi^2(x) = f(0) + \frac{f''(\theta(x))}{2}\xi^2(x).
\end{align*}
从而将$\int_0^x f(t)\mathrm{d}t = f(\xi(x))x$代入上式可得
\begin{align*}
\int_0^x f(t)\mathrm{d}t = x\left[f(0) + \frac{f''(\theta(x))}{2}\xi^2(x)\right].
\end{align*}
故$f''(\theta(x))\xi^2(x) = 2\left(\frac{\int_0^x f(t)\mathrm{d}t}{x} - f(0)\right)$。于是
\begin{align*}
\lim_{x\to 0^+}\theta(x) = 0 \Rightarrow \lim_{x\to 0^+}f''(\theta(x)) = f''(0).
\end{align*}
因此由L'Hospital法则可得
\begin{align*}
f''(0)\lim_{x\to 0^+}\frac{\xi^2(x)}{x^2} &= \lim_{x\to 0^+}\frac{f''(\theta(x))\xi^2(x)}{x^2} = \lim_{x\to 0^+}\frac{2\left(\int_0^x f(t)\mathrm{d}t - xf(0)\right)}{x^3}\\
&= \lim_{x\to 0^+}\frac{2\left(f(x) - f(0)\right)}{3x^2} = \lim_{x\to 0^+}\frac{f'(x)}{3x} = \frac{f''(0)}{3}.
\end{align*}
又$f''(0)\neq 0$,故$\lim_{x\to 0^+}\frac{\xi^2(x)}{x^2} = \frac{1}{3}$,因此$\lim_{x\to 0^+}\frac{\xi(x)}{x} = \frac{1}{\sqrt{3}}$。 
\end{proof}

\begin{example}
设$f$在$x = a$的邻域$n + p$阶可导且$p\geqslant  1$,于是有
\begin{align}
f(x) = f(a) + f'(a)(x - a) + \cdots + \frac{f^{(n - 1)}(a)}{(n - 1)!}(x - a)^{n - 1} + \frac{f^{(n)}(c)}{n!}(x - a)^n.\label{example0.90.0}
\end{align}
如果对于$j = 1,2,\cdots,p - 1$都有$f^{(n + j)}(a) = 0$,$f^{(n + p)}(a)\neq 0$,求极限$\lim_{x\to a}\frac{c - a}{x - a}$。 
\end{example}
\begin{proof}
由Taylor中值定理及条件可知,存在$\theta\in U(a)$,使得
\begin{align}
f^{(n)}(c) = f^{(n)}(a) + \frac{f^{(n + p)}(\theta)}{p!}(c - a)^p. \label{example0.9-1.1}
\end{align}
从而结合上式,再利用带Peano余项的Taylor公式可得
\begin{align*}
\lim_{x\to a^+}f^{(n + p)}(\theta) &= \lim_{x\to a^+}p!\frac{f^{(n)}(c) - f^{(n)}(a)}{(c - a)^p} = \lim_{x\to a^+}p!\frac{\frac{f^{(n + p)}(a)}{p!}(c - a)^p + o((c - a)^p)}{(c - a)^p} = f^{(n + p)}(a).
\end{align*}
于是利用\eqref{example0.90.0}\eqref{example0.9-1.1}式,再结合带Peano余项的Taylor公式可得
\begin{align*}
\lim_{x\to a^+}\left(\frac{c - a}{x - a}\right)^p &= \lim_{x\to a^+}\left[p!\cdot\frac{f^{(n)}(c) - f^{(n)}(a)}{(x - a)^pf^{(n + p)}(\theta)}\right] = \lim_{x\to a^+}\left[p!\cdot\frac{\frac{n![f(x) - \sum_{j = 0}^{n - 1}\frac{f^{(j)}(a)}{j!}(x - a)^j]}{(x - a)^n} - f^{(n)}(a)}{(x - a)^pf^{(n + p)}(\theta)}\right]\\
&= \lim_{x\to a^+}\left[n!p!\cdot\frac{f(x) - \sum_{j = 0}^n\frac{f^{(j)}(a)}{j!}(x - a)^j}{(x - a)^{n + p}f^{(n + p)}(\theta)}\right] = \frac{n!p!}{f^{(n + p)}(a)}\lim_{x\to a^+}\frac{\frac{f^{(n + p)}(a)}{(n + p)!}(x - a)^{n + p} + o[(x - a)^{n + p}]}{(x - a)^{n + p}} 
\\
&= \frac{n!p!}{(n + p)!}.
\end{align*}
故$\lim_{x\to a^+}\frac{c - a}{x - a} = \sqrt[p]{\frac{n!p!}{(n + p)!}}$。 
\end{proof}


\end{document}
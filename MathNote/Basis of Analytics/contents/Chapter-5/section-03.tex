\documentclass[../../main.tex]{subfiles}
\graphicspath{{\subfix{../../image/}}} % 指定图片目录,后续可以直接使用图片文件名。

% 例如:
% \begin{figure}[H]
% \centering
% \includegraphics[scale=0.3]{image-01.01}
% \caption{图片标题}
% \label{figure:image-01.01}
% \end{figure}
% 注意:上述\label{}一定要放在\caption{}之后,否则引用图片序号会只会显示??.

\begin{document}

\section{函数构造类}

\subsection{单中值点问题(一阶构造类)}\label{section单中值点问题}

\begin{example}
\begin{enumerate}
\item 设\(f\in C[0,2]\cap D(0,2)\)满足
\(f(0)=f(2)=0,\lim_{x\rightarrow 1}\frac{f(x)-2}{x - 1}=5\).
则存在\(\xi\in(0,2)\)使得
\begin{align*}
f'(\xi)=\frac{2\xi - f(\xi)}{\xi}.
\end{align*}

\item 设\(f\in C[0,1]\cap D(0,1),f(0)=0\), 证明: 存在\(u\in(0,1)\), 使得
\begin{align*}
f'(u)=\frac{uf(u)}{1 - u}.
\end{align*}

\item 设\(f\in C[-1,2]\cap D(-1,2)\)且有\(f(-1)=f(2)=-\frac{1}{2},f(\frac{1}{2}) = 1\). 证明对任何实数\(\lambda\in\mathbb{R}\), 都存在\(\xi\in(-1,2)\), 使得
\begin{align*}
f'(\xi)=\lambda\left[f(\xi)-\frac{\xi}{2}\right]+\frac{1}{2}.
\end{align*}
\end{enumerate}
\end{example}
\begin{note}
我们在草稿纸上构造函数, 构造过程无需展示给别人或者卷面. 构造的本质是猜测, 所以无需严格的逻辑.
\end{note}
\begin{remark}
\begin{enumerate}
\item $\mathbf{Step}\mathbf{1}$考虑微分方程\(y'=\frac{2x - y}{x}\), 解得\(y=\frac{c}{x}+x\).

$\mathbf{Step}\mathbf{2}$分离常数\(c\)得\(c = x(y - x)\), 常数变易得构造函数\(c(x)=x(f(x)-x)\).

\item $\mathbf{Step}\mathbf{1}$考虑微分方程\(y'=\frac{xy}{1 - x}\), 解得\(y=\frac{ce^{-x}}{x - 1}\).

$\mathbf{Step}\mathbf{2}$分离常数\(c\)得\(c = e^{x}(x - 1)y\), 常数变易得构造函数\(c(x)=e^{x}(x - 1)f(x)\).

\item $\mathbf{Step}\mathbf{1}$考虑微分方程\(y'=\lambda\left[y-\frac{x}{2}\right]+\frac{1}{2}\), 解得\(y = ce^{\lambda x}+\frac{x}{2}\).

$\mathbf{Step}\mathbf{2}$分离常数\(c\)得\(c=\frac{y-\frac{x}{2}}{e^{\lambda x}}\), 常数变易得构造函数\(c(x)=\frac{f(x)-\frac{x}{2}}{e^{\lambda x}}\).
\end{enumerate}
\end{remark}
\begin{proof}
\begin{enumerate}
\item 由\(\lim_{x\rightarrow 1}\frac{f(x) - 2}{x - 1}=5\)及\(f\in C[0,2]\)可知
\begin{align*}
f(1)=\lim_{x\rightarrow 1}f(x)=5\lim_{x\rightarrow 1}(x - 1)+2=2.
\end{align*}
从而
\begin{align*}
f'(1)=\lim_{x\rightarrow 1}\frac{f(x)-f(1)}{x - 1}=\lim_{x\rightarrow 1}\frac{f(x)-2}{x - 1}=5.
\end{align*}
构造函数\(c(x)=x(f(x)-x)\), 我们求导得
\begin{align}\label{equation-11.25}
c'(x)=f(x)-2x+xf'(x).
\end{align}
注意到
\begin{align*}
c(0)=0,c(1)=1,c(2)=-4.
\end{align*}
于是由Lagrange中值定理得\(\alpha\in(0,1),\beta\in(1,2)\), 使得
\begin{align*}
c'(\alpha)=\frac{c(1)-c(0)}{1 - 0}=1,c'(\beta)=\frac{c(1)-c(2)}{1 - 2}=-5.
\end{align*}
由\hyperref[theorem:导数介值定理]{导数介值定理}知存在\(\xi\in(0,\eta)\)使得\(c'(\xi)=0\). 由\eqref{equation-11.25}知
\begin{align*}
f'(\xi)=\frac{2\xi - f(\xi)}{\xi}.
\end{align*}
这就完成了证明.

\item 构造\(c(x)=e^{x}(x - 1)f(x)\), 则\(c(0)=c(1)=0\), 由罗尔中值定理, 存在\(u\in(0,1)\), 使得\(c'(u)=0\), 这恰好是
\begin{align*}
f'(u)=\frac{uf(u)}{1 - u}.
\end{align*}

\item 构造\(c(x)=\frac{f(x)-\frac{x}{2}}{e^{\lambda x}}\). 注意到
\begin{align*}
c(-1)=0,c(2)=-\frac{3}{2e^{2\lambda}},c\left(\frac{1}{2}\right)=\frac{3}{4e^{\frac{\lambda}{2}}}.
\end{align*}
由零点定理知存在\(\theta\in(\frac{1}{2},2)\), 使得\(c(\theta)=0\).再由罗尔中值定理知存在\(\xi\in(-1,\theta)\subset(-1,2)\), 使\(c'(\xi)=0\), 即
\begin{align*}
f'(\xi)=\lambda\left[f(\xi)-\frac{\xi}{2}\right]+\frac{1}{2}.
\end{align*}
\end{enumerate}
\end{proof}

\begin{example}
设\(f\in D[0,1]\)且\(f(0)>0,f(1)>0,\int_{0}^{1} f(x)\mathrm{d}x = 0\),证明存在\(\xi\in(0,1)\),使得
\begin{align*}
f'(\xi)+3f^{3}(\xi)=0.
\end{align*}
\end{example}
\begin{remark}
虽然本题直接考虑微分方程:$y'+3y^2=0$解出$y$,然后常数变易也不难得到构造函数.但是下述证明的方法旨在介绍一种新的解决这类问题的方法.
\end{remark}
\begin{note}
此类构造虽然仍然是一阶构造,但是要把部分\(f\)视为已知函数来构造,对于本题,即\(3f^{2}\)视为已知的函数. 考虑\(y'+3f^{2}y = 0\). 解得\(y = ce^{-\int_{0}^{x} 3f^{2}(t)dt}\),分离变量得构造函数\(c(x)=f(x)e^{\int_{0}^{x} 3f^{2}(t)dt}\).
\end{note}
\begin{proof}
{\color{blue}证法一:}
构造函数\(c(x)=f(x)e^{\int_{0}^{x} 3f^{2}(t)dt}\),注意到
\begin{align*}
c'(x)=e^{\int_{0}^{x} 3f^{2}(t)dt}[f'(x)+3f^{3}(x)].
\end{align*}
以及由\hyperref[theorem:积分中值定理]{积分中值定理},我们知道存在\(\theta\in(0,1)\),使得
\begin{align*}
f(\theta)=\int_{0}^{1} f(x)\mathrm{d}x = 0.
\end{align*}
注意到若\(f\)只有一个零点,则因为\(f(0)>0,f(1)>0\),我们知道\(f(x)>0,\forall x\in[0,\theta)\cup(\theta,1]\),从而\(\int_{0}^{1} f(x)\mathrm{d}x>0\),这就是一个矛盾!
于是存在\(\theta_1\neq\theta_2\),使得\(f(\theta_1)=f(\theta_2)=0\). 现在就有\(c(\theta_1)=c(\theta_2)=0\),由罗尔中值定理,存在\(\xi\in(0,1)\),使得\(c'(\xi)=0\),即
\begin{align*}
f'(\xi)+3f^{3}(\xi)=0.
\end{align*}

{\color{blue}证法二:}
构造函数\(c(x)=f(x)e^{\int_{0}^{x} 3f^{2}(t)dt}\),注意到
\begin{align*}
c'(x)=e^{\int_{0}^{x} 3f^{2}(t)dt}[f'(x)+3f^{3}(x)].
\end{align*}
以及由\hyperref[theorem:积分中值定理]{积分中值定理},我们知道存在\(\theta\in(0,1)\),使得
\begin{align*}
f(\theta)=\int_{0}^{1} f(x)\mathrm{d}x = 0.
\end{align*}
从而$c(\theta)=0$.
因为$f(0),f(1)>0$,所以$c(0),c(1)>0$.又由$c\in C[0,1]$,故$c(x)$在$[0,1]$上可取到最大、最小值.由于$c(\theta)=0<c(0),c(1)$,因此$c(x)$只能在$(0,1)$上可取到最小值,即存在\(\xi\in(0,1)\),使得$c(\xi)\leq c(x),\forall x\in[0,1]$.由费马引理可知\(c'(\xi)=0\),即
\begin{align*}
f'(\xi)+3f^{3}(\xi)=0.
\end{align*}
\end{proof}

\begin{example}
设\(f\in C^{1}[0,1]\),证明存在\(\xi\in[0,1]\),使得
\begin{align*}
f'(\xi)=\int_{0}^{1} (12x - 6)f(x)\mathrm{d}x.
\end{align*} 
\end{example}
\begin{note}
核心想法:\textbf{分部积分转移导数,但是需要控制非积分部分为零}.
\end{note}
\begin{remark}
\(\int_0^1{(12x - 6)f(x)\mathrm{d}x}=\int_0^1{(6x^2 - 6x)^\prime f(x)\mathrm{d}x}\)的原因:
我们希望利用分部积分后能够直接转移导数而没有多余部分,因此我们待定
\(\int_0^1{(12x - 6)f(x)\mathrm{d}x}=\int_0^1{(ax^2 + bx + c)^\prime f(x)\mathrm{d}x}\),即\(12x - 6=(ax^2 + bx + c)^\prime\).
分部积分得到
\begin{align*}
\int_0^1{(12x - 6)f(x)\mathrm{d}x}=\int_0^1{(ax^2 + bx + c)^\prime f(x)\mathrm{d}x}
=(ax^2 + bx + c)f(x)\big|_{0}^{1}-\int_0^1{(ax^2 + bx + c)f^\prime(x)\mathrm{d}x}.
\end{align*}
我们希望\((ax^2 + bx + c)f(x)\big|_{0}^{1}=(a + b + c)f(1)-cf(0)=0\),即希望\(x = 0,1\)恰好是\(ax^2 + bx + c\)的根,并且\(12x - 6=(ax^2 + bx + c)^\prime\).从而
\[\begin{cases}
a + b + c = 0\\
c = 0\\
2a = 12\\
b = -6\\
\end{cases}\Rightarrow \begin{cases}
a = 6\\
b = -6\\
c = 0\\
\end{cases}.\]
由此可知,满足我们期望的二次函数只有\(6x^2 - 6x\),即\(\int_0^1{(12x - 6)f(x)\mathrm{d}x}=\int_0^1{(6x^2 - 6x)^\prime f(x)\mathrm{d}x}\).
\end{remark}
\begin{proof}
\begin{align*}
\int_0^1{\left( 12x-6 \right) f\left( x \right) \mathrm{d}x}&=\int_0^1{\left( 6x^2-6x \right) \prime f\left( x \right) \mathrm{d}x}\xlongequal{\text{分部积分}}-\int_0^1{\left( 6x^2-6x \right) f\prime \left( x \right) \mathrm{d}x}
\\
&\xlongequal{\text{积分中值定理}}f\prime \left( \xi \right) \int_0^1{\left( 6x-6x^2 \right) \mathrm{d}x}=f\prime \left( \xi \right) ,\xi \in \left[ 0,1 \right] .
\end{align*}
\end{proof}

\begin{example}
\begin{enumerate}
\item 设\(f\in D^{2}[0,1]\)使得\(f(0)=f(1)\),证明存在\(\eta\in(0,1)\)使得
\begin{align*}
f''(\eta)=\frac{2f'(\eta)}{1 - \eta}.
\end{align*}

\item 设\(f\in D^{2}[0,\frac{\pi}{4}],f(0)=0,f'(0)=1,f(\frac{\pi}{4})=1\),证明存在\(\xi\in(0,\frac{\pi}{4})\),使得
\begin{align*}
f''(\xi)=2f(\xi)f'(\xi).
\end{align*}
\end{enumerate} 
\end{example}
\begin{remark}
\begin{enumerate}
\item 考虑微分方程\(y'' =\frac{2y'}{1-x}\),解得\(y' =\frac{c}{(1 - x)^2}\),常数变易得到构造函数\(c(x)=(1 - x)^2f'(x)\).

\item 虽然我们可以通过解微分方程得到构造函数,但是也不要忘记直接猜测构造函数的想法.当需要考虑的微分方程比较难解时,就只能猜测构造函数. 

考虑微分方程:$y''=2yy'$,解得$y'=y^2+c$,得到构造函数$c(x)=f'(x)-f^2(x)$.但是根据这个构造函数结合已知条件再利用中值定理无法得到结论.($f(\frac{\pi}{4})=1$用不了)因此需要构造更加具体的函数才行.

然而原问题等价于利用Rolle中值定理找一个中值点$\xi\in (0,\frac{\pi }{4})$,使得$c'(\xi)=0$.但由条件只能得到$c(0)=1$,$c(\frac{\pi}{4})$无法确定.因此我们希望还能找一个点$x_0\in(0,\frac{\pi }{4})$,使得$c(x_0)=f'(x_0)-f^2(x_0)=1$.

将这个看作一个新的中值问题,即已知设\(f\in D^{2}[0,\frac{\pi}{4}],f(0)=0,f'(0)=1,f(\frac{\pi}{4})=1\),证明:存在$x_0\in(0,\frac{\pi }{4})$,使得\[c(x_0)=f'(x_0)-f^2(x_0)=1.\]
考虑微分方程:$y'-y^2=1$,解得$\arctan y=x+C$,常数变易得到新的构造函数$g(x)=\arctan f(x)-x$.由条件可知$g(0)=g(\frac{\pi}{4})=0$,从而由Rolle中值定理可知,存在$x_0\in(0,\frac{\pi}{4})$,使得$g'(x_0)=0\Leftrightarrow f'(x_0)-f^2(x_0)=1$.从而找到了满足我们需求的中值点$x_0$,故结论得证.(具体证明见下述证明)
\end{enumerate}
\end{remark}
\begin{proof}
\begin{enumerate}
\item 令\(c(x)=(1 - x)^2f'(x)\),则\(c'(x)=2(x - 1)f'(x)+(1 - x)^2f''(x)\).由\(f(0)=f(1)\)及Rolle中值定理可得,存在\(\xi\in(0,1)\),使得\(f'(\xi)=0\).从而\(c(1)=c(\xi)=0\),再根据Rolle中值定理可得,存在\(\eta\in(0,1)\),使得
\begin{align*}
c'(\eta)=0\Leftrightarrow f''(\eta)=\frac{2f'(\eta)}{1 - \eta}.
\end{align*} 

\item 令\(c(x)=f'(x)-f^{2}(x),g(x)=\arctan f(x)-x\),则\(g'(x)=\frac{f'(x)-f^{2}(x)-1}{1 + f^{2}(x)}\).进而由条件可得
\(g(0)=g(\frac{\pi}{4}) = 0,g'(0)=0\).
于是由Rolle中值定理可知,存在\(a\in(0,\frac{\pi}{4})\),使得\(g'(a)=0\),即\(f'(a)=f^{2}(a)+1\).从而\(c(a)=1,c(0)=f'(0)-f^{2}(0)=1\),故再由Rolle中值定理可得,存在\(\xi\in(0,\frac{\pi}{4})\),使得
\begin{align*}
c(1)=0\Leftrightarrow f''(\xi)=2f(\xi)f'(\xi).
\end{align*} 
\end{enumerate}
\end{proof}

\subsection{多中值点问题}

\begin{example}
设\(f\in C[0,1]\cap D(0,1)\)且\(f(0)=0,f(1)=1\). 证明存在互不相同的\(\lambda,\mu\in(0,1)\)使得
\begin{align*}
f'(\lambda)(1 + f'(\mu))=2.
\end{align*} 
\end{example}
\begin{note}
核心想法:插入一个点$c$,将两个中值点问题转换为如何确定这单个插入点$c$的问题.
\end{note}
\begin{remark}
思路分析:
待定\(c\in(0,1)\),运用拉格朗日中值定理,我们知道存在\(\lambda\in(0,c),\mu\in(c,1)\),使得
\begin{align*}
f'(\lambda)=\frac{f(c)-f(0)}{c - 0}=\frac{f(c)}{c},f'(\mu)=\frac{f(c)-f(1)}{c - 1}=\frac{f(c)-1}{c - 1}.
\end{align*}
需要
\begin{align*}
2=f'(\lambda)(1 + f'(\mu))=\frac{f(c)}{c}\left[1+\frac{f(c)-1}{c - 1}\right],
\end{align*}
只需找到一个\(c\in(0,1)\)使得上式成立.但是直接考虑上式比较困难,我们希望能找到一个特殊的$c$从而将上式化简.因此待定$k$,我们希望$f(c)$同时满足$\frac{f(c)-1}{c-1}=k$(这里期望$f(c)$满足$\frac{f(c)}{c}=k$也可以),从而上式就转化为
\begin{align*}
&2=\frac{kc-k+1}{c}\cdot \left( k+1 \right) \Leftrightarrow \left( k^2+k-2 \right) c-\left( k^2-1 \right) =0
\\
&\Leftrightarrow \left( k-1 \right) \left[ \left( k+2 \right) c-k-1 \right] =0\Leftrightarrow k=1\text{或}k=\frac{1-2c}{c-1}.
\end{align*}

若取\(k = 1\),则我们只需找到一个\(c\in(0,1)\),使得\(\frac{f(c)-1}{c - 1}=1\),即\(f(c)=c\).此时令\(g(x)=f(x)-x\),则现在我们只需找到一个\(c\in(0,1)\),使得\(g(c)=0\)即可.但是由条件可知\(g(0)=g(1)=0\),无法用中值定理直接找出\(c\in(0,1)\),使得\(g(c)=0\).故取$k=1$不能找到满足我们的需求的$c$.

若取\(k=\frac{1 - 2c}{c - 1}\),则我们只需找到一个\(c\in(0,1)\),使得\(\frac{f(c)-1}{c - 1}=\frac{1 - 2c}{c - 1}\),即\(f(c)=2 - 2c\).此时令\(g(x)=f(x)+2x - 2\),则现在我们只需找到一个\(c\in(0,1)\),使得\(g(c)=0\)即可.由条件可知\(g(0)=-2,g(1)=1\),从而由连续函数介值定理可得,存在\(c\in(0,1)\),使得\(g(c)=0\).故取$k=\frac{1 - 2c}{c - 1}$能找到满足我们的需求的$c$,进而就确定了满足题目要求的\(\lambda\)和\(\mu\). 
\end{remark}
\begin{proof}
令\(g(x)=f(x)+2x - 2\),则由条件可知\(g(0)=-2,g(1)=1\),从而由连续函数介值定理可得,存在\(c\in(0,1)\),使得\(g(c)=0\),即\(f(c)=2 - 2c\). 
运用Lagrange中值定理,我们知道存在\(\lambda\in(0,c),\mu\in(c,1)\),使得
\begin{align*}
f'(\lambda)=\frac{f(c)-f(0)}{c - 0}=\frac{f(c)}{c},f'(\mu)=\frac{f(c)-f(1)}{c - 1}=\frac{f(c)-1}{c - 1}.
\end{align*}
再结合$f(c)=2-2c$可得
\begin{align*}
f' (\lambda )(1+f' (\mu ))=\frac{f(c)}{c}\left[ 1+\frac{f(c)-1}{c-1} \right] =2.
\end{align*}
故结论得证.
\end{proof}


\begin{example}
设\(f\in C[0,1]\cap D(0,1)\)使得\(f(0)=0,f(1)=1\),正实数满足\(\lambda_1 + \lambda_2 + \cdots + \lambda_n = 1\)。证明存在互不相同的\(x_1,x_2,\cdots,x_n\in(0,1)\),使得
\begin{align*}
\sum_{i = 1}^{n}\frac{\lambda_i}{f'(x_i)} = 1.
\end{align*} 
\end{example}
\begin{note}
核心想法:插入$n-1$个点$y_i$,将$n$个中值点问题转换为如何确定这些插入点$y_i$的问题.
\end{note}
\begin{remark}
思路分析:证明的想法就是插入\(n - 1\)个点\(0 = y_0 < y_1 < y_2 < \cdots < y_{n - 1} < y_n = 1\)之后用Lagrange定理得
\begin{align*}
f'(x_i)=\frac{f(y_i)-f(y_{i - 1})}{y_i - y_{i - 1}},x_i\in(y_{i - 1},y_i),i = 1,2,\cdots,n.
\end{align*}
于是需要满足
\begin{align*}
1=\sum_{i = 1}^{n}\frac{\lambda_i}{f'(x_i)}=\sum_{i = 1}^{n}\frac{\lambda_i(y_i - y_{i - 1})}{f(y_i)-f(y_{i - 1})}.
\end{align*}
自然期望
\begin{align}
f(y_i)-f(y_{i - 1})=\lambda_i,i = 1,2,\cdots,n.\label{11.36}
\end{align}
此时就有
\begin{align*}
\sum_{i = 1}^{n}\frac{\lambda_i}{f'(x_i)}=\sum_{i = 1}^{n}(y_i - y_{i - 1}) = 1.
\end{align*}
而为了得到\eqref{11.36},我们只需反复用介值定理即可.由条件可知$0=f(0)<\lambda_1<f(1)=1$,从而由连续函数介值定理可得,存在$y_1\in (0,1)$,使得$f(y_1)=\lambda_1$.进而$\lambda_1=f(y_1)<\lambda_1+\lambda_2<f(1)=1$,于是再由连续函数介值定理可得,存在$y_2\in (y_1,1)$,使得$f(y_2)=\lambda_1+\lambda_2$.以此类推,反复利用介值定理即可得到
\begin{align*}
f(y_i)=\lambda_1+\lambda_2+\cdots+\lambda_i,i = 1,2,\cdots,n - 1.
\end{align*}
其中\(0 = y_0 < y_1 < y_2 < \cdots < y_{n - 1} < y_n = 1\).
\end{remark}
\begin{proof}
由条件可知$0=f(0)<\lambda_1<f(1)=1$,从而由连续函数介值定理可得,存在$y_1\in (0,1)$,使得$f(y_1)=\lambda_1$.进而$\lambda_1=f(y_1)<\lambda_1+\lambda_2<f(1)=1$,于是再由连续函数介值定理可得,存在$y_2\in (y_1,1)$,使得$f(y_2)=\lambda_1+\lambda_2$.以此类推,反复利用介值定理即可得到
\begin{align*}
f(y_i)=\lambda_1+\lambda_2+\cdots+\lambda_i,i = 1,2,\cdots,n - 1.
\end{align*}
其中\(0 = y_0 < y_1 < y_2 < \cdots < y_{n - 1} < y_n = 1\).再利用Lagrange定理得
\begin{align*}
f'(x_i)=\frac{f(y_i)-f(y_{i - 1})}{y_i - y_{i - 1}},x_i\in(y_{i - 1},y_i),i = 1,2,\cdots,n.
\end{align*}
于是
\begin{align*}
\sum_{i = 1}^{n}\frac{\lambda_i}{f'(x_i)}=\sum_{i = 1}^{n}\frac{\lambda_i(y_i - y_{i - 1})}{f(y_i)-f(y_{i - 1})}\sum_{i = 1}^{n}(y_i - y_{i - 1}) = 1.
\end{align*}
故结论得证.
\end{proof}

\subsection{只能猜的类型}
来看一种很无趣的需要自己猜的类型. 此类问题的核心是两个中值参数相互制约,此时需要你自己复原中值参数.

\begin{example}
设\(f\in C[0,1]\cap D(0,1)\)且\(f(0)=0\)且\(f(x)\neq0,\forall x\in(0,1]\),证明对任何\(\alpha>0\),存在\(\xi\in(0,1)\)使得
\begin{align*}
\alpha\frac{f'(\xi)}{f(\xi)}=\frac{f'(1 - \xi)}{f(1 - \xi)}.
\end{align*} 
\end{example}
\begin{remark}
注意到
\begin{align*}
\alpha \frac{f' \left( \xi \right)}{f\left( \xi \right)}=\frac{f' \left( 1-\xi \right)}{f\left( 1-\xi \right)}\Leftrightarrow \alpha f' \left( \xi \right) f\left( 1-\xi \right) -f\left( \xi \right) f' \left( 1-\xi \right) =0.
\end{align*}
因此想到构造函数$g(x)=f^{\alpha}(x)f(1 - x)$.
\end{remark}
\begin{note}
不妨设\(f(x)>0,\forall x\in(0,1]\)的原因:如果$f(x)<0$,则$f^{\alpha}(x)$可能无意义.

由\(f\in C[0,1]\)且\(f(x)\neq0,\forall x\in(0,1]\)可知,\(f(x)\)在\((0,1]\)要么恒大于零,要么恒小于零. 否则由零点存在定理得到矛盾! 假设结论对\(f(x)>0,\forall x\in(0,1]\)成立,则当\(f(x)<0,\forall x\in(0,1]\)时,令\(F(x)=-f(x)>0,\forall x\in(0,1]\),则\(F(0)=0\). 从而由假设可知,对\(\forall \alpha>0\),存在\(\xi\in(0,1)\),使得
\begin{align*}
\alpha \frac{F'(\xi)}{F(\xi)}=\frac{F'(1 - \xi)}{F(1 - \xi)}\Leftrightarrow \alpha \frac{f'(\xi)}{f(\xi)}=\frac{f'(1 - \xi)}{f(1 - \xi)}.
\end{align*}
故不妨设成立. 
\end{note}
\begin{proof}
不妨设\(f(x)>0,\forall x\in(0,1]\). 对$\forall \alpha>0$,令\(g(x)=f^{\alpha}(x)f(1 - x)\),则\(g(0)=g(1)=0\). 从而由Rolle中值定理可知,存在\(\xi\in(0,1)\),使得
\begin{align*}
g'(\xi) = 0\Rightarrow g'(\xi)=\alpha f^{\alpha - 1}(\xi)f'(\xi)f(1 - \xi)-f^{\alpha}(\xi)f'(1 - \xi)
\Rightarrow \alpha \frac{f'(\xi)}{f(\xi)}=\frac{f'(1 - \xi)}{f(1 - \xi)}.
\end{align*}
\end{proof}











\end{document}
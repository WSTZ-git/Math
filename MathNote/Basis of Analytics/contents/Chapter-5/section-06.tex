\documentclass[../../main.tex]{subfiles}
\graphicspath{{\subfix{../../image/}}} % 指定图片目录,后续可以直接使用图片文件名。

% 例如:
% \begin{figure}[H]
% \centering
% \includegraphics[scale=0.4]{image-01.01}
% \caption{图片标题}
% \label{figure:image-01.01}
% \end{figure}
% 注意:上述\label{}一定要放在\caption{}之后,否则引用图片序号会只会显示??.

\begin{document}

\section{微分不等式问题}

\subsection{一阶/二阶构造类}

\begin{example}[$\,\,$Gronwall不等式]\label{example:Gronwall不等式}
设 $\alpha,\beta,\mu\in C[a,b]$ 且 $\beta$ 非负,若还有
\begin{align}
\mu(t)\leqslant\alpha(t)+\int_{a}^{t}\beta(s)\mu(s)ds,\forall t\in[a,b].
\label{equation---12.6}
\end{align}
证明:
\begin{align*}
\mu(t)\leqslant\alpha(t)+\int_{a}^{t}\beta(s)\alpha(s)e^{\int_{s}^{t}\beta(u)du}ds,\forall t\in[a,b].
\end{align*}
若还有 $\alpha$ 递增,我们有
\begin{align*}
\mu(t)\leqslant\alpha(t)e^{\int_{a}^{t}\beta(s)ds},\forall t\in[a,b].
\end{align*}
\end{example}
\begin{note}
解微分方程即得构造函数. 参考\hyperref[section单中值点问题]{单中值点问题}. 考虑 $F(t)=\int_{a}^{t}\beta(s)\mu(s)ds$,则
\begin{align*}
F'(t)=\beta(t)\mu(t)\leqslant\beta(t)\alpha(t)+\beta(t)F(t).
\end{align*}
于是考虑微分方程
\begin{align*}
y'=\beta(t)\alpha(t)+\beta(t)y\Rightarrow y=ce^{\int_{a}^{t}\beta(s)ds}+\int_{a}^{t}\beta(s)\alpha(s)e^{\int_{s}^{t}\beta(u)du}ds.
\end{align*}
故得到构造函数
\begin{align*}
c(t)=\frac{F(t)-\int_{a}^{t}\beta(s)\alpha(s)e^{\int_{s}^{t}\beta(u)du}ds}{e^{\int_{a}^{t}\beta(s)ds}}=F(t)e^{-\int_{a}^{t}\beta(s)ds}-\int_{a}^{t}\beta(s)\alpha(s)e^{\int_{s}^{a}\beta(u)du}ds,t\in[a,b].
\end{align*} 
\end{note}
\begin{proof}
令
\begin{align}
c(t)=F(t)e^{-\int_{a}^{t}\beta(s)ds}-\int_{a}^{t}\beta(s)\alpha(s)e^{\int_{s}^{a}\beta(u)du}ds,t\in[a,b],\label{equation---12.001}
\end{align}
这里 $F(t)=\int_{a}^{t}\beta(s)\mu(s)ds$. 由不等式\eqref{equation---12.6}知
\begin{align}
F'(t)\leqslant\alpha(t)\beta(t)+F(t)\beta(t),\forall t\in[a,b].\label{equation---12.17}
\end{align}
于是由\eqref{equation---12.001}和\eqref{equation---12.17}可知
\begin{align*}
c'(t)=[F'(t)-\alpha(t)\beta(t)-\beta(t)F(t)]e^{\int_{t}^{a}\beta(s)ds}\leqslant0,
\end{align*}
因此$c(t)$在$[a,b]$上单调递减,从而
\begin{align*}
c(t)\leqslant c(a)=0,
\end{align*}
这就得到了
\begin{align*}
F(t)e^{-\int_{a}^{t}\beta(s)ds}\leqslant\int_{a}^{t}\beta(s)\alpha(s)e^{\int_{s}^{a}\beta(u)du}ds.
\end{align*}
再用一次不等式\eqref{equation---12.6},即得
\begin{align*}
\mu(t)\leqslant\alpha(t)+F(t)\leqslant\alpha(t)+\int_{a}^{t}\beta(s)\alpha(s)e^{\int_{s}^{t}\beta(u)du}ds,\forall t\in[a,b].
\end{align*}
特别的,当 $\alpha$ 递增,对$\forall t\in [a,b]$,固定$t$,记$G(s)=\int_s^t{\beta (u)du}$,我们有不等式
\begin{align*}
\mu (t)&\leqslant \alpha (t)+\alpha (t)\int_a^t{\beta (s)e^{\int_s^t{\beta (u)du}}ds}=\alpha (t)+\alpha (t)\int_a^t{-G' \left( s \right) e^{G\left( s \right)}ds}
\\
&=\alpha (t)-\alpha (t)\int_a^t{e^{G\left( s \right)}dG\left( s \right)}=\alpha (t)+\alpha (t)\left[ e^{G\left( a \right)}-1 \right] =\alpha (t)e^{\int_a^t{\beta (s)ds}}.
\end{align*} 
\end{proof}


\begin{example}
设 $f$ 在 $[0,+\infty)$ 二阶可微且
\begin{align}
f(0),f'(0)\geqslant0,f''(x)\geqslant f(x),\forall x\geqslant0.
\label{equation12.19}
\end{align}
证明:
\begin{align}
f(x)\geqslant f(0)+f'(0)x,\forall x\geqslant0.
\label{equation12.20}
\end{align}
\end{example}
\begin{note}
通过 $f'' - f' = f - f'$ 视为一阶构造类来构造函数. (也可以尝试考虑$f''f'=ff'$,但是这样得到的构造函数处理本题可能不太方便)注意双曲三角函数和三角函数有着类似的不等式关系.
\end{note}
\begin{proof}
令 $h(x)=[f'(x)-f(x)]e^x$, 由\eqref{equation12.19}知
\begin{align*}
h' \left( x \right) =\left( f'' \left( x \right) -f' \left( x \right) +f' \left( x \right) -f\left( x \right) \right) e^x=\left( f'' \left( x \right) -f\left( x \right) \right) e^x \geqslant 0.
\end{align*}
故
\begin{align*}
h(x)\geq h(0)=f'(0)-f(0)\Rightarrow [f'(x)-f(x)]e^x\geq f'(0)-f(0)=h(0).
\end{align*}
继视为一阶构造类可得
\begin{align*}
c(x)=\frac{f(x)+\frac{1}{2}e^{-x}h(0)}{e^{x}},c'(x)=\frac{[f'(x)-f(x)]e^x-h(0)}{e^{3x}}\geqslant0.
\end{align*}
于是
\begin{align*}
\frac{f(x)+\frac{1}{2}e^{-x}h(0)}{e^{x}}\geqslant f(0)+\frac{1}{2}h(0)=\frac{f'(0)+f(0)}{2}.
\end{align*}
继续利用\eqref{equation12.19}即得
\begin{align*}
f(x)\geqslant\frac{e^{x}+e^{-x}}{2}f(0)+\frac{e^{x}-e^{-x}}{2}f'(0)\geqslant f(0)+f'(0)x,
\end{align*}
这里
\begin{align*}
\cosh x=\frac{e^{x}+e^{-x}}{2}\geqslant1,\sinh x=\frac{e^{x}-e^{-x}}{2}\geqslant x.
\end{align*}
可以分别了利用均值不等式和求导进行证明. 
\end{proof}

\begin{example}
设 $f\in C^1[0,+\infty)\cap D^2(0,+\infty)$ 且满足
\begin{align}
f''(x)-5f'(x)+6f(x)\geqslant0,f(0)=1,f'(0)=0.
\label{equation:::::12.23}
\end{align}
证明:
\begin{align}
f(x)\geqslant3e^{2x}-2e^{3x},\forall x\geqslant0.
\label{equation:::::12.24}
\end{align}
\end{example}
\begin{note}
显然如果把式\eqref{equation:::::12.23}得不等号改为等号, 则微分方程的解为 $3e^{2x}-2e^{3x}$. 现在对于不等号, 自然应该期望有不等式\eqref{equation:::::12.24}成立. 我们一阶一阶的视为一阶微分不等式来证明即可. 注意到 2,3 是微分方程的特征值根来改写命题. 本结果可以视为微分方程比较定理.
\end{note}
\begin{proof}
把不等式\eqref{equation:::::12.23}改写为
\begin{align*}
f''(x)-2f'(x)\geqslant3(f'(x)-2f(x)).
\end{align*}
考虑$g_1(x)=f'(x)-2f(x),$,则上式可化为
\begin{align*}
g_1'(x)\geqslant 3g_1(x).
\end{align*}
视为一阶构造类来构造函数,解得构造函数为$g_2(x)=\frac{g_1(x)}{e^{3x}}$.
于是有
\begin{align*}
g_2'(x)\geqslant0\Rightarrow g_2(x)\geqslant g_1(0)=-2\Rightarrow f'(x)-2f(x)\geqslant -2e^{3x}.
\end{align*}
进一步视为一阶构造类来构造函数,解得构造函数:
\begin{align*}
g_3(x)=\frac{f(x)}{e^{2x}}+2e^{x},g_3'(x)=\frac{f'(x)-2f(x)+2e^{3x}}{e^{2x}}\geqslant0,
\end{align*}
于是
\begin{align*}
g_3(x)\geqslant g_3(0)=3\Rightarrow f(x)\geqslant3e^{2x}-2e^{3x}.
\end{align*}
我们完成了证明. 
\end{proof}

\begin{example}
设 $f$ 在 $\mathbb{R}$ 上二阶可导且满足等式
\begin{align}\label{equation-----12.26}
f(x)+f''(x)=-xg(x)f'(x),g(x)\geqslant0.
\end{align}
证明 $f$ 在 $\mathbb{R}$ 上有界.
\end{example}
\begin{note}
$f + f''$ 的出现暗示我们构造 $|f(x)|^2 + |f'(x)|^2$, 这已是频繁出现的事实.因为等式右边有一个未知函数$g(x)$,所以我们考虑局部的微分方程,即只考虑等式左边,以此来得到构造函数.考虑$f+f''=0\Leftrightarrow ff'=-f''f'$,两边同时积分得到$\frac{1}{2}f^2=-\frac{1}{2}(f')^2+C$.由此得到构造函数$C(x)=|f(x)|^2+|f'(x)|^2$.
\end{note}
\begin{proof}
构造$h(x)=|f(x)|^2 + |f'(x)|^2$, 则由\eqref{equation-----12.26}知
\begin{align*}
h'(x)=2f'(x)[f(x)+f''(x)]=-2xg(x)[f'(x)]^2.
\end{align*}
于是 $h$ 在 $(-\infty,0]$ 递增,$[0,+\infty)$ 递减.
现在我们有
\begin{align*}
h(x)\leqslant h(0)\Rightarrow|f(x)|^2\leqslant h(0),
\end{align*}
即 $f$ 有界. 
\end{proof}

\subsection{双绝对值问题}

注意区分齐次微分不等式问题和双绝对值问题.

\begin{example}\label{example:双绝对值经典问题}
对某个 $D > 0$,
\begin{enumerate}
\item 设 $f\in D(\mathbb{R}),f(0)=0$, 使得
\begin{align}\label{equation-----1229}
|f'(x)|\leqslant D|f(x)|,\forall x\in\mathbb{R}.
\end{align}
证明 $f\equiv0$.

\item 设 $f\in C^\infty(\mathbb{R}),f^{(j)}(0)=0,\forall j\in\mathbb{N}_0$, 使得
\begin{align}\label{equation-----1230}
|xf'(x)|\leqslant D|f(x)|,\forall x\in\mathbb{R}.
\end{align}
证明 $f(x)=0,\forall x\geqslant0$.
\end{enumerate}
\end{example}
\begin{note}
双绝对值技巧除了正常解微分方程构造函数外, 还需要对构造函数平方进行处理. 对于第一题, 解微分方程 $y' = Dy,y' = -Dy$ 得构造函数
\begin{align*}
C_1(x)=\frac{y(x)}{e^{Dx}},C_2(x)=y(x)e^{Dx}.
\end{align*}
但我们还要手动平方一下. 第二题是类似的.
\end{note}
\begin{proof}
\begin{enumerate}
\item 构造 $C_1(x)=\frac{f^2(x)}{e^{2Dx}},C_2(x)=f^2(x)e^{2Dx}$, 我们有
\begin{align*}
C_1'(x)=\frac{2f(x)f'(x)-2Df^2(x)}{e^{2Dx}},C_2'(x)=[2f(x)f'(x)+2Df^2(x)]e^{2Dx}.
\end{align*}
由条件\eqref{equation-----1229}, 我们知道
\begin{align*}
\pm f'(x)f(x)\leqslant|f'(x)||f(x)|\leqslant D|f(x)|^2,
\end{align*}
于是 $C_1$ 递减, $C_2$ 递增, 故
\begin{align*}
\frac{f^2(x)}{e^{2Dx}}\leqslant\frac{f^2(0)}{e^{20}}=0,\forall x\geqslant0,f^2(x)e^{2Dx}\leqslant f^2(0)e^{20}=0,\forall x\leqslant0,
\end{align*}
于是就得到了 $f\equiv0$,$\forall x\in \mathbb{R}$.

\item 构造 $C(x)=\frac{f^2(x)}{x^{2D}},x>0$(因为只需证明$f(x)=0,\forall x\geqslant0$,所以我们只考虑一边), 则
\begin{align*}
C'(x)=\frac{2f(x)f'(x)x - 2Df^2(x)}{x^{2D + 1}}.
\end{align*}
由\eqref{equation-----1230}, 我们有
\begin{align*}
xf'(x)f(x)\leqslant x|f'(x)||f(x)|\leqslant D|f(x)|^2,
\end{align*}
即 $C$ 递减. 由 Taylor 公式的 Peano 余项, 我们有 $f(x)=o(x^m),\forall m\in\mathbb{N}\cap (D,+\infty)$, 于是 
\begin{align*}
C(x)\leqslant \lim_{x\rightarrow 0^+} \frac{f^2(x)}{x^{2D}}=\lim_{x\rightarrow 0^+} \frac{o\left( x^m \right)}{x^{2D}}=0,
\end{align*}
故 $f(x)=0,\forall x\geqslant0$. 
\end{enumerate}
\end{proof}

\begin{example}\label{example:齐次微分不等式}
设 $f\in D^2[0,+\infty)$ 满足 $f(0)=f'(0)=0$ 以及
\begin{align*}
|f''(x)|^2\leqslant|f(x)f'(x)|,\forall x\geqslant0.
\end{align*}
证明 $f(x)=0,\forall x\geqslant0$.
\end{example}
\begin{note}
本题的加强版本见\refpro{proposition:齐次化方法/关于导数乘积不等式问题}.
\end{note}
\begin{proof}
令$M>0$,考虑
\begin{align*}
g(x)=e^{-Mx}\left[|f(x)|^2+|f'(x)|^2\right],x\geqslant0.
\end{align*}
利用 $1 + t^2\geqslant\sqrt{t},\forall t\geqslant0$, 我们有
\begin{align}
1+\frac{|f|^2}{|f'|^2}\geqslant\sqrt{\frac{|f|}{|f'|}}\Rightarrow|f'|^2+|f|^2\geqslant|f|^{\frac{1}{2}}|f'|^{\frac{M}{2}}=|f'|\sqrt{|ff'|}.\label{equation-89574592387}
\end{align}
于是
\begin{align*}
g'(x)&=e^{-Mx}\left[2ff'+2f'f''-Mf^2-M(f')^2\right]\\
&\leqslant e^{-Mx}\left[2|ff'|+2|f'|\sqrt{|ff'|}-Mf^2-M(f')^2\right]\\
&\stackrel{\eqref{equation-89574592387}}{\leqslant}e^{-Mx}\left[2|ff'|+2|f'|^2+2|f|^2-Mf^2-M(f')^2\right]\\
&\stackrel{\text{均值不等式}}{\leqslant}e^{-Mx}\left[|f|^2+|f'|^2+2|f'|^2+2|f|^2-Mf^2-M(f')^2\right]=0.
\end{align*}
只要取充分大的$M$,就有 $g$ 递减,从而 $0\leqslant g(x)\leqslant g(0)=0$, 故 $f(x)\equiv0$. 
\end{proof}

\begin{example}
设 $f\in D^2(\mathbb{R})$ 满足 $f(0)=f'(0)=0$ 且
\begin{align*}
|f''(x)|\leqslant|f'(x)|+|f(x)|,\forall x\in\mathbb{R}.
\end{align*}
证明:
\begin{align*}
f(x)=0,\forall x\in\mathbb{R}.
\end{align*} 
\end{example}
\begin{note}
本题的加强版本见\refpro{proposition:关于导数求和不等式问题}.
\end{note}
\begin{proof}
令 $g(x)=e^{-Mx}\left[|f(x)|^2+|f'(x)|^2\right]$, 则
\begin{align*}
g'(x)&=e^{-Mx}\left[2ff'+2f'f''-Mf^2-M(f')^2\right]\\
&\leqslant e^{-Mx}\left[f^2+(f')^2+2f'\left(|f|+|f'|\right)-Mf^2-M(f')^2\right]\\
&\leqslant e^{-Mx}\left[f^2+(f')^2+2(f')^2+f^2+(f')^2-Mf^2-M(f')^2\right]\\
&=e^{-Mx}\left[(2 - M)f^2+(4 - M)(f')^2\right].
\end{align*}
取充分大的 $M$, 就有 $g'(x)\leqslant0$. 于是 $g(x)\leqslant g(0)=0,\forall x\geqslant0$.
又注意到 $g(x)=e^{-Mx}\left[|f(x)|^2+|f'(x)|^2\right]\geqslant0$, 因此 $g(x)\equiv0,\forall x\geqslant0$.
故 $f(x)=0,\forall x\geqslant0$. 
\end{proof}

\begin{example}
设 $f\in D^2(\mathbb{R})$ 满足 $f(0)=f'(0)=0$ 且
\begin{align*}
|f''(x)|\leqslant|f'(x)f(x)|,\forall x\in\mathbb{R}.
\end{align*}
证明:
\begin{align*}
f(x)=0,\forall x\geqslant 0.
\end{align*} 
\end{example}
\begin{remark}
与\refexa{example:齐次微分不等式}不同的是,本题的不等式左右两边并不齐次,如果还使用\refexa{example:齐次微分不等式}的方法,那么在放缩过程中会使得系数不含$M$的项的次数大于系数含$M$的项,从而无法直接通过控制$M$的取值,使得$g'(x)\leqslant 0$.因此本题我们需要使用另外的方法.

这里我们将本题与\refexa{example:双绝对值经典问题}类比,采用同样的方法. 因为只需证明$f(x)=0,\forall x\geqslant 0$,所以将原不等式视为(等式)函数构造类.此时需要考虑的微分方程是$f''=ff'$.我们将其中的$f$看作已知函数,考虑的微分方程转化为$y''=fy'$,则
\begin{align*}
y'' =fy' \Rightarrow \frac{y''}{y'}=f\Rightarrow \ln y' =\int_0^x{f\left( t \right) \mathrm{d}t}+C\Rightarrow y' =Ce^{\int_0^x{f\left( t \right) \mathrm{d}t}}.
\end{align*}
于是常数变易,再开平方得到构造函数$C\left( x \right) =\frac{\left[ f'\left( x \right) \right] ^2}{e^{2\int_0^x{|f\left( t \right)| \mathrm{d}t}}}.$
\end{remark}
\begin{proof}
令 $C(x)=\frac{[f'(x)]^2}{e^{2\int_0^x{|f(t)|\mathrm{d}t}}}$, 则
\begin{align*}
C'(x)=\frac{2f'(x)f''(x)-2|f(x)|[f'(x)]^2}{e^{2\int_0^x{|f(t)|\mathrm{d}t}}}.
\end{align*}
又因为
\begin{align*}
f'f''\leqslant|f'f''|\leqslant|f|(f')^2.
\end{align*}
所以 $C'(x)\leqslant0$, 故 $C(x)\leqslant C(0)=0$. 又注意到 $C(x)=\frac{[f'(x)]^2}{e^{2\int_0^x{|f(t)|\mathrm{d}t}}}\geqslant0$, 故 $C(x)\equiv0$. 于是 $f'(x)=0,\forall x\geqslant0$.
进而 $f$ 就是常值函数, 又 $f(0)=0$, 故 $f(x)=0,\forall x\geqslant0$. 
\end{proof}


\subsection{极值原理}

\begin{example}
设 $f\in C^2[0,1]$ 且 $f(0)=f(1)=0$, 若还有
\begin{align}
f''(x)-g(x)f'(x)=f(x).
\label{equation---12.3445}
\end{align}
证明:
$f(x)=0,\forall x\in[0,1]$.
\end{example}
\begin{proof}
如果 $f$ 在 $(0,1)$ 取得在 $[0,1]$ 上的正的最大值,设最大值点为 $c$ 且 $f(c)>0,f'(c)=0,c\in(0,1)$, 代入\eqref{equation---12.3445}式知 $f''(c)=f(c)>0$. 又由极值的充分条件,我们知道 $c$ 是严格极小值点,这就是一个矛盾!

同样的考虑 $f$ 在 $(0,1)$ 取得在 $[0,1]$ 上的负的最小值,设最小值点为 $c$ 且 $f(c)<0,f'(c)=0,c\in(0,1)$, 代入\eqref{equation---12.3445}式知 $f''(c)=f(c)<0$. 又由极值的充分条件,我们知道 $c$ 是严格极大值点,这就是一个矛盾!

综上,$f$在$(0,1)$上没有正的最大值,也没有负的最小值.即
\begin{align*}
0\leqslant f(x)\leqslant0.
\end{align*}
$f(x)=0,\forall x\in[0,1]$. 
\end{proof}






\end{document}
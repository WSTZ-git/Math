\documentclass[../../main.tex]{subfiles}
\graphicspath{{\subfix{../../image/}}} % 指定图片目录,后续可以直接使用图片文件名。

% 例如:
% \begin{figure}[H]
% \centering
% \includegraphics[scale=0.4]{图.png}
% \caption{}
% \label{figure:图}
% \end{figure}
% 注意:上述\label{}一定要放在\caption{}之后,否则引用图片序号会只会显示??.

\begin{document}

\section{常用积分公式}

\subsection{不定积分}

\begin{enumerate}
\item $\int\frac{\mathrm{d}x}{x^2+a^2}=\frac{1}{a}\arctan\frac{x}{a}+C\,(a>0)$.
\\
\item $\int\frac{\mathrm{d}x}{x^2-a^2}=\frac{1}{2a}\ln\left|\frac{x-a}{x+a}\right|+C\,(a>0)$.
3. $\int\frac{\mathrm{d}x}{\sqrt{a^2-x^2}}=\arcsin\frac{x}{a}+C\,(a>0)$.
\\
\item $\int\frac{\mathrm{d}x}{\sqrt{x^2\pm a^2}}=\ln\left|x+\sqrt{x^2\pm a^2}\right|+C\,(a>0)$.
\\
\item $\int\ln x\mathrm{d}x=x\ln x-x+C$.
\\
\item (1)$\int\sec x\mathrm{d}x=\ln|\sec x+\tan x|+C$;

(2)$\int\csc x\mathrm{d}x=\frac{1}{2}\ln \left| \frac{\cos x-1}{\cos x+1} \right|+C=\ln|\csc x-\cot x|+C=\ln|\tan \frac{x}{2}| + C.$
\\
\item $\int\sqrt{x^2\pm a^2}\mathrm{d}x=\frac{1}{2}\left[x\sqrt{x^2\pm a^2}\pm a^2\ln\left|x+\sqrt{x^2\pm a^2}\right|\right]+C\,(a>0)$;

$\int\sqrt{a^2-x^2}\mathrm{d}x=\frac{1}{2}\left[x\sqrt{a^2-x^2}+a^2\arcsin\frac{x}{a}\right]+C\,(a>0)$.
\\
\item $\int e^{ax}\cos bx\mathrm{d}x=\frac{e^{ax}}{a^2+b^2}(a\cos bx+b\sin bx)+C\,(ab\neq0)$;

$\int e^{ax}\sin bx\mathrm{d}x=\frac{e^{ax}}{a^2+b^2}(a\sin bx-b\cos bx)+C\,(ab\neq0)$.
\\
\item $\int x\cos nx\mathrm{d}x=\frac{1}{n^2}\cos nx+\frac{x}{n}\sin nx+C\,(n\neq0)$;

$\int x\sin nx\mathrm{d}x=\frac{1}{n^2}\sin nx-\frac{x}{n}\cos nx+C\,(n\neq0)$.
\\
\item 记 $I(m,n)=\int\cos^m x\sin^n x\mathrm{d}x$,$\forall n,m\in\mathbb{N}$,则
\begin{align*}
I(m,n)&=\frac{\cos^{m-1}x\sin^{n+1}x}{m+n}+\frac{m-1}{m+n}I(m-2,n)\quad(m\geqslant 2,n\geqslant 0);\\
&=-\frac{\cos^{m+1}x\sin^{n-1}x}{m+n}+\frac{n-1}{m+n}I(m,n-2)\quad(m\geqslant 0,n\geqslant 2).
\end{align*}
\end{enumerate}
\begin{proof}
\begin{enumerate}
\item 

\item 

\item 

\item 

\item (1)

(2)第一种:
\begin{align*}
\int \csc x \mathrm{d}x &= \int \frac{\sin x}{1 - \cos^2 x} \mathrm{d}x = \int \frac{1}{\cos^2 x - 1} \mathrm{d}(\cos x) = \frac{1}{2}
\int \frac{1}{\cos x - 1} - \frac{1}{\cos x + 1} \mathrm{d}(\cos x) \\
&= \frac{1}{2}\ln|\cos x - 1| - \frac{1}{2}\ln|\cos x + 1| + C=\frac{1}{2}\ln \left| \frac{\cos x-1}{\cos x+1} \right|+C.
\end{align*}

第二种:
\begin{align*}
\int \csc x \mathrm{d}x &= \frac{1}{2}\ln|\cos x - 1| - \frac{1}{2}\ln|\cos x + 1| + C = \ln|\tan \frac{x}{2}| + C.
\end{align*}

第三种:
\begin{align*}
\int \csc x \mathrm{d}x &= \int \frac{\csc x(\csc x - \cot x)}{\csc x - \cot x} \mathrm{d}x = \ln|\csc x - \cot x| + C\\
&=\xlongequal{\csc ^2x-\cot ^2x=1}\ln \left| \frac{1}{\csc x+\cot x} \right|+C
=-\ln \left| \csc x+\cot x \right|+C.
\end{align*}

\item 

\item 

\item 

\item 
\end{enumerate}
\end{proof}


\subsection{定积分}

\begin{enumerate}
\item 记$I_n=\int_0^{\frac{\pi}{2}}\sin^n x\mathrm{d}x=\int_0^{\frac{\pi}{2}}\cos^n x\mathrm{d}x,\forall n\in \mathbb{N}$,则
\begin{align*}
I_n=\frac{n-1}{n}I_{n-2},\quad \forall n\geqslant 2.
\end{align*}
从而
\begin{align}\label{eq:点火公式1}
I_n=
\begin{cases}
\frac{(n-1)!!}{n!!}I_0=\frac{(n-1)!!}{n!!}\cdot \frac{\pi}{2}&,n\text{为偶数}\\
\frac{(n-1)!!}{n!!}I_1=\frac{(n-1)!!}{n!!}&,n\text{为奇数}
\end{cases}.
\end{align}

\item 记$J(m,n)=\int_0^{\frac{\pi}{2}}\sin^m x\cos^n x\mathrm{d}x,\forall n,m\in \mathbb{N}$,则
\begin{align*}
J(m,n)=\frac{m-1}{m+n}J(m-2,n),\quad \forall n,m\geqslant 2.
\end{align*}
\begin{align*}
J(m,n)=\frac{n-1}{m+n}J(m,n-2),\quad \forall n,m\geqslant 2.
\end{align*}
从而
\begin{align}\label{eq:点火公式2}
J(m,n)=
\begin{cases}
\frac{(m-1)!!(n-1)!!}{(m+n)!!}&,m,n\text{不全为偶数}\\
\frac{(m-1)!!(n-1)!!}{(m+n)!!}\cdot \frac{\pi}{2}&,m,n\text{全为偶数}
\end{cases}.
\end{align}

\item 
\end{enumerate}
\begin{remark}
公式\eqref{eq:点火公式1}\eqref{eq:点火公式2}通常称为“点火公式”.
\end{remark}



















\end{document}
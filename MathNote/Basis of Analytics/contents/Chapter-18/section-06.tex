\documentclass[../../main.tex]{subfiles}
\graphicspath{{\subfix{../../image/}}} % 指定图片目录,后续可以直接使用图片文件名。

% 例如:
% \begin{figure}[H]
% \centering
% \includegraphics[scale=0.4]{图.png}
% \caption{}
% \label{figure:图}
% \end{figure}
% 注意:上述\label{}一定要放在\caption{}之后,否则引用图片序号会只会显示??.

\begin{document}

\section{三角函数相关}

\subsection{三角函数}

\begin{theorem}[三角平方差公式]\label{theorem:三角平方差}
$\sin^2 x-\sin^2 y=\sin(x-y)\sin(x+y)=\cos(y-x)\cos(y+x)=\cos^2 y-\cos^2 x.$
\end{theorem}
\begin{proof}
首先,我们有
\begin{align*}
\cos ^2x-\cos ^2y=1-\sin ^2x-\left( 1-\sin ^2y \right) =\sin ^2y-\sin ^2x.
\end{align*}
接着,我们有
\begin{align*}
\sin(x-y)\sin(x+y) &= (\sin x \cos y - \cos x \sin y)(\sin x \cos y + \cos x \sin y) \\
&= \sin^2 x \cos^2 y - \cos^2 x \sin^2 y \\
&= \sin^2 x (1 - \sin^2 y) - (1 - \sin^2 x) \sin^2 y \\
&= \sin^2 x - \sin^2 y;
\end{align*}
\begin{align*}
\cos(y-x)\cos(y+x) &= (\cos x \cos y + \sin x \sin y)(\cos x \cos y - \sin x \sin y) \\
&= \cos^2 x \cos^2 y - \sin^2 x \sin^2 y \\
&= \cos^2 x \cos^2 y - (1 - \cos^2 x)(1 - \cos^2 y) \\
&= \cos^2 x - \cos^2 y.
\end{align*}
故结论得证.
\end{proof}

\begin{theorem}\label{theorem:sinnx和cosnx以及sin^nx和cos^nx的展开式}
\begin{enumerate}
\item $$ \sin(n\theta) = \sum_{\substack{r=0 \\ 2r + 1 \leq n}} (-1)^r \binom{n}{2r + 1} \cos^{n - 2r - 1}(\theta) \sin^{2r + 1}(\theta) .$$

\item $$ \cos(n\theta) = \sum_{\substack{r=0 \\ 2r \leq n}} (-1)^r \binom{n}{2r} \cos^{n - 2r}(\theta) \sin^{2r}(\theta) .$$

\item $$\tan(n\theta) = \frac{\sum\limits_{\substack{r=0 \\ 2r + 1 \leq n}} (-1)^r \binom{n}{2r + 1} \tan^{2r + 1}(\theta)}{\sum\limits_{\substack{r=0 \\ 2r \leq n}} (-1)^r \binom{n}{2r} \tan^{2r}(\theta)} .$$

\item $$\cos^n\theta =\begin{cases}
\frac{1}{2^{n-1}}\sum_{\substack{r=0 \\ 2r<n}} \dbinom{n}{2r} \cos\left((n - 2r)\theta\right) + \frac{1}{2^n}\dbinom{n}{\frac{n}{2}}, & n\text{为偶数} \\
\frac{1}{2^{n-1}}\sum_{\substack{r=0 \\ 2r<n}} \dbinom{n}{2r} \cos\left((n - 2r)\theta\right), & n\text{为奇数} \\
\end{cases}.$$

\item $$\sin^n\theta =\begin{cases}
\frac{(-1)^{\frac{n}{2}}}{2^{n-1}}\sum_{\substack{r=0 \\ 2r<n}} (-1)^r \dbinom{n}{2r} \sin\left((n - 2r)\theta\right), & n\text{为偶数} \\
\frac{(-1)^{\lfloor \frac{n}{2} \rfloor}}{2^{n-1}}\sum_{\substack{r=0 \\ 2r<n}} (-1)^r \dbinom{n}{2r} \cos\left((n - 2r)\theta\right) + \frac{1}{2^n}\dbinom{n}{\frac{n}{2}}, & n\text{为奇数} \\
\end{cases}.$$


\end{enumerate}
\end{theorem}
\begin{note}
上述结论4表明:
$\cos ^nx$可以表示为$1,\cos x,\cdots,\cos nx$的线性组合.
\end{note}
\begin{proof}
具体证明见\href{https://brilliant.org/wiki/expansions-of-certain-trigonometric-functions/}{Expansions of sin(nx) and cos(nx)}.
% \begin{enumerate}

% \item 

% \item 

% \item 

% \item 

% \item 
% \end{enumerate}
\end{proof}

% \begin{theorem}
% (1) $\cos ^{2n+1}x$可以写成$\cos x,\cos 3x,\cdots ,\cos (2n+1)x$的线性组合 ,即$\cos ^{2n+1}x\in L(\cos x,\cos 3x,\cdots ,\cos (2n+1)x)$ ,也即$\cos ^{2n+1}x=\sum_{k=0}^{n}{a_k\cos (2k+1)x}$ ,其中$a_k\in \mathbb{R}$ ,$k=0,1,\cdots ,n$.

% (2) $\cos ^{2n}x=\frac{1}{2^{2n-1}}\sum_{k=0}^{n-1}{\mathrm{C}_{2n}^{k}\cos 2(n-k)x}+\frac{\mathrm{C}_{2n}^{n}}{2^{2n}}$.

% 综上,$\cos ^nx$可表示为$1,\cos x,\cdots,\cos nx$的线性组合.
% \end{theorem}
% \begin{proof}
% (1) 利用数学归纳法,当$n=1$时,结论显然成立.假设结论对$n-1$成立,则  
% \begin{align*}
% \cos ^{2n+1}x&=\cos ^2x\cdot \cos ^{2n-1}x=\frac{1+\cos 2x}{2}\cdot \sum_{k=0}^{n-1}{a_k\cos \left( 2k+1 \right) x}
% \\
% &=\frac{1}{2}\sum_{k=0}^{n-1}{a_k\cos \left( 2k+1 \right) x}+\frac{1}{2}\sum_{k=0}^{n-1}{a_k\cos 2x\cos \left( 2k+1 \right) x}
% \\
% &=\frac{1}{2}\sum_{k=0}^{n-1}{a_k\cos \left( 2k+1 \right) x}+\frac{1}{2}\sum_{k=0}^{n-1}{a_k\left[ \cos \left( 2k+3 \right) x+\cos \left( 2k-1 \right) x \right]}
% \\
% &=\frac{1}{2}\sum_{k=0}^{n-1}{a_k\cos \left( 2k+1 \right) x}+\frac{1}{2}\sum_{k=1}^{n-1}{a_k\left[ \cos \left( 2k+5 \right) x+\cos \left( 2k+1 \right) x \right]}+\frac{1}{2}a_0\left[ \cos 3x+\cos \left( -x \right) \right] 
% \\
% &=\frac{1}{2}\sum_{k=0}^{n-1}{a_k\cos \left( 2k+1 \right) x}+\frac{1}{2}\sum_{k=1}^{n-1}{a_k\left[ \cos \left( 2k+5 \right) x+\cos \left( 2k+1 \right) x \right]}+\frac{1}{2}a_0\left[ \cos 3x+\cos x \right] .
% \end{align*}  
% 故$\cos ^{2n+1}x\in L(\cos x,\cos 3x,\cdots ,\cos (2n+1)x)$  

% (2) 由二项式定理可得  
% $$
% (1+t^2)^{2n}=\sum_{k=0}^{2n}{\mathrm{C}_{2n}^{k}t^{2k}}
% $$  
% 令$t=e^{\mathrm{i}x}$,则  
% \begin{align*}
% (1+e^{2\mathrm{i}x})^{2n}&=\sum_{k=0}^{2n}{\mathrm{C}_{2n}^{k}e^{2\mathrm{i}kx}} \Rightarrow 2^{2n}\left( \frac{e^{-\mathrm{i}x}+e^{\mathrm{i}x}}{2e^{-\mathrm{i}x}} \right) ^{2n}=\sum_{k=0}^{2n}{\mathrm{C}_{2n}^{k}e^{2\mathrm{i}kx}} \Rightarrow 2^{2n}\left( \frac{e^{-\mathrm{i}x}+e^{\mathrm{i}x}}{2} \right) ^{2n}=e^{-2\mathrm{i}nx}\sum_{k=0}^{2n}{\mathrm{C}_{2n}^{k}e^{2\mathrm{i}kx}}\\
% &\Rightarrow 2^{2n}\cos ^{2n}x=\sum_{k=0}^{2n}{\mathrm{C}_{2n}^{k}e^{2\mathrm{i}(k-n)x}}=\sum_{k=0}^{n-1}[\mathrm{C}_{2n}^{k}e^{2\mathrm{i}(k-n)x}+\mathrm{C}_{2n}^{2n-k}e^{2\mathrm{i}((2n-k)-n)x}]+\mathrm{C}_{2n}^{n}\\
% &=\sum_{k=0}^{n-1}\mathrm{C}_{2n}^{k}(e^{2\mathrm{i}(k-n)x}+e^{2\mathrm{i}(n-k)x})+\mathrm{C}_{2n}^{n}\\
% &\Rightarrow 2^{2n}\cos ^{2n}x=2\sum_{k=0}^{n-1}\mathrm{C}_{2n}^{k}\left( \frac{e^{2\mathrm{i}(k-n)x}+e^{2\mathrm{i}(n-k)x}}{2} \right)+\mathrm{C}_{2n}^{n}=2\sum_{k=0}^{n-1}\mathrm{C}_{2n}^{k}\cos 2(n-k)x+\mathrm{C}_{2n}^{n}\\
% &\Rightarrow \cos ^{2n}x=\frac{1}{2^{2n-1}}\sum_{k=0}^{n-1}\mathrm{C}_{2n}^{k}\cos 2(n-k)x+\frac{\mathrm{C}_{2n}^{n}}{2^{2n}}
% \end{align*}
% \end{proof}







\subsection{反三角函数}

\begin{theorem}[常用反三角函数性质]\label{theorem:常用反三角函数性质}
\begin{enumerate}
\item $$\arcsin x+\arcsin y=\begin{cases}
\arcsin \left( x\sqrt{1-y^2}+y\sqrt{1-x^2} \right) &,xy<0\text{或}x^2+y^2\leqslant 1\\
\pi -\arcsin \left( x\sqrt{1-y^2}+y\sqrt{1-x^2} \right) &,x>0,y>0,x^2+y^2>1\\
-\pi -\arcsin \left( x\sqrt{1-y^2}+y\sqrt{1-x^2} \right) &,x<0,y<0,x^2+y^2>1\\
\end{cases}.$$

\item $$\arcsin x-\arcsin y=\begin{cases}
\arcsin \left( x\sqrt{1-y^2}-y\sqrt{1-x^2} \right) &,xy\geqslant 0\text{或}x^2+y^2\leqslant 1\\
\pi -\arcsin \left( x\sqrt{1-y^2}-y\sqrt{1-x^2} \right) &,x>0,y<0,x^2+y^2>1\\
-\pi -\arcsin \left( x\sqrt{1-y^2}-y\sqrt{1-x^2} \right) &,x<0,y>0,x^2+y^2>1\\
\end{cases}.$$

\item $$\arccos x+\arccos y=\begin{cases}
\arccos \left( xy-\sqrt{1-x^2}\sqrt{1-y^2} \right) &,x+y\geqslant 0\\
2\pi -\arccos \left( xy-\sqrt{1-x^2}\sqrt{1-y^2} \right) &,x+y<0\\
\end{cases}.$$

\item $$\arccos x-\arccos y=\begin{cases}
-\arccos \left( xy+\sqrt{1-x^2}\sqrt{1-y^2} \right) &,x\geqslant y\\
\arccos \left( xy+\sqrt{1-x^2}\sqrt{1-y^2} \right) &,x<y\\
\end{cases}.$$

\item $$\arctan x+\arctan y=\begin{cases}
\arctan \frac{x+y}{1-xy}&,xy<1\\
\pi +\arctan \frac{x+y}{1-xy},x>0&,xy>1\\
-\pi +\arctan \frac{x+y}{1-xy},x<0&,xy>1\\
\end{cases}.$$

\item $$\arctan x-\arctan y=\begin{cases}
\arctan \frac{x-y}{1+xy}&,xy>-1\\
\pi +\arctan \frac{x-y}{1+xy}&,x>0,xy<-1\\
-\pi +\arctan \frac{x-y}{1+xy}&,x<0,xy<-1\\
\end{cases}.$$

\item $$2\arcsin x=\begin{cases}
\arcsin \left( 2x\sqrt{1-x^2} \right) &,\left| x \right|\leqslant \frac{\sqrt{2}}{2}\\
\pi -\arcsin \left( 2x\sqrt{1-x^2} \right) &,\frac{\sqrt{2}}{2}<x\leqslant 1\\
-\pi -\arcsin \left( 2x\sqrt{1-x^2} \right) &,-1\leqslant x<-\frac{\sqrt{2}}{2}\\
\end{cases}.$$

\item $$2\arccos x=\begin{cases}
\arccos \left( 2x^2-1 \right) &,0\leqslant x\leqslant 1\\
2\pi -\arccos \left( 2x^2-1 \right) &,-1\leqslant x<0\\
\end{cases}.$$

\item $$2\arctan x=\begin{cases}
\arctan \frac{2x}{1-x^2},\left| x \right|\leqslant 1\\
\pi +\arctan \frac{2x}{1-x^2}&,\left| x \right|>1\\
-\pi +\arctan \frac{2x}{1-x^2}&,x<-1\\
\end{cases}.$$

\item $$\cos \left( n\arccos x \right) =\frac{\left( x+\sqrt{x^2-1} \right) ^n+\left( x-\sqrt{x^2-1} \right) ^n}{2}\left( n\geqslant 1 \right) .$$
\end{enumerate}
\end{theorem}
\begin{proof}

\end{proof}

\begin{proposition}\label{proposition:arctan相关等式}
$$\arctan x+\arctan \frac{1}{x}=\begin{cases}
\frac{\pi}{2},&x>0\\
-\frac{\pi}{2},&x<0\\
\end{cases}.$$
\end{proposition}
\begin{proof}
令 \( f(x)=\arctan x+\arctan\frac{1}{x} \),则
\begin{align*}
f'(x)&=\frac{1}{x^2 + 1}+\frac{1}{(\frac{1}{x})^2 + 1}(-\frac{1}{x^2})=\frac{1}{x^2 + 1}-\frac{1}{x^2 + 1}=0
\end{align*}
故 \( f(x) \) 为常函数,于是就有 \( f(x)=f(1)=\frac{\pi}{2},\forall x>0 \) ;\( f(x)=f(-1)=-\frac{\pi}{2},\forall x<0 \).
\end{proof}


\subsection{双曲三角函数}

\begin{proposition}
\begin{enumerate}[(1)]
\item $\cosh x=\frac{e^{x}+e^{-x}}{2}\geqslant1,$

\item $\sinh x=\frac{e^{x}-e^{-x}}{2}\geqslant x.$
\end{enumerate}
\end{proposition}
\begin{proof}
可以分别利用均值不等式和求导进行证明. 
\end{proof}

\begin{proposition}
\begin{enumerate}
\item $\cosh^2 x-\sinh^2 x=1$.

\item $\cosh(2x)=2\cosh^2 x-1=1-2\sinh^2 x$.

\item $\sinh(2x)=2\sinh x\cosh x$. 
\end{enumerate}
\end{proposition}
\begin{proof}

\end{proof}




\end{document}
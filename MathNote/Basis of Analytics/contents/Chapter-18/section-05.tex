\documentclass[../../main.tex]{subfiles}
\graphicspath{{\subfix{../../image/}}} % 指定图片目录,后续可以直接使用图片文件名。

% 例如:
% \begin{figure}[H]
% \centering
% \includegraphics[scale=0.4]{图.png}
% \caption{}
% \label{figure:图}
% \end{figure}
% 注意:上述\label{}一定要放在\caption{}之后,否则引用图片序号会只会显示??.

\begin{document}

\section{基本组合学公式}

\begin{definition}[组合数定义的扩充]\label{definition:组合数定义的扩充}
对 \( \forall n \in \mathbb{R} \),\( k \in \mathbb{N} \),定义
\[
\mathrm{C}_{n}^{k} = \begin{pmatrix} n \\ k \end{pmatrix} \triangleq \begin{cases}
0&,k<0,\\
\frac{n(n - 1) \cdots (n - k + 1)}{k!}&,0< k\leqslant n ,\\
0&,k>n.\\
\end{cases}
\]
特别地,\( \mathrm{C}_{n}^{0} \triangleq 1 \)。若 \( n, k \in \mathbb{N} \)且$0\leqslant k\leqslant n$,则还有
\[
\mathrm{C}_{n}^{k} = \begin{pmatrix} n \\ k \end{pmatrix} = \frac{n!}{k! (n - k)!}.
\]
\end{definition}

\begin{theorem}[二项式定理的推广]\label{theorem:二项式定理的推广}
$\left( a_1+b_1 \right)\cdots \left( a_n+b_n \right) =\sum_{I\subset \left\{ 1,2,\cdots ,n \right\}}{\left( \prod_{i\in I}{a_i}\prod_{j\in \left\{ 1,2,\cdots ,n \right\} -I}{b_j} \right)}.$
\end{theorem}
\begin{proof}
用数学归纳法证明即可.
\end{proof}




\end{document}
\documentclass[../../main.tex]{subfiles}
\graphicspath{{\subfix{../../image/}}} % 指定图片目录,后续可以直接使用图片文件名。

% 例如:
% \begin{figure}[H]
% \centering
% \includegraphics[scale=0.4]{图.png}
% \caption{}
% \label{figure:图}
% \end{figure}
% 注意:上述\label{}一定要放在\caption{}之后,否则引用图片序号会只会显示??.

\begin{document}

\section{实数基本定理}

\subsection{定理介绍}

\begin{theorem}[实数基本定理]\label{theorem:实数基本定理}
\begin{enumerate}
\item 确界存在定理:有上界的非空数集一定有上确界.
\item 单调有界原理:单调有界数列一定收敛.
\item 柯西收敛准则:数列\(\{x_n\}\)收敛当且仅当任意\(\varepsilon > 0\),存在\(N\)使得任意\(m,n > N\)都有\(\vert x_m - x_n\vert < \varepsilon\).
\item 闭区间套定理:闭区间套\(I_n = [a_n,b_n]\)满足\(I_{n + 1} \subset I_n\)并且\(\lim\limits_{n \to \infty} (a_n - b_n) = 0\),则存在唯一的\(\xi\),使得\(\xi\)属于每一个\(I_n\).
\item 聚点定理:有界数列必有收敛子列.
\item 有限覆盖定理:有界闭集的任意一族开覆盖,都存在有限子覆盖.
\end{enumerate}
\end{theorem}

\begin{definition}[点集相关概念]\label{definition:点集相关概念}
\begin{enumerate}
\item 如果存在\(r > 0\)使得\((a - r,a + r) \subset A\),则称\(a\)是集合\(A\)的内点(高维改为开球即可).
\item 如果一个集合\(A\)中的每一个点都是内点,则称\(A\)是开集.
\item 如果集合\(A\)中的任意一个收敛序列\(x_n\)的极限点\(x\),都有\(x\in A\),则称\(A\)是闭集.
\item 设\(B\subset A\),如果对任意\(r > 0\)和任意\(x\in A\),都有\((x - r,x + r)\cap B\neq\varnothing\),则称\(B\)在\(A\)中稠密.
\end{enumerate}
\end{definition}

\subsection{综合应用}

\begin{example}
设\(f(x):[0,1]\to[0,1]\)单调递增且\(f(0)>0,f(1)<1\),证明:存在\(x\)使得\(f(x)=x\).
\end{example}
\begin{note}
因为题目条件中的函数$f$只是一个实值函数,并没有其他更进一步的性质(连续性、可微性、凸性等).所以我们只能利用最基本的实数基本定理证明.证明存在性,考虑反证法会更加简便.
\end{note}
\begin{remark}
$f$并不是连续函数,不能用介值定理.
\end{remark}
\begin{proof}
(反证法)假设对$\forall x\in[0,1]$,都有$f(x)\ne x$.将闭区间\([0,1]\)记作\([a_1,b_1]\),且由条件可知\(f(a_1)>a_1\),\(f(b_1)<b_1\).令\(c_1=\frac{a_1 + b_1}{2}\),若\(f(c_1)>c_1\),则取\([a_2,b_2]=[c_1,b_1]\);若\(f(c_1)<c_1\),则取\([a_2,b_2]=[a_1,c_1]\).从而得到闭区间\([a_2,b_2]\subset [a_1,b_1]\),并且\(f(a_2)>a_2\),\(f(b_2)<b_2\).以此类推,可得到一列闭区间\(\{[a_n,b_n]\}\),并且\([a_n,b_n]\subset [a_{n + 1},b_{n + 1}]\),\(f(a_n)>a_n\),\(f(b_n)<b_n\),\(\forall n\in\mathbb{N}_+\).

根据闭区间套定理,可知存在唯一\(\xi=\lim_{n\rightarrow\infty}a_n=\lim_{n\rightarrow\infty}b_n\),且\(\xi\in [a_n,b_n]\),\(\forall n\in\mathbb{N}_+\).又由\(f(x)\)在\([0,1]\)上单调递增及\(f(a_n)>a_n\),\(f(b_n)<b_n\),\(\forall n\in\mathbb{N}_+\),可知\(a_n<f(a_n)\leqslant f(\xi)\leqslant f(b_n)<b_n\).令\(n\rightarrow\infty\)可得\(\xi\leqslant f(\xi)\leqslant\xi\),即\(f(\xi)=\xi\).这与假设矛盾.
\end{proof}

\begin{lemma}[Lebesgue数引理]\label{lemma:Lebesgue数引理}
如果\(\{\mathcal{O}_{\alpha}\}\)是区间\([a,b]\)的一个开覆盖,则存在一个正数\(\delta>0\),使得对于区间\([a,b]\)中的任何两个点\(x',x''\),只要\(\vert x' - x''\vert<\delta\),就存在开覆盖中的一个开区间,它覆盖\(x',x''\).(称这个数\(\delta\)为开覆盖的Lebesgue数.)
\end{lemma}
\begin{note}
本题谢惠民上的证明是利用有限覆盖定理,而$CMC$红宝书上通过直接构造出$\delta$进行证明.这里我们采用的是聚点定理进行证明.
\end{note}
\begin{proof}
(反证法)假设对$\forall n\in\mathbb{N}_+$,取$\delta=\frac{1}{n}>0$,都存在相应的$x_n,y_n\in [a,b]$且$\left| x_n-y_n \right|<\delta$,使得对$\forall I\in\{\mathcal{O} _{\alpha}\}$,要么$x_n \notin I$,要么$y_n \notin I$.由聚点定理可知,有界数列$\{x_n\},\{y_n\}$一定存在收敛子列.设$\{x_{n_k}\},\{y_{m_k}\}$为相应的收敛子列,则由$\left| x_n-y_n \right|<\delta=\frac{1}{n},\forall n\in \mathbb{N}_+$可知${x_{n_k}},{y_{m_k}}$收敛于同一个极限点.故设$\underset{k\rightarrow \infty}{\lim}x_{n_k}=\underset{k\rightarrow \infty}{\lim}y_{m_k}=x_0\in[a,b]$.

因为\(\{\mathcal{O}_{\alpha}\}\)是区间\([a,b]\)的一个开覆盖,所以存在$I_0\in \{\mathcal{O}_{\alpha}\}$,使得$x_0 \in I_0$.又由于$I_0$是开集,因此存在$\eta>0$,使得$(x_0-\eta,x_0+\eta)\subset I_0$.从而由$\underset{k\rightarrow \infty}{\lim}x_{n_k}=\underset{k\rightarrow \infty}{\lim}y_{m_k}=x_0\in[a,b]$可知,存在充分大的$K$,使得$\left| x_{n_K}-x_0 \right|<\eta ,\left| y_{m_K}-x_0 \right|<\eta$.于是$x_{n_K},y_{m_K}\in \left( x_0-\eta ,x_0+\eta \right) \subset I_0$.即开区间$I_0\in \{\mathcal{O}_{\alpha}\}$同时覆盖了$x_{n_K},y_{m_K}$这两个点,与假设矛盾.
\end{proof}
\begin{remark}
注意对于两个收敛子列$\{x_{n_k}\},\{y_{m_k}\}$,此时$n_k = m_k$并不一定对$\forall k\in \mathbb{N}_+$都成立,即这两个收敛子列的指标集$\left\{ n_k \right\} _{k=1}^{\infty},\left\{ m_k \right\} _{k=1}^{\infty}$不相同也不一定有交集,故无法利用聚点定理反复取子列的方法取到两个指标相同且同时收敛的子列$\left\{ x_{n_k} \right\} _{k=1}^{\infty},\left\{ y_{n_k} \right\} _{k=1}^{\infty}$(取$\{x_n\}$为一个奇子列收敛,偶子列发散的数列;取$\{y_n\}$为一个奇子列发散,偶子列收敛的数列就能得到反例.).
\end{remark}

\begin{example}
\begin{enumerate}
\item 设\(f(x)\)定义在\(\mathbb{R}\)中且对任意\(x\),都存在与\(x\)有关的\(r > 0\),使得\(f(x)\)在区间\((x - r,x + r)\)中为常值函数,证明:\(f(x)\)是常值函数.
\item 设\(f(x)\)是定义在\([a,b]\)中的实值函数,如果对任意\(x\in[a,b]\),均存在\(\delta_x>0\)以及\(M_x\),使得\(\vert f(y)\vert\leqslant  M_x\),\(\forall y\in(x - \delta_x,x + \delta_x)\cap [a,b]\),证明:\(f(x)\)是有界的.
\item 设\(f(x)\)定义在\(\mathbb{R}\)上,对任意\(x_0\in\mathbb{R}\)均存在与\(x_0\)有关的\(\delta>0\),使得\(f(x)\)在\((x_0 - \delta,x_0 + \delta)\)是单调递增的,证明:\(f\)在整个\(\mathbb{R}\)上也是单调递增的.
\end{enumerate}
\end{example}
\begin{note}
这个结果说明:局部常值函数就是常值函数,闭区间上局部有界的函数都是有界函数,局部单调递增函数在整个区间上也是单调递增的,\textbf{实数基本定理能够将局部性质扩充为整体性质}.
\end{note}
\begin{proof}
\begin{enumerate}
\item 
{\color{blue}证法一(有限覆盖定理)(不建议使用):}对任意\(x\in[a,b]\),存在\(r_x>0\)使得\(f(t)\)在区间\((x - r_x,x + r_x)\)为常值函数,则\(\bigcup_{x\in[a,b]} (x - r_x,x + r_x) \supset [a,b]\),故存在其中有限个区间\((x_k - r_k,x_k + r_k),1\leqslant  k\leqslant  n\)使得他们的并集包含\([a,b]\).

直观来看只需要将这些区间“从小到大”排列,就可以依次推出每一个区间上都是相同的一个常值函数,但是所谓“从小到大”排列目前是无法准确定义的,所以这样说不清楚,优化如下:

方案1:选择其中个数尽可能少的区间,使得它们的并集可以覆盖\([a,b]\)但是任意删去一个都不可以(这是能够准确定义的一个操作),此时区间具备性质“任意一个不能被其余的并集盖住”,接下来将这些区间按照左端点的大小关系来排序,去论证它们确实是如你所想的那样“从小到大”排列的(关注右端点),进而得证.

方案2:利用\hyperref[lemma:Lebesgue数引理]{Lebesgue数引理},将区间\([a,b]\)分为有限个\([a,a + \delta],[a + \delta,a + 2\delta],\cdots,[a + n\delta,b]\),其中$\delta$是Lebesgue数.则每一个闭区间都可以被开覆盖中的某一个开区间覆盖住,于是分段常值函数,并且还能拼接起来,所以是常值函数.

{\color{blue}证法二(确界存在定理):}(反证法)假设存在$a,b\in \mathbb{R}$,使得$f(a)\ne f(b)$.构造数集
\begin{align*}
E=\left\{ x\in \left[ a,b \right] |f\left( t \right) =f\left( a \right) ,\forall t\in \left[ a,x \right] \right\}.
\end{align*}
从而$E\ne \varnothing$且$E\in[a,b]$.于是由确界存在定理,可知数集$E$存在上确界,设$x_0=\mathrm{sup}E$.

如果\(f(a)\neq f(x_0)\),则由条件可知,存在\(r_0 > 0\),使得\(f(t)=f(x_0)\),\(\forall t\in(x_0 - r_0,x_0 + r_0)\).由\(x_0 = \sup E\)可知,存在\(x_1\in(x_0 - r_0,x_0)\)且\(x_1\in E\).于是\(f(t)=f(a)\),\(\forall t\in[a,x_1]\).从而\(f(t)=f(a)=f(x_0)\),\(\forall t\in(x_0 - r_0,x_1)\).这与\(f(x_0)\neq f(a)\)矛盾.

如果\(f(a)=f(x_0)\),则由条件可知,存在\(r_1 > 0\),使得\(f(t)=f(x_0)=f(a)\),\(\forall t\in(x_0 - r_1,x_0 + r_1)\).又由\(x_0 = \sup E\)可知,存在\(x_2\in(x_0 - r_1,x_0)\)且\(x_2\in E\).于是\(f(t)=f(a)\),\(\forall t\in[a,x_2]\).进而对\(\forall t\in[a,x_2]\cup(x_0 - r_1,x_0+\frac{r_1}{2}]=[a,x_0+\frac{r_1}{2}]\),有\(f(t)=f(a)\).从而\(x_0+\frac{r_1}{2}\in E\),这与\(x_0 = \sup E\)矛盾.

故假设不成立,命题得证.

{\color{blue}证法三(闭区间套定理):}(反证法)假设存在$a,b\in \mathbb{R}$,使得$f(a)\ne f(b)$.不妨设\(f(a) < f(b)\),则记闭区间\([a,b]=[a_1,b_1]\).若\(f(\frac{a_1 + b_1}{2}) > f(a_1)\),则记闭区间\([a_1,\frac{a_1 + b_1}{2}]=[a_2,b_2]\);若\(f(\frac{a_1 + b_1}{2}) < f(b_1)\),则记闭区间\([\frac{a_1 + b_1}{2},b_1]=[a_2,b_2]\).以此类推,可以得到一列闭区间\(\{[a_n,b_n]\}\),满足\([a_n,b_n]\subset [a_{n + 1},b_{n + 1}]\),\(f(a_n) < f(b_n)\),\(\forall n\in\mathbb{N}_+\).由闭区间套定理,可知存在唯一\(\xi=\lim_{n\rightarrow\infty}a_n=\lim_{n\rightarrow\infty}b_n\),且\(\xi\in [a_n,b_n]\).又由条件可知,存在\(r > 0\),使得\(f(t)=f(\xi)\),\(\forall t\in (\xi - r,\xi + r)\).从而存在充分大的\(N\in\mathbb{N}_+\),使得\(\vert a_N - \xi\vert < r\),\(\vert b_N - \xi\vert < r\),即\(a_N,b_N\in (\xi - r,\xi + r)\).于是\(f(a_N)=f(b_N)\),这与\(f(a_N) < f(b_N)\)矛盾.

\item {\color{blue}(聚点定理):}(反证法)假设$f(x)$在$[a,b]$上无界,则对$\forall n>0$,都存在$x_n\in[a,b]$,使得$\left| f\left( x_n \right) \right|>n$.从而得到一个有界数列$\{x_n\}$.由聚点定理,可知其存在收敛子列$\{x_{n_k}\}$,设$\underset{k\rightarrow \infty}{\lim}x_{n_k}=x_0$.由条件可知,存在\(\delta _{x_0}>0\)以及\(M_{x_0}\),使得\(\vert f(y)\vert\leqslant M_{x_0}\),\(\forall y\in(x_0 - \delta _{x_0},x_0+\delta _{x_0})\).
又由\(\lim_{k\rightarrow\infty}x_{n_k}=x_0\)可知,存在\(K > M_{x_0}\),使得\(\vert x_{n_K}-x_0\vert<\delta _{x_0}\),即\(x_{n_K}\in(x_0 - \delta _{x_0},x_0+\delta _{x_0})\).于是\(\vert f(x_{n_K})\vert\leqslant M_{x_0}\).
而\(\vert f(x_{n_K})\vert>n_K\geqslant K > M_{x_0}\)矛盾.

\item {\color{blue}(闭区间套定理):}(反证法)假设存在$a,b\in \mathbb{R}$,使得$f(a)\geqslant  f(b)$.记闭区间\([a,b]=[a_1,b_1]\),若\(f\left(\frac{a_1 + b_1}{2}\right) \leqslant f\left(a_1\right)\),则记闭区间\(\left[a_1,\frac{a_1 + b_1}{2}\right]=[a_2,b_2]\);若\(f\left(\frac{a_1 + b_1}{2}\right) \geqslant f\left(b_1\right)\),则记闭区间\(\left[\frac{a_1 + b_1}{2},b_1\right]=[a_2,b_2]\).以此类推,可以得到一列闭区间\(\{[a_n,b_n]\}\),满足\([a_n,b_n]\subset [a_{n + 1},b_{n + 1}]\),\(f(a_n)\geqslant f(b_n)\),\(\forall n\in \mathbb{N}_+\).
由闭区间套定理,可知存在唯一\(\xi = \lim_{n\rightarrow \infty} a_n = \lim_{n\rightarrow \infty} b_n\),且\(\xi \in [a_n,b_n]\).由条件可知,存在\(\delta > 0\),使得\(f\left(x\right)\)在区间\(\left(\xi - \delta,\xi + \delta\right)\)上单调递增.
又由\(\xi = \lim_{n\rightarrow \infty} a_n = \lim_{n\rightarrow \infty} b_n\)可知,存在\(N > 0\),使得\(\left|a_N - \xi\right| < \delta\),\(\left|b_N - \xi\right| < \delta\),即\(a_N,b_N\in \left(\xi - \delta,\xi + \delta\right)\),且\(a_N < b_N\).于是\(f\left(a_N\right) \leqslant f\left(b_N\right)\).而\(f\left(a_N\right) \geqslant f\left(b_N\right)\),这就产生了矛盾.
\end{enumerate}
\end{proof}

\begin{lemma}\label{lemma:R中的极值点集至多可数}
设\(f(x)\)定义在区间\(I\)中,则\(f(x)\)的全体极值构成的集合是至多可数集.
\end{lemma}
\begin{proof}
极值只有极大值和极小值,因此只要证明极大值全体与极小值全体都是至多可数的即可.

设\(f(x)\)的全体极小值构成的集合为\(A\),则
\[
A = \{f(x)|\exists\delta > 0,\forall t\in(x - \delta,x + \delta),f(t)\geqslant f(x)\}.
\]
故对\(\forall y\in A\),都存在\(x\in I\),使得\(y = f(x)\),并且\(\exists\delta > 0,\forall t\in(x - \delta,x + \delta),f(t)\geqslant f(x)\).由有理数的稠密性可知,存在\(r\in(x - \delta,x)\cap\mathbb{Q}\),\(s\in(x,x + \delta)\cap\mathbb{Q}\).从而\((r,s)\subset(x - \delta,x + \delta)\),于是对\(\forall t\in(r,s)\),同样有\(f(t)\geqslant f(x)\).

再设全体有理开区间构成的集合为\(B\),现在定义一个映射
\[
\varphi:A\longrightarrow B;\quad y\longmapsto(r,s).
\]
任取\(y_1,y_2\in A\)且\(y_1\neq y_2\),则存在\(x_1,x_2\in I\),使得\(f(x_1)=y_1\),\(f(x_2)=y_2\).假设\(\varphi(y_1)=\varphi(y_2)=(r_0,s_0)\),则对\(\forall t\in(r_0,s_0)\),都有\(f(t)\geqslant y_1,y_2\).于是\(y_1 = f(x_1)\geqslant y_2\),\(y_2 = f(x_2)\geqslant y_1\),从而\(y_1 = y_2\),这产生了矛盾.故\(\varphi(y_1)\neq\varphi(y_2)\),因此\(\varphi\)是单射.

而由全体有理开区间构成的集合\(B\)是至多可数的,因此\(f(x)\)的全体极小值构成的集合\(A\)也是至多可数的.同理,\(f(x)\)的全体极大值构成的集合也是至多可数的.
\end{proof}
\begin{remark}
由全体有理开区间构成的集合\(B\)是可数集的原因:

构造一个映射
\begin{align*}
\phi :B\longrightarrow \mathbb{Q} \times \mathbb{Q} ;\quad \left( r,s \right) \longmapsto \left( r,s \right) .
\end{align*}
显然$\phi$是一个双射,而$\mathbb{Q} \times \mathbb{Q}$是可数集,故$B$也是可数集.
\end{remark}

\begin{example}
设\(f(x)\)在区间\(I\)中连续,并且在每一点\(x\in I\)处都取到极值,证明:\(f(x)\)是常值函数.
\end{example}
\begin{remark}
连续这一条件不可删去,也不可减弱为至多在可数个点不连续.反例:考虑黎曼函数即可,它处处取极值,并且在有理点不连续,无理点连续.
\end{remark}
\begin{proof}
{\color{blue}证法一(\hyperref[lemma:R中的极值点集至多可数]{引理\ref{lemma:R中的极值点集至多可数}}):}(反证)假设$f(x)$不是常值函数,则存在$a,b \in I$,使得$f(a)\ne f(b)$.由$f$的连续性及连续函数的介值性可知,$f(x)$可以取到$f(a),f(b)$中的一切值.故$f(x)$的值域是不可数集(区间都是不可数集).又由条件可知,$f(x)$的值域就是由$f(x)$的全体极值构成的.于是根据\hyperref[lemma:R中的极值点集至多可数]{引理\ref{lemma:R中的极值点集至多可数}}可得,$f(x)$的值域是至多可数集.这与$f(x)$的值域是不可数集矛盾.

{\color{blue}证法二(闭区间套定理):}假设$f(x)$不是常值函数,则存在$a_1,b_1 \in I$,使得$f(a_1)\ne f(b_1)$.不妨设$f(a_1)<f(b_1)$.因为\(f\)在\(I\)上连续,所以由介值定理可知,存在\(c_1\in [a_1,b_1]\),使得\(f(a_1) < f(c_1)=\frac{f(a_1) + f(b_1)}{2}<f(b_1)\).若\(b_1 - c_1\leqslant\frac{b_1 - a_1}{2}\),则令\([a_2,b_2]=[c_1,b_1]\);若\(c_1 - a_1\leqslant\frac{b_1 - a_1}{2}\),则令\([a_2,b_2]=[a_1,c_1]\).无论哪种情况,都有\(f(a_2) < f(b_2)\).

在\([a_2,b_2]\)上重复上述操作,并依次类推下去,得到一列闭区间套\(\{[a_n,b_n]\}\)满足
\[
[a_n,b_n]\subset [a_{n + 1},b_{n + 1}], f(a_n) < f(b_n),\forall n\in\mathbb{N}_+.
\]
由闭区间套定理可知,存在唯一\(x_0\in\bigcap_{n = 1}^{\infty}[a_n,b_n]\),使得\(x_0=\lim_{n\rightarrow\infty}a_n=\lim_{n\rightarrow\infty}b_n\).再由\(f\)的连续性以及\(Heine\)归结原则可知,\(f(a_n)\)严格递增收敛于\(f(x_0)\),\(f(b_n)\)严格递减收敛于\(f(x_0)\).故\(f(a_n) < f(x_0) < f(b_n)\),\(\forall n\in\mathbb{N}_+\).
因此对\(\forall\delta > 0\),都存在\(N > 0\),使得\(\vert a_N - x_0\vert<\delta\),\(\vert b_N - x_0\vert<\delta\),并且\(f(a_N) < f(x_0) < f(b_N)\).从而\(x_0\in I\)不是\(f(x)\)的极值点,这与\(f\)在\(I\)上处处取极值矛盾.
\end{proof}

\begin{theorem}[Baire纲定理]\label{theorem:Baire纲定理}
\begin{enumerate}
\item 设 \( A_n \subset \mathbb{R}\) 是一列没有内点的闭集,则 \(\bigcup_{n=1}^{\infty}{A_n} \)
也没有内点.

\item 设 \( A_n \subset \mathbb{R}\) 是一列开集并且都在 \(\mathbb{R}\) 稠密,则 \(\bigcap_{n=1}^{\infty}A_n \)
也在 \(\mathbb{R}\) 中稠密.

\item 设 \( A_n \subset \mathbb{R}\) 是一列闭集,并且 $A=\bigcup_{n=1}^{\infty}{A_n}$
也是闭集,则存在开区间 \((a, b)\)(可以无穷区间)和正整数 \( N \) 使得 \((a,b) \cap A \subset A_N\).

\item 设 \( A_n \) 是一列无处稠密集(闭包没有内点),则 \(\bigcup_{n=1}^{\infty} A_n \)
也没有内点.
\end{enumerate}
\end{theorem}
\begin{proof}
\begin{enumerate}
\item 用反证法. 设\(x_0\in A=\bigcup_{n=1}^{\infty} A_n\)为内点, 则存在\(\delta_0 > 0\), 使得\([x_0 - \delta_0,x_0 + \delta_0]\subset A\).
因为\(A_1\)没有内点,故存在\(x_1\in(x_0 - \delta_0,x_0 + \delta_0)-A_1\).由于\(A_1\)为闭集,故存在\(\delta_1 > 0\),使得
\[
[x_1 - \delta_1,x_1 + \delta_1]\subset(x_0 - \delta_0,x_0 + \delta_0),\quad [x_1 - \delta_1,x_1 + \delta_1]\cap A_1=\varnothing
\]
不妨设\(\delta_1 < 1\). 因为\(A_2\)没有内点, 故存在\(x_2\in(x_1 - \delta_1,x_1 + \delta_1)-A_2\). 由于\(A_2\)为闭集,故存在\(\delta_2 > 0\),使得
\[
[x_2 - \delta_2,x_2 + \delta_2]\subset(x_1 - \delta_1,x_1 + \delta_1),\quad [x_2 - \delta_2,x_2 + \delta_2]\cap A_2=\varnothing
\]
不妨设\(\delta_2 < \frac{1}{2}\). 如此继续,我们得到闭区间套
\[
[x_1 - \delta_1,x_1 + \delta_1]\supset[x_2 - \delta_2,x_2 + \delta_2]\supset\cdots\supset[x_n - \delta_n,x_n + \delta_n]\supset\cdots,
\]
使得\([x_n - \delta_n,x_n + \delta_n]\cap A_n=\varnothing,\delta_n < \frac{1}{n}(n\geqslant1)\).
根据闭区间套原理, 存在\(\xi\in[x_n - \delta_n,x_n + \delta_n],\forall n\geqslant1\). 因此\(\xi\notin\bigcup_{n\geqslant1}A_n = A\),
这和\(\xi\in[x_1 - \delta_1,x_1 + \delta_1]\subset(x_0 - \delta_0,x_0 + \delta_0)\subset A\)相矛盾.

\item 

\item 

\item 
\end{enumerate}
\end{proof}

\begin{example}
设数列\(a_n\)单调递增趋于正无穷,并且\(\lim_{n \to \infty} \frac{a_{n + 1}}{a_n} \leqslant 1\),函数\(f(x)\)定义在\((0,+\infty)\)中且对任意\(x\geqslant 1\)都有\(\lim_{n \to \infty} f(a_nx) = 0\).
\begin{enumerate}
\item  若\(f(x)\)是连续函数,证明:\(\lim_{x \to +\infty} f(x) = 0\);
\item 若删去连续这一条件,或者虽然连续,但是\({\varlimsup_{n \to \infty}} \frac{a_{n + 1}}{a_n} > 1\),则上述结论均不成立.
\end{enumerate}
\end{example}
\begin{proof}
\begin{enumerate}
\item 对任意\(\varepsilon > 0\),定义\(E_n=\{x\geqslant 1|\forall k\geqslant  n,|f(a_kx)|\leqslant \varepsilon\}\),则\(E_n\)是一列闭集,根据条件有\(\bigcup_{n = 1}^{\infty}E_n=[1,+\infty)\).
于是根据baire纲定理可知存在正整数\(N\)和区间\((u,v)\)使得\((u,v)\subset E_N\),也就是说,任意\(x\in(u,v)\),任意\(n\geqslant  N\)都有\(|f(a_nx)|\leqslant \varepsilon\),换句话说我们得到了一个一致的\(N\).
因此\(|f(x)|\)在区间\((a_Nu,a_Nv),(a_{N + 1}u,a_{N+1}v),\cdots\)中都是不超过\(\varepsilon\)的,只要这些区间在\(n\)很大之后能够相互有重叠,一个接着下一个,全覆盖就行了.
换句话说,我们要证明:存在\(N_0\)使得任意\(n\geqslant  N_0\)都有\(a_{n + 1}u < a_nv\),这等价于\(\frac{a_{n + 1}}{a_n}<\frac{v}{u}\),注意条件:极限等于\(1\)并且右端\(\frac{v}{u}>1\),所以上式成立.
将前面推导的东西梳理一下,就是说:任意\(\varepsilon > 0\),存在\(M\)使得\(x > M\)时恒有\(|f(x)|<\varepsilon\),结论得证.

\item 例如考虑\(a_n = n\),定义\(f(x)\)为:当\(x = m\cdot2^{\frac{1}{k}},m\in\mathbb{N}^+\)时候取\(1\),其余情况都取\(0\),则对任意的\(x > 0\),数列\(f(nx)\)中都至多只有一项为\(1\),因此极限总是\(0\),但是很明显\(f(x)\)的极限并不存在.
另外一个反例,可以考虑\(a_n = e^n\),现在有连续性,条件为
\[
\lim_{n\to\infty}f(e^n)=\lim_{n\to\infty}f(e^{n+\ln x}) = 0
\]
将\(\ln x\in\mathbb{R}\)看成一个变量,相应的考虑\(g(x)=f(e^x)\),则连续函数\(g(x)\)定义在\(\mathbb{R}\)上且满足\(\lim_{n\to\infty}g(y + n)=\lim_{n\to\infty}f(e^{y + n}) = 0,\forall y\in\mathbb{R}\),我们构造一个例子使得\(g(x)\)在无穷处极限非零或者不存在即可.
这与经典的命题有关:设\(f(x)\)一致连续且\(f(x + n)\to0\)对任意\(x\)成立,则\(f(x)\to0\),现在删去了一致连续性命题自然是错的,具体构造留作习题.
\end{enumerate}
\end{proof}
\begin{remark}
通常,点态收敛(上题)或者数列极限(本题)这种非一致性的条件,描述起来是“任意\(x\in(0,1)\),任意\(\varepsilon > 0\),存在\(N\)使得任意\(n > N\)都有\(|f_n(x)-f(x)|<\varepsilon\)”或者“任意\(x > 0\),任意\(\varepsilon > 0\),存在\(N\)使得任意\(n > N\)都有\(|f(a_nx)|<\varepsilon\)”,很明显这里的\(N\)是与\(x,\varepsilon\)都有关系的,如果我们事先取定\(\varepsilon > 0\),那么这个过程可以说是“给定\(x\),去找对应的\(N\)”.
而baire纲定理的想法就是反过来找:不同的\(x\)对应的\(N\)确实可以不一样,那就先取好\(N\),我们看都有哪些\(x\)对应到这一个\(N\),也就是说事先取定\(\varepsilon > 0\),然后对每一个\(n\)去定义集合,反找\(x\).所有baire纲定理相关的问题,思想都是如此,根据定理便能得到一个一致的东西,拿来做事情.
\end{remark}

\begin{example}
设\(f(x)\)在区间\((0,1)\)中可导,证明:\(f'(x)\)在\((0,1)\)中的一个稠密子集中连续.
\end{example}
\begin{proof}

\end{proof}

\begin{lemma}
有界数列\(x_n\)如果满足\(\lim_{n\to\infty}(x_{n + 1}-x_n)=0\),则\(x_n\)的全体聚点构成一个闭区间.
\end{lemma}
\begin{proof}

\end{proof}

\begin{example}
设连续函数\(f(x):[0,1]\to[0,1],x_1\in[0,1],x_{n + 1}=f(x_n)\),证明:数列\(\{x_n\}\)收敛的充要条件是\(\lim_{n\to\infty}(x_{n + 1}-x_n)=0\).
\end{example}
\begin{proof}
必要性($\Rightarrow$):若$\{x_n\}$收敛,则$\underset{n\rightarrow \infty}{\lim}\left( x_{n+1}-x_n \right) =0$显然成立.

充分性($\Leftarrow$):
\end{proof}


\end{document}
\documentclass[../../main.tex]{subfiles}
\graphicspath{{\subfix{../../image/}}} % 指定图片目录,后续可以直接使用图片文件名。

% 例如:
% \begin{figure}[h]
% \centering
% \includegraphics{image-01.01}
% \label{fig:image-01.01}
% \caption{图片标题}
% \end{figure}

\begin{document}

\section{上下极限}

\begin{proposition}[子列极限命题]\label{proposition:子列极限命题}
(a):给定\(x \in \mathbb{R} \cup \{+\infty, -\infty\}\),\(\lim_{n \to \infty} x_n = x\)的充分必要条件是对任何广义存在的\(\lim_{k \to \infty} x_{n_k}\),都有\(\lim_{k \to \infty} x_{n_k} = x\).

(b):设\(m \in \mathbb{N}\),若\(\lim_{n \to \infty} x_{mn + r}\),\(\forall r = 0,1,2,\cdots,m - 1\)相同,则\(\lim_{n \to \infty} x_n\)存在且
\(\lim_{n \to \infty} x_n = \lim_{n \to \infty} x_{mn}\).
\end{proposition}
\begin{note}
当\(m = 2\),上述命题是在说如果序列奇偶子列极限存在且为同一个值,则序列的极限存在且极限和偶子列极限值相同.所谓奇偶,就是看除以\(2\)的余数是\(1\)还是\(0\).对一般的\(m \in \mathbb{N}\),我们也可以看除以\(m\)的余数是\(\{0,1,2,\cdots,m - 1\}\)中的哪一个来对整数进行分类,即\(\text{mod } m\)分类.严格的说,我们有无交并
\begin{align*}
\mathbb{Z} = \bigcup_{r = 0}^{m - 1} \{mk + r : k \in \mathbb{Z}\}.
\end{align*}
\end{note}
\begin{proof}
对(a):考虑上下极限即可.

对(b):记\(A \triangleq \lim_{n \to \infty} x_{mn}\).
事实上对任何\(\varepsilon > 0\),存在\(N \in \mathbb{N}\),使得当\(k > N\)时,我们有
\begin{align}\label{equation:2.121}
\vert x_{mk + r} - A \vert < \varepsilon, \forall r \in \{0,1,2,\cdots,m - 1\}.
\end{align}
我们知道对任何正整数\(n > mN + m - 1\),存在唯一的\(r \in \{0,1,2,\cdots,m - 1\}\)和\(k > N\),使得\(n = km + r\),于是运用\eqref{equation:2.121}我们有\(\vert x_n - A \vert < \varepsilon\),因此我们证明了
\begin{align*}
\lim_{n \to \infty} x_n = A = \lim_{n \to \infty} x_{mn} .
\end{align*}
\end{proof}

\begin{definition}[上下极限的定义]\label{theorem:上下极限的定义}
我们定义
\begin{align}\label{eq:上下极限的定义}
\varlimsup_{n \to \infty} a_n \triangleq \lim_{n \to \infty} \sup_{k \geq n} a_k, \varliminf_{n \to \infty} a_n \triangleq \lim_{n \to \infty} \inf_{k \geq n} a_k. 
\end{align}
\end{definition}
\begin{note}
注意到由定义,\(\sup_{k \geq n} a_k\)是单调递减的,\(\inf_{k \geq n} a_k\)是单调递增的.因此\eqref{eq:上下极限的定义}式的极限存在或为确定符号的\(\infty\).
\end{note}

\begin{proposition}[上下极限的等价定义]\label{proposition:上下极限的等价定义}
假定\(\{a_n\}\)是个实数列,则有

(1):设\(A\)是某个实数,则\(\varlimsup_{n \to \infty} a_n = A\)的充分必要条件是对任何\(\varepsilon > 0\),存在$N>0$,使得当$n>N$时,有$x_n<A+\varepsilon$且存在子列$\{x_{n_k}\}$,使得$x_{n_k}>A-\varepsilon,k=1,2,\cdots$.

(2):\(\varlimsup_{n \to \infty} a_n = +\infty\)的充分必要条件是对任何\(A > 0\),存在\(n\),使得\(a_n > A\).

(3):设\(A\)是某个实数,则\(\varliminf_{n \to \infty} a_n = A\)的充分必要条件是对任何\(\varepsilon > 0\),存在$N>0$,使得当$n>N$时,有$x_n>A-\varepsilon$且存在子列$\{x_{n_k}\}$,使得$x_{n_k}<A+\varepsilon,k=1,2,\cdots$.

(4):\(\varliminf_{n \to \infty} a_n = -\infty\)的充分必要条件是对任何\(A < 0\),存在\(n\),使得\(a_n < A\).
\end{proposition}

\begin{proposition}[上下极限的性质]\label{proposition:上下极限的性质}
我们有如下的

1. $\varlimsup_{n \to \infty} (a_n + b_n) \leq \varlimsup_{n \to \infty} a_n + \varlimsup_{n \to \infty} b_n.$

2. $-\varlimsup_{n \to \infty} a_n = \varliminf_{n \to \infty} (-a_n).$

3. $\varliminf_{n \to \infty} (a_n + b_n) \geq \varliminf_{n \to \infty} a_n + \varliminf_{n \to \infty} b_n.$

\hypertarget{上极限的性质命题(4)}{4}.若\(\lim_{n \to +\infty} b_n = b\),\(\varlimsup_{n \to +\infty} a_n = a\),则\(\varlimsup_{n \to +\infty} a_n b_n = ab\).
\end{proposition}
\begin{note}
上下极限的性质都可以通过考虑其子列的极限快速得到证明.因此我们一般不需要额外记忆上下极限的性质,只需要熟悉通过考虑子列极限直观地得到结论即可.并且\textbf{因为上下极限就是(最大/最小)子列极限,所以一般极限的性质对于上下极限都成立}.
\end{note}
\begin{proof}
1.

2.

3.

4.由于\(\varlimsup_{n \to +\infty} a_n = a\),因此我们可设\(\lim_{k \to +\infty} a_{n_k} = a\).
根据极限的四则运算法则,可知\(\lim_{n \to +\infty} a_{n_k} b_{n_k} = ab\).
从而\(\varlimsup_{n \to +\infty} a_n b_n \geqslant \lim_{n \to +\infty} a_{n_k} b_{n_k} = ab\).
又由上下极限的性质,可知\(\varlimsup_{n \to +\infty} a_n b_n \leqslant \varlimsup_{n \to +\infty} a_n \cdot \varlimsup_{n \to +\infty} b_n = ab\).
故\(\varlimsup_{n \to +\infty} a_n b_n = ab\).
\end{proof}

\begin{example}
求上极限
\begin{align*}
\underset{n\rightarrow +\infty}{{\varlimsup }}n\sin \left( \pi \sqrt{n^2+1} \right) .
\end{align*}
\end{example}
\begin{solution}
注意到
\begin{align*}
n\sin \left( \pi \sqrt{n^2+1} \right) =n\sin \left( \pi \sqrt{n^2+1}-n\pi +n\pi \right) =\left( -1 \right) ^nn\sin \left( \pi \sqrt{n^2+1}-n\pi \right) =\left( -1 \right) ^nn\sin \frac{\pi}{\sqrt{n^2+1}+n}.
\end{align*}
又因为
\begin{align*}
\underset{n\rightarrow +\infty}{\lim}n\sin \frac{\pi}{\sqrt{n^2+1}+n}=\underset{n\rightarrow +\infty}{\lim}\frac{n\pi}{\sqrt{n^2+1}+n}=\underset{n\rightarrow +\infty}{\lim}\frac{\pi}{\sqrt{1+\frac{1}{n^2}}+1}=\frac{\pi}{2}.
\end{align*}
所以
\begin{align*}
\underset{n\rightarrow +\infty}{{\varlimsup }}n\sin \left( \pi \sqrt{n^2+1} \right) =\underset{n\rightarrow +\infty}{{\varlimsup }}\left( -1 \right) ^nn\sin \frac{\pi}{\sqrt{n^2+1}+n}\hyperlink{本题最后一个等号}{\hypertarget{本题最后一个等号target}{=}}\frac{\pi}{2}.
\end{align*}
\end{solution}
\begin{remark}
\hypertarget{本题最后一个等号}{\hyperlink{本题最后一个等号target}{本题最后一个等号}}其实是直接套用了一个\hyperlink{上极限的性质命题(4)}{上极限的性质}得到的.
\end{remark}

\begin{proposition}
对任何\(\varepsilon > 0\),存在\(N \in \mathbb{N}\),使得
\[
f_1(n,\varepsilon) \leq a_n \leq f_2(n,\varepsilon), \forall n \geq N,
\]
这里
\[
\lim_{\varepsilon \to 0^+} \lim_{n \to \infty} f_2(n,\varepsilon) = \lim_{\varepsilon \to 0^+} \lim_{n \to \infty} f_1(n,\varepsilon) = A \in \mathbb{R}.
\]
证明\(\lim_{n \to \infty} a_n = A\).
\end{proposition}
\begin{note}
以后可以直接使用这个命题.但是要按照证法一的格式书写.
\end{note}
\begin{proof}
{\color{blue}证法一(利用上下极限)(也是实际做题中直接使用这个命题的书写步骤):}

已知对$\forall \varepsilon>0$,存在$N\in \mathbb{N}$,使得
\begin{align*}
f_1(n,\varepsilon) \leq a_n \leq f_2(n,\varepsilon), \forall n \geq N,
\end{align*}
上式两边令$n\to+\infty$,则有
\begin{align*}
\underset{n\rightarrow +\infty}{{\varliminf }}f_1(n,\varepsilon )\le \underset{n\rightarrow +\infty}{{\varliminf }}a_n,\underset{n\rightarrow +\infty}{{\varlimsup }}a_n\le \underset{n\rightarrow +\infty}{{\varlimsup }}f_2(n,\varepsilon ),\forall \varepsilon >0.
\end{align*}
由$\varepsilon$的任意性,两边令$\varepsilon\to 0^+$,可得
\begin{align*}
A=\underset{\varepsilon \rightarrow 0^+}{\lim}\underset{n\rightarrow +\infty}{{\varliminf }}f_1(n,\varepsilon )\le \underset{n\rightarrow +\infty}{{\varliminf }}a_n,\underset{n\rightarrow +\infty}{{\varlimsup }}a_n\le \underset{\varepsilon \rightarrow 0^+}{\lim}\underset{n\rightarrow +\infty}{{\varlimsup }}f_2(n,\varepsilon )=A.
\end{align*}
又显然有$\underset{n\rightarrow +\infty}{{\varliminf }}a_n\le \underset{n\rightarrow +\infty}{{\varlimsup }}a_n$,于是
\begin{align*}
A=\underset{\varepsilon \rightarrow 0^+}{\lim}\underset{n\rightarrow +\infty}{{\varliminf }}f_1(n,\varepsilon )\le \underset{n\rightarrow +\infty}{{\varliminf }}a_n\le \underset{n\rightarrow +\infty}{{\varlimsup }}a_n\le \underset{\varepsilon \rightarrow 0^+}{\lim}\underset{n\rightarrow +\infty}{{\varlimsup }}f_2(n,\varepsilon )=A.
\end{align*}
故由夹逼准则可得$\lim_{n \to \infty} a_n = A$.

{\color{blue}证法二($\varepsilon-\delta$语言):}

$\forall \varepsilon>0$,记$g_1\left( \varepsilon \right) =\underset{n\rightarrow +\infty}{\lim}f_1(n,\varepsilon ),g_2\left( \varepsilon \right) =\underset{n\rightarrow +\infty}{\lim}f_2(n,\varepsilon ).$由\(\lim_{\varepsilon \to 0^+}g_1(\varepsilon)=\lim_{\varepsilon \to 0^+}g_2(\varepsilon)=A\),可知
对\(\forall \eta > 0\),存在\(\delta > 0\),使得
\begin{align*}
g_1(\delta) > A - \frac{\eta}{2},g_2(\delta) < A + \frac{\eta}{2}.
\end{align*}
由于\(g_1(\delta)=\lim_{n \to +\infty}f_1(n,\delta)\),\(g_2(\delta)=\lim_{n \to +\infty}f_2(n,\delta)\),因此存在\(N'\in \mathbb{N}\),使得
\begin{align*}
f_1(n,\delta) > g_1(\delta) - \frac{\eta}{2},f_2(n,\delta) < g_2(\delta) + \frac{\eta}{2},\forall n > N'.
\end{align*}
又由条件可知,存在\(N\in \mathbb{N}\),使得
\begin{align*}
f_1(n,\delta) \leqslant a_n \leqslant f_2(n,\delta),\forall n > N .
\end{align*}
于是当\(n > \max\{N,N'\}\)时,对\(\forall \eta > 0\),我们都有
\begin{align*}
A - \eta < g_1(\delta) - \frac{\eta}{2} < f_1(n,\delta) \leqslant a_n \leqslant f_2(n,\delta) < g_2(\delta) + \frac{\eta}{2} < A + \eta.
\end{align*}
故由夹逼准则可知\(\lim_{n \to +\infty}a_n = A\). 
\end{proof}




\end{document}
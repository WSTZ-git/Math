\documentclass[../../main.tex]{subfiles}
\graphicspath{{\subfix{../../image/}}} % 指定图片目录,后续可以直接使用图片文件名。

% 例如:
% \begin{figure}[H]
% \centering
% \includegraphics[scale=0.3]{image-01.01}
% \caption{图片标题}
% \label{figure:image-01.01}
% \end{figure}
% 注意:上述\label{}一定要放在\caption{}之后,否则引用图片序号会只会显示??.

\begin{document}

\section{求积符号}

\begin{definition}[求积符号]\label{definition:求积符号}
\begin{align*}
\prod_{k=1}^n{a_k}\xlongequal{\bigtriangleup}a_1a_2\cdots a_n. 
\end{align*}
\end{definition}

\begin{theorem}[基本结论]\label{theorem:求积符号基本结论}
当$p,q\in \mathbb{Z}$且$p\leq q$时,有
\begin{gather*}
\prod_{n=p}^q{\frac{a_{n+1}}{a_n}}=\frac{a_{q+1}}{a_p};
\\
\prod_{n=p}^q{\frac{a_n}{a_{n+1}}}=\frac{a_p}{a_{q+1}}.
\end{gather*}
\end{theorem}
\begin{proof}
由求积符号定义很容易得到证明.
\end{proof}
\begin{remark}
对于正数列的乘积,我们可以通过取对数的方式,将其转化为$\ln \prod_{k=1}^n{a_k}=\sum\limits_{k=1}^n{\ln a_k}$来研究.
\end{remark}
\begin{example}
计算:$\prod_{k=2}^n{\frac{k^3-1}{k^3+1}}$.
\end{example}
\begin{solution}
\begin{align*}
&\prod_{k=2}^n{\frac{k^3-1}{k^3+1}}=\prod_{k=2}^n{\left( \frac{k-1}{k+1}\cdot \frac{k^2+k+1}{k^2-k+1} \right)}=\prod_{k=2}^n{\frac{k-1}{k+1}\cdot}\prod_{k=2}^n{\frac{k\left( k+1 \right) +1}{k\left( k-1 \right) +1}}
\\
&=\frac{1\cdot 2\cdots n-1}{3\cdot 4\cdots n+1}\cdot \frac{n\left( n+1 \right) +1}{2+1}=\frac{2}{n+1}\cdot \frac{n\left( n+1 \right) +1}{3}
\\
&=\frac{2n^2+2n+2}{3n+3}
\end{align*}
\end{solution}

\begin{example}
证明:\begin{align*}
\frac{\left( 2n-1 \right) !!}{2n!!}<\frac{1}{\sqrt{2n+1}},\forall n\in \mathbb{N} .
\end{align*}
\end{example}
\begin{note}
利用\hypertarget{"糖水"不等式}{\textbf{"糖水"不等式}}:
对任意真分数$\frac{b}{a},a,b,c>0$,都有$\frac{b}{a}<\frac{b+c}{a+c}$成立.
\end{note}
\begin{proof}
根据\hypertarget{"糖水"不等式}{"糖水"不等式},对$\forall n\in \mathbb{N}_+$,我们有
\begin{align*}
&\left[ \frac{\left( 2n-1 \right) !!}{2n!!} \right] ^2=\left( \prod_{k=1}^n{\frac{2k-1}{2k}} \right) ^2=\prod_{k=1}^n{\frac{2k-1}{2k}}\cdot \prod_{k=1}^n{\frac{2k-1}{2k}}
\\
&<\prod_{k=1}^n{\frac{2k-1}{2k}}\cdot \prod_{k=1}^n{\frac{2k}{2k+1}}=\prod_{k=1}^n{\frac{2k-1}{2k+1}}=\frac{1}{2n+1}
\end{align*}
故对$\forall n\in \mathbb{N}_+$,都有$\frac{\left( 2n-1 \right) !!}{2n!!}<\frac{1}{\sqrt{2n+1}},\forall n\in \mathbb{N}$成立.
\end{proof}



\end{document}
\documentclass[../../main.tex]{subfiles}
\graphicspath{{\subfix{../../image/}}} % 指定图片目录,后续可以直接使用图片文件名。

% 例如:
% \begin{figure}[h]
% \centering
% \includegraphics{image-01.01}
% \caption{图片标题}
% \label{fig:image-01.01}
% \end{figure}
% 注意:上述\label{}一定要放在\caption{}之后,否则引用图片序号会只会显示??.

\begin{document}

\section{求和符号}

\begin{definition}[空和(Empty sum)]\label{definition:空和(Empty sum)}
\begin{align}
\sum\limits_{i=b+1}^b{f(i)}\xlongequal{\bigtriangleup}0,b\in \mathbb{Z}.
\end{align}
\end{definition}

\begin{theorem}[关于求和号下限大于上限的计算]\label{theorem:关于求和号下限大于上限的计算}
\begin{align}
\sum\limits_{i=a}^c{f(i)}\equiv -\sum\limits_{i=c+1}^{a-1}{f(i),a,c}\in \mathbb{Z}\text{且} a>c. 
\end{align}
\end{theorem}
\begin{note}
上述\hyperref[definition:空和(Empty sum)]{空和的定义}与\hyperref[theorem:关于求和号下限大于上限的计算]{关于求和号下限大于上限的计算定理}都来自论文:\href{https://vixra.org/pdf/1601.0207v1.pdf}{Interpreting the summation notation when
the lower limit is greater than the upper limit(Kunle Adegoke)}.
\end{note}

\begin{theorem}[求和号基本性质]\label{theorem:求和号基本性质}
\begin{enumerate}
\item\label{theorem:求和号基本性质1} (\textbf{倒序求和})当$n$为非负整数时,有
\begin{align*}
\sum\limits_{k=1}^n{a_k}=\sum\limits_{k=1}^n{a_{n-k+1}}.
\end{align*}
\end{enumerate}
\end{theorem}
\begin{note}
\begin{enumerate}
\item 看到求和号内部有两个变量,都可以尝试一下将其转化为倒序求和的形式.
\end{enumerate}
\end{note}


\subsection{求和号交换顺序}

\begin{theorem}[\hypertarget{关于求和号换序的基本结论}{基本结论}]\label{theorem:求和号换序基本结论}
\hypertarget{求和号换序结论1}{1.}当$n,m$均为非负整数时,有
\begin{align*}
\sum\limits_{\substack{1\le i\le n\\1\le j\le m\\}}a_{ij}=\sum\limits_{i=1}^n{\sum\limits_{j=1}^m{a_{ij}}}=\sum\limits_{j=1}^m{\sum\limits_{i=1}^n{a_{ij}}}.
\end{align*}
2.当$n,m$均为非负整数,$p\leq n,q\leq m\text{且}p,q\in \mathbb{N_+}$时,有
\begin{align*}
\sum\limits_{\substack{
p\le i\le n\\
q\le j\le m\\
}}{a_{ij}}=\sum\limits_{i=p}^n{\sum\limits_{j=q}^m{a_{ij}}}=\sum\limits_{j=q}^m{\sum\limits_{i=p}^n{a_{ij}}.}
\end{align*}
\hypertarget{求和号换序结论2}{3.}当$n$为非负整数时,有
\begin{align*}
\sum\limits_{1\le i\le j\le n}{a_{ij}}=\sum\limits_{i=1}^n{\sum\limits_{j=i}^n{a_{ij}}}=\sum\limits_{j=1}^n{\sum\limits_{i=1}^j{a_{ij}}}.
\end{align*}
4.当$n$为非负整数时,有
\begin{align*}
\sum\limits_{1\le i<j\le n}{a_{ij}}=\sum\limits_{i=1}^{n-1}{\sum\limits_{j=i+1}^n{a_{ij}}}=\sum\limits_{j=2}^n{\sum\limits_{i=1}^{j-1}{a_{ij}}}.
\end{align*}
5.当$n$为非负整数时,有
\begin{align*}
\sum\limits_{i=1}^n{a_i}\cdot \sum\limits_{j=1}^n{b_j}=\sum\limits_{i=1}^n{\sum\limits_{j=1}^n{a_ib_j}}.
\end{align*}
6.当$n$为非负整数时,有
\begin{align*}
\left( \sum\limits_{i=1}^n{a_i} \right) ^2=\sum\limits_{i=1}^n{a_i\cdot \sum\limits_{j=1}^n{a_j}}\geqslant 0,\forall a_1,a_2,\cdots ,a_n\in \mathbb{R}=\sum\limits_{i=1}^n{\sum\limits_{j=1}^n{a_ia_j}} .
\end{align*}
\end{theorem}
\begin{note}
如果上述命题第1条中的$n$或$m$取到无穷,第2条中的$n$取到无穷,则求和号不能直接交换顺序.此时,往往要添加一个条件,相应的交换和号的结论才能成立.
比如,著名的$Fubini$定理(见\hyperlink{关于无限和的Fubinin定理}{关于无限和的Fubinin定理}).
\end{note}
\begin{proof}
1.利用矩阵证明该结论.

设一个$m$行$n$列的矩阵$A$为
\begin{align*}
A=\left[ \begin{matrix}
a_{11}&		a_{12}&		\cdots&		a_{1m}\\
a_{21}&		a_{22}&		\cdots&		a_{2m}\\
\vdots&		\vdots&		\ddots&		\vdots\\
a_{n1}&		a_{n2}&		\cdots&		a_{nm}\\
\end{matrix} \right] .
\nonumber
\end{align*}
则矩阵$A$的第$i$行的和记为
\begin{align*}
r_i=\sum\limits_{j=1}^m{a_{ij}}\left( i=1,2,\cdots ,n \right) .
\nonumber
\end{align*}
矩阵$A$的第$j$列的和记为
\begin{align*}
c_j=\sum\limits_{i=1}^n{a_{ij}}\left( j=1,2,\cdots ,m \right) .
\nonumber
\end{align*}
易知,矩阵所有元素的和等于所有行和$r_i,i=1,2,\cdots,n$求和也等于所有列和$c_j,j=1,2,\cdots,m$求和,即
\begin{align*}
\sum\limits_{\substack{1\le i\le n\\1\le j\le n\\}}a_{ij}=\sum\limits_{i=1}^n{r_i}=\sum\limits_{i=1}^n{\sum\limits_{j=1}^m{a_{ij}}},
\\
\sum\limits_{\substack{1\le i\le n\\1\le j\le n\\}}a_{ij}=\sum\limits_{j=1}^m{c_j}=\sum\limits_{j=1}^m{\sum\limits_{i=1}^n{a_{ij}}}.
\nonumber
\end{align*}
故\begin{align*}
\sum\limits_{i=1}^n{\sum\limits_{j=1}^m{a_{ij}}}=\sum\limits_{j=1}^m{\sum\limits_{i=1}^n{a_{ij}}}=\sum\limits_{\substack{1\le i\le n\\1\le j\le n\\}}a_{ij}.
\nonumber
\end{align*}
2.同理利用矩阵证明该结论.

设一个$m$行$n$列的矩阵$A$为
\begin{align*}
A=\left[ \begin{matrix}
a_{pq}&		a_{p,q+1}&		\cdots&		a_{pm}\\
a_{p+1,q}&		a_{p+1,q+1}&		\cdots&		a_{p+1,m}\\
\vdots&		\vdots&		\ddots&		\vdots\\
a_{nq}&		a_{n,q+1}&		\cdots&		a_{nm}\\
\end{matrix} \right] .
\nonumber
\end{align*}
则矩阵$A$的第$i$行的和记为
\begin{align*}
r_i=\sum\limits_{j=q}^m{a_{ij}}\left( i=p,p+1,\cdots ,n \right) .
\nonumber
\end{align*}
矩阵$A$的第$j$列的和记为
\begin{align*}
c_j=\sum\limits_{i=p}^n{a_{ij}}\left( j=q,q+1,\cdots ,m \right) .
\end{align*}
易知,矩阵所有元素的和等于所有行和$r_i,i=p,p+1,\cdots ,n$求和也等于所有列和$c_j,j=q,q+1,\cdots ,m$求和,即
\begin{align*}
\sum\limits_{\substack{
p\le i\le n\\
q\le j\le n\\
}}{a_{ij}}=\sum\limits_{i=p}^n{r_i}=\sum\limits_{i=p}^n{\sum\limits_{j=q}^m{a_{ij}},}
\\
\sum\limits_{\substack{
p\le i\le n\\
q\le j\le n\\
}}{a_{ij}}=\sum\limits_{j=q}^m{c_j}=\sum\limits_{j=q}^m{\sum\limits_{i=p}^n{a_{ij}}.}
\end{align*}
故\begin{align*}
\sum\limits_{i=p}^n{\sum\limits_{j=q}^m{a_{ij}}}=\sum\limits_{j=q}^m{\sum\limits_{i=p}^n{a_{ij}}}=\sum\limits_{\substack{p\le i\le n\\q\le j\le n\\}}a_{ij}.
\end{align*}

3.根据(1)的结论可得
\begin{align*}
\sum\limits_{j=1}^n{\sum\limits_{i=1}^j{a_{ij}}}=\sum\limits_{j=1}^n{\sum\limits_{i=1}^n{a_{ij}\chi _{i\le j}}}\left( i \right) 
\xlongequal{\hyperlink{求和号换序结论1}{1.\text{的结论}}}\sum\limits_{i=1}^n{\sum\limits_{j=1}^n{a_{ij}\chi _{i\le j}\left( i \right)}}=\sum\limits_{i=1}^n{\sum\limits_{j=i}^n{a_{ij}}}.
\end{align*}

4.根据(1)的结论可得
\begin{align*}
\sum\limits_{j=2}^{n}{\sum\limits_{i=1}^{j-1}{a_{ij}}}=\sum\limits_{j=2}^{n}{\sum\limits_{i=1}^{n-1}{a_{ij}\chi _{i< j}}}\left( i \right) 
\xlongequal{\hyperlink{求和号换序结论1}{1.\text{的结论}}}\sum\limits_{i=1}^{n-1}{\sum\limits_{j=2}^{n}{a_{ij}\chi _{i< j}\left( i \right)}}=\sum\limits_{i=1}^{n-1}{\sum\limits_{j=i+1}^{n}{a_{ij}}}.
\end{align*}

5.结论是显然的.

6.结论是显然的.
\end{proof}
\begin{remark}
设$X$是全集,对任意集合$A\subset X$,把函数
\begin{align*}
\chi _A\left( x \right) =\begin{cases}
1,x\in A\\
0,x\notin A\\
\end{cases}.
\nonumber
\end{align*}
称为集合$A$的\textbf{示性函数}.
\end{remark}

\begin{example}
计算
\begin{align*}
\sum\limits_{j=1}^n{\sum\limits_{i=1}^n{\frac{i}{2^{i+j}\left( i+j \right)}}.}
\end{align*}
\end{example}
\begin{solution}
令$I=\sum\limits_{j=1}^n{\sum\limits_{i=1}^n{\frac{i}{2^{i+j}\left( i+j \right)}}}$,则
\begin{align*}
I&=\sum\limits_{j=1}^n{\sum\limits_{i=1}^n{\frac{i}{2^{i+j}\left( i+j \right)}}}\xlongequal[\left( \text{轮换换元} \right)]{\text{将}i\text{换成}j,\text{换成}i}\sum\limits_{i=1}^n{\sum\limits_{j=1}^n{\frac{j}{2^{i+j}\left( i+j \right)}}}
\\
&=\frac{1}{2}\left( \sum\limits_{j=1}^n{\sum\limits_{i=1}^n{\frac{i}{2^{i+j}\left( i+j \right)}}}+\sum\limits_{i=1}^n{\sum\limits_{j=1}^n{\frac{j}{2^{i+j}\left( i+j \right)}}} \right)
= \frac{1}{2}\left( \sum\limits_{i=1}^n{\sum\limits_{j=1}^n{\frac{i}{2^{i+j}\left( i+j \right)}}}+\sum\limits_{i=1}^n{\sum\limits_{j=1}^n{\frac{j}{2^{i+j}\left( i+j \right)}}} \right) 
\\
&=\frac{1}{2}\sum\limits_{i=1}^n{\sum\limits_{j=1}^n{\frac{i+j}{2^{i+j}\left( i+j \right)}}}
=\frac{1}{2}\sum\limits_{i=1}^n{\sum\limits_{j=1}^n{\frac{1}{2^{i+j}}}}=\frac{1}{2}\sum\limits_{i=1}^n{\frac{1}{2^i}\cdot \sum\limits_{j=1}^n{\frac{1}{2^j}}}=\frac{1}{2}\left( \sum\limits_{i=1}^n{\frac{1}{2^i}} \right) ^2
\\
&=\frac{1}{2}\left( \frac{\frac{1}{2}-\frac{1}{2^{n+1}}}{1-\frac{1}{2}} \right) ^2=\frac{1}{2}\left[ 1-\frac{1}{2^n} \right] ^2.
\end{align*}
\end{solution}

\begin{example}
记\begin{align*}
T=\left\{ \left( a,b,c \right) \in \mathbb{N} ^3:a,b,c\text{可以构成某个三角形的三边长} \right\} .
\end{align*}
证明:\begin{align*}
\sum\limits_{\left( a,b,c \right) \in T}{A_{a,b,c}}=\sum\limits_{\left( x,y,z \right) \in \mathbb{N} ^3\text{且有相同的奇偶性}}{A_{\frac{x+y}{2},\frac{y+z}{2},\frac{z+x}{2}}}.
\end{align*}   
\end{example}
\begin{note}
核心想法:两个集合间可以建立一一映射.
\end{note}
\begin{conclusion}
若$x,y,z\in \mathbb{N} _+,x,y,z$具有相同奇偶性的充要条件为
\begin{align*}
x+y=2a,y+z=2b,x+z=2c,\text{其中}a,b,\in \mathbb{N} _+.
\end{align*}
\begin{proof}
必要性显然.下面证明充分性.
假设$x,y,z$具有不同的奇偶性,则不妨设$x,z$为奇数,$y$为偶数.从而$x+y$一定为奇数,这与$x+y=2a$矛盾.故$x,y,z$具有相同奇偶性.
\end{proof}
\end{conclusion}
\begin{proof}
设\(T = \{ (a,b,c) \in \mathbb{N}^3 : a,b,c\text{ 可以构成某个三角形的三边长}\}\).
\begin{align*}
\sum\limits_{(a,b,c) \in T}A_{a,b,c} = \sum\limits_{(x,y,z) \in \mathbb{N}^3\text{ 且有相同的奇偶性}}A_{\frac{x + y}{2},\frac{y + z}{2},\frac{z + x}{2}}.
\end{align*}
记\(S = \{ (x,y,z) \in \mathbb{N}^3 : x,y,z\text{ 有相同的奇偶性}\}\),则对\(\forall (x,y,z) \in S\),取\(a = \frac{x + y}{2}\),\(b = \frac{y + z}{2}\),\(c = \frac{z + x}{2}\).此时我们有
\begin{align*}
a + b = \frac{x + 2y + z}{2} > \frac{z + x}{2} = c,
\\
b + c = \frac{x + y + 2z}{2} > \frac{x + y}{2} = a,
\\
a + c = \frac{2x + y + z}{2} > \frac{y + z}{2} = b.
\end{align*}
从而\(a,b,c\)可以构成某个三角形的三边长,即此时\((a,b,c) = (\frac{x + y}{2},\frac{y + z}{2},\frac{z + x}{2}) \in T\).

于是我们可以构造映射
\begin{align*}
\tau : S \to T,(x,y,z) \mapsto (a,b,c) = (\frac{x + y}{2},\frac{y + z}{2},\frac{z + x}{2}).
\end{align*}
反之,对\(\forall (a,b,c) \in T\),取\(x = a + c - b\),\(y = a + b - c\),\(z = b + c - a\).
此时我们有
\begin{align*}
x + y = 2a,y + z = 2b,x + z = 2c.
\end{align*}
从而\(x,y,z\)具有相同的奇偶性,即此时\((x,y,z) = (a + c - b,a + b - c,b + c - a) \in S\).

于是我们可以构造映射
\begin{align*}
\tau': T \to S,(a,b,c) \mapsto (x,y,z) = (a + c - b,a + b - c,b + c - a).
\end{align*}
因此对\(\forall (x,y,z) \in S\),都有
\(\tau\tau'(x,y,z) = \tau'\tau(x,y,z) = (x,y,z)\).
即\(\tau\tau' = I\).故映射\(\tau\)存在逆映射\(\tau'\).从而映射\(\tau\)是双射.

因此集合\(S\)中的每一个元素都能在集合\(T\)中找到与之一一对应的元素.于是两和式\(\sum\limits_{(x,y,z) \in S}A_{\frac{x + y}{2},\frac{y + z}{2},\frac{z + x}{2}}\)和\(\sum\limits_{(a,b,c) \in T}A_{a,b,c}\)的项数一定相同.并且任取\(\sum\limits_{(x,y,z) \in S}A_{\frac{x + y}{2},\frac{y + z}{2},\frac{z + x}{2}}\)中\((x,y,z)\)所对应的一项\(A_{\frac{x + y}{2},\frac{y + z}{2},\frac{z + x}{2}}\),\(\sum\limits_{(a,b,c) \in T}A_{a,b,c}\)中一定存在与之一一对应的\(\tau(x,y,z)\)所对应的一项\(A_{\tau(x,y,z)}\).而\(\tau(x,y,z) = (\frac{x + y}{2},\frac{y + z}{2},\frac{z + x}{2})\),因此\(A_{\tau(x,y,z)} = A_{\frac{x + y}{2},\frac{y + z}{2},\frac{z + x}{2}}\).故\(\sum\limits_{(x,y,z) \in S}A_{\frac{x + y}{2},\frac{y + z}{2},\frac{z + x}{2}} = \sum\limits_{(a,b,c) \in T}A_{a,b,c}\). 
\end{proof}
\begin{remark}
上述证明中逆映射的构造可以通过联立方程$a=\frac{x+y}{2},b=\frac{y+z}{2},c=\frac{z+x}{2}$解出$x=a+c-b,y=a+b-c,z=b+c-a$得到.
\end{remark}

\begin{theorem}[\hypertarget{关于无限和的Fubinin定理}{关于无限和的Fubinin定理}]\label{关于无限和的Fubinin定理}
设 \( f: \mathbb{N} \times \mathbb{N} \to \mathbb{R} \) 是一个使得 \(\sum\limits_{(n,m) \in \mathbb{N} \times \mathbb{N}} f(n,m)\) 绝对收敛的函数.那么

1.\begin{align*}
\sum\limits_{n=0}^{\infty}{\sum\limits_{m=0}^{\infty}{f(n,m)}}=\sum\limits_{m=0}^{\infty}{\sum\limits_{n=0}^{\infty}{f(n,m)}.}
\nonumber
\end{align*}
2.\begin{align*}
\sum\limits_{n=1}^{\infty}{\sum\limits_{m=1}^n{f(n,m)}}=\sum\limits_{m=1}^{\infty}{\sum\limits_{n=m}^{\infty}{f(n,m)}}.
\nonumber
\end{align*}
\end{theorem}
\begin{note}
这个命题是\hyperlink{关于求和号换序的基本结论}{关于求和号换序的基本结论}的推广.
\end{note}
\begin{proof}

\end{proof}

\begin{example}
(PutnamA3)已知\(a_0, a_1, \ldots, a_n, x\)是实数,且\(0 < x < 1\),并且满足
\begin{align*}
\frac{a_0}{1 - x}+\frac{a_1}{1 - x^2}+\cdots+\frac{a_n}{1 - x^{n + 1}} = 0.
\nonumber
\end{align*}
证明:存在一个\(0 < y < 1\),使得
\begin{align*}
a_0 + a_1y + \cdots + a_ny^n = 0.
\nonumber
\end{align*}
\end{example}

\begin{proof}
由题意可知,将$\frac{1}{1-x^{k+1}}\left( k=0,1,\cdots ,n \right)$根据幂级数展开可得
\begin{align*}
\sum\limits_{k=0}^n{\frac{a_k}{1-x^{k+1}}}=\sum\limits_{k=0}^n{a_k\sum\limits_{i=0}^{+\infty}{x^{\left( k+1 \right) i}}}=\sum\limits_{k=0}^n{\sum\limits_{i=0}^{+\infty}{a_kx^{\left( k+1 \right) i}}}.
\nonumber
\end{align*}
又因为$0<x<1$,所以几何级数$\sum\limits_{i=0}^{+\infty}{x^{\left( k+1 \right) i}}$是绝对收敛的.
从而有限个绝对收敛的级数的线性组合$\sum\limits_{k=0}^n{a_k\sum\limits_{i=0}^{+\infty}{x^{\left( k+1 \right) i}}}$也是绝对收敛的.
于是根据\hyperlink{关于无限和的Fubinin定理}{关于无限和的Fubinin定理}可得
\begin{align*}
\sum\limits_{k=0}^n{\frac{a_k}{1-x^{k+1}}}=\sum\limits_{k=0}^n{\sum\limits_{i=0}^{+\infty}{a_kx^{\left( k+1 \right) i}}}=\sum\limits_{i=0}^{+\infty}{\sum\limits_{k=0}^n{a_kx^{\left( k+1 \right) i}}}=\sum\limits_{i=0}^{+\infty}{x^i\sum\limits_{k=0}^n{a_kx^{ki}}}.
\nonumber
\end{align*}
设$f(y)=a_0 + a_1y + \cdots + a_ny^n = 0,y\in(0,1)$,则$f\in \mathbb{C}(0,1)$.
假设对任意的$y\in(0,1)$,有$f(y)\ne0$.
则$f$要么恒为正数,要么恒为负数.
否则,存在$y_1,y_2\in(0,1)$,使得$f(y_1)>0,f(y_2)<0$.那么由连续函数介值定理可知,一定存在$y_0\in(0,1)$,使得$f(y_0)=0$.这与假设矛盾.
因此不失一般性,我们假设$f(y)>0,\forall y\in (0,1)$.又由$0<x<1$可知,$x^i\in(0,1)$.从而
\begin{align*}
\sum\limits_{k=0}^n{\frac{a_k}{1-x^{k+1}}}=\sum\limits_{i=0}^{+\infty}{x^i\sum\limits_{k=0}^n{a_kx^{ki}}=}\sum\limits_{i=0}^{+\infty}{x^if\left( x^i \right)}>0.
\nonumber
\end{align*}
这与题设矛盾.故原结论成立.
\end{proof}

\subsection{裂项求和}

\begin{theorem}[基本结论]\label{theorem:裂项求和基本结论}
(1)当$a,b\in \mathbb{Z}$且$a\leq b$时,有
\begin{gather*}
\sum\limits_{n=a}^b{\left[ f\left( n \right) -f\left( n+1 \right) \right] =f\left( a \right) -f\left( b+1 \right)};
\\
\sum\limits_{n=a}^b{\left[ f\left( n+1 \right) -f\left( n \right) \right] =f\left( b+1 \right) -f\left( a \right)};
\\
\sum\limits_{n=a}^b{\left[ f\left( n \right) -f\left( n-1 \right) \right]}=f\left( b \right) -f\left( a-1 \right) ;
\\
\sum\limits_{n=a}^b{\left[ f\left( n-1 \right) -f\left( n \right) \right]}=f\left( a-1 \right) -f\left( b \right) .
\end{gather*}
(2)当$a,b,m\in \mathbb{Z}$且$a\leq b$时,有
\begin{gather}\label{equation:隔m项的裂项求和式1}
\sum\limits_{n=a}^b{\left[ f\left( n+m \right) -f\left( n \right) \right] =\sum\limits_{n=b+1}^{b+m}{f\left( n \right)}-\sum\limits_{n=a}^{a+m-1}{f\left( n \right)}};
\\\label{equation:隔m项的裂项求和式2}
\sum\limits_{n=a}^b{\left[ f\left( n \right) -f\left( n+m \right) \right] =\sum\limits_{n=a}^{a+m-1}{f\left( n \right)}-\sum\limits_{n=b+1}^{b+m}{f\left( n \right)}}.
\end{gather}
\end{theorem}
\begin{proof}
(1)将求和展开后很容易得到证明.

(2)因为(2)中上下两个式子\eqref{equation:隔m项的裂项求和式1}\eqref{equation:隔m项的裂项求和式2}互为相反数,所以我们只证明\eqref{equation:隔m项的裂项求和式1}即可.

当\(m \geq 0\)时,若\(m \leq b - a\),则
\begin{align*}
&\sum\limits_{n = a}^{b}[f(n + m) - f(n)] 
\\
&= f(a + m) + \cdots + f(b) + f(b + 1) + \cdots + f(b + m) - f(a) - \cdots - f(a + m - 1) - f(a + m) - \cdots - f(b)\\
&= f(b + 1) + \cdots + f(b + m) - f(a) - \cdots - f(a + m - 1)\\
&= \sum\limits_{n = b + 1}^{b + m}f(n) - \sum\limits_{n = a}^{a + m - 1}f(n)
\end{align*}
若\(m > b - a\),则
\begin{align*}
&\sum\limits_{n = b + 1}^{b + m}f(n) - \sum\limits_{n = a}^{a + m - 1}f(n)\\
&= f(b + 1) + \cdots + f(a + m - 1) + f(a + m) + \cdots + f(b + m) - f(a) - \cdots - f(b) - f(b + 1) - \cdots - f(a + m - 1)\\
&= f(a + m) + \cdots + f(b + m) - f(a) - \cdots - f(b)\\
&= \sum\limits_{n = a}^{b}[f(n + m) - f(n)]
\end{align*}
综上,当\(m \geq 0\)时,有\(\sum\limits_{n = a}^{b}[f(n + m) - f(n)] = \sum\limits_{n = b + 1}^{b + m}f(n) - \sum\limits_{n = a}^{a + m - 1}f(n)\).

当\(m < 0\)时,我们有\(-m > 0\),从而
\begin{align*}
&\sum\limits_{n = a}^{b}[f(n + m) - f(n)] = \sum\limits_{n = a + m}^{b + m}[f(n) - f(n - m)]
= -\sum\limits_{n = a + m}^{b + m}[f(n - m) - f(n)]\\
&= -\left(\sum\limits_{n = b + m + 1}^{b + m - m}f(n) - \sum\limits_{n = a + m}^{a + m - m - 1}f(n)\right)
= \sum\limits_{n = a + m}^{a - 1}f(n) - \sum\limits_{n = b + m + 1}^{b}f(n)\\
&\xlongequal{\hyperref[theorem:关于求和号下限大于上限的计算]{\text{求和号下限大于上限}}}\sum\limits_{n = b + 1}^{b + m}f(n) - \sum\limits_{n = a}^{a + m - 1}f(n)
\end{align*}
综上所述,结论得证.
\end{proof}

\begin{example}
1.对$m\in \mathbb{N}$,计算$\sum\limits_{n=1}^m{\left( \sin n^2\cdot \sin n \right)}$.
\quad \quad
2.对$n,m\in \mathbb{N}$,计算$\sum\limits_{k=1}^n{\frac{1}{k\left( k+m \right)}}$.
\end{example}
\begin{solution}
1.\begin{align*}
&\sum\limits_{n=1}^m{\left( \sin n^2\cdot \sin n \right)}\xlongequal{\text{积化和差公式}}-\frac{1}{2}\sum\limits_{n=1}^m{\left[ \cos \left( n^2+n \right) -\cos \left( n^2-n \right) \right]}
\\
&=-\frac{1}{2}\sum\limits_{n=1}^m{\left[ \cos \left( n\left( n+1 \right) \right) -\cos \left( n\left( n-1 \right) \right) \right]}
\\
&=-\frac{1}{2}\left[ \cos \left( m\left( m+1 \right) \right) -1 \right] 
\end{align*}
2.\begin{align*}
&\sum\limits_{k=1}^n{\frac{1}{k\left( k+m \right)}}=\frac{1}{m}\sum\limits_{k=1}^n{\left( \frac{1}{k}-\frac{1}{k+m} \right)}
\\
&=\frac{1}{m}\left( 1+\frac{1}{2}\cdots +\frac{1}{m}-\frac{1}{n+1}-\frac{1}{n+2}-\cdots -\frac{1}{n+m} \right) 
\end{align*}
\end{solution}


\end{document}
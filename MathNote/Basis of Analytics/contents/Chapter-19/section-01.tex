\documentclass[../../main.tex]{subfiles}
\graphicspath{{\subfix{../../image/}}} % 指定图片目录,后续可以直接使用图片文件名。

% 例如:
% \begin{figure}[H]
% \centering
% \includegraphics[scale=0.4]{图.png}
% \caption{}
% \label{figure:图}
% \end{figure}
% 注意:上述\label{}一定要放在\caption{}之后,否则引用图片序号会只会显示??.

\begin{document}

\section{杂题}

\begin{example}
设$Y,x_0,\delta > 0$,计算
\begin{align*}
\lim_{n \to \infty} \sqrt{n} \int_{x_0 - \delta}^{x_0 + \delta} e^{-nY(x - x_0)^2} \,\mathrm{d}x.
\end{align*}
\end{example}
\begin{proof}
\begin{align*}
\underset{n\rightarrow \infty}{\lim}\sqrt{n}\int_{x_0-\delta}^{x_0+\delta}{e^{-nY(x-x_0)^2}\,\mathrm{d}x}&=\underset{n\rightarrow \infty}{\lim}\sqrt{n}\int_{-\delta}^{\delta}{e^{-nYx^2}\,\mathrm{d}x}=\underset{n\rightarrow \infty}{\lim}\frac{1}{\sqrt{Y}}\int_{-\delta \sqrt{nY}}^{\delta \sqrt{nY}}{e^{-x^2}\,\mathrm{d}x}
\\
&=\underset{n\rightarrow \infty}{\lim}\frac{2}{\sqrt{Y}}\int_0^{\delta \sqrt{nY}}{e^{-x^2}\,\mathrm{d}x}=\frac{2}{\sqrt{Y}}\int_0^{+\infty}{e^{-x^2}\,\mathrm{d}x}
\\
&=\sqrt{\frac{\pi}{Y}}.
\end{align*}
\end{proof}

\begin{example}
设$f \in C^3[0,x]$,$x > 0$,证明:存在$\xi \in (0,x)$使得
\begin{align}\label{equation:::::----16856486}
\int_0^x f(t) \,\mathrm{d}t = \frac{x}{2}[f(0) + f(x)] - \frac{x^3}{12}f''(\xi).
\end{align}
若还有$f'''(0) \neq 0$,计算$\lim_{x \to 0^+} \frac{\xi}{x}.$
\end{example}
\begin{note}
我们当然可以直接用\hyperref[proposition:Lagrange插值公式]{Lagrange插值公式}得到
\begin{align*}
f\left( t \right) =\left( f\left( x \right) -f\left( 0 \right) \right) t+f\left( 0 \right) +f'' \left( \xi \right) t\left( t-x \right) ,t\in \left[ 0,x \right] .
\end{align*}
两边同时对$t$在$[0,x]$上积分就能得到\eqref{equation:::::----16856486}式.
\end{note}
\begin{proof}
设$K \in \mathbb{R}$使得
\begin{align*}
\int_0^x f(t) \,\mathrm{d}t = \frac{x}{2}[f(0) + f(x)] - \frac{x^3}{12}K,
\end{align*}
则考虑
\begin{align*}
g(y) \triangleq \int_0^y f(t) \,\mathrm{d}t - \frac{y}{2}[f(0) + f(y)] + \frac{y^3}{12}K,
\end{align*}
于是
\begin{align*}
g'(y) = f(y) - \frac{1}{2}[f(0) + f(y)] - \frac{y f'(y)}{2} + \frac{y^2 K}{4} = \frac{f(y) - f(0)}{2} - \frac{y f'(y)}{2} + \frac{y^2 K}{4}
\end{align*}
以及
\begin{align*}
g''(y) = -\frac{y f''(y)}{2} + \frac{y K}{2}.
\end{align*}
由$g(x) = g(0) = 0$和罗尔中值定理得$\xi_1 \in (0,x)$使得$g'(\xi_1) = 0$. 注意到$g'(0) = 0$. 再次由罗尔中值定理得$\xi \in (0,x)$使得
\begin{align*}
g''(\xi) = -\frac{\xi f''(\xi)}{2} + \frac{\xi K}{2} = 0,
\end{align*}
即$K = f''(\xi)$, 这就得到了\eqref{equation:::::----16856486}式.
由\eqref{equation:::::----16856486}式得
\begin{align*}
f''(\xi) = -12 \frac{\int_0^x f(t) \,\mathrm{d}t - \frac{x}{2}[f(0) + f(x)]}{x^3}
\end{align*}
由Lagrange中值定理得
\begin{align*}
f''(\xi) = f''(0) + f'''(\eta) \xi, \eta \in (0,\xi).
\end{align*}
于是
\begin{align*}
f'''(\eta) \frac{\xi}{x} = \frac{-12 \frac{\int_0^x f(t) \,\mathrm{d}t - \frac{x}{2}[f(0) + f(x)]}{x^3} - f''(0)}{x}
\end{align*}
现在利用L'Hospital法则就有
\begin{align*}
\lim_{x \to 0^+} f'''(\eta) \frac{\xi}{x} &= \lim_{x \to 0^+} \frac{-12 \frac{\int_0^x f(t) \,\mathrm{d}t - \frac{x}{2}[f(0) + f(x)]}{x^3} - f''(0)}{x} \\
&= \lim_{x \to 0^+} \frac{-12 \int_0^x f(t) \,\mathrm{d}t + 6x [f(0) + f(x)] - f''(0) x^3}{x^4} \\
&= \lim_{x \to 0^+} \frac{-12 f(x) + 6 [f(x) + f(0)] + 6x f'(x) - 3 f''(0) x^2}{4x^3} \\
&= \lim_{x \to 0^+} \frac{6x f''(x) - 6 f''(0) x}{12x^2} \\
&= \lim_{x \to 0^+} \frac{f''(x) - f''(0)}{2x} = \frac{1}{2} f'''(0).
\end{align*}
因为$0<\eta<\xi<x$,所以
\begin{align*}
\lim_{x \to 0^+} f'''(\eta) = f'''(0),
\end{align*}
我们有
\begin{align*}
\lim_{x \to 0^+} \frac{\xi}{x} = \frac{1}{2}.
\end{align*}
\end{proof}

\begin{example}
设 \( f \) 是 \( [0,+\infty) \) 上的递增正函数. 若 \( g \in C^2[0,+\infty) \) 满足
\begin{align}
g''(x) + f(x)g(x) &= 0.\label{equation:::--107..8}
\end{align}
证明: 存在 \( M > 0 \) 使得
\begin{align}
|g(x)| &\leqslant M, \quad |g'(x)| \leqslant M\sqrt{f(x)}, \quad \forall x > 0.\label{equation:::--107..9}
\end{align}
\end{example}
\begin{proof}
对$\forall x>0,$有$f$在$[0,x]$上单调递增,从而由\hyperref[theorem:闭区间上单调函数必可积]{闭区间上单调函数必可积}可知$f\in R[0,x],\forall x>0$,$f$在$[0,+\infty)$上内闭连续.
由\eqref{equation:::--107..8}知
\begin{align}
\int_0^x g''(y)g'(y)\,\mathrm{d}y + \int_0^x f(y)g'(y)g(y)\,\mathrm{d}y &= 0, \forall x > 0 \label{eq:486861611651--8234}
\end{align}
利用 \( f \) 递增和\hyperref[theorem:积分中值定理(2)]{第二积分中值定理}和 \(\eqref{eq:486861611651--8234}\),我们有
\[
\int_0^x g''(y)g'(y)\,\mathrm{d}y + f(x)\int_\xi^x g'(y)g(y)\,\mathrm{d}y = 0, \xi \in [0, x].
\]
即
\[
\frac{1}{2}\lvert g'(x)\rvert^2 - \frac{1}{2}\lvert g'(0)\rvert^2 + \frac{[f(x)]^2}{2}\left[g^2(x) - g^2(\xi)\right] = 0.
\]
现在一方面
\begin{align}
\lvert g'(x)\rvert^2 = \lvert g'(0)\rvert^2 - f(x)g^2(x) + f(x)g^2(\xi) \leqslant \lvert g'(0)\rvert^2 + f(x)g^2(\xi).\label{eq:::--123124523-1214} 
\end{align}
另外一方面由\eqref{equation:::--107..8}得
\begin{align*}
\frac{g''(x)g'(x)}{f(x)} + g'(x)g(x) = 0, \forall x > 0. 
\end{align*}
即
\[
\int_0^x \frac{g''(y)g'(y)}{f(y)}\,\mathrm{d}y + \frac{1}{2}g^2(x) - \frac{1}{2}g^2(0) = 0, \forall x > 0
\]
由 \( f \) 递增和\hyperref[theorem:积分中值定理(1)]{第二积分中值定理},我们有
\[
\frac{1}{f(0)}\int_0^\eta g''(y)g'(y)\,\mathrm{d}y + \frac{1}{2}g^2(x) - \frac{1}{2}g^2(0) = 0, \eta \in [0, x]
\]
从而
\[
\frac{1}{2f(0)}\left[\lvert g'(\eta)\rvert^2 - \lvert g'(0)\rvert^2\right] + \frac{1}{2}g^2(x) - \frac{1}{2}g^2(0) = 0
\]
即
\begin{align}
\lvert g(x)\rvert^2 = g^2(0) - \frac{1}{f(0)}\left[\lvert g'(\eta)\rvert^2 - \lvert g'(0)\rvert^2\right] \leqslant g^2(0) + \frac{\lvert g'(0)\rvert^2}{f(0)},\forall x>0.\label{eq:::--123124523-1211}
\end{align}
由 \( g\in C[0,+\infty) \)知$g$ 有界,即存在$C_1>0,$使得$|g(x)|<C_1,\forall x>0.$于是由\eqref{eq:::--123124523-1214}式知
\begin{align}
\left| g' (x) \right|^2\leqslant \left| g' (0) \right|^2+f(x)g^2(\xi )\leqslant \left| g' (0) \right|^2+C_1f(x),\forall x>0.\label{eq::---1--3-1-3-2314--12-3}
\end{align}
又因为$f$是递增正函数,所以$f(x)\geqslant f(0)>0,\forall x>0.$从而存在$C_2>0,$使得
$$
|g'(0)|^2\leqslant C_2f(0)\leqslant f(x),\forall x>0.
$$
于是取$M=\max \left\{ C_1+C_2,g^2(0)+\frac{\left| g' (0) \right|^2}{f(0)} \right\},$则由\eqref{eq::---1--3-1-3-2314--12-3}式和\eqref{eq:::--123124523-1211}式可得,对$\forall x>0,$有
\begin{gather*}
\left| g\left( x \right) \right|^2\leqslant M\leqslant M^2,
\\
\left| g' (x) \right|^2\leqslant C_2f\left( x \right) +C_1f(x)\leqslant Mf\left( x \right) \leqslant M^2f\left( x \right) .
\end{gather*}
进而
\begin{align*}
\left| g\left( x \right) \right|\leqslant M,\left| g'(x) \right|\leqslant M\sqrt{f\left( x \right)},\forall x>0.
\end{align*}
这就证明了\eqref{equation:::--107..9}.
\end{proof}

\begin{example}
设 \( f \in C^2[0,1] \),证明
\begin{enumerate}[(a)]
\item 
\begin{align}
|f'(x)| &\leqslant 4\int_0^1 |f(x)| \mathrm{d}x + \int_0^1 |f''(x)| \mathrm{d}x. \label{eq::::::::::::::7151--------------5}
\end{align}

\item 
\begin{align}
\int_0^1 |f'(x)| \mathrm{d}x &\leqslant 4\int_0^1 |f(x)| \mathrm{d}x + \int_0^1 |f''(x)| \mathrm{d}x .\label{eq::::::::::::::7151--------------6}
\end{align}

\item  若 \( f(0)f(1) \geqslant 0 \),则
\begin{align}
\int_0^1 |f'(x)| \mathrm{d}x &\leqslant 2\int_0^1 |f(x)| \mathrm{d}x + \int_0^1 |f''(x)| \mathrm{d}x. \label{eq::::::::::::::7151--------------7}
\end{align}
\end{enumerate}
\end{example}
\begin{proof}
\begin{enumerate}[(a)]
\item 注意到对任何 \( \theta \in [0,1] \),我们有
\begin{align*}
|f' (x)|&\leqslant |f' (x)-f' (\theta )|+|f' (\theta )|\leqslant \left| \int_{\theta}^x{f'' (y)\mathrm{d}y} \right|+|f' (\theta )|
\\
&\leqslant \int_0^1{|f'' (y)|\mathrm{d}y}+|f' (\theta )|.
\end{align*}
于是只需证明存在 \( \theta \in [0,1] \) 使得
\begin{align}
|f'(\theta)| \leqslant 4 \int_0^1 |f(x)| \mathrm{d}x. \label{eq:10-------:::21161}
\end{align}
如果 \( f' \) 有零点,则显然存在$\theta \in [0,1],$使得$f(\theta)=0$,从而满足\(\eqref{eq:10-------:::21161}\)式.下设 \( f' \) 没有零点.由$f'$的介值性可知,$f'$要么恒正,要么恒负.不妨设 \( f \) 严格递增.
若 \( f \) 没有零点,不妨设 \( f > 0 \),则由Lagrange中值定理可得
\[
f(x) = f(0) + x f'(\eta) \geqslant x f'(\eta) \geqslant x \min_{[0,1]} |f'| \implies \int_0^1 |f(x)| \mathrm{d}x \geqslant \min_{[0,1]} |f'|\geqslant \frac{1}{4} \min_{[0,1]} |f'|,
\]
这也给出了\(\eqref{eq:10-------:::21161}\)式.
若 存在$t \in [0,1]$,使得\( f(t) = 0\).由Lagrange中值定理可知
\begin{align*}
f(x)=f' (\theta )(x-t).
\end{align*}
从而
\begin{align*}
\int_0^1{|f(x)|\mathrm{d}x}\geqslant \min_{[0,1]} |f' |\cdot \int_0^1{|x}-t|\mathrm{d}x\overset{\text{\refpro{proposition:单调函数减任意常数积分小于减其中点的积分}}}{\geqslant}\min_{[0,1]} |f' |\cdot \int_0^1{\left| x-\frac{1}{2} \right|\mathrm{d}x}=\frac{1}{4}\min_{[0,1]} |f' |.
\end{align*}
这也给出了\(\eqref{eq:10-------:::21161}\)式.于是我们证明了不等式\eqref{eq::::::::::::::7151--------------5}式.

\item 直接对\eqref{eq::::::::::::::7151--------------5}式两边关于$x$在$[0,1]$上积分得\eqref{eq::::::::::::::7151--------------6}式.

\item 由(a)同理只需证明存在 \( \theta \in [0,1] \) 使得
\begin{align}
|f'(\theta)| \leqslant 2 \int_0^1 |f(x)| \mathrm{d}x. \label{eq:10-------:::286161}
\end{align}
不妨假定 \( f' \) 没有零点且 \( f(0) \geqslant 0 \),则当 \( f \) 递增,由Lagrange中值定理,我们有
\[
f(x)=f(0)+xf' (\eta )\geqslant xf' (\eta )\geqslant x\cdot \min |f' |\Longrightarrow \int_0^1{|f(x)|\mathrm{d}x}\geqslant \min |f' \geqslant \frac{1}{2}\min |f' |.
\]
当 \( f \) 递减,由Lagrange中值定理,我们有
\[
f(x) = f(1) + (x - 1) f'(\alpha) \geqslant (1 - x) \min |f'| \implies \int_0^1 |f(x)| \mathrm{d}x \geqslant \frac{1}{2} \min |f'|.
\]
于是必有\(\eqref{eq:10-------:::286161}\)式成立,这就给出了\eqref{eq::::::::::::::7151--------------7}式.
\end{enumerate}
\end{proof}

\begin{example}
设函数$f(x)$在$(a,+\infty)$上严格单调下降,证明:若$\lim\limits_{n \to \infty} f(x_n)=\lim\limits_{x \to +\infty} f(x)$,则$\lim\limits_{n \to \infty} x_n=+\infty$. 
\end{example}
\begin{proof}
反证,假设$\varliminf_{n\rightarrow \infty}x_n=c\in \left( a,+\infty \right)$,则存在子列$\{ x_{n_k} \}$,满足$x_{n_k}\rightarrow c$。记
\begin{align*}
\lim_{n\rightarrow \infty}f\left( x_n \right) =\lim_{x\rightarrow +\infty}f\left( x \right) =A,
\end{align*}
则$f\left( x_n \right)$的子列极限也收敛到$A$,即$\lim_{k\rightarrow \infty}f\left( x_{n_k} \right) =A$。由$x_{n_k}\rightarrow c$知,存在$K\in \mathbb{N}$,使得
\begin{align*}
x_{n_k}\in \left( c-\delta ,c+\delta \right),\forall k>K.
\end{align*}
其中$\delta =\min \left\{ \frac{c-a}{2},\frac{1}{2} \right\}$。任取$x_1,x_2\in \left( c+\delta ,+\infty \right)$且$x_1<x_2$,则由$f$严格递减知
\begin{align*}
f\left( x_{n_k} \right) >f\left( x_1 \right) >f\left( x_2 \right) >f\left( x \right),\forall x>x_2,\forall k>K.
\end{align*}
左边令$k\rightarrow +\infty$,右边令$x\rightarrow +\infty$得
\begin{align*}
A=\lim_{k\rightarrow \infty}f\left( x_{n_k} \right) \geqslant f\left( x_1 \right) >f\left( x_2 \right) \geqslant \lim_{x\rightarrow +\infty}f\left( x \right) =A,
\end{align*}
显然矛盾!
\end{proof}

\begin{example}
设 $\{x_n\} \subset (0,1)$ 满足对 $i \neq j$, 有 $x_i \neq x_j$, 讨论函数 $f(x) = \sum_{n=1}^\infty \frac{\operatorname{sgn}(x - x_n)}{2^n}$ 连续性.
\end{example}
\begin{proof}
由
\[
\sum_{n=1}^\infty \left| \frac{\operatorname{sgn}(x - x_n)}{2^n} \right| \leqslant \sum_{n=1}^\infty \frac{1}{2^n} < \infty,
\]
故级数一致收敛. 注意到对$\forall n\in \mathbb{N} $,都有$\mathrm{sgn} \left( x-x_n \right) $在$x=x_n$处间断,在$x\ne x_n$处连续.

当$x\ne x_k,\forall k\in \mathbb{N} $时,$f\left( x \right) $的每一项都连续.又$f\left( x \right) $一致收敛,故$f$在$x\ne x_k,\forall k\in \mathbb{N} $处都连续.

当$x=x_k,\forall k\in \mathbb{N} $时,有
\begin{align*}
f\left( x \right) =\frac{\mathrm{sgn} \left( x-x_k \right)}{2^k}+\sum_{n\ne k}{\frac{\mathrm{sgn} \left( x-x_n \right)}{2^n}}
\end{align*}
在$x=x_k$处间断.故$f\left( x \right) $在$x=x_k,\forall k\in \mathbb{N} $处都间断.
\end{proof}

\begin{example}
证明 $\sum_{t=1}^{\infty} (-1)^t \frac{t}{t^2 + x}$ 在 $x \in [0, +\infty)$ 一致收敛性.
\end{example}
\begin{proof}
由\hyperref[theorem:Abel变换]{Abel变换}得,对$\forall m \in \mathbb{N},\forall x\geqslant 0$成立
\begin{align*}
\sum_{t=m}^{\infty} (-1)^t \frac{t}{t^2 + x} &= \lim_{n \to \infty} \sum_{t=m}^n (-1)^t \frac{t}{t^2 + x} \\
&= \lim_{n \to \infty} \left[ \sum_{t=m}^{n-1} \left( \frac{t}{t^2 + x} - \frac{t + 1}{(t + 1)^2 + x} \right) s_t + \frac{n}{n^2 + x} s_n \right] \\
&= \sum_{t=m}^{\infty} \left( \frac{t}{t^2 + x} - \frac{t + 1}{(t + 1)^2 + x} \right) s_t \\
&= \sum_{t=m}^{\infty} \frac{t^2 + t}{(x + t^2)(x + t^2 + 2t + 1)} s_t - \sum_{t=m}^{\infty} \frac{x}{(x + t^2)(x + t^2 + 2t + 1)} s_t,
\end{align*}
这里 $s_t =\sum_{i=1}^t{\left( -1 \right) ^i}=\left( -1 \right) ^t \in \{1, -1\}$.
一方面
\[
\left| \sum_{t=m}^{\infty} \frac{t^2 + t}{(x + t^2)(x + t^2 + 2t + 1)} s_t \right| \leqslant \sum_{t=m}^{\infty} \frac{t^2 + t}{t^2(t^2 + 2t + 1)},
\]
另外一方面
\[
\left| \sum_{t=m}^{\infty} \frac{x}{(x + t^2)(x + t^2 + 2t + 1)} s_t \right| \leqslant \sum_{t=m}^{\infty} \frac{1}{t^2 + t + 1}.
\]
而由$\sum_{t=1}^{\infty} \frac{t^2 + t}{t^2(t^2 + 2t + 1)}$和$\sum_{t=1}^{\infty} \frac{1}{t^2 + t + 1}$都收敛知
\begin{align*}
\underset{m\rightarrow \infty}{\lim}\sum_{t=m}^{\infty}{\frac{1}{t^2+t+1}}=\underset{m\rightarrow \infty}{\lim}\sum_{t=m}^{\infty}{\frac{t^2+t}{t^2(t^2+2t+1)}}=0.
\end{align*}
于是我们有
\[
\lim_{m \to \infty} \sum_{t=m}^{\infty} (-1)^t \frac{t}{t^2 + x} = 0, \text{关于} x \in [0, +\infty) \text{一致},
\]
这就证明了 $\sum\limits_{t=1}^{\infty} (-1)^t \frac{t}{t^2 + x}$ 在 $x \in [0, +\infty)$ 一致收敛.
\end{proof}

\begin{proposition}\label{proposition:右导数小于等于0函数值小于端点值}
设\( f(x) \)是\([a,b]\)上连续实值右可导函数,记\( D^+f(x) \)为\( f(x) \)的右导函数,如果\( f(a)=0 \),且\( D^+f(x) \leqslant0 \),则\( f(x) \leqslant0, x \in [a,b] \)。
\end{proposition}
\begin{proof}
(1) 先假定\( D^+f(x) < 0 \),如果结论不成立,则存在\( x_1 \in (a,b) \),使\( f(x_1) > 0 \)。
记
\[
x_0 = \inf\{ x \mid f(x) > 0 \}.
\]
由\( x_0 \)的定义,我们有序列\(\{ x_n \}\),使\( x_n \)单调递减趋于\( x_0 \),且\( f(x_n) > 0 \)。从而由\( f(x) \)的连续性知
\begin{align}
f\left( x_0 \right)=\underset{n\rightarrow \infty}{\lim}f\left( x_n \right) \geqslant 0.\label{eq:::::-----2348001456564156}
\end{align}
根据$x_0$的定义可知,对$\forall x<x_0,$都有$f(x)<f(x_0)$,否则与下确界定义矛盾!于是有序列$\{x'_n\}$单调递增趋于$x_0$,且$f(x'_n).$于是由\( f(x) \)的连续性知
\begin{align}
f\left( x_0 \right)=\underset{n\rightarrow \infty}{\lim}f\left( x'_n \right)  \leqslant 0.\label{eq:::::-----23480091456564156}
\end{align}
故由\eqref{eq:::::-----2348001456564156}\eqref{eq:::::-----23480091456564156}知\( f(x_0)=0 \)。于是
\[
D^+f(x_0) = \lim_{n \to \infty} \frac{f(x_n) - f(x_0)}{x_n - x_0} \geqslant 0,
\]
这与\( D^+f(x_0) < 0 \)矛盾,于是\( f(x) \leqslant 0, x \in [a,b] \)。

(2) 若\( D^+f(x) \leqslant 0 \),对任给的\( \varepsilon > 0 \)构造函数
\[
f_\varepsilon(x) = f(x) - \varepsilon(x - a),
\]
对\( f_\varepsilon(x) \)有\( f_\varepsilon(a)=0 \)且
\[
D^+f_\varepsilon(x) \leqslant-\varepsilon < 0.
\]
从而由(1)得\( f_\varepsilon(x) \leqslant0, x \in [a,b] \)。因此\( f(x) \leqslant\varepsilon(x - a)\leqslant \varepsilon (b-a) \),由\( \varepsilon \)的任意性,得\( f(x) \leqslant0, x \in [a,b] \).
\end{proof}

\begin{example}
设\( \varphi(x) \)是\([a,b)\)上连续且右可导的函数,如果\( D^+\varphi(x) \)在\([a,b)\)上连续,证明:\( \varphi(x) \)在\([a,b)\)上连续可导,\( \varphi'(x)=D^+\varphi(x) \)。
\end{example}
\begin{proof}
设
\[
f(x) = \varphi(a) + \int_a^x D^+\varphi(t)\mathrm{d}t - \varphi(x), \quad x \in [a,b).
\]
则\( f(x) \)在\([a,b)\)上连续且右可导,并且
\[
D^+f(x) = D^+\varphi(x) - D^+\varphi(x) = 0.
\]
又\( f(a)=0 \),由\refpro{proposition:右导数小于等于0函数值小于端点值}得\( f(x) \leqslant0 \)。
又\(-f(x)\)满足\(-f(a)=0, D^+[-f(x)] = 0\),同理由\refpro{proposition:右导数小于等于0函数值小于端点值}得\(-f(x) \leqslant0\) ,故\(f(x)=0\)。于是
\[
\varphi(x) = \varphi(a) + \int_a^x D^+\varphi(t)\mathrm{d}t.
\]
由\( D^+\varphi(x) \)的连续性,得\( \varphi'(x)=D^+\varphi(x) \)。
\end{proof}

\begin{example}
证明:
\begin{align*}
\sum_{k=1}^{n-1}{\frac{1}{\sin \frac{k\pi}{n}}}=\frac{2n}{\pi}\left( \ln 2n+\gamma -\ln \pi \right) +o\left( 1 \right).
\end{align*}
\end{example}
\begin{proof}
见\href{https://math.stackexchange.com/questions/3216414/sum-of-reciprocal-sine-function-sum-limits-k-1n-1-frac1-sin-frack-p?noredirect=1}{here}.
\end{proof}

\begin{example}
$\lim_{n\rightarrow \infty} \frac{\sum\limits_{k=1}^n{\left( -1 \right) ^k\mathrm{C}_{n}^{k}\ln k}}{\ln \left( \ln n \right)}=1.$
\end{example}
\begin{proof}
{\color{blue}证法一:}对任意充分大的$n$,由\hyperref[theorem:Frullani(傅汝兰尼)积分]{Frullani(傅汝兰尼)积分}知
\begin{align*}
\ln k = \int_0^{+\infty}{\frac{e^{-x}-e^{kx}}{x}\mathrm{d}x}.
\end{align*}
再结合二项式定理可得
\begin{align*}
A&\triangleq \sum_{k=1}^n{\left( -1 \right) ^k\mathrm{C}_{n}^{k}\ln k}=\sum\limits_{k=1}^n{\left[ \left( -1 \right) ^k\mathrm{C}_{n}^{k}\left( \int_0^{+\infty}{\frac{e^{-x}-e^{-kx}}{x}\mathrm{d}x} \right) \right]}=\int_0^{+\infty}{\frac{\sum\limits_{k=1}^n{\left( -1 \right) ^k\mathrm{C}_{n}^{k}}\left( e^{-x}-e^{-kx} \right)}{x}\mathrm{d}x} \\
&=\int_0^{+\infty}{\frac{\sum\limits_{k=1}^n{\left( -1 \right) ^k\mathrm{C}_{n}^{k}}\left( e^{-x}-e^{-kx} \right)}{x}\mathrm{d}x}=\int_0^{+\infty}{\frac{1-e^{-x}+\sum\limits_{k=0}^n{\left( -1 \right) ^k\mathrm{C}_{n}^{k}}\left( e^{-x}-e^{-kx} \right)}{x}\mathrm{d}x} \\
&=\int_0^{+\infty}{\frac{1-e^{-x}+e^{-x}\sum\limits_{k=0}^n{\left( -1 \right) ^k\mathrm{C}_{n}^{k}}-\sum\limits_{k=0}^n{\left( -1 \right) ^k\mathrm{C}_{n}^{k}}e^{-kx}}{x}\mathrm{d}x}=\int_0^{+\infty}{\frac{1-e^{-x}+e^{-x}\left( 1-1 \right) ^n-\left( 1-e^{-x} \right) ^n}{x}\mathrm{d}x} \\
&=\int_0^{+\infty}{\frac{1-e^{-x}-\left( 1-e^{-x} \right) ^n}{x}\mathrm{d}x}.
\end{align*}
由\hyperref[theorem:Bernoulli不等式]{Bernoulli不等式}知
\begin{align*}
\left( 1-e^{-x} \right) ^n\geqslant 1-ne^{-x}.
\end{align*}
取$M_n>1$,满足$M_ne^{M_n}=n$.于是
\begin{align*}
0&\leqslant \int_{M_n}^{+\infty}{\frac{1-e^{-x}-\left( 1-e^{-x} \right) ^n}{x}\mathrm{d}x}\leqslant \int_{M_n}^{+\infty}{\frac{1-e^{-x}-\left( 1-ne^{-x} \right)}{M_n}\mathrm{d}x}=\frac{n}{M_n}\int_{M_n}^{+\infty}{e^{-x}\mathrm{d}x}=\frac{n}{M_ne^{M_n}}=1.
\end{align*}
从而
\begin{align}
A&=\int_0^{M_n}{\frac{1-e^{-x}-\left( 1-e^{-x} \right) ^n}{x}\mathrm{d}x}+\int_{M_n}^{+\infty}{\frac{1-e^{-x}-\left( 1-e^{-x} \right) ^n}{x}\mathrm{d}x}=\int_0^{M_n}{\frac{1-e^{-x}-\left( 1-e^{-x} \right) ^n}{x}\mathrm{d}x}+O\left( 1 \right) .\label{eq:109.10009}
\end{align}
因为$M_ne^{M_n}=n$,所以由\refpro{proposition:Lampert W 的渐进估计}知
\begin{align}
M_n&=\ln n+o\left( \ln n \right) ,n\rightarrow \infty .\label{eq:109.71}
\end{align}
于是
\begin{align*}
\left( 1-e^{-x} \right) ^{n-1}&=e^{\left( n-1 \right) \ln \left( 1-e^{-x} \right)}\leqslant e^{-\left( n-1 \right) e^{-x}}\leqslant e^{-\left( n-1 \right) e^{-M_n}}=e^{-\frac{M_n\left( n-1 \right)}{n}}\rightarrow 0,\forall x\in \left[ 0,M_n \right] .
\end{align*}
从而
\begin{align*}
\frac{\int_0^{M_n}{\frac{\left( 1-e^{-x} \right) ^n}{x}\mathrm{d}x}}{\int_0^{M_n}{\frac{1-e^{-x}}{x}\mathrm{d}x}}&\leqslant \frac{e^{-\frac{M_n\left( n-1 \right)}{n}}\int_0^{M_n}{\frac{1-e^{-x}}{x}\mathrm{d}x}}{\int_0^{M_n}{\frac{1-e^{-x}}{x}\mathrm{d}x}}=e^{-\frac{M_n\left( n-1 \right)}{n}}\rightarrow 0,n\rightarrow \infty .
\end{align*}
即$\int_0^{M_n}{\frac{\left( 1-e^{-x} \right) ^n}{x}\mathrm{d}x}=o\left( \int_0^{M_n}{\frac{1-e^{-x}}{x}\mathrm{d}x} \right) ,n\rightarrow \infty .$故
\begin{align}
\int_0^{M_n}{\frac{1-e^{-x}-\left( 1-e^{-x} \right) ^n}{x}\mathrm{d}x}&=\int_0^{M_n}{\frac{1-e^{-x}}{x}\mathrm{d}x}-\int_0^{M_n}{\frac{\left( 1-e^{-x} \right) ^n}{x}\mathrm{d}x}=\left( 1+o\left( 1 \right) \right) \int_0^{M_n}{\frac{1-e^{-x}}{x}\mathrm{d}x},n\rightarrow \infty .\label{eq:108.1001}
\end{align}
注意到
\begin{align*}
\lim_{x\rightarrow 0}\frac{1-e^{-x}}{x}\xlongequal{\text{L'Hospital}}\lim_{x\rightarrow 0}e^x=1,
\end{align*}
故$\frac{1-e^{-x}}{x}$在$\left[ 0,1 \right]$上有界,进而$\int_0^1{\frac{1-e^{-x}}{x}\mathrm{d}x}=O(1)$.又注意到
\begin{align*}
\int_1^{M_n}{\frac{-e^{-x}}{x}\mathrm{d}x}\leqslant -e^{-M_n}\int_1^{M_n}{\frac{1}{x}\mathrm{d}x}\rightarrow 0,n\rightarrow \infty ,
\end{align*}
故$\int_1^{M_n}{\frac{-e^{-x}}{x}\mathrm{d}x}=O(1)$.于是再结合\eqref{eq:109.71}式可知
\begin{align*}
\int_0^{M_n}{\frac{1-e^{-x}}{x}\mathrm{d}x}&=\int_0^1{\frac{1-e^{-x}}{x}\mathrm{d}x}+\int_1^{M_n}{\frac{-e^{-x}}{x}\mathrm{d}x}+\int_1^{M_n}{\frac{1}{x}\mathrm{d}x} \\
&=O\left( 1 \right) +\ln M_n=\ln \left( \ln n+o\left( \ln n \right) \right) +O\left( 1 \right) \\
&=\ln\ln n+o\left( 1 \right) +O\left( 1 \right) =\ln\ln n+O\left( 1 \right) ,n\rightarrow \infty .
\end{align*}
因此再由\eqref{eq:108.1001}式可知
\begin{align*}
\int_0^{M_n}{\frac{1-e^{-x}-\left( 1-e^{-x} \right) ^n}{x}\mathrm{d}x}&=\left( 1+o\left( 1 \right) \right) \int_0^{M_n}{\frac{1-e^{-x}}{x}\mathrm{d}x}=\left( 1+o\left( 1 \right) \right) \left( \ln\ln n+O\left( 1 \right) \right) =\ln\ln n+o\left( \ln\ln n \right) ,n\rightarrow \infty .
\end{align*}
故由\eqref{eq:109.10009}可得
\begin{align*}
\lim_{n\rightarrow \infty} \frac{\sum\limits_{k=1}^n{\left( -1 \right) ^k\mathrm{C}_{n}^{k}\ln k}}{\ln \left( \ln n \right)}&=\lim_{n\rightarrow \infty} \frac{A}{\ln \left( \ln n \right)}=\lim_{n\rightarrow \infty} \frac{\int_0^{M_n}{\frac{1-e^{-x}-\left( 1-e^{-x} \right) ^n}{x}\mathrm{d}x}+O\left( 1 \right)}{\ln \left( \ln n \right)} \\
&=\lim_{n\rightarrow \infty} \frac{\ln\ln n+o\left( \ln\ln n \right) +O\left( 1 \right)}{\ln \left( \ln n \right)}=1.
\end{align*}

{\color{blue}证法二:}注意到
\begin{align*}
S\triangleq \sum_{k=1}^n{(-1) ^k\binom{n}{k}\ln k}&=\sum_{k=1}^n{(-1) ^k\left[\binom{n-1}{k}+\binom{n-1}{k-1}\right]\ln k}
\\
&=\sum_{k=1}^n{(-1) ^k\binom{n-1}{k}\ln k}+\sum_{k=1}^n{(-1) ^k\binom{n-1}{k-1}\ln k}
\\
&=\sum_{k=1}^{n-1}{(-1) ^k\binom{n-1}{k}\ln k}+\sum_{k=0}^{n-1}{(-1) ^{k+1}\binom{n-1}{k}\ln (k+1)}
\\
&=\sum_{k=1}^{n-1}{(-1) ^k\binom{n-1}{k}\ln k}+\sum_{k=1}^{n-1}{(-1) ^{k+1}\binom{n-1}{k}\ln (k+1)}
\\
&=-\sum_{k=1}^{n-1}{(-1) ^k\binom{n-1}{k}\left(\ln (k+1)-\ln k\right)}
\\
&=-\sum_{k=1}^{n-1}{(-1) ^k\binom{n-1}{k}\int_0^1{\frac{1}{k+x}\mathrm{d}x}}.
\end{align*}
又由二项式定理可知
\begin{align*}
\sum_{k=1}^{n-1}{(-1) ^k\binom{n-1}{k}\frac{1}{k+y}}&=\sum_{k=1}^{n-1}{(-1) ^k\binom{n-1}{k}\int_0^1{t^{k+y-1}\mathrm{d}t}}=\int_0^1{\sum_{k=1}^{n-1}{(-1) ^k\binom{n-1}{k}}t^{k+y-1}\mathrm{d}t}
\\
&=\int_0^1{t^{y-1}\sum_{k=1}^{n-1}{(-1) ^k\binom{n-1}{k}}t^k\mathrm{d}t}=\int_0^1{t^{y-1}\left[(1-t) ^{n-1}-1\right] \mathrm{d}t}.
\end{align*}
故
\begin{align*}
S&=-\int_0^1{\sum_{k=1}^{n-1}{(-1) ^k\binom{n-1}{k}\frac{1}{k+y}}\mathrm{d}y}=\int_0^1{\int_0^1{t^{y-1}\left[1-(1-t) ^{n-1}\right] \mathrm{d}t}\mathrm{d}y}
\\
&=\int_0^1{\int_0^1{t^{y-1}\left[1-(1-t) ^{n-1}\right] \mathrm{d}y}\mathrm{d}t}=\int_0^1{\frac{t-1}{t\ln t}\left[1-(1-t) ^{n-1}\right] \mathrm{d}t}
\\
&\xlongequal{t=e^{-x}}\int_0^{+\infty}{\frac{(1-e^{-x})\left[ 1-(1-e^{-x})^{n-1} \right]}{x}\mathrm{d}x}.
\end{align*}
后续估阶与证法一相同.

{\color{blue}证法三:}注意到
\begin{align*}
S\triangleq \sum_{k=1}^n{(-1) ^k\binom{n}{k}\ln k}&=\sum_{k=1}^n{(-1) ^k\left[\binom{n-1}{k}+\binom{n-1}{k-1}\right]\ln k}
\\
&=\sum_{k=1}^n{(-1) ^k\binom{n-1}{k}\ln k}+\sum_{k=1}^n{(-1) ^k\binom{n-1}{k-1}\ln k}
\\
&=\sum_{k=1}^{n-1}{(-1) ^k\binom{n-1}{k}\ln k}+\sum_{k=0}^{n-1}{(-1) ^{k+1}\binom{n-1}{k}\ln (k+1)}
\\
&=\sum_{k=1}^{n-1}{(-1) ^k\binom{n-1}{k}\ln k}+\sum_{k=1}^{n-1}{(-1) ^{k+1}\binom{n-1}{k}\ln (k+1)}
\\
&=-\sum_{k=1}^{n-1}{(-1) ^k\binom{n-1}{k}\left(\ln (k+1)-\ln k\right)}
\\
&=-\sum_{k=1}^{n-1}{(-1) ^k\binom{n-1}{k}\int_0^1{\frac{1}{k+x}\mathrm{d}x}}
\\
&=-\int_0^1{\sum_{k=1}^{n-1}{(-1) ^k\binom{n-1}{k}\frac{1}{k+x}}\mathrm{d}x}
\\
&=\int_0^1{\left( \frac{1}{x}-\sum_{k=0}^{n-1}{(-1) ^k\binom{n-1}{k}\frac{1}{k+x}} \right) \mathrm{d}x}
\\
&\xlongequal{\text{\refpro{proposition:组合数相关常用恒等式}}}\int_0^1{\left( \frac{1}{x}-\frac{(n-1)!}{x(x+1)\cdots(x+(n-1))} \right) \mathrm{d}x}
\\
&=\int_0^1{\frac{1}{x}\left( 1-\frac{(n-1)!}{(x+1)(x+2)\cdots(x+(n-1))} \right) \mathrm{d}x}
\\
&=\int_0^1{\frac{1}{x}\left( 1-\frac{1}{(1+x)\left(1+\frac{x}{2}\right)\cdots\left(1+\frac{x}{n-1}\right)} \right) \mathrm{d}x}.
\end{align*}
由\nrefpro{proposition:常用不等式4}{(4)}知
$$e^{x^2-x}\geqslant \frac{1}{1+x}\geqslant e^{-x},\forall x>0.$$
于是
$$e^{x^2-x}\cdot e^{\left( \frac{x}{2} \right) ^2-\frac{x}{2}}\cdots e^{\left( \frac{x}{n-1} \right) ^2-\frac{x}{n-1}}\geqslant \frac{1}{(1+x)\left(1+\frac{x}{2}\right)\cdots\left(1+\frac{x}{n-1}\right)}\geqslant e^{-x}\cdot e^{-\frac{x}{2}}\cdots e^{-\frac{x}{n-1}},$$
即
$$e^{x^2\left( 1+\frac{1}{2^2}+\cdots +\frac{1}{(n-1)^2} \right) -x\left( 1+\frac{1}{2}+\cdots +\frac{1}{n-1} \right)}\geqslant \frac{1}{(1+x)\left(1+\frac{x}{2}\right)\cdots\left(1+\frac{x}{n-1}\right)}\geqslant e^{-x\left( 1+\frac{1}{2}+\cdots +\frac{1}{n-1} \right)}.$$
注意到
$$x^2\left( 1+\frac{1}{2^2}+\cdots +\frac{1}{(n-1)^2} \right) \leqslant x\sum_{k=1}^{\infty}{\frac{1}{k^2}}=\frac{\pi ^2}{6}x<2x,\forall x\in [0,1],$$
故
$$e^{-x\left( -2+\sum\limits_{j=1}^{n-1}{\frac{1}{j}} \right)}\geqslant \frac{1}{(1+x)\left(1+\frac{x}{2}\right)\cdots\left(1+\frac{x}{n-1}\right)}\geqslant e^{-x\sum\limits_{j=1}^{n-1}{\frac{1}{j}}}.$$
从而由连续函数$e^{-x}$的介值性知,存在$C_n\in \left[ -2+\sum_{j=1}^{n-1}{\frac{1}{j}},\sum_{j=1}^{n-1}{\frac{1}{j}} \right]$,使得
$$\frac{1}{(1+x)\left(1+\frac{x}{2}\right)\cdots\left(1+\frac{x}{n-1}\right)}=e^{-C_nx}.$$
于是由$-2+\sum_{j=1}^{n-1}{\frac{1}{j}}\leqslant C_n\leqslant \sum_{j=1}^{n-1}{\frac{1}{j}}$知
$$C_n=\ln n+O(1),n\rightarrow \infty.$$
因此
\begin{align*}
S&=\int_0^1{\frac{1}{x}\left( 1-\frac{1}{(1+x)\left(1+\frac{x}{2}\right)\cdots\left(1+\frac{x}{n-1}\right)} \right) \mathrm{d}x}=\int_0^1{\frac{1}{x}\left( 1-e^{-C_nx} \right) \mathrm{d}x}
\\
&=\int_0^{C_n}{\frac{1-e^{-t}}{t}\mathrm{d}t}=\int_0^1{\frac{1-e^{-t}}{t}\mathrm{d}t}+\int_1^{C_n}{\frac{1-e^{-t}}{t}\mathrm{d}t}+\int_1^{C_n}{\frac{1}{t}\mathrm{d}t}.
\end{align*}
注意到
$$\lim_{t\rightarrow 0}\frac{1-e^{-t}}{t}\xlongequal{\text{L'Hospital}}\lim_{t\rightarrow 0}e^t=1,$$
故$\frac{1-e^{-t}}{t}$在$[0,1]$上有界,进而$\int_0^1{\frac{1-e^{-t}}{t}\mathrm{d}t}=O(1)$.
又注意到
$$\int_1^{C_n}{\frac{1-e^{-t}}{t}\mathrm{d}t}\leqslant 1-e^{-C_n}=1-e^{-\ln n+O(1)}\rightarrow 1,n\rightarrow \infty,$$
故$\int_1^{C_n}{\frac{1-e^{-t}}{t}\mathrm{d}t}=O(1)$.从而
\begin{align*}
S&=\int_0^1{\frac{1-e^{-t}}{t}\mathrm{d}t}+\int_1^{C_n}{\frac{1-e^{-t}}{t}\mathrm{d}t}+\int_1^{C_n}{\frac{1}{t}\mathrm{d}t}=\ln C_n+O(1)
\\
&=\ln \left( \ln n+O(1) \right) +O(1)=\ln\ln n+O(1),n\rightarrow \infty.
\end{align*}
因此
\begin{align*}
\lim_{n\rightarrow \infty}\frac{\sum\limits_{k=1}^n{(-1) ^k\binom{n}{k}\ln k}}{\ln\ln n}&=\lim_{n\rightarrow \infty}\frac{S}{\ln\ln n}=\lim_{n\rightarrow \infty}\frac{\ln\ln n+O(1)}{\ln\ln n}=1.
\end{align*}
\end{proof}



















\end{document}
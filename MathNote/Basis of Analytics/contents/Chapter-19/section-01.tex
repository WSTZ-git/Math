\documentclass[../../main.tex]{subfiles}
\graphicspath{{\subfix{../../image/}}} % 指定图片目录,后续可以直接使用图片文件名。

% 例如:
% \begin{figure}[H]
% \centering
% \includegraphics[scale=0.4]{图.png}
% \caption{}
% \label{figure:图}
% \end{figure}
% 注意:上述\label{}一定要放在\caption{}之后,否则引用图片序号会只会显示??.

\begin{document}

\section{杂题}

\begin{example}
设$Y,x_0,\delta > 0$,计算
\begin{align*}
\lim_{n \to \infty} \sqrt{n} \int_{x_0 - \delta}^{x_0 + \delta} e^{-nY(x - x_0)^2} \,\mathrm{d}x.
\end{align*}
\end{example}
\begin{proof}
\begin{align*}
\underset{n\rightarrow \infty}{\lim}\sqrt{n}\int_{x_0-\delta}^{x_0+\delta}{e^{-nY(x-x_0)^2}\,\mathrm{d}x}&=\underset{n\rightarrow \infty}{\lim}\sqrt{n}\int_{-\delta}^{\delta}{e^{-nYx^2}\,\mathrm{d}x}=\underset{n\rightarrow \infty}{\lim}\frac{1}{\sqrt{Y}}\int_{-\delta \sqrt{nY}}^{\delta \sqrt{nY}}{e^{-x^2}\,\mathrm{d}x}
\\
&=\underset{n\rightarrow \infty}{\lim}\frac{2}{\sqrt{Y}}\int_0^{\delta \sqrt{nY}}{e^{-x^2}\,\mathrm{d}x}=\frac{2}{\sqrt{Y}}\int_0^{+\infty}{e^{-x^2}\,\mathrm{d}x}
\\
&=\sqrt{\frac{\pi}{Y}}.
\end{align*}
\end{proof}

\begin{example}
设$f \in C^3[0,x]$,$x > 0$,证明:存在$\xi \in (0,x)$使得
\begin{align}\label{equation:::::----16856486}
\int_0^x f(t) \,\mathrm{d}t = \frac{x}{2}[f(0) + f(x)] - \frac{x^3}{12}f''(\xi).
\end{align}
若还有$f'''(0) \neq 0$,计算$\lim_{x \to 0^+} \frac{\xi}{x}.$
\end{example}
\begin{note}
我们当然可以直接用\hyperref[proposition:Lagrange插值公式]{Lagrange插值公式}得到
\begin{align*}
f\left( t \right) =\left( f\left( x \right) -f\left( 0 \right) \right) t+f\left( 0 \right) +f'' \left( \xi \right) t\left( t-x \right) ,t\in \left[ 0,x \right] .
\end{align*}
两边同时对$t$在$[0,x]$上积分就能得到\eqref{equation:::::----16856486}式.
\end{note}
\begin{proof}
设$K \in \mathbb{R}$使得
\begin{align*}
\int_0^x f(t) \,\mathrm{d}t = \frac{x}{2}[f(0) + f(x)] - \frac{x^3}{12}K,
\end{align*}
则考虑
\begin{align*}
g(y) \triangleq \int_0^y f(t) \,\mathrm{d}t - \frac{y}{2}[f(0) + f(y)] + \frac{y^3}{12}K,
\end{align*}
于是
\begin{align*}
g'(y) = f(y) - \frac{1}{2}[f(0) + f(y)] - \frac{y f'(y)}{2} + \frac{y^2 K}{4} = \frac{f(y) - f(0)}{2} - \frac{y f'(y)}{2} + \frac{y^2 K}{4}
\end{align*}
以及
\begin{align*}
g''(y) = -\frac{y f''(y)}{2} + \frac{y K}{2}.
\end{align*}
由$g(x) = g(0) = 0$和罗尔中值定理得$\xi_1 \in (0,x)$使得$g'(\xi_1) = 0$. 注意到$g'(0) = 0$. 再次由罗尔中值定理得$\xi \in (0,x)$使得
\begin{align*}
g''(\xi) = -\frac{\xi f''(\xi)}{2} + \frac{\xi K}{2} = 0,
\end{align*}
即$K = f''(\xi)$, 这就得到了\eqref{equation:::::----16856486}式.
由\eqref{equation:::::----16856486}式得
\begin{align*}
f''(\xi) = -12 \frac{\int_0^x f(t) \,\mathrm{d}t - \frac{x}{2}[f(0) + f(x)]}{x^3}
\end{align*}
由Lagrange中值定理得
\begin{align*}
f''(\xi) = f''(0) + f'''(\eta) \xi, \eta \in (0,\xi).
\end{align*}
于是
\begin{align*}
f'''(\eta) \frac{\xi}{x} = \frac{-12 \frac{\int_0^x f(t) \,\mathrm{d}t - \frac{x}{2}[f(0) + f(x)]}{x^3} - f''(0)}{x}
\end{align*}
现在利用L'Hospital法则就有
\begin{align*}
\lim_{x \to 0^+} f'''(\eta) \frac{\xi}{x} &= \lim_{x \to 0^+} \frac{-12 \frac{\int_0^x f(t) \,\mathrm{d}t - \frac{x}{2}[f(0) + f(x)]}{x^3} - f''(0)}{x} \\
&= \lim_{x \to 0^+} \frac{-12 \int_0^x f(t) \,\mathrm{d}t + 6x [f(0) + f(x)] - f''(0) x^3}{x^4} \\
&= \lim_{x \to 0^+} \frac{-12 f(x) + 6 [f(x) + f(0)] + 6x f'(x) - 3 f''(0) x^2}{4x^3} \\
&= \lim_{x \to 0^+} \frac{6x f''(x) - 6 f''(0) x}{12x^2} \\
&= \lim_{x \to 0^+} \frac{f''(x) - f''(0)}{2x} = \frac{1}{2} f'''(0).
\end{align*}
因为$0<\eta<\xi<x$,所以
\begin{align*}
\lim_{x \to 0^+} f'''(\eta) = f'''(0),
\end{align*}
我们有
\begin{align*}
\lim_{x \to 0^+} \frac{\xi}{x} = \frac{1}{2}.
\end{align*}
\end{proof}

\begin{example}
设 \( f \) 是 \( [0,+\infty) \) 上的递增正函数. 若 \( g \in C^2[0,+\infty) \) 满足
\begin{align}
g''(x) + f(x)g(x) &= 0.\label{equation:::--107..8}
\end{align}
证明: 存在 \( M > 0 \) 使得
\begin{align}
|g(x)| &\leq M, \quad |g'(x)| \leq M\sqrt{f(x)}, \quad \forall x > 0.\label{equation:::--107..9}
\end{align}
\end{example}
\begin{proof}
对$\forall x>0,$有$f$在$[0,x]$上单调递增,从而由\hyperref[theorem:闭区间上单调函数必可积]{闭区间上单调函数必可积}可知$f\in R[0,x],\forall x>0$,$f$在$[0,+\infty)$上内闭连续.
由\eqref{equation:::--107..8}知
\begin{align}
\int_0^x g''(y)g'(y)\,\mathrm{d}y + \int_0^x f(y)g'(y)g(y)\,\mathrm{d}y &= 0, \forall x > 0 \label{eq:486861611651--8234}
\end{align}
利用 \( f \) 递增和\hyperref[theorem:积分中值定理(2)]{第二积分中值定理}和 \(\eqref{eq:486861611651--8234}\),我们有
\[
\int_0^x g''(y)g'(y)\,\mathrm{d}y + f(x)\int_\xi^x g'(y)g(y)\,\mathrm{d}y = 0, \xi \in [0, x].
\]
即
\[
\frac{1}{2}\lvert g'(x)\rvert^2 - \frac{1}{2}\lvert g'(0)\rvert^2 + \frac{[f(x)]^2}{2}\left[g^2(x) - g^2(\xi)\right] = 0.
\]
现在一方面
\begin{align}
\lvert g'(x)\rvert^2 = \lvert g'(0)\rvert^2 - f(x)g^2(x) + f(x)g^2(\xi) \leqslant \lvert g'(0)\rvert^2 + f(x)g^2(\xi).\label{eq:::--123124523-1214} 
\end{align}
另外一方面由\eqref{equation:::--107..8}得
\begin{align*}
\frac{g''(x)g'(x)}{f(x)} + g'(x)g(x) = 0, \forall x > 0. 
\end{align*}
即
\[
\int_0^x \frac{g''(y)g'(y)}{f(y)}\,\mathrm{d}y + \frac{1}{2}g^2(x) - \frac{1}{2}g^2(0) = 0, \forall x > 0
\]
由 \( f \) 递增和\hyperref[theorem:积分中值定理(1)]{第二积分中值定理},我们有
\[
\frac{1}{f(0)}\int_0^\eta g''(y)g'(y)\,\mathrm{d}y + \frac{1}{2}g^2(x) - \frac{1}{2}g^2(0) = 0, \eta \in [0, x]
\]
从而
\[
\frac{1}{2f(0)}\left[\lvert g'(\eta)\rvert^2 - \lvert g'(0)\rvert^2\right] + \frac{1}{2}g^2(x) - \frac{1}{2}g^2(0) = 0
\]
即
\begin{align}
\lvert g(x)\rvert^2 = g^2(0) - \frac{1}{f(0)}\left[\lvert g'(\eta)\rvert^2 - \lvert g'(0)\rvert^2\right] \leqslant g^2(0) + \frac{\lvert g'(0)\rvert^2}{f(0)},\forall x>0.\label{eq:::--123124523-1211}
\end{align}
由 \( g\in C[0,+\infty) \)知$g$ 有界,即存在$C_1>0,$使得$|g(x)|<C_1,\forall x>0.$于是由\eqref{eq:::--123124523-1214}式知
\begin{align}
\left| g' (x) \right|^2\leqslant \left| g' (0) \right|^2+f(x)g^2(\xi )\leqslant \left| g' (0) \right|^2+C_1f(x),\forall x>0.\label{eq::---1--3-1-3-2314--12-3}
\end{align}
又因为$f$是递增正函数,所以$f(x)\geqslant f(0)>0,\forall x>0.$从而存在$C_2>0,$使得
$$
|g'(0)|^2\leqslant C_2f(0)\leqslant f(x),\forall x>0.
$$
于是取$M=\max \left\{ C_1+C_2,g^2(0)+\frac{\left| g' (0) \right|^2}{f(0)} \right\},$则由\eqref{eq::---1--3-1-3-2314--12-3}式和\eqref{eq:::--123124523-1211}式可得,对$\forall x>0,$有
\begin{gather*}
\left| g\left( x \right) \right|^2\leqslant M\leqslant M^2,
\\
\left| g' (x) \right|^2\leqslant C_2f\left( x \right) +C_1f(x)\leqslant Mf\left( x \right) \leqslant M^2f\left( x \right) .
\end{gather*}
进而
\begin{align*}
\left| g\left( x \right) \right|\leqslant M,\left| g'(x) \right|\leqslant M\sqrt{f\left( x \right)},\forall x>0.
\end{align*}
这就证明了\eqref{equation:::--107..9}.
\end{proof}

\begin{example}
设 \( f \in C^2[0,1] \),证明
\begin{enumerate}[(a)]
\item 
\begin{align}
|f'(x)| &\leqslant 4\int_0^1 |f(x)| \mathrm{d}x + \int_0^1 |f''(x)| \mathrm{d}x. \label{eq::::::::::::::7151--------------5}
\end{align}

\item 
\begin{align}
\int_0^1 |f'(x)| \mathrm{d}x &\leqslant 4\int_0^1 |f(x)| \mathrm{d}x + \int_0^1 |f''(x)| \mathrm{d}x .\label{eq::::::::::::::7151--------------6}
\end{align}

\item  若 \( f(0)f(1) \geqslant 0 \),则
\begin{align}
\int_0^1 |f'(x)| \mathrm{d}x &\leqslant 2\int_0^1 |f(x)| \mathrm{d}x + \int_0^1 |f''(x)| \mathrm{d}x. \label{eq::::::::::::::7151--------------7}
\end{align}
\end{enumerate}
\end{example}
\begin{proof}
\begin{enumerate}[(a)]
\item 注意到对任何 \( \theta \in [0,1] \),我们有
\begin{align*}
|f' (x)|&\leqslant |f' (x)-f' (\theta )|+|f' (\theta )|\leqslant \left| \int_{\theta}^x{f'' (y)\mathrm{d}y} \right|+|f' (\theta )|
\\
&\leqslant \int_0^1{|f'' (y)|\mathrm{d}y}+|f' (\theta )|.
\end{align*}
于是只需证明存在 \( \theta \in [0,1] \) 使得
\begin{align}
|f'(\theta)| \leqslant 4 \int_0^1 |f(x)| \mathrm{d}x. \label{eq:10-------:::21161}
\end{align}
如果 \( f' \) 有零点,则显然存在$\theta \in [0,1],$使得$f(\theta)=0$,从而满足\(\eqref{eq:10-------:::21161}\)式.下设 \( f' \) 没有零点.由$f'$的介值性可知,$f'$要么恒正,要么恒负.不妨设 \( f \) 严格递增.
若 \( f \) 没有零点,不妨设 \( f > 0 \),则由Lagrange中值定理可得
\[
f(x) = f(0) + x f'(\eta) \geqslant x f'(\eta) \geqslant x \min_{[0,1]} |f'| \implies \int_0^1 |f(x)| \mathrm{d}x \geqslant \min_{[0,1]} |f'|\geqslant \frac{1}{4} \min_{[0,1]} |f'|,
\]
这也给出了\(\eqref{eq:10-------:::21161}\)式.
若 存在$t \in [0,1]$,使得\( f(t) = 0\).由Lagrange中值定理可知
\begin{align*}
f(x)=f' (\theta )(x-t).
\end{align*}
从而
\begin{align*}
\int_0^1{|f(x)|\mathrm{d}x}\geqslant \min_{[0,1]} |f' |\cdot \int_0^1{|x}-t|\mathrm{d}x\overset{\text{\refpro{proposition:单调函数减任意常数积分小于减其中点的积分}}}{\geqslant}\min_{[0,1]} |f' |\cdot \int_0^1{\left| x-\frac{1}{2} \right|\mathrm{d}x}=\frac{1}{4}\min_{[0,1]} |f' |.
\end{align*}
这也给出了\(\eqref{eq:10-------:::21161}\)式.于是我们证明了不等式\eqref{eq::::::::::::::7151--------------5}式.

\item 直接对\eqref{eq::::::::::::::7151--------------5}式两边关于$x$在$[0,1]$上积分得\eqref{eq::::::::::::::7151--------------6}式.

\item 由(a)同理只需证明存在 \( \theta \in [0,1] \) 使得
\begin{align}
|f'(\theta)| \leqslant 2 \int_0^1 |f(x)| \mathrm{d}x. \label{eq:10-------:::286161}
\end{align}
不妨假定 \( f' \) 没有零点且 \( f(0) \geqslant 0 \),则当 \( f \) 递增,由Lagrange中值定理,我们有
\[
f(x)=f(0)+xf' (\eta )\geqslant xf' (\eta )\geqslant x\cdot \min |f' |\Longrightarrow \int_0^1{|f(x)|\mathrm{d}x}\geqslant \min |f' \geqslant \frac{1}{2}\min |f' |.
\]
当 \( f \) 递减,由Lagrange中值定理,我们有
\[
f(x) = f(1) + (x - 1) f'(\alpha) \geqslant (1 - x) \min |f'| \implies \int_0^1 |f(x)| \mathrm{d}x \geqslant \frac{1}{2} \min |f'|.
\]
于是必有\(\eqref{eq:10-------:::286161}\)式成立,这就给出了\eqref{eq::::::::::::::7151--------------7}式.
\end{enumerate}
\end{proof}

\begin{example}

\end{example}
\begin{proof}

\end{proof}












\end{document}
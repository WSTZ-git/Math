\documentclass[../../main.tex]{subfiles}
\graphicspath{{\subfix{../../image/}}} % 指定图片目录,后续可以直接使用图片文件名。

% 例如:
% \begin{figure}[H]
% \centering
% \includegraphics[scale=0.4]{图.png}
% \caption{}
% \label{figure:图}
% \end{figure}
% 注意:上述\label{}一定要放在\caption{}之后,否则引用图片序号会只会显示??.

\begin{document}

\section{杂题}

\begin{example}
设$Y,x_0,\delta > 0$,计算
\begin{align*}
\lim_{n \to \infty} \sqrt{n} \int_{x_0 - \delta}^{x_0 + \delta} e^{-nY(x - x_0)^2} \,\mathrm{d}x.
\end{align*}
\end{example}
\begin{proof}
\begin{align*}
\underset{n\rightarrow \infty}{\lim}\sqrt{n}\int_{x_0-\delta}^{x_0+\delta}{e^{-nY(x-x_0)^2}\,\mathrm{d}x}&=\underset{n\rightarrow \infty}{\lim}\sqrt{n}\int_{-\delta}^{\delta}{e^{-nYx^2}\,\mathrm{d}x}=\underset{n\rightarrow \infty}{\lim}\frac{1}{\sqrt{Y}}\int_{-\delta \sqrt{nY}}^{\delta \sqrt{nY}}{e^{-x^2}\,\mathrm{d}x}
\\
&=\underset{n\rightarrow \infty}{\lim}\frac{2}{\sqrt{Y}}\int_0^{\delta \sqrt{nY}}{e^{-x^2}\,\mathrm{d}x}=\frac{2}{\sqrt{Y}}\int_0^{+\infty}{e^{-x^2}\,\mathrm{d}x}
\\
&=\sqrt{\frac{\pi}{Y}}.
\end{align*}

\end{proof}

\begin{example}
设$f \in C^3[0,x]$,$x > 0$,证明:存在$\xi \in (0,x)$使得
\begin{align}\label{equation:::::----16856486}
\int_0^x f(t) \,\mathrm{d}t = \frac{x}{2}[f(0) + f(x)] - \frac{x^3}{12}f''(\xi).
\end{align}
若还有$f'''(0) \neq 0$,计算$\lim_{x \to 0^+} \frac{\xi}{x}.$
\end{example}
\begin{note}
我们当然可以直接用\hyperref[proposition:Lagrange插值公式]{Lagrange插值公式}得到
\begin{align*}
f\left( t \right) =\left( f\left( x \right) -f\left( 0 \right) \right) t+f\left( 0 \right) +f'' \left( \xi \right) t\left( t-x \right) ,t\in \left[ 0,x \right] .
\end{align*}
两边同时对$t$在$[0,x]$上积分就能得到\eqref{equation:::::----16856486}式.
\end{note}
\begin{proof}
设$K \in \mathbb{R}$使得
\begin{align*}
\int_0^x f(t) \,\mathrm{d}t = \frac{x}{2}[f(0) + f(x)] - \frac{x^3}{12}K,
\end{align*}
则考虑
\begin{align*}
g(y) \triangleq \int_0^y f(t) \,\mathrm{d}t - \frac{y}{2}[f(0) + f(y)] + \frac{y^3}{12}K,
\end{align*}
于是
\begin{align*}
g'(y) = f(y) - \frac{1}{2}[f(0) + f(y)] - \frac{y f'(y)}{2} + \frac{y^2 K}{4} = \frac{f(y) - f(0)}{2} - \frac{y f'(y)}{2} + \frac{y^2 K}{4}
\end{align*}
以及
\begin{align*}
g''(y) = -\frac{y f''(y)}{2} + \frac{y K}{2}.
\end{align*}
由$g(x) = g(0) = 0$和罗尔中值定理得$\xi_1 \in (0,x)$使得$g'(\xi_1) = 0$. 注意到$g'(0) = 0$. 再次由罗尔中值定理得$\xi \in (0,x)$使得
\begin{align*}
g''(\xi) = -\frac{\xi f''(\xi)}{2} + \frac{\xi K}{2} = 0,
\end{align*}
即$K = f''(\xi)$, 这就得到了\eqref{equation:::::----16856486}式.
由\eqref{equation:::::----16856486}式得
\begin{align*}
f''(\xi) = -12 \frac{\int_0^x f(t) \,\mathrm{d}t - \frac{x}{2}[f(0) + f(x)]}{x^3}
\end{align*}
由Lagrange中值定理得
\begin{align*}
f''(\xi) = f''(0) + f'''(\eta) \xi, \eta \in (0,\xi).
\end{align*}
于是
\begin{align*}
f'''(\eta) \frac{\xi}{x} = \frac{-12 \frac{\int_0^x f(t) \,\mathrm{d}t - \frac{x}{2}[f(0) + f(x)]}{x^3} - f''(0)}{x}
\end{align*}
现在利用L'Hospital法则就有
\begin{align*}
\lim_{x \to 0^+} f'''(\eta) \frac{\xi}{x} &= \lim_{x \to 0^+} \frac{-12 \frac{\int_0^x f(t) \,\mathrm{d}t - \frac{x}{2}[f(0) + f(x)]}{x^3} - f''(0)}{x} \\
&= \lim_{x \to 0^+} \frac{-12 \int_0^x f(t) \,\mathrm{d}t + 6x [f(0) + f(x)] - f''(0) x^3}{x^4} \\
&= \lim_{x \to 0^+} \frac{-12 f(x) + 6 [f(x) + f(0)] + 6x f'(x) - 3 f''(0) x^2}{4x^3} \\
&= \lim_{x \to 0^+} \frac{6x f''(x) - 6 f''(0) x}{12x^2} \\
&= \lim_{x \to 0^+} \frac{f''(x) - f''(0)}{2x} = \frac{1}{2} f'''(0).
\end{align*}
因为$0<\eta<\xi<x$,所以
\begin{align*}
\lim_{x \to 0^+} f'''(\eta) = f'''(0),
\end{align*}
我们有
\begin{align*}
\lim_{x \to 0^+} \frac{\xi}{x} = \frac{1}{2}.
\end{align*}

\end{proof}

\begin{example}
设 \( f \) 是 \( [0,+\infty) \) 上的递增正函数. 若 \( g \in C^2[0,+\infty) \) 满足
\begin{align}
g''(x) + f(x)g(x) &= 0.\label{equation:::--107..8}
\end{align}
证明: 存在 \( M > 0 \) 使得
\begin{align}
|g(x)| &\leqslant M, \quad |g'(x)| \leqslant M\sqrt{f(x)}, \quad \forall x > 0.\label{equation:::--107..9}
\end{align}
\end{example}
\begin{proof}
对$\forall x>0,$有$f$在$[0,x]$上单调递增,从而由\hyperref[theorem:闭区间上单调函数必可积]{闭区间上单调函数必可积}可知$f\in R[0,x],\forall x>0$,$f$在$[0,+\infty)$上内闭连续.
由\eqref{equation:::--107..8}知
\begin{align}
\int_0^x g''(y)g'(y)\,\mathrm{d}y + \int_0^x f(y)g'(y)g(y)\,\mathrm{d}y &= 0, \forall x > 0 \label{eq:486861611651--8234}
\end{align}
利用 \( f \) 递增和\hyperref[theorem:积分中值定理(2)]{第二积分中值定理}和 \(\eqref{eq:486861611651--8234}\),我们有
\[
\int_0^x g''(y)g'(y)\,\mathrm{d}y + f(x)\int_\xi^x g'(y)g(y)\,\mathrm{d}y = 0, \xi \in [0, x].
\]
即
\[
\frac{1}{2}\lvert g'(x)\rvert^2 - \frac{1}{2}\lvert g'(0)\rvert^2 + \frac{[f(x)]^2}{2}\left[g^2(x) - g^2(\xi)\right] = 0.
\]
现在一方面
\begin{align}
\lvert g'(x)\rvert^2 = \lvert g'(0)\rvert^2 - f(x)g^2(x) + f(x)g^2(\xi) \leqslant \lvert g'(0)\rvert^2 + f(x)g^2(\xi).\label{eq:::--123124523-1214} 
\end{align}
另外一方面由\eqref{equation:::--107..8}得
\begin{align*}
\frac{g''(x)g'(x)}{f(x)} + g'(x)g(x) = 0, \forall x > 0. 
\end{align*}
即
\[
\int_0^x \frac{g''(y)g'(y)}{f(y)}\,\mathrm{d}y + \frac{1}{2}g^2(x) - \frac{1}{2}g^2(0) = 0, \forall x > 0
\]
由 \( f \) 递增和\hyperref[theorem:积分中值定理(1)]{第二积分中值定理},我们有
\[
\frac{1}{f(0)}\int_0^\eta g''(y)g'(y)\,\mathrm{d}y + \frac{1}{2}g^2(x) - \frac{1}{2}g^2(0) = 0, \eta \in [0, x]
\]
从而
\[
\frac{1}{2f(0)}\left[\lvert g'(\eta)\rvert^2 - \lvert g'(0)\rvert^2\right] + \frac{1}{2}g^2(x) - \frac{1}{2}g^2(0) = 0
\]
即
\begin{align}
\lvert g(x)\rvert^2 = g^2(0) - \frac{1}{f(0)}\left[\lvert g'(\eta)\rvert^2 - \lvert g'(0)\rvert^2\right] \leqslant g^2(0) + \frac{\lvert g'(0)\rvert^2}{f(0)},\forall x>0.\label{eq:::--123124523-1211}
\end{align}
由 \( g\in C[0,+\infty) \)知$g$ 有界,即存在$C_1>0,$使得$|g(x)|<C_1,\forall x>0.$于是由\eqref{eq:::--123124523-1214}式知
\begin{align}
\left| g' (x) \right|^2\leqslant \left| g' (0) \right|^2+f(x)g^2(\xi )\leqslant \left| g' (0) \right|^2+C_1f(x),\forall x>0.\label{eq::---1--3-1-3-2314--12-3}
\end{align}
又因为$f$是递增正函数,所以$f(x)\geqslant f(0)>0,\forall x>0.$从而存在$C_2>0,$使得
$$
|g'(0)|^2\leqslant C_2f(0)\leqslant f(x),\forall x>0.
$$
于是取$M=\max \left\{ C_1+C_2,g^2(0)+\frac{\left| g' (0) \right|^2}{f(0)} \right\},$则由\eqref{eq::---1--3-1-3-2314--12-3}式和\eqref{eq:::--123124523-1211}式可得,对$\forall x>0,$有
\begin{gather*}
\left| g\left( x \right) \right|^2\leqslant M\leqslant M^2,
\\
\left| g' (x) \right|^2\leqslant C_2f\left( x \right) +C_1f(x)\leqslant Mf\left( x \right) \leqslant M^2f\left( x \right) .
\end{gather*}
进而
\begin{align*}
\left| g\left( x \right) \right|\leqslant M,\left| g'(x) \right|\leqslant M\sqrt{f\left( x \right)},\forall x>0.
\end{align*}
这就证明了\eqref{equation:::--107..9}.

\end{proof}

\begin{example}
设函数$f(x)$在$(a,+\infty)$上严格单调下降,证明:若$\lim\limits_{n \to \infty} f(x_n)=\lim\limits_{x \to +\infty} f(x)$,则$\lim\limits_{n \to \infty} x_n=+\infty$. 
\end{example}
\begin{proof}
反证,假设$\varliminf_{n\rightarrow \infty}x_n=c\in \left( a,+\infty \right)$,则存在子列$\{ x_{n_k} \}$,满足$x_{n_k}\rightarrow c$。记
\begin{align*}
\lim_{n\rightarrow \infty}f\left( x_n \right) =\lim_{x\rightarrow +\infty}f\left( x \right) =A,
\end{align*}
则$f\left( x_n \right)$的子列极限也收敛到$A$,即$\lim_{k\rightarrow \infty}f\left( x_{n_k} \right) =A$。由$x_{n_k}\rightarrow c$知,存在$K\in \mathbb{N}$,使得
\begin{align*}
x_{n_k}\in \left( c-\delta ,c+\delta \right),\forall k>K.
\end{align*}
其中$\delta =\min \left\{ \frac{c-a}{2},\frac{1}{2} \right\}$。任取$x_1,x_2\in \left( c+\delta ,+\infty \right)$且$x_1<x_2$,则由$f$严格递减知
\begin{align*}
f\left( x_{n_k} \right) >f\left( x_1 \right) >f\left( x_2 \right) >f\left( x \right),\forall x>x_2,\forall k>K.
\end{align*}
左边令$k\rightarrow +\infty$,右边令$x\rightarrow +\infty$得
\begin{align*}
A=\lim_{k\rightarrow \infty}f\left( x_{n_k} \right) \geqslant f\left( x_1 \right) >f\left( x_2 \right) \geqslant \lim_{x\rightarrow +\infty}f\left( x \right) =A,
\end{align*}
显然矛盾!

\end{proof}

\begin{example}
设 $\{x_n\} \subset (0,1)$ 满足对 $i \neq j$, 有 $x_i \neq x_j$, 讨论函数 $f(x) = \sum\limits_{n=1}^\infty \frac{\operatorname{sgn}(x - x_n)}{2^n}$ 连续性.
\end{example}
\begin{proof}
由
\[
\sum\limits_{n=1}^\infty \left| \frac{\operatorname{sgn}(x - x_n)}{2^n} \right| \leqslant \sum\limits_{n=1}^\infty \frac{1}{2^n} < \infty,
\]
故级数一致收敛. 注意到对$\forall n\in \mathbb{N} $,都有$\mathrm{sgn} \left( x-x_n \right) $在$x=x_n$处间断,在$x\ne x_n$处连续.

当$x\ne x_k,\forall k\in \mathbb{N} $时,$f\left( x \right) $的每一项都连续.又$f\left( x \right) $一致收敛,故$f$在$x\ne x_k,\forall k\in \mathbb{N} $处都连续.

当$x=x_k,\forall k\in \mathbb{N} $时,有
\begin{align*}
f\left( x \right) =\frac{\mathrm{sgn} \left( x-x_k \right)}{2^k}+\sum\limits_{n\ne k}{\frac{\mathrm{sgn} \left( x-x_n \right)}{2^n}}
\end{align*}
在$x=x_k$处间断.故$f\left( x \right) $在$x=x_k,\forall k\in \mathbb{N} $处都间断.

\end{proof}

\begin{example}
证明 $\sum\limits_{t=1}^{\infty} (-1)^t \frac{t}{t^2 + x}$ 在 $x \in [0, +\infty)$ 一致收敛性.
\end{example}
\begin{proof}
由\hyperref[theorem:Abel变换]{Abel变换}得,对$\forall m \in \mathbb{N},\forall x\geqslant 0$成立
\begin{align*}
\sum\limits_{t=m}^{\infty} (-1)^t \frac{t}{t^2 + x} &= \lim_{n \to \infty} \sum\limits_{t=m}^n (-1)^t \frac{t}{t^2 + x} \\
&= \lim_{n \to \infty} \left[ \sum\limits_{t=m}^{n-1} \left( \frac{t}{t^2 + x} - \frac{t + 1}{(t + 1)^2 + x} \right) s_t + \frac{n}{n^2 + x} s_n \right] \\
&= \sum\limits_{t=m}^{\infty} \left( \frac{t}{t^2 + x} - \frac{t + 1}{(t + 1)^2 + x} \right) s_t \\
&= \sum\limits_{t=m}^{\infty} \frac{t^2 + t}{(x + t^2)(x + t^2 + 2t + 1)} s_t - \sum\limits_{t=m}^{\infty} \frac{x}{(x + t^2)(x + t^2 + 2t + 1)} s_t,
\end{align*}
这里 $s_t =\sum\limits_{i=1}^t{\left( -1 \right) ^i}=\left( -1 \right) ^t \in \{1, -1\}$.
一方面
\[
\left| \sum\limits_{t=m}^{\infty} \frac{t^2 + t}{(x + t^2)(x + t^2 + 2t + 1)} s_t \right| \leqslant \sum\limits_{t=m}^{\infty} \frac{t^2 + t}{t^2(t^2 + 2t + 1)},
\]
另外一方面
\[
\left| \sum\limits_{t=m}^{\infty} \frac{x}{(x + t^2)(x + t^2 + 2t + 1)} s_t \right| \leqslant \sum\limits_{t=m}^{\infty} \frac{1}{t^2 + t + 1}.
\]
而由$\sum\limits_{t=1}^{\infty} \frac{t^2 + t}{t^2(t^2 + 2t + 1)}$和$\sum\limits_{t=1}^{\infty} \frac{1}{t^2 + t + 1}$都收敛知
\begin{align*}
\underset{m\rightarrow \infty}{\lim}\sum\limits_{t=m}^{\infty}{\frac{1}{t^2+t+1}}=\underset{m\rightarrow \infty}{\lim}\sum\limits_{t=m}^{\infty}{\frac{t^2+t}{t^2(t^2+2t+1)}}=0.
\end{align*}
于是我们有
\[
\lim_{m \to \infty} \sum\limits_{t=m}^{\infty} (-1)^t \frac{t}{t^2 + x} = 0, \text{关于} x \in [0, +\infty) \text{一致},
\]
这就证明了 $\sum\limits_{t=1}^{\infty} (-1)^t \frac{t}{t^2 + x}$ 在 $x \in [0, +\infty)$ 一致收敛.

\end{proof}

\begin{proposition}\label{proposition:右导数小于等于0函数值小于端点值}
设\( f(x) \)是\([a,b]\)上连续实值右可导函数,记\( D^+f(x) \)为\( f(x) \)的右导函数,如果\( f(a)=0 \),且\( D^+f(x) \leqslant0 \),则\( f(x) \leqslant0, x \in [a,b] \)。
\end{proposition}
\begin{proof}
(1) 先假定\( D^+f(x) < 0 \),如果结论不成立,则存在\( x_1 \in (a,b) \),使\( f(x_1) > 0 \)。
记
\[
x_0 = \inf\{ x \mid f(x) > 0 \}.
\]
由\( x_0 \)的定义,我们有序列\(\{ x_n \}\),使\( x_n \)单调递减趋于\( x_0 \),且\( f(x_n) > 0 \)。从而由\( f(x) \)的连续性知
\begin{align}
f\left( x_0 \right)=\underset{n\rightarrow \infty}{\lim}f\left( x_n \right) \geqslant 0.\label{eq:::::-----2348001456564156}
\end{align}
根据$x_0$的定义可知,对$\forall x<x_0,$都有$f(x)<f(x_0)$,否则与下确界定义矛盾!于是有序列$\{x'_n\}$单调递增趋于$x_0$,且$f(x'_n).$于是由\( f(x) \)的连续性知
\begin{align}
f\left( x_0 \right)=\underset{n\rightarrow \infty}{\lim}f\left( x'_n \right)  \leqslant 0.\label{eq:::::-----23480091456564156}
\end{align}
故由\eqref{eq:::::-----2348001456564156}\eqref{eq:::::-----23480091456564156}知\( f(x_0)=0 \)。于是
\[
D^+f(x_0) = \lim_{n \to \infty} \frac{f(x_n) - f(x_0)}{x_n - x_0} \geqslant 0,
\]
这与\( D^+f(x_0) < 0 \)矛盾,于是\( f(x) \leqslant 0, x \in [a,b] \)。

(2) 若\( D^+f(x) \leqslant 0 \),对任给的\( \varepsilon > 0 \)构造函数
\[
f_\varepsilon(x) = f(x) - \varepsilon(x - a),
\]
对\( f_\varepsilon(x) \)有\( f_\varepsilon(a)=0 \)且
\[
D^+f_\varepsilon(x) \leqslant-\varepsilon < 0.
\]
从而由(1)得\( f_\varepsilon(x) \leqslant0, x \in [a,b] \)。因此\( f(x) \leqslant\varepsilon(x - a)\leqslant \varepsilon (b-a) \),由\( \varepsilon \)的任意性,得\( f(x) \leqslant0, x \in [a,b] \).

\end{proof}

\begin{example}
设\( \varphi(x) \)是\([a,b)\)上连续且右可导的函数,如果\( D^+\varphi(x) \)在\([a,b)\)上连续,证明:\( \varphi(x) \)在\([a,b)\)上连续可导,\( \varphi'(x)=D^+\varphi(x) \)。
\end{example}
\begin{proof}
设
\[
f(x) = \varphi(a) + \int_a^x D^+\varphi(t)\mathrm{d}t - \varphi(x), \quad x \in [a,b).
\]
则\( f(x) \)在\([a,b)\)上连续且右可导,并且
\[
D^+f(x) = D^+\varphi(x) - D^+\varphi(x) = 0.
\]
又\( f(a)=0 \),由\refpro{proposition:右导数小于等于0函数值小于端点值}得\( f(x) \leqslant0 \)。
又\(-f(x)\)满足\(-f(a)=0, D^+[-f(x)] = 0\),同理由\refpro{proposition:右导数小于等于0函数值小于端点值}得\(-f(x) \leqslant0\) ,故\(f(x)=0\)。于是
\[
\varphi(x) = \varphi(a) + \int_a^x D^+\varphi(t)\mathrm{d}t.
\]
由\( D^+\varphi(x) \)的连续性,得\( \varphi'(x)=D^+\varphi(x) \)。

\end{proof}

\begin{example}
证明:
\begin{align*}
\sum\limits_{k=1}^{n-1}{\frac{1}{\sin \frac{k\pi}{n}}}=\frac{2n}{\pi}\left( \ln 2n+\gamma -\ln \pi \right) +o\left( 1 \right).
\end{align*}
\end{example}
\begin{proof}
见\href{https://math.stackexchange.com/questions/3216414/sum-of-reciprocal-sine-function-sum-limits-k-1n-1-frac1-sin-frack-p?noredirect=1}{here}.

\end{proof}

\begin{example}
$\lim_{n\rightarrow \infty} \frac{\sum\limits_{k=1}^n{\left( -1 \right) ^k\mathrm{C}_{n}^{k}\ln k}}{\ln \left( \ln n \right)}=1.$
\end{example}
\begin{proof}
{\color{blue}证法一:}对任意充分大的$n$,由\hyperref[theorem:Frullani(傅汝兰尼)积分]{Frullani(傅汝兰尼)积分}知
\begin{align*}
\ln k = \int_0^{+\infty}{\frac{e^{-x}-e^{kx}}{x}\mathrm{d}x}.
\end{align*}
再结合二项式定理可得
\begin{align*}
A&\triangleq \sum\limits_{k=1}^n{\left( -1 \right) ^k\mathrm{C}_{n}^{k}\ln k}=\sum\limits_{k=1}^n{\left[ \left( -1 \right) ^k\mathrm{C}_{n}^{k}\left( \int_0^{+\infty}{\frac{e^{-x}-e^{-kx}}{x}\mathrm{d}x} \right) \right]}=\int_0^{+\infty}{\frac{\sum\limits_{k=1}^n{\left( -1 \right) ^k\mathrm{C}_{n}^{k}}\left( e^{-x}-e^{-kx} \right)}{x}\mathrm{d}x} \\
&=\int_0^{+\infty}{\frac{\sum\limits_{k=1}^n{\left( -1 \right) ^k\mathrm{C}_{n}^{k}}\left( e^{-x}-e^{-kx} \right)}{x}\mathrm{d}x}=\int_0^{+\infty}{\frac{1-e^{-x}+\sum\limits_{k=0}^n{\left( -1 \right) ^k\mathrm{C}_{n}^{k}}\left( e^{-x}-e^{-kx} \right)}{x}\mathrm{d}x} \\
&=\int_0^{+\infty}{\frac{1-e^{-x}+e^{-x}\sum\limits_{k=0}^n{\left( -1 \right) ^k\mathrm{C}_{n}^{k}}-\sum\limits_{k=0}^n{\left( -1 \right) ^k\mathrm{C}_{n}^{k}}e^{-kx}}{x}\mathrm{d}x}=\int_0^{+\infty}{\frac{1-e^{-x}+e^{-x}\left( 1-1 \right) ^n-\left( 1-e^{-x} \right) ^n}{x}\mathrm{d}x} \\
&=\int_0^{+\infty}{\frac{1-e^{-x}-\left( 1-e^{-x} \right) ^n}{x}\mathrm{d}x}.
\end{align*}
由\hyperref[theorem:Bernoulli不等式]{Bernoulli不等式}知
\begin{align*}
\left( 1-e^{-x} \right) ^n\geqslant 1-ne^{-x}.
\end{align*}
取$M_n>1$,满足$M_ne^{M_n}=n$.于是
\begin{align*}
0&\leqslant \int_{M_n}^{+\infty}{\frac{1-e^{-x}-\left( 1-e^{-x} \right) ^n}{x}\mathrm{d}x}\leqslant \int_{M_n}^{+\infty}{\frac{1-e^{-x}-\left( 1-ne^{-x} \right)}{M_n}\mathrm{d}x}=\frac{n}{M_n}\int_{M_n}^{+\infty}{e^{-x}\mathrm{d}x}=\frac{n}{M_ne^{M_n}}=1.
\end{align*}
从而
\begin{align}
A&=\int_0^{M_n}{\frac{1-e^{-x}-\left( 1-e^{-x} \right) ^n}{x}\mathrm{d}x}+\int_{M_n}^{+\infty}{\frac{1-e^{-x}-\left( 1-e^{-x} \right) ^n}{x}\mathrm{d}x}=\int_0^{M_n}{\frac{1-e^{-x}-\left( 1-e^{-x} \right) ^n}{x}\mathrm{d}x}+O\left( 1 \right) .\label{eq:eq:::23589u2385tt10009}
\end{align}
因为$M_ne^{M_n}=n$,所以由\refpro{proposition:Lampert W 的渐进估计}知
\begin{align}
M_n&=\ln n+o\left( \ln n \right) ,n\rightarrow \infty .\label{eq:eq:::23589u2385tt71}
\end{align}
于是
\begin{align*}
\left( 1-e^{-x} \right) ^{n-1}&=e^{\left( n-1 \right) \ln \left( 1-e^{-x} \right)}\leqslant e^{-\left( n-1 \right) e^{-x}}\leqslant e^{-\left( n-1 \right) e^{-M_n}}=e^{-\frac{M_n\left( n-1 \right)}{n}}\rightarrow 0,\forall x\in \left[ 0,M_n \right] .
\end{align*}
从而
\begin{align*}
\frac{\int_0^{M_n}{\frac{\left( 1-e^{-x} \right) ^n}{x}\mathrm{d}x}}{\int_0^{M_n}{\frac{1-e^{-x}}{x}\mathrm{d}x}}&\leqslant \frac{e^{-\frac{M_n\left( n-1 \right)}{n}}\int_0^{M_n}{\frac{1-e^{-x}}{x}\mathrm{d}x}}{\int_0^{M_n}{\frac{1-e^{-x}}{x}\mathrm{d}x}}=e^{-\frac{M_n\left( n-1 \right)}{n}}\rightarrow 0,n\rightarrow \infty .
\end{align*}
即$\int_0^{M_n}{\frac{\left( 1-e^{-x} \right) ^n}{x}\mathrm{d}x}=o\left( \int_0^{M_n}{\frac{1-e^{-x}}{x}\mathrm{d}x} \right) ,n\rightarrow \infty .$故
\begin{align}
\int_0^{M_n}{\frac{1-e^{-x}-\left( 1-e^{-x} \right) ^n}{x}\mathrm{d}x}&=\int_0^{M_n}{\frac{1-e^{-x}}{x}\mathrm{d}x}-\int_0^{M_n}{\frac{\left( 1-e^{-x} \right) ^n}{x}\mathrm{d}x}=\left( 1+o\left( 1 \right) \right) \int_0^{M_n}{\frac{1-e^{-x}}{x}\mathrm{d}x},n\rightarrow \infty .\label{eq:108.1001}
\end{align}
注意到
\begin{align*}
\lim_{x\rightarrow 0}\frac{1-e^{-x}}{x}\xlongequal{\text{L'Hospital}}\lim_{x\rightarrow 0}e^x=1,
\end{align*}
故$\frac{1-e^{-x}}{x}$在$\left[ 0,1 \right]$上有界,进而$\int_0^1{\frac{1-e^{-x}}{x}\mathrm{d}x}=O(1)$.又注意到
\begin{align*}
\int_1^{M_n}{\frac{-e^{-x}}{x}\mathrm{d}x}\leqslant -e^{-M_n}\int_1^{M_n}{\frac{1}{x}\mathrm{d}x}\rightarrow 0,n\rightarrow \infty ,
\end{align*}
故$\int_1^{M_n}{\frac{-e^{-x}}{x}\mathrm{d}x}=O(1)$.于是再结合\eqref{eq:eq:::23589u2385tt71}式可知
\begin{align*}
\int_0^{M_n}{\frac{1-e^{-x}}{x}\mathrm{d}x}&=\int_0^1{\frac{1-e^{-x}}{x}\mathrm{d}x}+\int_1^{M_n}{\frac{-e^{-x}}{x}\mathrm{d}x}+\int_1^{M_n}{\frac{1}{x}\mathrm{d}x} \\
&=O\left( 1 \right) +\ln M_n=\ln \left( \ln n+o\left( \ln n \right) \right) +O\left( 1 \right) \\
&=\ln\ln n+o\left( 1 \right) +O\left( 1 \right) =\ln\ln n+O\left( 1 \right) ,n\rightarrow \infty .
\end{align*}
因此再由\eqref{eq:108.1001}式可知
\begin{align*}
\int_0^{M_n}{\frac{1-e^{-x}-\left( 1-e^{-x} \right) ^n}{x}\mathrm{d}x}&=\left( 1+o\left( 1 \right) \right) \int_0^{M_n}{\frac{1-e^{-x}}{x}\mathrm{d}x}=\left( 1+o\left( 1 \right) \right) \left( \ln\ln n+O\left( 1 \right) \right) =\ln\ln n+o\left( \ln\ln n \right) ,n\rightarrow \infty .
\end{align*}
故由\eqref{eq:eq:::23589u2385tt10009}可得
\begin{align*}
\lim_{n\rightarrow \infty} \frac{\sum\limits_{k=1}^n{\left( -1 \right) ^k\mathrm{C}_{n}^{k}\ln k}}{\ln \left( \ln n \right)}&=\lim_{n\rightarrow \infty} \frac{A}{\ln \left( \ln n \right)}=\lim_{n\rightarrow \infty} \frac{\int_0^{M_n}{\frac{1-e^{-x}-\left( 1-e^{-x} \right) ^n}{x}\mathrm{d}x}+O\left( 1 \right)}{\ln \left( \ln n \right)} \\
&=\lim_{n\rightarrow \infty} \frac{\ln\ln n+o\left( \ln\ln n \right) +O\left( 1 \right)}{\ln \left( \ln n \right)}=1.
\end{align*}

{\color{blue}证法二:}注意到
\begin{align*}
S\triangleq \sum\limits_{k=1}^n{(-1) ^k\binom{n}{k}\ln k}&=\sum\limits_{k=1}^n{(-1) ^k\left[\binom{n-1}{k}+\binom{n-1}{k-1}\right]\ln k}
\\
&=\sum\limits_{k=1}^n{(-1) ^k\binom{n-1}{k}\ln k}+\sum\limits_{k=1}^n{(-1) ^k\binom{n-1}{k-1}\ln k}
\\
&=\sum\limits_{k=1}^{n-1}{(-1) ^k\binom{n-1}{k}\ln k}+\sum\limits_{k=0}^{n-1}{(-1) ^{k+1}\binom{n-1}{k}\ln (k+1)}
\\
&=\sum\limits_{k=1}^{n-1}{(-1) ^k\binom{n-1}{k}\ln k}+\sum\limits_{k=1}^{n-1}{(-1) ^{k+1}\binom{n-1}{k}\ln (k+1)}
\\
&=-\sum\limits_{k=1}^{n-1}{(-1) ^k\binom{n-1}{k}\left(\ln (k+1)-\ln k\right)}
\\
&=-\sum\limits_{k=1}^{n-1}{(-1) ^k\binom{n-1}{k}\int_0^1{\frac{1}{k+x}\mathrm{d}x}}.
\end{align*}
又由二项式定理可知
\begin{align*}
\sum\limits_{k=1}^{n-1}{(-1) ^k\binom{n-1}{k}\frac{1}{k+y}}&=\sum\limits_{k=1}^{n-1}{(-1) ^k\binom{n-1}{k}\int_0^1{t^{k+y-1}\mathrm{d}t}}=\int_0^1{\sum\limits_{k=1}^{n-1}{(-1) ^k\binom{n-1}{k}}t^{k+y-1}\mathrm{d}t}
\\
&=\int_0^1{t^{y-1}\sum\limits_{k=1}^{n-1}{(-1) ^k\binom{n-1}{k}}t^k\mathrm{d}t}=\int_0^1{t^{y-1}\left[(1-t) ^{n-1}-1\right] \mathrm{d}t}.
\end{align*}
故
\begin{align*}
S&=-\int_0^1{\sum\limits_{k=1}^{n-1}{(-1) ^k\binom{n-1}{k}\frac{1}{k+y}}\mathrm{d}y}=\int_0^1{\int_0^1{t^{y-1}\left[1-(1-t) ^{n-1}\right] \mathrm{d}t}\mathrm{d}y}
\\
&=\int_0^1{\int_0^1{t^{y-1}\left[1-(1-t) ^{n-1}\right] \mathrm{d}y}\mathrm{d}t}=\int_0^1{\frac{t-1}{t\ln t}\left[1-(1-t) ^{n-1}\right] \mathrm{d}t}
\\
&\xlongequal{t=e^{-x}}\int_0^{+\infty}{\frac{(1-e^{-x})\left[ 1-(1-e^{-x})^{n-1} \right]}{x}\mathrm{d}x}.
\end{align*}
后续估阶与证法一相同.

{\color{blue}证法三:}注意到
\begin{align*}
S\triangleq \sum\limits_{k=1}^n{(-1) ^k\binom{n}{k}\ln k}&=\sum\limits_{k=1}^n{(-1) ^k\left[\binom{n-1}{k}+\binom{n-1}{k-1}\right]\ln k}
\\
&=\sum\limits_{k=1}^n{(-1) ^k\binom{n-1}{k}\ln k}+\sum\limits_{k=1}^n{(-1) ^k\binom{n-1}{k-1}\ln k}
\\
&=\sum\limits_{k=1}^{n-1}{(-1) ^k\binom{n-1}{k}\ln k}+\sum\limits_{k=0}^{n-1}{(-1) ^{k+1}\binom{n-1}{k}\ln (k+1)}
\\
&=\sum\limits_{k=1}^{n-1}{(-1) ^k\binom{n-1}{k}\ln k}+\sum\limits_{k=1}^{n-1}{(-1) ^{k+1}\binom{n-1}{k}\ln (k+1)}
\\
&=-\sum\limits_{k=1}^{n-1}{(-1) ^k\binom{n-1}{k}\left(\ln (k+1)-\ln k\right)}
\\
&=-\sum\limits_{k=1}^{n-1}{(-1) ^k\binom{n-1}{k}\int_0^1{\frac{1}{k+x}\mathrm{d}x}}
\\
&=-\int_0^1{\sum\limits_{k=1}^{n-1}{(-1) ^k\binom{n-1}{k}\frac{1}{k+x}}\mathrm{d}x}
\\
&=\int_0^1{\left( \frac{1}{x}-\sum\limits_{k=0}^{n-1}{(-1) ^k\binom{n-1}{k}\frac{1}{k+x}} \right) \mathrm{d}x}
\\
&\xlongequal{\text{\refpro{proposition:组合数相关常用恒等式}}}\int_0^1{\left( \frac{1}{x}-\frac{(n-1)!}{x(x+1)\cdots(x+(n-1))} \right) \mathrm{d}x}
\\
&=\int_0^1{\frac{1}{x}\left( 1-\frac{(n-1)!}{(x+1)(x+2)\cdots(x+(n-1))} \right) \mathrm{d}x}
\\
&=\int_0^1{\frac{1}{x}\left( 1-\frac{1}{(1+x)\left(1+\frac{x}{2}\right)\cdots\left(1+\frac{x}{n-1}\right)} \right) \mathrm{d}x}.
\end{align*}
由\nrefpro{proposition:常用不等式4}{(4)}知
$$e^{x^2-x}\geqslant \frac{1}{1+x}\geqslant e^{-x},\forall x>0.$$
于是
$$e^{x^2-x}\cdot e^{\left( \frac{x}{2} \right) ^2-\frac{x}{2}}\cdots e^{\left( \frac{x}{n-1} \right) ^2-\frac{x}{n-1}}\geqslant \frac{1}{(1+x)\left(1+\frac{x}{2}\right)\cdots\left(1+\frac{x}{n-1}\right)}\geqslant e^{-x}\cdot e^{-\frac{x}{2}}\cdots e^{-\frac{x}{n-1}},$$
即
$$e^{x^2\left( 1+\frac{1}{2^2}+\cdots +\frac{1}{(n-1)^2} \right) -x\left( 1+\frac{1}{2}+\cdots +\frac{1}{n-1} \right)}\geqslant \frac{1}{(1+x)\left(1+\frac{x}{2}\right)\cdots\left(1+\frac{x}{n-1}\right)}\geqslant e^{-x\left( 1+\frac{1}{2}+\cdots +\frac{1}{n-1} \right)}.$$
注意到
$$x^2\left( 1+\frac{1}{2^2}+\cdots +\frac{1}{(n-1)^2} \right) \leqslant x\sum\limits_{k=1}^{\infty}{\frac{1}{k^2}}=\frac{\pi ^2}{6}x<2x,\forall x\in [0,1],$$
故
$$e^{-x\left( -2+\sum\limits_{j=1}^{n-1}{\frac{1}{j}} \right)}\geqslant \frac{1}{(1+x)\left(1+\frac{x}{2}\right)\cdots\left(1+\frac{x}{n-1}\right)}\geqslant e^{-x\sum\limits_{j=1}^{n-1}{\frac{1}{j}}}.$$
从而由连续函数$e^{-x}$的介值性知,存在$C_n\in \left[ -2+\sum\limits_{j=1}^{n-1}{\frac{1}{j}},\sum\limits_{j=1}^{n-1}{\frac{1}{j}} \right]$,使得
$$\frac{1}{(1+x)\left(1+\frac{x}{2}\right)\cdots\left(1+\frac{x}{n-1}\right)}=e^{-C_nx}.$$
于是由$-2+\sum\limits_{j=1}^{n-1}{\frac{1}{j}}\leqslant C_n\leqslant \sum\limits_{j=1}^{n-1}{\frac{1}{j}}$知
$$C_n=\ln n+O(1),n\rightarrow \infty.$$
因此
\begin{align*}
S&=\int_0^1{\frac{1}{x}\left( 1-\frac{1}{(1+x)\left(1+\frac{x}{2}\right)\cdots\left(1+\frac{x}{n-1}\right)} \right) \mathrm{d}x}=\int_0^1{\frac{1}{x}\left( 1-e^{-C_nx} \right) \mathrm{d}x}
\\
&=\int_0^{C_n}{\frac{1-e^{-t}}{t}\mathrm{d}t}=\int_0^1{\frac{1-e^{-t}}{t}\mathrm{d}t}+\int_1^{C_n}{\frac{1-e^{-t}}{t}\mathrm{d}t}+\int_1^{C_n}{\frac{1}{t}\mathrm{d}t}.
\end{align*}
注意到
$$\lim_{t\rightarrow 0}\frac{1-e^{-t}}{t}\xlongequal{\text{L'Hospital}}\lim_{t\rightarrow 0}e^t=1,$$
故$\frac{1-e^{-t}}{t}$在$[0,1]$上有界,进而$\int_0^1{\frac{1-e^{-t}}{t}\mathrm{d}t}=O(1)$.
又注意到
$$\int_1^{C_n}{\frac{1-e^{-t}}{t}\mathrm{d}t}\leqslant 1-e^{-C_n}=1-e^{-\ln n+O(1)}\rightarrow 1,n\rightarrow \infty,$$
故$\int_1^{C_n}{\frac{1-e^{-t}}{t}\mathrm{d}t}=O(1)$.从而
\begin{align*}
S&=\int_0^1{\frac{1-e^{-t}}{t}\mathrm{d}t}+\int_1^{C_n}{\frac{1-e^{-t}}{t}\mathrm{d}t}+\int_1^{C_n}{\frac{1}{t}\mathrm{d}t}=\ln C_n+O(1)
\\
&=\ln \left( \ln n+O(1) \right) +O(1)=\ln\ln n+O(1),n\rightarrow \infty.
\end{align*}
因此
\begin{align*}
\lim_{n\rightarrow \infty}\frac{\sum\limits_{k=1}^n{(-1) ^k\binom{n}{k}\ln k}}{\ln\ln n}&=\lim_{n\rightarrow \infty}\frac{S}{\ln\ln n}=\lim_{n\rightarrow \infty}\frac{\ln\ln n+O(1)}{\ln\ln n}=1.
\end{align*}

\end{proof}

\begin{example}
已知$f(x)\in C[a,b]$,且
\begin{align*}
\int_a^b{f\left( x \right) \mathrm{d}x}=\int_a^b{xf\left( x \right) \mathrm{d}x}=0.
\end{align*}
证明:$f(x)$在$(a,b)$上至少2个零点.
\end{example}
\begin{proof}
设\(F_1(x)=\int_a^x f(t)\mathrm{d}t\),则\(F_1(a)=F_1(b)=0\).再设\(F_2(x)=\int_a^x F_1(t)\mathrm{d}t=\int_a^x\left[\int_a^t f(s)ds\right]\mathrm{d}t\),则\(F_2(a)=0\),\(F_{2}^{\prime}(x)=F_1(x)\),\(F_{2}^{''}(x)=F_{1}^{\prime}(x)=f(x)\).由条件可知
\begin{align*}
0=\int_a^b xf(x)\mathrm{d}x
=\int_a^b xF_{1}^{\prime}(x)\mathrm{d}x
=\int_a^b xdF_1(x)
=xF_1(x)\Big|_{a}^{b}-\int_a^b F_1(x)\mathrm{d}x
=-F_2(b).
\end{align*}
于是由\(Rolle\)中值定理可知,存在\(\xi\in(a,b)\),使得\(F_{2}^{\prime}(\xi)=F_1(\xi)=0\).从而再由\(Rolle\)中值定理可知,存在\(\eta_1\in(a,\xi)\),\(\eta_2\in(\xi,b)\),使得\(F_{1}^{\prime}(\eta_1)=F_{1}^{\prime}(\eta_2)=0\).即\(f(\eta_1)=f(\eta_2)=0\).

\end{proof}

\begin{proposition}\label{proposition:x^kf积分为0则f至少有n+1个零点}
已知$f(x)\in C[a,b]$,且
\begin{align*}
\int_a^b{x^kf\left( x \right) \mathrm{d}x}=0, k=0,1,2,\cdots ,n .
\end{align*}
证明:$f(x)$在$(a,b)$上至少$n+1$个零点.
\end{proposition}
\begin{note}
利用分部积分转换导数的技巧或匹配零点,得到不变号的被积函数.
\end{note}
\begin{proof}
{\color{blue}证法一:}令\(F(x)=\int_a^x\int_a^{x_n}\cdots\int_a^{x_3}\left[\int_a^{x_2}f(x_1)\mathrm{d}x_1\right]\mathrm{d}x_2\cdots \mathrm{d}x_n\).则\(F(a)=F^\prime(a)=\cdots=F^{(n)}(a)=0\),\(F^{(n + 1)}(x)=f(x)\).由已知条件,再反复分部积分,可得当\(1\leqslant k\leqslant n\)且\(k\in\mathbb{N}\)时,有
\begin{align*}
&0=\int_a^b{f\left( x \right) \mathrm{d}x}=\int_a^b{F^{\left( n+1 \right)}\left( x \right) \mathrm{d}x}=F^{\left( n \right)}\left( x \right) \Big|_{a}^{b}=F^{\left( n \right)}\left( b \right) ,
\\
&0=\int_a^b{xf\left( x \right) \mathrm{d}x}=\int_a^b{xF^{\left( n+1 \right)}\left( x \right) \mathrm{d}x}=\int_a^b{xdF^{\left( n \right)}\left( x \right)}=xF^{\left( n \right)}\left( x \right) \Big| _{a}^{b}-\int_a^b{F^{\left( n \right)}\left( x \right) \mathrm{d}x}=-F^{\left( n-1 \right)}\left( b \right) ,
\\
&\cdots \cdots 
\\
&0=\int_a^b{x^nf\left( x \right) \mathrm{d}x}=\int_a^b{x^nF^{\left( n+1 \right)}\left( x \right) \mathrm{d}x}=\int_a^b{x^ndF^{\left( n \right)}\left( x \right)}=x^nF^{\left( n \right)}\left( x \right) \Big| _{a}^{b}-n\int_a^b{x^{n-1}F^{\left( n \right)}\left( x \right) \mathrm{d}x}
\\
&=-n\int_a^b{x^{n-1}F^{\left( n \right)}\left( x \right) \mathrm{d}x}=\cdots =\left( -1 \right) ^nn!\int_a^b{F'\left( x \right) \mathrm{d}x}=\left( -1 \right) ^nn!F\left( b \right) .
\end{align*}
从而\(F(b)=F^\prime(b)=\cdots=F^{(n)}(b)=0\).于是由\(Rolle\)中值定理可知,存在\(\xi_1^1\in(a,b)\),使得\(F^\prime(\xi_1^1)=0\).再利用\(Rolle\)中值定理可知存在\(\xi_1^2,\xi_2^2\in(a,b)\),使得\(F^{\prime\prime}(\xi_1^2)=F^{\prime\prime}(\xi_2^2)=0\).
反复利用\(Rolle\)中值定理可得,存在\(\xi_1^{n + 1},\xi_2^{n + 1},\cdots,\xi_{n + 1}^{n + 1}\in(a,b)\),使得\(F^{(n + 1)}(\xi_1^{n + 1})=F^{(n + 1)}(\xi_2^{n + 1})=\cdots=F^{(n + 1)}(\xi_{n + 1}^{n + 1})=0\).即\(f(\xi_1^{n + 1})=f(\xi_2^{n + 1})=\cdots=f(\xi_{n + 1}^{n + 1})=0\).

{\color{blue}证法二:}假设$f$在$\left( a,b \right)$上只有$s\leqslant n$个不同的零点
$a<x_1<x_2<\cdots <x_s<b.$
由$f$的介值性知,$f$在$\left( x_{i-1},x_i \right) ,i=1,2,\cdots ,s$上不变号.
不妨设$f$在相邻两个区间变号,否则把这两个区间合并.现在
\begin{align*}
\left( x-x_1 \right) \left( x-x_2 \right) \cdots \left( x-x_s \right) f\left( x \right)
\end{align*}
在$\left[ a,b \right]$上不变号.又由条件得
\begin{align*}
\int_a^b{\left( x-x_1 \right) \left( x-x_2 \right) \cdots \left( x-x_s \right) f\left( x \right) \mathrm{d}x}=0,
\end{align*}
故$f\left( x \right) =0,\forall x\in \left[ a,b \right]$.这与$f$在$\left( a,b \right)$上只有$s$个不同的零点矛盾!故结论得证.

\end{proof}

\begin{example}
设
\[
f \in C[a,b], \int_a^b f(x)e^{kx}\mathrm{d}x = 0, k = 1,2,\cdots,n + 1,
\]
证明:$f$在$(a,b)$至少有$n + 1$个不同零点.
\end{example}
\begin{proof}
注意到
\begin{align*}
\int_a^b{f\left( x \right) e^{kx}\mathrm{d}x}\xlongequal{x=\ln t}\int_{e^a}^{e^b}{f\left( \ln t \right) t^{k-1}\mathrm{d}t}=0,\forall k=1,2,\cdots ,n+1.
\end{align*}
即
\begin{align*}
\int_{e^a}^{e^b}{f\left( \ln t \right) t^k\mathrm{d}t}=0,\forall k=0,1,\cdots ,n.
\end{align*}
且$f\left( \ln t \right) \in C\left[ e^a,e^b \right]$,故由\refpro{proposition:x^kf积分为0则f至少有n+1个零点}知$f\left( \ln t \right)$在$\left( e^a,e^b \right)$上至少有$n+1$个不同零点.因此$f$在$\left( a,b \right)$上至少有$n+1$个不同零点.

\end{proof}

\begin{example}
已知$f(x)\in D^2[0,1]$,且
\begin{align*}
\int_0^1{f\left( x \right) \mathrm{d}x}=\frac{1}{6},\int_0^1{xf\left( x \right) \mathrm{d}x}=0,\int_0^1{x^2f\left( x \right) \mathrm{d}x}=\frac{1}{60}.
\end{align*}
证明:存在$\xi \in (0,1)$,使得$f''(\xi)=16$.
\end{example}
\begin{note}
构造$g(x)=f(x)-(8x^2 - 9x + 2)$的原因:受到\hyperref[proposition:x^kf积分为0则f至少有n+1个零点]{上一题}的启发,我们希望找到一个$g(x)=f(x)-p(x)$,使得
\begin{align*}
\int_0^1 x^k g(x)\mathrm{d}x =\int_0^1 x^k [f(x)-p(x)]\mathrm{d}x = 0, \quad k = 0,1,2.
\end{align*}
成立.即
\begin{align*}
\int_0^1{x^kf(x)\mathrm{d}x}=\int_0^1{x^kp(x)\mathrm{d}x},\quad k=0,1,2.
\end{align*}
待定$p(x)=ax^2+bx+c$,则代入上述公式,再结合已知条件可得
\begin{gather*}
\frac{1}{6}=\int_0^1{p(x)\mathrm{d}x}=\int_0^1{\left( ax^2+bx+c \right) \mathrm{d}x}=\frac{a}{3}+\frac{b}{2}+c,
\\
0=\int_0^1{xp(x)\mathrm{d}x}=\int_0^1{\left( ax^3+bx^2+cx \right) \mathrm{d}x}=\frac{a}{4}+\frac{b}{3}+\frac{c}{2},
\\
\frac{1}{60}=\int_0^1{x^2p(x)\mathrm{d}x}=\int_0^1{\left( ax^4+bx^3+cx^2 \right) \mathrm{d}x}=\frac{a}{5}+\frac{b}{4}+\frac{c}{3}.
\end{gather*}
解得:$a=8,b=-9,c=2$.于是就得到\(g(x)=f(x)-(8x^2 - 9x + 2)\).
\end{note}
\begin{proof}
令\(g(x)=f(x)-(8x^2 - 9x + 2)\),则由条件可得
\[
\int_0^1 x^k g(x)\mathrm{d}x = 0, \quad k = 0,1,2.
\]
再令\(G(x)=\int_0^x\left[\int_0^t\left(\int_0^s g(y)\mathrm{d}y\right)ds\right]\mathrm{d}t\),则\(G(0)=G^\prime(0)=G^{\prime\prime}(0)=0\),\(G^{\prime\prime\prime}(x)=g(x)\).利用分部积分可得
\begin{align*}
&0=\int_0^1{g\left( x \right) \mathrm{d}x}=\int_0^1{G'''\left( x \right) \mathrm{d}x}=G''\left( 1 \right) ,
\\
&0=\int_0^1{xg\left( x \right) \mathrm{d}x}=\int_0^1{xG'''\left( x \right) \mathrm{d}x}=\int_0^1{xdG''\left( x \right)}=xG''\left( x \right) \Big |_{0}^{1}-\int_0^1{G''\left( x \right) \mathrm{d}x}=-G'\left( 1 \right) ,
\\
&0=\int_0^1{x^2g\left( x \right) \mathrm{d}x}=\int_0^1{x^2G'''\left( x \right) \mathrm{d}x}=\int_0^1{x^2dG''\left( x \right)}=x^2G''\left( x \right) \Big |_{0}^{1}-2\int_0^1{xG''\left( x \right) \mathrm{d}x}
\\
&=-2\int_0^1{xdG'\left( x \right)}=2\int_0^1{G'\left( x \right) \mathrm{d}x}-2xG'\left( x \right) \Big |_{0}^{1}=2G\left( 1 \right) .
\end{align*}
从而\(G(1)=G^\prime(1)=G^{\prime\prime}(1)=0\).于是由\(Rolle\)中值定理可知,存在\(\xi_1^1\in(0,1)\),使得\(G^\prime(\xi_1^1)=0\).
再利用\(Rolle\)中值定理可知,存在\(\xi_1^2,\xi_2^2\in(0,1)\),使得\(G^{\prime\prime}(\xi_1^2)=G^{\prime\prime}(\xi_2^2)=0\).
反复利用\(Rolle\)中值定理可得,存在\(\xi_1^3,\xi_2^3,\xi_3^3\in(0,1)\),使得\(G^{\prime\prime\prime}(\xi_1^3)=G^{\prime\prime\prime}(\xi_2^3)=G^{\prime\prime\prime}(\xi_3^3)=0\).即\(g(\xi_1^3)=g(\xi_2^3)=g(\xi_3^3)=0\).
再反复利用\(Rolle\)中值定理可得,存在\(\xi\in(0,1)\),使得\(g^{\prime\prime}(\xi)=0\).即\(f^{\prime\prime}(\xi)=16\).

\end{proof}

\begin{example}
设
\[
x_n = \int_0^1 \ln(1 + x + \cdots + x^n) \cdot \ln\frac{1}{1 - x} \mathrm{d}x, n = 1, 2, \cdots
\]
\begin{enumerate}[(1)]
\item 证明: $\lim\limits_{n \to \infty} x_n = 2$.

\item 计算: $\lim\limits_{n \to \infty} \left[ \frac{n}{\ln n} (2 - x_n) \right]$.
\end{enumerate}
\end{example}
\begin{proof}
\begin{enumerate}[(1)]
\item 注意到
\begin{align*}
x_n&=\int_0^1 \ln \frac{1-x^{n+1}}{1-x} \cdot \ln \frac{1}{1-x}\mathrm{d}x,
\end{align*}
于是
\begin{align*}
\int_0^1 \left| \ln \frac{1-x^{n+1}}{1-x} \cdot \ln \frac{1}{1-x} \right| \mathrm{d}x&\leqslant \int_0^1 \ln ^2\frac{1}{1-x}\mathrm{d}x=\int_0^1 \ln ^2x\mathrm{d}x\\
&\xlongequal{\text{分部积分}}-2\int_0^1 \ln x\mathrm{d}x\xlongequal{\text{分部积分}}2.
\end{align*}
从而
\begin{align*}
\lim\limits_{n\rightarrow \infty}x_n&=\int_0^1 \lim\limits_{n\rightarrow \infty}\ln \frac{1-x^{n+1}}{1-x} \cdot \ln \frac{1}{1-x}\mathrm{d}x=\int_0^1 \ln ^2\frac{1}{1-x}\mathrm{d}x\\
&=\int_0^1 \ln ^2x\mathrm{d}x=2.
\end{align*}

\item 注意到
\begin{align*}
x_n&=\int_0^1 \ln \left( 1-x^{n+1} \right) \cdot \ln \frac{1}{1-x}\mathrm{d}x+\int_0^1 \ln ^2\frac{1}{1-x}\mathrm{d}x\\
&=\int_0^1 \ln \left( 1-x^{n+1} \right) \cdot \ln \frac{1}{1-x}\mathrm{d}x+2,
\end{align*}
从而
\begin{align}
2-x_n&=-\int_0^1 \ln \left( 1-x^{n+1} \right) \cdot \ln \frac{1}{1-x}\mathrm{d}x \nonumber \\
&\xlongequal{x=e^{-\frac{y}{n+1}}}\frac{1}{n+1}\int_0^{+\infty} \ln \left( 1-e^{-y} \right) \cdot \ln \left( 1-e^{-\frac{y}{n+1}} \right) \cdot e^{-\frac{y}{n+1}}\mathrm{d}y\nonumber \\
&=\frac{1}{n+1}\int_0^{+\infty} \ln \left( 1-e^{-y} \right) \cdot \ln \left( \frac{1-e^{-\frac{y}{n+1}}}{\frac{y}{n+1}} \right) \cdot e^{-\frac{y}{n+1}}\mathrm{d}y \nonumber \\
&\quad +\frac{1}{n+1}\int_0^{+\infty} \ln \left( 1-e^{-y} \right) \cdot \ln y\cdot e^{-\frac{y}{n+1}}\mathrm{d}y \nonumber \\
&\quad -\frac{\ln \left( n+1 \right)}{n+1}\int_0^{+\infty} \ln \left( 1-e^{-y} \right) e^{-\frac{y}{n+1}}\mathrm{d}y. \label{eq:123.123123123}
\end{align}
注意到$\lim\limits_{x\rightarrow 0}e^{-x}\ln \frac{1-e^x}{x}=\lim\limits_{x\rightarrow +\infty}e^{-x}\ln \frac{1-e^x}{x}=0,$故存在$M>0,$使得
\begin{align*}
\left| e^{-x}\ln \frac{1-e^x}{x} \right|\leqslant M,\forall y\in \left( 0,+\infty \right) .
\end{align*}
又注意到
\begin{align*}
\left| \ln \left( 1-e^{-y} \right) \cdot \ln y\cdot e^{-\frac{y}{n+1}} \right|\leqslant \ln \left( 1-e^{-y} \right) \cdot \ln y,\forall y\in \left( 0,+\infty \right) .
\end{align*}
因此
\begin{align*}
\lim\limits_{n\rightarrow \infty}\int_0^{+\infty} \ln \left( 1-e^{-y} \right) \cdot \ln \left( \frac{1-e^{-\frac{y}{n+1}}}{\frac{y}{n+1}} \right) \cdot e^{-\frac{y}{n+1}}\mathrm{d}y&=\int_0^{+\infty} \lim\limits_{n\rightarrow \infty}\ln \left( 1-e^{-y} \right) \cdot \ln \left( \frac{1-e^{-\frac{y}{n+1}}}{\frac{y}{n+1}} \right) \cdot e^{-\frac{y}{n+1}}\mathrm{d}y\\
&=\int_0^{+\infty} \ln \left( 1-e^{-y} \right) \cdot 0\cdot 1\mathrm{d}y=0.
\end{align*}

\begin{align*}
\lim\limits_{n\rightarrow \infty}\int_0^{+\infty} \ln \left( 1-e^{-y} \right) \cdot \ln y\cdot e^{-\frac{y}{n+1}}\mathrm{d}y&=\int_0^{+\infty} \lim\limits_{n\rightarrow \infty}\ln \left( 1-e^{-y} \right) \cdot \ln y\cdot e^{-\frac{y}{n+1}}\mathrm{d}y\\
&=\int_0^{+\infty} \ln \left( 1-e^{-y} \right) \cdot \ln y\mathrm{d}y\xlongequal{x=e^{-y}}-\int_0^1 \ln \left( 1-x \right) \mathrm{d}x\\
&=\int_0^1 \ln x\mathrm{d}x=1.
\end{align*}

\begin{align*}
\lim\limits_{n\rightarrow \infty}\int_0^{+\infty} \ln \left( 1-e^{-y} \right) e^{-\frac{y}{n+1}}\mathrm{d}y&=\int_0^{+\infty} \lim\limits_{n\rightarrow \infty}\ln \left( 1-e^{-y} \right) e^{-\frac{y}{n+1}}\mathrm{d}y\\
&=\int_0^{+\infty} \ln \left( 1-e^{-y} \right) \mathrm{d}y=-\int_0^{+\infty} \sum\limits_{n=1}^{\infty} \frac{e^{-ny}}{n}\mathrm{d}y\\
&=-\sum\limits_{n=1}^{\infty} \int_0^{+\infty} \frac{e^{-ny}}{n}\mathrm{d}y=-\sum\limits_{n=1}^{\infty} \frac{1}{n^2}=-\frac{\pi ^2}{6}.
\end{align*}
故再由\eqref{eq:123.123123123}式可得
\begin{align*}
\lim\limits_{n\rightarrow \infty}\frac{n}{\ln n}\left( 2-x_n \right)&=\lim\limits_{n\rightarrow \infty}\frac{n}{\left( n+1 \right) \ln n}\int_0^{+\infty} \ln \left( 1-e^{-y} \right) \cdot \ln \left( \frac{1-e^{-\frac{y}{n+1}}}{\frac{y}{n+1}} \right) \cdot e^{-\frac{y}{n+1}}\mathrm{d}y\\
&\quad +\lim\limits_{n\rightarrow \infty}\frac{n}{\left( n+1 \right) \ln n}\int_0^{+\infty} \ln \left( 1-e^{-y} \right) \cdot \ln y\cdot e^{-\frac{y}{n+1}}\mathrm{d}y\\
&\quad -\lim\limits_{n\rightarrow \infty}\frac{n\ln \left( n+1 \right)}{\left( n+1 \right) \ln n}\int_0^{+\infty} \ln \left( 1-e^{-y} \right) e^{-\frac{y}{n+1}}\mathrm{d}y\\
&=\lim\limits_{n\rightarrow \infty}\int_0^{+\infty} \ln \left( 1-e^{-y} \right) e^{-\frac{y}{n+1}}\mathrm{d}y\\
&=\frac{\pi ^2}{6}.
\end{align*}
\end{enumerate}

\end{proof}

\begin{example}
设$f$在$\left[ 0,+\infty \right)$的任意闭区间上Riemann可积.对于$x\geqslant 0$,定义$F\left( x \right) =\int_0^x{t^{\alpha}}f\left( t+x \right) \mathrm{d}t$.
\begin{enumerate}[(1)]
\item 若$\alpha \in \left( -1,0 \right)$ 且$\lim_{x\rightarrow +\infty} f\left( x \right) =A$,证明:$F$在$\left[ 0,+\infty \right) $上一致连续.

\item 若$\alpha \in \left( 0,1 \right)$,$f$以$T>0$为周期,$\int_0^3{f\left( t \right) \mathrm{d}t}=2022$.证明:$F$在$\left[ 0,+\infty \right) $上非一致连续.
\end{enumerate}
\end{example}
\begin{note}
本题(1)中的$\lim_{x\rightarrow +\infty} f\left( x \right) =A$可以削弱为$\exists M,X>0,$使得$\left| f\left( x \right) \right|\leqslant M,x\in \left[ X,+\infty \right) $.
\end{note}
\begin{proof}
\begin{enumerate}[(1)]
\item 由于$f$在$[0,+\infty)$的任意闭区间上Riemann可积且$\lim_{x\rightarrow +\infty} f\left( x \right) =A$,所以
$\exists M>0,$使得$$
\left| f\left( x \right) \right|\leqslant M,\forall x\in \left[ 0,+\infty \right).
$$
对$\forall \varepsilon >0$,取$\delta = \left[ \frac{\left( \alpha +1 \right) \varepsilon}{3M} \right] ^{\frac{1}{\alpha +1}}$,则当\(0\leqslant x<\delta\)时,有
\begin{align*}
x^{\alpha +1}-y^{\alpha +1}<\delta ^{1+\alpha}.
\end{align*}
当\(x\geqslant \delta\)时,有
\begin{align*}
x^{\alpha +1}-y^{\alpha +1}&=\frac{x-y}{\left[ \left( x^{\alpha +1} \right) ^{\frac{1}{\alpha +1}-1}+\left( x^{\alpha +1} \right) ^{\frac{1}{\alpha +1}-2}y^{\alpha +1}+\cdots +\left( y^{\alpha +1} \right) ^{\frac{1}{\alpha +1}-1} \right]} \\
&<\frac{\delta}{\left( x^{\alpha +1} \right) ^{\frac{1}{\alpha +1}-1}}<\frac{\delta}{\left( \delta ^{\alpha +1} \right) ^{\frac{1}{\alpha +1}-1}}=\delta ^{1+\alpha}.
\end{align*}
因此对\(\forall x,y\in \left[ 0,+\infty \right)\)且\(0<x-y<\delta\),都有
\begin{align*}
x^{\alpha +1}-y^{\alpha +1}<\delta ^{1+\alpha}.
\end{align*}
从而对\(\forall x,y\in \left[ 0,+\infty \right)\)且\(0<x-y<\delta\),都有
\begin{align*}
\left| F\left( x \right) -F\left( y \right) \right|&=\left| \int_0^x{t^{\alpha}f\left( t+y \right) \mathrm{d}t}-\int_0^y{t^{\alpha}f\left( t+x \right) \mathrm{d}t} \right|=\left| \int_x^{2x}{\left( t-x \right) ^{\alpha}f\left( t \right) \mathrm{d}t}-\int_y^{2y}{\left( t-y \right) ^{\alpha}f\left( t \right) \mathrm{d}t} \right| \\
&=\left| \int_{2y}^{2x}{\left( t-x \right) ^{\alpha}f\left( t \right) \mathrm{d}t}-\int_y^x{\left( t-y \right) ^{\alpha}f\left( t \right) \mathrm{d}t}+\int_x^{2y}{\left[ \left( t-x \right) ^{\alpha}-\left( t-y \right) ^{\alpha} \right] f\left( t \right) \mathrm{d}t} \right| \\
&\leqslant \int_{2y}^{2x}{\left( t-x \right) ^{\alpha}\left| f\left( t \right) \right|\mathrm{d}t}+\int_y^x{\left( t-y \right) ^{\alpha}\left| f\left( t \right) \right|\mathrm{d}t}+\int_x^{2y}{\left[ \left( t-x \right) ^{\alpha}-\left( t-y \right) ^{\alpha} \right] \left| f\left( t \right) \right|\mathrm{d}t} \\
&\leqslant M\left[ \int_{2y}^{2x}{\left( t-x \right) ^{\alpha}\mathrm{d}t}+\int_y^x{\left( t-y \right) ^{\alpha}\mathrm{d}t}+\int_x^{2y}{\left[ \left( t-x \right) ^{\alpha}-\left( t-y \right) ^{\alpha} \right] \mathrm{d}t} \right]  \\
&=\frac{M}{\alpha +1}\left[ x^{\alpha +1}-\left( 2y-x \right) ^{\alpha +1}+\left( x-y \right) ^{\alpha +1}+\left( 2y-x \right) ^{\alpha +1}-y^{\alpha +1}+\left( x-y \right) ^{\alpha +1} \right]  \\
&=\frac{M}{\alpha +1}\left( x^{\alpha +1}-y^{\alpha +1}+2\left( x-y \right) ^{\alpha +1} \right)  \\
&<\frac{3M}{\alpha +1}\delta ^{1+\alpha}<\varepsilon .
\end{align*}
故$F$在$\left[ 0,+\infty \right) $上一致连续.

\item 假设$F(x)$在$\left[ 0,+\infty \right) $上一致连续.
那么存在$a,b>0$,使得$F\left( x \right) <a\left| x \right|+b$.

从而$\left| \frac{F\left( x \right)}{x^{\alpha +1}} \right|<\frac{a\left| x \right|+b}{\left| x \right|^{\alpha +1}}$,
进而$\lim_{x\rightarrow +\infty} \frac{F\left( x \right)}{x^{\alpha +1}}=0$.

于是
\begin{equation}
\begin{split}
\lim_{x\rightarrow +\infty} \frac{F\left( x \right)}{x^{\alpha +1}}&=\frac{\int_0^x{t^{\alpha}}f\left( t+x \right) \mathrm{d}t}{x^{\alpha +1}}\xlongequal{\text{换元}}\frac{\int_x^{2x}{\left( t-x \right) ^{\alpha}}f\left( t \right) \mathrm{d}t}{x^{\alpha +1}}\xlongequal{\text{换元}}\frac{\int_1^2{x^{\alpha +1}\left( t-1 \right) ^{\alpha}}f\left( tx \right) \mathrm{d}t}{x^{\alpha +1}}
\\
&\xlongequal{\text{\hyperref[theorem:Riemann引理]{Riemann引理}}}\int_1^2{\left( t-1 \right) ^{\alpha}}f\left( tx \right) \mathrm{d}t=\frac{1}{T}\int_0^T{f\left( x \right)}\mathrm{d}t\int_1^2{\left( t-1 \right) ^{\alpha}}\mathrm{d}t=0    
\end{split}
\nonumber
\end{equation}
再结合$\int_1^2{\left( t-1 \right) ^{\alpha}}\mathrm{d}t>0$,
知$\int_0^T{f\left( x \right)}\mathrm{d}t=0$.

现在有
\begin{equation}
\begin{split}
F\left( x \right) &=\int_0^x{t^{\alpha}f\left( t+x \right)}\mathrm{d}t=\int_0^x{t^{\alpha}d}\left[ \int_0^{x+t}{f\left( y \right) dy} \right] 
\\
&\xlongequal{\text{分部积分}}x^{\alpha}\int_0^{2x}{f\left( y \right) dy}-\alpha \int_0^x{t^{\alpha -1}\left[ \int_0^{x+t}{f\left( y \right) dy} \right]}\mathrm{d}t
\\
&=x^{\alpha}\int_0^{2x}{f\left( y \right) dy}-\alpha \int_0^x{t^{\alpha -1}F\left( x+t \right)}\mathrm{d}t
\end{split}
\nonumber
\end{equation}

设$G(x)=\int_0^x{f\left( x \right)}\mathrm{d}t$,则由$f$在$\left[ 0,+\infty \right)$的任意闭区间上Riemann可积知,
$G\in C\left[ 0,+\infty \right) $.
又由$\int_0^T{f\left( x \right)}\mathrm{d}t=0$,得
\begin{equation}
\begin{split}
G\left( x+T \right) -G\left( x \right) =\int_0^x{f\left( x+T \right)}\mathrm{d}t-\int_0^x{f\left( x \right)}\mathrm{d}t=\int_x^{x+T}{f\left( x \right)}\mathrm{d}t=\int_0^T{f\left( x \right)}\mathrm{d}t=0
\end{split}
\nonumber
\end{equation}
因为连续的周期函数必有界,所以$G(x)$有界.

又$\alpha -1\in \left( -1,0 \right) $,
故由(1)可得,$-\alpha \int_0^x{t^{\alpha -1}F\left( x+t \right)}\mathrm{d}t$
在$\left[ 0,+\infty \right) $上一致连续.

下面证明$x^{\alpha}\int_0^{2x}{f\left( y \right) dy}$
不一致连续.

由于$G(2x)$在$\left[ 0,\frac{T}{2} \right] $上连续,所以由连续函数最大、最小值定理知

记$M=\underset{x\in \left[ 0,\frac{T}{2} \right]}{\max}G\left( 2x \right) $,则存在$x_2\in \left[ 0,\frac{T}{2} \right] $,
使得$M=G\left( 2x_2 \right) \geqslant G\left( 2x \right) ,x\in \left[ 0,\frac{T}{2} \right] $.

又因为$G(3)=\int_0^3{f\left( t \right) \mathrm{d}t}=2022$,且$G(2x)$以$\frac{T}{2}$为周期
,所以存在$x_1\in \left[ 0,\frac{T}{2} \right] $,使得$G(2x_1)=G(3)>0$.

因此,$M=G{\left( 2x_2 \right)}\geqslant G(2x_1)=G\left( 3 \right) =\int_0^3{f\left( t \right) \mathrm{d}t}>0$.

构造数集$E=\left\{ x'\in \left[ 0,\frac{T}{2} \right] \mid G\left( 2x^{\prime} \right) =M \right\} $,
由$x_2\in E$知,
$E\ne \varnothing $.
又因为$E$有界,所以由确界存在定理知,$E$必有上确界,取$x_0=supE$.

假设$x_0\notin E$,取$\varepsilon _0=\frac{1}{2}\left| G\left( 2x_0 \right) -M \right|$,
则$\varepsilon _0>0$,否则$x_0\in E$矛盾.

从而$\forall \delta ^{\prime}>0$,$\exists x_{\delta'}\in E$,
使得$ x_0-\delta ^{\prime}<x_{\delta'}<x_0$,
都有$\left| G\left( 2x_0 \right) -G\left( 2x'_{\delta'} \right) \right|\geqslant \varepsilon _0$.

这与G(2x)在闭区间$\left[ 0,\frac{T}{2} \right]$上连续,进而一致连续矛盾.
故$x_0\in E$.

任取$\delta \in \left( 0,\frac{1}{2}\left( \frac{T}{2}-x_0 \right) \right) $,
则$G\left( 2x_0+\delta \right) <M=G\left( 2x_0 \right) $,
否则与$x_0=supE$矛盾.

进而$\left| \int_{2x_0}^{2x_0+\delta}{f\left( y \right) dy} \right|=\left| G\left( 2x_0+\delta \right) -G\left( 2x_0 \right) \right|>0$.

从而当$n>\left( \frac{2}{\delta} \right) ^{\frac{2}{\alpha}}$时,由积分中值定理,得

存在$\xi _n\in \left( 2x_0,2x_0+\frac{2}{n^{\frac{\alpha}{2}}} \right) $,使得
\begin{equation}\label{A}
\left| \int_{2x_0}^{2x_0+\frac{2}{n^{\frac{\alpha}{2}}}}{f\left( y \right) dy} \right|=\frac{2}{n^{\frac{\alpha}{2}}}\left| f\left( \xi _n \right) \right|>0
\end{equation}

又因为$f$在$\left[ 0,+\infty \right)$的任意闭区间上Riemann可积,
所以$f$在$\left( 2x_0,2x_0+\frac{2}{n^{\frac{\alpha}{2}}} \right) $
上有界.

于是存在$K,L> 0$,使得
\begin{equation}\label{B}
K\leqslant \left| f\left( \xi _n \right) \right|\leqslant L
\end{equation}

取数列$\left\{ x_n \right\} \text{、}\left\{ y_n \right\} $,
其中$x_n=x_0+n\frac{T}{2},y_n=x_0+n\frac{T}{2}+\frac{2}{n^{\frac{\alpha}{2}}},n\in \mathbb{N} _+$.
并且$\underset{n\rightarrow +\infty}{\lim}\left( x_n-y_n \right) =\underset{n\rightarrow +\infty}{\lim}\left( \frac{2}{n^{\frac{\alpha}{2}}} \right) =0$.

由拉格朗日中值定理,得
对$\forall n\in \mathbb{N} _+$,

存在$\xi _n\in \left( x_0+n\frac{T}{2},x_0+n\frac{T}{2}+\frac{2}{n^{\frac{\alpha}{2}}} \right) $,使得
$\left( x_0+n\frac{T}{2} \right) ^{\alpha}-\left( x_0+n\frac{T}{2}+\frac{2}{n^{\frac{\alpha}{2}}} \right) ^{\alpha}=\frac{2\alpha}{n^{\frac{\alpha}{2}}}{\xi _n}^{\alpha -1}$

从而
\begin{equation}
\begin{split}
\frac{2\alpha}{n^{\frac{\alpha}{2}}}\left( x_0+n\frac{T}{2} \right) ^{\alpha -1}\leqslant \frac{2\alpha}{n^{\frac{\alpha}{2}}}{\xi _n}^{\alpha -1}\leqslant \frac{2\alpha}{n^{\frac{\alpha}{2}}}\left( x_0+n\frac{T}{2}+\frac{2}{n^{\frac{\alpha}{2}}} \right) ^{\alpha -1}
\end{split}
\nonumber
\end{equation}
令$n\rightarrow +\infty$,有$\lim_{n\rightarrow +\infty} \left[ \left( x_0+n\frac{T}{2} \right) ^{\alpha}-\left( x_0+n\frac{T}{2}+\frac{2}{n^{\frac{\alpha}{2}}} \right) ^{\alpha} \right] =\lim_{n\rightarrow +\infty} \frac{2\alpha}{n^{\frac{\alpha}{2}}}{\xi _n}^{\alpha -1}=0$.

于是存在$N>0$,使得$\forall n>N$,有
\begin{equation}\label{C}
\left( x_0+n\frac{T}{2} \right) ^{\alpha}-\left( x_0+n\frac{T}{2}+\frac{2}{n^{\frac{\alpha}{2}}} \right) ^{\alpha}<\frac{\varepsilon}{\int_0^{2x_0}{f\left( y \right) dy}}
\end{equation}

现在,当$n>\max \left\{ N,\left( \frac{2}{\delta} \right) ^{\frac{2}{\alpha}} \right\} $时,
结合\eqref{A}\eqref{B}\eqref{C},我们有
\begin{align*}
&\left| {x_n}^{\alpha}\int_0^{2x_n}{f\left( y \right) dy}-{y_n}^{\alpha}\int_0^{2y_n}{f\left( y \right) dy} \right|
\\
&=\left| \left( x_0+n\frac{T}{2} \right) ^{\alpha}\int_0^{2\left( x_0+n\frac{T}{2} \right)}{f\left( y \right) dy}-\left( x_0+n\frac{T}{2}+\frac{2}{n^{\frac{\alpha}{2}}} \right) ^{\alpha}\int_0^{2\left( x_0+n\frac{T}{2}+\frac{2}{n^{\frac{\alpha}{2}}} \right)}{f\left( y \right) dy} \right|
\\
&=\left| \left( x_0+n\frac{T}{2} \right) ^{\alpha}\int_0^{2\left( x_0+n\frac{T}{2} \right)}{f\left( y \right) dy}-\left[ \left( x_0+n\frac{T}{2}+\frac{2}{n^{\frac{\alpha}{2}}} \right) ^{\alpha}\int_0^{2\left( x_0+n\frac{T}{2} \right)}{f\left( y \right) dy}+\left( x_0+n\frac{T}{2}+\frac{2}{n^{\frac{\alpha}{2}}} \right) ^{\alpha}\int_{2\left( x_0+n\frac{T}{2} \right)}^{2\left( x_0+n\frac{T}{2}+\frac{2}{n^{\frac{\alpha}{2}}} \right)}{f\left( y \right) dy} \right] \right|
\\
&=\left| \left[ \left( x_0+n\frac{T}{2} \right) ^{\alpha}-\left( x_0+n\frac{T}{2}+\frac{2}{n^{\frac{\alpha}{2}}} \right) ^{\alpha} \right] \int_0^{2\left( x_0+n\frac{T}{2} \right)}{f\left( y \right) dy}-\left( x_0+n\frac{T}{2}+\frac{2}{n^{\frac{\alpha}{2}}} \right) ^{\alpha}\int_{2\left( x_0+n\frac{T}{2} \right)}^{2\left( x_0+n\frac{T}{2}+\frac{2}{n^{\frac{\alpha}{2}}} \right)}{f\left( y \right) dy} \right|
\\
&\geqslant \left| \left( x_0+n\frac{T}{2}+\frac{2}{n^{\frac{\alpha}{2}}} \right) ^{\alpha}\int_{2\left( x_0+n\frac{T}{2} \right)}^{2\left( x_0+n\frac{T}{2}+\frac{2}{n^{\frac{\alpha}{2}}} \right)}{f\left( y \right) dy} \right|-\left| \left[ \left( x_0+n\frac{T}{2} \right) ^{\alpha}-\left( x_0+n\frac{T}{2}+\frac{2}{n^{\frac{\alpha}{2}}} \right) ^{\alpha} \right] \int_0^{2\left( x_0+n\frac{T}{2} \right)}{f\left( y \right) dy} \right|
\\
&=\left| \left( x_0+n\frac{T}{2}+\frac{2}{n^{\frac{\alpha}{2}}} \right) ^{\alpha}\int_{2x_0}^{2x_0+\frac{2}{n^{\frac{\alpha}{2}}}}{f\left( y \right) dy} \right|-\left| \left[ \left( x_0+n\frac{T}{2} \right) ^{\alpha}-\left( x_0+n\frac{T}{2}+\frac{2}{n^{\frac{\alpha}{2}}} \right) ^{\alpha} \right] \int_0^{2x_0}{f\left( y \right) dy} \right|
\\
&=\left| \left( x_0+n\frac{T}{2}+\frac{2}{n^{\frac{\alpha}{2}}} \right) ^{\alpha}\right|\cdot\left| \frac{2}{n^{\frac{\alpha}{2}}}f\left( \xi _n \right) \right|-\left| \left[ \left( x_0+n\frac{T}{2} \right) ^{\alpha}-\left( x_0+n\frac{T}{2}+\frac{2}{n^{\frac{\alpha}{2}}} \right) ^{\alpha} \right] \int_0^{2x_0}{f\left( y \right) dy} \right|
\\
&\geqslant 2\left( \frac{T}{2} \right) ^{\alpha}\left|f\left( \xi _n \right)\right| \cdot n^{\frac{\alpha}{2}}-\varepsilon 
\\
&\geqslant 2\left( \frac{T}{2} \right) ^{\alpha}K\cdot n^{\frac{\alpha}{2}}-\varepsilon 
\\
&\,\,\,\,\,\,\,\,\,\,\,\, \text{令}n\rightarrow +\infty ,\text{有}\underset{n\rightarrow +\infty}{\lim}\left( {x_n}^{\alpha}\int_0^{2x_n}{f\left( y \right) dy}-{y_n}^{\alpha}\int_0^{2y_n}{f\left( y \right) dy} \right) =+\infty .  
\text{故}x^{\alpha}\int_0^{2x}{f\left( y \right) dy}\text{在}\left[ 0,+\infty \right)\text{上非一致连续.}
\\
&\text{这与$F(x)$在$\left[ 0,+\infty \right) $上一致连续矛盾.因此,}F\text{在}\left[ 0,+\infty \right)\text{上非一致连续.}
\nonumber
\end{align*}
\end{enumerate}

\end{proof}

\begin{example}
计算
\begin{align*}
\lim_{t\rightarrow 1^-} (1-t)\left( \frac{t}{1+t}+\frac{t^2}{1+t^2}+\cdots +\frac{t^n}{1+t^n}+\cdots \right)
\end{align*}
\end{example}
\begin{solution}
对$\forall t \in (0,1)$,一方面,我们有
\begin{align*}
\left( 1-t \right) \sum\limits_{k=1}^{\infty}{\frac{t^k}{1+t^k}}=\sum\limits_{k=1}^{\infty}{\int_{t^{k+1}}^{t^k}{\frac{1}{1+t^k}\mathrm{d}x}}\leqslant \sum\limits_{k=1}^{\infty}{\int_{t^{k+1}}^{t^k}{\frac{1}{1+x}\mathrm{d}x}}=\int_0^t{\frac{1}{1+x}\mathrm{d}x}=\ln \left( 1+t \right)
\end{align*}
另一方面,我们有
\begin{align*}
\left( 1-t \right) \sum\limits_{k=1}^{\infty}{\frac{t^k}{1+t^k}}=\sum\limits_{k=1}^{\infty}{\int_{t^k}^{t^{k-1}}{\frac{t}{1+t^k}\mathrm{d}x}}\geqslant \sum\limits_{k=1}^{\infty}{\int_{t^k}^{t^{k-1}}{\frac{t}{1+x}\mathrm{d}x}}=\int_0^1{\frac{t}{1+x}\mathrm{d}x}=t\ln 2.
\end{align*}
故
\begin{align*}
\ln 2=\lim_{t\rightarrow 1^-} \left[ \ln \left( 1+t \right) \right] \leqslant \lim_{t\rightarrow 1^-} \left( 1-t \right) \sum\limits_{k=1}^{\infty}{\frac{t^k}{1+t^k}}\leqslant \lim_{t\rightarrow 1^-} \left( t\ln 2 \right) =\ln 2.
\end{align*}

\end{solution}

\begin{example}
计算极限 $\lim\limits_{n \to \infty} \int_{0}^{n} \frac{\mathrm{d}x}{1 + n^2 \cos^2 x}$.
\end{example}
\begin{solution}
对$\forall n\in \mathbb{N}$,存在$t_n\in \mathbb{N}$,使得$\left( t_n+1 \right) \pi <n<\left( t_n+2 \right) \pi$.从而
$n-7<t_n\pi <n,\quad \forall n\in \mathbb{N}$.
于是$\lim\limits_{n\rightarrow \infty}\frac{t_n}{n}=\frac{1}{\pi}$.现在我们有
\begin{align}
\int_0^n{\frac{\mathrm{d}x}{1+n^2\cos ^2x}}&=\sum\limits_{k=0}^{t_n}{\int_{k\pi}^{\left( k+1 \right) \pi}{\frac{\mathrm{d}x}{1+n^2\cos ^2x}}}+\int_{\left( t_n+1 \right) \pi}^n{\frac{\mathrm{d}x}{1+n^2\cos ^2x}} \nonumber \\
&=\sum\limits_{k=0}^{t_n}{\int_0^{\pi}{\frac{\mathrm{d}x}{1+n^2\cos ^2\left( x+k\pi \right)}}}+\int_0^{n-\left( t_n+1 \right) \pi}{\frac{\mathrm{d}x}{1+n^2\cos ^2\left( x+\left( t_n+1 \right) \pi \right)}} \nonumber \\
&=t_n\int_0^{\pi}{\frac{\mathrm{d}x}{1+n^2\cos ^2x}}+\int_0^{n-\left( t_n+1 \right) \pi}{\frac{\mathrm{d}x}{1+n^2\cos ^2x}}. \label{eq:177.12}
\end{align}
注意到对$\forall n\in \mathbb{N}$,都有
\begin{align*}
\left| \frac{1}{1+n^2\cos ^2x} \right|&\leqslant \frac{1}{1+\cos ^2x}, \\
n\left| \frac{1}{1+n^2\cos ^2x}-\frac{1}{1+n^2x^2} \right|&=\left| \frac{n^3\left( x^2-\cos ^2x \right)}{\left( 1+n^2\cos ^2x \right) \left( 1+n^2x^2 \right)} \right| \\
&\leqslant \left| \frac{n^3\left( x^2-\cos ^2x \right)}{n^4x^2\cos ^2x} \right|\leqslant \frac{\left| x^2-\cos ^2x \right|}{x^2\cos ^2x},
\end{align*}
故由\hyperref[Real Analysis-theorem:控制收敛定理]{控制收敛定理}知
\begin{align}
\lim\limits_{n\rightarrow \infty}\int_0^{n-\left( t_n+1 \right) \pi}{\frac{\mathrm{d}x}{1+n^2\cos ^2x}}\leqslant \lim\limits_{n\rightarrow \infty}\int_0^{\pi}{\frac{\mathrm{d}x}{1+n^2\cos ^2x}}=\int_0^{\pi}{\lim\limits_{n\rightarrow \infty}\frac{\mathrm{d}x}{1+n^2\cos ^2x}}=0, \label{eq:177.13}
\end{align}
\begin{align*}
\lim\limits_{n\rightarrow \infty}n\int_0^{\pi}{\left( \frac{1}{1+n^2\cos ^2x}-\frac{1}{1+n^2x^2} \right) \mathrm{d}x}&=\int_0^{\pi}{\lim\limits_{n\rightarrow \infty}\left( \frac{n}{1+n^2\cos ^2x}-\frac{n}{1+n^2x^2} \right) \mathrm{d}x}=0.
\end{align*}
因此
\begin{align}
&\lim\limits_{n\rightarrow \infty}t_n\int_0^{\pi}{\frac{\mathrm{d}x}{1+n^2\cos ^2x}}=\lim\limits_{n\rightarrow \infty}\frac{t_n}{n}\cdot n\int_0^{\pi}{\frac{\mathrm{d}x}{1+n^2\cos ^2x}} \nonumber \\
&=\frac{1}{\pi}\lim\limits_{n\rightarrow \infty}\left[ n\int_0^{\pi}{\frac{1}{1+n^2x^2}}+n\int_0^{\pi}{\left( \frac{1}{1+n^2\cos ^2x}-\frac{1}{1+n^2x^2} \right) \mathrm{d}x} \right] \nonumber \\
&=\frac{1}{\pi}\lim\limits_{n\rightarrow \infty}n\int_0^{\pi}{\frac{\mathrm{d}x}{1+n^2x^2}}=\frac{1}{\pi}\lim\limits_{n\rightarrow \infty}\int_0^{n\pi}{\frac{\mathrm{d}x}{1+x^2}}=1. \label{eq:177.14}
\end{align}
综上,由\eqref{eq:177.12}\eqref{eq:177.13}\eqref{eq:177.14}式知
\begin{align*}
\lim_{n\rightarrow \infty} \int_0^n{\frac{\mathrm{d}x}{1+n^2\cos ^2x}}=1.
\end{align*}

\end{solution}

\begin{example}
计算积分$\int_{0}^{+\infty} \frac{x - x^2 + x^3 - x^4 + \cdots - x^{2018}}{(1 + x)^{2021}} \mathrm{d}x$.
\end{example}
\begin{proof}
注意到对$\forall k\in \left[ 1,2018 \right] \cap \mathbb{N}$,都有
$$\int_0^{+\infty}{\frac{x^k}{\left( 1+x \right) ^{2021}}\mathrm{d}x}\xlongequal{t=\frac{1}{x}}\int_0^{+\infty}{\frac{t^{2018-k}}{\left( 1+t \right) ^{2021}}\mathrm{d}t}=\int_0^{+\infty}{\frac{x^{2018-k}}{\left( 1+x \right) ^{2021}}\mathrm{d}x}.$$
故
$$\int_0^{+\infty}{\frac{x^k-x^{2018-k}}{\left( 1+x \right) ^{2021}}\mathrm{d}x}=0,\quad \forall k\in \left[ 1,2018 \right] \cap \mathbb{N} .$$
因此
$$\int_0^{+\infty}{\frac{x-x^2+x^3-x^4+\cdots -x^{2018}}{\left( 1+x \right) ^{2021}}\mathrm{d}x}=0.$$

\end{proof}

\begin{example}
设$I(f) = \int_{0}^{\pi} (\sin x - f(x))f(x) \mathrm{d}x$,求当遍历$[0, \pi]$上所有连续函数$f$时$I(f)$的最大值.
\end{example}
\begin{solution}
对函数配方,有
\begin{align*}
(\sin x - f(x))f(x) = -\left(f(x) - \frac{\sin x}{2}\right)^2 + \frac{\sin^2 x}{4}.
\end{align*}
代入积分式,得
\begin{align*}
I(f) &= \int_{0}^{\pi} \frac{\sin^2 x}{4} \mathrm{d}x - \int_{0}^{\pi} \left(f(x) - \frac{\sin x}{2}\right)^2 \mathrm{d}x \\
&= \frac{\pi}{8} - \int_{0}^{\pi} \left(f(x) - \frac{\sin x}{2}\right)^2 \mathrm{d}x.
\end{align*}
故当\(f(x) = \frac{\sin x}{2}\)时,\(I(f)\)取得最大值\(\frac{\pi}{8}\).

\end{solution}

\begin{example}
设$\alpha > 1$,$\Gamma_k = \left[ k^\alpha, \left( k + \frac{1}{2} \right)^\alpha \right) \cap \mathbb{N} (k \geq 1)$。试判断级数$\sum\limits_{n=1}^{\infty} a_n$和$\sum\limits_{n=1}^{\infty} b_n$的敛散性,其中
$$a_n = \begin{cases}
\frac{1}{n}, & \text{存在} k \text{使得} n = \min \Gamma_k, \\
\frac{1}{n^\alpha}, & \text{其他},
\end{cases} \quad b_n = \begin{cases}
\frac{1}{n}, & \text{存在} k \text{使得} n \in \Gamma_k, \\
\frac{1}{n^\alpha}, & \text{其他}.
\end{cases}$$
\end{example}
\begin{proof}
由$\alpha >1$和条件直接可得
\begin{align*}
\sum\limits_{n=1}^{\infty}a_n=
\sum\limits_{\substack{n\ne \min\,\Gamma_k,\\\forall k\in \mathbb{N}}}^{\infty}\frac{1}{n^{\alpha}}+
\sum\limits_{\substack{\exists k\in \mathbb{N},\\n=\min\,\Gamma_k}}^{\infty}\frac{1}{n}
\leqslant \sum\limits_{n=1}^{\infty}\frac{1}{n^{\alpha}}+
\sum\limits_{k=1}^{\infty}\frac{1}{\min\,\Gamma_k}=
\sum\limits_{n=1}^{\infty}\frac{1}{n^{\alpha}}+
\sum\limits_{k=1}^{\infty}\frac{1}{k^{\alpha}}<\infty.
\end{align*}
注意到
\begin{align}
\sum\limits_{n=1}^{\infty}{b_n}&=\sum\limits_{n\notin \bigcup_{k\in \mathbb{N}}{\Gamma _k}}^{\infty}{\frac{1}{n^{\alpha}}}+\sum\limits_{n\in \bigcup_{k\in \mathbb{N}}{\Gamma _k}}^{\infty}{\frac{1}{n}}=\sum\limits_{n\notin \bigcup_{k\in \mathbb{N}}{\Gamma _k}}^{\infty}{\frac{1}{n^{\alpha}}}+\sum\limits_{k=1}^{\infty}{\sum\limits_{n\in \Gamma _k}{\frac{1}{n}}}
\nonumber \\
&\geqslant \sum\limits_{n\notin \bigcup_{k\in \mathbb{N}}{\Gamma _k}}^{\infty}{\frac{1}{n^{\alpha}}}+\sum\limits_{k=1}^{\infty}{\sum\limits_{n\in \Gamma _k}{\frac{1}{\left( k+\frac{1}{2} \right) ^{\alpha}}}}.\label{eq:188.897}
\end{align}
记$N_k$为$\Gamma_k$所含元素的个数,则
$$\lfloor \left( k+\frac{1}{2} \right) ^{\alpha}-k^{\alpha} \rfloor \leqslant N_k<\lfloor \left( k+\frac{1}{2} \right) ^{\alpha}-k^{\alpha} \rfloor +1,\quad \forall k\in \mathbb{N} .$$
再结合Lagrange中值定理知
$$N_k\sim \left( k+\frac{1}{2} \right) ^{\alpha}-k^{\alpha}\sim \frac{\alpha}{2}k^{\alpha -1},\quad k\rightarrow \infty .$$
而
$$\sum\limits_{k=1}^{\infty}{\frac{\frac{\alpha}{2}k^{\alpha -1}}{\left( k+\frac{1}{2} \right) ^{\alpha}}}\geqslant \frac{\alpha}{2}\sum\limits_{k=1}^{\infty}{\frac{k^{\alpha -1}}{2^{\alpha}k^{\alpha}}}=\frac{\alpha}{2^{\alpha +1}}\sum\limits_{k=1}^{\infty}{\frac{1}{k}}=+\infty .$$
因此
$$\sum\limits_{k=1}^{\infty}{\sum\limits_{n\in \Gamma _k}{\frac{1}{\left( k+\frac{1}{2} \right) ^{\alpha}}}}=\sum\limits_{k=1}^{\infty}{\frac{N_k}{\left( k+\frac{1}{2} \right) ^{\alpha}}}=+\infty .$$
故由$\eqref{eq:188.897}$式知$\sum\limits_{n=1}^{\infty}{b_n}$发散.

\end{proof}

\begin{example}
证明: 
\begin{enumerate}[(1)]
\item $\lim\limits_{\alpha \to 0^+} \sum\limits_{n=1}^{\infty} \frac{\cos\left(n + \frac{1}{2}\right)}{n^{1+\alpha}} = \sum\limits_{n=1}^{\infty} \frac{\cos\left(n + \frac{1}{2}\right)}{n}$;

\item 计算 $\lim\limits_{\alpha \to 0^+} \sum\limits_{n=1}^{\infty} \frac{\sin n}{n^{\alpha}}$,并说明理由.
\end{enumerate}
\end{example}
\begin{proof}
\begin{enumerate}[(1)]
\item 注意到对$\forall \alpha \geqslant 0$,都有
\begin{align*}
\left| \frac{\cos \left( n+\frac{1}{2} \right)}{n^{1+\alpha}} \right| \leqslant \frac{|\cos \left( n+\frac{1}{2} \right)|}{n},
\end{align*}
并且对$\forall n\in \mathbb{N}$,有
\begin{align*}
\sum\limits_{k=1}^n \cos \left( k+\frac{1}{2} \right)&=\frac{\sum\limits_{k=1}^n 2\sin \frac{1}{2}\cos \left( k+\frac{1}{2} \right)}{2\sin \frac{1}{2}}=\frac{\sum\limits_{k=1}^n \left[ \sin \left( k+1 \right) -\sin k \right]}{2\sin \frac{1}{2}}\\
&=\frac{\sin \left( n+1 \right) -\sin 1}{2\sin \frac{1}{2}} \leqslant \frac{1}{\sin \frac{1}{2}}.
\end{align*}
故由Dirichlet判别法知$\sum\limits_{n=1}^{\infty} \frac{\cos \left( n+\frac{1}{2} \right)}{n}$收敛,因此$\sum\limits_{n=1}^{\infty} \frac{\cos \left( n+\frac{1}{2} \right)}{n^{1+\alpha}}$关于$\alpha \geqslant 0$一致收敛.从而
\begin{align*}
\lim_{\alpha \rightarrow 0^+} \sum\limits_{n=1}^{\infty} \frac{\cos \left( n+\frac{1}{2} \right)}{n^{1+\alpha}}=\sum\limits_{n=1}^{\infty} \lim_{\alpha \rightarrow 0^+} \frac{\cos \left( n+\frac{1}{2} \right)}{n^{1+\alpha}}=\sum\limits_{n=1}^{\infty} \frac{\cos \left( n+\frac{1}{2} \right)}{n}.
\end{align*}

\item 注意到对$\forall \alpha \in \left( 0,1 \right)$,都有
\begin{align*}
\sum\limits_{n=1}^{\infty} \frac{\sin n}{n^{\alpha}}&=\sum\limits_{n=1}^{\infty} \frac{2\sin \frac{1}{2}\sin n}{2n^{\alpha}\sin \frac{1}{2}}=\sum\limits_{n=1}^{\infty} \frac{\cos \left( n-\frac{1}{2} \right) -\cos \left( n+\frac{1}{2} \right)}{2n^{\alpha}\sin \frac{1}{2}}\\
&=\sum\limits_{n=1}^{\infty} \frac{\cos \left( n-\frac{1}{2} \right)}{2n^{\alpha}\sin \frac{1}{2}}-\sum\limits_{n=1}^{\infty} \frac{\cos \left( n+\frac{1}{2} \right)}{2n^{\alpha}\sin \frac{1}{2}}\\
&=\frac{\cos \frac{1}{2}}{2\sin \frac{1}{2}}+\sum\limits_{n=2}^{\infty} \frac{\cos \left( n-\frac{1}{2} \right)}{2n^{\alpha}\sin \frac{1}{2}}-\sum\limits_{n=1}^{\infty} \frac{\cos \left( n+\frac{1}{2} \right)}{2n^{\alpha}\sin \frac{1}{2}}\\
&=\frac{\cos \frac{1}{2}}{2\sin \frac{1}{2}}+\sum\limits_{n=1}^{\infty} \frac{\cos \left( n+\frac{1}{2} \right)}{2\left( n+1 \right) ^{\alpha}\sin \frac{1}{2}}-\sum\limits_{n=1}^{\infty} \frac{\cos \left( n+\frac{1}{2} \right)}{2n^{\alpha}\sin \frac{1}{2}}\\
&=\frac{\cos \frac{1}{2}}{2\sin \frac{1}{2}}+\frac{1}{2\sin \frac{1}{2}}\sum\limits_{n=1}^{\infty} \left[ \frac{1}{\left( n+1 \right) ^{\alpha}}-\frac{1}{n^{\alpha}} \right] \cos \left( n+\frac{1}{2} \right).
\end{align*}
由Lagrange中值定理知,对$\forall n\in \mathbb{N}$,存在$\xi \in \left( n,n+1 \right)$,使得
\begin{align*}
\sum\limits_{n=1}^{\infty} \left[ \frac{1}{\left( n+1 \right) ^{\alpha}}-\frac{1}{n^{\alpha}} \right] \cos \left( n+\frac{1}{2} \right)=-\alpha \sum\limits_{n=1}^{\infty} \frac{\cos \left( n+\frac{1}{2} \right)}{\xi ^{1+\alpha}} \leqslant \alpha \sum\limits_{n=1}^{\infty} \frac{|\cos \left( n+\frac{1}{2} \right)|}{n^{1+\alpha}}.
\end{align*}
于是再结合(1)的结论可得
\begin{align*}
\lim_{\alpha \rightarrow 0^+} \sum\limits_{n=1}^{\infty} \left[ \frac{1}{\left( n+1 \right) ^{\alpha}}-\frac{1}{n^{\alpha}} \right] \cos \left( n+\frac{1}{2} \right)=-\lim_{\alpha \rightarrow 0^+} \alpha \sum\limits_{n=1}^{\infty} \frac{\cos \left( n+\frac{1}{2} \right)}{\xi ^{1+\alpha}}=0.
\end{align*}
故
\begin{align*}
\lim_{\alpha \rightarrow 0^+} \sum\limits_{n=1}^{\infty} \frac{\sin n}{n^{\alpha}}=\frac{\cos \frac{1}{2}}{2\sin \frac{1}{2}}.
\end{align*}
\end{enumerate}

\end{proof}

\begin{example}
计算广义积分$\int_{1}^{+\infty} \frac{(x)}{x^3} \mathrm{d}x$,这里$(x)$表示$x$的小数部分(例如:当$n$为正整数且$x \in [n, n+1)$时,则$(x) = x - n$).
\end{example}
\begin{proof}
注意到$(x)$是周期为1的函数,并且在$[0,1)$上恒有$(x)=x$.因此
\begin{align*}
\int_1^{+\infty}{\frac{\left( x \right)}{x^3}\mathrm{d}x}&=\sum\limits_{k=1}^{\infty}{\int_k^{k+1}{\frac{\left( x \right)}{x^3}\mathrm{d}x}}=\sum\limits_{k=1}^{\infty}{\int_0^1{\frac{\left( x+k \right)}{\left( x+k \right) ^3}\mathrm{d}x}}
\\
&=\sum\limits_{k=1}^{\infty}{\int_0^1{\frac{\left( x \right)}{\left( x+k \right) ^3}\mathrm{d}x}}=\sum\limits_{k=1}^{\infty}{\int_0^1{\frac{x}{\left( x+k \right) ^3}\mathrm{d}x}}
\\
&=\sum\limits_{k=1}^{\infty}{\int_0^1{\left[ \frac{1}{\left( x+k \right) ^2}-\frac{k}{\left( x+k \right) ^3} \right] \mathrm{d}x}}
\\
&=\sum\limits_{k=1}^{\infty}{\left[ \frac{1}{k}-\frac{1}{k+1}+\frac{1}{2}\left( \frac{k}{\left( 1+k \right) ^2}-\frac{1}{k} \right) \right]}
\\
&=\sum\limits_{k=1}^{\infty}{\left[ \frac{1}{2k}-\frac{1}{2\left( k+1 \right)}+\frac{1}{2}\left( \frac{k}{\left( 1+k \right) ^2}-\frac{1}{1+k} \right) \right]}
\\
&=\sum\limits_{k=1}^{\infty}{\left[ \frac{1}{2k}-\frac{1}{2\left( k+1 \right)}-\frac{1}{2\left( 1+k \right) ^2} \right]}
\\
&=\sum\limits_{k=1}^{\infty}{\left[ \frac{1}{2k}-\frac{1}{2\left( k+1 \right)} \right]}-\frac{1}{2}\sum\limits_{k=2}^{\infty}{\frac{1}{k ^2}}
\\
&=\frac{1}{2}-\frac{1}{2}\left( \sum\limits_{k=1}^{\infty}{\frac{1}{k ^2}}-1 \right) 
\\
&=1-\frac{1}{2}\cdot \frac{\pi ^2}{6}=1-\frac{\pi ^2}{12}.
\end{align*}

\end{proof}

\begin{example}
设$f(x)$在$\mathbb{R}$上连续,且$g(x)=f(x)\int_0^x f(t)\mathrm{d}t$单调递减,证明$f(x)\equiv 0$。
\end{example}
\begin{proof}
{\color{blue}证法一:}
令$F\left( x \right) =\int_0^x{f\left( t \right) \mathrm{d}t}$,$G\left( x \right) =\frac{F^2\left( x \right)}{2}$,则
\[
g\left( x \right) =F\left( x \right) F^\prime \left( x \right) =\left[ \frac{F^2\left( x \right)}{2} \right] ^\prime =G^\prime \left( x \right),
\]
由条件知$g\left( x \right) =G^\prime \left( x \right)$单调递减,故$G\left( x \right)$是上凸函数。注意到$G^\prime \left( 0 \right) =g\left( 0 \right) =0$,由$g$递减知,$G^\prime \left( x \right) =g\left( x \right) \leqslant 0$,$\forall x>0$。从而$G\left( x \right)$在$\left[ 0,+\infty \right)$上递减。故
\[
0\leqslant G\left( x \right) =\frac{F^2\left( x \right)}{2}\leqslant G\left( 0 \right) =0.
\]
因此$G\left( x \right) =0$,$\forall x\geqslant 0$。于是$F\left( x \right) \equiv 0$,故$f\left( x \right) =F^\prime \left( x \right) =0$,$\forall x\geqslant 0$。

{\color{blue}证法二:}
证明: 若$\exists X_0>0$, s.t. $f(X_0)\neq 0$. 不妨设$f(X_0)>0$,
则由$g$递减知
$$g(X_0)=f(X_0)\int_0^{X_0} f(t)\mathrm{d}t \leqslant g(0)=0 \Rightarrow \int_0^{X_0} f(t)\mathrm{d}t \leqslant 0.$$
从而由积分中值定理知, $\exists \xi \in (0,X_0), s.t. f(\xi) \leqslant 0$
于是由介值定理知, $\exists x_1 \in (0,X_0), s.t. f(x_1)=0.$
记
$$x_2 \triangleq \sup\{x \in [x_1,X_0) \mid f(x)=0\}.$$
则$f(x)>0$, $\forall x \in (x_2,X_0)$, 否则, $$\exists \eta \in (x_2,X_0), s.t. f(\eta) \leqslant 0.$$
由介值定理知, $\exists \eta' \in (x_2,X_0)$, s.t. $f(\eta')=0$. 这与上确界定义矛盾! 再记
$f(x') = \max\limits_{x \in [x_2,X_0]} f(x)$,
任取$x_3 \in (x_2,x')$. 则$f(x_3)>0$, 进而$\int_{x_3}^{x'} f(t)\mathrm{d}t >0$. 于是
\begin{align*}
g(x_3) &= f(x_3)\int_0^{x_3} f(t)\mathrm{d}t < f(x')\left( \int_0^{x_3} f(t)\mathrm{d}t + \int_{x_3}^{x'} f(t)\mathrm{d}t \right)\\
&= f(x')\int_0^{x'} f(t)\mathrm{d}t = g(x').
\end{align*}
这与$g$递减矛盾! 故$f(x)=0$, $\forall x>0$. 同理可得$f(x)=0$, $\forall x<0$.
再由$f$的连续性可知$f(0)=0$, 故$f(x) \equiv 0$.


\end{proof}

\begin{example}
设 \( f \in C^1[0,+\infty) \) 满足
\[
|f(x)| \leqslant e^{-\sqrt{x}}, f'(x) = -3f(x) + 6f(2x), \forall x \geqslant 0.
\]
证明:
\[
\underset{n\rightarrow \infty}{\lim}\frac{3^n}{n!}\int_0^{\infty}{x^nf(x)\,\mathrm{d}x}<\infty.
\]
并且证明
\[
\underset{n\rightarrow \infty}{\lim}\frac{3^n}{n!}\int_0^{\infty}{x^nf(x)\,\mathrm{d}x} = 0
\]
的充要条件是 \( \int_0^{\infty} f(x) \, \mathrm{d}x = 0 \).
\end{example}
\begin{proof}
由 \(f'(x)=-3f(x)+6f(2x)\) 可得
\begin{align*}
I_n&\triangleq \frac{3^n}{n!}\int_0^{\infty}{x^nf(x)\mathrm{d}x\xlongequal{\text{分部积分}}}-\frac{3^n}{(n+1)!}\int_0^{\infty}{x^{n+1}f\prime (x)\mathrm{d}x}
\\
&=\frac{3^{n+1}}{(n+1)!}\int_0^{\infty}{x^{n+1}\left[ f(x)-2f(2x) \right] \mathrm{d}x}
\\
&=\frac{3^{n+1}}{(n+1)!}\left[ \int_0^{\infty}{x^{n+1}f(x)\mathrm{d}x}-2\int_0^{\infty}{x^{n+1}f(2x)\mathrm{d}x} \right] 
\\
&=\frac{3^{n+1}}{(n+1)!}\left[ \int_0^{\infty}{x^{n+1}f(x)\mathrm{d}x}-\frac{1}{2^{n+1}}\int_0^{\infty}{x^{n+1}f(x)\mathrm{d}x} \right] 
\\
&=\frac{2^{n+1}-1}{2^{n+1}}\cdot \frac{3^{n+1}}{(n+1)!}\int_0^{\infty}{x^{n+1}f(x)\mathrm{d}x}=\frac{2^{n+1}-1}{2^{n+1}}\cdot I_{n+1}.
\end{align*}
故
\[
I_{n+1}=\frac{2^{n+1}}{2^{n+1}-1}I_n,\forall n\in \mathbb{N}.
\]
从而对 \(\forall n\in \mathbb{N}\),都有
\begin{align*}
I_n=\frac{2^n}{2^n-1}I_{n-1}=\cdots =\frac{2^n}{2^n-1}\cdot \frac{2^{n-1}}{2^{n-1}-1}\cdots \frac{2}{2-1}I_0=I_0\prod_{k=1}^n{\frac{2^k}{\left( 2^k-1 \right)}}.
\end{align*}
注意到
\begin{align*}
\sum\limits_{k=1}^n{\ln \left( 1-\frac{1}{2^k} \right)}\geqslant \sum\limits_{k=1}^n{\left( -\frac{2}{2^k} \right)}=-2\cdot \frac{1-\frac{1}{2^n}}{1-\frac{1}{2}}=-4\left( 1-\frac{1}{2^n} \right),\forall n\in \mathbb{N} .
\end{align*}
因此对$\forall n\in \mathbb{N}$,都有
\begin{align*}
\prod_{k=1}^n{\frac{2^k}{\left( 2^k-1 \right)}}=\frac{1}{\prod\limits_{k=1}^n{\left( 1-\frac{1}{2^k} \right)}}=e^{-\sum\limits_{k=1}^n{\ln \left( 1-\frac{1}{2^k} \right)}}\leqslant e^{4\left( 1-\frac{1}{2^n} \right)}\leqslant e^4.
\end{align*}
故$\prod_{k=1}^{\infty}{\frac{2^k}{\left( 2^k-1 \right)}}$收敛.于是
\begin{align*}
0=\underset{n\rightarrow \infty}{\lim}I_n=I_0\prod_{k=1}^{\infty}{\frac{2^k}{\left( 2^k-1 \right)}}\Longleftrightarrow I_0=0.
\end{align*}

\end{proof}

\begin{example}
设连续函数 $g:[0,+\infty)\to(0,+\infty)$ 满足 $g$ 单调递减. 设
\[
x_0>0, x_{n+1}=x_n+g(x_n), n=0,1,2,\cdots,
\]
以及 $x\in D^1[0,+\infty)$ 满足
\[
x(0)=x_0 ,x'(t)=g(x(t)), \forall t\geqslant0.
\]
证明
\[
x_n=x(n)+O(1), n\to\infty.
\]
\end{example}
\begin{proof}
由条件可知
\begin{align*}
x_{n+1}-x_n=g\left( x_n \right) >0,\quad x^\prime \left( t \right) =g\left( x\left( t \right) \right) >0.
\end{align*}
故$\left\{ x_n \right\}$严格递增,$x\left( t \right)$在$\left[ 0,+\infty \right)$上也严格递增.由条件知$x_0\geqslant x\left( 0 \right)$,假设$x_n\geqslant x\left( n \right)$,则由条件知
\begin{align*}
n=\int_0^n{\frac{x^\prime \left( t \right)}{g\left( x(t) \right)}\mathrm{d}t}=\int_{x\left( 0 \right)}^{x\left( n \right)}{\frac{1}{g\left( t \right)}\mathrm{d}t},\quad n=0,1,2,\cdots .
\end{align*}
从而
\begin{align*}
\int_{x\left( 0 \right)}^{x_{n+1}}{\frac{1}{g\left( t \right)}\mathrm{d}t}=\int_{x\left( 0 \right)}^{x_n}{\frac{1}{g\left( t \right)}\mathrm{d}t}+\int_{x_n}^{x_{n+1}}{\frac{1}{g\left( t \right)}\mathrm{d}t}\geqslant n+\frac{x_{n+1}-x_n}{g\left( x_n \right)}=n+1=\int_{x\left( 0 \right)}^{x\left( n+1 \right)}{\frac{1}{g\left( t \right)}\mathrm{d}t}.
\end{align*}
又因为$g$非负,所以$x_{n+1}\geqslant x\left( n+1 \right)$.故由数学归纳法知,$x_n\geqslant x\left( n \right),\forall n\in \mathbb{N}$.于是
\begin{align*}
x_n&=x_0+\sum\limits_{k=0}^{n-1}{\left( x_{k+1}-x_k \right)}=x_0+\sum\limits_{k=0}^{n-1}{g\left( x_k \right)}=x_0+g\left( x_0 \right) +\sum\limits_{k=1}^{n-1}{\int_{k-1}^k{g\left( x_k \right) \mathrm{d}t}}
\\
&\leqslant x_0+g\left( x_0 \right) +\sum\limits_{k=1}^{n-1}{\int_{k-1}^k{g\left( x\left( t \right) \right) \mathrm{d}t}}=x_0+g\left( x_0 \right) +\int_0^{n-1}{g\left( x\left( t \right) \right) \mathrm{d}t}
\\
&=x_0+g\left( x_0 \right) +\int_0^{n-1}{x^\prime \left( t \right) \mathrm{d}t}=x_0+g\left( x_0 \right) +x\left( n-1 \right) 
\\
&<x_0+g\left( x_0 \right) +x\left( n \right) .
\end{align*}
故
\begin{align*}
x\left( n \right) \leqslant x_n\leqslant x\left( n \right) +x_0+g\left( x_0 \right) \Longleftrightarrow x_n=x\left( n \right) +O\left( 1 \right) ,\quad n\rightarrow \infty .
\end{align*}

\end{proof}

\begin{example}
设 $\gamma:[0,1]\to[0,1]^2$ 是连续满射且满足对某个 $\alpha\in(0,1)$ 有
\[
|\gamma(s)-\gamma(t)|\leqslant M|s-t|^\alpha,\forall s,t\in[0,1].
\]
证明:$\alpha\leqslant\frac{1}{2}$.
\end{example}
\begin{proof}
记
\begin{align*}
B_k\triangleq \left\{ x\in \mathbb{R} ^2:\left| x-\gamma \left( \frac{k}{n} \right) \right|\leqslant \frac{M}{n^{\alpha}} \right\},\quad k=0,1,2\cdots,n-1.
\end{align*}
对$\forall x\in \left[ 0,1 \right]$,存在$k\in \left[ 0,n-1 \right] \cap \mathbb{N}$,使得$x\in \left[ \frac{k-1}{n},\frac{k}{n} \right]$.由条件可知
\begin{align*}
\left| \gamma \left( x \right) -\gamma \left( \frac{k}{n} \right) \right|\leqslant M\left| x-\frac{k}{n} \right|^{\alpha}\leqslant \frac{M}{n^{\alpha}}\Longrightarrow \gamma \left( x \right) \in B_k.
\end{align*}
故$\gamma$的值域$\left[ 0,1 \right] ^2\subseteq \bigcup_{k=0}^{n-1}{B_k}$.于是
\begin{align*}
1\leqslant S\left( \bigcup_{k=0}^{n-1}{B_k} \right) \leqslant \sum\limits_{k=0}^{n-1}{S\left( B_k \right)}=\sum\limits_{k=0}^{n-1}{\frac{\pi M^2}{n^{2\alpha}}}=\frac{\pi M^2}{n^{2\alpha -1}},\forall n\in \mathbb{N}.
\end{align*}
因此$\alpha \leqslant \frac{1}{2}$.

\end{proof}

\begin{example}
设$f \in D^2(\mathbb{R})$满足对某个$M > 0$成立
$$0 < f(x) < M, \forall x \in \mathbb{R}.$$
证明存在$\xi \in \mathbb{R}$使得
$$f(\xi)f''(\xi) + 2[f'(\xi)]^2 = 0.$$
\end{example}
\begin{proof}
注意到
$$(f^3(x))'' = 3f(x)\left[f''(x)f(x) + 2[f'(x)]^2\right].$$
若$\xi$不存在,则$(f^3(x))''$不变号,从而$f^3(x)$是上凸或者下凸函数.但是\hyperref[proposition:无穷区间上的有界凸函数必为常函数]{无穷区间上的有界凸函数必为常函数},因此$f$为常值函数,这就和$\xi$不存在矛盾!现在我们知道存在$\xi \in \mathbb{R}$使得
$$f(\xi)f''(\xi) + 2[f'(\xi)]^2 = 0.$$

\end{proof}

\begin{example}
已知:存在连续正函数\( f : [1, +\infty) \to (0, +\infty) \),满足

(1) \( \lim\limits_{x \to +\infty} \frac{f(x)}{x} = 0 \),

(2) \( \frac{1}{x} \int_1^x f(t)\mathrm{d}t \)有界。

证明:当\( \alpha > 1 \)时,
\[
\lim_{x \to +\infty} \frac{\int_1^x f^\alpha(t) \ln(1 + f(t))\mathrm{d}t}{x^\alpha \ln(1 + x)} = 0.
\]
\end{example}
\begin{proof}
由\( x^{\alpha}\ln(1 + x) \)递增可得,对\( \forall A > 1 \),都有
\begin{align}
\varlimsup_{x \to +\infty} \frac{\int_1^x f^{\alpha}(t)\ln(1 + f(t))\mathrm{d}t}{x^{\alpha}\ln(1 + x)} &= \varlimsup_{x \to +\infty} \frac{\int_A^x f^{\alpha}(t)\ln(1 + f(t))\mathrm{d}t}{x^{\alpha}\ln(1 + x)}  \nonumber \\
&\leqslant \varlimsup_{x \to +\infty} \frac{1}{x} \int_A^x f(t) \cdot \left( \frac{f(t)}{t} \right)^{\alpha - 1} \frac{\ln(1 + f(t))}{\ln(1 + t)}\mathrm{d}t \nonumber \\
&\leqslant \sup_{t \geqslant A} \left( \frac{f(t)}{t} \right)^{\alpha - 1} \cdot \sup_{t \geqslant A} \frac{\ln(1 + f(t))}{\ln(1 + t)} \cdot \varlimsup_{x \to +\infty} \frac{\int_A^x f(t)\mathrm{d}t}{x}. \label{eq:::23589u2385tt12}
\end{align}
由条件可知
\begin{align*}
\lim_{A \to +\infty} \sup_{t \geqslant A} \left( \frac{f(t)}{t} \right)^{\alpha - 1} = \varlimsup_{t \to +\infty} \left( \frac{f(t)}{t} \right)^{\alpha - 1} = 0,
\end{align*}
\begin{align*}
\lim_{x \to +\infty} \frac{f(t)}{t} = 0 \implies f(t) \leqslant t, \quad t \to +\infty.
\end{align*}
从而
\begin{align*}
\lim_{A \to +\infty} \sup_{t \geqslant A} \frac{\ln(1 + f(t))}{\ln(1 + t)} = \varlimsup_{t \to +\infty} \frac{\ln(1 + f(t))}{\ln(1 + t)} \leqslant \varlimsup_{t \to +\infty} \frac{\ln(1 + t)}{\ln(1 + t)} = 1.
\end{align*}
于是令\eqref{eq:::23589u2385tt12}式\( A \to +\infty \)得
\begin{align*}
\varlimsup_{x \to +\infty} \frac{\int_1^x f^{\alpha}(t)\ln(1 + f(t))\mathrm{d}t}{x^{\alpha}\ln(1 + x)} \leqslant 0.
\end{align*}

\end{proof}

\begin{example}
设\( f(x) \)是闭区间\( [a,b] \)上的连续下凸函数,实数\( m \)满足
\[
\int_{a}^{b}|f(x) - m|\mathrm{d}x = \min_{t \in [a,b]}\int_{a}^{b}|f(x) - f(t)|\mathrm{d}x,
\]
证明:\( m \geqslant f\left( \frac{a + b}{2} \right) \).
\end{example}
\begin{proof}
不妨设\( a=0, b=1 \),否则用\( f(a+(b-a)x) \)代替\( f(x) \)即可。再不妨设\( f\left( \frac{1}{2} \right) =0 \),否则用\( f(x)-f\left( \frac{1}{2} \right) \)代替即可。现在只需证明\( m \geqslant 0 \)。因为\( f \)在\( [0,1] \)上连续下凸,所以\( f \)在\( [0,1] \)上存在最小值,不妨设
\begin{align*}
\min_{x \in [0,1]} f(x) = f(x_0), \quad x_0 \in \left[ 0, \frac{1}{2} \right].
\end{align*}
根据下凸性知,\( f \)在\( \left[ \frac{1}{2},1 \right] \)上非负递增。反证,假设\( m < 0 \),则由条件可知(取\( t=\frac{1}{2} \))
\begin{align*}
\int_0^1 |f(x)| \mathrm{d}x \geqslant \int_0^1 |f(x)-m| \mathrm{d}x.
\end{align*}
但是另一方面,根据绝对值不等式,我们有
\begin{align*}
\int_0^1 \left( |f(x)-m| - |f(x)| \right) \mathrm{d}x &= \int_0^{\frac{1}{2}} \left( |f(x)-m| - |f(x)| \right) \mathrm{d}x + \int_{\frac{1}{2}}^1 \left[ (f(x)-m) - f(x) \right] \mathrm{d}x \\
&= \int_0^{\frac{1}{2}} \left( |f(x)-m| - |f(x)| \right) \mathrm{d}x - \frac{m}{2} \\
&= \int_0^{\frac{1}{2}} \left( |m| + |f(x)-m| - |f(x)| \right) \mathrm{d}x \\
&\geqslant \int_0^{\frac{1}{2}} \left( |m + f(x) - m| - |f(x)| \right) \mathrm{d}x = 0.
\end{align*}
因此上述绝对值不等式取等,即对\( \forall x \in \left[ 0, \frac{1}{2} \right] \),都有
\begin{align*}
|m| + |f(x)-m| &= |m + f(x) - m| = |f(x)| \\
&\implies m^2 + f^2(x) - 2mf(x) + m^2 = f^2(x) \\
&\implies f(x) = m.
\end{align*}
这与\( f\left( \frac{1}{2} \right) = 0 \)矛盾!

\end{proof}

\begin{example}
设$x_1, x_2 \in \mathbb{R}$,定义
$$x_{n+1} = \left(1 - \frac{1}{\sqrt{n}}\right)x_{n-1} + \frac{1}{\sqrt{n}}x_n, n \geq 2.$$
请问 $\{x_n\}$ 是否收敛?若收敛,请证明;若不收敛,请举反例。
\end{example}
\begin{proof}
由条件可知,对$\forall n\in \mathbb{N}$,都有
$$x_{n+1} - x_n = -\left(1 - \frac{1}{\sqrt{n}}\right)\left(x_n - x_{n-1}\right) = \cdots = (-1)^{n-1}\left(x_2 - x_1\right)\prod_{k=2}^n \left(1 - \frac{1}{\sqrt{k}}\right).$$
于是对$\forall n\in \mathbb{N}$,有
$$x_n = x_1 + \sum\limits_{m=1}^{n-1} \left(x_{m+1} - x_m\right) = x_1 + \left(x_2 - x_1\right)\sum\limits_{m=1}^{n-1} (-1)^{m-1}\prod_{k=2}^m \left(1 - \frac{1}{\sqrt{k}}\right).$$
注意到对$\forall m\in \mathbb{N}$,都有
$$\prod_{k=2}^m \left(1 - \frac{1}{\sqrt{k}}\right) < \prod_{k=2}^{m-1} \left(1 - \frac{1}{\sqrt{k}}\right) \Longrightarrow \left\{ \prod_{k=2}^m \left(1 - \frac{1}{\sqrt{k}}\right) \right\} \text{递减}.$$
并且
$$\prod_{k=2}^m \left(1 - \frac{1}{\sqrt{k}}\right) = e^{\sum\limits_{k=2}^m \ln\left(1 - \frac{1}{\sqrt{k}}\right)} \leqslant e^{-\sum\limits_{k=2}^m \frac{1}{\sqrt{k}}}, \quad \forall m\in \mathbb{N}.$$
令$m\rightarrow \infty$得
$$\varlimsup_{m\rightarrow \infty} \prod_{k=2}^m \left(1 - \frac{1}{\sqrt{k}}\right) \leqslant \varlimsup_{m\rightarrow \infty} e^{-\sum\limits_{k=2}^m \frac{1}{\sqrt{k}}} = 0,$$
故$\left\{ \prod_{k=2}^m \left(1 - \frac{1}{\sqrt{k}}\right) \right\}$单调递减趋于$0$。由leibniz判别法知
$$\sum\limits_{m=2}^{\infty} (-1)^{m-1}\prod_{k=2}^m \left(1 - \frac{1}{\sqrt{k}}\right) < +\infty.$$
故
$$\lim_{n\rightarrow \infty} x_n = x_1 + \left(x_2 - x_1\right)\sum\limits_{m=1}^{\infty} (-1)^{m-1}\prod_{k=1}^m \left(1 - \frac{1}{\sqrt{k}}\right) < +\infty.$$

\end{proof}

\begin{example}
设$a_n = \int_0^{\frac{\pi}{4}} \tan^n x \mathrm{d}x, n = 0,1,2,\cdots$.
\begin{enumerate}[(1)]
\item 求$\lim\limits_{n \to \infty} \frac{n}{2}(a_{n-2} + a_n).$

\item 求$\sum\limits_{n=0}^{\infty} a_n x^n$的收敛域.
\end{enumerate}
\end{example}
\begin{proof}
\begin{enumerate}[(1)]
\item 由条件可知
\begin{align*}
a_{n-2} + a_n &= \int_0^{\frac{\pi}{4}} \tan^{n-2}x \left( \tan^2 x + 1 \right) \mathrm{d}x = \int_0^{\frac{\pi}{4}} \tan^{n-2}x \mathrm{d}(\tan x) = \frac{1}{n-1}.
\end{align*}
故
\begin{align*}
\lim_{n \to \infty} \frac{n}{2} \left( a_{n-2} + a_n \right) = \lim_{n \to \infty} \frac{n}{2(n-1)} = \frac{1}{2}.
\end{align*}

\item 注意到对$\forall n\in \mathbb{N}$,都有
\begin{align*}
a_n = \int_0^{\frac{\pi}{4}} \tan^n x \mathrm{d}x \xlongequal{t = \tan x} \int_0^1 \frac{t^n}{1 + t^2} \mathrm{d}t,
\end{align*}
又因为
\begin{align*}
\left( \frac{t}{1 + t^2} \right)' = \frac{1 - t^2}{(1 + t^2)^2} \geqslant 0,\quad \forall t \in [0,1],
\end{align*}
所以
\begin{align*}
\frac{t^n}{2} \leqslant \frac{t^n}{1 + t^2} = t^{n-1} \cdot \frac{t}{1 + t^2} \leqslant \frac{t^{n-1}}{2}, \quad \forall t \in [0,1],
\end{align*}
故对$\forall n\in \mathbb{N}$,都有
\begin{align*}
\frac{1}{2(n+1)} = \int_0^1 \frac{t^n}{2} \mathrm{d}t \leqslant a_n = \int_0^1 \frac{t^n}{1 + t^2} \mathrm{d}t \leqslant \int_0^1 \frac{t^{n-1}}{2} \mathrm{d}t = \frac{1}{2n}.
\end{align*}
因此$a_n \sim \frac{1}{2n}, n \to \infty$。而$\sum\limits_{n=0}^{\infty} \frac{x^n}{2n}$的收敛域为$[-1,1)$,故$\sum\limits_{n=0}^{\infty} a_n x^n$的收敛域为$[-1,1)$.
\end{enumerate}

\end{proof}

\begin{example}
设$f: \mathbb{R} \to (0, +\infty)$是一可微函数,且对所有$x, y \in \mathbb{R}$,有
$$|f'(x) - f'(y)| \leq |x - y|^\alpha,$$
其中$\alpha \in (0, 1]$是常数.
求证:对所有$x \in \mathbb{R}$,有
$$|f'(x)|^{\frac{\alpha + 1}{\alpha}} < \frac{\alpha + 1}{\alpha} f(x).$$
\end{example}
\begin{proof}
对$\forall x \in \mathbb{R}$,固定$x$.

(i)若$f'(x) = 0$,则结论显然成立.

(ii)若$f'(x) < 0$,则令$h = \left( -f'(x) \right)^{\frac{1}{\alpha}} > 0$. 由Newton-Leibniz公式可得
\begin{align*}
0 &< f(x + h) = f(x) + \int_x^{x + h} f'(t) \mathrm{d}t \\
&= f(x) + \int_x^{x + h} \left[ f'(t) - f'(x) \right] \mathrm{d}t + f'(x) h \\
&\leq f(x) + \int_x^{x + h} \left( t - x \right)^{\alpha} \mathrm{d}t + f'(x) h \\
&= f(x) + \frac{h^{\alpha + 1}}{\alpha + 1} + f'(x) h \\
&= f(x) + \frac{\left( -f'(x) \right)^{\frac{\alpha + 1}{\alpha}}}{\alpha + 1} + f'(x) \left( -f'(x) \right)^{\frac{1}{\alpha}}.
\end{align*}
于是
$$\left[ f'(x) - \frac{1}{\alpha + 1} f'(x) \right] \left( -f'(x) \right)^{\frac{1}{\alpha}} < f(x)$$
$$\Longleftrightarrow f'(x) \left( -f'(x) \right)^{\frac{1}{\alpha}} < \frac{\alpha + 1}{\alpha} f(x).$$
从而
$$\left| f'(x) \right|^{\frac{\alpha + 1}{\alpha}} < \frac{\alpha + 1}{\alpha} f(x).$$

(iii)若$f'(x) > 0$,则令$h = \left( f'(x) \right)^{\frac{1}{\alpha}} > 0$. 由Newton-Leibniz公式可得
\begin{align*}
0 &< f(x - h) = -\int_{x - h}^x f'(t) \mathrm{d}t + f(x) \\
&= \int_{x - h}^x \left[ f'(x) - f'(t) \right] \mathrm{d}t + f(x) - f'(x) h \\
&\leq \int_{x - h}^x \left( x - t \right)^{\alpha} \mathrm{d}t + f(x) - f'(x) h \\
&= \frac{h^{\alpha + 1}}{\alpha + 1} + f(x) - f'(x) h \\
&= \frac{\left( f'(x) \right)^{\frac{\alpha + 1}{\alpha}}}{\alpha + 1} + f(x) - f'(x) \left( f'(x) \right)^{\frac{1}{\alpha}}.
\end{align*}
于是
$$\left[ f'(x) - \frac{1}{\alpha + 1} f'(x) \right] \left( f'(x) \right)^{\frac{1}{\alpha}} < f(x)$$
$$\Longleftrightarrow \left( f'(x) \right)^{\frac{\alpha + 1}{\alpha}} < \frac{\alpha + 1}{\alpha} f(x).$$
从而
$$\left| f'(x) \right|^{\frac{\alpha + 1}{\alpha}} < \frac{\alpha + 1}{\alpha} f(x).$$

\end{proof}

\begin{example}
给定$f,g:[0,+\infty) \to [0,+\infty)$且$f$连续,$\lim_{x \to +\infty} g(x) = 0$. 若有不等式
\begin{align*}
|f(x) - f(y)| \leqslant g(x) \sup_{t \in [x,y]} f(t), \forall 0 \leqslant x < y.
\end{align*}
证明$\lim_{x \to +\infty} f(x)$存在.
\end{example}
\begin{proof}
(i)若$f(x)$有界,设$|f(x)| \leqslant M, \forall x \in [0,+\infty)$.由$\lim_{x \to +\infty} g(x) = 0$知,对$\forall \varepsilon > 0$,存在$X > 0$,使得
\begin{align*}
g(x) < \varepsilon, \quad \forall x > X.
\end{align*}
于是对$\forall y > x > X$,都有
\begin{align*}
|f(x) - f(y)| \leqslant g(x) \sup_{t \in [x,y]} f(t) < M\varepsilon.
\end{align*}
故由Cauchy收敛准则知$\lim_{x \to +\infty} f(x)$存在.

(ii)若$f(x)$无界,记$K \triangleq \lfloor f(0) \rfloor + 1$,
\begin{align*}
y_n \triangleq \inf \left\{ y \in [0,+\infty) : f(y) = n \right\},\quad  \forall n > K.
\end{align*}
由$f$的连续性知$f(y_n) = n, \forall n > K$.断言$y_{n+1} > y_n$.若$y_{n+1} \leqslant y_n$,则由介值定理知,存在$y_n' \in (y_{n+1}, y_n)$,使得$f(y_n') = f(y_n) = n$,这与$y_n$的下确界定义矛盾!于是$\lim_{n \to \infty} y_n = a \in (K, +\infty]$.又$f \in C[0,+\infty)$,故
\begin{align*}
f\left( \lim_{n \to \infty} y_n \right) = \lim_{n \to \infty} f(y_n) = \lim_{n \to \infty} n = +\infty.
\end{align*}
因此$\lim_{n \to \infty} y_n = +\infty$,否则$f\left( \lim_{n \to \infty} y_n \right) = f(a) < +\infty$矛盾!我们还可以断言
\begin{align}
f(y_n) = \sup_{t \in [0,y_n]} f(t), \quad \forall n > K. \label{eq:eq::382wer346456yt34t3y4g44j76k67opp242rwfdsg3432}
\end{align}
否则,对$\forall n > K$,存在$t_n \in [0,y_n)$,使得$f(t_n) > f(y_n)$.由介值定理知,存在$\xi_n \in (t_n, y_n)$,使得$f(\xi_n) = f(y_n) = n$,这与$y_n$的下确界定义矛盾!由$\lim_{x\to +\infty}g(x)=0$知,存在$x_0 \in [0,+\infty)$,使得$g(x_0)<1$.又由$\lim_{n\to\infty}y_n=+\infty$知,存在$N\in \mathbb{N},$使得$y_n>x_0$.于是由条件和\eqref{eq:eq::382wer346456yt34t3y4g44j76k67opp242rwfdsg3432}式可得,对$\forall n > N$,有
\begin{align*}
f(y_n) - f(x_0) = |f(y_n) - f(x_0)| \leqslant g(x_0) \sup_{t \in [x_0,y_n]} f(t) \leqslant g(x_0) f(y_n).
\end{align*}
进而
\begin{align*}
f(x_0) \geqslant f(y_n) \left[ 1 - g(x_0) \right] = n \left[ 1 - g(x_0) \right], \quad \forall n > N.
\end{align*}
令$n \to +\infty$得$f(x_0) = +\infty$,显然矛盾!故$f$必有界.再由(i)可知$\lim_{x \to +\infty} f(x)$存在.

\end{proof}

\begin{example}
设
\begin{align*}
\varphi(x) \triangleq \begin{cases} 
\sin x, & x \geqslant 0 \\
\cos x, & x < 0 
\end{cases},
\end{align*}
证明不存在$(-\infty,+\infty)$上的可微函数$f$使得对任何$x \in \mathbb{R}$都有
\begin{align*}
f(f(x)) - |x| f'(x) = \varphi(x).
\end{align*}
\end{example}
\begin{note}
实际上,可以直接看出
\begin{align*}
f(0) - f(-1) = \int_{-1}^0 \frac{\varphi(x) - f(f(x))}{x} \, \mathrm{d}x \sim \int_{-1}^0 \frac{1}{x} \, \mathrm{d}x
\end{align*}
显然矛盾!
\end{note}
\begin{proof}
假设存在这样的$f(x)$,则
由条件得$f(f(0))=0$. 注意到
\begin{align*}
f'(x) = \frac{\varphi(x) - f(f(x))}{x}, \, \forall x < 0.
\end{align*}
于是我们有
\begin{align*}
f(-\varepsilon) - f(-1) = \int_{-1}^{-\varepsilon} \frac{\varphi(x) - f(f(x))}{x} \, \mathrm{d}x, \, \forall \varepsilon \in (0,1).
\end{align*}
现在让$\varepsilon \to 0^+$得
\begin{align*}
f(0) - f(-1) = \int_{-1}^0 \frac{\varphi(x) - f(f(x))}{x} \, \mathrm{d}x.
\end{align*}
由$f \in D(\mathbb{R})$知, 存在$M > 0$, 使得$|f(x)| \leqslant M,\ \forall x \in [-1,0]$. 再结合$f,\varphi$在$x=0$处的连续性知, 存在$\delta \in (0,1)$, 使得
\begin{align*}
f(0)-f(-1) &= \int_{-1}^0 \frac{\varphi(x) - f(f(x))}{x} \, \mathrm{d}x = \int_{-\delta}^0 \frac{\varphi(x) - f(f(x))}{x} \, \mathrm{d}x + \int_{-1}^{-\delta} \frac{\varphi(x) - f(f(x))}{x} \, \mathrm{d}x \\
&\geqslant \int_{-\delta}^0 \frac{\varphi(0) - f(f(0))}{x} \, \mathrm{d}x + \int_{-1}^{-\delta} \frac{1 + M}{-\delta} \, \mathrm{d}x = \int_{-\delta}^0 \frac{1}{x} \, \mathrm{d}x - \frac{1 + M}{\delta}(1 - \delta) = +\infty
\end{align*}
这就是一个矛盾!因此满足条件的可微函数$f$不存在.

\end{proof}

\begin{example}
设$\varphi$是$\mathbb{R}$上严格单调增加的连续函数,$\psi$是$\varphi$的反函数,实数列$\{x_n\}$满足
\begin{align*}
x_{n+2} = \psi\left( \left(1 - \frac{1}{\sqrt{n}}\right)\varphi(x_n) + \frac{1}{\sqrt{n}}\varphi(x_{n+1}) \right), \, n \geq 2.
\end{align*}
证明:$\{x_n\}$收敛或举例说明$\{x_n\}$有可能发散.
\end{example}
\begin{proof}
由条件可得
\begin{align*}
\varphi(x_{n+2}) = \left(1 - \frac{1}{\sqrt{n}}\right)\varphi(x_n) + \frac{1}{\sqrt{n}}\varphi(x_{n+1}),\ \ n \geq 2.
\end{align*}
于是
\begin{align*}
\varphi(x_{n+2}) - \varphi(x_{n+1}) &= \left( \frac{1}{\sqrt{n}} - 1 \right)\left[ \varphi(x_{n+1}) - \varphi(x_n) \right] \\
&= \left( \frac{1}{\sqrt{n}} - 1 \right)\left( \frac{1}{\sqrt{n-1}} - 1 \right)\left[ \varphi(x_n) - \varphi(x_{n-1}) \right] \\
&= \cdots = \prod_{k=2}^n \left( \frac{1}{\sqrt{k}} - 1 \right)\left[ \varphi(x_3) - \varphi(x_2) \right] > 0,\ \forall n \geq 2.
\end{align*}
从而$\{\varphi(x_n)\}$递增,并且
\begin{align}
\varphi(x_n) &= \varphi(x_3) + \sum\limits_{k=3}^{n-1} \left[ \varphi(x_{k+1}) - \varphi(x_k) \right]\nonumber \\
&= \varphi(x_3) + \left[ \varphi(x_3) - \varphi(x_2) \right] \sum\limits_{k=3}^{n-1} \prod_{m=2}^{k-1} \left( \frac{1}{\sqrt{m}} - 1 \right) \nonumber \\
&= \varphi(x_3) + \left[ \varphi(x_3) - \varphi(x_2) \right] \sum\limits_{k=3}^{n-1} e^{\sum\limits_{m=2}^{k-1} \ln\left( \frac{1}{\sqrt{m}} - 1 \right)},\ \forall n \geq 4. \label{eq::382wer346456yt34t3y4g44j76k67opp}
\end{align}
注意到$\lim\limits_{x \to 0^+} \frac{\ln(x - 1)}{-x} = 1$,故
\begin{align*}
\ln\left( \frac{1}{\sqrt{m}} - 1 \right) \sim -\frac{1}{\sqrt{m}},\ m \to +\infty.
\end{align*}
因此
\begin{align*}
e^{\sum\limits_{m=2}^{k-1} \ln\left( \frac{1}{\sqrt{m}} - 1 \right)} \sim e^{-\sum\limits_{m=2}^{k-1} \frac{1}{\sqrt{m}}},\ m \to +\infty.
\end{align*}
由Stolz公式可得
\begin{align*}
\lim_{k \to \infty} \frac{2\ln k}{\sum\limits_{m=2}^{k-1} \frac{1}{\sqrt{m}}} = \lim_{k \to \infty} \frac{2\ln\left( 1 + \frac{1}{k} \right)}{\frac{1}{\sqrt{k}}} = \lim_{k \to \infty} \frac{2\sqrt{k}}{k} = 0.
\end{align*}
从而当$k$充分大时,有
\begin{align*}
\sum\limits_{m=2}^{k-1} \frac{1}{\sqrt{m}} > 2\ln k \Longleftrightarrow e^{-\sum\limits_{m=2}^{k-1} \frac{1}{\sqrt{m}}} < e^{-2\ln k} = \frac{1}{k^2}.
\end{align*}
于是
\begin{align*}
\sum\limits_{k=3}^{\infty} e^{\sum\limits_{m=2}^{k-1} \ln\left( \frac{1}{\sqrt{m}} - 1 \right)} < \sum\limits_{k=3}^{\infty} \frac{1}{k^2} < +\infty.
\end{align*}
再由\eqref{eq::382wer346456yt34t3y4g44j76k67opp}式知$\varphi(x_n)$收敛. 设$\lim\limits_{n \to \infty} \varphi(x_n) = a$,由$\{\varphi(x_n)\}$递增知$\lim\limits_{n \to \infty} x_n \in (x_1, +\infty]$. 由$\varphi$连续且存在反函数知
\begin{align*}
\varphi\left( \lim_{n \to \infty} x_n \right) = \lim_{n \to \infty} \varphi(x_n) = a \Longrightarrow \lim_{n \to \infty} x_n = \psi(a) < +\infty.
\end{align*}

\end{proof}

\begin{example}
证明:
\begin{align*}
\lim_{x \to 0^+} \frac{1}{x \ln x} \sum_{n=1}^{\infty} \frac{|\sin(nx)|}{n^2} = -1.
\end{align*}
\end{example}
\begin{proof}
对$\forall x\in (0,1)$,我们有
\begin{align*}
\sum_{n=1}^{\infty}\frac{|\sin(nx)|}{n^2} &= \sin x + \sum_{n=2}^{\lfloor \frac{1}{x} \rfloor}\frac{|\sin(nx)|}{n^2} + \sum_{n=\lfloor \frac{1}{x} \rfloor +1}^{\infty}\frac{|\sin(nx)|}{n^2} \leqslant \sin x + \sum_{n=2}^{\lfloor \frac{1}{x} \rfloor}\frac{nx}{n^2} + \sum_{n=\lfloor \frac{1}{x} \rfloor +1}^{\infty}\frac{1}{n^2} \\
&= \sin x + x\sum_{n=2}^{\lfloor \frac{1}{x} \rfloor}\frac{1}{n} + \sum_{n=\lfloor \frac{1}{x} \rfloor +1}^{\infty}\frac{1}{n(n-1)} \leqslant \sin x + x\sum_{n=2}^{\lfloor \frac{1}{x} \rfloor}\int_{n-1}^n\frac{1}{t}\mathrm{d}t + \sum_{n=\lfloor \frac{1}{x} \rfloor +1}^{\infty}\left( \frac{1}{n-1} - \frac{1}{n} \right) \\
&= \sin x + x\int_1^{\lfloor \frac{1}{x} \rfloor}\frac{1}{t}\mathrm{d}t + \frac{1}{\lfloor \frac{1}{x} \rfloor} \leqslant \sin x + x\ln \lfloor \frac{1}{x} \rfloor + \frac{x}{1-x}.
\end{align*}
注意到
\begin{align*}
\sin x\geqslant x-x^3,\ \forall x\in (0,1].
\end{align*}
故另一方面,对$\forall x\in (0,\frac{1}{N})$,我们有
\begin{align*}
\sum_{n=1}^{\infty}\frac{|\sin(nx)|}{n^2}&=\sum_{n=1}^{\lfloor \frac{1}{x} \rfloor}\frac{|\sin(nx)|}{n^2}+\sum_{n=\lfloor \frac{1}{x} \rfloor +1}^{\infty}\frac{|\sin(nx)|}{n^2} \\
&\geqslant \sum_{n=1}^{\lfloor \frac{1}{x} \rfloor}\frac{nx-(nx)^3}{n^2}=x\sum_{n=1}^{\lfloor \frac{1}{x} \rfloor}\frac{1}{n}-x^4\sum_{n=1}^{\lfloor \frac{1}{x} \rfloor}n \\
&\geqslant x\sum_{n=1}^{\lfloor \frac{1}{x} \rfloor}\int_n^{n+1}\frac{1}{t}\mathrm{d}t-x^4\cdot \frac{1}{x^2}=x\int_1^{\lfloor \frac{1}{x} \rfloor}\frac{1}{t}\mathrm{d}t-x^2 \\
&=x\ln \lfloor \frac{1}{x} \rfloor -x^2\geqslant x\ln \left( \frac{1}{x}-1 \right) -x^2\\
&=-x\ln x+x\ln \left( 1-x \right) -x^2.
\end{align*}
综上,对$\forall x\in (0,1)$,我们有
\begin{align*}
-x\ln x-x^2\leqslant \sum_{n=1}^{\infty}\frac{|\sin(nx)|}{n^2}\leqslant \sin x-x\ln x+\frac{x}{1-x}.
\end{align*}
从而对$\forall x\in (0,1)$,我们有
\begin{align*}
-1+\frac{\ln \left( 1-x \right)}{\ln x}-\frac{x^2}{x\ln x} \leqslant \frac{1}{x\ln x}\sum_{n=1}^{\infty}\frac{|\sin(nx)|}{n^2}\leqslant \frac{\sin x}{x\ln x}-1\leqslant \frac{1}{\ln x}-1+\frac{1}{(1-x)\ln x}.
\end{align*}
令$x\rightarrow 0^+$得
\begin{align*}
\lim_{x\rightarrow 0^+}\frac{1}{x\ln x}\sum_{n=1}^{\infty}\frac{|\sin(nx)|}{n^2}=-1.
\end{align*}

\end{proof}

\begin{example}
设 $f \in C(0,1]$ 且对某个 $\mu > 1$ 有
\begin{align*}
\lim_{x \to 0^+} x[f(\mu x) - f(x)] = A.
\end{align*}
证明:$\lim_{x \to 0^+} x f(x)$ 存在.
\end{example}
\begin{note}
取$n=\lfloor \frac{\ln \frac{\delta}{x}}{\ln \mu} + 1 \rfloor$的原因:
\begin{align*}
\mu^n x > \delta, \quad \mu^k x \leq \delta, \ k=0,1,\cdots,n-1
\end{align*}
由$\mu > 1$知上式等价于
\begin{align*}
\mu^n x > \delta \text{且}\mu^{n-1} x \leqslant \delta \Longleftrightarrow \frac{\ln \frac{\delta}{x}}{\ln \mu} < n \leqslant \frac{\ln \frac{\delta}{x}}{\ln \mu} + 1.
\end{align*}
故取$n=\lfloor \frac{\ln \frac{\delta}{x}}{\ln \mu} + 1 \rfloor$即可满足要求.
\end{note}
\begin{proof}
由条件知,对$\forall \varepsilon \in (0,1)$,存在$\delta > 0$,使得
\begin{align}
A - \varepsilon < x\left[ f\left( \mu x \right) - f\left( x \right) \right] < A + \varepsilon, \ \forall x \in (0,\delta). \label{eq::9033489t41}
\end{align}
由$f \in C\left[ \delta,1 \right]$知,存在$M > 0$,使得
\begin{align}
\left| f\left( x \right) \right| < M, \quad x \in \left[ \delta,1 \right]. \label{eq::9033489t42}
\end{align}
注意到$\lim_{n \to \infty} -A\sum_{k=1}^n \frac{1}{\mu^{k-1}} = -\frac{\mu A}{\mu - 1}$,故存在$N > 0$,使得
\begin{align}
-\frac{\mu A}{\mu - 1} - \varepsilon < -A\sum_{k=1}^n \frac{1}{\mu^{k-1}} < -\frac{\mu A}{\mu - 1} + \varepsilon. \label{eq::9033489t43}
\end{align}
对$\forall x \in (0,\delta)$,取$n_x = \lfloor \frac{\ln \frac{\delta}{x}}{\ln \mu} + 1 \rfloor$,则
\begin{gather*}
\mu^{n_x} x > \delta, \quad \mu^k x \leqslant \delta, \ k = 0,1,\cdots,n_x - 1.\\
x < \frac{\delta}{\mu^N} \Longrightarrow \lfloor \frac{\ln \frac{\delta}{x}}{\ln \mu} + 1 \rfloor > \frac{\ln \frac{\delta}{x}}{\ln \mu} > N \Longrightarrow n_x > N.
\end{gather*}
于是由\eqref{eq::9033489t41}\eqref{eq::9033489t42}\eqref{eq::9033489t43}式知,对$\forall x \in \left( 0,\min \left\{ \delta, \frac{\varepsilon}{M}, \frac{\delta}{\mu^N} \right\} \right)$,一方面,我们有
\begin{align*}
xf\left( x \right) &= x\sum_{k=1}^{n_x} \left[ f\left( \mu^{k-1}x \right) - f\left( \mu^k x \right) \right] + xf\left( \mu^{n_x} x \right) \\
&< \sum_{k=1}^{n_x} \frac{\varepsilon - A}{\mu^{k-1}} + xM < -A\sum_{k=1}^{n_x} \frac{1}{\mu^{k-1}} + \sum_{k=1}^{\infty} \frac{\varepsilon}{\mu^{k-1}} + \varepsilon \\
&< -\frac{\mu A}{\mu - 1} + \varepsilon + \frac{\mu \varepsilon}{\mu - 1} + \varepsilon.
\end{align*}
另一方面,我们有
\begin{align*}
xf\left( x \right) &= x\sum_{k=1}^{n_x} \left[ f\left( \mu^{k-1}x \right) - f\left( \mu^k x \right) \right] + xf\left( \mu^{n_x} x \right) \\
&> \sum_{k=1}^{n_x} \frac{-\varepsilon - A}{\mu^{k-1}} - xM > -A\sum_{k=1}^{n_x} \frac{1}{\mu^{k-1}} - \sum_{k=1}^{\infty} \frac{\varepsilon}{\mu^{k-1}} - \varepsilon \\
&> -\frac{\mu A}{\mu - 1} - \varepsilon - \frac{\mu \varepsilon}{\mu - 1} - \varepsilon.
\end{align*}
综上,对$\forall x \in \left( 0,\min \left\{ \delta, \frac{\varepsilon}{M} \right\} \right)$,我们有
\begin{align*}
\left| xf\left( x \right) + \frac{\mu A}{\mu - 1} \right| < \varepsilon \left( 2 + \frac{\mu}{\mu - 1} \right).
\end{align*}
故$\lim_{x \to 0^+} xf\left( x \right) = -\frac{\mu A}{\mu - 1}.$

\end{proof}

\begin{example}

\end{example}
\begin{proof}


\end{proof}

\begin{example}

\end{example}
\begin{proof}


\end{proof}

\begin{example}

\end{example}
\begin{proof}


\end{proof}

\begin{example}

\end{example}
\begin{proof}


\end{proof}






































\end{document}
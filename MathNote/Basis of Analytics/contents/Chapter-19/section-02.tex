\documentclass[../../main.tex]{subfiles}% 注意这里的文件路径不能用 ./main.tex ,否则用latexmk编译子文件会报错
\graphicspath{{\subfix{./image/}}} % 指定图片目录,后续可以直接使用图片文件名
% 注意这里的文件路径不能用 ../../image/ ,否则用latexmk编译子文件会报错

% 例如:
% \begin{figure}[H]
% \centering
% \includegraphics[scale=0.3]{图.png}
% \caption{}
% \label{figure:图}
% \end{figure}
% 注意:上述\label{}一定要放在\caption{}之后,否则引用图片序号会只会显示??.

\begin{document}

\section{未解决的习题}

\begin{example}
设$\varphi$是$(0, +\infty)$上正严格递减的连续函数,且
\begin{align*}
\lim\limits_{x \to 0} \varphi(x) = +\infty, \quad
\int_{0}^{+\infty} \varphi(t)\mathrm{d}t = a < +\infty.
\end{align*}
设$\psi$是$\varphi$的反函数,求证:$\int_{0}^{+\infty} \psi(t)\mathrm{d}t = a$,且
\begin{align*}
\int_{0}^{+\infty} (\varphi(t))^2\mathrm{d}t + \int_{0}^{+\infty} (\psi(t))^2\mathrm{d}t &\geqslant \frac{1}{2} a^{\frac{3}{2}}.
\end{align*}
\end{example}
\begin{note}
思路:利用不动点估计$\int_{0}^{p} (\varphi(t))^2\mathrm{d}t + \int_{0}^{p} (\psi(t))^2\mathrm{d}t.$
\end{note}
\begin{proof}


\end{proof}

\begin{example}

\end{example}
\begin{proof}


\end{proof}

\begin{example}

\end{example}
\begin{proof}


\end{proof}










\end{document}
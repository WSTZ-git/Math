\documentclass[../../main.tex]{subfiles}
\graphicspath{{\subfix{../../image/}}} % 指定图片目录,后续可以直接使用图片文件名。

% 例如:
% \begin{figure}[H]
% \centering
% \includegraphics[scale=0.4]{image-01.01}
% \caption{图片标题}
% \label{figure:image-01.01}
% \end{figure}
% 注意:上述\label{}一定要放在\caption{}之后,否则引用图片序号会只会显示??.

\begin{document}

\section{数值比较类}

\begin{example}
证明如下积分不等式:
\begin{enumerate}
\item $\int_{0}^{\sqrt{2\pi}} \sin x^{2}\mathrm{d}x > 0$.

\item $\int_{0}^{\frac{\pi}{2}} \frac{\cos x}{1 + x^{2}}\mathrm{d}x \geqslant \int_{0}^{\frac{\pi}{2}} \frac{\sin x}{1 + x^{2}}\mathrm{d}x.$

\item $\int_{0}^{1} \frac{\cos x}{\sqrt{1 - x^{2}}}\mathrm{d}x > \int_{0}^{1} \frac{\sin x}{\sqrt{1 - x^{2}}}\mathrm{d}x.$
\end{enumerate}
\end{example}
\begin{note}
此类问题都是考虑分母更小的时候正的更多,通过换元把负的区间转化到正的同一个区间.
\end{note}
\begin{proof}
\begin{enumerate}
\item \begin{align*}
\int_0^{\sqrt{2\pi}}{\sin x^2\mathrm{d}x}&\xlongequal{x=\sqrt{y}}\int_0^{2\pi}{\frac{\sin y}{2\sqrt{y}}\mathrm{d}y}=\frac{1}{2}\int_0^{\pi}{\frac{\sin y}{2\sqrt{y}}\mathrm{d}y}+\frac{1}{2}\int_{\pi}^{2\pi}{\frac{\sin y}{2\sqrt{y}}\mathrm{d}y}
\\
&=\frac{1}{2}\int_0^{\pi}{\frac{\sin y}{2\sqrt{y}}\mathrm{d}y}+\frac{1}{2}\int_0^{\pi}{\frac{\sin \left( y+\pi \right)}{2\sqrt{y+\pi}}\mathrm{d}y}
\\
&=\frac{1}{2}\int_0^{\pi}{\sin y\left( \frac{1}{2\sqrt{y}}-\frac{1}{2\sqrt{y+\pi}} \right) \mathrm{d}y}>0.
\end{align*}

\item \begin{align*}
\int_0^{\frac{\pi}{2}}{\frac{\cos x-\sin x}{1+x^2}\mathrm{d}x}&=\int_0^{\frac{\pi}{4}}{\frac{\cos x-\sin x}{1+x^2}\mathrm{d}x}+\int_{\frac{\pi}{4}}^{\frac{\pi}{2}}{\frac{\cos x-\sin x}{1+x^2}\mathrm{d}x}=\sqrt{2}\int_0^{\frac{\pi}{4}}{\frac{\sin \left( \frac{\pi}{4}-x \right)}{1+x^2}\mathrm{d}x}+\sqrt{2}\int_{\frac{\pi}{4}}^{\frac{\pi}{2}}{\frac{\sin \left( \frac{\pi}{4}-x \right)}{1+x^2}\mathrm{d}x}
\\
&=\sqrt{2}\int_0^{\frac{\pi}{4}}{\frac{\sin y}{1+\left( \frac{\pi}{4}-y \right) ^2}\mathrm{d}y}+\sqrt{2}\int_{\frac{\pi}{4}}^{\frac{\pi}{2}}{\frac{\sin \left( -y \right)}{1+\left( \frac{\pi}{4}+y \right) ^2}\mathrm{d}y}
\\
&=\sqrt{2}\int_0^{\frac{\pi}{4}}{\sin y\left[ \frac{1}{1+\left( \frac{\pi}{4}-y \right) ^2}-\frac{1}{1+\left( \frac{\pi}{4}+y \right) ^2} \right] \mathrm{d}y}>0.
\end{align*}

\item 本题稍有不同,注意到
\begin{align*}
\int_0^1{\frac{\cos x}{\sqrt{1-x^2}}\mathrm{d}x\xlongequal{x=\sin y}\int_0^{\frac{\pi}{2}}{\mathrm{cos(}\sin y)dy,\int_0^1{\frac{\sin x}{\sqrt{1-x^2}}\mathrm{d}x\xlongequal{x=\cos y}\int_0^{\frac{\pi}{2}}{\sin\mathrm{(}\cos y)dy}}}}.
\end{align*}
现在利用\(\sin x < x, \forall x \in (0, \frac{\pi}{2})\)可得不等式链
\(\cos\sin x > \cos x > \sin\cos x, \forall x \in (0, \frac{\pi}{2})\),
于是
\begin{align*}
\int_{0}^{1}\frac{\cos x}{\sqrt{1 - x^{2}}}\mathrm{d}x > \int_{0}^{1}\frac{\sin x}{\sqrt{1 - x^{2}}}\mathrm{d}x.
\end{align*} 
\end{enumerate}
\end{proof}

\begin{theorem}[Jordan不等式]\label{theorem:Jordan不等式}
$sinx\geqslant \frac{2}{\pi}x,\forall x\in [0,\frac{\pi}{2}]$
\end{theorem}
\begin{proof}
利用$\sin x$的上凸性及割线放缩可得
\begin{align*}
\frac{\sin x-\sin 0}{x-0}\geqslant \frac{\sin \frac{\pi}{2}-\sin x}{\frac{\pi}{2}-x},\forall x\in \left[ 0,\frac{\pi}{2} \right] .
\end{align*}
\end{proof}

\begin{example}
证明如下积分不等式
\begin{enumerate}
\item \(\frac{\pi}{6} < \int_{0}^{1}\frac{1}{\sqrt{4 - x^{2}-x^{3}}}\mathrm{d}x < \frac{\pi}{4\sqrt{2}}\).

\item \(\int_{0}^{\pi}e^{\sin^{2}x}\mathrm{d}x \geqslant \sqrt{e}\pi\).

\item \(\frac{\pi}{2}e^{-R} < \int_{0}^{\frac{\pi}{2}}e^{-R\sin x}\mathrm{d}x < \frac{\pi(1 - e^{-R})}{2R}, R > 0\).

\item $\int_0^{n\pi}{\frac{\left| \sin x \right|}{x}\mathrm{d}x}>\frac{2}{\pi}\ln \left( n+1 \right) ,n\geqslant 2.$
\end{enumerate}
\end{example}
\begin{remark}
$(2n)!!=2^n\cdot n!.$
\end{remark}
\begin{proof}
\begin{enumerate}
\item \begin{align*}
\frac{\pi}{6}=\int_{0}^{1}\frac{1}{\sqrt{4 - x^{2}}}\mathrm{d}x < \int_{0}^{1}\frac{1}{\sqrt{4 - x^{2}-x^{3}}}\mathrm{d}x < \int_{0}^{1}\frac{1}{\sqrt{4 - x^{2}-x^{2}}}\mathrm{d}x=\frac{\pi}{4\sqrt{2}}.
\end{align*}

\item \begin{align*}
\int_0^{\pi}{e^{\sin ^2x}\mathrm{d}x}&=\int_0^{\pi}{\sum_{n=0}^{\infty}{\frac{\sin ^{2n}x}{n!}}\mathrm{d}x}=\sum_{n=0}^{\infty}{\frac{1}{n!}\int_0^{\pi}{\sin ^{2n}x\mathrm{d}x}}
\\
&=\pi \left[ 1+\sum_{n=1}^{\infty}{\frac{(2n-1)!!}{n!(2n)!!}} \right] =\pi \left[ 1+\sum_{n=1}^{\infty}{\frac{(2n-1)!!}{2^n\left( n! \right) ^2}} \right] 
\\
&\overset{(2n-1)!!\geqslant n!}{\geqslant}\pi \sum_{n=0}^{\infty}{\frac{1}{2^nn!}}=\sqrt{e}\pi .
\end{align*}

\item \begin{align*}
\frac{\pi}{2}e^{-R}=\int_{0}^{\frac{\pi}{2}}e^{-R}\mathrm{d}x < \int_{0}^{\frac{\pi}{2}}e^{-R\sin x}\mathrm{d}x\stackrel{\hyperref[theorem:Jordan不等式]{Jordan\text{不等式}}}{<}\int_{0}^{\frac{\pi}{2}}e^{-\frac{2R}{\pi}x}\mathrm{d}x=\frac{\pi(1 - e^{-R})}{2R}, R > 0.
\end{align*} 

\item \begin{align*}
\int_0^{n\pi}{\frac{\left| \sin x \right|}{x}\mathrm{d}x}&=\sum_{k=0}^{n-1}{\int_{k\pi}^{\left( k+1 \right) \pi}{\frac{\left| \sin x \right|}{x}\mathrm{d}x}}\xlongequal{x=k\pi +y}\sum_{k=0}^{n-1}{\int_0^{\pi}{\frac{\left| \sin y \right|}{k\pi +y}\mathrm{d}y}}
\\
&>\sum_{k=0}^{n-1}{\int_0^{\pi}{\frac{\left| \sin y \right|}{\left( k+1 \right) \pi}\mathrm{d}y}}=\frac{2}{\pi}\sum_{k=0}^{n-1}{\frac{1}{k+1}}
\\
&>\frac{2}{\pi}\sum_{k=0}^{n-1}{\ln \left( 1+\frac{1}{k+1} \right)}=\frac{2}{\pi}\sum_{k=0}^{n-1}{\left[ \ln \left( k+2 \right) -\ln \left( k+1 \right) \right]}
\\
&=\frac{2}{\pi}\ln \left( n+1 \right) .
\end{align*}
还可以使用积分放缩法处理$\frac{2}{\pi}\sum_{k=0}^{n-1}{\frac{1}{k+1}}$,如下所示:
\begin{align*}
\frac{2}{\pi}\sum_{k=0}^{n-1}{\frac{1}{k+1}}=\frac{2}{\pi}\sum_{k=0}^{n-1}{\int_k^{k+1}{\frac{1}{k+1}\mathrm{d}x}}\geqslant \frac{2}{\pi}\sum_{k=0}^{n-1}{\int_k^{k+1}{\frac{1}{x+1}\mathrm{d}x}}=\frac{2}{\pi}\int_0^n{\frac{1}{x+1}\mathrm{d}x}=\frac{2}{\pi}\ln \left( n+1 \right) .
\end{align*}
\end{enumerate}
\end{proof}


















\end{document}
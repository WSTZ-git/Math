\documentclass[../../main.tex]{subfiles}
\graphicspath{{\subfix{../../image/}}} % 指定图片目录,后续可以直接使用图片文件名。

% 例如:
% \begin{figure}[H]
% \centering
% \includegraphics[scale=0.4]{图.png}
% \caption{}
% \label{figure:图}
% \end{figure}
% 注意:上述\label{}一定要放在\caption{}之后,否则引用图片序号会只会显示??.

\begin{document}

\section{其他}

\begin{example}
设$f:[0,1]\to(0,+\infty)$是连续递增函数,记$s = \frac{\int_{0}^{1}xf(x)\mathrm{d}x}{\int_{0}^{1}f(x)\mathrm{d}x}$。证明
\begin{align*}
\int_{0}^{s}f(x)\mathrm{d}x&\leqslant\int_{s}^{1}f(x)\mathrm{d}x\leqslant\frac{s}{1 - s}\int_{0}^{s}f(x)\mathrm{d}x.
\end{align*}
\end{example}
\begin{note}
看到函数复合积分就联想\hyperref[theorem:Jensen不等式积分形式]{Jensen不等式(积分形式)},不过\hyperref[theorem:Jensen不等式积分形式]{Jensen不等式(积分形式)}考试中不能直接使用.因此仍需要利用函数的凸性相关不等式进行证明.
\end{note}
\begin{proof}
令$F(t) = \int_0^t f(x) \, \mathrm{d}x$,则$F'(t) = f(t)$连续递增,故$F$是下凸的。显然$s \in [0,1]$,于是
\begin{align*}
F(x) \geqslant  F(s) + F'(s)(x - s) = F(s) + f(s)(x - s), \quad \forall x \in [0,1].
\end{align*}
从而
\begin{align*}
\int_0^1 F(x) f(x) \, \mathrm{d}x &\geqslant  \int_0^1 \left[ F(s) f(x) + f(s) f(x)(x - s) \right] \, \mathrm{d}x \\
&= F(s) \int_0^1 f(x) \, \mathrm{d}x + f(s) \int_0^1 \left[ x f(x) - s f(x) \right] \, \mathrm{d}x \\
&= F(s) \int_0^1 f(x) \, \mathrm{d}x + f(s) \left[ \int_0^1 x f(x) \, \mathrm{d}x - \frac{\int_0^1 x f(x) \, \mathrm{d}x}{\int_0^1 f(x) \, \mathrm{d}x} \int_0^1 f(x) \, \mathrm{d}x \right] \\
&= F(s) \int_0^1 f(x) \, \mathrm{d}x.
\end{align*}
又注意到
\begin{align*}
\int_0^1 F(x) f(x) \, \mathrm{d}x = \int_0^1 F(x) \, \mathrm{d}F(x) = \frac{1}{2} \left( \int_0^1 f(x) \, \mathrm{d}x \right)^2.
\end{align*}
故
\begin{align*}
&\frac{1}{2} \left( \int_0^1 f(x) \, \mathrm{d}x \right)^2 \geqslant  F(s) \int_0^1 f(x) \, \mathrm{d}x \implies \frac{1}{2} \int_0^1 f(x) \, \mathrm{d}x \geqslant  F(s) = \int_0^s f(x) \, \mathrm{d}x \\
&\implies \int_0^s f(x) \, \mathrm{d}x + \int_s^1 f(x) \, \mathrm{d}x = \int_0^1 f(x) \, \mathrm{d}x \geqslant  2 \int_0^s f(x) \, \mathrm{d}x \\
&\implies \int_0^s f(x) \, \mathrm{d}x \leqslant  \int_s^1 f(x) \, \mathrm{d}x.
\end{align*}
由分部积分可得
\begin{align*}
s = \frac{\int_0^1 x f(x) \, \mathrm{d}x}{\int_0^1 f(x) \, \mathrm{d}x} = \frac{\int_0^1 x \, \mathrm{d}F(x)}{F(1)} = 1 - \frac{\int_0^1 F(x) \, \mathrm{d}x}{F(1)},
\end{align*}
即$\int_0^1 F(x) \, \mathrm{d}x = (1 - s) F(1)$。又由$F$的下凸性可知
\begin{align*}
F(x) \leqslant  
\begin{cases}
\frac{F(1) - F(s)}{1 - s}(x - s) + F(s), & x \in [s,1] \\
\frac{F(s) - F(0)}{s}x + F(0), & x \in [0,s]
\end{cases}.
\end{align*}
于是
\begin{align*}
(1 - s) F(1) = \int_0^1 F(x) \, \mathrm{d}x &\leqslant  \int_0^s \left[ \frac{F(1) - F(s)}{1 - s}(x - s) + F(s) \right] \, \mathrm{d}x + \int_s^1 \left[ \frac{F(s) - F(0)}{s}x + F(0) \right] \, \mathrm{d}x \\
&= \frac{1}{2} F(s) + \frac{1 - s}{2} F(1).
\end{align*}
因此
\begin{align*}
\frac{1 - s}{2} F(1) \leqslant  \frac{1}{2} F(s) \implies F(1) \leqslant  \frac{1}{1 - s} F(s),
\end{align*}
故
\begin{align*}
\int_s^1 f(x) \, \mathrm{d}x = F(1) - F(s) \leqslant  \left( \frac{1}{1 - s} - 1 \right) F(s) = \frac{s}{1 - s} F(s) = \frac{s}{1 - s} \int_0^s f(x) \, \mathrm{d}x.
\end{align*}
\end{proof}

\begin{example}
求最小实数$C$,使得对一切满足$\int_{0}^{1}|f(x)|\mathrm{d}x = 1$的连续函数$f$,都有
\begin{align*}
\int_{0}^{1}|f(\sqrt{x})|\mathrm{d}x \leqslant C.
\end{align*}
\end{example}
\begin{remark}
这类证明最佳系数的问题,我们一般只需要找一个函数列,是其达到逼近取等即可.

本题将要找的函数列需要满足其积分值集中在$x=1$处,联想到Laplace方法章节具有类似性质的被积函数(即指数部分是$n$的函数),类似进行构造函数列即可.
\end{remark}
\begin{proof}
显然有
\begin{align*}
\int_0^1 |f(\sqrt{x})| \, \mathrm{d}x = 2 \int_0^1 t |f(t)| \, \mathrm{d}t \leqslant 2 \int_0^1 |f(t)| \, \mathrm{d}t = 2.
\end{align*}
令$f_n(t) = (n+1) t^n$,则$\int_0^1 f_n(t) \, \mathrm{d}t = 1$。于是
\begin{align*}
\int_0^1 |f_n(\sqrt{x})| \, \mathrm{d}x &= 2 \int_0^1 t |f(t)| \, \mathrm{d}t = 2 \int_0^1 t (n+1) t^n \, \mathrm{d}t 
= 2 (n+1) \int_0^1 t^{n+1} \, \mathrm{d}t = \frac{2(n+1)}{n+2} \to 2, n \to \infty.
\end{align*}
因此若$C < 2$,都存在$N \in \mathbb{N}$,使得$\int_0^1 |f_N(\sqrt{x})| \, \mathrm{d}x > C$。故$C = 2$就是最佳上界.
\end{proof}

\begin{example}
设$f\in C[0,1]$使得$\int_{0}^{1}x^{k}f(x)\mathrm{d}x = 1$,$k = 0,1,2,\cdots,n - 1$. 证明
\begin{align*}
\int_{0}^{1}|f(x)|^{2}\mathrm{d}x\geqslant n^{2}.
\end{align*}
\end{example}
\begin{proof}
设$a=(a_0,a_1,\cdots,a_{n-1})^T\in\mathbb{R}^n\setminus\{0\}$. 由Cauchy不等式及条件可知
\begin{align*}
\int_0^1 |f(x)|^2\mathrm{d}x \int_0^1 (a_0 + a_1x + \cdots + a_{n-1}x^{n-1})^2\mathrm{d}x &\geqslant \left[ \int_0^1 f(x)(a_0 + a_1x + \cdots + a_{n-1}x^{n-1})\mathrm{d}x \right]^2 \\
&= (a_0 + a_1 + \cdots + a_{n-1})^2 = \left( \sum_{j=0}^{n-1}a_j \right)^2.
\end{align*}
注意到
\begin{align*}
\int_0^1 (a_0 + a_1x + \cdots + a_{n-1}x^{n-1})^2\mathrm{d}x &= \int_0^1 \left( \sum_{j=0}^{n-1}a_jx^j \right)^2\mathrm{d}x = \int_0^1 \sum_{j=0}^{n-1}\sum_{i=0}^{n-1}a_ja_ix^{i+j}\mathrm{d}x \\
&= \sum_{j=0}^{n-1}\sum_{i=0}^{n-1}a_ja_i \int_0^1 x^{i+j}\mathrm{d}x = \sum_{j=0}^{n-1}\sum_{i=0}^{n-1}\frac{a_ja_i}{i+j+1}.
\end{align*}
因此
\begin{align*}
\int_0^1 |f(x)|^2\mathrm{d}x \geqslant \frac{\left( \sum\limits_{j=0}^{n-1}a_j \right)^2}{\sum\limits_{j=0}^{n-1}\sum\limits_{i=0}^{n-1}\frac{a_ja_i}{i+j+1}} = \frac{a^TJa}{a^THa},
\end{align*}
其中$J=\begin{pmatrix}1 & 1 & \cdots & 1 \\ 1 & 1 & \cdots & 1 \\ \vdots & \vdots & \ddots & \vdots \\ 1 & 1 & \cdots & 1\end{pmatrix}_{n\times n}$,$H=\left( \frac{1}{i+j+1} \right)_{n\times n}$. 于是我们只需求$\sup_{a\neq 0}\frac{a^TJa}{a^THa}$. 设$\lambda$为$\frac{a^TJa}{a^THa}$的一个大于$0$的上界,由\nrefexa{Basis of Algebra-example:例8.48}{(3)}可知$H$正定,则
\begin{align*}
\lambda \text{为}\frac{a^TJa}{a^THa}\text{的一个上界} &\Longleftrightarrow \lambda \geqslant \frac{a^TJa}{a^THa}, \forall a\in\mathbb{R}^n \\
&\Longleftrightarrow a^TJa \leqslant \lambda a^THa, \forall a\in\mathbb{R}^n \\
&\Longleftrightarrow a^T(\lambda H - J)a \geqslant 0, \forall a\in\mathbb{R}^n \\
&\Longleftrightarrow \lambda H - J\text{半正定}.
\end{align*}
因此$\sup_{a\neq 0}\frac{a^TJa}{a^THa} = \min\{\lambda \mid \lambda H - J\text{半正定}\} = \inf\{\lambda \mid \lambda H - J\text{半正定}\}$. 设$H_k,J_k$分别为$H,J$的$k$阶顺序主子阵,再根据\hyperref[Basis of Algebra-corollary:打洞原理推论]{打洞原理}及\nrefexa{Basis of Algebra-proposition:一些能写成两个向量乘积的矩阵}{(1)}可得
\begin{align*}
|\lambda H_k - J_k| &= |H_k| |\lambda I_k - H_k^{-1}J_k| = |H_k| |\lambda I_k - H_k^{-1}\mathbf{1}_k\mathbf{1}_k^T| \\
&= \lambda^{k-1} |H_k| (\lambda - \mathbf{1}_k^T H_k^{-1}\mathbf{1}_k).
\end{align*}
其中$\mathbf{1}_k^T = (1,1,\cdots,1)_{1\times k}$. 由$H$正定可知$|H_k| > 0$,又因为$\lambda > 0$,所以再由\reflem{Basis of Algebra-lemma:Hilbert矩阵逆矩阵元素和}可得
\begin{align*}
|\lambda H_k - J_k| > 0 \Longleftrightarrow \lambda > \mathbf{1}_k^T H_k^{-1}\mathbf{1}_k \xlongequal{\text{\reflem{Basis of Algebra-lemma:Hilbert矩阵逆矩阵元素和}}} n^2.
\end{align*}
因此对$\forall \lambda > n^2$,都有$\lambda H - J$的顺序主子式都大于$0$,故此时$\lambda H - J$正定. 于是对$\forall a\in\mathbb{R}^n\setminus\{0\}$,固定$a$,都有
\begin{align*}
a^T(\lambda H - J)a > 0, \forall \lambda > n^2.
\end{align*}
令$\lambda \to n^2$,则由$a^T(\lambda H - J)a$的连续性可知
\begin{align*}
a^T(n^2 H - J)a \geqslant 0.
\end{align*}
故$n^2 H - J$半正定. 因此$n^2 = \inf\{\lambda \mid \lambda H - J\text{半正定}\} = \sup_{a\neq 0}\frac{a^TJa}{a^THa}$. 结论得证.
\end{proof}

\begin{example}
设$A,B$都是$n$级实对称矩阵,若$B$正定,证明
\begin{align*}
\max_{\alpha\in\mathbb{R}^n\setminus\{0\}} \frac{\alpha^TA\alpha}{\alpha^TB\alpha} = \lambda_{\max}(AB^{-1}).
\end{align*}
\end{example}
\begin{proof}

\end{proof}

\begin{lemma}\label{lemma:g'(x)的Hölder连续相关结论}
设$\alpha>0, g\in C^1(\mathbb{R})$. 存在$a\in\mathbb{R}$使得$g(a)=\min_{x\in\mathbb{R}}g(x)$,如果
\begin{align}
|g'(x) - g'(y)| \leqslant M|x - y|^{\alpha}, \forall x,y\in\mathbb{R}, \label{17.39}
\end{align}
证明
\begin{align}
|g'(x)|^{\alpha + 1} \leqslant \left(\frac{\alpha + 1}{\alpha}\right)^{\alpha}[g(x) - g(a)]^{\alpha}M, \forall x\in\mathbb{R}. \label{17.40}
\end{align}
\end{lemma}
\begin{proof}
不妨设$g(a)=0$,否则用$g(x) - g(a)$代替$g(x)$. 当$M = 0$,则不等式\eqref{17.40}显然成立. 当$M\neq0$可以不妨设$M = 1$.

现在对非负函数$g$,现在我们正式开始我们的证明,当$g'(x_0)=0$,不等式\eqref{17.40}显然成立. 当$g'(x_0)>0$,则利用\eqref{17.39}有
\begin{align*}
g(x_0) &\geqslant g(x_0) - g(h)=\int_{h}^{x_0}g'(t)\mathrm{d}t \\
&\geqslant \int_{h}^{x_0}[g'(x_0) - |t - x_0|^{\alpha}]\mathrm{d}t \\
&= g'(x_0)(x_0 - h)-\frac{(x_0 - h)^{\alpha + 1}}{\alpha + 1},
\end{align*}
取$h = x_0 - |g'(x_0)|^{\frac{1}{\alpha}}$,就得到了$g(x_0)>\frac{\alpha}{\alpha + 1}|g'(x_0)|^{1 + \frac{1}{\alpha}}$,即不等式\eqref{17.40}成立. 类似的考虑$g'(x_0)<0$可得\eqref{17.40}.

当$g'(x_0)<0$,则利用\eqref{17.39}有
\begin{align*}
g(x_0) &\geqslant -g(h) + g(x_0)=-\int_{x_0}^{h}g'(t)\mathrm{d}t \\
&\geqslant -\int_{x_0}^{h}[g'(x_0) + |t - x_0|^{\alpha}]\mathrm{d}t \\
&= -g'(x_0)(h - x_0)-\frac{(h - x_0)^{\alpha + 1}}{\alpha + 1},
\end{align*}
取$h = x_0 + |g'(x_0)|^{\frac{1}{\alpha}}$,就得到了$g(x_0)>\frac{\alpha}{\alpha + 1}|g'(x_0)|^{1 + \frac{1}{\alpha}}$,即不等式\eqref{17.40}成立. 
\end{proof}

\begin{proposition}[Heisenberg(海森堡)不等式]\label{proposition:Heisenberg不等式}
设$f\in C^1(\mathbb{R})$,证明不等式
\begin{align}\label{17.29}
\left( \int_{\mathbb{R}}|f(x)|^2\mathrm{d}x \right)^2 \leqslant 4\int_{\mathbb{R}}x^2|f(x)|^2\mathrm{d}x\cdot\int_{\mathbb{R}}|f'(x)|^2\mathrm{d}x.
\end{align} 
\end{proposition}
\begin{remark}
直观上,直接Cauchy不等式,我们有
\begin{align*}
\left( \int_{\mathbb{R}}|f(x)|^2\mathrm{d}x \right)^2 \xlongequal{\textbf{分部积分}} 4\left( \int_{\mathbb{R}}xf(x)f'(x)\mathrm{d}x \right)^2 \leqslant 4\int_{\mathbb{R}}x^2|f(x)|^2\mathrm{d}x\cdot\int_{\mathbb{R}}|f'(x)|^2\mathrm{d}x.
\end{align*}
但是上述\textbf{分部积分}部分需要零边界条件(即需要$\lim_{x\to\infty}x|f(x)|^2 = 0$上式才成立). 但是其实专业数学知识告诉我们在$\mathbb{R}$上只要可积其实就可以分部积分的. 且看我们两种操作.
\end{remark}
\begin{proof}
{\heiti Method 1专业技术:}
对一般的$f\in C^1(\mathbb{R})$,假定
\begin{align*}
4\int_{\mathbb{R}}x^2|f(x)|^2\mathrm{d}x\cdot\int_{\mathbb{R}}|f'(x)|^2\mathrm{d}x < \infty.
\end{align*}
取紧化序列 $h_n, n\in\mathbb{N}$,则对每一个$n\in\mathbb{N}$,都有
\begin{align*}
\left( \int_{\mathbb{R}}|h_n(x)f(x)|^2\mathrm{d}x \right)^2 &\leqslant 4\int_{\mathbb{R}}x^2|h_n(x)f(x)|^2\mathrm{d}x\cdot\int_{\mathbb{R}}|(h_nf)'(x)|^2\mathrm{d}x \\
&= 4\int_{\mathbb{R}}x^2|h_n(x)f(x)|^2\mathrm{d}x\cdot\int_{\mathbb{R}}|h_n'(x)f(x) + h_n(x)f'(x)|^2\mathrm{d}x.
\end{align*}
右边让$n\to +\infty$,就有
\begin{align*}
\lim_{n\to\infty}\left[4\int_{\mathbb{R}}x^2|h_n(x)f(x)|^2\mathrm{d}x\cdot\int_{\mathbb{R}}|h_n'(x)f(x) + h_n(x)f'(x)|^2\mathrm{d}x\right] =\left[4\int_{\mathbb{R}}x^2|f(x)|^2\mathrm{d}x\cdot\int_{\mathbb{R}}|f'(x)|^2\mathrm{d}x\right].
\end{align*}
但是左边暂时不知道是否有$\left(\int_{\mathbb{R}}|f(x)|^2\mathrm{d}x\right)^2 < \infty$,因此不能直接换序. 但是\hyperref[Real Analysis-lemma:Fatou引理]{Fatou引理}告诉我们
\begin{align*}
\left( \int_{\mathbb{R}}|f(x)|^2\mathrm{d}x \right)^2 &= \left( \int_{\mathbb{R}}\lim_{n\to\infty}|h_n(x)f(x)|^2\mathrm{d}x \right)^2 \leqslant \lim_{n\to\infty}\left( \int_{\mathbb{R}}|h_n(x)f(x)|^2\mathrm{d}x \right)^2 \\
&\leqslant 4\int_{\mathbb{R}}x^2|f(x)|^2\mathrm{d}x\cdot\int_{\mathbb{R}}|f'(x)|^2\mathrm{d}x,
\end{align*}
从而不等式\eqref{17.29}成立.

{\heiti Method 2 正常方法:}对一般的$f\in C^1(\mathbb{R})$,假定
\begin{align*}
4\int_{\mathbb{R}}x^2|f(x)|^2\mathrm{d}x\cdot\int_{\mathbb{R}}|f'(x)|^2\mathrm{d}x < \infty.
\end{align*}
从分部积分需要看到,我们只需证明
\begin{align*}
\lim_{x\to\infty}x|f(x)|^2 = 0.
\end{align*}
我们以正无穷为例. 注意到
\begin{align}
\infty >\sqrt{\int_{x}^{\infty}y^2f^2(y)\mathrm{d}y\cdot\int_{x}^{\infty}|f'(y)|^2\mathrm{d}y} \stackrel{\text{Cauchy不等式}}{\geqslant} \int_{x}^{\infty}y|f'(y)f(y)|\mathrm{d}y \geqslant x\int_{x}^{\infty}|f'(y)f(y)|\mathrm{d}y, \label{17.30}
\end{align}
于是$\int_{x}^{\infty}f(y)f'(y)\mathrm{d}y=\lim_{y\to +\infty}\frac{1}{2}|f(y)|^2-\frac{1}{2}|f(x)|^2$收敛. 因此$\lim_{y\to +\infty}\frac{1}{2}|f(y)|^2$存在. 注意
$\int_{\mathbb{R}}x^2|f(x)|^2\mathrm{d}x < \infty$,因此由\hyperref[proposition:积分收敛必有子列趋于0]{积分收敛必有子列趋于0}可知, 存在$x_n\to \infty$,使得$\lim_{n\to\infty}x_n|f(x_n)| = 0$,
于是再结合$\lim_{y\to +\infty}\frac{1}{2}|f(y)|^2$存在可得
\begin{align*}
\lim_{n\to\infty}f(x_n) = 0 \Rightarrow \lim_{y\to +\infty}f(y) = 0.
\end{align*}
现在继续用\eqref{17.30},我们知道
\begin{align*}
\sqrt{\int_{x}^{\infty}y^2f^2(y)\mathrm{d}y\cdot\int_{x}^{\infty}|f'(y)|^2\mathrm{d}y} \geqslant x\int_{x}^{\infty}f'(y)f(y)\mathrm{d}y = \frac{x}{2}|f(x)|^2,
\end{align*}
令$x\to +\infty$,由Cauchy收敛准则即得$\sqrt{\int_{x}^{\infty}y^2f^2(y)\mathrm{d}y\cdot\int_{x}^{\infty}|f'(y)|^2\mathrm{d}y}\to 0$,
从而$\lim_{x\to +\infty}x|f(x)|^2 = 0$,这就完成了证明.于是由分部积分和Cauchy不等式可知,对$f\in C^{\infty}(\mathbb{R})$,我们有
\begin{align*}
\left( \int_{\mathbb{R}}|f(x)|^2\mathrm{d}x \right)^2 \xlongequal{\text{分部积分}} 4\left( \int_{\mathbb{R}}xf(x)f'(x)\mathrm{d}x \right)^2 \leqslant 4\int_{\mathbb{R}}x^2|f(x)|^2\mathrm{d}x\cdot\int_{\mathbb{R}}|f'(x)|^2\mathrm{d}x,
\end{align*}
即不等式\eqref{17.29}成立.
\end{proof}

\begin{example}
设$f:[0,+\infty)\to (0,1)$是内闭Riemman可积函数,若$\int_0^{+\infty} f(x) \mathrm{d}x$与$\int_0^{+\infty} x f(x) \mathrm{d}x$均收敛,证明
\begin{align}
\left( \int_0^{+\infty} f(x) \mathrm{d}x \right)^2 < 2 \int_0^{+\infty} x f(x) \mathrm{d}x.\label{eq::103.1}
\end{align}
\end{example}
\begin{proof}
记$a=\int_0^{\infty} f(x) \mathrm{d}x>0$,待定$s>0$,则不等式\eqref{eq::103.1}等价于
\begin{align*}
\int_0^{\infty} x f(x) \mathrm{d}x = \int_0^s x f(x) \mathrm{d}x + \int_s^{\infty} x f(x) \mathrm{d}x > \frac{a^2}{2}.
\end{align*}
于是
\begin{align*}
&\quad \quad \int_0^s x f(x) \mathrm{d}x + s \int_s^{\infty} f(x) \mathrm{d}x \geqslant \frac{a^2}{2} \Longleftrightarrow \int_0^s x f(x) \mathrm{d}x + s \left( a - \int_0^s f(x) \mathrm{d}x \right) \geqslant \frac{a^2}{2} \\
&\Longleftrightarrow \frac{a^2}{2} - sa + s \int_0^s f(x) \mathrm{d}x - \int_0^s x f(x) \mathrm{d}x \leqslant 0 \Longleftrightarrow \frac{a^2}{2} - sa + \int_0^s (s - x) f(x) \mathrm{d}x \leqslant 0.
\end{align*}
利用$f<1$,取$s=a$,则我们有
\begin{align*}
\frac{a^2}{2} - sa + \int_0^s (s - x) f(x) \mathrm{d}x = -\frac{a^2}{2} + \int_0^a (a - x) f(x) \mathrm{d}x < -\frac{a^2}{2} + \int_0^a (a - x) \mathrm{d}x = 0.
\end{align*}
从而
\begin{align*}
\int_0^a x f(x) \mathrm{d}x + a \int_a^{\infty} f(x) \mathrm{d}x > \frac{a^2}{2}
\end{align*}
成立.因此
\begin{align*}
\int_0^{\infty} x f(x) \mathrm{d}x = \int_0^a x f(x) \mathrm{d}x + \int_a^{\infty} x f(x) \mathrm{d}x \geqslant \int_0^a x f(x) \mathrm{d}x + a \int_a^{\infty} f(x) \mathrm{d}x > \frac{a^2}{2}.
\end{align*}
这就证明了不等式\eqref{eq::103.1}.
\end{proof}

\begin{proposition}\label{proposition:单调函数减任意常数积分小于减其中点的积分}
设 $f$ 是 $[0,1]$ 上的单调函数. 求证: 对任意实数 $a$ 有
\begin{align}\label{eq:5.1.1}
\int_{0}^{1}|f(x)-a|\,\mathrm{d}x\geqslant\int_{0}^{1}\left|f(x)-f\left(\frac{1}{2}\right)\right|\,\mathrm{d}x.
\end{align}
\end{proposition}
\begin{proof}
不妨设 $f$ 是单调递增函数. 注意到 $\frac{1}{2}$ 是积分区间的中点, 将式 \eqref{eq:5.1.1} 右端的积分从 $\frac{1}{2}$ 处分成两部分来处理.
\begin{align*}
\int_{0}^{1}\left|f(x)-f\left(\frac{1}{2}\right)\right|\,\mathrm{d}x&=\int_{0}^{\frac{1}{2}}\left(f\left(\frac{1}{2}\right)-f(x)\right)\,\mathrm{d}x+\int_{\frac{1}{2}}^{1}\left(f(x)-f\left(\frac{1}{2}\right)\right)\,\mathrm{d}x\\
&=\int_{0}^{\frac{1}{2}}(-f(x))\,\mathrm{d}x+\int_{\frac{1}{2}}^{1}f(x)\,\mathrm{d}x\\
&=\int_{0}^{\frac{1}{2}}(a-f(x))\,\mathrm{d}x+\int_{\frac{1}{2}}^{1}(f(x)-a)\,\mathrm{d}x\\
&\leqslant\int_{0}^{\frac{1}{2}}|a-f(x)|\,\mathrm{d}x+\int_{\frac{1}{2}}^{1}|f(x)-a|\,\mathrm{d}x\\
&=\int_{0}^{1}|f(x)-a|\,\mathrm{d}x.
\end{align*}
故式 \eqref{eq:5.1.1} 成立.
\end{proof}

\begin{example}
若 $[a,b]$ 上的可积函数列 $\{f_n\}$ 在 $[a,b]$ 上一致收敛于函数 $f$, 则 $f$ 在 $[a,b]$ 上可积.
\end{example}
\begin{proof}
由已知条件, 对任意正数 $\varepsilon$, 存在正整数 $k$ 使得
$$|f_k(x)-f(x)|<\frac{\varepsilon}{4(b-a)}, \quad x \in[a,b].$$
因为 $f_k \in R([a,b])$, 所以存在 $[a,b]$ 的一个分割
$$T: a=x_0<x_1<\cdots<x_n=b$$
使得
$$\sum_{j=1}^n \omega_j(f_k)(x_j-x_{j-1})<\frac{\varepsilon}{2},$$
这里 $\omega_j(f_k)$ 是 $f_k$ 在区间 $[x_{j-1},x_j]$ 上的振幅. 因为
$$
\begin{aligned}
|f(x)-f(y)| & \leqslant |f(x)-f_k(x)|+\left|f_k(x)-f_k(y)\right|+\left|f_k(y)-f(y)\right| \\
& \leqslant  \frac{\varepsilon}{2(b-a)}+\left|f_k(x)-f_k(y)\right|,
\end{aligned}
$$
所以
$$\omega_j(f) \leqslant  \frac{\varepsilon}{2(b-a)}+\omega_j(f_k).$$
于是
$$\sum_{j=1}^n \omega_j(f)\left(x_j-x_{j-1}\right) \leqslant  \frac{\varepsilon}{2}+\sum_{j=1}^n \omega_j\left(f_k\right)\left(x_j-x_{j-1}\right)<\varepsilon.$$
故 $f$ 在 $[a,b]$ 上可积.
\end{proof}

\begin{example}
设 $f$ 在 $[a,b]$ 上非负可积. 求证: 数列 $I_n=\left(\frac{1}{b-a}\int_a^b f^n(x)\mathrm{d}x\right)^{\frac{1}{n}}$ 是单调递增的.
\end{example}
\begin{remark}
当 $f$ 是连续函数时, 可以进一步证明 $\lim\limits_{n\to+\infty}I_n=\max\limits_{x\in[a,b]}f(x)$(见\refexa{example-3.31}).
\end{remark}
\begin{proof}
要比较 $I_n$ 与 $I_{n+1}$ 的大小, 就要比较 $f^n$ 的积分与 $f^{n+1}$ 之间的关系. 这可以利用\hyperref[theorem:Hold(赫尔德)不等式(积分形式)]{Hölder不等式}:
$$
\begin{aligned}
&\int_a^b f^n(x)\mathrm{d}x = \int_a^b 1\cdot f^n(x)\mathrm{d}x \\
&\leqslant  \left(\int_a^b 1^{n+1}\mathrm{d}x\right)^{\frac{1}{n+1}} \left(\int_a^b (f^n(x))^{\frac{n+1}{n}}\mathrm{d}x\right)^{\frac{n}{n+1}} \\
&= (b-a)^{\frac{1}{n+1}} \left(\int_a^b f^{n+1}(x)\mathrm{d}x\right)^{\frac{n}{n+1}},
\end{aligned}
$$
即
$$\left(\frac{1}{b-a}\int_a^b f^n(x)\mathrm{d}x\right)^{\frac{1}{n}} \leqslant  \left(\frac{1}{b-a}\int_a^b f^{n+1}(x)\mathrm{d}x\right)^{\frac{1}{n+1}}.$$
故 $\{I_n\}$ 是单调递增数列.
\end{proof}

\begin{example}
设 $f$ 在 $[a,b]$ 上连续可导, 且 $f(a)=0$. 求证: 对 $p\geqslant 1$ 有
$$\int_a^b|f(x)|^p\mathrm{d}x\leqslant \frac{1}{p}\int_a^b\left[(b-a)^p-(x-a)^p\right]|f'(x)|^p\mathrm{d}x.$$
\end{example}
\begin{proof}
为了建立 $|f|^p$ 的积分与 $|f'|^p$ 的积分之间的关系, 先建立 $|f|$ 与 $|f'|$ 的积分的关系. 根据 Newton-Leibniz 公式, 有
$$f(x)=f(x)-f(a)=\int_a^x f'(t)\mathrm{d}t,\quad x\in[a,b].$$
所以对于 $p>1$ 应用\hyperref[theorem:Hold(赫尔德)不等式(积分形式)]{Hölder积分不等式}, 可得
$$
\begin{aligned}
|f(x)| &= \left|\int_a^x f'(t)\mathrm{d}t\right|\leqslant  \left(\int_a^x 1^q\mathrm{d}t\right)^{\frac{1}{q}} \left(\int_a^x |f'(t)|^p\mathrm{d}t\right)^{\frac{1}{p}} \\
&= (x-a)^{\frac{1}{q}} \left(\int_a^x |f'(t)|^p\mathrm{d}t\right)^{\frac{1}{p}}.
\end{aligned}
$$
其中$\frac{1}{p}+\frac{1}{q}=1$.因而
$$|f(x)|^p\leqslant (x-a)^{p-1}\int_a^x |f'(t)|^p\mathrm{d}t,\quad x\in[a,b].$$
注意到上式对 $p=1$ 也是成立的. 上式两边在 $[a,b]$ 上积分, 可得
$$\int_a^b|f(x)|^p\mathrm{d}x\leqslant \int_a^b(x-a)^{p-1}\left(\int_a^x |f'(t)|^p\mathrm{d}t\right)\mathrm{d}x.$$
注意到 $\int_a^x |f'(t)|^p\mathrm{d}t$ 是 $|f'|^p$ 的一个原函数. 对上式右端分部积分, 可得
$$
\begin{aligned}
\int_a^b|f(x)|^p\mathrm{d}x &\leqslant  \frac{1}{p}(x-a)^p\int_a^x |f'(t)|^p\mathrm{d}t\bigg|_a^b - \frac{1}{p}\int_a^b(x-a)^p|f'(x)|^p\mathrm{d}x \\
&= \frac{1}{p}(b-a)^p\int_a^b |f'(t)|^p\mathrm{d}t - \frac{1}{p}\int_a^b(x-a)^p|f'(x)|^p\mathrm{d}x \\
&= \frac{1}{p}\int_a^b\left[(b-a)^p-(x-a)^p\right]|f'(x)|^p\mathrm{d}x.
\end{aligned}
$$
\end{proof}

\begin{example}
设 $f$ 是 $[0,a]$ 上的连续函数, 且存在正常数 $M,c$ 使得
$$|f(x)|\leqslant  M+c\int_0^x|f(t)|\mathrm{d}t,$$
求证: $|f(x)|\leqslant  M\mathrm{e}^{cx}\,(\forall x\in[0,a])$.
\end{example}
\begin{proof}
证明 注意对于包含变上限积分的不等式常可以转化为微分的不等式. 令
$$F(x)=\int_0^x|f(t)|\mathrm{d}t,$$
则条件中的不等式就是
$$F'(x)\leqslant  M+cF(x).$$
令
$$G(x)=F(x)\mathrm{e}^{-cx}+\frac{M}{c}\mathrm{e}^{-cx},$$
则有
$$
\begin{aligned}
G'(x) &= F'(t)\mathrm{e}^{-cx}-cF(x)\mathrm{e}^{-cx}-M\mathrm{e}^{-cx} \\
&= |f(x)|\mathrm{e}^{-cx}-cF(x)\mathrm{e}^{-cx}-M\mathrm{e}^{-cx} \\
&\leqslant  (M+cF(x))\mathrm{e}^{-cx}-cF(x)\mathrm{e}^{-cx}-M\mathrm{e}^{-cx}=0.
\end{aligned}
$$
这说明 $G$ 在 $[0,a]$ 上单调递减. 因为 $G(0)=\frac{M}{c}$, 所以 $G\leqslant \frac{M}{c}$. 因而
$$F(x)+\frac{M}{c}\leqslant \frac{M}{c}\mathrm{e}^{cx}.$$
再结合条件可得 $|f(x)|\leqslant  M+cF(x)\leqslant  M\mathrm{e}^{cx}$.
\end{proof}

\begin{example}
设 $f$ 在区间 $[0,1]$ 上连续且对任意 $x,y\in[0,1]$, 有
$$xf(y)+yf(x)\leqslant 1.$$
求证: $\int_0^1 f(x)\mathrm{d}x\leqslant \frac{\pi}{4}$.
\end{example}
\begin{proof}
结论中出现 $\pi$ 且条件中要求 $x,y\in[0,1]$. 因此将条件中的 $x,y$ 分别换成 $\sin t$ 和 $\cos t$, 有
$$f(\cos t)\sin t+f(\sin t)\cos t\leqslant 1,\quad t\in\left[0,\frac{\pi}{2}\right].$$
将此式在 $\left[0,\frac{\pi}{2}\right]$ 上积分, 得
$$\int_0^{\frac{\pi}{2}}f(\cos t)\sin t\mathrm{d}t+\int_0^{\frac{\pi}{2}}f(\sin t)\cos t\mathrm{d}t\leqslant \frac{\pi}{2}.$$
由对称性可知上式左端的两个积分相等. 因而
$$\int_0^{\frac{\pi}{2}}f(\sin t)\cos t\mathrm{d}t\leqslant \frac{\pi}{4}.$$
作变换 $\sin t=x$ 即得 $\int_0^1 f(x)\mathrm{d}x\leqslant \frac{\pi}{4}$.
\end{proof}

\begin{example}
设 $f$ 在区间 $[0,1]$ 上连续且对任意 $x,y\in[0,1]$, 有
$$xf(y)+yf(x)\leqslant 1.$$
求证: $\int_0^1 f(x)\mathrm{d}x\leqslant \frac{\pi}{4}$.
\end{example}
\begin{remark}
结论中的 $\frac{\pi}{4}$ 是最佳的, 这只要取 $f(x)=\sqrt{1-x^2}$ 即可验证.
\end{remark}
\begin{proof}
结论中出现 $\pi$ 且条件中要求 $x,y\in[0,1]$. 因此将条件中的 $x,y$ 分别换成 $\sin t$ 和 $\cos t$, 有
$$f(\cos t)\sin t+f(\sin t)\cos t\leqslant 1,\quad t\in\left[0,\frac{\pi}{2}\right].$$
将此式在 $\left[0,\frac{\pi}{2}\right]$ 上积分, 得
$$\int_0^{\frac{\pi}{2}}f(\cos t)\sin t\mathrm{d}t+\int_0^{\frac{\pi}{2}}f(\sin t)\cos t\mathrm{d}t\leqslant \frac{\pi}{2}.$$
由\hyperref[theorem:区间再现恒等式]{区间再现恒等式}可知上式左端的两个积分相等. 因而
$$\int_0^{\frac{\pi}{2}}f(\sin t)\cos t\mathrm{d}t\leqslant \frac{\pi}{4}.$$
作变换 $\sin t=x$ 即得 $\int_0^1 f(x)\mathrm{d}x\leqslant \frac{\pi}{4}$.
\end{proof}

\begin{example}
设 $f$ 在区间 $[0,1]$ 上有可积的导函数且满足 $f(0)=0,f(1)=1$. 求证: 对任意 $a\geqslant 0$ 有
$$\int_0^1|af(x)+f'(x)|\mathrm{d}x\geqslant 1.$$
\end{example}
\begin{proof}
因为 $\mathrm{e}^{-ax}\geqslant \mathrm{e}^{-a}\,(0\leqslant  x\leqslant 1)$, 所以
$$
\begin{aligned}
\int_0^1|af(x)+f'(x)|\mathrm{d}x &= \int_0^1|(\mathrm{e}^{ax}f(x))'\mathrm{e}^{-ax}|\mathrm{d}x \geqslant  \mathrm{e}^{-a}\int_0^1|(\mathrm{e}^{ax}f(x))'|\mathrm{d}x \\
&\geqslant  \mathrm{e}^{-a}\left|\int_0^1(\mathrm{e}^{ax}f(x))'\mathrm{d}x\right| = \mathrm{e}^{-a}|\mathrm{e}^{a}f(1)-f(0)| = 1.
\end{aligned}
$$
\end{proof}

\begin{example}
设 \( f \) 在 \([0,2]\) 上可导且 \(|f'| \leqslant 1\),\( f(0) = f(2) = 1 \)。求证:
\[
1 \leqslant \int_{0}^{2} f(x) \, \mathrm{d}x \leqslant 3
\]
\end{example}
\begin{proof}
由Taylor中值定理可知,存在$\xi_1\in[0,1]$,$\xi_2\in[1,2]$,使得
\begin{align*}
f(x)&=1+f'(\xi_1)x,\forall x\in[0,1].\\
f(x)&=1+f'(\xi_2)(x-2),\forall x\in[1,2].
\end{align*}
于是
\begin{align*}
\int_0^2 f(x)\mathrm{d}x&=\int_0^1 f(x)\mathrm{d}x+\int_1^2 f(x)\mathrm{d}x\\
&=\int_0^1 [1+f'(\xi_1)x]\mathrm{d}x+\int_1^2 [1+f'(\xi_2)(x-2)]\mathrm{d}x\\
&=2+\frac{f'(\xi_1)}{2}-\frac{1}{2}f'(\xi_2).
\end{align*}
由$|f'|\leqslant1$可知
\begin{align*}
1=2-\frac{1}{2}-\frac{1}{2}\leqslant 2+\frac{f'(\xi_1)}{2}-\frac{1}{2}f'(\xi_2)\leqslant 2+\frac{1}{2}+\frac{1}{2}=3.
\end{align*}
故
\begin{align*}
1\leqslant \int_0^2 f(x)\mathrm{d}x\leqslant 3.
\end{align*}
\end{proof}

\begin{example}
设 \( f \) 在区间 \([0,1]\) 上连续可导,且 \( f(0) = f(1) = 0 \)。求证:
\[
\left( \int_{0}^{1} f(x) \, \mathrm{d}x \right)^{2} \leqslant \frac{1}{12} \int_{0}^{1} \left( f'(x) \right)^{2} \, \mathrm{d}x
\]
且等号成立当且仅当 \( f(x) = Ax(1 - x) \),其中 \( A \) 是常数。
\end{example}
\begin{note}
对于在两个端点取零值的连续可导函数, 可以考虑 \((ax + b)f'(x)\) 的积分, 并利用分部积分公式得到一些结果.
\end{note}
\begin{proof}
设 \(t\) 是任意常数, 有
\[
\int_0^1 (x + t)f'(x)\mathrm{d}x = (x + t)f(x) \Big|_0^1 - \int_0^1 f(x)\mathrm{d}x = -\int_0^1 f(x)\mathrm{d}x.
\]
于是利用 Cauchy 积分不等式, 可得
\begin{align*}
\left(\int_0^1 f(x)\mathrm{d}x\right)^2 &= \left(\int_0^1 (x + t)f'(x)\mathrm{d}x\right)^2\\
&\leqslant \int_0^1 (x + t)^2\mathrm{d}x \int_0^1 (f'(x))^2\mathrm{d}x\\
&= \left(\frac{1}{3} + t + t^2\right)\int_0^1 (f'(x))^2\mathrm{d}x.
\end{align*}
取 \(t = -\frac{1}{2}\), 即得所证不等式. 当所证不等式成为等式时, 上面所用的 Cauchy 不等式应为等式. 因此, 存在常数 \(C\) 使得 \(f'(x) = C\left(x - \frac{1}{2}\right)\). 注意到 \(f(0) = f(1) = 0\), 可得 \(f(x) = Ax(1 - x)\), 这里 \(A\) 为任意常数.
\end{proof}

\begin{example}
设 \( f,g \) 是区间 \([0,1]\) 上的连续函数,使得对 \([0,1]\) 上任意满足 \( \varphi(0) = \varphi(1) = 0 \) 的连续可导函数 \( \varphi \) 有
\[
\int_{0}^{1} \left[ f(x)\varphi'(x) + g(x)\varphi(x) \right] \mathrm{d}x = 0
\]
求证:\( f \) 可导,且 \( f' = g \).
\end{example}
\begin{proof}
证明 设
\[
c = \int_{0}^{1} f(t)\mathrm{d}t - \int_{0}^{1} g(t)\mathrm{d}t + \int_{0}^{1} tg(t)\mathrm{d}t
\]
考察函数
\[
G(x) = \int_{0}^{x} g(t)\mathrm{d}t + c
\]
显然 \( G \) 可导且 \( G'(x) = g(x) \),\( G(1) = \int_{0}^{1} g(t)\mathrm{d}t + c \)。只需证明 \( f = G \)。令
\[
\varphi(x) = \int_{0}^{x} \left[ f(t) - G(t) \right] \mathrm{d}t
\]
则 \( \varphi \) 可导,且 \( \varphi(0) = 0 \),
\begin{align*}
\varphi(1) &= \int_{0}^{1} f(t)\mathrm{d}t - \int_{0}^{1} G(t)\mathrm{d}t \\
&= \int_{0}^{1} f(t)\mathrm{d}t - \left[ tG(t)\big|_{0}^{1} - \int_{0}^{1} tg(t)\mathrm{d}t \right] \\
&= \int_{0}^{1} f(t)\mathrm{d}t - G(1) + \int_{0}^{1} tg(t)\mathrm{d}t \\
&= \int_{0}^{1} f(t)\mathrm{d}t - \int_{0}^{1} g(t)\mathrm{d}t - c + \int_{0}^{1} tg(t)\mathrm{d}t \\
&= 0
\end{align*}
根据条件有
\[
\int_{0}^{1} \left[ f(x)\varphi'(x) + g(x)\varphi(x) \right] \mathrm{d}x = 0
\]
因为
\[
\int_{0}^{1} g(x)\varphi(x)\mathrm{d}x = G(x)\varphi(x)\big|_{0}^{1} - \int_{0}^{1} G(x)\varphi'(x)\mathrm{d}x = -\int_{0}^{1} G(x)\varphi'(x)\mathrm{d}x
\]
所以
\[
\int_{0}^{1} \left[ f(x) - G(x) \right] \varphi'(x)\mathrm{d}x = 0
\]
注意到 \( \varphi' = f - G \)。我们有
\[
\int_{0}^{1} \left[ f(x) - G(x) \right]^{2} \mathrm{d}x = 0
\]
于是 \( f = G \)。
\end{proof}

\begin{proposition}
设 \( f \) 是区间 \([a,b]\) 上的严格单调递减连续函数,\( f(a) = b \),\( f(b) = a \),\( g \) 是 \( f \) 的反函数。求证:
\[
\int_{a}^{b} f(x) \, \mathrm{d}x = \int_{a}^{b} g(x) \, \mathrm{d}x.
\]
特别地,对 \( p > 0 \),\( q > 0 \) 取 \( f(x) = (1 - x^{q})^{\frac{1}{p}} \),\( g(x) = (1 - x^{p})^{\frac{1}{q}} \),可得
\[
\int_{0}^{1} (1 - x^{p})^{\frac{1}{q}} \, \mathrm{d}x = \int_{0}^{1} (1 - x^{q})^{\frac{1}{p}} \, \mathrm{d}x.
\]
\end{proposition}
\begin{proof}
因为可以用在 \( a,b \) 分别插值于 \( f(a),f(b) \) 的严格单调递减的多项式(也可以用Bernstein多项式)在 \([a,b]\) 上一致逼近 \( f(x) \),所以只需对 \( f \) 是连续可微函数的情况证明。

作变换 \( x = f(t) \),有
\begin{align*}
\int_{a}^{b} g(x) \, \mathrm{d}x &= \int_{b}^{a} g(f(t))f'(t) \, \mathrm{d}t = \int_{b}^{a} t f'(t) \, \mathrm{d}t \\
&= t f(t) \Big |_{b}^{a} - \int_{b}^{a} f(t) \, \mathrm{d}t = \int_{a}^{b} f(t) \, \mathrm{d}t
\end{align*}
故所证等式成立。
\end{proof}

\begin{example}
设 \( f \) 是区间 \([a,b]\) 上的连续可微函数。求证:
\[
\max_{a\leqslant x\leqslant b} f(x)\leqslant \frac{1}{b-a}\int_a^b{f(x)\,\mathrm{d}x}+\int_a^b{|f' (x)|\,\mathrm{d}x}.
\]
\end{example}
\begin{proof}
由于有限闭区间上连续函数可取到最大值,可设 \( \max_{a \leqslant x \leqslant b} f(x) = f(y) \)。因此对任意 \( x \in [a,b] \),有
\[
\max_{a \leqslant x \leqslant b} f(x) - f(x) = f(y) - f(x) = \int_{x}^{y} f'(t) \, \mathrm{d}t \leqslant \int_{a}^{b} |f'(t)| \, \mathrm{d}t
\]
关于 \( x \) 在 \([a,b]\) 上积分,即得
\[
(b - a) \max_{a \leqslant x \leqslant b} f(x) - \int_{a}^{b} f(x) \, \mathrm{d}x \leqslant (b-a)\int_{a}^{b} |f'(t)| \, \mathrm{d}t.
\]
两边除以 \( b - a \) 即得所证。
\end{proof}

\begin{example}
设 \( \alpha \in \left[0, \frac{1}{2}\right] \),\( f \in C^1[0,1] \) 且满足 \( f(1) = 0 \)。求证:
\[
\int_{0}^{1} |f(x)|^2 \, \mathrm{d}x + \left( \int_{0}^{1} |f(x)| \, \mathrm{d}x \right)^2 \leqslant \frac{4}{3 - 4\alpha} \int_{0}^{1} x^{2\alpha + 1} |f'(x)|^2 \, \mathrm{d}x.
\]
\end{example}
\begin{proof}
设 \( \alpha \in [0,1) \) 且 \( \alpha \neq \frac{1}{2} \)。根据 Newton-Leibniz 公式和 Cauchy 不等式,对 \( x \in [0,1] \) 有
\begin{align*}
f^2(x) &= \left( \int_{x}^{1} f'(t) \, \mathrm{d}t \right)^2 = \left( \int_{x}^{1} t^{-\alpha} \cdot t^{\alpha} f'(t) \, \mathrm{d}t \right)^2 \\
&\leqslant \int_{x}^{1} t^{-2\alpha} \, \mathrm{d}t \int_{x}^{1} t^{2\alpha} |f'(t)|^2 \, \mathrm{d}t = \frac{1}{1 - 2\alpha} (1 - x^{1 - 2\alpha}) \int_{x}^{1} t^{2\alpha} |f'(t)|^2 \, \mathrm{d}t
\end{align*}
因此,由分部积分得
\begin{align*}
\int_{0}^{1} f^2(x) \, \mathrm{d}x &\leqslant \frac{1}{1 - 2\alpha} \int_{0}^{1} (1 - x^{1 - 2\alpha}) \left( \int_{x}^{1} t^{2\alpha} |f'(t)|^2 \, \mathrm{d}t \right) \, \mathrm{d}x \\
&= \frac{1}{1 - 2\alpha} \left[ \left. \left( x - \frac{x^{2 - 2\alpha}}{2 - 2\alpha} \right) \int_{x}^{1} t^{2\alpha} |f'(t)|^2 \, \mathrm{d}t \right|_{0}^{1} \right. \\
&\quad \left. + \int_{0}^{1} \left( x - \frac{x^{2 - 2\alpha}}{2 - 2\alpha} \right) x^{2\alpha} |f'(x)|^2 \, \mathrm{d}x \right]
\end{align*}
即
\begin{align}
\int_{0}^{1} f^2(x) \, \mathrm{d}x \leqslant \frac{1}{1 - 2\alpha} \int_{0}^{1} x^{2\alpha + 1} |f'(x)|^2 \, \mathrm{d}x - \frac{1}{(1 - 2\alpha)(2 - 2\alpha)} \int_{0}^{1} x^2 |f'(x)|^2 \, \mathrm{d}x .\label{5.1.5}
\end{align}
另一方面,有
\begin{align*}
\int_{0}^{1} |f(x)| \, \mathrm{d}x &= \int_{0}^{1} \left| \int_{1}^{x} f'(t) \, \mathrm{d}t \right| \, \mathrm{d}x \leqslant \int_{0}^{1} \left( \int_{x}^{1} |f'(t)| \, \mathrm{d}t \right) \, \mathrm{d}x \\
&= \left. x \left( \int_{x}^{1} |f'(t)| \, \mathrm{d}t \right) \right|_{0}^{1} + \int_{0}^{1} x |f'(x)| \, \mathrm{d}x
\end{align*}
因此
\begin{align}
\int_{0}^{1} |f(x)| \, \mathrm{d}x \leqslant \int_{0}^{1} x |f'(x)| \, \mathrm{d}x. \label{5.1.6}
\end{align}
再由 Cauchy 不等式,有
\begin{align*}
\left( \int_{0}^{1} |f(x)| \, \mathrm{d}x \right)^2 &\leqslant \left( \int_{0}^{1} x^{\frac{1 - 2\alpha}{2}} \cdot x^{\frac{2\alpha + 1}{2}} |f'(x)| \, \mathrm{d}x \right)^2 \\
&\leqslant \left( \int_{0}^{1} x^{1 - 2\alpha} \, \mathrm{d}x \right) \left( \int_{0}^{1} x^{2\alpha + 1} |f'(x)|^2 \, \mathrm{d}x \right)
\\
&= \frac{1}{2 - 2\alpha} \int_{0}^{1} x^{2\alpha + 1} |f'(x)|^2 \, \mathrm{d}x
\end{align*}
结合式 \eqref{5.1.5},可得
\begin{align}
\int_{0}^{1} |f(x)|^2 \, \mathrm{d}x + \left( \int_{0}^{1} |f(x)| \, \mathrm{d}x \right)^2 &\leqslant \frac{1}{(2\alpha - 1)(2 - 2\alpha)} \int_{0}^{1} x^2 |f'(x)|^2 \, \mathrm{d}x - \frac{3 - 4\alpha}{(2\alpha - 1)(2 - 2\alpha)} \int_{0}^{1} x^{2\alpha + 1} |f'(x)|^2 \, \mathrm{d}x \label{5.1.7}
\end{align}
在上式中取 \( \alpha = \frac{3}{4} \),即得
\begin{align}
\int_{0}^{1} |f(x)|^2 \, \mathrm{d}x + \left( \int_{0}^{1} |f(x)| \, \mathrm{d}x \right)^2 \leqslant 4 \int_{0}^{1} x^2 |f'(x)|^2 \, \mathrm{d}x .\label{5.1.8}
\end{align}
对 \( \alpha \in \left[0, \frac{1}{2}\right) \),将式 \eqref{5.1.7} 两边乘以 \( 4(1 - 2\alpha)(2 - 2\alpha) \) 再与式 \eqref{5.1.8} 相加可得
\[
\int_{0}^{1} |f(x)|^2 \, \mathrm{d}x + \left( \int_{0}^{1} |f(x)| \, \mathrm{d}x \right)^2 \leqslant \frac{4}{3 - 4\alpha} \int_{0}^{1} x^{2\alpha + 1} |f'(x)|^2 \, \mathrm{d}x.
\]
\end{proof}

\begin{example}
设 \( f \) 在 \([0,1]\) 上非负且连续可导. 求证:
\[
\left| \int_{0}^{1} f^3(x) \, \mathrm{d}x - f^2(0) \int_{0}^{1} f(x) \, \mathrm{d}x \right| \leqslant \max_{0 \leqslant x \leqslant 1} |f'(x)| \left( \int_{0}^{1} f(x) \, \mathrm{d}x \right)^2
\]
\end{example}
\begin{proof}
记 \( M = \max_{0 \leqslant x \leqslant 1} |f'(x)| \),则有
\[
-M f(x) \leqslant f(x) f'(x) \leqslant M f(x), \quad \forall x \in [0,1]
\]
因此
\[
-M \int_{0}^{x} f(t) \, \mathrm{d}t \leqslant \frac{1}{2} f^2(x) - \frac{1}{2} f^2(0) \leqslant M \int_{0}^{x} f(t) \, \mathrm{d}t, \quad \forall x \in [0,1]
\]
上式两边乘以 \( f \) 得
\[
-M f(x) \int_{0}^{x} f(t) \, \mathrm{d}t \leqslant \frac{1}{2} f^3(x) - \frac{1}{2} f^2(0) f(x) \leqslant M f(x) \int_{0}^{x} f(t) \, \mathrm{d}t, \quad \forall x \in [0,1]
\]
将上式关于变量 \( x \) 在 \([0,1]\) 上积分,得
\[
-M \left( \int_{0}^{1} f(x) \, \mathrm{d}x \right)^2 \leqslant \int_{0}^{1} f^3(x) \, \mathrm{d}x - f^2(0) \int_{0}^{1} f(x) \, \mathrm{d}x \leqslant M \left( \int_{0}^{1} f(x) \, \mathrm{d}x \right)^2
\]
结论得证.
\end{proof}

\begin{example}
设 \( f \) 在 \([0,1]\) 上非负单调递增连续函数,\( 0 < \alpha < \beta < 1 \)。求证:
\[
\int_{0}^{1} f(x) \, \mathrm{d}x \geqslant \frac{1 - \alpha}{\beta - \alpha} \int_{\alpha}^{\beta} f(x) \, \mathrm{d}x
\]
并且 \( \frac{1 - \alpha}{\beta - \alpha} \) 不能换为更大的数。
\end{example}
\begin{remark}
当函数具有单调性时,小区间上的积分与整体区间上的积分可比较大小.
\end{remark}
\begin{proof}
根据积分中值定理,存在 \( \xi \in (\alpha, \beta) \) 使得
\[
\int_{\alpha}^{\beta} f(x) \, \mathrm{d}x = f(\xi)(\beta - \alpha)
\]
因而由 \( f \) 的递增性,有
\[
\int_{\alpha}^{\beta} f(x) \, \mathrm{d}x \leqslant (\beta - \alpha)f(\beta)
\]
于是
\begin{align*}
\int_{0}^{1} f(x) \, \mathrm{d}x &= \int_{0}^{\alpha} f(x) \, \mathrm{d}x + \int_{\alpha}^{\beta} f(x) \, \mathrm{d}x + \int_{\beta}^{1} f(x) \, \mathrm{d}x \\
&\geqslant \int_{\alpha}^{\beta} f(x) \, \mathrm{d}x + \int_{\beta}^{1} f(x) \, \mathrm{d}x \geqslant \int_{\alpha}^{\beta} f(x) \, \mathrm{d}x + \int_{\beta}^{1} f(\beta) \, \mathrm{d}x \\
&= \int_{\alpha}^{\beta} f(x) \, \mathrm{d}x + (1 - \beta)f(\beta) \geqslant \int_{\alpha}^{\beta} f(x) \, \mathrm{d}x + \frac{1 - \beta}{\beta - \alpha} \int_{\alpha}^{\beta} f(x) \, \mathrm{d}x \\
&= \frac{1 - \alpha}{\beta - \alpha} \int_{\alpha}^{\beta} f(x) \, \mathrm{d}x.
\end{align*}
取正整数 \( n \) 使得 \( \alpha + \frac{1}{n} < \beta \)。构造函数
\[
f_n(x) = \begin{cases} 
0, & 0 \leqslant x \leqslant \alpha, \\
n(x - \alpha), & \alpha < x \leqslant \alpha + \frac{1}{n}, \\
1, & \alpha + \frac{1}{n} < x \leqslant 1 .
\end{cases}
\]
显然这是一个连续函数,且
\[
\int_{0}^{1} f_n(x) \, \mathrm{d}x = 1 - \alpha - \frac{1}{2n}, \quad \int_{\alpha}^{\beta} f_n(x) \, \mathrm{d}x = \beta - \alpha - \frac{1}{2n}.
\]
因而
\[
\lim_{n \to +\infty} \frac{\int_{0}^{1} f_n(x) \, \mathrm{d}x}{\int_{\alpha}^{\beta} f_n(x) \, \mathrm{d}x} = \frac{1 - \alpha}{\beta - \alpha}
\]
故题中 \( \frac{1 - \alpha}{\beta - \alpha} \) 不能换成更大的数。
\end{proof}

\begin{example}
设函数 \( f \) 在 \([0,1]\) 上连续的二阶导函数,\( f(0) = f(1) = 0 \),\( f'(1) = \frac{a}{2} \)。求证:
\[
\int_{0}^{1} x(f''(x))^2 \, \mathrm{d}x \geqslant \frac{a^2}{2}
\]
并求上式成为等式的 \( f \)。
\end{example}
\begin{remark}
当 \( f \) 在端点的值为零,\( f' \) 在端点的值确定时,可以考虑 \( f'' \) 与线性函数的乘积的积分。
\end{remark}
\begin{proof}
根据分部积分,Newton-Leibniz 公式和题设条件,有
\begin{align*}
0 &\leqslant \int_{0}^{1} x(f''(x) - a)^2 \, \mathrm{d}x = \int_{0}^{1} x(f''(x))^2 \, \mathrm{d}x - 2a \int_{0}^{1} x f''(x) \, \mathrm{d}x + a^2 \int_{0}^{1} x \, \mathrm{d}x \\
&= \int_{0}^{1} x(f''(x))^2 \, \mathrm{d}x - 2a \left( x f'(x)\big|_{0}^{1} - \int_{0}^{1} f'(x) \, \mathrm{d}x \right) + \frac{a^2}{2} \\
&= \int_{0}^{1} x(f''(x))^2 \, \mathrm{d}x - 2a \left( f'(1) - f(1) + f(0) \right) + \frac{a^2}{2} \\
&= \int_{0}^{1} x(f''(x))^2 \, \mathrm{d}x - \frac{a^2}{2}
\end{align*}
所以
\[
\int_{0}^{1} x(f''(x))^2 \, \mathrm{d}x \geqslant \frac{a^2}{2}
\]
等式成立时,有
\[
f''(x) = a
\]
即 \( f(x) = \frac{1}{2} a x^2 + b x + c \)。因为 \( f(0) = f(1) = 0 \),\( f'(1) = \frac{a}{2} \),所以 \( c = 0 \),\( b = -\frac{a}{2} \)。因此
\[
f(x) = \frac{1}{2} a x(x - 1).
\]
\end{proof}

\begin{example}
设 \( n \) 是正整数,且 \( m > 2 \)。求证:
\begin{align*}
\int_{0}^{\pi/2} t \left| \frac{\sin nt}{\sin t} \right|^m \, \mathrm{d}t \leqslant \left( \frac{m \cdot n^{m - 2}}{8(m - 2)} - \frac{1}{4(m - 2)} \right) \pi^2.
\end{align*}
\end{example}
\begin{remark}
当利用积分的可加性把区间 \( [a, b] \) 上的积分分为区间 \( [a, c] \) 和区间 \( [c, b] \) 上的积分之和时,为了得到较好的估计,可以根据情况选择适当的 \( c \)。
\end{remark}
\begin{proof}
用数学归纳法容易证明 \( |\sin nt| \leqslant n \sin t \),\( t \in \left[ 0, \frac{\pi}{2} \right] \)。另外又有
\[
|\sin nt| \leqslant 1, \quad \sin t \geqslant \frac{2t}{\pi}, \, t \in \left[ 0, \frac{\pi}{2} \right].
\]
设 \( a \in \left( 0, \frac{\pi}{2} \right) \)。则有
\begin{align*}
\int_{0}^{\pi/2} t \left| \frac{\sin nt}{\sin t} \right|^m \, \mathrm{d}t &= \int_{0}^{a} t \left( \frac{\sin nt}{\sin t} \right)^m \, \mathrm{d}t + \int_{a}^{\pi/2} t \left( \frac{\sin nt}{\sin t} \right)^m \, \mathrm{d}t \\
&\leqslant \int_{0}^{a} tn^m \, \mathrm{d}t + \int_{a}^{\pi/2} t \left( \frac{1}{2t/\pi} \right)^m \, \mathrm{d}t \\
&= \frac{1}{2} n^m a^2 + \frac{1}{m - 2} \left( \frac{\pi}{2} \right)^m \left( \frac{1}{a^{m - 2}} - \frac{1}{(\pi/2)^{m - 2}} \right).
\end{align*}
易知函数 \( g(a) = \frac{1}{2} n^m a^2 + \frac{1}{m - 2} \left( \frac{\pi}{2} \right)^m \frac{1}{a^{m - 2}} \) 当 \( a = \frac{\pi}{2n} \) 时取最小值。于是将上面的 \( a \) 换成 \( \frac{\pi}{2n} \) 可得
\[
\int_{0}^{\pi/2} t \left| \frac{\sin nt}{\sin t} \right|^m \, \mathrm{d}t \leqslant \left( \frac{m \cdot n^{m - 2}}{8(m - 2)} - \frac{1}{4(m - 2)} \right) \pi^2.
\]
\end{proof}

\begin{example}
设 \( n \geqslant 1 \) 是自然数。求证:
\begin{align*}
\frac{1}{\pi} \int_{0}^{\pi/2} \frac{|\sin(2n + 1)t|}{\sin t} \, \mathrm{d}t < \frac{2n^2 + 1}{2n^2 + n} + \frac{1}{2} \ln n.
\end{align*}
\end{example}
\begin{proof}
注意到
\[
\int_{0}^{\pi/2} \frac{|\sin(2n + 1)t|}{\sin t} \, \mathrm{d}t = \int_{0}^{\pi/2n} \frac{|\sin(2n + 1)t|}{\sin t} \, \mathrm{d}t + \int_{\pi/2n}^{\pi/2} \frac{|\sin(2n + 1)t|}{\sin t} \, \mathrm{d}t.
\]
因为当 \( x \in \left( 0, \frac{\pi}{2} \right) \) 时,\( \sin x > \frac{2x}{\pi} \),所以
\[
\int_{\pi/2n}^{\pi/2} \frac{|\sin(2n + 1)t|}{\sin t} \, \mathrm{d}t < \int_{\pi/2n}^{\pi/2} \frac{1}{2t/\pi} \, \mathrm{d}t = \frac{\pi}{2} \ln n.
\]
另一方面,
\[
\int_{0}^{\pi/2n} \frac{|\sin(2n + 1)t|}{\sin t} \, \mathrm{d}t = \int_{0}^{\pi/(2n + 1)} \frac{\sin(2n + 1)t}{\sin t} \, \mathrm{d}t - \int_{\pi/(2n + 1)}^{\pi/2n} \frac{\sin(2n + 1)t}{\sin t} \, \mathrm{d}t.
\]
用数学归纳法容易证明当 \( t \in \left[ 0, \frac{\pi}{2} \right] \) 时,有 \( |\sin nt| \leqslant n \sin t \)。因此
\begin{align*}
\int_0^{\pi/(2n+1)}{\frac{\sin\mathrm{(}2n+1)t}{\sin t}\,\mathrm{d}t}&=\int_0^{\pi/(2n+1)}{\left( \frac{\sin 2nt\cos t}{\sin t}+\cos 2nt \right) \,\mathrm{d}t=\int_0^{\pi/(2n+1)}{\frac{\sin 2nt\cos t}{\sin t}\,\mathrm{d}t}}+\frac{1}{2n}\sin \frac{2n\pi}{2n+1}
\\
&=\int_0^{\pi/(2n+1)}{2n\cos t\,\mathrm{d}t}+\frac{1}{2n}\sin \frac{2n\pi}{2n+1}<2n\sin \frac{\pi}{2n+1}+\frac{1}{2n}\sin \frac{2n\pi}{2n+1}
\\
&=\left( 2n+\frac{1}{2n} \right) \sin \frac{\pi}{2n+1},
\end{align*}
\begin{align*}
-\int_{\pi/(2n+1)}^{\pi/2n}{\frac{\sin\mathrm{(}2n+1)t}{\sin t}\,\mathrm{d}t}=-\int_{\pi/(2n+1)}^{\pi/2n}{\left( \frac{\sin 2nt\cos t}{\sin t}+\cos 2nt \right) \,\mathrm{d}t<}-\int_{\pi/(2n+1)}^{\pi/2n}{\cos 2nt\,\mathrm{d}t=\frac{1}{2n}\sin \frac{\pi}{2n+1}.}
\end{align*}
因此
\[
\int_{0}^{\pi/2n} \frac{|\sin(2n + 1)t|}{\sin t} \, \mathrm{d}t < \left( 2n + \frac{1}{n} \right) \sin \frac{\pi}{2n + 1} < \left( 2n + \frac{1}{n} \right) \frac{\pi}{2n + 1} = \frac{2n^2 + 1}{2n^2 + n} \pi.
\]
于是
\[
\int_{0}^{\pi/2} \frac{|\sin(2n + 1)t|}{\sin t} \, \mathrm{d}t < \frac{2n^2 + 1}{2n^2 + n} \pi + \frac{\pi}{2} \ln n.
\]
两边同时除以 \( \pi \) 得:
\[
\frac{1}{\pi} \int_{0}^{\pi/2} \frac{|\sin(2n + 1)t|}{\sin t} \, \mathrm{d}t < \frac{2n^2 + 1}{2n^2 + n} + \frac{1}{2} \ln n.
\]
\end{proof}

\begin{example}
设 \( f \not\equiv 0 \),在 \( [a,b] \) 上可微,\( f(a) = f(b) = 0 \)。求证:至少存在一点 \( c \in [a,b] \) 使
\begin{align}
|f'(c)| > \frac{4}{(b - a)^2} \int_{a}^{b} |f(x)| \, \mathrm{d}x. \label{5.1.10}
\end{align}
\end{example}
\begin{proof}
记上式右端为 \( M \)。假设对一切 \( c \in [a,b] \) 有 \( |f'(c)| \leqslant M \),下面推出矛盾。首先根据微分中值定理,对于 \( x \in \left[ a, \frac{a + b}{2} \right] \) 存在 \( \xi \in (a,x) \),使
\[
f(x) = f(x) - f(a) = f'(\xi)(x - a),
\]
由假设,有
\begin{align}
|f(x)| \leqslant M(x - a), \quad x \in \left[ a, \frac{a + b}{2} \right], \label{5.1.11}
\end{align}
因而
\begin{align}
\int_{a}^{\frac{a + b}{2}} |f(x)| \, \mathrm{d}x \leqslant \frac{1}{2} \left( \frac{b - a}{2} \right)^2 M. \label{5.1.12}
\end{align}
再根据微分中值定理,对于 \( x \in \left[ \frac{a + b}{2}, b \right] \),存在 \( \eta \in (x,b) \),使得
\[
f(x) = f(x) - f(b) = f'(\eta)(x - b),
\]
由假设,有
\begin{align}
|f(x)| \leqslant M(b - x), \quad x \in \left[ \frac{a + b}{2}, b \right], \label{5.1.13}
\end{align}
因而
\begin{align}
\int_{\frac{a + b}{2}}^{b} |f(x)| \, \mathrm{d}x \leqslant \frac{1}{2} \left( \frac{b - a}{2} \right)^2 M. \label{5.1.14}
\end{align}
将式 \(\eqref{5.1.12}\) 与式 \(\eqref{5.1.14}\) 相加可得
\[
\int_{a}^{b} |f(x)| \, \mathrm{d}x \leqslant \left( \frac{b - a}{2} \right)^2 M = \int_{a}^{b} |f(x)| \, \mathrm{d}x.
\]
这说明式 \(\eqref{5.1.12}\) 与式 \(\eqref{5.1.14}\) 必须是等式,因而式 \(\eqref{5.1.11}\) 与式 \(\eqref{5.1.13}\) 必须成为等式。于是
\[
f^2(x) = 
\begin{cases} 
M^2(x - a)^2, & x \in \left[ a, \frac{a + b}{2} \right], \\
M^2(b - x)^2, & x \in \left( \frac{a + b}{2}, b \right], 
\end{cases}
\]
此分段函数在 \( x = \frac{a + b}{2} \) 不可导,这与 \( f \) 在 \([a,b]\) 可导矛盾!
\end{proof}

\begin{example}
设 \( f \) 是区间 \([0,1]\) 上的下凸函数。求证:对一切 \( t \in [0,1] \),有
\[
t(1 - t)f(t) \leqslant (1 - t)^2 \int_{0}^{t} f(x) \, \mathrm{d}x + t^2 \int_{t}^{1} f(x) \, \mathrm{d}x.
\]
\end{example}
\begin{remark}
从本题结论知:当 \( f \) 是区间 \([0,1]\) 上的下凸函数时,有
\[
\int_{0}^{1} t(1 - t)f(t) \, \mathrm{d}t \leqslant \frac{1}{3}\int_0^1{\,\left[ t^3+\left( 1-t \right) \right] ^3f\left( x \right) \mathrm{d}t}.
\]
因为
\begin{align*}
\int_0^1{t\left( 1-t \right) f\left( t \right) \mathrm{d}t}&\leqslant \int_0^1{\,\left( 1-t \right) ^2\left( \int_0^t{f\left( x \right) \,\mathrm{d}x} \right) \mathrm{d}t}+\int_0^1{\,t^2\left( \int_t^1{f\left( x \right) \,\mathrm{d}x} \right) \mathrm{d}t}
\\
&=-\frac{1}{3}\int_0^1{\,\left( \int_0^t{f\left( x \right) \,\mathrm{d}x} \right) \mathrm{d}\left( 1-t \right) ^3}+\frac{1}{3}\int_0^1{\,\left( \int_t^1{f\left( x \right) \,\mathrm{d}x} \right) \mathrm{d}t^3}
\\
&\xlongequal{\text{分部积分}}\frac{1}{3}\int_0^1{\,\left( 1-t \right) ^3f\left( x \right) \mathrm{d}t}+\frac{1}{3}\int_0^1{\,t^3f\left( t \right) \mathrm{d}t}
\\
&=\frac{1}{3}\int_0^1{\,\left[ t^3+\left( 1-t \right) \right] ^3f\left( x \right) \mathrm{d}t}.
\end{align*}
\end{remark}
\begin{note}
构造思路:待定 \( a = a(t,x) \),\( b = b(t,x) \),使得 \( t = ta + (1 - t)b \)。由 \( f \) 是下凸函数可知
\begin{align*}
f(t) \leqslant tf(a) + (1 - t)f(b), \forall t \in (0,1).
\end{align*}
并且上式两边对 \( x \) 在 \( [0,1] \) 上积分,得
\begin{align*}
&\quad \quad t\int_0^1{f\left( a \right) \mathrm{d}x}+\left( 1-t \right) \int_0^1{f\left( b \right) \mathrm{d}x}=\frac{t}{1-t}\int_t^1{f\left( x \right) \,\mathrm{d}x}+\frac{1-t}{t}\int_0^t{f\left( x \right) \,\mathrm{d}x}
\\
&\Longrightarrow \int_0^1{f\left( b \right) \mathrm{d}x}=\frac{1}{t}\int_0^t{f\left( x \right) \,\mathrm{d}x}=\int_0^1{f\left( tx \right) \,\mathrm{d}x}
\\
&\Longrightarrow b=tx,t=ta+\left( 1-t \right) b
\\
&\Longrightarrow a=t-tx+t^2x=t\left( 1-x+tx \right) .
\end{align*}
\end{note}
\begin{proof}
对于 \( t = 0 \) 和 \( t = 1 \) 所证不等式是显然的。设 \( t \in (0,1) \),由\refthe{theorem:开区间下凸函数左右导数处处存在}可知,下凸函数在 \( t \) 点是连续的,所以 \( f \) 在 \([0,1]\) 上可积。对于 \( x \in [0,1] \),有 \( t = (1 - t)(tx) + t(1 - x + tx) \)。因此根据下凸函数的定义,得
\[
f(t) \leqslant (1 - t)f(tx) + tf(1 - x + tx).
\]
上式对变量 \( x \) 在 \([0,1]\) 上积分,得
\begin{align*}
f(t) &\leqslant (1 - t) \int_{0}^{1} f(tx) \, \mathrm{d}x + t \int_{0}^{1} f(1 - x + tx) \, \mathrm{d}x
\\
&= \frac{1 - t}{t} \int_{0}^{t} f(x) \, \mathrm{d}x + \frac{t}{1 - t} \int_{t}^{1} f(x) \, \mathrm{d}x.
\end{align*}
\end{proof}

\begin{proposition}
设 \( f \) 在区间 \([0,a)\) 上有二阶连续导数,满足 \( f(0) = f'(0) = 0 \) 且 \( f''(x) > 0 \, (0 < x < a) \)。求证:对任意 \( x \in (0,a) \),有
\begin{align}
\int_{0}^{x} \sqrt{1 + (f'(t))^2} \, \mathrm{d}t < x + \frac{f(x)f'(x)}{\sqrt{1 + (f'(x))^2} + 1}. \label{5.1.15}
\end{align}
\end{proposition}
\begin{remark}
式 \(\eqref{5.1.15}\) 左端是弧长计算公式,不等式 \(\eqref{5.1.15}\) 的几何意义是:光滑下凸曲线段的起点 \( A \) 和终点 \( B \) 处的切线在曲线凸出的一侧相交于 \( C \) 点,则直线段 \( AC \) 与 \( BC \) 的长度之和大于这条曲线段的长度。
\end{remark}
\begin{proof}
将式 \(\eqref{5.1.15}\) 右端第一项 \( x \) 移到左端,有
\[
\int_{0}^{x} \left( \sqrt{1 + (f'(t))^2} - 1 \right) \, \mathrm{d}t = \int_{0}^{x} \frac{f'(t)}{\sqrt{1 + (f'(t))^2} + 1} \cdot f'(t) \, \mathrm{d}t.
\]
因为 \( f'(t) \) 和 \( \frac{t}{\sqrt{1 + t^2} + 1} \) 都是单调递增函数,所以 \( \frac{f'(t)}{\sqrt{1 + (f'(t))^2} + 1} \) 是单调递增函数。因此
\[
\int_{0}^{x} \left( \sqrt{1 + (f'(t))^2} - 1 \right) \, \mathrm{d}t < \frac{f'(x)}{\sqrt{1 + (f'(x))^2} + 1} \cdot \int_{0}^{x} f'(t) \, \mathrm{d}t = \frac{f(x)f'(x)}{\sqrt{1 + (f'(x))^2} + 1}.
\]
\end{proof}

\begin{example}
\( f \) 是区间 \([0,1]\) 上的正连续函数,\( k \geqslant 1 \)。求证:
\begin{align}
\int_{0}^{1} \frac{1}{1 + f(x)} \, \mathrm{d}x \int_{0}^{1} f(x) \, \mathrm{d}x \leqslant \int_{0}^{1} \frac{f^{k + 1}(x)}{1 + f(x)} \, \mathrm{d}x \int_{0}^{1} \frac{1}{f^k(x)} \, \mathrm{d}x, \label{5.1.19}
\end{align}
并讨论等号成立的条件。
\end{example}
\begin{proof}
当 \( k \geqslant 1 \) 时,函数 \( \frac{t^k}{1 + t} \) 和 \( t^{k + 1} \) 都是单调递增的。因此对于任意 \( x, y \in [0,1] \),有
\begin{align}
\frac{1}{f^k(x)f^k(y)} \left( \frac{f^k(x)}{1 + f(x)} - \frac{f^k(y)}{1 + f(y)} \right) \left( f^{k + 1}(x) - f^{k + 1}(y) \right) \geqslant 0, \label{5.1.20}
\end{align}
即
\[
\frac{f(x)}{1 + f(y)} + \frac{f(y)}{1 + f(x)} \leqslant \frac{f^{k + 1}(x)}{1 + f(x)} \cdot \frac{1}{f^k(y)} + \frac{f^{k + 1}(y)}{1 + f(y)} \cdot \frac{1}{f^k(x)}.
\]
在上式两端分别关于变量 \( x, y \) 在区间 \([0,1]\) 上积分,即得所证。

要使式 \(\eqref{5.1.19}\) 成为等式,必须式 \(\eqref{5.1.20}\) 成为等式。因此对任意 \( x, y \in [0,1] \),有 \( f(x) = f(y) \),即 \( f \) 在 \([0,1]\) 上为常数。
\end{proof}

\begin{example}
设 \( b \geqslant a + 2 \)。函数 \( f \) 在 \([a,b]\) 上为正连续函数,且
\[
\int_{a}^{b} \frac{1}{1 + f(x)} \, \mathrm{d}x = 1.
\]
求证:
\begin{align}
\int_{a}^{b} \frac{f(x)}{b - a - 1 + f^2(x)} \, \mathrm{d}x \leqslant 1. \label{5.1.21}
\end{align}
并求式 \(\eqref{5.1.21}\) 成为等式的条件。
\end{example}
\begin{proof}
令 \( g(x) = \frac{b - a}{1 + f(x)} \),则 \( g \) 在 \([a,b]\) 上连续且 \( \int_{a}^{b} g(x) \, \mathrm{d}x = b - a \)。从 \( g \) 的定义可得 \( f(x) = \frac{b - a - g(x)}{g(x)} \)。因此
\begin{align*}
\frac{f(x)}{b-a-1+f^2(x)}&=\frac{\frac{b-a-g(x)}{g(x)}}{b-a-1+\left( \frac{b-a-g(x)}{g(x)} \right) ^2}=\frac{1}{b-a}\cdot \frac{g(x)(b-a-g(x))}{g^2(x)-2g(x)+b-a}
\\
&=\frac{1}{b-a}\left[ -1+\frac{(b-a-2)g(x)+b-a}{(g(x)-1)^2+b-a-1} \right] \leqslant \frac{1}{b-a}\left[ -1+\frac{(b-a-2)g(x)+b-a}{b-a-1} \right] 
\\
&=\frac{1}{b-a}\cdot \frac{(b-a-2)g(x)+1}{b-a-1},
\end{align*}
故
\begin{align*}
\int_a^b{\frac{f(x)}{b-a-1+f^2(x)}\,\mathrm{d}x}&\leqslant \int_a^b{\frac{1}{b-a}}\cdot \frac{(b-a-2)g(x)+1}{b-a-1}\,\mathrm{d}x
\\
&=\frac{1}{b-a}\cdot \frac{(b-a-2)(b-a)+b-a}{b-a-1}=1.
\end{align*}
等号成立当且仅当 \( g(x) = 1 \),即 \( f(x) = b - a - 1 \) 时成立。
\end{proof}

\begin{example}
设 \( f \) 是 \( (-\infty,+\infty) \) 上连续函数,且在 \( (-\infty,a] \cup [b,+\infty) \) 上等于零。又设
\[
\varphi(x) = \frac{1}{2h} \int_{x-h}^{x+h} f(t) \, \mathrm{d}t \quad (h > 0).
\]
求证:
\[
\int_{a}^{b} |\varphi(x)| \, \mathrm{d}x \leq \int_{a}^{b} |f(x)| \, \mathrm{d}x.
\]
\end{example}
\begin{proof}
作变换 \( u = t - x \),得
\[
\int_{x-h}^{x+h} |f(t)| \, \mathrm{d}t = \int_{-h}^{h} |f(u + x)| \, \mathrm{d}u.
\]
因此
\[
\int_{a}^{b} \int_{-h}^{h} |f(u + x)| \, \mathrm{d}u \, \mathrm{d}x = \int_{-h}^{h} \int_{a}^{b} |f(u + x)| \, \mathrm{d}x \, \mathrm{d}u.
\]
作变换 \( v = u + x \),得
\begin{align*}
\int_a^b{|f(u}+x)|\,\mathrm{d}x=\int_{a+u}^{b+u}{|f(v)|\,\mathrm{d}v}=\begin{cases}
\int_{a+u}^b{|f(v)|\,\mathrm{d}v,}&		u\ge 0,\\
\int_a^{b+u}{|f(v)|\,\mathrm{d}v,}&		u<0\\
\end{cases}\leqslant \int_a^b{|f(v)|\,\mathrm{d}v}.
\end{align*}
由此可知
\begin{align*}
\int_a^b{|\varphi (x)|\,\mathrm{d}x}&=\int_a^b{\left| \frac{1}{2h}\int_{x-h}^{x+h}{f(t)\,\mathrm{d}t} \right|\,\mathrm{d}x}\leqslant \frac{1}{2h}\int_a^b{\int_{x-h}^{x+h}{|f(t)|\,\mathrm{d}t\,\mathrm{d}x}}
\\
&=\frac{1}{2h}\int_a^b{\int_{-h}^h{|f(u}}+x)|\,\mathrm{d}u\,\mathrm{d}x=\frac{1}{2h}\int_{-h}^h{\int_a^b{|f(u}}+x)|\,\mathrm{d}x\,\mathrm{d}u
\\
&\leqslant \frac{1}{2h}\int_{-h}^h{\int_a^b{|f(v)|\,\mathrm{d}v\,\mathrm{d}u}}=\int_a^b{|f(v)|\,\mathrm{d}v}.
\end{align*}
\end{proof}

\begin{example}
设 \( f \) 在区间 \([1,+\infty)\) 上连续并满足
\begin{align}
x \int_{1}^{x} f(t) \, \mathrm{d}t = (x + 1) \int_{1}^{x} t f(t) \, \mathrm{d}t. \label{5.1.23}
\end{align}
求 \( f \)。
\end{example}
\begin{solution}
假设 \( f \) 是满足条件的连续函数,则对式 \eqref{5.1.23}  两边求导得
\begin{align}
\int_{1}^{x} f(t) \, \mathrm{d}t = \int_{1}^{x} t f(t) \, \mathrm{d}t + x^2 f(x). \label{5.1.24}
\end{align}
由此可知,\( f(1) = 0 \),且当 \( x \geq 1 \) 时,\( f \) 可导。对式 \eqref{5.1.24} 两边求导得
\[
f(x) = x f(x) + 2x f(x) + x^2 f'(x),
\]
即
\begin{align}
f'(x) = \frac{1 - 3x}{x^2} f(x), \quad x \geq 1. \label{5.1.25}
\end{align}
所以
\begin{align}
|f'(x)| \leq 2 |f(x)|. \label{5.1.26}
\end{align}
令 \( g(x) = \mathrm{e}^{-4x} f^2(x) \),则有
\[
g'(x) = 2 \mathrm{e}^{-4x} \left( f(x) f'(x) - 2 f^2(x) \right).
\]
结合式\eqref{5.1.26} 可知 \( g' \leq 0 \),这说明 \( g \) 单调递减。因为 \( g(1) = 0 \),所以 \( g \leq 0 \)。但从 \( g \) 的定义知 \( g \geq 0 \)。于是 \( g = 0 \),从而 \( f = 0 \)。

实际上,由\eqref{5.1.25}可解得$f\left( x \right) =Ce^{\int_1^x{\frac{1-3t}{t^2}\mathrm{d}t}}=Ce^{1-\frac{1}{x}-3\ln x},$再将$f(1)=0$代入得$C=0.$故$f\equiv 0.$

总之,原方程 \eqref{5.1.23} 的解只有 \( f \equiv 0 \)。
\end{solution}

\begin{example}
设 \( f \) 在任意有限区间上可积, 且对任意 \( x \) 及任意 \( a \neq 0 \) 满足
\[
\frac{1}{2a} \int_{x - a}^{x + a} f(t) \, \mathrm{d}t = f(x).
\]
试求函数 \( f \).
\end{example}
\begin{solution}
易知线性函数满足上面的式子. 下面证明满足上式的函数必是线性函数. 由条件知, 对任意 \( x \) 和 \( a \), 有
\[
\int_{x - a}^{x + a} f(t) \, \mathrm{d}t = 2a f(x).
\]
因此
\[
2a f(x + y) = \int_{x + y - a}^{x + y + a} f(t) \, \mathrm{d}t = \int_{y + x - a}^{y + a - x} f(t) \, \mathrm{d}t + \int_{y + a - x}^{x + y + a} f(t) \, \mathrm{d}t
= 2(a - x) f(y) + 2x f(y + a).
\]
取 \( a = 1, y = 0 \) 就得
\[
f(x) = (f(1) - f(0))x + f(0),
\]
即 \( f \) 是线性函数.
\end{solution}

\begin{example}
设 \( f \) 是 \( \mathbb{R} \) 上有下界的连续函数. 若存在常数 \( a \in (0,1] \) 使得  
\[
f(x) - a \int_{x}^{x + 1} f(t) \, \mathrm{d}t
\]  
为常数, 则 \( f \) 无穷可微且它的任意阶导函数都是非负的.  
\end{example}
\begin{proof}
不妨设 \( m = \inf_{x \in \mathbb{R}} f(x) = 0 \) (不然将 \( f \) 换为 \( f - m \) 之后再证明). 此时 \( f \geq 0 \). 记  
\begin{align}
A = f(x) - a \int_{x}^{x + 1} f(t) \, \mathrm{d}t, \label{5.1.27}
\end{align}
则 \( f \geq A \). 因此, \( A \leq 0 \). 由式 \(\eqref{5.1.27}\) 知 \( f \) 无穷可微, 且  
\begin{align}
f'(x) = a f(x + 1) - a f(x). \label{5.1.28}
\end{align}  
记 \( a_1 = a \), 则  
\[
f'(x) + a_1 f(x) \geq 0.
\]  

假设存在 \( a_n > 0 \) 使得  
\begin{align}
f'(x) + a_n f(x) \geq 0. \label{5.1.29}
\end{align}  
则 \( (e^{a_n x} f(x))' \geq 0 \). 这说明函数 \( e^{a_n x} f(x) \) 是递增的. 由式 \(\eqref{5.1.27}\) 可得  
\[
\begin{aligned}
f(x) &\leq a \int_{x}^{x + 1} f(t) \, \mathrm{d}t = a \int_{x}^{x + 1} e^{a_n t} f(t) e^{-a_n t} \, \mathrm{d}t \\
&\leq a e^{a_n (x + 1)} f(x + 1) \int_{x}^{x + 1} e^{-a_n t} \, \mathrm{d}t \\
&= \frac{e^{a_n} - 1}{a_n} a f(x + 1) \\
&= \frac{e^{a_n} - 1}{a_n} (f'(x) + a f(x)).
\end{aligned}
\]  
由此可得  
\begin{align}
f'(x) + a_{n + 1} f(x) \geq 0, \label{5.1.30}
\end{align}  
其中  
\[
a_{n + 1} = a - \frac{a_n}{e^{a_n} - 1}.
\]  
若 \( a_{n + 1} \leq 0 \), 则由 \(\eqref{5.1.30}\) 得 \( f' \geq 0 \). 若 \( a_{n + 1} > 0 \), 则接着可构造 \( a_{n + 2} \). 若 \( \{a_n\} \) 均为正的, 则 \( \{a_n\} \) 为递减正数列, 设其极限为 \( r \geq 0 \). 若 \( r > 0 \), 则从上式得 \( r = a - \frac{r}{e^r - 1} \), 即 \( a = \frac{r e^r}{e^r - 1} > 1 \). 这与条件不符, 因此必有 \( r = 0 \). 在式 \(\eqref{5.1.29}\) 中令 \( n \to +\infty \), 即得对一切 \( x \) 有 \( f'(x) \geq 0 \). 注意到  
\[
f^{(n)}(x) - a \int_{x}^{x + 1} f^{(n)}(t) \, \mathrm{d}t = 0, \quad n = 1, 2, \cdots,
\]  
因而将前面的 \( f \) 换为 \( f' \), 可以得到 \( f''(x) \geq 0 \), 依次可以证明 \( f^{(n)}(x) \geq 0 \).
\end{proof}

\begin{example}
求所有连续函数 \( f : \mathbb{R} \to \mathbb{R} \) 使得对任意 \( x \in \mathbb{R} \) 和任意正整数 \( n \), 有  
\[
n^2 \int_{x}^{x + \frac{1}{n}} f(t) \, \mathrm{d}t = n f(x) + \frac{1}{2}.
\]  
\end{example}
\begin{solution}
设 \( f \) 是要求的一个连续函数, 则 \( f \) 是可导的且  
\begin{align}
n \left[ f \left( x + \frac{1}{n} \right) - f(x) \right] = f'(x). \label{5.1.31}
\end{align}  
由此知 \( f \) 二阶可导, 且  
\begin{align}
n \left[ f' \left( x + \frac{1}{n} \right) - f'(x) \right] = f''(x). \label{5.1.32}
\end{align}  
将 \(\eqref{5.1.31}\) 中的 \( n \) 换成 \( 2n \), 得  
\begin{align}
2n \left[ f \left( x + \frac{1}{2n} \right) - f(x) \right] = f'(x). \label{5.1.33}
\end{align}  
将上式中的 \( x \) 换成 \( x + \frac{1}{2n} \) 得  
\begin{align}
2n \left[ f \left( x + \frac{1}{n} \right) - f \left( x + \frac{1}{2n} \right) \right] = f' \left( x + \frac{1}{2n} \right). \label{5.1.34}
\end{align}  
将式 \(\eqref{5.1.31}\) 两边乘以 \( 2 \) 再减去式 \(\eqref{5.1.33}\) 两边, 得  
\begin{align}
2n \left[ f \left( x + \frac{1}{n} \right) - f \left( x + \frac{1}{2n} \right) \right] = f'(x). \label{5.1.35}
\end{align}  
从式 \(\eqref{5.1.34}\) 和式 \(\eqref{5.1.35}\) 得  
\[
f'(x) = f' \left( x + \frac{1}{2n} \right), \quad \forall n \in \mathbb{Z}^+, \forall x \in \mathbb{R}.
\]  
由\eqref{5.1.32}式可知 \( f'' = 0 \). 因而存在常数 \( a, b \) 使得 \( f(x) = ax + b \). 代入题设条件可得 \( a = 1 \). 于是 \( f(x) = x + b \), 这里 \( b \) 是任意常数.
\end{solution}

\begin{example}
设 \( f \in C[-1,1] \) 且对任意整数 \( n \) 满足  
\begin{align}
\int_{0}^{1} f(\sin(nx)) \, \mathrm{d}x = 0. \label{5.1.36}
\end{align}  
求证: 对任意 \( x \in [-1,1] \) 有 \( f(x) = 0 \).  
\end{example}
\begin{proof}
在式 \(\eqref{5.1.36}\) 中取 \( n = 0 \), 可得 \( f(0) = 0 \). 对任意非零整数 \( n \), 将式 \(\eqref{5.1.36}\) 中的积分作变换 \( t = nx \) 可得  
\[
\int_{0}^{n} f(\sin t) \, \mathrm{d}t = 0.
\]  
令 
\[
F(x) = \int_{x}^{x + 1} f(\sin t) \, \mathrm{d}t,
\]  
则 \( F \) 可导, 且 \( F(n) = 0 \). 对整数 \( k \) 有  
\[
\begin{aligned}
F(x + 2k\pi) &= \int_{x + 2k\pi}^{x + 2k\pi + 1} f(\sin t) \, \mathrm{d}t = \int_{x}^{x + 1} f(\sin(t + 2k\pi)) \, \mathrm{d}t \\
&= \int_{x}^{x + 1} f(\sin t) \, \mathrm{d}t = F(x).
\end{aligned}
\]  
因而 \( F(n + 2k\pi) = F(n) = 0 \). 这说明 \( F \) 在集合 \( A = \{ n + 2k\pi \mid n, k \in \mathbb{Z} \} \) 上取值为 \( 0 \). 由于集合 \( A \) 在 \( \mathbb{R} \) 上是稠密的, 由 \( F \) 的连续性可知 \( F(x) = 0 \, (x \in \mathbb{R}) \). 于是  
\[
F'(x) = f(\sin(x + 1)) - f(\sin x) = 0.
\]  
这说明 \( f(\sin x) \) 是以 \( 1 \) 和 \( 2\pi \) 为周期的连续函数. 仍由集合 \( A \) 的稠密性可知 \( f(\sin x) \) 是常数. 因此 \( f \) 在 \( [-1,1] \) 上是常数. 故 \( f(x) = f(0) = 0 \).
\end{proof}

\begin{example}
设 \( f \) 是 \( [0, 2\pi] \) 上可导的凸函数, \( f' \) 有界, 试证  
\[
a_n = \frac{1}{\pi} \int_{0}^{2\pi} f(x) \cos nx \, \mathrm{d}x \geq 0.
\]  
\end{example}
\begin{proof}
因为 \( f \) 是可导的凸函数, 所以 \( f' \) 是单调递增的函数. 由 \( f' \) 的单调有界性, 知 \( f' \) 在 \( [0, 2\pi] \) 上可积. 根据分部积分公式, 得  
\[
\begin{aligned}
\pi a_n &= \int_{0}^{2\pi} f(x) \cos nx \, \mathrm{d}x = f(x) \frac{\sin nx}{n} \bigg|_{0}^{2\pi} - \frac{1}{n} \int_{0}^{2\pi} f'(x) \sin nx \, \mathrm{d}x \\
&= -\frac{1}{n} \int_{0}^{2\pi} f'(x) \sin nx \, \mathrm{d}x = -\frac{1}{n} \sum_{k=1}^{2n} \int_{(k - 1)\pi/n}^{k\pi/n} f'(x) \sin nx \, \mathrm{d}x \\
&= -\frac{1}{n} \sum_{k=1}^{2n} \int_{0}^{\frac{\pi}{n}} f' \left( x + \frac{(k - 1)\pi}{n} \right) \sin \big( (k - 1)\pi + x \big) \, \mathrm{d}x \\
&= -\frac{1}{n} \sum_{k=1}^{2n} \int_{0}^{\frac{\pi}{n}} f' \left( x + \frac{(k - 1)\pi}{n} \right) (-1)^{k - 1} \sin x \, \mathrm{d}x \\
&= -\frac{1}{n} \int_{0}^{\frac{\pi}{n}} \sum_{k=1}^{2n} (-1)^{k - 1} f' \left( x + \frac{(k - 1)\pi}{n} \right) \sin x \, \mathrm{d}x \\
&= -\frac{1}{n} \int_{0}^{\frac{\pi}{n}} \sum_{k=1}^{n} \left( f' \left( x + \frac{(2k - 2)\pi}{n} \right) - f' \left( x + \frac{(2k - 1)\pi}{n} \right) \right) \sin x \, \mathrm{d}x.
\end{aligned}
\]  
注意到 \( f' \) 是单调递增的, 即知 \( a_n \geq 0 \).
\end{proof}

\begin{example}
设 $f$ 在 $[0,1]$ 上连续可微, $f(0) = 0$. 求证:
\begin{align}
\int_0^1 \frac{f^2(x)}{x^2} \mathrm{d}x \leqslant 4 \int_0^1 \left(f'(x)\right)^2 \mathrm{d}x, \label{5.1.47}
\end{align}
且右边的系数 $4$ 是最佳的.
\end{example}
\begin{proof}
{\color{blue}证法一:}因为
$$f'(x) = x^{\frac{1}{2}} \left(x^{-\frac{1}{2}} f(x)\right)' + \frac{f(x)}{2x},$$
所以
$$\left( f' (x) \right) ^2=\left[ x^{\frac{1}{2}}\left( x^{-\frac{1}{2}}f(x) \right) ' \right] ^2+\left( x^{-\frac{1}{2}}f(x) \right) \left( x^{-\frac{1}{2}}f(x) \right) ' +\frac{f^2(x)}{4x^2}\geqslant \left( x^{-\frac{1}{2}}f(x) \right) \left( x^{-\frac{1}{2}}f(x) \right) ' +\frac{f^2(x)}{4x^2}.$$
因而
$$\int_0^1 \left(f'(x)\right)^2 \mathrm{d}x \geqslant \frac{1}{2} f^2(1) + \int_0^1 \frac{f^2(x)}{4x^2} \mathrm{d}x \geqslant \int_0^1 \frac{f^2(x)}{4x^2} \mathrm{d}x,$$
即所证不等式 \eqref{5.1.47} 成立.

若存在常数 $c \in (0,4)$ 使得
\begin{align}
\int_0^1 \frac{f^2(x)}{x^2} \mathrm{d}x \leqslant c \int_0^1 \left(f'(x)\right)^2 \mathrm{d}x \label{5.1.48}
\end{align}
对任意满足条件的 $f$ 成立, 则对 $\delta \in (0,1)$ 取
$$f(x) = \begin{cases} 
\sqrt{x}, & x \in [\delta, 1], \\
\frac{3}{2\sqrt{\delta}} x - \frac{1}{2\delta^{\frac{3}{2}}} x^2, & x \in [0, \delta).
\end{cases}$$
此时, 有
$$
\begin{aligned}
\int_0^1 \frac{f^2(x)}{x^2} \mathrm{d}x &= \int_0^\delta \left( \frac{3}{2\sqrt{\delta}} - \frac{1}{2\delta^{\frac{3}{2}}} x \right)^2 \mathrm{d}x + \int_\delta^1 \frac{1}{x} \mathrm{d}x \\
&= \int_0^\delta \left( \frac{9}{4\delta} - \frac{3x}{2\delta^2} + \frac{x^2}{4\delta^3} \right) \mathrm{d}x + \int_\delta^1 \frac{1}{x} \mathrm{d}x \\
&= \frac{19}{12} + \int_\delta^1 \frac{1}{x} \mathrm{d}x,
\end{aligned}
$$
$$
\begin{aligned}
\int_0^1 \left(f'(x)\right)^2 \mathrm{d}x &= \int_0^\delta \left( \frac{3}{2\sqrt{\delta}} - \frac{1}{\delta^{\frac{3}{2}}} x \right)^2 \mathrm{d}x + \int_\delta^1 \left( \frac{1}{2\sqrt{x}} \right)^2 \mathrm{d}x \\
&= \int_0^\delta \left( \frac{9}{4\delta} - \frac{3x}{\delta^2} + \frac{x^2}{\delta^3} \right) \mathrm{d}x + \frac{1}{4} \int_\delta^1 \frac{1}{x} \mathrm{d}x \\
&= \frac{13}{12} + \frac{1}{4} \int_\delta^1 \frac{1}{x} \mathrm{d}x.
\end{aligned}
$$
因此式\eqref{5.1.48}导致
$$\left(1 - \frac{c}{4}\right) \int_\delta^1 \frac{1}{x} \mathrm{d}x \leqslant \frac{13}{12} c - \frac{19}{12}.$$
此式当 $\delta$ 充分小时是不成立的. 这个矛盾说明 $4$ 是最佳的.

{\color{blue}证法二:} 利用 \hyperref[theorem:Minkowski(闵可夫斯基)不等式]{Minkowski不等式}, 可得
\begin{align*}
&\left( \int_0^1 \frac{|f(x)|^2}{x^2} \mathrm{d}x \right)^{\frac{1}{2}} = \left[ \int_0^1 \left( \int_0^1 f'(xt) \mathrm{d}t \right)^2 \mathrm{d}x \right]^{\frac{1}{2}}
\\
&\leqslant \int_0^1 \left( \int_0^1 |f'(xt)|^2 \mathrm{d}x \right)^{\frac{1}{2}} \mathrm{d}t \xlongequal{\text{换元}} \int_0^1 \left( \frac{\int_0^t |f'(x)|^2 \mathrm{d}x}{t} \right)^{\frac{1}{2}} \mathrm{d}t \\
&\leqslant \left( \int_0^1 |f'(x)|^2 \mathrm{d}x \right)^{\frac{1}{2}} \int_0^1 \frac{1}{\sqrt{t}} \mathrm{d}t = 2 \left( \int_0^1 |f'(x)|^2 \mathrm{d}x \right)^{\frac{1}{2}}.
\end{align*}
从上式推导可以看出, 对于不恒为零的 $f$, 严格不等号成立.

为说明相关常数不可改进, 任取 $\varepsilon \in (0,1)$, 考察不恒为零的 $\bar{f} \in C[\varepsilon, 1]$ 使得
$$\frac{\int_\varepsilon^1 \frac{|\bar{f}(x)|^2}{x^2} \mathrm{d}x}{\int_0^1 |\bar{f}'(x)|^2 \mathrm{d}x} = \lambda \equiv \sup_{\substack{f \in C[\varepsilon, 1] \\ f \not\equiv 0}} \frac{\int_\varepsilon^1 \frac{|f(x)|^2}{x^2} \mathrm{d}x}{\int_\varepsilon^1 |f'(x)|^2 \mathrm{d}x}.$$
这样的 $\bar{f}$ 的存在性一般需要用泛函分析. 这里只作形式推导. 任取 $\varphi \in C_c^1(\varepsilon, 1)$, 则
$$
\begin{aligned}
0 &= \left. \frac{\mathrm{d}}{\mathrm{d}s} \frac{\int_\varepsilon^1 \frac{|\bar{f}(x) + s\varphi(x)|^2}{x^2} \mathrm{d}x}{\int_\varepsilon^1 |\bar{f}'(x) + s\varphi'(x)|^2 \mathrm{d}x} \right|_{s=0} \\
&= \frac{2\lambda}{\int_\varepsilon^1 |\bar{f}'(x)|^2 \mathrm{d}x} \left( \frac{1}{\lambda} \int_\varepsilon^1 \frac{\bar{f}(x)\varphi(x)}{x^2} \mathrm{d}x - \int_\varepsilon^1 \bar{f}'(x)\varphi'(x) \mathrm{d}x \right) \\
&= \frac{2\lambda}{\int_\varepsilon^1 |\bar{f}'(x)|^2 \mathrm{d}x} \int_\varepsilon^1 \left( \bar{f}''(x) + \frac{1}{\lambda} \frac{\bar{f}(x)}{(x + \varepsilon)^2} \right) \varphi(x) \mathrm{d}x.
\end{aligned}
$$
因此, 尝试寻找 $\bar{f}$ 满足
$$\bar{f}''(x) + \frac{1}{\lambda} \frac{\bar{f}(x)}{x^2} = 0, \quad x \in [\varepsilon, 1].$$
若取 $\alpha \in (0,1)$, 则 $\bar{f}(x) = x^\alpha$ 满足上述方程. 对应的 $\lambda = \frac{1}{\alpha(1 - \alpha)}$, 为使得 $\lambda$ 最大, 取 $\alpha = \frac{1}{2}$.

以上讨论启发我们考虑
$$f_\varepsilon' = \begin{cases} 
\frac{1}{2\sqrt{\varepsilon}}, & x \in [0, \varepsilon], \\
\frac{1}{2\sqrt{x}}, & x \in (\varepsilon, 1].
\end{cases}$$
则
$$f_\varepsilon = \begin{cases} 
\frac{x}{2\sqrt{\varepsilon}}, & x \in [0, \varepsilon], \\
\sqrt{x} - \frac{\sqrt{\varepsilon}}{2}, & x \in (\varepsilon, 1].
\end{cases}$$
直接计算得到
$$\lim_{\varepsilon \to 0^+} \frac{\int_0^1 \frac{|f_\varepsilon(x)|^2}{x^2} \mathrm{d}x}{\int_0^1 |f_\varepsilon'(x)|^2 \mathrm{d}x} = \lim_{\varepsilon \to 0^+} \frac{2\sqrt{\varepsilon} - \frac{\varepsilon}{4} - \ln \varepsilon - \frac{3}{2}}{\frac{1}{4} - \frac{\ln \varepsilon}{4}} = 4.$$
这就表明不等式中的常数 $4$ 是最佳的.
\end{proof}

\begin{example}
设 \( f,g:[a,b]\to(0,+\infty) \) 都是连续函数,且 \( f\neq g \),\( \int_{a}^{b}f(x)\mathrm{d}x = \int_{a}^{b}g(x)\mathrm{d}x \)。定义数列
\[
I_{n}=\int_{a}^{b}\frac{f^{n + 1}(x)}{g^{n}(x)}\mathrm{d}x,\quad n = 0,1,\cdots
\]
求证:\(\{I_{n}\}\) 严格单调递增,且 \(\lim\limits_{n\to+\infty}I_{n}=+\infty\)。
\end{example}
\begin{proof}
由 Cauchy 不等式,得
\[
\int_{a}^{b}f(x)\mathrm{d}x=\int_{a}^{b}\frac{f(x)}{\sqrt{g(x)}}\cdot\sqrt{g(x)}\mathrm{d}x\leqslant\left(\int_{a}^{b}\frac{f^{2}(x)}{g(x)}\mathrm{d}x\right)^{\frac{1}{2}}\left(\int_{a}^{b}g(x)\mathrm{d}x\right)^{\frac{1}{2}}
\]
\[
=\left(\int_{a}^{b}\frac{f^{2}(x)}{g(x)}\mathrm{d}x\right)^{\frac{1}{2}}\left(\int_{a}^{b}f(x)\mathrm{d}x\right)^{\frac{1}{2}}
\]
故
\[
\int_{a}^{b}f(x)\mathrm{d}x\leqslant\int_{a}^{b}\frac{f^{2}(x)}{g(x)}\mathrm{d}x
\]
即 \(I_{0}\leqslant I_{1}\),等号成立当且仅当存在常数 \(c\) 使得 \(\frac{f(x)}{\sqrt{g(x)}}=c\sqrt{g(x)}\),即 \(f(x)=cg(x)\)。再由条件 \(\int_{a}^{b}f(x)\mathrm{d}x=\int_{a}^{b}g(x)\mathrm{d}x\) 可得 \(c = 1\)。这与 \(f\neq g\) 矛盾,故 \(I_{0}<I_{1}\)。

假设 \(I_{0}<I_{1}<\cdots<I_{n}\),根据 \hyperref[theorem:Hold(赫尔德)不等式(积分形式)]{Hölder 不等式},有
\begin{align*}
I_{n}&=\int_{a}^{b}\frac{f^{n + 1}(x)}{g^{\frac{(n + 1)^{2}}{n + 2}}(x)}\cdot g^{\frac{(n + 1)^{2}}{n + 2}-n}(x)\mathrm{d}x
\\
&\leqslant\left(\int_{a}^{b}\left(\frac{f^{n + 1}(x)}{g^{\frac{(n + 1)^{2}}{n + 2}}(x)}\right)^{\frac{n + 2}{n + 1}}\mathrm{d}x\right)^{\frac{n + 1}{n + 2}}\left(\int_{a}^{b}\left(g^{\frac{(n + 1)^{2}}{n + 2}-n}(x)\right)^{n + 2}\mathrm{d}x\right)^{\frac{1}{n + 2}}
\\
&=I_{n + 1}^{\frac{n + 1}{n + 2}}\cdot I_{0}^{\frac{1}{n + 2}}<I_{n + 1}^{\frac{n + 1}{n + 2}}\cdot I_{n}^{\frac{1}{n + 2}}
\end{align*}
因而 \(I_{n}<I_{n + 1}\),这样,根据数学归纳法原理,就证明了 \(\{I_{n}\}\) 严格单调递增。

若对任意 \(x\in(a,b)\),有 \(g(x)\geqslant f(x)\),则 \(g(x)-f(x)\geqslant0\)。根据条件 \(g(x)-f(x)\) 连续且满足 \(\int_{a}^{b}(g(x)-f(x))\mathrm{d}x = 0\),这可推出 \(f = g\),与条件矛盾!因此必存在 \(x_{0}\in(a,b)\) 使得 \(f(x_{0})>g(x_{0})\),因而存在正数 \(\delta<\min\{x_{0}-a,b - x_{0}\}\) 使得
\[
f(x)>g(x),\quad x\in[x_{0}-\delta,x_{0}+\delta]
\]
记 \(m=\min\limits_{x\in[x_{0}-\delta,x_{0}+\delta]}\frac{f(x)}{g(x)}\),则 \(m>1\),因此
\[
I_{n}\geqslant\int_{x_{0}-\delta}^{x_{0}+\delta}\left(\frac{f(x)}{g(x)}\right)^{n}f(x)\mathrm{d}x\geqslant m^{n}\int_{x_{0}-\delta}^{x_{0}+\delta}f(x)\mathrm{d}x
\]
令 \(n\to+\infty\) 得 \(\lim\limits_{n\to+\infty}I_{n}=+\infty\)。
\end{proof}

\begin{example}
设 \( f : \mathbb{R} \to \mathbb{R} \) 连续,定义 \( g(x) = f(x)\int_{0}^{x} f(t) \, \mathrm{d}t \) \((x \in \mathbb{R})\)。如果 \( g \) 是 \( \mathbb{R} \) 上的递减函数,求证:\( f \equiv 0 \)。
\end{example}
\begin{proof}
记 \( F(x) = \int_{0}^{x} f(t) \, \mathrm{d}t \),则 \( F \) 可导且 \( F' = f \)。由条件知
\[
(F^2(x))' = 2F'(x)F(x) = 2g(x)
\]
是单调递减函数。注意到 \( F(0) = 0 \)。有 \( (F^2(x))' \leqslant 0 \, (x > 0) \),\( (F^2(x))' \geqslant 0 \, (x < 0) \)。这说明 \( F^2(x) \) 当 \( x \geqslant 0 \) 时单调递减,当 \( x \leqslant 0 \) 时单调递增。因此 \( F^2 \) 的最大值为 \( F^2(0) = 0 \)。但显然 \( F^2 \geqslant 0 \)。故 \( F = 0 \),于是 \( f =F'= 0 \)。
\end{proof}

\begin{example}
设 \( f \in C[0,1] \)。如果对任意 \( x \in [0,1] \) 有
\[
\int_{0}^{x} f(t) \, \mathrm{d}t \geqslant f(x) \geqslant 0,
\]
求证:\( f(x) \equiv 0 \)。
\end{example}
\begin{proof}
记 \( F(x) = \int_{0}^{x} f(t) \, \mathrm{d}t \)。则 \( F \) 可导且 \( F' = f \)。由条件知 \( F(x) \geqslant F'(x) \)。因此 \( (\mathrm{e}^x F(x))' \leqslant 0 \),即 \( \mathrm{e}^x F(x) \) 单调递减。由 \( F(0) = 0 \),得 \( F(x) \leqslant 0 \)。但由条件 \( F(x) \geqslant f(x) \geqslant 0 \),故 \( F(x) = 0 \),于是 \( f(x) =F'(x)= 0 \)。
\end{proof}

\begin{proposition}
设 $g(x) \in C^2[0,1]$ 是递增的下凸函数, 则有
\begin{align}
\inf_{\substack{f \in C[0,1], \\ \int_{x^2}^x f(y) \mathrm{d}y \geqslant g(x) - g(x^2)}} \int_0^1 |f(x)|^2 \mathrm{d}x &= \int_0^1 |g'(x)|^2 \mathrm{d}x, \label{17.17eq1} \\
\inf_{\substack{f \in C[0,1], \\ \int_x^1 f(y) \mathrm{d}y \geqslant g(1) - g(x)}} \int_0^1 |f(x)|^2 \mathrm{d}x &= \int_0^1 |g'(x)|^2 \mathrm{d}x. \label{17.17eq2}
\end{align}
\end{proposition}
\begin{note}
这题的下确界$\inf$可以改成最小值$\min$,因为可取到等号.
\end{note}
\begin{proof}
我们令
$$F(x) = \int_x^1 f(y) \mathrm{d}y + g(x),$$
则 $F(x^2) \geqslant F(x), \forall x \in [0,1]$, 因此由 $F$ 连续性,就有
$$F(x) \geqslant F\left(x^{\frac{1}{2}}\right) \geqslant F\left(x^{\frac{1}{4}}\right) \geqslant \cdots \geqslant \lim_{n \to \infty} F\left(x^{\frac{1}{2^n}}\right) = F(1), \forall x \in (0,1],$$
于是我们有 $F(x) \geqslant F(1), \forall x \in [0,1]$, 现在就有
$$\int_x^1 f(y) \mathrm{d}y \geqslant g(1) - g(x), \forall x \in [0,1],$$
因此
\begin{scriptsize}
\begin{align*}
\left\{ \int_0^1{\left| f\left( x \right) \right|^2\mathrm{d}x}:f\in C\left[ 0,1 \right] ,\int_{x^2}^x{f\left( y \right) \mathrm{d}y}\geqslant g\left( x \right) -g\left( x^2 \right) \right\} \subset \left\{ \int_0^1{\left| f\left( x \right) \right|^2\mathrm{d}x}:f\in C\left[ 0,1 \right] ,\int_{x^2}^1{f\left( y \right) \mathrm{d}y}\geqslant g\left( 1 \right) -g\left( x^2 \right) \right\} .
\end{align*}
\end{scriptsize}
故
\begin{align*}
\inf_{\substack{f \in C[0,1], \\ \int_{x^2}^x f(y) \mathrm{d}y \geqslant g(x) - g(x^2)}} \int_0^1 |f(x)|^2 \mathrm{d}x \geqslant
\inf_{\substack{f \in C[0,1], \\ \int_x^1 f(y) \mathrm{d}y \geqslant g(1) - g(x)}} \int_0^1 |f(x)|^2 \mathrm{d}x .
\end{align*}
取 $f(y) = g'(y)$, 可以知道\eqref{17.17eq1}\eqref{17.17eq2}式等号都成立.从而
\begin{align*}
\int_0^1 |g'(x)|^2 \mathrm{d}x \geqslant \inf_{\substack{f \in C[0,1], \\ \int_{x^2}^x f(y) \mathrm{d}y \geqslant g(x) - g(x^2)}} \int_0^1 |f(x)|^2 \mathrm{d}x \geqslant
\inf_{\substack{f \in C[0,1], \\ \int_x^1 f(y) \mathrm{d}y \geqslant g(1) - g(x)}} \int_0^1 |f(x)|^2 \mathrm{d}x .
\end{align*}
故只须证明
\begin{align*}
&\quad \quad \quad \inf_{\substack{f \in C[0,1], \\ \int_x^1 f(y) \mathrm{d}y \geqslant g(1) - g(x)}} \int_0^1 |f(x)|^2 \mathrm{d}x \geqslant \int_0^1 |g'(x)|^2 \mathrm{d}x\\
&\iff \text{对}\forall f\in C\left[ 0,1 \right]\text{且}\int_x^1{f\left( y \right) \mathrm{d}y}\geqslant g\left( 1 \right) -g\left( x \right) ,\text{都有}\int_0^1{|f(x)|^2\mathrm{d}x}\geqslant \int_0^1{|g'(x)|^2\mathrm{d}x}.
\end{align*}
于是设$f\in C\left[ 0,1 \right]\text{且}\int_x^1{f\left( y \right) \mathrm{d}y}$,由Cauchy不等式得
\begin{align*}
\int_0^1 |g'(x)|^2 \mathrm{d}x \int_0^1 |f(x)|^2 \mathrm{d}x &\geqslant \left( \int_0^1 f(x) g'(x) \mathrm{d}x \right)^2 = \left( \int_0^1 g'(x) \mathrm{d} \int_x^1 f(y) \mathrm{d}y \right)^2 \\
&= \left( -g'(0) \int_0^1 f(y) \mathrm{d}y - \int_0^1 \left( \int_x^1 f(y) \mathrm{d}y \right) g''(x) \mathrm{d}x \right)^2 \\
&= \left( g'(0) \int_0^1 f(y) \mathrm{d}y + \int_0^1 \left( \int_x^1 f(y) \mathrm{d}y \right) g''(x) \mathrm{d}x \right)^2 \\
&\geqslant \left( g'(0) \int_0^1 f(y) \mathrm{d}y + \int_0^1 (g(1) - g(x)) g''(x) \mathrm{d}x \right)^2 \\
&= \left( g'(0) \int_0^1 f(y) \mathrm{d}y - g'(0) (g(1) - g(0)) + \int_0^1 |g'(x)|^2 \mathrm{d}x \right)^2 \\
&\geqslant \left( \int_0^1 |g'(x)|^2 \mathrm{d}x \right)^2
\end{align*}
因此 $\int_0^1 |f(x)|^2 \mathrm{d}x \geqslant \int_0^1 |g'(x)|^2 \mathrm{d}x$.这样我们就完成了证明.
\end{proof}

\begin{corollary}
\begin{align*}
\inf_{\substack{f(x) \in C[0,1], \\ \int_{x^2}^x f(y) \mathrm{d}y \geqslant \frac{x^2 - x^4}{2} - 0}} \int_0^1 |f(x)|^2 \mathrm{d}x &= \inf_{\substack{f(x) \in C[0,1], \\ \int_x^1 f(y) \mathrm{d}y \geqslant \frac{1 - x^2}{2} - 0}} \int_0^1 |f(x)|^2 \mathrm{d}x = \frac{1}{3} \\
\inf_{\substack{f(x) \in C[0,1], \\ \int_{x^2}^x f(y) \mathrm{d}y \geqslant \frac{x^3 - x^6}{2} - 0}} \int_0^1 |f(x)|^2 \mathrm{d}x &= \inf_{\substack{f(x) \in C[0,1], \\ \int_x^1 f(y) \mathrm{d}y \geqslant \frac{1 - x^3}{2} - 0}} \int_0^1 |f(x)|^2 \mathrm{d}x = \frac{9}{20}
\end{align*}
\end{corollary}

\begin{proposition}\label{proposition:积分不等式零点拟合}
设$f(x)$为$[0,1]$的上凸函数,且$f(0)=1$,则
$$\int_0^1 \left( \frac{2}{p+2} - x^p \right) f(x) \mathrm{d}x \geqslant \frac{p}{(p+1)(p+2)},\quad p>0.$$
\end{proposition}
\begin{proof}

\end{proof}



















































\end{document}
\documentclass[../../main.tex]{subfiles}
\graphicspath{{\subfix{../../image/}}} % 指定图片目录,后续可以直接使用图片文件名。

% 例如:
% \begin{figure}[H]
% \centering
% \includegraphics[scale=0.4]{图.png}
% \caption{}
% \label{figure:图}
% \end{figure}
% 注意:上述\label{}一定要放在\caption{}之后,否则引用图片序号会只会显示??.

\begin{document}

\section{其他}

\begin{example}
设$f:[0,1]\to(0,+\infty)$是连续递增函数,记$s = \frac{\int_{0}^{1}xf(x)\mathrm{d}x}{\int_{0}^{1}f(x)\mathrm{d}x}$。证明
\begin{align*}
\int_{0}^{s}f(x)\mathrm{d}x&\leqslant\int_{s}^{1}f(x)\mathrm{d}x\leqslant\frac{s}{1 - s}\int_{0}^{s}f(x)\mathrm{d}x.
\end{align*}
\end{example}
\begin{note}
看到函数复合积分就联想Jensen不等式(积分形式),不过Jensen不等式(积分形式)考试中不能直接使用.因此仍需要利用函数的凸性相关不等式进行证明.
\end{note}
\begin{proof}
令$F(t) = \int_0^t f(x) \, \mathrm{d}x$,则$F'(t) = f(t)$连续递增,故$F$是下凸的。显然$s \in [0,1]$,于是
\begin{align*}
F(x) \geq F(s) + F'(s)(x - s) = F(s) + f(s)(x - s), \quad \forall x \in [0,1].
\end{align*}
从而
\begin{align*}
\int_0^1 F(x) f(x) \, \mathrm{d}x &\geq \int_0^1 \left[ F(s) f(x) + f(s) f(x)(x - s) \right] \, \mathrm{d}x \\
&= F(s) \int_0^1 f(x) \, \mathrm{d}x + f(s) \int_0^1 \left[ x f(x) - s f(x) \right] \, \mathrm{d}x \\
&= F(s) \int_0^1 f(x) \, \mathrm{d}x + f(s) \left[ \int_0^1 x f(x) \, \mathrm{d}x - \frac{\int_0^1 x f(x) \, \mathrm{d}x}{\int_0^1 f(x) \, \mathrm{d}x} \int_0^1 f(x) \, \mathrm{d}x \right] \\
&= F(s) \int_0^1 f(x) \, \mathrm{d}x.
\end{align*}
又注意到
\begin{align*}
\int_0^1 F(x) f(x) \, \mathrm{d}x = \int_0^1 F(x) \, \mathrm{d}F(x) = \frac{1}{2} \left( \int_0^1 f(x) \, \mathrm{d}x \right)^2.
\end{align*}
故
\begin{align*}
&\frac{1}{2} \left( \int_0^1 f(x) \, \mathrm{d}x \right)^2 \geq F(s) \int_0^1 f(x) \, \mathrm{d}x \implies \frac{1}{2} \int_0^1 f(x) \, \mathrm{d}x \geq F(s) = \int_0^s f(x) \, \mathrm{d}x \\
&\implies \int_0^s f(x) \, \mathrm{d}x + \int_s^1 f(x) \, \mathrm{d}x = \int_0^1 f(x) \, \mathrm{d}x \geq 2 \int_0^s f(x) \, \mathrm{d}x \\
&\implies \int_0^s f(x) \, \mathrm{d}x \leq \int_s^1 f(x) \, \mathrm{d}x.
\end{align*}
由分部积分可得
\begin{align*}
s = \frac{\int_0^1 x f(x) \, \mathrm{d}x}{\int_0^1 f(x) \, \mathrm{d}x} = \frac{\int_0^1 x \, \mathrm{d}F(x)}{F(1)} = 1 - \frac{\int_0^1 F(x) \, \mathrm{d}x}{F(1)},
\end{align*}
即$\int_0^1 F(x) \, \mathrm{d}x = (1 - s) F(1)$。又由$F$的下凸性可知
\begin{align*}
F(x) \leq 
\begin{cases}
\frac{F(1) - F(s)}{1 - s}(x - s) + F(s), & x \in [s,1] \\
\frac{F(s) - F(0)}{s}x + F(0), & x \in [0,s]
\end{cases}.
\end{align*}
于是
\begin{align*}
(1 - s) F(1) = \int_0^1 F(x) \, \mathrm{d}x &\leq \int_0^s \left[ \frac{F(1) - F(s)}{1 - s}(x - s) + F(s) \right] \, \mathrm{d}x + \int_s^1 \left[ \frac{F(s) - F(0)}{s}x + F(0) \right] \, \mathrm{d}x \\
&= \frac{1}{2} F(s) + \frac{1 - s}{2} F(1).
\end{align*}
因此
\begin{align*}
\frac{1 - s}{2} F(1) \leq \frac{1}{2} F(s) \implies F(1) \leq \frac{1}{1 - s} F(s),
\end{align*}
故
\begin{align*}
\int_s^1 f(x) \, \mathrm{d}x = F(1) - F(s) \leq \left( \frac{1}{1 - s} - 1 \right) F(s) = \frac{s}{1 - s} F(s) = \frac{s}{1 - s} \int_0^s f(x) \, \mathrm{d}x.
\end{align*}
\end{proof}

\begin{example}
求最小实数$C$,使得对一切满足$\int_{0}^{1}|f(x)|\mathrm{d}x = 1$的连续函数$f$,都有
\begin{align*}
\int_{0}^{1}|f(\sqrt{x})|\mathrm{d}x \leqslant C.
\end{align*}
\end{example}
\begin{remark}
这类证明最佳系数的问题,我们一般只需要找一个函数列,是其达到逼近取等即可.

本题将要找的函数列需要满足其积分值集中在$x=1$处,联想到Laplace方法章节具有类似性质的被积函数(即指数部分是$n$的函数),类似进行构造函数列即可.
\end{remark}
\begin{proof}
显然有
\begin{align*}
\int_0^1 |f(\sqrt{x})| \, \mathrm{d}x = 2 \int_0^1 t |f(t)| \, dt \leqslant 2 \int_0^1 |f(t)| \, dt = 2.
\end{align*}
令$f_n(t) = (n+1) t^n$,则$\int_0^1 f_n(t) \, dt = 1$。于是
\begin{align*}
\int_0^1 |f_n(\sqrt{x})| \, \mathrm{d}x &= 2 \int_0^1 t |f(t)| \, dt = 2 \int_0^1 t (n+1) t^n \, dt 
= 2 (n+1) \int_0^1 t^{n+1} \, dt = \frac{2(n+1)}{n+2} \to 2, n \to \infty.
\end{align*}
因此若$C < 2$,都存在$N \in \mathbb{N}$,使得$\int_0^1 |f_N(\sqrt{x})| \, \mathrm{d}x > C$。故$C = 2$就是最佳上界.
\end{proof}

\begin{example}
设$f\in C[0,1]$使得$\int_{0}^{1}x^{k}f(x)\mathrm{d}x = 1$,$k = 0,1,2,\cdots,n - 1$. 证明
\begin{align*}
\int_{0}^{1}|f(x)|^{2}\mathrm{d}x\geqslant n^{2}.
\end{align*}
\end{example}
\begin{proof}
设$a=(a_0,a_1,\cdots,a_{n-1})^T\in\mathbb{R}^n\setminus\{0\}$. 由Cauchy不等式及条件可知
\begin{align*}
\int_0^1 |f(x)|^2\mathrm{d}x \int_0^1 (a_0 + a_1x + \cdots + a_{n-1}x^{n-1})^2\mathrm{d}x &\geqslant \left[ \int_0^1 f(x)(a_0 + a_1x + \cdots + a_{n-1}x^{n-1})\mathrm{d}x \right]^2 \\
&= (a_0 + a_1 + \cdots + a_{n-1})^2 = \left( \sum_{j=0}^{n-1}a_j \right)^2.
\end{align*}
注意到
\begin{align*}
\int_0^1 (a_0 + a_1x + \cdots + a_{n-1}x^{n-1})^2\mathrm{d}x &= \int_0^1 \left( \sum_{j=0}^{n-1}a_jx^j \right)^2\mathrm{d}x = \int_0^1 \sum_{j=0}^{n-1}\sum_{i=0}^{n-1}a_ja_ix^{i+j}\mathrm{d}x \\
&= \sum_{j=0}^{n-1}\sum_{i=0}^{n-1}a_ja_i \int_0^1 x^{i+j}\mathrm{d}x = \sum_{j=0}^{n-1}\sum_{i=0}^{n-1}\frac{a_ja_i}{i+j+1}.
\end{align*}
因此
\begin{align*}
\int_0^1 |f(x)|^2\mathrm{d}x \geqslant \frac{\left( \sum\limits_{j=0}^{n-1}a_j \right)^2}{\sum\limits_{j=0}^{n-1}\sum\limits_{i=0}^{n-1}\frac{a_ja_i}{i+j+1}} = \frac{a^TJa}{a^THa},
\end{align*}
其中$J=\begin{pmatrix}1 & 1 & \cdots & 1 \\ 1 & 1 & \cdots & 1 \\ \vdots & \vdots & \ddots & \vdots \\ 1 & 1 & \cdots & 1\end{pmatrix}_{n\times n}$,$H=\left( \frac{1}{i+j+1} \right)_{n\times n}$. 于是我们只需求$\sup_{a\neq 0}\frac{a^TJa}{a^THa}$. 设$\lambda$为$\frac{a^TJa}{a^THa}$的一个大于$0$的上界,由\nrefexa{Basis of Algebra-example:例8.48}{(3)}可知$H$正定,则
\begin{align*}
\lambda \text{为}\frac{a^TJa}{a^THa}\text{的一个上界} &\Longleftrightarrow \lambda \geqslant \frac{a^TJa}{a^THa}, \forall a\in\mathbb{R}^n \\
&\Longleftrightarrow a^TJa \leqslant \lambda a^THa, \forall a\in\mathbb{R}^n \\
&\Longleftrightarrow a^T(\lambda H - J)a \geqslant 0, \forall a\in\mathbb{R}^n \\
&\Longleftrightarrow \lambda H - J\text{半正定}.
\end{align*}
因此$\sup_{a\neq 0}\frac{a^TJa}{a^THa} = \min\{\lambda \mid \lambda H - J\text{半正定}\} = \inf\{\lambda \mid \lambda H - J\text{半正定}\}$. 设$H_k,J_k$分别为$H,J$的$k$阶顺序主子阵,再根据\hyperref[Basis of Algebra-corollary:打洞原理推论]{打洞原理}及\nrefexa{Basis of Algebra-proposition:一些能写成两个向量乘积的矩阵}{(1)}可得
\begin{align*}
|\lambda H_k - J_k| &= |H_k| |\lambda I_k - H_k^{-1}J_k| = |H_k| |\lambda I_k - H_k^{-1}\mathbf{1}_k\mathbf{1}_k^T| \\
&= \lambda^{k-1} |H_k| (\lambda - \mathbf{1}_k^T H_k^{-1}\mathbf{1}_k).
\end{align*}
其中$\mathbf{1}_k^T = (1,1,\cdots,1)_{1\times k}$. 由$H$正定可知$|H_k| > 0$,又因为$\lambda > 0$,所以再由\reflem{Basis of Algebra-lemma:Hilbert矩阵逆矩阵元素和}可得
\begin{align*}
|\lambda H_k - J_k| > 0 \Longleftrightarrow \lambda > \mathbf{1}_k^T H_k^{-1}\mathbf{1}_k \xlongequal{\text{\reflem{Basis of Algebra-lemma:Hilbert矩阵逆矩阵元素和}}} n^2.
\end{align*}
因此对$\forall \lambda > n^2$,都有$\lambda H - J$的顺序主子式都大于$0$,故此时$\lambda H - J$正定. 于是对$\forall a\in\mathbb{R}^n\setminus\{0\}$,固定$a$,都有
\begin{align*}
a^T(\lambda H - J)a > 0, \forall \lambda > n^2.
\end{align*}
令$\lambda \to n^2$,则由$a^T(\lambda H - J)a$的连续性可知
\begin{align*}
a^T(n^2 H - J)a \geqslant 0.
\end{align*}
故$n^2 H - J$半正定. 因此$n^2 = \inf\{\lambda \mid \lambda H - J\text{半正定}\} = \sup_{a\neq 0}\frac{a^TJa}{a^THa}$. 结论得证.
\end{proof}

\begin{example}
设$A,B$都是$n$级实对称矩阵,若$B$正定,证明
\begin{align*}
\max_{\alpha\in\mathbb{R}^n\setminus\{0\}} \frac{\alpha^TA\alpha}{\alpha^TB\alpha} = \lambda_{\max}(AB^{-1}).
\end{align*}
\end{example}
\begin{proof}

\end{proof}

\begin{lemma}\label{lemma:g'(x)的Hölder连续相关结论}
设$\alpha>0, g\in C^1(\mathbb{R})$. 存在$a\in\mathbb{R}$使得$g(a)=\min_{x\in\mathbb{R}}g(x)$,如果
\begin{align}
|g'(x) - g'(y)| \leqslant M|x - y|^{\alpha}, \forall x,y\in\mathbb{R}, \label{17.39}
\end{align}
证明
\begin{align}
|g'(x)|^{\alpha + 1} \leqslant \left(\frac{\alpha + 1}{\alpha}\right)^{\alpha}[g(x) - g(a)]^{\alpha}M, \forall x\in\mathbb{R}. \label{17.40}
\end{align}
\end{lemma}
\begin{proof}
不妨设$g(a)=0$,否则用$g(x) - g(a)$代替$g(x)$. 当$M = 0$,则不等式\eqref{17.40}显然成立. 当$M\neq0$可以不妨设$M = 1$.

现在对非负函数$g$,现在我们正式开始我们的证明,当$g'(x_0)=0$,不等式\eqref{17.40}显然成立. 当$g'(x_0)>0$,则利用\eqref{17.39}有
\begin{align*}
g(x_0) &\geqslant g(x_0) - g(h)=\int_{h}^{x_0}g'(t)\mathrm{d}t \\
&\geqslant \int_{h}^{x_0}[g'(x_0) - |t - x_0|^{\alpha}]\mathrm{d}t \\
&= g'(x_0)(x_0 - h)-\frac{(x_0 - h)^{\alpha + 1}}{\alpha + 1},
\end{align*}
取$h = x_0 - |g'(x_0)|^{\frac{1}{\alpha}}$,就得到了$g(x_0)>\frac{\alpha}{\alpha + 1}|g'(x_0)|^{1 + \frac{1}{\alpha}}$,即不等式\eqref{17.40}成立. 类似的考虑$g'(x_0)<0$可得\eqref{17.40}.

当$g'(x_0)<0$,则利用\eqref{17.39}有
\begin{align*}
g(x_0) &\geqslant -g(h) + g(x_0)=-\int_{x_0}^{h}g'(t)\mathrm{d}t \\
&\geqslant -\int_{x_0}^{h}[g'(x_0) + |t - x_0|^{\alpha}]\mathrm{d}t \\
&= -g'(x_0)(h - x_0)-\frac{(h - x_0)^{\alpha + 1}}{\alpha + 1},
\end{align*}
取$h = x_0 + |g'(x_0)|^{\frac{1}{\alpha}}$,就得到了$g(x_0)>\frac{\alpha}{\alpha + 1}|g'(x_0)|^{1 + \frac{1}{\alpha}}$,即不等式\eqref{17.40}成立. 
\end{proof}

\begin{proposition}[Heisenberg(海森堡)不等式]\label{proposition:Heisenberg不等式}
设$f\in C^1(\mathbb{R})$,证明不等式
\begin{align}\label{17.29}
\left( \int_{\mathbb{R}}|f(x)|^2\mathrm{d}x \right)^2 \leqslant 4\int_{\mathbb{R}}x^2|f(x)|^2\mathrm{d}x\cdot\int_{\mathbb{R}}|f'(x)|^2\mathrm{d}x.
\end{align} 
\end{proposition}
\begin{remark}
直观上,直接Cauchy不等式,我们有
\begin{align*}
\left( \int_{\mathbb{R}}|f(x)|^2\mathrm{d}x \right)^2 \xlongequal{\textbf{分部积分}} 4\left( \int_{\mathbb{R}}xf(x)f'(x)\mathrm{d}x \right)^2 \leqslant 4\int_{\mathbb{R}}x^2|f(x)|^2\mathrm{d}x\cdot\int_{\mathbb{R}}|f'(x)|^2\mathrm{d}x.
\end{align*}
但是上述\textbf{分部积分}部分需要零边界条件. 但是其实专业数学知识告诉我们在$\mathbb{R}$上只要可积其实就可以分部积分的. 且看我们两种操作.
\end{remark}
\begin{proof}
{\heiti Method 1专业技术:}注意到对$f\in C_0^{\infty}(\mathbb{R})$,我们有
\begin{align*}
\left( \int_{\mathbb{R}}|f(x)|^2\mathrm{d}x \right)^2 \xlongequal{\text{分部积分}} 4\left( \int_{\mathbb{R}}xf(x)f'(x)\mathrm{d}x \right)^2 \leqslant 4\int_{\mathbb{R}}x^2|f(x)|^2\mathrm{d}x\cdot\int_{\mathbb{R}}|f'(x)|^2\mathrm{d}x,
\end{align*}
即不等式\eqref{17.29}成立.

对一般的$f\in C^1(\mathbb{R})$,假定
\begin{align*}
4\int_{\mathbb{R}}x^2|f(x)|^2\mathrm{d}x\cdot\int_{\mathbb{R}}|f'(x)|^2\mathrm{d}x < \infty.
\end{align*}
取紧化序列 $h_n, n\in\mathbb{N}$,则对每一个$n\in\mathbb{N}$,都有
\begin{align*}
\left( \int_{\mathbb{R}}|h_n(x)f(x)|^2\mathrm{d}x \right)^2 &\leqslant 4\int_{\mathbb{R}}x^2|h_n(x)f(x)|^2\mathrm{d}x\cdot\int_{\mathbb{R}}|(h_nf)'(x)|^2\mathrm{d}x \\
&= 4\int_{\mathbb{R}}x^2|h_n(x)f(x)|^2\mathrm{d}x\cdot\int_{\mathbb{R}}|h_n'(x)f(x) + h_n(x)f'(x)|^2\mathrm{d}x.
\end{align*}
右边让$n\to +\infty$,就有
\begin{align*}
&\lim_{n\to\infty}\left[4\int_{\mathbb{R}}x^2|h_n(x)f(x)|^2\mathrm{d}x\cdot\int_{\mathbb{R}}|h_n'(x)f(x) + h_n(x)f'(x)|^2\mathrm{d}x\right] \\
&=\left[4\int_{\mathbb{R}}x^2|f(x)|^2\mathrm{d}x\cdot\int_{\mathbb{R}}|f'(x)|^2\mathrm{d}x\right].
\end{align*}
但是左边暂时不知道是否有$\left(\int_{\mathbb{R}}|f(x)|^2\mathrm{d}x\right)^2 < \infty$,因此不能直接换序. 但是法图引理告诉我们
\begin{align*}
\left( \int_{\mathbb{R}}|f(x)|^2\mathrm{d}x \right)^2 &= \left( \int_{\mathbb{R}}\lim_{n\to\infty}|h_n(x)f(x)|^2\mathrm{d}x \right)^2 \leqslant \lim_{n\to\infty}\left( \int_{\mathbb{R}}|h_n(x)f(x)|^2\mathrm{d}x \right)^2 \\
&\leqslant 4\int_{\mathbb{R}}x^2|f(x)|^2\mathrm{d}x\cdot\int_{\mathbb{R}}|f'(x)|^2\mathrm{d}x,
\end{align*}
从而不等式\eqref{17.29}成立.

{\heiti Method 2正常技术:}对一般的$f\in C^1(\mathbb{R})$,假定
\begin{align*}
4\int_{\mathbb{R}}x^2|f(x)|^2\mathrm{d}x\cdot\int_{\mathbb{R}}|f'(x)|^2\mathrm{d}x < \infty.
\end{align*}
从分部积分需要看到,我们只需证明
\begin{align*}
\lim_{x\to\infty}x|f(x)|^2 = 0.
\end{align*}
我们以正无穷为例. 注意到
\begin{align}
\infty >\sqrt{\int_{x}^{\infty}y^2f^2(y)\mathrm{d}y\cdot\int_{x}^{\infty}|f'(y)|^2\mathrm{d}y} \stackrel{\text{Cauchy不等式}}{\geqslant} \int_{x}^{\infty}y|f'(y)f(y)|\mathrm{d}y \geqslant x\int_{x}^{\infty}|f'(y)f(y)|\mathrm{d}y, \label{17.30}
\end{align}
于是$\int_{x}^{\infty}f(y)f'(y)\mathrm{d}y=\lim_{y\to +\infty}\frac{1}{2}|f(y)|^2-\frac{1}{2}|f(x)|^2$收敛. 因此$\lim_{y\to +\infty}\frac{1}{2}|f(y)|^2$存在. 注意
$\int_{\mathbb{R}}x^2|f(x)|^2\mathrm{d}x < \infty$,因此由\hyperref[proposition:积分收敛必有子列趋于0]{积分收敛必有子列趋于0}可知, 存在$x_n\to \infty$,使得$\lim_{n\to\infty}x_n|f(x_n)| = 0$,
于是再结合$\lim_{y\to +\infty}\frac{1}{2}|f(y)|^2$存在可得
\begin{align*}
\lim_{n\to\infty}f(x_n) = 0 \Rightarrow \lim_{y\to +\infty}f(y) = 0.
\end{align*}
现在继续用\eqref{17.30},我们知道
\begin{align*}
\sqrt{\int_{x}^{\infty}y^2f^2(y)\mathrm{d}y\cdot\int_{x}^{\infty}|f'(y)|^2\mathrm{d}y} \geqslant x\int_{x}^{\infty}f'(y)f(y)\mathrm{d}y = \frac{x}{2}|f(x)|^2,
\end{align*}
令$x\to +\infty$,由Cauchy收敛准则即得$\sqrt{\int_{x}^{\infty}y^2f^2(y)\mathrm{d}y\cdot\int_{x}^{\infty}|f'(y)|^2\mathrm{d}y}\to 0$,
从而$\lim_{x\to +\infty}x|f(x)|^2 = 0$,这就完成了证明. 
\end{proof}


\end{document}
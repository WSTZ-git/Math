\documentclass[../../main.tex]{subfiles}% 注意这里的文件路径不能用 ./main.tex ,否则用latexmk编译子文件会报错
\graphicspath{{\subfix{./image/}}} % 指定图片目录,后续可以直接使用图片文件名
% 注意这里的文件路径不能用 ../../image/ ,否则用latexmk编译子文件会报错

% 例如:
% \begin{figure}[H]
% \centering
% \includegraphics[scale=0.3]{图.png}
% \caption{}
% \label{figure:图}
% \end{figure}
% 注意:上述\label{}一定要放在\caption{}之后,否则引用图片序号会只会显示??.

\begin{document}

\section{积分不等式的应用}

\begin{example}
设 $f$ 在区间 $[0,1]$ 上可积且满足
$$\int_0^1 f(x)\mathrm{d}x=\int_0^1 xf(x)\mathrm{d}x=1.$$
求证: $\int_0^1 f^2(x)\mathrm{d}x\geqslant 4$.
\end{example}
\begin{proof}
{\color{blue}证法一:}
对于任意常数 $a$ 和 $b$ 有 $\int_0^1 (f(x)-ax-b)^2\mathrm{d}x\geqslant 0$. 由此并根据条件可得
$$
\begin{aligned}
&\quad \quad \int_0^1{\left( f\left( x \right) -ax-b \right) ^2\mathrm{d}x}=\int_0^1{f^2\left( x \right) \mathrm{d}x}-2\int_0^1{\left( ax+b \right) f\left( x \right) \mathrm{d}x}+\int_0^1{\left( ax+b \right) \mathrm{d}x}\geqslant 0
\\
&\Longrightarrow \int_0^1{f^2\left( x \right) \mathrm{d}x}\geqslant 2\int_0^1{(ax}+b)f\left( x \right) \mathrm{d}x-\int_0^1{(ax}+b)^2\mathrm{d}x=2(a+b)-\frac{1}{3}a^2-ab-b^2.
\end{aligned}
$$
取 $a=6,b=-2$ 即得所证.

{\color{blue}证法二:}对$\forall a,b\in \mathbb{R}$,由Cauchy不等式可知
\begin{align*}
\int_0^1 (ax+b)^2\mathrm{d}x\int_0^1 f^2(x)\mathrm{d}x&\geqslant  \left[ \int_0^1 (ax+b)f(x)\mathrm{d}x \right]^2=(a+b)^2.
\end{align*}
从而
\begin{align*}
\int_0^1 f^2(x)\mathrm{d}x&\geqslant  \frac{(a+b)^2}{\int_0^1 (ax+b)^2\mathrm{d}x}=\frac{a^2+2ab+b^2}{\frac{a^2}{3}+ab+b^2}=3-\frac{3\frac{a}{b}+6}{\frac{a^2}{b^2}+3\frac{a}{b}+3}.
\end{align*}
再由$a,b$的任意性知
\begin{align}
\int_0^1 f^2(x)\mathrm{d}x&\geqslant  3+\sup_{a,b\in \mathbb{R}}\left\{ -\frac{3\frac{a}{b}+6}{\frac{a^2}{b^2}+3\frac{a}{b}+3} \right\}.\label{eq:103.20}
\end{align}
令$g(x)=-\frac{3x+6}{x^2+3x+3}$,则
\begin{align*}
g'(x)=\frac{3(x+1)(x+3)}{(x^2+3x+3)^2}=0\Rightarrow x=-1,-3.
\end{align*}
又$g(-1)=-3<1=g(-3)$,故$\max_{\mathbb{R}}g(x)=1$.因此
\begin{align*}
\sup_{a,b\in \mathbb{R}}\left\{ -\frac{3\frac{a}{b}+6}{\frac{a^2}{b^2}+3\frac{a}{b}+3} \right\}=\max_{\mathbb{R}}g(x)=1.
\end{align*}
再由\eqref{eq:103.20}式可知
\begin{align*}
\int_0^1 f^2(x)\mathrm{d}x\geqslant  4.
\end{align*}
并且这个不等式右边不可改进.

\end{proof}

\begin{example}
设\(f \in C^1[0, 1]\),解决下列问题.
\begin{enumerate}
\item 若\(f(0) = 0\),证明:
\begin{align*}
\int_{0}^{1}|f(x)|^{2}\mathrm{d}x \leqslant \frac{1}{2}\int_{0}^{1}|f'(x)|^{2}\mathrm{d}x.
\end{align*}

\item 若\(f(0) = f(1) = 0\),证明:
\begin{align*}
\int_{0}^{1}|f(x)|^{2}\mathrm{d}x \leqslant \frac{1}{8}\int_{0}^{1}|f'(x)|^{2}\mathrm{d}x.
\end{align*}
\end{enumerate}
\end{example}
\begin{remark}
牛顿莱布尼兹公式也可以看作带积分余项的插值公式(插一个点).
\end{remark}
\begin{proof}
\begin{enumerate}
\item 由牛顿莱布尼兹公式可知
\begin{align*}
f(x) = f(0) + \int_{0}^{x}f'(y)\mathrm{d}y=\int_{0}^{x}f'(y)\mathrm{d}y.
\end{align*}
从而
\begin{align*}
|f(x)|^2 = \left| \int_{0}^{x}f'(y)\mathrm{d}y \right|^2 \leqslant \int_{0}^{x}1^2\mathrm{d}y \int_{0}^{x}|f'(y)|^2\mathrm{d}y = x\int_{0}^{x}|f'(y)|^2\mathrm{d}y \leqslant x\int_{0}^{1}|f'(y)|^2\mathrm{d}y.
\end{align*}
于是对上式两边同时积分可得
\begin{align*}
\int_{0}^{1}|f(x)|^2\mathrm{d}x \leqslant \int_{0}^{1}x\mathrm{d}x \int_{0}^{1}|f'(y)|^2\mathrm{d}y = \frac{1}{2}\int_{0}^{1}|f'(y)|^2\mathrm{d}y.
\end{align*}

\item 由牛顿莱布尼兹公式(带积分型余项的插值公式)可得
\begin{align*}
f(x) = \int_{0}^{x}f(y)\mathrm{d}y, x \in \left[0, \frac{1}{2}\right]; \quad f(x) = \int_{x}^{1}f'(y)\mathrm{d}y, x \in \left[\frac{1}{2}, 1\right].
\end{align*}
从而
\begin{align*}
|f(x)|^2 = \left| \int_{0}^{x}f'(y)\mathrm{d}y \right|^2 \leqslant \int_{0}^{x}1^2\mathrm{d}y\int_{0}^{x}|f'(y)|^2\mathrm{d}y = x\int_{0}^{x}|f'(y)|^2\mathrm{d}y \leqslant x\int_{0}^{\frac{1}{2}}|f'(y)|^2\mathrm{d}y, x \in \left[0, \frac{1}{2}\right].
\end{align*}
\begin{align*}
|f(x)|^2=\left| \int_x^1{f' (y)\mathrm{d}y} \right|^2\leqslant \int_x^1{1^2\mathrm{d}y\int_x^1{|f' (y)|^2\mathrm{d}y}}\leqslant (1-x)\int_{\frac{1}{2}}^1{|f' (y)|^2\mathrm{d}y,x}\in \left[ \frac{1}{2},1 \right] .
\end{align*}
于是对上面两式两边同时积分可得
\begin{align*}
\int_{0}^{\frac{1}{2}}|f(x)|^2\mathrm{d}x \leqslant \int_{0}^{\frac{1}{2}}x\mathrm{d}x\int_{0}^{\frac{1}{2}}|f'(y)|^2\mathrm{d}y = \frac{1}{8}\int_{0}^{\frac{1}{2}}|f'(y)|^2\mathrm{d}y.
\end{align*}
\begin{align*}
\int_{\frac{1}{2}}^{1}|f(x)|^2\mathrm{d}x \leqslant \int_{\frac{1}{2}}^{1}(1 - x)\mathrm{d}x\int_{\frac{1}{2}}^{1}|f'(y)|^2\mathrm{d}y = \frac{1}{8}\int_{\frac{1}{2}}^{1}|f'(y)|^2\mathrm{d}y.
\end{align*}
将上面两式相加得
\begin{align*}
\int_{0}^{1}|f(x)|^2\mathrm{d}x \leqslant \frac{1}{8}\int_{0}^{1}|f'(y)|^2\mathrm{d}y.
\end{align*}
\end{enumerate}

\end{proof}

\begin{example}[$\,\,$opial不等式]\label{example:opial不等式}

\textbf{特例:}
\begin{enumerate}
\item 设\(f \in C^1[a,b]\)且\(f(a) = 0\),证明
\begin{align*}
\int_{a}^{b}|f(x)f'(x)|\mathrm{d}x \leqslant \frac{b - a}{2}\int_{a}^{b}|f'(x)|^{2}\mathrm{d}x.
\end{align*}

\item 设\(f \in C^1[a,b]\)且\(f(a) = 0\),\(f(b) = 0\),证明
\begin{align*}
\int_{a}^{b}|f(x)f'(x)|\mathrm{d}x \leqslant \frac{b - a}{4}\int_{a}^{b}|f'(x)|^{2}\mathrm{d}x.
\end{align*} 
\end{enumerate}

\textbf{一般情况:}
\begin{enumerate}
\item 设\(f \in C^1[a,b], p\geqslant0, q\geqslant1\)且\(f(a) = 0\). 证明
\begin{align}
\int_{a}^{b}|f(x)|^{p}|f'(x)|^{q}\mathrm{d}x \leqslant \frac{q(b - a)^{p}}{p + q}\int_{a}^{b}|f'(x)|^{p + q}\mathrm{d}x. \label{equation----:::16.37}
\end{align}

\item 若还有\(f(b) = 0\). 证明
\begin{align}
\int_{a}^{b}|f(x)|^{p}|f'(x)|^{q}\mathrm{d}x \leqslant \frac{q(b - a)^{p}}{(p + q)2^{p}}\int_{a}^{b}|f'(x)|^{p + q}\mathrm{d}x. \label{equation----:::16.38}
\end{align}
\end{enumerate}
\end{example}
\begin{note}
说明了证明的想法就是注意变限积分为整体凑微分.
\end{note}
\begin{proof}
{\heiti 特例:}
\begin{enumerate}
\item 令\(F(x) \triangleq \int_{a}^{x}|f'(y)|\mathrm{d}y\),则\(F'(x) = |f'(x)|\),\(F(a) = 0\)。从而
\begin{align*}
f(x) = \int_{0}^{x}f'(y)\mathrm{d}y \Rightarrow |f(x)| \leqslant \int_{a}^{x}|f'(y)|\mathrm{d}y = F(x).
\end{align*}
于是
\begin{align*}
\int_{a}^{b}|f(x)f'(x)|\mathrm{d}x \leqslant \int_{a}^{b}F(x)F'(x)\mathrm{d}x = \frac{1}{2}F^2(x)\big|_{a}^{b} = \frac{1}{2}F^2(b) = \frac{1}{2}\left( \int_{a}^{b}|f'(y)|\mathrm{d}x \right)^2 \\
\overset{\text{Cauchy不等式}}{\leqslant} \frac{1}{2}\int_{a}^{b}1^2\mathrm{d}x \int_{a}^{b}|f'(y)|^2\mathrm{d}x = \frac{b - a}{2}\int_{a}^{b}|f'(y)|^2\mathrm{d}x.
\end{align*}

\item 由第1问可知
\begin{align*}
\int_{a}^{\frac{a + b}{2}}|f(x)f'(x)|\mathrm{d}x \leqslant \frac{\frac{a + b}{2} - a}{2}\int_{a}^{\frac{a + b}{2}}|f'(y)|^2\mathrm{d}y = \frac{b - a}{4}\int_{a}^{\frac{a + b}{2}}|f'(y)|^2\mathrm{d}y.
\end{align*}
\begin{align*}
\int_{\frac{a + b}{2}}^{b}|f(x)f'(x)|\mathrm{d}x \leqslant \frac{\frac{a + b}{2} - a}{2}\int_{\frac{a + b}{2}}^{b}|f'(y)|^2\mathrm{d}y = \frac{b - a}{4}\int_{\frac{a + b}{2}}^{b}|f'(y)|^2\mathrm{d}y.
\end{align*}
将上面两式相加可得
\begin{align*}
\int_{a}^{b}|f(x)f'(x)|\mathrm{d}x \leqslant \frac{b - a}{4}\int_{a}^{b}|f'(y)|^2\mathrm{d}y.
\end{align*}
\end{enumerate}

{\heiti 一般情况:}
\begin{enumerate}
\item 只证\(q>1\). \(q = 1\)可类似得到. 考虑
\[f(x)=\int_{a}^{x}f'(y)\mathrm{d}y, F(x)=\int_{a}^{x}|f'(y)|^{q}\mathrm{d}y.\]
则由\hyperref[theorem:Hölder(赫尔德)不等式(积分形式)]{Hölder不等式}, 我们知道
\begin{align*}
|f(x)|^{p}&\leqslant \left(\int_{a}^{x}|f'(y)|\mathrm{d}y\right)^{p} \leqslant \left(\int_{a}^{x}|f'(y)|^{q}\mathrm{d}y\right)^{\frac{p}{q}}\left(\int_{a}^{x}1^{\frac{q}{q - 1}}\mathrm{d}y\right)^{\frac{p(q - 1)}{q}} = F^{\frac{p}{q}}(x)(x - a)^{\frac{p(q - 1)}{q}},
\end{align*}
这里\(\frac{1}{p}+\frac{1}{q}=1\).
于是
\begin{align*}
\int_{a}^{b}|f(x)|^{p}|f'(x)|^{q}\mathrm{d}x &\leqslant \int_{a}^{b}F^{\frac{p}{q}}(x)(x - a)^{\frac{p(q - 1)}{q}}|f'(x)|^{q}\mathrm{d}x = \int_{a}^{b}F^{\frac{p}{q}}(x)(x - a)^{\frac{p(q - 1)}{q}}dF(x)\\
&\leqslant (b - a)^{\frac{p(q - 1)}{q}}\int_{a}^{b}F^{\frac{p}{q}}(x)dF(x) = \frac{q}{q + p}(b - a)^{\frac{p(q - 1)}{q}}F^{\frac{p + q}{q}}(b)\\
&=\frac{q}{q + p}(b - a)^{\frac{p(q - 1)}{q}}\left(\int_{a}^{b}|f'(y)|^{q}\mathrm{d}y\right)^{\frac{p + q}{q}}\\
&\stackrel{\text{Cauchy不等式}}{\leqslant} \frac{q}{q + p}(b - a)^{\frac{p(q - 1)}{q}}\left(\int_{a}^{b}|f'(y)|^{q(\frac{p + q}{q})}\mathrm{d}y\right)^{\frac{q}{q + p}}\left(\int_{a}^{b}1^{(\frac{p + q}{q - 1})}\mathrm{d}y\right)^{\frac{q - 1}{q + p}}\\
&=\frac{q(b - a)^{p}}{p + q}\int_{a}^{b}|f'(y)|^{p + q}\mathrm{d}y,
\end{align*}
这就证明了不等式\eqref{equation----:::16.37}.

\item 由第一问得
\[\int_{a}^{\frac{a + b}{2}}|f(x)|^{p}|f'(x)|^{q}\mathrm{d}x \leqslant \frac{q(b - a)^{p}}{(p + q)2^{p}}\int_{a}^{\frac{a + b}{2}}|f'(x)|^{p + q}\mathrm{d}x,\]
对称得
\[\int_{\frac{a + b}{2}}^{b}|f(x)|^{p}|f'(x)|^{q}\mathrm{d}x \leqslant \frac{q(b - a)^{p}}{(p + q)2^{p}}\int_{\frac{a + b}{2}}^{b}|f'(x)|^{p + q}\mathrm{d}x.\] 
故上面两式相加得到\eqref{equation----:::16.38}式.
\end{enumerate}

\end{proof}

\begin{example}
设\(f \in C[0,1]\)满足\(\int_{0}^{1}f(x)\mathrm{d}x = 0\),证明:
\begin{align}
\left(\int_{0}^{1}xf(x)\mathrm{d}x\right)^2 \leqslant \frac{1}{12}\int_{0}^{1}f^{2}(x)\mathrm{d}x. \label{equation----:::::16.28}
\end{align}
\end{example}
\begin{note}
从条件\(\int_{0}^{1}f(x)\mathrm{d}x = 0\)来看,我们待定\(a \in \mathbb{R}\),一定有
\begin{align*}
\int_{0}^{1}xf(x)\mathrm{d}x = \int_{0}^{1}(x - a)f(x)\mathrm{d}x.
\end{align*}
然后利用Cauchy不等式得
\begin{align*}
\left(\int_{0}^{1}(x - a)f(x)\mathrm{d}x\right)^2 \leqslant \int_{0}^{1}(x - a)^2\mathrm{d}x\int_{0}^{1}f^{2}(x)\mathrm{d}x.
\end{align*}
为了使得不等式最精确,我们自然希望\(\int_{0}^{1}(x - a)^2\mathrm{d}x\)达到最小值. 读者也可以直接根据对称性猜测出\(a = \frac{1}{2}\)就是达到最小值的\(a\).
\end{note}
\begin{proof}
利用 Cauchy 不等式得
\begin{align*}
\frac{1}{12}\int_{0}^{1}f^{2}(x)\mathrm{d}x=\int_{0}^{1}\left(x - \frac{1}{2}\right)^2\mathrm{d}x\int_{0}^{1}f^{2}(x)\mathrm{d}x\geqslant \left(\int_{0}^{1}\left(x - \frac{1}{2}\right)f(x)\mathrm{d}x\right)^2=\left(\int_{0}^{1}xf(x)\mathrm{d}x\right)^2,
\end{align*}
这就证明了\eqref{equation----:::::16.28}式.

\end{proof}

\begin{example}
设\(f \in C^1[0,1]\),\(\int_{\frac{1}{3}}^{\frac{2}{3}}f(x)\mathrm{d}x = 0\),证明
\begin{align*}
\int_0^1{|f' (x)|^2\mathrm{d}x}\geqslant 27\left( \int_0^1{f(x)\mathrm{d}x} \right) ^2.
\end{align*}
\end{example}
\begin{note}
为了分部积分提供\(0\)边界且求导之后不留下东西,设\(g(0) = g(1) = 0\)且\(g\)是一次函数,这不可能,于是只能是分段函数\(g(x)=\begin{cases}x - 1, & c\leqslant x\leqslant 1 \\ x, & 0\leqslant x\leqslant c\end{cases}\)。为了让\(g\)连续会发现\(c = c - 1\),这不可能。结合\(\int_{\frac{1}{3}}^{\frac{2}{3}}f(x)\mathrm{d}x = 0\),所以我们插入一段来使得连续,因此真正构造的函数为
\[g(x)=\begin{cases}x - 1, & \frac{2}{3}\leqslant x\leqslant 1 \\ 1 - 2x, & \frac{1}{3}\leqslant x\leqslant \frac{2}{3} \\ x, & 0\leqslant x\leqslant \frac{1}{3}\end{cases}.\]
\end{note}
\begin{proof}
令
\[g(x)=\begin{cases}x - 1, & \frac{2}{3}\leqslant x\leqslant 1 \\ 1 - 2x, & \frac{1}{3}\leqslant x\leqslant \frac{2}{3} \\ x, & 0\leqslant x\leqslant \frac{1}{3}\end{cases}.\]
于是由 Cauchy 不等式,我们有
\begin{align*}
&\int_{0}^{1}|f'(x)|^{2}\mathrm{d}x\int_{0}^{1}|g(x)|^{2}\mathrm{d}x\geqslant \left(\int_{0}^{1}f'(x)g(x)\mathrm{d}x\right)^{2}\xlongequal{\text{分部积分}}\left(\int_{0}^{1}f(x)g'(x)\mathrm{d}x\right)^{2}\\
&=\left(\int_{0}^{\frac{1}{3}}f(x)\mathrm{d}x - 2\int_{\frac{1}{3}}^{\frac{2}{3}}f(x)\mathrm{d}x+\int_{\frac{2}{3}}^{1}f(x)\mathrm{d}x\right)^{2}=\left(\int_{0}^{\frac{1}{3}}f(x)\mathrm{d}x+\int_{\frac{1}{3}}^{\frac{2}{3}}f(x)\mathrm{d}x+\int_{\frac{2}{3}}^{1}f(x)\mathrm{d}x\right)^{2}=\left(\int_{0}^{1}f(x)\mathrm{d}x\right)^{2},
\end{align*}
结合\(\int_{0}^{1}|g(x)|^{2}\mathrm{d}x = \frac{1}{27}\),这就完成了证明.

\end{proof}

\begin{example}
设\(f \in C[a,b]\cap D(a,b)\)且\(f(a) = f(b) = 0\)且\(f\)不恒为\(0\),证明存在一点\(\xi \in (a,b)\)使得
\begin{align}
|f'(\xi)| > \frac{4}{(b - a)^2}\int_{a}^{b}\left|f(x)\right|\mathrm{d}x.\label{eq:::23523523424jrioj53313}
\end{align} 
\end{example}
\begin{remark}
\hypertarget{example0.5----:不妨设的原因}{不妨设}\(\int_{a}^{b}f(x)\mathrm{d}x > 0\)的原因:若\(\int_{a}^{b}f(x)\mathrm{d}x < 0\)则用\(-f\)代替\(f\),\(\int_{a}^{b}f(x)\mathrm{d}x = 0\)是平凡的。
\end{remark}
\begin{proof}
反证,若\(|f'(x)| \leqslant \frac{4}{(b - a)^2}\int_{a}^{b}\left|f(x)\right| \mathrm{d}x\triangleq M\),则\hyperlink{example0.5----:不妨设的原因}{不妨设\(\int_{a}^{b}f(x)\mathrm{d}x > 0\)},由 Hermite 插值定理可知,存在\(\theta_1 \in (a, x)\),\(\theta_2 \in (x, b)\),使得
\begin{align*}
f(x) = f(a) + f'(\theta_1)(x - a) \leqslant M(x - a), \forall x \in \left[a, \frac{a + b}{2}\right].
\end{align*}
\begin{align*}
f(x) = f(b) + f'(\theta_2)(x - b) \leqslant -M(x - b) = M(b - x), \forall x \in \left[\frac{a + b}{2}, b\right].
\end{align*}
从而
\begin{align*}
\int_{a}^{b}|f(x)|\mathrm{d}x \leqslant \int_{a}^{\frac{a + b}{2}}M(x - a)\mathrm{d}x + \int_{\frac{a + b}{2}}^{b}M(b - x)\mathrm{d}x = \frac{M(b - a)^2}{4} = \int_{a}^{b}|f(x)|\mathrm{d}x.
\end{align*}
于是结合\(f\)的连续性及$M(x-a)-f(x),M(b-x)-f(x)\geqslant 0(\forall x\in [a,b])$可得
\begin{align*}
\int_a^{\frac{a+b}{2}}{f(x)\mathrm{d}x}=\int_a^{\frac{a+b}{2}}{M(x}-a)\mathrm{d}x\Longrightarrow \int_a^{\frac{a+b}{2}}{\left[ M\left( x-a \right) -f\left( x \right) \right] \mathrm{d}x}=0\Longrightarrow f(x)=M(x-a),\forall x\in \left[ a,\frac{a+b}{2} \right] .
\end{align*}
\begin{align*}
\int_{\frac{a+b}{2}}^b{f(x)\mathrm{d}x}=\int_{\frac{a+b}{2}}^b{M(b}-x)\mathrm{d}x\Longrightarrow \int_{\frac{a+b}{2}}^b{\left[ M\left( b-x \right) -f\left( x \right) \right] \mathrm{d}x}=0\Longrightarrow f(x)=M(b-x),\forall x\in \left[ \frac{a+b}{2},b \right] .
\end{align*}
故\(f\)在\(x = \frac{a + b}{2}\)处不可导,这与\(f \in D(a, b)\)矛盾!

\end{proof}

\begin{example}设 \(f \in C^1[0, \pi]\) 且满足 \(\int_{0}^{\pi} f(x) \mathrm{d}x = 0\),证明:
\begin{align*}
|f(x)|\leqslant \sqrt{\frac{\pi}{3}\int_0^{\pi}{|f' (t)|^2\mathrm{d}t}},\forall x\in [0,\pi ].
\end{align*} 
\end{example}
\begin{remark}
原不等式等价于
\begin{align*}
f^2(x) \leqslant \frac{\pi}{3}\int_0^{\pi} |f'(t)|^2\mathrm{d}t, \forall x\in [0, \pi].
\end{align*}
显然要利用Cauchy不等式,先待定\(g(x)\),由Cauchy不等式可得
\begin{align}
\int_0^{\pi} |f'(t)|^2\mathrm{d}t\int_0^{\pi} g^2(t) \mathrm{d}t\geqslant \left( \int_0^{\pi} f'(t)g(t) \mathrm{d}t \right) ^2, \forall x\in [0, \pi]. \label{eq:101.2}
\end{align}
此时,我们希望对\(\forall x\in [0, \pi]\),固定\(x\),都有\(\int_0^{\pi} f'(t)g(t) \mathrm{d}t = kf(x)\),其中\(k\)为某一常数。因此\(g(t)\)必和\(x\)有关,于是令
\[g(t) = 
\begin{cases}
t - \pi, & t\in [x, \pi]\\
t, & t\in [0, x]
\end{cases},
\]
再代入\eqref{eq:101.2}式验证即可。

实际上,回忆\refthe{theorem:带积分型余项的Lagrange插值公式}中的Green函数,可以发现上述构造的$g(x)=\frac{\mathrm{d}k\left( x,t \right)}{\mathrm{d}x},\,x,t\in [0,\pi].$

希望 \(\int_0^{\pi} f(t)g'(t) \mathrm{d}t = f(x)\),考虑广义导数,使得 \(g'(x) = \delta(x)\)。实际上,这里的 \(g\) 就是 \(H\) 函数(详细参考rudin的泛函分析).
\end{remark}
\begin{proof}
对\(\forall x\in [0, \pi]\),令
\[g(t) = 
\begin{cases}
t - \pi, & t\in [x, \pi]\\
t, & t\in [0, x]
\end{cases},
\]
则
\begin{align*}
\left( \int_0^{\pi} f'(t)g(t) \mathrm{d}t \right) ^2&=\left( \int_x^{\pi} (t - \pi)f'(t) \mathrm{d}t+\int_0^x tf(t) \mathrm{d}t \right) ^2\\
&\xlongequal{\text{分部积分}}\left( -(x - \pi)f'(x) - \int_x^{\pi} f(t) \mathrm{d}t+xf(x) - \int_0^x f(t) \mathrm{d}t \right) ^2\\
&=\pi ^2|f(x)|^2.
\end{align*}
\begin{align*}
\int_0^{\pi} g^2(t) \mathrm{d}t=\int_x^{\pi} (t - \pi)^2\mathrm{d}t+\int_0^x t^2\mathrm{d}t=\frac{\pi}{3}(3x^2 - 3\pi x + \pi ^2).
\end{align*}
故由Cauchy不等式可得
\begin{align*}
\frac{\pi}{3}(3x^2 - 3\pi x + \pi ^2)\int_0^{\pi} |f'(t)|^2\mathrm{d}t=\int_0^{\pi} |f'(t)|^2\mathrm{d}t\int_0^{\pi} g^2(t) \mathrm{d}t
\geqslant \left( \int_0^{\pi} f'(t)g(t) \mathrm{d}t \right) ^2=\pi ^2|f(x)|^2, \forall x\in [0, \pi]
\end{align*}
即
\begin{align*}
|f(x)|^2\leqslant \frac{1}{3\pi}(3x^2 - 3\pi x + x^2)\int_0^{\pi} |f'(t)|^2\mathrm{d}t
\leqslant \frac{\pi}{3}\int_0^{\pi} |f'(t)|^2\mathrm{d}t, \forall x\in [0, \pi]
\end{align*}

\end{proof}

\begin{proposition}[反向Cauchy不等式]\label{proposition:反向Cauchy不等式}
设 \(f,g \in R[a,b]\),\(g \geqslant  0\),\(0 < m \leqslant  f \leqslant  M\),证明
\begin{align*}
\left( \int_a^b{g\left( x \right) \,\mathrm{d}x} \right) ^2\leqslant \int_a^b{\frac{g\left( x \right)}{f\left( x \right)}\,\mathrm{d}x\int_a^b{f\left( x \right) g\left( x \right) \,\mathrm{d}x\leqslant \frac{1}{4}\left( \sqrt{\frac{M}{m}}+\sqrt{\frac{m}{M}} \right) ^2\left( \int_a^b{g\left( x \right) \,\mathrm{d}x} \right) ^2}}.
\end{align*} 
\end{proposition}
\begin{proof}
由 Cauchy 不等式可得
\begin{align*}
\left( \int_a^b g(x) \, \mathrm{d}x \right)^2 = \left( \int_a^b \sqrt{f(x) g(x)} \cdot \sqrt{\frac{g(x)}{f(x)}} \, \mathrm{d}x \right)^2 
\leqslant \int_a^b \left[ \sqrt{f(x) g(x)} \right]^2 \, \mathrm{d}x \int_a^b \left[ \sqrt{\frac{g(x)}{f(x)}} \right]^2 \, \mathrm{d}x 
= \int_a^b f(x) g(x) \, \mathrm{d}x \int_a^b \frac{g(x)}{f(x)} \, \mathrm{d}x.
\end{align*}
故第一个不等式成立。下证第二个不等式。由条件和均值不等式可知
\begin{align*}
&\quad \quad \quad \int_a^b{\frac{\left[ f\left( x \right) -m \right] \left[ M-f\left( x \right) \right]}{f\left( x \right)}g\left( x \right) \mathrm{d}x}\geqslant 0\Longleftrightarrow \int_a^b{\frac{Mf\left( x \right) +mf\left( x \right) -mM-f^2\left( x \right)}{f\left( x \right)}g\left( x \right) \mathrm{d}x}\geqslant 0
\\
&\Longleftrightarrow \left( M+m \right) \int_a^b{g\left( x \right) \mathrm{d}x}\geqslant mM\int_a^b{\frac{g\left( x \right)}{f\left( x \right)}\mathrm{d}x}+\int_a^b{f\left( x \right) g\left( x \right) \mathrm{d}x}\geqslant 2\sqrt{mM}\sqrt{\int_a^b{\frac{g\left( x \right)}{f\left( x \right)}\mathrm{d}x}\int_a^b{f\left( x \right) g\left( x \right) \mathrm{d}x}}.
\end{align*}
故
\begin{align*}
\int_a^b \frac{g(x)}{f(x)} \, \mathrm{d}x \int_a^b f(x) g(x) \, \mathrm{d}x &\leqslant \left[ \frac{(M + m)}{2 \sqrt{m M}} \int_a^b g(x) \, \mathrm{d}x \right]^2.
\end{align*}
即
\begin{align*}
\int_a^b{\frac{g\left( x \right)}{f\left( x \right)}\,\mathrm{d}x\int_a^b{f\left( x \right) g\left( x \right) \,\mathrm{d}x\leqslant \frac{1}{4}\left( \sqrt{\frac{M}{m}}+\sqrt{\frac{m}{M}} \right) ^2\left( \int_a^b{g\left( x \right) \,\mathrm{d}x} \right) ^2}}.
\end{align*}

\end{proof}

\begin{example}
设 \(f,g \in R[a,b]\) 满足
\begin{align*}
0 < m \leqslant f(x) \leqslant M, \quad \int_{a}^{b} g(x)\mathrm{d}x = 0.
\end{align*}
证明:
\begin{align*}
\left( \int_a^b{f\left( x \right) g\left( x \right) \mathrm{d}x} \right) ^2\leqslant \left( \frac{M-m}{M+m} \right) ^2\int_a^b{f^2\left( x \right) \mathrm{d}x\int_a^b{g^2\left( x \right) \mathrm{d}x.}}
\end{align*} 
\end{example}
\begin{remark}
待定常数 \(k\),由条件 \(\int_a^b g(x) \, \mathrm{d}x = 0\) 和 Cauchy 不等式可得
\begin{align*}
\left( \int_a^b f(x) g(x) \, \mathrm{d}x \right)^2 = \left( \int_a^b \left( f(x) - k \right) g(x) \, \mathrm{d}x \right)^2 \leqslant \int_a^b \left( f(x) - k \right)^2 \, \mathrm{d}x \int_a^b g^2(x) \, \mathrm{d}x.
\end{align*}
于是我们希望
\begin{align*}
\int_a^b \left( f(x) - k \right)^2 \, \mathrm{d}x \int_a^b g^2(x) \, \mathrm{d}x \leqslant \left( \frac{M - m}{M + m} \right)^2 \int_a^b f^2(x) \, \mathrm{d}x \int_a^b g^2(x) \, \mathrm{d}x.
\end{align*}
从而希望
\begin{align*}
\left( f(x) - k \right)^2 \leqslant \left( \frac{M - m}{M + m} \right)^2 f^2(x).
\end{align*}
又因为 \(m \leqslant f(x) \leqslant M\),所以只需要下式成立即可
\begin{align}
\left( t - k \right)^2 \leqslant \left( \frac{M - m}{M + m} \right)^2 t^2, \quad \forall t \in [m, M]. \label{eq:100.5}
\end{align}
我们只需要找到出一个合适的 \(k\),使这个 \(k\) 满足上式即可。

现在,我们先求不等式 \(\left( t - k \right)^2 \leqslant C t^2\),\(\forall t \in [m, M]\) 的最佳系数 \(C\)。即求最小的 \(C > 0\),存在 \(k \in \mathbb{R}\),使得
\begin{align*}
\left( t - k \right)^2 \leqslant C t^2, \quad \forall t \in [m, M].
\end{align*}
上式等价于
\begin{align*}
\left( 1 - \frac{k}{t} \right)^2 \leqslant C, \quad \forall t \in [m, M] \Longleftrightarrow \left( 1 - \frac{k}{M} \right)^2, \left( 1 - \frac{k}{m} \right)^2 \leqslant C.
\end{align*}
令 \(h(x) \triangleq \max \left\{ \left( 1 - \frac{x}{M} \right)^2, \left( 1 - \frac{x}{m} \right)^2 \right\}\),则 \(C = \min_{x \in \mathbb{R}} h(x)\),\(k\) 是 \(h(x)\) 的最小值点。

(画图)易知 \(h(x)\) 的最小值就在 \(\left( 1 - \frac{x}{M} \right)^2\) 和 \(\left( 1 - \frac{x}{m} \right)^2\) 中间的一个交点处取到,即 \(k \in \left( \frac{1}{M}, \frac{1}{m} \right)\)。于是由 \(\left( 1 - \frac{x}{M} \right)^2 = \left( 1 - \frac{x}{m} \right)^2\) 可得
\begin{align*}
\text{(i)} \quad 1 - \frac{x}{M} &= 1 - \frac{x}{m} \Longrightarrow x = 0, \quad h(0) = 1, \\
\text{(ii)} \quad 1 - \frac{x}{M} &= \frac{x}{m} - 1 \Longrightarrow 2 = \left( \frac{1}{m} + \frac{1}{M} \right) x \Longrightarrow x = \frac{2mM}{M + m}, \quad h\left( \frac{2mM}{M + m} \right) = \left( \frac{M - m}{M + m} \right)^2.
\end{align*}
故 \(k = \frac{2mM}{M + m}\),\(C = \left( \frac{M - m}{M + m} \right)^2\)。再结合 \eqref{eq:100.5} 式,可知原不等式的系数就是最佳系数,并且此时我们找到了证明需要的 \(k = \frac{2mM}{M + m}\)。证明只需要将 \(k = \frac{2mM}{M + m}\) 代入上述步骤验证即可。
\end{remark}
\begin{proof}
由条件 \(\int_a^b g(x) \, \mathrm{d}x = 0\) 和 Cauchy 不等式可得
\begin{align}
\left( \int_a^b f(x) g(x) \, \mathrm{d}x \right)^2 &= \left( \int_a^b \left( f(x) - \frac{2mM}{M + m} \right) g(x) \, \mathrm{d}x \right)^2 \nonumber \\
&\leqslant \int_a^b \left( f(x) - \frac{2mM}{M + m} \right)^2 \, \mathrm{d}x \int_a^b g^2(x) \, \mathrm{d}x. \label{eq:100.6}
\end{align}
注意到
\begin{align*}
\left( t - \frac{2mM}{M + m} \right)^2 - \left( \frac{M - m}{M + m} \right)^2 t &= \frac{4mM \left( t - M \right) \left( m - t \right)}{\left( m + M \right)^2} \leqslant 0, \quad \forall t \in [m, M].
\end{align*}
因此由 \(f(x) \in [m, M]\),\(\forall x \in \mathbb{R}\) 可得
\begin{align*}
\left( f(x) - \frac{2mM}{M + m} \right)^2 \leqslant \left( \frac{M - m}{M + m} \right)^2 f^2(x), \quad \forall x \in \mathbb{R}.
\end{align*}
于是再结合 \eqref{eq:100.6} 式可得
\begin{align*}
\left( \int_a^b f(x) g(x) \, \mathrm{d}x \right)^2 \leqslant \int_a^b \left( f(x) - \frac{2mM}{M + m} \right)^2 \, \mathrm{d}x \int_a^b g^2(x) \, \mathrm{d}x 
\leqslant \left( \frac{M - m}{M + m} \right)^2 \int_a^b f^2(x) \, \mathrm{d}x \int_a^b g^2(x) \, \mathrm{d}x, \forall x \in \mathbb{R}.
\end{align*}

\end{proof}

\begin{example}
设 \(f \in C^2[0,1]\) 满足 \(f(0) = f(1) = f'(0) = 0\),\(f'(1) = 1\)。证明
\begin{align*}
\int_{0}^{1} |f''(x)|^2 \mathrm{d}x \geqslant 4.
\end{align*} 
\end{example}
\begin{remark}
待定 \(g(x)\),由 Cauchy 不等式及条件可得
\begin{align*}
\int_0^1 \left| f''(x) \right|^2 \, \mathrm{d}x \int_0^1 g^2(x) \, \mathrm{d}x &\geqslant \left( \int_0^1 f''(x) g(x) \, \mathrm{d}x \right)^2 \\
&\xlongequal{\text{分部积分}} \left( g(1) - \int_0^1 f'(x) g'(x) \, \mathrm{d}x \right)^2 \\
&\xlongequal{\text{分部积分}} \left( g(1) + \int_0^1 f(x) g''(x) \, \mathrm{d}x \right)^2.
\end{align*}
将上式两边与要证不等式对比,我们希望 \(g''(x) \equiv 0\),从而 \(\int_0^1 f(x) g''(x) \, \mathrm{d}x = 0\),于是上式可化为
\begin{align}
\int_0^1 \left| f''(x) \right|^2 \, \mathrm{d}x \int_0^1 g^2(x) \, \mathrm{d}x &\geqslant g^2(1) \nonumber \\
\Longleftrightarrow \int_0^1 \left| f''(x) \right|^2 \, \mathrm{d}x &\geqslant \frac{g^2(1)}{\int_0^1 g^2(x) \, \mathrm{d}x}. \label{eq:100.7}
\end{align}
因此只要 \(g(x)\) 还满足 \(\frac{g^2(1)}{\int_0^1 g^2(x) \, \mathrm{d}x} \geqslant 4\) 即可。

因为 \(g''(x) \equiv 0\),所以我们可以设 \(g(x)\) 为一次函数,即 \(g(x) = ax + b\),\(a \ne 0\)。又因为 \(\frac{g^2(1)}{\int_0^1 g^2(x) \, \mathrm{d}x}\) 越大,不等式 \eqref{eq:100.7} 越强,所以现在我们想要找到一个一次函数 \(g(x)\) 使得 \(\frac{g^2(1)}{\int_0^1 g^2(x) \, \mathrm{d}x}\) 达到最大值。

不妨设 \(g(x) = ax - 1\),\(a \ne 0\),否则用 \(-bg(x)\) 代替 \(g(x)\),不改变 \(\frac{g^2(1)}{\int_0^1 g^2(x) \, \mathrm{d}x}\) 的取值。
此时,我们有
\begin{align*}
\frac{g^2(1)}{\int_0^1 g^2(x) \, \mathrm{d}x} = 3 \cdot \frac{a^2 - 2a + 1}{a^2 - 3a + 3} = 3 \left( 1 + \frac{a}{a^2 - 3a + 3} \right).
\end{align*}
令 \(h(a) = \frac{a}{a^2 - 3a + 3}\),则由 \(h'(a) = \frac{3 - a^2}{(a^2 - 3a + 3)^2} = 0\) 可得 \(h\) 的极大值点为 \(a = \sqrt{3}\)。
又因为
\begin{align*}
\lim_{a \to -\infty} h(a) = \lim_{a \to -\infty} \frac{a}{a^2 - 3a + 3} = 0, \quad h(\sqrt{3}) = \frac{\sqrt{3}}{6 - 3\sqrt{3}} = \frac{3 + 2\sqrt{3}}{3}.
\end{align*}
所以 \(\max_{a \in \mathbb{R}} h(a) = \frac{2\sqrt{3} + 3}{3}\)。从而
\begin{align*}
\max_{a \in \mathbb{R}} \frac{g^2(1)}{\int_0^1 g^2(x) \, \mathrm{d}x} = \max_{a \in \mathbb{R}} 3 \left( 1 + \frac{a}{a^2 - 3a + 3} \right) = 3 \left( 1 + \max_{a \in \mathbb{R}} h(a) \right) = 6 + 2\sqrt{3} > 4.
\end{align*}
综上,取 \(g(x) = \sqrt{3}x - 1\),就能得到
\begin{align*}
\int_0^1 \left| f''(x) \right|^2 \, \mathrm{d}x \geqslant \frac{g^2(1)}{\int_0^1 g^2(x) \, \mathrm{d}x} = 6 + 2\sqrt{3} > 4.
\end{align*}
实际上,\(6 + 2\sqrt{3}\) 就是原不等式的最佳下界。只需要再将 \(g(x) = \sqrt{3}x - 1\) 代入最开始的 Cauchy 不等式验证即可。
\end{remark}
\begin{proof}
令 \(g(x) = \sqrt{3}x - 1\),则
\begin{align*}
g''(x) \equiv 0, \quad g(1) = \sqrt{3} - 1.
\end{align*}
于是由 Cauchy 不等式及条件可得
\begin{align*}
&\int_0^1 \left| f''(x) \right|^2 \, \mathrm{d}x \int_0^1 g^2(x) \, \mathrm{d}x \geqslant \left( \int_0^1 f''(x) g(x) \, \mathrm{d}x \right)^2 \\
&\xlongequal{\text{分部积分}} \left( g(1) - \int_0^1 f'(x) g'(x) \, \mathrm{d}x \right)^2 
\xlongequal{\text{分部积分}} \left( g(1) + \int_0^1 f(x) g''(x) \, \mathrm{d}x \right)^2.
\end{align*}
从而
\begin{align*}
\int_0^1 \left| f''(x) \right|^2 \, \mathrm{d}x &\geqslant \frac{g^2(1)}{\int_0^1 g^2(x) \, \mathrm{d}x} = \frac{(\sqrt{3} - 1)^2}{\int_0^1 (\sqrt{3}x - 1)^2 \, \mathrm{d}x} = 6 + 2\sqrt{3} > 4.
\end{align*}

\end{proof}

\begin{example}
设 \(f \in C^2[0,2]\),证明:
\begin{align*}
\int_{0}^{2} |f''(x)|^2 \mathrm{d}x \geqslant \frac{3}{2} [f(0) + f(2) - 2f(1)]^2.
\end{align*} 
\end{example}
\begin{remark}
不妨设 \(f(0) = f(2) = 0\),\(f(1) = 1\) 的原因:

(1) 当 \(f(0) + f(2) - 2f(1) = 0\) 时,结论显然成立。  

(2) 当 \(f(0) + f(2) - 2f(1) \ne 0\) 时,则待定 \(a, b, c\),令 \(g(x) = c f(x) - a x - b\),希望 \(g(0) = g(2) = 0\),\(g(1) = 1\),即
\begin{align}
\begin{pmatrix}
-2 & -1 & f(2) \\
0 & -1 & f(0) \\
-1 & -1 & f(1)
\end{pmatrix}
\begin{pmatrix}
a \\
b \\
c
\end{pmatrix}
=
\begin{pmatrix}
0 \\
0 \\
1
\end{pmatrix}. \label{eq:100.8}
\end{align}
注意到上述方程的系数行列式为
\begin{align*}
\begin{vmatrix}
-2 & -1 & f(2) \\
0 & -1 & f(0) \\
-1 & -1 & f(1)
\end{vmatrix}
= f(0) + f(2) - 2f(1) \ne 0.
\end{align*}
故由 Cramer 法则可知,存在唯一的解 \(a = a_0\),\(b = b_0\),\(c = c_0\) 满足方程组 \eqref{eq:100.8}。即 \(g(x) = c_0 f(x) - a_0 x - b_0\) 满足 \(g(0) = g(2) = 0\),\(g(1) = 1\)。

下证不妨设成立。假设原不等式已经对 \(f(0) = f(2) = 0\),\(f(1) = 1\) 的的情况成立,则对一般的 \(f(x)\) 而言,令 \(g(x) = c_0 f(x) - a_0 x - b_0\),显然 \(g''(x) = c_0 f''(x)\),并且由上述推导可知 \(g(0) = g(2) = 0\),\(g(1) = 1\)。从而此时由假设可得
\begin{align*}
\int_0^2 \left| g''(x) \right|^2 \, \mathrm{d}x \geqslant \frac{3}{2} [g(0) + g(2) - 2g(1)]^2.
\end{align*}
于是
\begin{align*}
&\left| c_0 \right|^2 \int_0^2 \left| f''(x) \right|^2 \, \mathrm{d}x = \int_0^2 \left| g''(x) \right|^2 \, \mathrm{d}x \geqslant \frac{3}{2} \left[ g(0) + g(2) - 2g(1) \right]^2 \\
&= \frac{3}{2} \left[ \left( c_0 f(0) - b_0 \right) + \left( c_0 f(2) - 2a_0 - b_0 \right) - 2 \left( c_0 f(1) - a_0 - b_0 \right) \right]^2 \\
&= \frac{3 \left| c_0 \right|^2}{2} \left[ f(0) + f(2) - 2f(1) \right]^2.
\end{align*}
故
\begin{align*}
\int_0^2 \left| f''(x) \right|^2 \, \mathrm{d}x \geqslant \frac{3}{2} \left[ f(0) + f(2) - 2f(1) \right]^2.
\end{align*}
因此不妨设成立。

于是我们可以不妨设 \(f(0) = f(2) = 0\),\(f(1) = 1\),否则用 \(c_0f(x) - a_0x - b_0\) 代替即可。从而只须证
\begin{align*}
\int_0^2 \left| f''(x) \right|^2 \, \mathrm{d}x \geqslant \frac{3}{2} \left[ f(0) + f(2) - 2f(1) \right]^2 = 6.
\end{align*}
显然要利用 Cauchy 不等式,因此待定 \(g(x)\),由 Cauchy 不等式可得
\begin{align*}
\int_0^2 \left| f''(x) \right|^2 \, \mathrm{d}x \int_0^2 g^2(x) \, \mathrm{d}x \geqslant \left( \int_0^2 f''(x) g(x) \, \mathrm{d}x \right)^2.
\end{align*}
对上式右边分部积分可得
\begin{align}
\left( \int_0^2 f''(x) g(x) \, \mathrm{d}x \right)^2 = \left( f'(2) g(2) - f'(0) g(0) - \int_0^2 f'(x) g'(x) \, \mathrm{d}x \right)^2. \label{eq:100.9}
\end{align}
于是我们希望 \(g'(x) \equiv C\),其中 \(C\) 为某一常数,\(g(2) = g(0) = 0\),从而设 \(g(x)\) 为一次函数,即设 \(g(x) = px + q\)。从而由 \(g(2) = g(0) = 0\) 可得 \(q = p = 0\),进而 \(g \equiv 0\),显然不行!

因此我们猜测 \(g(x)\) 为满足 \(g(2) = g(0) = 0\) 的分段一次函数,则待定 \(m\),令
\[
g(x) = 
\begin{cases}
x, & 0 \leqslant x \leqslant 1 \\
m(x - 2), & 1 < x \leqslant 2
\end{cases}.
\]
(因为有 \(f(1) = 1\) 这个条件,所以选先 \(x = 1\) 为分段点)
又由 \eqref{eq:100.9} 式可知需要 \(f\) 和 \(g\) 都连续才能分部积分,因此 \(g\) 在 \(x = 1\) 处要连续,故 \(m = -1\),即
\[
g(x) = 
\begin{cases}
x, & 0 \leqslant x \leqslant 1 \\
2 - x, & 1 < x \leqslant 2
\end{cases}.
\]
再代入\eqref{eq:100.9} 式中验证即可得到证明。
\end{remark}
\begin{proof}
不妨设 \(f(0) = f(2) = 0\),\(f(1) = 1\),否则用 \(c_0f(x) - a_0x - b_0\) 代替即可。令
\[
g(x) = 
\begin{cases}
x, & 0 \leqslant x \leqslant 1 \\
2-x, & 1 < x \leqslant 2
\end{cases},
\]
则
\begin{align*}
\int_0^2 g^2(x) \, \mathrm{d}x = \frac{2}{3}, \quad \left( \int_0^2 f'(x) g'(x) \, \mathrm{d}x \right)^2 = \left( f(1)-f(0)-f(2)+f(1) \right)^2 = 4.
\end{align*}
于是由 Cauchy 不等式可得
\begin{align*}
&\int_0^2 \left| f''(x) \right|^2 \, \mathrm{d}x \int_0^2 g^2(x) \, \mathrm{d}x \geqslant \left( \int_0^2 f''(x) g(x) \, \mathrm{d}x \right)^2 
\xlongequal{\text{分部积分}} \left( \int_0^2 f'(x) g'(x) \, \mathrm{d}x \right)^2 \\
&\Longleftrightarrow \frac{2}{3} \int_0^2 \left| f''(x) \right|^2 \, \mathrm{d}x \geqslant 4 
\Longleftrightarrow \int_0^2 \left| f''(x) \right|^2 \, \mathrm{d}x \geqslant 6.
\end{align*}

\end{proof}

\begin{example}
设 \(f \in C^1[0,1]\),\(f(0) = f(1) = -\frac{1}{6}\),证明
\begin{align*}
\int_0^1{|f' \left( x \right) |^2\mathrm{d}x}\geqslant 2\int_0^1{f\left( x \right) \mathrm{d}x}+\frac{1}{4}.
\end{align*}
\end{example}
\begin{note}
注意到不等式左右不是齐次的,不是自然的不等式,但我们一定可以得到一个自然的不等式。
\end{note}
\begin{remark}
显然要利用 Cauchy 不等式,待定 \(g(x)\),由 Cauchy 不等式和条件可得
\begin{align}
\int_0^1{\left| f' (x) \right|^2\,\mathrm{d}x\int_0^1{g^2(x)\,\mathrm{d}x}}\geqslant \left( \int_0^1{f'(x)g(x)\,\mathrm{d}x} \right) ^2\xlongequal{\text{分部积分}}\left( -\frac{1}{6}g(1)+\frac{1}{6}g(0)-\int_0^1{f(x)g' (x)\,\mathrm{d}x} \right) ^2. \label{eq:100.10}
\end{align}
将上式与要证不等式对比,于是我们希望 \(g'(x) = C\),其中 \(C\) 为某一常数。这样才能使
\begin{align*}
\int_0^1 f(x) g'(x) \, \mathrm{d}x = C \int_0^1 f(x) \, \mathrm{d}x,
\end{align*}
进而不等式右边才会出现我们需要的 \(\int_0^1 f(x) \, \mathrm{d}x\)。从而待定的 \(g(x)\) 为线性函数。
设 \(g(x) = ax + c\),\(a \ne 0\),进而不妨设 \(g(x) = x + c\),否则用 \(\frac{1}{a}g\) 代替 \(g\) 仍有不等式 \eqref{eq:100.10}(因为不等式两边齐次)。
于是不等式 \eqref{eq:100.10} 可化为
\begin{align*}
\frac{3c^2 + 3c + 1}{3} \int_0^1 \left| f'(x) \right|^2 \, \mathrm{d}x &= \int_0^1 \left| f'(x) \right|^2 \, \mathrm{d}x \int_0^1 (x + c)^2 \, \mathrm{d}x  \\
&\geqslant \left( -\frac{1}{6}(1 + c) + \frac{1}{6}c - \int_0^1 f(x) \, \mathrm{d}x \right)^2  \\
&= \left( \frac{1}{6} + \int_0^1 f(x) \, \mathrm{d}x \right)^2 
\end{align*}
\begin{align}
\Longleftrightarrow \int_0^1 \left| f'(x) \right|^2 \, \mathrm{d}x &\geqslant \frac{3}{3c^2 + 3c + 1} \left( \frac{1}{6} + \int_0^1 f(x) \, \mathrm{d}x \right)^2. \label{eq:100.11}
\end{align}
因此只需要找到一个合适的 \(c\),使得上述不等式右边满足
\begin{align}
\frac{3}{3c^2 + 3c + 1} \left( \frac{1}{6} + \int_0^1 f(x) \, \mathrm{d}x \right)^2 \geqslant 2 \int_0^1 f(x) \, \mathrm{d}x + \frac{1}{4}. \label{eq:100.12}
\end{align}
即对 \(\forall t = \int_0^1 f(x) \, \mathrm{d}x \in \mathbb{R}\),找到一个 \(c\),记 \(K = \frac{3}{3c^2 + 3c + 1} \in \mathbb{R}\),使得
\begin{align*}
K \left( \frac{1}{6} + t \right)^2 \geqslant 2t + \frac{1}{4} \Longleftrightarrow \Delta = \frac{12 - K}{3} \leqslant 0 \Longleftrightarrow K \geqslant 12.
\end{align*}
因此取 \(c = -\frac{1}{2}\),得 \(K = \frac{3}{3c^2 + 3c + 1} = 12\)。

综上,令 \(g(x) = x - \frac{1}{2}\),则由 \eqref{eq:100.11}和\eqref{eq:100.12} 式可知
\begin{align*}
\int_0^1 \left| f'(x) \right|^2 \, \mathrm{d}x \geqslant \frac{3}{3c^2 + 3c + 1} \left( \frac{1}{6} + \int_0^1 f(x) \, \mathrm{d}x \right)^2 \geqslant 2 \int_0^1 f(x) \, \mathrm{d}x + \frac{1}{4}.
\end{align*}
只需要将 \(g(x) = x - \frac{1}{2}\) 代入上述步骤进行验证即可得到证明。
\end{remark}
\begin{proof}
令 \(g(x) = x - \frac{1}{2}\),则
\begin{align*}
\int_0^1 g^2(x) \, \mathrm{d}x = \frac{1}{12}, \quad g(1) = \frac{1}{2}, \quad g(0) = -\frac{1}{2}.
\end{align*}
于是由 Cauchy 不等式和条件可得
\begin{align*}
&\int_0^1 \left| f'(x) \right|^2 \, \mathrm{d}x \int_0^1 g^2(x) \, \mathrm{d}x \geqslant \left( \int_0^1 f'(x) g(x) \, \mathrm{d}x \right)^2 
\xlongequal{\text{分部积分}} \left( \frac{1}{6} + \int_0^1 f(x) \, \mathrm{d}x \right)^2 \\
&\Longleftrightarrow \int_0^1 \left| f'(x) \right|^2 \, \mathrm{d}x \geqslant 12 \left( \frac{1}{6} + \int_0^1 f(x) \, \mathrm{d}x \right)^2.
\end{align*}
注意到 \(12 \left( \frac{1}{6} + t \right)^2 \geqslant 2t + \frac{1}{4}\) 对 \(\forall t \in \mathbb{R}\) 恒成立,故
\begin{align*}
\int_0^1 \left| f'(x) \right|^2 \, \mathrm{d}x \geqslant 12 \left( \frac{1}{6} + \int_0^1 f(x) \, \mathrm{d}x \right)^2 \geqslant 2 \int_0^1 f(x) \, \mathrm{d}x + \frac{1}{4}.
\end{align*}

\end{proof}

\begin{example}[(一类)Hilbert不等式]\label{example:(一类)Hilbert(希尔伯特)不等式}
\begin{enumerate}
\item 设 $f(x), g(x)$ 在 $[0, +\infty)$ 中可积,证明:
\begin{align*}
\iint_{[0, +\infty)} \frac{f(x)g(y)}{(\sqrt{x} + \sqrt{y})^2} \mathrm{d}x\mathrm{d}y \leqslant 2 \sqrt{\int_0^\infty f^2(x) \mathrm{d}x \int_0^\infty g^2(x) \mathrm{d}x}.
\end{align*}

\item 设 $N$ 为正整数,$a_k, b_k$ 为实数,证明:
\begin{align*}
\sum_{m,n=1}^N \frac{a_m b_n}{(\sqrt{m} + \sqrt{n})^2} \leqslant  2 \sqrt{\sum_{m=1}^N a_m^2 \cdot \sum_{n=1}^N b_n^2}.
\end{align*}
\end{enumerate}
\end{example}
\begin{proof}
\begin{enumerate}
\item 

\item 
\end{enumerate}

\end{proof}




























\end{document}
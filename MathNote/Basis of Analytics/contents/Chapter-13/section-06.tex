\documentclass[../../main.tex]{subfiles}
\graphicspath{{\subfix{../../image/}}} % 指定图片目录,后续可以直接使用图片文件名。

% 例如:
% \begin{figure}[h]
% \centering
% \includegraphics{image-01.01}
% \label{fig:image-01.01}
% \caption{图片标题}
% \end{figure}

\begin{document}

\section{Cauchy不等式的应用}

\begin{example}
设\(f \in C^1[0, 1]\),解决下列问题.
\begin{enumerate}
\item 若\(f(0) = 0\),证明:
\begin{align*}
\int_{0}^{1}|f(x)|^{2}dx \leqslant \frac{1}{2}\int_{0}^{1}|f'(x)|^{2}dx.
\end{align*}

\item 若\(f(0) = f(1) = 0\),证明:
\begin{align*}
\int_{0}^{1}|f(x)|^{2}dx \leqslant \frac{1}{8}\int_{0}^{1}|f'(x)|^{2}dx.
\end{align*}
\end{enumerate}
\end{example}
\begin{remark}
牛顿莱布尼兹公式也可以看作带积分余项的插值公式(插一个点).
\end{remark}
\begin{proof}
\begin{enumerate}
\item 由牛顿莱布尼兹公式可知
\begin{align*}
f(x) = f(0) + \int_{0}^{x}f'(y)dy=\int_{0}^{x}f'(y)dy.
\end{align*}
从而
\begin{align*}
|f(x)|^2 = \left| \int_{0}^{x}f'(y)dy \right|^2 \leqslant \int_{0}^{x}1^2dy \int_{0}^{x}|f'(y)|^2dy = x\int_{0}^{x}|f'(y)|^2dy \leqslant x\int_{0}^{1}|f'(y)|^2dy.
\end{align*}
于是对上式两边同时积分可得
\begin{align*}
\int_{0}^{1}|f(x)|^2dx \leqslant \int_{0}^{1}xdx \int_{0}^{1}|f'(y)|^2dy = \frac{1}{2}\int_{0}^{1}|f'(y)|^2dy.
\end{align*}

\item 由牛顿莱布尼兹公式(带积分型余项的插值公式)可得
\begin{align*}
f(x) = \int_{0}^{x}f(y)dy, x \in \left[0, \frac{1}{2}\right]; \quad f(x) = \int_{x}^{1}f'(y)dy, x \in \left[\frac{1}{2}, 1\right].
\end{align*}
从而
\begin{align*}
|f(x)|^2 = \left| \int_{0}^{x}f'(y)dy \right|^2 \leqslant \int_{0}^{x}1^2dy\int_{0}^{x}|f'(y)|^2dy = x\int_{0}^{x}|f'(y)|^2dy \leqslant x\int_{0}^{\frac{1}{2}}|f'(y)|^2dy, x \in \left[0, \frac{1}{2}\right].
\end{align*}
\begin{align*}
|f(x)|^2 = \left| \int_{x}^{1}f'(y)dy \right|^2 \leqslant \int_{0}^{x}1^2dy\int_{x}^{1}|f'(y)|^2dy \leqslant (1 - x)\int_{\frac{1}{2}}^{1}|f'(y)|^2dy, x \in \left[\frac{1}{2}, 1\right].
\end{align*}
于是对上面两式两边同时积分可得
\begin{align*}
\int_{0}^{\frac{1}{2}}|f(x)|^2dx \leqslant \int_{0}^{\frac{1}{2}}xdx\int_{0}^{\frac{1}{2}}|f'(y)|^2dy = \frac{1}{8}\int_{0}^{\frac{1}{2}}|f'(y)|^2dy.
\end{align*}
\begin{align*}
\int_{\frac{1}{2}}^{1}|f(x)|^2dx \leqslant \int_{\frac{1}{2}}^{1}(1 - x)dx\int_{\frac{1}{2}}^{1}|f'(y)|^2dy = \frac{1}{8}\int_{0}^{\frac{1}{2}}|f'(y)|^2dy.
\end{align*}
将上面两式相加得
\begin{align*}
\int_{0}^{1}|f(x)|^2dx \leqslant \frac{1}{8}\int_{0}^{1}|f'(y)|^2dy.
\end{align*}
\end{enumerate}
\end{proof}

\begin{example}[$\,\,$opial不等式]\label{example:opial不等式}

\textbf{特例:}
\begin{enumerate}
\item 设\(f \in C^1[a,b]\)且\(f(a) = 0\),证明
\begin{align*}
\int_{a}^{b}|f(x)f'(x)|dx \leqslant \frac{b - a}{2}\int_{a}^{b}|f'(x)|^{2}dx.
\end{align*}

\item 设\(f \in C^1[a,b]\)且\(f(a) = 0\),\(f(b) = 0\),证明
\begin{align*}
\int_{a}^{b}|f(x)f'(x)|dx \leqslant \frac{b - a}{4}\int_{a}^{b}|f'(x)|^{2}dx.
\end{align*} 
\end{enumerate}

\textbf{一般情况:}
\begin{enumerate}
\item 设\(f \in C^1[a,b], p\geqslant0, q\geqslant1\)且\(f(a) = 0\). 证明
\begin{align}
\int_{a}^{b}|f(x)|^{p}|f'(x)|^{q}dx \leqslant \frac{q(b - a)^{p}}{p + q}\int_{a}^{b}|f'(x)|^{p + q}dx. \label{equation----:::16.37}
\end{align}

\item 若还有\(f(b) = 0\). 证明
\begin{align}
\int_{a}^{b}|f(x)|^{p}|f'(x)|^{q}dx \leqslant \frac{q(b - a)^{p}}{(p + q)2^{p}}\int_{a}^{b}|f'(x)|^{p + q}dx. \label{equation----:::16.38}
\end{align}
\end{enumerate}
\end{example}
\begin{note}
说明了证明的想法就是注意变限积分为整体凑微分.
\end{note}
\begin{proof}
{\heiti 特例:}
\begin{enumerate}
\item 令\(F(x) \triangleq \int_{a}^{x}|f'(y)|dy\),则\(F'(x) = |f'(x)|\),\(F(a) = 0\)。从而
\begin{align*}
f(x) = \int_{0}^{x}f'(y)dy \Rightarrow |f(x)| \leqslant \int_{a}^{x}|f'(y)|dy = F(x).
\end{align*}
于是
\begin{align*}
\int_{a}^{b}|f(x)f'(x)|dx \leqslant \int_{a}^{b}F(x)F'(x)dx = \frac{1}{2}F^2(x)\big|_{a}^{b} = \frac{1}{2}F^2(b) = \frac{1}{2}\left( \int_{a}^{b}|f'(y)|dx \right)^2 \\
\overset{\text{Cauchy不等式}}{\leqslant} \frac{1}{2}\int_{a}^{b}1^2dx \int_{a}^{b}|f'(y)|^2dx = \frac{b - a}{2}\int_{a}^{b}|f'(y)|^2dx.
\end{align*}

\item 由第1问可知
\begin{align*}
\int_{a}^{\frac{a + b}{2}}|f(x)f'(x)|dx \leqslant \frac{\frac{a + b}{2} - a}{2}\int_{a}^{\frac{a + b}{2}}|f'(y)|^2dy = \frac{b - a}{4}\int_{a}^{\frac{a + b}{2}}|f'(y)|^2dy.
\end{align*}
\begin{align*}
\int_{\frac{a + b}{2}}^{b}|f(x)f'(x)|dx \leqslant \frac{\frac{a + b}{2} - a}{2}\int_{\frac{a + b}{2}}^{b}|f'(y)|^2dy = \frac{b - a}{4}\int_{\frac{a + b}{2}}^{b}|f'(y)|^2dy.
\end{align*}
将上面两式相加可得
\begin{align*}
\int_{a}^{b}|f(x)f'(x)|dx \leqslant \frac{b - a}{4}\int_{a}^{b}|f'(y)|^2dy.
\end{align*}
\end{enumerate}

{\heiti 一般情况:}
\begin{enumerate}
\item 只证\(q>1\). \(q = 1\)可类似得到. 考虑
\[f(x)=\int_{a}^{x}f'(y)dy, F(x)=\int_{a}^{x}|f'(y)|^{q}dy.\]
则由Holder不等式, 我们知道
\begin{align*}
|f(x)|^{p}&\leqslant \left(\int_{a}^{x}|f'(y)|dy\right)^{p} \leqslant \left(\int_{a}^{x}|f'(y)|^{q}dy\right)^{\frac{p}{q}}\left(\int_{a}^{x}1^{\frac{q}{q - 1}}dy\right)^{\frac{p(q - 1)}{q}} = F^{\frac{p}{q}}(x)(x - a)^{\frac{p(q - 1)}{q}},
\end{align*}
这里\(\frac{1}{p}+\frac{1}{q}=1\).
于是
\begin{align*}
\int_{a}^{b}|f(x)|^{p}|f'(x)|^{q}dx &\leqslant \int_{a}^{b}F^{\frac{p}{q}}(x)(x - a)^{\frac{p(q - 1)}{q}}|f'(x)|^{q}dx = \int_{a}^{b}F^{\frac{p}{q}}(x)(x - a)^{\frac{p(q - 1)}{q}}dF(x)\\
&\leqslant (b - a)^{\frac{p(q - 1)}{q}}\int_{a}^{b}F^{\frac{p}{q}}(x)dF(x) = \frac{q}{q + p}(b - a)^{\frac{p(q - 1)}{q}}F^{\frac{p + q}{q}}(b)\\
&=\frac{q}{q + p}(b - a)^{\frac{p(q - 1)}{q}}\left(\int_{a}^{b}|f'(y)|^{q}dy\right)^{\frac{p + q}{q}}\\
&\stackrel{Cauchy\text{不等式}}{\leqslant} \frac{q}{q + p}(b - a)^{\frac{p(q - 1)}{q}}\left(\int_{a}^{b}|f'(y)|^{q(\frac{p + q}{q})}dy\right)^{\frac{q}{q + p}}\left(\int_{a}^{b}1^{(\frac{p + q}{q - 1})}dy\right)^{\frac{q - 1}{q + p}}\\
&=\frac{q(b - a)^{p}}{p + q}\int_{a}^{b}|f'(y)|^{p + q}dy,
\end{align*}
这就证明了不等式\eqref{equation----:::16.37}.

\item 由第一问得
\[\int_{a}^{\frac{a + b}{2}}|f(x)|^{p}|f'(x)|^{q}dx \leqslant \frac{q(b - a)^{p}}{(p + q)2^{p}}\int_{a}^{\frac{a + b}{2}}|f'(x)|^{p + q}dx,\]
对称得
\[\int_{\frac{a + b}{2}}^{b}|f(x)|^{p}|f'(x)|^{q}dx \leqslant \frac{q(b - a)^{p}}{(p + q)2^{p}}\int_{\frac{a + b}{2}}^{b}|f'(x)|^{p + q}dx.\] 
故上面两式相加得到\eqref{equation----:::16.3}式.
\end{enumerate}
\end{proof}

\begin{example}
设\(f \in C[0,1]\)满足\(\int_{0}^{1}f(x)dx = 0\),证明:
\begin{align*}
\left(\int_{0}^{1}xf(x)dx\right)^2 \leqslant \frac{1}{12}\int_{0}^{1}f^{2}(x)dx. \label{equation----:::::16.28}
\end{align*}
\end{example}
\begin{note}
从条件\(\int_{0}^{1}f(x)dx = 0\)来看,我们待定\(a \in \mathbb{R}\),一定有
\begin{align*}
\int_{0}^{1}xf(x)dx = \int_{0}^{1}(x - a)f(x)dx.
\end{align*}
然后利用Cauchy不等式得
\begin{align*}
\left(\int_{0}^{1}(x - a)f(x)dx\right)^2 \leqslant \int_{0}^{1}(x - a)^2dx\int_{0}^{1}f^{2}(x)dx.
\end{align*}
为了使得不等式最精确,我们自然希望\(\int_{0}^{1}(x - a)^2dx\)达到最小值. 读者也可以直接根据对称性猜测出\(a = \frac{1}{2}\)就是达到最小值的\(a\).
\end{note}
\begin{proof}
利用 Cauchy 不等式得
\begin{align*}
\frac{1}{12}\int_{0}^{1}f^{2}(x)dx&=\int_{0}^{1}\left(x - \frac{1}{2}\right)^2dx\int_{0}^{1}f^{2}(x)dx\\
&\geqslant \left(\int_{0}^{1}\left(x - \frac{1}{2}\right)f(x)dx\right)^2\\
&=\left(\int_{0}^{1}xf(x)dx\right)^2,
\end{align*}
这就证明了\eqref{equation----:::::16.28}式.
\end{proof}

\begin{example}
设\(f \in C^1[0,1]\),\(\int_{\frac{1}{3}}^{\frac{2}{3}}f(x)dx = 0\),证明
\begin{align*}
\int_0^1{|f' (x)|^2dx}\geqslant 27\left( \int_0^1{f(x)dx} \right) ^2.
\end{align*}
\end{example}
\begin{note}
为了分部积分提供\(0\)边界且求导之后不留下东西,设\(g(0) = g(1) = 0\)且\(g\)是一次函数,这不可能,于是只能是分段函数\(g(x)=\begin{cases}x - 1, & c\leqslant x\leqslant 1 \\ x, & 0\leqslant x\leqslant c\end{cases}\)。为了让\(g\)连续会发现\(c = c - 1\),这不可能。结合\(\int_{\frac{1}{3}}^{\frac{2}{3}}f(x)dx = 0\),所以我们插入一段来使得连续,因此真正构造的函数为
\[g(x)=\begin{cases}x - 1, & \frac{2}{3}\leqslant x\leqslant 1 \\ 1 - 2x, & \frac{1}{3}\leqslant x\leqslant \frac{2}{3} \\ x, & 0\leqslant x\leqslant \frac{1}{3}\end{cases}.\]
\end{note}
\begin{proof}
令
\[g(x)=\begin{cases}x - 1, & \frac{2}{3}\leqslant x\leqslant 1 \\ 1 - 2x, & \frac{1}{3}\leqslant x\leqslant \frac{2}{3} \\ x, & 0\leqslant x\leqslant \frac{1}{3}\end{cases}.\]
于是由 Cauchy 不等式,我们有
\begin{align*}
&\int_{0}^{1}|f'(x)|^{2}dx\int_{0}^{1}|g(x)|^{2}dx\geqslant \left(\int_{0}^{1}f'(x)g(x)dx\right)^{2}\xlongequal{\text{分部积分}}\left(\int_{0}^{1}f(x)g'(x)dx\right)^{2}\\
&=\left(\int_{0}^{\frac{1}{3}}f(x)dx - 2\int_{\frac{1}{3}}^{\frac{2}{3}}f(x)dx+\int_{\frac{2}{3}}^{1}f(x)dx\right)^{2}\\
&=\left(\int_{0}^{\frac{1}{3}}f(x)dx+\int_{\frac{1}{3}}^{\frac{2}{3}}f(x)dx+\int_{\frac{2}{3}}^{1}f(x)dx\right)^{2}=\left(\int_{0}^{1}f(x)dx\right)^{2},
\end{align*}
结合\(\int_{0}^{1}|g(x)|^{2}dx = \frac{1}{27}\),这就完成了证明.
\end{proof}

\begin{example}
设\(f \in C[a,b]\cap D(a,b)\)且\(f(a) = f(b) = 0\)且\(f\)不恒为\(0\),证明存在一点\(\xi \in (a,b)\)使得
\begin{align*}
|f'(\xi)| > \frac{4}{(b - a)^2}\left|\int_{a}^{b}f(x)dx\right|.
\end{align*} 
\end{example}
\begin{remark}
\hypertarget{example0.5----:不妨设的原因}{不妨设}\(\int_{a}^{b}f(x)dx > 0\)的原因:若\(\int_{a}^{b}f(x)dx < 0\)则用\(-f\)代替\(f\),\(\int_{a}^{b}f(x)dx = 0\)是平凡的。
\end{remark}
\begin{proof}
反证,若\(|f'(x)| \leqslant \frac{4}{(b - a)^2}\left|\int_{a}^{b}f(x)dx\right| \triangleq M\),则\hyperlink{example0.5----:不妨设的原因}{不妨设\(\int_{a}^{b}f(x)dx > 0\)},由 Hermite 插值定理可知,存在\(\theta_1 \in (a, x)\),\(\theta_2 \in (x, b)\),使得
\begin{align*}
f(x) = f(a) + f'(\theta_1)(x - a) \leqslant M(x - a), \forall x \in \left[a, \frac{a + b}{2}\right].
\end{align*}
\begin{align*}
f(x) = f(b) + f'(\theta_2)(x - b) \leqslant -M(x - b) = M(b - x), \forall x \in \left[\frac{a + b}{2}, b\right].
\end{align*}
从而
\begin{align*}
\int_{a}^{b}|f(x)|dx \leqslant \int_{a}^{\frac{a + b}{2}}M(x - a)dx + \int_{\frac{a + b}{2}}^{b}M(b - x)dx = \frac{M(b - a)^2}{4} = \int_{a}^{b}|f(x)|dx.
\end{align*}
于是结合\(f\)的连续性可得
\begin{align*}
\int_{a}^{\frac{a + b}{2}}f(x)dx = \int_{a}^{\frac{a + b}{2}}M(x - a)dx \Rightarrow f(x) = M(x - a), \forall x \in \left[a, \frac{a + b}{2}\right].
\end{align*}
\begin{align*}
\int_{\frac{a + b}{2}}^{b}f(x)dx = \int_{\frac{a + b}{2}}^{b}M(b - x)dx \Rightarrow f(x) = M(b - x), \forall x \in \left[\frac{a + b}{2}, b\right].
\end{align*}
故\(f\)在\(x = \frac{a + b}{2}\)处不可导,这与\(f \in D(a, b)\)矛盾!
\end{proof}

















\end{document}
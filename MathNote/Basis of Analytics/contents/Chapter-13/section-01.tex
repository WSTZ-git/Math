\documentclass[../../main.tex]{subfiles}
\graphicspath{{\subfix{../../image/}}} % 指定图片目录,后续可以直接使用图片文件名。

% 例如:
% \begin{figure}[H]
% \centering
% \includegraphics[scale=0.4]{图.png}
% \caption{}
% \label{figure:图}
% \end{figure}
% 注意:上述\label{}一定要放在\caption{}之后,否则引用图片序号会只会显示??.

\begin{document}

\section{著名积分不等式}

\begin{theorem}[Young不等式初等形式]\label{theorem:Young不等式初等形式}
设 $(x_i)_{i = 1}^n \subset [0, +\infty)$,$(p_i)_{i = 1}^n \subset (1, +\infty)$,$\sum_{i = 1}^n \frac{1}{p_i} = 1$, 则有
\begin{align*}
\prod_{i = 1}^n x_i \leqslant \sum_{i = 1}^n \frac{x_i^{p_i}}{p_i}.
\end{align*}
且等号成立条件为所有 $x_i$,$i = 1,2,\cdots,n$ 相等.
\end{theorem}
\begin{note}
最常用的是Young不等式的二元情形:

对任何$a,b\geq0,\frac{1}{p}+\frac{1}{q}=1,p>1$有$ab\leqslant \frac{a^p}{p}+\frac{b^q}{q}.$
\end{note}
\begin{proof}
不妨设 $x_i \neq 0$,$(i = 1,2,\cdots,n)$. 本结果可以取对数用\hyperref[theorem:Jensen不等式1111]{Jensen不等式}证明, 即
\begin{align*}
\prod_{i = 1}^n x_i \leqslant \sum_{i = 1}^n \frac{x_i^{p_i}}{p_i}\Leftrightarrow \sum_{i = 1}^n \ln x_i \leqslant \ln \left(\sum_{i = 1}^n \frac{x_i^{p_i}}{p_i}\right) \Leftrightarrow \sum_{i = 1}^n \frac{1}{p_i} \ln x_i^{p_i} \leqslant \ln \left(\sum_{i = 1}^n \frac{x_i^{p_i}}{p_i}\right),
\end{align*}
而最后一个等价之后就是 $\ln$ 的上凸性结合\hyperref[theorem:Jensen不等式1111]{Jensen不等式}给出.
\end{proof}

\begin{definition}
\begin{enumerate}[(1)]
\item $d\mu = g(x)\mathrm{d}x$, 这里 $g$ 是一个在区间上内闭黎曼可积的函数.
\item 若 $E \subset \mathbb{Z}$, 则 $\int_E f(x) d\mu = \sum_{n \in E} f(n)$.
\end{enumerate}
\end{definition}

\begin{theorem}[Cauchy不等式]\label{theorem:Cauchy不等式(一般版本)}
\begin{align*}
\left(\int_E f(x)g(x) d\mu\right)^2 \leqslant \int_E |f(x)|^2 d\mu \int_E |g(x)|^2 d\mu.
\end{align*} 
\end{theorem}
\begin{proof}
只需证
\begin{align*}
\int_E |f(x)g(x)| d\mu \leqslant \sqrt{\int_E |f(x)|^2 d\mu \int_E |g(x)|^2 d\mu}.
\end{align*}
当 $\int_E |f(x)| d\mu$ 或 $\int_E |g(x)| d\mu = 0$ 时, 不等式右边为 $0$, 结论显然成立.

当 $\int_E |f(x)| d\mu \ne 0$ 且 $\int_E |g(x)| d\mu \ne 0$ 时, 不妨设 $\int_E |f(x)|^2 d\mu = \int_E |g(x)|^2 d\mu = 1$, 否则, 用 $\frac{f(x)}{\sqrt{\int_E |f(x)|^2 d\mu}}$ 代替 $f(x)$, $\frac{g(x)}{\sqrt{\int_E |g(x)|^2 d\mu}}$ 代替 $g(x)$即可.
利用\hyperref[theorem:Young不等式初等形式]{Young不等式}可得
\begin{align*}
\int_E |f(x)||g(x)| d\mu \leqslant \int_E \frac{|f(x)|^2 + |g(x)|^2}{2} d\mu = \frac{1}{2} + \frac{1}{2} = 1.
\end{align*}
等号成立当且仅当存在不全为零的$c_1,c_2$,使得$c_1f(x)+c_2g(x)=0.$
\end{proof}

\begin{theorem}[Jensen不等式(积分形式)]\label{theorem:Jensen不等式积分形式}
设 $\varphi$ 是下凸函数且 $p(x) \geqslant 0$, $\int_a^b p(x) \mathrm{d}x > 0$, 则在有意义时, 必有
\begin{align}\label{equation--Jensen不等式积分形式16.11}
\varphi\left(\frac{\int_a^b p(x)f(x)\mathrm{d}x}{\int_a^b p(x)\mathrm{d}x}\right) \leqslant \frac{\int_a^b p(x)\varphi(f(x))\mathrm{d}x}{\int_a^b p(x)\mathrm{d}x}.
\end{align}
\end{theorem}
\begin{note}
1. 类似的对上凸函数, 不等式\eqref{equation--Jensen不等式积分形式16.11}反号.

2.一般情况可利用下凸函数可以被 $C^2$ 的下凸函数逼近得到, 例如定理 Bernstein 多项式保凸性一致逼近. 

3.Jensen不等式(积分形式)考试中不能直接使用,需要证明.
\end{note}
\begin{proof}
为书写简便, 我们记 $d\mu = \frac{p(x)}{\int_a^b p(y)\mathrm{d}y}\mathrm{d}x$, 那么有 $\int_a^b 1d\mu = 1$. 于是我们记 $x_0 = \int_a^b f(x)d\mu$ 并利用下凸函数恒在切线上方
\begin{align*}
\varphi(x) \geqslant \varphi(x_0) + \varphi'(x_0)(x - x_0),
\end{align*}
就有
\begin{align*}
\int_a^b \varphi(f(x))d\mu \geqslant \int_a^b [\varphi(x_0) + \varphi'(x_0)(f(x) - x_0)]d\mu = \varphi(x_0) = \varphi\left(\int_a^b f(x)d\mu\right),
\end{align*}
这就完成了证明. 
\end{proof}

\begin{example}
对连续正值函数 $f$, 我们有
\begin{align*}
\ln\left(\frac{1}{b - a}\int_a^b f(x)\mathrm{d}x\right) \geqslant \frac{1}{b - a}\int_a^b \ln f(x)\mathrm{d}x.
\end{align*} 
\end{example}
\begin{proof}
令 $d\mu = \frac{1}{b - a}\mathrm{d}x$, 则 $\int_a^b d\mu = 1$, 再令 $x_0 \triangleq \int_a^b f(x) d\mu>0$, 则由 $\ln x$ 的上凸性可知
\begin{align*}
\ln x \leqslant \ln x_0 + \frac{1}{x_0}(x - x_0), \forall x > 0.
\end{align*}
从而
\begin{align*}
\int_a^b \ln f(x) d\mu &\leqslant \int_a^b \ln x_0 d\mu + \frac{1}{x_0}\int_a^b (f(x) - x_0) d\mu \\
&= \ln x_0 + \frac{1}{x_0}\left(\int_a^b f(x) d\mu - x_0\int_a^b d\mu\right) \\
&= \ln x_0 = \ln \int_a^b f(x) d\mu.
\end{align*}
故结论得证.
\end{proof}

\begin{theorem}[Hold不等式]\label{theorem:Hold(赫尔德)不等式(积分形式)}
设 $V$ 是 $\mathbb{R}^n$ 中有体积的有界集,$f$ 和 $g$ 都在 $V$ 上可积,又设 $p,q$ 是满足 $\frac{1}{p} + \frac{1}{q} = 1$ 的正数,且$p>1$,则有
\begin{align*}
\int_V |f(x)g(x)| \, dx \leqslant \left( \int_V |f(x)|^p \, dx \right)^{\frac{1}{p}} \left( \int_V |g(x)|^q \, dx \right)^{\frac{1}{q}}
\end{align*}
当且仅当 $\frac{f^p(x)}{g^q(x)}$ 几乎处处为同一个常数时取等(若一个取零,则另一个也取零).
\end{theorem}
\begin{remark}
这是最重要的基本结论了(必须掌握),很多需要“调幂次”的积分不等式,都得用赫尔德不等式,同时这也是用来证明很多定理或者题目的工具,也包括下面两个,对于 $p \in (0,1)$ 的情况会有反向赫尔德不等式.
\end{remark}
\begin{proof}
不妨设$f,g\geqslant 0$,否则用$|f|,|g|$代替$f,g$.由\hyperref[theorem:Young不等式]{Young不等式}可知
\begin{align*}
f(x)g(x) \leqslant \frac{f^p(x)}{p}+\frac{g^q(x)}{q}.
\end{align*}
由于$f,g$在$V$上都可积,故可不妨设$\int_V f^p(x) \mathrm{d}x=\int_V g^q(x) \mathrm{d}x=1$,否则用$\frac{f}{\left( \int_V f^p(x) \mathrm{d}x \right)^{\frac{1}{p}}},\frac{g}{\left( \int_V g^q(x) \mathrm{d}x \right)^{\frac{1}{q}}}$代替$f,g$.从而
\begin{align*}
\int_V f(x)g(x) \mathrm{d}x \leqslant \frac{1}{p}\int_V f^p(x) \mathrm{d}x+\frac{1}{q}\int_V g^q(x) \mathrm{d}x=1=\left( \int_V f^p(x) \mathrm{d}x \right)^{\frac{1}{p}}\left( \int_V g^q(x) \mathrm{d}x \right)^{\frac{1}{q}}.
\end{align*}
如果上述不等式等号成立,那么
\begin{align*}
f(x)g(x) \leqslant \frac{f^p(x)}{p}+\frac{g^q(x)}{q}
\end{align*}
在$V$上几乎处处取等.根据\hyperref[theorem:Young不等式]{Young不等式}的取等条件可知,此即$\frac{f^p(x)}{g^q(x)}$几乎处处为一个常数(若一个取零,则另一个也取零).
\end{proof}

\begin{theorem}[Minkowski不等式]\label{theorem:Minkowski(闵可夫斯基)不等式}
若 $f$ 是 $[a,b] \times [c,d]$ 上的非负连续函数,则对 $p \geqslant 1$ 有(若 $p \in (0,1)$ 则不等式反向)
\begin{align*}
\left( \int_a^b \left( \int_c^d f(x,y) dy \right)^p dx \right)^{\frac{1}{p}} \leqslant \int_c^d \left( \int_a^b f^p(x,y) dx \right)^{\frac{1}{p}} dy.
\end{align*}
\end{theorem}
\begin{note}
证明的核心就一句话:拆一个幂次出来,然后换序,再用赫尔德不等式.
\end{note}
\begin{remark}
注意观察,积分顺序变了,另外,可以简单的记为“绝对值不等式”,就像直觉那样,先取绝对值再算积分要大(先算积分再取绝对值要小),用 $p$ 范数来写会好记并且清晰:
\begin{align*}
\left\| \int_c^d f(x,y) dy \right\|_p \leq \int_c^d \|f(x,y)\|_p dy.
\end{align*}
对于 $p \in (0,1)$ 的情形,证明方法是完全类似的,只需要运用反向赫尔德不等式.
\end{remark}
\begin{proof}
假设 $p \geq 1$,记 $g(x) = \int_c^d f(x,y) dy$,换序并利用\hyperref[theorem:Hold(赫尔德)不等式(积分形式)]{赫尔德不等式}有
\begin{align*}
&\int_a^b \left( \int_c^d f(x,y) dy \right)^p dx = \int_a^b \int_c^d f(x,y) dy \cdot \left( \int_c^d f(x,y) dy \right)^{p-1} dx
\\
&= \int_a^b \int_c^d f(x,y) g^{p-1}(x) dy dx = \int_c^d \int_a^b f(x,y) g^{p-1}(x) dx dy
\\
&\leqslant \int_c^d \left( \int_a^b f^p(x,y) dx \right)^{\frac{1}{p}} \cdot \left( \int_a^b g^{q(p-1)}(x) dx \right)^{\frac{1}{q}} dy
\\
&= \left( \int_a^b g^p(x) dx \right)^{1-\frac{1}{p}} \cdot \int_c^d \left( \int_a^b f^p(x,y) dx \right)^{\frac{1}{p}} dy
\\
&= \left( \int_a^b \left( \int_c^d f(x,y) dy \right)^p dx \right)^{1-\frac{1}{p}} \cdot \int_c^d \left( \int_a^b f^p(x,y) dx \right)^{\frac{1}{p}} dy.
\end{align*}
其中$\frac{1}{p}+\frac{1}{q}=1$,进而$q(p-1)=p$.两边约掉 $\int_a^b \left( \int_c^d f(x,y) dy \right)^p dx$ 就有
\begin{align*}
\left( \int_a^b \left( \int_c^d f(x,y) dy \right)^p dx \right)^{\frac{1}{p}} \leqslant \int_c^d \left( \int_a^b f^p(x,y) dx \right)^{\frac{1}{p}} dy.
\end{align*}
\end{proof}

\begin{theorem}[Hardy不等式]\label{theorem:Hardy(哈代)不等式}
设 $p > 1$ 或 $p < 0$,$f(x)$ 恒正且连续,记 $F(x) = \int_0^x f(t) dt$,则
\begin{align*}
\int_0^\infty \left( \frac{F(x)}{x} \right)^p dx \leqslant \left( \frac{p}{p - 1} \right)^p \int_0^\infty f^p(x) dx.
\end{align*}
\end{theorem}
\begin{remark}
这个不等式及其离散形式经常会考,证明的方法就是分部积分然后赫尔德(连续版),或者作差(离散版)然后求和再赫尔德,结构是类似的,系数也是最佳的,不过并不能找到一个函数使得刚刚好取等,只能是逼近取等,另外 $p < 0$ 的情况证明完全类似,利用反向赫尔德即可.
\end{remark}
\begin{proof}
假设 $p > 1$,对任意 $M > 0$,利用分部积分和\hyperref[theorem:Hold(赫尔德)不等式(积分形式)]{赫尔德不等式}有
\begin{align*}
&\int_0^M \left( \frac{F(x)}{x} \right)^p dx = -\frac{1}{p - 1} \int_0^M F^p(x) d\frac{1}{x^{p - 1}} = -\frac{1}{p - 1} \left( \frac{F^p(x)}{x^{p - 1}} \bigg|_0^M - \int_0^M \frac{1}{x^{p - 1}} dF^p(x) \right)
\\
&= -\frac{1}{p - 1} \frac{F^p(M)}{M^{p - 1}} + \frac{p}{p - 1} \int_0^M \frac{F^{p - 1}(x) f(x)}{x^{p - 1}} dx \leqslant \frac{p}{p - 1} \int_0^M \left( \frac{F(x)}{x} \right)^{p - 1} f(x) dx
\\
&\leqslant \frac{p}{p - 1} \left( \int_0^M \left( \frac{F(x)}{x} \right)^p dx \right)^{\frac{p - 1}{p}} \left( \int_0^M f^p(x) dx \right)^{\frac{1}{p}}.
\end{align*}
其中利用了
\begin{align*}
\lim_{x \to 0^+} \frac{F^p(x)}{x^{p - 1}} = \lim_{x \to 0^+} F(x) \left( \frac{F(x)}{x} \right)^{p - 1}, \lim_{x \to 0^+} \frac{F(x)}{x} \xlongequal{\text{L'Hospital}} \lim_{x \to 0^+} f(x) = f(0).
\end{align*}
所以
\begin{align*}
\frac{F^p(x)}{x^{p - 1}} \bigg|_0^M = \frac{F^p(M)}{M^{p - 1}} - \lim_{x \to 0^+} \frac{F^p(x)}{x^{p - 1}} = \frac{F^p(M)}{M^{p - 1}}.
\end{align*}
现在约掉相同的部分,再令 $M \to \infty$ 就有
\begin{align*}
\int_0^\infty \left( \frac{F(x)}{x} \right)^p dx \leqslant \left( \frac{p}{p - 1} \right)^p \int_0^\infty f^p(x) dx.
\end{align*}
\end{proof}

\begin{corollary}[离散版Hardy不等式]\label{corollary:离散版Hardy(哈代)不等式}
设数列 $a_n$ 非负,对任意 $p > 1$ 或者 $p < 0$,都有
\begin{align*}
\sum_{k=1}^n \left( \frac{a_1 + a_2 + \dots + a_k}{k} \right)^p \leqslant \left( \frac{p}{p - 1} \right)^p \sum_{k=1}^n a_k^p.
\end{align*}
\end{corollary}
\begin{remark}
如果 $p < 0$,则同样使用反向赫尔德不等式即可完成
\end{remark}
\begin{proof}
记 $S_k = a_1 + a_2 + \dots + a_k$,不妨设 $p > 1$,利用均值不等式或者 Young 不等式容易证明
\begin{align*}
\frac{S_k^p}{k^p} - \frac{p}{p - 1} \frac{S_k^{p - 1}}{k^{p - 1}} a_k \leqslant \frac{1}{p - 1} \left( (k - 1) \frac{S_{k - 1}^p}{(k - 1)^p} - k \frac{S_k^p}{k^p} \right)
\end{align*}
求和有
\begin{align*}
\sum_{k=1}^n \left( \frac{a_1 + a_2 + \dots + a_k}{k} \right)^p \leqslant \frac{p}{p - 1} \sum_{k=1}^n \left( \frac{a_1 + a_2 + \dots + a_k}{k} \right)^{p - 1} a_k.
\end{align*}
效果上就和前面分部积分完全一样,然后再用赫尔德不等式即可.
\end{proof}










\end{document}
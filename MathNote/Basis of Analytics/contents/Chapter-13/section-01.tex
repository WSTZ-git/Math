\documentclass[../../main.tex]{subfiles}
\graphicspath{{\subfix{../../image/}}} % 指定图片目录,后续可以直接使用图片文件名。

% 例如:
% \begin{figure}[H]
% \centering
% \includegraphics{image-01.01}
% \caption{图片标题}
% \label{figure:image-01.01}
% \end{figure}
% 注意:上述\label{}一定要放在\caption{}之后,否则引用图片序号会只会显示??.

\begin{document}

\section{著名积分不等式}

\begin{theorem}[Young不等式初等形式]\label{theorem:Young不等式初等形式}
设 $(x_i)_{i = 1}^n \subset [0, +\infty)$,$(p_i)_{i = 1}^n \subset (1, +\infty)$,$\sum_{i = 1}^n \frac{1}{p_i} = 1$, 则有
\begin{align*}
\prod_{i = 1}^n x_i \leqslant \sum_{i = 1}^n \frac{x_i^{p_i}}{p_i}.
\end{align*}
且等号成立条件为所有 $x_i$,$i = 1,2,\cdots,n$ 相等.
\end{theorem}
\begin{note}
最常用的是Young不等式的二元情形:

对任何$a,b\geq0,\frac{1}{p}+\frac{1}{q}=1,p>1$有$ab\leqslant \frac{a^p}{p}+\frac{b^q}{q}.$
\end{note}
\begin{proof}
不妨设 $x_i \neq 0$,$(i = 1,2,\cdots,n)$. 本结果可以取对数用\hyperref[theorem:Jensen不等式1111]{Jensen不等式}证明, 即
\begin{align*}
\prod_{i = 1}^n x_i \leqslant \sum_{i = 1}^n \frac{x_i^{p_i}}{p_i}\Leftrightarrow \sum_{i = 1}^n \ln x_i \leqslant \ln \left(\sum_{i = 1}^n \frac{x_i^{p_i}}{p_i}\right) \Leftrightarrow \sum_{i = 1}^n \frac{1}{p_i} \ln x_i^{p_i} \leqslant \ln \left(\sum_{i = 1}^n \frac{x_i^{p_i}}{p_i}\right),
\end{align*}
而最后一个等价之后就是 $\ln$ 的上凸性结合\hyperref[theorem:Jensen不等式1111]{Jensen不等式}给出.
\end{proof}

\begin{definition}
\begin{enumerate}[(1)]
\item $d\mu = g(x)dx$, 这里 $g$ 是一个在区间上内闭黎曼可积的函数.
\item 若 $E \subset \mathbb{Z}$, 则 $\int_E f(x) d\mu = \sum_{n \in E} f(n)$.
\end{enumerate}
\end{definition}

\begin{theorem}[Cauchy不等式]\label{theorem:Cauchy不等式(一般版本)}
\begin{align*}
\left(\int_E f(x)g(x) d\mu\right)^2 \leqslant \int_E |f(x)|^2 d\mu \int_E |g(x)|^2 d\mu.
\end{align*} 
\end{theorem}
\begin{proof}
只需证
\begin{align*}
\int_E |f(x)g(x)| d\mu \leqslant \sqrt{\int_E |f(x)|^2 d\mu \int_E |g(x)|^2 d\mu}.
\end{align*}
当 $\int_E |f(x)| d\mu$ 或 $\int_E |g(x)| d\mu = 0$ 时, 不等式右边为 $0$, 结论显然成立.

当 $\int_E |f(x)| d\mu \ne 0$ 且 $\int_E |g(x)| d\mu \ne 0$ 时, 不妨设 $\int_E |f(x)|^2 d\mu = \int_E |g(x)|^2 d\mu = 1$, 否则, 用 $\frac{f(x)}{\sqrt{\int_E |f(x)|^2 d\mu}}$ 代替 $f(x)$, $\frac{g(x)}{\sqrt{\int_E |g(x)|^2 d\mu}}$ 代替 $g(x)$即可.
利用\hyperref[theorem:Young不等式初等形式]{Young不等式}可得
\begin{align*}
\int_E |f(x)||g(x)| d\mu \leqslant \int_E \frac{|f(x)|^2 + |g(x)|^2}{2} d\mu = \frac{1}{2} + \frac{1}{2} = 1.
\end{align*}
等号成立当且仅当存在不全为零的$c_1,c_2$,使得$c_1f(x)+c_2g(x)=0.$
\end{proof}

\begin{theorem}[Jensen不等式(积分形式)]\label{theorem:Jensen不等式积分形式}
设 $\varphi$ 是下凸函数且 $p(x) \geqslant 0$, $\int_a^b p(x) dx > 0$, 则在有意义时, 必有
\begin{align}\label{equation--Jensen不等式积分形式16.11}
\varphi\left(\frac{\int_a^b p(x)f(x)dx}{\int_a^b p(x)dx}\right) \leqslant \frac{\int_a^b p(x)\varphi(f(x))dx}{\int_a^b p(x)dx}.
\end{align}
\end{theorem}
\begin{note}
1. 类似的对上凸函数, 不等式\eqref{equation--Jensen不等式积分形式16.11}反号.

2.一般情况可利用下凸函数可以被 $C^2$ 的下凸函数逼近得到, 例如定理 Bernstein 多项式保凸性一致逼近. 

3.Jensen不等式(积分形式)考试中不能直接使用,需要证明.
\end{note}
\begin{proof}
为书写简便, 我们记 $d\mu = \frac{p(x)}{\int_a^b p(y)dy}dx$, 那么有 $\int_a^b 1d\mu = 1$. 于是我们记 $x_0 = \int_a^b f(x)d\mu$ 并利用下凸函数恒在切线上方
\begin{align*}
\varphi(x) \geqslant \varphi(x_0) + \varphi'(x_0)(x - x_0),
\end{align*}
就有
\begin{align*}
\int_a^b \varphi(f(x))d\mu \geqslant \int_a^b [\varphi(x_0) + \varphi'(x_0)(f(x) - x_0)]d\mu = \varphi(x_0) = \varphi\left(\int_a^b f(x)d\mu\right),
\end{align*}
这就完成了证明. 
\end{proof}

\begin{example}
对连续正值函数 $f$, 我们有
\begin{align*}
\ln\left(\frac{1}{b - a}\int_a^b f(x)dx\right) \geqslant \frac{1}{b - a}\int_a^b \ln f(x)dx.
\end{align*} 
\end{example}
\begin{proof}
令 $d\mu = \frac{1}{b - a}dx$, 则 $\int_a^b d\mu = 1$, 再令 $x_0 \triangleq \int_a^b f(x) d\mu>0$, 则由 $\ln x$ 的上凸性可知
\begin{align*}
\ln x \leqslant \ln x_0 + \frac{1}{x_0}(x - x_0), \forall x > 0.
\end{align*}
从而
\begin{align*}
\int_a^b \ln f(x) d\mu &\leqslant \int_a^b \ln x_0 d\mu + \frac{1}{x_0}\int_a^b (f(x) - x_0) d\mu \\
&= \ln x_0 + \frac{1}{x_0}\left(\int_a^b f(x) d\mu - x_0\int_a^b d\mu\right) \\
&= \ln x_0 = \ln \int_a^b f(x) d\mu.
\end{align*}
故结论得证.
\end{proof}
















\end{document}
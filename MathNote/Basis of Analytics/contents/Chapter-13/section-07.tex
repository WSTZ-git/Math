\documentclass[../../main.tex]{subfiles}
\graphicspath{{\subfix{../../image/}}} % 指定图片目录,后续可以直接使用图片文件名。

% 例如:
% \begin{figure}[H]
% \centering
% \includegraphics[scale=0.4]{图.png}
% \caption{}
% \label{figure:图}
% \end{figure}
% 注意:上述\label{}一定要放在\caption{}之后,否则引用图片序号会只会显示??.

\begin{document}

\section{Fourier积分不等式}

\begin{theorem}[Fourier型积分不等式]\label{theorem:Fourier型积分不等式}
若$f(x)\in C^1[a,b]$,则
\begin{enumerate}[(1)]
\item \begin{align*}
\int_{a}^{b} |f(x)|^2 \mathrm{d}x - \frac{1}{b - a} \left( \int_{a}^{b} f(x) \mathrm{d}x \right)^2 \leqslant  \frac{(b - a)^2}{\pi^2} \int_{a}^{b} |f'(x)|^2 \mathrm{d}x,
\end{align*}
等号成立条件为
\begin{align*}
f(x) = c_1 + c_2 \cos\left( \frac{\pi(x - a)}{b - a} \right), c_1, c_2 \in \mathbb{R}.
\end{align*}

\item 若$f(a) = f(b)$,则
\begin{align*}
\int_{a}^{b} |f(x)|^2 \mathrm{d}x - \frac{1}{b - a} \left( \int_{a}^{b} f(x) \mathrm{d}x \right)^2 \leqslant  \frac{(b - a)^2}{4\pi^2} \int_{a}^{b} |f'(x)|^2 \mathrm{d}x,
\end{align*}
等号成立条件是
\begin{align*}
f(x) = c_1 + c_2 \cos\left( \frac{2\pi x}{b - a} \right) + c_3 \sin\left( \frac{2\pi x}{b - a} \right), c_1, c_2, c_3 \in \mathbb{R}.
\end{align*}

\item 若$f(a) = f(b) = 0$,则
\begin{align*}
\int_{a}^{b} |f(x)|^2 \mathrm{d}x \leqslant  \frac{(b - a)^2}{\pi^2} \int_{a}^{b} |f'(x)|^2 \mathrm{d}x,
\end{align*}
等号成立条件是
\begin{align*}
f(x) = c \sin\left( \frac{\pi (x - a)}{b - a} \right), c \in \mathbb{R}.
\end{align*} 
\end{enumerate}
\end{theorem}
\begin{remark}
(1)中对$f$进行偶延拓的原因是:使延拓后的区间端点函数值相等,从而就能利用\hyperref[theorem:Fourier级数的逐项微分定理]{Fourier级数的逐项微分定理}.

(2)已经有区间端点函数值相等的条件了,所以不需要进行延拓.

(3)中对$f$进行奇延拓的原因是:$f$满足$f(a)=f(b)=0$,此时对$f$做奇延拓后能使得$f\in C^1[2a-b,b]$,进而就能得到更好的结论.(如果只有$f(a)=f(b)\ne0$,那么$f$奇延拓后在$x=a$处间断.)
\end{remark}
\begin{proof}
\begin{enumerate}[(1)]
\item 把$f(x)$延拓到$[2a - b,b]$,使得$f(x)=f(2a - x)$,$x\in[a,b)$,则$f(b)=f(2a - b)$,$f\in C[2a - b,b]$且分段可微,并且此时$f$关于$x=a$轴对称.因此设$f(x)$有傅立叶级数
\begin{align*}
f(x) &\sim \frac{a_0}{2}+\sum_{n = 1}^{\infty}a_n\cos\left(\frac{\pi n(x - a)}{b - a}\right),
\end{align*}
进而由\hyperref[theorem:Fourier级数的逐项微分定理]{Fourier级数的逐项微分定理}可得
\begin{align*}
f^{\prime}(x) &\sim -\frac{\pi}{b - a}\sum_{n = 1}^{\infty}[na_n\sin\left(\frac{\pi n(x - a)}{b - a}\right)].
\end{align*}
这里
\begin{align*}
a_n=\frac{1}{b - a}\int_{2a - b}^{b}f(x)\cos\left(\frac{\pi n(x - a)}{b - a}\right)\mathrm{d}x, n\in\mathbb{N}_0.
\end{align*}
我们由\hyperref[theorem:Parseval恒等式]{Parseval恒等式}可得
\begin{align*}
\int_{2a - b}^{b}|f(x)|^2\mathrm{d}x&=(b - a)\left[\frac{a_0^2}{2}+\sum_{n = 1}^{\infty}a_n^2\right],\\
\int_{2a - b}^{b}|f^{\prime}(x)|^2\mathrm{d}x&=\frac{\pi^2}{b - a}\sum_{n = 1}^{\infty}n^2a_n^2.
\end{align*}
从而有
\begin{gather*}
\int_{2a-b}^b{|f(x)|^2\mathrm{d}x}-\left( b-a \right) \frac{a_{0}^{2}}{2}=\left( b-a \right) \sum_{n=1}^{\infty}{a_{n}^{2}}\leqslant \left( b-a \right) \sum_{n=1}^{\infty}{n^2a_{n}^{2}}=\frac{(b-a)^2}{\pi ^2}\int_{2b-a}^b{|f^{\prime}(x)|^2\mathrm{d}x}
\\
\iff \int_{2a-b}^b{|f(x)|^2\mathrm{d}x}-\frac{1}{2(b-a)}\left( \int_{2a-b}^b{f(x)\mathrm{d}x} \right) ^2\le \frac{(b-a)^2}{\pi ^2}\int_{2b-a}^b{|f^{\prime}(x)|^2\mathrm{d}x}.
\end{gather*}
利用对称性,就有
\begin{align*}
\int_{a}^{b}|f(x)|^2\mathrm{d}x-\frac{1}{(b - a)}\left(\int_{a}^{b}f(x)\mathrm{d}x\right)^2\leqslant \frac{(b - a)^2}{\pi^2}\int_{a}^{b}|f^{\prime}(x)|^2\mathrm{d}x,
\end{align*}
等号成立条件为
\begin{align*}
f(x)=c_1 + c_2\cos\left(\frac{\pi(x - a)}{b - a}\right), c_1,c_2\in\mathbb{R}.
\end{align*}

\item 设
\begin{align*}
f(x) &\sim \frac{a_0}{2}+\sum_{n = 1}^{\infty}\left(a_n\cos\left(\frac{2\pi nx}{b - a}\right)+b_n\sin\left(\frac{2\pi nx}{b - a}\right)\right),
\end{align*}
由\hyperref[theorem:Fourier级数的逐项微分定理]{Fourier级数的逐项微分定理}可得
\begin{align*}
f^{\prime}(x) &\sim \frac{2\pi}{b - a}\sum_{n = 1}^{\infty}\left(-na_n\sin\left(\frac{2\pi nx}{b - a}\right)+nb_n\cos\left(\frac{2\pi nx}{b - a}\right)\right).
\end{align*}
这里
\begin{align*}
a_n&=\frac{2}{b - a}\int_{a}^{b}f(x)\cos\left(\frac{2\pi nx}{b - a}\right)\mathrm{d}x,\\
b_n&=\frac{2}{b - a}\int_{a}^{b}f(x)\sin\left(\frac{2\pi nx}{b - a}\right)\mathrm{d}x.
\end{align*}
由\hyperref[theorem:Parseval恒等式]{Parseval恒等式},我们有
\begin{align*}
\int_{a}^{b}|f(x)|^2\mathrm{d}x&=\frac{b - a}{2}\left[\frac{a_0^2}{2}+\sum_{n = 1}^{\infty}(a_n^2 + b_n^2)\right],\\
\int_{a}^{b}|f^{\prime}(x)|^2\mathrm{d}x&=\frac{2\pi^2}{b - a}\sum_{n = 1}^{\infty}n^2(a_n^2 + b_n^2).
\end{align*}
因此
\begin{gather*}
\int_a^b{|f(x)|^2\mathrm{d}x}-\frac{\left( b-a \right) a_{0}^{2}}{4}=\frac{b-a}{2}\sum_{n=1}^{\infty}{\left( a_{n}^{2}+b_{n}^{2} \right)}\leqslant \frac{b-a}{2}\sum_{n=1}^{\infty}{n^2\left( a_{n}^{2}+b_{n}^{2} \right)}=\frac{(b-a)^2}{4\pi ^2}\int_a^b{|f^{\prime}(x)|^2\mathrm{d}x}
\\
\Longleftrightarrow \int_a^b{|f(x)|^2\mathrm{d}x}-\frac{1}{b-a}\left( \int_a^b{f(x)\mathrm{d}x} \right) ^2\le \frac{(b-a)^2}{4\pi ^2}\int_a^b{|f^{\prime}(x)|^2\mathrm{d}x},
\end{gather*}
等号成立条件是
\begin{align*}
f(x)=c_1 + c_2\cos\left(\frac{2\pi x}{b - a}\right)+c_3\sin\left(\frac{2\pi x}{b - a}\right).
\end{align*}

\item 令
\begin{align*}
f(x)=-f(2a - x), x\in[2a - b,a),
\end{align*}
则$f(x)\in C^1[2a - b,b]$,并且此时$f$关于$(a,0)$点中心对称.
设$f(x)$有傅立叶级数
\begin{align*}
f(x) &\sim \sum_{n = 1}^{\infty}b_n\sin\left(\frac{\pi n(x - a)}{b - a}\right),
\end{align*}
由\hyperref[theorem:Fourier级数的逐项微分定理]{Fourier级数的逐项微分定理}可得
\begin{align*}
f^{\prime}(x) &\sim \frac{\pi}{b - a}\sum_{n = 1}^{\infty}nb_n\cos\left(\frac{\pi n(x - a)}{b - a}\right).
\end{align*}
这里
\begin{align*}
b_n=\frac{1}{b - a}\int_{2a - b}^{b}f(x)\sin\left(\frac{\pi n(x - a)}{b - a}\right)\mathrm{d}x, n\in\mathbb{N}_0.
\end{align*}
我们由\hyperref[theorem:Parseval恒等式]{Parseval恒等式}可得
\begin{align*}
\int_{2a - b}^{b}|f(x)|^2\mathrm{d}x&=(b - a)\sum_{n = 1}^{\infty}b_n^2,\\
\int_{2a - b}^{b}|f^{\prime}(x)|^2\mathrm{d}x&=\frac{\pi^2}{b - a}\sum_{n = 1}^{\infty}n^2b_n^2.
\end{align*}
从而有
\begin{gather*}
\int_{2a-b}^b{|f(x)|^2\mathrm{d}x}=\left( b-a \right) \sum_{n=1}^{\infty}{b_{n}^{2}}\leqslant \left( b-a \right) \sum_{n=1}^{\infty}{n^2b_{n}^{2}}=\frac{(b-a)^2}{\pi ^2}\int_{2b-a}^b{|f^{\prime}(x)|^2\mathrm{d}x}
\\
\Longleftrightarrow \int_{2a-b}^b{|f(x)|^2\mathrm{d}x}\leqslant \frac{(b-a)^2}{\pi ^2}\int_{2b-a}^b{|f^{\prime}(x)|^2\mathrm{d}x}.
\end{gather*}
利用对称性,我们有
\begin{align*}
\int_{a}^{b}|f(x)|^2\mathrm{d}x\leqslant \frac{(b - a)^2}{\pi^2}\int_{a}^{b}|f^{\prime}(x)|^2\mathrm{d}x,
\end{align*}
等号成立条件是
\begin{align*}
f(x)=c\sin\left(\frac{\pi(x - a)}{b - a}\right).
\end{align*} 
\end{enumerate}

\end{proof}







\end{document}
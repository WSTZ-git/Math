\documentclass[../../main.tex]{subfiles}
\graphicspath{{\subfix{../../image/}}} % 指定图片目录,后续可以直接使用图片文件名。

% 例如:
% \begin{figure}[H]
% \centering
% \includegraphics[scale=0.4]{图.png}
% \caption{}
% \label{figure:图}
% \end{figure}
% 注意:上述\label{}一定要放在\caption{}之后,否则引用图片序号会只会显示??.

\begin{document}

\section{重积分方法}

\begin{theorem}[\hypertarget{Chebeshev不等式积分形式}{Chebeshev不等式积分形式}]\label{Chebeshev不等式积分形式}
设 \( p \in R[a,b] \) 且非负,\( f,g \) 在 \([a,b]\) 上是单调函数,则
\begin{align*}
\left( \int_{a}^{b} p(x) f(x) \,\mathrm{d}x \right) \left( \int_{a}^{b} p(x) g(x) \,\mathrm{d}x \right) \leqslant \left( \int_{a}^{b} p(x) \,\mathrm{d}x \right) \left( \int_{a}^{b} p(x) f(x) g(x) \,\mathrm{d}x \right), & f,g\text{单调性相同} \\
\\
\left( \int_{a}^{b} p(x) f(x) \,\mathrm{d}x \right) \left( \int_{a}^{b} p(x) g(x) \,\mathrm{d}x \right) \geqslant \left( \int_{a}^{b} p(x) \,\mathrm{d}x \right) \left( \int_{a}^{b} p(x) f(x) g(x) \,\mathrm{d}x \right), & f,g\text{单调性相反}
\nonumber
\end{align*}
\end{theorem}
\begin{note}
本不等式要牢记于心,它是很多不等式的基本模型,其特征就是出现单调性.
\end{note}
\begin{remark}
{\color{blue}证法二}中的$\mathrm{d}\mu$应该看作测度.
\end{remark}
\begin{proof}
{\color{blue}证法一:}
\begin{align*}
& \left( \int_{a}^{b} p(x) f(x) \mathrm{d}x \right) \left( \int_{a}^{b} p(x) g(x) \mathrm{d}x \right) - \left( \int_{a}^{b} p(x) \mathrm{d}x \right) \left( \int_{a}^{b} p(x) f(x) g(x) \mathrm{d}x \right) \\
& = \left( \int_{a}^{b} p(x) f(x) \mathrm{d}x \right) \left( \int_{a}^{b} p(y) g(y) \mathrm{d}y \right) - \left( \int_{a}^{b} p(x) \mathrm{d}x \right) \left( \int_{a}^{b} p(y) f(y) g(y) \mathrm{d}y \right) \\
& = \iint_{[a,b]^2} p(x) p(y) g(y) [f(x) - f(y)] \mathrm{d}x \mathrm{d}y \\
& \xlongequal{\text{对称性}} \iint_{[a,b]^2} p(y) p(x) g(x) [f(y) - f(x)] \mathrm{d}x \mathrm{d}y \\
& = \frac{1}{2} \iint_{[a,b]^2} p(x) p(y) [g(y) - g(x)] [f(x) - f(y)] \mathrm{d}x \mathrm{d}y,
\end{align*}
故结论得证.

{\color{blue}证法二:}
令$\frac{p\left( x \right)}{\int_a^b{p\left( x \right) \mathrm{d}x}}\mathrm{d}x=\mathrm{d}\mu $,则$\int_a^b{\mathrm{d}\mu}=\int_a^b{\frac{p\left( x \right)}{\int_a^b{p\left( x \right) \mathrm{d}x}}\mathrm{d}x}=1.$于是原不等式等价于
\begin{align*}
&\int_a^b f(x)\mathrm{d}\mu \int_a^b g(x)\mathrm{d}\mu - \int_a^b f(x)g(x)\mathrm{d}\mu \\
=&\int_a^b f(x)\mathrm{d}\mu \int_a^b g(y)\mathrm{d}\mu - \int_a^b \int_a^b f(y)g(y)\mathrm{d}\mu(y)\mathrm{d}\mu(x) \\
=&\int_a^b \int_a^b [f(x) - f(y)]g(y)\mathrm{d}\mu(y)\mathrm{d}\mu(x) \\
=&\int_a^b \int_a^b [f(y) - f(x)]g(x)\mathrm{d}\mu(y)\mathrm{d}\mu(x) \\
=&\frac{1}{2}\int_a^b \int_a^b [f(x) - f(y)][g(y) - g(x)]
\end{align*} 
故结论得证.
\end{proof}

\begin{example}
设 $f \in C[0,1]$ 递减恒正, 证明
\begin{align*}
\frac{\int_0^1 f^2(x)\mathrm{d}x}{\int_0^1 f(x)\mathrm{d}x} \geqslant \frac{\int_0^1 xf^2(x)\mathrm{d}x}{\int_0^1 xf(x)\mathrm{d}x}.
\end{align*}
\end{example}
\begin{proof}
\begin{align*}
\frac{\int_0^1 f^2(x)\mathrm{d}x}{\int_0^1 f(x)\mathrm{d}x} \geqslant \frac{\int_0^1 xf^2(x)\mathrm{d}x}{\int_0^1 xf(x)\mathrm{d}x}.
\end{align*}
原不等式等价于
\begin{align*}
\left(\int_0^1 f^2(x)\mathrm{d}x\right)\left(\int_0^1 xf(x)\mathrm{d}x\right) \geqslant \left(\int_0^1 xf^2(x)\mathrm{d}x\right)\left(\int_0^1 f(x)\mathrm{d}x\right).
\end{align*}
令 $\frac{f(x)}{\int_0^1 f(x)\mathrm{d}x}\mathrm{d}x = \mathrm{d}\mu$, 则上式等价于
\begin{align*}
\int_0^1 f(x)\mathrm{d}\mu \int_0^1 x\mathrm{d}\mu \geqslant \int_0^1 xf(x)\mathrm{d}\mu.
\end{align*}
上式由\hyperref[Chebeshev不等式积分形式]{Chebeshev不等式积分形式}可直接得到. 
\end{proof}

\begin{proposition}[反向切比雪夫不等式]\label{proposition:反向切比雪夫不等式}
设 $f,g \in R[a,b]$ 且 $m_1 \leqslant  f(x) \leqslant  M_1$, $m_2 \leqslant  g(x) \leqslant  M_2$, 证明
\begin{align*}
\left|\frac{1}{b - a}\int_{a}^{b}f(x)g(x)\mathrm{d}x - \frac{1}{(b - a)^2}\int_{a}^{b}f(x)\mathrm{d}x\int_{a}^{b}g(x)\mathrm{d}x\right| \leqslant  \frac{(M_2 - m_2)(M_1 - m_1)}{4}.
\end{align*}
\end{proposition}
\begin{remark}
不妨设 $a = 0$, $b = 1$的原因:假设当 $a = 0$, $b = 1$ 时,
\begin{align*}
\left|\int_0^1{f(x)g(x) \mathrm{d}x}-\int_0^1{f(x) \mathrm{d}x}\int_0^1{g(x) \mathrm{d}x}\right| &\leqslant \frac{(M_2 - m_2)(M_1 - m_1)}{4}
\end{align*}
成立. 则对一般的 $[a,b]$, 原不等式等价于
\begin{align}\label{equation:反向切比雪夫不等式-1.1}
\left|\int_0^1{f(a + (b - a)x)g(a + (b - a)x) \mathrm{d}x}-\int_0^1{f(a + (b - a)x) \mathrm{d}x}\int_0^1{g(a + (b - a)x) \mathrm{d}x}\right| &\leqslant \frac{(M_2 - m_2)(M_1 - m_1)}{4}.
\end{align}
又注意到 $f(a + (b - a)x),g(a + (b - a)x) \in R[0,1]$, 且 $f(x) \in [m_1,M_1]$, $g(x) \in [m_2,M_2]$.
故由假设可知\eqref{equation:反向切比雪夫不等式-1.1}式成立. 因此不妨设也成立. 
\end{remark}
\begin{note}
积累本题的想法.
\end{note}
\begin{proof}
不妨设 $a = 0$, $b = 1$, 则记 $A = \int_{0}^{1}f(x)\mathrm{d}x$, $B = \int_{0}^{1}g(x)\mathrm{d}x$. 于是
\begin{align*}
&\left|\int_{0}^{1}f(x)g(x)\mathrm{d}x - \int_{0}^{1}f(x)\mathrm{d}x\int_{0}^{1}g(x)\mathrm{d}x\right|^2 = \left|\int_{0}^{1}(f(x) - A)(g(x) - B)\mathrm{d}x\right|^2
\\
&\stackrel{\hyperref[theorem:Cauchy不等式(一般版本)]{Cauchy\text{不等式}}}{\leqslant } \int_{0}^{1}|f(x) - A|^2\mathrm{d}x \cdot \int_{0}^{1}|g(x) - B|^2\mathrm{d}x
\\
&= \left(\int_{0}^{1}|f(x)|^2\mathrm{d}x - \left(\int_{0}^{1}f(x)\mathrm{d}x\right)^2\right) \cdot \left(\int_{0}^{1}|g(x)|^2\mathrm{d}x - \left(\int_{0}^{1}g(x)\mathrm{d}x\right)^2\right).
\end{align*}
注意到
\[
\int_{0}^{1}(M_1 - f)(f - m_1)\mathrm{d}x = M_1A + m_1A - M_1m_1 - \int_{0}^{1}|f(x)|^2\mathrm{d}x,
\]
于是我们有
\begin{align*}
\int_{0}^{1}|f(x)|^2\mathrm{d}x - \left(\int_{0}^{1}f(x)\mathrm{d}x\right)^2 &=\int_{0}^{1}|f(x)|^2\mathrm{d}x -A^2
\\
&= (M_1 - A)(A - m_1) - \int_{0}^{1}(M_1 - f)(f - m_1)\mathrm{d}x\\
&\leqslant  (M_1-A)(A-m_1)\leqslant  \frac{(M_1-m_1)^2}{4}.
\end{align*}
最后一个不等号可由均值不等式或看出二次函数取最值得到.
类似的有
\[
\int_{0}^{1}|g(x)|^2\mathrm{d}x - \left(\int_{0}^{1}g(x)\mathrm{d}x\right)^2 \leqslant  \frac{(M_2 - m_2)^2}{4},
\]
这就证明了
\[
\left|\int_{0}^{1}f(x)g(x)\mathrm{d}x - \int_{0}^{1}f(x)\mathrm{d}x\int_{0}^{1}g(x)\mathrm{d}x\right|^2 \leqslant  \frac{(M_1 - m_1)^2}{4}\frac{(M_2 - m_2)^2}{4},
\]
即原不等式成立. 
\end{proof}

\begin{example}
设 $f \in C[a,b]$ 且
\[0 \leqslant f(x) \leqslant M, \forall x \in [a,b].\]
证明
\begin{align}\label{example:::::16.20}
\left(\int_{a}^{b}f(x)\cos x\mathrm{d}x\right)^2 + \left(\int_{a}^{b}f(x)\sin x\mathrm{d}x\right)^2 + \frac{M^2(b - a)^4}{12} &\geqslant \left(\int_{a}^{b}f(x)\mathrm{d}x\right)^2.
\end{align}
\end{example}
\begin{remark}
由Taylor公式可得不等式:
\begin{align}\label{cos放小成幂函数}
\cos x\geqslant  1-\frac{x^2}{2},\forall x\in \mathbb{R}.
\end{align}
$\sin x<x$两边同时在$[0,1]$上积分也可得$1-\cos x\leqslant \frac{x^2}{2}$.
\end{remark}
\begin{proof}
一方面
\begin{align*}
&\left( \int_a^b{f(x)\cos x\mathrm{d}x} \right) ^2+\left( \int_a^b{f(x)\sin x\mathrm{d}x} \right) ^2=\int_a^b{f(x)\cos x\mathrm{d}x}\int_a^b{f(y)\cos y\mathrm{d}y}+\int_a^b{f(x)\sin x\mathrm{d}x}\int_a^b{f(y)\sin y\mathrm{d}y}
\\
&=\iint_{[a,b]^2}{f(x)f(y)[\cos x\cos y}+\sin x\sin y]\mathrm{d}x\mathrm{d}y=\iint_{[a,b]^2}{f(x)f(y)\cos\mathrm{(}x}-y)\mathrm{d}x\mathrm{d}y.
\end{align*}
另外一方面
\begin{align*}
\left( \int_a^b{f(x)\mathrm{d}x} \right) ^2=\int_a^b{f(x)\cos x\mathrm{d}x}\int_a^b{f(y)\cos y\mathrm{d}y}=\iint_{[a,b]^2}{f(x)f(y)\mathrm{d}x\mathrm{d}y.}
\end{align*}
于是不等式\eqref{example:::::16.20}变为
\begin{align}\label{example:::::16.21}
\iint_{[a,b]^2}f(x)f(y)[1 - \cos(x - y)]\mathrm{d}x\mathrm{d}y &\leqslant \frac{M^2(b - a)^4}{12}.
\end{align}
事实上
\begin{align*}
\iint_{[a,b]^2}{f(x)f(y)[1}-\cos\mathrm{(}x-y)]\mathrm{d}x\mathrm{d}y\overset{\eqref{cos放小成幂函数}}{\leqslant}M^2\iint_{[a,b]^2}{\frac{(x-y)^2}{2}\mathrm{d}x\mathrm{d}y}=\frac{M^2(b-a)^4}{12},
\end{align*}
这就得到了不等式\eqref{example:::::16.21}. 
\end{proof}



















\end{document}
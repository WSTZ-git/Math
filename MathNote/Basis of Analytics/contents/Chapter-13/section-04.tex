\documentclass[../../main.tex]{subfiles}
\graphicspath{{\subfix{../../image/}}} % 指定图片目录,后续可以直接使用图片文件名。

% 例如:
% \begin{figure}[H]
% \centering
% \includegraphics[scale=0.4]{图.png}
% \caption{}
% \label{figure:图}
% \end{figure}
% 注意:上述\label{}一定要放在\caption{}之后,否则引用图片序号会只会显示??.

\begin{document}

\section{凸性相关题型}

\begin{example}
设 $f$ 是 $[a,b]$ 上的非负上凸函数. 证明对任何 $x \in (a,b)$, 都有
\begin{align}\label{equation-凸性相关积分-16.40}
f(x) &\leqslant \frac{2}{b - a}\int_{a}^{b}f(y)\mathrm{d}y.
\end{align}
特别的, 若 $f \in C[a,b]$, 则对 $x = a,b$, 也有\eqref{equation-凸性相关积分-16.40}式成立.
\end{example}
\begin{remark}
$\mathbf{Step}\mathbf{2}$中的$g(x)$的构造可以类比Lagrange中值定理的构造函数(关键是这个构造函数的几何直观).
\end{remark}
\begin{note}
这种只考虑函数端点函数值同为0的情形,再
通过构造$g(x)=f(x)-p(x)$(其中$p(x)$是$f$过两个端点的直线),将其推广到一般情况的想法很重要!
\end{note}
\begin{proof}
由\hyperref[corollary:开集上的下凸函数必连续]{开集上的凸函数必连续}可知, 开集上的上凸函数连续且有限个点不影响积分值. 又由\hyperref[proposition:下凸函数的单调性刻画]{凸函数单调性的刻画}, 我们知道
\[\lim_{x \to a^+}f(x), \lim_{x \to b^-}f(x)\]
是存在的. 因此不妨设 $f \in C[a,b]$.不妨设$a=0,b=1$,否则用$f(a+(b-a)x)$代替$f(x)$即可.

$\mathbf{Step}\mathbf{1}$ 当
\[f(a) = f(b) = 0, x_0\text{是}f(x)\text{最大值点}, x_0 \in (a,b),\]
我们利用上凸函数一定在割线上放缩得不等式
\[
\begin{cases}
f(x) \geqslant   \frac{f(x_0) }{x_0 }x, &x \in [0,x_0]\\
f(x) \geqslant \frac{f(x_0) }{x_0 - 1}(x - 1), &x \in [x_0,1]
\end{cases}.
\]
运用得到的不等式就有
\begin{align*}
\int_0^1{f(x)\mathrm{d}x\geqslant \int_0^{x_0}{\frac{f(x_0)}{x_0}x}}\mathrm{d}x+\int_{x_0}^1{\frac{f(x_0)}{x_0-1}(x}-1)\mathrm{d}x=\frac{1}{2}f(x_0),
\end{align*}
这就相当于得到了不等式\eqref{equation-凸性相关积分-16.40}.

当$x_0=a\text{或}b$时,由$f(a)=f(b)=0$且$f$非负可知,此时$f(x)\equiv 0$结论显然成立.

$\mathbf{Step}\mathbf{2}$ 一般情况可设
\begin{align*}
g\left( x \right) =f\left( x \right) -\left[ f\left( 1 \right) -f\left( 0 \right) \right] x-f\left( 0 \right) ,
\end{align*}
从而$g(0)=g(1)=0,$
于是 $g$ 就满足 $\mathbf{Step}\mathbf{1}$ 中的条件. 因此由\eqref{equation-凸性相关积分-16.40}知
\begin{align}\label{equation-凸性相关积分-16.41}
g\left( x \right) \leqslant 2\int_0^1{g\left( y \right) \mathrm{d}y,}\forall x\in [0,1].
\end{align}
于是利用\eqref{equation-凸性相关积分-16.41}知
\begin{align*}
f\left( x \right) -\left[ \left( f\left( 1 \right) -f\left( 0 \right) \right) x+f\left( 0 \right) \right] \leqslant 2\int_0^1{f\left( y \right) \mathrm{d}y}-2\int_0^1{\left[ \left( f\left( 1 \right) -f\left( 0 \right) \right) y+f\left( 0 \right) \right] \mathrm{d}y},\forall x\in [0,1].
\end{align*}
从而
\begin{align*}
f\left( x \right) -2\int_0^1{f\left( y \right) \mathrm{d}y}\leqslant \left[ \left( f\left( 1 \right) -f\left( 0 \right) \right) x+f\left( 0 \right) \right] -2\int_0^1{\left[ \left( f\left( 1 \right) -f\left( 0 \right) \right) y+f\left( 0 \right) \right] \mathrm{d}y},\forall x\in [0,1].
\end{align*}
注意到对$\forall x\in [0,1]$,都有
\begin{align*}
&\quad \quad \left[ \left( f\left( 1 \right) -f\left( 0 \right) \right) x+f\left( 0 \right) \right] -2\int_0^1{\left[ \left( f\left( 1 \right) -f\left( 0 \right) \right) y+f\left( 0 \right) \right] \mathrm{d}y}\leqslant 0
\\
&\Leftrightarrow \left[ f\left( 1 \right) -f\left( 0 \right) \right] x+f\left( 0 \right) \leqslant 2\int_0^1{\left[ \left( f\left( 1 \right) -f\left( 0 \right) \right) x+f\left( 0 \right) \right] \mathrm{d}x}
\\
&\Leftrightarrow \left[ f\left( 1 \right) -f\left( 0 \right) \right] x+f\left( 0 \right) \leqslant f\left( 1 \right) +f\left( 0 \right) 
\\
&\Leftrightarrow \left[ f\left( 1 \right) -f\left( 0 \right) \right] x\leqslant f\left( 1 \right) 
\\
&\Leftrightarrow f\left( 1 \right) \left( 1-x \right) +f\left( 0 \right) x\geqslant 0
\end{align*}
上述最后一个不等式可由$x\in[0,1],f(1),f(0)\geqslant  0$直接得到.
于是我们完成了证明. 
\end{proof}






\end{document}
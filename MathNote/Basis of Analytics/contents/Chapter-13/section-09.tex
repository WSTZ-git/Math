\documentclass[../../main.tex]{subfiles}
\graphicspath{{\subfix{./image/}}} % 指定图片目录,后续可以直接使用图片文件名
% 注意这里的文件路径不能用 ../../image/ ,否则用latexmk编译子文件会报错

% 例如:
% \begin{figure}[H]
% \centering
% \includegraphics[scale=0.4]{图.png}
% \caption{}
% \label{figure:图}
% \end{figure}
% 注意:上述\label{}一定要放在\caption{}之后,否则引用图片序号会只会显示??.

\begin{document}

\section{局部展开和能量积分法}

\begin{proposition}\label{proposition:g'(x)的Hölder连续相关结论}
设$\alpha>0, g\in C^1(\mathbb{R})$. 存在$a\in\mathbb{R}$使得$g(a)=\min_{x\in\mathbb{R}}g(x)$,如果
\begin{align}
|g'(x) - g'(y)| \leqslant M|x - y|^{\alpha}, \forall x,y\in\mathbb{R}, \label{17.39}
\end{align}
证明
\begin{align}
|g'(x)|^{\alpha + 1} \leqslant \left(\frac{\alpha + 1}{\alpha}\right)^{\alpha}[g(x) - g(a)]^{\alpha}M, \forall x\in\mathbb{R}. \label{17.40}
\end{align}
\end{proposition}
\begin{proof}
不妨设$g(a)=0$,否则用$g(x) - g(a)$代替$g(x)$. 当$M = 0$,则不等式\eqref{17.40}显然成立. 当$M\neq0$可以不妨设$M = 1$.

现在对非负函数$g$,当$g'(x_0)=0$,不等式\eqref{17.40}显然成立. 当$g'(x_0)>0$,则利用\eqref{17.39}有
\begin{align*}
g(x_0) &\geqslant g(x_0) - g(h)=\int_{h}^{x_0}g'(t)\mathrm{d}t \\
&\geqslant \int_{h}^{x_0}[g'(x_0) - |t - x_0|^{\alpha}]\mathrm{d}t \\
&= g'(x_0)(x_0 - h)-\frac{(x_0 - h)^{\alpha + 1}}{\alpha + 1},
\end{align*}
取$h = x_0 - |g'(x_0)|^{\frac{1}{\alpha}}$,就得到了$g(x_0)>\frac{\alpha}{\alpha + 1}|g'(x_0)|^{1 + \frac{1}{\alpha}}$,即不等式\eqref{17.40}成立. 类似的考虑$g'(x_0)<0$可得\eqref{17.40}.

当$g'(x_0)<0$,则利用\eqref{17.39}有
\begin{align*}
g(x_0) &\geqslant -g(h) + g(x_0)=-\int_{x_0}^{h}g'(t)\mathrm{d}t \\
&\geqslant -\int_{x_0}^{h}[g'(x_0) + |t - x_0|^{\alpha}]\mathrm{d}t \\
&= -g'(x_0)(h - x_0)-\frac{(h - x_0)^{\alpha + 1}}{\alpha + 1},
\end{align*}
取$h = x_0 + |g'(x_0)|^{\frac{1}{\alpha}}$,就得到了$g(x_0)>\frac{\alpha}{\alpha + 1}|g'(x_0)|^{1 + \frac{1}{\alpha}}$,即不等式\eqref{17.40}成立. 

\end{proof}

\begin{corollary}
设$f: \mathbb{R} \to (0, +\infty)$是一可微函数,且对所有$x, y \in \mathbb{R}$,有
$$|f'(x) - f'(y)| \leqslant |x - y|^\alpha,$$
其中$\alpha \in (0, 1]$是常数.
求证:对所有$x \in \mathbb{R}$,有
$$|f'(x)|^{\frac{\alpha + 1}{\alpha}} < \frac{\alpha + 1}{\alpha} f(x).$$
\end{corollary}
\begin{proof}
对$\forall x \in \mathbb{R}$,固定$x$.

(i)若$f'(x) = 0$,则结论显然成立.

(ii)若$f'(x) < 0$,则令$h = \left( -f'(x) \right)^{\frac{1}{\alpha}} > 0$. 由微积分基本定理可得
\begin{align*}
0 &< f(x + h) = f(x) + \int_x^{x + h} f'(t) \mathrm{d}t = f(x) + \int_x^{x + h} \left[ f'(t) - f'(x) \right] \mathrm{d}t + f'(x) h \\
&\leqslant f(x) + \int_x^{x + h} \left( t - x \right)^{\alpha} \mathrm{d}t + f'(x) h = f(x) + \frac{h^{\alpha + 1}}{\alpha + 1} + f'(x) h \\
&= f(x) + \frac{\left( -f'(x) \right)^{\frac{\alpha + 1}{\alpha}}}{\alpha + 1} + f'(x) \left( -f'(x) \right)^{\frac{1}{\alpha}}.
\end{align*}
于是
$$\left[ f'(x) - \frac{1}{\alpha + 1} f'(x) \right] \left( -f'(x) \right)^{\frac{1}{\alpha}} < f(x)\Longleftrightarrow f'(x) \left( -f'(x) \right)^{\frac{1}{\alpha}} < \frac{\alpha + 1}{\alpha} f(x).$$
从而
$$\left| f'(x) \right|^{\frac{\alpha + 1}{\alpha}} < \frac{\alpha + 1}{\alpha} f(x).$$

(iii)若$f'(x) > 0$,则令$h = \left( f'(x) \right)^{\frac{1}{\alpha}} > 0$. 由Newton-Leibniz公式可得
\begin{align*}
0 &< f(x - h) = -\int_{x - h}^x f'(t) \mathrm{d}t + f(x) = \int_{x - h}^x \left[ f'(x) - f'(t) \right] \mathrm{d}t + f(x) - f'(x) h \\
&\leqslant \int_{x - h}^x \left( x - t \right)^{\alpha} \mathrm{d}t + f(x) - f'(x) h = \frac{h^{\alpha + 1}}{\alpha + 1} + f(x) - f'(x) h \\
&= \frac{\left( f'(x) \right)^{\frac{\alpha + 1}{\alpha}}}{\alpha + 1} + f(x) - f'(x) \left( f'(x) \right)^{\frac{1}{\alpha}}.
\end{align*}
于是
$$\left[ f'(x) - \frac{1}{\alpha + 1} f'(x) \right] \left( f'(x) \right)^{\frac{1}{\alpha}} < f(x)\Longleftrightarrow \left( f'(x) \right)^{\frac{\alpha + 1}{\alpha}} < \frac{\alpha + 1}{\alpha} f(x).$$
从而
$$\left| f'(x) \right|^{\frac{\alpha + 1}{\alpha}} < \frac{\alpha + 1}{\alpha} f(x).$$

\end{proof}

\begin{example}
设$f(x)$是$(-\infty,+\infty)$上具有连续导数的非负函数,且存在$M > 0$使得对任意的$x,y \in (-\infty,+\infty)$,有
\begin{align*}
|f'(x) - f'(y)| \leqslant M|x - y|.
\end{align*}
证明:对于任意实数$x$,恒有$(f'(x))^2 \leqslant 2Mf(x)$.
\end{example}
\begin{proof}
对$\forall x \in \mathbb{R}$,固定$x$。由$f \geqslant 0$可得,对$\forall h > 0$,有
\begin{align*}
\int_{x-h}^x [f'(x) - f'(t)] \mathrm{d}t = f'(x)h - [f(x) - f(x - h)] \geqslant f'(x)h - f(x).
\end{align*}
又由条件可得,对$\forall h > 0$,有
\begin{align*}
\int_{x-h}^x |f'(x) - f'(t)| \mathrm{d}t \leqslant M \int_{x-h}^x |x - t| \mathrm{d}t = \frac{M}{2}h^2.
\end{align*}
于是对$\forall h > 0$,有
\begin{align*}
f'(x)h - f(x) \leqslant \int_{x-h}^x [f'(x) - f'(t)] \mathrm{d}t \leqslant \int_{x-h}^x |f'(x) - f'(t)| \mathrm{d}t \leqslant \frac{M}{2}h^2.
\end{align*}
故对$\forall h > 0$,都有
\begin{align*}
\frac{M}{2}h^2 - f'(x)h + f(x) \geqslant 0.
\end{align*}
因此
\begin{align*}
\Delta = (f'(x))^2 - 2Mf(x) \leqslant 0 \Longleftrightarrow (f'(x))^2 \leqslant 2Mf(x).
\end{align*}
再由$x$的任意性可知结论成立。

\end{proof}

\begin{example}
设$f$在$\mathbb{R}$上三阶可导,且$\forall x\in\mathbb{R}$成立
\begin{align*}
f(x),f'(x),f''(x),f'''(x)>0,\quad f'''(x)\leqslant f(x).
\end{align*}
证明:$\forall x\in\mathbb{R}$成立
\begin{align*}
f'(x)<2f(x).
\end{align*}
\end{example}
\begin{proof}
{\color{blue}证法一:}由Taylor定理可知,对$\forall x,t\in \mathbb{R}$,都存在$\xi$在$x$与$x+t$之间,使得
\begin{align}
0<f(x+t)=f(x)+f'(x)t+\frac{f''(x)}{2}t^2+\frac{f'''(\xi)}{6}t^3. \label{eq:::--23r8923f43t3g34223r2f22322.1}
\end{align}
当$t\leqslant 0$时,由\eqref{eq:::--23r8923f43t3g34223r2f22322.1}式和条件可得
\begin{align*}
0<f(x)+f'(x)t+\frac{f''(x)}{2}t^2+\frac{f'''(\xi)}{6}t^3\leqslant f(x)+f'(x)t+\frac{f''(x)}{2}t^2.
\end{align*}
当$t>0$时,由条件可得
\begin{align*}
f(x)+f'(x)t+\frac{f''(x)}{2}t^2>0.
\end{align*}
故
\begin{align*}
f(x)+f'(x)t+\frac{f''(x)}{2}t^2>0,\quad \forall t\in \mathbb{R}.
\end{align*}
由二次函数的性质可知
\begin{align}
\Delta =\left[ f'(x) \right]^2-2f''(x)f(x)<0\Longrightarrow \left[ f'(x) \right]^2<2f''(x)f(x),\quad \forall x\in \mathbb{R}. \label{eq:::--23r8923f43t3g34223r2f22322.2}
\end{align}
同理,由Taylor定理可知,对$\forall x,t\in \mathbb{R}$,都存在$\eta$在$x$与$x+t$之间,使得
\begin{align*}
0<f'(x+t)=f'(x)+f''(x)t+\frac{f'''(\eta)}{2}t^2.
\end{align*}
由$f'>0$知$f$递增,再结合$f'''(x)<f(x)$,由上式可得,对$\forall x,t\in \mathbb{R}$,都有
\begin{align*}
0<f'(x)+f''(x)t+\frac{f'''(\eta)}{2}t^2<f'(x)+f''(x)t+\frac{f(\eta)}{2}t^2\leqslant f'(x)+f''(x)t+\frac{f(x)}{2}t^2.
\end{align*}
于是由二次函数的性质可知
\begin{align}
\Delta' =\left[ f''(x) \right]^2-2f'(x)f(x)<0\Longrightarrow \left[ f''(x) \right]^2<2f'(x)f(x),\quad \forall x\in \mathbb{R}. \label{eq:::--23r8923f43t3g34223r2f22322.3}
\end{align}
由\eqref{eq:::--23r8923f43t3g34223r2f22322.2}\eqref{eq:::--23r8923f43t3g34223r2f22322.3}式可得,对$\forall x\in \mathbb{R}$,有
\begin{align*}
\left[ f'(x) \right]^4<4\left[ f''(x) \right]^2f^2(x)<8f'(x)f^3(x)\Longrightarrow \left[ f'(x) \right]^3<8f^3(x)\Longrightarrow f'(x)<2f(x).
\end{align*}

{\color{blue}证法二(能量积分法):}由条件知$f,f',f''$都是递增函数且有下界$0$,故
\begin{align*}
f(-\infty),f'(-\infty),f''(-\infty)\in [0,+\infty).
\end{align*}
若$f'(-\infty)=A>0$,则存在$-M<0$,使得
\begin{align*}
f'(x)>\frac{A}{2},\quad \forall x\leqslant -M.
\end{align*}
于是对$\forall x<-M$,有
\begin{align*}
f\left( x \right) &=f\left( -M \right) +\int_{-M}^x{f' \left( t \right) \mathrm{d}t}=f\left( -M \right) -\int_x^{-M}{f' \left( t \right) \mathrm{d}t}
\\
&<f\left( -M \right) -\int_x^{-M}{\frac{A}{2}\mathrm{d}t}=f\left( -M \right) -\frac{A}{2}\left( -M-x \right) 
\\
&=f\left( -M \right) +\frac{A}{2}\left( x+M \right) .
\end{align*}
令$x\rightarrow -\infty$得$f(-\infty)=-\infty$,这与$f(-\infty)\in [0,+\infty)$矛盾!故$f'(-\infty)=0$.同理可证$f''(-\infty)=0$.由条件可得
\begin{align*}
\frac{1}{2}\left[ \left( f''(x) \right)^2 \right]'=f'''(x)f''(x)<f(x)f''(x)=\left[ f(x)f'(x) \right]'-\left[ f'(x) \right]^2<\left[ f(x)f'(x) \right]'.
\end{align*}
两边同时积分得,对$\forall x\in \mathbb{R}$,都有
\begin{align}
\int_{-\infty}^x \frac{1}{2}\left[ \left( f''(x) \right)^2 \right]'\mathrm{d}t<\int_{-\infty}^x \left[ f(x)f'(x) \right]'\mathrm{d}t\Longleftrightarrow \left[ f''(x) \right]^2<2f(x)f'(x). \label{eq:::--23r8923f43t3g34223r22322.1}
\end{align}
同理,由条件可得
\begin{align*}
\left[ f''(x)f'(x) \right]'-\left[ f''(x) \right]^2=f'''(x)f'(x)<f(x)f'(x)=\frac{1}{2}\left[ f(x) \right]^2.
\end{align*}
从而
\begin{align*}
\left[ f''(x)f'(x) \right]'<\frac{3}{2}\left[ \left( f(x) \right)^2 \right]'.
\end{align*}
两边同时积分得,对$\forall x\in \mathbb{R}$,都有
\begin{align}
\int_{-\infty}^x \left[ f''(x)f'(x) \right]'\mathrm{d}t<\int_{-\infty}^x \frac{3}{2}\left[ \left( f(x) \right)^2 \right]'\mathrm{d}t\Longleftrightarrow f''(x)f'(x)<\frac{3}{2}f^2(x). \label{eq:::--23r8923f43t3g34223r22322.2}
\end{align}
将\eqref{eq:::--23r8923f43t3g34223r22322.1}\eqref{eq:::--23r8923f43t3g34223r22322.2}两式相乘得
\begin{align*}
\left[ f''(x) \right]^3<3f^3(x)\Longrightarrow f''(x)<\sqrt[3]{3}f(x)\Longrightarrow f''(x)f'(x)<\sqrt[3]{3}f(x)f'(x),\quad \forall x\in \mathbb{R}.
\end{align*}
两边再同时积分得
\begin{align*}
\int_{-\infty}^x f''(t)f'(t)\mathrm{d}t<\sqrt[3]{3}\int_{-\infty}^x f(t)f'(t)\mathrm{d}t\Longleftrightarrow \left[ f'(x) \right]^2<\sqrt[3]{3}f^2(x)\Longleftrightarrow f'(x)<\sqrt[6]{3}f(x),\quad \forall x\in \mathbb{R}.
\end{align*}

\end{proof}






















\end{document}
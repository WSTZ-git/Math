\documentclass[../../main.tex]{subfiles}
\graphicspath{{\subfix{../../image/}}} % 指定图片目录,后续可以直接使用图片文件名。

% 例如:
% \begin{figure}[H]
% \centering
% \includegraphics[scale=0.4]{图.png}
% \caption{}
% \label{figure:图}
% \end{figure}
% 注意:上述\label{}一定要放在\caption{}之后,否则引用图片序号会只会显示??.

\begin{document}

\section{直接求导法}

\begin{example}
\begin{enumerate}
\item 设 $f \in C^1[0,1]$, $f(0) = 0$, $0 \leqslant slant f'(x) \leqslant slant 1$, 证明
\begin{align*}
\left[\int_{0}^{1}f(x)\mathrm{d}x\right]^2 \geqslant slant \int_{0}^{1}f^3(x)\mathrm{d}x,
\end{align*}
并判断取等条件.

\item 设 $f$ 在 $[0,a]$ 可导且 $f(0) = 0$, $0 \leqslant slant f'(x) \leqslant slant \lambda$, $\lambda > 0$ 为常数, 证明
\begin{align}\label{equation-section03-16.24}
\left[\int_{0}^{a}f(x)\mathrm{d}x\right]^m \geqslant slant \frac{m}{(2\lambda)^{m - 1}}\int_{0}^{a}f^{2m - 1}(x)\mathrm{d}x,
\end{align}
并判断取等条件.
\end{enumerate}
\end{example}
\begin{proof}
\begin{enumerate}
\item 由 $0<f'(x)\,(x>0)$ 及 $f(0)=0$ 可知 $f(x)>0\,(0<x\leqslant 1)$. 设
$$g(t)=\int_0^t f^3(x)\mathrm{d}x-\left(\int_0^t f(x)\mathrm{d}x\right)^2\quad(t\in[0,1]),$$
则
$$g'(t)=f(t)\left(f^2(t)-2\int_0^t f(x)\mathrm{d}x\right).$$
令$h(t)=f^2(t)-2\int_0^t f(x)\mathrm{d}x$,则由$0<f'(x)\leqslant slant 1(x>0)$可知
\[
h' \left( t \right) =2f\left( t \right) \left[ f' \left( t \right) -1 \right] \leqslant slant 0,\forall t\in \left[ 0,1 \right] .
\]
从而$h(t)\leqslant slant h(0)=0,\forall t\in \left[ 0,1 \right] $.于是$g'(t)\leqslant slant 0,\forall t\in \left[ 0,1 \right]$.
因而 $g$ 在 $[0,1]$ 上单调递减. 由 $g(0)=0$ 知 $g\leqslant 0$. 若
$$\int_0^1 f^3(x)\mathrm{d}x=\left(\int_0^1 f(x)\mathrm{d}x\right)^2,$$
则 $g(1)=0$, 因而 $g(t)\equiv0$. 所以
$$g'(t)=f(t)\left(f^2(t)-2\int_0^t f(x)\mathrm{d}x\right)=0.$$
这推出 $f\equiv 0$或$f^2(t)=2\int_0^t f(x)\mathrm{d}x$. 因而
$$2f(t)f'(t)=2f(t)\quad(0<t\leqslant 1).$$
这推出 $f'(t)=1$,即$f(t)=t$.故当$f(t)\equiv 0$或$f(t)=t$时等号成立.

\item 定义
\begin{align*}
g(x) &= \left(\int_{0}^{x}f(t)\mathrm{d}t\right)^m - \frac{m}{(2\lambda)^{m - 1}}\int_{0}^{x}f^{2m - 1}(t)\mathrm{d}t.
\end{align*}
求导得
\begin{align*}
g'(x) &= mf(x)\left(\int_{0}^{x}f(t)\mathrm{d}t\right)^{m - 1} - \frac{m}{(2\lambda)^{m - 1}}f^{2m - 1}(x)\\
&= mf(x)\left[\left(\int_{0}^{x}f(t)\mathrm{d}t\right)^{m - 1} - \frac{1}{(2\lambda)^{m - 1}}f^{2m - 2}(x)\right].
\end{align*}
令$h(x)=\int_{0}^{x}f(t)\mathrm{d}t - \frac{f^2(x)}{2\lambda}$,则
\begin{align*}
h'(x)=\left[\int_{0}^{x}f(t)\mathrm{d}t - \frac{f^2(x)}{2\lambda}\right]' &= f(x) - \frac{f(x)f'(x)}{\lambda} = \frac{f(x)}{\lambda}[\lambda - f'(x)] \geqslant slant 0,
\end{align*}
从而$h(x)\geqslant  h(0)=0.$进而
\begin{align*}
h^{m-1}(x)\geqslant slant \left(\int_{0}^{x}f(t)\mathrm{d}t\right)^{m - 1} - \frac{1}{(2\lambda)^{m - 1}}f^{2m - 2}(x)\geqslant slant 0.
\end{align*}
于是我们有
\[g'(x) \geqslant slant g'(0) = 0,\]
从而 $g$ 递增且
\[g(a) \geqslant slant g(0) = 0,\]
这就是不等式\eqref{equation-section03-16.24}.
要使得等号成立, 我们需要 $g$ 为常数, 因此需要$g'\equiv 0$,故需要 $f \equiv 0$ 或者
\[\int_{0}^{x}f(t)\mathrm{d}t - \frac{f^2(x)}{2\lambda} \equiv 0,\]
令$y=\int_{0}^{x}f(t)\mathrm{d}t,$则上式等价于
\begin{align*}
y-\frac{(y')^2}{2\lambda}=0
\end{align*}
从而解上述微分方程得到取等条件是
\[f(x) = 0\text{或者}f(x) = \lambda x.\]
\end{enumerate}
\end{proof}

\begin{example}
设 $f,g \in C[a,b]$ 使得 $f$ 递增且 $0 \leqslant slant g \leqslant slant 1$, 证明
\begin{align}\label{example-239048-16.25}
\int_{a}^{a + \int_{a}^{b}g(t)\mathrm{d}t}f(x)\mathrm{d}x &\leqslant slant \int_{a}^{b}f(x)g(x)\mathrm{d}x \leqslant slant \int_{b - \int_{a}^{b}g(t)\mathrm{d}t}^{b}f(x)\mathrm{d}x.
\end{align}
\end{example}
\begin{proof}
考虑
\begin{align*}
h(y) &= \int_{a}^{a + \int_{a}^{y}g(t)\mathrm{d}t}f(x)\mathrm{d}x - \int_{a}^{y}f(x)g(x)\mathrm{d}x.
\end{align*}
则利用
\begin{align*}
a + \int_{a}^{y}g(x)\mathrm{d}x &\leqslant slant a + \int_{a}^{y}1\mathrm{d}x = y,
\end{align*}
再结合$f$递增,我们有
\begin{align*}
h'(y) &= g(y)f\left(a + \int_{a}^{y}g(t)\mathrm{d}t\right) - f(y)g(y) \leqslant slant 0 \to h(b) \leqslant slant h(a) = 0,
\end{align*}
故不等式\eqref{example-239048-16.25}左侧得证. 另一侧不等式同理可得, 这就证明了不等式\eqref{example-239048-16.25}. 
\end{proof}

\begin{proposition}
设 $f$ 是 $[a,b]$ 上单调递增的连续函数. 求证
$$\int_a^b xf(x)\mathrm{d}x\geqslant \frac{a+b}{2}\int_a^b f(x)\mathrm{d}x.$$
\end{proposition}
\begin{note}
许多有关连续函数积分的不等式可以通过变上限积分的性质来证明.
\end{note}
\begin{proof}
令
$$F(t)=\int_a^t xf(x)\mathrm{d}x-\frac{a+t}{2}\int_a^t f(x)\mathrm{d}x.$$
只需证明 $F(b)\geqslant 0$. 由于 $f$ 是连续函数, $F$ 在 $[a,b]$ 上可微, 且
$$
\begin{aligned}
F'(t) &= tf(t)-\frac{1}{2}\int_a^t f(x)\mathrm{d}x-\frac{a+t}{2}f(t) \\
&= \frac{t-a}{2}f(t)-\frac{1}{2}\int_a^t f(x)\mathrm{d}x \\
&\geqslant  \frac{t-a}{2}f(t)-\frac{1}{2}(t-a)f(t)=0.
\end{aligned}
$$
这说明 $f$ 在 $[a,b]$ 上单调递增. 因为 $F(a)=0$, 所以 $F(b)\geqslant 0$.
\end{proof}

\begin{example}
设 $f$ 是区间 $[0,1]$ 上的连续函数并满足 $0\leqslant  f(x)\leqslant  x$. 求证:
$$\int_0^1 f(x)\mathrm{d}x-\left(\int_0^1 f(x)\mathrm{d}x\right)^2\geqslant \int_0^1 x^2f(x)\mathrm{d}x\geqslant \left(\int_0^1 f(x)\mathrm{d}x\right)^2.$$
并且上式成为等式当且仅当$f(x)=x$.
\end{example}
\begin{proof}
设 $f$ 是连续函数满足所给的条件, $F(x)=\int_0^x f(t)\mathrm{d}t$, 则 $F'=f$. 由 $0<f(x)\leqslant  x$ 得 $F(x)\leqslant \int_0^x t\mathrm{d}t=\frac{1}{2}x^2$. 因而
$$\int_0^1 x^2f(x)\mathrm{d}x\geqslant \int_0^1 2F(x)F'(x)\mathrm{d}x=F^2(x)\bigg|_0^1=\left(\int_0^1 f(x)\mathrm{d}x\right)^2.$$
利用分部积分, 得
$$
\begin{aligned}
\int_0^1 x^2f(x)\mathrm{d}x &= x^2F(x)\bigg|_0^1-\int_0^1 2xF(x)\mathrm{d}x \\
&= \int_0^1 f(x)\mathrm{d}x-\int_0^1 2xF(x)\mathrm{d}x \\
&\leqslant  \int_0^1 f(x)\mathrm{d}x-\int_0^1 2f(x)F(x)\mathrm{d}x \\
&= \int_0^1 f(x)\mathrm{d}x-F^2(x)\bigg|_0^1 \\
&= \int_0^1 f(x)\mathrm{d}x-\left(\int_0^1 f(x)\mathrm{d}x\right)^2.
\end{aligned}
$$
由证明过程可知只有当 $f(x)=x$ 时, 所证不等式成为等式.
\end{proof}




\end{document}
\documentclass[../../main.tex]{subfiles}
\graphicspath{{\subfix{../../image/}}} % 指定图片目录,后续可以直接使用图片文件名。

% 例如:
% \begin{figure}[H]
% \centering
% \includegraphics[scale=0.4]{image-01.01}
% \caption{图片标题}
% \label{figure:image-01.01}
% \end{figure}
% 注意:上述\label{}一定要放在\caption{}之后,否则引用图片序号会只会显示??.

\begin{document}

\section{直接求导法}

\begin{example}
\begin{enumerate}
\item 设 $f \in C^1[0,1]$, $f(0) = 0$, $0 \leqslant f'(x) \leqslant 1$, 证明
\begin{align*}
\left[\int_{0}^{1}f(x)\mathrm{d}x\right]^2 \geqslant \int_{0}^{1}f^3(x)\mathrm{d}x,
\end{align*}
并判断取等条件.

\item 设 $f$ 在 $[0,a]$ 可导且 $f(0) = 0$, $0 \leqslant f'(x) \leqslant \lambda$, $\lambda > 0$ 为常数, 证明
\begin{align}\label{equation-section03-16.24}
\left[\int_{0}^{a}f(x)\mathrm{d}x\right]^m \geqslant \frac{m}{(2\lambda)^{m - 1}}\int_{0}^{a}f^{2m - 1}(x)\mathrm{d}x,
\end{align}
并判断取等条件.
\end{enumerate}
\end{example}
\begin{proof}
因为第一题是第二题的特例了,所以我们只证第二题.
定义
\begin{align*}
g(x) &= \left(\int_{0}^{x}f(t)dt\right)^m - \frac{m}{(2\lambda)^{m - 1}}\int_{0}^{x}f^{2m - 1}(t)dt.
\end{align*}
求导得
\begin{align*}
g'(x) &= mf(x)\left(\int_{0}^{x}f(t)dt\right)^{m - 1} - \frac{m}{(2\lambda)^{m - 1}}f^{2m - 1}(x)\\
&= mf(x)\left[\left(\int_{0}^{x}f(t)dt\right)^{m - 1} - \frac{1}{(2\lambda)^{m - 1}}f^{2m - 2}(x)\right].
\end{align*}
令$h(x)=\int_{0}^{x}f(t)dt - \frac{f^2(x)}{2\lambda}$,则
\begin{align*}
h'(x)=\left[\int_{0}^{x}f(t)dt - \frac{f^2(x)}{2\lambda}\right]' &= f(x) - \frac{f(x)f'(x)}{\lambda} = \frac{f(x)}{\lambda}[\lambda - f'(x)] \geqslant 0,
\end{align*}
从而$h(x)\geq h(0)=0.$进而
\begin{align*}
h^{m-1}(x)\geqslant \left(\int_{0}^{x}f(t)dt\right)^{m - 1} - \frac{1}{(2\lambda)^{m - 1}}f^{2m - 2}(x)\geqslant 0.
\end{align*}
于是我们有
\[g'(x) \geqslant g'(0) = 0,\]
从而 $g$ 递增且
\[g(a) \geqslant g(0) = 0,\]
这就是不等式\eqref{equation-section03-16.24}.
要使得等号成立, 我们需要 $g$ 为常数, 因此需要$g'\equiv 0$,故需要 $f \equiv 0$ 或者
\[\int_{0}^{x}f(t)dt - \frac{f^2(x)}{2\lambda} \equiv 0,\]
令$y=\int_{0}^{x}f(t)dt,$则上式等价于
\begin{align*}
y-\frac{(y')^2}{2\lambda}=0
\end{align*}
从而解上述微分方程得到取等条件是
\[f(x) = 0\text{或者}f(x) = \lambda x.\]
\end{proof}

\begin{example}
设 $f,g \in C[a,b]$ 使得 $f$ 递增且 $0 \leqslant g \leqslant 1$, 证明
\begin{align}\label{example-239048-16.25}
\int_{a}^{a + \int_{a}^{b}g(t)dt}f(x)\mathrm{d}x &\leqslant \int_{a}^{b}f(x)g(x)\mathrm{d}x \leqslant \int_{b - \int_{a}^{b}g(t)dt}^{b}f(x)\mathrm{d}x.
\end{align}
\end{example}
\begin{proof}
考虑
\begin{align*}
h(y) &= \int_{a}^{a + \int_{a}^{y}g(t)dt}f(x)\mathrm{d}x - \int_{a}^{y}f(x)g(x)\mathrm{d}x.
\end{align*}
则利用
\begin{align*}
a + \int_{a}^{y}g(x)\mathrm{d}x &\leqslant a + \int_{a}^{y}1\mathrm{d}x = y,
\end{align*}
再结合$f$递增,我们有
\begin{align*}
h'(y) &= g(y)f\left(a + \int_{a}^{y}g(t)dt\right) - f(y)g(y) \leqslant 0 \to h(b) \leqslant h(a) = 0,
\end{align*}
故不等式\eqref{example-239048-16.25}左侧得证. 另一侧不等式同理可得, 这就证明了不等式\eqref{example-239048-16.25}. 
\end{proof}




\end{document}
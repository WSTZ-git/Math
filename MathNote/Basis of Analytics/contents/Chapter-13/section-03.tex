\documentclass[../../main.tex]{subfiles}
\graphicspath{{\subfix{../../image/}}} % 指定图片目录,后续可以直接使用图片文件名。

% 例如:
% \begin{figure}[H]
% \centering
% \includegraphics[scale=0.4]{图.png}
% \caption{}
% \label{figure:图}
% \end{figure}
% 注意:上述\label{}一定要放在\caption{}之后,否则引用图片序号会只会显示??.

\begin{document}

\section{直接求导法}

\begin{example}
\begin{enumerate}
\item 设 $f \in C^1[0,1]$, $f(0) = 0$, $0 \leqslant f'(x) \leqslant 1$, 证明
\begin{align*}
\left[\int_{0}^{1}f(x)\mathrm{d}x\right]^2 \geqslant \int_{0}^{1}f^3(x)\mathrm{d}x,
\end{align*}
并判断取等条件.

\item 设 $f$ 在 $[0,a]$ 可导且 $f(0) = 0$, $0 \leqslant f'(x) \leqslant \lambda$, $\lambda > 0$ 为常数, 证明
\begin{align}\label{equation-section03-16.24}
\left[\int_{0}^{a}f(x)\mathrm{d}x\right]^m \geqslant \frac{m}{(2\lambda)^{m - 1}}\int_{0}^{a}f^{2m - 1}(x)\mathrm{d}x,
\end{align}
并判断取等条件.
\end{enumerate}
\end{example}
\begin{proof}
\begin{enumerate}
\item 由 $0<f'(x)\,(x>0)$ 及 $f(0)=0$ 可知 $f(x)>0\,(0<x\leqslant 1)$. 设
$$g(t)=\int_0^t f^3(x)\mathrm{d}x-\left(\int_0^t f(x)\mathrm{d}x\right)^2\quad(t\in[0,1]),$$
则
$$g'(t)=f(t)\left(f^2(t)-2\int_0^t f(x)\mathrm{d}x\right).$$
令$h(t)=f^2(t)-2\int_0^t f(x)\mathrm{d}x$,则由$0<f'(x)\leqslant 1(x>0)$可知
\[
h' \left( t \right) =2f\left( t \right) \left[ f' \left( t \right) -1 \right] \leqslant 0,\forall t\in \left[ 0,1 \right] .
\]
从而$h(t)\leqslant h(0)=0,\forall t\in \left[ 0,1 \right] $.于是$g'(t)\leqslant 0,\forall t\in \left[ 0,1 \right]$.
因而 $g$ 在 $[0,1]$ 上单调递减. 由 $g(0)=0$ 知 $g\leqslant 0$. 若
$$\int_0^1 f^3(x)\mathrm{d}x=\left(\int_0^1 f(x)\mathrm{d}x\right)^2,$$
则 $g(1)=0$, 因而 $g(t)\equiv0$. 所以
$$g'(t)=f(t)\left(f^2(t)-2\int_0^t f(x)\mathrm{d}x\right)=0.$$
这推出 $f\equiv 0$或$f^2(t)=2\int_0^t f(x)\mathrm{d}x$. 因而
$$2f(t)f'(t)=2f(t)\quad(0<t\leqslant 1).$$
这推出 $f'(t)=1$,即$f(t)=t$.故当$f(t)\equiv 0$或$f(t)=t$时等号成立.

\item 定义
\begin{align*}
g(x) &= \left(\int_{0}^{x}f(t)\mathrm{d}t\right)^m - \frac{m}{(2\lambda)^{m - 1}}\int_{0}^{x}f^{2m - 1}(t)\mathrm{d}t.
\end{align*}
求导得
\begin{align*}
g'(x) &= mf(x)\left(\int_{0}^{x}f(t)\mathrm{d}t\right)^{m - 1} - \frac{m}{(2\lambda)^{m - 1}}f^{2m - 1}(x)\\
&= mf(x)\left[\left(\int_{0}^{x}f(t)\mathrm{d}t\right)^{m - 1} - \frac{1}{(2\lambda)^{m - 1}}f^{2m - 2}(x)\right].
\end{align*}
令$h(x)=\int_{0}^{x}f(t)\mathrm{d}t - \frac{f^2(x)}{2\lambda}$,则
\begin{align*}
h'(x)=\left[\int_{0}^{x}f(t)\mathrm{d}t - \frac{f^2(x)}{2\lambda}\right]' &= f(x) - \frac{f(x)f'(x)}{\lambda} = \frac{f(x)}{\lambda}[\lambda - f'(x)] \geqslant 0,
\end{align*}
从而$h(x)\geqslant  h(0)=0.$进而
\begin{align*}
h^{m-1}(x)\geqslant \left(\int_{0}^{x}f(t)\mathrm{d}t\right)^{m - 1} - \frac{1}{(2\lambda)^{m - 1}}f^{2m - 2}(x)\geqslant 0.
\end{align*}
于是我们有
\[g'(x) \geqslant g'(0) = 0,\]
从而 $g$ 递增且
\[g(a) \geqslant g(0) = 0,\]
这就是不等式\eqref{equation-section03-16.24}.
要使得等号成立, 我们需要 $g$ 为常数, 因此需要$g'\equiv 0$,故需要 $f \equiv 0$ 或者
\[\int_{0}^{x}f(t)\mathrm{d}t - \frac{f^2(x)}{2\lambda} \equiv 0,\]
令$y=\int_{0}^{x}f(t)\mathrm{d}t,$则上式等价于
\begin{align*}
y-\frac{(y')^2}{2\lambda}=0
\end{align*}
从而解上述微分方程得到取等条件是
\[f(x) = 0\text{或者}f(x) = \lambda x.\]
\end{enumerate}

\end{proof}

\begin{example}
设 $f,g \in C[a,b]$ 使得 $f$ 递增且 $0 \leqslant g \leqslant 1$, 证明
\begin{align}\label{example-239048-16.25}
\int_{a}^{a + \int_{a}^{b}g(t)\mathrm{d}t}f(x)\mathrm{d}x &\leqslant \int_{a}^{b}f(x)g(x)\mathrm{d}x \leqslant \int_{b - \int_{a}^{b}g(t)\mathrm{d}t}^{b}f(x)\mathrm{d}x.
\end{align}
\end{example}
\begin{proof}
考虑
\begin{align*}
h(y) &= \int_{a}^{a + \int_{a}^{y}g(t)\mathrm{d}t}f(x)\mathrm{d}x - \int_{a}^{y}f(x)g(x)\mathrm{d}x.
\end{align*}
则利用
\begin{align*}
a + \int_{a}^{y}g(x)\mathrm{d}x &\leqslant a + \int_{a}^{y}1\mathrm{d}x = y,
\end{align*}
再结合$f$递增,我们有
\begin{align*}
h'(y) &= g(y)f\left(a + \int_{a}^{y}g(t)\mathrm{d}t\right) - f(y)g(y) \leqslant 0 \to h(b) \leqslant h(a) = 0,
\end{align*}
故不等式\eqref{example-239048-16.25}左侧得证. 另一侧不等式同理可得, 这就证明了不等式\eqref{example-239048-16.25}. 

\end{proof}

\begin{proposition}
设 $f$ 是 $[a,b]$ 上单调递增的连续函数. 求证
$$\int_a^b xf(x)\mathrm{d}x\geqslant \frac{a+b}{2}\int_a^b f(x)\mathrm{d}x.$$
\end{proposition}
\begin{note}
许多有关连续函数积分的不等式可以通过变上限积分的性质来证明.
\end{note}
\begin{proof}
令
$$F(t)=\int_a^t xf(x)\mathrm{d}x-\frac{a+t}{2}\int_a^t f(x)\mathrm{d}x.$$
只需证明 $F(b)\geqslant 0$. 由于 $f$ 是连续函数, $F$ 在 $[a,b]$ 上可微, 且
$$
\begin{aligned}
F'(t) &= tf(t)-\frac{1}{2}\int_a^t f(x)\mathrm{d}x-\frac{a+t}{2}f(t) \\
&= \frac{t-a}{2}f(t)-\frac{1}{2}\int_a^t f(x)\mathrm{d}x \\
&\geqslant  \frac{t-a}{2}f(t)-\frac{1}{2}(t-a)f(t)=0.
\end{aligned}
$$
这说明 $f$ 在 $[a,b]$ 上单调递增. 因为 $F(a)=0$, 所以 $F(b)\geqslant 0$.

\end{proof}



\begin{example}
设 \( f \) 是 \([0,1]\) 上正的可导函数,且满足 \(|f'| \leqslant 1\)。记  
\begin{align}
m = \min f(x), \quad M = \max f(x), \quad \beta = \int_{0}^{1} \frac{1}{f(x)} \, \mathrm{d}x. \label{5.1.37}
\end{align}
\begin{enumerate}
\item 求证: \( M \leqslant m e^{\beta} \).

\item 求证:对 \( n > -1 \),有  
\begin{align}
\int_{0}^{1} f^n(x) \, \mathrm{d}x \leqslant \frac{m^{n + 1}}{n + 1} \big( e^{(n + 1)\beta} - 1 \big). \label{5.1.38}
\end{align}  
\end{enumerate}  
\end{example}
\begin{remark}
第2问中,令 \( n = 0 \), 可得 \( \frac{m + 1}{m} \leqslant e^{\beta} \). 式 \(\eqref{5.1.38}\) 两边开 \( n \) 次方根, 再令 \( n \to +\infty \), 可得 \( M \leqslant m e^{\beta} \).
\end{remark}
\begin{proof}
\begin{enumerate}
\item 设 \( m = f(x), M = f(y) \), 则有  
\[
\ln M - \ln m = \ln f(y) - \ln f(x) = \int_{x}^{y} \frac{f'(t)}{f(t)} \, \mathrm{d}t \leqslant \int_{0}^{1} \frac{1}{f(t)} \, \mathrm{d}t = \beta.
\]  
因而有 \( M \leqslant m e^{\beta} \).

\item 设  
\[
h_1(t) = \frac{e^{(n + 1)\beta_1(t)} - 1}{n + 1} f^{n + 1}(t) - \int_{0}^{t} f^n(x) \, \mathrm{d}x, \quad t \in [0,1],
\]  
\[
h_2(t) = \frac{e^{(n + 1)\beta_2(t)} - 1}{n + 1} f^{n + 1}(t) - \int_{t}^{1} f^n(x) \, \mathrm{d}x, \quad t \in [0,1],
\]  
其中  
\[
\beta_1(t) = \int_{0}^{t} \frac{1}{f(x)} \, \mathrm{d}x, \quad \beta_2(t) = \int_{t}^{1} \frac{1}{f(x)} \, \mathrm{d}x,
\]  
则有 \( \beta_1 \geqslant 0, \beta_2 \geqslant 0, h_1(0) = 0, h_2(1) = 0 \),且  
\[
\begin{aligned}
h_1'(t) &= e^{(n + 1)\beta_1(t)} f^n(t) + \big( e^{(n + 1)\beta_1(t)} - 1 \big) f^n(t) f'(t) - f^n(t) \\
&= f^n(t) \big( e^{(n + 1)\beta_1(t)} - 1 \big) \big( 1 + f'(t) \big) \geqslant 0,
\end{aligned}
\]  
\[
\begin{aligned}
h_2'(t) &= -e^{(n + 1)\beta_2(t)} f^n(t) + \big( e^{(n + 1)\beta_2(t)} - 1 \big) f^n(t) f'(t) + f^n(t) \\
&= f^n(t) \big( e^{(n + 1)\beta_2(t)} - 1 \big) \big( -1 + f'(t) \big) \leqslant 0,
\end{aligned}
\]  
这说明 \( h_1 \) 在 \([0,1]\) 上单调递增,而 \( h_2 \) 在 \([0,1]\) 上单调递减。于是 \( h_1 \) 和 \( h_2 \) 都是非负函数,即  
\begin{align}
\int_{0}^{t} f^n(x) \, \mathrm{d}x \leqslant \frac{e^{(n + 1)\beta_1(t)} - 1}{n + 1} f^{n + 1}(t), \label{5.1.39}
\end{align}  
\begin{align}
\int_{t}^{1} f^n(x) \, \mathrm{d}x \leqslant \frac{e^{(n + 1)\beta_2(t)} - 1}{n + 1} f^{n + 1}(t). \label{5.1.40}
\end{align}  
将以上两式相加,可得  
\begin{align}
\int_{0}^{1} f^n(x) \, \mathrm{d}x \leqslant \frac{e^{(n + 1)\beta_1(t)} + e^{(n + 1)\beta_2(t)} - 2}{n + 1} f^{n + 1}(t). \label{5.1.41}
\end{align}  
\hyperref[proposition:常用不等式3]{容易证明}对任意 \( x > 0, y > 0 \) 有  
\[
e^x + e^y - 2 < e^{x + y} - 1.
\]  
因此从式 \(\eqref{5.1.41}\) 可得  
\[
\int_{0}^{1} f^n(x) \, \mathrm{d}x \leqslant \frac{e^{(n + 1)(\beta_1(t) + \beta_2(t))} - 1}{n + 1} f^{n + 1}(t) = \frac{e^{(n + 1)\beta} - 1}{n + 1} f^{n + 1}(t),
\]  
这里 \( t \in [0,1] \) 是任意的。故式 \(\eqref{5.1.38}\) 成立。
\end{enumerate}

\end{proof}

\begin{example}
设 $f \in C[a,b]$ 是一个正的连续函数,且满足 Lipschitz 条件
$$|f(x) - f(y)| \leqslant L|x - y|.$$
对于区间 $[c,d] \subset [a,b]$,记
$$\beta = \int_a^b \frac{1}{f(x)} \mathrm{d}x, \quad \alpha = \int_c^d \frac{1}{f(x)} \mathrm{d}x.$$
求证:
\begin{align}
\int_a^b f(x) \mathrm{d}x \leqslant \frac{\mathrm{e}^{2L\beta} - 1}{2L\alpha} \int_c^d f(x) \mathrm{d}x. \label{5.1.42}
\end{align}
\end{example}
\begin{proof}
只需证明对任意的 $t \in [a,b]$,有
\begin{align}
\int_a^b f(x) \mathrm{d}x \leqslant \frac{\mathrm{e}^{2L\beta} - 1}{2L} f^2(t), \label{5.1.43}
\end{align}
这是因为将式 \eqref{5.1.43} 两端除以 $f(t)$,然后关于变量 $t$ 在区间 $[c,d]$ 上积分,即得式 \eqref{5.1.42}. 不妨假设 $a = 0, b = 1$,不然考虑新的函数 $g(t) =(b - a)f(a(1-t) + bt)= (b - a)f(a + (b - a)t), t \in [0,1]$. $g$ 满足 Lipschitz 条件 $|g(x_1) - g(x_2)| \leqslant L_1|x_1 - x_2|, L_1 = (b - a)^2L$. 由于 $f$ 的 Bernstein 多项式 $B_n(f)$ 保持 $f$ 的 Lipschitz 常数,而且在 $[0,1]$ 上一致收敛于 $f$,我们一开始就可以假设 $f$ 是可导的,此时 $|f'| \leqslant L$.

以下就在 $a = 0, b = 1$ 且 $|f'| \leqslant L$ 的条件下证明式 \eqref{5.1.43}. 设
$$h_1(t) = \frac{\mathrm{e}^{2L\beta_1(t)} - 1}{2L} f^2(t) - \int_0^t f(x) \mathrm{d}x, \quad t \in [0,1],$$
$$h_2(t) = \frac{\mathrm{e}^{2L\beta_2(t)} - 1}{2L} f^2(t) - \int_t^1 f(x) \mathrm{d}x, \quad t \in [0,1],$$
其中
$$\beta_1(t) = \int_0^t \frac{1}{f(x)} \mathrm{d}x, \quad \beta_2(t) = \int_t^1 \frac{1}{f(x)} \mathrm{d}x.$$
则有 $h_1(0) = 0, h_2(1) = 0$,且
$$
\begin{aligned}
h_1'(t) &= \mathrm{e}^{2L\beta_1(t)} f(t) + \frac{\mathrm{e}^{2L\beta_1(t)} - 1}{L} f(t) f'(t) - f(t) \\
&= \frac{\mathrm{e}^{2L\beta_1(t)} - 1}{L} f(t) (L + f'(t)) \geqslant 0,
\end{aligned}
$$
$$
\begin{aligned}
h_2'(t) &= -\mathrm{e}^{2L\beta_2(t)} f(t) + \frac{\mathrm{e}^{2L\beta_2(t)} - 1}{L} f(t) f'(t) + f(t) \\
&= \frac{\mathrm{e}^{2L\beta_2(t)} - 1}{L} f(t) (f'(t) - L) \leqslant 0.
\end{aligned}
$$
这说明 $h_1$ 在 $[0,1]$ 上单调递增,而 $h_2$ 在 $[0,1]$ 上单调递减. 于是 $h_1$ 和 $h_2$ 都是非负函数,即
\begin{align}
\int_0^t f(x) \mathrm{d}x \leqslant \frac{\mathrm{e}^{2L\beta_1(t)} - 1}{2L} f^2(t), \label{5.1.44}
\end{align}
\begin{align}
\int_t^1 f(x) \mathrm{d}x \leqslant \frac{\mathrm{e}^{2L\beta_2(t)} - 1}{2L} f^2(t). \label{5.1.45}
\end{align}
将此两式相加,可得
\begin{align}
\int_0^1 f(x) \mathrm{d}x \leqslant \frac{\mathrm{e}^{2L\beta_1(t)} + \mathrm{e}^{2L\beta_2(t)} - 2}{2L} f^2(t). \label{5.1.46}
\end{align}
\hyperref[proposition:常用不等式3]{容易证明}对任意 $x > 0, y > 0$ 有
$$\mathrm{e}^x + \mathrm{e}^y - 2 < \mathrm{e}^{x + y} - 1.$$
因此从式 \eqref{5.1.46} 可得
$$\int_0^1 f(x) \mathrm{d}x \leqslant \frac{\mathrm{e}^{2L(\beta_1(t) + \beta_2(t))} - 1}{2L} f^2(t) = \frac{\mathrm{e}^{2L\beta} - 1}{2L} f^2(t).$$
即式 \eqref{5.1.43} 成立.

\end{proof}


\end{document}
\documentclass[../../main.tex]{subfiles}
\graphicspath{{\subfix{../../image/}}} % 指定图片目录,后续可以直接使用图片文件名。

% 例如:
% \begin{figure}[H]
% \centering
% \includegraphics[scale=0.4]{图.png}
% \caption{}
% \label{figure:图}
% \end{figure}
% 注意:上述\label{}一定要放在\caption{}之后,否则引用图片序号会只会显示??.

\begin{document}

\section{级数证明}

\begin{example}
设 $f \in \mathbb{R}[x]$ 是只有正实根的多项式,求 $\frac{f'(x)}{f(x)}$ 在 $x = 0$ 幂级数展开和收敛域.
\end{example}
\begin{proof}
设$f(x) =a(x-x_1)^{k_1}(x-x_2)^{k_2}\cdots(x-x_n)^{k_n}$,其中$a\ne 0$,并且
\begin{align*}
0<x_1<x_2<\cdots <x_n,k_i\in \mathbb{N}.
\end{align*}
从而
\begin{align*}
\frac{f'(x)}{f(x)}&=\left[ \ln f(x) \right]'=\left[ \ln a+k_1\ln(x-x_1)+k_2\ln(x-x_2)+\cdots +k_n\ln(x-x_n) \right]' \\
&=\frac{k_1}{x-x_1}+\frac{k_2}{x-x_2}+\cdots +\frac{k_n}{x-x_n} =\sum_{j=1}^n{\frac{k_j}{x-x_j}}\\
&=-\frac{k_j}{x_j}\sum_{j=1}^n{\frac{1}{1-\frac{x}{x_j}}} =-\sum_{j=1}^n{\frac{k_j}{x_j}\sum_{m=0}^{\infty}{\left( \frac{x}{x_j} \right)^m}}\\
&=-\sum_{m=0}^{\infty}{\sum_{j=1}^n{\frac{k_j}{x_j^{m+1}}}x^m}.
\end{align*}
显然收敛半径就是$x_1$,注意到
\begin{align*}
\lim_{m\rightarrow +\infty}\sum_{j=1}^n{\frac{k_j}{x_j^{m+1}}}x_1^m=\frac{k_1}{x_1}\ne 0,
\end{align*}
故收敛域为$(-x_1,x_1)$.
\end{proof}

\begin{example}
设 $e^{a_n} = a_n + e^{b_n},a_n > 0$, 若 $\sum\limits_{n=1}^{\infty}a_n$ 收敛, 证明: $\sum\limits_{n=1}^{\infty}b_n$ 收敛.
\end{example}
\begin{proof}
显然$e^{b_n}=e^{a_n}-a_n\geqslant 1$,故$b_n\geqslant 0$,并且由$\sum_{n=1}^{\infty}a_n$收敛知$a_n\rightarrow 0$。于是
\begin{align*}
b_n&=\ln\left( e^{a_n}-a_n \right)=\ln e^{a_n}+\ln\left( 1-a_ne^{-a_n} \right)\\
&=a_n+O\left( a_ne^{-a_n} \right),n\rightarrow \infty.
\end{align*}
注意到$O\left( a_ne^{-a_n} \right)\leqslant a_n$,故$\sum_{n=1}^{\infty}O\left( a_ne^{-a_n} \right)$也收敛,因此$\sum_{n=1}^{\infty}b_n$收敛。
\end{proof}

\begin{example}
设 $\{a_n\}$ 是递减正数列且 $\sum\limits_{n=1}^{\infty}a_n = +\infty$, 证明
\begin{align*}
\lim_{n\rightarrow \infty}\frac{a_2+a_4+\cdots +a_{2n}}{a_1+a_3+\cdots +a_{2n-1}} = 1.
\end{align*}
\end{example}
\begin{proof}
由条件可知对$\forall n\in \mathbb{N}$,都有
\begin{align*}
a_2+a_4+\cdots +a_{2n}\leqslant a_1+a_3+\cdots +a_{2n-1},
\end{align*}
故$A\leqslant 1$.注意到
\begin{align*}
&\frac{a_2+a_4+\cdots +a_{2n}}{a_1+a_3+\cdots +a_{2n-1}}\geqslant \frac{a_3+a_5+\cdots +a_{2n+1}}{a_1+a_3+\cdots +a_{2n-1}}=1-\frac{a_1-a_{2n+1}}{a_1+a_3+\cdots +a_{2n-1}}
\\
&\geqslant 1-\frac{a_1}{\frac{a_2+a_4+\cdots +a_{2n}}{2}+\frac{a_1+a_3+\cdots +a_{2n-1}}{2}}=1-\frac{2a_1}{\sum\limits_{i=1}^n{a_i}}\rightarrow 1,n\rightarrow \infty .
\end{align*}
故$A\geqslant 1$.因此$A=1$.
\end{proof}

\begin{proposition}\label{proposition:级数经典命题1}
设$a_n$递减到$0$,证明:$\sum\limits_{n=1}^\infty n\left(a_n-a_{n+1}\right)$收敛的充要条件是$\sum\limits_{n=1}^\infty a_n$收敛,并且$\sum\limits_{n=1}^\infty n\left(a_n-a_{n+1}\right)=\sum\limits_{n=1}^\infty a_n$.
\end{proposition}
\begin{note}
\eqref{eq:105.12}式可由\hyperref[theorem:Abel变换]{Abel变换}直接得到,也可以采用下述证明一样的强行凑裂项的思路.
\end{note}
\begin{proof}
注意到
\begin{align}
\sum_{k=1}^n{k\left( a_k-a_{k+1} \right)}&=\sum_{k=1}^n{\left[ ka_k-\left( k+1 \right) a_{k+1} \right]}+\sum_{k=1}^n{\left[ \left( k+1 \right) a_{k+1}-ka_{k+1} \right]} \nonumber \\
&=a_1-\left( n+1 \right) a_{n+1}+\sum_{k=1}^n{a_{k+1}}=\sum_{k=1}^{n+1}{a_k}-\left( n+1 \right) a_{n+1}. \label{eq:105.12}
\end{align}

{\heiti 充分性:}若$\sum_{n=1}^{\infty}{a_n}$收敛,则由\refpro{proposition:单调收敛级数的阶}可知$\lim\limits_{n\rightarrow \infty}na_n=0$.再由\eqref{eq:105.12}式可得
$$\sum_{k=1}^{\infty}{k\left( a_k-a_{k+1} \right)}=\sum_{k=1}^{\infty}{a_k}<+\infty .$$

{\heiti 必要性:}若$\sum_{n=1}^{\infty}{n\left( a_n-a_{n+1} \right)}$收敛,则由$\{ a_n \}$的单调性知,对$\forall m\in \mathbb{N}$,当$n\geqslant m$时,有
$$\sum_{k=1}^n{a_k}=\sum_{k=1}^m{a_k}+\sum_{k=m+1}^n{a_k}\geqslant \sum_{k=1}^m{a_k}+\left( n-m \right) a_n.$$
又由\eqref{eq:105.12}式和$\sum_{n=1}^{\infty}{n\left( a_n-a_{n+1} \right)}$收敛知,存在$A>0$,使得
$$\sum_{k=1}^{n-1}{k\left( a_k-a_{k+1} \right)}=\sum_{k=1}^n{a_k}-na_n\leqslant A,\forall n\in \mathbb{N} .$$
故
$$A\geqslant \sum_{k=1}^n{a_k}-na_n\geqslant \sum_{k=1}^m{a_k}+\left( n-m \right) a_n-na_n=\sum_{k=1}^m{a_k}-ma_n.$$
令$n\rightarrow +\infty$得$\sum_{k=1}^m{a_k}\leqslant A$.再由$m$的任意性可知$\sum_{k=1}^{\infty}{a_k}$收敛.此时由\refpro{proposition:单调收敛级数的阶}可知$\lim\limits_{n\rightarrow \infty}na_n=0$,再由\eqref{eq:105.12}式可知
$$\sum_{k=1}^{\infty}{k\left( a_k-a_{k+1} \right)}=\sum_{k=1}^{\infty}{a_k}.$$
\end{proof}

\begin{example}
设$a_n$递减到$0$,且$\sum\limits_{n=1}^\infty a_n$发散,证明
$$\int_1^\infty \frac{\ln f(x)}{x^2}dx$$
发散,这里$f(x)=\sum\limits_{n=1}^\infty a_n^n x^n$.
\end{example}
\begin{proof}
由$\lim\limits_{n\rightarrow \infty}a_n$可知$\lim\limits_{n\rightarrow \infty}\sqrt[n]{a_{n}^{n}}=\lim\limits_{n\rightarrow \infty}a_n=0$,故$f\left( x \right)$的收敛域为$\mathbb{R}$.显然$f>0,x>0$,且$f$在$\left( 0,+\infty \right)$上递增.
待定$\left\{ b_n \right\}$满足:$b_n\nearrow +\infty$.从而
\begin{align*}
\int_{b_n}^{b_{n+1}}{\frac{\ln f\left( x \right)}{x^2}\mathrm{d}x}&\geqslant \int_{b_n}^{b_{n+1}}{\frac{\ln f\left( b_n \right)}{x^2}\mathrm{d}x}=\ln f\left( b_n \right) \left( \frac{1}{b_n}-\frac{1}{b_{n+1}} \right) \\
&\geqslant \ln \left( a_{n}^{n}b_{n}^{n} \right) \left( \frac{1}{b_n}-\frac{1}{b_{n+1}} \right) =n\ln \left( a_nb_n \right) \left( \frac{1}{b_n}-\frac{1}{b_{n+1}} \right) .
\end{align*}
取$b_n=\frac{C}{a_n},C>\max \left\{ 1,a_1 \right\}$,则
\begin{align*}
\int_{b_n}^{b_{n+1}}{\frac{\ln f\left( x \right)}{x^2}\mathrm{d}x}&\geqslant n\ln \left( a_nb_n \right) \left( \frac{1}{b_n}-\frac{1}{b_{n+1}} \right) =\frac{\ln C}{C}n\left( a_n-a_{n+1} \right) .
\end{align*}
由\refpro{proposition:级数经典命题1}可知$\sum_{n=1}^{\infty}{n\left( a_n-a_{n+1} \right)}$发散.故
\begin{align*}
\int_1^{+\infty}{\frac{\ln f\left( x \right)}{x^2}\mathrm{d}x}&\geqslant \sum_{n=1}^{\infty}{\int_{b_n}^{b_{n+1}}{\frac{\ln f\left( x \right)}{x^2}\mathrm{d}x}}\geqslant \frac{\ln C}{C}\sum_{n=1}^{\infty}{n\left( a_n-a_{n+1} \right)}=+\infty .
\end{align*}
\end{proof}

\begin{example}
证明:
\begin{enumerate}
\item $$\sum_{n=1}^\infty \frac{1}{(n+1)\sqrt[n]{n}} \leqslant p,\forall p \in (1,+\infty).$$

\item $$\sum_{n=1}^\infty \frac{1}{(n+1)\sqrt[n]{n}} \geqslant p,\forall p \in (0,1).$$
\end{enumerate}
\end{example}
\begin{note}
注意强行凑裂项和熟悉\hyperref[theorem:Bernoulli不等式]{Bernoulli不等式}.
\end{note}
\begin{proof}
\begin{enumerate}
\item \begin{align}
&\quad \quad \frac{1}{\left( n+1 \right) \sqrt[p]{n}}\leqslant p\left( \frac{1}{\sqrt[p]{n}}-\frac{1}{\sqrt[p]{n+1}} \right) \label{eq:104.11} \\
&\Longleftrightarrow \sqrt[p]{n}\left( \frac{1}{\sqrt[p]{n}}-\frac{1}{\sqrt[p]{n+1}} \right) =1-\sqrt[p]{1-\frac{1}{n+1}}\geqslant \frac{1}{p\left( n+1 \right)} \nonumber \\
&\Longleftrightarrow \sqrt[p]{1-\frac{1}{n+1}}\leqslant 1-\frac{1}{p\left( n+1 \right)}.\nonumber
\end{align}
下证$\sqrt[p]{1-\frac{1}{n+1}}\leqslant 1-\frac{1}{p\left( n+1 \right)}$.令$f\left( x \right) \triangleq \sqrt[p]{1-x}-\frac{x}{p}$,则
\begin{align*}
f\prime \left( x \right) &=-\frac{1}{p}\left( 1-x \right) ^{\frac{1}{p}-1}+\frac{1}{p}=\frac{1}{p}\left[ 1-\left( 1-x \right) ^{\frac{1}{p}-1} \right] <0.
\end{align*}
故
\begin{align*}
f\left( x \right) &\leqslant f\left( 0 \right) =1\Longleftrightarrow \sqrt[p]{1-x}\leqslant 1-\frac{x}{p}.
\end{align*}
令$x=\frac{1}{n+1}$得$\sqrt[p]{1-\frac{1}{n+1}}\leqslant 1-\frac{1}{p\left( n+1 \right)}$,从而\eqref{eq:104.11}式成立.故
\begin{align*}
\sum_{n=1}^{\infty}{\frac{1}{\left( n+1 \right) \sqrt[p]{n}}}&\leqslant \sum_{n=1}^{\infty}{p\left( \frac{1}{\sqrt[p]{n}}-\frac{1}{\sqrt[p]{n+1}} \right)}=p.
\end{align*}

\item \begin{align}
&\quad \quad \frac{1}{\left( n+1 \right) \sqrt[p]{n}}\geqslant p\left( \frac{1}{\sqrt[p]{n}}-\frac{1}{\sqrt[p]{n+1}} \right) \label{eq:104.13} \\
&\Longleftrightarrow \sqrt[p]{n}\left( \frac{1}{\sqrt[p]{n}}-\frac{1}{\sqrt[p]{n+1}} \right) =1-\sqrt[p]{1-\frac{1}{n+1}}\leqslant \frac{1}{p\left( n+1 \right)}\nonumber \\
&\Longleftrightarrow \sqrt[p]{1-\frac{1}{n+1}}\geqslant 1-\frac{1}{p\left( n+1 \right)}.\nonumber
\end{align}
下证$\sqrt[p]{1-\frac{1}{n+1}}\geqslant 1-\frac{1}{p\left( n+1 \right)}$.令$f\left( x \right) \triangleq \sqrt[p]{1-x}-\frac{x}{p}$,则
\begin{align*}
f\prime \left( x \right) &=-\frac{1}{p}\left( 1-x \right) ^{\frac{1}{p}-1}+\frac{1}{p}=\frac{1}{p}\left[ 1-\left( 1-x \right) ^{\frac{1}{p}-1} \right] >0.
\end{align*}
故
\begin{align*}
f\left( x \right) &\geqslant f\left( 0 \right) =1\Longleftrightarrow \sqrt[p]{1-x}\geqslant 1-\frac{x}{p}.
\end{align*}
令$x=\frac{1}{n+1}$得$\sqrt[p]{1-\frac{1}{n+1}}\geqslant 1-\frac{1}{p\left( n+1 \right)}$,从而\eqref{eq:104.13}式成立.故
\begin{align*}
\sum_{n=1}^{\infty}{\frac{1}{\left( n+1 \right) \sqrt[p]{n}}}&\geqslant \sum_{n=1}^{\infty}{p\left( \frac{1}{\sqrt[p]{n}}-\frac{1}{\sqrt[p]{n+1}} \right)}=p.
\end{align*}
\end{enumerate}
\end{proof}

\begin{example}
对$t \in \mathbb{R}$,证明:
$$\sum_{n=1}^\infty \frac{t^{n-1}}{n^n}=\int_0^1 \frac{1}{x^{tx}}dx.$$
\end{example}
\begin{proof}
\begin{align*}
&\int_0^1{\frac{1}{x^{tx}}\mathrm{d}x}=\int_0^1{e^{-tx\ln x}\mathrm{d}x}=\int_0^1{\sum_{n=0}^{\infty}{\frac{\left( -tx\ln x \right) ^n}{n!}\mathrm{d}x}}
\\
&=\sum_{n=0}^{\infty}{\int_0^1{\frac{\left( -tx\ln x \right) ^n}{n!}\mathrm{d}x}}=\sum_{n=0}^{\infty}{\frac{\left( -t \right) ^n}{n!}\int_0^1{x^n\ln ^nx\mathrm{d}x}}
\\
&\xlongequal{x=e^{-y}}\sum_{n=0}^{\infty}{\frac{t^n}{n!}\int_0^{+\infty}{e^{-\left( n+1 \right) y}y^n\mathrm{d}y}}=\sum_{n=0}^{\infty}{\frac{t^n}{n!\left( n+1 \right) ^{n+1}}\int_0^{+\infty}{e^{-y}y^n\mathrm{d}y}}
\\
&=\sum_{n=0}^{\infty}{\frac{t^n\Gamma \left( n+1 \right)}{n!\left( n+1 \right) ^{n+1}}}=\sum_{n=0}^{\infty}{\frac{t^n}{\left( n+1 \right) ^{n+1}}}=\sum_{n=1}^{\infty}{\frac{t^n}{n^n}}.
\end{align*}
\end{proof}

\begin{proposition}\label{proposition:没有收敛最慢的级数也没有发散最慢的级数}
\begin{enumerate}
\item 设正项级数$\sum\limits_{n=1}^\infty a_n$收敛,$a_n>0$,则存在$A_n$使得$a_n=o(A_n)$和$\sum\limits_{n=1}^\infty A_n$收敛.

\item 设正项级数$\sum\limits_{n=1}^\infty a_n$发散,$a_n>0$,则存在$A_n$使得$A_n=o(a_n)$和$\sum\limits_{n=1}^\infty A_n$发散.
\end{enumerate}
\end{proposition}
\begin{note}
这个命题说明:\textbf{没有收敛最慢的级数,也没有发散最慢的级数.}
\end{note}
\begin{proof}
\begin{enumerate}
\item 令
\begin{align*}
A_n\triangleq \sqrt{\sum_{k=n}^{\infty}{a_k}}-\sqrt{\sum_{k=n+1}^{\infty}{a_k}},
\end{align*}
则
\begin{align*}
\sum_{n=1}^{\infty}{A_n}=\sqrt{\sum_{k=1}^{\infty}{a_k}}<+\infty .
\end{align*}
\begin{align*}
\lim_{n\rightarrow \infty}\frac{a_n}{A_n}=\lim_{n\rightarrow \infty}\frac{a_n}{\sqrt{\sum\limits_{k=n}^{\infty}{a_k}}-\sqrt{\sum\limits_{k=n+1}^{\infty}{a_k}}}=\lim_{n\rightarrow \infty}\frac{a_n\left( \sqrt{\sum\limits_{k=n}^{\infty}{a_k}}+\sqrt{\sum\limits_{k=n+1}^{\infty}{a_k}} \right)}{a_n}=0.
\end{align*}
故$a_n=o\left( A_n \right) ,n\rightarrow \infty .$

\item 令
\begin{align*}
A_1=1,\quad A_n\triangleq \sqrt{\sum_{k=1}^n{a_k}}-\sqrt{\sum_{k=1}^{n-1}{a_k}},\,\,n=2,3,\cdots .
\end{align*}
则
\begin{align*}
\sum_{n=2}^{\infty}{A_n}=\lim_{n\rightarrow \infty}\left( \sqrt{\sum_{k=1}^n{a_k}}-\sqrt{a_1} \right) =+\infty .
\end{align*}
\begin{align*}
\lim_{n\rightarrow \infty}\frac{A_n}{a_n}=\lim_{n\rightarrow \infty}\frac{\sqrt{\sum\limits_{k=1}^n{a_k}}-\sqrt{\sum\limits_{k=1}^{n-1}{a_k}}}{a_n}=\lim_{n\rightarrow \infty}\frac{a_n}{a_n\left( \sqrt{\sum\limits_{k=1}^n{a_k}}+\sqrt{\sum\limits_{k=1}^{n-1}{a_k}} \right)}=0.
\end{align*}
故$A_n=o\left( a_n \right) ,n\rightarrow \infty .$
\end{enumerate}
\end{proof}

\begin{example}
设正项级数$\sum\limits_{n=1}^\infty \frac{1}{p_n}<\infty$,证明
$$\sum_{n=1}^\infty \frac{n^2 p_n}{(p_1+p_2+\cdots +p_n)^2}<\infty.$$
\end{example}
\begin{remark}
本题的想法就是把$\sum\limits_{n=1}^\infty \frac{n^2 p_n}{(p_1+p_2+\cdots +p_n)^2}$放大为阶更小的量,从而其收敛.
\end{remark}
\begin{proof}
记$S_0=0,S_n=\sum_{k=1}^n{p_k}$,则对$N\geqslant 2$,有
\begin{align*}
&\sum_{n=2}^N{\frac{n^2p_n}{(p_1+p_2+\cdots +p_n)^2}}=\sum_{n=2}^N{\frac{n^2p_n}{S_{n}^{2}}}=\sum_{n=2}^N{\frac{n^2\left( S_n-S_{n-1} \right)}{S_{n}^{2}}} \\
&=\sum_{n=2}^N{n^2\int_{S_{n-1}}^{S_n}{\frac{1}{S_{n}^{2}}\mathrm{d}x}}\leqslant \sum_{n=2}^N{n^2\int_{S_{n-1}}^{S_n}{\frac{1}{x^2}\mathrm{d}x}}=\sum_{n=2}^N{n^2\left( \frac{1}{S_{n-1}}-\frac{1}{S_n} \right)} \\
&=\sum_{n=2}^N{\left[ \frac{n^2}{S_{n-1}}-\frac{\left( n+1 \right) ^2}{S_n} \right]}+\sum_{n=2}^N{\frac{\left( n+1 \right) ^2-n^2}{S_n}} \\
&=\frac{4}{S_1}-\frac{\left( N+1 \right) ^2}{S_N}+\sum_{n=2}^N{\frac{2n+1}{S_n}} \\
&\leqslant \frac{4}{S_1}+3\sum_{n=2}^N{\frac{n}{S_n}}=\frac{4}{S_1}+3\sum_{n=2}^N{\left( \frac{n\sqrt{p_n}}{S_n}\cdot \frac{1}{\sqrt{p_n}} \right)} \\
&\overset{Cauchy\text{不等式}}{\leqslant}\frac{4}{S_1}+3\sqrt{\sum_{n=2}^N{\frac{n^2p_n}{S_{n}^{2}}}\cdot \sum_{n=2}^N{\frac{1}{p_n}}} \\
&\leqslant \frac{4}{S_1}+C\sqrt{\sum_{n=2}^N{\frac{n^2p_n}{S_{n}^{2}}}}.
\end{align*}
从而
\begin{align*}
\sqrt{\sum_{n=2}^N{\frac{n^2p_n}{S_{n}^{2}}}}\leqslant \frac{4}{S_1}\frac{1}{\sqrt{\sum\limits_{n=2}^N{\frac{n^2p_n}{S_{n}^{2}}}}}+C.
\end{align*}
若$\sum_{n=2}^{\infty}{\frac{n^2p_n}{S_{n}^{2}}}$发散,则对上式令$N\rightarrow +\infty$得$+\infty \leqslant C$矛盾!故$\sum_{n=2}^{\infty}{\frac{n^2p_n}{S_{n}^{2}}}<+\infty .$
\end{proof}




































\end{document}
\documentclass[../../main.tex]{subfiles}% 注意这里的文件路径不能用 ./main.tex ,否则用latexmk编译子文件会报错
\graphicspath{{\subfix{./image/}}} % 指定图片目录,后续可以直接使用图片文件名
% 注意这里的文件路径不能用 ../../image/ ,否则用latexmk编译子文件会报错

% 例如:
% \begin{figure}[H]
% \centering
% \includegraphics[scale=0.3]{图.png}
% \caption{}
% \label{figure:图}
% \end{figure}
% 注意:上述\label{}一定要放在\caption{}之后,否则引用图片序号会只会显示??.

\begin{document}

\section{级数证明}

\begin{example}
设 $f \in \mathbb{R}[x]$ 是只有正实根的多项式,求 $\frac{f'(x)}{f(x)}$ 在 $x = 0$ 幂级数展开和收敛域.
\end{example}
\begin{proof}
设$f(x) =a(x-x_1)^{k_1}(x-x_2)^{k_2}\cdots(x-x_n)^{k_n}$,其中$a\ne 0$,并且
\begin{align*}
0<x_1<x_2<\cdots <x_n,k_i\in \mathbb{N}.
\end{align*}
从而
\begin{align*}
\frac{f'(x)}{f(x)}&=\left[ \ln f(x) \right]'=\left[ \ln a+k_1\ln(x-x_1)+k_2\ln(x-x_2)+\cdots +k_n\ln(x-x_n) \right]' \\
&=\frac{k_1}{x-x_1}+\frac{k_2}{x-x_2}+\cdots +\frac{k_n}{x-x_n} =\sum\limits_{j=1}^n{\frac{k_j}{x-x_j}}\\
&=-\frac{k_j}{x_j}\sum\limits_{j=1}^n{\frac{1}{1-\frac{x}{x_j}}} =-\sum\limits_{j=1}^n{\frac{k_j}{x_j}\sum\limits_{m=0}^{\infty}{\left( \frac{x}{x_j} \right)^m}}\\
&=-\sum\limits_{m=0}^{\infty}{\sum\limits_{j=1}^n{\frac{k_j}{x_j^{m+1}}}x^m}.
\end{align*}
显然收敛半径就是$x_1$,注意到
\begin{align*}
\lim_{m\rightarrow +\infty}\sum_{j=1}^n{\frac{k_j}{x_j^{m+1}}}x_1^m=\frac{k_1}{x_1}\ne 0,
\end{align*}
故收敛域为$(-x_1,x_1)$.

\end{proof}

\begin{example}
设 $e^{a_n} = a_n + e^{b_n},a_n > 0$, 若 $\sum\limits_{n=1}^{\infty}a_n$ 收敛, 证明: $\sum\limits_{n=1}^{\infty}b_n$ 收敛.
\end{example}
\begin{proof}
显然$e^{b_n}=e^{a_n}-a_n\geqslant 1$,故$b_n\geqslant 0$,并且由$\sum_{n=1}^{\infty}a_n$收敛知$a_n\rightarrow 0$。于是
\begin{align*}
b_n&=\ln\left( e^{a_n}-a_n \right)=\ln e^{a_n}+\ln\left( 1-a_ne^{-a_n} \right)\\
&=a_n+O\left( a_ne^{-a_n} \right),n\rightarrow \infty.
\end{align*}
注意到$O\left( a_ne^{-a_n} \right)\leqslant a_n$,故$\sum_{n=1}^{\infty}O\left( a_ne^{-a_n} \right)$也收敛,因此$\sum_{n=1}^{\infty}b_n$收敛。

\end{proof}

\begin{example}
设 $\{a_n\}$ 是递减正数列且 $\sum\limits_{n=1}^{\infty}a_n = +\infty$, 证明
\begin{align*}
\lim_{n\rightarrow \infty}\frac{a_2+a_4+\cdots +a_{2n}}{a_1+a_3+\cdots +a_{2n-1}} = 1.
\end{align*}
\end{example}
\begin{proof}
由条件可知对$\forall n\in \mathbb{N}$,都有
\begin{align*}
a_2+a_4+\cdots +a_{2n}\leqslant a_1+a_3+\cdots +a_{2n-1},
\end{align*}
故$A\leqslant 1$.注意到
\begin{align*}
&\frac{a_2+a_4+\cdots +a_{2n}}{a_1+a_3+\cdots +a_{2n-1}}\geqslant \frac{a_3+a_5+\cdots +a_{2n+1}}{a_1+a_3+\cdots +a_{2n-1}}=1-\frac{a_1-a_{2n+1}}{a_1+a_3+\cdots +a_{2n-1}}
\\
&\geqslant 1-\frac{a_1}{\frac{a_2+a_4+\cdots +a_{2n}}{2}+\frac{a_1+a_3+\cdots +a_{2n-1}}{2}}=1-\frac{2a_1}{\sum\limits_{i=1}^n{a_i}}\rightarrow 1,n\rightarrow \infty .
\end{align*}
故$A\geqslant 1$.因此$A=1$.

\end{proof}

\begin{proposition}\label{proposition:级数经典命题1}
设$a_n$递减到$0$,证明:$\sum\limits_{n=1}^\infty n\left(a_n-a_{n+1}\right)$收敛的充要条件是$\sum\limits_{n=1}^\infty a_n$收敛,并且$\sum\limits_{n=1}^\infty n\left(a_n-a_{n+1}\right)=\sum\limits_{n=1}^\infty a_n$.
\end{proposition}
\begin{note}
\eqref{eq:105.12}式可由\hyperref[theorem:Abel变换]{Abel变换}直接得到,也可以采用下述证明一样的强行凑裂项的思路.
\end{note}
\begin{proof}
注意到
\begin{align}
\sum_{k=1}^n{k\left( a_k-a_{k+1} \right)}&=\sum_{k=1}^n{\left[ ka_k-\left( k+1 \right) a_{k+1} \right]}+\sum_{k=1}^n{\left[ \left( k+1 \right) a_{k+1}-ka_{k+1} \right]} \nonumber \\
&=a_1-\left( n+1 \right) a_{n+1}+\sum_{k=1}^n{a_{k+1}}=\sum_{k=1}^{n+1}{a_k}-\left( n+1 \right) a_{n+1}. \label{eq:105.12}
\end{align}

{\heiti 充分性:}若$\sum_{n=1}^{\infty}{a_n}$收敛,则由\refpro{proposition:单调收敛级数的阶}可知$\lim\limits_{n\rightarrow \infty}na_n=0$.再由\eqref{eq:105.12}式可得
$$\sum_{k=1}^{\infty}{k\left( a_k-a_{k+1} \right)}=\sum_{k=1}^{\infty}{a_k}<+\infty .$$

{\heiti 必要性:}若$\sum_{n=1}^{\infty}{n\left( a_n-a_{n+1} \right)}$收敛,则由$\{ a_n \}$的单调性知,对$\forall m\in \mathbb{N}$,当$n\geqslant m$时,有
$$\sum_{k=1}^n{a_k}=\sum_{k=1}^m{a_k}+\sum_{k=m+1}^n{a_k}\geqslant \sum_{k=1}^m{a_k}+\left( n-m \right) a_n.$$
又由\eqref{eq:105.12}式和$\sum_{n=1}^{\infty}{n\left( a_n-a_{n+1} \right)}$收敛知,存在$A>0$,使得
$$\sum_{k=1}^{n-1}{k\left( a_k-a_{k+1} \right)}=\sum_{k=1}^n{a_k}-na_n\leqslant A,\forall n\in \mathbb{N} .$$
故
$$A\geqslant \sum_{k=1}^n{a_k}-na_n\geqslant \sum_{k=1}^m{a_k}+\left( n-m \right) a_n-na_n=\sum_{k=1}^m{a_k}-ma_n.$$
令$n\rightarrow +\infty$得$\sum_{k=1}^m{a_k}\leqslant A$.再由$m$的任意性可知$\sum_{k=1}^{\infty}{a_k}$收敛.此时由\refpro{proposition:单调收敛级数的阶}可知$\lim\limits_{n\rightarrow \infty}na_n=0$,再由\eqref{eq:105.12}式可知
$$\sum_{k=1}^{\infty}{k\left( a_k-a_{k+1} \right)}=\sum_{k=1}^{\infty}{a_k}.$$

\end{proof}

\begin{example}
设$a_n$递减到$0$,且$\sum\limits_{n=1}^\infty a_n$发散,证明
$$\int_1^\infty \frac{\ln f(x)}{x^2}\mathrm{d}x$$
发散,这里$f(x)=\sum\limits_{n=1}^\infty a_n^n x^n$.
\end{example}
\begin{proof}
由$\lim\limits_{n\rightarrow \infty}a_n$可知$\lim\limits_{n\rightarrow \infty}\sqrt[n]{a_{n}^{n}}=\lim\limits_{n\rightarrow \infty}a_n=0$,故$f\left( x \right)$的收敛域为$\mathbb{R}$.显然$f>0,x>0$,且$f$在$\left( 0,+\infty \right)$上递增.
待定$\left\{ b_n \right\}$满足:$b_n\nearrow +\infty$.从而
\begin{align*}
\int_{b_n}^{b_{n+1}}{\frac{\ln f\left( x \right)}{x^2}\mathrm{d}x}&\geqslant \int_{b_n}^{b_{n+1}}{\frac{\ln f\left( b_n \right)}{x^2}\mathrm{d}x}=\ln f\left( b_n \right) \left( \frac{1}{b_n}-\frac{1}{b_{n+1}} \right) \\
&\geqslant \ln \left( a_{n}^{n}b_{n}^{n} \right) \left( \frac{1}{b_n}-\frac{1}{b_{n+1}} \right) =n\ln \left( a_nb_n \right) \left( \frac{1}{b_n}-\frac{1}{b_{n+1}} \right) .
\end{align*}
取$b_n=\frac{C}{a_n},C>\max \left\{ 1,a_1 \right\}$,则
\begin{align*}
\int_{b_n}^{b_{n+1}}{\frac{\ln f\left( x \right)}{x^2}\mathrm{d}x}&\geqslant n\ln \left( a_nb_n \right) \left( \frac{1}{b_n}-\frac{1}{b_{n+1}} \right) =\frac{\ln C}{C}n\left( a_n-a_{n+1} \right) .
\end{align*}
由\refpro{proposition:级数经典命题1}可知$\sum_{n=1}^{\infty}{n\left( a_n-a_{n+1} \right)}$发散.故
\begin{align*}
\int_1^{+\infty}{\frac{\ln f\left( x \right)}{x^2}\mathrm{d}x}&\geqslant \sum_{n=1}^{\infty}{\int_{b_n}^{b_{n+1}}{\frac{\ln f\left( x \right)}{x^2}\mathrm{d}x}}\geqslant \frac{\ln C}{C}\sum_{n=1}^{\infty}{n\left( a_n-a_{n+1} \right)}=+\infty .
\end{align*}

\end{proof}

\begin{example}
证明:
\begin{enumerate}
\item $$\sum_{n=1}^\infty \frac{1}{(n+1)\sqrt[n]{n}} \leqslant p,\forall p \in (1,+\infty).$$

\item $$\sum_{n=1}^\infty \frac{1}{(n+1)\sqrt[n]{n}} \geqslant p,\forall p \in (0,1).$$
\end{enumerate}
\end{example}
\begin{note}
注意强行凑裂项和熟悉\hyperref[theorem:Bernoulli不等式]{Bernoulli不等式}.
\end{note}
\begin{proof}
\begin{enumerate}
\item \begin{align}
&\quad \quad \frac{1}{\left( n+1 \right) \sqrt[p]{n}}\leqslant p\left( \frac{1}{\sqrt[p]{n}}-\frac{1}{\sqrt[p]{n+1}} \right) \label{eq:104.11} \\
&\Longleftrightarrow \sqrt[p]{n}\left( \frac{1}{\sqrt[p]{n}}-\frac{1}{\sqrt[p]{n+1}} \right) =1-\sqrt[p]{1-\frac{1}{n+1}}\geqslant \frac{1}{p\left( n+1 \right)} \nonumber \\
&\Longleftrightarrow \sqrt[p]{1-\frac{1}{n+1}}\leqslant 1-\frac{1}{p\left( n+1 \right)}.\nonumber
\end{align}
下证$\sqrt[p]{1-\frac{1}{n+1}}\leqslant 1-\frac{1}{p\left( n+1 \right)}$.令$f\left( x \right) \triangleq \sqrt[p]{1-x}-\frac{x}{p}$,则
\begin{align*}
f\prime \left( x \right) &=-\frac{1}{p}\left( 1-x \right) ^{\frac{1}{p}-1}+\frac{1}{p}=\frac{1}{p}\left[ 1-\left( 1-x \right) ^{\frac{1}{p}-1} \right] <0.
\end{align*}
故
\begin{align*}
f\left( x \right) &\leqslant f\left( 0 \right) =1\Longleftrightarrow \sqrt[p]{1-x}\leqslant 1-\frac{x}{p}.
\end{align*}
令$x=\frac{1}{n+1}$得$\sqrt[p]{1-\frac{1}{n+1}}\leqslant 1-\frac{1}{p\left( n+1 \right)}$,从而\eqref{eq:104.11}式成立.故
\begin{align*}
\sum_{n=1}^{\infty}{\frac{1}{\left( n+1 \right) \sqrt[p]{n}}}&\leqslant \sum_{n=1}^{\infty}{p\left( \frac{1}{\sqrt[p]{n}}-\frac{1}{\sqrt[p]{n+1}} \right)}=p.
\end{align*}

\item \begin{align}
&\quad \quad \frac{1}{\left( n+1 \right) \sqrt[p]{n}}\geqslant p\left( \frac{1}{\sqrt[p]{n}}-\frac{1}{\sqrt[p]{n+1}} \right) \label{eq:104.13} \\
&\Longleftrightarrow \sqrt[p]{n}\left( \frac{1}{\sqrt[p]{n}}-\frac{1}{\sqrt[p]{n+1}} \right) =1-\sqrt[p]{1-\frac{1}{n+1}}\leqslant \frac{1}{p\left( n+1 \right)}\nonumber \\
&\Longleftrightarrow \sqrt[p]{1-\frac{1}{n+1}}\geqslant 1-\frac{1}{p\left( n+1 \right)}.\nonumber
\end{align}
下证$\sqrt[p]{1-\frac{1}{n+1}}\geqslant 1-\frac{1}{p\left( n+1 \right)}$.令$f\left( x \right) \triangleq \sqrt[p]{1-x}-\frac{x}{p}$,则
\begin{align*}
f\prime \left( x \right) &=-\frac{1}{p}\left( 1-x \right) ^{\frac{1}{p}-1}+\frac{1}{p}=\frac{1}{p}\left[ 1-\left( 1-x \right) ^{\frac{1}{p}-1} \right] >0.
\end{align*}
故
\begin{align*}
f\left( x \right) &\geqslant f\left( 0 \right) =1\Longleftrightarrow \sqrt[p]{1-x}\geqslant 1-\frac{x}{p}.
\end{align*}
令$x=\frac{1}{n+1}$得$\sqrt[p]{1-\frac{1}{n+1}}\geqslant 1-\frac{1}{p\left( n+1 \right)}$,从而\eqref{eq:104.13}式成立.故
\begin{align*}
\sum_{n=1}^{\infty}{\frac{1}{\left( n+1 \right) \sqrt[p]{n}}}&\geqslant \sum_{n=1}^{\infty}{p\left( \frac{1}{\sqrt[p]{n}}-\frac{1}{\sqrt[p]{n+1}} \right)}=p.
\end{align*}
\end{enumerate}

\end{proof}

\begin{example}
对$t \in \mathbb{R}$,证明:
$$\sum_{n=1}^\infty \frac{t^{n-1}}{n^n}=\int_0^1 \frac{1}{x^{tx}}\mathrm{d}x.$$
\end{example}
\begin{proof}
\begin{align*}
&\int_0^1{\frac{1}{x^{tx}}\mathrm{d}x}=\int_0^1{e^{-tx\ln x}\mathrm{d}x}=\int_0^1{\sum_{n=0}^{\infty}{\frac{\left( -tx\ln x \right) ^n}{n!}\mathrm{d}x}}
\\
&=\sum_{n=0}^{\infty}{\int_0^1{\frac{\left( -tx\ln x \right) ^n}{n!}\mathrm{d}x}}=\sum_{n=0}^{\infty}{\frac{\left( -t \right) ^n}{n!}\int_0^1{x^n\ln ^nx\mathrm{d}x}}
\\
&\xlongequal{x=e^{-y}}\sum_{n=0}^{\infty}{\frac{t^n}{n!}\int_0^{+\infty}{e^{-\left( n+1 \right) y}y^n\mathrm{d}y}}=\sum_{n=0}^{\infty}{\frac{t^n}{n!\left( n+1 \right) ^{n+1}}\int_0^{+\infty}{e^{-y}y^n\mathrm{d}y}}
\\
&=\sum_{n=0}^{\infty}{\frac{t^n\Gamma \left( n+1 \right)}{n!\left( n+1 \right) ^{n+1}}}=\sum_{n=0}^{\infty}{\frac{t^n}{\left( n+1 \right) ^{n+1}}}=\sum_{n=1}^{\infty}{\frac{t^n}{n^n}}.
\end{align*}

\end{proof}

\begin{proposition}\label{proposition:没有收敛最慢的级数也没有发散最慢的级数}
\begin{enumerate}
\item 设正项级数$\sum\limits_{n=1}^\infty a_n$收敛,$a_n>0$,则存在$A_n$使得$a_n=o(A_n)$和$\sum\limits_{n=1}^\infty A_n$收敛.

\item 设正项级数$\sum\limits_{n=1}^\infty a_n$发散,$a_n>0$,则存在$A_n$使得$A_n=o(a_n)$和$\sum\limits_{n=1}^\infty A_n$发散.
\end{enumerate}
\end{proposition}
\begin{note}
这个命题说明:\textbf{没有收敛最慢的级数,也没有发散最慢的级数.}
\end{note}
\begin{proof}
\begin{enumerate}
\item 令
\begin{align*}
A_n\triangleq \sqrt{\sum_{k=n}^{\infty}{a_k}}-\sqrt{\sum_{k=n+1}^{\infty}{a_k}},
\end{align*}
则
\begin{align*}
\sum_{n=1}^{\infty}{A_n}=\sqrt{\sum_{k=1}^{\infty}{a_k}}<+\infty .
\end{align*}
\begin{align*}
\lim_{n\rightarrow \infty}\frac{a_n}{A_n}=\lim_{n\rightarrow \infty}\frac{a_n}{\sqrt{\sum\limits_{k=n}^{\infty}{a_k}}-\sqrt{\sum\limits_{k=n+1}^{\infty}{a_k}}}=\lim_{n\rightarrow \infty}\frac{a_n\left( \sqrt{\sum\limits_{k=n}^{\infty}{a_k}}+\sqrt{\sum\limits_{k=n+1}^{\infty}{a_k}} \right)}{a_n}=0.
\end{align*}
故$a_n=o\left( A_n \right) ,n\rightarrow \infty .$

\item 令
\begin{align*}
A_1=1,\quad A_n\triangleq \sqrt{\sum_{k=1}^n{a_k}}-\sqrt{\sum_{k=1}^{n-1}{a_k}},\,\,n=2,3,\cdots .
\end{align*}
则
\begin{align*}
\sum_{n=2}^{\infty}{A_n}=\lim_{n\rightarrow \infty}\left( \sqrt{\sum_{k=1}^n{a_k}}-\sqrt{a_1} \right) =+\infty .
\end{align*}
\begin{align*}
\lim_{n\rightarrow \infty}\frac{A_n}{a_n}=\lim_{n\rightarrow \infty}\frac{\sqrt{\sum\limits_{k=1}^n{a_k}}-\sqrt{\sum\limits_{k=1}^{n-1}{a_k}}}{a_n}=\lim_{n\rightarrow \infty}\frac{a_n}{a_n\left( \sqrt{\sum\limits_{k=1}^n{a_k}}+\sqrt{\sum\limits_{k=1}^{n-1}{a_k}} \right)}=0.
\end{align*}
故$A_n=o\left( a_n \right) ,n\rightarrow \infty .$
\end{enumerate}

\end{proof}

\begin{example}
设正项级数$\sum\limits_{n=1}^\infty \frac{1}{p_n}<\infty$,证明
$$\sum_{n=1}^\infty \frac{n^2 p_n}{(p_1+p_2+\cdots +p_n)^2}<\infty.$$
\end{example}
\begin{remark}
本题的想法就是把$\sum\limits_{n=1}^\infty \frac{n^2 p_n}{(p_1+p_2+\cdots +p_n)^2}$放大为阶更小的量,从而其收敛.
\end{remark}
\begin{proof}
记$S_0=0,S_n=\sum_{k=1}^n{p_k}$,则对$N\geqslant 2$,有
\begin{align*}
&\sum_{n=2}^N{\frac{n^2p_n}{(p_1+p_2+\cdots +p_n)^2}}=\sum_{n=2}^N{\frac{n^2p_n}{S_{n}^{2}}}=\sum_{n=2}^N{\frac{n^2\left( S_n-S_{n-1} \right)}{S_{n}^{2}}} \\
&=\sum_{n=2}^N{n^2\int_{S_{n-1}}^{S_n}{\frac{1}{S_{n}^{2}}\mathrm{d}x}}\leqslant \sum_{n=2}^N{n^2\int_{S_{n-1}}^{S_n}{\frac{1}{x^2}\mathrm{d}x}}=\sum_{n=2}^N{n^2\left( \frac{1}{S_{n-1}}-\frac{1}{S_n} \right)} \\
&=\sum_{n=2}^N{\left[ \frac{n^2}{S_{n-1}}-\frac{\left( n+1 \right) ^2}{S_n} \right]}+\sum_{n=2}^N{\frac{\left( n+1 \right) ^2-n^2}{S_n}} \\
&=\frac{4}{S_1}-\frac{\left( N+1 \right) ^2}{S_N}+\sum_{n=2}^N{\frac{2n+1}{S_n}} \\
&\leqslant \frac{4}{S_1}+3\sum_{n=2}^N{\frac{n}{S_n}}=\frac{4}{S_1}+3\sum_{n=2}^N{\left( \frac{n\sqrt{p_n}}{S_n}\cdot \frac{1}{\sqrt{p_n}} \right)} \\
&\overset{\text{Cauchy不等式}}{\leqslant}\frac{4}{S_1}+3\sqrt{\sum_{n=2}^N{\frac{n^2p_n}{S_{n}^{2}}}\cdot \sum_{n=2}^N{\frac{1}{p_n}}} \\
&\leqslant \frac{4}{S_1}+C\sqrt{\sum_{n=2}^N{\frac{n^2p_n}{S_{n}^{2}}}}.
\end{align*}
从而
\begin{align*}
\sqrt{\sum_{n=2}^N{\frac{n^2p_n}{S_{n}^{2}}}}\leqslant \frac{4}{S_1}\frac{1}{\sqrt{\sum\limits_{n=2}^N{\frac{n^2p_n}{S_{n}^{2}}}}}+C.
\end{align*}
若$\sum_{n=2}^{\infty}{\frac{n^2p_n}{S_{n}^{2}}}$发散,则对上式令$N\rightarrow +\infty$得$+\infty \leqslant C$矛盾!故$\sum_{n=2}^{\infty}{\frac{n^2p_n}{S_{n}^{2}}}<+\infty .$

\end{proof}

\begin{example}
\begin{enumerate}
\item 设$\{a_n\}_{n=1}^\infty,\{b_n\}_{n=1}^\infty\subset(0,+\infty)$且
$$
\lim_{n\to\infty}\frac{b_n}{n}=0,\lim_{n\to\infty}b_n\left(\frac{a_n}{a_{n+1}}-1\right)>0.
$$
证明级数$\sum_{n=1}^\infty a_n$收敛.

\item 设$\alpha\in(0,1),\{a_n\}_{n=1}^\infty\subset(0,+\infty)$且满足
$$
\varliminf_{n\to\infty}n^\alpha\left(\frac{a_n}{a_{n+1}}-1\right)=\lambda\in(0,+\infty).
$$
证明: $\lim_{n\to\infty}n^k a_n=0$.
\end{enumerate}
\end{example}
\begin{remark}
由$\lim_{k\rightarrow \infty}\frac{o\left( \frac{1}{k^2} \right)}{\frac{1}{k^2}}=0$,$\sum_{k=1}^{\infty}{\frac{1}{k^2}}$收敛可知$\sum_{k=1}^{\infty}{o\left( \frac{1}{k^2} \right)}$收敛,故$\sum_{k=1}^n{o\left( \frac{1}{k^2} \right)}=O\left( 1 \right)$,$\forall n\in \mathbb{N}$.
\end{remark}
\begin{proof}
\begin{enumerate}
\item 由条件可知,存在$c>0,N\in \mathbb{N}$,使得对$\forall n\geqslant N$有
\begin{align*}
b_n\leqslant \frac{c}{2}n,\quad b_n\left( \frac{a_n}{a_{n+1}}-1 \right) >c>0.
\end{align*}
不妨设$N=1$,则
\begin{align*}
\frac{a_n}{a_{n+1}}>1+\frac{c}{b_n}\Longrightarrow a_1>a_{n+1}\prod_{k=1}^n{\left( 1+\frac{c}{b_k} \right)}.
\end{align*}
于是
\begin{align*}
a_{n+1}&<a_1\prod_{k=1}^n{\frac{1}{1+\frac{c}{b_k}}}=a_1e^{-\sum\limits_{k=1}^n{\ln \left( 1+\frac{c}{b_k} \right)}}
\\&\leqslant a_1e^{-\sum\limits_{k=1}^n{\ln \left( 1+\frac{2}{k} \right)}}=a_1e^{-\sum\limits_{k=1}^n{\left[ \frac{2}{k}+o\left( \frac{1}{k^2} \right) \right]}}
\\&=a_1e^{-2\sum\limits_{k=1}^n{\frac{1}{k}}+O\left( 1 \right)}=a_1e^{-2\left[ \ln n+O\left( 1 \right) \right] +O\left( 1 \right)}
\\&=a_1e^{-2\ln n+O\left( 1 \right)}\sim \frac{C}{n^2},n\rightarrow \infty .
\end{align*}
故$\sum_{n=1}^{\infty}{a_n}$收敛.

\item 由条件可知,当$n$充分大时,有
\begin{align*}
n^{\alpha}\left( \frac{a_n}{a_{n+1}}-1 \right) >0\Rightarrow \frac{a_n}{a_{n+1}}>1.
\end{align*}
从而不妨设$\left\{ a_n \right\}$递减.再根据条件可知,存在$N\in \mathbb{N}$,使得
\begin{align*}
n^{\alpha}\left( \frac{a_n}{a_{n+1}}-1 \right) >\frac{\lambda}{2},\forall n\geqslant N.
\end{align*}
故
\begin{align*}
\frac{a_n}{a_{n+1}}>1+\frac{\lambda}{2n^{\alpha}},\forall n\geqslant N.
\end{align*}
于是对$\forall n\geqslant N$,有
\begin{align*}
a_{n+1}&<a_N\prod_{k=N}^n{\frac{1}{1+\frac{\lambda}{2k^{\alpha}}}}=a_Ne^{-\sum\limits_{k=N}^n{\ln \left( 1+\frac{\lambda}{2k^{\alpha}} \right)}}
\\&=a_Ne^{-\sum\limits_{k=N}^n{\left[ \frac{\lambda}{2k^{\alpha}}+O\left( \frac{1}{k^{2\alpha}} \right) \right]}}\xlongequal{\text{\refpro{proposition:幂次小于1的级数的阶}}}a_Ne^{-\left[ \frac{\lambda n^{1-\alpha}}{2\left( 1-\alpha \right)}+O\left( n^{1-\alpha} \right) +O\left( n^{1-2\alpha} \right) \right]}
\\&=a_Ne^{-\left[ \frac{\lambda n^{1-\alpha}}{2\left( 1-\alpha \right)}+O\left( n^{1-\alpha} \right) \right]}\leqslant a_Ne^{-Cn^{1-\alpha}},n\rightarrow \infty .
\end{align*}
注意到$\sum_{n=1}^{\infty}{e^{-Cn^{1-\alpha}}}<+\infty$,故$\sum_{n=1}^{\infty}{a_{n}}$收敛.
\end{enumerate}

\end{proof}

\begin{proposition}\label{proposition:A-D判别法是级数收敛的“充要条件”}
证明:
\begin{enumerate}
\item 实级数$\sum_{n=1}^\infty u_n$收敛等价于存在分解$u_n=a_nb_n,n\in \mathbb{N}$使得$\{a_n\}$单调趋于$0$且$\sum b_n$部分和有界.

\item 实级数$\sum_{n=1}^\infty u_n$收敛等价于存在分解$u_n=a_nb_n,n\in \mathbb{N}$使得$\{a_n\}$单调有界且$\sum b_n$部分和收敛.
\end{enumerate}
\end{proposition}
\begin{note}
这个命题说明:A-D判别法是级数收敛的“充要条件”.
积分版本见\refpro{proposition:A-D判别法是积分收敛的“充要条件”}.
\end{note}
\begin{proof}
充分性就是由级数收敛的A-D判别法.下证必要性.
\begin{enumerate}
\item 设$\sum_{n=1}^{\infty}{u_n}$收敛,由Cauchy收敛准则,对$\forall i\in \mathbb{N}$,存在$n_i\in \mathbb{N}$,使得
\begin{align*}
\left| \sum_{n=k}^{k+p}{u_n} \right|\leqslant \frac{1}{i^3},\forall k\geqslant n_i,p\in \mathbb{N}.
\end{align*}
定义
\begin{align*}
a_0=1,\quad a_n\triangleq \begin{cases}
1,1\leqslant n\leqslant n_1\\
\frac{1}{i},n_i<n\leqslant n_{i+1}\\
\end{cases},i=1,2,\cdots .\quad b_n=\frac{u_n}{a_n}.
\end{align*}
显然$a_n\searrow 0$。当$1\leqslant n\leqslant n_1$时,我们有
\begin{align*}
\left| \sum_{k=1}^n{b_k} \right|=\left| \sum_{k=1}^n{u_k} \right|\leqslant \sum_{k=1}^{n_1}{\left| u_k \right|}.
\end{align*}
当$n>n_1$时,存在$k\in \mathbb{N}$,使得$n_k<n\leqslant n_{k+1}$,于是
\begin{align*}
\left| \sum_{j=1}^n{b_j} \right|&=\left| \sum_{j=1}^{n_1}{u_j}+\sum_{i=1}^{k-1}{\sum_{j=n_i+1}^{n_{i+1}}{iu_j}}+k\sum_{j=n_k+1}^n{u_j} \right|\leqslant \sum_{j=1}^{n_1}{\left| u_j \right|}+\sum_{i=1}^{k-1}{i\left| \sum_{j=n_i+1}^{n_{i+1}}{u_j} \right|}+k\left| \sum_{j=n_k+1}^n{u_j} \right|\\
&\leqslant \sum_{j=1}^{n_1}{\left| u_j \right|}+\sum_{i=1}^{k-1}{\frac{i}{i^3}}+\frac{k}{k^3}\leqslant \sum_{j=1}^{n_1}{\left| u_j \right|}+\sum_{i=1}^{\infty}{\frac{1}{i^2}}=\sum_{j=1}^{n_1}{\left| u_j \right|}+\frac{\pi ^2}{6}.
\end{align*}

\item 设$\sum_{n=1}^{\infty}{u_n}$收敛,由第1问可知,存在$\left\{ \alpha _n \right\}$,$\left\{ \beta _n \right\}$使得$u_n=\alpha _n\beta _n$,并且$\left\{ \alpha _n \right\}$单调递减趋于0,$\sum{\beta _n}$部分和有界。
令
\begin{align*}
a_n\triangleq \sqrt{\alpha _n},\quad b_n\triangleq \beta _n\sqrt{\alpha _n}=\beta _na_n,\quad n=1,2,\cdots .
\end{align*}
显然$a_n\searrow 0$,进而$\left\{ a_n \right\}$单调有界。又$\sum{\beta _n}$部分和有界,故由Dirichlet判别法知,$\sum{b_n}$部分和收敛。
\end{enumerate}

\end{proof}

\begin{example}
设实级数$\sum_{n=1}^{\infty} a_n = s$条件收敛,$\sum_{n=1}^{\infty} a_{f(n)} = t \neq s$是一个重排。证明:对任何$N \in \mathbb{N}$,存在$n \in \mathbb{N}$使得$|n - f(n)| > N$.
\end{example}
\begin{proof}
若$\exists N\in \mathbb{N}$,使得$\forall n\in \mathbb{N}$,有$\left| n-f\left( n \right) \right|\leqslant N$,那么对$\forall m>N$,就有
\begin{align*}
&\sum_{k=1}^{m+N}{a_{f\left( k \right)}}-\sum_{k=1}^m{a_k}\text{中一定不包含}a_1,a_2,\cdots ,a_m,a_{f\left( 1 \right)},a_{f\left( 2 \right)},\cdots ,a_{f\left( m-N \right)}.
\\
&\text{并且}\sum_{k=1}^{m+N}{a_{f\left( k \right)}}-\sum_{k=1}^m{a_k}\text{至多含有}m+N+m-\left( 2m-N \right) =2N\text{项}.
\end{align*}
故对$\forall \varepsilon >0$,存在$N_1\in \mathbb{N}$,使得当$n>N_1$时,有$\left| a_n \right|\leqslant \varepsilon$。于是对$\forall m>N_1$,就有
\begin{align*}
\left| \sum_{k=1}^{m+N}{a_{f\left( k \right)}}-\sum_{k=1}^m{a_k} \right|\leqslant 2N\varepsilon.
\end{align*}
令$m\rightarrow +\infty$得$\left| s-t \right|\leqslant 2N\varepsilon$。由$\varepsilon$的任意性知$s=t$,矛盾!

\end{proof}

\begin{example}
\begin{enumerate}
\item 设$f$满足:对任何绝对收敛级数$\sum_{n=1}^{\infty} a_n$,都有$\sum_{n=1}^{\infty} f(a_n)$绝对收敛,证明$f(x) = O(x),x\to 0$.

\item 设$f$满足:对任何收敛级数$\sum_{n=1}^{\infty} a_n$,都有$\sum_{n=1}^{\infty} f(a_n)$收敛,证明存在$k \in \mathbb{R}$使得在0的某个邻域内有$f(x) = kx$。
\end{enumerate}
\end{example}
\begin{proof}
\begin{enumerate}
\item 反证,若$\frac{f\left( x \right)}{x}$在$x=0$邻域内无界,则$\exists x_n\rightarrow 0$,使得
\begin{align}
\left| \frac{f\left( x_n \right)}{x_n} \right|>n,\,\,n=1,2,\cdots .
\label{eq:103.134}
\end{align}
取$\left\{ x_n \right\}$的子列$\left\{ x_{n_k} \right\}$,使得
\begin{align*}
\left| x_{n_k} \right|<\frac{1}{k^2},k=1,2,\cdots .
\end{align*}
从而对$\forall k\in \mathbb{N}$,都有
\begin{align*}
\frac{2}{k^2\left| x_{n_k} \right|}-\frac{1}{k^2\left| x_{n_k} \right|}=\frac{1}{k^2\left| x_{n_k} \right|}>1,
\end{align*}
于是存在正整数$m_k$,使得
\begin{align}
\frac{1}{k^2\left| x_{n_k} \right|}<m_k<\frac{2}{k^2\left| x_{n_k} \right|}.
\label{eq:103.135}
\end{align}
令
\begin{align*}
a_n\triangleq x_{n_k},\quad m_{k-1}<n\leqslant m_k.
\end{align*}
则由\eqref{eq:103.135}式可得
\begin{align*}
\sum_{k=1}^{\infty}{\left| a_k \right|}=\sum_{k=1}^{\infty}{m_k\left| x_{n_k} \right|}<\sum_{k=1}^{\infty}{\frac{2}{k^2}}<+\infty .
\end{align*}
由条件可知
\begin{align}
\sum_{k=1}^{\infty}{\left| f\left( a_n \right) \right|}=\sum_{k=1}^{\infty}{m_k\left| f\left( x_{n_k} \right) \right|}<+\infty .
\label{eq:103.136}
\end{align}
又由\eqref{eq:103.134}\eqref{eq:103.135}式可得
\begin{align*}
\sum_{k=1}^{\infty}{\left| f\left( a_n \right) \right|}&=\sum_{k=1}^{\infty}{m_k\left| f\left( x_{n_k} \right) \right|}\geqslant \sum_{k=1}^{\infty}{m_kn_k\left| x_{n_k} \right|}\\
&\geqslant \sum_{k=1}^{\infty}{km_k\left| x_{n_k} \right|}\geqslant \sum_{k=1}^{\infty}{\frac{1}{k}}=+\infty .
\end{align*}
这与\eqref{eq:103.136}式矛盾!

\item 目标证明$f$在$x=0$邻域满足Cauchy方程。
考虑$g(x,y)=f(x+y)+f(-x)+f(-y)$。如果对任何$0$的开邻域$U$都有$g$在$U\times U$上不恒为$0$,那么存在$(x_n,y_n)\to(0,0)$使得$g(x_n,y_n)\neq0$。

考虑
\[
\underbrace{(x_1+y_1)-x_1-y_1,(x_1+y_1)-x_1-y_1,\cdots,(x_1+y_1)-x_1-y_1}_{m_1次}+
\]
\[
\underbrace{(x_2+y_2)-x_2-y_2,(x_2+y_2)-x_2-y_2,\cdots,(x_2+y_2)-x_2-y_2}_{m_2次}+
\]
\[
\vdots
\]
上述级数的部分和只可能出现$x_n+y_n,y_n,0$,而当$n\to+\infty$时它们都趋于$0$,因此上述级数收敛。由题目条件,我们有
\[
\underbrace{(f(x_1)+f(y_1))+f(-x_1)+f(-y_1),\cdots,(f(x_1)+f(y_1))+f(-x_1)+f(-y_1)}_{m_1次}+
\]
\[
\underbrace{(f(x_2)+f(y_2))+f(-x_2)+f(-y_2),\cdots,(f(x_2)+f(y_2))+f(-x_2)+f(-y_2)}_{m_2次}+
\]
\[
\vdots
\]
收敛。由\hyperref[theorem:级数加括号的理解]{收敛级数加括号也收敛},我们知道对任何一组$\{m_n\}_{n=1}^{\infty}\subset\mathbb{N}$都有$\sum_{n=1}^{\infty}m_ng(x_n,y_n)$收敛。这显然不可能!因此我们证明了$f(x+y)+f(-x)+f(-y)$在某个$U\times U$上恒为$0$,这里$U$是一个开区间。现在由
\[
f(0)+f(0)+f(0)=0\Rightarrow f(0)=0
\]
知
\[
f(x-x)+f(x)+f(-x)=0\Rightarrow f\text{是奇函数},
\]
即$f(x+y)=f(x)+f(y)$。

再证明$f$在$x=0$连续。设$x_n\to0$,我们考虑收敛级数$x_1-x_1+x_2-x_2+\cdots$,故级数
\[
f(x_1)-f(x_1)+f(x_2)-f(x_2)+\cdots
\]
收敛。考虑上述级数部分和可得$\lim_{n\to\infty}f(x_n)=0$,从而$f$在$x=0$连续。现在由\refthe{theorem:Cauchy方程基本定理}知存在$k\in\mathbb{R}$使得在$0$的某个邻域内有$f(x)=kx$。
\end{enumerate}

\end{proof}

\begin{example}
给定$\{a_n\}_{n=0}^{\infty} \subset \mathbb{R}$,设
\[
f(x) = \sum_{n=0}^{\infty} a_n x^n, \quad x \in (-1, 1).
\]
若
\[
\lim_{n \to \infty} \sum_{k=0}^n a_k = +\infty (-\infty),
\]
证明
\[
\lim_{x \to 1^-} f(x) = +\infty (-\infty),
\]
并指出
\[
\lim_{n \to \infty} \left| \sum_{k=0}^n a_k \right| = +\infty \nRightarrow \lim_{x \to 1^-} |f(x)| = +\infty.
\]
\end{example}
\begin{note}
熟记\refpro{proposition:幂级数Cauchy积常用技巧}.
\end{note}
\begin{proof}
记$S_n = \sum_{k=0}^n a_k$,不妨设
\[
\lim_{n \to \infty} S_n = +\infty.
\]
于是对任意$C > 0$,存在$N \in \mathbb{N}$,使得对任意$n > N$,成立$S_n \geqslant  C$。注意到
\[
\begin{aligned}
\varliminf_{x \to 1^-} f(x) &= \varliminf_{x \to 1^-} (1 - x) \frac{f(x)}{1 - x} \xlongequal{\text{\refpro{proposition:幂级数Cauchy积常用技巧}}} \varliminf_{x \to 1^-} (1 - x) \left[ \sum_{n=0}^{\infty} S_n x^n \right] \\
&= \varliminf_{x \to 1^-} (1 - x) \left[ \sum_{n=0}^{N} S_n x^n + \sum_{n=N+1}^{\infty} S_n x^n \right] \\
&\geqslant \varliminf_{x \to 1^-} (1 - x) \left[ \sum_{n=0}^{N} S_n x^n + C \sum_{n=N+1}^{\infty} x^n \right] \\
&= C \varliminf_{x \to 1^-} (1 - x) \frac{x^{N+1}}{1 - x} = C,
\end{aligned}
\]
由$C$任意性,我们证明了
\[
\lim_{x \to 1^-} f(x) = +\infty.
\]
对于反例,考虑下面的函数即可
\[
f(x) = \frac{x - 1}{(1 + x)^2} = \sum_{n=0}^{\infty} (-1)^{n+1} (2n + 1) x^n.
\]

\end{proof}

\begin{proposition}\label{proposition:三角求和极限结论1}
\begin{enumerate}
\item 设$\lambda_1,\lambda_2,\cdots,\lambda_n$是两两不同的实数,$C_1,C_2,\cdots,C_n$为复数。证明:
\begin{align*}
\lim_{x\to+\infty}\sum_{j=1}^n C_j e^{\lambda_j x}=0 \Leftrightarrow C_j=0,j=1,2,\cdots,n.
\end{align*}

\item 设$m\geqslant 2,\lambda_1,\lambda_2,\cdots,\lambda_m\in\mathbb{R},C_1,C_2,\cdots,C_m\in\mathbb{C}$。若
\[
\lambda_j-\lambda_k\neq2\ell\pi,\forall1\leqslant  j<k\leqslant  m,\ell\in\mathbb{Z},
\]
证明:
\begin{align*}
\lim_{n\to\infty}\sum_{j=1}^m C_j e^{ni\lambda_j}=0 \Leftrightarrow C_j=0,j=1,2,\cdots,m.
\end{align*}
\end{enumerate}
\end{proposition}
\begin{note}
想法即类比傅立叶系数,做积分使得系数暴露出来。离散版本可以类似连续版本证明,连续的处理方式核心是乘上某个$e^{-i\lambda x}$均值形式的积分取极限,从而离散的时候应该是乘上某个$e^{-i\lambda y}$均值的取和.
\end{note}
\begin{proof}
\begin{enumerate}
\item 充分性显然,只需证明必要性。考虑$f(x)\triangleq\sum_{j=1}^n C_j e^{i\lambda_j x}$,其中$i$是虚数单位.对$T>0,k=1,2,\cdots,n$,我们有
\begin{align*}
&\int_T^{2T} e^{-i\lambda_k x}f(x)\,\mathrm{d}x=\int_T^{2T} e^{-i\lambda_k x}\left(\sum_{j=1}^n C_j e^{i\lambda_j x}\right)\,\mathrm{d}x\\
&=\sum_{j=1}^n C_j\int_T^{2T} e^{i(\lambda_j-\lambda_k)x}\,\mathrm{d}x=TC_k+\sum_{j\neq k} C_j\frac{e^{i(\lambda_j-\lambda_k)2T}-e^{i(\lambda_j-\lambda_k)T}}{\lambda_j-\lambda_k},
\end{align*}
从而
\begin{align*}
&|C_k|=\frac{\left|\int_T^{2T} e^{-i\lambda_k x}f(x)\,\mathrm{d}x-\sum\limits_{j\neq k} C_j\frac{e^{i(\lambda_j-\lambda_k)2T}-e^{i(\lambda_j-\lambda_k)T}}{\lambda_j-\lambda_k}\right|}{T}
\\
&\leqslant \frac{\int_T^{2T}|f(x)|\,\mathrm{d}x+\sum\limits_{j\neq k} |C_j|\frac{2}{|\lambda_j-\lambda_k|}}{T}=\frac{|f(\theta_T)|T+\sum\limits_{j\neq k} |C_j|\frac{2}{|\lambda_j-\lambda_k|}}{T},
\end{align*}
这里最后一个等号来自\hyperref[theorem:积分中值定理]{积分中值定理}且$\theta_T\in(T,2T)$。现在由$\lim_{x\to+\infty}f(x)=0$可知
\[
\lim_{T\to+\infty} C_k=0,k=1,2,\cdots,n,
\]
这就证明了$C_j=0,j=1,2,\cdots,n$,必要性得证。

\item 充分性显然,只需证明必要性。对$k=1,2,\cdots,m$,我们有
\[
\lim_{n\to\infty}\left(C_k+\sum_{j\neq k} C_j e^{in(\lambda_j-\lambda_k)}\right)=\lim_{n\to\infty}\left(e^{-in\lambda_k}\sum_{j=1}^m C_j e^{in\lambda_j}\right)=0.
\]
现在由Stolz定理我们有
\begin{align*}
C_k&=-\lim_{n\rightarrow \infty} \sum_{j\ne k}{C_je^{in(\lambda _j-\lambda _k)}}=-\lim_{n\rightarrow \infty} \frac{\sum\limits_{\ell =0}^n{\sum\limits_{j\ne k}{C_je^{i\ell (\lambda _j-\lambda _k)}}}}{n+1}
\\
&=-\lim_{n\rightarrow \infty} \frac{\sum\limits_{j\ne k}{\sum\limits_{\ell =0}^n{C_je^{i\ell (\lambda _j-\lambda _k)}}}}{n+1}=-\lim_{n\rightarrow \infty} \frac{\sum\limits_{j\ne k}{C_j\frac{1-e^{i(n+1)(\lambda _j-\lambda _k)}}{1-e^{i(\lambda _j-\lambda _k)}}}}{n+1}.
\end{align*}
结合
\[
0\leqslant \lim_{n\to\infty}\frac{\left|\sum\limits_{j\neq k} C_j\frac{1-e^{i(n+1)(\lambda_j-\lambda_k)}}{1-e^{i(\lambda_j-\lambda_k)}}\right|}{n+1}\leqslant \lim_{n\to\infty}\frac{\sum\limits_{j\neq k}|C_j|\frac{2}{|1-e^{i(\lambda_j-\lambda_k)}|}}{n+1}=0,
\]
我们知道$C_j=0,j=1,2,\cdots,n$,这就证明了必要性!
\end{enumerate}

\end{proof}

\begin{example}
设$\{\alpha_i\}_{i=1}^m\subset\mathbb{R}$满足
\[
\lim_{n\to\infty}\prod_{i=1}^m\sin(n\alpha_i)=0.
\]
证明: 必有一个$i\in\{1,2,\cdots,m\}$使得$\frac{\alpha_i}{\pi}\in\mathbb{Z}$.
\end{example}
\begin{note}
本题是\refpro{proposition:三角求和极限结论1}的一个应用.
\end{note}
\begin{proof}
由Euler公式得
\[
\lim_{n\to\infty}\prod_{j=1}^m\frac{e^{in\alpha_j}-e^{-in\alpha_j}}{2i}=0\Rightarrow\lim_{n\to\infty}\prod_{j=1}^m(e^{in\alpha_j}-e^{-in\alpha_j})=0.
\]
打开括号得
\begin{align}\label{eq:105.19}
\lim_{n\to\infty}\sum_{\varepsilon_i\in\{-1,1\}}(-1)^{|\{i\in\{1,2,\cdots,m\}:\varepsilon_i=-1\}|}e^{in(\varepsilon_1\alpha_1+\varepsilon_2\alpha_2+\cdots+\varepsilon_m\alpha_m)}=0.
\end{align}
注意到若有
\begin{align}\label{eq:105.20}
\varepsilon _1\alpha _1+\varepsilon _2\alpha _2+\cdots +\varepsilon _m\alpha _m=\varepsilon _1\prime \alpha _1+\varepsilon _2\prime \alpha _2+\cdots +\varepsilon _m\prime \alpha _m+2\ell \pi ,\ell \in \mathbb{Z} ,
\end{align}
则
\begin{align*}
e^{in\left( \varepsilon _1\alpha _1+\varepsilon _2\alpha _2+\cdots +\varepsilon _m\alpha _m \right)}=e^{in\left( \varepsilon _1\prime \alpha _1+\varepsilon _2\prime \alpha _2+\cdots +\varepsilon _m\prime \alpha _m \right)}.
\end{align*}
因此将\eqref{eq:105.19}式中满足\eqref{eq:105.20}式的项合并,得到新的和式的任意两项中的$\varepsilon _1\alpha _1+\varepsilon _2\alpha _2+\cdots +\varepsilon _m\alpha _m$的差值都不等于$2\ell \pi,\ell \in \mathbb{Z}$.于是由\refpro{proposition:三角求和极限结论1}知
\begin{align*}
\sum_{\varepsilon_i\in\{-1,1\}}(-1)^{|\{i\in\{1,2,\cdots,m\}:\varepsilon_i=-1\}|}e^{in(\varepsilon_1\alpha_1+\varepsilon_2\alpha_2+\cdots+\varepsilon_m\alpha_m)}
\end{align*}
恒为0.否则,上式每项系数$(-1)^{|\{i\in\{1,2,\cdots,m\}:\varepsilon_i=-1\}|}=0$矛盾!
故现在就有$\prod_{j=1}^m(e^{in\alpha_j}-e^{-in\alpha_j})=0,\forall n\in\mathbb{N}$, 取$n=1$,则必存在一个$j\in\{1,2,\cdots,m\}$,使得
\begin{align*}
e^{i\alpha _j}-e^{-i\alpha _j}=0\Longrightarrow e^{2i\alpha _j}=0\Longrightarrow 2\alpha _j=2k\pi ,k\in \mathbb{Z} \Longrightarrow \frac{\alpha _j}{\pi}=k\in \mathbb{Z} .
\end{align*}

\end{proof}

\begin{example}
设对 \( n \in \mathbb{N} \) 都有
\[
u_n = \lim_{m \to \infty} (u_{n+1}^2 + u_{n+2}^2 + \cdots + u_{n+m}^2).
\]
证明: 若 \( \lim_{n \to \infty} (u_1 + u_2 + \cdots + u_n) \) 存在, 则 \( u_n = 0, \forall n \in \mathbb{N} \).
\end{example}
\begin{remark}
题目条件中写了极限等于$u_n$就是默认这个极限存在.
\end{remark}
\begin{proof}
注意到
\begin{align}
u_n - u_{n+1} = u_{n+1}^2 \Longrightarrow u_{n+1} = \frac{\sqrt{1 + 4u_n} - 1}{2}, \, n = 1, 2, \cdots, \label{eq:107.41}
\end{align}
由$\lim\limits_{n \to \infty} \sum_{k=1}^n u_k$存在知,$\lim\limits_{n \to \infty} u_n = 0$。若$u_n \neq 0 \, (n \in \mathbb{N})$,由Stolz定理可得
\begin{align*}
\lim_{n \to \infty} n u_n &= \lim_{n \to \infty} \left( \frac{1}{u_{n+1}} - \frac{1}{u_n} \right) = \lim_{n \to \infty} \left( \frac{2}{\sqrt{1 + 4u_n} - 1} - \frac{1}{u_n} \right) \\
&= \lim_{x \to 0^+} \left( \frac{2}{\sqrt{1 + 4x} - 1} - \frac{1}{x} \right) = 1.
\end{align*}
故$u_n \sim \frac{1}{n}$,从而$\sum u_n$发散,矛盾!故存在$n_0 \in \mathbb{N}$,使得$u_{n_0} = 0$。于是由\eqref{eq:107.41}式,利用数学归纳法可知
\[
u_{n+1} = \frac{\sqrt{1 + 4u_n} - 1}{2} = \cdots = 0, \, \forall n > n_0.
\]
\[
u_n = u_{n+1}^2 + u_{n+1} = \cdots = u_{n_0}^2 + u_{n_0} = 0, \, \forall n < n_0.
\]
故此时$u_n = 0, \, \forall n \in \mathbb{N}$。

\end{proof}

\begin{example}
设 \( a_0 = 1, a_1 = \frac{1}{2}, a_{n+1} = \frac{n a_n^2}{1 + (n + 1) a_n}, n \in \mathbb{N} \). 证明: \( \sum_{k=0}^{\infty} \frac{a_{k+1}}{a_k} \) 收敛并求值.
\end{example}
\begin{remark}
$\mathbb{N}\triangleq \{1,2,\cdots,\}$.
\end{remark}
\begin{remark}
级数可求值的情况只有两种:

1.级数通项可求.

2.级数通项可以写成裂项形式.
\end{remark}
\begin{proof}
归纳易得$a_n\geqslant 0,n\in \mathbb{N}$。注意到
\begin{align*}
a_{n+1}+\left( n+1 \right) a_na_{n+1}=na_{n}^{2}\Longrightarrow \frac{a_{n+1}}{a_n}=na_n-\left( n+1 \right) a_{n+1}\geqslant 0,\forall n\in \mathbb{N}。
\end{align*}
故$\{ na_n \}$递减且有下界$0$。设$\lim\limits_{n\rightarrow \infty}na_n=a$,则
\begin{align*}
\sum_{k=0}^{\infty}{\frac{a_{k+1}}{a_k}}=\frac{a_1}{a_0}+\sum_{k=1}^{\infty}{\left[ ka_k-\left( k+1 \right) a_{k+1} \right]}=\frac{1}{2}+a_1-\lim\limits_{n\rightarrow \infty}na_n=1-a<+\infty。
\end{align*}
由$\sum_{k=0}^{\infty}{\frac{a_{k+1}}{a_k}}<+\infty$知,$\lim\limits_{k\rightarrow \infty}\frac{a_{k+1}}{a_k}=0$,从而存在$N\in \mathbb{N}$,使得
\begin{align*}
a_{k+1}\leqslant \frac{1}{2}a_k,\forall k\geqslant N。
\end{align*}
于是对$n\geqslant N$,有
\begin{align*}
a_{n+1}\leqslant \frac{1}{2}a_n\leqslant \frac{1}{2^2}a_n\leqslant \cdots \leqslant \frac{1}{2^{n-N+1}}a_N。
\end{align*}
因此
\begin{align*}
na_n\leqslant \frac{n}{2^{n-N+1}}a_N\rightarrow 0,n\rightarrow \infty。
\end{align*}
故$a=0$,从而$\sum_{k=0}^{\infty}{\frac{a_{k+1}}{a_k}}=1$。

\end{proof}

\begin{example}
设$\{ x_n \}_{n=1}^{\infty}$是单调递减的正数列并且满足$\sum_{n=1}^{\infty} x_n = +\infty$,并证明
\begin{align*}
\sum_{n=1}^{\infty} x_n e^{-\frac{x_n}{x_{n+1}}} = +\infty
\end{align*}
\end{example}
\begin{note}
利用分组判别的想法.
\end{note}
\begin{proof}
任取$M>1$,定义
\begin{align*}
A_M\triangleq \left\{ n\in \mathbb{N} \mid \frac{x_n}{x_{n+1}}\leqslant M \right\} .
\end{align*}
则
\begin{align*}
+\infty =\sum_{n=1}^{\infty}{x_n}=\sum_{n\in A_M}{x_n}+\sum_{n\in \mathbb{N} \setminus A_M}{x_n}.
\end{align*}
若$\sum_{n\in A_M}{x_n}=+\infty$,则
\begin{align*}
\sum_{n\in A_M}{x_ne^{-\frac{x_n}{x_{n+1}}}}\geqslant \sum_{n\in A_M}{x_ne^{-M}}=+\infty .
\end{align*}
若$\sum_{n\in \mathbb{N} \setminus A_M}{x_n}=+\infty$,显然$\mathbb{N} \setminus A_M$中有无穷项,记$\mathbb{N} \setminus A_M=\left\{ n_k \right\}$,且满足$n_k\nearrow +\infty$,则
\begin{align*}
x_{n_{k+1}}\leqslant x_{n_k+1}<\frac{1}{M}x_{n_k}.
\end{align*}
从而对$\forall k\in \mathbb{N}$,有
\begin{align*}
x_{n_k}\leqslant \frac{1}{M}x_{n_{k-1}}\leqslant \cdots \leqslant \frac{1}{M^{k-1}}x_{n_1}\rightarrow 0.
\end{align*}
故
\begin{align*}
\sum_{n\in \mathbb{N} \setminus A_M}{x_n}=\sum_{k=1}^{\infty}{x_{n_k}}\leqslant C\sum_{k=1}^{\infty}{\frac{1}{M^{k-1}}x_{n_1}}<+\infty ,
\end{align*}
这与$\sum_{n\in \mathbb{N} \setminus A_M}{x_n}=+\infty$矛盾!
综上可知
\begin{align*}
\sum_{n=1}^{\infty}{x_ne^{-\frac{x_n}{x_{n+1}}}}\geqslant \sum_{n\in A_M}{x_ne^{-\frac{x_n}{x_{n+1}}}}=+\infty .
\end{align*}

\end{proof}

\begin{example}
设正项级数$\sum_{n=1}^{\infty} a_n$收敛,证明$\sum_{n=1}^{\infty} \frac{a_n \ln \frac{1}{a_n}}{\ln (1 + n)}$收敛。
\end{example}
\begin{note}
利用分组判别的想法.
\end{note}
\begin{proof}


\end{proof}

\begin{example}
设$\{ \lambda_n \},\{ a_n \} \subset \mathbb{R}$。
\begin{enumerate}
\item 如果对任何收敛于 0 的数列$\{ \lambda_n \}$都有$\sum_{n=1}^{\infty} \lambda_n a_n$收敛,证明:$\sum_{n=1}^{\infty} a_n$绝对收敛。

\item 设$p > 1$,如果对任何$\sum_{n=1}^{\infty} |\lambda_n|^p < \infty$都有$\sum_{n=1}^{\infty} \lambda_n a_n$收敛,证明:$\sum_{n=1}^{\infty} |a_n|^q < \infty$,这里$\frac{1}{p} + \frac{1}{q} = 1$。
\end{enumerate}
\end{example}
\begin{proof}
\begin{enumerate}
\item 若$\sum_{n=1}^{\infty}{| a_n |}=+\infty$,记$S_n=\sum_{k=1}^n{| a_k |}$,取$\lambda _n=\frac{\mathrm{sgn} a_n}{S_n}$,故由\refpro{proposition:部分和相关级数重要性质}可知
\begin{align*}
\sum_{n=1}^{\infty}{\lambda _na_n}=\sum_{n=1}^{\infty}{\frac{| a_n |}{S_n}}=+\infty
\end{align*}
矛盾!

\item 若$\sum_{n=1}^{\infty}{| a_n |^q}=+\infty$,记$S_n=\sum_{k=1}^n{| a_k |^q}$,取$\lambda _n=\frac{| a_n |^{q-1}\mathrm{sgn} a_n}{S_n}$,则由$p>1$和\refpro{proposition:部分和相关级数重要性质}可知
\begin{align*}
\sum_{n=1}^{\infty}{| \lambda _n |^p}=\sum_{n=1}^{\infty}{\frac{| a_n |^{\frac{p}{p-1}}}{S_{n}^{p}}}=\sum_{n=1}^{\infty}{\frac{| a_n |^q}{S_{n}^{p}}}<+\infty.
\end{align*}
再由\refpro{proposition:部分和相关级数重要性质}可知
\begin{align*}
\sum_{n=1}^{\infty}{\lambda _na_n}=\sum_{n=1}^{\infty}{\frac{| a_n |^q}{S_n}}=+\infty.
\end{align*}
矛盾!
\end{enumerate}

\end{proof}

\begin{example}
设$a_n$递减到$0$且$b_n = a_n - 2a_{n + 1} + a_{n + 2} \geqslant 0$,$n = 1,2,\cdots$,证明:$\sum_{n = 1}^{\infty} nb_n$收敛并计算值.
\end{example}
\begin{proof}
注意到
\begin{align*}
\sum_{n=1}^m{nb_n}&=\sum_{n=1}^m{n\left( a_n-2a_{n+1}+a_{n+2} \right)}=\sum_{n=1}^m{n\left[ \left( a_n-a_{n+1} \right) -\left( a_{n+1}-a_{n+2} \right) \right]} \\
&\xlongequal{\hyperref[theorem:Abel变换]{\text{Abel变换}}}\sum_{j=1}^{m-1}{\left( j-\left( j+1 \right) \right) \sum_{i=1}^j{\left[ \left( a_i-a_{i+1} \right) +\left( a_{i+1}-a_{i+2} \right) \right]}}+m\sum_{i=1}^m{\left[ \left( a_i-a_{i+1} \right) -\left( a_{i+1}-a_{i+2} \right) \right]} \\
&=-\sum_{j=1}^{m-1}{\left[ \left( a_1-a_2 \right) -\left( a_{j+1}-a_{j+2} \right) \right]}+m\left[ \left( a_1-a_2 \right) -\left( a_{m+1}-a_{m+2} \right) \right] \\
&=a_1-a_2+a_2-a_{m+1}-m\left( a_{m+1}-a_{m+2} \right) \\
&=a_1-\left( m+1 \right) a_{m+1}+ma_{m+2} \\
&=a_1-\left( m+1 \right) \left( a_{m+1}-a_{m+2} \right) -a_{m+2}.
\end{align*}
由$b_n=\left( a_n-a_{n+1} \right) -\left( a_{n+1}-a_{n+2} \right) \geqslant 0$可知,$\left\{ a_n-a_{n+1} \right\}$单调递减.又$\lim\limits_{n\rightarrow \infty}a_n=0$,故
\begin{align*}
\sum_{n=1}^{\infty}{\left( a_n-a_{n+1} \right)}=\lim\limits_{m\rightarrow \infty}\sum_{n=1}^m{\left( a_n-a_{n+1} \right)}=a_1-\lim\limits_{m\rightarrow \infty}a_m=a_1<+\infty .
\end{align*}
因此由\refpro{proposition:单调收敛级数的阶}可知
\begin{align*}
\lim\limits_{m\rightarrow \infty}\left( m+1 \right) \left( a_{m+1}-a_{m+2} \right) =0.
\end{align*}
于是
\begin{align*}
\sum_{n=1}^m{nb_n}=a_1-\left( m+1 \right) \left( a_{m+1}-a_{m+2} \right) -a_{m+2}\rightarrow a_1,m\rightarrow \infty .
\end{align*}

\end{proof}

\begin{example}
设$\{a_n\},\{b_n\} \subset (0,+\infty)$满足
\[ b_{n + 1} - b_n \geqslant \delta > 0, \, n = 1,2,\cdots. \]
若$\sum_{n = 1}^{\infty} a_n$收敛,证明:
\[ \sum_{n = 1}^{\infty} \frac{n\sqrt[n]{(a_1 a_2 \cdots a_n)(b_1 b_2 \cdots b_n)}}{b_n b_{n + 1}} < +\infty. \]
\end{example}
\begin{proof}
由均值不等式可知
\begin{align*}
\frac{n\sqrt[n]{(a_1a_2\cdots a_n)(b_1b_2\cdots b_n)}}{b_nb_{n+1}}&\leqslant \frac{\sum\limits_{j=1}^n{a_jb_j}}{b_nb_{n+1}}=\frac{\delta}{\delta}\frac{\sum\limits_{j=1}^n{a_jb_j}}{b_nb_{n+1}} \\
&\leqslant \frac{b_{n+1}-b_n}{\delta b_nb_{n+1}}\sum_{j=1}^n{a_jb_j}=\frac{1}{\delta}\left( \frac{1}{b_{n+1}}-\frac{1}{b_n} \right) \sum_{j=1}^n{a_jb_j}.
\end{align*}
由$b_{n+1}-b_n>\delta >0$可得
\begin{align*}
b_n\geqslant b_1+\left( n-1 \right) \delta \Longrightarrow b_n\rightarrow +\infty .
\end{align*}
于是
\begin{align*}
\sum_{n=1}^{\infty}{\frac{n\sqrt[n]{(a_1a_2\cdots a_n)(b_1b_2\cdots b_n)}}{b_nb_{n+1}}}&\leqslant \frac{1}{\delta}\sum_{n=1}^{\infty}{\sum_{j=1}^n{\left( \frac{1}{b_{n+1}}-\frac{1}{b_n} \right) a_jb_j}}\leqslant \frac{1}{\delta}\sum_{j=1}^{\infty}{\sum_{n=j}^{\infty}{\left( \frac{1}{b_{n+1}}-\frac{1}{b_n} \right) a_jb_j}}
\\
&=\frac{1}{\delta}\sum_{j=1}^{\infty}{a_jb_j\sum_{n=j}^{\infty}{\left( \frac{1}{b_{n+1}}-\frac{1}{b_n} \right)}}=\frac{1}{\delta}\sum_{j=1}^{\infty}{a_jb_j\sum_{n=j}^{\infty}{\left( \frac{1}{b_{n+1}}-\frac{1}{b_n} \right)}}
\\
&=\frac{1}{\delta}\sum_{j=1}^{\infty}{a_j}<+\infty .
\end{align*}

\end{proof}

\begin{example}
设
\[ S_n = \sum_{k = 1}^n a_k, \, T_n = \sum_{k = 1}^n \left( 1 - \frac{k}{n + 1} \right) a_k, \, n = 1,2,\cdots \]
若\(\sum_{k = 1}^{\infty} |S_k - T_k|^2 < \infty\),证明:\(\sum_{k = 1}^{\infty} a_k < \infty\)。
\end{example}
\begin{proof}
注意到
\begin{align*}
T_n=S_n-\frac{1}{n+1}\sum_{k=1}^n{ka_k},
\end{align*}
故由条件可知
\begin{align}
\sum_{n=1}^{\infty}{\left| S_n-T_n \right|^2}=\sum_{n=1}^{\infty}{\frac{\left( \sum\limits_{k=1}^n{ka_k} \right) ^2}{\left( n+1 \right) ^2}}<+\infty . \label{107.100}
\end{align}
注意到
\begin{align}
&\sum_{n=1}^m{\frac{\sum\limits_{k=1}^n{ka_k}}{n\left( n+1 \right)}}=\sum_{n=1}^m{\left( \frac{1}{n}-\frac{1}{n+1} \right) \sum_{k=1}^n{ka_k}} \nonumber \\
&\xlongequal{\hyperref[theorem:Abel变换]{\text{Abel变换}}}\sum_{j=1}^{m-1}{\left( -\left( j+1 \right) a_{j+1} \right) \sum_{i=1}^j{\left( \frac{1}{i}-\frac{1}{i+1} \right)}}+\sum_{j=1}^m{ja_j}\sum_{i=1}^m{\left( \frac{1}{i}-\frac{1}{i+1} \right)} \nonumber \\
&=-\sum_{j=1}^{m-1}{ja_{j+1}}+\frac{m}{m+1}\sum_{j=1}^m{ja_j} \nonumber \\
&=\frac{m}{m+1}a_1+\frac{m}{m+1}\sum_{j=2}^m{ja_j}-\sum_{j=2}^m{\left( j-1 \right) a_j} \nonumber \\
&=\frac{m}{m+1}a_1+\sum_{j=2}^m{a_j}-\frac{1}{m+1}\sum_{j=2}^m{ja_j} \nonumber \\
&=\sum_{j=1}^m{a_j}-\frac{1}{m+1}\sum_{j=1}^m{ja_j} \nonumber \\
&=S_m-\frac{1}{m+1}\sum_{j=1}^m{ja_j}. \label{107.101}
\end{align}
由\(\eqref{107.100}\)式得
\begin{align*}
0=\lim_{n\rightarrow \infty} \frac{\left( \sum\limits_{k=1}^n{ka_k} \right) ^2}{\left( n+1 \right) ^2}=\lim_{n\rightarrow \infty} \frac{\left( \sum\limits_{k=1}^n{ka_k} \right) ^2}{n^2}\Longrightarrow \lim_{n\rightarrow \infty} \frac{\sum\limits_{k=1}^n{ka_k}}{n}=0.
\end{align*}
因此只须证\(\sum_{n=1}^{\infty}{\frac{\sum\limits_{k=1}^n{ka_k}}{n\left( n+1 \right)}}<+\infty\),即可由\(\eqref{107.101}\)式得
\begin{align*}
\sum_{n=1}^{\infty}{a_n}=\lim_{m\rightarrow \infty} S_m =\sum_{n=1}^{\infty}{\frac{\sum\limits_{k=1}^n{ka_k}}{n\left( n+1 \right)}}+\lim_{m\rightarrow \infty} \frac{1}{m+1}\sum\limits_{j=1}^m{ja_j}=\sum_{n=1}^{\infty}{\frac{\sum\limits_{k=1}^n{ka_k}}{n\left( n+1 \right)}}<+\infty .
\end{align*}
下证\(\sum_{n=1}^{\infty}{\frac{\sum\limits_{k=1}^n{ka_k}}{n\left( n+1 \right)}}<+\infty\)。由均值不等式和\(\eqref{107.100}\)式可得
\begin{align*}
\sum_{n=1}^{\infty}{\frac{\sum\limits_{k=1}^n{ka_k}}{n\left( n+1 \right)}}\leqslant \frac{1}{2}\sum_{n=1}^{\infty}{\left[ \frac{\left( \sum\limits_{k=1}^n{ka_k} \right) ^2}{\left( n+1 \right) ^2}+\frac{1}{n^2} \right]}<+\infty .
\end{align*}
故结论得证。

\end{proof}

\begin{example}
设$\{x_n\}_{n = 1}^{\infty} \subset (0,+\infty)$满足$\sum_{n = 1}^{\infty} \frac{x_n}{2^{n - 1}} = 1$,证明
\[
\sum_{n = 1}^{\infty} \sum_{k = 1}^n \frac{x_k}{n^2} \leqslant 2, \sum_{n = 1}^{\infty} \frac{x_n}{n^2} < 1.
\]
\end{example}
\begin{proof}


\end{proof}

\begin{example}
设$a_n > 0$,$n \in \mathbb{N}$满足

(1)$a_n - a_{n + 1}$递减;

(2) $\sum_{n = 1}^{\infty} a_n$收敛。

证明:
\[
\lim_{n \to \infty} \left( \frac{1}{a_{n + 1}} - \frac{1}{a_n} \right) = +\infty
\]
\end{example}
\begin{proof}
注意到
\begin{align*}
\sum_{n=1}^{\infty}{a_n}<+\infty &\Longrightarrow \lim_{n\rightarrow \infty}a_n=0\Longrightarrow \lim_{n\rightarrow \infty}\left( a_n-a_{n+1} \right) =0\Longrightarrow a_n-a_{n+1}\geqslant 0.
\end{align*}
从而$\{ a_n \} \searrow 0$。于是
\begin{align*}
\lim_{n\rightarrow \infty}\left( \frac{1}{a_{n+1}}-\frac{1}{a_n} \right) &=\lim_{n\rightarrow \infty}\frac{a_n-a_{n+1}}{a_{n+1}a_n}\geqslant \lim_{n\rightarrow \infty}\frac{a_n-a_{n+1}}{a_{n}^{2}}.
\end{align*}
注意到
\begin{align*}
a_{n}^{2}&=\sum_{k=n}^{\infty}{\left( a_{k}^{2}-a_{k+1}^{2} \right)}=\sum_{k=n}^{\infty}{\left( a_k-a_{k+1} \right) \left( a_k+a_{k+1} \right)} \\
&\leqslant \left( a_n-a_{n+1} \right) \sum_{k=n}^{\infty}{\left( a_k+a_{k+1} \right)},
\end{align*}
又$\sum_{n=1}^{\infty}{a_n}<+\infty$,故$\lim_{n\rightarrow \infty}\sum_{k=n}^{\infty}{a_k}=0$。故
\begin{align*}
\lim_{n\rightarrow \infty}\left( \frac{1}{a_{n+1}}-\frac{1}{a_n} \right) &\geqslant \lim_{n\rightarrow \infty}\frac{a_n-a_{n+1}}{a_{n}^{2}}\geqslant \lim_{n\rightarrow \infty}\frac{1}{\sum\limits_{k=n}^{\infty}{\left( a_k+a_{k+1} \right)}}=+\infty .
\end{align*}

\end{proof}

\begin{example}
设\(\phi(x) = \sum_{k = 0}^{\infty} \frac{a_k}{x^k}\),\(x > 0\),证明\(\sum_{n = 1}^{\infty} \phi(n)\)收敛的充要条件是\(a_0 = a_1 = 0\).
\end{example}
\begin{proof}
令\(g\left( x \right) =\phi \left( \frac{1}{x} \right) =\sum_{k=0}^{\infty}{a_kx^k}\),则\(g\in C^{\infty}\left( \mathbb{R} \right)\)。于是
\begin{align*}
g\left( x \right) =a_0+a_1x+O\left( x^2 \right) ,x\rightarrow 0.
\end{align*}
从而
\begin{align*}
\phi \left( n \right) =g\left( \frac{1}{n} \right) =a_0+\frac{a_1}{n}+O\left( \frac{1}{n^2} \right) ,n\rightarrow \infty .
\end{align*}
故
\begin{align*}
\sum_{n=1}^{\infty}{\phi \left( n \right)}<+\infty \Longleftrightarrow \sum_{n=1}^{\infty}{\left( a_0+\frac{a_1}{n} \right)}<+\infty \Longleftrightarrow a_0=a_1=0.
\end{align*}

\end{proof}














\end{document}
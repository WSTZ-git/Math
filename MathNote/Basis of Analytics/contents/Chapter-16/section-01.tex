\documentclass[../../main.tex]{subfiles}
\graphicspath{{\subfix{../../image/}}} % 指定图片目录,后续可以直接使用图片文件名。

% 例如:
% \begin{figure}[H]
% \centering
% \includegraphics[scale=0.4]{图.png}
% \caption{}
% \label{figure:图}
% \end{figure}
% 注意:上述\label{}一定要放在\caption{}之后,否则引用图片序号会只会显示??.

\begin{document}

\section{级数基本结论}

\subsection{级数的敛散性}

\begin{theorem}[交错级数不等式]\label{theorem:交错级数不等式}
设\(\{a_n\}\)递减非负数列,则对\(m,p\in \mathbb{N}_0\),必有
\begin{equation}\label{eq:18.7}
\left|\sum_{n = m}^{m + p} (-1)^n a_n\right| \leqslant a_m.
\end{equation}
\end{theorem}
\begin{note}
本不等式是最容易被遗忘的不等式,应该牢记于心.
\end{note}
\begin{proof}
不妨设\(m = 0\),则
\begin{align*}
\sum_{n = 0}^{p} (-1)^n a_n = 
\begin{cases}
a_0 - (a_1 - a_2) - (a_3 - a_4) - \cdots - (a_{p - 1} - a_p) &, p\text{为偶数} \\
a_0 - (a_1 - a_2) - (a_3 - a_4) - \cdots - (a_{p - 2} - a_{p - 1}) - a_p &, p\text{为奇数}
\end{cases}
\leqslant a_0.
\end{align*}
此外
\begin{align*}
\sum_{n = 0}^{p} (-1)^n a_n = 
\begin{cases}
(a_0 - a_1) + (a_2 - a_3) + \cdots + (a_{p - 2} - a_{p - 1}) + a_p &, p\text{为偶数} \\
(a_0 - a_1) + (a_2 - a_3) + \cdots + (a_{p - 1} - a_p) &, p\text{为奇数}
\end{cases}
\geqslant 0,
\end{align*}
这就证明了不等式\eqref{eq:18.7}.
\end{proof}

\begin{theorem}[A-D判别法]
级数 \( \sum_{n=1}^\infty a_n b_n \) 满足下列条件之一时收敛.
\begin{enumerate}
\item \( \left\{ \sum_{k=1}^n a_k \right\}_{n=1}^\infty \) 有界, \( b_n \) 递减到 \( 0 \);

\item \( \sum_{n=1}^\infty a_n \) 收敛, \( b_n \) 单调有界.
\end{enumerate}
\end{theorem}
\begin{proof}
由 \hyperref[theorem:Abel变换]{Abel变换}, 注意到
\[
\sum_{k=n}^m a_k b_k = \sum_{k=n}^{m - 1} (b_k - b_{k + 1}) \sum_{j=n}^k a_j + b_m \sum_{k=n}^m a_k.
\]
于是对于第一种情况, 设
\[
M = 2 \sup_{n \geqslant 1} \left| \sum_{k=1}^n a_k \right|,
\]
我们有
\[
\left| \sum_{k=n}^m a_k b_k \right| \leqslant M \sum_{k=n}^{m - 1} |b_k - b_{k + 1}| + M |b_m| = M b_n \to 0, \text{当} n, m \to \infty.
\]
对于第二种情况, 因为 \( \sum_{n=1}^\infty a_n \) 收敛, 故对任何 \( \varepsilon > 0 \), 当 \( n \) 充分大, 对任何 \( p \in \mathbb{N}_0 \), 必有
\[
\left| \sum_{k=n}^{n + p} a_k \right| \leqslant \varepsilon.
\]
于是当 \( n, m \) 充分大, 我们有
\[
\left| \sum_{k=n}^m a_k b_k \right| \leqslant \varepsilon \sum_{k=n}^{m - 1} |b_k - b_{k + 1}| + \varepsilon |b_m| = \varepsilon |b_m - b_n| + \varepsilon |b_m| \leqslant 3 \varepsilon \sup_{n \geqslant 1} |b_n|.
\]
因此无论如何都有级数 \( \sum_{n=1}^\infty a_n b_n \) 收敛.
\end{proof}

\begin{theorem}[积分判别法]\label{theorem:积分判别法}
若 \( f \) 是 \( [1, +\infty) \) 的单调不变号函数, 则 \( \sum_{n=1}^\infty f(n) \) 和 \( \int_1^\infty f(x) \, \mathrm{d}x \) 同敛散.
\end{theorem}
\begin{note}
注意有限项不影响级数收敛性, 有限区间不影响积分收敛性. 方法是我们之前已经反复训练的.
\end{note}
\begin{proof}
不妨设 \( f \) 非负递减, 注意到
\[
\int_1^\infty f(x) \, \mathrm{d}x = \sum_{n=1}^\infty \int_n^{n + 1} f(x) \, \mathrm{d}x \leqslant \sum_{n=1}^\infty f(n) \leqslant f(1) + \sum_{n=2}^\infty \int_{n - 1}^n f(x) \, \mathrm{d}x = f(1) + \int_1^\infty f(x) \, \mathrm{d}x.
\]
由夹逼准则即证.
\end{proof}

\begin{theorem}[比值判别法]\label{theorem:级数-比值判别法}
对级数 \( \sum_{n=1}^\infty a_n, a_n > 0, \forall n \in \mathbb{N} \), 有如下判别法:

极限版:
\begin{enumerate}
\item 若 \( \varlimsup_{n \to \infty} \frac{a_{n + 1}}{a_n} < 1 \), 则 \( \sum_{n=1}^\infty a_n \) 收敛;

\item 若 \( \varliminf_{n \to \infty} \frac{a_{n + 1}}{a_n} > 1 \), 则 \( \sum_{n=1}^\infty a_n \) 发散.
\end{enumerate}

不等式版:
\begin{enumerate}
\item 若存在 \( N \in \mathbb{N}, \delta \in (0, 1) \) 使得 \( \frac{a_{n + 1}}{a_n} \leqslant \delta, \forall n \geqslant N \), 则 \( \sum_{n=1}^\infty a_n \) 收敛;

\item 若存在 \( N \in \mathbb{N} \) 使得 \( \frac{a_{n + 1}}{a_n} \geqslant 1, \forall n \geqslant N \), 则 \( \sum_{n=1}^\infty a_n \) 发散.
\end{enumerate}
\end{theorem}
\begin{remark}
极限版的1和不等式版的1是等价的,极限版的2能推出不等式版的2,但不等式版的2不能推出极限版的2.
\end{remark}

\begin{theorem}[Cauchy链]\label{theorem:Cauchy链}
设正值递增函数 \( F \in C^1[a, +\infty) \),\(\frac{F'}{F}\) 在 \([a, +\infty)\) 递减。若满足 \(\sum_{n=1}^\infty F'(n)\) 发散,则对正项级数 \(\sum_{n=1}^\infty a_n, a_n > 0, \forall n \in \mathbb{N}\) 有如下判别法:

极限版:
\begin{enumerate}
\item 若
\begin{align}
\varliminf_{n \to \infty} \frac{\ln \frac{F'(n)}{a_n}}{\ln F(n)} > 1, \label{eq::18.9--1}
\end{align}
则 \(\sum_{n=1}^\infty a_n\) 收敛;

\item 若
\begin{align}
\varlimsup_{n \to \infty} \frac{\ln \frac{F'(n)}{a_n}}{\ln F(n)} < 1, \label{eq::18.10--1}
\end{align}
则 \(\sum_{n=1}^\infty a_n\) 发散。
\end{enumerate}

不等式版:
\begin{enumerate}
\item 若存在 \( c > 1, N \in \mathbb{N} \) 使得
\[
\frac{\ln \frac{F'(n)}{a_n}}{\ln F(n)} \geqslant c, \forall n \geqslant N,
\]
则 \(\sum_{n=1}^\infty a_n\) 收敛;

\item 若存在 \( c \leqslant 1, N \in \mathbb{N} \) 使得
\[
\frac{\ln \frac{F'(n)}{a_n}}{\ln F(n)} \leqslant c, \forall n \geqslant N,
\]
则 \(\sum_{n=1}^\infty a_n\) 发散。
\end{enumerate}
\end{theorem}
\begin{note}
极限版和不等式版的第 1 个结果的条件是等价的,第 2 个结果不等式版条件要更弱,因为如果改 \eqref{eq::18.10--1}为 \(\varliminf_{n \to \infty} \frac{\ln \frac{F'(n)}{a_n}}{\ln F(n)} \leqslant 1\),则 \(\frac{\ln \frac{F'(n)}{a_n}}{\ln F(n)}\) 仍然可能在 \( n \) 充分大严格超过 1。
\end{note}
\begin{remark}
取 \( F(x) = e^x \),则
\[
\frac{\ln \frac{F'(n)}{a_n}}{\ln F(n)} = \frac{n - \ln a_n}{n} = 1 - \ln \sqrt[n]{a_n},
\]
这恰好是\hyperref[theorem:级数-根值判别法]{根值判别法}。

取 \( F(x) = x \),则
\[
\frac{\ln \frac{F'(n)}{a_n}}{\ln F(n)} = \frac{-\ln a_n}{\ln n},
\]
这恰好是\hyperref[theorem:级数-对数判别法]{对数判别法}。
\end{remark}
\begin{proof}
$\mathbf{Step}\,\,\mathbf{1}$先证明
\begin{align}
\lim_{x \to +\infty} F(x) = +\infty. \label{eq:::18.11---11}
\end{align}
设 \(\lim_{x \to +\infty} F(x) = A\),则\hyperref[theorem:积分判别法]{积分判别法}表明
\[
\sum_{n=1}^\infty \frac{F'(n)}{F(n)} \sim \int_a^\infty \frac{F'(x)}{F(x)} \, \mathrm{d}x = \ln F(x) \big|_a^\infty,
\]
即 \(\sum_{n=1}^\infty \frac{F'(n)}{F(n)}\) 收敛。但 \(\sum_{n=1}^\infty \frac{F'(n)}{F(n)} \geqslant \sum_{n=1}^\infty \frac{F'(n)}{A}\),这就和 \(\sum_{n=1}^\infty F'(n)\) 发散矛盾!故我们证明了 \(\eqref{eq:::18.11---11}\)。

$\mathbf{Step}\,\,\mathbf{2}$ 当 \(\eqref{eq::18.9--1}\) 成立,再利用\eqref{eq:::18.11---11}式,存在 \( c > 1, N \in \mathbb{N} \) 使得
\[
\frac{\ln \frac{F'(n)}{a_n}}{\ln F(n)} \geqslant c,F(N)>1, \forall n \geqslant N.
\]
因此
\[
\frac{F'(n)}{a_n} \geqslant e^{c \ln F(n)} \Rightarrow \frac{F'(n)}{F^c(n)} \geqslant a_n, \forall n \geqslant N.
\]
结合 \(\frac{F'(n)}{F^c(n)} = \frac{F'(n)}{F(n)} \cdot \frac{1}{F^{c - 1}(n)}\) 递减,由\hyperref[theorem:积分判别法]{积分判别法},我们有
\[
\sum_{n=1}^\infty \frac{F'(n)}{F^c(n)} \sim \int_N^\infty \frac{F'(x)}{F^c(x)} \, \mathrm{d}x = \int_{F(N)}^\infty \frac{1}{y^c} \, \mathrm{d}y < \infty,
\]
因此 \(\sum_{n=1}^\infty a_n\) 收敛。

$\mathbf{Step}\,\,\mathbf{3}$ 若存在 \( c \leqslant 1, N \in \mathbb{N} \) 使得
\[
\frac{\ln \frac{F'(n)}{a_n}}{\ln F(n)} \leqslant c, F(n) \geqslant 1, \forall n \geqslant N.
\]
根据$\mathbf{Step}\,\,\mathbf{2}$,同样的我们有 \(\frac{F'(n)}{F(n)} \leqslant \frac{F'(n)}{F^c(n)} \leqslant a_n, \forall n \geqslant N\) 以及由\hyperref[theorem:积分判别法]{积分判别法}有
\[
\sum_{n=1}^\infty \frac{F'(n)}{F(n)} \sim \int_N^\infty \frac{F'(x)}{F(x)} \, \mathrm{d}x = \int_{F(N)}^\infty \frac{1}{y} \, \mathrm{d}y = \infty,
\]
因此 \(\sum_{n=1}^\infty a_n\) 发散。
\end{proof}

\begin{theorem}[对数判别法]\label{theorem:级数-对数判别法}
对正项级数 \( \sum_{n=1}^\infty a_n, a_n > 0, \forall n \in \mathbb{N} \), 则有如下判别法:

极限版:
\begin{enumerate}
\item 若 \( \lim_{n \to \infty} \frac{\ln \frac{1}{a_n}}{\ln n} > 1 \), 则 \( \sum_{n=1}^\infty a_n \) 收敛;

\item 若 \( \varlimsup_{n \to \infty} \frac{\ln \frac{1}{a_n}}{\ln n} < 1 \), 则 \( \sum_{n=1}^\infty a_n \) 发散.
\end{enumerate}

不等式版:
\begin{enumerate}
\item 若存在 \( c > 1, N \in \mathbb{N} \) 使得
\[
\frac{\ln \frac{1}{a_n}}{\ln n} \geqslant c, \forall n \geqslant N,
\]
则 \( \sum_{n=1}^\infty a_n \) 收敛;

\item 若存在 \( c \leqslant 1, N \in \mathbb{N} \) 使得
\[
\frac{\ln \frac{1}{a_n}}{\ln n} \leqslant c, \forall n \geqslant N,
\]
则 \( \sum_{n=1}^\infty a_n \) 发散.
\end{enumerate}
\end{theorem}

\begin{theorem}[根值判别法]\label{theorem:级数-根值判别法}
对正项级数 \(\sum_{n=1}^\infty a_n\),则有如下判别法:

极限版:
\begin{enumerate}
\item 若 \(\varlimsup_{n \to \infty} \sqrt[n]{a_n} < 1\),则 \(\sum_{n=1}^\infty a_n\) 收敛;

\item 若 \(\varliminf_{n \to \infty} \sqrt[n]{a_n} > 1\),则 \(\sum_{n=1}^\infty a_n\) 发散。
\end{enumerate}

不等式版:
\begin{enumerate}
\item 若存在 \( c < 1, N \in \mathbb{N} \) 使得
\[
\sqrt[n]{a_n} \leqslant c, \forall n \geqslant N,
\]
则 \(\sum_{n=1}^\infty a_n\) 收敛;

\item 若存在 \( c \geqslant 1 \) 和无穷多个 \( n \) 使得
\[
\sqrt[n]{a_n} \geqslant c,
\]
则 \(\sum_{n=1}^\infty a_n\) 发散。
\end{enumerate}
\end{theorem}
\begin{remark}
值得注意的是,对于\hyperref[theorem:级数-根值判别法]{根值判别法},这里通过 \hyperref[theorem:Cauchy链]{Cauchy 链}的叙述,不应该是 \(\varlimsup_{n \to \infty} \sqrt[n]{a_n} > 1\),而应该是 \(\varliminf_{n \to \infty} \sqrt[n]{a_n} > 1\)。也不应是无穷多个 \( n \),而是任何 \( n \geqslant N \)。所以我们需要一些加强的证明。
\end{remark}
\begin{proof}
若存在 \( c \geqslant 1 \) 和无穷多个 \( n \) 使得
\[
\sqrt[n]{a_n} \geqslant c,
\]
则存在 \( n_k \to \infty \),使得
\[
\sqrt[n_k]{|a_{n_k}|} \geqslant c \geqslant 1 \Rightarrow |a_{n_k}| \geqslant 1 \Rightarrow \lim_{k \to \infty} |a_{n_k}| \neq 0,
\]
于是 \(\sum_{n=1}^\infty a_n\) 发散。
\end{proof}
 
\begin{theorem}[Kummer链]\label{theorem:Kummer链}
对正项级数 \( \sum_{n=1}^\infty a_n, a_n > 0, \forall n \in \mathbb{N} \), 设
\[
K_n = \frac{1}{d_n} \cdot \frac{a_n}{a_{n + 1}} - \frac{1}{d_{n + 1}}, n = 1, 2, \cdots, d_n > 0, \sum_{n=1}^\infty d_n = +\infty,
\]
有如下判别法:

极限版:
\begin{enumerate}
\item 若 \( \varliminf_{n \to \infty} K_n > 0 \), 则 \( \sum_{n=1}^\infty a_n \) 收敛;
\item 若 \( \varlimsup_{n \to \infty} K_n < 0 \), 则 \( \sum_{n=1}^\infty a_n \) 发散.
\end{enumerate}

不等式版:
\begin{enumerate}
\item 若存在 \( N \in \mathbb{N}, \delta > 0 \) 使得 \( K_n \geqslant \delta, \forall n \geqslant N \), 则 \( \sum_{n=1}^\infty a_n \) 收敛;

\item 若存在 \( N \in \mathbb{N} \) 使得 \( K_n \leqslant 0, \forall n \geqslant N \), 则 \( \sum_{n=1}^\infty a_n \) 发散.
\end{enumerate}
\end{theorem}
\begin{note}
极限版和不等式版的第 1 个结果的条件是等价的, 第 2 个结果不等式版条件要更弱.从证明可以看到, 无论是极限版还是不等式版的1, 没用到条件 \( \sum_{n=1}^\infty d_n = +\infty \).
\end{note}
\begin{remark}
当 \( d_n = 1, n \in \mathbb{N} \). 我们有 \( K_n = \frac{a_n}{a_{n + 1}} - 1 \), 这恰好就是\hyperref[theorem:级数-比值判别法]{比值判别法}.

当 \( d_n = \frac{1}{n}, n \in \mathbb{N} \), 我们有 \( K_n = n \frac{a_n}{a_{n + 1}} - (n + 1) \), 这恰好是\hyperref[theorem:拉比判别法]{拉比判别法}.

当 \( d_n = \frac{1}{n \ln n}, n \in \mathbb{N} \), 我们有
\begin{align*}
K_n &= n \ln n \cdot \frac{a_n}{a_{n + 1}} - (n + 1) \ln (n + 1)
\\
&= n \ln n \cdot \frac{a_n}{a_{n + 1}} - (n + 1) \ln n - (n + 1) \ln \left( 1 + \frac{1}{n} \right)
\\
&= \ln n \cdot \left[ n \left( \frac{a_n}{a_{n + 1}} - 1 \right) - 1 \right] - (n + 1) \ln \left( 1 + \frac{1}{n} \right),
\end{align*}
即得一个\hyperref[theorem:较为广泛的判别法]{较为广泛的判别法}. 要注意我们在阶的层面对 \( K_n \) 做了变形, 因此不再给出不等式版本的\hyperref[theorem:较为广泛的判别法]{较为广泛的判别法}.
\end{remark}
\begin{proof}
若存在 \( N \in \mathbb{N}, \delta > 0 \) 使得 \( K_n \geqslant \delta, \forall n \geqslant N \), 则
\[
\frac{1}{\delta} \left( \frac{a_n}{d_n} - \frac{a_{n + 1}}{d_{n + 1}} \right) \geqslant a_{n + 1}, \forall n \geqslant N.
\]
现在
\[
\sum_{k=N}^m a_{k + 1} \leqslant \sum_{k=N}^m \frac{1}{\delta} \left( \frac{a_k}{d_k} - \frac{a_{k + 1}}{d_{k + 1}} \right) = \frac{1}{\delta} \left( \frac{a_N}{d_N} - \frac{a_{m + 1}}{d_{m + 1}} \right) \leqslant \frac{1}{\delta} \cdot \frac{a_N}{d_N},
\]
所以 \( \sum_{n=1}^\infty a_n \) 收敛.

若存在 \( N \in \mathbb{N} \) 使得 \( K_n \leqslant 0, \forall n \geqslant N \). 则 \( \frac{a_{n + 1}}{d_{n + 1}} \geqslant \frac{a_n}{d_n}, \forall n \geqslant N \). 现在
\[
a_{n + 1} \geqslant \frac{a_N}{d_N} d_{n + 1}, \forall n \geqslant N, \sum_{n=1}^\infty d_n = +\infty \Rightarrow \sum_{n=1}^\infty a_n = +\infty,
\]
这就完成了证明.
\end{proof}

\begin{theorem}[拉比判别法]\label{theorem:拉比判别法}
对正项级数 \( \sum_{n=1}^\infty a_n, a_n > 0, \forall n \in \mathbb{N} \), 有如下判别法:

极限版:
\begin{enumerate}
\item 若 \( \varliminf_{n \to \infty} n \left( \frac{a_n}{a_{n + 1}} - 1 \right) > 1 \), 则 \( \sum_{n=1}^\infty a_n \) 收敛;
\item 若 \( \varlimsup_{n \to \infty} n \left( \frac{a_n}{a_{n + 1}} - 1 \right) < 1 \), 则 \( \sum_{n=1}^\infty a_n \) 发散.
\end{enumerate}

不等式版:
\begin{enumerate}
\item 若存在 \( N \in \mathbb{N}, \delta > 1 \) 使得 \( n \left( \frac{a_n}{a_{n + 1}} - 1 \right) \geqslant \delta, \forall n \geqslant N \), 则 \( \sum_{n=1}^\infty a_n \) 收敛;
\item 若存在 \( N \in \mathbb{N} \) 使得 \( n \left( \frac{a_n}{a_{n + 1}} - 1 \right) \leqslant 1, \forall n \geqslant N \), 则 \( \sum_{n=1}^\infty a_n \) 发散.
\end{enumerate}
\end{theorem}
\begin{proof}

\end{proof}

\begin{theorem}[较为广泛的判别法]\label{theorem:较为广泛的判别法}
对正项级数 \( \sum_{n=1}^\infty a_n, a_n > 0, \forall n \in \mathbb{N} \), 有如下判别法:

极限版 1:
\begin{enumerate}
\item 若 \( \varliminf_{n \to \infty} \ln n \cdot \left[ n \left( \frac{a_n}{a_{n + 1}} - 1 \right) - 1 \right] > 1 \), 则 \( \sum_{n=1}^\infty a_n \) 收敛;

\item 若 \( \varlimsup_{n \to \infty} \ln n \cdot \left[ n \left( \frac{a_n}{a_{n + 1}} - 1 \right) - 1 \right] < 1 \), 则 \( \sum_{n=1}^\infty a_n \) 发散.
\end{enumerate}

极限版 2:
\begin{enumerate}
\item 若 \( \varliminf_{n \to \infty} \ln n \left[ n \ln \frac{a_n}{a_{n + 1}} - 1 \right] > 1 \), 则 \( \sum_{n=1}^\infty a_n \) 收敛;
\item 若 \( \varlimsup_{n \to \infty} \ln n \left[ n \ln \frac{a_n}{a_{n + 1}} - 1 \right] < 1 \), 则 \( \sum_{n=1}^\infty a_n \) 发散.
\end{enumerate}
\end{theorem}
\begin{note}
极限版 2 和极限版 1 在很多情况下是等价的, 极限版 1 就是\hyperref[theorem:Kummer链]{Kummer链}的 \( d_n = \frac{1}{n \ln n} \) 的情况. 我们这里以大家更熟悉的主流方法来书写一遍判别法证明, 以极限版 2 为例, 考场会更优先使用这种做法.
\end{note}
\begin{proof}
\begin{enumerate}
\item 设 \( t > 1, N \in \mathbb{N} \) 使得
\[
\ln n \left[ n \ln \frac{a_n}{a_{n + 1}} - 1 \right] > t, \forall n \geqslant N.
\]
然后
\[
\ln \frac{a_n}{a_{n + 1}} > \frac{1}{n} + \frac{t}{n \ln n}, \forall n \geqslant N.
\]
现在求和得
\[
\ln \frac{a_N}{a_{n + 1}} > \sum_{k=N}^n \left( \frac{1}{k} + \frac{t}{k \ln k} \right), \forall n \geqslant N.
\]
于是
\[
a_{n + 1} < a_N e^{-\sum\limits_{k=N}^n \left( \frac{1}{k} + \frac{t}{k \ln k} \right)}, \forall n \geqslant N.
\]
现在由\nrefexa{example:8.61846546}{(2)}和\refexa{example:例题8.5021514}, 我们有
\[
\sum_{k=N}^n \frac{1}{k} = \ln n + O(1), \sum_{k=N}^n \frac{1}{k \ln k} = \ln \ln n + O(1), n \to \infty.
\]
于是
\[
e^{-\sum\limits_{k=N}^n \left( \frac{1}{k} + \frac{t}{k \ln k} \right)} = e^{-\ln n - \ln \ln n + O(1)} = \frac{e^{O(1)}}{n \ln^t n}.
\]
结合\hyperref[theorem:积分判别法]{积分判别法}有
\[
\sum_{n=N}^\infty \frac{1}{n \ln^t n} \sim \int_{10}^\infty \frac{1}{x \ln^t x} \, \mathrm{d}x = \int_{\ln 10}^\infty \frac{1}{y^t} \, \mathrm{d}y < \infty,
\]
我们知道 \( \sum_{n=1}^\infty a_n \) 收敛.

\item 设 \( 0 < t < 1, N \in \mathbb{N} \) 使得
\[
\ln n \left[ n \ln \frac{a_n}{a_{n + 1}} - 1 \right] < t, \forall n \geqslant N.
\]
然后相似第 1 问的证明和
\[
\sum_{n=N}^\infty \frac{1}{n \ln^t n} \sim \int_{10}^\infty \frac{1}{x \ln^t x} \, \mathrm{d}x = \int_{\ln 10}^\infty \frac{1}{y^t} \, \mathrm{d}y = +\infty,
\]
我们有 \( \sum_{n=1}^\infty a_n \) 发散.
\end{enumerate}
\end{proof}

\begin{theorem}[Herschfeld判别法]\label{theorem:Herschfeld判别法}
设 \( p > 1 \) 且 \( \{a_n\}_{n = 1}^{\infty} \subset [0, +\infty) \)。定义
\[
t_n = \sqrt[p]{a_1 + \sqrt[p]{a_2 + \cdots + \sqrt[p]{a_n}}}, \, n \in \mathbb{N},
\]
然后 \( \{t_n\}_{n = 1}^{\infty} \) 收敛的充要条件是 \( a_n^{\frac{1}{p^n}} \) 有界。
显然 \( \{t_n\}_{n = 1}^{\infty} \) 单调递增。
\end{theorem}
\begin{proof}
{\heiti 必要性:} 若 \( \{t_n\}_{n = 1}^{\infty} \) 收敛, 则由
\[
t_n = \sqrt[p]{a_1 + \sqrt[p]{a_2 + \cdots + \sqrt[p]{a_n}}} \geqslant \sqrt[p]{0 + \sqrt[p]{0 + \cdots + \sqrt[p]{a_n}}} = a_n^{\frac{1}{p^n}}
\]
和 \( \{t_n\}_{n = 1}^{\infty} \) 有界知 \( a_n^{\frac{1}{p^n}} \) 有界。

{\heiti 充分性:} 若 \( a_n^{\frac{1}{p^n}} \) 有界, 则设 \( a_n^{\frac{1}{p^n}} \leqslant M, \, \forall n \in \mathbb{N} \), 于是我们有 \( a_n \leqslant M^{p^n}, \, \forall n \in \mathbb{N} \)。因此
\begin{align*}
t_n &= \sqrt[p]{a_1 + \sqrt[p]{a_2 + \cdots + \sqrt[p]{a_n}}} \leqslant \sqrt[p]{M^{p} + \sqrt[p]{M^{p^2} + \cdots + \sqrt[p]{M^{p^n}}}} \\
&= M\sqrt[p]{1 + \sqrt[p]{1 + \cdots + \sqrt[p]{1}}} \leqslant M \lim_{n \to \infty} \underbrace{\sqrt[p]{1 + \sqrt[p]{1 + \cdots + \sqrt[p]{1}}}}_{n \text{ 个根号}},
\end{align*}
其中最后一个等号的极限存在性可以考虑递增函数确定的递推
\[
x_1 = \sqrt[p]{1}, \, x_{n + 1} = \sqrt[p]{1 + x_n}, \, n \in \mathbb{N}.
\]
注意到 \( x_2 = \sqrt[p]{2} > 1 = x_1 \), 不动点 \( x_0 > 1 \) 满足 \( x_0^p - x_0 - 1 = 0 \)。因此由\refpro{proposition:递增函数递推数列}知 \( \{x_n\}_{n = 1}^{\infty} \) 递增有上界, 从而极限存在。
\end{proof}

\begin{proposition}\label{proposition:级数收敛平均级数必收敛}
若$\sum_{n=1}^{\infty}a_n$收敛,则$\lim_{n\rightarrow \infty}\frac{\sum\limits_{k=1}^n{ka_k}}{n}=0$.
\end{proposition}
\begin{note}
这个命题是一个重要的需要记忆的结论,在很多难题时可能是一个很微不足道的中间步骤,但却会把人卡住.

这个命题是\refpro{proposition:反常积分收敛则其平均值极限为0}的离散版本.
\end{note}
\begin{remark}
此外,此类问题还不是直接应用 Stolz 定理就可以的.
笔记 如果我们直接使用Stolz 定理,就有
\begin{align*}
\lim_{n \to \infty} \frac{\sum\limits_{k=1}^n k a_k}{n} = \lim_{n \to \infty} \frac{n a_n}{n - (n - 1)} = \lim_{n \to \infty} n a_n.
\end{align*}
遗憾的是,上式最后的极限可能不存在,而 Stolz 定理不可以逆用.
\end{remark}
\begin{proof}
记$s_k=\sum_{i=1}^k{a_i}$,则由\hyperref[theorem:Abel变换]{Abel变换}及Stolz公式可得
\begin{align*}
\lim_{n\rightarrow \infty}\frac{\sum\limits_{k=1}^n{ka_k}}{n}&=\lim_{n\rightarrow \infty}\frac{\sum\limits_{k=1}^{n-1}\left[ k-\left( k+1 \right) \right] s_k+ns_n}{n}\\
&=\lim_{n\rightarrow \infty}\left( s_n-\frac{\sum\limits_{k=1}^{n-1}s_k}{n} \right)\\
&=\lim_{n\rightarrow \infty}s_n-\lim_{n\rightarrow \infty}s_n=0.
\end{align*}
\end{proof}

\begin{proposition}\label{proposition:单调收敛级数的阶}
设 \( \sum_{n = 1}^{\infty} a_n \) 收敛, 则
\begin{enumerate}
\item 若 \( a_n \) 单调, 则 \( \lim_{n \to \infty} n a_n = 0 \)。

\item 若 \( n a_n \) 单调, 则 \( \lim_{n \to \infty} n \ln n \cdot a_n = 0 \)。

\item 若 \( n \ln n \cdot a_n \) 单调, 则 \( \lim_{n \to \infty} n \ln n \cdot \ln \ln n \cdot a_n = 0 \)。
\end{enumerate}
\end{proposition}
\begin{proof}
\begin{enumerate}
\item 不妨设 \( a_n \) 递减, 否则考虑 \( -a_n \) 即可. 因为收敛级数末项趋于 0, 所以我们知道 \( a_n \) 递减到 0. 注意到由 \( a_n \) 递减知
\[
0 \leqslant 2n a_{2n} \leqslant 2 \sum_{k = n + 1}^{2n} a_k, \, 0 \leqslant (2n - 1) a_{2n - 1} \leqslant 2n a_{2n - 1} \leqslant 2 \sum_{k = n}^{2n - 1} a_k.
\]
现在由 Cauchy 收敛准则知
\[
\lim_{n \to \infty} 2n a_{2n} = \lim_{n \to \infty} (2n - 1) a_{2n - 1} = 0.
\]
由\refpro{proposition:子列极限命题}知 \( \lim_{n \to \infty} n a_n = 0 \)。

\item 不妨设 \( n a_n \) 递减, 否则考虑 \( -a_n \) 即可. 因为 \( \lim_{n \to \infty} n a_n = c \neq 0 \) 会导致$a_n\sim \frac{c}{n}$,进而\( \sum_{n = 1}^{\infty} a_n \) 发散, 所以我们知道 \( n a_n \) 递减到 0.

我们有
\begin{align*}
\sum_{\sqrt{n} - 1 \leqslant k \leqslant n - 1} a_k &= \sum_{\sqrt{n} - 1 \leqslant k \leqslant n - 1} \frac{k a_k}{k} \geqslant n a_n \sum_{\sqrt{n} - 1 \leqslant k \leqslant n - 1} \frac{1}{k} \geqslant n a_n \sum_{\sqrt{n} - 1 \leqslant k \leqslant n - 1} \int_{k}^{k + 1} \frac{1}{x} \mathrm{d}x \\
&= n a_n \int_{\lfloor \sqrt{n} \rfloor}^{n} \frac{1}{x} \mathrm{d}x = n a_n \ln \frac{n}{\lfloor \sqrt{n} \rfloor} \geqslant n a_n \ln \frac{n}{\sqrt{n}} = \frac{1}{2} n a_n \ln n \geqslant 0,
\end{align*}
利用 Cauchy 收敛准则和夹逼准则我们得到 \( \lim_{n \to \infty} n \ln n \cdot a_n = 0 \)。

\item 不妨设 \( n \ln n \cdot a_n \) 递减, 否则考虑 \( -a_n \) 即可. 若 \( \lim_{n \to \infty} (n \ln n \cdot a_n) = c \neq 0 \). 注意到 \( \sum_{n = 2}^{\infty} \frac{1}{n \ln n} \) 发散, \( \sum_{n = 2}^{\infty} a_n \) 收敛, 这就和比较判别法矛盾! 因此 \( \lim_{n \to \infty} (n \ln n \cdot a_n) = 0 \), 从而 \( a_n \geqslant 0 \)。

注意到
\begin{align*}
\sum_{[\ln n] \leqslant k \leqslant n} a_k &= \sum_{[\ln n] \leqslant k \leqslant n} \frac{k \ln k \cdot a_k}{k \ln k} \geqslant n \ln n \cdot a_n \sum_{[\ln n] \leqslant k \leqslant n} \frac{1}{k \ln k} \\
&\geqslant n \ln n \cdot a_n \sum_{[\ln n] \leqslant k \leqslant n} \int_{k}^{k + 1} \frac{1}{x \ln x} \mathrm{d}x = n \ln n \cdot a_n \int_{[\ln n]}^{n + 1} \frac{1}{x \ln x} \mathrm{d}x \\
&= n \ln n \cdot a_n \cdot \ln \frac{\ln (n + 1)}{\ln [\ln n]} \geqslant n \ln n \cdot a_n \cdot \ln \frac{\ln n}{\ln \ln n} \sim n \ln n \cdot \ln \ln n \cdot a_n,
\end{align*}
利用 Cauchy 收敛准则就证明了 \( \lim_{n \to \infty} n \ln n \cdot \ln \ln n \cdot a_n = 0 \)。
\end{enumerate}
\end{proof}

\begin{theorem}[级数的控制收敛定理]\label{theorem:级数的控制收敛定理}
设 \( a_n(s), n = 1, 2, \cdots \) 满足
\[
|a_n(s)| \leqslant c_n, \sum_{n = 1}^{\infty} c_n < \infty,
\]
以及 \( \lim_{s} a_n(s) = b_n \in \mathbb{R} \)。

则
\[
\lim_{s} \sum_{n = 1}^{\infty} a_n(s) = \sum_{n = 1}^{\infty} b_n,
\]
这里 \( \lim_{s} \) 表示 \( s \) 趋于某个 \( s_0 \in \mathbb{R} \cup \{-\infty, +\infty\} \)。
\end{theorem}
\begin{proof}
事实上由极限保号性,我们知道 \( |b_n| \leqslant c_n, n = 1, 2, \cdots \),因此 \( \sum_{n = 1}^{\infty} b_n \) 绝对收敛,从而
\begin{align*}
\left| \sum_{n = 1}^{\infty} a_n(s) - \sum_{n = 1}^{\infty} b_n \right| &\leqslant \left| \sum_{n = 1}^{m} (a_n(s) - b_n) \right| + \sum_{n = m + 1}^{\infty} |a_n(s) - b_n| \\
&\leqslant \left| \sum_{n = 1}^{m} (a_n(s) - b_n) \right| + 2 \sum_{n = m + 1}^{\infty} c_n.
\end{align*}
对 \( s \) 取极限得
\[
\lim_{s} \left| \sum_{n = 1}^{\infty} a_n(s) - \sum_{n = 1}^{\infty} b_n \right| \leqslant 2 \sum_{n = m + 1}^{\infty} c_n.
\]
由 \( m \) 任意性及$\sum_{n=1}^{\infty}{c_n}$收敛的Cauchy收敛准则得
\[
\lim_{s} \left| \sum_{n = 1}^{\infty} a_n(s) - \sum_{n = 1}^{\infty} b_n \right| = 0.
\]
我们完成了级数控制收敛定理的证明。
\end{proof}

\begin{example}
求$\lim_{n\rightarrow \infty}\sum_{k=1}^{n-1}\left( \frac{k}{n} \right) ^n$.
\end{example}
\begin{solution}
注意到
\begin{align}
\lim_{n\rightarrow \infty}\sum_{k=1}^{n-1}\left( \frac{k}{n} \right) ^n=\lim_{n\rightarrow \infty}\sum_{k=1}^{n-1}\left( \frac{n-k}{n} \right) ^n\nonumber=\lim_{n\rightarrow \infty}\sum_{k=1}^{\infty}\left( 1-\frac{k}{n} \right) ^n\chi _{\{1,2,\cdots ,n-1\}}(k),\label{eq:101.678}
\end{align}
并且
\[
\left| \left( 1-\frac{k}{n} \right) ^n\chi _{\{1,2,\cdots ,n-1\}}(k) \right|\leqslant e^{n\ln \left( 1-\frac{k}{n} \right)}\leqslant e^{n\cdot \left( -\frac{k}{n} \right)}=e^{-k}.
\]
又$\sum_{k=1}^{\infty}e^{-k}<\infty$,故由\hyperref[theorem:级数的控制收敛定理]{级数的控制收敛定理}及\eqref{eq:101.678}式可知
\begin{align*}
\lim_{n\rightarrow \infty}\sum_{k=1}^{n-1}\left( \frac{k}{n} \right) ^n&=\sum_{k=1}^{\infty}\lim_{n\rightarrow \infty}\left( 1-\frac{k}{n} \right) ^n\chi _{\{1,2,\cdots ,n-1\}}(k)=\sum_{k=1}^{\infty}e^{-k}
\\
&=\frac{e^{-1}}{1-e^{-1}}=\frac{1}{e-1}.
\end{align*}
\end{solution}

\begin{theorem}[级数的Levi定理]\label{theorem:级数的Levi定理}
若非负 \( a_n(s), n = 1, 2, \cdots \) 满足 \( a_n(s) \) 是 \( s \) 的关于趋近方向的递增函数 (注意如果取极限的方式是 \( s \to s_0^+ \),那么应该是关于 \( s \) 的递减函数) 且
\[
\lim_{s} a_n(s) = b_n \in \mathbb{R} \bigcup \{+\infty\}.
\]
证明
\[
\lim_{s} \sum_{n = 1}^{\infty} a_n(s) = \sum_{n = 1}^{\infty} b_n.
\]
\end{theorem}
\begin{note}
本定理即使级数发散,极限数列发散,也能使用。
\end{note}
\begin{proof}
若 \( \sum_{n = 1}^{\infty} b_n \) 收敛,那么由于 \( 0 \leqslant a_n(s) \leqslant b_n \),取控制级数 \( \sum_{n = 1}^{\infty} b_n \) 即可使用\hyperref[theorem:级数的控制收敛定理]{控制收敛定理}得到
\[
\lim_{s} \sum_{n = 1}^{\infty} a_n(s) = \sum_{n = 1}^{\infty} b_n.
\]

若 \( \sum_{n = 1}^{\infty} b_n \) 发散,由于$\sum_{n = 1}^{\infty} a_n(s)$也单调递增,故$\lim_{s} \sum_{n = 1}^{\infty} a_n(s) $广义存在.假设
\[
\lim_{s} \sum_{n = 1}^{\infty} a_n(s) = m < \infty,
\]
此时对任何 \( N \in \mathbb{N} \),都有
\[
\sum_{n = 1}^{N} b_n = \lim_{s} \sum_{n = 1}^{N} a_n(s) \leqslant \lim_{s} \sum_{n = 1}^{\infty} a_n(s) = m < \infty,
\]
矛盾!我们完成了Levi定理的证明。
\end{proof}

\begin{lemma}[级数的Fatou引理]\label{lemma:级数的Fatou引理}
设非负数列 \( a_n(s), n = 1, 2, \cdots \),则
\[
\sum_{n = 1}^{\infty} \varliminf_{s} a_n(s) \leqslant \varliminf_{s} \sum_{n = 1}^{\infty} a_n(s).
\]
\end{lemma}
\begin{note}
本定理即使级数发散,极限数列发散,也能使用。
\end{note}
\begin{proof}
不妨设 \( s \to +\infty \),考虑 \( g_n(s) \triangleq \inf_{t \geqslant s} a_n(t) \),则 \( g_n \) 关于趋于方向递增非负,所以由\hyperref[theorem:级数的Levi定理]{级数的Levi定理}知
\[
\sum_{n = 1}^{\infty} \varliminf_{s} a_n(s) = \sum_{n = 1}^{\infty} \lim_{s} g_n(s) = \lim_{s} \sum_{n = 1}^{\infty} g_n(s) = \lim_{s} \sum_{n = 1}^{\infty} \inf_{t \geqslant s} a_k(t) \leqslant \varliminf_{s} \sum_{n = 1}^{\infty} a_k(s),
\]
这就完成了证明。
\end{proof}

\begin{theorem}[级数的Fubini定理]\label{theorem:级数的Fubini定理}
满足下述条件之一时,必有
\begin{align}
\sum_{m = 1}^{\infty} \sum_{n = 1}^{\infty} a_{m,n} = \sum_{n = 1}^{\infty} \sum_{m = 1}^{\infty} a_{m,n}. \label{eq:::---12380-18.13}
\end{align}
\begin{enumerate}
\item \( a_{m,n} \geqslant 0, 
\forall m, n \in \mathbb{N} \);

\item \[
\sum_{m = 1}^{\infty} \sum_{n = 1}^{\infty} |a_{m,n}| < \infty.
\]
\end{enumerate}
\end{theorem}
\begin{note}
第一个条件级数发散也能用,再一次体现思想:非负级数无脑换。
\end{note}
\begin{proof}
\begin{enumerate}
\item 由\hyperref[theorem:级数的Levi定理]{级数的 Levi 定理}.我们注意到 \( \{\sum_{n = 1}^{N} a_{m,n} \}\) 关于 \( N \) 非负递增,于是有
\begin{align}
\sum_{m = 1}^{\infty} \lim_{N \to \infty} \sum_{n = 1}^{N} a_{m,n} = \lim_{N \to \infty} \sum_{m = 1}^{\infty} \sum_{n = 1}^{N} a_{m,n} = \lim_{N \to \infty} \sum_{n = 1}^{N} \sum_{m = 1}^{\infty} a_{m,n} = \sum_{n = 1}^{\infty} \sum_{m = 1}^{\infty} a_{m,n}, \label{eq:::---12380-18.14}
\end{align}
这就是 \eqref{eq:::---12380-18.13}。

\item 注意到
\[
\sum_{m = 1}^{\infty} \left| \sum_{n = 1}^{N} a_{m,n} \right| \leqslant \sum_{m = 1}^{\infty} \sum_{n = 1}^{\infty} |a_{m,n}| < \infty,
\]
于是由\hyperref[theorem:级数的控制收敛定理]{级数的控制收敛定理}知 \eqref{eq:::---12380-18.14} 仍然成立,这就是 \eqref{eq:::---12380-18.13}。
\end{enumerate}
\end{proof}

\begin{theorem}[级数加括号的理解]\label{theorem:级数加括号的理解}
\begin{enumerate}
\item 收敛级数任意加括号也收敛且收敛到同一个值。

\item 级数加括号之后收敛,且括号内每个元素符号相同,则原级数收敛,且级数值和如此加括号后一致。
\end{enumerate}
\end{theorem}
\begin{proof}
\begin{enumerate}
\item 设加括号后新的级数是 \( \sum_{k = 1}^{\infty} \sum_{j = n_k + 1}^{n_{k + 1}} a_j \),其中$n_k$递增趋于$+\infty.$则
\[
\sum_{k = 1}^{\infty} \sum_{j = n_k + 1}^{n_{k + 1}} a_j = \lim_{m \to \infty} \sum_{k = 1}^{m} \sum_{j = n_k + 1}^{n_{k + 1}} a_j = \lim_{m \to \infty} \sum_{j = n_1 + 1}^{n_{m + 1}} a_j = \sum_{j = n_1 + 1}^{\infty} a_j,
\]
这就完成了证明。

\item 即证明对严格递增的 \( \{n_k\}_{k = 1}^{\infty} \subset \mathbb{N}, n_1 = 0 \),如果 \( \sum_{k = 1}^{\infty} \sum_{j = n_k + 1}^{n_{k + 1}} a_j \) 收敛且对任何 \( k \in \mathbb{N} \) 都有 \( a_{n_k + 1}, a_{n_k + 2}, \cdots, a_{n_{k + 1}} \) 将符号相同,则 \( \sum_{j = 1}^{\infty} a_j \) 收敛且
\begin{align}
\sum_{j = 1}^{\infty} a_j = \sum_{k = 1}^{\infty} \sum_{j = n_k + 1}^{n_{k + 1}} a_j. \label{eq:::---12380-18.15}
\end{align}
事实上,对每个 \( n \in \mathbb{N} \),存在唯一的 \( m \in \mathbb{N} \),使得 \( n_m < n \leqslant n_{m + 1} \),此时
\[
\sum_{j = 1}^{n} a_j = \sum_{k = 1}^{m - 1} \sum_{j = n_k + 1}^{n_{k + 1}} a_j + \sum_{j = n_m + 1}^{n} a_j.
\]
则当 \( a_j \geqslant 0, n_m < j \leqslant n_{m + 1} \),我们有
\begin{align}
\sum_{j = 1}^{n} a_j \geqslant \sum_{k = 1}^{m - 1} \sum_{j = n_k + 1}^{n_{k + 1}} a_j, \sum_{j = 1}^{n} a_j = \sum_{k = 1}^{m} \sum_{j = n_k + 1}^{n_{k + 1}} a_j - \sum_{j = n + 1}^{n_{m + 1}} a_j \leqslant \sum_{k = 1}^{m} \sum_{j = n_k + 1}^{n_{k + 1}} a_j. \label{eq:::---12380-18.16}
\end{align}
若 \( a_j \leqslant 0, n_m < j \leqslant n_{m + 1} \),可得 \eqref{eq:::---12380-18.16} 的类似式
\begin{align}
\sum_{k = 1}^{m} \sum_{j = n_k + 1}^{n_{k + 1}} a_j \leqslant \sum_{j = 1}^{n} a_j \leqslant \sum_{k = 1}^{m - 1} \sum_{j = n_k + 1}^{n_{k + 1}} a_j. \label{eq:::---12380-18.17}
\end{align}
让 \( n \to +\infty \),我们由 \eqref{eq:::---12380-18.16},\eqref{eq:::---12380-18.17} 和夹逼准则得 \eqref{eq:::---12380-18.15}。这就完成了证明。
\end{enumerate}
\end{proof}

\begin{proposition}
设 \( \{a_n\}_{n = 1}^{\infty} \subset [0, +\infty) \),\( S_n = \sum_{k = 1}^{n} a_k, n \in \mathbb{N} \)。若 \( \{a_n\}_{n = 1}^{\infty} \) 不恒为 0,不妨设 \( a_1 \neq 0 \),则有
\begin{align*}
\sum_{n = 1}^{\infty} \frac{a_n}{S_n^p}
\begin{cases} 
\text{收敛}, & p > 1 \\
\text{和} \sum_{n = 1}^{\infty} a_n \text{同敛散}, & 0 < p \leqslant 1 
\end{cases}. 
\end{align*}
\end{proposition}
\begin{note}
本结果虽然不能直接使用,但连同证明方法却要记住!并且要学会联想和转化到本题的样子,例如
\[
\sum \left( 1 - \frac{a_{n + 1}}{a_n} \right), \sum \left( \frac{\ln \frac{a_n}{a_{n + 1}}}{\ln a_n} \right)
\]
等结构。
\end{note}
\begin{proof}
当 \( p > 1 \),注意到
\[
\sum_{n = 2}^{\infty} \frac{a_n}{S_n^p} = \sum_{n = 2}^{\infty} \frac{S_n - S_{n - 1}}{S_n^p} = \sum_{n = 2}^{\infty} \int_{S_{n - 1}}^{S_n} \frac{1}{S_n^p} \mathrm{d}x \leqslant \sum_{n = 2}^{\infty} \int_{S_{n - 1}}^{S_n} \frac{1}{x^p} \mathrm{d}x = \int_{S_1}^{\sum\limits_{n = 1}^{\infty} a_n} \frac{1}{x^p} \mathrm{d}x,
\]
可以看到无论 \( \sum_{n = 1}^{\infty} a_n \) 收敛性如何都有 \( \sum_{n = 1}^{\infty} \frac{a_n}{S_n^p} \) 收敛。

当 \( 0 < p \leqslant 1 \),若 \( \sum_{n = 1}^{\infty} a_n \) 收敛,则有 \( \frac{a_n}{S_n^p} \sim \frac{a_n}{\left( \sum\limits_{n = 1}^{\infty} a_n \right)^p}=ca_n, n \to \infty \),其中$c$是某个常数,故 \( \sum_{n = 1}^{\infty} \frac{a_n}{S_n^p} \) 收敛。当 \( \sum_{n = 1}^{\infty} a_n \) 发散,我们对任何充分大的 \( m, k \in \mathbb{N} \) 都有
\[
1 - \frac{S_k}{S_{k + m}} = \frac{S_{k + m} - S_k}{S_{k + m}} = \sum_{n = k + 1}^{k + m} \frac{a_n}{S_{k + m}} \leqslant \sum_{n = k + 1}^{k + m} \frac{a_n}{S_n} \leqslant \sum_{n = k + 1}^{k + m} \frac{a_n}{S_n^p}.
\]
让 \( m \to +\infty \),利用 \( S_{k + m} \to +\infty \),于是我们有余项不能任意小,因此由 Cauchy 收敛准则知 \( \sum_{n = 1}^{\infty} \frac{a_n}{S_n^p} \) 发散。这就完成了证明。
\end{proof}

\subsection{幂级数阶与系数阶的关系}

\begin{theorem}[幂级数系数的阶蕴含幂级数和函数的阶]\label{theorem:幂级数系数的阶蕴含幂级数和函数的阶}
\begin{enumerate}[(1)]
\item 设
\begin{align}
f(x) = \sum_{n=0}^{\infty} a_n x^n, g(x) = \sum_{n=0}^{\infty} b_n x^n, x \in (-1,1)
\end{align}
满足
\begin{align}
b_n > 0, \lim_{n \to \infty} \frac{a_n}{b_n} = 0, \lim_{x \to 1^-} g(x) = +\infty, \label{eq::::--::--12301749-19.1}
\end{align}
则
\begin{align}
\lim_{x \to 1^-} \frac{f(x)}{g(x)} = 0. \label{eq::::--::--12301349-19.1}
\end{align}

\item 设
\[
f(x) = \sum_{n=0}^{\infty} a_n x^n, g(x) = \sum_{n=0}^{\infty} b_n x^n, x \in (-1,1)
\]
满足
\begin{align}
b_n > 0, \lim_{n \to \infty} \frac{a_n}{b_n} = 1, \lim_{x \to 1^-} g(x) = +\infty, \label{eq::::--::--123349-19.1}
\end{align}
则
\begin{align}
\lim_{x \to 1^-} \frac{f(x)}{g(x)} = 1. \label{eq::::--::--1233241449-19.1}
\end{align}

\item 设
\begin{align}
f(x) = \sum_{n=0}^{\infty} a_n x^n, g(x) = \sum_{n=0}^{\infty} b_n x^n, x \in \mathbb{R}
\end{align}
满足
\begin{align}
b_n > 0, \lim_{n \to \infty} \frac{a_n}{b_n} = 0, \label{eq::::--::-2341233241449-19.1}
\end{align}
则
\begin{align}
\lim_{x \to +\infty} \frac{f(x)}{g(x)} = 0. \label{eq::::--::-233241449-19.1}
\end{align}

\item 设
\begin{align}
f(x) = \sum_{n=0}^{\infty} a_n x^n, g(x) = \sum_{n=0}^{\infty} b_n x^n, x \in \mathbb{R}
\end{align}
满足
\begin{align}
b_n > 0, \lim_{n \to \infty} \frac{a_n}{b_n} = 1, \label{eq::::--::-23412332449-19.1}
\end{align}
则
\begin{align}
\lim_{x \to +\infty} \frac{f(x)}{g(x)} = 1. \label{eq::::--::-2332419-19.1}
\end{align}
\end{enumerate}
\end{theorem}
\begin{remark}
一句话总结本结论: 即幂级数系数的阶蕴含幂级数和函数的阶.
\end{remark}
\begin{proof}
\begin{enumerate}[(1)]
\item 注意到
\[
\frac{f(x)}{g(x)} = \sum_{n=0}^{\infty} \frac{a_n}{b_n} \frac{b_n x^n}{\sum\limits_{k=0}^{\infty} b_k x^k}.
\]
我们有
\[
0 \leqslant \lim_{x \to 1^-} \frac{b_n x^n}{\sum\limits_{k=0}^{\infty} b_k x^k} = \lim_{x \to 1^-} \frac{b_n x^n}{g(x)} = 0.
\]
由\hyperref[theorem:Toeplitz定理]{Toeplitz定理(b)}以及\(\eqref{eq::::--::--12301749-19.1}\)即得\(\eqref{eq::::--::--12301349-19.1}\).

\item 由\(\lim_{n \to \infty} \frac{a_n - b_n}{b_n} = 0\)和(1)问知
\[
\lim_{x \to 1^-} \frac{f(x) - g(x)}{g(x)} = 0,
\]
即得\(\eqref{eq::::--::--1233241449-19.1}\).

\item 注意到
\[
\frac{f(x)}{g(x)} = \sum_{n=0}^{\infty} \frac{a_n}{b_n} \frac{b_n x^n}{\sum\limits_{k=0}^{\infty} b_k x^k}.
\]
我们有
\[
0 \leqslant \frac{b_n x^n}{\sum\limits_{k=0}^{\infty} b_k x^k} \leqslant \frac{b_n x^n}{b_n x^n + b_{n+1} x^{n+1}} = \frac{b_n}{b_n + b_{n+1} x},
\]
即\(\lim_{x \to +\infty} \frac{b_n x^n}{\sum\limits_{k=0}^{\infty} b_k x^k} = 0\). 由\hyperref[theorem:Toeplitz定理]{Toeplitz定理(b)}以及\(\eqref{eq::::--::-2341233241449-19.1}\)我们就得到\(\eqref{eq::::--::-233241449-19.1}\).

\item 由\(\lim_{n \to \infty} \frac{a_n - b_n}{b_n} = 0\)和(3)问知
\[
\lim_{x \to 1^-} \frac{f(x) - g(x)}{g(x)} = 0,
\]
即得\(\eqref{eq::::--::-2332419-19.1}\).
\end{enumerate}
\end{proof}

\begin{example}
设\(p\)是\(\mathbb{R}\)上实解析函数且\(0 < \prod_{n=0}^{\infty} p^{(n)}(0) < \infty\),求\(\lim_{x \to +\infty} \frac{p'(x)}{p(x)}\).
\end{example}
\begin{proof}
注意到
\begin{align*}
1=\underset{m\rightarrow \infty}{\lim}\frac{\prod\limits_{n=0}^{m+1}{p^{\left( n \right)}\left( 0 \right)}}{\prod\limits_{n=0}^m{p^{\left( n \right)}\left( 0 \right)}}=\lim_{m\rightarrow \infty} p^{\left( m \right)}\left( 0 \right),
\end{align*}
所以\(\{p^{(n)}(0)\}_{n=0}^{\infty}\)是有界数列,故
\[
p(x)=\sum_{n=0}^{\infty}{\frac{p^{(n)}(0)}{n!}x^n,}x\in \mathbb{R} .
\]
在$\mathbb{R}$上有定义且收敛.于是
\begin{align*}
p'(x)=\sum_{n=0}^{\infty}{\frac{p^{(n+1)}(0)}{n!}x^n,}x\in \mathbb{R} .
\end{align*}
由\nrefthe{theorem:幂级数系数的阶蕴含幂级数和函数的阶}{(3)},我们有
\[
\lim_{n \to \infty} \frac{\frac{p^{(n)}(0)}{n!}}{\frac{p^{(n+1)}(0)}{n!}} = 1 \Rightarrow \lim_{x \to +\infty} \frac{p'(x)}{p(x)} = 1.
\]
\end{proof}

\begin{example}
计算
\[
\sum_{n=1}^{\infty} \ln n \cdot x^n \sim \frac{\ln \frac{1}{1 - x}}{1 - x}, x \to 1^{-}.
\]
\end{example}
\begin{solution}
注意到
\begin{align*}
\ln n\sim 1+\frac{1}{2}+\cdots +\frac{1}{n},n\rightarrow \infty .
\end{align*}
由\refthe{theorem:幂级数系数的阶蕴含幂级数和函数的阶}可知
\begin{align*}
\sum_{n=1}^{\infty}{\ln n\cdot x^n}\sim \sum_{n=1}^{\infty}{\left( 1+\frac{1}{2}+\cdots +\frac{1}{n} \right) x^n}\xlongequal{\text{\refexa{example:例题11.24556415}}} \frac{-\ln \left( 1-x \right)}{1-x}=\frac{\ln \frac{1}{1-x}}{1-x},x\rightarrow 1^-.
\end{align*}
\end{solution}

\begin{example}
证明:
\[
\lim_{y\rightarrow 1^{-}}\frac{1}{\ln(1 - y)}\int_{0}^{1}\frac{\mathrm{d}x}{\sqrt{(1 - x^2)(1 - y^2x^2)}}=-\frac{1}{2}.
\]
\end{example}
\begin{proof}
注意到
\begin{align*}
(1+x)^{-\frac{1}{2}}&=\sum_{k=0}^{\infty}\mathrm{C}_{-\frac{1}{2}}^{k}x^k=1+\sum_{k=1}^{\infty}\frac{\left( -\frac{1}{2} \right) \left( -\frac{1}{2}-1 \right) \cdots \left( -\frac{1}{2}-k+1 \right)}{k!}x^k=1+\sum_{k=1}^{\infty}\frac{(-1)^k(2k-1)!!}{2^kk!}x^k=1+\sum_{k=1}^{\infty}\frac{(-1)^k(2k-1)!!}{(2k)!!}x^k.
\end{align*}
于是
\begin{align*}
\int_0^1\frac{\mathrm{d}x}{\sqrt{1-x^2}\sqrt{1-y^2x^2}}&=\int_0^1\frac{1+\sum\limits_{k=1}^{\infty}\frac{(2k-1)!!}{2^kk!}x^{2k}y^{2k}}{\sqrt{1-x^2}}\mathrm{d}x=\int_0^1\frac{1}{\sqrt{1-x^2}}\mathrm{d}x+\sum_{k=1}^{\infty}\left[ \frac{(2k-1)!!}{(2k)!!}\int_0^1\frac{x^{2k}}{\sqrt{1-x^2}}\mathrm{d}x \right]y^{2k}\\
&\xlongequal{x=\cos \theta}\frac{\pi}{2}+\sum_{k=1}^{\infty}\left[ \frac{(2k-1)!!}{(2k)!!}\int_0^{\frac{\pi}{2}}\sin ^{2k}\theta \mathrm{d}\theta \right]y^{2k}=\frac{\pi}{2}+\frac{\pi}{2}\sum_{k=1}^{\infty}\left[ \frac{(2k-1)!!}{(2k)!!} \right]^2y^{2k}.
\end{align*}
又由\hyperref[theorem:Wallis公式]{Wallis公式}知
\begin{align*}
\frac{(2k)!!}{(2k-1)!!}\sim \sqrt{\pi k},k\rightarrow \infty .
\end{align*}
故由\refthe{theorem:幂级数系数的阶蕴含幂级数和函数的阶}可得
\begin{align*}
\int_0^1\frac{\mathrm{d}x}{\sqrt{1-x^2}\sqrt{1-y^2x^2}}&\sim \frac{\pi}{2}\sum_{k=1}^{\infty}\frac{y^{2k}}{\pi k}=\frac{1}{2}\sum_{k=1}^{\infty}\frac{(y^2)^k}{k}=-\frac{1}{2}\ln(1-y^2)\\
&=-\frac{1}{2}\ln(1-y)-\frac{1}{2}\ln(1+y)\sim -\frac{1}{2}\ln(1-y),y\rightarrow 1^-.
\end{align*}
\end{proof}

\begin{example}
证明
\[
\sum_{n=2}^{\infty}{\frac{x^n}{\ln n}}\sim \frac{1}{\left( 1-x \right) \ln \frac{1}{1-x}},x\rightarrow 1^-.
\]
\end{example}
\begin{proof}
注意到$\sum_{n=2}^{\infty}{\frac{x^n}{\ln n}}$在$(-1,1)$上绝对收敛,由\hyperref[theorem:Cauchy积收敛定理]{Cauchy积收敛定理}及\refcor{corollary:收敛级数Cauchy积收敛则就等于级数积}可知
\[
-\ln \left( 1-x \right) \sum_{n=2}^{\infty}{\frac{x^n}{\ln n}}=\sum_{n=1}^{\infty}{\frac{x^n}{n}}\cdot \sum_{n=2}^{\infty}{\frac{x^n}{\ln n}}\xlongequal{\text{\refcor{corollary:收敛级数Cauchy积收敛则就等于级数积}}}\sum_{n=3}^{\infty}{\left( \sum_{k=2}^{n-1}{\frac{1}{\ln k\left( n-k \right)}} \right) x^n}.
\]
下证$\lim_{n\rightarrow +\infty}\sum_{k=2}^{n-1}{\frac{1}{\left( n-k \right) \ln k}}=1$。一方面,我们有
\[
\sum_{k=2}^{n-1}{\frac{1}{\left( n-k \right) \ln k}}\geqslant \sum_{k=2}^{n-1}{\frac{1}{\left( n-k \right) \ln \left( n-1 \right)}}=\frac{\sum\limits_{k=1}^{n-2}{\frac{1}{k}}}{\ln \left( n-1 \right)}\rightarrow 1,n\rightarrow \infty .
\]
另一方面,对$\forall \varepsilon \in (0,1)$,我们有
\begin{align*}
\sum_{k=2}^{n-1}{\frac{1}{\left( n-k \right) \ln k}}&\leqslant \sum_{2\leqslant k\leqslant \varepsilon n}{\frac{1}{\left( n-k \right) \ln k}}+\sum_{\varepsilon n\leqslant k\leqslant n-1}{\frac{1}{\left( n-k \right) \ln k}}
\\
&\leqslant \frac{1}{n\left( 1-\varepsilon \right)}\sum_{2\leqslant k\leqslant \varepsilon n}{\frac{1}{\ln 2}}+\sum_{\varepsilon n\leqslant k\leqslant n-1}{\frac{1}{\left( n-k \right) \ln \varepsilon n}}
\\
&\leqslant \frac{\varepsilon n}{n\left( 1-\varepsilon \right) \ln 2}+\frac{\sum\limits_{\varepsilon n\leqslant k\leqslant n-1}{\frac{1}{k}}}{\ln \varepsilon +\ln n}.
\end{align*}
令$n\rightarrow \infty$得
\[
\varlimsup_{n\rightarrow \infty}\sum_{k=2}^{n-1}{\frac{1}{\left( n-k \right) \ln k}}\leqslant \frac{\varepsilon}{\left( 1-\varepsilon \right) \ln 2}+1.
\]
再令$\varepsilon \rightarrow 0^+$得
\[
\varlimsup_{n\rightarrow \infty}\sum_{k=2}^{n-1}{\frac{1}{\left( n-k \right) \ln k}}\leqslant 1.
\]
故由夹逼准则知$\lim_{n\rightarrow +\infty}\sum_{k=2}^{n-1}{\frac{1}{\left( n-k \right) \ln k}}=1$。于是由\refthe{theorem:幂级数系数的阶蕴含幂级数和函数的阶}可知
\[
-\ln \left( 1-x \right) \sum_{n=2}^{\infty}{\frac{x^n}{\ln n}}=\sum_{n=3}^{\infty}{\left( \sum_{k=2}^{n-1}{\frac{1}{\ln k\left( n-k \right)}} \right) x^n}\sim \sum_{n=3}^{\infty}{x^n}=\frac{1}{1-x},x\rightarrow 1^-.
\]
即$\sum_{n=2}^{\infty}{\frac{x^n}{\ln n}}\sim \frac{1}{\left( 1-x \right) \ln \frac{1}{1-x}},x\rightarrow 1^-.$
\end{proof}

\begin{example}
设
\[
a_0=1,a_1=\frac{5}{4},a_n=\frac{(2n+3)a_{n-1}+(2n-3)a_{n-2}}{4n},n=2,3,\cdots.
\]
求$\lim_{n\rightarrow \infty}a_n$.
\end{example}
\begin{note}
注意到形式幂级数法我们不需要担心考虑的 \( f \) 的幂级数是否收敛的问题. 因为这个方法最后往往可以算出一个 具体的\( f \), 对这个 \( f \) 来说直接用数学归纳法计算验证会发现其Taylor多项式的系数恰好就是条件中的数列, 从而整个逻辑严谨. 因此这又是一个从逻辑上来说属于\textbf{先猜后证}的方法.

从证明可以看到本题实质上是通过幂级数法求出了$a_n$的通项.此外考虑$\frac{1}{1-x}f(x)$的幂级数并用Cauchy积可以导出$\sum_{k=0}^n a_k$的信息.

如果要严谨地证明,就是用数学归纳法证明下述求出来的$a_n$通项表达式(其实就是下面解出来的$f$的Taylor展开式中的通项)就是满足题目条件的$a_n$,再直接计算其极限即可.
\end{note}
\begin{proof}
记$f(x) = \sum_{n=0}^{\infty} a_n x^n$,则$f'(x) = \sum_{n=1}^{\infty} n a_n x^{n-1}$。由条件可得
\begin{align*}
&\quad \,\,\, 4n a_n = (2n+3)a_{n-1} + (2n-3)a_{n-2}, \quad n=2,3,\cdots .
\\
&\Rightarrow 4\sum_{n=2}^{\infty} n a_n x^n = \sum_{n=2}^{\infty} \left[ (2n+3)a_{n-1} + (2n-3)a_{n-2} \right] x^n.
\\
&\Rightarrow 4\sum_{n=1}^{\infty} n a_n x^n - 4a_1 x = \sum_{n=1}^{\infty} (2n+5)a_n x^{n+1} + \sum_{n=0}^{\infty} (2n+1)a_n x^{n+2}
\\
&\Rightarrow 4x\sum_{n=1}^{\infty} n a_n x^{n-1} - 5x = 2x^2\sum_{n=1}^{\infty} n a_n x^{n-1} + 2x^3\sum_{n=1}^{\infty} n a_n x^{n-1} + 5x\sum_{n=1}^{\infty} a_n x^n + x^2\sum_{n=0}^{\infty} a_n x^n
\\
&\Rightarrow 4x\sum_{n=1}^{\infty} n a_n x^{n-1} - 5x = 2x^2\sum_{n=1}^{\infty} n a_n x^{n-1} + 2x^3\sum_{n=1}^{\infty} n a_n x^{n-1} + 5x\sum_{n=0}^{\infty} a_n x^n + x^2\sum_{n=0}^{\infty} a_n x^n - 5x
\\
&\Rightarrow (2x^3 + 2x^2 - 4x) \sum_{n=1}^{\infty} n a_n x^{n-1} + (x^2 + 5x) \sum_{n=1}^{\infty} a_n x^n = 0
\\
&\Rightarrow (2x^3 + 2x^2 - 4x) f'(x) + (x^2 + 5x) f(x) = 0.
\end{align*}
又注意到$f(0) = a_0 = 1$,$f'(0) = a_1 = \frac{5}{4}$,故分离变量解上述微分方程得
\begin{align*}
f(x) = \frac{1}{\sqrt{2}} \frac{\sqrt{x+2}}{1-x}.
\end{align*}
因为$\sqrt{x+2} \in C^{\infty}(\mathbb{R})$,所以可记$\sqrt{x+2} = \sum_{n=0}^{\infty} b_n x^n$,则$\sqrt{3} = \sum_{n=0}^{\infty} b_n$。由\hyperref[theorem:Cauchy积收敛定理]{Cauchy积收敛定理}及\refcor{corollary:收敛级数Cauchy积收敛则就等于级数积}知
\begin{align*}
f(x) = \frac{1}{\sqrt{2}} \cdot \frac{1}{1-x} \cdot \sqrt{x+2} = \frac{1}{\sqrt{2}} \sum_{n=0}^{\infty} \left( \sum_{k=0}^n b_k \right) x^n.
\end{align*}
因此$a_n = \frac{1}{\sqrt{2}} \sum_{k=0}^n b_k$,故$\lim_{n\rightarrow \infty} a_n = \frac{1}{\sqrt{2}} \sum_{n=0}^{\infty} b_n = \frac{\sqrt{3}}{\sqrt{2}} = \frac{\sqrt{6}}{2}.$
\end{proof}








\subsection{Cauchy积}

\begin{definition}[Cauchy积]\label{definition:Cauchy积}
设\(\sum_{n = 0}^{\infty} a_n\),\(\sum_{n = 0}^{\infty} b_n\)是两个收敛级数,我们称
\[
\sum_{n = 0}^{\infty} c_n, c_n = \sum_{k = 0}^{n} a_k b_{n - k}
\]
为\(\sum_{n = 0}^{\infty} a_n\),\(\sum_{n = 0}^{\infty} b_n\)的 \textbf{Cauchy(乘)积}. 我们记
\[
A_n = \sum_{k = 0}^{n} a_k, B_n = \sum_{k = 0}^{n} b_k, S_n = \sum_{k = 0}^{n} c_k.
\]
\end{definition}
\begin{remark}
我们暂时并不清楚\(\sum_{n = 0}^{\infty} c_n\)是否收敛,更不知道是否有
\[
\sum_{n = 0}^{\infty} c_n = \sum_{n = 0}^{\infty} a_n \cdot \sum_{n = 0}^{\infty} b_n. 
\]
\end{remark}
\begin{conclusion}
延续\refdef{definition:Cauchy积},我们有
\[
\begin{cases}
a_0 b_0 = c_0 \\
a_0 b_1 + a_1 b_0 = c_1 \\
a_0 b_2 + a_1 b_1 + a_2 b_0 = c_2 \\
\vdots \\
a_0 b_n + a_1 b_{n-1} + a_2 b_{n-2} + \cdots + a_n b_0 = c_n
\end{cases}
\]
这可以看做一个线性方程组
\[
\begin{pmatrix}
a_0 \\
a_1 & a_0 \\
a_2 & a_1 & a_0 \\
\vdots & \vdots & \vdots & \ddots \\
a_n & a_{n-1} & a_{n-2} & \cdots & a_0
\end{pmatrix}
\begin{pmatrix}
b_0 \\
b_1 \\
b_2 \\
\vdots \\
b_n
\end{pmatrix}
=
\begin{pmatrix}
c_0 \\
c_1 \\
c_2 \\
\vdots \\
c_n
\end{pmatrix}
\]
则当 $a_0 \neq 0$, 我们有
\[
\begin{pmatrix}
b_0 \\
b_1 \\
b_2 \\
\vdots \\
b_n
\end{pmatrix}
=
\begin{pmatrix}
a_0 \\
a_1 & a_0 \\
a_2 & a_1 & a_0 \\
\vdots & \vdots & \vdots & \ddots \\
a_n & a_{n-1} & a_{n-2} & \cdots & a_0
\end{pmatrix}^{-1}
\begin{pmatrix}
c_0 \\
c_1 \\
c_2 \\
\vdots \\
c_n
\end{pmatrix}
\]
本结论可以帮我们计算已知函数的倒数的 Taylor 展开.
\end{conclusion}


\begin{example}
设\(a_n = b_n = \frac{(-1)^n}{\sqrt{n + 1}}, n = 0,1,\cdots\),则
\[
\sum_{n = 0}^{\infty} \sum_{k = 0}^{n} \frac{(-1)^n}{\sqrt{(n - k + 1)(k + 1)}}
\]
发散.
\end{example}
\begin{remark}
这是一组Cauchy积不收敛的反例.
\end{remark}
\begin{proof}
事实上,我们有
\[
\left| \sum_{k = 0}^{n} \frac{(-1)^n}{\sqrt{(n - k + 1)(k + 1)}} \right| = \sum_{k = 0}^{n} \frac{1}{\sqrt{(n - k + 1)(k + 1)}} \geqslant \sum_{k = 0}^{n} \frac{1}{\sqrt{\left(n - \frac{n}{2} + 1\right)\left(\frac{n}{2} + 1\right)}} = \frac{n + 1}{\frac{n}{2} + 1} \to 2,
\]
上式的放缩实际上利用了二次函数$\left( n-k+1 \right) \left( k+1 \right) =-k^2+nk+n+1$的最值大值点$k=\frac{n}{2}$.这就证明了
\[
\sum_{n = 0}^{\infty} \sum_{k = 0}^{n} \frac{(-1)^n}{\sqrt{(n - k + 1)(k + 1)}}
\]
发散.
\end{proof}

\begin{proposition}\label{proposition:命题3....4}
延续\refdef{definition:Cauchy积},我们有
\begin{align}
\lim\limits_{n\to\infty} \frac{\sum\limits_{j=0}^{n} S_j}{n} = \sum\limits_{n=0}^{\infty} a_n \cdot \sum\limits_{n=0}^{\infty} b_n
\label{eq:---::161asf19.8}
\end{align}
\end{proposition}
\begin{proof}
注意到
\begin{align*}
S_n = \sum\limits_{k=0}^{n} \sum\limits_{i=0}^{k} a_i b_{k-i} = \sum\limits_{i=0}^{n} \sum\limits_{k=i}^{n} a_i b_{k-i} = \sum\limits_{i=0}^{n} a_i B_{n-i} = \sum\limits_{i=0}^{n} a_{n-i} B_i,
\end{align*}
于是我们有
\begin{align*}
\sum\limits_{j=0}^{n} S_j = \sum\limits_{j=0}^{n} \sum\limits_{i=0}^{j} a_{j-i} B_i = \sum\limits_{i=0}^{n} \sum\limits_{j=i}^{n} a_{j-i} B_i = \sum\limits_{i=0}^{n} A_{n-i} B_i
\end{align*}
由\refpro{proposition:命题3.44}可得\eqref{eq:---::161asf19.8}.
\end{proof}

\begin{corollary}\label{corollary:收敛级数Cauchy积收敛则就等于级数积}
设级数\(\sum_{n = 0}^{\infty} a_n\),\(\sum_{n = 0}^{\infty} b_n\)都收敛,则它们的Cauchy积\(\sum_{n = 0}^{\infty} c_n\)收敛的充要条件是
\[
\sum_{n = 0}^{\infty} c_n = \sum_{n = 0}^{\infty} a_n \cdot \sum_{n = 0}^{\infty} b_n. 
\]
\end{corollary}
\begin{proof}
延续\refdef{definition:Cauchy积},充分性显然成立,下证必要性.由\refpro{proposition:命题3....4}及Stolz定理可得
\begin{align*}
\sum_{n=0}^{\infty}{a_n}\cdot \sum_{n=0}^{\infty}{b_n}=\lim_{n\rightarrow \infty} \frac{\sum\limits_{j=0}^n{S_j}}{n}=\lim_{n\rightarrow \infty} S_n=\sum_{n=0}^{\infty}{c_n}.
\end{align*}
\end{proof}

\begin{theorem}[Cauchy积收敛定理]\label{theorem:Cauchy积收敛定理}
延续\refdef{definition:Cauchy积},我们有

1. 若$\sum\limits_{n=0}^{\infty} a_n,\sum\limits_{n=0}^{\infty} b_n$有一个绝对收敛,则$\sum\limits_{n=0}^{\infty} c_n$收敛.

2. 若$\sum\limits_{n=0}^{\infty} a_n,\sum\limits_{n=0}^{\infty} b_n$都绝对收敛,则$\sum\limits_{n=0}^{\infty} c_n$绝对收敛.
\end{theorem}
\begin{proof}
1. 注意到
\begin{align*}
S_n = \sum\limits_{k=0}^{n} \sum\limits_{i=0}^{k} a_i b_{k-i} = \sum\limits_{i=0}^{n} \sum\limits_{k=i}^{n} a_i b_{k-i} = \sum\limits_{i=0}^{n} a_i B_{n-i} = \sum\limits_{i=0}^{n} a_{n-i} B_i,
\end{align*}
因此我们只需证明
\begin{align*}
\lim\limits_{n\to\infty} S_n = \lim\limits_{n\to\infty}\sum\limits_{i=0}^{n} a_i B_{n-i}
\end{align*}
收敛.不妨设(否则,若$\lim_{n\to\infty}B_n=B$,则用$B_n-B$代替$B_n$)
\begin{align*}
\sum\limits_{n=1}^{\infty} |a_n| < \infty,\lim\limits_{n\to\infty} B_n = 0
\end{align*}
于是运用\refpro{proposition:命题:3.422164}就有
\begin{align*}
\lim\limits_{n\to\infty} \left| \sum\limits_{i=0}^{n} a_i B_{n-i} \right| \leqslant \lim\limits_{n\to\infty} \sum\limits_{i=0}^{n} |a_i| \cdot |B_{n-i}| = \sum\limits_{i=0}^{\infty} |a_i| \cdot 0 = 0
\end{align*}
这就证明了$\sum\limits_{n=0}^{\infty} c_n$收敛.

2. 若$\sum\limits_{n=0}^{\infty} a_n,\sum\limits_{n=0}^{\infty} b_n$都绝对收敛.注意到
\begin{align*}
\sum\limits_{k=0}^{n} |c_k| \leqslant \sum\limits_{k=0}^{n} \sum\limits_{i=0}^{k} |a_i b_{k-i}| = \sum\limits_{i=0}^{n} \sum\limits_{k=i}^{n} |a_i b_{k-i}| = \sum\limits_{i=0}^{n} \left( |a_i| \sum\limits_{k=i}^{n} |b_{k-i}| \right)
\end{align*}
于是由\refpro{proposition:命题:3.422164}就有
\begin{align*}
\sum\limits_{k=0}^{\infty} |c_k| \leqslant \lim\limits_{n\to\infty} \sum\limits_{i=0}^{n} \left( |a_i| \sum\limits_{k=i}^{n} |b_{k-i}| \right) = \sum\limits_{i=0}^{\infty} |a_i| \cdot \sum\limits_{i=0}^{\infty} |b_i| < \infty
\end{align*}
这就证明了$\sum\limits_{n=0}^{\infty} c_n$绝对收敛.
\end{proof}


接下来我们研究 Cauchy 积和两个级数的积差距有多少.
\begin{proposition}\label{proposition:命题19....2}
延续\refdef{definition:Cauchy积},我们有
\begin{align}
\lim\limits_{n\to\infty} \sum\limits_{k=1}^{n} a_k \sum\limits_{j=0}^{k-1} b_{n-j} = 0 \iff \sum\limits_{n=0}^{\infty} c_n \text{收敛}.
\end{align}
\end{proposition}
\begin{proof}
注意到
\begin{align*}
S_n = \sum\limits_{k=0}^{n} \sum\limits_{i=0}^{k} a_i b_{k-i} = \sum\limits_{i=0}^{n} \sum\limits_{k=i}^{n} a_i b_{k-i} = \sum\limits_{i=0}^{n} a_i B_{n-i} = \sum\limits_{i=0}^{n} a_{n-i} B_i,
\end{align*}
即$\sum\limits_{j=0}^{n} b_j \sum\limits_{k=0}^{n} a_k =\sum\limits_{k=0}^{n} c_k.$于是
\begin{align*}
\sum\limits_{k=1}^{n} a_k \sum\limits_{j=0}^{k-1} b_{n-j} &= \sum\limits_{k=1}^{n} a_k \left( \sum\limits_{j=0}^{n} b_{n-j} - \sum\limits_{j=k}^{n} b_{n-j} \right) \\
&= \sum\limits_{j=0}^{n} b_j \left( \sum\limits_{k=0}^{n} a_k - a_0 \right) - \sum\limits_{k=1}^{n} a_k \sum\limits_{j=0}^{n-k} b_j \\
&= \sum\limits_{j=0}^{n} b_j \sum\limits_{k=0}^{n} a_k - \sum\limits_{k=0}^{n} \sum\limits_{j=0}^{n-k} a_k b_j \\
&= \sum\limits_{j=0}^{n} b_j \sum\limits_{k=0}^{n} a_k - \sum\limits_{k=0}^{n} c_k
\end{align*}
由于Cauchy积收敛,则由\refcor{corollary:收敛级数Cauchy积收敛则就等于级数积},我们有
\begin{align*}
\lim\limits_{n\to\infty} \sum\limits_{k=1}^{n} a_k \sum\limits_{j=0}^{k-1} b_{n-j} = 0 \iff \sum\limits_{n=0}^{\infty} c_n \text{收敛}
\end{align*}
\end{proof}

\begin{example}
设递减数列 $a_n, b_n > 0, n = 0, 1, 2, \cdots$, 且 $\sum\limits_{n=0}^{\infty} (-1)^n a_n, \sum\limits_{n=0}^{\infty} (-1)^n b_n$ 收敛, 记 $c_n = \sum\limits_{j=0}^{n} a_j b_{n-j}$, 证明
\begin{align}
\sum\limits_{n=0}^{\infty} (-1)^n c_n \text{收敛} \iff \lim\limits_{n\to\infty} c_n = 0
\label{eq:19.10}
\end{align}
\end{example}
\begin{proof}
左推右显然,现在假设 $\lim\limits_{n\to\infty} c_n = 0$, 由\refpro{proposition:命题19....2}, 我们只需证明
\begin{align*}
\lim\limits_{n\to\infty} \sum\limits_{k=1}^{n} (-1)^k a_k \sum\limits_{j=0}^{k-1} (-1)^{n-j} b_{n-j} = 0
\end{align*}
现在
\begin{align*}
\left| \sum\limits_{k=1}^{n} (-1)^k a_k \sum\limits_{j=0}^{k-1} (-1)^{n-j} b_{n-j} \right| &\leqslant \sum\limits_{k=1}^{n} a_k \left| \sum\limits_{j=0}^{k-1} (-1)^{n-j} b_{n-j} \right| \\
&\leqslant \sum\limits_{k=1}^{n} a_k b_{n - k + 1} \leqslant \sum\limits_{k=0}^{n+1} a_k b_{n - k + 1} = c_{n+1}
\end{align*}
其中第二个不等号来自于\hyperref[theorem:交错级数不等式]{交错级数不等式}. 于是我们有
\begin{align*}
\lim\limits_{n\to\infty} \left| \sum\limits_{k=1}^{n} (-1)^k a_k \sum\limits_{j=0}^{k-1} (-1)^{n-j} b_{n-j} \right| = 0
\end{align*}
我们证明了\eqref{eq:19.10}.
\end{proof}

\begin{proposition}\label{proposition:幂级数Cauchy积常用技巧}
设$\left\{ a_n \right\} _{n=0}^{\infty}\subset \mathbb{R}$,设
\begin{align*}
f\left( x \right) =\sum_{n=0}^{\infty}{a_nx^n},x\in \left( -1,1 \right) .
\end{align*}
记$S_n\triangleq \sum_{k=0}^n{a_k}$,则
\begin{align*}
\frac{f\left( x \right)}{1-x}=\sum_{n=0}^{\infty}{S_nx^n},x\in \left( -1,1 \right) .
\end{align*}
\end{proposition}
\begin{proof}
由Taylor级数可知
\begin{align*}
\frac{1}{1-x}=\sum_{n=0}^{\infty}{x^n},x\in \left( -1,1 \right) .
\end{align*}
显然$\sum_{n=0}^{\infty}{x^n}$在$\left( -1,1 \right)$上绝对收敛,故由\hyperref[theorem:Cauchy积收敛定理]{Cauchy积收敛定理}可知$f\left( x \right) =\sum_{n=0}^{\infty}{a_nx^n}$和$\frac{1}{1-x}=\sum_{n=0}^{\infty}{x^n}$的Cauchy积也收敛,即
\begin{align*}
\sum_{n=0}^{\infty}{\sum_{k=0}^n{\left( a_kx^k \right) x^{n-k}}}=\sum_{n=0}^{\infty}{\sum_{k=0}^n{a_kx^n}}=\sum_{n=0}^{\infty}{S_nx^n}<+\infty .
\end{align*}
故由\refcor{corollary:收敛级数Cauchy积收敛则就等于级数积}可知
\begin{align*}
\frac{f\left( x \right)}{1-x}=\sum_{n=0}^{\infty}{a_nx^n}\cdot \sum_{n=0}^{\infty}{x^n}=\sum_{n=0}^{\infty}{S_nx^n}.
\end{align*}
\end{proof}


















\end{document}
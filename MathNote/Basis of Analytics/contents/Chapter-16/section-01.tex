\documentclass[../../main.tex]{subfiles}
\graphicspath{{\subfix{../../image/}}} % 指定图片目录,后续可以直接使用图片文件名。

% 例如:
% \begin{figure}[H]
% \centering
% \includegraphics[scale=0.3]{image-01.01}
% \caption{图片标题}
% \label{figure:image-01.01}
% \end{figure}
% 注意:上述\label{}一定要放在\caption{}之后,否则引用图片序号会只会显示??.

\begin{document}

\section{级数常用结论}

\begin{theorem}[交错级数不等式]\label{theorem:交错级数不等式}
设\(\{a_n\}\)递减非负数列,则对\(m,p\in \mathbb{N}_0\),必有
\begin{equation}\label{eq:18.7}
\left|\sum_{n = m}^{m + p} (-1)^n a_n\right| \leqslant a_m.
\end{equation}
\end{theorem}
\begin{note}
本不等式是最容易被遗忘的不等式,应该牢记于心.
\end{note}
\begin{proof}
不妨设\(m = 0\),则
\begin{align*}
\sum_{n = 0}^{p} (-1)^n a_n = 
\begin{cases}
a_0 - (a_1 - a_2) - (a_3 - a_4) - \cdots - (a_{p - 1} - a_p) &, p\text{为偶数} \\
a_0 - (a_1 - a_2) - (a_3 - a_4) - \cdots - (a_{p - 2} - a_{p - 1}) - a_p &, p\text{为奇数}
\end{cases}
\leqslant a_0.
\end{align*}
此外
\begin{align*}
\sum_{n = 0}^{p} (-1)^n a_n = 
\begin{cases}
(a_0 - a_1) + (a_2 - a_3) + \cdots + (a_{p - 2} - a_{p - 1}) + a_p &, p\text{为偶数} \\
(a_0 - a_1) + (a_2 - a_3) + \cdots + (a_{p - 1} - a_p) &, p\text{为奇数}
\end{cases}
\geqslant 0,
\end{align*}
这就证明了不等式\eqref{eq:18.7}.
\end{proof}






\end{document}
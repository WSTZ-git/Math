\documentclass[../../main.tex]{subfiles}
\graphicspath{{\subfix{../../image/}}} % 指定图片目录,后续可以直接使用图片文件名。

% 例如:
% \begin{figure}[H]
% \centering
% \includegraphics[scale=0.4]{图.png}
% \caption{}
% \label{figure:图}
% \end{figure}
% 注意:上述\label{}一定要放在\caption{}之后,否则引用图片序号会只会显示??.

\begin{document}

\section{级数计算}

\subsection{裂项方法}

\begin{example}
计算$\sum_{n=1}^{\infty} \frac{1}{2^n (1 + \sqrt[2^n]{2})}.$
\end{example}
\begin{note}
继续采用强行裂项的想法,猜出裂项之后的模样之后还原看看差什么.
\end{note}
\begin{proof}
注意到
\begin{align*}
\frac{1}{2^{n-1}\left( 2^{\frac{1}{2^{n-1}}}-1 \right)}-\frac{1}{2^n\left( 2^{\frac{1}{2^n}}-1 \right)}&=\frac{2}{2^n\left( 2^{\frac{1}{2^{n-1}}}-1 \right)}-\frac{2^{\frac{1}{2^n}}+1}{2^n\left( 2^{\frac{1}{2^{n-1}}}-1 \right)}=-\frac{2^{\frac{1}{2^n}}-1}{2^n\left( 2^{\frac{1}{2^{n-1}}}-1 \right)}
\\
&=-\frac{2^{\frac{1}{2^n}}-1}{2^n\left( 2^{\frac{1}{2^n}}+1 \right) \left( 2^{\frac{1}{2^n}}-1 \right)}=-\frac{1}{2^n\left( 2^{\frac{1}{2^n}}+1 \right)},
\end{align*}
我们有
\begin{align*}
\sum_{n=1}^{\infty} \frac{1}{2^n \left(2^{\frac{1}{2^n}} + 1\right)} = \lim_{n \to \infty} \left( \frac{1}{2^n \left(2^{\frac{1}{2^n}} - 1\right)} \right) - 1 = \frac{1}{\ln 2} - 1.
\end{align*}

\end{proof}

\begin{example}
计算
\begin{align*}
\sum_{k=1}^{\infty} \frac{1}{k(k + 1)(k + 1)!}
\end{align*}
\end{example}
\begin{note}
想法的关键是强行裂项.
\end{note}
\begin{proof}
\begin{align*}
\sum_{k=1}^{\infty} \frac{1}{k(k + 1)(k + 1)!} &= \sum_{k=1}^{\infty} \left( \frac{1}{k} - \frac{1}{k + 1} \right) \frac{1}{(k + 1)!} = \sum_{k=1}^{\infty} \left( \frac{1}{k(k + 1)!} - \frac{1}{(k + 1)(k + 1)!} \right) \\
&= \sum_{k=1}^{\infty} \left( \frac{1}{k(k + 1)!} - \frac{1}{(k + 1)(k + 2)!} \right) + \sum_{k=1}^{\infty} \left( \frac{1}{(k + 1)(k + 2)!} - \frac{1}{(k + 1)(k + 1)!} \right) \\
&= \frac{1}{2} - \sum_{k=1}^{\infty} \frac{1}{(k + 2)!} \xlongequal{e\text{的Taylor展开}} \frac{1}{2} - \left( e - 1-1 - \frac{1}{2} \right) = 3 - e.
\end{align*}

\end{proof}

\begin{example}
计算级数
\begin{align*}
\sum_{n=1}^{\infty} (-1)^{n - 1} \frac{\ln n}{n}
\end{align*}
\end{example}
\begin{note}
此类问题化部分和之后估阶.
\end{note}
\begin{proof}
注意到
\begin{align*}
\sum_{n=1}^{2m+1}{\left( -1 \right) ^{n-1}\frac{\ln n}{n}}=\sum_{n=1}^{2m}{\left( -1 \right) ^{n-1}\frac{\ln n}{n}}+\frac{\ln \left( 2m+1 \right)}{2m+1}.
\end{align*}
令$m\rightarrow \infty ,$得
\begin{align*}
\underset{m\rightarrow \infty}{\lim}\sum_{n=1}^{2m+1}{\left( -1 \right) ^{n-1}\frac{\ln n}{n}}=\underset{m\rightarrow \infty}{\lim}\sum_{n=1}^{2m}{\left( -1 \right) ^{n-1}\frac{\ln n}{n}}.
\end{align*}
于是由\hyperref[proposition:子列极限命题]{子列极限命题(b)}可得
\begin{align*}
&\sum_{n=1}^{\infty} (-1)^{n - 1} \frac{\ln n}{n} = \lim_{m \to \infty} \sum_{n=1}^{2m} (-1)^{n - 1} \frac{\ln n}{n} = \lim_{m \to \infty} \sum_{n=1}^{m} \left( \frac{\ln(2n - 1)}{2n - 1} - \frac{\ln(2n)}{2n} \right) \\
&= \lim_{m \to \infty} \left( \sum_{n=1}^{2m} \frac{\ln n}{n} - \sum_{n=1}^{m} \frac{\ln(2n)}{2n} - \sum_{n=1}^{m} \frac{\ln(2n)}{2n} \right) = \lim_{m \to \infty} \left( \sum_{n=1}^{2m} \frac{\ln n}{n} - \sum_{n=1}^{m} \frac{\ln n}{n} - \sum_{n=1}^{m} \frac{\ln 2}{n} \right)
\end{align*}
利用\nrefexa{example:8.61846546}{(2)},我们知道
\begin{align*}
\sum_{n=1}^{m} \frac{\ln 2}{n} = \ln 2 \cdot \ln m + \ln 2 \cdot \gamma + o(1), m \to \infty
\end{align*}
由\hyperref[theorem:0阶Euler-Maclaurin公式(0阶E-M公式)]{0阶E-M公式}知道
\begin{align*}
\sum_{n=1}^{m} \frac{\ln n}{n} = \frac{\ln m}{2m} + \int_{1}^{m} \frac{\ln x}{x} \mathrm{d}x + \int_{1}^{m} \left( x - [x] - \frac{1}{2} \right) \left( \frac{\ln x}{x} \right)' \mathrm{d}x
\end{align*}
注意到$\int_{1}^{m} \frac{\ln x}{x} \mathrm{d}x = \frac{1}{2} \ln^2 m$以及
\begin{align*}
\left| \int_{1}^{m} \left( x - [x] - \frac{1}{2} \right) \left( \frac{\ln x}{x} \right)' \mathrm{d}x \right| = \left| \int_{1}^{m} \left( x - [x] - \frac{1}{2} \right) \frac{1 - \ln x}{x^2} \mathrm{d}x \right| \leqslant \frac{1}{2} \int_{1}^{\infty} \frac{|1 - \ln x|}{x^2} \mathrm{d}x < \infty
\end{align*}
于是我们有
\begin{align*}
\sum_{n=1}^{m} \frac{\ln n}{n} = \frac{1}{2} \ln^2 m + C + o(1), m \to \infty
\end{align*}
这里$C = \int_{1}^{\infty} \left( x - [x] - \frac{1}{2} \right) \left( \frac{\ln x}{x} \right)' \mathrm{d}x$. 现在结合上述渐近估计式就有
\begin{align*}
\sum_{n=1}^{\infty} (-1)^{n - 1} \frac{\ln n}{n} = \lim_{m \to \infty} \left[ \frac{1}{2} \ln^2 (2m) - \frac{1}{2} \ln^2 m - \ln 2 \cdot \ln m - \ln 2 \cdot \gamma + o(1) \right] = \frac{\ln^2 2}{2} - \ln 2 \cdot \gamma
\end{align*}

\end{proof}

\begin{example}
\begin{enumerate}
\item 计算
\begin{align*}
\sum_{n=1}^{\infty} \frac{1 + \frac{1}{2} + \cdots + \frac{1}{n}}{(n + 1)(n + 2)}
\end{align*}

\item 计算
\begin{align*}
\sum_{n=1}^{\infty} \frac{1 + \frac{1}{2} + \cdots + \frac{1}{n}}{n(n + 1)}
\end{align*}
\end{enumerate}
\end{example}
\begin{note}
证明的想法即强行裂项.
\end{note}
\begin{proof}
\begin{enumerate}
\item 记$H_n \triangleq 1 + \frac{1}{2} + \cdots + \frac{1}{n}$, 我们有
\begin{align*}
&\sum_{n=1}^{\infty} \frac{1 + \frac{1}{2} + \cdots + \frac{1}{n}}{(n + 1)(n + 2)} = \lim_{m \to \infty} \left( \sum_{n=1}^{m} \frac{H_n}{n + 1} - \sum_{n=1}^{m} \frac{H_n}{n + 2} \right) \\
&= \lim_{m \to \infty} \left( \sum_{n=1}^{m} \frac{H_n}{n + 1} - \sum_{n=1}^{m} \frac{H_{n + 1}}{n + 2} \right) + \lim_{m \to \infty} \left( \sum_{n=1}^{m} \frac{H_{n + 1}}{n + 2} - \sum_{n=1}^{m} \frac{H_n}{n + 2} \right) \\
&= \lim_{m \to \infty} \left( \frac{H_1}{2} - \frac{H_{m + 1}}{m + 2} \right) + \sum_{n=1}^{\infty} \frac{1}{(n + 1)(n + 2)} = \frac{1}{2} + \sum_{n=1}^{\infty} \left( \frac{1}{n + 1} - \frac{1}{n + 2} \right) = 1.
\end{align*}

\item 我们有
\begin{align*}
&\sum_{n=1}^{\infty} \frac{1 + \frac{1}{2} + \cdots + \frac{1}{n}}{n(n + 1)} = \sum_{n=1}^{\infty} \left( \frac{H_n}{n} - \frac{H_n}{n + 1} \right) \\
&= \sum_{n=1}^{\infty} \left( \frac{H_n}{n} - \frac{H_{n + 1}}{n + 1} \right) + \sum_{n=1}^{\infty} \left( \frac{H_{n + 1}}{n + 1} - \frac{H_n}{n + 1} \right) \\
&= H_1 + \sum_{n=1}^{\infty} \frac{1}{(n + 1)^2} = \frac{\pi^2}{6}.
\end{align*}
\end{enumerate}

\end{proof}

\begin{example}
计算
\begin{align*}
\sum_{n=1}^{\infty} \arctan \frac{1}{2n^2}
\end{align*}
\end{example}
\begin{note}
证明的想法即利用合适范围内都成立的恒等式
\begin{align*}
\arctan x - \arctan y = \arctan \frac{x - y}{1 + xy}
\end{align*}
来裂项.
\end{note}
\begin{proof}
我们有
\begin{align*}
\sum_{n=1}^{\infty} \arctan \frac{1}{2n^2} = \sum_{n=1}^{\infty} \left( \arctan \frac{n}{n + 1} - \arctan \frac{n - 1}{n} \right) = \lim_{n \to \infty} \arctan \frac{n}{n + 1} = \frac{\pi}{4}.
\end{align*}

\end{proof}



\subsection{凑已知函数}

\begin{example}
对 \(|x| < 1\), 计算
\begin{align*}
\sum_{n=0}^{\infty} \frac{4n^2 + 4n + 3}{2n + 1} x^{2n}
\end{align*}
\end{example}
\begin{proof}
我们有
\begin{align*}
\sum_{n=0}^{\infty} \frac{4n^2 + 4n + 3}{2n + 1} x^{2n} &= \sum_{n=0}^{\infty} (2n + 1) x^{2n} + 2 \sum_{n=0}^{\infty} \frac{x^{2n}}{2n + 1} \\
&= \left( \sum_{n=0}^{\infty} x^{2n + 1} \right)' + \frac{2}{x} \int_{0}^{x} \sum_{n=0}^{\infty} y^{2n} \mathrm{d}y \\
&= \left( \frac{x}{1 - x^2} \right)' + \frac{2}{x} \int_{0}^{x} \frac{1}{1 - y^2} \mathrm{d}y \\
&=\begin{cases}
\frac{1+x^2}{(1-x^2)^2}+\frac{1}{x}\ln \frac{1+x}{1-x}&,x\ne 0\\
3&,x=0\\
\end{cases}.
\end{align*}

\end{proof}

\begin{example}
计算
\[
1 - \frac{1}{6} - \sum_{k=2}^{\infty} \frac{(3k - 4)(3k - 7) \cdots 5 \cdot 2}{6^k k!}.
\]
\end{example}
\begin{proof}
我们有
\begin{align*}
&1 - \frac{1}{6} - \sum_{k=2}^{\infty} \frac{(3k - 4)(3k - 7) \cdots 5 \cdot 2}{6^k k!}= 1 - \frac{1}{6} - \sum_{k=2}^{\infty} \frac{3^{k - 1} \prod\limits_{j=2}^{k} \left(j - \frac{4}{3}\right)}{6^k k!}\\
&= 1 - \sum_{k=1}^{\infty} \frac{(-3)^{k - 1} 3 \prod\limits_{j=1}^{k} \left(\frac{1}{3} - j + 1\right)}{6^k k!}= 1 + \sum_{k=1}^{\infty} \binom{\frac{1}{3}}{k} \left(-\frac{1}{2}\right)^k = \left(1 - \frac{1}{2}\right)^{\frac{1}{3}} = 2^{-\frac{1}{3}}.
\end{align*}

\end{proof}

\begin{example}
对 \(|x| < 1\),计算
\[
\sum_{k=1}^{\infty} \frac{1}{2^k} \tan \frac{x}{2^k}.
\]
\end{example}
\begin{proof}
相似\refexa{example:例题3.82}的计算,我们有恒等式
\[
\frac{\sin x}{2^n \sin \frac{x}{2^n}} = \prod_{k=1}^{n} \cos \frac{x}{2^k}.
\]
于是
\[
\sum_{k=1}^{n} \ln \cos \frac{x}{2^k}=\ln \sin x - n \ln 2 - \ln \sin \frac{x}{2^n}.
\]
两边求导有
\[
- \sum_{k=1}^{n} \frac{1}{2^k} \tan \frac{x}{2^k} = \frac{\cos x}{\sin x} - \frac{\cos \frac{x}{2^n}}{2^n \sin \frac{x}{2^n}}.
\]
于是就有
\[
\sum_{k=1}^{\infty} \frac{1}{2^k} \tan \frac{x}{2^k} = - \frac{\cos x}{\sin x} + \lim_{n \to \infty} \frac{\cos \frac{x}{2^n}}{2^n \sin \frac{x}{2^n}} = 
\begin{cases} 
- \frac{\cos x}{\sin x} + \frac{1}{x}, & 0 < |x| < 1 \\
0, & x = 0 
\end{cases}.
\]

\end{proof}

\begin{example}
设\(S_{n}=\sum_{k = 1}^{\infty}\frac{1}{k^{n}}(n = 2,3,\cdots)\)。证明:
\[
\sum_{n = 2}^{\infty}(-1)^{n}\frac{S_{n}}{n}=\gamma
\]
式中\(\gamma\)是Euler常数。
\end{example}
\begin{proof}
注意到
\begin{align*}
\sum_{n=2}^{\infty}{\left( -1 \right) ^n\frac{S_n}{n}}&=\sum_{n=2}^{\infty}{\frac{\left( -1 \right) ^n}{n}\sum_{k=1}^{\infty}{\frac{1}{k^n}}}=\sum_{k=1}^{\infty}{\sum_{n=2}^{\infty}{\frac{\left( -1 \right) ^n}{nk^n}}},
\end{align*}
又由$\ln \left( 1+x \right)$的幂级数展开知
\begin{align*}
\sum_{n=2}^{\infty}{\frac{\left( -1 \right) ^n}{nk^n}}&=\frac{1}{k}+\sum_{n=1}^{\infty}{\frac{\left( -1 \right) ^n}{nk^n}}=\frac{1}{k}-\ln \left( 1+\frac{1}{k} \right) ,\quad \forall k\in \mathbb{N} .
\end{align*}
故对$\forall m\in \mathbb{N} ,$都有
\begin{align*}
\sum_{n=2}^m{\left( -1 \right) ^n\frac{S_n}{n}}&=\sum_{k=1}^m{\left[ \frac{1}{k}-\ln \left( 1+\frac{1}{k} \right) \right]}=\sum_{k=1}^m{\frac{1}{k}}-\sum_{k=1}^m{\ln \left( 1+\frac{1}{k} \right)}\\
&=\sum_{k=1}^m{\frac{1}{k}}-\sum_{k=1}^m{\ln \frac{k+1}{k}}=\sum_{k=1}^m{\frac{1}{k}}-\sum_{k=1}^m{\left[ \ln \left( k+1 \right) -\ln k \right]}\\
&=\ln \frac{m}{m+1}+\gamma +O\left( \frac{1}{m} \right) ,\quad m\rightarrow \infty .
\end{align*}
令$m\rightarrow \infty$得
\begin{align*}
\sum_{n=2}^{\infty}{\left( -1 \right) ^n\frac{S_n}{n}}&=\gamma .
\end{align*}

\end{proof}


\subsection{生成函数和幂级数计算方法}

\begin{example}\label{example:例题11.24556415}
计算
\[
\sum_{n=1}^{\infty} \left(1 + \frac{1}{2} + \cdots + \frac{1}{n}\right) x^n.
\]
\end{example}
\begin{note}
使用 \hyperref[definition:Cauchy积]{Cauchy积}计算幂级数有一个特点,即系数往往出现求和结构.
\end{note}
\begin{proof}
考虑 \(a_n = 1, n \in \mathbb{N}_0\),\(b_n = \begin{cases} \frac{1}{n}, & n \in \mathbb{N} \\ 0 & n = 0 \end{cases}\). 注意到
\[
\sum_{n=0}^{\infty} a_n x^n = \frac{1}{1 - x}, \sum_{n=0}^{\infty} b_n x^n = -\ln (1 - x),
\]
并且上述级数在$(-1,1)$上绝对收敛,于是由\hyperref[theorem:Cauchy积收敛定理]{Cauchy积收敛定理}及\refcor{corollary:收敛级数Cauchy积收敛则就等于级数积}可知
\[
\sum_{n=1}^{\infty} \left(1 + \frac{1}{2} + \cdots + \frac{1}{n}\right) x^n = -\frac{\ln (1 - x)}{1 - x}, |x| < 1.
\]
收敛域可以直接注意到
\[
\lim_{n \to \infty} \frac{1 + \frac{1}{2} + \cdots + \frac{1}{n} + \frac{1}{n + 1}}{1 + \frac{1}{2} + \cdots + \frac{1}{n}} = 1,
\]
以及
\[
\lim_{n \to \infty} \left(1 + \frac{1}{2} + \cdots + \frac{1}{n}\right) 1^n = \infty, \lim_{n \to \infty} \left(1 + \frac{1}{2} + \cdots + \frac{1}{n}\right) (-1)^n = \infty
\]
故收敛域就是$(-1,1)$.

\end{proof}

\begin{example}
设
\[
f(x)=\frac{1}{1 - x - x^2}, a_n=\frac{f^{(n)}(0)}{n!}, n = 0,1,2,\cdots,
\]
计算
\[
\sum_{n = 1}^{\infty}\frac{a_{n + 1}}{a_n a_{n + 2}}.
\]
\end{example}
\begin{note}
注意到形式幂级数法我们不需要担心考虑的 \( f \) 的幂级数是否收敛的问题. 因为这个方法最后往往可以算出一个 具体的\( f \), 对这个 \( f \) 来说直接用数学归纳法计算验证会发现其Taylor多项式的系数恰好就是条件中的数列, 从而整个逻辑严谨. 因此这又是一个从逻辑上来说属于\textbf{先猜后证}的方法.

对本题而言,$f(x)$已知,且容易求出其收敛半径.此时用形式幂级数法本身就是严谨地.
\end{note}
\begin{proof}
考虑 \( f(x)=\sum_{n = 0}^{\infty}a_n x^n \), 则对任意在其收敛域内的$x$我们有
\begin{align}
1&=(1 - x - x^2)\sum_{n = 0}^{\infty}a_n x^n=\sum_{n = 0}^{\infty}a_n x^n-\sum_{n = 0}^{\infty}a_n x^{n + 1}-\sum_{n = 0}^{\infty}a_n x^{n + 2}\nonumber \\
&=\sum_{n = 0}^{\infty}a_n x^n-\sum_{n = 1}^{\infty}a_{n - 1}x^n-\sum_{n = 2}^{\infty}a_{n - 2}x^n\nonumber \\
&=a_0 + a_1 x - a_0 x + \sum_{n = 2}^{\infty}(a_n - a_{n - 1} - a_{n - 2})x^n,\nonumber
\end{align}
于是对比系数得
\[
a_0 = 1,\ a_1 = a_0 = 1,\ a_n = a_{n - 1} + a_{n - 2},\ n = 2,3,\cdots. 
\]
显然有
\[
a_n\in\mathbb{N}\Rightarrow\lim_{n\rightarrow\infty}a_n=\infty,
\]
于是
\[
\sum_{n = 1}^{\infty}\frac{a_{n + 1}}{a_n a_{n + 2}}=\sum_{n = 1}^{\infty}\left(\frac{1}{a_n}-\frac{1}{a_{n + 2}}\right)=\frac{1}{a_1}+\frac{1}{a_2}=\frac{3}{2}.
\]
\begin{remark}
(证明可见复分析教材)为了求出 \( f \) 收敛半径, 可以展开点为中心作圆并一直扩大直到接触到和函数在 \( \mathbb{C} \) 上第一个奇点为止. 对于 \( f(x)=\frac{1}{1 - x - x^2} \), 第一个奇点即使得 \( \frac{1}{1 - x - x^2} \) 分母为 0 且模更小的点 \( \frac{\sqrt{5}-1}{2} \). 于是 \( f(x)=\sum_{n = 0}^{\infty}a_n x^n \) 收敛半径为 \( \frac{\sqrt{5}-1}{2} \). 于是我们得到一个极限
\[
\lim_{n\rightarrow\infty}\sqrt[n]{|a_n|}=\frac{2}{\sqrt{5}-1}=\frac{\sqrt{5}+1}{2}.
\]
\end{remark}

\end{proof}

\begin{example}
设 $a_0 = 0$, $a_1 = \frac{2}{3}$, $(n + 1)a_{n+1} = 2a_n + (n - 1)a_{n-1}$, $n \in \mathbb{N}$, 计算 $\sum_{n=0}^{\infty} n a_n x^n$ 收敛域和和函数.
\end{example}
\begin{proof}
记 $f(x) = \sum_{n=0}^{\infty} a_n x^n$, 形式的, 我们有
\begin{align*}
\sum_{n=1}^{\infty} (n + 1) a_{n+1} x^{n - 1} = 2 \sum_{n=1}^{\infty} a_n x^{n - 1} + \sum_{n=1}^{\infty} (n - 1) a_{n-1} x^{n - 1}.
\end{align*}
于是
\begin{align*}
\frac{1}{x} [f'(x) - a_1] = \frac{2}{x} [f(x) - a_0] + x f'(x) \Rightarrow \frac{1}{x} \left[ f'(x) - \frac{2}{3} \right] = \frac{2}{x} f(x) + x f'(x).
\end{align*}
故解微分方程得 $f(x) = \frac{2x}{3 - 3x}$, $x \in (-1,1)$. 这给出了 $a_n = \frac{2}{3}$, $n \in \mathbb{N}$. 于是
\begin{align*}
\sum_{n=0}^{\infty} n a_n x^n = \frac{2}{3} \sum_{n=0}^{\infty} n x^n = \frac{2}{3} x \left( \sum_{n=0}^{\infty} x^n \right)' = \frac{2}{3} x \left( \frac{1}{1 - x} \right)' = \frac{2x}{3(1 - x)^2}, x \in (-1,1).
\end{align*}

\end{proof}

\begin{example}
\begin{enumerate}
\item 计算
\begin{align*}
\sum_{n=0}^{\infty} \frac{(2n)!!}{(2n + 1)!!} x^{2n + 1}.
\end{align*}
\item 计算
\begin{align*}
\sum_{n=1}^{\infty} \left[ \left( \frac{1 \cdot 2 \cdot 3 \cdot \cdots \cdot n}{3 \cdot 5 \cdot 7 \cdot \cdots \cdot 2n + 1} \right) \cdot \frac{1}{n + 1} \right].
\end{align*}
\end{enumerate}
\end{example}
\begin{remark}
第 2 问是第十届大学生数学竞赛非数学类决赛得分率非常低的一个题. 可以看到如果我们平时记忆 $\arcsin^2 x$ 展开, 就能快速解题而规避掉最容易考的构造微分方程求解幂级数的技巧. 这一点我们在\refpro{proposition:一个额外记忆的级数}中也提到过.
\end{remark}
\begin{proof}
\begin{enumerate}
\item 考虑
\begin{align*}
g(x) = \sum_{n=0}^{\infty} \frac{(2n)!!}{(2n + 1)!!} x^{2n + 1}, g'(x) = 1 + \sum_{n=1}^{\infty} \frac{(2n)!!}{(2n - 1)!!} x^{2n},
\end{align*}
我们有
\begin{align*}
g'(x) &= 1 + x \sum_{n=1}^{\infty} \frac{(2n - 2)!!}{(2n - 1)!!} 2n x^{2n - 1} = 1 + x \left( \sum_{n=1}^{\infty} \frac{(2n - 2)!!}{(2n - 1)!!} x^{2n} \right)' \\
&= 1 + x \left( x \sum_{n=1}^{\infty} \frac{(2n - 2)!!}{(2n - 1)!!} x^{2n - 1} \right)' = 1 + x [x g(x)]',
\end{align*}
即
\begin{align*}
g'(x) - \frac{x}{1 - x^2} g(x) = \frac{1}{1 - x^2}, g(0) = 0.
\end{align*}
由常数变易法得 $g(x) = \frac{\arcsin x}{\sqrt{1 - x^2}}$. 于是收敛区间为 $|x| < 1$. 由
\begin{align*}
\lim_{x \to 1^-} g(x) = +\infty, \lim_{x \to -1^+} g(x) = +\infty
\end{align*}
知幂级数在 $x = \pm 1$ 不收敛, 故收敛域为 $|x| < 1$.
\item 首先把级数写成
\begin{align*}
\sum_{n=1}^{\infty} \left[ \left( \frac{1 \cdot 2 \cdot 3 \cdot \cdots \cdot n}{3 \cdot 5 \cdot 7 \cdot \cdots \cdot 2n + 1} \right) \cdot \frac{1}{n + 1} \right] = \sum_{n=1}^{\infty} \left[ \frac{n!}{(2n + 1)!!} \cdot \frac{1}{n + 1} \right].
\end{align*}
然后利用等式$(2n)!!=2^nn!$可考虑
\begin{align*}
f(x) = \sum_{n=1}^{\infty} \frac{(2n)!!}{(2n + 1)!!} \frac{x^{2n + 2}}{n + 1} = \frac{1}{2} \sum_{n=1}^{\infty} \left[ \frac{n!}{(2n + 1)!!} \cdot \frac{(\sqrt{2}x)^{2n + 2}}{n + 1} \right].
\end{align*}
现在所求级数为 $2f\left( \frac{1}{\sqrt{2}} \right)$.
我们利用第 1 问有
\begin{align*}
f'(x) = 2 \sum_{n=1}^{\infty} \frac{(2n)!!}{(2n + 1)!!} x^{2n + 1} = 2 \left( \frac{\arcsin x}{\sqrt{1 - x^2}} - x \right).
\end{align*}
于是我们有
\begin{align*}
2f\left( \frac{1}{\sqrt{2}} \right) - 2f(0) = 2 \int_0^{\frac{1}{\sqrt{2}}} \left[ 2 \left( \frac{\arcsin x}{\sqrt{1 - x^2}} - x \right) \right] \mathrm{d}x = \frac{\pi^2}{8} - 1.
\end{align*}
\end{enumerate}

\end{proof}

\begin{example}
计算 $\sum_{n=0}^{\infty} \frac{1}{(4n)!}$.
\end{example}
\begin{proof}
注意到
\begin{align*}
f(x) = \sum_{n=0}^{\infty} \frac{x^{4n}}{(4n)!}, f'(x) = \sum_{n=1}^{\infty} \frac{x^{4n - 1}}{(4n - 1)!}, f''(x) = \sum_{n=1}^{\infty} \frac{x^{4n - 2}}{(4n - 2)!}, f'''(x) = \sum_{n=1}^{\infty} \frac{x^{4n - 3}}{(4n - 3)!}.
\end{align*}
于是我们有
\begin{align*}
f + f' + f'' + f''' = \sum_{n=0}^{\infty} \frac{x^n}{n!} = e^x.
\end{align*}
注意到 $f(0) = 1$, $f'(0) = f''(0) = f'''(0) = 0$, 求解微分方程(Euler待定指数法)得解 $f(x) = \frac{1}{4} (e^{-x} + e^x + 2\cos x)$.

\end{proof}





\subsection{多重求和}

\begin{example}
计算
\[
\sum_{m = 1}^{\infty}\sum_{n = 1}^{\infty}\frac{m^2 n}{3^m\left(n 3^m + m 3^n\right)}.
\]
\end{example}
\begin{note}
二重级数一类题型往往会用对称性来简化结构.
\end{note}
\begin{proof}
直接计算有
\[
\begin{aligned}
\sum_{m = 1}^{\infty}\sum_{n = 1}^{\infty}\frac{m^2 n}{3^m\left(n 3^m + m 3^n\right)}&\xlongequal{\text{\hyperref[theorem:级数的Fubini定理]{级数的Fubini定理}}}\sum_{m = 1}^{\infty}\sum_{n = 1}^{\infty}\frac{m n^2}{3^n\left(m 3^n + n 3^m\right)}=\frac{1}{2}\sum_{m = 1}^{\infty}\sum_{n = 1}^{\infty}\frac{3^m m n^2 + 3^n m^2 n}{3^{n + m}\left(m 3^n + n 3^m\right)}\\
&=\frac{1}{2}\sum_{m = 1}^{\infty}\sum_{n = 1}^{\infty}\frac{m n}{3^{n + m}}=\frac{1}{2}\left(\sum_{m = 1}^{\infty}\frac{m}{3^m}\right)^2=\frac{9}{32},
\end{aligned}
\]
这里最后的级数是一个差比数列, 高中数学的错位相减可以直接算出结果.或者利用凑已知函数的方法计算 :
\begin{align*}
\sum_{m=1}^{\infty}mx^m&=x\left( \sum_{m=1}^{\infty}x^m \right)' =x\cdot \left( \frac{1}{1-x} \right)' =\frac{x}{\left( 1-x \right) ^2}.
\end{align*}
将$x=\frac{1}{3}$代入得
\begin{align*}
\sum_{m=1}^{\infty}\frac{m}{3^m}&=\frac{3}{4}\Rightarrow \frac{1}{2}\left( \sum_{m=1}^{\infty}\frac{m}{3^m} \right) ^2=\frac{9}{32}.
\end{align*}

\end{proof}

\begin{example}
计算
\[
\left[\sum_{m = 1}^{\infty}\sum_{n = 1}^{\infty}\frac{100m^2 n}{2^m\left(n 2^m + m 2^n\right)}\right].
\]
\end{example}
\begin{proof}
我们有
\[
\begin{aligned}
\sum_{m = 1}^{\infty}\sum_{n = 1}^{\infty}\frac{100m^2 n}{2^m\left(n 2^m + m 2^n\right)}&\xlongequal{\text{\hyperref[theorem:级数的Fubini定理]{级数的Fubini定理}}}\sum_{m=1}^{\infty}{\sum_{n=1}^{\infty}{\frac{100mn^2}{2^n\left( m2^n+n2^m \right)}}}=50\sum_{m = 1}^{\infty}\sum_{n = 1}^{\infty}\left(\frac{m^2 n}{2^m\left(n 2^m + m 2^n\right)}+\frac{n^2 m}{2^n\left(m 2^n + n 2^m\right)}\right)\\
&=50\sum_{m = 1}^{\infty}\sum_{n = 1}^{\infty}\left(\frac{mn\left(\frac{m}{2^m}+\frac{n}{2^n}\right)}{n 2^m + m 2^n}\right)=50\sum_{m = 1}^{\infty}\sum_{n = 1}^{\infty}\frac{mn}{2^{m + n}}
=50\left(\sum_{n = 1}^{\infty}\frac{n}{2^n}\right)^2=200.
\end{aligned}
\]

\end{proof}



\subsection{级数特殊算法(换序法)}

\begin{example}
\begin{enumerate}
\item 证明: $\sum_{n=1}^{\infty} \frac{1}{(2n - 1)^2} = \frac{\pi^2}{8}$.
\item 证明: $\sum_{n=1}^{\infty} \frac{(-1)^{n - 1}}{(2n - 1)^3} = \frac{\pi^3}{32}$.
\item 证明: $\sum_{n=1}^{\infty} \frac{1}{n^2 2^n} = \frac{\pi^2}{12} - \frac{\ln^2 2}{2}$.
\end{enumerate}
\end{example}
\begin{remark}
熟知$\sum_{n=1}^{\infty}{\frac{1}{n^2}}=\frac{\pi ^2}{6}.$
\end{remark}
\begin{proof}
\begin{enumerate}
\item 我们有
\begin{align*}
\sum_{n=1}^{\infty} \frac{1}{(2n - 1)^2} = \sum_{n=1}^{\infty} \frac{1}{n^2} - \sum_{n=1}^{\infty} \frac{1}{(2n)^2} = \frac{3}{4} \sum_{n=1}^{\infty} \frac{1}{n^2} = \frac{\pi^2}{8}.
\end{align*}
\item (考试肯定会给提示或多设置一问)注意到傅立叶展开$f(x) = x^3 - \pi^2 x$, $x \in [-\pi, \pi]$ 得
\begin{align*}
x^3 - \pi^2 x \sim 12 \sum_{n=1}^{\infty} \frac{(-1)^n}{n^3} \sin(nx).
\end{align*}
考虑 $x = \frac{\pi}{2}$ 即得
\begin{align*}
12\sum_{n=1}^{\infty}{\frac{(-1)^n}{n^3}\sin\mathrm{(}nx)}=-12\sum_{n=1}^{\infty}{\frac{1}{\left( 2n-1 \right) ^3}\sin \left( \frac{\pi}{2}\left( 2n-1 \right) \right)}=-12\sum_{n=1}^{\infty}{\frac{1}{\left( 2n-1 \right) ^3}\sin \left( n\pi -\frac{\pi}{2} \right)}=12\sum_{n=1}^{\infty}{\frac{\left( -1 \right) ^n}{\left( 2n-1 \right) ^3}}.
\end{align*}
故$12\sum_{n=1}^{\infty}{\frac{\left( -1 \right) ^n}{\left( 2n-1 \right) ^3}}=\frac{\pi ^3}{32}.$
\item 由\nrefpro{proposition:Li_2函数的性质}{(2)}得到 $\sum_{n=1}^{\infty} \frac{1}{n^2 2^n} = \frac{\pi^2}{12} - \frac{\ln^2 2}{2}$.
\end{enumerate}

\end{proof}

\begin{example}
设 $f \in C^1[0,1]$, $f(x) \geqslant 0$, 证明下述级数收敛且求值
\begin{align*}
\sum_{n=1}^{\infty} (-1)^{n - 1} \int_0^1 x^n f(x) \mathrm{d}x.
\end{align*}
\end{example}
\begin{note}
为了有换序
\begin{align*}
\sum_{n=1}^{\infty} \int f_n(x) \mathrm{d}x = \int \sum_{n=1}^{\infty} f_n(x) \mathrm{d}x,
\end{align*}
我们只需要
\begin{align*}
\lim_{m \to \infty} \sum_{n=1}^{m} \int f_n(x) \mathrm{d}x = \lim_{m \to \infty} \int \sum_{n=1}^{m} f_n(x) \mathrm{d}x = \int \sum_{n=1}^{\infty} f_n(x) \mathrm{d}x,
\end{align*}
即需要证明
\begin{align*}
\lim_{m \to \infty} \int \sum_{n=m + 1}^{\infty} f_n(x) \mathrm{d}x = 0.
\end{align*}
\end{note}
\begin{remark}
实际上,这里的换序就是\hyperref[Real Analysis-theorem:控制收敛定理]{控制收敛定理}.
\end{remark}
\begin{proof}
显然 $\int_0^1 x^n f(x) \mathrm{d}x$ 递减且
\begin{align*}
0 \leqslant \int_0^1 x^n f(x) \mathrm{d}x \leqslant \max f \cdot \int_0^1 x^n \mathrm{d}x \to 0, n \to \infty,
\end{align*}
故由交错级数判别法知 $\sum_{n=1}^{\infty} (-1)^{n - 1} \int_0^1 x^n f(x) \mathrm{d}x$ 收敛. 故
\begin{align*}
\sum_{n=1}^{\infty} (-1)^{n - 1} \int_0^1 x^n f(x) \mathrm{d}x = - \int_0^1 \sum_{n=1}^{\infty} (-x)^n f(x) \mathrm{d}x = \int_0^1 \frac{x f(x)}{1 + x} \mathrm{d}x,
\end{align*}
这里换序来自
\begin{align*}
\left| \int_0^1 \sum_{n=m}^{\infty} (-x)^n f(x) \mathrm{d}x \right| \stackrel{\text{\hyperref[theorem:交错级数不等式]{交错级数不等式}}}{\leqslant} \int_0^1 x^m f(x) \mathrm{d}x \to 0, m \to \infty.
\end{align*}

\end{proof}

\begin{proposition}[组合数的无穷和技巧]\label{proposition:组合数的无穷和技巧}
1. 我们有
\begin{align*}
\sum_{n=0}^{\infty} a_n (y + x)^n &= \sum_{k=0}^{\infty} b_k y^k \Rightarrow b_k = x^{-k} \sum_{n=k}^{\infty} C_n^k a_n x^n.
\end{align*}

2. 我们有
\begin{align*}
\sum_{n=0}^{m} a_n (y + x)^n &= \sum_{k=0}^{m} b_k y^k \Rightarrow b_k = x^{-k} \sum_{n=k}^{m} C_n^k a_n x^n.
\end{align*}
\end{proposition}
\begin{proof}


\end{proof}

\begin{example}
计算
\begin{align*}
\sum_{n=k}^{\infty} C_n^k \left(1 + \frac{1}{2} + \cdots + \frac{1}{n}\right) x^n, k \in \mathbb{N}. 
\end{align*}
\end{example}
\begin{proof}
取 $a_n = 1 + \frac{1}{2} + \cdots + \frac{1}{n}, n \in \mathbb{N}$. 由\refexa{example:例题11.24556415},我们有
\begin{align*}
\sum_{n=1}^{\infty} \left(1 + \frac{1}{2} + \cdots + \frac{1}{n}\right) (y + x)^n &= -\frac{\ln(1 - x - y)}{1 - x - y} = -\frac{\ln(1 - x)}{1 - x} \frac{1}{1 - \frac{y}{1 - x}} - \frac{1}{1 - x} \frac{\ln\left(1 - \frac{y}{1 - x}\right)}{1 - \frac{y}{1 - x}} \\
&= -\frac{\ln(1 - x)}{1 - x} \sum_{k=0}^{\infty} \frac{y^k}{(1 - x)^k} + \frac{1}{1 - x} \sum_{k=1}^{\infty} \left(1 + \frac{1}{2} + \cdots + \frac{1}{k}\right) \frac{y^k}{(1 - x)^k} \\
&= -\frac{\ln(1 - x)}{1 - x} + \sum_{k=1}^{\infty} \left[\frac{1 + \frac{1}{2} + \cdots + \frac{1}{k} - \ln(1 - x)}{(1 - x)^{k + 1}}\right] y^k
\end{align*}
于是由\refpro{proposition:组合数的无穷和技巧}, 我们有
\begin{align*}
\sum_{n=k}^{\infty} C_n^k \left(1 + \frac{1}{2} + \cdots + \frac{1}{n}\right) x^n = b_k x^k = \left[\frac{1 + \frac{1}{2} + \cdots + \frac{1}{k} - \ln(1 - x)}{(1 - x)^{k + 1}}\right] x^k 
\end{align*}
注意到和函数第一个奇点是 $x = 1$, 所以幂级数收敛半径是 1. 注意到和函数在 $x = 1$ 的左极限发散, 因此幂级数在 $x = 1$ 不收敛. 虽然和函数在 $x = -1$ 的右极限收敛, 但并不能一定能推出幂级数在 $x = -1$ 收敛, 为了判断 $x = -1$ 的收敛性, 我们要使用小 o Tauber定理.

若 $\lim_{x \to 1^-} \sum_{n=k}^{\infty} C_n^k \left(1 + \frac{1}{2} + \cdots + \frac{1}{n}\right) (-x)^n$ 存在, 则由小 o Tauber定理知
\begin{align*}
\lim_{m \to \infty} \frac{1}{m} \sum_{n=k}^{k + m} \left[(-1)^n n C_n^k \left(1 + \frac{1}{2} + \cdots + \frac{1}{n}\right)\right] = 0 .
\end{align*}
注意到
\begin{align*}
\lim_{m \to \infty} \frac{1}{2m} \sum_{n=k}^{k + 2m} \left[(-1)^n n C_n^k \left(1 + \frac{1}{2} + \cdots + \frac{1}{n}\right)\right] = 0.
\end{align*}
\begin{align*}
\lim_{m \to \infty} \frac{1}{2m + 1} \sum_{n=k}^{k + 2m + 1} \left[(-1)^n n C_n^k \left(1 + \frac{1}{2} + \cdots + \frac{1}{n}\right)\right] = 0 .
\end{align*}
我们有
\begin{align*}
\lim_{m \to \infty} \frac{(-1)^{k + 2m + 1} (k + 2m + 1) C_{k + 2m + 1}^k \left(1 + \frac{1}{2} + \cdots + \frac{1}{k + 2m + 1}\right)}{2m + 1} = 0.
\end{align*}
又
\begin{align*}
\lim_{m \to \infty} C_{k + 2m + 1}^k = \lim_{m \to \infty} \frac{(k + 2m + 1)!}{k! (2m + 1)!} = +\infty .
\end{align*}
矛盾! 因此我们证明了原幂级数收敛域是 $(-1, 1)$.

\end{proof}





































\end{document}
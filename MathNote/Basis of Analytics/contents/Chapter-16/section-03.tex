\documentclass[../../main.tex]{subfiles}
\graphicspath{{\subfix{../../image/}}} % 指定图片目录,后续可以直接使用图片文件名。

% 例如:
% \begin{figure}[H]
% \centering
% \includegraphics[scale=0.4]{图.png}
% \caption{}
% \label{figure:图}
% \end{figure}
% 注意:上述\label{}一定要放在\caption{}之后,否则引用图片序号会只会显示??.

\begin{document}

\section{级数计算}

\subsection{裂项方法}

\begin{example}
计算$\sum_{n=1}^{\infty} \frac{1}{2^n (1 + \sqrt[2^n]{2})}.$
\end{example}
\begin{note}
继续采用强行裂项的想法,猜出裂项之后的模样之后还原看看差什么.
\end{note}
\begin{proof}
注意到
\begin{align*}
\frac{1}{2^{n-1}\left( 2^{\frac{1}{2^{n-1}}}-1 \right)}-\frac{1}{2^n\left( 2^{\frac{1}{2^n}}-1 \right)}&=\frac{2}{2^n\left( 2^{\frac{1}{2^{n-1}}}-1 \right)}-\frac{2^{\frac{1}{2^n}}+1}{2^n\left( 2^{\frac{1}{2^{n-1}}}-1 \right)}=-\frac{2^{\frac{1}{2^n}}-1}{2^n\left( 2^{\frac{1}{2^{n-1}}}-1 \right)}
\\
&=-\frac{2^{\frac{1}{2^n}}-1}{2^n\left( 2^{\frac{1}{2^n}}+1 \right) \left( 2^{\frac{1}{2^n}}-1 \right)}=-\frac{1}{2^n\left( 2^{\frac{1}{2^n}}+1 \right)},
\end{align*}
我们有
\begin{align*}
\sum_{n=1}^{\infty} \frac{1}{2^n \left(2^{\frac{1}{2^n}} + 1\right)} = \lim_{n \to \infty} \left( \frac{1}{2^n \left(2^{\frac{1}{2^n}} - 1\right)} \right) - 1 = \frac{1}{\ln 2} - 1.
\end{align*}
\end{proof}

\begin{example}
计算
\begin{align*}
\sum_{k=1}^{\infty} \frac{1}{k(k + 1)(k + 1)!}
\end{align*}
\end{example}
\begin{note}
想法的关键是强行裂项.
\end{note}
\begin{proof}
\begin{align*}
\sum_{k=1}^{\infty} \frac{1}{k(k + 1)(k + 1)!} &= \sum_{k=1}^{\infty} \left( \frac{1}{k} - \frac{1}{k + 1} \right) \frac{1}{(k + 1)!} = \sum_{k=1}^{\infty} \left( \frac{1}{k(k + 1)!} - \frac{1}{(k + 1)(k + 1)!} \right) \\
&= \sum_{k=1}^{\infty} \left( \frac{1}{k(k + 1)!} - \frac{1}{(k + 1)(k + 2)!} \right) + \sum_{k=1}^{\infty} \left( \frac{1}{(k + 1)(k + 2)!} - \frac{1}{(k + 1)(k + 1)!} \right) \\
&= \frac{1}{2} - \sum_{k=1}^{\infty} \frac{1}{(k + 2)!} \xlongequal{e\text{的Taylor展开}} \frac{1}{2} - \left( e - 1-1 - \frac{1}{2} \right) = 3 - e.
\end{align*}
\end{proof}

\begin{example}
计算级数
\begin{align*}
\sum_{n=1}^{\infty} (-1)^{n - 1} \frac{\ln n}{n}
\end{align*}
\end{example}
\begin{note}
此类问题化部分和之后估阶.
\end{note}
\begin{proof}
注意到
\begin{align*}
\sum_{n=1}^{2m+1}{\left( -1 \right) ^{n-1}\frac{\ln n}{n}}=\sum_{n=1}^{2m}{\left( -1 \right) ^{n-1}\frac{\ln n}{n}}+\frac{\ln \left( 2m+1 \right)}{2m+1}.
\end{align*}
令$m\rightarrow \infty ,$得
\begin{align*}
\underset{m\rightarrow \infty}{\lim}\sum_{n=1}^{2m+1}{\left( -1 \right) ^{n-1}\frac{\ln n}{n}}=\underset{m\rightarrow \infty}{\lim}\sum_{n=1}^{2m}{\left( -1 \right) ^{n-1}\frac{\ln n}{n}}.
\end{align*}
于是由\hyperref[proposition:子列极限命题]{子列极限命题(b)}可得
\begin{align*}
&\sum_{n=1}^{\infty} (-1)^{n - 1} \frac{\ln n}{n} = \lim_{m \to \infty} \sum_{n=1}^{2m} (-1)^{n - 1} \frac{\ln n}{n} = \lim_{m \to \infty} \sum_{n=1}^{m} \left( \frac{\ln(2n - 1)}{2n - 1} - \frac{\ln(2n)}{2n} \right) \\
&= \lim_{m \to \infty} \left( \sum_{n=1}^{2m} \frac{\ln n}{n} - \sum_{n=1}^{m} \frac{\ln(2n)}{2n} - \sum_{n=1}^{m} \frac{\ln(2n)}{2n} \right) = \lim_{m \to \infty} \left( \sum_{n=1}^{2m} \frac{\ln n}{n} - \sum_{n=1}^{m} \frac{\ln n}{n} - \sum_{n=1}^{m} \frac{\ln 2}{n} \right)
\end{align*}
利用\nrefexa{example:8.61846546}{(2)},我们知道
\begin{align*}
\sum_{n=1}^{m} \frac{\ln 2}{n} = \ln 2 \cdot \ln m + \ln 2 \cdot \gamma + o(1), m \to \infty
\end{align*}
由\hyperref[proposition:0阶欧拉麦克劳林公式(0阶E-M公式)]{0阶E-M公式}知道
\begin{align*}
\sum_{n=1}^{m} \frac{\ln n}{n} = \frac{\ln m}{2m} + \int_{1}^{m} \frac{\ln x}{x} \mathrm{d}x + \int_{1}^{m} \left( x - [x] - \frac{1}{2} \right) \left( \frac{\ln x}{x} \right)' \mathrm{d}x
\end{align*}
注意到$\int_{1}^{m} \frac{\ln x}{x} \mathrm{d}x = \frac{1}{2} \ln^2 m$以及
\begin{align*}
\left| \int_{1}^{m} \left( x - [x] - \frac{1}{2} \right) \left( \frac{\ln x}{x} \right)' \mathrm{d}x \right| = \left| \int_{1}^{m} \left( x - [x] - \frac{1}{2} \right) \frac{1 - \ln x}{x^2} \mathrm{d}x \right| \leqslant \frac{1}{2} \int_{1}^{\infty} \frac{|1 - \ln x|}{x^2} \mathrm{d}x < \infty
\end{align*}
于是我们有
\begin{align*}
\sum_{n=1}^{m} \frac{\ln n}{n} = \frac{1}{2} \ln^2 m + C + o(1), m \to \infty
\end{align*}
这里$C = \int_{1}^{\infty} \left( x - [x] - \frac{1}{2} \right) \left( \frac{\ln x}{x} \right)' \mathrm{d}x$. 现在结合上述渐近估计式就有
\begin{align*}
\sum_{n=1}^{\infty} (-1)^{n - 1} \frac{\ln n}{n} = \lim_{m \to \infty} \left[ \frac{1}{2} \ln^2 (2m) - \frac{1}{2} \ln^2 m - \ln 2 \cdot \ln m - \ln 2 \cdot \gamma + o(1) \right] = \frac{\ln^2 2}{2} - \ln 2 \cdot \gamma
\end{align*}
\end{proof}

\begin{example}
\begin{enumerate}
\item 计算
\begin{align*}
\sum_{n=1}^{\infty} \frac{1 + \frac{1}{2} + \cdots + \frac{1}{n}}{(n + 1)(n + 2)}
\end{align*}

\item 计算
\begin{align*}
\sum_{n=1}^{\infty} \frac{1 + \frac{1}{2} + \cdots + \frac{1}{n}}{n(n + 1)}
\end{align*}
\end{enumerate}
\end{example}
\begin{note}
证明的想法即强行裂项.
\end{note}
\begin{proof}
\begin{enumerate}
\item 记$H_n \triangleq 1 + \frac{1}{2} + \cdots + \frac{1}{n}$, 我们有
\begin{align*}
&\sum_{n=1}^{\infty} \frac{1 + \frac{1}{2} + \cdots + \frac{1}{n}}{(n + 1)(n + 2)} = \lim_{m \to \infty} \left( \sum_{n=1}^{m} \frac{H_n}{n + 1} - \sum_{n=1}^{m} \frac{H_n}{n + 2} \right) \\
&= \lim_{m \to \infty} \left( \sum_{n=1}^{m} \frac{H_n}{n + 1} - \sum_{n=1}^{m} \frac{H_{n + 1}}{n + 2} \right) + \lim_{m \to \infty} \left( \sum_{n=1}^{m} \frac{H_{n + 1}}{n + 2} - \sum_{n=1}^{m} \frac{H_n}{n + 2} \right) \\
&= \lim_{m \to \infty} \left( \frac{H_1}{2} - \frac{H_{m + 1}}{m + 2} \right) + \sum_{n=1}^{\infty} \frac{1}{(n + 1)(n + 2)} = \frac{1}{2} + \sum_{n=1}^{\infty} \left( \frac{1}{n + 1} - \frac{1}{n + 2} \right) = 1.
\end{align*}

\item 我们有
\begin{align*}
&\sum_{n=1}^{\infty} \frac{1 + \frac{1}{2} + \cdots + \frac{1}{n}}{n(n + 1)} = \sum_{n=1}^{\infty} \left( \frac{H_n}{n} - \frac{H_n}{n + 1} \right) \\
&= \sum_{n=1}^{\infty} \left( \frac{H_n}{n} - \frac{H_{n + 1}}{n + 1} \right) + \sum_{n=1}^{\infty} \left( \frac{H_{n + 1}}{n + 1} - \frac{H_n}{n + 1} \right) \\
&= H_1 + \sum_{n=1}^{\infty} \frac{1}{(n + 1)^2} = \frac{\pi^2}{6}.
\end{align*}
\end{enumerate}
\end{proof}

\begin{example}
计算
\begin{align*}
\sum_{n=1}^{\infty} \arctan \frac{1}{2n^2}
\end{align*}
\end{example}
\begin{note}
证明的想法即利用合适范围内都成立的恒等式
\begin{align*}
\arctan x - \arctan y = \arctan \frac{x - y}{1 + xy}
\end{align*}
来裂项.
\end{note}
\begin{proof}
我们有
\begin{align*}
\sum_{n=1}^{\infty} \arctan \frac{1}{2n^2} = \sum_{n=1}^{\infty} \left( \arctan \frac{n}{n + 1} - \arctan \frac{n - 1}{n} \right) = \lim_{n \to \infty} \arctan \frac{n}{n + 1} = \frac{\pi}{4}.
\end{align*}
\end{proof}



\subsection{凑已知函数}

\begin{example}
对 \(|x| < 1\), 计算
\begin{align*}
\sum_{n=0}^{\infty} \frac{4n^2 + 4n + 3}{2n + 1} x^{2n}
\end{align*}
\end{example}
\begin{proof}
我们有
\begin{align*}
\sum_{n=0}^{\infty} \frac{4n^2 + 4n + 3}{2n + 1} x^{2n} &= \sum_{n=0}^{\infty} (2n + 1) x^{2n} + 2 \sum_{n=0}^{\infty} \frac{x^{2n}}{2n + 1} \\
&= \left( \sum_{n=0}^{\infty} x^{2n + 1} \right)' + \frac{2}{x} \int_{0}^{x} \sum_{n=0}^{\infty} y^{2n} dy \\
&= \left( \frac{x}{1 - x^2} \right)' + \frac{2}{x} \int_{0}^{x} \frac{1}{1 - y^2} dy \\
&=\begin{cases}
\frac{1+x^2}{(1-x^2)^2}+\frac{1}{x}\ln \frac{1+x}{1-x}&,x\ne 0\\
3&,x=0\\
\end{cases}.
\end{align*}
\end{proof}

\begin{example}
计算
\[
1 - \frac{1}{6} - \sum_{k=2}^{\infty} \frac{(3k - 4)(3k - 7) \cdots 5 \cdot 2}{6^k k!}.
\]
\end{example}
\begin{proof}
我们有
\begin{align*}
&1 - \frac{1}{6} - \sum_{k=2}^{\infty} \frac{(3k - 4)(3k - 7) \cdots 5 \cdot 2}{6^k k!}= 1 - \frac{1}{6} - \sum_{k=2}^{\infty} \frac{3^{k - 1} \prod\limits_{j=2}^{k} \left(j - \frac{4}{3}\right)}{6^k k!}\\
&= 1 - \sum_{k=1}^{\infty} \frac{(-3)^{k - 1} 3 \prod\limits_{j=1}^{k} \left(\frac{1}{3} - j + 1\right)}{6^k k!}= 1 + \sum_{k=1}^{\infty} \binom{\frac{1}{3}}{k} \left(-\frac{1}{2}\right)^k = \left(1 - \frac{1}{2}\right)^{\frac{1}{3}} = 2^{-\frac{1}{3}}.
\end{align*}
\end{proof}

\begin{example}
对 \(|x| < 1\),计算
\[
\sum_{k=1}^{\infty} \frac{1}{2^k} \tan \frac{x}{2^k}.
\]
\end{example}
\begin{proof}
相似\refexa{example:例题3.82}的计算,我们有恒等式
\[
\frac{\sin x}{2^n \sin \frac{x}{2^n}} = \prod_{k=1}^{n} \cos \frac{x}{2^k}.
\]
于是
\[
\sum_{k=1}^{n} \ln \cos \frac{x}{2^k}=\ln \sin x - n \ln 2 - \ln \sin \frac{x}{2^n}.
\]
两边求导有
\[
- \sum_{k=1}^{n} \frac{1}{2^k} \tan \frac{x}{2^k} = \frac{\cos x}{\sin x} - \frac{\cos \frac{x}{2^n}}{2^n \sin \frac{x}{2^n}}.
\]
于是就有
\[
\sum_{k=1}^{\infty} \frac{1}{2^k} \tan \frac{x}{2^k} = - \frac{\cos x}{\sin x} + \lim_{n \to \infty} \frac{\cos \frac{x}{2^n}}{2^n \sin \frac{x}{2^n}} = 
\begin{cases} 
- \frac{\cos x}{\sin x} + \frac{1}{x}, & 0 < |x| < 1 \\
0, & x = 0 
\end{cases}.
\]
\end{proof}



\subsection{生成函数和幂级数计算方法}

\begin{example}
计算
\[
\sum_{n=1}^{\infty} \left(1 + \frac{1}{2} + \cdots + \frac{1}{n}\right) x^n.
\]
\end{example}
\begin{note}
使用 \hyperref[definition:Cauchy积]{Cauchy积}计算幂级数有一个特点,即系数往往出现求和结构.
\end{note}
\begin{proof}
考虑 \(a_n = 1, n \in \mathbb{N}_0\),\(b_n = \begin{cases} \frac{1}{n}, & n \in \mathbb{N} \\ 0 & n = 0 \end{cases}\). 注意到
\[
\sum_{n=0}^{\infty} a_n x^n = \frac{1}{1 - x}, \sum_{n=0}^{\infty} b_n x^n = -\ln (1 - x),
\]
于是
\[
\sum_{n=1}^{\infty} \left(1 + \frac{1}{2} + \cdots + \frac{1}{n}\right) x^n = -\frac{\ln (1 - x)}{1 - x}, |x| < 1.
\]
收敛域可以直接注意到
\[
\lim_{n \to \infty} \frac{1 + \frac{1}{2} + \cdots + \frac{1}{n} + \frac{1}{n + 1}}{1 + \frac{1}{2} + \cdots + \frac{1}{n}} = 1,
\]
以及
\[
\lim_{n \to \infty} \left(1 + \frac{1}{2} + \cdots + \frac{1}{n}\right) 1^n = \infty, \lim_{n \to \infty} \left(1 + \frac{1}{2} + \cdots + \frac{1}{n}\right) (-1)^n = \infty
\]
来得到.
\end{proof}




















































\end{document}
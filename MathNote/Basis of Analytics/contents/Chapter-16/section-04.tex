\documentclass[../../main.tex]{subfiles}
\graphicspath{{\subfix{../../image/}}} % 指定图片目录,后续可以直接使用图片文件名。

% 例如:
% \begin{figure}[H]
% \centering
% \includegraphics[scale=0.4]{图.png}
% \caption{}
% \label{figure:图}
% \end{figure}
% 注意:上述\label{}一定要放在\caption{}之后,否则引用图片序号会只会显示??.

\begin{document}

\section{级数一致收敛性判断}

\begin{theorem}[函数列一致收敛的柯西准则]\label{theorem:函数列一致收敛的柯西准则}
函数列\(\{f_n\}\)在数集\(D\)上一致收敛的充要条件是:对任给正数\(\varepsilon\),总存在正数\(N\),使得当\(n, m > N\)时,对一切\(x \in D\),都有
\begin{align}
|f_n(x) - f_m(x)| < \varepsilon.\label{4}
\end{align}
\end{theorem}
\begin{proof}
{\heiti 必要性} 设\(f_n(x) \rightrightarrows f(x) \ (n \to \infty)\),\(x \in D\),即对任给\(\varepsilon > 0\),存在正数\(N\),使得当\(n > N\)时,对一切\(x \in D\),都有
\begin{align}
|f_n(x) - f(x)| < \frac{\varepsilon}{2}. \label{5}
\end{align}
于是当\(n, m > N\),由\(\eqref{5}\)就有
\begin{align*}
|f_n(x) - f_m(x)| \leqslant |f_n(x) - f(x)| + |f(x) - f_m(x)| < \frac{\varepsilon}{2} + \frac{\varepsilon}{2} = \varepsilon.
\end{align*}

{\heiti 充分性} 若条件\(\eqref{4}\)成立,由数列收敛的柯西准则,\(\{f_n\}\)在\(D\)上任一点都收敛,记其极限函数为\(f(x)\),\(x \in D\).现固定\(\eqref{4}\)式中的\(n\),让\(m \to \infty\),于是当\(n > N\)时,对一切\(x \in D\),都有
\[
|f_n(x) - f(x)| \leqslant \varepsilon.
\]
因此,\(f_n(x) \rightrightarrows f(x) \ (n \to \infty)\),\(x \in D\).

\end{proof} 

\begin{theorem}
函数列\(\{f_n\}\)在区间\(D\)上一致收敛于\(f\)的充要条件是:
\begin{align}
\lim_{n \to \infty} \sup_{x \in D} |f_n(x) - f(x)| = 0. \label{6}
\end{align}
\end{theorem} 
\begin{proof}
{\heiti 必要性} 若\(f_n(x) \rightrightarrows f(x) \ (n \to \infty)\),\(x \in D\).则对任给的正数\(\varepsilon\),存在不依赖于\(x\)的正整数\(N\),当\(n > N\)时,有
\[
|f_n(x) - f(x)| < \varepsilon, \ x \in D.
\]
由上确界的定义,亦有
\[
\sup_{x \in D} |f_n(x) - f(x)| \leqslant \varepsilon.
\]
这就证得\(\eqref{6}\)式成立.

 由假设,对任给\(\varepsilon > 0\),存在正整数\(N\),使得当\(n > N\)时,有
\begin{align}
\sup_{x \in D} |f_n(x) - f(x)| < \varepsilon. \label{7}
\end{align}
因为对一切\(x \in D\),总有
\[
|f_n(x) - f(x)| \leqslant \sup_{x \in D} |f_n(x) - f(x)|,
\]
故由\(\eqref{7}\)式得
\[
|f_n(x) - f(x)| < \varepsilon.
\]
于是\(\{f_n\}\)在\(D\)上一致收敛于\(f\).

\end{proof}

\begin{corollary}\label{corollary:函数列不一致收敛的充要条件}
函数列\(\{f_n\}\)在\(D\)上不一致收敛于\(f\)的充分且必要条件是:存在\(\{x_n\} \subset D\),使得\(\{f_n(x_n) - f(x_n)\}\)不收敛于\(0\).
\end{corollary}

\begin{theorem}[一致收敛的柯西准则]\label{theorem:一致收敛的柯西准则}
函数项级数\(\sum u_n(x)\)在数集\(D\)上一致收敛的充要条件为:对任给的正数\(\varepsilon\),总存在某正整数\(N\),使得当\(n > N\)时,对一切\(x \in D\)和一切正整数\(p\),都有
\[
|S_{n + p}(x) - S_n(x)| < \varepsilon
\]
或
\[
|u_{n + 1}(x) + u_{n + 2}(x) + \cdots + u_{n + p}(x)| < \varepsilon.
\]
\end{theorem}

\begin{corollary}
函数项级数\(\sum u_n(x)\)在数集\(D\)上一致收敛的必要条件是函数列\(\{u_n(x)\}\)在\(D\)上一致收敛于零.
\end{corollary} 

\begin{theorem}
函数项级数\(\sum u_n(x)\)在数集\(D\)上一致收敛于\(S(x)\)的充要条件是
\[
\lim_{n \to \infty} \sup_{x \in D} |R_n(x)| = \lim_{n \to \infty} \sup_{x \in D} |S(x) - S_n(x)| = 0.
\]
\end{theorem}

\begin{theorem}[A-D判别法]\label{theorem:级数一致收敛的A-D判别法}
若\(\sum_{i = 1}^{\infty} a_n(x)b_n(x)\)在定义域内满足下列两条件之一,则其在定义域上一致收敛

(1) \(\{a_n(x)\}\)对于固定的\(x\)关于\(n\)单调,且在定义域内一致有界;\(\sum_{i = 1}^{n} b_n\)一致收敛.(Abel 判别法)

(2) \(\{a_n(x)\}\)对于固定的\(x\)关于\(n\)单调,且一致趋于\(0\);\(\sum_{i = 1}^{n} b_n\)一致有界.(Dirichlet判别法)
\end{theorem}

\begin{theorem}\label{theorem:定理13.8123}
设函数列\(\{f_n\}\)在\((a, x_0) \cup (x_0, b)\)上一致收敛于\(f(x)\),且对每个\(n\),\(\lim_{x \to x_0} f_n(x) = a_n\),则\(\lim_{n \to \infty} a_n\)和\(\lim_{x \to x_0} f(x)\)均存在且相等.
\end{theorem}
\begin{proof}
先证\(\{a_n\}\)是收敛数列.对任意\(\varepsilon > 0\),由于\(\{f_n\}\)一致收敛,故有\(N\),当\(n > N\)时,对任意正整数\(p\)和对一切\(x \in (a, x_0) \cup (x_0, b)\),有
\begin{align}
|f_n(x) - f_{n + p}(x)| < \varepsilon. \label{eq::--1}
\end{align}
从而
\[
|a_n - a_{n + p}| = \lim_{x \to x_0} |f_n(x) - f_{n + p}(x)| \leqslant \varepsilon.
\]
这样由柯西准则可知\(\{a_n\}\)是收敛数列.

设\(\lim_{n \to \infty} a_n = A\).再证\(\lim_{x \to x_0} f(x) = A\).

由于\(f_n(x)\)一致收敛于\(f(x)\)及\(a_n\)收敛于\(A\),因此对任意\(\varepsilon > 0\),存在正数\(N\),当\(n > N\)时,对任意\(x \in (a, x_0) \cup (x_0, b)\),
\[
|f_n(x) - f(x)| < \frac{\varepsilon}{3} \quad \text{和} \quad |a_n - A| < \frac{\varepsilon}{3}
\]
同时成立.特别取\(n = N + 1\),有
\[
|f_{N + 1}(x) - f(x)| < \frac{\varepsilon}{3}, \quad |a_{N + 1} - A| < \frac{\varepsilon}{3}.
\]
又\(\lim_{x \to x_0} f_{N + 1}(x) = a_{N + 1}\),故存在\(\delta > 0\),当\(0 < |x - x_0| < \delta\)时,
\[
|f_{N + 1}(x) - a_{N + 1}| < \frac{\varepsilon}{3}.
\]
这样,当\(x\)满足\(0 < |x - x_0| < \delta\)时,
\begin{align*}
|f(x) - A| &\leqslant |f(x) - f_{N + 1}(x)| + |f_{N + 1}(x) - a_{N + 1}| + |a_{N + 1} - A| \\
&< \frac{\varepsilon}{3} + \frac{\varepsilon}{3} + \frac{\varepsilon}{3} = \varepsilon,
\end{align*}
即\(\lim_{x \to x_0} f(x) = A\).

\end{proof}

\begin{theorem}[连续性]\label{theorem:定理13.98648}
若函数列\(\{f_n\}\)在区间\(I\)上一致收敛,且每一项都连续,则其极限函数\(f\)在\(I\)上也连续.
\end{theorem}
\begin{note}
由这个定理可知,\textbf{若各项为连续函数的函数列在区间\(I\)上其极限函数不连续,则此函数列在区间\(I\)上不一致收敛.}
\end{note}
\begin{proof}
设\(x_0\)为\(I\)上任一点.由于\(\lim_{x \to x_0} f_n(x) = f_n(x_0)\),于是由\refthe{theorem:定理13.8123}知\(\lim_{x \to x_0} f(x)\)亦存在,且\(\lim_{x \to x_0} f(x) = \lim_{n \to \infty} f_n(x_0) = f(x_0)\),因此\(f(x)\)在\(x_0\)上连续.

\end{proof}

\begin{corollary}
若连续函数列\(\{f_n\}\)在区间\(I\)上内闭一致收敛于\(f\),则\(f\)在\(I\)上连续.
\end{corollary}

\begin{theorem}[可积性]\label{theorem:定理13.15648640}
若函数列\(\{f_n\}\)在\([a, b]\)上一致收敛,且每一项都连续,则
\begin{align}
\int_a^b \lim_{n \to \infty} f_n(x) \mathrm{d}x = \lim_{n \to \infty} \int_a^b f_n(x) \mathrm{d}x. \label{eq::--3}
\end{align}
\end{theorem}
\begin{proof}
设\(f\)为函数列\(\{f_n\}\)在\([a, b]\)上的极限函数.由\refthe{theorem:定理13.98648},\(f\)在\([a, b]\)上连续,从而\(f_n \ (n = 1, 2, \cdots)\)与\(f\)在\([a, b]\)上都可积.

因为在\([a, b]\)上\(f_n \rightrightarrows f \ (n \to \infty)\),故对任给正数\(\varepsilon\),存在\(N\),当\(n > N\)时,对一切\(x \in [a, b]\),都有
\[
|f_n(x) - f(x)| < \varepsilon.
\]
再根据定积分的性质,当\(n > N\)时有
\begin{align*}
\left| \int_a^b f_n(x) \mathrm{d}x - \int_a^b f(x) \mathrm{d}x \right| &= \left| \int_a^b (f_n(x) - f(x)) \mathrm{d}x \right| \\
&\leqslant \int_a^b |f_n(x) - f(x)| \mathrm{d}x \\
&\leqslant \varepsilon(b - a).
\end{align*}
这就证明了等式\(\eqref{eq::--3}\).

\end{proof}

\begin{theorem}[可微性]\label{theorem:定理13.1154661}
设\(\{f_n\}\)为定义在\([a, b]\)上的函数列,若\(x_0 \in [a, b]\)为\(\{f_n\}\)的收敛点,\(\{f_n\}\)的每一项在\([a, b]\)上有连续的导数,且\(\{f'_n\}\)在\([a, b]\)上一致收敛,则
\begin{align}
\frac{\mathrm{d}}{\mathrm{d}x} \left( \lim_{n \to \infty} f_n(x) \right) = \lim_{n \to \infty} \frac{\mathrm{d}}{\mathrm{d}x} f_n(x). \label{eq::--4}
\end{align}
\end{theorem}
\begin{proof}
设\(f_n(x_0) \to A \ (n \to \infty)\),\(f'_n \rightrightarrows g \ (n \to \infty)\),\(x \in [a, b]\).我们要证明函数列\(\{f_n\}\)在区间\([a, b]\)上收敛,且其极限函数的导数存在且等于\(g\).

由定理条件,对任一\(x \in [a, b]\),总有
\[
f_n(x) = f_n(x_0) + \int_{x_0}^x f'_n(t) \mathrm{d}t.
\]
当\(n \to \infty\)时,右边第一项极限为\(A\),第二项极限为\(\int_{x_0}^x g(t) \mathrm{d}t\)(\refthe{theorem:定理13.15648640}),所以左边极限存在,记为\(f\),则有
\[
f(x) = \lim_{n \to \infty} f_n(x) = f(x_0) + \int_{x_0}^x g(t) \mathrm{d}t,
\]
其中\(f(x_0) = A\).由\(g\)的连续性及微积分学基本定理推得
\[
f' = g.
\]
这就证明了等式\(\eqref{eq::--4}\).

\end{proof}

\begin{theorem}[连续性]\label{theorem:函数项级数连续性定理}
若函数项级数\(\sum u_n(x)\)在区间\([a, b]\)上一致收敛,且每一项都连续,则其和函数在\([a, b]\)上也连续.
\end{theorem}
\begin{note}
这个定理指出:在一致收敛条件下,(无限项)求和运算与求极限运算可以交换顺序,即
\[
\sum \left( \lim_{x \to x_0} u_n(x) \right) = \lim_{x \to x_0} \left( \sum u_n(x) \right). 
\]
\end{note}

\begin{theorem}[逐项求积]\label{theorem:函数项级数逐项求积定理}
若函数项级数\(\sum u_n(x)\)在\([a, b]\)上一致收敛,且每一项\(u_n(x)\)都连续,则
\[
\sum \int_a^b u_n(x) \mathrm{d}x = \int_a^b \sum u_n(x) \mathrm{d}x. 
\]
\end{theorem}

\begin{theorem}[逐项求导]\label{theorem:级数逐项求导定理}
若函数项级数\(\sum u_n(x)\)在\([a, b]\)上每一项都有连续的导函数,\(x_0 \in [a, b]\)为\(\sum u_n(x)\)的收敛点,且\(\sum u'_n(x)\)在\([a, b]\)上一致收敛,则
\[
\sum \left( \frac{\mathrm{d}}{\mathrm{d}x} u_n(x) \right) = \frac{\mathrm{d}}{\mathrm{d}x} \left( \sum u_n(x) \right). 
\]
\end{theorem}


\begin{example}
判断下列级数在指定区间一致收敛性:
\begin{enumerate}
\item  \(\sum_{n=1}^{\infty} \frac{(-1)^n}{\sqrt[3]{n + \sqrt{x}}}, [0, +\infty)\);

\item \(\sum_{n=1}^{\infty} (-1)^n \frac{n + x^2}{n^2}, [-a, a], a > 0\);

\item \(\sum_{n=1}^{\infty} \left( \frac{x^{n - 1}}{n} - \frac{x^n}{n + 1} \right), [-1, 1]\);

\item \(\sum_{n=1}^{\infty} \frac{x^n}{(1 + x)(1 + x^2) \cdots (1 + x^n)}, [0, +\infty)\);

\item \(\sum_{n=1}^{\infty} \frac{x}{1 + nx^2} \sin \frac{1}{\sqrt{nx}} \arctan \left( \sqrt{\frac{x}{n}} \right), (0, +\infty)\);

\item \(\sum_{n=1}^{\infty} \frac{\sin^2 x}{x + n^3 x^2}, (0, +\infty)\);

\item \(\sum_{n=1}^{\infty} \left( \arctan \frac{x}{n^2 + x^2} \right)^2, [0, +\infty)\).
\end{enumerate}
\end{example}
\begin{remark}
第4问可以通过裂项算出级数的和函数:
\begin{align*}
\sum_{n=1}^{\infty}{\frac{x^n}{(1+x)(1+x^2)\cdots (1+x^n)}}&=\frac{x}{1+x}+\sum_{n=2}^{\infty}{\frac{x^n}{(1+x)(1+x^2)\cdots (1+x^n)}}
\\
&=\frac{x}{1+x}+\sum_{n=2}^{\infty}{\left[ \frac{1}{(1+x)(1+x^2)\cdots (1+x^{n-1})}-\frac{1}{(1+x)(1+x^2)\cdots (1+x^n)} \right]}
\\
&=\frac{x}{1+x}+\frac{1}{1+x}-\underset{n\rightarrow \infty}{\lim}\frac{1}{(1+x)(1+x^2)\cdots (1+x^n)}
\\
&=1-\underset{n\rightarrow \infty}{\lim}\frac{1}{(1+x)(1+x^2)\cdots (1+x^n)}=1-\prod_{n=1}^{\infty}{\frac{1}{1+x^n}}.
\end{align*}
但$\prod_{n=1}^{\infty}{\frac{1}{1+x^n}}$的一致收敛性不好判断(这个方法比较复杂),因此我们不采取这个方法.
\end{remark}
\begin{proof}
\begin{enumerate}
\item 显然\(\left| \sum_{j=1}^{n} (-1)^j \right| \leqslant 2\)以及对每一个\(x \in [0, +\infty)\)都有\(\frac{1}{\sqrt[3]{n + \sqrt{x}}}\)单调递减. 又
\[
\frac{1}{\sqrt[3]{n + \sqrt{x}}} \leqslant \frac{1}{\sqrt[3]{n}} \to 0, n \to \infty,
\]
我们由\hyperref[theorem:级数一致收敛的A-D判别法]{一致收敛的A-D判别法}有\(\sum_{n=1}^{\infty} \frac{(-1)^n}{\sqrt[3]{n + \sqrt{x}}}\)在\([0, +\infty)\)一致收敛.

\item 显然\(\left| \sum_{j=1}^{n} (-1)^j \right| \leqslant 2\)以及对每一个\(x \in [-a, a]\)都有\(\frac{n + x^2}{n^2}\)单调递减. 又
\[
\frac{n + x^2}{n^2} \leqslant \frac{n + a^2}{n^2} \to 0, n \to \infty,
\]
我们由\hyperref[theorem:级数一致收敛的A-D判别法]{一致收敛的A-D判别法}有\(\sum_{n=1}^{\infty} (-1)^n \frac{n + x^2}{n^2}\)在\([-a, a]\)一致收敛.

\item 注意到
\[
\lim_{m \to \infty} \sup_{x \in [-1, 1]} \left| \sum_{n=m}^{\infty} \left( \frac{x^{n - 1}}{n} - \frac{x^n}{n + 1} \right) \right| = \lim_{m \to \infty} \sup_{x \in [-1, 1]} \frac{x^m}{m + 1} = \lim_{m \to \infty} \frac{1}{m + 1} = 0,
\]
我们有\(\sum_{n=1}^{\infty} \left( \frac{x^{n - 1}}{n} - \frac{x^n}{n + 1} \right)\)在\([-1, 1]\)一致收敛.

\item 一方面, 对\(x \in [1, +\infty), n \in \mathbb{N}\), 我们有
\[
\frac{x^n}{(1 + x)(1 + x^2) \cdots (1 + x^{n - 1})(1 + x^n)} \leqslant \frac{1}{2^{n - 1}}, \forall x \in [1, +\infty).
\]
另外一方面, 对\(n \geqslant 2, x \in [0, 1)\), 我们有
\[
\frac{x^{2n + 1}}{(1 + x)(1 + x^2) \cdots (1 + x^{2n + 1})} \leqslant \frac{x^{2n}}{(1 + x)(1 + x^2) \cdots (1 + x^{2n})}
\]
\[
\leqslant \frac{x^{2n}}{\underbrace{(1 + x^n)(1 + x^n) \cdots (1 + x^n)}_{n \text{个}}} \leqslant \frac{x^{2n}}{C_n^2 x^{2n}} = \frac{2}{n(n - 1)}.
\]
即由Weierstrass判别法和
\[
\sum_{n=1}^{\infty} \frac{1}{2^{n - 1}} < \infty, \sum_{n=2}^{\infty} \frac{2}{n(n - 1)} < \infty,
\]
我们知道原级数一致收敛.

\item 首先
\[
\left| \sin \frac{1}{\sqrt{nx}} \arctan \left( \sqrt{\frac{x}{n}} \right) \right| \leqslant \sqrt{\frac{x}{n}}, \forall x > 0, n \in \mathbb{N}.
\]
然后
\[
\left( \frac{\sqrt{x}}{n} \frac{x}{1 + nx^2} \right)' = \frac{\sqrt{x}(3 - nx^2)}{2n(1 + x^2 n)^2} \Rightarrow \frac{\sqrt{x}}{n} \frac{x}{1 + nx^2} \leqslant \frac{\sqrt{x}}{n} \frac{x}{1 + nx^2} \bigg|_{x = \sqrt{\frac{3}{n}}} = \frac{3^{\frac{3}{4}}}{4} \left( \frac{1}{n} \right)^{\frac{7}{4}}.
\]
于是
\begin{align*}
\sum_{n=1}^{\infty}{\frac{x}{1+nx^2}\sin \frac{1}{\sqrt{nx}}\arctan  \left( \sqrt{\frac{x}{n}} \right)}\leqslant \sum_{n=1}^{\infty}{\sqrt{\frac{x}{n}}\frac{x}{1+nx^2}}\leqslant \sum_{n=1}^{\infty}{\frac{3^{\frac{3}{4}}}{4}\left( \frac{1}{n} \right) ^{\frac{7}{4}}}<+\infty.
\end{align*}
这就证明了\(\sum_{n=1}^{\infty} \frac{x}{1 + nx^2} \sin \frac{1}{\sqrt{nx}} \arctan \left( \sqrt{\frac{x}{n}} \right)\)在\((0, +\infty)\)一致收敛.

\item 我们有
\[
\sum_{n=1}^{\infty} \frac{\sin^2 x}{x + n^3 x^2} \leqslant \sum_{n=1}^{\infty} \frac{\sin^2 x}{2n^{\frac{3}{2}} x^{\frac{3}{2}}} \leqslant \sup_{x \in (0, +\infty)} \frac{\sin^2 x}{2x^{\frac{3}{2}}} \cdot \sum_{n=1}^{\infty} \frac{1}{n^{\frac{3}{2}}} < \infty,
\]
即\(\sum_{n=1}^{\infty} \frac{\sin^2 x}{x + n^3 x^2}\)在\((0, +\infty)\)一致收敛.

\item 因为
\[
\left( \frac{x}{n^2 + x^2} \right)' = \frac{n^2 - x^2}{(x^2 + n^2)^2},
\]
于是我们有
\[
\sum_{n=1}^{\infty} \left( \arctan \frac{x}{n^2 + x^2} \right)^2 \leqslant \sum_{n=1}^{\infty} \left( \frac{x}{n^2 + x^2} \right)^2 \leqslant \sum_{n=1}^{\infty} \left( \frac{n}{n^2 + n^2} \right)^2 < \infty,
\]
即\(\sum_{n=1}^{\infty} \left( \arctan \frac{x}{n^2 + x^2} \right)^2\)在\([0, +\infty)\)一致收敛.
\end{enumerate}

\end{proof}

\begin{example}
证明
\begin{align*}
\int_{e^3}^{\infty} \frac{\sin(xy)}{\ln \ln y} \, dy
\end{align*}
在$x \in (0, +\infty)$非一致收敛.
\end{example}
\begin{proof}
注意到对$\forall A= \frac{n}{4}>e^3$,有$\frac{\pi}{2}n>\frac{\pi}{4}n>A$,再取$x=\frac{1}{n}$,则
\begin{align*}
\left| \int_{\frac{\pi}{4}n}^{\frac{\pi}{2}n} \frac{\sin \left( \frac{1}{n}y \right)}{\ln\ln y} \,\mathrm{d}y \right| \geqslant \int_{\frac{\pi}{4}n}^{\frac{\pi}{2}n} \frac{\sin \frac{\pi}{4}}{\ln\ln \left( \frac{\pi}{2}n \right)} \,\mathrm{d}y = \frac{\pi n}{4\sqrt{2}\ln\ln \left( \frac{\pi}{2}n \right)} \rightarrow +\infty,\,n\rightarrow \infty.
\end{align*}
故由不一致收敛的Cauchy收敛准则知,$\int_{e^3}^{+\infty} \frac{\sin \left( xy \right)}{\ln\ln y} \,\mathrm{d}y$在$\left( 0,+\infty \right)$上不一致收敛.

\end{proof}

\begin{example}
级数\(\sum_{n=1}^{\infty} \frac{\sin(nx)}{n}\)在\((0, \frac{\pi}{2})\)是否一致收敛.
\end{example}
\begin{note}
连续函数列\(\{f_n\}\)在区间\(I\)一致收敛,则在\(\overline{I}\)也一致收敛,这是因为有等式
\[
\sup_{x \in I} |f_n(x) - f_m(x)| = \sup_{x \in \overline{I}} |f_n(x) - f_m(x)|.
\]
我们可以猜测级数值
\begin{align*}
\sum_{n=1}^{\infty} \frac{\sin(nx)}{n} &= \sum_{n=1}^{\infty} \Im \left( \frac{e^{inx}}{n} \right) = \Im \sum_{n=1}^{\infty} \frac{e^{inx}}{n} = \Im (-\ln(1 - e^{ix})) = -\arg(1 - e^{ix}) \\
&= -\mathrm{arg(}1-\cos x-i\sin x)=-\arctan  \frac{-\sin x}{1-\cos x}=\arctan  \frac{2\sin \frac{x}{2}\cos \frac{x}{2}}{2\sin ^2\frac{x}{2}}\\
&= \arctan  \frac{1}{\tan \frac{x}{2}}=\frac{\pi}{2}-\arctan \tan \frac{x}{2}=\frac{\pi -x}{2},
\end{align*}
然后对\(\frac{\pi - x}{2}, x \in (0, \pi)\)做奇延拓之后在\([-\pi, \pi]\)展开为傅立叶级数,从而得到
\[
\sum_{n=1}^{\infty} \frac{\sin(nx)}{n} = \frac{\pi - x}{2}, x \in (0, \pi).
\]
这个级数结果应当记忆.注意到上述和函数与$x$有关,故原级数一定不一致收敛,下面将证明严格化.
\end{note}
\begin{proof}
对\(\frac{\pi - x}{2}, x \in (0, \pi)\)做奇延拓之后在\([-\pi, \pi]\)展开为傅立叶级数,得到
\[
\sum_{n=1}^{\infty} \frac{\sin(nx)}{n} = \frac{\pi - x}{2}, x \in (0, \pi).
\]
若\(\sum_{n=1}^{\infty} \frac{\sin(nx)}{n}\)在\((0, \frac{\pi}{2})\)一致收敛,则在\([0, \frac{\pi}{2})\)也一致收敛. 但是
\[
\sum_{n=1}^{\infty} \frac{\sin(n \cdot 0)}{n} =0 \neq \lim_{x \to 0^+} \sum_{n=1}^{\infty} \frac{\sin(nx)}{n} = \lim_{x \to 0^+} \left( \frac{\pi - x}{2} \right) = \frac{\pi}{2},
\]
这就和\(\sum_{n=1}^{\infty} \frac{\sin(nx)}{n}\)在\(x = 0\)应该连续矛盾!因此\(\sum_{n=1}^{\infty} \frac{\sin(nx)}{n}\)在\((0, \frac{\pi}{2})\)不一致收敛.

\end{proof}

\begin{example}
设\(f \in C^1(\mathbb{R})\),令
\[
f_n(x) = n \left[ f \left( x + \frac{1}{n} \right) - f(x) \right], n = 1, 2, \cdots.
\]
试证明对任给区间\([a, b]\)都有\(f_n(x)\)一致收敛到\(f'(x)\).
\end{example}
\begin{proof}
由\hyperref[theorem:Cantor定理]{Cantor定理}及\(f \in C^1(\mathbb{R})\)可知\(f'\)内闭一致连续性,于是对任何\(\varepsilon > 0\),存在\(\delta > 0\)使得当\(x \in [a, b]\),\(t \in [0, \delta]\),我们有
\[
|f'(x + t) - f'(x)| \leqslant \varepsilon.
\]
当\(n > \frac{1}{\delta}\),我们对任何\(x \in [a, b]\)有
\begin{align*}
&|f_n(x) - f'(x)| = n \left| \int_{x}^{x + \frac{1}{n}} f'(y) - f'(x) \mathrm{d}y \right| \\
&\leqslant n \int_{x}^{x + \frac{1}{n}} |f'(y) - f'(x)| \mathrm{d}y = n \int_{0}^{\frac{1}{n}} |f'(x + t) - f'(x)| \mathrm{d}t \\
&\leqslant \varepsilon n \int_{0}^{\frac{1}{n}} 1 \mathrm{d}t = \varepsilon,
\end{align*}
这就证明了\(f_n(x)\)在\([a, b]\)一致收敛到\(f'(x)\).

\end{proof}

\begin{example}
讨论下列函数在给定区间可微性.
\begin{enumerate}
\item \(\sum_{n=1}^{\infty} e^{-n^2 \pi x}, (0, +\infty)\);

\item \(\sum_{n=1}^{\infty} \frac{\sin(nx)}{n^3}, (-\infty, +\infty)\);

\item \(\sum_{n=1}^{\infty} \frac{(-1)^{n+1}}{\sqrt{n}} \arctan \left( \frac{x}{\sqrt{n}} \right), (-\infty, +\infty)\);

\item \(\sum_{n=1}^{\infty} \sqrt{n} \tan^n x, \left( -\frac{\pi}{4}, \frac{\pi}{4} \right)\).
\end{enumerate}
\end{example}
\begin{proof}
\begin{enumerate}
\item \(\sum_{n=1}^{\infty} e^{-n^2 \pi x}\)显然收敛. 考虑逐项微分级数
\[
\sum_{n=1}^{\infty} \left( e^{-n^2 \pi x} \right)' = \sum_{n=1}^{\infty} -n^2 \pi e^{-n^2 \pi x}.
\]
对任何\([a, b] \subset (0, +\infty)\),我们有
\[
\sum_{n=1}^{\infty} \left| -n^2 \pi e^{-n^2 \pi x} \right| \leqslant \sum_{n=1}^{\infty} n^2 \pi e^{-n^2 \pi a} < \infty,
\]
即内闭一致收敛,因此由\refthe{theorem:级数逐项求导定理}可知\(\sum_{n=1}^{\infty} e^{-n^2 \pi x}\)在\((0, +\infty)\)可微.

\item  注意到
\[
\sum_{n=1}^{\infty} \left| \frac{\sin(nx)}{n^3} \right| \leqslant \sum_{n=1}^{\infty} \frac{1}{n^3} < \infty, \sum_{n=1}^{\infty} \left| \frac{\cos(nx)}{n^2} \right| \leqslant \sum_{n=1}^{\infty} \frac{1}{n^2} < \infty,
\]
于是我们有原级数和逐项微分级数一致收敛,因此\(\sum_{n=1}^{\infty} \frac{\sin(nx)}{n^3}\)在\((-\infty, +\infty)\)可微.

\item 显然
\[
\frac{1}{\sqrt{n}} \arctan \left( \frac{x}{\sqrt{n}} \right) \rightrightarrows 0, \forall x \in \mathbb{R},
\]
于是由莱布尼兹判别法知原级数收敛. 考虑逐项微分级数
\[
\sum_{n=1}^{\infty} \left( \frac{(-1)^{n+1}}{\sqrt{n}} \arctan \left( \frac{x}{\sqrt{n}} \right) \right)' = \sum_{n=1}^{\infty} \frac{(-1)^{n+1}}{n} \frac{1}{1 + \frac{x^2}{n}} = \sum_{n=1}^{\infty} \frac{(-1)^{n+1}}{n + x^2}.
\]
注意到
\[
\left| \sum_{j=1}^{n} (-1)^{j+1} \right| \leqslant 1, \left| \frac{1}{n + x^2} \right| \leqslant 1, \forall x \in \mathbb{R}, n \in \mathbb{N},
\]
以及对任何\(x \in \mathbb{R}\)都有\(\frac{1}{n + x^2}\)递减,因此由一致收敛的 A-D 判别法我们知道逐项微分级数一致收敛. 因此\(\sum_{n=1}^{\infty} \frac{(-1)^{n+1}}{\sqrt{n}} \arctan \left( \frac{x}{\sqrt{n}} \right)\)在\((-\infty, +\infty)\)可微.

\item 显然\(\sum_{n=1}^{\infty} \sqrt{n} \tan^n x\)在\(\left( -\frac{\pi}{4}, \frac{\pi}{4} \right)\)收敛. 因为可微性是局部的概念,我们来证明逐项微分级数
\[
\sum_{n=1}^{\infty} n \sqrt{n} \tan^{n - 1} x \left( \tan^2 x + 1 \right)
\]
在任何\([a, b] \subset \left( -\frac{\pi}{4}, \frac{\pi}{4} \right)\)一致收敛.

显然存在\(c_{a,b} \in (0, 1)\)使得
\[
|\tan x| \leqslant c_{a,b}, \forall x \in [a, b].
\]
于是我们知道
\[
\sum_{n=1}^{\infty} \left| n \sqrt{n} \tan^{n - 1} x \left( \tan^2 x + 1 \right) \right| \leqslant 2 \sum_{n=1}^{\infty} n \sqrt{n} c_{a,b}^{n - 1} < \infty.
\]
因此逐项微分级数
\[
\sum_{n=1}^{\infty} n \sqrt{n} \tan^{n - 1} x \left( \tan^2 x + 1 \right)
\]
在\([a, b]\)一致收敛,从而\(\sum_{n=1}^{\infty} \sqrt{n} \tan^n x\)在\(\left( -\frac{\pi}{4}, \frac{\pi}{4} \right)\)可微.
\end{enumerate}

\end{proof}

\begin{example}
判断$\sum\limits_{n=1}^{\infty}\frac{nx}{(1+x)(1+2x)\cdots(1+nx)}$在$[0,\lambda],\lambda>0$的一致收敛性.
\end{example}
\begin{proof}
注意到
$$
\frac{nx}{(1+x)(1+2x)\cdots(1+nx)}=\left[\frac{1}{(1+x)(1+2x)\cdots(1+(n-1)x)}-\frac{1}{(1+x)(1+2x)\cdots(1+nx)}\right],n=2,3,\cdots.
$$
于是我们有
$$
\sum_{n=2}^{m}\frac{nx}{(1+x)(1+2x)\cdots(1+nx)}=\frac{1}{1+x}-\frac{1}{(1+x)(1+2x)\cdots(1+mx)}.
$$
现在
\begin{align*}
\sum_{n=1}^{\infty}{\frac{nx}{(1+x)(1+2x)\cdots (1+nx)}}&=\frac{x}{1+x}+\sum_{n=2}^{\infty}{\frac{nx}{(1+x)(1+2x)\cdots (1+nx)}}
\\
&=\begin{cases}
\frac{x}{1+x}+\frac{1}{1+x},&		x>0\\
0,&		x=0\\
\end{cases}.
\end{align*}
于是由级数和函数不连续知其在$[0,\lambda],\lambda>0$不一致收敛.

\end{proof}

\begin{example}
判断$\sum\limits_{n=1}^{\infty}\frac{1}{n}\left[e^x-\left(1+\frac{x}{n}\right)^n\right]$在$[0,b]$和$[0,+\infty)$的一致收敛性.
\end{example}
\begin{proof}
首先注意到
$$
\left[e^x-\left(1+\frac{x}{n}\right)^n\right]'=e^x-\left(1+\frac{x}{n}\right)^{n-1}\geqslant0,
$$
我们有
$$
e^x-\left(1+\frac{x}{n}\right)^n\leqslant e^b-\left(1+\frac{b}{n}\right)^n.
$$
由 Taylor 公式得
\begin{align*}
e^b-\left( 1+\frac{b}{n} \right) ^n&=e^b\left[ 1-e^{n\ln \left( 1+\frac{b}{n} \right) -b} \right] =e^b\left[ 1-e^{n\left[ \frac{b}{n}+O\left( \frac{1}{n^2} \right) \right] -b} \right] 
\\
&=e^b\left[ 1-e^{O\left( \frac{1}{n} \right)} \right] =O\left( \frac{1}{n} \right) ,n\rightarrow \infty ,
\end{align*}
(实际上我们可以写出具体的等价量$e^b-\left( 1+\frac{b}{n} \right) ^n\sim \frac{e^bb^2}{2n},n\rightarrow \infty$)于是我们有
$$
\sum_{n=1}^{\infty}\frac{1}{n}\left[e^b-\left(1+\frac{b}{n}\right)^n\right]<\infty.
$$
故$\sum\limits_{n=1}^{\infty}\frac{1}{n}\left[e^x-\left(1+\frac{x}{n}\right)^n\right]$在$[0,b]$一致收敛. 注意到
$$
\sup_{x\in[0,+\infty)}\left|\frac{1}{n}\left[e^x-\left(1+\frac{x}{n}\right)^n\right]\right|=+\infty,
$$
我们知道$\sum\limits_{n=1}^{\infty}\frac{1}{n}\left[e^x-\left(1+\frac{x}{n}\right)^n\right]$在$[0,+\infty)$不一致收敛.

\end{proof}

\begin{example}
对$\alpha>0$, 判断$\sum\limits_{n=1}^{\infty}x^\alpha e^{-nx}$在$[0,+\infty)$一致收敛性.
\end{example}
\begin{proof}
注意到
$$
\left( x^{\alpha}e^{-nx} \right) '=(\alpha -nx)x^{\alpha -1}e^{-nx}=0\Rightarrow x=\frac{\alpha}{n}.
$$
我们有
$$
x^\alpha e^{-nx}\leqslant\left(\frac{\alpha}{n}\right)^\alpha e^{-\alpha}.
$$
当$\alpha>1$, 我们由$\sum\limits_{n=1}^{\infty}\frac{1}{n^\alpha}<\infty$知原级数在$[0,+\infty)$一致收敛.

当$\alpha\in[0,1)$, 注意到
$$
\sum_{n=1}^{\infty}x^\alpha e^{-nx}=
\begin{cases}
\frac{x^\alpha}{e^x-1},&x>0\\
0,&x=0
\end{cases}.
$$
如果原级数在$[0,+\infty)$一致收敛, 那么上述和函数在$x=0$应该连续. 但是
$$
\lim_{x\to0^+}\frac{x^\alpha}{e^x-1}\neq0,
$$
故原级数在$[0,+\infty)$不一致收敛.

\end{proof}

\begin{example}
求$\alpha$的范围, 使得$\sum\limits_{n=1}^{\infty}\left(x-\frac{1}{n}\right)^n(1-x)^\alpha$在$x\in[0,1]$一致收敛.
\end{example}
\begin{note}
我们只需保证
$$
\sum_{n=1}^{\infty}\sup_{x\in[0,1]}\left(x-\frac{1}{n}\right)^n(1-x)^\alpha
$$
收敛. 虽然一般情况这并不能说明
$$
\sum_{n=1}^{\infty}\sup_{x\in[0,1]}\left(x-\frac{1}{n}\right)^n(1-x)^\alpha=+\infty
$$
时一定有$\sum\limits_{n=1}^{\infty}\left(x-\frac{1}{n}\right)^n(1-x)^\alpha$在$x\in[0,1]$不一致收敛, 但是对具体例子, 我们通过对$x$赋值往往能实现这一点.
\end{note}
\begin{proof}
当$\alpha>1$, 首先由
$$
\left[\left(x-\frac{1}{n}\right)^n(1-x)^\alpha\right]'=\left(x-\frac{1}{n}\right)^{n-1}(1-x)^{\alpha-1}\left[n+\frac{\alpha}{n}-(n+\alpha)x\right]=0\Rightarrow x=\frac{n+\frac{\alpha}{n}}{n+\alpha}\text{为函数列通项最大值点}.
$$
知
\begin{align*}
\mathop {\mathrm{sup}} \limits_{x\in [0,1]}\left( x-\frac{1}{n} \right) ^n(1-x)^{\alpha}&=\left( \frac{n+\frac{\alpha}{n}}{n+\alpha}-\frac{1}{n} \right) ^n\left( 1-\frac{n+\frac{\alpha}{n}}{n+\alpha} \right) ^{\alpha}=\left( \frac{n^3-n^2}{n^3+\alpha n^2} \right) ^n\left( \frac{\alpha n-\alpha}{n^2+\alpha n} \right) ^{\alpha}
\\
&\sim \frac{\alpha ^{\alpha}}{n^{\alpha}}\left( \frac{n^3-n^2}{n^3+\alpha n^2} \right) ^n=\frac{\alpha ^{\alpha}}{n^{\alpha}}\left( 1-\frac{\left( 1+\alpha \right) n^2}{n^3+\alpha n^2} \right) ^n
\\
&\sim \frac{\alpha ^{\alpha}}{n^{\alpha}}e^{-\frac{\left( 1+\alpha \right) n^3}{n^3+\alpha n^2}}\sim \frac{e^{-(1+\alpha)}\alpha ^{\alpha}}{n^{\alpha}},n\rightarrow \infty .
\end{align*}
故由Weierstrass判别法可知$\sum\limits_{n=1}^{\infty}\left(x-\frac{1}{n}\right)^n(1-x)^\alpha$在$x\in[0,1]$一致收敛.

当$\alpha\leqslant0$, 原级数在$x=1$时通项极限不等于0,故此时级数在$x=1$时发散.

当$0<\alpha\leqslant1$, 当$N\to+\infty$,取$x=N+\frac{\alpha}{N}$,我们有
$$
\begin{aligned}
\sum_{n=N}^{2N-1}{\left( \frac{N+\frac{\alpha}{N}}{N+\alpha}-\frac{1}{n} \right) ^n\left( 1-\frac{N+\frac{\alpha}{N}}{N+\alpha} \right) ^{\alpha}}&\geqslant \sum_{n=N}^{2N-1}{\left( \frac{N+\frac{\alpha}{N}}{N+\alpha}-\frac{1}{N} \right) ^n\left( 1-\frac{N+\frac{\alpha}{N}}{N+\alpha} \right) ^{\alpha}}
=\sum_{n=N}^{2N-1}{\left( \frac{N^2-N}{(N+\alpha )N} \right) ^n\left( 1-\frac{N+\frac{\alpha}{N}}{N+\alpha} \right) ^{\alpha}}
\\
&\geqslant N\left( \frac{N^2-N}{(N+\alpha )N} \right) ^{2N-1}\left( 1-\frac{N+\frac{\alpha}{N}}{N+\alpha} \right) ^{\alpha}
\geqslant N\left( \frac{N^2-N}{(N+\alpha )N} \right) ^{2N-1}\left( \frac{\alpha -\frac{\alpha}{N}}{N+\alpha} \right) ^{\alpha}
\\
&\sim Ne^{-\frac{\left( 1+\alpha \right) \left( 2N-1 \right)}{N+\alpha}}\cdot \frac{\alpha ^{\alpha}}{N^{\alpha}}\sim \frac{\alpha ^{\alpha}e^{-2\left( 1+\alpha \right)}}{N^{\alpha -1}}\rightarrow +\infty ,
\end{aligned}
$$
即$\sum\limits_{n=1}^{\infty}\left(x-\frac{1}{n}\right)^n(1-x)^\alpha$在$x\in[0,1]$不一致收敛.

\end{proof}

\begin{example}
设$f_1\in C[a,b],x_0\in[a,b]$. 考虑函数列
$$
f_{n+1}(x)=\int_{x_0}^{x}f_n(t)\mathrm{d}t,n=1,2,\cdots.
$$
讨论$\{f_n\}$在$[a,b]$一致收敛性.
\end{example}
\begin{note}
注意到
\begin{gather*}
\left| f_1\left( x \right) \right|\leqslant M\Rightarrow \left| f_2\left( x \right) \right|\leqslant \int_{x_0}^x{M\mathrm{d}x}=M\left| x-x_0 \right|
\\
\Rightarrow \left| f_3\left( x \right) \right|\leqslant \int_{x_0}^x{M\left| x-x_0 \right|\mathrm{d}x}=\frac{M}{2}\left| x-x_0 \right|^2\Rightarrow \cdots \cdots 
\end{gather*}
于是就有下述归纳.
\end{note}
\begin{remark}
要注意积分上下限大小问题. 如果积分上限小于下限, 则绝对值不等式要反一下上下限使得上限大于下限,因此我们放缩时在积分号外面再加了一个绝对值(当然也可以分类讨论).
\end{remark}
\begin{proof}
设$M\triangleq\sup\limits_{x\in[a,b]}|f_1(x)|$, 我们归纳证明
\begin{align}
|f_n(x)|\leqslant\frac{M}{(n-1)!}|x-x_0|^{n-1}\label{eq::----1564156418691}.
\end{align}
现在\eqref{eq::----1564156418691}对$n=1$已经成立. 假设$n$时成立, 我们对$x_0\in [a,b]$有
$$
|f_{n+1}(x)|=\left| \int_{x_0}^x{f_n(t)\mathrm{d}t} \right|\leqslant \left| \int_{x_0}^x{\left| f_n(t) \right|\mathrm{d}t} \right|\leqslant \frac{M}{(n-1)!}\int_{x_0}^x{|x}-x_0|^{n-1}\mathrm{d}x=\frac{M}{n!}|x-x_0|^n.
$$
现在由数学归纳法知对一切$n\in\mathbb{N}$都有\eqref{eq::----1564156418691}成立, 故
$$
|f_n(x)|\leqslant\frac{M}{(n-1)!}|x-x_0|^{n-1}\leqslant\frac{M}{(n-1)!}\max\left\{|b-x_0|^{n-1},|x_0-a|^{n-1}\right\},
$$
即$\{f_n\}$在$[a,b]$一致收敛到$0$.

\end{proof}

\begin{example}
设$f,f_1 \in C[a,b]$满足
\begin{align*}
f_{n+1}(x) = f(x) + \int_a^x \sin f_n(t)\mathrm{d}t, n = 1,2,\cdots.
\end{align*}
证明:$\{f_n\}$在$[a,b]$一致收敛.
\end{example}
\begin{proof}
首先对$x \in [a,b]$,$n = 2,3,\cdots$,由Lagrange中值定理得
\begin{align}
&|f_{n+1}(x) - f_n(x)| \leqslant \int_a^x |\sin f_n(t) - \sin f_{n-1}(t)| \mathrm{d}t \nonumber \\
&= \int_a^x |\cos \theta| \cdot |f_n(t) - f_{n-1}(t)| \mathrm{d}t \leqslant \int_a^x |f_n(t) - f_{n-1}(t)| \mathrm{d}t.\label{eq:::r9023r304r34t}
\end{align}
然后记
\begin{align*}
M \triangleq \sup_{x \in [a,b]} |f_2(x) - f_1(x)|.
\end{align*}
若
\begin{align*}
|f_n(x) - f_{n-1}(x)| \leqslant \frac{M(x - a)^{n-1}}{(n - 1)!},
\end{align*}
则由\eqref{eq:::r9023r304r34t}式可得
\begin{align*}
|f_{n+1}(x)-f_n(x)|\leqslant |f_n(x)-f_{n-1}(x)|=\int_a^x{\frac{M(t-a)^{n-1}}{(n-1)!}\mathrm{d}t}=\frac{M(x-a)^n}{n!}.
\end{align*}
由数学归纳法可得
\begin{align*}
|f_{n+1}(x) - f_n(x)| \leqslant \frac{M(x - a)^n}{n!}, n = 1,2,\cdots.
\end{align*}
现在
\begin{align*}
\left| f_n\left( x \right) \right|\leqslant \left| f_1\left( x \right) \right|+\sum_{n=1}^{\infty}{|f_{n+1}(x)}-f_n(x)|\leqslant M+\sum_{n=1}^{\infty}{\frac{M(b-a)^n}{n!}}<\infty ,
\end{align*}
即$\{f_n\}$在$[a,b]$一致收敛.

\end{proof}

\begin{example}
设\( [0,1] \)上的连续函数列\( \{f_n\} \)满足\( f_n \geqslant 0 \),且
\[
f_{n+1}(x) = \int_0^x \frac{\mathrm{d}t}{1 + f_n(t)}, \forall n \in \mathbb{N}.
\]
证明:\( \{f_n\} \)在\( [0,1] \)上一致收敛,并求出极限函数
\end{example}
\begin{note}
极限函数的求法:设\( \{f_n\} \)的极限函数为\( f \),则对条件等式两边同时求导并令\( n\rightarrow \infty \)得
\[
f^\prime(x) = \frac{1}{1 + f(x)} \Longrightarrow f(x) = \sqrt{2x + 1} - 1.
\]
\end{note}
\begin{proof}
记\( f(x) = \sqrt{2x + 1} - 1 \),则
\[
f(x) = \int_0^x{\frac{\mathrm{d}t}{1 + f(t)}}.
\]
由\( f,f_1 \in C[0,1] \)知,存在\( M > 0 \),使得
\[
|f_1(x) - f(x)| \leqslant M,\quad \forall x \in [0,1].
\]
假设\( |f_n(x) - f(x)| \leqslant \frac{Mx^{n - 1}}{n!},\forall x \in [0,1] \). 则
\begin{align*}
&|f_{n + 1}(x) - f(x)| \leqslant \int_0^x{\left| \frac{1}{1 + f_n(t)} - \frac{1}{1 + f(t)} \right|}\mathrm{d}t \\
&= \int_0^x{\frac{|f_n(t) - f(t)|}{(1 + f_n(t))(1 + f(t))}\mathrm{d}t} \leqslant \int_0^x{|f_n(t) - f(t)|\mathrm{d}t} \\
&\leqslant \int_0^x{\frac{Mt^{n - 1}}{n!}\mathrm{d}t} = \frac{Mx^n}{(n + 1)!}.
\end{align*}
故由数学归纳法知
\[
|f_n(x) - f(x)| \leqslant \frac{Mx^{n - 1}}{n!} \leqslant \frac{M}{n!},\quad \forall x \in [0,1].
\]
因此
\[
\lim_{n \to \infty} \sup_{x \in [0,1]} |f_n(x) - f(x)| \leqslant \lim_{n \to \infty} \frac{M}{n!} = 0.
\]
故\( \{f_n\} \)在\( [0,1] \)上一致收敛到\( f(x) = \sqrt{2x + 1} - 1 \).

\end{proof}

\begin{example}
\begin{enumerate}
\item 设$\sum\limits_{n = 1}^{\infty}a_{n}$是收敛的正项级数,令$f(x)=\sum\limits_{n = 1}^{\infty}a_{n}|\sin nx|$,已知$f(x)$在$(-\infty,+\infty)$上满足Lipschitz条件,即存在常数$L>0$,使对任意实数$x,y$,都有$|f(x) - f(y)|\leqslant L|x - y|$,证明级数$\sum\limits_{n = 1}^{\infty}na_{n}$收敛.

\item 设$b_{n}\geqslant0$,且$\phi(x)=\sum\limits_{n = 1}^{+\infty}b_{n}\sin nx$在$\mathbb{R}$上一致收敛,证明$\phi(x)\in C^{1}(\mathbb{R})$的充要条件为$\sum\limits_{n = 1}^{+\infty}nb_{n}$收敛.
\end{enumerate}
\end{example}
\begin{remark}
第2问中不能与第1问一样直接得到
\begin{align*}
\frac{\phi \left( x \right)}{x}=\frac{1}{x}\sum_{n=1}^{\infty}{b_n\sin nx}\geqslant \frac{1}{x}\sum_{n=1}^m{b_n\sin nx},
\end{align*}
这个放缩是错误的.因为$\sum_{n=1}^{\infty}{b_n\sin nx}$不是正项级数,第1问中有绝对值,从而是正项级数才能这样放缩.
\end{remark}
\begin{proof}
\begin{enumerate}
\item 注意到$\frac{\sin x}{x}$在$\left( 0,1 \right)$上单调递减,故对$\forall m\in \mathbb{N}$,有
\begin{align*}
\frac{f\left( \frac{1}{m} \right)}{\frac{1}{m}}&=\sum_{n=1}^{\infty}{a_n\frac{\left| \sin \frac{n}{m} \right|}{\frac{1}{m}}}\geqslant \sum_{n=1}^m{na_n\frac{\left| \sin \frac{n}{m} \right|}{\frac{n}{m}}}
\\
&=\sum_{n=1}^m{na_n\frac{\sin \frac{n}{m}}{\frac{n}{m}}}\geqslant \sum_{n=1}^m{na_n\frac{\sin 1}{1}}
\\
&=\sin 1\sum_{n=1}^m{na_n.}
\end{align*}
于是
\begin{align*}
\sum_{n=1}^m{na_n}&\leqslant \frac{1}{\sin 1}m f\left( \frac{1}{m} \right),\quad \forall m\in \mathbb{N}.
\end{align*}
因为$f$在$\mathbb{R}$上满足Lipschitz条件且$f\left( 0 \right) =0$,所以
\begin{align*}
\sum_{n=1}^m{na_n}&\leqslant \frac{1}{\sin 1}m \cdot L \cdot \frac{1}{m}=\frac{L}{\sin 1},\quad \forall m\in \mathbb{N}.
\end{align*}
令$m\rightarrow +\infty$得
\begin{align*}
\sum_{n=1}^{+\infty}{na_n}&\leqslant \frac{L}{\sin 1}<+\infty.
\end{align*}

\item 由条件和\hyperref[theorem:函数项级数连续性定理]{函数项级数连续性定理}可知$\phi \left( x \right) \in C\left( \mathbb{R} \right)$.

{\heiti 充分性:}设$\sum_{n=1}^{+\infty}{nb_n}$收敛,则
\begin{align*}
\sum_{n=1}^{+\infty}{nb_n\cos nx}\leqslant \sum_{n=1}^{+\infty}{nb_n}<+\infty.
\end{align*}
故由\hyperref[theorem:函数项级数逐项求导定理]{函数项级数逐项求导定理}知
\begin{align*}
\phi'\left( x \right) =\sum_{n=1}^{+\infty}{\frac{\mathrm{d}}{\mathrm{d}x}b_n\sin nx}=\sum_{n=1}^{+\infty}{nb_n\cos nx}.
\end{align*}
由\hyperref[theorem:函数项级数连续性定理]{函数项级数连续性定理}知$\phi' \left( x \right) \in C\left( \mathbb{R} \right)$.因此$\phi \left( x \right) \in C^1\left( \mathbb{R} \right)$.

{\heiti 必要性:}任取\( x > 0 \),由\hyperref[theorem:函数项级数逐项求积定理]{函数项级数逐项求积定理}知
\begin{align*}
\frac{\int_{0}^{x} \phi(t) \, \mathrm{d}t}{x^2} &= \frac{1}{x^2} \sum_{n=1}^{+\infty} \frac{b_n (1 - \cos nx)}{n} \geqslant \frac{1}{x^2} \sum_{n=1}^{m} \frac{b_n (1 - \cos nx)}{n}, \quad \forall m \geqslant 1.
\end{align*}
上式中令\( x \to 0^+ \)再结合L'Hospital法则得到
\begin{align*}
\frac{1}{2} \phi'(0) &= \lim_{x \to 0^+} \frac{\int_{0}^{x} \phi(t) \, \mathrm{d}t}{x^2} \geqslant \lim_{x \to 0^+} \frac{1}{x^2} \sum_{n=1}^{m} \frac{b_n (1 - \cos nx)}{n} = \frac{1}{2} \sum_{n=1}^{m} nb_n.
\end{align*}
因此\( \sum_{n=1}^{+\infty} nb_n \)收敛.
\end{enumerate}

\end{proof}

\begin{example}
求$p,q$的范围使得
\begin{align*}
\sum_{n=1}^{\infty} \frac{x}{n^p + n^q x^2}
\end{align*}
在$\mathbb{R}$上一致收敛.
\end{example}
\begin{proof}
当$p+q>2$时,我们有
\begin{align*}
\sum_{n=1}^{\infty}{\frac{x}{n^p+n^qx^2}}\leqslant \sum_{n=1}^{\infty}{\frac{x}{2\sqrt{n^p\cdot n^qx^2}}}=\sum_{n=1}^{\infty}{\frac{1}{2n^{\frac{p+q}{2}}}}<+\infty .
\end{align*}
当$p+q\leqslant 2$时,对$\forall m\in \mathbb{N}$,取$x=m^{\frac{p-q}{2}}$,则
\begin{align*}
\sum_{n=m}^{2m}{\frac{m^{\frac{p-q}{2}}}{n^p+n^q\left( m^{\frac{p-q}{2}} \right) ^2}}&\geqslant \sum_{n=m}^{2m}{\frac{m^{\frac{p-q}{2}}}{\left( 2m \right) ^p+\left( 2m \right) ^qm^{p-q}}}=\sum_{n=m}^{2m}{\frac{1}{2^pm^{\frac{p+q}{2}}+2^qm^{\frac{p+q}{2}}}}
\\
&=\frac{1}{2^p+2^q}\cdot \frac{1}{m^{\frac{p+q}{2}-1}}\geqslant \frac{1}{2^p+2^q}>0.
\end{align*}
故由不一致收敛的Cauchy收敛准则知,此时$\sum_{n=1}^{\infty}{\frac{x}{n^p+n^qx^2}}$不一致收敛.

\end{proof}












































\end{document}
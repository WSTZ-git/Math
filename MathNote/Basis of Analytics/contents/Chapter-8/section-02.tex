\documentclass[../../main.tex]{subfiles}
\graphicspath{{\subfix{../../image/}}} % 指定图片目录,后续可以直接使用图片文件名。

% 例如:
% \begin{figure}[H]
% \centering
% \includegraphics[scale=0.4]{图.png}
% \caption{}
% \label{figure:图}
% \end{figure}
% 注意:上述\label{}一定要放在\caption{}之后,否则引用图片序号会只会显示??.

\begin{document}

\section{积分性态分析}

\begin{example}
已知$f(x)\in C[a,b]$,且
\begin{align*}
\int_a^b{f\left( x \right) \mathrm{d}x}=\int_a^b{xf\left( x \right) \mathrm{d}x}=0.
\end{align*}
证明:$f(x)$在$(a,b)$上至少2个零点.
\end{example}
\begin{proof}
设\(F_1(x)=\int_a^x f(t)dt\),则\(F_1(a)=F_1(b)=0\).再设\(F_2(x)=\int_a^x F_1(t)dt=\int_a^x\left[\int_a^t f(s)ds\right]dt\),则\(F_2(a)=0\),\(F_{2}^{\prime}(x)=F_1(x)\),\(F_{2}^{''}(x)=F_{1}^{\prime}(x)=f(x)\).由条件可知
\begin{align*}
0=\int_a^b xf(x)\mathrm{d}x
=\int_a^b xF_{1}^{\prime}(x)\mathrm{d}x
=\int_a^b xdF_1(x)
=xF_1(x)\Big|_{a}^{b}-\int_a^b F_1(x)\mathrm{d}x
=-F_2(b).
\end{align*}
于是由\(Rolle\)中值定理可知,存在\(\xi\in(a,b)\),使得\(F_{2}^{\prime}(\xi)=F_1(\xi)=0\).从而再由\(Rolle\)中值定理可知,存在\(\eta_1\in(a,\xi)\),\(\eta_2\in(\xi,b)\),使得\(F_{1}^{\prime}(\eta_1)=F_{1}^{\prime}(\eta_2)=0\).即\(f(\eta_1)=f(\eta_2)=0\).
\end{proof}

\begin{example}\label{example245574}
已知$f(x)\in C[a,b]$,且
\begin{align*}
\int_a^b{x^kf\left( x \right) \mathrm{d}x}=0, k=0,1,2,\cdots ,n .
\end{align*}
证明:$f(x)$在$(a,b)$上至少n+1个零点.
\end{example}
\begin{note}
利用分部积分转换导数的技巧.
\end{note}
\begin{proof}
令\(F(x)=\int_a^x\int_a^{x_n}\cdots\int_a^{x_3}\left[\int_a^{x_2}f(x_1)\mathrm{d}x_1\right]\mathrm{d}x_2\cdots \mathrm{d}x_n\).则\(F(a)=F^\prime(a)=\cdots=F^{(n)}(a)=0\),\(F^{(n + 1)}(x)=f(x)\).由已知条件,再反复分部积分,可得当\(1\leqslant k\leqslant n\)且\(k\in\mathbb{N}\)时,有
\begin{align*}
&0=\int_a^b{f\left( x \right) \mathrm{d}x}=\int_a^b{F^{\left( n+1 \right)}\left( x \right) \mathrm{d}x}=F^{\left( n \right)}\left( x \right) \Big|_{a}^{b}=F^{\left( n \right)}\left( b \right) ,
\\
&0=\int_a^b{xf\left( x \right) \mathrm{d}x}=\int_a^b{xF^{\left( n+1 \right)}\left( x \right) \mathrm{d}x}=\int_a^b{xdF^{\left( n \right)}\left( x \right)}=xF^{\left( n \right)}\left( x \right) \Big| _{a}^{b}-\int_a^b{F^{\left( n \right)}\left( x \right) \mathrm{d}x}=-F^{\left( n-1 \right)}\left( b \right) ,
\\
&\cdots \cdots 
\\
&0=\int_a^b{x^nf\left( x \right) \mathrm{d}x}=\int_a^b{x^nF^{\left( n+1 \right)}\left( x \right) \mathrm{d}x}=\int_a^b{x^ndF^{\left( n \right)}\left( x \right)}=x^nF^{\left( n \right)}\left( x \right) \Big| _{a}^{b}-n\int_a^b{x^{n-1}F^{\left( n \right)}\left( x \right) \mathrm{d}x}
\\
&=-n\int_a^b{x^{n-1}F^{\left( n \right)}\left( x \right) \mathrm{d}x}=\cdots =\left( -1 \right) ^nn!\int_a^b{F\prime\left( x \right) \mathrm{d}x}=\left( -1 \right) ^nn!F\left( b \right) .
\end{align*}
从而\(F(b)=F^\prime(b)=\cdots=F^{(n)}(b)=0\).于是由\(Rolle\)中值定理可知,存在\(\xi_1^1\in(a,b)\),使得\(F^\prime(\xi_1^1)=0\).再利用\(Rolle\)中值定理可知存在\(\xi_1^2,\xi_2^2\in(a,b)\),使得\(F^{\prime\prime}(\xi_1^2)=F^{\prime\prime}(\xi_2^2)=0\).
反复利用\(Rolle\)中值定理可得,存在\(\xi_1^{n + 1},\xi_2^{n + 1},\cdots,\xi_{n + 1}^{n + 1}\in(a,b)\),使得\(F^{(n + 1)}(\xi_1^{n + 1})=F^{(n + 1)}(\xi_2^{n + 1})=\cdots=F^{(n + 1)}(\xi_{n + 1}^{n + 1})=0\).即\(f(\xi_1^{n + 1})=f(\xi_2^{n + 1})=\cdots=f(\xi_{n + 1}^{n + 1})=0\).
\end{proof}

\begin{example}
已知$f(x)\in D^2[0,1]$,且
\begin{align*}
\int_0^1{f\left( x \right) \mathrm{d}x}=\frac{1}{6},\int_0^1{xf\left( x \right) \mathrm{d}x}=0,\int_0^1{x^2f\left( x \right) \mathrm{d}x}=\frac{1}{60}.
\end{align*}
证明:存在$\xi \in (0,1)$,使得$f''(\xi)=16$.
\end{example}
\begin{note}
构造$g(x)=f(x)-(8x^2 - 9x + 2)$的原因:受到\hyperref[example245574]{上一题}的启发,我们希望找到一个$g(x)=f(x)-p(x)$,使得
\begin{align*}
\int_0^1 x^k g(x)\mathrm{d}x =\int_0^1 x^k [f(x)-p(x)]\mathrm{d}x = 0, \quad k = 0,1,2.
\end{align*}
成立.即
\begin{align*}
\int_0^1{x^kf(x)\mathrm{d}x}=\int_0^1{x^kp(x)\mathrm{d}x},\quad k=0,1,2.
\end{align*}
待定$p(x)=ax^2+bx+c$,则代入上述公式,再结合已知条件可得
\begin{gather*}
\frac{1}{6}=\int_0^1{p(x)\mathrm{d}x}=\int_0^1{\left( ax^2+bx+c \right) \mathrm{d}x}=\frac{a}{3}+\frac{b}{2}+c,
\\
0=\int_0^1{xp(x)\mathrm{d}x}=\int_0^1{\left( ax^3+bx^2+cx \right) \mathrm{d}x}=\frac{a}{4}+\frac{b}{3}+\frac{c}{2},
\\
\frac{1}{60}=\int_0^1{x^2p(x)\mathrm{d}x}=\int_0^1{\left( ax^4+bx^3+cx^2 \right) \mathrm{d}x}=\frac{a}{5}+\frac{b}{4}+\frac{c}{3}.
\end{gather*}
解得:$a=8,b=-9,c=2$.于是就得到\(g(x)=f(x)-(8x^2 - 9x + 2)\).
\end{note}
\begin{proof}
令\(g(x)=f(x)-(8x^2 - 9x + 2)\),则由条件可得
\[
\int_0^1 x^k g(x)\mathrm{d}x = 0, \quad k = 0,1,2.
\]
再令\(G(x)=\int_0^x\left[\int_0^t\left(\int_0^s g(y)\mathrm{d}y\right)ds\right]dt\),则\(G(0)=G^\prime(0)=G^{\prime\prime}(0)=0\),\(G^{\prime\prime\prime}(x)=g(x)\).利用分部积分可得
\begin{align*}
&0=\int_0^1{g\left( x \right) \mathrm{d}x}=\int_0^1{G'''\left( x \right) \mathrm{d}x}=G''\left( 1 \right) ,
\\
&0=\int_0^1{xg\left( x \right) \mathrm{d}x}=\int_0^1{xG'''\left( x \right) \mathrm{d}x}=\int_0^1{xdG''\left( x \right)}=xG''\left( x \right) \Big |_{0}^{1}-\int_0^1{G''\left( x \right) \mathrm{d}x}=-G'\left( 1 \right) ,
\\
&0=\int_0^1{x^2g\left( x \right) \mathrm{d}x}=\int_0^1{x^2G'''\left( x \right) \mathrm{d}x}=\int_0^1{x^2dG''\left( x \right)}=x^2G''\left( x \right) \Big |_{0}^{1}-2\int_0^1{xG''\left( x \right) \mathrm{d}x}
\\
&=-2\int_0^1{xdG'\left( x \right)}=2\int_0^1{G'\left( x \right) \mathrm{d}x}-2xG'\left( x \right) \Big |_{0}^{1}=2G\left( 1 \right) .
\end{align*}
从而\(G(1)=G^\prime(1)=G^{\prime\prime}(1)=0\).于是由\(Rolle\)中值定理可知,存在\(\xi_1^1\in(0,1)\),使得\(G^\prime(\xi_1^1)=0\).
再利用\(Rolle\)中值定理可知,存在\(\xi_1^2,\xi_2^2\in(0,1)\),使得\(G^{\prime\prime}(\xi_1^2)=G^{\prime\prime}(\xi_2^2)=0\).
反复利用\(Rolle\)中值定理可得,存在\(\xi_1^3,\xi_2^3,\xi_3^3\in(0,1)\),使得\(G^{\prime\prime\prime}(\xi_1^3)=G^{\prime\prime\prime}(\xi_2^3)=G^{\prime\prime\prime}(\xi_3^3)=0\).即\(g(\xi_1^3)=g(\xi_2^3)=g(\xi_3^3)=0\).
再反复利用\(Rolle\)中值定理可得,存在\(\xi\in(0,1)\),使得\(g^{\prime\prime}(\xi)=0\).即\(f^{\prime\prime}(\xi)=16\).
\end{proof}










\end{document}
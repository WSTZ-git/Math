\documentclass[../../main.tex]{subfiles}% 注意这里的文件路径不能用 ./main.tex ,否则用latexmk编译子文件会报错
\graphicspath{{\subfix{./image/}}} % 指定图片目录,后续可以直接使用图片文件名
% 注意这里的文件路径不能用 ../../image/ ,否则用latexmk编译子文件会报错

% 例如:
% \begin{figure}[H]
% \centering
% \includegraphics[scale=0.3]{图.png}
% \caption{}
% \label{figure:图}
% \end{figure}
% 注意:上述\label{}一定要放在\caption{}之后,否则引用图片序号会只会显示??.

\begin{document}

\section{第二型曲线积分}

\begin{definition}\label{definecolor:平面上的第二型曲线积分}
设函数$P(x,y)$与$Q(x,y)$定义在平面有向可求长度曲线$L:\wideparen{AB}$上.对$L$的任一分割$T$,它把$L$分成$n$个小弧段
\[
\wideparen{M_{i-1}M_i}\quad(i=1,2,\cdots,n),
\]
其中$M_0=A,M_n=B$.记各小弧段$\wideparen{M_{i-1}M_i}$的弧长为$\Delta s_i$,分割$T$的细度$\|T\|=\max\limits_{1 \leqslant i \leqslant n}\Delta s_i$.又设$T$的分点$M_i$的坐标为$(x_i,y_i)$,并记$\Delta x_i=x_i-x_{i-1},\Delta y_i=y_i-y_{i-1}\ (i=1,2,\cdots,n)$.在每个小弧段$\wideparen{M_{i-1}M_i}$上任取一点$(\xi_i,\eta_i)$,若极限
\begin{align*}
\lim\limits_{\|T\| \to 0} \sum_{i=1}^n P(\xi_i,\eta_i)\Delta x_i + \lim\limits_{\|T\| \to 0} \sum_{i=1}^n Q(\xi_i,\eta_i)\Delta y_i
\end{align*}
存在且与分割$T$和点$(\xi_i,\eta_i)$的取法无关,则称此极限为函数$P(x,y),Q(x,y)$沿有向曲线$L$上的\textbf{第二型曲线积分},记为
\begin{align}
\int_L P(x,y)\mathrm{d}x + Q(x,y)\mathrm{d}y\quad\text{或}\quad\int_{\wideparen{AB}} P(x,y)\mathrm{d}x + Q(x,y)\mathrm{d}y. \label{eq::::32dfw:f9hq98u892d2a2r1}
\end{align}
上述积分\eqref{eq::::32dfw:f9hq98u892d2a2r1}也可写作
\[
\int_L P(x,y)\mathrm{d}x + \int_L Q(x,y)\mathrm{d}y
\]
或
\[
\int_{\wideparen{AB}} P(x,y)\mathrm{d}x + \int_{\wideparen{AB}} Q(x,y)\mathrm{d}y.
\]
为书写简洁起见,\eqref{eq::::32dfw:f9hq98u892d2a2r1}式常简写成
\[
\int_L P\mathrm{d}x + Q\mathrm{d}y\quad\text{或}\quad\int_{\wideparen{AB}} P\mathrm{d}x + Q\mathrm{d}y.
\]
若$L$为封闭的有向曲线,则记为
\begin{align*}
\oint_L P\mathrm{d}x + Q\mathrm{d}y. 
\end{align*}

若记$\boldsymbol{F}(x,y)=(P(x,y),Q(x,y)),\mathrm{d}\boldsymbol{s}=(\mathrm{d}x,\mathrm{d}y)$,则\eqref{eq::::32dfw:f9hq98u892d2a2r1}式可写成向量形式
\begin{align*}
\int_L \boldsymbol{F} \cdot \mathrm{d}\boldsymbol{s}\quad\text{或}\quad\int_{\wideparen{AB}} \boldsymbol{F} \cdot \mathrm{d}\boldsymbol{s}. 
\end{align*}
\end{definition}

\begin{definition}
倘若$L$为空间有向可求长度曲线,$P(x,y,z),Q(x,y,z),R(x,y,z)$为定义在$L$上的函数,则可按\refdef{definecolor:平面上的第二型曲线积分}类似地定义沿空间有向曲线$L$上的\textbf{第二型曲线积分},并记为
\begin{align}
\int_L P(x,y,z)\mathrm{d}x + Q(x,y,z)\mathrm{d}y + R(x,y,z)\mathrm{d}z, \label{eq::::32dwssgszzzzzzzz:f9hq98u892d2a2r1}
\end{align}
或简写成
\[
\int_L P\mathrm{d}x + Q\mathrm{d}y + R\mathrm{d}z.
\]
当把$\boldsymbol{F}(x,y,z)=(P(x,y,z),Q(x,y,z),R(x,y,z))$与$\mathrm{d}\boldsymbol{s}=(\mathrm{d}x,\mathrm{d}y,\mathrm{d}z)$看作三维向量时,\eqref{eq::::32dwssgszzzzzzzz:f9hq98u892d2a2r1}式也可表示成向量形式
\begin{align*}
\int_L \boldsymbol{F} \cdot \mathrm{d}\boldsymbol{s}. 
\end{align*}
\end{definition}

\begin{theorem}\label{theorem:第二型曲线积分的基本性质}
\begin{enumerate}[(1)]
\item 若$\int_{\wideparen{AB}} P\mathrm{d}x + Q\mathrm{d}y$存在,则
\begin{align*}
\int_{\wideparen{AB}} P\mathrm{d}x + Q\mathrm{d}y = -\int_{\wideparen{BA}} P\mathrm{d}x + Q\mathrm{d}y. 
\end{align*}

\item 若$\int_L P_i\mathrm{d}x + Q_i\mathrm{d}y\ (i=1,2,\cdots,k)$存在,则$\int_L \left( \sum\limits_{i=1}^k c_i P_i \right)\mathrm{d}x + \left( \sum\limits_{i=1}^k c_i Q_i \right)\mathrm{d}y$也存在,且
\begin{align*}
\int_L \left( \sum_{i=1}^k c_i P_i \right)\mathrm{d}x + \left( \sum_{i=1}^k c_i Q_i \right)\mathrm{d}y = \sum_{i=1}^k c_i \left( \int_L P_i\mathrm{d}x + Q_i\mathrm{d}y \right),
\end{align*}
其中$c_i\ (i=1,2,\cdots,k)$为常数.

\item 若有向曲线$L$是由有向曲线$L_1,L_2,\cdots,L_k$首尾相接而成,且$\int_{L_i} P\mathrm{d}x + Q\mathrm{d}y\ (i=1,2,\cdots,k)$存在,则$\int_L P\mathrm{d}x + Q\mathrm{d}y$也存在,且
\begin{align*}
\int_L P\mathrm{d}x + Q\mathrm{d}y = \sum_{i=1}^k \int_{L_i} P\mathrm{d}x + Q\mathrm{d}y.
\end{align*}
\end{enumerate}
\end{theorem}
\begin{proof}


\end{proof}

\begin{theorem}\label{theorem:第二型曲线积分的换元积分公式}
\begin{enumerate}[(1)]
\item 设平面曲线
\begin{align*}
L:\begin{cases} x = \varphi(t), \\ y = \psi(t), \end{cases} \quad t \in [\alpha,\beta],
\end{align*}
其中$\varphi(t),\psi(t)$在$[\alpha,\beta]$上具有一阶连续导函数,且点$A$与$B$的坐标分别为$(\varphi(\alpha),\psi(\alpha))$与$(\varphi(\beta),\psi(\beta))$.又设$P(x,y)$与$Q(x,y)$为$L$上的连续函数,则沿$L$从$A$到$B$的第二型曲线积分
\begin{align}\label{eq::fionsfioqfifkldfsfiojdlszji}
\int_L P(x,y)\mathrm{d}x + Q(x,y)\mathrm{d}y = \int_{\alpha}^{\beta} \left[ P(\varphi(t),\psi(t))\varphi'(t) + Q(\varphi(t),\psi(t))\psi'(t) \right] \mathrm{d}t.
\end{align}

\item 设空间有向光滑曲线$L$的参量方程为
\begin{align*}
\begin{cases} x = x(t), \\ y = y(t), \\ z = z(t), \end{cases} \quad \alpha \leqslant t \leqslant \beta,
\end{align*}
起点为$(x(\alpha),y(\alpha),z(\alpha))$,终点为$(x(\beta),y(\beta),z(\beta))$,则
\begin{align*}
\int_L P\mathrm{d}x + Q\mathrm{d}y + R\mathrm{d}z = \int_{\alpha}^{\beta} \left[ P(x(t),y(t),z(t))x'(t) + Q(x(t),y(t),z(t))y'(t) + R(x(t),y(t),z(t))z'(t) \right] \mathrm{d}t.
\end{align*}
这里要注意曲线方向与积分上下限的确定应该一致.
\end{enumerate}
\end{theorem}
\begin{proof}


\end{proof}

\begin{theorem}[两类曲线积分的联系]
设$L$为从$A$到$B$的有向光滑曲线,它以弧长$s$为参数,于是
\begin{align*}
L:\begin{cases} x = x(s), \\ y = y(s), \end{cases} \quad 0 \leqslant s \leqslant l,
\end{align*}
其中$l$为曲线$L$的全长,且点$A$与$B$的坐标分别为$(x(0),y(0))$与$(x(l),y(l))$.曲线$L$上每一点的切线方向指向弧长增加的一方.现以$(\wideparen{t,x}),(\wideparen{t,y})$分别表示切线方向$t$与$x$轴和$y$轴正向的夹角,则在曲线上的每一点的切线方向余弦是
\begin{align}
\frac{\mathrm{d}x}{\mathrm{d}s} = \cos(\wideparen{t,x}),\quad \frac{\mathrm{d}y}{\mathrm{d}s} = \cos(\wideparen{t,y}). \label{8}
\end{align}
若$P(x,y),Q(x,y)$为曲线$L$上的连续函数,则由\eqref{eq::fionsfioqfifkldfsfiojdlszji}式得
\begin{align}
\int_L P\mathrm{d}x + Q\mathrm{d}y
&= \int_0^l \left[ P(x(s),y(s))\cos(\wideparen{t,x}) + Q(x(s),y(s))\cos(\wideparen{t,y}) \right] \mathrm{d}s\nonumber \\
&= \int_L \left[ P(x,y)\cos(\wideparen{t,x}) + Q(x,y)\cos(\wideparen{t,y}) \right] \mathrm{d}s,\label{eq::iom2wj9r2jrwpq3-99}
\end{align}
这里必须指出,当\eqref{eq::iom2wj9r2jrwpq3-99}式左边第二型曲线积分中$L$改变方向时,积分值改变符号,相应在\eqref{eq::iom2wj9r2jrwpq3-99}式右边第一型曲线积分中,曲线上各点的切线方向指向相反的方向(即指向弧长减少的方向).这时夹角$(\wideparen{t,x})$和$(\wideparen{t,y})$分别与原来的夹角相差一个弧度$\pi$,从而$\cos(\wideparen{t,x})$和$\cos(\wideparen{t,y})$都要变号.因此,一旦方向确定了,公式\eqref{eq::iom2wj9r2jrwpq3-99}总是成立的.

类似讨论可以得到
\begin{align*}
\int_L P\mathrm{d}x + Q\mathrm{d}y + R\mathrm{d}z
= \int_L \left[ P(x,y,z)\cos(\wideparen{t,x}) + Q(x,y,z)\cos(\wideparen{t,y}) + R(x,y,z)\cos(\wideparen{t,z}) \right] \mathrm{d}s,
\end{align*}
其中$P,Q,R$为空间有向曲线$L$上的连续函数,$(\cos(\wideparen{t,x}),\cos(\wideparen{t,y}),\cos(\wideparen{t,z}))$为曲线$L$正切向的方向余弦.
\end{theorem}
\begin{proof}


\end{proof}








\end{document}
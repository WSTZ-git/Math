\documentclass[../../main.tex]{subfiles}% 注意这里的文件路径不能用 ./main.tex ,否则用latexmk编译子文件会报错
\graphicspath{{\subfix{./image/}}} % 指定图片目录,后续可以直接使用图片文件名
% 注意这里的文件路径不能用 ../../image/ ,否则用latexmk编译子文件会报错

% 例如:
% \begin{figure}[H]
% \centering
% \includegraphics[scale=0.3]{图.png}
% \caption{}
% \label{figure:图}
% \end{figure}
% 注意:上述\label{}一定要放在\caption{}之后,否则引用图片序号会只会显示??.

\begin{document}

\section{第一型曲线积分}

\begin{definition}
设$L$为平面上可求长度的曲线段,$f(x,y)$为定义在$L$上的函数.对曲线$L$作分割$T$,它把$L$分成$n$个可求长度的小曲线段$L_i$ $(i=1,2,\cdots,n)$,$L_i$的弧长记为$\Delta s_i$,分割$T$的细度为$\| T \| = \max\limits_{1 \leqslant i \leqslant n} \Delta s_i$,在$L_i$上任取一点$(\xi_i,\eta_i)$ $(i=1,2,\cdots,n)$.若有极限
\begin{align*}
\lim\limits_{\| T \| \to 0} \sum_{i=1}^n f(\xi_i,\eta_i)\Delta s_i = J,
\end{align*}
且$J$的值与分割$T$和点$(\xi_i,\eta_i)$的取法无关,则称此极限为$f(x,y)$在$L$上的\textbf{第一型曲线积分},记作
\begin{align*}
\int_L f(x,y)\mathrm{d}s.
\end{align*}
若$L$为空间可求长曲线段,$f(x,y,z)$为定义在$L$上的函数,则可类似地定义$f(x,y,z)$在空间曲线$L$上的\textbf{第一型曲线积分},并且记作
\begin{align*}
\int_L f(x,y,z)\mathrm{d}s.
\end{align*}
\end{definition}
\begin{remark}
第一型曲线积分的几何意义:
若$L$为平面$Oxy$上分段光滑曲线,$f(x,y)$为定义在$L$上非负连续函数.由第一型曲面积分的定义,以$L$为准线,母线平行于$z$轴的柱面上截取$0 \leqslant z \leqslant f(x,y)$的部分面积就是$\int_L f(x,y)\mathrm{d}s$.
\end{remark}

\begin{theorem}
\begin{enumerate}[(1)]
\item 若$\int_L f_i(x,y)\mathrm{d}s$ $(i=1,2,\cdots,k)$存在,$c_i$ $(i=1,2,\cdots,k)$为常数,则$\int_L \sum\limits_{i=1}^k c_i f_i(x,y)\mathrm{d}s$也存在,且
\begin{align*}
\int_L \sum\limits_{i=1}^k c_i f_i(x,y)\mathrm{d}s = \sum\limits_{i=1}^k c_i \int_L f_i(x,y)\mathrm{d}s.
\end{align*}

\item 若曲线段$L$由曲线$L_1,L_2,\cdots,L_k$首尾相接而成,且$\int_{L_i} f(x,y)\mathrm{d}s$ $(i=1,2,\cdots,k)$都存在,则$\int_L f(x,y)\mathrm{d}s$也存在,且
\begin{align*}
\int_L f(x,y)\mathrm{d}s = \sum\limits_{i=1}^k \int_{L_i} f(x,y)\mathrm{d}s.
\end{align*}

\item 若$\int_L f(x,y)\mathrm{d}s$与$\int_L g(x,y)\mathrm{d}s$都存在,且在$L$上$f(x,y) \leqslant g(x,y)$,则
\begin{align*}
\int_L f(x,y)\mathrm{d}s \leqslant \int_L g(x,y)\mathrm{d}s.
\end{align*}

\item 若$\int_L f(x,y)\mathrm{d}s$存在,则$\int_L |f(x,y)|\mathrm{d}s$也存在,且
\begin{align*}
\left| \int_L f(x,y)\mathrm{d}s \right| \leqslant \int_L |f(x,y)|\mathrm{d}s.
\end{align*}

\item 若$\int_L f(x,y)\mathrm{d}s$存在,$L$的弧长为$s$,则存在常数$c$,使得
\begin{align*}
\int_L f(x,y)\mathrm{d}s = cs,
\end{align*}
这里$\inf\limits_L f(x,y) \leqslant c \leqslant \sup\limits_L f(x,y)$.
\end{enumerate}
\end{theorem}
\begin{proof}


\end{proof}

\begin{theorem}
设有光滑曲线
\begin{align*}
L:\begin{cases} x = \varphi(t), \\ y = \psi(t), \end{cases} \quad t \in [\alpha,\beta],
\end{align*}
函数$f(x,y)$为定义在$L$上的连续函数,则
\begin{align}
\int_L f(x,y)\mathrm{d}s = \int_{\alpha}^{\beta} f(\varphi(t),\psi(t)) \sqrt{\varphi'^2(t) + \psi'^2(t)} \mathrm{d}t. \label{eq:::f9hq98u892d2a2r1}
\end{align}
特别地,当曲线$L$由方程
\begin{align*}
y = \psi(x), \quad x \in [a,b]
\end{align*}
表示,且$\psi(x)$在$[a,b]$上有连续的导函数时,\eqref{eq:::f9hq98u892d2a2r1}式成为
\begin{align*}
\int_L f(x,y)\mathrm{d}s = \int_{a}^{b} f(x,\psi(x)) \sqrt{1 + \psi'^2(x)} \mathrm{d}x;
\end{align*}
当曲线$L$由方程
\begin{align*}
x = \varphi(y), \quad y \in [c,d]
\end{align*}
表示,且$\varphi(y)$在$[c,d]$上有连续导函数时,\eqref{eq:::f9hq98u892d2a2r1}式成为
\begin{align*}
\int_L f(x,y)\mathrm{d}s = \int_{c}^{d} f(\varphi(y),y) \sqrt{1 + \varphi'^2(y)} \mathrm{d}y.
\end{align*}
\end{theorem}
\begin{proof}
由弧长公式知道,$L$上由$t=t_{i-1}$到$t=t_i$的弧长
\begin{align*}
\Delta s_i = \int_{t_{i-1}}^{t_i} \sqrt{\varphi'^2(t) + \psi'^2(t)} \mathrm{d}t.
\end{align*}
由$\sqrt{\varphi'^2(t)+\psi'^2(t)}$的连续性与积分中值定理,有
\begin{align*}
\Delta s_i = \sqrt{\varphi'^2(\tau_i') + \psi'^2(\tau_i')} \Delta t_i \quad (t_{i-1} < \tau_i' < t_i).
\end{align*}
所以
\begin{align*}
\sum_{i=1}^n f(\xi_i,\eta_i) \Delta s_i = \sum_{i=1}^n f(\varphi(\tau_i''),\psi(\tau_i'')) \sqrt{\varphi'^2(\tau_i') + \psi'^2(\tau_i')} \Delta t_i,
\end{align*}
这里$t_{i-1} \leqslant \tau_i',\tau_i'' \leqslant t_i$.设
\begin{align*}
\sigma = \sum_{i=1}^n f(\varphi(\tau_i''),\psi(\tau_i'')) \left[ \sqrt{\varphi'^2(\tau_i') + \psi'^2(\tau_i')} - \sqrt{\varphi'^2(\tau_i'') + \psi'^2(\tau_i'')} \right] \Delta t_i,
\end{align*}
则有
\begin{align}
\sum_{i=1}^n f(\xi_i,\eta_i) \Delta s_i = \sum_{i=1}^n f(\varphi(\tau_i''),\psi(\tau_i'')) \sqrt{\varphi'^2(\tau_i'') + \psi'^2(\tau_i'')} \Delta t_i + \sigma. \label{eq:::f9hq98u892d21}
\end{align}
令$\Delta t = \max\{ \Delta t_1,\Delta t_2,\cdots,\Delta t_n \}$,则当$\| T \| \to 0$时,必有$\Delta t \to 0$.现在证明$\lim\limits_{\Delta t \to 0} \sigma = 0$.

因为复合函数$f(\varphi(t),\psi(t))$关于$t$连续,所以在闭区间$[\alpha,\beta]$上有界,即存在常数$M$,使对一切$t \in [\alpha,\beta]$,都有
\begin{align*}
|f(\varphi(t),\psi(t))| \leqslant M.
\end{align*}
再由$\sqrt{\varphi'^2(t)+\psi'^2(t)}$在$[\alpha,\beta]$上连续,所以它在$[\alpha,\beta]$上一致连续,即对任给的$\varepsilon > 0$,必存在$\delta > 0$,使当$\Delta t < \delta$时有
\begin{align*}
\left| \sqrt{\varphi'^2(\tau_i'') + \psi'^2(\tau_i'')} - \sqrt{\varphi'^2(\tau_i') + \psi'^2(\tau_i')} \right| < \varepsilon,
\end{align*}
从而
\begin{align*}
|\sigma| \leqslant \varepsilon M \sum_{i=1}^n \Delta t_i = \varepsilon M (\beta - \alpha),
\end{align*}
所以
\begin{align*}
\lim_{\Delta t \to 0} \sigma = 0.
\end{align*}

再由定积分定义,
\begin{align*}
\lim_{\Delta t \to 0} \sum_{i=1}^n f(\varphi(\tau_i''),\psi(\tau_i'')) \sqrt{\varphi'^2(\tau_i'') + \psi'^2(\tau_i'')} \Delta t_i
= \int_{\alpha}^{\beta} f(\varphi(t),\psi(t)) \sqrt{\varphi'^2(t) + \psi'^2(t)} \mathrm{d}t.
\end{align*}
因此当在\eqref{eq:::f9hq98u892d21}式两边取极限后,即得所要证的\eqref{eq:::f9hq98u892d2a2r1}式.


\end{proof}







\end{document}
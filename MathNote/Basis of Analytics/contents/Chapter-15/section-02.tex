\documentclass[../../main.tex]{subfiles}
\graphicspath{{\subfix{../../image/}}} % 指定图片目录,后续可以直接使用图片文件名。

% 例如:
% \begin{figure}[h]
% \centering
% \includegraphics{image-01.01}
% \label{fig:image-01.01}
% \caption{图片标题}
% \end{figure}

\begin{document}

\section{反常积分收敛的相关结论}

\begin{proposition}\label{proposition:反常积分收敛的基本结论1}
\begin{enumerate}[(1)]
\item 设\(\int_a^{+\infty}{f(x)\,\mathrm{d}x}\)收敛,且\(f(x)\)单调,则\(\lim\limits_{x\rightarrow+\infty}xf(x)=0\)。

\item 若\(\int_a^{+\infty} f(x)\,\mathrm{d}x\)收敛且\(xf(x)\)单调,则\(\lim\limits_{x \to +\infty}x\ln x f(x)=0\)。
\end{enumerate}
\end{proposition}
\begin{proof}
\begin{enumerate}[(1)]
\item 不妨设\(f\)递减,否则用\(-f\)代替\(f\),
从而
\begin{align*}
Af(A)\geqslant \int_A^{2A}{f(x)\,\mathrm{d}x},\quad \frac{A}{2}f(A)\leqslant \int_{\frac{A}{2}}^A{f(x)\,\mathrm{d}x.}
\end{align*}
进而
\begin{align*}
\int_A^{2A}{f(x)\,\mathrm{d}x}\leqslant Af(A)\leqslant 2\int_{\frac{A}{2}}^A{f(x)\,\mathrm{d}x}.
\end{align*}
由\(\int_a^{+\infty}{f(x)\,\mathrm{d}x}\)收敛的Cauchy收敛准则可知
\begin{align*}
\int_{\frac{A}{2}}^A{f(x)\,\mathrm{d}x}\rightarrow 0, A\rightarrow +\infty,  \quad
\int_A^{2A}{f(x)\,\mathrm{d}x}\rightarrow 0, A\rightarrow +\infty.
\end{align*}
故\(\lim\limits_{A\rightarrow+\infty}Af(A)=0\)。

\item 不妨设\(xf\)递减,否则用\(-f\)代替\(f\)即可。于是
\begin{align*}
\frac{1}{2}A\ln A f(A) = Af(A)\int_{\sqrt{A}}^A\frac{1}{x}\,\mathrm{d}x 
\leq \int_{\sqrt{A}}^A\frac{xf(x)}{x}\,\mathrm{d}x 
= \int_{\sqrt{A}}^A f(x)\,\mathrm{d}x,
\\ 
\int_A^{A^2} f(x)\,\mathrm{d}x = \int_A^{A^2}\frac{xf(x)}{x}\,\mathrm{d}x 
\leq Af(A)\int_A^{A^2}\frac{1}{x}\,\mathrm{d}x 
= A\ln A f(A).
\end{align*}
从而
\begin{align*}
\int_A^{A^2} f(x)\,\mathrm{d}x \leq A\ln A f(A) 
\leq 2\int_{\sqrt{A}}^A f(x)\,\mathrm{d}x
\end{align*}
又由\(\int_a^{+\infty} f(x)\,\mathrm{d}x\)收敛的Cauchy收敛准则可知
\begin{align*}
\int_{\sqrt{A}}^A f(x)\,\mathrm{d}x \to 0, A \to +\infty. \quad
\int_A^{A^2} f(x)\,\mathrm{d}x \to 0,  A \to +\infty.
\end{align*}
故由夹逼准则可知\(\lim\limits_{A \to +\infty}A\ln A f(A)=0\)。 
\end{enumerate}
\end{proof}

\begin{example}
设 \(f\in D^{1}(0,+\infty)\) 且 \(|f'|\) 在 \((0,+\infty)\) 递减。若 \(\lim\limits_{x \to +\infty} f(x)\) 存在,证明: \(\lim\limits_{x \to +\infty} x f'(x)=0\)。 
\end{example}
\begin{proof}
若存在 \(a > 0\),使得 \(f'(a) = 0\),则由 \(|f'|\) 在 \((0, +\infty)\) 递减可得
\begin{align*}
f'(x) = 0, \quad \forall x > a.
\end{align*}
此时结论显然成立。

若 \(f' \neq 0\),\(\forall x \in (0, +\infty)\),则由导数介值性可知,\(f'\) 在 \((0, +\infty)\) 上要么恒大于零,要么恒小于零。
于是不妨设 \(f' > 0\),\(\forall x \in (0, +\infty)\),故此时 \(f\) 在 \((0, +\infty)\) 上严格递增。
并且此时 \(f' = |f'|\) 在 \((0, +\infty)\) 递减,故此时 \(f'\) 在 \((0, +\infty)\) 内闭 Riemann 可积。
从而由微积分基本定理可知
\begin{align*}
\int_1^x f'(y) \, \mathrm{d}y = f(x) - f(1).
\end{align*}
又因为 \(\lim\limits_{x \to +\infty} f(x)\) 存在,所以 \(\int_1^{+\infty} f'(y) \, \mathrm{d}y\) 收敛。于是由\nrefpro{proposition:反常积分收敛的基本结论1}{(1)}可知 \(\lim\limits_{x \to +\infty} x f'(x) = 0\)。
\end{proof}

\begin{example}
设 \(f\) 在 \((a,+\infty)\) 可导。如果 \(f\) 有界且 \(xf'\) 为单调函数,证明
\begin{align*}
\lim_{x \to +\infty} x\ln x f'(x) = 0.
\end{align*}
\end{example}
\begin{proof}
由 \(xf'\) 单调可知,\(g(x)\triangleq xf'\) 在 \((a, +\infty)\) 上内闭 Riemann 可积。从而 \(f' = \frac{g(x)}{x}\) 在 \((a, +\infty)\) 上也内闭 Riemann 可积。
不妨设 \(xf'\) 单调递增,由单调有界定理可知,\(\lim\limits_{x \to +\infty}xf'(x)\) 存在或 \(+\infty\)。
由 \(f\) 有界可得 \(\lim\limits_{x \to +\infty}xf'(x)\leqslant 0\)。否则,若 \(\lim\limits_{x \to +\infty}xf'(x)>0\),则存在 \(C > 0\),使得
\begin{align}
xf'(x)>C \Rightarrow f'(x)>\frac{C}{x}, \quad x\in (\max\{a,0\}, +\infty). \label{eq::::1.1}
\end{align}
对 \((\ref{eq::::1.1})\) 式两边同时积分得到
\begin{align*}
f(x)>\int_a^x\frac{c}{t}\,\mathrm{d}t = c\ln|x|-c\ln a.
\end{align*}
令 \(x \to +\infty\),得到 \(\lim\limits_{x \to +\infty}f(x)=+\infty\),这与 \(f\) 有界矛盾!
于是由 \(\lim\limits_{x \to +\infty}xf'(x)\leqslant 0\) 可知存在 \(X > \max\{a,0\}\),使得
\begin{align*}
xf'(x)\leqslant 0 \Rightarrow f'(x)\leqslant 0, \quad x\in (X, +\infty).
\end{align*}
故 \(f\) 在 \((X, +\infty)\) 上递减。又因为 \(f\) 有界,所以 \(\lim\limits_{x \to +\infty}f(x)\) 存在。
根据微积分基本定理可得
\begin{align*}
\int_a^x f'(t)\,\mathrm{d}t = f(x)-f(a).
\end{align*}
令 \(x \to +\infty\),则由 \(\lim\limits_{x \to +\infty}f(x)\) 存在可得 \(\int_a^{+\infty}f'(t)\,\mathrm{d}t\) 收敛。又 \(xf'(x)\) 单调,于是由\nrefpro{proposition:反常积分收敛的基本结论1}{(2)}可知 \(\lim\limits_{x \to +\infty}x\ln x f'(x)=0\)。 
\end{proof}










\end{document}
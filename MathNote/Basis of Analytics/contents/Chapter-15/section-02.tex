\documentclass[../../main.tex]{subfiles}
\graphicspath{{\subfix{../../image/}}} % 指定图片目录,后续可以直接使用图片文件名。

% 例如:
% \begin{figure}[H]
% \centering
% \includegraphics{image-01.01}
% \caption{图片标题}
% \label{figure:image-01.01}
% \end{figure}
% 注意:上述\label{}一定要放在\caption{}之后,否则引用图片序号会只会显示??.

\begin{document}

\section{反常积分收敛抽象问题}

\begin{proposition}\label{proposition:反常积分收敛的基本结论2}
设$f$为$[a,+\infty)$上的非负可积函数,则$\int_a^{+\infty}f(x)\,\mathrm{d}x$ 收敛的充要条件是存在一个数列$\{x_n\}$,满足$x_n\to+\infty$,使得$\lim_{n\to\infty}\int_a^{x_n}f(y)\,\mathrm{d}y$存在.
\end{proposition}
\begin{proof}
必要性显然成立,下证充分性
令$g(x)=\int_a^xf(y)\,\mathrm{d}y$,则$g(x)$在$[a,+\infty)$上非负单调递增.由单调收敛定理可知$\lim_{x\to+\infty}g(x)\in\mathbb{R}\cup\{+\infty\}$.
\\
又$\lim_{n\to\infty}\int_a^{x_n}f(y)\,\mathrm{d}y$存在,即$\lim_{n\to\infty}g(x_n)$存在.从而
\begin{align*}
\lim_{x\to+\infty}g(x)=\lim_{n\to\infty}g(x_n)<+\infty.
\end{align*}
因此
\begin{align*}
\int_a^{+\infty}f(x)\,\mathrm{d}x=\lim_{x\to+\infty}g(x)<+\infty.
\end{align*}
\end{proof}

\begin{proposition}[积分收敛必有子列趋于0]\label{proposition:积分收敛必有子列趋于0}
设连续函数满足$\int_{0}^{\infty}f(x)\mathrm{d}x$收敛,则
\begin{enumerate}[(1)]
\item 存在趋于$+\infty$的$\{x_n\}_{n = 1}^{\infty}\subset(0,+\infty)$,使得$\lim_{n\to\infty}f(x_n)=0$.

\item 若$f$不一定连续,但有$\int_{0}^{\infty}|f(x)|\mathrm{d}x < \infty$,则存在严格递增的$\lim_{n\to\infty}x_n = +\infty$,使得
$\lim_{n\to\infty}x_n\ln x_nf(x_n)=0$.
\end{enumerate}
\end{proposition}
\begin{note}
连续性是否可以去掉构成一个有趣的话题. 第一问结论可以直接用,第二问主要告诉我们积分绝对收敛性,我们总能找到很好的子列极限.并且(2)中结论的$x_nlnx_n$可以换成任意数列$\{a_n\}$,只要满足$\int_{a}^{\infty}a_n\mathrm{d}x=+\infty$即可
\end{note}
\begin{proof}
\begin{enumerate}[(1)]
\item 运用积分中值定理,我们知道
\begin{align*}
\int_{A}^{A + 1}f(x)\mathrm{d}x = f(\theta(A)), A + 1 > \theta(A) > A.
\end{align*}
由Cauchy收敛准则,我们知道
\begin{align*}
0 = \lim_{A\to +\infty}\int_{A}^{A + 1}f(x)\mathrm{d}x = \lim_{A\to +\infty}f(\theta(A)), \lim_{A\to +\infty}\theta(A) = +\infty.
\end{align*}
这就完成了证明.

\item 若$|f(x)| > \frac{1}{x\ln x}, \forall x > e$,则由$\int_{e}^{\infty}\frac{1}{x\ln x}\mathrm{d}x = +\infty$可得$\int_e^{\infty}{\left| f\left( x \right) \right|\mathrm{d}x}=+\infty $矛盾! 故存在$x_1 > e$使得$|f(x_1)| \leqslant \frac{1}{x_1\ln x_1}$.
同样的,如果$|f(x)| > \frac{1}{2x\ln x}, \forall x > x_1 + 1$,同理可得矛盾!因此必然存在$x_2 > x_1 + 1$使得$|f(x_2)| \leqslant \frac{1}{2x_2\ln x_2}$. 依次下去我们得到
\begin{align*}
|f(x_n)| \leqslant \frac{1}{nx_n\ln x_n}, n = 1,2,\cdots,
\end{align*}
即
\begin{align*}
\lim_{n\to\infty}x_n\ln x_n\cdot|f(x_n)| = 0.
\end{align*} 
\end{enumerate}
\end{proof}

\begin{proposition}\label{proposition:反常积分收敛的基本结论1}
\begin{enumerate}[(1)]
\item 设\(\int_a^{+\infty}{f(x)\,\mathrm{d}x}\)收敛,且\(f(x)\)单调,则\(\lim\limits_{x\rightarrow+\infty}xf(x)=0\)。

\item 若\(\int_a^{+\infty} f(x)\,\mathrm{d}x\)收敛且\(xf(x)\)单调,则\(\lim\limits_{x \to +\infty}x\ln x f(x)=0\)。
\end{enumerate}
\end{proposition}
\begin{proof}
\begin{enumerate}[(1)]
\item 不妨设\(f\)递减,否则用\(-f\)代替\(f\),
从而
\begin{align*}
Af(A)\geqslant \int_A^{2A}{f(x)\,\mathrm{d}x},\quad \frac{A}{2}f(A)\leqslant \int_{\frac{A}{2}}^A{f(x)\,\mathrm{d}x.}
\end{align*}
进而
\begin{align*}
\int_A^{2A}{f(x)\,\mathrm{d}x}\leqslant Af(A)\leqslant 2\int_{\frac{A}{2}}^A{f(x)\,\mathrm{d}x}.
\end{align*}
由\(\int_a^{+\infty}{f(x)\,\mathrm{d}x}\)收敛的Cauchy收敛准则可知
\begin{align*}
\int_{\frac{A}{2}}^A{f(x)\,\mathrm{d}x}\rightarrow 0, A\rightarrow +\infty,  \quad
\int_A^{2A}{f(x)\,\mathrm{d}x}\rightarrow 0, A\rightarrow +\infty.
\end{align*}
故\(\lim\limits_{A\rightarrow+\infty}Af(A)=0\)。

\item 不妨设\(xf\)递减,否则用\(-f\)代替\(f\)即可。于是
\begin{align*}
\frac{1}{2}A\ln A f(A) = Af(A)\int_{\sqrt{A}}^A\frac{1}{x}\,\mathrm{d}x 
\leq \int_{\sqrt{A}}^A\frac{xf(x)}{x}\,\mathrm{d}x 
= \int_{\sqrt{A}}^A f(x)\,\mathrm{d}x,
\\ 
\int_A^{A^2} f(x)\,\mathrm{d}x = \int_A^{A^2}\frac{xf(x)}{x}\,\mathrm{d}x 
\leq Af(A)\int_A^{A^2}\frac{1}{x}\,\mathrm{d}x 
= A\ln A f(A).
\end{align*}
从而
\begin{align*}
\int_A^{A^2} f(x)\,\mathrm{d}x \leq A\ln A f(A) 
\leq 2\int_{\sqrt{A}}^A f(x)\,\mathrm{d}x
\end{align*}
又由\(\int_a^{+\infty} f(x)\,\mathrm{d}x\)收敛的Cauchy收敛准则可知
\begin{align*}
\int_{\sqrt{A}}^A f(x)\,\mathrm{d}x \to 0, A \to +\infty. \quad
\int_A^{A^2} f(x)\,\mathrm{d}x \to 0,  A \to +\infty.
\end{align*}
故由夹逼准则可知\(\lim\limits_{A \to +\infty}A\ln A f(A)=0\)。 
\end{enumerate}
\end{proof}

\begin{proposition}
若$f$在$[0, +\infty)$上一致连续, 且$\int_0^{+\infty} f(x) \, \mathrm{d}x < \infty$, 则$\lim_{x \to +\infty} f(x) = 0$.
\end{proposition}
\begin{proof}

\end{proof}

\begin{example}
设 \(f\in D^{1}(0,+\infty)\) 且 \(|f'|\) 在 \((0,+\infty)\) 递减。若 \(\lim\limits_{x \to +\infty} f(x)\) 存在,证明: \(\lim\limits_{x \to +\infty} x f'(x)=0\)。 
\end{example}
\begin{proof}
若存在 \(a > 0\),使得 \(f'(a) = 0\),则由 \(|f'|\) 在 \((0, +\infty)\) 递减可得
\begin{align*}
f'(x) = 0, \quad \forall x > a.
\end{align*}
此时结论显然成立。

若 \(f' \neq 0\),\(\forall x \in (0, +\infty)\),则由导数介值性可知,\(f'\) 在 \((0, +\infty)\) 上要么恒大于零,要么恒小于零。
于是不妨设 \(f' > 0\),\(\forall x \in (0, +\infty)\),故此时 \(f\) 在 \((0, +\infty)\) 上严格递增。
并且此时 \(f' = |f'|\) 在 \((0, +\infty)\) 递减,故此时 \(f'\) 在 \((0, +\infty)\) 内闭 Riemann 可积。
从而由微积分基本定理可知
\begin{align*}
\int_1^x f'(y) \, \mathrm{d}y = f(x) - f(1).
\end{align*}
又因为 \(\lim\limits_{x \to +\infty} f(x)\) 存在,所以 \(\int_1^{+\infty} f'(y) \, \mathrm{d}y\) 收敛。于是由\nrefpro{proposition:反常积分收敛的基本结论1}{(1)}可知 \(\lim\limits_{x \to +\infty} x f'(x) = 0\)。
\end{proof}

\begin{example}
设 \(f\) 在 \((a,+\infty)\) 可导。如果 \(f\) 有界且 \(xf'\) 为单调函数,证明
\begin{align*}
\lim_{x \to +\infty} x\ln x f'(x) = 0.
\end{align*}
\end{example}
\begin{proof}
由 \(xf'\) 单调可知,\(g(x)\triangleq xf'\) 在 \((a, +\infty)\) 上内闭 Riemann 可积。从而 \(f' = \frac{g(x)}{x}\) 在 \((a, +\infty)\) 上也内闭 Riemann 可积。
不妨设 \(xf'\) 单调递增,由单调有界定理可知,\(\lim\limits_{x \to +\infty}xf'(x)\) 存在或 \(+\infty\)。
由 \(f\) 有界可得 \(\lim\limits_{x \to +\infty}xf'(x)\leqslant 0\)。否则,若 \(\lim\limits_{x \to +\infty}xf'(x)>0\),则存在 \(C > 0\),使得
\begin{align}
xf'(x)>C \Rightarrow f'(x)>\frac{C}{x}, \quad x\in (\max\{a,0\}, +\infty). \label{eq::::1.1}
\end{align}
对 \((\ref{eq::::1.1})\) 式两边同时积分得到
\begin{align*}
f(x)>\int_a^x\frac{c}{t}\,\mathrm{d}t = c\ln|x|-c\ln a.
\end{align*}
令 \(x \to +\infty\),得到 \(\lim\limits_{x \to +\infty}f(x)=+\infty\),这与 \(f\) 有界矛盾!
于是由 \(\lim\limits_{x \to +\infty}xf'(x)\leqslant 0\) 可知存在 \(X > \max\{a,0\}\),使得
\begin{align*}
xf'(x)\leqslant 0 \Rightarrow f'(x)\leqslant 0, \quad x\in (X, +\infty).
\end{align*}
故 \(f\) 在 \((X, +\infty)\) 上递减。又因为 \(f\) 有界,所以 \(\lim\limits_{x \to +\infty}f(x)\) 存在。
根据微积分基本定理可得
\begin{align*}
\int_a^x f'(t)\,\mathrm{d}t = f(x)-f(a).
\end{align*}
令 \(x \to +\infty\),则由 \(\lim\limits_{x \to +\infty}f(x)\) 存在可得 \(\int_a^{+\infty}f'(t)\,\mathrm{d}t\) 收敛。又 \(xf'(x)\) 单调,于是由\nrefpro{proposition:反常积分收敛的基本结论1}{(2)}可知 \(\lim\limits_{x \to +\infty}x\ln x f'(x)=0\)。 

\end{proof}

\begin{example}
设
\(
f \in D[a, +\infty), \lim_{x \to +\infty} f'(x) = +\infty \text{ 且 } f' \text{ 严格递增},
\)
证明 \(\int_{0}^{\infty} \sin f(x) \, \mathrm{d}x\) 收敛.
\end{example}
\begin{proof}
由\refpro{proposition:导函数没有第一类间断点与无穷间断点}和\refpro{proposition:单调函数只有第一类间断点}可知,$f'\in C[a,+\infty)$.又由\refpro{proposition:导数有正增长率则函数爆炸}可知$\underset{x\rightarrow +\infty}{\lim}f\left( x \right) =+\infty$.故存在$X>0$,使得$f',f$在$[X,+\infty)$上恒正,且$f$在$[X,+\infty)$上严格单调递增.从而由\hyperref[theorem:反函数存在定理]{反函数存在定理}可知,$f$存在严格单调递增的反函数
\begin{align*}
g:[f(X),+\infty)\to [X,+\infty).
\end{align*}
于是令$x=g(y)$,则
\begin{align*}
\int_X^{+\infty}{\sin f\left( x \right) \mathrm{d}x}=\int_{f\left( X \right)}^{+\infty}{\sin y}g'\left( y \right) \mathrm{d}y.
\end{align*}
又由\hyperref[theorem:反函数求导定理]{反函数求导定理}可知$g'(y)f'(g(y))=1$,并且$f(g(y))=y$,故上式可化为
\begin{align*}
\int_X^{+\infty}{\sin f\left( x \right) \mathrm{d}x}=\int_{f\left( X \right)}^{+\infty}{\sin y}g' \left( y \right) \mathrm{d}y=\int_{f\left( X \right)}^{+\infty}{\frac{\sin y}{f'\left( g\left( y \right) \right)}}\mathrm{d}y.
\end{align*}
因为$f',g$都严格递增趋于$+\infty$,所以$\frac{1}{f'(g(x))}$严格递增趋于0.又注意到
\begin{align*}
\left| \int_{f\left( X \right)}^A{\sin y}\mathrm{d}y \right|\leqslant 2,\forall A\geqslant f\left( X \right) .
\end{align*}
故由Dirchlet判别法可知\(\int_{0}^{\infty} \sin f(x) \, \mathrm{d}x\) 收敛.
\end{proof}

\begin{example}
\begin{enumerate}[(1)]
\item 设 \( f \) 内闭可积且 \( f(x) > 0 \),\( x_0 > 0 \)。若
\[
\lim_{x \to +\infty} \frac{f(x + x_0)}{f(x)} = \ell \in [0, +\infty) \bigcup \{ +\infty \}
\]
我们就有
\[
\int_{a}^{\infty} f(x) dx \text{ 是 } 
\begin{cases} 
\text{收敛,} & \ell < 1 \\
\text{发散,} & \ell > 1 
\end{cases}.
\]

\item 设$f>0$内闭可积, 若有常数$k>1$使得
\begin{align*}
\lim_{x \to +\infty} \frac{f(kx)}{f(x)} = \ell \in [0, +\infty) \bigcup \{+\infty\},
\end{align*}
则
\begin{align*}
\int_a^{+\infty} f(x) \, dx \text{是} 
\begin{cases} 
\text{收敛}, & \ell < \tfrac{1}{k} \\
\text{发散}, & \ell > \tfrac{1}{k}
\end{cases}.
\end{align*}

\item 设$f>0$内闭可积, 若
\begin{align*}
\lim_{x \to +\infty} \frac{\ln f(x)}{\ln x} = p,
\end{align*}
则
\begin{align*}
\int_a^{+\infty} f(x) \, dx 
\begin{cases} 
\text{收敛}, & -\infty \leqslant p < -1 \\
\text{发散}, & -1 < p \leqslant +\infty
\end{cases}.
\end{align*}
\end{enumerate}
\end{example}
\begin{remark}
第(1)题中当$\ell =1$时无法判断反常积分的敛散性!

第(3)题中当$p =1$时无法判断反常积分的敛散性!
\end{remark}
\begin{remark}
第(3)题的条件$\lim_{x \to +\infty} \frac{\ln f(x)}{\ln x} = p$可改为$\underset{x\rightarrow +\infty}{\lim}\frac{xf' \left( x \right)}{f\left( x \right)}=p$.因为由L'Hospital法则可知
\begin{align*}
\lim_{x\rightarrow +\infty} \frac{\ln f(x)}{\ln x}=\underset{x\rightarrow +\infty}{\lim}\frac{xf' \left( x \right)}{f\left( x \right)}=p.
\end{align*}
\end{remark}
\begin{proof}
\begin{enumerate}[(1)]
\item 注意到
\begin{align*}
\int_a^{\infty}{f\left( x \right) \mathrm{d}x}=\sum_{n=1}^{\infty}{\int_{a+\left( n-1 \right) x_0}^{a+nx_0}{f\left( x \right) \mathrm{d}x}}\triangleq \sum_{n=1}^{\infty}{a_n}.
\end{align*}
对$\forall \varepsilon >0$,由题设可知,存在$X>a$,使得
\begin{align*}
\ell -\varepsilon \leqslant \frac{f\left( x_0+x \right)}{f\left( x \right)}\leqslant \ell +\varepsilon ,\forall x\geqslant X.
\end{align*}
从而当$n>\frac{X-a}{x_0}$时,就有$a+nx_0>X$,进而
\begin{align*}
a_{n+1}=\int_{a+nx_0}^{a+\left( n+1 \right) x_0}{f\left( x \right) \mathrm{d}x}=\int_{a+\left( n-1 \right) x_0}^{a+nx_0}{f\left( x+x_0 \right) \mathrm{d}x}\in \left[ \left( \ell -\varepsilon \right) a_n,\left( \ell +\varepsilon \right) a_n \right] ,
\end{align*}
故
\begin{align*}
\ell -\varepsilon \leqslant \frac{a_{n+1}}{a_n}\leqslant \ell +\varepsilon ,\forall n>\frac{X-a}{x_0}.
\end{align*}
因此$\underset{n\rightarrow \infty}{\lim}\frac{a_{n+1}}{a_n}=\ell$,再由比值判别法得证.

\item 根据题设,令$x=e^t$,任取$c>0$,则
\begin{align*}
\int_c^{\infty}{f\left( x \right) \mathrm{d}x}=\int_{\ln c}^{\infty}{f\left( e^t \right) e^t\mathrm{d}t}.
\end{align*}
\begin{align*}
\underset{t\rightarrow +\infty}{\lim}\frac{f\left( e^{t+\ln k} \right) e^{t+\ln k}}{f\left( e^t \right) e^t}=k\underset{t\rightarrow +\infty}{\lim}\frac{f\left( ke^t \right)}{f\left( e^t \right) e^t}=k\ell .
\end{align*}
于是由$\eqref{1}$可知结论成立.

\item 只讨论$p\in \mathbb{R}$的情况,其余$p=\pm \infty$情况类似.由题意可知,$\forall \varepsilon >0,\text{存在}X>e,\text{使得当}x>X\text{时},\text{有}$
\begin{align*}
p-\varepsilon \leqslant \frac{\ln f\left( x \right)}{\ln x}\leqslant p+\varepsilon \Longleftrightarrow x^{p-\varepsilon}\leqslant f\left( x \right) \leqslant x^{p+\varepsilon}.
\end{align*}
于是
\begin{align*}
\frac{1}{x^{-p+\varepsilon}}\leqslant f\left( x \right) \leqslant \frac{1}{x^{-p-\varepsilon}},\forall x>X.
\end{align*}
再由比较判别法即得结论.
\end{enumerate}
\end{proof}
\begin{remark}
上述例题第(3)题的证明中,令$\varepsilon \to 0$,并不能得到
\begin{align*}
\frac{1}{x^{-p}}\leqslant f(x)\leqslant \frac{1}{x^{-p}},\forall x>X.
\end{align*}
因为$X$是与$\varepsilon$有关的.因此只有固定$\varepsilon$时,才有$\frac{1}{x^{-p+\varepsilon}}\leqslant f\left( x \right) \leqslant \frac{1}{x^{-p-\varepsilon}},\forall x>X$成立.故再利用比较判别法式,不能令$\varepsilon\to 0$.
\end{remark}

\begin{example}
若$f \in C^1[0, +\infty)$且$f(0) > 0$, $f'(x) > 0$. 若$\int_0^{\infty} \frac{1}{f(x) + f'(x)} dx < \infty$, 证明$\int_0^{\infty} \frac{1}{f(x)} dx < \infty$.
\end{example}
\begin{note}
利用拟合法的想法证明反常积分收敛.
\end{note}
\begin{proof}
由条件可知$f(x)$严格递增且恒正,从而$f(+\infty) \in \mathbb{R}^+ \cup \{+\infty\}$,进而
\begin{align*}
\frac{1}{f(+\infty)} \in \mathbb{R}^+ \cup \{0\}.
\end{align*}
于是
\begin{align*}
\int_0^{\infty} \left| \frac{1}{f(x) + f'(x)} - \frac{1}{f(x)} \right| \, dx = \int_0^{\infty} \frac{f'(x)}{f(x) \left[ f(x) + f'(x) \right]} \, dx \leqslant \int_0^{\infty} \frac{f'(x)}{f^2(x)} \, dx = \int_0^{\infty} \frac{1}{f^2(x)} \, df(x) = \frac{1}{f(0)} - \frac{1}{f(+\infty)} < +\infty.
\end{align*}
注意到
\begin{align*}
\frac{1}{f(x)} \leqslant \left| \frac{1}{f(x)} - \frac{1}{f(x) + f'(x)} \right| + \frac{1}{f(x) + f'(x)}.
\end{align*}
又$\int_0^{\infty} \frac{1}{f(x) + f'(x)} \, dx < +\infty$,故$\int_0^{\infty} \frac{1}{f(x)} \, dx < \infty$.
\end{proof}

\begin{example}
设非负函数$f \in C(\mathbb{R})$使得对任何$k \in \mathbb{N}$都有$\int_{-\infty}^{\infty} e^{-\frac{|x|}{k}} f(x) dx \leqslant M$, 证明$\int_{-\infty}^{\infty} f(x) dx$收敛且$\int_{-\infty}^{\infty} f(x) dx \leqslant M$.
\end{example}
\begin{note}
利用拟合法的想法证明反常积分收敛.
\end{note}
\begin{proof}
{\color{blue} 证法一:}$\forall a < b$, 注意到 $1 - x \leqslant e^{-x}, \forall x \in \mathbb{R}$. 从而对 $\forall k \in \mathbb{N}$, 都有
\begin{align*}
\int_a^b \left( 1 - \frac{|x|}{k} \right) f(x) \, \mathrm{d}x \leqslant \int_a^b e^{-\frac{|x|}{k}} f(x) \, \mathrm{d}x \leqslant M.
\end{align*}
于是
\begin{align*}
\int_a^b f(x) \, \mathrm{d}x \leqslant \frac{1}{k} \int_a^b |x| f(x) \, \mathrm{d}x + M.
\end{align*}
令 $k \to +\infty$ 得 $\int_a^b f(x) \, \mathrm{d}x \leqslant M$. 再由 $a, b$ 的任意性可得 $\int_{-\infty}^{+\infty} f(x) \, \mathrm{d}x \leqslant M$.

{\color{blue} 证法二:}由\hyperref[Real Analysis-lemma:Fatou引理]{Fatou引理}可得
\begin{align*}
\int_{-\infty}^{+\infty}{f\left( x \right) \mathrm{d}x}=\int_{-\infty}^{+\infty}{\underset{k\rightarrow +\infty}{\underline{\lim }}e^{-\frac{\left| x \right|}{k}}f\left( x \right) \mathrm{d}x}\leqslant \underset{k\rightarrow +\infty}{\underline{\lim }}\int_{-\infty}^{+\infty}{e^{-\frac{\left| x \right|}{k}}f\left( x \right) \mathrm{d}x}\leqslant M.
\end{align*}
\end{proof}

\begin{example}
设$f \in C^1[0, +\infty)$ 满足
\begin{align*}
|f'(x)| \leqslant M, \forall x \geqslant 0, \int_0^{\infty} |f(x)|^2 dx < \infty.
\end{align*}
证明
\begin{align*}
\lim_{x \to +\infty} f(x) = 0.
\end{align*}
\end{example}
\begin{proof}
由条件可得
\begin{align*}
\int_0^{+\infty} \left| f^2(x) f'(x) \right| \, \mathrm{d}x \leqslant M \int_0^{+\infty} \left| f(x) \right|^2 \, \mathrm{d}x < +\infty.
\end{align*}
故$\int_0^{+\infty} f^2(x) f'(x) \, \mathrm{d}x$收敛. 于是
\begin{align*}
\int_0^{+\infty} f^2(x) f'(x) \, \mathrm{d}x = \lim_{x \to +\infty} f^3(x) - f^3(0) < \infty.
\end{align*}
从而$\lim_{x \to +\infty} f^3(x)$存在. 由$\int_0^{+\infty} \left| f(x) \right|^2 \, \mathrm{d}x < \infty$及\nrefpro{proposition:积分收敛必有子列趋于0}{(1)}可知, 存在$\{x_n\}$, 满足$x_n \to +\infty$, 使得$f(x_n) \to 0$.
故$\lim_{x \to +\infty} f^3(x) = \lim_{n \to \infty} f^3(x_n) = 0$, 因此$\lim_{x \to +\infty} f(x) = 0$.
\end{proof}

\begin{example}
设 $f \in D^2[0, +\infty)$ 且
\begin{align*}
\int_0^{\infty} |f(x)|^2 dx < \infty, \int_0^{\infty} |f''(x)|^2 dx < \infty.
\end{align*}
证明$\int_0^{\infty} |f'(x)|^2 dx < \infty$.
\end{example}
\begin{proof}
由Cauchy不等式得
\begin{align*}
\int_0^{+\infty}|f(x)f''(x)|\,\mathrm{d}x\leqslant\sqrt{\int_0^{+\infty}|f(x)|^2\,\mathrm{d}x\int_0^{+\infty}|f''(x)|^2\,\mathrm{d}x}<+\infty.
\end{align*}
故$\int_0^{+\infty}|f(x)f''(x)|\,\mathrm{d}x$收敛.利用分部积分得
\begin{align}\label{100.112}
\int_0^x|f'(y)|^2\,\mathrm{d}y=f(x)f'(x)-f(0)f'(0)-\int_0^x f(y)f''(y)\,\mathrm{d}y
\end{align}
由\refpro{proposition:反常积分收敛的基本结论2}可知,只须找一个$x_n\rightarrow+\infty$,使$f(x_n)f'(x_n)$极限存在即可.

由于$\int_0^{\infty}|f(x)|^2\,\mathrm{d}x<+\infty$,故由\nrefpro{proposition:积分收敛必有子列趋于0}{(1)}可知存在$a_n\rightarrow+\infty$,使得$\lim_{n\rightarrow\infty}|f(a_n)|^2=0$,从而$\lim_{x\rightarrow+\infty}|f(x)|^2\ne+\infty$.
于是再由\refpro{proposition:函数不爆破则各阶导数必然有趋于 0 的子列}可知,存在$x_n\rightarrow+\infty$,使得
\begin{align*}
\lim_{n\rightarrow\infty}[f^2(x_n)]'=0\Longleftrightarrow\lim_{n\rightarrow\infty}2f(x_n)f'(x_n)=0.
\end{align*}
从而由\refpro{proposition:反常积分收敛的基本结论2}及\eqref{100.112}式可知结论成立.
\end{proof}

\begin{example}

\end{example}
\begin{proof}

\end{proof}





















\end{document}
\documentclass[../../main.tex]{subfiles}
\graphicspath{{\subfix{../../image/}}} % 指定图片目录,后续可以直接使用图片文件名。

% 例如:
% \begin{figure}[H]
% \centering
% \includegraphics{image-01.01}
% \caption{图片标题}
% \label{figure:image-01.01}
% \end{figure}
% 注意:上述\label{}一定要放在\caption{}之后,否则引用图片序号会只会显示??.

\begin{document}

\section{反常积分敛散性判别}

\begin{theorem}[Cauchy收敛准则]
广义积分 \(\int_{a}^{\infty} f(x) \mathrm{d}x\) 收敛等价于对任意 \(\varepsilon > 0\),存在 \(A > a\) 使得任意 \(x_1,x_2 > A\) 都有 \(\left|\int_{x_1}^{x_2} f(t) \mathrm{d}t\right| < \varepsilon\)。 
\end{theorem}

\begin{theorem}[A-D判别法]
设 \(f(x),g(x)\) 在任何闭区间上黎曼可积,
\begin{enumerate}
\item {\heiti Abel判别法}:若 \(\int_{a}^{+\infty} f(x) \mathrm{d}x\) 收敛,并且$g(x)$在$[a,+\infty)$上单调有界,则 \(\int_{a}^{\infty} f(x)g(x) \mathrm{d}x\) 收敛.

\item {\heiti Dirichlet判别法}:若 \(\int_{a}^{x} f(x) \mathrm{d}x\)在$[a,+\infty)$上有界,并且\(g(x)\)在\([a,+\infty)\)上单调,$\underset{x\rightarrow +\infty}{\lim}g\left( x \right) =0$,则 \(\int_{a}^{\infty} f(x)g(x) \mathrm{d}x\) 收敛.
\end{enumerate}
\end{theorem}

\begin{example}
设 \(f(x)\) 在 \([0,+\infty)\) 中非负且递减,证明:\(\int_{0}^{+\infty} f(x) \mathrm{d}x\),\(\int_{0}^{+\infty} f(x)\sin^{2}x \mathrm{d}x\) 同敛散性. 
\end{example}
\begin{proof}
(i) 若 \(\int_0^\infty f(x) \, \mathrm{d}x < \infty\),则由条件可知
\begin{align*}
f(x) \sin^2 x \leqslant f(x), \quad \forall x \in [0, +\infty).
\end{align*}
故由比较判别法可得 \(\int_0^\infty f(x) \sin^2 x \, \mathrm{d}x < \infty\)。

(ii) 若 \(\int_0^\infty f(x) \sin^2 x \, \mathrm{d}x < \infty\),则由 \(f\) 非负递减,故 \(\lim_{x \to +\infty} f(x)\) 存在且 \(\lim_{x \to +\infty} f(x) \geqslant 0\)。若 \(\lim_{x \to +\infty} f(x) \triangleq a > 0\),则存在 \(M > 0\),使得
\begin{align}
f(x) \sin^2 x > \frac{a}{2} \sin^2 x, \quad \forall x \in [M, +\infty). \label{eq:100.31}
\end{align}
又因为
\begin{align*}
\int_0^\infty \sin^2 x \, \mathrm{d}x = \lim_{b \to +\infty} \int_0^b \frac{1 - \cos 2x}{2} \, \mathrm{d}x = \frac{1}{2} \lim_{b \to +\infty} \left( b - \frac{\sin 2b}{2} \right),
\end{align*}
而上式右边极限不存在,所以 \(\int_0^\infty \sin^2 x \, \mathrm{d}x\) 发散。从而结合 \eqref{eq:100.31} 式,由比较判别法可知 \(\int_0^\infty f(x) \sin^2 x \, \mathrm{d}x\) 发散,矛盾!故 \(\lim_{x \to +\infty} f(x) = 0\)。

注意到
\begin{align*}
\int_0^\infty f(x) \sin^2 x \, \mathrm{d}x = \frac{1}{2} \int_0^\infty f(x) (1 - \cos 2x) \, \mathrm{d}x < \infty.
\end{align*}
即 \(\int_0^\infty f(x) (1 - \cos 2x) \, \mathrm{d}x < \infty\)。考虑 \(\int_0^\infty f(x) \cos 2x \, \mathrm{d}x\),注意到
\begin{align*}
\int_0^C \cos 2x \, \mathrm{d}x = \frac{\sin 2C}{2} < 1, \quad \forall C > 0.
\end{align*}
又由于 \(f(x)\) 在 \([0, +\infty)\) 上单调递减趋于 0,故由狄利克雷判别法可知 \(\int_0^\infty f(x) \cos 2x \, \mathrm{d}x < \infty\)。因此
\begin{align*}
\int_0^\infty f(x) \, \mathrm{d}x = \int_0^\infty f(x) (1 - \cos 2x) \, \mathrm{d}x + \int_0^\infty f(x) \cos 2x \, \mathrm{d}x < \infty.
\end{align*}

(iii)当\(\int_0^\infty f(x) \, \mathrm{d}x\) 或 \(\int_0^\infty f(x) \sin^2 x \, \mathrm{d}x\) 发散时,实际上,\(\int_0^\infty f(x) \, \mathrm{d}x\) 或 \(\int_0^\infty f(x) \sin^2 x \, \mathrm{d}x\) 发散的情形就是 (i)(ii) 的逆否命题。故结论得证。
\end{proof}

\begin{example}
设\(f(x)\)在\(\mathbb{R}\)上非负连续,对任意正整数\(k\)有$\int_{-\infty}^{+\infty} e^{-\frac{|x|}{k}}f(x)dx \leq 1,$
证明:$\int_{-\infty}^{+\infty}f(x)dx \leq 1.$
\end{example}
\begin{remark}
实际上,由实变函数相关结论可直接得到
\begin{align*}
\lim_{k\rightarrow \infty}\int_{\mathbb{R}}e^{-\frac{| x |}{k}}f( x ) \mathrm{d}x=\int_{\mathbb{R}}\left[ \lim_{k\rightarrow \infty}e^{-\frac{| x |}{k}}f( x ) \right] \mathrm{d}x=\int_{\mathbb{R}}f( x ) \mathrm{d}x.
\end{align*}
\end{remark}
\begin{proof}
由条件可得,对$\forall A>0$,我们有
\begin{align*}
1\geqslant \int_{-A}^Ae^{-\frac{| x |}{k}}f( x ) \mathrm{d}x\geqslant e^{-\frac{1}{k}}\int_{-A}^Af( x ) \mathrm{d}x.\Rightarrow \int_{-A}^Af( x ) \mathrm{d}x\leqslant e^{-\frac{1}{k}},\forall k\in \mathbb{N} .
\end{align*}
令$k\rightarrow \infty$,则$\int_{-A}^Af( x ) \mathrm{d}x\leqslant 1,\forall A>0$。于是再令$A\rightarrow +\infty$,可得$\int_{\mathbb{R}}f( x ) \mathrm{d}x\leqslant 1.$

实际上再由单调有界可知$\int_{\mathbb{R}}f( x ) \mathrm{d}x$收敛。
\end{proof}

\begin{example}
对实数\(a\),讨论$\int_{0}^{\infty}\frac{x}{\cos^{2}x + x^{a}\sin^{2}x}\mathrm{d}x$
的敛散性. 
\end{example}
\begin{proof}
先讨论$\int_0^1\frac{x}{\cos^2x+x^a\sin^2x}\mathrm{d}x$的敛散性. 注意到
\begin{align*}
\int_0^1\frac{x}{\cos^2x+x^a\sin^2x}\mathrm{d}x\leqslant \int_0^1\frac{1}{\cos^2x}\mathrm{d}x=\tan x\mid_{0}^{1}=\tan 1<\infty,\quad \forall a\in \mathbb{R}.
\end{align*}
故$\forall a\in \mathbb{R}$,都有$\int_0^1\frac{x}{\cos^2x+x^a\sin^2x}\mathrm{d}x$收敛.
再讨论$\int_1^{\infty}\frac{x}{\cos^2x+x^a\sin^2x}\mathrm{d}x$的敛散性.
\begin{enumerate}[(i)]
\item 当$a\leqslant 2$时,
\begin{align*}
\int_1^{\infty}\frac{x}{\cos^2x+x^a\sin^2x}\mathrm{d}x\geqslant \int_1^{\infty}\frac{x}{\cos^2x+x^2\sin^2x}\mathrm{d}x\geqslant \int_1^{\infty}\frac{x}{x^2+1}\mathrm{d}x=\frac{1}{2}\int_1^{\infty}\frac{1}{x^2+1}\mathrm{d}(x^2+1)=+\infty.
\end{align*}
\item 当$a>2$时,我们有(等价关系直观上是显然的,可由拟合法或放缩严谨证明)
\begin{align}
\int_{n\pi}^{(n+1)\pi}\frac{x}{\cos^2x+x^a\sin^2x}\mathrm{d}x=\int_0^{\pi}\frac{x+n\pi}{\cos^2x+(x+n\pi)^a\sin^2x}\mathrm{d}x\sim n\pi \int_0^{\pi}\frac{1}{\cos^2x+(n\pi)^a\sin^2x}\mathrm{d}x,\quad n\rightarrow \infty.\label{eq:100.34}
\end{align}
注意到对$\forall \lambda >0$,我们都有
\begin{align*}
\int_0^{\pi}\frac{1}{\cos^2x+\lambda \sin^2x}\mathrm{d}x=2\int_0^{\frac{\pi}{2}}\frac{1}{\cos^2x+\lambda \sin^2x}\mathrm{d}x=2\int_0^{\frac{\pi}{2}}\frac{1}{1+\lambda \tan^2x}\cdot \frac{1}{\cos^2x}\mathrm{d}x=2\int_0^{\infty}\frac{1}{1+\lambda t^2}\mathrm{d}t=\frac{\pi}{\sqrt{\lambda}}.
\end{align*}
故再结合\eqref{eq:100.34}式可知
\begin{align*}
\int_{n\pi}^{(n+1)\pi}\frac{x}{\cos^2x+x^a\sin^2x}\mathrm{d}x\sim n\pi \int_0^{\pi}\frac{1}{\cos^2x+(n\pi)^a\sin^2x}\mathrm{d}x\sim n\pi \frac{\pi}{(n\pi)^{\frac{a}{2}}}\sim \frac{1}{n^{\frac{a}{2}-1}},\quad n\rightarrow \infty.
\end{align*}
于是
\begin{align*}
\int_1^{\infty}\frac{x}{\cos^2x+x^a\sin^2x}\mathrm{d}x\sim \int_{\pi}^{\infty}\frac{x}{\cos^2x+x^a\sin^2x}\mathrm{d}x=\sum_{n=1}^{\infty}\int_{n\pi}^{(n+1)\pi}\frac{x}{\cos^2x+x^a\sin^2x}\mathrm{d}x\sim \sum_{n=1}^{\infty}\frac{1}{n^{\frac{a}{2}-1}},\quad n\rightarrow \infty.
\end{align*}
从而当$\frac{a}{2}-1\leqslant 1$时,即$2<a\leqslant 4$,$\int_1^{\infty}\frac{x}{\cos^2x+x^a\sin^2x}\mathrm{d}x$发散;当$\frac{a}{2}-1>1$,即$a>4$时,$\int_1^{\infty}\frac{x}{\cos^2x+x^a\sin^2x}\mathrm{d}x$收敛.
\end{enumerate}

综上,当$a>4$时,$\int_0^{\infty}\frac{x}{\cos^2x+x^a\sin^2x}\mathrm{d}x$收敛;当$a\leqslant 4$时,$\int_0^{\infty}\frac{x}{\cos^2x+x^a\sin^2x}\mathrm{d}x$发散.
\end{proof}

\begin{example}
对正整数\(n\),讨论$\int_{0}^{+\infty}x^{n}e^{-x^{12}\sin^{2}x}dx$
的敛散性. 
\end{example}
\begin{proof}
注意到
\begin{align}\label{equation:100.35}
\int_{k\pi}^{(k+1)\pi}x^ne^{-x^{12}\sin^2x}\mathrm{d}x=\int_0^{\pi}(x+k\pi)^ne^{-(x+k\pi)^{12}\sin^2x}\mathrm{d}x\sim (k\pi)^n\int_0^{\pi}e^{-(x+k\pi)^{12}\sin^2x}\mathrm{d}x,\quad k\rightarrow \infty.
\end{align}
又注意到
\begin{align*}
\int_0^{\pi}e^{-\lambda \sin^2x}\mathrm{d}x=2\int_0^{\frac{\pi}{2}}e^{-\lambda \sin^2x}\mathrm{d}x\geqslant 2\int_0^{\frac{\pi}{2}}e^{-\lambda x^2}\mathrm{d}x=\frac{2}{\sqrt{\lambda}}\int_0^{\frac{\pi}{2}\sqrt{\lambda}}e^{-x^2}\mathrm{d}x\sim \sqrt{\frac{\pi}{\lambda}},\quad \lambda \rightarrow +\infty,
\end{align*}
\begin{align*}
\int_0^{\pi}e^{-\lambda \sin^2x}\mathrm{d}x=2\int_0^{\frac{\pi}{2}}e^{-\lambda \sin^2x}\mathrm{d}x\leqslant 2\int_0^{\frac{\pi}{2}}e^{-\lambda \frac{4}{\pi^2}x^2}\mathrm{d}x=\frac{\pi}{\sqrt{\lambda}}\int_0^{\sqrt{\lambda}}e^{-x^2}\mathrm{d}x\sim \frac{\pi \sqrt{\pi}}{2\sqrt{\lambda}},\quad \lambda \rightarrow +\infty.
\end{align*}
故$\int_0^{\pi}e^{-\lambda \sin^2x}\mathrm{d}x\sim \frac{C}{\sqrt{\lambda}}$,$\lambda \rightarrow +\infty$,其中$C$为某一常数. 因此
\begin{align*}
\int_0^{\pi}e^{-(k\pi)^{12}\sin^2x}\mathrm{d}x\sim \frac{C}{(k\pi)^6},\quad k\rightarrow +\infty,
\end{align*}
\begin{align*}
\int_0^{\pi}e^{-[(k+1)\pi]^{12}\sin^2x}\mathrm{d}x\sim \frac{C}{[(k+1)\pi]^6},\quad k\rightarrow +\infty.
\end{align*}
又因为
\begin{align*}
\int_0^{\pi}e^{-[(k+1)\pi]^{12}\sin^2x}\mathrm{d}x\leqslant \int_0^{\pi}e^{-(x+k\pi)^{12}\sin^2x}\mathrm{d}x\leqslant \int_0^{\pi}e^{-(k\pi)^{12}\sin^2x}\mathrm{d}x,
\end{align*}
所以$\int_0^{\pi}e^{-(x+k\pi)^{12}\sin^2x}\mathrm{d}x\sim \frac{C_1}{k^6}$,$k\rightarrow +\infty$,其中$C_1$为某一常数. 于是结合\eqref{equation:100.35}式可知
\begin{align*}
\int_{k\pi}^{(k+1)\pi}x^ne^{-x^{12}\sin^2x}\mathrm{d}x=\int_0^{\pi}(x+k\pi)^ne^{-(x+k\pi)^{12}\sin^2x}\mathrm{d}x\sim (k\pi)^n\int_0^{\pi}e^{-(x+k\pi)^{12}\sin^2x}\mathrm{d}x\sim C_2k^{n-6},\quad k\rightarrow \infty.
\end{align*}
其中$C_2$为某一常数. 因此
\begin{align*}
\int_{\pi}^{\infty}x^ne^{-x^{12}\sin^2x}\mathrm{d}x=\sum_{k=1}^{\infty}\int_{k\pi}^{(k+1)\pi}x^ne^{-x^{12}\sin^2x}\mathrm{d}x\sim \sum_{k=1}^{\infty}C_2k^{n-6},\quad k\rightarrow \infty.
\end{align*}
故当$n<5$时,$\int_{\pi}^{\infty}x^ne^{-x^{12}\sin^2x}\mathrm{d}x$收敛;当$n\geqslant 5$时,$\int_{\pi}^{\infty}x^ne^{-x^{12}\sin^2x}\mathrm{d}x$发散. 又因为
\begin{align*}
\int_0^{\pi}x^ne^{-x^{12}\sin^2x}\mathrm{d}x\leqslant \pi^n,
\end{align*}
所以$\int_0^{\pi}x^ne^{-x^{12}\sin^2x}\mathrm{d}x$对$\forall n\in \mathbb{N}$都收敛. 从而由
\begin{align*}
\int_0^{\infty}x^ne^{-x^{12}\sin^2x}\mathrm{d}x=\int_0^{\pi}x^ne^{-x^{12}\sin^2x}\mathrm{d}x+\int_{\pi}^{\infty}x^ne^{-x^{12}\sin^2x}\mathrm{d}x,
\end{align*}
可知当$n<5$时,$\int_0^{\infty}x^ne^{-x^{12}\sin^2x}\mathrm{d}x$收敛;当$n\geqslant 5$时,$\int_0^{\infty}x^ne^{-x^{12}\sin^2x}\mathrm{d}x$发散.
\end{proof}

\begin{lemma}\label{lemma:cos^2nx和cos^2n+1x的展开形式}
(1) $\cos ^{2n+1}x$可以写成$\cos x,\cos 3x,\cdots ,\cos (2n+1)x$的线性组合 ,即$\cos ^{2n+1}x\in L(\cos x,\cos 3x,\cdots ,\cos (2n+1)x)$ ,也即$\cos ^{2n+1}x=\sum_{k=0}^{n}{a_k\cos (2k+1)x}$ ,其中$a_k\in \mathbb{R}$ ,$k=0,1,\cdots ,n$.

(2) $\cos ^{2n}x=\frac{1}{2^{2n-1}}\sum_{k=0}^{n-1}{\mathrm{C}_{2n}^{k}\cos 2(n-k)x}+\frac{\mathrm{C}_{2n}^{n}}{2^{2n}}$.
\end{lemma}
\begin{proof}
(1) 利用数学归纳法,当$n=1$时,结论显然成立.假设结论对$n-1$成立,则  
\begin{align*}
\cos ^{2n+1}x&=\cos ^2x\cdot \cos ^{2n-1}x=\frac{1+\cos 2x}{2}\cdot \sum_{k=0}^{n-1}{a_k\cos \left( 2k+1 \right) x}
\\
&=\frac{1}{2}\sum_{k=0}^{n-1}{a_k\cos \left( 2k+1 \right) x}+\frac{1}{2}\sum_{k=0}^{n-1}{a_k\cos 2x\cos \left( 2k+1 \right) x}
\\
&=\frac{1}{2}\sum_{k=0}^{n-1}{a_k\cos \left( 2k+1 \right) x}+\frac{1}{2}\sum_{k=0}^{n-1}{a_k\left[ \cos \left( 2k+3 \right) x+\cos \left( 2k-1 \right) x \right]}
\\
&=\frac{1}{2}\sum_{k=0}^{n-1}{a_k\cos \left( 2k+1 \right) x}+\frac{1}{2}\sum_{k=1}^{n-1}{a_k\left[ \cos \left( 2k+5 \right) x+\cos \left( 2k+1 \right) x \right]}+\frac{1}{2}a_0\left[ \cos 3x+\cos \left( -x \right) \right] 
\\
&=\frac{1}{2}\sum_{k=0}^{n-1}{a_k\cos \left( 2k+1 \right) x}+\frac{1}{2}\sum_{k=1}^{n-1}{a_k\left[ \cos \left( 2k+5 \right) x+\cos \left( 2k+1 \right) x \right]}+\frac{1}{2}a_0\left[ \cos 3x+\cos x \right] .
\end{align*}  
故$\cos ^{2n+1}x\in L(\cos x,\cos 3x,\cdots ,\cos (2n+1)x)$  

(2) 由二项式定理可得  
$$
(1+t^2)^{2n}=\sum_{k=0}^{2n}{\mathrm{C}_{2n}^{k}t^{2k}}
$$  
令$t=e^{\mathrm{i}x}$,则  
\begin{align*}
(1+e^{2\mathrm{i}x})^{2n}&=\sum_{k=0}^{2n}{\mathrm{C}_{2n}^{k}e^{2\mathrm{i}kx}} \Rightarrow 2^{2n}\left( \frac{e^{-\mathrm{i}x}+e^{\mathrm{i}x}}{2e^{-\mathrm{i}x}} \right) ^{2n}=\sum_{k=0}^{2n}{\mathrm{C}_{2n}^{k}e^{2\mathrm{i}kx}} \Rightarrow 2^{2n}\left( \frac{e^{-\mathrm{i}x}+e^{\mathrm{i}x}}{2} \right) ^{2n}=e^{-2\mathrm{i}nx}\sum_{k=0}^{2n}{\mathrm{C}_{2n}^{k}e^{2\mathrm{i}kx}}\\
&\Rightarrow 2^{2n}\cos ^{2n}x=\sum_{k=0}^{2n}{\mathrm{C}_{2n}^{k}e^{2\mathrm{i}(k-n)x}}=\sum_{k=0}^{n-1}[\mathrm{C}_{2n}^{k}e^{2\mathrm{i}(k-n)x}+\mathrm{C}_{2n}^{2n-k}e^{2\mathrm{i}((2n-k)-n)x}]+\mathrm{C}_{2n}^{n}\\
&=\sum_{k=0}^{n-1}\mathrm{C}_{2n}^{k}(e^{2\mathrm{i}(k-n)x}+e^{2\mathrm{i}(n-k)x})+\mathrm{C}_{2n}^{n}\\
&\Rightarrow 2^{2n}\cos ^{2n}x=2\sum_{k=0}^{n-1}\mathrm{C}_{2n}^{k}\left( \frac{e^{2\mathrm{i}(k-n)x}+e^{2\mathrm{i}(n-k)x}}{2} \right)+\mathrm{C}_{2n}^{n}=2\sum_{k=0}^{n-1}\mathrm{C}_{2n}^{k}\cos 2(n-k)x+\mathrm{C}_{2n}^{n}\\
&\Rightarrow \cos ^{2n}x=\frac{1}{2^{2n-1}}\sum_{k=0}^{n-1}\mathrm{C}_{2n}^{k}\cos 2(n-k)x+\frac{\mathrm{C}_{2n}^{n}}{2^{2n}}
\end{align*}
\end{proof}

\begin{example}
设\(p,q\)为正整数,求反常积分$I(p,q)=\int_{0}^{+\infty}\frac{\cos^{p}x - \cos^{q}x}{x}dx$
收敛的充要条件.
\end{example}
\begin{proof}
因为当$p=q$时,积分显然收敛,所以只需考虑$p\ne q$的情形. 由$I(q,p) =-I(p,q)$可知,可以不妨设$p>q$,否则用$I(q,p) =-I(p,q)$代替$I(p,q)$即可

先讨论$\int_0^1{\frac{\cos ^px-\cos ^qx}{x}\mathrm{d}x}$的敛散性.由Taylor定理可知,对$\forall \varepsilon \in (0,1)$,存在$\delta >0$ ,使得  
\[
-\frac{x^2}{2}-\varepsilon x^2\leqslant \cos x\leqslant 1-\frac{x^2}{2}+\varepsilon x^2 ,\quad \forall x\in [0,\delta].
\]  
于是  
\begin{align*}
\int_0^1{\frac{\cos ^px-\cos ^qx}{x}\mathrm{d}x}&=\int_0^{\delta}{\frac{\cos ^px-\cos ^qx}{x}\mathrm{d}x}+\int_{\delta}^1{\frac{\cos ^px-\cos ^qx}{x}\mathrm{d}x}\\
&\leqslant \int_0^{\delta}{\frac{(1-\frac{x^2}{2}+\varepsilon x^2)^p-(1-\frac{x^2}{2}-\varepsilon x^2)^q}{x}\mathrm{d}x}+\frac{2}{\delta}(1-\delta)\\
&\leqslant \int_0^{\delta}{\frac{\frac{q-p+(p-q)\varepsilon}{2}x^2+(p+q)\mathrm{C}_{p}^{2}x^4}{x}\mathrm{d}x}+\frac{2}{\delta}(1-\delta)\\
&=\frac{q-p+(p-q)\varepsilon}{4}\delta +\frac{(p+q)\mathrm{C}_{p}^{2}}{4}\delta +\frac{2}{\delta}(1-\delta).
\end{align*}  
令$\varepsilon \rightarrow 0^+$,得$\int_0^1{\frac{\cos ^px-\cos ^qx}{x}\mathrm{d}x}\leqslant \frac{q-p}{4}\delta +\frac{(p+q)\mathrm{C}_{p}^{2}}{4}\delta +\frac{2}{\delta}(1-\delta)$.故对$\forall p>q$且$p,q\in \mathbb{N}$,都有$\int_0^1{\frac{\cos ^px-\cos ^qx}{x}\mathrm{d}x}$收敛.

再讨论$\int_1^{\infty}{\frac{\cos ^px-\cos ^qx}{x}\mathrm{d}x}$的敛散性.

(i) 当$p,q$都是奇数时,由\reflem{lemma:cos^2nx和cos^2n+1x的展开形式}可知  
$$
\cos ^px=\sum_{k=1}^p{p_k\cos kx} ,\quad \text{其中}p_k\in \mathbb{R} , k=1,2,\cdots,p.
$$  
$$
\cos ^qx=\sum_{k=1}^q{q_k\cos kx} ,\quad \text{其中}q_k\in \mathbb{R} , k=1,2,\cdots,q.
$$  
从而此时  
\begin{align*}
\int_1^{\infty}{\frac{\cos ^px-\cos ^qx}{x}\mathrm{d}x}&=\int_1^{\infty}{\frac{\sum\limits_{k=1}^p{p_k\cos kx}-\sum\limits_{k=1}^q{q_k\cos kx}}{x}\mathrm{d}x}\\
&=\sum\limits_{k=1}^q{(p_k-q_k)\int_1^{\infty}{\frac{\cos kx}{x}\mathrm{d}x}}+\sum\limits_{k=q+1}^p{p_k\int_1^{\infty}{\frac{\cos kx}{x}\mathrm{d}x}} .\label{100.36}
\end{align*}  
注意到对$\forall k\in \mathbb{N}$ 都有  
$$
\int_1^x{\cos kt\mathrm{d}t}=\frac{\sin kx-\sin k}{k}<2 ,\quad \forall x>1.
$$  
并且$\frac{1}{x}$在$[1,+\infty)$上单调递减趋于0,故由Dirichlet判别法可知,$\int_1^{\infty}{\frac{\cos kx}{x}\mathrm{d}x}(k\in \mathbb{N})$都收敛.因此再结合\eqref{100.36}式可知,$\int_1^{\infty}{\frac{\cos ^px-\cos ^qx}{x}\mathrm{d}x}$收敛 . 

(ii) 当$p,q$中至少有一个是偶数时,不妨设$p$是偶数 $q$不是偶数 ,则由\reflem{lemma:cos^2nx和cos^2n+1x的展开形式}可知  
\[
\cos ^px=\frac{1}{2^{p-1}}\sum_{k=0}^{\frac{p}{2}-1}{\mathrm{C}_{p}^{k}\cos 2\left( \frac{p}{2}-k \right) x}+\frac{\mathrm{C}_{p}^{\frac{p}{2}}}{2^p}.
\]  
\[
\cos ^qx=\sum_{k=1}^q{q_k\cos kx} \quad \text{其中}q_k\in \mathbb{R} \quad k=1,2,\cdots,q.
\] 
于是
\begin{align*}
\int_1^{\infty}{\frac{\cos ^px-\cos ^qx}{x}\mathrm{d}x}&=\int_1^{\infty}{\frac{\frac{1}{2^{p-1}}\sum\limits_{k=0}^{\frac{p}{2}-1}{\mathrm{C}_{p}^{k}\cos 2\left( \frac{p}{2}-k \right) x}-\sum\limits_{k=1}^q{q_k\cos kx}+\frac{\mathrm{C}_{p}^{\frac{p}{2}}}{2^p}}{x}\mathrm{d}x}\\
&=\int_1^{\infty}{\frac{\frac{1}{2^{p-1}}\sum\limits_{k=0}^{\frac{p}{2}-1}{\mathrm{C}_{p}^{k}\cos 2\left( \frac{p}{2}-k \right) x}-\sum\limits_{k=1}^q{q_k\cos kx}}{x}\mathrm{d}x}+\frac{\mathrm{C}_{p}^{\frac{p}{2}}}{2^p}\int_1^{\infty}{\frac{1}{x}\mathrm{d}x}.
\end{align*}  
由于$\int_1^{\infty}{\frac{1}{x}\mathrm{d}x}$发散, 故此时$\int_1^{\infty}{\frac{\cos ^px-\cos ^qx}{x}\mathrm{d}x}$也发散  .

综上,由  
\[
\int_0^{\infty}{\frac{\cos ^px-\cos ^qx}{x}\mathrm{d}x}=\int_0^1{\frac{\cos ^px-\cos ^qx}{x}\mathrm{d}x}+\int_1^{\infty}{\frac{\cos ^px-\cos ^qx}{x}\mathrm{d}x}.
\]
可知当$p=q$或$p,q$均为奇数时, $\int_0^{\infty}{\frac{\cos ^px-\cos ^qx}{x}\mathrm{d}x}$收敛 ,其余情形均发散.
\end{proof}

\begin{example}
对实数$p\ne 0$,讨论$I = \int_0^1{\frac{\cos(\frac{1}{1 - x})}{\sqrt[p]{1 - x^2}}\mathrm{d}x}$的敛散性.
\end{example}
\begin{proof}
对$I$进行积分换元可得
\begin{align}
I=\int_0^1{\frac{\cos\left(\frac{1}{1-x}\right)}{\sqrt[p]{1-x^2}}\mathrm{d}x}\xlongequal{u=\frac{1}{1-x}}\int_1^{\infty}{\frac{\cos u}{\left( 1-\left( 1-\frac{1}{u} \right) ^2 \right) ^{\frac{1}{p}}}\cdot \frac{1}{u^2}\mathrm{d}u} \notag \\
=\int_1^{\infty}{\frac{\cos u}{\left( \frac{2}{u}-\frac{1}{u^2} \right) ^{\frac{1}{p}}u^2}\mathrm{d}u}=\int_1^{\infty}{\frac{\cos u}{\left( 2-\frac{1}{u} \right) ^{\frac{1}{p}}u^{2-\frac{1}{p}}}\mathrm{d}u}.\label{eq:integral-transform111}
\end{align}

(i) 当$p>\frac{1}{2}$时, 令$f\left( u \right) =\left[ \left( 2-\frac{1}{u} \right) ^{\frac{1}{p}}u^{2-\frac{1}{p}} \right] ^p=\left( 2-\frac{1}{u} \right) u^{2p-1}$, 则显然有$\underset{u\rightarrow +\infty}{\lim}f\left( u \right) =+\infty$且$f\left( u \right)$递增. 于是$\frac{1}{\left( \frac{2}{u}-\frac{1}{u^2} \right) ^{\frac{1}{p}}u^2}=\frac{1}{\sqrt[p]{f\left( u \right)}}$在$\left[ 1,+\infty \right)$上单调递减趋于$0$. 又显然有$\int_1^A{\cos x\mathrm{d}x}$关于$A$有界, 所以结合\eqref{eq:integral-transform111}式, 再由Dirichlet判别法可知$I$收敛.

(ii) 当$p\in \left[ 0,\frac{1}{2} \right]$时, 若$p=\frac{1}{2}$, 则$\underset{u\rightarrow +\infty}{\lim}\frac{1}{\left( 2-\frac{1}{u} \right) ^{\frac{1}{p}}u^{2-\frac{1}{p}}}=2$; 若$p\in \left[ 0,\frac{1}{2} \right)$, 则$\underset{u\rightarrow +\infty}{\lim}\frac{1}{\left( 2-\frac{1}{u} \right) ^{\frac{1}{p}}u^{2-\frac{1}{p}}}=+\infty$. 因此对$\forall p\in \left[ 0,\frac{1}{2} \right]$, 都存在$K>0$, 使得
$$\frac{1}{\left( 2-\frac{1}{u} \right) ^{\frac{1}{p}}u^{2-\frac{1}{p}}}\geqslant 1,\forall u>K.$$
于是对$\forall k\in \mathbb{N} \cap \left( K,+\infty \right)$, 都有
$$\left| \int_{\frac{k\pi}{2}}^{\frac{\left( k+1 \right) \pi}{2}}{\frac{\cos u}{\left( 2-\frac{1}{u} \right) ^{\frac{1}{p}}u^{2-\frac{1}{p}}}\mathrm{d}u} \right|\geqslant \left| \int_{\frac{k\pi}{2}}^{\frac{\left( k+1 \right) \pi}{2}}{\cos u\mathrm{d}u} \right|=1.$$
故由Cauchy收敛准则可知,$I=\int_1^{\infty}{\frac{\cos u}{\left( 2-\frac{1}{u} \right) ^{\frac{1}{p}}u^{2-\frac{1}{p}}}\mathrm{d}u}$发散.

(iii) 当$p<0$时, 显然有$\underset{u\rightarrow +\infty}{\lim}\frac{1}{\left( 2-\frac{1}{u} \right) ^{\frac{1}{p}}u^{2-\frac{1}{p}}}=0$. 令$g\left( u \right) =\left( 2-\frac{1}{u} \right) ^{\frac{1}{p}}u^{2-\frac{1}{p}}$, 则
\begin{align*}
g'\left( u \right) =\frac{2}{p}u^{-\frac{1}{p}}\left( 2-\frac{1}{u} \right) ^{\frac{1}{p}-1}+\left( 2-\frac{1}{p} \right) \left( 2-\frac{1}{u} \right) ^{\frac{1}{p}}u^{1-\frac{1}{p}}>0,\forall u\in \left[ 1,+\infty \right).
\end{align*}
因此$g\left( u \right)$单调递增, 于是$\frac{1}{\left( 2-\frac{1}{u} \right) ^{\frac{1}{p}}u^{2-\frac{1}{p}}}=\frac{1}{g\left( u \right)}$单调递减趋于$0$. 又显然有$\int_1^A{\cos x\mathrm{d}x}$关于$A$有界, 所以结合\eqref{eq:integral-transform}式, 再由Dirichlet判别法可知$I$收敛.
\end{proof}

\begin{example}
对实数 \(p\),讨论反常积分 \(\int_0^{\infty}{\frac{\sin \left( x+\frac{1}{x} \right)}{x^p}\mathrm{d}x}\) 的敛散性.
\end{example}
\begin{remark}
令\(u=x+\frac{1}{x}\),则
\begin{align*}
\int_1^{\infty}{\frac{\sin \left( x+\frac{1}{x} \right)}{x^p}\mathrm{d}x}=\int_u^{\infty}{\frac{\sin u}{\left( u+\sqrt{u^2-4} \right) ^p}\left( 1+\frac{u}{\sqrt{u^2-4}} \right) \mathrm{d}u}。
\end{align*}
显然\(\int_0^A{\sin u\mathrm{d}u}\)关于\(A\)有界。再证明\(\frac{1+\frac{u}{\sqrt{u^2-4}}}{\left( u+\sqrt{u^2-4} \right) ^p}\)单调递减趋于0,就能利用Dirichlet判别法得到\(\int_1^{\infty}{\frac{\sin \left( x+\frac{1}{x} \right)}{x^p}\mathrm{d}x}\)收敛。再同理讨论\(\int_0^1{\frac{\sin \left( x+\frac{1}{x} \right)}{x^p}\mathrm{d}x}\)即可。这种方法虽然能做,但是比较繁琐,不适合考场中使用。
\end{remark}
\begin{proof}
显然\(\int_0^{\infty}{\frac{\sin \left( x+\frac{1}{x} \right)}{x^p}\mathrm{d}x}\)有两个奇点\(x=0,+\infty\)。

(1) 当\(p\leqslant 0\)时,考虑区间\(\left[ 2n\pi +\frac{\pi}{4},2n\pi +\frac{3\pi}{4} \right]\),则
\begin{align*}
x+\frac{1}{x}\in \left[ 2n\pi +\frac{\pi}{4}+\frac{1}{2n\pi +\frac{\pi}{4}},2n\pi +\frac{3\pi}{4}+\frac{1}{2n\pi +\frac{3\pi}{4}} \right]。
\end{align*}
于是当\(n>10\)时,我们有
\begin{align*}
\int_{2n\pi +\frac{\pi}{4}}^{2n\pi +\frac{3\pi}{4}}{\frac{\sin \left( x+\frac{1}{x} \right)}{x^p}\mathrm{d}x}&\geqslant \int_{2n\pi +\frac{\pi}{4}}^{2n\pi +\frac{3\pi}{4}}{\sin \left( x+\frac{1}{x} \right) \mathrm{d}x}\\
&\geqslant \int_{2n\pi +\frac{\pi}{4}}^{2n\pi +\frac{3\pi}{4}}{\sin \left( 2n\pi +\frac{3\pi}{4}+\frac{1}{2n\pi +\frac{3\pi}{4}} \right) \mathrm{d}x}\\
&=\frac{\pi}{2}\sin \left( \frac{3\pi}{4}+\frac{1}{2n\pi +\frac{3\pi}{4}} \right) >0。
\end{align*}
因此由Cauchy收敛准则可知\(\int_1^{\infty}{\frac{\sin \left( x+\frac{1}{x} \right)}{x^p}\mathrm{d}x}\)发散。故此时\(\int_0^{\infty}{\frac{\sin \left( x+\frac{1}{x} \right)}{x^p}\mathrm{d}x}\)发散。

(2) 当\(p>0\)时,先考虑\(\int_1^{\infty}{\frac{\sin \left( x+\frac{1}{x} \right)}{x^p}\mathrm{d}x}\)。

(i) 若\(p>1\),则
\begin{align*}
\int_1^{\infty}{\left| \frac{\sin \left( x+\frac{1}{x} \right)}{x^p} \right|\mathrm{d}x}\leqslant \int_1^{\infty}{\frac{1}{x^p}\mathrm{d}x}<\infty。
\end{align*}
因此\(\int_1^{\infty}{\frac{\sin \left( x+\frac{1}{x} \right)}{x^p}\mathrm{d}x}\)绝对收敛。

(ii) 若\(p\in \left( 0,1 \right]\),则
\begin{align}
\int_1^{\infty}{\frac{\sin \left( x+\frac{1}{x} \right)}{x^p}\mathrm{d}x}=\int_1^{\infty}{\sin x\frac{\cos \frac{1}{x}}{x^p}\mathrm{d}x}+\int_1^{\infty}{\cos x\frac{\sin \frac{1}{x}}{x^p}\mathrm{d}x}。\label{100.40}
\end{align}
显然\(\int_1^A{\cos x\mathrm{d}x}\)关于\(A\)有界,并且\(\frac{\sin \frac{1}{x}}{x^p}\)在\(\left[ 1,+\infty \right)\)上单调递减趋于0,故由Dirichlet判别法可知\(\int_1^{\infty}{\frac{\cos x}{x^p}\sin \frac{1}{x}\mathrm{d}x}\)收敛。令\(f\left( u \right) =u^p\cos u\),则当\(u\in \left( 0,\frac{4p}{\pi} \right)\)时,有
\begin{align*}
f'\left( u \right) =pu^{p-1}\cos u-u^p\sin u=u^{p-1}\cos u\left( p-u\tan u \right) >0。
\end{align*}
于是\(f\left( u \right)\)在\(\left( 0,\frac{4p}{\pi} \right)\)上单调递增,从而\(\frac{\cos \frac{1}{x}}{x^p}=f\left( \frac{1}{x} \right)\)在\(\left( \frac{\pi}{4}p,+\infty \right)\)上单调递减趋于0。又显然\(\int_{\frac{\pi}{4}p}^A{\sin x\mathrm{d}x}\)关于\(A\)有界,故由Dirichlet判别法可知\(\int_{\frac{\pi}{4}p}^{\infty}{\frac{\sin x}{x^p}\cos \frac{1}{x}\mathrm{d}x}\)收敛,又\(\frac{\pi}{4}p<1\),故此时\(\int_1^{\infty}{\frac{\sin x}{x^p}\cos \frac{1}{x}\mathrm{d}x}\)收敛。因此再由\eqref{100.40}式可知\(\int_1^{\infty}{\frac{\sin \left( x+\frac{1}{x} \right)}{x^p}\mathrm{d}x}\)收敛。

注意到
\begin{align*}
\int_1^{\infty}{\frac{\left| \sin \left( x+\frac{1}{x} \right) \right|}{x^p}\mathrm{d}x}\geqslant \int_1^{\infty}{\frac{\sin ^2\left( x+\frac{1}{x} \right)}{x^p}\mathrm{d}x}
=\frac{1}{2}\int_1^{\infty}{\frac{1}{x^p}\mathrm{d}x}+\frac{1}{2}\int_1^{\infty}{\frac{\cos \left( 2x+\frac{2}{x} \right)}{x^p}\mathrm{d}x}。
\end{align*}
显然\(\int_1^{\infty}{\frac{1}{x^p}\mathrm{d}x}\)发散。故此时\(\int_1^{\infty}{\frac{\sin \left( x+\frac{1}{x} \right)}{x^p}\mathrm{d}x}\)条件收敛,但不绝对收敛。

再考虑\(\int_0^1{\frac{\sin \left( x+\frac{1}{x} \right)}{x^p}\mathrm{d}x}\)。

(i) 若\(p\in \left( 0,1 \right)\),则
\begin{align*}
\int_0^1{\frac{\left| \sin \left( x+\frac{1}{x} \right) \right|}{x^p}\mathrm{d}x}\leqslant \int_0^1{\frac{1}{x^p}\mathrm{d}x}<\infty。
\end{align*}
故此时\(\int_0^1{\frac{\sin \left( x+\frac{1}{x} \right)}{x^p}\mathrm{d}x}\)绝对收敛。

(ii) 若\(p\geqslant 1\),则
\begin{align*}
\int_0^1{\frac{\sin \left( x+\frac{1}{x} \right)}{x^p}\mathrm{d}x}\xlongequal{x=\frac{1}{t}}\int_1^{\infty}{\frac{\sin \left( t+\frac{1}{t} \right)}{t^{2-p}}\mathrm{d}t}。
\end{align*}
此时\(2-p\leqslant 1\)。于是当\(2-p\leqslant 0\)即\(p\geqslant 2\)时,由(1)可知\(\int_0^1{\frac{\sin \left( x+\frac{1}{x} \right)}{x^p}\mathrm{d}x}\)发散。当\(2-p\in \left( 0,1 \right]\)即\(p\in \left[ 1,2 \right)\)时,由(i)可知\(\int_0^1{\frac{\sin \left( x+\frac{1}{x} \right)}{x^p}\mathrm{d}x}\)条件收敛,但不绝对收敛。

综上,当\(p\leqslant 0\)时,\(\int_0^{\infty}{\frac{\sin \left( x+\frac{1}{x} \right)}{x^p}\mathrm{d}x}\)发散;当\(p\in \left( 0,2 \right)\)时,\(\int_0^{\infty}{\frac{\sin \left( x+\frac{1}{x} \right)}{x^p}\mathrm{d}x}\)条件收敛;当\(p\geqslant 2\)时,\(\int_0^{\infty}{\frac{\sin \left( x+\frac{1}{x} \right)}{x^p}\mathrm{d}x}\)发散。
\end{proof}

\begin{example}
判断广义积分\(\int_{1}^{\infty}\frac{1}{x}e^{\cos x}\cos(2\sin x)\mathrm{d}x\),\(\int_{0}^{\infty}\frac{1}{x}e^{\cos x}\sin(\sin x)\mathrm{d}x\)的敛散性。 
\end{example}
\begin{proof}
(1) 由于\(e^{\cos x}\sin \left( 2\sin x \right)\)是周期为\(2\pi\)的奇函数,故
\begin{align*}
\int_0^{2\pi}{e^{\cos x}\sin \left( 2\sin x \right) \mathrm{d}x}=\int_{-\pi}^{\pi}{e^{\cos x}\sin \left( 2\sin x \right) \mathrm{d}x}=0。
\end{align*}
\begin{align*}
\int_0^{2\pi}{\left| e^{\cos x}\sin \left( 2\sin x \right) \right|\mathrm{d}x}\leqslant \int_0^{2\pi}{e\mathrm{d}x}=2\pi e。
\end{align*}
于是
\begin{align*}
\int_0^A{e^{\cos x}\sin \left( 2\sin x \right) \mathrm{d}x}&=\int_0^{2\pi \left[ \frac{A}{2\pi} \right]}{e^{\cos x}\sin \left( 2\sin x \right) \mathrm{d}x}+\int_{2\pi \left[ \frac{A}{2\pi} \right]}^A{e^{\cos x}\sin \left( 2\sin x \right) \mathrm{d}x} \\
&\leqslant 0+\int_{2\pi \left[ \frac{A}{2\pi} \right]}^A{\left| e^{\cos x}\sin \left( 2\sin x \right) \right|\mathrm{d}x} 
\leqslant \int_{2\pi \left[ \frac{A}{2\pi} \right]}^{2\pi \left( \left[ \frac{A}{2\pi} \right] +1 \right)}{\left| e^{\cos x}\sin \left( 2\sin x \right) \right|\mathrm{d}x} \\
&=\int_0^{2\pi}{\left| e^{\cos x}\sin \left( 2\sin x \right) \right|\mathrm{d}x}\leqslant 2\pi e,\forall A>2\pi。
\end{align*}
又显然有\(\frac{1}{x}\)单调趋于\(0\),故由Dirichlet判别法可知\(\int_0^{\infty}{e^{\cos x}\sin \left( 2\sin x \right) \mathrm{d}x}\)收敛。

(2) 对\(\forall n\in \mathbb{N}\),我们有
\begin{align*}
\int_{2n\pi}^{2\left( n+2 \right) \pi}{\frac{1}{x}e^{\cos x}\cos \left( 2\sin x \right) \mathrm{d}x}\geqslant \frac{C}{n},
\end{align*}
其中\(C\)为某一常数。(这里需要对上述积分进行数值估计,\(C\)需要具体确定出来,太麻烦暂时省略) 于是
\begin{align*}
\int_1^{\infty}{\frac{1}{x}e^{\cos x}\cos(2\sin x)\mathrm{d}x}&=\sum_{n=1}^{\infty}{\int_{2n\pi}^{2\left( n+2 \right) \pi}{\frac{1}{x}e^{\cos x}\cos \left( 2\sin x \right) \mathrm{d}x}} 
\geqslant \sum_{n=1}^{\infty}{\frac{C}{n}}=\infty。
\end{align*}
故\(\int_1^{\infty}{\frac{1}{x}e^{\cos x}\cos(2\sin x)\mathrm{d}x}\)发散。
\end{proof}













\end{document}
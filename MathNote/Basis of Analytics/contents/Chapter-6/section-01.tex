\documentclass[../../main.tex]{subfiles}
\graphicspath{{\subfix{../../image/}}} % 指定图片目录,后续可以直接使用图片文件名。

% 例如:
% \begin{figure}[H]
% \centering
% \includegraphics[scale=0.4]{图.png}
% \caption{}
% \label{figure:图}
% \end{figure}
% 注意:上述\label{}一定要放在\caption{}之后,否则引用图片序号会只会显示??.

\begin{document}

\section{基本性态分析模型}

\begin{proposition}[多个函数取最值或者中间值]\label{proposition:多个函数取最值或者中间值}
设\(f,g,h\)是定义域上的连续函数,则
\((a)\):\(\max\{f,g\}, \min\{f,g\}\)是定义域上的连续函数.
\((b)\):\(\text{mid}\{f,g,h\}\)是定义域上的连续函数.
\end{proposition}
\begin{remark}
这里\(\text{mid}\{f,g,h\}\)表示取中间值函数,显然这个命题可以推广到多个函数的情况.
\end{remark}
\begin{proof}
只需要注意到
\begin{align*}
\max\{f,g\}&=\frac{f + g+\vert f - g\vert}{2},\\
\min\{f,g\}&=\frac{f + g-\vert f - g\vert}{2},\\
\text{mid}\{f,g,h\}&=f + g + h-\max\{f,g,h\}-\min\{f,g,h\}.
\end{align*}

\end{proof}

\begin{proposition}\label{proposition:连续函数无零点则一定恒大于零或恒小于零}
若\(f\)是区间\(I\)上处处不为零的连续函数,则\(f\)在区间\(I\)上要么恒大于零,要么恒小于零.
\end{proposition}
\begin{proof}
用反证法,若存在$x_1,x_2\in I$,使得$f(x_1)=f(x_2)=0$,则由零点存在定理可知,存在$\xi \in (\min{x_1,x_2},\max{x_1,x_2})$,使得$f(\xi)=0$矛盾.

\end{proof}

\begin{proposition}\label{proposition:导数为常数的函数必是线性函数}
设\(f\)为区间\(I\)上的可微函数.证明:\(f'\)为\(I\)上的常值函数的充分必要条件是\(f\)为线性函数.
\end{proposition}
\begin{proof}
充分性显然,下证必要性.设$f'(x)\equiv C$,其中$C$为某一常数.
$\forall x\in I$,任取固定点$x_0\in I$,由$Lagrange$中值定理可知,
存在$\xi\in(\min\{x_0,x\},\max\{x_0,x\})$,使得
\begin{align*}
f(x)=f'(\xi)(x-x_0)+f(x_0)=C(x-x_0)+f(x_0).
\end{align*}
故$f(x)$为线性函数.

\end{proof}

\begin{theorem}[闭区间上单调函数必可积]\label{theorem:闭区间上单调函数必可积}
设$f$在$[a,b]$上单调,则$f\in R[a,b].$
\end{theorem}
\begin{proof}


\end{proof}

\begin{proposition}[连续的周期函数的基本性质]\label{proposition:连续的周期函数的基本性质}
设$f\in C(\mathbb{R})$且以$T>0$为周期,则
\begin{enumerate}[(1)]
\item $f$在$\mathbb{R}$上有界.

\item $f$在$\mathbb{R}$上一致连续.
\end{enumerate}
\end{proposition}
\begin{proof}
\begin{enumerate}[(1)]
\item 由$f\in C\left[ 0,T \right]$知,存在$M>0$,使得
\begin{align*}
\left| f\left( x \right) \right|<M,\quad \forall x\in \left[ 0,T \right]
\end{align*}
对$\forall y\in \mathbb{R}$,存在$n\in \mathbb{Z}$,$x\in \left[ 0,T \right]$,使得$y=nT+x$。又$f$以$T$为周期,故$\left| f\left( y \right) \right|=\left| f\left( x \right) \right|<M$。

\item 由\refcor{corollary:闭区间上的连续函数一定一致连续}知$f$在$\left[ nT,\left( n+1 \right) T \right]$,$\forall n\in \mathbb{Z}$上一致连续,从而由\hyperref[proposition:一致连续函数的拼接]{一致连续的拼接}同理可知$f$在$\left( -\infty ,+\infty \right)$上也一致连续。
\end{enumerate}

\end{proof}

\begin{proposition}[导数有正增长率则函数爆炸]\label{proposition:导数有正增长率则函数爆炸}
设\(f\)在\([a,+\infty)\)可微且\(\lim_{x\rightarrow +\infty}f'(x)=c > 0\),证明
\[\lim_{x\rightarrow +\infty}f(x)=+\infty.\]
\end{proposition}
\begin{note}
类似的还有趋于\(-\infty\)或者非极限形式的结果,读者应该准确理解含义并使得各种情况都能复现,我们引用本结论时未必就是本结论本身,而是其蕴含的思想.
\end{note}
\begin{proof}
因为\(\lim_{x\rightarrow +\infty}f'(x)=c > 0\),所以存在\(X > a\),使得\(f'(x)>\frac{c}{2},\forall x\geqslant X\)。于是由Lagrange中值定理得到,对$\forall x\geqslant X$,存在$\theta \in (X,x)$,使得
\[f(x)=f(X)+f'(\theta)(x - X)\geqslant f(X)+\frac{c}{2}(x - X),\forall x\geqslant X.\]
让\(x\rightarrow +\infty\)就得到
\[\lim_{x\rightarrow +\infty}f(x)=+\infty.\]

\end{proof}

\begin{proposition}[函数不爆破则各阶导数必然有趋于 0 的子列]\label{proposition:函数不爆破则各阶导数必然有趋于 0 的子列}
设\(k\in\mathbb{N},a\in\mathbb{R}\)且\(f\in D^{k}[a,+\infty)\),若\(\lim_{x\rightarrow +\infty}|f(x)|\neq +\infty\),那么存在趋于正无穷的\(\{x_n\}_{n = 1}^{\infty}\subset [a,+\infty)\)使得
\[
\lim_{n\rightarrow\infty}f^{(k)}(x_n)=0.
\]
\end{proposition}
\begin{note}
\begin{enumerate}[(1)]
\item \hypertarget{proposition5.4由假设得到这个结论的原因}{存在\(X > 0\)使得\(f^{(k)}\)在\((X,+\infty)\)要么恒正,要么恒负的原因:}否则,对\(\forall X>0\),存在\(x_1,x_2\in (X,+\infty)\),使得\(f^{(k)}(x_1)>0\),\(f^{(k)}(x_2)<0\)。从而由导数的介值性可知,存在\(\xi_X\in (x_1,x_2)\),使得\(f^{(k)}(\xi_X) = 0\)。
于是
\begin{align*}
&\text{令}X = 1\text{,则存在}y_1>1\text{,使得}f^{(k)}(y_1)=0;\\
&\text{令}X=\max\{2,y_1\}\text{,则存在}y_2>\max\{2,y_1\}\text{,使得}f^{(k)}(y_2)=0;\\
&\cdots\cdots\\
&\text{令}X=\max\{n,y_{n - 1}\}\text{,则存在}y_n>\max\{n,y_{n - 1}\}\text{,使得}f^{(k)}(y_n)=0;\\
&\cdots\cdots
\end{align*}
这样得到一个数列\(\{y_n\}_{n = 1}^{\infty}\)满足
\[
\lim_{n\rightarrow \infty}y_n=+\infty\text{且}f^{(k)}(y_n)=0,\forall n\in \mathbb{N}_+.
\]
这与假设矛盾!

\item \hypertarget{m的存在性}{存在$m>0$,使得$f^{(k)}(x)\geqslant m>0,\forall x\geqslant X$的原因:}假设对\(\forall m>0\),有\(m > f^{(k)}(x) > 0\),\(\forall x\geqslant X\)。再令\(m\rightarrow 0^+\),则由夹逼准则可得\(f^{(k)}(x) = 0\),\(\forall x\geqslant X\)。这与假设矛盾!
(也可以用下极限证明)
\end{enumerate}
\end{note}
\begin{proof}
注意到若不存在\(\{x_n\}_{n = 1}^{\infty}\)使得\(\lim_{n\rightarrow\infty}f^{(k)}(x_n)=0\)成立那么将\hyperlink{proposition5.4由假设得到这个结论的原因}{存在\(X > 0\)使得\(f^{(k)}\)在\((X,+\infty)\)要么恒正,要么恒负(见笔记(1)).}如果找不到子列使得\(\lim_{n\rightarrow\infty}f^{(k)}(x_n)=0\)成立,那么不妨设存在\(X> 0\)使得
\begin{align*}
f^{(k)}(x)>0,\forall x\geqslant X.
\end{align*}
从而\hyperlink{m的存在性}{一定存在$m>0$(见笔记(2))},使得
\begin{align}\label{proposition5.4-12.5}
f^{(k)}(x)\geqslant m>0,\forall x\geqslant X.
\end{align}
则由Taylor中值定理,我们知道对每个\(x > X\),运用\eqref{proposition5.4-12.5},都有
\[f(x)=\sum_{j = 0}^{k - 1}\frac{f^{(j)}(X)}{j!}(x - X)^j+\frac{f^{(k)}(\theta)}{k!}(x - X)^k\geqslant\sum_{j = 0}^{k - 1}\frac{f^{(j)}(X)}{j!}(x - X)^j+\frac{m}{k!}(x - X)^k,\]
于是\(\lim_{x\rightarrow +\infty}f(x)=+\infty\),这就是一个矛盾!因此我们证明了必有子列使得\(\lim_{n\rightarrow\infty}f^{(k)}(x_n)=0\)成立.

\end{proof}

\begin{theorem}[严格单调和导数的关系]\label{theorem:严格单调和导数的关系}
\begin{enumerate}
\item 设\(f\in C[a,b]\cap D(a,b)\)且\(f\)递增,则\(f\)在\([a,b]\)严格递增的充要条件是对任何\([x_1,x_2]\subset [a,b]\)都存在\(c\in(x_1,x_2)\)使得\(f'(c)>0\)。

\item 设\(f\in C[a,b]\cap D(a,b)\)且\(f\)递减,则\(f\)在\([a,b]\)严格递减的充要条件是对任何\([x_1,x_2]\subset [a,b]\)都存在\(c\in(x_1,x_2)\)使得\(f'(c)<0\)。
\end{enumerate}
\end{theorem}
\begin{proof}
若\(f\)在\([a,b]\)严格递增,则对任何\([x_1,x_2]\subset [a,b]\),由Lagrange中值定理可知,存在\(c\in(x_1,x_2)\),使得
\[
\frac{f(x_2)-f(x_1)}{x_2 - x_1}=f'(c)>0.
\]
反之对任何\([x_1,x_2]\subset [a,b]\)都存在\(c\in(x_1,x_2)\)使得\(f'(c)>0\)。
任取\([s,t]\subset [a,b]\),现在有\(c\in(s,t)\)使得\(f'(c)>0\),则根据$f'\left( c \right) =\underset{h\rightarrow 0}{\lim}\frac{f\left( c+h \right) -f\left( c \right)}{h}=\underset{h\rightarrow 0}{\lim}\frac{f\left( c \right) -f\left( c-h \right)}{h}>0$,再结合$f$递增,可知存在充分小的\(h > 0\)使得
\[
f(s)\leqslant f(c - h)<f(c)<f(c + h)\leqslant f(t),
\]
这就证明了\(f\)严格递增。严格递减是类似的,我们完成了证明。

\end{proof}

\begin{theorem}[单侧导数极限定理]\label{theorem:单侧导数极限定理}
设\(f\in C[a,b]\cap D^{1}(a,b]\)且\(\lim_{x\rightarrow a^{+}}f'(x)=c\)存在,证明\(f\)在\(a\)右可导且\(f_{+}'(a)=c\)。
\end{theorem}
\begin{remark}
本结果当然也可对应写出左可导的版本和可导的版本,以及对应的无穷版本(即$a,b,c$相应的取$\pm \infty$).
\end{remark}
\begin{note}
本结果告诉我们可在\(f\)连续的时候用\(f'\)的左右极限存在性来推\(f\)可导性.
\end{note}
\begin{proof}
运用Lagrange中值定理,我们知道
\[\lim_{x\rightarrow a^{+}}\frac{f(x)-f(a)}{x - a}=\lim_{x\rightarrow a^{+}}f'(\theta(x))=c,\]
其中
\(\theta(x)\in(a,x),\lim_{x\rightarrow a^{+}}\theta(x)=a.\)
这就完成了这个定理的证明.

\end{proof}

\begin{example}[\,\,经典光滑函数]\label{example:经典光滑函数}
考虑
\[
f(x)=
\begin{cases}
e^{-\frac{1}{x^2}}, &|x|>0 \\
0, &|x| = 0
\end{cases}
\]
则\(f\in C^{\infty}(\mathbb{R})\)且\(f^{(n)}(0)=0,\forall n\in\mathbb{N}\)。
\end{example}
\begin{proof}
我们归纳证明,首先\(f\in C^{0}(\mathbb{R}) = C(\mathbb{R})\),假定\(f\in C^{k}(\mathbb{R}),k\in\mathbb{N}\)。注意到存在多项式\(p_{k + 1}\in\mathbb{R}[x]\),使得
\[
f^{(k + 1)}(x)=p_{k + 1}\left(\frac{1}{x}\right)e^{-\frac{1}{x^2}},\forall x\neq0.
\]
于是
\[
\lim_{x\rightarrow0}f^{(k + 1)}(x)=\lim_{x\rightarrow0}p_{k + 1}\left(\frac{1}{x}\right)e^{-\frac{1}{x^2}}=\lim_{x\rightarrow\infty}p_{k + 1}(x)e^{-x^2}=0,
\]
运用\hyperref[theorem:导数极限定理]{导数极限定理},我们知道\(f^{(k + 1)}(0)=0\)。由数学归纳法我们知道\(f^{(n)}(0)=0,\forall n\in\mathbb{N}\),这就完成了证明。

\end{proof}

\begin{theorem}[连续函数中间值定理]\label{theorem:连续函数中间值定理}
设\(p_1,p_2,\cdots,p_n\geqslant0\)且\(\sum_{j = 1}^{n}p_j = 1\)。则对有介值性函数\(f:[a,b]\to\mathbb{R}\)和\(a\leqslant x_1\leqslant x_2\leqslant\cdots\leqslant x_n\leqslant b\),必然存在\(\theta\in[x_1,x_n]\),使得
\begin{align*}
f(\theta)=\sum_{j = 1}^{n}p_jf(x_j).
\end{align*}
\end{theorem}
\begin{note}
中间值可以通过介值定理取到是非常符合直观的。特别的当\(p_1 = p_2=\cdots=p_n=\frac{1}{n}\),就是所谓的平均值定理
\[
f(\theta)=\frac{1}{n}\sum_{j = 1}^{n}f(x_j).
\]
\end{note}
\begin{proof}
设
\[
M=\max_{1\leqslant i\leqslant n}f(x_i),m=\min_{1\leqslant i\leqslant n}f(x_i).
\]
于是
\[
m = m\sum_{j = 1}^{n}p_j\leqslant\sum_{j = 1}^{n}p_jf(x_j)\leqslant M\sum_{j = 1}^{n}p_j = M.
\]
因此由\(f\)的介值性知:必然存在\(\theta\in[x_1,x_n]\),使得$f(\theta)=\sum_{j = 1}^{n}p_jf(x_j)$成立。

\end{proof}

\begin{proposition}\label{proposition:导函数没有第一类间断点与无穷间断点}
若$f\in C[a,b]\cap D(a,b)$,则$f'$没有第一类间断点与无穷间断点.
\end{proposition}
\begin{remark}
也可以利用\hyperref[theorem:导数介值定理]{Darboux定理}进行证明.
\end{remark}
\begin{proof}
若$f'$存在第一类间断点$c\in [a,b]$,则由\hyperref[theorem:单侧导数极限定理]{单侧导数极限定理}可知
\begin{align*}
f' \left( c^- \right) =f_{-}'\left( c \right) ,\quad f' \left( c^+ \right) =f_{+}'\left( c \right) .
\end{align*}
又因为$f$在$x=c$处可导,所以$f_{-}'\left( c \right) =f_{+}'\left( c \right)$.从而\
\begin{align*}
f' \left( c^- \right) =f_{-}'\left( c \right) =f_{+}'\left( c \right) =f'\left( c^+ \right) .
\end{align*}
即$f$在$x=c$处既左连续又右连续,故$f$在$x=c$处连续,矛盾!

由于\hyperref[theorem:单侧导数极限定理]{单侧导数极限定理}同样适用于单侧导数为无穷大的情况,因此对于无穷大的情况可同理证明.

\end{proof}

\begin{proposition}\label{proposition:定义是区间的单调函数值域还是区间就必是连续函数}
设$f$是一个定义在区间$I\subset \mathbb{R}$上的单调函数,并且满足$f(I)=I'$,其中$I'\subset \mathbb{R}$是一个区间,则$f$在区间$I$上连续,即$f\in C(I)$.
\end{proposition}
\begin{proof}
反证,假设$f$在某个点$c\in I$处间断。若$c$在区间$I$的内部,则由$f$在区间$I$上单调递增,利用单调有界定理可知$\lim_{x\rightarrow c^+}f(x)$和$\lim_{x\rightarrow c^-}f(x)$存在,并且
\begin{align*}
\lim_{x\rightarrow c^-}f(x) \leqslant f(c) \leqslant \lim_{x\rightarrow c^+}f(x).
\end{align*}
又因为$f(x)$在$x=c$处间断,所以上式至少有一个严格不等号成立,故不妨设
\begin{align*}
\lim_{x\rightarrow c^-}f(x) \leqslant f(c) < \lim_{x\rightarrow c^+}f(x).
\end{align*}
对$\forall x>c$,固定$x$,由$f$在$I$上递增可知
\begin{align*}
f(x) > f(y), \quad \forall y \in (c, x).
\end{align*}
令$y\rightarrow c^+$,得$f(x) \geqslant \lim_{x\rightarrow c^+}f(x)$。对$\forall x<c$,由$f$在$I$上递增可知$f(x) \leqslant f(c)$。
因此$f(I) \subset (-\infty, f(c)] \cup [\lim_{x\rightarrow c^+}f(x), +\infty)$,故$(f(c), \lim_{x\rightarrow c^+}f(x)) \not\subset f(I)$,但$(f(c), \lim_{x\rightarrow c^+}f(x)) \subset I'$。这与$f(I) = I'$矛盾!

若$c$是区间$I$的端点,则同理可得矛盾!

\end{proof}

\begin{proposition}\label{proposition:单调函数只有第一类间断点}
定义在区间$I$上的单调函数$f$只有第一类间断点,特别地,若$x_0$在区间$I$的内部,则$x_0$要么是跳跃间断点,要么就是连续点.
\end{proposition}
\begin{proof}


\end{proof}

\begin{proposition}[连续单射等价严格单调]\label{proposition:连续单射等价严格单调}
设\(f\)是区间\(I\)上的连续函数,证明\(f\)在\(I\)上严格单调的充要条件是\(f\)是单射。
\end{proposition}
\begin{proof}
必要性是显然的,只证充分性.如若不然,不妨考虑\(f(x_3)<f(x_1)<f(x_2),x_1<x_2<x_3\)(其他情况要么类似,要么平凡),于是由连续函数介值定理知存在\(\theta\in[x_2,x_3]\)使得\(f(\theta)=f(x_1)\),这就和\(f\)在\(I\)上单射矛盾!故\(f\)严格单调.

\end{proof}

\begin{example}
证明不存在\(\mathbb{R}\)上的连续函数\(f\)满足方程
\[
f(f(x)) = e^{-x}.
\]
\end{example}
\begin{note}
注意积累二次复合的常用处理手法,即运用\hyperref[proposition:连续单射等价严格单调]{命题\ref{proposition:连续单射等价严格单调}}.
\end{note}
\begin{proof}
假设存在满足条件的函数$f$.设\(f(x)=f(y)\),则
\[
e^{-x}=f(f(x)) = f(f(y)) = e^{-y}.
\]
由\(e^{-x}\)的严格单调性我们知\(x = y\),于是\(f\)是单射。由\hyperref[proposition:连续单射等价严格单调]{命题\ref{proposition:连续单射等价严格单调}}知\(f\)严格单调。又递增和递增复合递增,递减和递减复合也递增,我们知道\(f(f(x)) = e^{-x}\)递增,这和\(e^{-x}\)严格递减矛盾!故这样的\(f\)不存在。

\end{proof}

\begin{example}
求\(k\in\mathbb{R}\)的范围,使得存在\(f\in C(\mathbb{R})\)使得\(f(f(x)) = kx^{9}\)。
\end{example}
\begin{note}
\hypertarget{取这个函数的原因}{\textbf{取$\boldsymbol{f}\mathbf{(}\boldsymbol{x}\mathbf{)}=\sqrt[\mathbf{4}]{\boldsymbol{k}}\boldsymbol{x}^{\mathbf{3}}$的原因:}}当\(k\geqslant0\)时,我们可待定\(f(x)=cx^{3}\),需要\(c^{4}x^{9}=kx^{9}\),从而可取\(c = \sqrt[4]{k}\).
\end{note}
\begin{proof}
当$k<0$时,假设存在满足条件的函数$f$.设\(f(x)=f(y)\),则
\[
kx^{9}=f(f(x)) = f(f(y)) = ky^{9}.
\]
由\(kx^{9}\)的严格单调性我们知\(x = y\),于是\(f\)是单射。由\hyperref[proposition:连续单射等价严格单调]{命题\ref{proposition:连续单射等价严格单调}}知\(f\)严格单调。又递增和递增复合递增,递减和递减复合也递增,我们知道\(f(f(x)) = kx^{9}\)递增,这和\(kx^{9}\)严格递减矛盾!故这样的\(f\)不存在。

当\(k\geqslant0\)时,\hyperlink{取这个函数的原因}{取$f(x)=\sqrt[4]{k}x^3$},此时$f(x)$满足条件.

\end{proof}

\begin{proposition}[\([a,b]\)到\([a,b]\)的连续函数必有不动点]\label{proposition:[a,b]到[a,b]的连续函数必有不动点}
设\(f:[a,b]\to[a,b]\)是连续函数,证明\(f\)在\([a,b]\)上有不动点。
\end{proposition}
\begin{note}
注意\([a,b]\to[a,b]\)表示\(f\)是从\([a,b]\to[a,b]\)的映射,右端的\([a,b]\)是像集而不是值域,\(f\)可能取不到整个\([a,b]\)。
\end{note}
\begin{proof}
考虑\(g(x)=f(x)-x\in C[a,b]\),注意到\(g(a)\geqslant0,g(b)\leqslant0\),由连续函数的零点定理知道\(f\)在\([a,b]\)上有不动点。

\end{proof}

\begin{proposition}[没有极值点则严格单调]\label{proposition:没有极值点则严格单调}
设\(f\in C[a,b]\)且\(f\)在\((a,b)\)没有极值点,证明\(f\)在\([a,b]\)严格单调。
\end{proposition}
\begin{proof}
因为闭区间上连续函数必然取得最值,且在\((a,b)\)的最值点必然是极值点,因此由假设我们不妨设\(f\)在\([a,b]\)端点取得最值。不失一般性假设
\[
f(a)=\min_{x\in[a,b]}f(x),f(b)=\max_{x\in[a,b]}f(x).
\]
此时若在\([a,b]\)上\(f\)严格单调,则只能是严格单调递增.
若在\([a,b]\)上\(f\)不严格递增,则存在\(x_2>x_1\),使得\(f(x_2)\leqslant f(x_1)\)。

若\(x_1>a\),在\([a,x_2]\)上我们注意到\(f(x_1)\geqslant\max\{f(a),f(x_2)\}\),又由$f$的连续性可知,$f$一定能在$[a,x_2]$上取到最大值.于是\(f\)只能在\((a,x_2)\)达到最大值,从而$f$在\((a,x_2)\)存在极大值点,这和\(f\)在\((a,b)\)没有极值点矛盾!

若\(x_1 = a,x_2 < b\),则注意到\(f(x_2)\leqslant\min\{f(a),f(b)\}\),同样的\(f\)在\((a,b)\)取得极小值而矛盾。

若\(x_1 = a,x_2 = b\),则\(f\)恒为常数而矛盾!这就完成了证明。

\end{proof}

\begin{proposition}[函数值相同的点导数值相同就一定单调]\label{proposition:函数值相同的点导数值相同就一定单调}
设\(f\in D(a,b)\)满足\(f(x_1)=f(x_2),x_1,x_2\in(a,b)\),必有\(f'(x_1)=f'(x_2)\),证明\(f\)在\((a,b)\)是单调函数。
\end{proposition}
\begin{note}
\hypertarget{令
sigma=max{x in[c,xi]:f(x)=f(d)}的原因
}{\textbf{令
\(
\boldsymbol{\sigma }=\mathbf{max}\left\{ \boldsymbol{x}\in \left[ \boldsymbol{c},\boldsymbol{\xi } \right] :\boldsymbol{f}\left( \boldsymbol{x} \right) =\boldsymbol{f}\left( \boldsymbol{d} \right) \right\}    
\)的原因:}}设$E=\{x\in[c,\xi]:f(x)=f(d)\}$.实际上,这里取$\sigma=\sup\{x\in[c,\xi]:f(x)=f(d)\}$也可以,效果类似.
\begin{enumerate}[(1)]
\item \textbf{$\boldsymbol{\sigma }$的存在性证明:}由$f$的介值性知,存在$\eta\in (c,\xi)$,使得
\begin{align*}
f(\xi)\leqslant  f(\eta)=f(d)\leqslant  f(c).
\end{align*}
从而$\eta \in E=\{x\in[c,\xi]:f(x)=f(d)\}$,故$E$非空.又由$E$的定义,显然$E$有界,故由确界存在定理可知,$E$存在上确界.于是令$\sigma=\sup\{x\in[c,\xi]:f(x)=f(d)\}\leqslant  \in[c,\xi]$.下证$\sigma=\sup\{x\in[c,\xi]:f(x)=f(d)\}=\max\{x\in[c,\xi]:f(x)=f(d)\}$,即$\sigma\in E=\{x\in[c,\xi]:f(x)=f(d)\}$.

由上确界的性质可知,存在$\{x_n\}_{n=1}^\infty$满足$x_n \in E$且$\underset{n\rightarrow \infty}{\lim}x_n=\sigma $.从而$f(x_n)=f(d)$.于是由$f$的连续性可得
\begin{align*}
\underset{n\rightarrow \infty}{\lim}f\left( x_n \right) =f\left( \underset{n\rightarrow \infty}{\lim}x_n \right) =f\left( \sigma \right) =f\left( d \right) .
\end{align*}
故$\sigma \in E$.这样就完成了证明.

\item \textbf{取$\boldsymbol{\sigma }=\mathbf{max}\left\{ \boldsymbol{x}\in \left[ \boldsymbol{c},\boldsymbol{\xi } \right] :\boldsymbol{f}\left( \boldsymbol{x} \right) =\boldsymbol{f}\left( \boldsymbol{d} \right) \right\} $的原因:}当\(f(c)\geqslant f(d)\)时,$E=\{x\in[c,\xi]:f(x)=f(d)\}$中的其他点$a\in E$,可能有$f'\left( a \right) >0$,也可能有$f'\left( a \right) \leqslant 0$.而$\sigma$一定只满足$f'\left( \sigma \right) \leqslant 0$.
\end{enumerate}
\end{note}
\begin{proof}
若\(f\)不在\((a,b)\)是单调,则不妨设\(a < c < d < b\),使得\(f'(c)<0<f'(d)\)。

由\(f'(d)=\lim_{x\rightarrow d^{-}}\frac{f(x)-f(d)}{x - d}>0\)知在\(d\)的左邻域内,\(f(x)<f(d)\)。由\(f'(c)=\lim_{x\rightarrow c^{+}}\frac{f(x)-f(c)}{x - c}<0\)知\(f\)在\(c\)的右邻域内有\(f(x)<f(c)\),于是$f(c),f(d)$不是$f$在$[c,d]$上的最小值,又由$f\in C[c,d]$可知$f$在$[c,d]$上一定存在最小值.故可以设\(f\)在\([c,d]\)最小值点为\(\xi\in(c,d)\)。

当\(f(c)\geqslant f(d)\)时,\hyperlink{令
sigma=max{x in[c,xi]:f(x)=f(d)}的原因
}{令
\[
\sigma=\max\{x\in[c,\xi]:f(x)=f(d)\}.
\]}
注意到\(\sigma<\xi\)。显然\(f'(\sigma)\leqslant0\),因为如果\(f'(\sigma)>0\)会导致在\(\sigma\)右邻域内有大于\(f(d)\)的点,由介值定理可以找到\(\xi>\sigma'>\sigma\),使得\(f(\sigma')=f(d)\)而和\(\sigma\)是最大值矛盾!而函数值相同的点导数值也相同,因此\(f'(\sigma)=f'(d)>0\),这与\(f'(\sigma)\leqslant0\)矛盾!

当\(f(c)\leqslant f(d)\)时类似可得矛盾!我们完成了证明。

\end{proof}

\begin{proposition}[一个经典初等不等式]\label{proposition:一个经典初等不等式}
设\(a,b\geqslant0\),证明:
\begin{align}\label{equation-12.777}
\begin{cases}
a^p + b^p\leqslant(a + b)^p\leqslant2^{p - 1}(a^p + b^p),& p\geqslant1,p\leqslant0\\
a^p + b^p\geqslant(a + b)^p\geqslant2^{p - 1}(a^p + b^p),& 0 < p < 1
\end{cases}
\end{align}
\end{proposition}
\begin{note}
不等式左右是奇次对称的,我们可以设\(t = \frac{a}{b}\in[0,1]\),于是\eqref{equation-12.777}两边同时除以$b^p$得
\[
\begin{cases}
t^p + 1\leqslant(t + 1)^p\leqslant2^{p - 1}(t^p + 1),& p\geqslant1,p\leqslant0\\
t^p + 1\geqslant(t + 1)^p\geqslant2^{p - 1}(t^p + 1),& 0 < p < 1
\end{cases}.
\]
\end{note}
\begin{proof}
考虑\(f(t)\triangleq\frac{(t + 1)^p}{1 + t^p},t\in[0,1]\),我们有
\[
f'(t)=p(t + 1)^{p - 1}\frac{1 - t^{p - 1}}{(1 + t^p)^2}
\begin{cases}
\geqslant0,& p\geqslant1,p\leqslant0\\
<0,& 0 < p < 1
\end{cases}
\]
于是
\[
\begin{cases}
2^{p - 1}=f(1)\geqslant f(t)\geqslant f(0)=1,& p\geqslant1,p\leqslant0\\
2^{p - 1}=f(1)\leqslant f(t)\leqslant f(0)=1,& 0 < p < 1
\end{cases}
\]
这就完成了证明.

\end{proof}

\begin{theorem}[反函数存在定理]\label{theorem:反函数存在定理}
设 \( y = f(x), x \in D \) 为严格增(减)函数,则 \( f \) 必有反函数 \( f^{-1} \),且 \( f^{-1} \) 在其定义域 \( f(D) \) 上也是严格增(减)函数。
\end{theorem}
\begin{proof}
设 \( f \) 在 \( D \) 上严格增。对任一 \( y \in f(D) \),有 \( x \in D \) 使 \( f(x) = y \)。下面证明这样的 \( x \) 只能有一个。事实上,对于 \( D \) 中任一 \( x_1 \neq x \),由 \( f \) 在 \( D \) 上的严格增性,当 \( x_1 < x \) 时,\( f(x_1) < y \),当 \( x_1 > x \) 时,有 \( f(x_1) > y \),总之 \( f(x_1) \neq y \)。这就说明,对每一个 \( y \in f(D) \),都只存在唯一的一个 \( x \in D \),使得 \( f(x) = y \),从而函数 \( f \) 存在反函数 \( x = f^{-1}(y), y \in f(D) \)。

现证 \( f^{-1} \) 也是严格增的。任取 \( y_1, y_2 \in f(D) \),\( y_1 < y_2 \)。设 \( x_1 = f^{-1}(y_1) \),\( x_2 = f^{-1}(y_2) \),则 \( y_1 = f(x_1) \),\( y_2 = f(x_2) \)。由 \( y_1 < y_2 \) 及 \( f \) 的严格增性,显然有 \( x_1 < x_2 \),即 \( f^{-1}(y_1) < f^{-1}(y_2) \)。所以反函数 \( f^{-1} \) 是严格增的。

\end{proof}

\begin{theorem}[反函数连续定理]\label{theorem:反函数连续定理}
若函数 \( f \) 在 \([a, b]\) 上严格单调并连续,则反函数 \( f^{-1} \) 在其定义域 \([f(a), f(b)]\) 或 \([f(b), f(a)]\) 上连续。
\end{theorem}
\begin{proof}
不妨设 \( f \) 在 \([a, b]\) 上严格增。此时 \( f \) 的值域即反函数 \( f^{-1} \) 的定义域为 \([f(a), f(b)]\)。任取 \( y_0 \in (f(a), f(b)) \),设 \( x_0 = f^{-1}(y_0) \),则 \( x_0 \in (a, b) \)。于是对任给的 \( \varepsilon > 0 \),可在 \( (a, b) \) 上 \( x_0 \) 的两侧各取异于 \( x_0 \) 的点 \( x_1, x_2 \)(\( x_1 < x_0 < x_2 \)),使它们与 \( x_0 \) 的距离小于 \( \varepsilon \)。

设与 \( x_1, x_2 \) 对应的函数值分别为 \( y_1, y_2 \),由 \( f \) 的严格增性知 \( y_1 < y_0 < y_2 \)。令
\[
\delta = \min\{ y_2 - y_0, y_0 - y_1 \}
\]
则当 \( y \in U(y_0; \delta) \) 时,对应的 \( x = f^{-1}(y) \) 的值都落在 \( x_1 \) 与 \( x_2 \) 之间,故有
\[
| f^{-1}(y) - f^{-1}(y_0) | = | x - x_0 | < \varepsilon
\]
这就证明了 \( f^{-1} \) 在点 \( y_0 \) 连续,从而 \( f^{-1} \) 在 \( (f(a), f(b)) \) 上连续。

类似地可证 \( f^{-1} \) 在其定义区间的端点 \( f(a) \) 与 \( f(b) \) 分别为右连续与左连续。所以 \( f^{-1} \) 在 \([f(a), f(b)]\) 上连续。

\end{proof}

\begin{theorem}[反函数求导定理]\label{theorem:反函数求导定理}
设 \( y = f(x) \) 为 \( x = \varphi(y) \) 的反函数,若 \( \varphi(y) \) 在点 \( y_0 \) 的某邻域上连续,严格单调且 \( \varphi'(y_0) \neq 0 \),则 \( f(x) \) 在点 \( x_0 (x_0 = \varphi(y_0)) \) 可导,且
\[
f'(x_0) = \frac{1}{\varphi'(y_0)}.
\]
\end{theorem}
\begin{proof}
设 \( \Delta x = \varphi(y_0 + \Delta y) - \varphi(y_0) \),\( \Delta y = f(x_0 + \Delta x) - f(x_0) \)。因为 \( \varphi \) 在 \( y_0 \) 的某邻域上连续且严格单调,故 \( f = \varphi^{-1} \) 在 \( x_0 \) 的某邻域上连续且严格单调。从而当且仅当 \( \Delta y = 0 \) 时 \( \Delta x = 0 \),并且当且仅当 \( \Delta y \to 0 \) 时 \( \Delta x \to 0 \)。由 \( \varphi'(y_0) \neq 0 \),可得
\[
f'(x_0) = \lim_{\Delta x \to 0} \frac{\Delta y}{\Delta x} = \lim_{\Delta y \to 0} \frac{\Delta y}{\Delta x} = \frac{1}{\lim\limits_{\Delta y \to 0} \frac{\Delta x}{\Delta y}} = \frac{1}{\varphi'(y_0)}
\]

\end{proof}





\end{document}
\documentclass[../../main.tex]{subfiles}
\graphicspath{{\subfix{../../image/}}} % 指定图片目录,后续可以直接使用图片文件名。

% 例如:
% \begin{figure}[H]
% \centering
% \includegraphics[scale=0.4]{图.png}
% \caption{}
% \label{figure:图}
% \end{figure}
% 注意:上述\label{}一定要放在\caption{}之后,否则引用图片序号会只会显示??.

\begin{document}

\section{函数列极限}

\begin{theorem}[Dini定理]\label{theorem:Dini定理(数分版本)}
若 \(\{f_n\}_{n\in\mathbb{N}}\subset C([a,b])\),\(f\in C([a,b])\) 且对每一个 \(x\in [a,b]\),都有 \(f_n(x)\) 关于$n$单调并成立
\begin{align*}
\lim_{n\rightarrow\infty}f_n(x)=f(x).
\end{align*}
证明:\(\lim_{n\rightarrow\infty}f_n(x)=f(x)\) 关于 \(x\in [a,b]\) 一致。 即$f_n(x)$一致收敛到$f(x).$
\end{theorem}
\begin{remark}
不妨设 \(f(x) = 0\) 的原因:假设当 \(f(x) = 0\) 时结论已经成立,则当 \(f(x)\ne 0\) 时,令 \(g_n(x)=f_n(x)-f(x)\),
此时 \(\{g_n\}_{n\in \mathbb{N}}\in C[a,b]\),且 \(\lim_{n\rightarrow \infty}g_n(x)=0\)。因为对任意 \(x\in [a,b]\),都有 \(f_n(x)\) 关于 \(n\) 单调,
所以对任意 \(x\in [a,b]\),也有 \(g_n(x)\) 关于 \(n\) 单调。于是由假设可知,\(g_n(x)\) 一致收敛到 \(0\)。
因此 \(f_n(x)\) 一致收敛到 \(f(x)\)。故不妨设成立.
\end{remark}
\begin{proof}
不妨设 \(f(x) = 0\),不妨设对 \(\forall x\in [a,b]\),都有 \(f_n(x)\) 关于 \(n\) 单调递减,则由 \(\lim_{n\rightarrow \infty}f_n(x)=0\) 可知,对 \(\forall x\in [a,b]\),都有
\begin{align*}
f_n(x)\geqslant 0,\forall n\in \mathbb{N}_1.
\end{align*}
对 \(\forall \varepsilon>0\),考虑 \(U_n\triangleq \{x\in [a,b]|f_n(x)<\varepsilon\}\),由 \(\lim_{n\rightarrow \infty}f_n(x)=0\) 可得
\begin{align*}
[a,b]\subset \bigcup_{n = 1}^{+\infty}U_n.
\end{align*}
因为 \(\{f_n\}_{n\in \mathbb{N}}\in C[a,b]\),又注意 \(f_{n}^{-1}(-\varepsilon,\varepsilon)=U_n\),所以 \(U_n\) 是开集。
又由于对 \(\forall x\in [a,b]\),都有 \(f_n(x)\) 关于 \(n\) 单调递减,因此 \(U_n\subset U_{n + 1},\forall n\in \mathbb{N}_1\)。这是因为对 \(\forall x\in U_n\),都有 \(f_{n + 1}(x)\leqslant f_n(x)<\varepsilon\),于是 \(x\in U_{n + 1}\)。
从而由有限覆盖定理可知,存在 \(n_1,n_2,\cdots,n_m\in \mathbb{N}_1\),使得
\begin{align*}
[a,b]\subset \bigcup_{k = 1}^mU_{n_k}.
\end{align*}
取 \(N\triangleq \max\{n_1,n_2,\cdots,n_m\}\),则此时 \([a,b]\subset U_N\)。
故对 \(\forall n\geqslant N\),由 \(U_n\subset U_{n + 1},\forall n\in \mathbb{N}_1\) 可知,\([a,b]\subset U_N\subset U_n\),即对 \(\forall n\geqslant N\),都有 \(f_n(x)<\varepsilon,\forall x\in [a,b]\)。
因此 \(f_n(x)\) 一致收敛到 $0$。故原定理得证. 
\end{proof}

\begin{theorem}[Dini 定理函数单调版本]\label{theorem:Dini 定理函数单调版本}
设 \(f_n\in C[a,b],n = 1,2,\cdots\) 都是单调函数。若
\begin{align*}
\lim_{n\rightarrow\infty}f_n(x)=f(x)\in C[a,b].
\end{align*}
则 \(\lim_{n\rightarrow\infty}f_n(x)=f(x)\) 是一致的。即$f_n(x)$一致收敛到$f(x).$ 
\end{theorem}
\begin{proof}
由 Cantor 定理可知,对 \(\forall n\in \mathbb{N}\),都有 \(f_n\) 在 \([a,b]\) 上一致连续。
从而对 \(\forall \varepsilon>0\),存在 \(\delta>0\),使得
\begin{align}
|f(y)-f(x)|<\varepsilon,\forall |y - x|\leqslant \delta.\label{equationDnin0.2-1.1}
\end{align}
设 \(a = x_0<x_1<\cdots <x_m = b\),使得 \(x_i - x_{i + 1}\leqslant \delta,i = 0,1,2,\cdots,m\)。
由 \(\lim_{n\rightarrow \infty}f_n(x)=f(x)\) 可知,存在 \(N\in \mathbb{N}\),使得当 \(n\geqslant N\) 时,有
\begin{align}
|f_n(x_i)-f(x_i)|<\varepsilon,\forall i\in \{0,1,2,\cdots,m\}.\label{equationDnin0.2-1.2}
\end{align}
对 \(\forall x\in [a,b]\),当 \(n\geqslant N\) 时,一定存在 \(i\in \{1,2,\cdots,m\}\),使得 \(x\in [x_{i - 1},x_i]\)。
从而当 \(n\geqslant N\) 时,利用\eqref{equationDnin0.2-1.1}和 \eqref{equationDnin0.2-1.2} 式以及 \(f_n\) 的单调性可得
\begin{align*}
|f_n(x)-f(x)|&\leqslant |f_n(x)-f_n(x_i)|+|f_n(x_i)-f(x_i)|+|f(x_i)-f(x)|<|f_n(x_{i + 1})-f_n(x_i)|+\varepsilon+\varepsilon\\
&\leqslant |f_n(x_{i + 1})-f(x_{i + 1})|+|f(x_{i + 1})-f(x_i)|+|f(x_i)-f_n(x_i)|+2\varepsilon\\
&<\varepsilon+\varepsilon+\varepsilon+2\varepsilon = 5\varepsilon.
\end{align*}
故 \(f_n(x)\) 一致收敛到 \(f(x)\)。 
\end{proof}


\end{document}
\documentclass[../../main.tex]{subfiles}
\graphicspath{{\subfix{../../image/}}} % 指定图片目录,后续可以直接使用图片文件名。

% 例如:
% \begin{figure}[h]
% \centering
% \includegraphics{image-01.01}
% \label{fig:image-01.01}
% \caption{图片标题}
% \end{figure}

\begin{document}

\section{函数性态分析综合问题}

\begin{proposition}\label{proposition:齐次化方法/关于导数乘积不等式问题}
设 $f \in D^s(0,+\infty) \cap C[0,+\infty)$,$s \in \mathbb{N}$ 且满足
\begin{align*}
f^{(j)}(0) &= 0, j = 0, 1, 2, \cdots, s - 1.
\end{align*}
若还存在 $\lambda_1, \lambda_2, \cdots, \lambda_s \geq 0, \sum_{i=1}^{s} \lambda_i = 1, C > 0$,满足
\begin{align}\label{example7.10-0.1}
\left| f^{(s)}(x) \right| &\leq C \left| f(x) \right|^{\lambda_1} \left| f'(x) \right|^{\lambda_2} \cdots \left| f^{(s-1)}(x) \right|^{\lambda_s}, \forall x \geq 0.
\end{align}
证明 $f(x) = 0, \forall x \geq 0$。
\end{proposition}
\begin{note}
我们把下述证明中左右两边各项次数均相同的不等式:$x_{1}^{2\lambda _1}x_{2}^{2\lambda _2}\cdots x_{n}^{2\lambda _n}\leqslant K\left( x_{1}^{2}+x_{2}^{2}+\cdots +x_{n}^{2} \right) ,\forall x_1,x_2,\cdots ,x_n\geqslant 0$称为\textbf{齐次不等式}.(虽然也可以直接利用幂平均不等式得到,但这里我们旨在介绍如何利用\textbf{齐次化方法}证明\textbf{一般的齐次不等式}.)
\end{note}
\begin{proof}
令 $g\left( x \right) =e^{-Mx}\left[ f^2+\left( f' \right) ^2+\left( f'' \right) ^2+\cdots +\left( f^{\left( s-1 \right)} \right) ^2 \right] ,M>0$,显然 $g\left( x \right) \geqslant 0,\forall x\geqslant 0$。则利用均值不等式和条件 \eqref{example7.10-0.1} 式可得,对 $\forall x\geqslant 0$,都有
\begin{align}
g'\left( x \right) &=e^{-Mx}\left[ 2ff' +2f'f'' +2f''f''' +\cdots +2f^{\left( s-1 \right)}f^{\left( s \right)}-Mf^2-M\left( f' \right) ^2-\cdots -M\left( f^{\left( s-1 \right)} \right) ^2 \right] \nonumber \\
&\overset{\text{均值不等式}}{\leqslant}e^{-Mx}\left[ f^2+\left( f' \right) ^2+\left( f' \right) ^2+\left( f'' \right) ^2+\cdots +\left( f^{\left( s-1 \right)} \right) ^2+\left| f^{\left( s \right)} \right|^2-Mf^2-M\left( f' \right) ^2-\cdots -M\left( f^{\left( s-1 \right)} \right) ^2 \right] \nonumber \\
&\overset{\eqref{example7.10-0.1} \text{式}}{\leqslant}e^{-Mx}\left[ \left( 1-M \right) f^2+\left( 2-M \right) \left( f' \right) ^2+\cdots +\left( 2-M \right) \left( f^{\left( s-1 \right)} \right) ^2+C^2\left| f(x) \right|^{2\lambda _1}\left| f'(x) \right|^{2\lambda _2}\cdots \left| f^{(s-1)}(x) \right|^{2\lambda _s} \right] .\label{example7.10-2.1}
\end{align}
我们先证明 $x_{1}^{2\lambda _1}x_{2}^{2\lambda _2}\cdots x_{n}^{2\lambda _n}\leqslant K\left( x_{1}^{2}+x_{2}^{2}+\cdots +x_{n}^{2} \right) ,\forall x_1,x_2,\cdots ,x_n\geqslant 0$。

令 $S\triangleq \left\{ \left( x_1,x_2,\cdots ,x_n \right) |x_{1}^{2}+x_{2}^{2}+\cdots +x_{n}^{2}=1 \right\}$,则 $S$ 是 $\mathbb{R} ^n$ 上的有界闭集,从而 $S$ 是紧集。于是 $x_{1}^{2\lambda _1}x_{2}^{2\lambda _2}\cdots x_{n}^{2\lambda _n}$ 为紧集 $S$ 上的连续函数,故一定存在 $K>0$,使得
\begin{align}
x_{1}^{2\lambda _1}x_{2}^{2\lambda _2}\cdots x_{n}^{2\lambda _n}\leqslant K,\forall \left( x_1,x_2,\cdots ,x_n \right) \in S.\label{example7.10-1.1}
\end{align}
对 $\forall x_1,x_2,\cdots ,x_n\geqslant 0$,固定 $x_1,x_2,\cdots ,x_n$。不妨设 $x_1,x_2,\cdots ,x_n$ 不全为零,否则结论显然成立。取 $L=\frac{1}{\sqrt{x_{1}^{2}+x_{2}^{2}+\cdots +x_{n}^{2}}}>0$,考虑 $\left( Lx_1,Lx_2,\cdots ,Lx_n \right)$,则此时 $\left( Lx_1 \right) ^2+\left( Lx_2 \right) ^2+\cdots +\left( Lx_n \right) ^2=1$,因此 $\left( Lx_1,Lx_2,\cdots ,Lx_n \right) \in S$。
从而由\eqref{example7.10-1.1}式可知
\begin{align*}
\left( Lx_1 \right) ^{2\lambda _1}\left( Lx_2 \right) ^{2\lambda _2}\cdots \left( Lx_n \right) ^{2\lambda _n}\leqslant K.
\end{align*}
于是
\begin{align*}
x_{1}^{2\lambda _1}x_{2}^{2\lambda _2}\cdots x_{n}^{2\lambda _n}\leqslant \frac{K}{L^{2\lambda _1+2\lambda _2+\cdots +2\lambda _n}}=\frac{K}{L^2}=K\left( x_{1}^{2}+x_{2}^{2}+\cdots +x_{n}^{2} \right) .
\end{align*}
故由 $x_1,x_2,\cdots ,x_n$ 的任意性可得
\begin{align}
x_{1}^{2\lambda _1}x_{2}^{2\lambda _2}\cdots x_{n}^{2\lambda _n}\leqslant K\left( x_{1}^{2}+x_{2}^{2}+\cdots +x_{n}^{2} \right) ,\forall x_1,x_2,\cdots ,x_n\geqslant 0.\label{example7.10-1.2}
\end{align}
因此由\eqref{example7.10-2.1} \eqref{example7.10-1.2} 式可得,对 $\forall x\geqslant 0$,都有
\begin{align*}
g'\left( x \right) &\leqslant e^{-Mx}\left[ \left( 1-M \right) f^2+\left( 2-M \right) \left( f' \right) ^2+\cdots +\left( 2-M \right) \left( f^{\left( s-1 \right)} \right) ^2+C^2\left| f(x) \right|^{2\lambda _1}\left| f' (x) \right|^{2\lambda _2}\cdots \left| f^{(s-1)}(x) \right|^{2\lambda _s} \right] \\
&\leqslant e^{-Mx}\left[ \left( 1-M \right) f^2+\left( 2-M \right) \left( f' \right) ^2+\cdots +\left( 2-M \right) \left( f^{\left( s-1 \right)} \right) ^2+KC^2\left( f^2+\left( f' \right) ^2+\left( f' \right) ^2+\left( f'' \right) ^2+\cdots +\left( f^{\left( s-1 \right)} \right) ^2 \right) \right] \\
&=e^{-Mx}\left[ \left( KC^2+1-M \right) f^2+\left( KC^2+2-M \right) \left( f' \right) ^2+\cdots +\left( KC^2+2-M \right) \left( f^{\left( s-1 \right)} \right) ^2 \right] .
\end{align*}
于是任取 $M>KC^2+2$,利用上式就有 $g'\left( x \right) \leqslant 0,\forall x\geq 0$.故 $g\left( x \right)$ 在 $[0,+\infty )$ 上单调递减,因此 $g\left( x \right) \leqslant g\left( 0 \right) =0$。又因为 $g\left( x \right) \geqslant 0,\forall x\geqslant 0$,所以 $g\left( x \right) =0,\forall x\geqslant 0$。故
$f\left( x \right) =f'\left( x \right) =\cdots =f^{\left( s-1 \right)}\left( x \right) =0,\forall x\geqslant 0$。
\end{proof}

\begin{proposition}\label{proposition:关于导数求和不等式问题}
设 $f \in D^s(0,+\infty) \cap C[0,+\infty)$,$s \in \mathbb{N}$ 且满足
\begin{align*}
f^{(j)}(0) &= 0, j = 0, 1, 2, \cdots, s - 1.
\end{align*}
若还存在 $\lambda_1, \lambda_2, \cdots, \lambda_s \geq 0$,满足
\begin{align}\label{aaaaproposition7.19-0.1}
\left| f^{(s)}(x) \right| &\leq \lambda_1 \left| f(x) \right| + \lambda_2 \left| f'(x) \right| + \cdots + \lambda_s \left| f^{(s-1)}(x) \right|, \forall x \geq 0.
\end{align}
证明 $f(x) = 0, \forall x \geq 0$。
\end{proposition}
\begin{proof}
令 $g\left( x \right) =e^{-Mx}\left[ f^2+\left( f' \right) ^2+\left( f'' \right) ^2+\cdots +\left( f^{\left( s-1 \right)} \right) ^2 \right] ,M>0$,显然 $g\left( x \right) \geqslant 0,\forall x\geqslant 0$。则利用均值不等式和条件\eqref{aaaaproposition7.19-0.1} 式可得,对 $\forall x\geqslant 0$,都有
\begin{align}
g'\left( x \right) &=e^{-Mx}\left[ 2ff' +2f'f'' +2f'' f''' +\cdots +2f^{\left( s-1 \right)}f^{\left( s \right)}-Mf^2-M\left( f' \right) ^2-\cdots -M\left( f^{\left( s-1 \right)} \right) ^2 \right] \nonumber \\
&\overset{\text{均值不等式}}{\leqslant}e^{-Mx}\left[ f^2+\left( f' \right) ^2+\left( f' \right) ^2+\left( f'' \right) ^2+\cdots +\left( f^{\left( s-1 \right)} \right) ^2+\left| f^{\left( s \right)} \right|^2-Mf^2-M\left( f' \right) ^2-\cdots -M\left( f^{\left( s-1 \right)} \right) ^2 \right] \nonumber \\
&\overset{\eqref{aaaaproposition7.19-0.1}\text{式}}{\leqslant}e^{-Mx}\left[ \left( 1-M \right) f^2+\left( 2-M \right) \left( f' \right) ^2+\cdots +\left( 2-M \right) \left( f^{\left( s-1 \right)} \right) ^2+\left( \lambda _1\left| f \right|+\lambda _2\left| f' \right|+\cdots +\lambda _s\left| f^{(s-1)} \right| \right) ^2 \right] .\label{aaaaproposition7.19-2.1}
\end{align}
我们先证明 $\left( \lambda _1x_1+\lambda _2x_2+\cdots +\lambda _sx_s \right) ^2\leqslant K\left( x_{1}^{2}+x_{2}^{2}+\cdots +x_{n}^{2} \right) ,\forall x_1,x_2,\cdots ,x_n\geqslant 0$。

令 $S\triangleq \left\{ \left( x_1,x_2,\cdots ,x_n \right) |x_{1}^{2}+x_{2}^{2}+\cdots +x_{n}^{2}=1 \right\} $,则 $S$ 是 $\mathbb{R} ^n$ 上的有界闭集,从而 $S$ 是紧集。于是 $\left( \lambda _1x_1+\lambda _2x_2+\cdots +\lambda _sx_s \right) ^2$ 为紧集 $S$ 上的连续函数,故一定存在 $K>0$,使得
\begin{align}
x_{1}^{2}+x_{2}^{2}+\cdots +x_{n}^{2}\leqslant K,\forall \left( x_1,x_2,\cdots ,x_n \right) \in S。\label{aaaaproposition7.19-1.1}
\end{align}
对 $\forall x_1,x_2,\cdots ,x_n\geqslant 0$,固定 $x_1,x_2,\cdots ,x_n$。不妨设 $x_1,x_2,\cdots ,x_n$ 不全为零,否则结论显然成立。取 $L=\frac{1}{\sqrt{x_{1}^{2}+x_{2}^{2}+\cdots +x_{n}^{2}}}>0$,考虑 $\left( Lx_1,Lx_2,\cdots ,Lx_n \right) $,则此时 $\left( Lx_1 \right) ^2+\left( Lx_2 \right) ^2+\cdots +\left( Lx_n \right) ^2=1$,因此 $\left( Lx_1,Lx_2,\cdots ,Lx_n \right) \in S$。从而由 \eqref{aaaaproposition7.19-1.1} 式可知
\begin{align*}
\left( \lambda _1Lx_1+\lambda _2Lx_2+\cdots +\lambda _sLx_s \right) ^2\leqslant K.
\end{align*}
于是
\begin{align*}
\left( \lambda _1x_1+\lambda _2x_2+\cdots +\lambda _sx_s \right) ^2\leqslant \frac{K}{L^2}=K\left( x_{1}^{2}+x_{2}^{2}+\cdots +x_{n}^{2} \right) .
\end{align*}
故由 $x_1,x_2,\cdots ,x_n$ 的任意性可得
\begin{align}
\left( \lambda _1x_1+\lambda _2x_2+\cdots +\lambda _sx_s \right) ^2\leqslant K\left( x_{1}^{2}+x_{2}^{2}+\cdots +x_{n}^{2} \right) ,\forall x_1,x_2,\cdots ,x_n\geqslant 0.\label{aaaaproposition7.19-1.2}
\end{align}
因此由 \eqref{aaaaproposition7.19-2.1} \eqref{aaaaproposition7.19-1.2}式可得,对 $\forall x\geqslant 0$,都有
\begin{align*}
g'\left( x \right) &\leqslant e^{-Mx}\left[ \left( 1-M \right) f^2+\left( 2-M \right) \left( f' \right) ^2+\cdots +\left( 2-M \right) \left( f^{\left( s-1 \right)} \right) ^2+\left( \lambda _1\left| f \right|+\lambda _2\left| f' \right|+\cdots +\lambda _s\left| f^{(s-1)} \right| \right) ^2 \right] \\
&\leqslant e^{-Mx}\left[ \left( 1-M \right) f^2+\left( 2-M \right) \left( f' \right) ^2+\cdots +\left( 2-M \right) \left( f^{\left( s-1 \right)} \right) ^2+K\left( f^2+\left( f' \right) ^2+\cdots +\left( f^{\left( s-1 \right)} \right) ^2 \right) \right] \\
&=e^{-Mx}\left[ \left( K+1-M \right) f^2+\left( K+2-M \right) \left( f' \right) ^2+\cdots +\left( K+2-M \right) \left( f^{\left( s-1 \right)} \right) ^2 \right] .
\end{align*}
于是任取 $M>K+2$,利用上式就有 $g'\left( x \right) \leqslant 0,\forall x\geqslant 0$。故 $g\left( x \right) $ 在 $[0,+\infty )$ 上单调递减,因此 $g\left( x \right) \leqslant g\left( 0 \right) =0$。又因为 $g\left( x \right) \geqslant 0,\forall x\geqslant 0$,所以 $g\left( x \right) =0,\forall x\geqslant 0$。故 $f\left( x \right) =f'\left( x \right) =\cdots =f^{\left( s-1 \right)}\left( x \right) =0,\forall x\geqslant 0$.
\end{proof}

\begin{proposition}\label{proposition:经典导数一致连续问题}
设 $f'$ 在 $[0,+\infty)$ 一致连续且 $\lim_{x \to +\infty} f(x)$ 存在,证明 $\lim_{x \to +\infty} f'(x) = 0$。
\end{proposition}
\begin{note}
本题也有积分版本:设 $f$ 在 $[0,+\infty)$ 一致连续且 $\int_{0}^{\infty} f(x) \, dx$ 收敛,则 $\lim_{x \to +\infty} f(x) = 0$.(令$F=\int_0^x{f\left( x \right) \mathrm{d}x}$就可以将这个积分版本转化为上述命题)
\end{note}
\begin{proof}
反证,若 $\lim_{x\rightarrow +\infty} f'(x) \ne 0$,则可以不妨设存在 $\delta >0$,$\{ x_n \}_{n=1}^{\infty}$,使得
\begin{align*}
x_n\rightarrow +\infty \text{且} f'(x_n) \geqslant \delta,\forall n\in \mathbb{N}.    
\end{align*}
由 $f'$ 在 $[0,+\infty)$ 上一致连续可知,存在 $\eta >0$,使得对 $\forall n\in \mathbb{N}$,都有
\begin{align*}
f'(x) \geqslant f(x_n) -\frac{\delta}{2} \geqslant \frac{\delta}{2}, \forall x\in [x_n-\eta ,x_n+\eta] 。
\end{align*}
从而对 $\forall n\in \mathbb{N}$,都有
\begin{align*}
f(x_n+\eta) -f(x_n) =\int_{x_n}^{x_n+\eta} f'(x) \mathrm{d}x \geqslant \int_{x_n}^{x_n+\eta} \frac{\delta}{2} \mathrm{d}x = \frac{\delta \eta}{2} > 0, \forall x\in [x_n-\eta ,x_n+\eta] 。
\end{align*}
令$n\rightarrow \infty$,由 $\lim_{x\rightarrow +\infty} f(x) $存在可得 $0 \geqslant \frac{\delta \eta}{2} > 0$,矛盾!故 $\lim_{x\rightarrow +\infty} f'(x) =0$。
\end{proof}

\begin{example}[$\,\,$时滞方程]\label{example:时滞方程}
设 $f$ 在 $\mathbb{R}$ 上可微且满足
\begin{align*}
\lim_{x \to +\infty} f'(x) = 1, \quad
f(x+1) - f(x) = f'(x), \forall x \in \mathbb{R}.
\end{align*}
证明存在常数 $C \in \mathbb{R}$ 使得 $f(x) = x + C, \forall x \in \mathbb{R}$.
\end{example}
\begin{proof}
由 \( f(x+1)-f(x)=f'(x),\forall x\in \mathbb{R} \) 及 \( f\in D(\mathbb{R}) \) 可知 \( f' \in C(\mathbb{R}) \)。
对 \(\forall x_1\in \mathbb{R} \),固定 \( x_1 \),记
\[ A=\{ z>x_1 \mid f'(z) = f'(x_1) \} .\]
由 Lagrange 中值定理及 \( f(x+1)-f(x)=f'(x),\forall x\in \mathbb{R} \) 可知
\[ \exists x_2\in (x_1,x_1+1) \,\,\mathrm{s}.\mathrm{t}.\,\, f'(x_1) = f(x_1+1) - f(x_1) = f'(x_2) .\]
故 \( x_2 \in A \),从而 \( A \) 非空。现在考虑 \( y \triangleq \mathrm{sup}A \in (x_1,+\infty) \),下证 \( y=+\infty \)。
若 \( y<+\infty \),则存在 \(\{ z_{n}' \}_{n=1}^{\infty} \),使得
\[ z_{n}' \rightarrow y \text{且} f'(z_{n}') = f'(x_1) .\]
两边同时令 \( n \rightarrow \infty \),由 \( f' \in C(\mathbb{R}) \) 可得
\[ f'(x_1) = \lim_{n \rightarrow \infty} f'(z_{n}') = f'(y) .\]
又由 Lagrange 中值定理及 \( f(x+1)-f(x)=f'(x),\forall x\in \mathbb{R} \) 可得
\[ \exists y' \in (y,y+1) \,\,\mathrm{s}.\mathrm{t}.\,\, f'(y) = f(y+1) - f(y) = f'(y') .\]
从而 \( y' \in A \) 且 \( y' > y \),这与 \( y=\mathrm{sup}A \) 矛盾!故 \( y=+\infty \)。于是存在 \(\{ z_n \}_{n=1}^{\infty} \),使得
\[ z_n \rightarrow +\infty \text{且} f'(z_n) = f'(x_1) .\]
两边同时令 \( n \rightarrow \infty \),由 \( f' \in C(\mathbb{R}) \) 及 \(\lim_{x \rightarrow +\infty} f'(x) = 1 \) 可得
\[ f'(x_1) = \lim_{n \rightarrow \infty} f'(z_n) = \lim_{x \rightarrow +\infty} f'(x) = 1 .\]
因此由 \( x_1 \) 的任意性得,存在 \( C \) 为常数,使得 \( f(x) = x+C,\forall x\in \mathbb{R} \).
\end{proof}






\chapter{无理数初步}

\begin{theorem}[狄利克雷定理]\label{theorem:狄利克雷定理}
对于无理数\(a\),则存在无穷多对互素的整数\(p,q\)使得\(\left|a - \frac{p}{q}\right|\leq\frac{1}{q^2}\),而对有理数\(a\),这样的互素整数对\((p,q)\)只能是有限个.
\end{theorem}
\begin{note}
这通常称为“\textbf{齐次逼近}”,证明利用抽屉原理即可.
\end{note}

\begin{corollary}
对于实数\(a\),则\(a\)为无理数当且仅当任意\(\varepsilon>0\),存在整数\(x,y\)使得\(0 < |ax - y| < \varepsilon\).
\end{corollary}
\begin{proof}
对任意正整数\(N\),将\([0,1]\)均分为\(N\)个闭区间,每一个长度\(\frac{1}{N}\),则\(n + 1\)个数\(0,\{a\},\{2a\},\cdots,\{Na\}\)全部落在\([0,1]\)中,根据抽屉原理必定有两个数落入同一区间,也即存在\(0\leq i < j\leq N\)使得\(\{ia\},\{ja\}\in\left[\frac{k}{N},\frac{k + 1}{N}\right]\)。
注:因为\(a\)是无理数,所以任意\(i\neq j\)都一定有\(\{ia\}\neq\{ja\}\),否则\(ia - [ia]=ja - [ja]\)意味着\(a\)是有理数。
所以
\[|\{ia\} - \{ja\}|=|(j - i)a - M|\leq\frac{1}{N}\Rightarrow\left|a - \frac{M}{j - i}\right|\leq\frac{1}{N(j - i)}\]
这里\(M\)是一个整数,现在不一定有\(M\)与\(j - i\)互素,但是我们可以将其写成既约分数\(M = up,j - i = uq\),其中\((p,q)=1,u\in\mathbb{N}^+\),代入得到:对任意正整数\(N\),都存在互素的整数\(p,q\),其中\(1\leq q\leq N\)是正整数,使得\(\left|a - \frac{p}{q}\right|\leq\frac{1}{Nq}\leq\frac{1}{q^2}\)。
现在还没有说明“无穷多个”,采用反证法,假如使得\(\left|a - \frac{p}{q}\right|\leq\frac{1}{q^2}\)成立的互素的整数\((p,q)\)只有有限对,记为\((p_1,q_1),\cdots,(p_m,q_m)\),那么(在上面证明的结论里面)依次取\(N = 3,4,\cdots\),则每一个\(N\)都能够对应这\(m\)对\((p,q)\)中的某一个,而\(N = 3,4,\cdots\)是无限的,\(m\)是有限的,所以必定有一个\((p_i,q_i)\)对应了无穷多个正整数\(N\)。
不妨设\(i = 1\),换句话说:存在一列正整数\(N_k\)单调递增趋于正无穷,使得\(\left|a - \frac{p_1}{q_1}\right|\leq\frac{1}{N_kq_1}\)恒成立,令\(k\to\infty\)可知\(a = \frac{p}{q}\)是有理数,导致矛盾。

而如果\(a=\frac{m}{n}\)是有理数,但是有无穷个互素的\((p,q)\)使得\(\left|\frac{m}{n}-\frac{p}{q}\right|\leq\frac{1}{q^2}\),则当\(q\)充分大时,所有这些\((p,q)\)中的\(p\)也都会充分大(相当于同时趋于无穷),然而不等式等价于\(\frac{1}{q}\geq\frac{|mq - np|}{n}\),则当\(p,q\)都充分大时\(mq - np\neq0\)(不然会导致\(p|mq\)结合互素有\(p|m\)(对充分大的\(p\)均成立),显然矛盾),于是\(\frac{1}{q}\geq\frac{|mq - np|}{n}\geq\frac{1}{n}\)导致\(q\)有上界,还是矛盾,结论得证。
\end{proof}



\end{document}
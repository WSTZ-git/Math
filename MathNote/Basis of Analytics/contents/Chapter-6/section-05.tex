\documentclass[../../main.tex]{subfiles}
\graphicspath{{\subfix{../../image/}}} % 指定图片目录,后续可以直接使用图片文件名。

% 例如:
% \begin{figure}[h]
% \centering
% \includegraphics{image-01.01}
% \caption{图片标题}
% \label{fig:image-01.01}
% \end{figure}
% 注意:上述\label{}一定要放在\caption{}之后,否则引用图片序号会只会显示??.

\begin{document}

\section{函数性态分析综合问题}

\subsection{齐次微分不等式问题}

\begin{proposition}\label{proposition:齐次化方法/关于导数乘积不等式问题}
设 $f \in D^s(0,+\infty) \cap C[0,+\infty)$,$s \in \mathbb{N}$ 且满足
\begin{align*}
f^{(j)}(0) &= 0, j = 0, 1, 2, \cdots, s - 1.
\end{align*}
若还存在 $\lambda_1, \lambda_2, \cdots, \lambda_s \geq 0, \sum_{i=1}^{s} \lambda_i = 1, C > 0$,满足
\begin{align}\label{example7.10-0.1}
\left| f^{(s)}(x) \right| &\leq C \left| f(x) \right|^{\lambda_1} \left| f'(x) \right|^{\lambda_2} \cdots \left| f^{(s-1)}(x) \right|^{\lambda_s}, \forall x \geq 0.
\end{align}
证明 $f(x) = 0, \forall x \geq 0$。
\end{proposition}
\begin{note}
我们把下述证明中左右两边各项次数均相同的不等式:$x_{1}^{2\lambda _1}x_{2}^{2\lambda _2}\cdots x_{n}^{2\lambda _n}\leqslant K\left( x_{1}^{2}+x_{2}^{2}+\cdots +x_{n}^{2} \right) ,\forall x_1,x_2,\cdots ,x_n\geqslant 0$称为\textbf{齐次不等式}.(虽然也可以直接利用幂平均不等式得到,但这里我们旨在介绍如何利用\textbf{齐次化方法}证明\textbf{一般的齐次不等式}.)
\end{note}
\begin{proof}
令 $g\left( x \right) =e^{-Mx}\left[ f^2+\left( f' \right) ^2+\left( f'' \right) ^2+\cdots +\left( f^{\left( s-1 \right)} \right) ^2 \right] ,M>0$,显然 $g\left( x \right) \geqslant 0,\forall x\geqslant 0$。则利用均值不等式和条件 \eqref{example7.10-0.1} 式可得,对 $\forall x\geqslant 0$,都有
\begin{align}
g'\left( x \right) &=e^{-Mx}\left[ 2ff' +2f'f'' +2f''f''' +\cdots +2f^{\left( s-1 \right)}f^{\left( s \right)}-Mf^2-M\left( f' \right) ^2-\cdots -M\left( f^{\left( s-1 \right)} \right) ^2 \right] \nonumber \\
&\overset{\text{均值不等式}}{\leqslant}e^{-Mx}\left[ f^2+\left( f' \right) ^2+\left( f' \right) ^2+\left( f'' \right) ^2+\cdots +\left( f^{\left( s-1 \right)} \right) ^2+\left| f^{\left( s \right)} \right|^2-Mf^2-M\left( f' \right) ^2-\cdots -M\left( f^{\left( s-1 \right)} \right) ^2 \right] \nonumber \\
&\overset{\eqref{example7.10-0.1} \text{式}}{\leqslant}e^{-Mx}\left[ \left( 1-M \right) f^2+\left( 2-M \right) \left( f' \right) ^2+\cdots +\left( 2-M \right) \left( f^{\left( s-1 \right)} \right) ^2+C^2\left| f(x) \right|^{2\lambda _1}\left| f'(x) \right|^{2\lambda _2}\cdots \left| f^{(s-1)}(x) \right|^{2\lambda _s} \right] .\label{example7.10-2.1}
\end{align}
我们先证明 $x_{1}^{2\lambda _1}x_{2}^{2\lambda _2}\cdots x_{n}^{2\lambda _n}\leqslant K\left( x_{1}^{2}+x_{2}^{2}+\cdots +x_{n}^{2} \right) ,\forall x_1,x_2,\cdots ,x_n\geqslant 0$。

令 $S\triangleq \left\{ \left( x_1,x_2,\cdots ,x_n \right) |x_{1}^{2}+x_{2}^{2}+\cdots +x_{n}^{2}=1 \right\}$,则 $S$ 是 $\mathbb{R} ^n$ 上的有界闭集,从而 $S$ 是紧集。于是 $x_{1}^{2\lambda _1}x_{2}^{2\lambda _2}\cdots x_{n}^{2\lambda _n}$ 为紧集 $S$ 上的连续函数,故一定存在 $K>0$,使得
\begin{align}
x_{1}^{2\lambda _1}x_{2}^{2\lambda _2}\cdots x_{n}^{2\lambda _n}\leqslant K,\forall \left( x_1,x_2,\cdots ,x_n \right) \in S.\label{example7.10-1.1}
\end{align}
对 $\forall x_1,x_2,\cdots ,x_n\geqslant 0$,固定 $x_1,x_2,\cdots ,x_n$。不妨设 $x_1,x_2,\cdots ,x_n$ 不全为零,否则结论显然成立。取 $L=\frac{1}{\sqrt{x_{1}^{2}+x_{2}^{2}+\cdots +x_{n}^{2}}}>0$,考虑 $\left( Lx_1,Lx_2,\cdots ,Lx_n \right)$,则此时 $\left( Lx_1 \right) ^2+\left( Lx_2 \right) ^2+\cdots +\left( Lx_n \right) ^2=1$,因此 $\left( Lx_1,Lx_2,\cdots ,Lx_n \right) \in S$。
从而由\eqref{example7.10-1.1}式可知
\begin{align*}
\left( Lx_1 \right) ^{2\lambda _1}\left( Lx_2 \right) ^{2\lambda _2}\cdots \left( Lx_n \right) ^{2\lambda _n}\leqslant K.
\end{align*}
于是
\begin{align*}
x_{1}^{2\lambda _1}x_{2}^{2\lambda _2}\cdots x_{n}^{2\lambda _n}\leqslant \frac{K}{L^{2\lambda _1+2\lambda _2+\cdots +2\lambda _n}}=\frac{K}{L^2}=K\left( x_{1}^{2}+x_{2}^{2}+\cdots +x_{n}^{2} \right) .
\end{align*}
故由 $x_1,x_2,\cdots ,x_n$ 的任意性可得
\begin{align}
x_{1}^{2\lambda _1}x_{2}^{2\lambda _2}\cdots x_{n}^{2\lambda _n}\leqslant K\left( x_{1}^{2}+x_{2}^{2}+\cdots +x_{n}^{2} \right) ,\forall x_1,x_2,\cdots ,x_n\geqslant 0.\label{example7.10-1.2}
\end{align}
因此由\eqref{example7.10-2.1} \eqref{example7.10-1.2} 式可得,对 $\forall x\geqslant 0$,都有
\begin{align*}
g'\left( x \right) &\leqslant e^{-Mx}\left[ \left( 1-M \right) f^2+\left( 2-M \right) \left( f' \right) ^2+\cdots +\left( 2-M \right) \left( f^{\left( s-1 \right)} \right) ^2+C^2\left| f(x) \right|^{2\lambda _1}\left| f' (x) \right|^{2\lambda _2}\cdots \left| f^{(s-1)}(x) \right|^{2\lambda _s} \right] \\
&\leqslant e^{-Mx}\left[ \left( 1-M \right) f^2+\left( 2-M \right) \left( f' \right) ^2+\cdots +\left( 2-M \right) \left( f^{\left( s-1 \right)} \right) ^2+KC^2\left( f^2+\left( f' \right) ^2+\left( f' \right) ^2+\left( f'' \right) ^2+\cdots +\left( f^{\left( s-1 \right)} \right) ^2 \right) \right] \\
&=e^{-Mx}\left[ \left( KC^2+1-M \right) f^2+\left( KC^2+2-M \right) \left( f' \right) ^2+\cdots +\left( KC^2+2-M \right) \left( f^{\left( s-1 \right)} \right) ^2 \right] .
\end{align*}
于是任取 $M>KC^2+2$,利用上式就有 $g'\left( x \right) \leqslant 0,\forall x\geq 0$.故 $g\left( x \right)$ 在 $[0,+\infty )$ 上单调递减,因此 $g\left( x \right) \leqslant g\left( 0 \right) =0$。又因为 $g\left( x \right) \geqslant 0,\forall x\geqslant 0$,所以 $g\left( x \right) =0,\forall x\geqslant 0$。故
$f\left( x \right) =f'\left( x \right) =\cdots =f^{\left( s-1 \right)}\left( x \right) =0,\forall x\geqslant 0$。
\end{proof}

\begin{proposition}\label{proposition:关于导数求和不等式问题}
设 $f \in D^s(0,+\infty) \cap C[0,+\infty)$,$s \in \mathbb{N}$ 且满足
\begin{align*}
f^{(j)}(0) &= 0, j = 0, 1, 2, \cdots, s - 1.
\end{align*}
若还存在 $\lambda_1, \lambda_2, \cdots, \lambda_s \geq 0$,满足
\begin{align}\label{aaaaproposition7.19-0.1}
\left| f^{(s)}(x) \right| &\leq \lambda_1 \left| f(x) \right| + \lambda_2 \left| f'(x) \right| + \cdots + \lambda_s \left| f^{(s-1)}(x) \right|, \forall x \geq 0.
\end{align}
证明 $f(x) = 0, \forall x \geq 0$。
\end{proposition}
\begin{proof}
令 $g\left( x \right) =e^{-Mx}\left[ f^2+\left( f' \right) ^2+\left( f'' \right) ^2+\cdots +\left( f^{\left( s-1 \right)} \right) ^2 \right] ,M>0$,显然 $g\left( x \right) \geqslant 0,\forall x\geqslant 0$。则利用均值不等式和条件\eqref{aaaaproposition7.19-0.1} 式可得,对 $\forall x\geqslant 0$,都有
\begin{align}
g'\left( x \right) &=e^{-Mx}\left[ 2ff' +2f'f'' +2f'' f''' +\cdots +2f^{\left( s-1 \right)}f^{\left( s \right)}-Mf^2-M\left( f' \right) ^2-\cdots -M\left( f^{\left( s-1 \right)} \right) ^2 \right] \nonumber \\
&\overset{\text{均值不等式}}{\leqslant}e^{-Mx}\left[ f^2+\left( f' \right) ^2+\left( f' \right) ^2+\left( f'' \right) ^2+\cdots +\left( f^{\left( s-1 \right)} \right) ^2+\left| f^{\left( s \right)} \right|^2-Mf^2-M\left( f' \right) ^2-\cdots -M\left( f^{\left( s-1 \right)} \right) ^2 \right] \nonumber \\
&\overset{\eqref{aaaaproposition7.19-0.1}\text{式}}{\leqslant}e^{-Mx}\left[ \left( 1-M \right) f^2+\left( 2-M \right) \left( f' \right) ^2+\cdots +\left( 2-M \right) \left( f^{\left( s-1 \right)} \right) ^2+\left( \lambda _1\left| f \right|+\lambda _2\left| f' \right|+\cdots +\lambda _s\left| f^{(s-1)} \right| \right) ^2 \right] .\label{aaaaproposition7.19-2.1}
\end{align}
我们先证明 $\left( \lambda _1x_1+\lambda _2x_2+\cdots +\lambda _sx_s \right) ^2\leqslant K\left( x_{1}^{2}+x_{2}^{2}+\cdots +x_{n}^{2} \right) ,\forall x_1,x_2,\cdots ,x_n\geqslant 0$。

令 $S\triangleq \left\{ \left( x_1,x_2,\cdots ,x_n \right) |x_{1}^{2}+x_{2}^{2}+\cdots +x_{n}^{2}=1 \right\} $,则 $S$ 是 $\mathbb{R} ^n$ 上的有界闭集,从而 $S$ 是紧集。于是 $\left( \lambda _1x_1+\lambda _2x_2+\cdots +\lambda _sx_s \right) ^2$ 为紧集 $S$ 上的连续函数,故一定存在 $K>0$,使得
\begin{align}
x_{1}^{2}+x_{2}^{2}+\cdots +x_{n}^{2}\leqslant K,\forall \left( x_1,x_2,\cdots ,x_n \right) \in S。\label{aaaaproposition7.19-1.1}
\end{align}
对 $\forall x_1,x_2,\cdots ,x_n\geqslant 0$,固定 $x_1,x_2,\cdots ,x_n$。不妨设 $x_1,x_2,\cdots ,x_n$ 不全为零,否则结论显然成立。取 $L=\frac{1}{\sqrt{x_{1}^{2}+x_{2}^{2}+\cdots +x_{n}^{2}}}>0$,考虑 $\left( Lx_1,Lx_2,\cdots ,Lx_n \right) $,则此时 $\left( Lx_1 \right) ^2+\left( Lx_2 \right) ^2+\cdots +\left( Lx_n \right) ^2=1$,因此 $\left( Lx_1,Lx_2,\cdots ,Lx_n \right) \in S$。从而由 \eqref{aaaaproposition7.19-1.1} 式可知
\begin{align*}
\left( \lambda _1Lx_1+\lambda _2Lx_2+\cdots +\lambda _sLx_s \right) ^2\leqslant K.
\end{align*}
于是
\begin{align*}
\left( \lambda _1x_1+\lambda _2x_2+\cdots +\lambda _sx_s \right) ^2\leqslant \frac{K}{L^2}=K\left( x_{1}^{2}+x_{2}^{2}+\cdots +x_{n}^{2} \right) .
\end{align*}
故由 $x_1,x_2,\cdots ,x_n$ 的任意性可得
\begin{align}
\left( \lambda _1x_1+\lambda _2x_2+\cdots +\lambda _sx_s \right) ^2\leqslant K\left( x_{1}^{2}+x_{2}^{2}+\cdots +x_{n}^{2} \right) ,\forall x_1,x_2,\cdots ,x_n\geqslant 0.\label{aaaaproposition7.19-1.2}
\end{align}
因此由 \eqref{aaaaproposition7.19-2.1} \eqref{aaaaproposition7.19-1.2}式可得,对 $\forall x\geqslant 0$,都有
\begin{align*}
g'\left( x \right) &\leqslant e^{-Mx}\left[ \left( 1-M \right) f^2+\left( 2-M \right) \left( f' \right) ^2+\cdots +\left( 2-M \right) \left( f^{\left( s-1 \right)} \right) ^2+\left( \lambda _1\left| f \right|+\lambda _2\left| f' \right|+\cdots +\lambda _s\left| f^{(s-1)} \right| \right) ^2 \right] \\
&\leqslant e^{-Mx}\left[ \left( 1-M \right) f^2+\left( 2-M \right) \left( f' \right) ^2+\cdots +\left( 2-M \right) \left( f^{\left( s-1 \right)} \right) ^2+K\left( f^2+\left( f' \right) ^2+\cdots +\left( f^{\left( s-1 \right)} \right) ^2 \right) \right] \\
&=e^{-Mx}\left[ \left( K+1-M \right) f^2+\left( K+2-M \right) \left( f' \right) ^2+\cdots +\left( K+2-M \right) \left( f^{\left( s-1 \right)} \right) ^2 \right] .
\end{align*}
于是任取 $M>K+2$,利用上式就有 $g'\left( x \right) \leqslant 0,\forall x\geqslant 0$。故 $g\left( x \right) $ 在 $[0,+\infty )$ 上单调递减,因此 $g\left( x \right) \leqslant g\left( 0 \right) =0$。又因为 $g\left( x \right) \geqslant 0,\forall x\geqslant 0$,所以 $g\left( x \right) =0,\forall x\geqslant 0$。故 $f\left( x \right) =f'\left( x \right) =\cdots =f^{\left( s-1 \right)}\left( x \right) =0,\forall x\geqslant 0$.
\end{proof}

\subsection{其他}

\begin{proposition}\label{proposition:经典导数一致连续问题}
设 $f'$ 在 $[0,+\infty)$ 一致连续且 $\lim_{x \to +\infty} f(x)$ 存在,证明 $\lim_{x \to +\infty} f'(x) = 0$。
\end{proposition}
\begin{note}
本题也有积分版本:设 $f$ 在 $[0,+\infty)$ 一致连续且 $\int_{0}^{\infty} f(x) \, dx$ 收敛,则 $\lim_{x \to +\infty} f(x) = 0$.(令$F=\int_0^x{f\left( x \right) \mathrm{d}x}$就可以将这个积分版本转化为上述命题)
\end{note}
\begin{proof}
反证,若 $\lim_{x\rightarrow +\infty} f'(x) \ne 0$,则可以不妨设存在 $\delta >0$,$\{ x_n \}_{n=1}^{\infty}$,使得
\begin{align*}
x_n\rightarrow +\infty \text{且} f'(x_n) \geqslant \delta,\forall n\in \mathbb{N}.    
\end{align*}
由 $f'$ 在 $[0,+\infty)$ 上一致连续可知,存在 $\eta >0$,使得对 $\forall n\in \mathbb{N}$,都有
\begin{align*}
f'(x) \geqslant f'(x_n) -\frac{\delta}{2} \geqslant \frac{\delta}{2}, \forall x\in [x_n-\eta ,x_n+\eta] 。
\end{align*}
从而对 $\forall n\in \mathbb{N}$,都有
\begin{align*}
f(x_n+\eta) -f(x_n) =\int_{x_n}^{x_n+\eta} f'(x) \mathrm{d}x \geqslant \int_{x_n}^{x_n+\eta} \frac{\delta}{2} \mathrm{d}x = \frac{\delta \eta}{2} > 0, \forall x\in [x_n-\eta ,x_n+\eta] 。
\end{align*}
令$n\rightarrow \infty$,由 $\lim_{x\rightarrow +\infty} f(x) $存在可得 $0 \geqslant \frac{\delta \eta}{2} > 0$,矛盾!故 $\lim_{x\rightarrow +\infty} f'(x) =0$。
\end{proof}

\begin{example}[$\,\,$时滞方程]\label{example:时滞方程}
设 $f$ 在 $\mathbb{R}$ 上可微且满足
\begin{align*}
\lim_{x \to +\infty} f'(x) = 1, \quad
f(x+1) - f(x) = f'(x), \forall x \in \mathbb{R}.
\end{align*}
证明存在常数 $C \in \mathbb{R}$ 使得 $f(x) = x + C, \forall x \in \mathbb{R}$.
\end{example}
\begin{proof}
由 \( f(x+1)-f(x)=f'(x),\forall x\in \mathbb{R} \) 及 \( f\in D(\mathbb{R}) \) 可知 \( f' \in C(\mathbb{R}) \)。
对 \(\forall x_1\in \mathbb{R} \),固定 \( x_1 \),记
\[ A=\{ z>x_1 \mid f'(z) = f'(x_1) \} .\]
由 Lagrange 中值定理及 \( f(x+1)-f(x)=f'(x),\forall x\in \mathbb{R} \) 可知
\[ \exists x_2\in (x_1,x_1+1) \,\,\mathrm{s}.\mathrm{t}.\,\, f'(x_1) = f(x_1+1) - f(x_1) = f'(x_2) .\]
故 \( x_2 \in A \),从而 \( A \) 非空。现在考虑 \( y \triangleq \mathrm{sup}A \in (x_1,+\infty) \),下证 \( y=+\infty \)。
若 \( y<+\infty \),则存在 \(\{ z_{n}' \}_{n=1}^{\infty} \),使得
\[ z_{n}' \rightarrow y \text{且} f'(z_{n}') = f'(x_1) .\]
两边同时令 \( n \rightarrow \infty \),由 \( f' \in C(\mathbb{R}) \) 可得
\[ f'(x_1) = \lim_{n \rightarrow \infty} f'(z_{n}') = f'(y) .\]
又由 Lagrange 中值定理及 \( f(x+1)-f(x)=f'(x),\forall x\in \mathbb{R} \) 可得
\[ \exists y' \in (y,y+1) \,\,\mathrm{s}.\mathrm{t}.\,\, f'(y) = f(y+1) - f(y) = f'(y') .\]
从而 \( y' \in A \) 且 \( y' > y \),这与 \( y=\mathrm{sup}A \) 矛盾!故 \( y=+\infty \)。于是存在 \(\{ z_n \}_{n=1}^{\infty} \),使得
\[ z_n \rightarrow +\infty \text{且} f'(z_n) = f'(x_1) .\]
两边同时令 \( n \rightarrow \infty \),由 \( f' \in C(\mathbb{R}) \) 及 \(\lim_{x \rightarrow +\infty} f'(x) = 1 \) 可得
\[ f'(x_1) = \lim_{n \rightarrow \infty} f'(z_n) = \lim_{x \rightarrow +\infty} f'(x) = 1 .\]
因此由 \( x_1 \) 的任意性得,存在 \( C \) 为常数,使得 \( f(x) = x+C,\forall x\in \mathbb{R} \).
\end{proof}

\begin{example}
设 $f\in C^2(\mathbb{R})$ 满足 $f(1)\leqslant0$ 以及
\begin{align*}
\lim_{x\to\infty}[f(x)-|x|]=0.
\tag{12.27}
\end{align*}
证明:
(1):存在 $\xi\in(1,+\infty)$, 使得 $f'(\xi)>1$.

(2):存在 $\eta\in\mathbb{R}$, 使得 $f''(\eta)=0$.
\end{example}
\begin{proof}
(1)如果对任何 $x\in(1,+\infty)$, 都有 $f'(x)\leqslant1$, 那么 $[f(x)-x]'\leqslant0$ 知 $f(x)-x$ 在 $[1,+\infty)$ 单调递减. 从而
\begin{align*}
-1\geqslant f(1)-1\geqslant\lim_{x\to+\infty}[f(x)-x]=\lim_{x\to\infty}[f(x)-|x|]=0,
\end{align*}
这就是一个矛盾! 于是我们证明了 (1).

(2)
若对任何 $x\in\mathbb{R}$, 我们有 $f''(x)\neq0$.从而$f''(x)$要么恒大于零,要么恒小于零,否则由零点存在定理可得矛盾!任取$\xi \in \mathbb{R}$.

当 $f''(x)>0,\forall x\in\mathbb{R}$, 我们知道 $f$ 在 $\mathbb{R}$ 上是下凸函数. 由 (1) 和下凸函数切线总是在函数下方, 我们知道
\begin{align*}
f(x)\geqslant f(\xi)+f'(\xi)(x-\xi),\forall x>\xi.
\end{align*}
于是
\begin{align*}
0=\lim_{x\to+\infty}[f(x)-x]\geqslant\lim_{x\to+\infty}[f(\xi)-f'(\xi)\xi+(f'(\xi)-1)x]=+\infty,
\end{align*}
这就是一个矛盾!

当 $f''(x)<0,\forall x\in\mathbb{R}$, 我们知道 $f$ 在 $\mathbb{R}$ 上是上凸函数. 由 (1) 和上凸函数切线总是在函数上方, 我们有
\begin{align*}
f(x)\leqslant f(\xi)+f'(\xi)(x-\xi),\forall x<\xi.
\end{align*}
于是
\begin{align*}
0=\lim_{x\to-\infty}[f(x)+x]\leqslant\lim_{x\to-\infty}[f(\xi)-f'(\xi)\xi+(f'(\xi)+1)x]=-\infty,
\end{align*}
这就是一个矛盾! 因此我们证明了 (2). 
\end{proof}

\begin{example}
设 \(f\) 在 \([a,b]\) 上每一个点极限都存在,证明:\(f\) 在 \([a,b]\) 有界。
\end{example}
\begin{note}
极限存在必然局部有界,本题就是说局部有界可以推出在紧集上有界。在大量问题中会有一个公共现象:即\textbf{局部的性质等价于在所有紧集上的性质}。证明的想法就是有限覆盖。 
\end{note}
\begin{proof}
对 \(\forall c\in [a,b]\),由 \(\lim_{x\rightarrow c}f(x)\) 存在可知,存在 \(c\) 的邻域 \(U_c\) 和 \(M>0\),使得
\begin{align*}
\sup_{x\in U_c\cap [a,b]}|f(x)|\leqslant M_c.
\end{align*}
注意 \([a,b]\subset \bigcup_{c\in [a,b]}U_c\),由有限覆盖定理得,存在 \(c_1,c_2,\cdots,c_n\in [a,b]\),使得
\begin{align*}
[a,b]\subset \bigcup_{k = 1}^nU_{c_k}. 
\end{align*}
故 \(\sup_{x\in [a,b]}|f(x)|\leqslant \max_{1\leqslant k\leqslant n}M_k\)。 
\end{proof}

\begin{example}
设 \(f\) 是 \((a, +\infty)\) 有界连续函数, 证明对任何实数 \(T\) , 存在数列 \(\lim_{n \to \infty} x_n = +\infty\) 使得
\begin{align*}
\lim_{n \to \infty} [f(x_n + T) - f(x_n)] = 0.
\end{align*}
\end{example}
\begin{remark}
因为\(\vert f(x + T) - f(x)\vert \geqslant 0\) , 所以
\begin{align*}
0\leqslant \varliminf_{x \to +\infty} \vert f(x + T) - f(x)\vert < \varlimsup_{x \to +\infty} \vert f(x + T) - f(x)\vert
\end{align*}
原结论的反面只用考虑\(\varliminf_{x \to +\infty} \vert f(x + T) - f(x)\vert\) 即可. 若\(\varliminf_{x \to +\infty} \vert f(x + T) - f(x)\vert = 0\) , 则一定存在子列 \(x_n \to +\infty\) , 使得结论成立.
故原结论的反面就是\(\varliminf_{x \to +\infty} \vert f(x + T) - f(x)\vert > 0\) .
\end{remark}
\begin{note}
考虑反证法之后,再进行定性分析(画$f(x)$的大致走势图),就容易找到矛盾.
\end{note}
\begin{proof}
反证,
假设\(\varliminf_{x \to +\infty} \vert f(x + T) - f(x)\vert > 0\) , 则存在\(\varepsilon_0 > 0\) , \(X > 0\) , 使得
\begin{align}
\vert f(x + T) - f(x)\vert \geqslant \varepsilon_0,\quad \forall x \geqslant X \label{example0.4section05----1.1}
\end{align}
令\(g(x) \triangleq f(x + T) - f(x)\) , 则若存在 \(x_1, x_2 \geqslant X\) , 使得
\(g(x_1) = f(x_1 + T) - f(x_1) \geqslant \varepsilon_0 > 0\) , \(\quad g(x_2) = f(x_2 + T) - f(x_2) \leqslant -\varepsilon_0 < 0\) .
不妨设 \(x_1 < x_2\) , 由 \(g\) 连续及介值定理可知, 存在 \(x_3 \in (x_1, x_2)\) , 使得
\begin{align*}
g(x_3) = f(x_3 + T) - f(x_3) = 0
\end{align*}
与\eqref{example0.4section05----1.1} 式矛盾! 故 \(g(x) \triangleq f(x + T) - f(x)\) 在\([X, +\infty)\) 上要么恒大于\(\varepsilon_0\) , 要么恒小于\(\varepsilon_0\) . 于是不妨设
\begin{align}
f(x + T) - f(x) \geqslant \varepsilon_0,\quad \forall x \geqslant X \label{example0.4section05----1.2}
\end{align}
从而对\(\forall k \in \mathbb{N}\) , 存在 \(X_k \geqslant X\) , 使得当 \(x \geqslant X_1\) 时, 有
\(x + (k - 1)T > X\) .
于是由\eqref{example0.4section05----1.1}式可得
\begin{align}
f(x + kT) - f(x + (k - 1)T) \geqslant \varepsilon_0,\quad \forall x \geqslant X_k \label{example0.4section05----1.3}
\end{align}
因此对\(\forall n \in \mathbb{N}\) , 取 \(K_n = \max\{X_1, X_2, \cdots, X_k\}\) , 则由\eqref{example0.4section05----1.3}式可知
\(f(x + kT) - f(x + (k - 1)T)\) , \(\forall x \geqslant K_n\) , \(\forall k \in \{1, 2, \cdots, n\}\)
进而对上式求和可得, 对\(\forall n \in \mathbb{N}\) , 都有
\begin{align*}
\sum_{k = 1}^n [f(x + kT) - f(x + (k - 1)T)] = f(x + nT) - f(x) \geqslant n\varepsilon_0,\quad \forall x \geqslant K_n
\end{align*}
任取 \(x_0 \geqslant K_n\) , 则
\(f(x_0 + nT) - f(x_0) \geqslant n\varepsilon_0,\quad \forall n \in \mathbb{N}\) .
令 \(n \to \infty\) , 得\(\lim_{x \to +\infty} f(x) = +\infty\) . 这与 \(f\) 在\((a, +\infty)\) 上有界矛盾! 
\end{proof}

\begin{proposition}
\begin{enumerate}
\item 设 \(f_n\in C[a,b]\) 且关于 \([a,b]\) 一致的有
\begin{align*}
\lim_{n \to \infty}f_n(x)=f(x).
\end{align*}
证明: 对 \(\{x_n\}\subset [a,b]\), \(\lim_{n \to \infty}x_n = c\), 有
\begin{align*}
\lim_{n \to \infty}f_n(x_n)=f(c).
\end{align*}

\item 设 \(f_n(x):\mathbb{R}\to\mathbb{R}\) 满足对任何 \(x_0\in\mathbb{R}\) 和 \(\{x_n\}_{n = 1}^{\infty}\subset\mathbb{R}\), \(\lim_{n \to \infty}x_n = x_0\), 都有
\begin{align*}
\lim_{n \to \infty}f_n(x_n)=f(x_0),
\end{align*}
证明: \(f\in C(\mathbb{R})\). 
\end{enumerate}
\end{proposition}
\begin{proof}
\begin{enumerate}
\item 由\(f_n\)一致收敛到\(f(x)\)可知, 对\(\forall \varepsilon > 0\), 存在\(N_0\in \mathbb{N}\), 使得对\(\forall N\geqslant N_0\), 当\(n\geqslant N\)时, 对\(\forall x\in [a,b]\), 都有
\begin{align*}
|f_n(x) - f_N(x)| < \varepsilon.
\end{align*}
从而由上式可得
\begin{align*}
|f_n(x_n) - f(c)| &\leqslant |f_n(x_n) - f_N(x_n)| + |f_N(x_n) - f_N(c)| + |f_N(c) - f(c)|\\
&\leqslant \varepsilon + |f_N(x_n) - f_N(c)| + |f_N(c) - f(c)|.
\end{align*}
令\(n\rightarrow +\infty\), 由\(f\)的连续性及\(\lim_{n\rightarrow \infty}x_n = c\)可得
\begin{align*}
\overline{\lim_{n\rightarrow \infty}}|f_n(x_n) - f(c)| &\leqslant \varepsilon + |f_N(c) - f(c)|.
\end{align*}
再令\(N\rightarrow +\infty\), 由\(\lim_{n\rightarrow \infty}f_n(x) = f(x)\), \(\forall x\in [a,b]\)可知
\begin{align*}
\overline{\lim_{n\rightarrow \infty}}|f_n(x_n) - f(c)| &\leqslant \varepsilon.
\end{align*}
令\(\varepsilon \rightarrow 0^+\), 得\(\overline{\lim_{n\rightarrow \infty}}|f_n(x_n) - f(c)| \leqslant 0\). 故\(\lim_{n\rightarrow \infty}f_n(x_n) = f(c)\). 

\item 反证, 若\(f\)在\(x_0\in \mathbb{R}\)处不连续, 则存在\(\varepsilon_0 > 0\), 使得\(\forall m\in \mathbb{N}\), 存在\(y_m\in (x_0 - \frac{1}{m}, x_0 + \frac{1}{m})\), 使得
\begin{align}
|f(y_m) - f(x_0)| \geqslant \varepsilon_0. \label{equation--0.4section051.1}
\end{align}
对\(\forall m\in \mathbb{N}\), 令条件中的\(x_0 = y_m\), \(x_n\equiv y_m\), \(\forall n\in \mathbb{N}\), 从而由条件可知\(\lim_{n\rightarrow \infty}|f_n(y_m) - f(y_m)| = 0\), \(m = 1,2,\cdots\),
故对\(\forall m\in \mathbb{N}\), 存在严格递增的数列\(n_m\rightarrow +\infty\), 使得
\begin{align}
|f_{n_m}(y_m) - f(y_m)| < \frac{\varepsilon_0}{2}. \label{equation--0.4section051.2}
\end{align}
从而由\eqref{equation--0.4section051.1}\eqref{equation--0.4section051.2}式可知, 对\(\forall m\in \mathbb{N}\), 都有
\begin{align}
|f(y_{n_m}) - f(x_0)| &\geqslant \varepsilon_0, \label{equation--0.4section051.3}\\
|f_{n_m}(y_{n_m}) - f(y_{n_m})| &< \frac{\varepsilon_0}{2}. \label{equation--0.4section051.4}
\end{align}
因此由\eqref{equation--0.4section051.3}\eqref{equation--0.4section051.4}式可得, 对\(\forall m\in \mathbb{N}\), 都有
\begin{align}
|f_{n_m}(y_{n_m}) - f(x_0)| &\geqslant |f(y_{n_m}) - f(x_0)| - |f_{n_m}(y_{n_m}) - f(y_{n_m})| \geqslant \varepsilon_0 - \frac{\varepsilon_0}{2} = \frac{\varepsilon_0}{2}. \label{equation--0.4section051.5}
\end{align}
注意到\(y_m\rightarrow x_0\), 于是\(y_{n_m}\rightarrow x_0\). 从而由已知条件可知\(\lim_{m\rightarrow \infty}f_{n_m}(y_{n_m}) = f(x_0)\). 这与\eqref{equation--0.4section051.5}式矛盾! 故\(f\in C(\mathbb{R})\). 
\end{enumerate}
\end{proof}

\begin{example}
设 \(g\in C(\mathbb{R})\) 且以 \(T > 0\) 为周期, 且有
\begin{align}
f(f(x))=-x^3 + g(x).\label{equation0.5example-1.1}
\end{align}
证明:不存在 \(f\in C(\mathbb{R})\), 使得\eqref{equation0.5example-1.1}式成立. 
\end{example}
\begin{proof}
由\hyperref[proposition:连续的周期函数的基本性质]{连续的周期函数的基本性质}可知,存在$M>0,$使得$|g(x)|\leqslant M.$反证,假设存在 \(f\in C(\mathbb{R})\), 使得\eqref{equation0.5example-1.1}式成立. 则
\begin{align}
&\underset{x\rightarrow +\infty}{\lim}f\left( f\left( x \right) \right) =\underset{x\rightarrow +\infty}{\lim}\left( -x^3+g\left( x \right) \right) =-\infty ,\label{equation0.5example-2.1}
\\
&\underset{x\rightarrow -\infty}{\lim}f\left( f\left( x \right) \right) =\underset{x\rightarrow -\infty}{\lim}\left( -x^3+g\left( x \right) \right) =+\infty .\label{equation0.5example-2.2}
\end{align}
假设\(\lim_{x\rightarrow +\infty}f(x) = A\in \mathbb{R}\), 则存在\(x_n\rightarrow +\infty\), 使得\(f(x_n)\rightarrow A\). 从而由\eqref{equation0.5example-1.1}式可得
\begin{align*}
f(A) = \lim_{n\rightarrow \infty}f(f(x_n)) = \lim_{n\rightarrow \infty}(-x_{n}^{3}+g(x_n)) = -\infty.
\end{align*}
上式显然矛盾! 又因为\(f\in C(\mathbb{R})\), 所以\(\lim_{x\rightarrow +\infty}f(x) = +\infty\)或\(-\infty\).
否则, 当\(x\rightarrow +\infty\)时,\(f(x)\)振荡, 则由零点存在定理可知, 存在\(y_n\rightarrow +\infty\), 使得\(f(y_n) = 0\), \(n = 1,2,\cdots\).
从而由\eqref{equation0.5example-2.1}式可知
\begin{align*}
-\infty = \lim_{x\rightarrow +\infty}f(f(x)) = \lim_{n\rightarrow \infty}f(f(y_n)) = f(0).
\end{align*}
显然矛盾!

(i)若\(\lim_{x\rightarrow +\infty}f(x) = +\infty\), 则
\begin{align*}
+\infty = \lim_{x\rightarrow +\infty}f(x) = f(+\infty) = \lim_{x\rightarrow +\infty}f(f(x)) = \lim_{x\rightarrow +\infty}[-x^3 + g(x)] = -\infty.
\end{align*}
显然矛盾!

(ii)若\(\lim_{x\rightarrow +\infty}f(x) = -\infty\), 则
\begin{align}
f(-\infty) = \lim_{x\rightarrow +\infty}f(f(x)) = \lim_{x\rightarrow +\infty}[-x^3 + g(x)] = -\infty. \label{equation0.5example-0.1}
\end{align}
从而对上式两边同时作用\(f\)可得
\begin{align}
f(-\infty) = f(f(-\infty)) = \lim_{x\rightarrow -\infty}[-x^3 + g(x)] = +\infty. \label{equation0.5example-0.2}
\end{align}
于是\eqref{equation0.5example-0.1}式与\eqref{equation0.5example-0.2}式显然矛盾!

综上,\(f\in C(\mathbb{R})\)的解不存在. 
\end{proof}























\end{document}
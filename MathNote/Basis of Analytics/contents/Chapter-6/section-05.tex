\documentclass[../../main.tex]{subfiles}
\graphicspath{{\subfix{../../image/}}} % 指定图片目录,后续可以直接使用图片文件名。

% 例如:
% \begin{figure}[H]
% \centering
% \includegraphics[scale=0.4]{图.png}
% \caption{}
% \label{figure:图}
% \end{figure}
% 注意:上述\label{}一定要放在\caption{}之后,否则引用图片序号会只会显示??.

\begin{document}

\section{函数性态分析综合}

\begin{proposition}
设 $f : (a, b) \to (a, b)$ 满足对任意的 $x, y \in (a, b)$, 当 $x \neq y$ 时, 有 $\left| f(x) - f(y) \right| < \left| x - y \right|$. 任取 $x_1 \in (a, b)$, 令 $x_{n+1} = f(x_n), n = 1, 2, \cdots$, 证明: 数列 $\{x_n\}_{n=1}^{\infty}$ 收敛.
\end{proposition}
\begin{proof}
注意到 $x_1 \in (a, b)$,假设 $x_k \in (a, b)$,则 $x_{k+1} = f(x_k) \in (a, b)$,故由数学归纳法可知 $x_n \in (a, b), \forall n \in \mathbb{N}$。
又由条件可知,对 $\forall \varepsilon > 0$,令 $\delta = \varepsilon > 0$,当 $x, y \in (a, b)$ 且 $0 < |x - y| < \delta$ 时,有
\begin{align*}
|f(x) - f(y)| < |x - y| < \delta = \varepsilon。
\end{align*}
故 $f$ 在 $(a, b)$ 上一致连续。从而 $f \in C(a, b)$,令 $F(x) = f(x) - x$,则 $F \in C(a, b)$。下面我们对 $F$ 进行分类讨论。

\begin{enumerate}[(1)]
\item 若 $F$ 在 $(a, b)$ 上不变号,则由 $F \in C(a, b)$ 可知,$F$ 要么恒大于零,要么恒小于零。不妨设 $F$ 在 $(a, b)$ 上恒大于零,即 $f(x) > x, \forall x \in (a, b)$。从而
\begin{align*}
x_{n+1} = f(x_n) > x_n, \forall n \in \mathbb{N}。
\end{align*}
即 $\{x_n\}$ 单调递增。又因为 $x_n \in (a, b), \forall n \in \mathbb{N}$,所以由单调有界定理可知 $\{x_n\}$ 收敛。

\item 若 $F$ 在 $(a, b)$ 上变号,则由 $F \in C(a, b)$ 及介值定理可得,存在 $\xi \in (a, b)$,使得 $f(\xi) = \xi$。
若存在 $\xi' \in (a, b)$ 且 $\xi' \neq \xi$,使得 $f(\xi') = \xi'$,则由条件可得到
\begin{align*}
|\xi - \xi'| = |f(\xi) - f(\xi')| < |\xi - \xi'|。
\end{align*}
显然矛盾!因此存在唯一的 $\xi \in (a, b)$,使得 $f(\xi) = \xi$。从而
\begin{align*}
|x_{n+1} - \xi| = |f(x_n) - f(\xi)| < |x_n - \xi|, \forall n \in \mathbb{N}。
\end{align*}
于是 $\{|x_n - \xi|\}$ 单调递减且有下界 $0$,故由单调有界定理可知 $\lim_{n \to \infty} |x_n - \xi| = A \geqslant 0$ 存在。
\begin{enumerate}[(i)]
\item 当 $A = 0$ 时,则由 $\lim_{n \to \infty} |x_n - \xi| = A = 0$ 可得 $\lim_{n \to \infty} x_n = \xi$。
\item 当 $A > 0$ 时,若 $\{x_n\}$ 收敛,则结论已经成立。若 $\{x_n\}$ 发散,则由 $x_n \in (a, b), \forall n \in \mathbb{N}$ 及聚点定理可知,$\{x_n\}$ 至少有一个聚点。若 $\{x_n\}$ 只有一个聚点,则 $\{x_n\}$ 收敛与假设矛盾!因此 $\{x_n\}$ 至少有两个聚点。任取收敛子列 $\{x_{n_k}\} \subset \{x_n\}$,设 $\lim_{k \to \infty} x_{n_k} = B$,则
\begin{align*}
A=\lim_{n\rightarrow \infty} |x_n-\xi |=\lim_{k\rightarrow \infty} |x_{n_k}-\xi |=|B-\xi |.
\end{align*}
从而 $B = \xi - A$ 或 $\xi + A$。因此 $\{x_n\}$ 最多有两个聚点 $\xi - A,\,\xi + A\in [a,b]$。又因为 $\{x_n\}$ 至少有两个聚点,所以 $\{x_n\}$ 有且仅有两个聚点 $\xi - A$ 和 $\xi + A$。进而一定存在收敛子列 $\{x_{n_k}\}$,使得$\lim_{k \to \infty} x_{n_k} = \xi - A.$因为$\xi-A\ne \xi$不是$f$的不动点,而$\{x_n\}$只有两个聚点,所以
\begin{align*}
\lim_{k\rightarrow \infty} x_{n_k+1}=\lim_{k\rightarrow \infty} f\left( x_{n_k} \right) =f\left( \xi -A \right) \ne \xi -A\Longrightarrow \lim_{k\rightarrow \infty} x_{n_k+1}=f(\xi -A)=\xi +A.
\end{align*}
由
\begin{align*}
\lim_{k\rightarrow \infty}x_{n_k}=\xi -A<\xi ,\quad \lim_{k\rightarrow \infty}x_{n_k+1}=\xi +A>\xi 
\end{align*}
知,存在$K\in \mathbb{N}$,使得$\forall k>K$,有$x_{n_k}<\xi$,$x_{n_k+1}>\xi$。又$\{| x_n-\xi |\}$递减趋于$A$,故对$\forall k>K$有
\begin{gather*}
A\leqslant | x_{n_k}-\xi |=\xi -x_{n_k}<\xi -a\Longrightarrow \xi -A>a,
\\
A\leqslant | x_{n_k+1}-\xi |=x_{n_k+1}-\xi <b-\xi \Longrightarrow \xi +A<b.
\end{gather*}
因此$\xi -A,\xi +A\in (a,b)$,$\xi =f(\xi)$。
再由条件可得
\begin{align*}
A = |\xi - (\xi + A)| = |f(\xi) - f(\xi - A)| < |\xi - (\xi - A)| = A。
\end{align*}
显然矛盾!故 $A > 0$ 不成立,于是 $A = 0$。再由 (1) 可得 $\lim_{n \to \infty} x_n = \xi$,即 $\{x_n\}$ 收敛,与假设$\{x_n\}$发散矛盾!
\end{enumerate}
\end{enumerate}

\end{proof}

\begin{proposition}\label{proposition:经典导数一致连续问题}
设 $f'$ 在 $[0,+\infty)$ 一致连续且 $\lim_{x \to +\infty} f(x)$ 存在,证明 $\lim_{x \to +\infty} f'(x) = 0$。
\end{proposition}
\begin{note}
本题也有积分版本:设 $f$ 在 $[0,+\infty)$ 一致连续且 $\int_{0}^{\infty} f(x) \, \mathrm{d}x$ 收敛,则 $\lim_{x \to +\infty} f(x) = 0$.(令$F=\int_0^x{f\left( x \right) \mathrm{d}x}$就可以将这个积分版本转化为上述命题)
\end{note}
\begin{proof}
反证,若 $\lim_{x\rightarrow +\infty} f'(x) \ne 0$,则可以不妨设存在 $\delta >0$,$\{ x_n \}_{n=1}^{\infty}$,使得
\begin{align*}
x_n\rightarrow +\infty \text{且} f'(x_n) \geqslant \delta,\forall n\in \mathbb{N}.    
\end{align*}
由 $f'$ 在 $[0,+\infty)$ 上一致连续可知,存在 $\eta >0$,使得对 $\forall n\in \mathbb{N}$,都有
\begin{align*}
f'(x) \geqslant f'(x_n) -\frac{\delta}{2} \geqslant \frac{\delta}{2}, \forall x\in [x_n-\eta ,x_n+\eta] 。
\end{align*}
从而对 $\forall n\in \mathbb{N}$,都有
\begin{align*}
f(x_n+\eta) -f(x_n) =\int_{x_n}^{x_n+\eta} f'(x) \mathrm{d}x \geqslant \int_{x_n}^{x_n+\eta} \frac{\delta}{2} \mathrm{d}x = \frac{\delta \eta}{2} > 0, \forall x\in [x_n-\eta ,x_n+\eta] 。
\end{align*}
令$n\rightarrow \infty$,由 $\lim_{x\rightarrow +\infty} f(x) $存在可得 $0 \geqslant \frac{\delta \eta}{2} > 0$,矛盾!故 $\lim_{x\rightarrow +\infty} f'(x) =0$。

\end{proof}

\begin{example}[$\,\,$时滞方程]\label{example:时滞方程}
设 $f$ 在 $\mathbb{R}$ 上可微且满足
\begin{align*}
\lim_{x \to +\infty} f'(x) = 1, \quad
f(x+1) - f(x) = f'(x), \forall x \in \mathbb{R}.
\end{align*}
证明存在常数 $C \in \mathbb{R}$ 使得 $f(x) = x + C, \forall x \in \mathbb{R}$.
\end{example}
\begin{proof}
由 \( f(x+1)-f(x)=f'(x),\forall x\in \mathbb{R} \) 及 \( f\in D(\mathbb{R}) \) 可知 \( f' \in C(\mathbb{R}) \)。
对 \(\forall x_1\in \mathbb{R} \),固定 \( x_1 \),记
\[ A=\{ z>x_1 \mid f'(z) = f'(x_1) \} .\]
由 Lagrange 中值定理及 \( f(x+1)-f(x)=f'(x),\forall x\in \mathbb{R} \) 可知
\[ \exists x_2\in (x_1,x_1+1) \,\,\mathrm{s}.\mathrm{t}.\,\, f'(x_1) = f(x_1+1) - f(x_1) = f'(x_2) .\]
故 \( x_2 \in A \),从而 \( A \) 非空。现在考虑 \( y \triangleq \mathrm{sup}A \in (x_1,+\infty) \),下证 \( y=+\infty \)。
若 \( y<+\infty \),则存在 \(\{ z_{n}' \}_{n=1}^{\infty} \),使得
\[ z_{n}' \rightarrow y \text{且} f'(z_{n}') = f'(x_1) .\]
两边同时令 \( n \rightarrow \infty \),由 \( f' \in C(\mathbb{R}) \) 可得
\[ f'(x_1) = \lim_{n \rightarrow \infty} f'(z_{n}') = f'(y) .\]
又由 Lagrange 中值定理及 \( f(x+1)-f(x)=f'(x),\forall x\in \mathbb{R} \) 可得
\[ \exists y' \in (y,y+1) \,\,\mathrm{s}.\mathrm{t}.\,\, f'(y) = f(y+1) - f(y) = f'(y') .\]
从而 \( y' \in A \) 且 \( y' > y \),这与 \( y=\mathrm{sup}A \) 矛盾!故 \( y=+\infty \)。于是存在 \(\{ z_n \}_{n=1}^{\infty} \),使得
\[ z_n \rightarrow +\infty \text{且} f'(z_n) = f'(x_1) .\]
两边同时令 \( n \rightarrow \infty \),由 \( f' \in C(\mathbb{R}) \) 及 \(\lim_{x \rightarrow +\infty} f'(x) = 1 \) 可得
\[ f'(x_1) = \lim_{n \rightarrow \infty} f'(z_n) = \lim_{x \rightarrow +\infty} f'(x) = 1 .\]
因此由 \( x_1 \) 的任意性得,存在 \( C \) 为常数,使得 \( f(x) = x+C,\forall x\in \mathbb{R} \).

\end{proof}

\begin{example}
设 $f\in C^2(\mathbb{R})$ 满足 $f(1)\leqslant0$ 以及
\begin{align*}
\lim_{x\to\infty}[f(x)-|x|]=0.
\tag{12.27}
\end{align*}
证明:
(1):存在 $\xi\in(1,+\infty)$, 使得 $f'(\xi)>1$.

(2):存在 $\eta\in\mathbb{R}$, 使得 $f''(\eta)=0$.
\end{example}
\begin{proof}
(1)如果对任何 $x\in(1,+\infty)$, 都有 $f'(x)\leqslant1$, 那么 $[f(x)-x]'\leqslant0$ 知 $f(x)-x$ 在 $[1,+\infty)$ 单调递减. 从而
\begin{align*}
-1\geqslant f(1)-1\geqslant\lim_{x\to+\infty}[f(x)-x]=\lim_{x\to\infty}[f(x)-|x|]=0,
\end{align*}
这就是一个矛盾! 于是我们证明了 (1).

(2)
若对任何 $x\in\mathbb{R}$, 我们有 $f''(x)\neq0$.从而$f''(x)$要么恒大于零,要么恒小于零,否则由零点存在定理可得矛盾!任取$\xi \in \mathbb{R}$.

当 $f''(x)>0,\forall x\in\mathbb{R}$, 我们知道 $f$ 在 $\mathbb{R}$ 上是下凸函数. 由 (1) 和下凸函数切线总是在函数下方, 我们知道
\begin{align*}
f(x)\geqslant f(\xi)+f'(\xi)(x-\xi),\forall x>\xi.
\end{align*}
于是
\begin{align*}
0=\lim_{x\to+\infty}[f(x)-x]\geqslant\lim_{x\to+\infty}[f(\xi)-f'(\xi)\xi+(f'(\xi)-1)x]=+\infty,
\end{align*}
这就是一个矛盾!

当 $f''(x)<0,\forall x\in\mathbb{R}$, 我们知道 $f$ 在 $\mathbb{R}$ 上是上凸函数. 由 (1) 和上凸函数切线总是在函数上方, 我们有
\begin{align*}
f(x)\leqslant f(\xi)+f'(\xi)(x-\xi),\forall x<\xi.
\end{align*}
于是
\begin{align*}
0=\lim_{x\to-\infty}[f(x)+x]\leqslant\lim_{x\to-\infty}[f(\xi)-f'(\xi)\xi+(f'(\xi)+1)x]=-\infty,
\end{align*}
这就是一个矛盾! 因此我们证明了 (2). 

\end{proof}

\begin{example}
设 \(f\) 在 \([a,b]\) 上每一个点极限都存在,证明:\(f\) 在 \([a,b]\) 有界。
\end{example}
\begin{note}
极限存在必然局部有界,本题就是说局部有界可以推出在紧集上有界。在大量问题中会有一个公共现象:即\textbf{局部的性质等价于在所有紧集上的性质}。证明的想法就是有限覆盖。 
\end{note}
\begin{proof}
对 \(\forall c\in [a,b]\),由 \(\lim_{x\rightarrow c}f(x)\) 存在可知,存在 \(c\) 的邻域 \(U_c\) 和 \(M>0\),使得
\begin{align*}
\sup_{x\in U_c\cap [a,b]}|f(x)|\leqslant M_c.
\end{align*}
注意 \([a,b]\subset \bigcup_{c\in [a,b]}U_c\),由有限覆盖定理得,存在 \(c_1,c_2,\cdots,c_n\in [a,b]\),使得
\begin{align*}
[a,b]\subset \bigcup_{k = 1}^nU_{c_k}. 
\end{align*}
故 \(\sup_{x\in [a,b]}|f(x)|\leqslant \max_{1\leqslant k\leqslant n}M_k\)。 

\end{proof}

\begin{example}
设 \(f\) 是 \((a, +\infty)\) 有界连续函数, 证明对任何实数 \(T\) , 存在数列 \(\lim_{n \to \infty} x_n = +\infty\) 使得
\begin{align*}
\lim_{n \to \infty} [f(x_n + T) - f(x_n)] = 0.
\end{align*}
\end{example}
\begin{remark}
因为\(\vert f(x + T) - f(x)\vert \geqslant 0\) , 所以
\begin{align*}
0\leqslant \varliminf_{x \to +\infty} \vert f(x + T) - f(x)\vert \leqslant \varlimsup_{x \to +\infty} \vert f(x + T) - f(x)\vert.
\end{align*}
原结论的反面只用考虑\(\varliminf_{x \to +\infty} \vert f(x + T) - f(x)\vert\) 即可. 若\(\varliminf_{x \to +\infty} \vert f(x + T) - f(x)\vert = 0\) , 则一定存在子列 \(x_n \to +\infty\) , 使得结论成立.因此原结论等价于\(\varliminf_{x \to +\infty} \vert f(x + T) - f(x)\vert = 0\) .
故原结论的反面就是\(\varliminf_{x \to +\infty} \vert f(x + T) - f(x)\vert > 0\) .
\end{remark}
\begin{note}
考虑反证法之后,再进行定性分析(画$f(x)$的大致走势图),就容易找到矛盾.
\end{note}
\begin{proof}
当$T=0$时,显然存在这样的数列.不妨设$T>0,$
假设\(\varliminf_{x \to +\infty} \vert f(x + T) - f(x)\vert > 0\) , 则存在\(\varepsilon_0 > 0\) , \(X > 0\) , 使得
\begin{align}
\vert f(x + T) - f(x)\vert \geqslant \varepsilon_0>0,\quad \forall x \geqslant X \label{example0.4section05----1.1}
\end{align}
令\(g(x) \triangleq f(x + T) - f(x)\) , 则若存在 \(x_1, x_2 \geqslant X\) , 使得
\[g(x_1) = f(x_1 + T) - f(x_1) \geqslant \varepsilon_0 > 0 ,\quad g(x_2) = f(x_2 + T) - f(x_2) \leqslant -\varepsilon_0 < 0. \]
不妨设 \(x_1 < x_2\) , 由 \(g\) 连续及介值定理可知, 存在 \(x_3 \in (x_1, x_2)\) , 使得
\begin{align*}
g(x_3) = f(x_3 + T) - f(x_3) = 0
\end{align*}
与\eqref{example0.4section05----1.1} 式矛盾! 故 \(g(x) \triangleq f(x + T) - f(x)\) 在\([X, +\infty)\) 上要么恒大于\(\varepsilon_0\) , 要么恒小于\(\varepsilon_0\) . 于是不妨设
\begin{align}
f(x + T) - f(x) \geqslant \varepsilon_0,\quad \forall x \geqslant X .\label{example0.4section05----1.2}
\end{align}
从而对\(\forall k \in \mathbb{N}\) , 存在 \(X_k \geqslant X\) , 使得当 \(x \geqslant X_1\) 时, 有
\(x + (k - 1)T > X\) .
于是由\eqref{example0.4section05----1.2}式可得
\begin{align}
f(x + kT) - f(x + (k - 1)T) \geqslant \varepsilon_0,\quad \forall x \geqslant X_k. \label{example0.4section05----1.3}
\end{align}
因此对\(\forall n \in \mathbb{N}\) , 取 \(K_n = \max\{X_1, X_2, \cdots, X_k\}\) , 则由\eqref{example0.4section05----1.3}式可知
\[
f(x + kT) - f(x + (k - 1)T)\geqslant \varepsilon_0 , \forall x \geqslant K_n ,\forall k \in \{1, 2, \cdots, n\}.
\]
进而对上式求和可得, 对\(\forall n \in \mathbb{N}\) , 都有
\begin{align*}
\sum_{k = 1}^n [f(x + kT) - f(x + (k - 1)T)] = f(x + nT) - f(x) \geqslant n\varepsilon_0,\quad \forall x \geqslant K_n
\end{align*}
任取 \(x_0 \geqslant K_n\) , 则
\(f(x_0 + nT) - f(x_0) \geqslant n\varepsilon_0,\quad \forall n \in \mathbb{N}\) .
令 \(n \to \infty\) , 得\(\lim_{x \to +\infty} f(x) = +\infty\) . 这与 \(f\) 在\((a, +\infty)\) 上有界矛盾! 

\end{proof}

\begin{proposition}
\begin{enumerate}
\item 设 \(f_n\in C[a,b]\) 且关于 \([a,b]\) 一致的有
\begin{align*}
\lim_{n \to \infty}f_n(x)=f(x).
\end{align*}
证明: 对 \(\{x_n\}\subset [a,b]\), \(\lim_{n \to \infty}x_n = c\), 有
\begin{align*}
\lim_{n \to \infty}f_n(x_n)=f(c).
\end{align*}

\item 设 \(f_n(x):\mathbb{R}\to\mathbb{R}\) 满足对任何 \(x_0\in\mathbb{R}\) 和 \(\{x_n\}_{n = 1}^{\infty}\subset\mathbb{R}\), \(\lim_{n \to \infty}x_n = x_0\), 都有
\begin{align*}
\lim_{n \to \infty}f_n(x_n)=f(x_0),
\end{align*}
证明: \(f\in C(\mathbb{R})\). 
\end{enumerate}
\end{proposition}
\begin{proof}
\begin{enumerate}
\item 由\(f_n\)一致收敛到\(f(x)\)可知, 对\(\forall \varepsilon > 0\), 存在\(N_0\in \mathbb{N}\), 使得对\(\forall N\geqslant N_0\), 当\(n\geqslant N\)时, 对\(\forall x\in [a,b]\), 都有
\begin{align*}
|f_n(x) - f_N(x)| < \varepsilon.
\end{align*}
从而由上式可得
\begin{align*}
|f_n(x_n) - f(c)| &\leqslant |f_n(x_n) - f_N(x_n)| + |f_N(x_n) - f_N(c)| + |f_N(c) - f(c)|\\
&\leqslant \varepsilon + |f_N(x_n) - f_N(c)| + |f_N(c) - f(c)|.
\end{align*}
令\(n\rightarrow +\infty\), 由\(f\)的连续性及\(\lim_{n\rightarrow \infty}x_n = c\)可得
\begin{align*}
\varlimsup_{n\rightarrow \infty}|f_n(x_n) - f(c)| &\leqslant \varepsilon + |f_N(c) - f(c)|.
\end{align*}
再令\(N\rightarrow +\infty\), 由\(\lim_{n\rightarrow \infty}f_n(x) = f(x)\), \(\forall x\in [a,b]\)可知
\begin{align*}
\varlimsup_{n\rightarrow \infty}|f_n(x_n) - f(c)| &\leqslant \varepsilon.
\end{align*}
令\(\varepsilon \rightarrow 0^+\), 得\(\varlimsup_{n\rightarrow \infty}|f_n(x_n) - f(c)| \leqslant 0\). 故\(\lim_{n\rightarrow \infty}f_n(x_n) = f(c)\). 

\item 反证, 若\(f\)在\(x_0\in \mathbb{R}\)处不连续, 则存在\(\varepsilon_0 > 0\), 使得\(\forall m\in \mathbb{N}\), 存在\(y_m\in (x_0 - \frac{1}{m}, x_0 + \frac{1}{m})\), 使得
\begin{align}
|f(y_m) - f(x_0)| \geqslant \varepsilon_0. \label{equation--0.4section051.1}
\end{align}
对\(\forall m\in \mathbb{N}\), 令条件中的\(x_0 = y_m\), \(x_n\equiv y_m\), \(\forall n\in \mathbb{N}\), 从而由条件可知\(\lim_{n\rightarrow \infty}|f_n(y_m) - f(y_m)| = 0\), \(m = 1,2,\cdots\),
故对\(\forall m\in \mathbb{N}\), 存在严格递增的数列\(n_m\rightarrow +\infty\), 使得
\begin{align}
|f_{n_m}(y_m) - f(y_m)| < \frac{\varepsilon_0}{2}. \label{equation--0.4section051.2}
\end{align}
从而由\eqref{equation--0.4section051.1}\eqref{equation--0.4section051.2}式可知, 对\(\forall m\in \mathbb{N}\), 都有
\begin{gather}
|f(y_{n_m}) - f(x_0)| \geqslant \varepsilon_0, \label{equation--0.4section051.3}\\
|f_{n_m}(y_{n_m}) - f(y_{n_m})| < \frac{\varepsilon_0}{2}. \label{equation--0.4section051.4}
\end{gather}
因此由\eqref{equation--0.4section051.3}\eqref{equation--0.4section051.4}式可得, 对\(\forall m\in \mathbb{N}\), 都有
\begin{align}
|f_{n_m}(y_{n_m}) - f(x_0)| &\geqslant |f(y_{n_m}) - f(x_0)| - |f_{n_m}(y_{n_m}) - f(y_{n_m})| \geqslant \varepsilon_0 - \frac{\varepsilon_0}{2} = \frac{\varepsilon_0}{2}. \label{equation--0.4section051.5}
\end{align}
注意到\(y_m\rightarrow x_0\), 于是\(y_{n_m}\rightarrow x_0\). 从而由已知条件可知\(\lim_{m\rightarrow \infty}f_{n_m}(y_{n_m}) = f(x_0)\). 这与\eqref{equation--0.4section051.5}式矛盾! 故\(f\in C(\mathbb{R})\). 
\end{enumerate}

\end{proof}

\begin{example}
设 \(g\in C(\mathbb{R})\) 且以 \(T > 0\) 为周期, 且有
\begin{align}
f(f(x))=-x^3 + g(x).\label{equation0.5example-1.1}
\end{align}
证明:不存在 \(f\in C(\mathbb{R})\), 使得\eqref{equation0.5example-1.1}式成立. 
\end{example}
\begin{proof}
由\hyperref[proposition:连续的周期函数的基本性质]{连续的周期函数的基本性质}可知,存在$M>0,$使得$|g(x)|\leqslant M.$反证,假设存在 \(f\in C(\mathbb{R})\), 使得\eqref{equation0.5example-1.1}式成立. 则
\begin{align}
&\underset{x\rightarrow +\infty}{\lim}f\left( f\left( x \right) \right) =\underset{x\rightarrow +\infty}{\lim}\left( -x^3+g\left( x \right) \right) =-\infty ,\label{equation0.5example-2.1}
\\
&\underset{x\rightarrow -\infty}{\lim}f\left( f\left( x \right) \right) =\underset{x\rightarrow -\infty}{\lim}\left( -x^3+g\left( x \right) \right) =+\infty .\label{equation0.5example-2.2}
\end{align}
假设\(\lim_{x\rightarrow +\infty}f(x) = A\in \mathbb{R}\), 则存在\(x_n\rightarrow +\infty\), 使得\(f(x_n)\rightarrow A\). 从而由\eqref{equation0.5example-1.1}式可得
\begin{align*}
f(A) = \lim_{n\rightarrow \infty}f(f(x_n)) = \lim_{n\rightarrow \infty}(-x_{n}^{3}+g(x_n)) = -\infty.
\end{align*}
上式显然矛盾! 又因为\(f\in C(\mathbb{R})\), 所以\(\lim_{x\rightarrow +\infty}f(x) = +\infty\)或\(-\infty\).
否则, 当\(x\rightarrow +\infty\)时,\(f(x)\)振荡, 则由零点存在定理可知, 存在\(y_n\rightarrow +\infty\), 使得\(f(y_n) = 0\), \(n = 1,2,\cdots\).
从而由\eqref{equation0.5example-2.1}式可知
\begin{align*}
-\infty = \lim_{x\rightarrow +\infty}f(f(x)) = \lim_{n\rightarrow \infty}f(f(y_n)) = f(0).
\end{align*}
显然矛盾!

(i)若\(\lim_{x\rightarrow +\infty}f(x) = +\infty\), 则
\begin{align*}
+\infty = \lim_{x\rightarrow +\infty}f(x) = f(+\infty) = \lim_{x\rightarrow +\infty}f(f(x)) = \lim_{x\rightarrow +\infty}[-x^3 + g(x)] = -\infty.
\end{align*}
显然矛盾!

(ii)若\(\lim_{x\rightarrow +\infty}f(x) = -\infty\), 则
\begin{align}
f(-\infty) = \lim_{x\rightarrow +\infty}f(f(x)) = \lim_{x\rightarrow +\infty}[-x^3 + g(x)] = -\infty. \label{equation0.5example-0.1}
\end{align}
从而对上式两边同时作用\(f\)可得
\begin{align}
f(-\infty) = f(f(-\infty)) = \lim_{x\rightarrow -\infty}[-x^3 + g(x)] = +\infty. \label{equation0.5example-0.2}
\end{align}
于是\eqref{equation0.5example-0.1}式与\eqref{equation0.5example-0.2}式显然矛盾!
综上,\(f\in C(\mathbb{R})\)的解不存在. 

\end{proof}

\begin{example}
\begin{enumerate}
\item 设 \(f\in C[0,+\infty)\) 是有界的。若对任何 \(r\in\mathbb{R}\),都有 \(f(x)=r\) 在 \([0,+\infty)\) 只有有限个或者无根,证明:
$\lim_{x \to +\infty} f(x) \text{ 存在}.$

\item 设 \(f\in C(\mathbb{R})\),\(n\) 是一个非 \(0\) 正偶数,使得对任何 \(y\in\mathbb{R}\),都有 \(\{x\in\mathbb{R}:f(x)=y\}\) 是 \(n\) 元集。证明: 这样的 \(f\) 不存在。 
\end{enumerate}
\end{example}
\begin{proof}
\begin{enumerate}
\item 反证,设\(\lim\limits_{x \to +\infty} f(x)\)不存在,由\(f\)有界,可设\(\varlimsup_{x \to +\infty} f(x)=A > B=\varliminf_{x \to +\infty} f(x)\)。
任取\(C\in (B,A)\),则由\(\varlimsup_{x \to +\infty} f(x)=A > C\)可知,存在\(x_1\geqslant 0\),使得\(f(x_1)>C\)。
又由\(\varliminf_{x \to +\infty} f(x)=B < C\)可知,存在\(x_2 > x_1 + 1\),使得\(f(x_2)<C\)。

于是再由\(\varlimsup_{x \to +\infty} f(x)=A > C\)可知,存在\(x_3 > x_2 + 1\),使得\(f(x_3)>C\)。
又由\(\varliminf_{x \to +\infty} f(x)=B < C\)可知,存在\(x_4 > x_3 + 1\),使得\(f(x_4)<C\)。
依此类推,可得递增数列\(\{x_n\}\),使得
\begin{align*}
x_{n + 1} > x_n + 1, \quad f(x_{2n - 1}) > C, \quad f(x_{2n}) < C, \quad n = 1,2,\cdots.
\end{align*}
从而由\(f\in C[0,+\infty)\)及介值定理可得,对\(\forall n\in\mathbb{N}\),存在\(y_n\in (x_{2n - 1},x_{2n})\),使得\(f(y_n)=C\),矛盾! 

\item 设\(x_1<x_2<\cdots <x_n\)是\(f\)的所有零点,记\(x_0 = -\infty\),\(x_{n + 1} = +\infty\),
则由\(f\)的连续性及介值定理可知,\(f\)在\(( x_{i - 1},x_i )\)上不变号。
这里共有\(n + 1\)个区间,现在考虑\(( x_{i - 1},x_i )\),\(i = 2,3,\cdots,n\),这\(n - 1\)个区间。于是由抽屉原理可知,这\(n - 1\)个区间中必存在\(\frac{n}{2}\)区间,使\(f\)在这\(\frac{n}{2}\)个区间内都同号。

不妨设\(f\)在这\(\frac{n}{2}\)个区间内恒大于\(0\),记\(f\)在\([ x_{i - 1},x_i ]\)上的最大值记为\(f( m_i ) \triangleq M_i>0\),其中\(m_i\in ( x_{i - 1},x_i )\),\(i = 2,3,\cdots,n\)。由介值定理知,至少存在\(c_i\in ( x_{i - 1},m_i )\),\(c_{i}^{\prime}\in ( m_i,x_i )\),使得
\begin{align*}
f( c_i ) =f( c_{i}^{\prime} ) =\frac{1}{2}\underset{2\leqslant  i\leqslant  n}{\min}M_i>0,i = 2,3,\cdots,n.
\end{align*}
注意到在\(( x_0,x_1 )\),\(( x_n,x_{n + 1} )\)上\(f\)必不同号。否则,不妨设在\(( x_0,x_1 )\),\(( x_n,x_{n + 1} )\)上\(f\)恒大于\(0\),则由\(f\in C( \mathbb{R} )\)可知,存在\(M>0\),使得
\(\left| f( x ) \right|<M,\forall x\in [ x_1,x_{n + 1} ].\)
从而\(f( x ) \geqslant -M,\forall x\in \mathbb{R} \)。这与对\(\forall y\in \mathbb{R} \),\(f( x ) =y\)都有根矛盾!

不妨设\(f\)在\(( x_0,x_1 )\)上恒小于\(0\),在\(( x_n,x_{n + 1} )\)上恒大于\(0\),则\(f\)在\(( x_n,x_{n + 1} )\)上无上界。否则,存在\(K>\underset{2\leqslant  i\leqslant  n}{\max}M_i>0\),使得\(f( x ) <K,\forall x\in ( x_n,x_{n + 1} )\)。又因为
\(f( x ) <0<K,\forall x\in ( x_0,x_1 ),\quad f( x ) \leqslant \underset{2\leqslant  i\leqslant  n}{\max}M_i<K,\forall x\in ( x_1,x_n ).\)
所以\(f( x ) <K,\forall x\in \mathbb{R} \)。这与对\(\forall y\in \mathbb{R} \),\(f( x ) =y\)都有根矛盾!

又\(f( x_n ) =0\),故至少存在一个\(c\in ( x_n,x_{n + 1} )\),使得\(f( c ) =\frac{1}{2}\underset{2\leqslant  i\leqslant  n}{\min}M_i>0\)。
综上,至少有\(n + 1\)个点使得\(f( x ) =\frac{1}{2}\underset{2\leqslant  i\leqslant  n}{\min}M_i>0\)。这与\(\{ x\in \mathbb{R} :f( x ) =\frac{1}{2}\underset{2\leqslant  i\leqslant  n}{\min}M_i \}\)是\(n\)元集矛盾!
\end{enumerate}

\end{proof}

\begin{example}
设\(f \in C^{2}[0, +\infty)\),\(g \in C^{1}[0, +\infty)\)且存在\(\lambda > 0\)使得\(g(x) \geqslant \lambda\),\(\forall x \geqslant 0\)。若\(g'\)至多只有有限个零点且
\begin{align*}
f''(x)+g(x)f(x) = 0,\quad\forall x \geqslant 0,
\end{align*}
证明:\(f\)在\([0, +\infty)\)有界。
\end{example}
\begin{note}
形式计算分析需要的构造函数:由条件微分方程可得
\begin{align*}
y'y'' =-gyy' \Rightarrow \frac{( y' )^2}{2}=-\int{gyy'\mathrm{d}x}=-\frac{1}{2}\int{g\mathrm{d}y^2}=-\frac{1}{2}gy^2+\frac{1}{2}\int{y^2\mathrm{d}g}
\\
\Rightarrow ( y' )^2=-gy^2+\int{y^2\mathrm{d}g}\Rightarrow \frac{( y' )^2}{g}+y^2=\int{y^2\mathrm{d}g}.
\end{align*}
于是考虑构造函数\(F_1( x ) \triangleq \frac{| f'( x ) |^2}{g( x )}+f^2( x )\),\(F_2( x ) \triangleq | f'( x ) |^2+g( x ) f^2( x )\)。
\end{note}
\begin{proof}
因为\(g'\)至多只有有限个零点,所以存在\(X>0\),使得\(g'( x ) \ne 0\),\(\forall x\geqslant X\)。从而由导数介值性可知,\(g'\)在\([ X,+\infty )\)上要么恒大于\(0\),要么恒小于\(0\)。
令\(F_1( x ) \triangleq \frac{| f'( x ) |^2}{g( x )}+f^2( x )\),\(x\geqslant X\),则结合条件\(f'' =-gf\)可得
\begin{align}
F_{1}'( x ) =\frac{2f'f'' g-g'( f' )^2+2ff' g}{g^2}=\frac{-2f'fg^2-g'( f' )^2+2ff'g^2}{g^2}=-\frac{g'( f' )^2}{g^2}. \label{6.7}
\end{align}

(i) 若\(g'( x ) >0\),\(\forall x\geqslant X\),则由\eqref{6.7}式可知\(F'( x ) \leqslant 0\),即\(F( x )\)在\([ X,+\infty )\)上递减。
于是再结合\(g( x ) >\lambda >0\),\(\forall x>0\)可知,存在\(C>0\),使得
\[
f^2( x ) \leqslant F_1( x ) \leqslant C,\quad\forall x\geqslant X.
\]
故\(f( x )\)在\([ X,+\infty )\)上有界。又\(f\in C[ 0,+\infty )\),故\(f\)在\([ 0,X ]\)上必有界。因此\(f\)在\([ 0,+\infty )\)上有界。

(ii) 若\(g'( x ) <0\),\(\forall x\geqslant X\),令\(F_2( x ) \triangleq | f'( x ) |^2+g( x ) f^2( x )\),则结合条件\(f'' =-gf\)可得
\begin{align}
F_{2}'( x ) =2f'f'' +g'f^2+2gff' =-2f'fg+g'f^2+2gff' =g'f^2\leqslant 0. \label{6.8}
\end{align}
从而\(F_2( x )\)在\([ X,+\infty )\)上递减,于是存在\(C'>0\),使得
\[
g( x ) f^2( x ) \leqslant F_2( x ) \leqslant C,\quad\forall x\geqslant X.
\]
进而由\(g( x ) >\lambda >0\),\(\forall x>0\)可得
\[
f^2( x ) \leqslant \frac{C}{g( x )}\leqslant \frac{C}{\lambda},\quad\forall x\geqslant X.
\]
故\(f( x )\)在\([ X,+\infty )\)上有界。又\(f\in C[ 0,+\infty )\),故\(f\)在\([ 0,X ]\)上必有界。因此\(f\)在\([ 0,+\infty )\)上有界。

\end{proof}

\begin{example}
设 \(a,b > 1\) 且 \(f: \mathbb{R} \to \mathbb{R}\) 在 \(x = 0\) 邻域有界。若
\begin{align*}
f(ax) = bf(x),\quad \forall x \in \mathbb{R}
\end{align*}
证明:\(f\) 在 \(x = 0\) 连续。 
\end{example}
\begin{proof}
注意到
\begin{align*}
f(0) = bf(0) \Rightarrow f(0) = 0.
\end{align*}
由条件可得
\begin{align}
f(ax) = bf(x) \Rightarrow f(x) = \frac{f(ax)}{b} = \frac{f(a^2x)}{b^2} = \cdots = \frac{f(a^nx)}{b^n}, \forall n \in \mathbb{N}. \label{eq:100.1}
\end{align}
因为\(f\)在\(x=0\)邻域有界,所以存在\(\delta > 0\),使得
\begin{align}
|f(x)| \leqslant M, \forall x \in (-\delta, \delta). \label{eq:100.2}
\end{align}
从而对\(\forall \varepsilon > 0\),取\(N \in \mathbb{N}\),使得
\begin{align}
\frac{M}{b^N} < \varepsilon. \label{eq:100.3}
\end{align}
于是当\(x \in \left( -\frac{\delta}{a^N}, \frac{\delta}{a^N} \right)\)时,结合\eqref{eq:100.1}\eqref{eq:100.2}\eqref{eq:100.3}式,我们有
\begin{align*}
|f(x)| = \left| \frac{f(a^Nx)}{b^N} \right| \leqslant \frac{M}{b^N} < \varepsilon.
\end{align*}
故\(\lim_{x \to 0} f(x) = f(0) = 0.\)

\end{proof}

\begin{example}
设 \(f \in C(\mathbb{R})\) 满足 \(f(x), f(x^2)\) 都是周期函数,证明:\(f\) 为常值函数. 
\end{example}
\begin{proof}
由\hyperref[proposition:连续周期函数必一致连续]{连续周期函数必一致连续}可知,\(f(x),f(x^2)\) 在 \(\mathbb{R}\) 上一致连续。
于是对任意满足 \(\lim_{n \to \infty} (x_n' - x_n'') = 0\) 的数列 \(\{x_n'\}, \{x_n''\}\),都有
\begin{align}
|f(x_n') - f(x_n'')|, |f((x_n')^2) - f((x_n'')^2)| \to 0, \quad n \to \infty. \label{eq:101.1}
\end{align}
设 \(f(x)\) 的周期为 \(T > 0\),则对 \(\forall c \in \mathbb{R}\),取 \(x_n' = \sqrt{nT + c}, x_n'' = \sqrt{nT}\),显然 \(x_n' - x_n'' = \frac{c}{\sqrt{nT + c} + \sqrt{nT}} \to 0\)。从而由 \eqref{eq:101.1} 式可得
\begin{align*}
|f((x_n')^2) - f((x_n'')^2)| = f(nT + c) - f(nT) = f(c) - f(0) \to 0.
\end{align*}
故 \(f(c) = f(0)\),\(\forall c \in \mathbb{R}\)。故 \(f\) 为常值函数。

\end{proof}


\begin{example}
计算函数方程 \( f(\log_2 x) = f(\log_3 x) + \log_5 x \) 所有 \(\mathbb{R}\) 上的连续解.
\end{example}
\begin{note}
注意到
\[
f\left( \frac{\ln x}{\ln 2} \right) = f\left( \frac{\ln x}{\ln 3} \right) + \frac{\ln x}{\ln 5}, \, x > 0.
\]
为了凑裂项的形式, 我们待定
\[
f\left( \frac{\ln a_n}{\ln 2} \right) = f\left( \frac{\ln a_n}{\ln 3} \right) + \frac{\ln a_n}{\ln 5}, \, n \in \mathbb{N}.
\]
注意到我们有两种选择
\[
\frac{\ln a_n}{\ln 2} = \frac{\ln a_{n+1}}{\ln 3}, \quad \frac{\ln a_n}{\ln 3} = \frac{\ln a_{n+1}}{\ln 2}.
\]
前者公比 \(\frac{\ln 3}{\ln 2} > 1\), 后者公比 \(\frac{\ln 2}{\ln 3} < 1\), 为了求和方便我们选取后者.
\end{note}
\begin{proof}
设$f \in C(\mathbb{R})$,对$\forall x \in \mathbb{R}$,取$a_1 = x$,$\ln a_n = \left( \frac{\ln 2}{\ln 3} \right)^{n-1} \cdot \ln x$,$n \in \mathbb{N}$。则$\lim_{n \to \infty} \ln a_n = 0$。此时有
\begin{align*}
\frac{\ln a_n}{\ln 3} = \frac{\ln a_{n+1}}{\ln 2}, \forall n \in \mathbb{N}.
\end{align*}
于是由条件可得
\begin{align*}
f\left( \frac{\ln x}{\ln 2} \right) = f\left( \frac{\ln x}{\ln 3} \right) + \frac{\ln x}{\ln 5} \Rightarrow f\left( \frac{\ln a_n}{\ln 2} \right) = f\left( \frac{\ln a_n}{\ln 3} \right) + \frac{\ln a_n}{\ln 5}
\end{align*}
\begin{align*}
\Rightarrow f\left( \frac{\ln a_n}{\ln 2} \right) = f\left( \frac{\ln a_{n+1}}{\ln 2} \right) + \frac{\ln a_n}{\ln 5}, n = 1, 2, \cdots
\end{align*}
因此
\begin{align*}
\sum_{n=1}^{\infty} \left[ f\left( \frac{\ln a_n}{\ln 2} \right) - f\left( \frac{\ln a_{n+1}}{\ln 2} \right) \right] = \sum_{n=1}^{\infty} \frac{\ln a_n}{\ln 5} = \frac{1}{\ln 5} \cdot \frac{\ln x}{1 - \frac{\ln 2}{\ln 3}}.
\end{align*}
\begin{align*}
\sum_{n=1}^{\infty} \left[ f\left( \frac{\ln a_n}{\ln 2} \right) - f\left( \frac{\ln a_{n+1}}{\ln 2} \right) \right] = f\left( \frac{\ln a_1}{\ln 2} \right) - \lim_{n \to \infty} f\left( \frac{\ln a_{n+1}}{\ln 2} \right) = f\left( \frac{\ln x}{\ln 2} \right) - f(0).
\end{align*}
故
\begin{align*}
\frac{1}{\ln 5} \cdot \frac{\ln x}{1 - \frac{\ln 2}{\ln 3}} = f\left( \frac{\ln x}{\ln 2} \right) - f(0) \stackrel{y = \frac{\ln x}{\ln 2}}{\Rightarrow} f(y) = f(0) + \frac{y \ln 2 \ln 3}{\ln 5 \ln \frac{3}{2}}.
\end{align*}

\end{proof}

\begin{example}
设 \( n \in \mathbb{N} \),\( f \in C^{n+2}(\mathbb{R}) \) 使得存在 \( \theta \in \mathbb{R} \) 满足对任何 \( x, h \in \mathbb{R} \) 都有
\begin{align*}
f(x + h) = f(x) + f'(x)h + \frac{f''(x)}{2}h^2 + \cdots + \frac{f^{(n-1)}(x)}{(n-1)!}h^{n-1} + \frac{f^{(n)}(x + \theta h)}{n!}h^n
\end{align*}
证明: \( f \) 是不超过 \( n + 1 \) 次的多项式.
\end{example}
\begin{proof}
对$\forall x,h\in \mathbb{R}$,由Taylor公式可知
\begin{align*}
f^{(n)}(x+\theta h) =f^{(n)}(x) +f^{(n+1)}(x) \theta h+\frac{f^{(n+1)}(\xi)}{2}\theta^2 h^2.
\end{align*}
再结合条件可得
\begin{align}
f(x+h) &=\sum_{j=0}^{n-1}\frac{f^{(j)}(x)}{j!}h^j+\frac{f^{(n)}(x+\theta h)}{n!}h^n \notag \\
&=\sum_{j=0}^n\frac{f^{(j)}(x)}{j!}h^j+\frac{f^{(n+1)}(x) \theta h+\frac{f^{(n+1)}(\xi)}{2}\theta^2 h^2}{n!}h^n, \label{eq:101.313}
\end{align}
由Taylor公式可知
\begin{align}
f(x+h) =\sum_{j=0}^{n+1}\frac{f^{(j)}(x)}{j!}h^j+\frac{f^{(n+2)}(\eta)}{(n+2)!}h^{n+2}. \label{eq:101.1313}
\end{align}
比较\eqref{eq:101.313}式和\eqref{eq:101.1313}式得
\begin{align}
\left[ \frac{1}{(n+1)!}-\frac{\theta}{n!} \right] f^{(n+1)}(x) =\left[ \frac{\theta^2 f^{(n+2)}(\xi)}{2n!}-\frac{f^{(n+2)}(\eta)}{(n+2)!} \right] h. \label{eq:101.12131}
\end{align}
当$\theta =\frac{1}{n+1}$时,我们有
\begin{align*}
\frac{\theta^2 f^{(n+2)}(\xi)}{2n!}=\frac{f^{(n+2)}(\eta)}{(n+2)!}.
\end{align*}
对上式令$h\rightarrow 0$,则$\xi,\eta \rightarrow x$,故此时就有
\begin{align*}
\frac{f^{(n+2)}(x)}{2n!(n+1)^2}=\frac{f^{(n+2)}(x)}{(n+2)!}\Rightarrow f^{(n+2)}(x) =0.
\end{align*}
当$\theta \ne \frac{1}{n+1}$时,对\eqref{eq:101.12131}式令$h\rightarrow 0$,则$\xi,\eta \rightarrow x$,故此时就有
\begin{align*}
\left[ \frac{1}{(n+1)!}-\frac{\theta}{n!} \right] f^{(n+1)}(x) =0\Rightarrow f^{(n+1)}(x) =0.
\end{align*}
因此,无论如何都有$f$是不超过$n+1$次的多项式.

\end{proof}

\begin{example}
\begin{enumerate}
\item 设
\begin{align}
P_n(x)=\frac{1}{2^n n!}\frac{\mathrm{d}^n}{\mathrm{d}x^n}(x^2 - 1)^n, \quad n = 1, 2, \cdots
\end{align}
证明多项式 \( P_n \) 的全部根都是实数且分布在 \( (-1, 1) \)。

\item 设 \( g(x)=e^{x^2}\frac{\mathrm{d}^n}{\mathrm{d}x^n}(e^{-x^2}) \),证明 \( g \) 是只有实根的多项式。
\end{enumerate}
\end{example}
\begin{note}
本题第1问叫做Legendre(勒让德)多项式,第2问叫做Hermite多项式。第2问用Rolle定理时注意无穷远点也会提供零点。
\end{note}
\begin{proof}
\begin{enumerate}
\item 显然$P_n$是$n$次多项式,且$\pm 1$是$\frac{\mathrm{d}^k}{\mathrm{d}x^k}(x^2-1)^n$的$n-k$重根$(0\leqslant k\leqslant n)$.由Rolle定理可知,$\frac{\mathrm{d}}{\mathrm{d}x}(x^2-1)^n$在$(-1,1)$有一个实根.于是再由Rolle定理可知,$\frac{\mathrm{d}^2}{\mathrm{d}x^2}(x^2-1)^n$在$(-1,1)$有两个不同实根.反复利用Rolle定理可得,$\frac{\mathrm{d}^n}{\mathrm{d}x^n}(x^2-1)^n$在$(-1,1)$有$n$个不同实根.而$n$次多项式有且仅有$n$个根,故$P_n$的全部根都是实数且分布在$(-1,1)$上.

\item 设$\frac{\mathrm{d}^k}{\mathrm{d}x^k}(e^{-x^2}) = P_k(x)e^{-x^2}$,$P_k$是$k$次多项式,$k\in \mathbb{N}$,显然$P_0(x) = 1$,于是
\begin{align*}
\frac{\mathrm{d}^{k+1}}{\mathrm{d}x^{k+1}}(e^{-x^2}) = \left[ P_k'(x) - 2xP_k(x) \right] e^{-x^2}.
\end{align*}
令$P_{k+1}(x) = P_k'(x) - 2xP_k(x)$,则由$P_k$是$k$次多项式可知$P_{k+1}(x)$是$k+1$次多项式。
故由数学归纳法可知
\begin{align*}
\frac{\mathrm{d}^n}{\mathrm{d}x^n}(e^{-x^2}) = P_n(x)e^{-x^2}, \quad P_n \in \mathbb{R}[x], \quad n \in \mathbb{N}.
\end{align*}
因此$g(x) = e^{-x^2}\frac{\mathrm{d}^n}{\mathrm{d}x^n}(e^{-x^2}) = P_n(x)$是$n$次多项式$(n \in \mathbb{N})$。
设$P_k$是有$k$个不同实根的多项式,这$k$个根为
\begin{align*}
x_1 < x_2 < \cdots < x_k.
\end{align*}
从而这$k$个根也是$P_k(x)e^{-x^2}$的根。由Rolle定理可知
\begin{align*}
P_{k+1}(x) = e^{-x^2}\frac{\mathrm{d}^k}{\mathrm{d}x^k}(e^{-x^2}) = e^{-x^2}\frac{\mathrm{d}}{\mathrm{d}x}\left( P_k(x)e^{-x^2} \right)
\end{align*}
在$(x_{j-1}, x_j)$,$j=2,3,\cdots,k$有实根。而$\lim\limits_{x \rightarrow \pm \infty} P_k(x)e^{-x^2} = 0$,故由Rolle定理可知$P_{k+1}(x)$在$(-\infty, x_1)$,$(x_k, +\infty)$上还各有一个实根。因此$P_{k+1}(x)$有$k+1$个根。故由数学归纳法可知$P_n(x)$有$n$个实根$(n \in \mathbb{N})$。又因为$g(x) = P_n(x)$是$n$次多项式,而$n$次多项式有且仅有$n$个根,所以$g(x) = P_n(x)$是只有实根的多项式。
\end{enumerate}

\end{proof}

\begin{example}
设 $f \in C^2(\mathbb{R})$ 且 $f, f', f''$ 都是正值函数. 若存在 $a, b > 0$ 使得
\begin{align*}
f''(x) \leqslant a f(x) + b f'(x), \quad \forall x \in \mathbb{R}.
\end{align*}
求 $f'(x) \leqslant c f(x)$ 恒成立的最小的 $c$.
\end{example}
\begin{proof}
考虑微分方程$y'' = ay + by'$,其特征方程为
\begin{align*}
x^2 - bx - a = 0 \Rightarrow x_1 = \frac{b + \sqrt{b^2 + 4a}}{2} > 0, \quad x_2 = \frac{b - \sqrt{b^2 + 4a}}{2} < 0.
\end{align*}
于是
\begin{align*}
f''(x) \leqslant a f(x) + b f'(x) \Longleftrightarrow f''(x) - x_1 f'(x) \leqslant x_2 \left[ f'(x) - x_1 f(x) \right].
\end{align*}
令$g(x) \triangleq f'(x) - x_1 f(x)$,则$g'(x) \leqslant x_2 g(x)$。再令$c(x) \triangleq \frac{g(x)}{e^{x_2 x}}$,则
\begin{align*}
c'(x) = \frac{g'(x) - x_2 g(x)}{e^{x_2 x}} \leqslant 0 \Rightarrow c(x) \leqslant \lim_{x \rightarrow -\infty} c(x) = \lim_{x \rightarrow -\infty} \frac{f'(x) - x_1 f(x)}{e^{x_2 x}}, \quad \forall x \in \mathbb{R}.
\end{align*}
由$f''$, $f'$, $f > 0$可知$f, f'$递增有下界。故$\lim_{x \rightarrow -\infty} f(x)$和$\lim_{x \rightarrow -\infty} f'(x)$都存在。因此
\begin{align*}
c(x) \leqslant \lim_{x \rightarrow -\infty} \frac{f'(x) - x_1 f(x)}{e^{x_2 x}} = 0, \quad \forall x \in \mathbb{R}.
\end{align*}
即
\begin{align*}
f'(x) \leqslant x_1 f(x), \quad \forall x \in \mathbb{R}.
\end{align*}
取$f(x) = e^{x_1 x}$,此时等号成立。故$c_{\min} = x_1 = \frac{b + \sqrt{b^2 + 4a}}{2}$。

\end{proof}
\begin{remark}
若存在$c < x_1$,使得$f'(x) \leqslant c f(x)$,$\forall x \in \mathbb{R}$,则
\begin{align*}
f'(x) \leqslant c f(x) < x_1 f(x), \quad \forall x \in \mathbb{R}.
\end{align*}
但是取当$f(x) = e^{x_1 x}$时,$f'(x) = x_1 f(x)$矛盾!故$c_{\min} = x_1$。
\end{remark}

\begin{example}
设 $f \in C^n(\mathbb{R}), n \in \mathbb{N}, f^{(k)}(x_0) = 0, k = 0, 1, 2, \cdots, n - 1$. 若对某个 $M > 0$ 和 $\lambda_0, \lambda_1, \cdots, \lambda_{n - 2} \geqslant  0, \lambda_{n - 1} \geqslant  1$ 有不等式
\begin{align*}
|f^{(n)}(x)| \leqslant  M \prod_{k = 0}^{n - 1} |f^{(k)}(x)|^{\lambda_k}, \forall x \in \mathbb{R}.
\end{align*}
证明 $f(x) \equiv 0$.
\end{example}
\begin{note}
因为原不等式是绝对值不等式,所以考虑两个微分方程
\begin{align*}
f^{(n)} &= f^{(n-1)} \cdot g \Rightarrow \frac{f^{(n)}}{f^{(n-1)}} = g \Rightarrow \ln f^{(n-1)} = \int_{x_0}^x g(y) \mathrm{d}y + C \Rightarrow f^{(n-1)} = C e^{\int_{x_0}^x g(y) \mathrm{d}y}.
\end{align*}
\begin{align*}
f^{(n)} &= -f^{(n-1)} \cdot g \Rightarrow \frac{f^{(n)}}{f^{(n-1)}} = -g \Rightarrow \ln f^{(n-1)} = -\int_{x_0}^x g(y) \mathrm{d}y + C \Rightarrow f^{(n-1)} = C e^{-\int_{x_0}^x g(y) \mathrm{d}y}.
\end{align*}
分离常量得到构造函数 $c_1(x) \triangleq \frac{f^{(n-1)}(x)}{e^{\int_{x_0}^x g(y) \mathrm{d}y}}$, $c_2(x) \triangleq f^{(n-1)}(x) e^{\int_{x_0}^x g(y) \mathrm{d}y}$.
回顾\hyperref[subsection:双绝对值问题]{双绝对值问题}的构造函数,我们需要的构造函数应是 $C_1(x) \triangleq c_1^2(x) = \frac{[f^{(n-1)}(x)]^2}{e^{2\int_{x_0}^x g(y) \mathrm{d}y}}$, $C_2(x) \triangleq c_2^2(x) = [f^{(n-1)}(x)]^2 e^{2\int_{x_0}^x g(y) \mathrm{d}y}$.
\end{note}
\begin{proof}
由条件可知
\begin{align*}
|f^{(n)}(x)| \leqslant |f^{(n-1)}(x)| \cdot g(x),
\end{align*}
其中 $g(x) = M \prod_{k=1}^{n-1} |f^{(k)}(x)|^{\lambda_k - 1}$. 从而 $f^{(n)}(x) f^{(n-1)}(x) \leqslant |f^{(n)}(x) f^{(n-1)}(x)| \leqslant |f^{(n-1)}(x)|^2 \cdot g(x)$.
\begin{align}
\label{eq:10001.13131}
\end{align}
令 $C_1(x) \triangleq \frac{[f^{(n-1)}(x)]^2}{e^{2\int_{x_0}^x g(y) \mathrm{d}y}}$, 则由 \eqref{eq:10001.13131} 式可知
\begin{align*}
C_1'(x) = \frac{2f^{(n-1)}(x) f^{(n)}(x) - 2g(x) [f^{(n-1)}(x)]^2}{e^{2\int_{x_0}^x g(y) \mathrm{d}y}} \leqslant 0.
\end{align*}
故 $C_1(x) \leqslant C_1(x_0) = 0, \forall x \geqslant x_0$. 因此 $C_1(x) = 0, \forall x \geqslant x_0$. 进而 $f^{(n-1)}(x) = 0, \forall x \geqslant x_0$.
令 $C_2(x) \triangleq [f^{(n-1)}(x)]^2 e^{2\int_{x_0}^x g(y) \mathrm{d}y}$, 则由 \eqref{eq:10001.13131} 式可知
\begin{align*}
C_2'(x) = \left[ 2f^{(n-1)}(x) f^{(n)}(x) + 2g(x) (f^{(n-1)}(x))^2 \right] e^{2\int_{x_0}^x g(y) \mathrm{d}y} \geqslant 0.
\end{align*}
故 $C_2(x) \leqslant C_2(x_0) = 0, \forall x \leqslant x_0$. 因此 $C_2(x) = 0, \forall x \leqslant x_0$. 进而 $f^{(n-1)}(x) = 0, \forall x \leqslant x_0$.
综上, $f^{(n-1)}(x) \equiv 0, x \in \mathbb{R}$. 又 $f^{(k)}(x_0) = 0, k = 0, 1, \cdots, n-1$, 故 $f(x) \equiv 0$.

\end{proof}

\begin{example}
设 $f$ 是直线上的非常值连续周期函数. 若 $g \in C(\mathbb{R})$ 且 $\varlimsup_{x \to +\infty} \frac{|g(x)|}{x} = +\infty$, 证明: $f \circ g$ 不是周期函数.
\end{example}
\begin{note}
$\varlimsup_{x \to +\infty} |g(x + \delta) - g(x)| = +\infty$.的证明类似\hyperref[theorem:函数Stolz定理]{函数Stolz定理}的证明.实际上就是利用了上极限版的函数Stolz定理,只不过我们之前并没有写出这个定理.
\end{note}
\begin{proof}
若 $f \circ g$ 是周期函数,则由\hyperref[proposition:连续周期函数必一致连续]{连续周期函数必一致连续}可知 $f \circ g$ 在 $\mathbb{R}$ 上一致连续.
设 $f$ 的周期为 $T > 0$, 记 $a \triangleq \max f - \min f > 0$, 则存在 $\delta > 0$, 使
\begin{align}
|f(g(x)) - f(g(y))| < a, \forall |x - y| \leqslant \delta. \label{eq:102.12}
\end{align}
先证 $\varlimsup_{x \to +\infty} |g(x + \delta) - g(x)| = +\infty$. 若 $\varlimsup_{x \to +\infty} |g(x + \delta) - g(x)| \ne +\infty$, 则存在 $A > 0$, 使
$|g(x + \delta) - g(x)| \leqslant A, \forall x \geqslant 0.$
对 $x \in [n\delta, (n + 1)\delta), n \in \mathbb{N}$, 我们有
\begin{align*}
\left| \frac{g(x)}{x} \right| \leqslant \frac{|g(x - n\delta)|}{n\delta} + \sum_{k=0}^{n - 1} \left| \frac{g(x - k\delta) - g(x - (k + 1)\delta)}{n\delta} \right| \leqslant \frac{1}{n\delta} \sup_{x \in [0, \delta]} |g(x)| + \frac{A}{\delta}.
\end{align*}
故 $\varlimsup_{x \to +\infty} \left| \frac{g(x)}{x} \right| \leqslant \frac{A}{\delta}$ 矛盾! 因此 $\varlimsup_{x \to +\infty} |g(x + \delta) - g(x)| = +\infty$.
于是存在 $x_0 \in \mathbb{R}$, 使得 $|g(x_0 + \delta) - g(x_0)| \geqslant T$. 由介值定理可知, 存在 $s, t \in [x_0, x_0 + \delta]$, 使得
$f(g(s)) = \max f, \quad f(g(t)) = \min f.$
从而由 \eqref{eq:102.12} 式可知
\begin{align*}
a = |f(g(s)) - f(g(t))| < a
\end{align*}
矛盾! 故 $f \circ g$ 不是周期函数 ($f \circ g$ 甚至不是一致连续函数).

\end{proof}

\begin{example}
设 \( f \in C[0,+\infty) \bigcap D^1(0,+\infty) \) 满足
\[
f(0) \geqslant 0, f'(x) \geqslant f^3(x), \forall x > 0.
\]
证明:
\[
f(x) = 0, \forall x \geqslant 0.
\]
\end{example}
\begin{note}
$y' = y^3$这个微分方程有两种解法得到两个不同的构造函数,即分别考虑$\frac{\mathrm{d}y}{y^3}=\mathrm{d}x$和$\frac{y'}{y}=y^2$得到
\begin{align*}
\int{\frac{\mathrm{d}y}{y^3}}=\int{\mathrm{d}x}\Longrightarrow -\frac{1}{2y^2}=x+C_1\Longrightarrow C=x+\frac{1}{2y^2};
\end{align*}
\begin{align*}
\int{\frac{y'}{y}\mathrm{d}y}=\int{y^2\mathrm{d}y}\Longrightarrow \ln y=\int{y^2\mathrm{d}y}\Longrightarrow y=Ce^{\int{y^2\mathrm{d}y}}\Longrightarrow C=\frac{y}{e^{\int{y^2\mathrm{d}y}}}.
\end{align*}
\end{note}
\begin{proof}
由条件可知
\begin{align*}
\left[ \frac{f\left( x \right)}{e^{\int_0^x{f^2\left( y \right) \mathrm{d}y}}} \right]' =\frac{f' \left( x \right) -f^3\left( x \right)}{e^{\int_0^x{f^2\left( y \right) \mathrm{d}y}}}\geqslant 0.
\end{align*}
从而
\begin{align*}
\frac{f\left( x \right)}{e^{\int_0^x{f^2\left( y \right) \mathrm{d}y}}}\geqslant \frac{f\left( 0 \right)}{1}\geqslant 0\Longrightarrow f\left( x \right) \geqslant 0,\forall x\geqslant 0.
\end{align*}
于是
\begin{align*}
f' \left( x \right) \geqslant f^3\left( x \right) \geqslant 0,\forall x\geqslant 0.
\end{align*}
若存在$a\geqslant 0$,使得$f\left( a \right) >0$,则由$f' \geqslant 0$知
\begin{align}
f\left( x \right) \geqslant f\left( a \right) >0,\forall x>a. \label{100.12836}
\end{align}
注意到对$\forall x\in \left( a,A \right)$,有
\begin{align*}
\left[ x+\frac{1}{f^2\left( x \right)} \right]' =\frac{f^3\left( x \right) -f' \left( x \right)}{f^3\left( x \right)}\leqslant 0.
\end{align*}
故
\begin{align}
\lim_{x\rightarrow +\infty}\left[ x+\frac{1}{f^2\left( x \right)} \right] \leqslant a+\frac{1}{2f^2\left( a \right)}<+\infty . \label{100.1212123}
\end{align}
而由$f' \geqslant 0$可知$f$递增,再结合\eqref{100.12836}式知$\lim_{x\rightarrow +\infty}f\left( x \right) \in \left[ f\left( a \right) ,+\infty \right]$,从而
\begin{align*}
\lim_{x\rightarrow +\infty}\frac{1}{f^2\left( x \right)}\in \left[ 0,\frac{1}{f\left( a \right)} \right] .
\end{align*}
于是$\lim_{x\rightarrow +\infty}\left[ x+\frac{1}{f^2\left( x \right)} \right] =+\infty$,这与\eqref{100.1212123}式矛盾!

\end{proof}

\begin{example}
设可导函数$f:[0,+\infty)\to\mathbb{R}$满足$\int_0^1 f(x)\mathrm{d}x = f(1)$且 $x f'(x) + f(x - 1) = 0, \forall x \geq 1$. 证明:$\lim\limits_{x\to\infty} f(x) = 0$.
\end{example}
\begin{proof}
注意到
\begin{align*}
\left( xf\left( x \right) -\int_{x-1}^x{f\left( t \right) \mathrm{d}t} \right) ' &= xf'\left( x \right) +f\left( x-1 \right) =0,\forall x\geqslant 1. 
\end{align*}
故
\begin{align}
xf\left( x \right) -\int_{x-1}^x{f\left( t \right) \mathrm{d}t} &= 1\cdot f\left( 1 \right) -\int_0^1{f\left( t \right) \mathrm{d}t}=0,\forall x\geqslant 1. \label{eq::24080129824890280180873590238508239072052352}
\end{align}
下面不妨设$f$不恒为$0$,否则结论是平凡的。
对$\forall x\geqslant 1$,都有
\begin{align}
\left| f\left( x \right) \right| &\leqslant \max_{y\in \left[ 0,x \right]} \left| f\left( y \right) \right|. \label{eq::2408012982489028018087359023850823907205235235}
\end{align}
设$x^*\geqslant 1$,满足
\[
\left| f\left( x^* \right) \right|=\max_{y\in \left[ 0,x^* \right]} \left| f\left( y \right) \right|.
\]
由\eqref{eq::24080129824890280180873590238508239072052352}式可得
\[
\max_{y\in \left[ 0,x^* \right]} \left| f\left( y \right) \right|=\left| f\left( x^* \right) \right|=\frac{\int_{x^*-1}^{x^*}{f\left( t \right) \mathrm{d}t}}{x^*}\leqslant \frac{\max\limits_{y\in \left[ 0,x^* \right]} \left| f\left( y \right) \right|}{x^*}\leqslant \max_{y\in \left[ 0,x^* \right]} \left| f\left( y \right) \right|,
\]
故$\frac{\max\limits_{y\in \left[ 0,x^* \right]} \left| f\left( y \right) \right|}{x^*}=\max\limits_{y\in \left[ 0,x^* \right]} \left| f\left( y \right) \right|$,进而要么$x^*=1$,要么$\max_{y\in \left[ 0,x^* \right]} \left| f\left( y \right) \right|=0$。显然若$\max_{y\in \left[ 0,x^* \right]} \left| f\left( y \right) \right|=0$,则$f\left( x \right) =0,\forall x\in \left[ 0,x^* \right]$。
即\eqref{eq::2408012982489028018087359023850823907205235235}式等号成立的充要条件就是$x^*=1$或$f\left( x \right) =0,\forall x\in \left[ 0,x^* \right]$。故可以断言存在$X\geqslant 1$,使得对$\forall x\geqslant X$,都有
\begin{align}
\left| f\left( x \right) \right| &< \max_{y\in \left[ 0,x \right]} \left| f\left( y \right) \right|. \label{eq::2408012982489028018087359023850823907205235236}
\end{align}
否则,对$\forall X\geqslant 1$,都存在$x_X\geqslant X$,使得$\left| f\left( x \right) \right|=\max_{y\in \left[ 0,x \right]} \left| f\left( y \right) \right|$,即存在$x_X=1$或$f\left( x \right) =0,\forall x\in \left[ 0,x_X \right]$。令$X\rightarrow +\infty$,则$x_X\rightarrow +\infty$,此时$f\left( x \right) =0,\forall x\in \left[ 0,+\infty \right)$。这与$f$不恒为$0$矛盾!故\eqref{eq::2408012982489028018087359023850823907205235236}式成立。我们再断言
\begin{align}
\left| f\left( x \right) \right| &\leqslant \max_{y\in \left[ 0,X \right]} \left| f\left( y \right) \right|,\forall x>1. \label{eq::2408012982489028018087359023850823907205235237}
\end{align}
否则,存在$x_0>1$,使得
\[
\left| f\left( x_0 \right) \right|>\max_{y\in \left[ 0,X \right]} \left| f\left( y \right) \right|.
\]
记
\[
x_1\triangleq \inf\left\{ x\in \left[ X,x_0 \right] \mid \left| f\left( x \right) \right|>\max_{y\in \left[ 0,X \right]} \left| f\left( y \right) \right| \right\},
\]
则由$f$的连续性和\eqref{eq::2408012982489028018087359023850823907205235236}式知
\[
\max_{y\in \left[ 0,X \right]} \left| f\left( y \right) \right|\leqslant \left| f\left( x_1 \right) \right|<\max_{y\in \left[ 0,x_1 \right]} \left| f\left( y \right) \right|.
\]
于是再由$f$的连续性知,存在$x_2\in \left( X,x_1 \right)$,使得
\[
\max_{y\in \left[ 0,X \right]} \left| f\left( y \right) \right|\leqslant \left| f\left( x_1 \right) \right|<\max_{y\in \left[ 0,x_1 \right]} \left| f\left( y \right) \right|=\left| f\left( x_2 \right) \right|.
\]
这与$x_1$的下确界定义矛盾!故\eqref{eq::2408012982489028018087359023850823907205235237}式成立,即$f$在$\left[ 0,+\infty \right)$上有界。因此再由\eqref{eq::24080129824890280180873590238508239072052352}式可得
\[
\lim_{x\rightarrow +\infty} f\left( x \right) =\lim_{x\rightarrow +\infty} \frac{\int_{x-1}^x{f\left( t \right) \mathrm{d}t}}{x}\leqslant \lim_{x\rightarrow +\infty} \frac{\sup\limits_{x\in \left[ 0,+\infty \right)} \left| \,f\left( x \right) \right|}{x}=0.
\]

\end{proof}

\begin{example}
设$F(x)$是$[0,+\infty)$上的单调递减函数,且
\begin{align}
\lim\limits_{x \to +\infty} F(x) = 0,\quad
\lim\limits_{n \to +\infty} \int_0^{+\infty} F(t) \sin \frac{t}{n} \mathrm{d}t = 0.\label{eq::23945789023589028225}
\end{align}
证明:
\begin{enumerate}[(i)]
\item \begin{align}
\lim\limits_{x \to +\infty} xF(x) = 0;\label{eq::23945789023589028226}
\end{align}

\item \begin{align}
\lim\limits_{x \to 0} \int_0^{+\infty} F(t) \sin(xt) \mathrm{d}t = 0.\label{eq::23945789023589028227}
\end{align}
\end{enumerate}
\end{example}
\begin{proof}
\begin{enumerate}[(i)]
\item 由\eqref{eq::23945789023589028225}知$F$非负. 由A-D判别法知本题涉及的积分都收敛. 注意到
\begin{align*}
\int_0^\infty F(t)\sin(tx)\mathrm{d}t &= \frac{1}{x}\int_0^\infty F\left(\frac{u}{x}\right)\sin u\mathrm{d}u = \frac{1}{x}\sum_{k=0}^\infty \int_{2k\pi}^{(2k+2)\pi} F\left(\frac{u}{x}\right)\sin u\mathrm{d}u \\
&= \frac{1}{x}\sum_{k=0}^\infty \left[ \int_{2k\pi}^{(2k+1)\pi} F\left(\frac{u}{x}\right)\sin u\mathrm{d}u + \int_{(2k+1)\pi}^{(2k+2)\pi} F\left(\frac{u}{x}\right)\sin u\mathrm{d}u \right] \\
&= \frac{1}{x}\sum_{k=0}^\infty \left[ \int_{2k\pi}^{(2k+1)\pi} F\left(\frac{u}{x}\right)\sin u\mathrm{d}u - \int_{2k\pi}^{(2k+1)\pi} F\left(\frac{u + \pi}{x}\right)\sin u\mathrm{d}u \right] \\
&= \frac{1}{x}\sum_{k=0}^\infty \int_{2k\pi}^{(2k+1)\pi} \left[ F\left(\frac{u}{x}\right) - F\left(\frac{u + \pi}{x}\right) \right]\sin u\mathrm{d}u \\
&\geqslant \frac{1}{x}\int_0^\pi \left[ F\left(\frac{u}{x}\right) - F\left(\frac{u + \pi}{x}\right) \right]\sin u\mathrm{d}u = \frac{1}{x}\int_0^{2\pi} F\left(\frac{u}{x}\right)\sin u\mathrm{d}u \geqslant 0,
\end{align*}
以及
\begin{align*}
\int_0^\infty F(t)\sin(tx)\mathrm{d}t &= \frac{1}{x}\int_0^\infty F\left(\frac{u}{x}\right)\sin u\mathrm{d}u = \frac{1}{x}\int_0^\pi F\left(\frac{u}{x}\right)\sin u\mathrm{d}u + \frac{1}{x}\sum_{k=0}^\infty \int_{(2k+1)\pi}^{(2k+3)\pi} F\left(\frac{u}{x}\right)\sin u\mathrm{d}u \\
&= \frac{1}{x}\int_0^\pi F\left(\frac{u}{x}\right)\sin u\mathrm{d}u + \frac{1}{x}\sum_{k=0}^\infty \left[ \int_{(2k+1)\pi}^{(2k+2)\pi} F\left(\frac{u}{x}\right)\sin u\mathrm{d}u + \int_{(2k+2)\pi}^{(2k+3)\pi} F\left(\frac{u}{x}\right)\sin u\mathrm{d}u \right] \\
&= \frac{1}{x}\int_0^\pi F\left(\frac{u}{x}\right)\sin u\mathrm{d}u + \frac{1}{x}\sum_{k=0}^\infty \left[ \int_{(2k+1)\pi}^{(2k+2)\pi} F\left(\frac{u}{x}\right)\sin u\mathrm{d}u - \int_{(2k+1)\pi}^{(2k+2)\pi} F\left(\frac{u + \pi}{x}\right)\sin u\mathrm{d}u \right] \\
&= \frac{1}{x}\int_0^\pi F\left(\frac{u}{x}\right)\sin u\mathrm{d}u + \frac{1}{x}\sum_{k=0}^\infty \int_{(2k+1)\pi}^{(2k+2)\pi} \left[ F\left(\frac{u}{x}\right) - F\left(\frac{u + \pi}{x}\right) \right]\sin u\mathrm{d}u \\
&\leqslant \frac{1}{x}\int_0^\pi F\left(\frac{u}{x}\right)\sin u\mathrm{d}u.
\end{align*}
从而
\begin{align}
0 \leqslant \int_0^{2\pi} \frac{F\left(\frac{t}{x}\right)\sin t}{x}\mathrm{d}t \leqslant \int_0^\infty F(t)\sin(tx)\mathrm{d}t = \int_0^\infty \frac{F\left(\frac{t}{x}\right)\sin t}{x}\mathrm{d}t \leqslant \int_0^\pi \frac{F\left(\frac{t}{x}\right)\sin t}{x}\mathrm{d}t. \label{eq::23945789023589028232}
\end{align}
上式取$x = \frac{1}{n}$并结合
\begin{align*}
\int_0^{+\infty} F(t) \sin \frac{t}{n} \mathrm{d}t &\geqslant n\int_0^{2\pi} F(nu)\sin u\mathrm{d}u \xlongequal{\text{区间再现}} n\int_0^\pi [F(nu) - F(2\pi n - nu)]\sin u\mathrm{d}u \\
&\geqslant n\int_0^{\frac{\pi}{2}} [F(nu) - F(2\pi n - nu)]\sin u\mathrm{d}u \geqslant n\int_0^{\frac{\pi}{2}} \left[ F\left(\frac{\pi n}{2}\right) - F\left(\frac{3\pi n}{2}\right) \right]\sin u\mathrm{d}u \\
&= n\left[ F\left(\frac{\pi n}{2}\right) - F\left(\frac{3\pi n}{2}\right) \right] \geqslant 0,
\end{align*}
和\eqref{eq::23945789023589028225}知
\[
\lim_{n \to +\infty} n\left[ F\left(\frac{\pi n}{2}\right) - F\left(\frac{3\pi n}{2}\right) \right] = 0.
\]
现在对任何$\varepsilon > 0$, 存在$N \in \mathbb{N}$使得对任何$n \geqslant N$都有
\begin{align}
n\left[ F\left(\frac{\pi n}{2}\right) - F\left(\frac{3\pi n}{2}\right) \right] \leqslant \varepsilon. \label{eq::23945789023589028233}
\end{align}
当正整数$k$充分大, 我们考虑
\[
b_n \triangleq k3^{n-1} \left[ F\left(\frac{\pi k3^{n-1}}{2}\right) - F\left(\frac{\pi k3^n}{2}\right) \right] \leqslant \varepsilon,
\]
则利用$\lim\limits_{x \to +\infty} F(x) = 0$和\eqref{eq::23945789023589028233}得
\[
0 \leqslant kF\left(\frac{k\pi}{2}\right) = k\sum_{n=1}^\infty \left[ F\left(\frac{\pi k3^{n-1}}{2}\right) - F\left(\frac{\pi k3^n}{2}\right) \right] = \sum_{n=1}^\infty \frac{b_n}{3^{n-1}} \leqslant \varepsilon \sum_{n=1}^\infty \frac{1}{3^{n-1}} = \frac{3}{2}\varepsilon.
\]
现在我们有$\lim\limits_{k \to +\infty} kF\left(\frac{k\pi}{2}\right) = 0$. 对$x \in [0,+\infty)$, 存在唯一的$k \in \mathbb{N}$使得$x \in \left[ \frac{k\pi}{2}, \frac{k+1}{2}\pi \right)$, 于是
\[
0 \leqslant xF(x) \leqslant \frac{(k+1)\pi}{2}F\left(\frac{k\pi}{2}\right) \to 0, k \to +\infty,
\]
因此我们证明了\eqref{eq::23945789023589028226}.

\item 我们由\eqref{eq::23945789023589028232}知
\begin{align}
\int_0^\infty F(t)\sin(tx)\mathrm{d}t \geqslant 0 \Rightarrow \varliminf_{x \to 0^+} \int_0^\infty F(t)\sin(tx)\mathrm{d}t \geqslant 0, \label{eq::23945789023589028234}
\end{align}
以及对任何$\eta > 0$, 我们有
\begin{align*}
\varlimsup_{x \to 0^+} \int_0^\infty F(t)\sin(tx)\mathrm{d}t &\leqslant \varlimsup_{x \to 0^+} \int_0^\pi \frac{t}{x}F\left(\frac{t}{x}\right) \cdot \frac{\sin t}{t}\mathrm{d}t \\
&\leqslant \varlimsup_{x \to 0^+} \int_0^\eta \frac{t}{x}F\left(\frac{t}{x}\right) \cdot \frac{\sin t}{t}\mathrm{d}t + \varlimsup_{x \to 0^+} \int_\eta^\pi \frac{t}{x}F\left(\frac{t}{x}\right) \cdot \frac{\sin t}{t}\mathrm{d}t \\
&\leqslant \sup_{y \in [0,+\infty)} yF(y) \cdot \int_0^\eta \frac{\sin t}{t}\mathrm{d}t + \varlimsup_{x \to 0^+} \sup_{y \in \left[ \frac{\eta}{x}, \frac{\pi}{x} \right]} yF(y) \cdot \int_0^\pi \frac{\sin t}{t}\mathrm{d}t \\
&= \sup_{y \in [0,+\infty)} yF(y) \cdot \int_0^\eta \frac{\sin t}{t}\mathrm{d}t,
\end{align*}
这里最后一个等号用到了\eqref{eq::23945789023589028226}. 由$\eta$任意性得
\[
\varlimsup_{x \to 0^+} \int_0^\infty F(t)\sin(tx)\mathrm{d}t \leqslant 0.
\]
结合\eqref{eq::23945789023589028234}即得\eqref{eq::23945789023589028227}.
\end{enumerate}

\end{proof}

\begin{example}

\end{example}
\begin{proof}


\end{proof}

\begin{example}

\end{example}
\begin{proof}


\end{proof}

\begin{example}

\end{example}
\begin{proof}


\end{proof}

\begin{example}

\end{example}
\begin{proof}


\end{proof}

\begin{example}

\end{example}
\begin{proof}


\end{proof}

\begin{example}

\end{example}
\begin{proof}


\end{proof}

\begin{example}

\end{example}
\begin{proof}


\end{proof}

\begin{example}

\end{example}
\begin{proof}


\end{proof}

\begin{example}

\end{example}
\begin{proof}


\end{proof}

\begin{example}

\end{example}
\begin{proof}


\end{proof}

\begin{example}

\end{example}
\begin{proof}


\end{proof}

\begin{example}

\end{example}
\begin{proof}


\end{proof}








\end{document}
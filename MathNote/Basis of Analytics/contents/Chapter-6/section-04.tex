\documentclass[../../main.tex]{subfiles}
\graphicspath{{\subfix{../../image/}}} % 指定图片目录,后续可以直接使用图片文件名。

% 例如:
% \begin{figure}[H]
% \centering
% \includegraphics{image-01.01}
% \caption{图片标题}
% \label{figure:image-01.01}
% \end{figure}
% 注意:上述\label{}一定要放在\caption{}之后,否则引用图片序号会只会显示??.

\begin{document}

\section{一致连续}

\begin{theorem}\label{theorem:一致连续的充要条件1}
\(f\) 在区间 \(I\) 一致连续的充要条件是对任何 \(\{x_n'\}_{n = 1}^{\infty}\),\(\{x_n''\}_{n = 1}^{\infty}\subset I\) 且 \(\lim_{n\rightarrow\infty}(x_n'' - x_n') = 0\) 都有 \(\lim_{n\rightarrow\infty}(f(x_n'') - f(x_n'))=0\).
\end{theorem}

\begin{theorem}[Cantor定理]\label{theorem:Cantor定理}
\(f\in C(a,b)\) 一致连续的充要条件是 \(\lim_{x\rightarrow a^{+}}f(x)\),\(\lim_{x\rightarrow b^{-}}f(x)\) 存在.
\end{theorem}
\begin{remark}
这个定理对$f\in C(a,b]$和$f\in C[a,b)$也成立.
\end{remark}

\begin{corollary}\label{corollary:闭区间上的连续函数一定一致连续.}
若$f\in C[a,b]$,则$f$在$[a,b]$上一致连续.
\end{corollary}

\begin{proposition}\label{proposition一致连续的充分不必要条件}
设\(f\in C[0,+\infty)\)且\(\lim_{x\rightarrow +\infty}f(x)\)存在。证明:\(f\)在\([0,+\infty)\)一致连续。
\end{proposition}
\begin{remark}
这个命题反过来并不成立,反例:$f(x)=\sqrt{x}$.因此这个条件只是函数一致连续的充分不必要条件.
\end{remark}
\begin{proof}
\(\forall \varepsilon > 0\),由Cauchy收敛准则可知,存在\(A > 0\),对\(\forall x_1, x_2 \geqslant A\),有
\begin{align}
\left| f(x_2) - f(x_1) \right| < \varepsilon.  \label{equation5.1-1.1}
\end{align}
由Cantor定理可知,\(f\)在\([0, A + 1]\)上一致连续。故存在\(\delta \in (0, 1)\),使得\(\forall x_1, x_2 \in [0, A + 1]\)且\(\left| x_2 - x_1 \right| \leqslant \delta\),有
\begin{align}
\left| f(x_2) - f(x_1) \right| < \varepsilon.\label{equation5.1-1.2}
\end{align}
现在对\(\forall \left| x_1 - x_2 \right| \leqslant \delta < 1\),必然有\(x_1, x_2 \in [0, A + 1]\)或\(x_1, x_2 \in [A, +\infty)\),从而由\eqref{equation5.1-1.1}\eqref{equation5.1-1.2}式可知,此时一定有
\[
\left| f(x_2) - f(x_1) \right| < \varepsilon.
\]
故\(f\)在\([0, +\infty)\)上一致连续。
\end{proof}

\begin{proposition}\label{proposition:连续函数在无穷远处与一致连续极限相同则一定也一致连续}
设\(f\)在\([0,+\infty)\)一致连续且\(g\in C[0,+\infty)\)满足
\[
\lim_{x\rightarrow +\infty}[f(x)-g(x)] = 0.
\]
证明:\(g\)在\([0,+\infty)\)一致连续。
\end{proposition}
\begin{proof}
\(\forall \varepsilon > 0\),由\(f\)一致连续可知,存在\(\delta \in (0, 1)\),使得对\(\forall x, y \in [0, +\infty)\)且\(\vert x - y \vert \leq \delta\),有
\begin{align}\label{proposition-equation5.2-1.1}
\vert f(x) - f(y) \vert < \frac{\varepsilon}{3}. 
\end{align}
由\(\lim_{x \to +\infty}[f(x) - g(x)] = 0\)可知,存在\(A > 0\),使得对\(\forall x \geq A\),有
\begin{align}\label{proposition-equation5.2-1.2}
\vert f(x) - g(x) \vert < \frac{\varepsilon}{3}.  
\end{align}
由Cantor定理可知,\(g\)在\([0, A + 1]\)上一致连续。故存在\(\eta \in (0, 1)\),使得对\(\forall x, y \in [0, A + 1]\)且\(\vert x - y \vert \leq \eta\),有
\begin{align}\label{proposition-equation5.2-1.3}
\vert g(x) - g(y) \vert < \frac{\varepsilon}{3}. 
\end{align}
故对\(\forall x, y \geq 0\)且\(\vert x - y \vert \leq \eta\),要么都落在\([0, A + 1]\),要么都落在\([A, +\infty)\)。
\begin{enumerate}[(i)]
\item 若\(x, y \in [0, A + 1]\),则由\eqref{proposition-equation5.2-1.3}式可得\(\vert g(x) - g(y) \vert < \frac{\varepsilon}{3}\);
\item 若\(x, y \in [A, +\infty)\),则由\eqref{proposition-equation5.2-1.1}\eqref{proposition-equation5.2-1.2}式可得
\[
\vert g(x) - g(y) \vert \leq \vert g(x) - f(x) \vert + \vert f(x) - f(y) \vert + \vert f(y) - g(y) \vert < \frac{\varepsilon}{3} + \frac{\varepsilon}{3} + \frac{\varepsilon}{3} = \varepsilon.
\]
\end{enumerate}
故\(g\)在\([0, +\infty)\)上一致连续。
\end{proof}

\begin{proposition}[连续周期函数必一致连续]\label{proposition:连续周期函数必一致连续}
设 \(f\) 是周期 \(T > 0\) 的 \(\mathbb{R}\) 上的连续函数,则 \(f\) 在 \(\mathbb{R}\) 上一致连续。
\end{proposition}
\begin{proof}
由 \hyperref[theorem:Cantor定理]{Cantor 定理},\(f\) 在 \([0, 2T]\) 一致连续,所以对任何 \(\epsilon > 0\),存在 \(\delta \in (0, T)\) 使得对 \(|x_1 - x_2| < \delta\),\(x_1, x_2 \in [0, 2T]\) 都有
\begin{align*}
|f(x_1) - f(x_2)| \leqslant \epsilon.
\end{align*}
现在对 \(x_1, x_2 \in \mathbb{R}\) 使得 \(0 < x_2 - x_1 < \delta\)。注意到
\begin{align*}
x_1 - \left\lfloor \frac{x_1}{T} \right\rfloor T \in [0, T), x_2 - \left\lfloor \frac{x_1}{T} \right\rfloor T \in [0, 2T), |x_1 - x_2| < \delta,
\end{align*}
我们有
\begin{align*}
|f(x_1) - f(x_2)| = \left| f\left( x_1 - \left\lfloor \frac{x_1}{T} \right\rfloor T \right) - f\left( x_2 - \left\lfloor \frac{x_1}{T} \right\rfloor T \right) \right| \leqslant \epsilon,
\end{align*}
这就证明了 \(f\) 在 \(\mathbb{R}\) 上一致连续。 
\end{proof}

\begin{proposition}\label{proposition:一致连续的充要条件-一致连续与Lipschitz连续的关系}
设 \(f\) 定义在区间 \(I\) 的函数. 证明 \(f\) 在区间 \(I\) 一致连续的充要条件是对任何 \(\varepsilon>0\),存在 \(M > 0\),使得对任何 \(x_1,x_2\in I\),都有
\[|f(x_2)-f(x_1)|\leq M|x_1 - x_2|+\varepsilon. \]
\end{proposition}
\begin{remark}
这个命题相当重要!但是考试中不能直接使用,需要证明.
\end{remark}
\begin{proof}
{\heiti 充分性:} 由条件可知,\(\forall \varepsilon >0\),\(\exists M>0\),取\(\delta =\frac{\varepsilon}{M}\),则当\(\vert x_2 - x_1\vert\leqslant \delta\)且$x_1,x_2\in I$时,有
\begin{align*}
\vert f(x_1) - f(x_2)\vert \leqslant M\vert x_1 - x_2\vert+\varepsilon
\leqslant M\cdot\frac{\varepsilon}{M}+\varepsilon
= 2\varepsilon.
\end{align*}
故\(f\)在\(I\)上一致连续. 

{\heiti 必要性:}由\(f\)在\(I\)上一致连续可知,\(\forall \varepsilon >0\),存在\(\delta >0\),使得对\(\forall x_1,x_2\in I\)且\(\vert x_1 - x_2\vert\leqslant \delta\),有
\begin{align}
\vert f(x_1) - f(x_2)\vert<\varepsilon. \label{proposition5.1-1.1}
\end{align}
因此任取\(x,y\in I\),{\large \ding{192}}当\(\vert x - y\vert\leqslant \delta\)时,由\eqref{proposition5.1-1.1}式可知
\(
\vert f(x) - f(y)\vert<\varepsilon \leqslant M\vert x - y\vert+\varepsilon.
\)
由\(x,y\)的任意性可知结论成立.

{\large \ding{193}}当\(\vert x - y\vert>\delta\)时,\((\mathrm{i})\)当\(\vert f(x) - f(y)\vert\leqslant \varepsilon\)时,此时结论显然成立;

\((\mathrm{ii})\)当\(\vert f(x) - f(y)\vert>\varepsilon\)时,不妨设\(y > x\),\(f(y) > f(x)\)(其它情况类似),
\hyperlink{k,t的存在性}{令\(f(y) - f(x) = kt\),其中\(k\in \mathbb{N}\),\(t\in (\varepsilon,2\varepsilon]\).}由介值定理可知,存在\(x = x_0<x_1<\cdots <x_k = y\),使得
\[
f(x) \leqslant f(x_j) = f(x) + jt\leqslant f(x) + kt = f(y), j = 0,1,2,\cdots,k.
\]
于是
\[
f(x_j) - f(x_{j - 1}) = t>\varepsilon, j = 1,2,\cdots,k.
\]
此时由\eqref{proposition5.1-1.1}式可知\(x_j - x_{j - 1}>\delta\),\(j = 1,2,\cdots,k\)。从而我们有
\begin{align}
y - x=\sum_{j = 1}^k{(x_j - x_{j - 1})}>k\delta \Rightarrow k<\frac{y - x}{\delta}. \label{proposition5.1-1.2}  
\end{align}
取\(M = \frac{2\varepsilon}{\delta}>0\),于是结合\eqref{proposition5.1-1.2}式及\(t\in (\varepsilon,2\varepsilon]\)就有
\[
\vert f(y) - f(x)\vert=kt\leqslant \frac{t}{\delta}\vert y - x\vert\leqslant \frac{2\varepsilon}{\delta}\vert y - x\vert=M\vert y - x\vert.
\]
再由\(x,y\)的任意性可知结论成立.
\end{proof}
\begin{remark}
\hypertarget{k,t的存在性}{这里\(k,t\)的存在性可以如此得到:}考虑\((\varepsilon,+\infty)=\bigcup_{k\in \mathbb{N}}{(k\varepsilon,2k\varepsilon]}\)即可,又因为\((k + 1)\varepsilon\leqslant 2k\varepsilon\),所以相邻的\((k\varepsilon,2k\varepsilon]\)一定相交。
于是一定存在\(k\in \mathbb{N}\),使得\(f(y) - f(x)\in (k\varepsilon,2k\varepsilon]\),从而\(\frac{f(y) - f(x)}{k}\in (\varepsilon,2\varepsilon]\)。故取\(t = \frac{f(y) - f(x)}{k}\in (\varepsilon,2\varepsilon]\)。此时就有\(f(y) - f(x) = kt\)。 
\end{remark}

\begin{corollary}[一致连续函数被线性函数控制]\label{corollary:一致连续函数被线性函数控制}
若\(f\)在\(\mathbb{R}\)一致连续且\(f(0)=0\),证明存在\(M>0\)使得
\[
|f(x)|\leqslant1 + M|x|,\forall x\in\mathbb{R}.
\]
\end{corollary}
\begin{note}
读者应该积累大概的感觉:一致连续函数的增长速度不超过线性函数,这能帮助我们快速排除一些非一致连续函数。
\end{note}
\begin{proof}
取\hyperref[proposition:一致连续的充要条件-一致连续与Lipschitz连续的关系]{命题\ref{proposition:一致连续的充要条件-一致连续与Lipschitz连续的关系}}中的$\varepsilon=1,x_1=x\in \mathbb{R},x_2=0$,则一定存在$M>0$,使得\(
|f(x)|\leqslant1 + M|x|,\forall x\in\mathbb{R}\).
\end{proof}

\begin{corollary}\label{corollary:一致连续函数被线性函数控制1}
若\(f\)在$I$上一致连续,则存在\(M,c>0\)使得
\[
|f(x)|\leqslant c + M|x|,\forall x\in I.
\]
\end{corollary}

\begin{corollary}[一致连续函数的阶的提升]\label{corollary:一致连续函数的阶的提升}
若\(f\)在\([1,+\infty)\)一致连续,证明存在\(M > 0\)使得
\[
\left|\frac{f(x)}{x}\right|\leqslant M,\forall x\geqslant1.
\]
\end{corollary}
\begin{proof}
取\hyperref[proposition:一致连续的充要条件-一致连续与Lipschitz连续的关系]{命题\ref{proposition:一致连续的充要条件-一致连续与Lipschitz连续的关系}}中的$\varepsilon=1,x_1=x\geqslant 1,x_2=1$,则一定存在$C>0$,使得
\begin{align*}
|f(x)-f\left( 1 \right) |\leqslant C|x-1|+1,\forall x\geqslant 1.
\end{align*}
于是
\begin{align*}
\left| \frac{f\left( x \right)}{x} \right|\leqslant \left| \frac{f\left( x \right) -f\left( 1 \right)}{x} \right|+\frac{\left| f\left( 1 \right) \right|}{x}\leqslant \frac{C\left| x-1 \right|+1}{x}+\left| f\left( 1 \right) \right|,\forall x\geqslant 1.
\end{align*}
上式两边同时令$x\to +\infty$,得到
\begin{align*}
\underset{x\rightarrow +\infty}{\overline{\lim }}\left| \frac{f\left( x \right)}{x} \right|\leqslant C.
\end{align*}
由上极限的定义可知,存在$X>1$,使得$\underset{x\geqslant X}{\mathrm{sup}}\left| \frac{f\left( x \right)}{x} \right|\leqslant C$.从而我们有
\begin{align}\label{equation-586165}
\left| \frac{f\left( x \right)}{x} \right|\leqslant C,\forall x>X.
\end{align}
又因为$f$在$[1,+\infty)$上一致连续,所以由Cantor定理可知$f$在$[1,X]$上连续,从而$f$在$[1,X]$上有界,即存在$C'>0$,使得
\begin{align}\label{equation-586164}
\left| \frac{f\left( x \right)}{x} \right|\leqslant C',\forall x\in [1,X].
\end{align}
于是取$M=\max\{C,C'\}$,则由\eqref{equation-586165}\eqref{equation-586164}式可知
\begin{align*}
\left|\frac{f(x)}{x}\right|\leqslant M,\forall x\geqslant1.
\end{align*}
\end{proof}

\begin{proposition}\label{proposition:一致连续的充要条件-分式形式}
证明区间\(I\)上的函数\(f\)一致连续的充要条件是对任何\(\varepsilon > 0\),存在\(\ell > 0\),使得当\(x_1\neq x_2\in I\),就有:
\[
\left|\frac{f(x_2)-f(x_1)}{x_2 - x_1}\right|>\ell\Rightarrow|f(x_2)-f(x_1)|<\varepsilon.
\]
\end{proposition}
\begin{proof}
{\heiti 必要性:}由\hyperref[proposition:一致连续的充要条件-一致连续与Lipschitz连续的关系]{命题\ref{proposition:一致连续的充要条件-一致连续与Lipschitz连续的关系}}可知,
$\forall \varepsilon>0,\exists M > 0$,使得
\[
|f(x)-f(y)|\leq M|x - y|+\varepsilon,\forall x,y\in I.
\]
取\(\ell=\frac{\varepsilon}{\delta}+M\),任取\(x_1\neq x_2\in I\),当\(\left|\frac{f(x_2)-f(x_1)}{x_2 - x_1}\right|>\ell\)时,我们有
\[
\ell<\left|\frac{f(x_2)-f(x_1)}{x_2 - x_1}\right|\leq\frac{M|x_2 - x_1|}{|x_2 - x_1|}+\frac{\varepsilon}{|x_2 - x_1|}=M+\frac{\varepsilon}{|x_2 - x_1|}.
\]
从而
\begin{align}
|x_2 - x_1|<\frac{\varepsilon}{\ell - M}=\delta.  \label{proposition5.2-1.1}
\end{align}
又由\(f\)在\(I\)上一致连续可知
\begin{align}
|f(x') - f(x'')|<\varepsilon,\forall x',x''\in I\text{且}|x' - x''|<\delta. \label{proposition5.2-1.2} 
\end{align}
因此结合\eqref{proposition5.2-1.1}\eqref{proposition5.2-1.2}式可得\(|f(x_2)-f(x_1)|<\varepsilon\)。故必要性得证.

{\heiti 充分性:}已知对\(\forall \varepsilon >0\),存在\(\ell >0\),使得\(\forall x_1\neq x_2\in I\),有
\begin{align}
\left|\frac{f(x_2) - f(x_1)}{x_2 - x_1}\right|>\ell\Rightarrow|f(x_2) - f(x_1)|<\varepsilon. \label{proposition-5.2-1.1}
\end{align}
取\(\delta \in \left(0,\frac{\varepsilon}{\ell}\right)\),若\(|f(x_2) - f(x_1)|\geqslant \varepsilon\)但\(|x_2 - x_1|\leqslant \delta\),则我们有
\[
\left|\frac{f(x_2) - f(x_1)}{x_2 - x_1}\right|\geqslant \frac{\varepsilon}{\delta}>\ell.
\]
而由\eqref{proposition-5.2-1.1}式可得,此时\(|f(x_2) - f(x_1)|<\varepsilon\)。矛盾! 故\(f\)在\(I\)上一致连续。 
\end{proof}

\begin{proposition}[一致连续函数的拼接]\label{proposition:一致连续函数的拼接}
设\(f\in C[0,+\infty)\),若存在\(\delta>0\)使得\(f\)在\([\delta,+\infty)\)一致连续,则\(f\)在\([0,+\infty)\)一致连续。
\end{proposition}
\begin{note}
证明的想法比结论本身重要,在和本命题叙述形式不同的时候需要快速准确判断出来\(f\)在\([0,+\infty)\)一致连续.
\end{note}
\begin{proof}
\(\forall \varepsilon >0\),由Cantor定理可知,\(f\)在\([0,\delta + 1]\)上一致连续。故存在\(\eta \in (0,1)\),使得\(\forall x,y\in [0,\delta + 1]\)且\(\vert x - y\vert\leqslant \eta\),都有
\begin{align}
\vert f(x) - f(y)\vert<\varepsilon.\label{proposition5.3-1.1}
\end{align}
由\(f\)在\([\delta, +\infty)\)上一致连续可知,对\(\forall x,y\in [\delta, +\infty)\)且\(\vert x - y\vert\leqslant \eta\),都有
\begin{align}
\vert f(x) - f(y)\vert<\varepsilon. \quad \label{proposition5.3-1.2} 
\end{align}
现在对\(\forall x,y\in [0, +\infty)\),都有\(\vert x - y\vert\leqslant \eta\)。
\begin{enumerate}[(i)]
\item 若\(x,y\in [0,\delta + 1]\)或\([\delta, +\infty)\),则由\eqref{proposition5.3-1.1}\eqref{proposition5.3-1.2}式可直接得到\(\vert f(x) - f(y)\vert<\varepsilon\);

\item 若\(x\in [0,\delta + 1]\),\(y\in [\delta, +\infty)\),则\(\vert x - y\vert\geqslant 1>\eta\),这是不可能的。
\end{enumerate}
故原命题得证。
\end{proof}

\begin{example}
设\(f\)在\([1,+\infty)\)一致连续。证明:\(\frac{f(x)}{x}\)也在\([1,+\infty)\)一致连续。
\end{example}
\begin{proof}
由\(f\)在\([1, +\infty)\)上一致连续可知,\(\forall \varepsilon > 0\),存在\(\delta > 0\),使得对\(\forall x, y \geqslant 1\)且\(\vert x - y \vert \leqslant \delta\),有
\begin{align}\label{example5.1-1.1}
\vert f(x) - f(y) \vert < \frac{\varepsilon}{2}. 
\end{align}
由\hyperref[corollary:一致连续函数的阶的提升]{推论\ref{corollary:一致连续函数的阶的提升}}可知,$\left| \frac{f\left( x \right)}{x} \right|$有界.故可设$M\triangleq \underset{x\geqslant 1}{\mathrm{sup}}\left| \frac{f\left( x \right)}{x} \right|<+\infty$.取\(\delta' = \min \left\{ \delta, \frac{\varepsilon}{2M} \right\}\),则对\(\forall x, y \geqslant 1\)且\(\vert x - y \vert \leqslant \delta'\),由\eqref{example5.1-1.1}式可得
\begin{align*}
\left| \frac{f\left( x \right)}{x}-\frac{f\left( y \right)}{y} \right|&=\frac{\left| yf\left( x \right) -xf\left( y \right) \right|}{xy}\leqslant \frac{\left| yf\left( x \right) -yf\left( y \right) \right|+\left| y-x \right|\left| f\left( y \right) \right|}{xy}
\\
&=\frac{\left| f\left( x \right) -f\left( y \right) \right|}{x}+\frac{\left| y-x \right|}{xy}\left| f\left( y \right) \right|\leqslant \left| f\left( x \right) -f\left( y \right) \right|+M\left| y-x \right|
\\
&<\frac{\varepsilon}{2}+M\cdot \frac{\varepsilon}{2M}=\varepsilon .
\end{align*}
故\(\frac{f(x)}{x}\)也在\([1,+\infty)\)一致连续.
\end{proof}

\begin{proposition}[函数爆炸一定不一致连续]\label{proposition:函数爆炸一定不一致连续}
设\(f\)在\([a,+\infty)\)可微且\(\lim_{x\rightarrow +\infty}f^{\prime}(x)= +\infty\),证明:\(f\)在\([a,+\infty)\)不一致连续.
\end{proposition}
\begin{proof}
{\color{blue}证法一:}假设$f$在$\left[ a,+\infty \right) $上一致连续,则由\hyperref[corollary:一致连续函数被线性函数控制1]{推论\ref{corollary:一致连续函数被线性函数控制1}}可知,存在$c,d>0$,使得
\begin{align}
\left| f\left( x \right) \right|\leqslant c\left| x \right|+d,\forall x\in \left[ a,+\infty \right) .\label{example5.2-1..1}
\end{align}
从而
\begin{align}\label{example5.2-1..2}
\underset{x\rightarrow +\infty}{\underline{\lim }}\left| \frac{f\left( x \right)}{x} \right|\leqslant \underset{x\rightarrow +\infty}{\overline{\lim }}\left| \frac{f\left( x \right)}{x} \right|<+\infty .
\end{align}
由\hyperref[theorem:上下极限L'Hospital法则]{上下极限L'Hospital法则}可得
\begin{align*}
\underset{x\rightarrow +\infty}{\underline{\lim }}\frac{f\left( x \right)}{x}\geqslant \underset{x\rightarrow +\infty}{\underline{\lim }}f\prime\left( x \right) =+\infty .
\end{align*}
这与\eqref{example5.2-1..2}式矛盾.故\(f\)在\([a,+\infty)\)不一致连续.

{\color{blue}证法二:}假设$f$在$\left[ a,+\infty \right) $上一致连续,则由\hyperref[corollary:一致连续函数被线性函数控制1]{推论\ref{corollary:一致连续函数被线性函数控制1}}可知,存在$c,d>0$,使得
\begin{align}
\left| f\left( x \right) \right|\leqslant c\left| x \right|+d,\forall x\in \left[ a,+\infty \right) .\label{example5.2-0..1}
\end{align}
由\(\lim_{x\rightarrow +\infty}f^\prime(x) = +\infty\)可知,存在\(X > 0\),使得对\(\forall x\geqslant X\),有
\[
f^\prime(x) \geqslant c + 1 \Leftrightarrow f^\prime(x) - c + 1 \geqslant 0.
\]
从而\(f(x) - (c + 1)x\)在\([X, +\infty)\)上单调递增,于是就有
\[
f(x) - (c + 1)x \geqslant f(X) - (c + 1)X \triangleq D, \forall x\geqslant X.
\]
故\(f(x) \geqslant (c + 1)x + D, \forall x\geqslant X\)。再结合\eqref{example5.2-0..1}式可得
\[
(c + 1)x + D \leqslant f(x) \leqslant cx + d, \forall x\geqslant X > 0.
\]
即\(x \leqslant d - D, \forall x\geqslant X > 0\)。令\(x\rightarrow +\infty\),则
\[
+\infty = \lim_{x\rightarrow +\infty}x \leqslant d - D.
\]
矛盾。故\(f\)在\([a,+\infty)\)不一致连续.
\end{proof}


\begin{example}
判断下述函数的一致连续性:
\begin{enumerate}[(1)]
\item \( f(x) = \ln x, \quad x \in (0, 1]; \)

\item \( f(x) = e^x \cos \frac{1}{x}, \quad x \in (0, 1]; \)

\item \( f(x) = \frac{\sin x}{x}, \quad x \in (0, +\infty); \)

\item \( f(x) = \sin^2 x, \quad x \in \mathbb{R}; \)

\item \( f(x) = e^x, \quad x \in \mathbb{R}; \)

\item \( f(x) = \sin x^2, \quad x \in [0, +\infty); \)

\item \( f(x) = \sin (x \sin x), \quad x \in [0, +\infty); \)

\item \( f(x) = x \cos x, \quad x \in [0, +\infty); \)

\item 设 \( a > 0, \quad f(x) = \frac{x+2}{x+1} \sin \frac{1}{x}, \quad x \in (0, a) \) 和 \( x \in (a, +\infty); \)
\end{enumerate}
\end{example}
\begin{note}
关于三角函数找数列的问题,一般$\sin,\cos$函数就多凑一个$2n\pi$或$2n\pi+\frac{\pi}{2}$.
\end{note}
\begin{remark}
\hypertarget{找这两个数列的方法}{\textbf{\ref{example5.5(6)}中找这两个数列$\boldsymbol{x}_{\boldsymbol{n}}^{\prime}=\sqrt{\mathbf{2}\boldsymbol{n\pi }},\boldsymbol{x}_{\boldsymbol{n}}^{\prime\prime}=\sqrt{\mathbf{2}\boldsymbol{n\pi }}+\frac{\mathbf{1}}{\sqrt{\boldsymbol{n}}}$的方式:}}待定 $c_n$,令 $x_{n}^{\prime} = \sqrt{2n\pi}$,$x_{n}^{\prime\prime} = \sqrt{2n\pi} + c_n$,我们希望
\[
\lim_{n \rightarrow \infty} \left( x_{n}^{\prime\prime} - x_{n}^{\prime} \right) = \lim_{n \rightarrow \infty} c_n = 0,
\]
并且
\[
\lim_{n \rightarrow \infty} \left[ f\left( x_{n}^{\prime\prime} \right) - f\left( x_{n}^{\prime} \right) \right] = \lim_{n \rightarrow \infty} \sin \left( 2n\pi + c_{n}^{2} + 2c_n \sqrt{2n\pi} \right) = \lim_{n \rightarrow \infty} \sin \left( c_{n}^{2} + 2c_n \sqrt{2n\pi} \right) \ne 0.
\]
再结合 $\lim_{n \rightarrow \infty} c_n = 0$ 可得
\[
\lim_{n \rightarrow \infty} \sin \left( c_{n}^{2} + 2c_n \sqrt{2n\pi} \right) = \lim_{n \rightarrow \infty} \left( \sin c_{n}^{2} \cos 2c_n \sqrt{2n\pi} + \cos c_{n}^{2} \sin 2c_n \sqrt{2n\pi} \right) = \lim_{n \rightarrow \infty} \sin 2c_n \sqrt{2n\pi}.
\]
故我们希望 $\lim_{n \rightarrow \infty} c_n = 0$ 且 $\lim_{n \rightarrow \infty} \sin 2c_n \sqrt{2n\pi} \ne 0$。从而令 $c_n = \frac{1}{\sqrt{n}}$ 即可.

(7)(8)找数列的方式与(6)类似.
\end{remark}
\begin{solution}
\begin{enumerate}[(1)]
\item 不一致连续.由$\underset{x\rightarrow 0^+}{\lim}\ln x=+\infty $及\hyperref[theorem:Cantor定理]{Cantor定理}可得.

\item 不一致连续.由$\underset{x\rightarrow 0^+}{\lim}e^x\cos \frac{1}{x}$不存在及\hyperref[theorem:Cantor定理]{Cantor定理}可得.

\item 一致连续.由 $\lim_{x \rightarrow 0^+} f(1)$ 存在(连续性),$\lim_{x \rightarrow 0^+} \frac{\sin x}{x} = 1$ 及 \hyperref[theorem:Cantor定理]{Cantor定理}可知,$f$ 在 $(0, 1]$ 上一致连续。又因为 $\lim_{x \rightarrow +\infty} \frac{\sin x}{x} = 0$,所以由\hyperref[proposition一致连续的充分不必要条件]{命题\ref{proposition一致连续的充分不必要条件}}可知,$f$ 在 $[1, +\infty)$ 上一致连续。再根据\hyperref[proposition:一致连续函数的拼接]{一致连续函数的拼接}可知,$f$ 在 $(0, +\infty)$ 上一致连续.

\item 一致连续.由$(\sin^2 x)'=2\sin x \cos x\leq 2$及由Lagrange中值定理,易知$f(x)$是Lipschitz连续的,从而一致连续.

\item 不一致连续.由$\underset{x\rightarrow +\infty}{\lim}e^x=+\infty $及\hyperref[proposition:函数爆炸一定不一致连续]{命题\ref{proposition:函数爆炸一定不一致连续}}可得.

\item\label{example5.5(6)} 不一致连续.\hyperlink{找这两个数列的方法}{令 $x_{n}^{\prime} = \sqrt{2n\pi}$,$x_{n}^{\prime\prime} = \sqrt{2n\pi} + \frac{1}{\sqrt{n}}$},则
\(
\lim_{n \rightarrow \infty} \left( x_{n}^{\prime} - x_{n}^{\prime\prime} \right) = 0.
\)
但是
\begin{align*}
\lim_{n \rightarrow \infty} \left( f\left( x_{n}^{\prime\prime} \right) - f\left( x_{n}^{\prime} \right) \right) &= \lim_{n \rightarrow \infty} \sin \left( 2n\pi + \frac{1}{n} + 2\sqrt{2\pi} \right) = \lim_{n \rightarrow \infty} \sin \left( \frac{1}{n} + 2\sqrt{2\pi} \right)
\\
&= \lim_{n \rightarrow \infty} \left[ \sin 2\sqrt{2\pi} \cos \frac{1}{n} + \cos 2\sqrt{2\pi} \sin \frac{1}{n} \right] = \sin 2\sqrt{2\pi} \ne 0.
\end{align*}
故根据\hyperref[theorem:一致连续的充要条件1]{定理\ref{theorem:一致连续的充要条件1}}可知$f$不一致连续.

\item 不一致连续.\hyperlink{找这两个数列的方法}{令 $x_{n}^{\prime} = 2n\pi$,$x_{n}^{\prime\prime} = 2n\pi + \frac{\pi}{2n}$},则
\[
\lim_{n \rightarrow \infty} \left( x_{n}^{\prime} - x_{n}^{\prime\prime} \right) = 0.
\]
但是
\begin{align*}
\underset{n\rightarrow \infty}{\lim}\left( f\left( x_{n}^{''} \right) -f\left( x_{n}^{\prime} \right) \right) &=\underset{n\rightarrow \infty}{\lim}\sin \left[ \left( 2n\pi +\frac{\pi}{2n} \right) \sin \left( 2n\pi +\frac{\pi}{2n} \right) \right] =\underset{n\rightarrow \infty}{\lim}\sin \left[ \left( 2n\pi +\frac{\pi}{2n} \right) \sin \frac{\pi}{2n} \right] 
\\
&=\underset{n\rightarrow \infty}{\lim}\sin \left[ \left( 2n\pi +\frac{\pi}{2n} \right) \sin \frac{\pi}{2n} \right] =\underset{n\rightarrow \infty}{\lim}\sin \left[ \left( 2n\pi +\frac{\pi}{2n} \right) \left( \frac{\pi}{2n}+o\left( \frac{1}{n} \right) \right) \right] 
\\
&=\underset{n\rightarrow \infty}{\lim}\sin \left[ \pi ^2+o\left( \frac{1}{n^2} \right) \right] =\underset{n\rightarrow \infty}{\lim}\left[ \sin \pi ^2x\cos o\left( \frac{1}{n^2} \right) +\cos \pi ^2\sin o\left( \frac{1}{n^2} \right) \right] 
\\
&=\sin \pi ^2\ne 0.
\end{align*}
故根据\hyperref[theorem:一致连续的充要条件1]{定理\ref{theorem:一致连续的充要条件1}}可知$f$不一致连续.

\item 不一致连续.\hyperlink{找这两个数列的方法}{令 $x_{n}^{\prime} = 2n\pi + \frac{\pi}{2}$,$x_{n}^{\prime\prime} = 2n\pi + \frac{\pi}{2} + \frac{1}{n}$},则
\[
\lim_{n \rightarrow \infty} \left( x_{n}^{\prime} - x_{n}^{\prime\prime} \right) = 0.
\]
但是
\begin{align*}
\lim_{n \rightarrow \infty} \left( f\left( x_{n}^{\prime\prime} \right) - f\left( x_{n}^{\prime} \right) \right) = \lim_{n \rightarrow \infty} \left( 2n\pi + \frac{\pi}{2} + \frac{1}{n} \right) \cos \left( 2n\pi + \frac{\pi}{2} + \frac{1}{n} \right) 
= -\lim_{n \rightarrow \infty} \left( 2n\pi + \frac{\pi}{2} + \frac{1}{n} \right) \sin \frac{1}{n} 
= -2\pi.
\end{align*}
故根据\hyperref[theorem:一致连续的充要条件1]{定理\ref{theorem:一致连续的充要条件1}}可知$f$不一致连续.

\item 在$(0,a)$上不一致连续,在$(a,+\infty)$上一致连续.由$\lim_{x \rightarrow 0^+} \frac{x + 2}{x + 1} \sin \frac{1}{x}$不存在,$\lim_{x \rightarrow +\infty} \frac{x + 2}{x + 1} \sin \frac{1}{x} = 0$及\hyperref[theorem:Cantor定理]{Cantor定理}可得.
\end{enumerate}
\end{solution}

\begin{proposition}[一个重要不等式]\label{proposition:一个重要不等式1}
对 \(\alpha \in (0,1)\), 证明
\[
|x^\alpha - y^\alpha| \leq |x - y|^\alpha, \, \forall x, y \in [0, +\infty).
\]
\end{proposition}
\begin{proof}
不妨设 $y \geqslant x \geqslant 0$,则只须证 $y^{\alpha} - x^{\alpha} \leqslant \left( y - x \right)^{\alpha}$。则只须证 $\left( \frac{y}{x} \right)^{\alpha} - 1 \leqslant \left( \frac{y}{x} - 1 \right)^{\alpha}$。故只须证
\[
t^{\alpha} - 1 \leqslant \left( t - 1 \right)^{\alpha}, \forall t \geqslant 1.
\]
令 $g\left( t \right) = t^{\alpha} - 1 - \left( t - 1 \right)^{\alpha}$,则
\(
g'\left( t \right) = \alpha t^{\alpha - 1} - \alpha \left( t - 1 \right)^{\alpha - 1} \leqslant 0.
\)
从而 $g\left( t \right) \leqslant g\left( 1 \right) = 0$,$\forall t \geqslant 1$。故
\(
t^{\alpha} - 1 \leqslant \left( t - 1 \right)^{\alpha},\forall t \geqslant 1.
\)
\end{proof}

\begin{example}
证明:\( f(x) = x^{\alpha} \ln x \) 在 \( (0, +\infty) \) 一致连续的充要条件是 \( \alpha \in (0,1) \)。
\end{example}
\begin{proof}
当 $\alpha \geqslant 1$ 时,$f$ 不被线性函数控制,故由\hyperref[corollary:一致连续函数被线性函数控制1]{一致连续函数被线性函数控制}可知 $f$ 不一致连续。

当 $\alpha \leqslant 0$ 时,$\lim_{x \rightarrow 0^+} f\left( x \right)$ 不存在,由\hyperref[theorem:Cantor定理]{Cantor定理}可知,$f$ 在 $\left( 0,2 \right)$ 上不一致连续。故此时 $f$ 在 $\left( 0,+\infty \right)$ 上不一致连续。

当 $\alpha \in \left( 0,1 \right)$ 时,有
\(
f'\left( x \right) = x^{\alpha - 1} \left( \alpha \ln x - 1 \right).
\)
因此
\(
\lim_{x \rightarrow +\infty} f'\left( x \right) = 0,
\)
于是 $f'\left( x \right)$ 在 $\left[ 2,+\infty \right)$ 上有界,从而由Lagrange中值定理易得 $f$ 在 $\left[ 1,+\infty \right)$ 上 Lipschitz 连续,故 $f$ 在 $\left[ 2,+\infty \right)$ 上一致连续。此时,注意到
\(
\lim_{x \rightarrow 0^+} f\left( x \right) = 0,
\)
故由\hyperref[theorem:Cantor定理]{Cantor定理}可知,$f$ 在 $\left( 0,2 \right]$ 上一致连续。于是由\hyperref[proposition:一致连续函数的拼接]{一致连续的拼接}可得,$f$ 在 $\left( 0,+\infty \right)$ 上一致连续.
\end{proof}

\begin{example}
设 \( f(x) = \begin{cases} 
x^\alpha \cos \frac{1}{x}, & x > 0 \\ 
0, & x = 0 
\end{cases} \)。求 \(\alpha\) 的范围使得 \(f\) 在 \([0, +\infty)\) 一致连续。
\end{example}
\begin{note}
\hypertarget{找两个数列的方式123}{\textbf{找这两个数列$\boldsymbol{x}_{\boldsymbol{n}}^{\prime}=\mathbf{2}\boldsymbol{n\pi },\boldsymbol{x}_{\boldsymbol{n}}^{\prime\prime}=\mathbf{2}\boldsymbol{n\pi }+\boldsymbol{n}^{\mathbf{1}-\boldsymbol{\alpha }}$的方法:}}
当 $\alpha > 1$ 时,待定 $c_n$,令 $x_{n}^{\prime} = 2n\pi$,$x_{n}^{\prime\prime} = 2n\pi + c_n$。我们希望
\(
\lim_{n \rightarrow \infty} \left( x_{n}^{\prime\prime} - x_{n}^{\prime} \right) = \lim_{n \rightarrow \infty} c_n = 0,
\)
并且
\(
\lim_{n \rightarrow \infty} \left[ f\left( x_{n}^{\prime\prime} \right) - f\left( x_{n}^{\prime} \right) \right] \ne 0.
\)
注意到
\begin{align*}
f\left( x_{n}^{\prime\prime} \right) - f\left( x_{n}^{\prime} \right) &= \left( 2n\pi + c_n \right)^{\alpha} \cos \frac{1}{2n\pi + c_n} - \left( 2n\pi \right)^{\alpha} \cos \frac{1}{2n\pi} \\
&= \left( 2n\pi \right)^{\alpha} \left( 1 + \frac{c_n}{2n\pi} \right)^{\alpha} \cos \frac{1}{2n\pi + c_n} - \left( 2n\pi \right)^{\alpha} \cos \frac{1}{2n\pi} \\
&= \left( 2n\pi \right)^{\alpha} \left( 1 + \frac{c_n}{2n\pi} \right)^{\alpha} \left[ 1 + O\left( \frac{1}{\left( 2n\pi + c_n \right)^2} \right) \right] - \left( 2n\pi \right)^{\alpha} \left[ 1 + O\left( \frac{1}{n^2} \right) \right] \\
&= \left( 2n\pi \right)^{\alpha} \left( 1 + \frac{c_n}{2n\pi} \right)^{\alpha} \left[ 1 + O\left( \frac{1}{n^2} \right) \right] - \left( 2n\pi \right)^{\alpha} \left[ 1 + O\left( \frac{1}{n^2} \right) \right] \\
&= \left( 2n\pi \right)^{\alpha} \left[ \left( 1 + \frac{c_n}{2n\pi} \right)^{\alpha} - 1 \right] \left[ 1 + O\left( \frac{1}{n^2} \right) \right] \\
&= \left( 2n\pi \right)^{\alpha} \left[ \frac{\alpha c_n}{2n\pi} + O\left( \frac{c_n}{n^2} \right) \right] \left[ 1 + O\left( \frac{1}{n^2} \right) \right] \\
&= \left( 2n\pi \right)^{\alpha} \left[ \frac{\alpha c_n}{2n\pi} + O\left( \frac{c_n}{n^2} \right) \right], \quad n \rightarrow \infty.
\end{align*}
于是取 $c_n = n^{1 - \alpha}$,则 $\lim_{n \rightarrow \infty} c_n = 0$,并且由上式可得
\begin{align*}
f\left( x_{n}^{\prime\prime} \right) - f\left( x_{n}^{\prime} \right) &= \left( 2n\pi \right)^{\alpha} \left[ \frac{\alpha n^{-\alpha}}{2\pi} + O\left( n^{-\alpha - 1} \right) \right] \\
&= \alpha \left( 2\pi \right)^{\alpha - 1} + O\left( \frac{1}{n} \right) \rightarrow \alpha \left( 2\pi \right)^{\alpha - 1} \ne 0, \quad n \rightarrow \infty.
\end{align*}
故我们可取 $x_{n}^{\prime} = 2n\pi$,$x_{n}^{\prime\prime} = 2n\pi + n^{1 - \alpha}$。
\end{note}
\begin{proof}
当 $\alpha \leqslant 0$ 时,$\lim_{x \rightarrow 0^+} f\left( x \right)$ 不存在,由\hyperref[theorem:Cantor定理]{Cantor定理}可知,$f$ 在 $\left( 0,1 \right)$ 上不一致连续。故此时 $f$ 在 $\left( 0,+\infty \right)$ 上不一致连续。

当$\alpha \in (0,1]$时,由条件可知,对 $\forall x \geqslant 1$,都有
\[
\left| f'\left( x \right) \right| = \left| \left( x^{\alpha} \cos \frac{1}{x} \right)' \right| = \left| \alpha x^{\alpha - 1} \cos \frac{1}{x} - x^{\alpha - 2} \sin \frac{1}{x} \right| \leqslant \left| \alpha x^{\alpha - 1} \cos \frac{1}{x} \right| + \left| x^{\alpha - 2} \sin \frac{1}{x} \right| \leqslant \alpha + 1.
\]
因此 $f'\left( x \right)$ 在 $\left[ 1, +\infty \right)$ 上有界。从而由Lagrange中值定理易得 $f$ 在 $\left[ 1,+\infty \right)$ 上 Lipschitz 连续,故 $f$ 在 $\left[ 1,+\infty \right)$ 上一致连续。此时,注意到
\(
\lim_{x \rightarrow 0^+} f\left( x \right) = 0,
\)
故由\hyperref[theorem:Cantor定理]{Cantor定理}可知,$f$ 在 $\left( 0,1 \right]$ 上一致连续。于是由\hyperref[proposition:一致连续函数的拼接]{一致连续的拼接}可得,$f$ 在 $\left( 0,+\infty \right)$ 上一致连续.

当 $\alpha > 1$ 时,\hyperlink{找两个数列的方式123}{令 $x_{n}^{\prime} = 2n\pi$,$x_{n}^{\prime\prime} = 2n\pi + n^{1 - \alpha}$},则
\[
\lim_{n \rightarrow \infty} \left( x_{n}^{\prime\prime} - x_{n}^{\prime} \right) = \lim_{n \rightarrow \infty} n^{1 - \alpha} = 0.
\]
此时我们有
\begin{align*}
f\left( x_{n}^{\prime\prime} \right) - f\left( x_{n}^{\prime} \right) &= \left( 2n\pi + n^{1 - \alpha} \right)^{\alpha} \cos \frac{1}{2n\pi + n^{1 - \alpha}} - \left( 2n\pi \right)^{\alpha} \cos \frac{1}{2n\pi} \\
&= \left( 2n\pi \right)^{\alpha} \left( 1 + \frac{n^{-\alpha}}{2\pi} \right)^{\alpha} \cos \frac{1}{2n\pi + n^{1 - \alpha}} - \left( 2n\pi \right)^{\alpha} \cos \frac{1}{2n\pi} \\
&= \left( 2n\pi \right)^{\alpha} \left( 1 + \frac{n^{-\alpha}}{2\pi} \right)^{\alpha} \left[ 1 + O\left( \frac{1}{\left( 2n\pi + n^{1 - \alpha} \right)^2} \right) \right] - \left( 2n\pi \right)^{\alpha} \left[ 1 + O\left( \frac{1}{n^2} \right) \right] \\
&= \left( 2n\pi \right)^{\alpha} \left( 1 + \frac{n^{-\alpha}}{2\pi} \right)^{\alpha} \left[ 1 + O\left( \frac{1}{n^2} \right) \right] - \left( 2n\pi \right)^{\alpha} \left[ 1 + O\left( \frac{1}{n^2} \right) \right] \\
&= \left( 2n\pi \right)^{\alpha} \left[ \left( 1 + \frac{n^{-\alpha}}{2\pi} \right)^{\alpha} - 1 \right] \left[ 1 + O\left( \frac{1}{n^2} \right) \right] \\
&= \left( 2n\pi \right)^{\alpha} \left[ \frac{\alpha n^{-\alpha}}{2\pi} + O\left( n^{-\alpha - 1} \right) \right] \left[ 1 + O\left( \frac{1}{n^2} \right) \right] \\
&= \left( 2n\pi \right)^{\alpha} \left[ \frac{\alpha n^{-\alpha}}{2\pi} + O\left( n^{-\alpha - 1} \right) \right] \\
&= \alpha \left( 2\pi \right)^{\alpha - 1} + O\left( \frac{1}{n} \right) \rightarrow \alpha \left( 2\pi \right)^{\alpha - 1} \ne 0, \quad n \rightarrow \infty.
\end{align*}
故根据\hyperref[theorem:一致连续的充要条件1]{定理\ref{theorem:一致连续的充要条件1}}可知 $f$ 在 $\left[ 0, +\infty \right)$ 上不一致连续。
\end{proof}

\begin{example}
设 \( f_n : (0, +\infty) \to \mathbb{R}, n = 1, 2, \cdots \) 是一致连续函数且 \( f_n \rightarrow f \),证明:\( f \) 在 \( (0, +\infty) \) 一致连续。
\end{example}
\begin{proof}
$\forall \varepsilon > 0$,$\exists N \in \mathbb{N}$,使得当 $n \geqslant N$ 时,有
\begin{align}
\left| f_n\left( x \right) - f\left( x \right) \right| < \varepsilon, \quad \forall x \in \left( 0, +\infty \right). \label{example5.8-1.1}
\end{align}
由 $f_N$ 一致连续,可知 $\exists \delta > 0$,使得 $\forall x, y \in \left( 0, +\infty \right)$ 且 $\left| x - y \right| \leqslant \delta$,有
\begin{align}
\left| f_N\left( x \right) - f_N\left( y \right) \right| < \varepsilon. \label{example5.8-1.2}
\end{align}
于是对 $\forall x, y \in \left( 0, +\infty \right)$ 且 $\left| x - y \right| \leqslant \delta$ ,结合 \eqref{example5.8-1.1} 和 \eqref{example5.8-1.2} 式,我们有
\[
\left| f\left( x \right) - f\left( y \right) \right| \leqslant \left| f\left( x \right) - f_N\left( x \right) \right| + \left| f_N\left( x \right) - f_N\left( y \right) \right| + \left| f_N\left( y \right) - f\left( y \right) \right| < 3\varepsilon.
\]
故\( f \) 在 \( (0, +\infty) \) 一致连续。
\end{proof}

\begin{example}
设 \( f \) 在 \([0,+\infty)\) 一致连续且对任何 \( x \geq 0 \) 都有 \(\lim_{n \to \infty} f(x+n) = 0\),证明 \(\lim_{x \to +\infty} f(x) = 0\)。并说明如果去掉一致连续则结论不对。
\end{example}
\begin{note}
证明的想法即把点拉回到 \([0,1]\) 并用一致连续来解决。反例可积累
\[
f(x) = \frac{x \sin (\pi x)}{1 + x^2 \sin^2 (\pi x)}.
\]
\textbf{核心想法:分段放缩、取整平移、一致连续.}
\end{note}
\begin{proof}
由 $f$ 在 $\left[ 0, +\infty \right)$ 上一致连续可知,$\forall \varepsilon > 0$,$\exists \delta > 0$,使得当 $x, y \in \left[ 0, +\infty \right)$ 且 $\left| x - y \right| \leqslant \delta$ 时,有
\begin{align}
\left| f\left( x \right) - f\left( y \right) \right| < \varepsilon. \label{example5.9-1.1}
\end{align}
把 $\left[ 0,1 \right]$ 做 $N$ 等分,其中 $N = \frac{1}{\delta}$。由 $\lim_{n \rightarrow \infty} f\left( \frac{i}{N} + n \right) = 0$,$i = 0, 1, \cdots, N$ 可知,存在 $N' \in \mathbb{N}$,使得 $\forall n \geqslant N'$,有
\begin{align}
\left| f\left( \frac{i}{N} + n \right) \right| < \varepsilon, \quad i = 0, 1, \cdots, N. \label{example5.9-1.2} 
\end{align}
从而对 $\forall x \geqslant 1 + N'$,一定存在 $i \in \left\{ 0, 1, \cdots, N - 1 \right\}$,$n \in \mathbb{N} \cap \left[ N', +\infty \right)$,使得 $x \in \left[ \frac{i}{N} + n, \frac{i + 1}{N} + n \right]$。注意到此时
\[
\left| x - \left( \frac{i}{N} + n \right) \right| \leqslant \left| \left( \frac{i + 1}{N} + n \right) - \left( \frac{i}{N} + n \right) \right| = \frac{1}{N} = \delta.
\]
于是结合 \eqref{example5.9-1.1} 和 \eqref{example5.9-1.2} 式我们就有
\[
\left| f\left( x \right) \right| \leqslant \left| f\left( x \right) - f\left( \frac{i}{N} + n \right) \right| + \left| f\left( \frac{i}{N} + n \right) \right| < 2\varepsilon.
\]
故\(\lim_{x \to +\infty} f(x) = 0\).
\end{proof}


\end{document}
\documentclass[../../main.tex]{subfiles}
\graphicspath{{\subfix{../../image/}}} % 指定图片目录,后续可以直接使用图片文件名。

% 例如:
% \begin{figure}[H]
% \centering
% \includegraphics[scale=0.4]{图.png}
% \caption{}
% \label{figure:图}
% \end{figure}
% 注意:上述\label{}一定要放在\caption{}之后,否则引用图片序号会只会显示??.

\begin{document}

\section{微分不等式问题}

\subsection{一阶/二阶构造类}

\begin{proposition}[Gronwall不等式]\label{proposition:Gronwall不等式}
设 $\alpha,\beta,\mu\in C[a,b]$ 且 $\beta$ 非负,若还有
\begin{align}
\mu(t)\leqslant\alpha(t)+\int_{a}^{t}\beta(s)\mu(s)\mathrm{d}s,\forall t\in[a,b].
\label{equation---12.6}
\end{align}
证明:
\begin{align*}
\mu(t)\leqslant\alpha(t)+\int_{a}^{t}\beta(s)\alpha(s)e^{\int_{s}^{t}\beta(u)\mathrm{d}u}\mathrm{d}s,\forall t\in[a,b].
\end{align*}
若还有 $\alpha$ 递增,我们有
\begin{align*}
\mu(t)\leqslant\alpha(t)e^{\int_{a}^{t}\beta(s)\mathrm{d}s},\forall t\in[a,b].
\end{align*}
\end{proposition}
\begin{note}
解微分方程即得构造函数. 参考\hyperref[section单中值点问题]{单中值点问题}. 考虑 $F(t)=\int_{a}^{t}\beta(s)\mu(s)\mathrm{d}s$,则
\begin{align*}
F'(t)=\beta(t)\mu(t)\leqslant\beta(t)\alpha(t)+\beta(t)F(t).
\end{align*}
于是考虑微分方程
\begin{align*}
y'=\beta(t)\alpha(t)+\beta(t)y\Rightarrow y=ce^{\int_{a}^{t}\beta(s)\mathrm{d}s}+\int_{a}^{t}\beta(s)\alpha(s)e^{\int_{s}^{t}\beta(u)\mathrm{d}u}\mathrm{d}s.
\end{align*}
故得到构造函数
\begin{align*}
c(t)=\frac{F(t)-\int_{a}^{t}\beta(s)\alpha(s)e^{\int_{s}^{t}\beta(u)\mathrm{d}u}\mathrm{d}s}{e^{\int_{a}^{t}\beta(s)\mathrm{d}s}}=F(t)e^{-\int_{a}^{t}\beta(s)\mathrm{d}s}-\int_{a}^{t}\beta(s)\alpha(s)e^{\int_{s}^{a}\beta(u)\mathrm{d}u}\mathrm{d}s,t\in[a,b].
\end{align*} 
\end{note}
\begin{proof}
令
\begin{align}
c(t)=F(t)e^{-\int_{a}^{t}\beta(s)\mathrm{d}s}-\int_{a}^{t}\beta(s)\alpha(s)e^{\int_{s}^{a}\beta(u)\mathrm{d}u}\mathrm{d}s,t\in[a,b],\label{equation---12.001}
\end{align}
这里 $F(t)=\int_{a}^{t}\beta(s)\mu(s)\mathrm{d}s$. 由不等式\eqref{equation---12.6}知
\begin{align}
F'(t)\leqslant\alpha(t)\beta(t)+F(t)\beta(t),\forall t\in[a,b].\label{equation---12.17}
\end{align}
于是由\eqref{equation---12.001}和\eqref{equation---12.17}可知
\begin{align*}
c'(t)=[F'(t)-\alpha(t)\beta(t)-\beta(t)F(t)]e^{\int_{t}^{a}\beta(s)\mathrm{d}s}\leqslant0,
\end{align*}
因此$c(t)$在$[a,b]$上单调递减,从而
\begin{align*}
c(t)\leqslant c(a)=0,
\end{align*}
这就得到了
\begin{align*}
F(t)e^{-\int_{a}^{t}\beta(s)\mathrm{d}s}\leqslant\int_{a}^{t}\beta(s)\alpha(s)e^{\int_{s}^{a}\beta(u)\mathrm{d}u}\mathrm{d}s.
\end{align*}
再用一次不等式\eqref{equation---12.6},即得
\begin{align*}
\mu(t)\leqslant\alpha(t)+F(t)\leqslant\alpha(t)+\int_{a}^{t}\beta(s)\alpha(s)e^{\int_{s}^{t}\beta(u)\mathrm{d}u}\mathrm{d}s,\forall t\in[a,b].
\end{align*}
特别的,当 $\alpha$ 递增,对$\forall t\in [a,b]$,固定$t$,记$G(s)=\int_s^t{\beta (u)\mathrm{d}u}$,我们有不等式
\begin{align*}
\mu (t)&\leqslant \alpha (t)+\alpha (t)\int_a^t{\beta (s)e^{\int_s^t{\beta (u)\mathrm{d}u}}\mathrm{d}s}=\alpha (t)+\alpha (t)\int_a^t{-G' \left( s \right) e^{G\left( s \right)}\mathrm{d}s}
\\
&=\alpha (t)-\alpha (t)\int_a^t{e^{G\left( s \right)}\mathrm{d}G\left( s \right)}=\alpha (t)+\alpha (t)\left[ e^{G\left( a \right)}-1 \right] =\alpha (t)e^{\int_a^t{\beta (s)\mathrm{d}s}}.
\end{align*} 

\end{proof}

\begin{example}
设
\begin{align*}
E \triangleq \left\{ u \in C[0,1] : u^2(t) \leqslant 1 + 4 \int_0^t s|u(s)| \, \mathrm{d}s, \forall t \in [0,1] \right\},
\end{align*}
计算
\begin{align*}
\max_{u \in E} \int_0^1 \left[ u^2(s) - u(s) \right] \mathrm{d}s.
\end{align*}
\end{example}
\begin{note}
$g(x)$的构造思路:解微分方程常数变易法.具体如下
\begin{align*}
F'(x) \leqslant 4x\sqrt{F(x)} \Longrightarrow \frac{F'(x)}{2\sqrt{F(x)}} \leqslant 2x,
\end{align*}
两边同时积分得
\begin{align*}
\sqrt{F(x)} \leqslant x^2 + C \Longrightarrow C \geqslant \sqrt{F(x)} - x^2.
\end{align*}
故取$g(x) \triangleq \sqrt{F(x)} - x^2.$
\end{note}
\begin{proof}
记$F(x) = 1 + 4\int_0^x s|u(s)|\mathrm{d}s,$则
\begin{align*}
F'(x) = 4x|u(x)| \Longrightarrow |u(x)| = \frac{F'(x)}{4x}.
\end{align*}
由条件可得,当$u(x) \in E$时,对$\forall x \in [0,1]$,有
\begin{align*}
u^2(x) \leqslant F(x) \Longleftrightarrow |u(x)| \leqslant \sqrt{F(x)} \Longleftrightarrow \frac{F'(x)}{4x} \leqslant \sqrt{F(x)} \Longleftrightarrow F'(x) \leqslant 4x\sqrt{F(x)}.
\end{align*}
令$g(x) \triangleq \sqrt{F(x)} - x^2,$则
\begin{align*}
g'(x) = \frac{F'(x) - 4x\sqrt{F(x)}}{2\sqrt{F(x)}} \leqslant 0, \quad \forall x \in [0,1].
\end{align*}
故对$\forall x \in [0,1]$,有
\begin{align*}
g(x) \leqslant g(0) = 1 \Longrightarrow \sqrt{F(x)} \leqslant 1 + x^2.
\end{align*}
因此利用$|u(x)| \leqslant \sqrt{F(x)}$和上式可得
\begin{align*}
\int_0^1 \left[ u^2(x) - u(x) \right] \mathrm{d}x &\leqslant \int_0^1 u(x) \left[ u(x) - 1 \right] \mathrm{d}x \leqslant \int_0^1 |u(x)| \left( |u(x)| + 1 \right) \mathrm{d}x \\
&\leqslant \int_0^1 \sqrt{F(x)} \left( \sqrt{F(x)} + 1 \right) \mathrm{d}x \leqslant \int_0^1 \left( 1 + x^2 \right) \left( 2 + x^2 \right) \mathrm{d}x = \frac{16}{5}.
\end{align*}
当且仅当$u(x) = -1 - x^2$等号成立.故
\begin{align*}
\max_{u \in E} \int_0^1 \left[ u^2(x) - u(x) \right] \mathrm{d}x = \frac{16}{5}.
\end{align*}

\end{proof}

\begin{example}
设 $f$ 在 $[0,+\infty)$ 二阶可微且
\begin{align}
f(0),f'(0)\geqslant0,f''(x)\geqslant f(x),\forall x\geqslant0.
\label{equation12.19}
\end{align}
证明:
\begin{align}
f(x)\geqslant f(0)+f'(0)x,\forall x\geqslant0.
\label{equation12.20}
\end{align}
\end{example}
\begin{note}
通过 $f'' - f' = f - f'$ 视为一阶构造类来构造函数. (也可以尝试考虑$f''f'=ff'$,但是这样得到的构造函数处理本题可能不太方便)注意双曲三角函数和三角函数有着类似的不等式关系.
\end{note}
\begin{proof}
令 $h(x)=[f'(x)-f(x)]e^x$, 由\eqref{equation12.19}知
\begin{align*}
h' \left( x \right) =\left( f'' \left( x \right) -f' \left( x \right) +f' \left( x \right) -f\left( x \right) \right) e^x=\left( f'' \left( x \right) -f\left( x \right) \right) e^x \geqslant 0.
\end{align*}
故
\begin{align*}
h(x)\geqslant  h(0)=f'(0)-f(0)\Rightarrow [f'(x)-f(x)]e^x\geqslant  f'(0)-f(0)=h(0).
\end{align*}
继视为一阶构造类可得
\begin{align*}
c(x)=\frac{f(x)+\frac{1}{2}e^{-x}h(0)}{e^{x}},c'(x)=\frac{[f'(x)-f(x)]e^x-h(0)}{e^{3x}}\geqslant0.
\end{align*}
于是
\begin{align*}
\frac{f(x)+\frac{1}{2}e^{-x}h(0)}{e^{x}}\geqslant f(0)+\frac{1}{2}h(0)=\frac{f'(0)+f(0)}{2}.
\end{align*}
继续利用\eqref{equation12.19}即得
\begin{align*}
f(x)\geqslant\frac{e^{x}+e^{-x}}{2}f(0)+\frac{e^{x}-e^{-x}}{2}f'(0)\geqslant f(0)+f'(0)x,
\end{align*}
这里
\begin{align*}
\cosh x=\frac{e^{x}+e^{-x}}{2}\geqslant1,\sinh x=\frac{e^{x}-e^{-x}}{2}\geqslant x.
\end{align*}
可以分别了利用均值不等式和求导进行证明. 

\end{proof}

\begin{example}
设 $f\in C^1[0,+\infty)\cap D^2(0,+\infty)$ 且满足
\begin{align}
f''(x)-5f'(x)+6f(x)\geqslant0,f(0)=1,f'(0)=0.
\label{equation:::::12.23}
\end{align}
证明:
\begin{align}
f(x)\geqslant3e^{2x}-2e^{3x},\forall x\geqslant0.
\label{equation:::::12.24}
\end{align}
\end{example}
\begin{note}
显然如果把式\eqref{equation:::::12.23}得不等号改为等号, 则微分方程的解为 $3e^{2x}-2e^{3x}$. 现在对于不等号, 自然应该期望有不等式\eqref{equation:::::12.24}成立. 我们一阶一阶的视为一阶微分不等式来证明即可. 注意到 2,3 是微分方程的特征值根来改写命题. 本结果可以视为微分方程比较定理.
\end{note}
\begin{proof}
把不等式\eqref{equation:::::12.23}改写为
\begin{align*}
f''(x)-2f'(x)\geqslant3(f'(x)-2f(x)).
\end{align*}
考虑$g_1(x)=f'(x)-2f(x)$,则上式可化为
\begin{align*}
g_1'(x)\geqslant 3g_1(x).
\end{align*}
视为一阶构造类来构造函数,解得构造函数为$g_2(x)=\frac{g_1(x)}{e^{3x}}$.
于是有
\begin{align*}
g_2'(x)\geqslant0\Rightarrow g_2(x)\geqslant g_1(0)=-2\Rightarrow f'(x)-2f(x)\geqslant -2e^{3x}.
\end{align*}
进一步视为一阶构造类来构造函数,解得构造函数:
\begin{align*}
g_3(x)=\frac{f(x)}{e^{2x}}+2e^{x},g_3'(x)=\frac{f'(x)-2f(x)+2e^{3x}}{e^{2x}}\geqslant0,
\end{align*}
于是
\begin{align*}
g_3(x)\geqslant g_3(0)=3\Rightarrow f(x)\geqslant3e^{2x}-2e^{3x}.
\end{align*}
我们完成了证明. 

\end{proof}

\begin{example}
设 $f$ 在 $\mathbb{R}$ 上二阶可导且满足等式
\begin{align}\label{equation-----12.26}
f(x)+f''(x)=-xg(x)f'(x),g(x)\geqslant0.
\end{align}
证明 $f$ 在 $\mathbb{R}$ 上有界.
\end{example}
\begin{note}
$f + f''$ 的出现暗示我们构造 $|f(x)|^2 + |f'(x)|^2$, 这已是频繁出现的事实.因为等式右边有一个未知函数$g(x)$,所以我们考虑局部的微分方程,即只考虑等式左边,以此来得到构造函数.考虑$f+f''=0\Leftrightarrow ff'=-f''f'$,两边同时积分得到$\frac{1}{2}f^2=-\frac{1}{2}(f')^2+C$.由此得到构造函数$C(x)=|f(x)|^2+|f'(x)|^2$.
\end{note}
\begin{proof}
构造$h(x)=|f(x)|^2 + |f'(x)|^2$, 则由\eqref{equation-----12.26}知
\begin{align*}
h'(x)=2f'(x)[f(x)+f''(x)]=-2xg(x)[f'(x)]^2.
\end{align*}
于是 $h$ 在 $(-\infty,0]$ 递增,$[0,+\infty)$ 递减.
现在我们有
\begin{align*}
h(x)\leqslant h(0)\Longrightarrow |f(x)|^2\leqslant |f(x)|^2+|f' (x)|^2\leqslant h(0),
\end{align*}
即 $f$ 有界. 

\end{proof}

\begin{example}
设\(f \in C^{2}[0, +\infty)\),\(g \in C^{1}[0, +\infty)\)且存在\(\lambda > 0\)使得\(g(x) \geqslant \lambda\),\(\forall x \geqslant 0\)。若\(g'\)至多只有有限个零点且
\begin{align*}
f''(x)+g(x)f(x) = 0,\quad\forall x \geqslant 0,
\end{align*}
证明:\(f\)在\([0, +\infty)\)有界。
\end{example}
\begin{note}
形式计算分析需要的构造函数:由条件解微分方程可得
\begin{align*}
y' y'' =-gyy' \Longrightarrow \frac{(y' )^2}{2}=-\int{gyy' \mathrm{d}x}=-\frac{1}{2}\int{g\mathrm{d}y^2}=-\frac{1}{2}gy^2+\frac{1}{2}\int{y^2\mathrm{d}g}\Longrightarrow (y' )^2+gy^2=\int{y^2\mathrm{d}g};
\end{align*}
\begin{align*}
&y' y'' =-gyy' \Longrightarrow \frac{y' y''}{g}=-yy' \Longrightarrow \int{\frac{y' y''}{g}\mathrm{d}x}=-\int{yy' \mathrm{d}x}\Longrightarrow \int{\frac{1}{2g}\mathrm{d(}y' )^2}=-\frac{1}{2}y^2
\\
&\Longrightarrow \frac{(y' )^2}{2g}-\frac{1}{2}\int{(y' )^2\left( \frac{1}{g} \right) ' \mathrm{d}x}=-\frac{1}{2}y^2\Longrightarrow \frac{(y' )^2}{g}+y^2=\int{(y' )^2\left( \frac{1}{g} \right) ' \mathrm{d}x}.
\end{align*}
于是考虑构造函数\(F_1( x ) \triangleq \frac{| f'( x ) |^2}{g( x )}+f^2( x )\),\(F_2( x ) \triangleq | f'( x ) |^2+g( x ) f^2( x )\)。
\end{note}
\begin{proof}
因为\(g'\)至多只有有限个零点,所以存在\(X>0\),使得\(g'( x ) \ne 0\),\(\forall x\geqslant X\)。从而由导数介值性可知,\(g'\)在\([ X,+\infty )\)上要么恒大于\(0\),要么恒小于\(0\)。
令\(F_1( x ) \triangleq \frac{| f'( x ) |^2}{g( x )}+f^2( x )\),\(x\geqslant X\),则结合条件\(f'' =-gf\)可得
\begin{align}
F_{1}'( x ) =\frac{2f'f'' g-g'( f' )^2+2ff' g^2}{g^2}=\frac{-2f'fg^2-g'( f' )^2+2ff'g^2}{g^2}=-\frac{g'( f' )^2}{g^2}. \label{6.7}
\end{align}

(i) 若\(g'( x ) >0\),\(\forall x\geqslant X\),则由\eqref{6.7}式可知\(F'( x ) \leqslant 0\),即\(F( x )\)在\([ X,+\infty )\)上递减。
于是再结合\(g( x ) >\lambda >0\),\(\forall x>0\)可知,存在\(C>0\),使得
\[
f^2( x ) \leqslant F_1( x ) \leqslant C,\quad\forall x\geqslant X.
\]
故\(f( x )\)在\([ X,+\infty )\)上有界。又\(f\in C[ 0,+\infty )\),故\(f\)在\([ 0,X ]\)上必有界。因此\(f\)在\([ 0,+\infty )\)上有界。

(ii) 若\(g'( x ) <0\),\(\forall x\geqslant X\),令\(F_2( x ) \triangleq | f'( x ) |^2+g( x ) f^2( x )\),则结合条件\(f'' =-gf\)可得
\begin{align}
F_{2}'( x ) =2f'f'' +g'f^2+2gff' =-2f'fg+g'f^2+2gff' =g'f^2\leqslant 0. \label{6.8}
\end{align}
从而\(F_2( x )\)在\([ X,+\infty )\)上递减,于是存在\(C'>0\),使得
\[
g( x ) f^2( x ) \leqslant F_2( x ) \leqslant C,\quad\forall x\geqslant X.
\]
进而由\(g( x ) >\lambda >0\),\(\forall x>0\)可得
\[
f^2( x ) \leqslant \frac{C}{g( x )}\leqslant \frac{C}{\lambda},\quad\forall x\geqslant X.
\]
故\(f( x )\)在\([ X,+\infty )\)上有界。又\(f\in C[ 0,+\infty )\),故\(f\)在\([ 0,X ]\)上必有界。因此\(f\)在\([ 0,+\infty )\)上有界。

\end{proof}

\begin{example}
设 $f \in C^2(\mathbb{R})$ 且 $f, f', f''$ 都是正值函数. 若存在 $a, b > 0$ 使得
\begin{align*}
f''(x) \leqslant a f(x) + b f'(x), \quad \forall x \in \mathbb{R}.
\end{align*}
求 $f'(x) \leqslant c f(x)$ 恒成立的最小的 $c$.
\end{example}
\begin{remark}
若存在$c < x_1$,使得$f'(x) \leqslant c f(x)$,$\forall x \in \mathbb{R}$,则
\begin{align*}
f'(x) \leqslant c f(x) < x_1 f(x), \quad \forall x \in \mathbb{R}.
\end{align*}
但是取当$f(x) = e^{x_1 x}$时,从而$f'(x) =cf(x)= x_1 f(x)$,于是$c=x_1$矛盾!故$c_{\min} = x_1$。
\end{remark}
\begin{proof}
考虑微分方程$y'' = ay + by'$,其特征方程为
\begin{align*}
x^2 - bx - a = 0 \Rightarrow x_1 = \frac{b + \sqrt{b^2 + 4a}}{2} > 0, \quad x_2 = \frac{b - \sqrt{b^2 + 4a}}{2} < 0.
\end{align*}
于是
\begin{align*}
f''(x) \leqslant a f(x) + b f'(x) \Longleftrightarrow f''(x) - x_1 f'(x) \leqslant x_2 \left[ f'(x) - x_1 f(x) \right].
\end{align*}
令$g(x) \triangleq f'(x) - x_1 f(x)$,则$g'(x) \leqslant x_2 g(x)$。再令$c(x) \triangleq \frac{g(x)}{e^{x_2 x}}$,则
\begin{align*}
c'(x) = \frac{g'(x) - x_2 g(x)}{e^{x_2 x}} \leqslant 0 \Rightarrow c(x) \leqslant \lim_{x \rightarrow -\infty} c(x) = \lim_{x \rightarrow -\infty} \frac{f'(x) - x_1 f(x)}{e^{x_2 x}}, \quad \forall x \in \mathbb{R}.
\end{align*}
由$f''$, $f'$, $f > 0$可知$f, f'$递增有下界。故$\lim_{x \rightarrow -\infty} f(x)$和$\lim_{x \rightarrow -\infty} f'(x)$都存在。从而$\lim_{x \rightarrow -\infty} f'(x)=0$,否则由\refpro{proposition:导数有正增长率则函数爆炸}知$\lim_{x \rightarrow -\infty} f(x)=-\infty$矛盾!再结合$x_1,f>0,x_2<0$可得
\begin{align*}
c(x)\leqslant \lim_{x\rightarrow -\infty} \frac{f' (x)-x_1f(x)}{e^{x_2x}}\leqslant \lim_{x\rightarrow -\infty} \frac{f' (x)}{e^{x_2x}}=0,\quad \forall x\in \mathbb{R} .
\end{align*}
即
\begin{align*}
f'(x) \leqslant x_1 f(x), \quad \forall x \in \mathbb{R}.
\end{align*}
取$f(x) = e^{x_1 x}$,此时等号成立。故$c_{\min} = x_1 = \frac{b + \sqrt{b^2 + 4a}}{2}$。

\end{proof}

\begin{example}
设 \( f \in C[0,+\infty) \bigcap D^1(0,+\infty) \) 满足
\[
f(0) \geqslant 0, f'(x) \geqslant f^3(x), \forall x > 0.
\]
证明:
\[
f(x) = 0, \forall x \geqslant 0.
\]
\end{example}
\begin{note}
$y' = y^3$这个微分方程有三种解法得到三个不同的构造函数,即分别考虑$\frac{y'}{y^3}=1,\frac{y'}'{y^2}=y,\frac{y'}{y}=y^2$得到
\begin{align*}
\int{\frac{\mathrm{d}y}{y^3}}=\int{\mathrm{d}x}\Longrightarrow -\frac{1}{2y^2}=x+C_1\Longrightarrow C=x+\frac{1}{2y^2};
\end{align*}
\begin{align*}
\int{\frac{y\prime}{y^2}\mathrm{d}x}=\int{y\mathrm{d}x}\Longrightarrow \int{\frac{1}{y^2}\mathrm{d}y}=\int{y\mathrm{d}x}\Longrightarrow C-\frac{1}{y}=\int{y\mathrm{d}x}\Longrightarrow C=\frac{1}{y}+\int{y\mathrm{d}x};
\end{align*}
\begin{align*}
\int{\frac{y'}{y}\mathrm{d}x}=\int{y^2\mathrm{d}x}\Longrightarrow \ln y=\int{y^2\mathrm{d}x}\Longrightarrow y=Ce^{\int{y^2\mathrm{d}x}}\Longrightarrow C=\frac{y}{e^{\int{y^2\mathrm{d}x}}}.
\end{align*}
第二个构造函数实际上在本题中没发挥作用.
\end{note}
\begin{proof}
由条件可知
\begin{align*}
\left[ \frac{f\left( x \right)}{e^{\int_0^x{f^2\left( y \right) \mathrm{d}y}}} \right]' =\frac{f' \left( x \right) -f^3\left( x \right)}{e^{\int_0^x{f^2\left( y \right) \mathrm{d}y}}}\geqslant 0.
\end{align*}
从而
\begin{align*}
\frac{f\left( x \right)}{e^{\int_0^x{f^2\left( y \right) \mathrm{d}y}}}\geqslant \frac{f\left( 0 \right)}{1}\geqslant 0\Longrightarrow f\left( x \right) \geqslant 0,\forall x\geqslant 0.
\end{align*}
于是
\begin{align*}
f' \left( x \right) \geqslant f^3\left( x \right) \geqslant 0,\forall x\geqslant 0.
\end{align*}
若存在$a\geqslant 0$,使得$f\left( a \right) >0$,则由$f' \geqslant 0$知
\begin{align}
f\left( x \right) \geqslant f\left( a \right) >0,\forall x>a. \label{100.12836}
\end{align}
注意到对$\forall x\in \left( a,A \right)$,有
\begin{align*}
\left[ x+\frac{1}{f^2\left( x \right)} \right]' =\frac{f^3\left( x \right) -f' \left( x \right)}{f^3\left( x \right)}\leqslant 0.
\end{align*}
故
\begin{align}
\lim_{x\rightarrow +\infty}\left[ x+\frac{1}{f^2\left( x \right)} \right] \leqslant a+\frac{1}{2f^2\left( a \right)}<+\infty . \label{100.1212123}
\end{align}
而由$f' \geqslant 0$可知$f$递增,再结合\eqref{100.12836}式和\refpro{proposition:导数有正增长率则函数爆炸}知$\lim_{x\rightarrow +\infty} f\left( x \right) =+\infty $,从而$\lim_{x\rightarrow +\infty}\frac{1}{f^2\left( x \right)}=0 .$
于是$\lim_{x\rightarrow +\infty}\left[ x+\frac{1}{f^2\left( x \right)} \right] =+\infty$,这与\eqref{100.1212123}式矛盾!

\end{proof}

\begin{example}
设可导函数$f:[0,+\infty)\to\mathbb{R}$满足$\int_0^1 f(x)\mathrm{d}x = f(1)$且 $x f'(x) + f(x - 1) = 0, \forall x \geqslant 1$. 证明:$\lim\limits_{x\to+\infty} f(x) = 0$.
\end{example}
\begin{note}
本题关键是由$f\left( x \right) =\frac{\int_{x-1}^x{f\left( t \right) \mathrm{d}t}}{x}$看出只需证$f$有界,然后由
\begin{align*}
f\left( x \right) =\frac{\int_{x-1}^x{f\left( t \right) \mathrm{d}t}}{x}\leqslant \frac{\underset{y\in \left[ x-1,x \right]}{\max}f\left( y \right)}{x}<\frac{\underset{y\in \left[ 0,x \right]}{\max}f\left( y \right)}{x}\leqslant \underset{y\in \left[ 0,x \right]}{\max}f\left( y \right) ,\forall x>1.
\end{align*}
发现$f$的最大值就在在某个有限区间内取到.
\end{note}
\begin{proof}
注意到
\begin{align*}
\left( xf\left( x \right) -\int_{x-1}^x{f\left( t \right) \mathrm{d}t} \right) ' &= xf'\left( x \right) +f\left( x-1 \right) =0,\forall x\geqslant 1. 
\end{align*}
故
\begin{align}
xf\left( x \right) -\int_{x-1}^x{f\left( t \right) \mathrm{d}t} &= 1\cdot f\left( 1 \right) -\int_0^1{f\left( t \right) \mathrm{d}t}=0,\forall x\geqslant 1. \label{eq::24080129824890280180873590238508239072052352}
\end{align}
下面不妨设$f$不恒为$0$,否则结论是平凡的。
对$\forall x\geqslant 1$,都有
\begin{align}
\left| f\left( x \right) \right| &\leqslant \max_{y\in \left[ 0,x \right]} \left| f\left( y \right) \right|. \label{eq::2408012982489028018087359023850823907205235235}
\end{align}
设$x^*\geqslant 1$,满足
\[
\left| f\left( x^* \right) \right|=\max_{y\in \left[ 0,x^* \right]} \left| f\left( y \right) \right|.
\]
由\eqref{eq::24080129824890280180873590238508239072052352}式可得
\[
\max_{y\in \left[ 0,x^* \right]} \left| f\left( y \right) \right|=\left| f\left( x^* \right) \right|=\frac{\int_{x^*-1}^{x^*}{f\left( t \right) \mathrm{d}t}}{x^*}\leqslant \frac{\max\limits_{y\in \left[ 0,x^* \right]} \left| f\left( y \right) \right|}{x^*}\leqslant \max_{y\in \left[ 0,x^* \right]} \left| f\left( y \right) \right|,
\]
故$\frac{\max\limits_{y\in \left[ 0,x^* \right]} \left| f\left( y \right) \right|}{x^*}=\max\limits_{y\in \left[ 0,x^* \right]} \left| f\left( y \right) \right|=\frac{\int_{x^*-1}^{x^*}{f\left( t \right) \mathrm{d}t}}{x^*}$,进而要么$x^*=1$,要么$\max_{y\in \left[ 0,x^* \right]} \left| f\left( y \right) \right|=0$。显然若$\max_{y\in \left[ 0,x^* \right]} \left| f\left( y \right) \right|=0$,则$f\left( x \right) =0,\forall x\in \left[ 0,x^* \right]$。
即\eqref{eq::2408012982489028018087359023850823907205235235}式等号成立的充要条件就是$x^*=1$或$f\left( x \right) =0,\forall x\in \left[ 0,x^* \right]$。又$f$不恒为0,故存在$X> 1$,使得对$\forall x\geqslant X$,都有
\begin{align}
\left| f\left( x \right) \right| &< \max_{y\in \left[ 0,x \right]} \left| f\left( y \right) \right|. \label{eq::2408012982489028018087359023850823907205235236}
\end{align}
我们断言
\begin{align}
\left| f\left( x \right) \right| &\leqslant \max_{y\in \left[ 0,X \right]} \left| f\left( y \right) \right|,\forall x>1. \label{eq::2408012982489028018087359023850823907205235237}
\end{align}
否则,存在$x_0>1$,使得
\[
\left| f\left( x_0 \right) \right|>\max_{y\in \left[ 0,X \right]} \left| f\left( y \right) \right|.
\]
记
\[
x_1\triangleq \inf\left\{ x\in \left[ X,x_0 \right] \mid \left| f\left( x \right) \right|>\max_{y\in \left[ 0,X \right]} \left| f\left( y \right) \right| \right\},
\]
则由$f$的连续性和\eqref{eq::2408012982489028018087359023850823907205235236}式知
\[
\max_{y\in \left[ 0,X \right]} \left| f\left( y \right) \right|\leqslant \left| f\left( x_1 \right) \right|<\max_{y\in \left[ 0,x_1 \right]} \left| f\left( y \right) \right|.
\]
于是再由$f$的连续性知,存在$x_2\in \left( X,x_1 \right)$,使得
\[
\max_{y\in \left[ 0,X \right]} \left| f\left( y \right) \right|\leqslant \left| f\left( x_1 \right) \right|<\max_{y\in \left[ 0,x_1 \right]} \left| f\left( y \right) \right|=\left| f\left( x_2 \right) \right|.
\]
这与$x_1$的下确界定义矛盾!故\eqref{eq::2408012982489028018087359023850823907205235237}式成立,即$f$在$\left[ 0,+\infty \right)$上有界。因此再由\eqref{eq::24080129824890280180873590238508239072052352}式可得
\[
\lim_{x\rightarrow +\infty} f\left( x \right) =\lim_{x\rightarrow +\infty} \frac{\int_{x-1}^x{f\left( t \right) \mathrm{d}t}}{x}\leqslant \lim_{x\rightarrow +\infty} \frac{\sup\limits_{x\in \left[ 0,+\infty \right)} \left| \,f\left( x \right) \right|}{x}=0.
\]

\end{proof}


\subsection{双绝对值微分不等式问题}\label{subsection:双绝对值问题}

注意区分齐次微分不等式问题和双绝对值问题.

\begin{example}\label{example:双绝对值经典问题}
对某个 $D > 0$,
\begin{enumerate}
\item 设 $f\in D(\mathbb{R}),f(0)=0$, 使得
\begin{align}\label{equation-----1229}
|f'(x)|\leqslant D|f(x)|,\forall x\in\mathbb{R}.
\end{align}
证明 $f\equiv0$.

\item 设 $f\in C^\infty(\mathbb{R}),f^{(j)}(0)=0,\forall j\in\mathbb{N}_0$, 使得
\begin{align}\label{equation-----1230}
|xf'(x)|\leqslant D|f(x)|,\forall x\in\mathbb{R}.
\end{align}
证明 $f(x)=0,\forall x\geqslant0$.
\end{enumerate}
\end{example}
\begin{note}
双绝对值技巧除了正常解微分方程构造函数外, 还需要对构造函数平方进行处理. 对于第一题, 解微分方程 $y' = Dy,y' = -Dy$ 得构造函数
\begin{align*}
C_1(x)=\frac{y(x)}{e^{Dx}},C_2(x)=y(x)e^{Dx}.
\end{align*}
但我们还要手动平方一下. 第二题是类似的.
\end{note}
\begin{proof}
\begin{enumerate}
\item 构造 $C_1(x)=\frac{f^2(x)}{e^{2Dx}},C_2(x)=f^2(x)e^{2Dx}$, 我们有
\begin{align*}
C_1'(x)=\frac{2f(x)f'(x)-2Df^2(x)}{e^{2Dx}},C_2'(x)=[2f(x)f'(x)+2Df^2(x)]e^{2Dx}.
\end{align*}
由条件\eqref{equation-----1229}, 我们知道
\begin{align*}
\pm f'(x)f(x)\leqslant|f'(x)||f(x)|\leqslant D|f(x)|^2,
\end{align*}
于是 $C_1$ 递减, $C_2$ 递增, 故
\begin{align*}
\frac{f^2(x)}{e^{2Dx}}\leqslant\frac{f^2(0)}{e^{20}}=0,\forall x\geqslant0,f^2(x)e^{2Dx}\leqslant f^2(0)e^{20}=0,\forall x\leqslant0,
\end{align*}
于是就得到了 $f\equiv0$,$\forall x\in \mathbb{R}$.

\item 构造 $C(x)=\frac{f^2(x)}{x^{2D}},x>0$(因为只需证明$f(x)=0,\forall x\geqslant0$,所以我们只考虑一边), 则
\begin{align*}
C'(x)=\frac{2f(x)f'(x)x - 2Df^2(x)}{x^{2D + 1}}.
\end{align*}
由\eqref{equation-----1230}, 我们有
\begin{align*}
xf'(x)f(x)\leqslant x|f'(x)||f(x)|\leqslant D|f(x)|^2,
\end{align*}
即 $C$ 递减. 由 Taylor 公式的 Peano 余项, 我们有 $f(x)=o(x^m),\forall m\in\mathbb{N}\cap (2D,+\infty)$, 于是 
\begin{align*}
C(x)\leqslant \lim_{x\rightarrow 0^+} \frac{f^2(x)}{x^{2D}}=\lim_{x\rightarrow 0^+} \frac{o\left( x^m \right)}{x^{2D}}=0,
\end{align*}
故 $f(x)=0,\forall x\geqslant0$. 
\end{enumerate}

\end{proof}

\begin{example}\label{example:齐次微分不等式}
设 $f\in D^2[0,+\infty)$ 满足 $f(0)=f'(0)=0$ 以及
\begin{align*}
|f''(x)|^2\leqslant|f(x)f'(x)|,\forall x\geqslant0.
\end{align*}
证明 $f(x)=0,\forall x\geqslant0$.
\end{example}
\begin{note}
本题的加强版本见\refpro{proposition:齐次化方法/关于导数乘积不等式问题}.
\end{note}
\begin{proof}
令$M=3$,考虑
\begin{align*}
g(x)=e^{-Mx}\left[|f(x)|^2+|f'(x)|^2\right],x\geqslant0.
\end{align*}
利用 $1 + t^2\geqslant\sqrt{t},\forall t\geqslant0$, 我们有
\begin{align}
1+\frac{|f|^2}{|f'|^2}\geqslant\sqrt{\frac{|f|}{|f'|}}\Rightarrow|f'|^2+|f|^2\geqslant|f|^{\frac{1}{2}}|f'|^{\frac{M}{2}}=|f'|\sqrt{|ff'|}.\label{equation-89574592387}
\end{align}
于是
\begin{align*}
g'(x)&=e^{-Mx}\left[2ff'+2f'f''-Mf^2-M(f')^2\right]\\
&\leqslant e^{-Mx}\left[2|ff'|+2|f'|\sqrt{|ff'|}-Mf^2-M(f')^2\right]\\
&\stackrel{\eqref{equation-89574592387}}{\leqslant}e^{-Mx}\left[2|ff'|+2|f'|^2+2|f|^2-Mf^2-M(f')^2\right]\\
&\stackrel{\text{均值不等式}}{\leqslant}e^{-Mx}\left[|f|^2+|f'|^2+2|f'|^2+2|f|^2-Mf^2-M(f')^2\right]=0.
\end{align*}
于是 $g$ 递减,从而 $0\leqslant g(x)\leqslant g(0)=0$, 故 $f(x)\equiv0$. 

\end{proof}

\begin{example}
设 $f\in D^2(\mathbb{R})$ 满足 $f(0)=f'(0)=0$ 且
\begin{align*}
|f''(x)|\leqslant|f'(x)|+|f(x)|,\forall x\in\mathbb{R}.
\end{align*}
证明:
\begin{align*}
f(x)=0,\forall x\in\mathbb{R}.
\end{align*} 
\end{example}
\begin{note}
本题的加强版本见\refpro{proposition:关于导数求和不等式问题}.
\end{note}
\begin{proof}
令 $g(x)=e^{-Mx}\left[|f(x)|^2+|f'(x)|^2\right]$, 则
\begin{align*}
g'(x)&=e^{-Mx}\left[2ff'+2f'f''-Mf^2-M(f')^2\right]\\
&\leqslant e^{-Mx}\left[f^2+(f')^2+2f'\left(|f|+|f'|\right)-Mf^2-M(f')^2\right]\\
&\leqslant e^{-Mx}\left[f^2+(f')^2+2(f')^2+f^2+(f')^2-Mf^2-M(f')^2\right]\\
&=e^{-Mx}\left[(2 - M)f^2+(4 - M)(f')^2\right].
\end{align*}
取充分大的 $M$, 就有 $g'(x)\leqslant0$. 于是 $g(x)\leqslant g(0)=0,\forall x\geqslant0$.
又注意到 $g(x)=e^{-Mx}\left[|f(x)|^2+|f'(x)|^2\right]\geqslant0$, 因此 $g(x)\equiv0,\forall x\geqslant0$.
故 $f(x)=0,\forall x\geqslant0$. 

\end{proof}

\begin{example}
设 $f\in D^2(\mathbb{R})$ 满足 $f(0)=f'(0)=0$ 且
\begin{align*}
|f''(x)|\leqslant|f'(x)f(x)|,\forall x\in\mathbb{R}.
\end{align*}
证明:
\begin{align*}
f(x)=0,\forall x\geqslant 0.
\end{align*} 
\end{example}
\begin{remark}
与\refexa{example:齐次微分不等式}不同的是,本题的不等式左右两边并不齐次,如果还使用\refexa{example:齐次微分不等式}的方法,那么在放缩过程中会使得系数不含$M$的项的次数大于系数含$M$的项,从而无法直接通过控制$M$的取值,使得$g'(x)\leqslant 0$.因此本题我们需要使用另外的方法.

这里我们将本题与\refexa{example:双绝对值经典问题}类比,采用同样的方法. 因为只需证明$f(x)=0,\forall x\geqslant 0$,所以将原不等式视为(等式)函数构造类.此时需要考虑的微分方程是$f''=ff'$.我们将其中的$f$看作已知函数,考虑的微分方程转化为$y''=fy'$,则
\begin{align*}
y'' =fy' \Rightarrow \frac{y''}{y'}=f\Rightarrow \ln y' =\int_0^x{f\left( t \right) \mathrm{d}t}+C\Rightarrow y' =Ce^{\int_0^x{f\left( t \right) \mathrm{d}t}}.
\end{align*}
于是常数变易,再开平方得到构造函数$C\left( x \right) =\frac{\left[ f'\left( x \right) \right] ^2}{e^{2\int_0^x{|f\left( t \right)| \mathrm{d}t}}}.$
\end{remark}
\begin{proof}
令 $C(x)=\frac{[f'(x)]^2}{e^{2\int_0^x{|f(t)|\mathrm{d}t}}}$, 则
\begin{align*}
C'(x)=\frac{2f'(x)f''(x)-2|f(x)|[f'(x)]^2}{e^{2\int_0^x{|f(t)|\mathrm{d}t}}}.
\end{align*}
又因为
\begin{align*}
f'f''\leqslant|f'f''|\leqslant|f|(f')^2.
\end{align*}
所以 $C'(x)\leqslant0$, 故 $C(x)\leqslant C(0)=0$. 又注意到 $C(x)=\frac{[f'(x)]^2}{e^{2\int_0^x{|f(t)|\mathrm{d}t}}}\geqslant0$, 故 $C(x)\equiv0$. 于是 $f'(x)=0,\forall x\geqslant0$.
进而 $f$ 就是常值函数, 又 $f(0)=0$, 故 $f(x)=0,\forall x\geqslant0$. 

\end{proof}

\begin{proposition}\label{proposition:齐次化方法/关于导数乘积不等式问题}
设 $f \in D^s(0,+\infty) \cap C[0,+\infty)$,$s \in \mathbb{N}$ 且满足
\begin{align*}
f^{(j)}(0) &= 0, j = 0, 1, 2, \cdots, s - 1.
\end{align*}
若还存在 $\lambda_1, \lambda_2, \cdots, \lambda_s \geqslant  0, \sum_{i=1}^{s} \lambda_i = 1, C > 0$,满足
\begin{align}\label{example7.10-0.1}
\left| f^{(s)}(x) \right| &\leqslant  C \left| f(x) \right|^{\lambda_1} \left| f'(x) \right|^{\lambda_2} \cdots \left| f^{(s-1)}(x) \right|^{\lambda_s}, \forall x \geqslant  0.
\end{align}
证明 $f(x) = 0, \forall x \geqslant  0$。
\end{proposition}
\begin{note}
我们把下述证明中左右两边各项次数均相同的不等式:$x_{1}^{2\lambda _1}x_{2}^{2\lambda _2}\cdots x_{n}^{2\lambda _n}\leqslant K\left( x_{1}^{2}+x_{2}^{2}+\cdots +x_{n}^{2} \right) ,\forall x_1,x_2,\cdots ,x_n\geqslant 0$称为\textbf{齐次不等式}.(虽然也可以直接利用幂平均不等式得到,但这里我们旨在介绍如何利用\textbf{齐次化方法}证明\textbf{一般的齐次不等式}.)
\end{note}
\begin{proof}
令 $g\left( x \right) =e^{-Mx}\left[ f^2+\left( f' \right) ^2+\left( f'' \right) ^2+\cdots +\left( f^{\left( s-1 \right)} \right) ^2 \right] ,M>0$,显然 $g\left( x \right) \geqslant 0,\forall x\geqslant 0$。则利用均值不等式和条件 \eqref{example7.10-0.1} 式可得,对 $\forall x\geqslant 0$,都有
\begin{align}
g'\left( x \right) &=e^{-Mx}\left[ 2ff' +2f'f'' +2f''f''' +\cdots +2f^{\left( s-1 \right)}f^{\left( s \right)}-Mf^2-M\left( f' \right) ^2-\cdots -M\left( f^{\left( s-1 \right)} \right) ^2 \right] \nonumber \\
&\overset{\text{均值不等式}}{\leqslant}e^{-Mx}\left[ f^2+\left( f' \right) ^2+\left( f' \right) ^2+\left( f'' \right) ^2+\cdots +\left( f^{\left( s-1 \right)} \right) ^2+\left| f^{\left( s \right)} \right|^2-Mf^2-M\left( f' \right) ^2-\cdots -M\left( f^{\left( s-1 \right)} \right) ^2 \right] \nonumber \\
&\overset{\eqref{example7.10-0.1} \text{式}}{\leqslant}e^{-Mx}\left[ \left( 1-M \right) f^2+\left( 2-M \right) \left( f' \right) ^2+\cdots +\left( 2-M \right) \left( f^{\left( s-1 \right)} \right) ^2+C^2\left| f(x) \right|^{2\lambda _1}\left| f'(x) \right|^{2\lambda _2}\cdots \left| f^{(s-1)}(x) \right|^{2\lambda _s} \right] .\label{example7.10-2.1}
\end{align}
我们先证明 $x_{1}^{2\lambda _1}x_{2}^{2\lambda _2}\cdots x_{n}^{2\lambda _n}\leqslant K\left( x_{1}^{2}+x_{2}^{2}+\cdots +x_{n}^{2} \right) ,\forall x_1,x_2,\cdots ,x_n\geqslant 0$。

令 $S\triangleq \left\{ \left( x_1,x_2,\cdots ,x_n \right) |x_{1}^{2}+x_{2}^{2}+\cdots +x_{n}^{2}=1 \right\}$,则 $S$ 是 $\mathbb{R} ^n$ 上的有界闭集,从而 $S$ 是紧集。于是 $x_{1}^{2\lambda _1}x_{2}^{2\lambda _2}\cdots x_{n}^{2\lambda _n}$ 为紧集 $S$ 上的连续函数,故一定存在 $K>0$,使得
\begin{align}
x_{1}^{2\lambda _1}x_{2}^{2\lambda _2}\cdots x_{n}^{2\lambda _n}\leqslant K,\forall \left( x_1,x_2,\cdots ,x_n \right) \in S.\label{example7.10-1.1}
\end{align}
对 $\forall x_1,x_2,\cdots ,x_n\geqslant 0$,固定 $x_1,x_2,\cdots ,x_n$。不妨设 $x_1,x_2,\cdots ,x_n$ 不全为零,否则结论显然成立。取 $$L=\frac{1}{\sqrt{x_{1}^{2}+x_{2}^{2}+\cdots +x_{n}^{2}}}>0,$$
考虑 $\left( Lx_1,Lx_2,\cdots ,Lx_n \right)$,则此时 $\left( Lx_1 \right) ^2+\left( Lx_2 \right) ^2+\cdots +\left( Lx_n \right) ^2=1$,因此 $\left( Lx_1,Lx_2,\cdots ,Lx_n \right) \in S$。
从而由\eqref{example7.10-1.1}式可知
\begin{align*}
\left( Lx_1 \right) ^{2\lambda _1}\left( Lx_2 \right) ^{2\lambda _2}\cdots \left( Lx_n \right) ^{2\lambda _n}\leqslant K.
\end{align*}
于是
\begin{align*}
x_{1}^{2\lambda _1}x_{2}^{2\lambda _2}\cdots x_{n}^{2\lambda _n}\leqslant \frac{K}{L^{2\lambda _1+2\lambda _2+\cdots +2\lambda _n}}=\frac{K}{L^2}=K\left( x_{1}^{2}+x_{2}^{2}+\cdots +x_{n}^{2} \right) .
\end{align*}
故由 $x_1,x_2,\cdots ,x_n$ 的任意性可得
\begin{align}
x_{1}^{2\lambda _1}x_{2}^{2\lambda _2}\cdots x_{n}^{2\lambda _n}\leqslant K\left( x_{1}^{2}+x_{2}^{2}+\cdots +x_{n}^{2} \right) ,\forall x_1,x_2,\cdots ,x_n\geqslant 0.\label{example7.10-1.2}
\end{align}
因此由\eqref{example7.10-2.1} \eqref{example7.10-1.2} 式可得,对 $\forall x\geqslant 0$,都有
\begin{align*}
g'\left( x \right) &\leqslant e^{-Mx}\left[ \left( 1-M \right) f^2+\left( 2-M \right) \left( f' \right) ^2+\cdots +\left( 2-M \right) \left( f^{\left( s-1 \right)} \right) ^2+C^2\left| f(x) \right|^{2\lambda _1}\left| f' (x) \right|^{2\lambda _2}\cdots \left| f^{(s-1)}(x) \right|^{2\lambda _s} \right] \\
&\leqslant e^{-Mx}\left[ \left( 1-M \right) f^2+\left( 2-M \right) \left( f' \right) ^2+\cdots +\left( 2-M \right) \left( f^{\left( s-1 \right)} \right) ^2+KC^2\left( f^2+\left( f' \right) ^2+\left( f' \right) ^2+\left( f'' \right) ^2+\cdots +\left( f^{\left( s-1 \right)} \right) ^2 \right) \right] \\
&=e^{-Mx}\left[ \left( KC^2+1-M \right) f^2+\left( KC^2+2-M \right) \left( f' \right) ^2+\cdots +\left( KC^2+2-M \right) \left( f^{\left( s-1 \right)} \right) ^2 \right] .
\end{align*}
于是任取 $M>KC^2+2$,利用上式就有 $g'\left( x \right) \leqslant 0,\forall x\geqslant  0$.故 $g\left( x \right)$ 在 $[0,+\infty )$ 上单调递减,因此 $g\left( x \right) \leqslant g\left( 0 \right) =0$。又因为 $g\left( x \right) \geqslant 0,\forall x\geqslant 0$,所以 $g\left( x \right) =0,\forall x\geqslant 0$。故
$f\left( x \right) =f'\left( x \right) =\cdots =f^{\left( s-1 \right)}\left( x \right) =0,\forall x\geqslant 0$。

\end{proof}

\begin{example}
设 $f \in C^n(\mathbb{R}), n \in \mathbb{N}, f^{(k)}(x_0) = 0, k = 0, 1, 2, \cdots, n - 1$. 若对某个 $M > 0$ 和 $\lambda_0, \lambda_1, \cdots, \lambda_{n - 2} \geqslant  0, \lambda_{n - 1} \geqslant  1$ 有不等式
\begin{align*}
|f^{(n)}(x)| \leqslant  M \prod_{k = 0}^{n - 1} |f^{(k)}(x)|^{\lambda_k}, \forall x \in \mathbb{R}.
\end{align*}
证明 $f(x) \equiv 0$.
\end{example}
\begin{note}
因为原不等式是绝对值不等式,所以考虑两个微分方程
\begin{align*}
f^{(n)} &= f^{(n-1)} \cdot g \Rightarrow \frac{f^{(n)}}{f^{(n-1)}} = g \Rightarrow \ln f^{(n-1)} = \int_{x_0}^x g(y) \mathrm{d}y + C \Rightarrow f^{(n-1)} = C e^{\int_{x_0}^x g(y) \mathrm{d}y}.
\end{align*}
\begin{align*}
f^{(n)} &= -f^{(n-1)} \cdot g \Rightarrow \frac{f^{(n)}}{f^{(n-1)}} = -g \Rightarrow \ln f^{(n-1)} = -\int_{x_0}^x g(y) \mathrm{d}y + C \Rightarrow f^{(n-1)} = C e^{-\int_{x_0}^x g(y) \mathrm{d}y}.
\end{align*}
分离常量得到构造函数 $c_1(x) \triangleq \frac{f^{(n-1)}(x)}{e^{\int_{x_0}^x g(y) \mathrm{d}y}}$, $c_2(x) \triangleq f^{(n-1)}(x) e^{\int_{x_0}^x g(y) \mathrm{d}y}$.
回顾\hyperref[subsection:双绝对值问题]{双绝对值问题}的构造函数,我们需要的构造函数应是 $C_1(x) \triangleq c_1^2(x) = \frac{[f^{(n-1)}(x)]^2}{e^{2\int_{x_0}^x g(y) \mathrm{d}y}}$, $C_2(x) \triangleq c_2^2(x) = [f^{(n-1)}(x)]^2 e^{2\int_{x_0}^x g(y) \mathrm{d}y}$.
\end{note}
\begin{proof}
由条件可知
\begin{align*}
|f^{(n)}(x)| \leqslant |f^{(n-1)}(x)| \cdot g(x),
\end{align*}
其中 $g(x) = M \prod_{k=1}^{n-1} |f^{(k)}(x)|^{\lambda_k - 1}$. 从而 $f^{(n)}(x) f^{(n-1)}(x) \leqslant |f^{(n)}(x) f^{(n-1)}(x)| \leqslant |f^{(n-1)}(x)|^2 \cdot g(x)$.
\begin{align}
\label{eq:10001.13131}
\end{align}
令 $C_1(x) \triangleq \frac{[f^{(n-1)}(x)]^2}{e^{2\int_{x_0}^x g(y) \mathrm{d}y}}$, 则由 \eqref{eq:10001.13131} 式可知
\begin{align*}
C_1'(x) = \frac{2f^{(n-1)}(x) f^{(n)}(x) - 2g(x) [f^{(n-1)}(x)]^2}{e^{2\int_{x_0}^x g(y) \mathrm{d}y}} \leqslant 0.
\end{align*}
故 $C_1(x) \leqslant C_1(x_0) = 0, \forall x \geqslant x_0$. 因此 $C_1(x) = 0, \forall x \geqslant x_0$. 进而 $f^{(n-1)}(x) = 0, \forall x \geqslant x_0$.
令 $C_2(x) \triangleq [f^{(n-1)}(x)]^2 e^{2\int_{x_0}^x g(y) \mathrm{d}y}$, 则由 \eqref{eq:10001.13131} 式可知
\begin{align*}
C_2'(x) = \left[ 2f^{(n-1)}(x) f^{(n)}(x) + 2g(x) (f^{(n-1)}(x))^2 \right] e^{2\int_{x_0}^x g(y) \mathrm{d}y} \geqslant 0.
\end{align*}
故 $C_2(x) \leqslant C_2(x_0) = 0, \forall x \leqslant x_0$. 因此 $C_2(x) = 0, \forall x \leqslant x_0$. 进而 $f^{(n-1)}(x) = 0, \forall x \leqslant x_0$.
综上, $f^{(n-1)}(x) \equiv 0, x \in \mathbb{R}$.从而$f^{(n-2)}(x)=K,K\in \mathbb{R}$, 又 $f^{(n-2)}(x_0) = 0$,故$f^{(n-2)}(x) \equiv 0$. 又 $f^{(k)}(x_0) = 0, k = 0, 1, \cdots, n-1$, 依此类推可得 $f(x) \equiv 0$.

\end{proof}

\begin{proposition}\label{proposition:关于导数求和不等式问题}
设 $f \in D^s(0,+\infty) \cap C[0,+\infty)$,$s \in \mathbb{N}$ 且满足
\begin{align*}
f^{(j)}(0) &= 0, j = 0, 1, 2, \cdots, s - 1.
\end{align*}
若还存在 $\lambda_1, \lambda_2, \cdots, \lambda_s \geqslant  0$,满足
\begin{align}\label{aaaaproposition7.19-0.1}
\left| f^{(s)}(x) \right| &\leqslant  \lambda_1 \left| f(x) \right| + \lambda_2 \left| f'(x) \right| + \cdots + \lambda_s \left| f^{(s-1)}(x) \right|, \forall x \geqslant  0.
\end{align}
证明 $f(x) = 0, \forall x \geqslant  0$。
\end{proposition}
\begin{proof}
令 $g\left( x \right) =e^{-Mx}\left[ f^2+\left( f' \right) ^2+\left( f'' \right) ^2+\cdots +\left( f^{\left( s-1 \right)} \right) ^2 \right] ,M>0$,显然 $g\left( x \right) \geqslant 0,\forall x\geqslant 0$。则利用均值不等式和条件\eqref{aaaaproposition7.19-0.1} 式可得,对 $\forall x\geqslant 0$,都有
\begin{align}
g'\left( x \right) &=e^{-Mx}\left[ 2ff' +2f'f'' +2f'' f''' +\cdots +2f^{\left( s-1 \right)}f^{\left( s \right)}-Mf^2-M\left( f' \right) ^2-\cdots -M\left( f^{\left( s-1 \right)} \right) ^2 \right] \nonumber \\
&\overset{\text{均值不等式}}{\leqslant}e^{-Mx}\left[ f^2+\left( f' \right) ^2+\left( f' \right) ^2+\left( f'' \right) ^2+\cdots +\left( f^{\left( s-1 \right)} \right) ^2+\left| f^{\left( s \right)} \right|^2-Mf^2-M\left( f' \right) ^2-\cdots -M\left( f^{\left( s-1 \right)} \right) ^2 \right] \nonumber \\
&\overset{\eqref{aaaaproposition7.19-0.1}\text{式}}{\leqslant}e^{-Mx}\left[ \left( 1-M \right) f^2+\left( 2-M \right) \left( f' \right) ^2+\cdots +\left( 2-M \right) \left( f^{\left( s-1 \right)} \right) ^2+\left( \lambda _1\left| f \right|+\lambda _2\left| f' \right|+\cdots +\lambda _s\left| f^{(s-1)} \right| \right) ^2 \right] .\label{aaaaproposition7.19-2.1}
\end{align}
我们先证明 $\left( \lambda _1x_1+\lambda _2x_2+\cdots +\lambda _sx_s \right) ^2\leqslant K\left( x_{1}^{2}+x_{2}^{2}+\cdots +x_{n}^{2} \right) ,\forall x_1,x_2,\cdots ,x_n\geqslant 0$。

令 $S\triangleq \left\{ \left( x_1,x_2,\cdots ,x_n \right) |x_{1}^{2}+x_{2}^{2}+\cdots +x_{n}^{2}=1 \right\} $,则 $S$ 是 $\mathbb{R} ^n$ 上的有界闭集,从而 $S$ 是紧集。于是 $\left( \lambda _1x_1+\lambda _2x_2+\cdots +\lambda _sx_s \right) ^2$ 为紧集 $S$ 上的连续函数,故一定存在 $K>0$,使得
\begin{align}
x_{1}^{2}+x_{2}^{2}+\cdots +x_{n}^{2}\leqslant K,\forall \left( x_1,x_2,\cdots ,x_n \right) \in S。\label{aaaaproposition7.19-1.1}
\end{align}
对 $\forall x_1,x_2,\cdots ,x_n\geqslant 0$,固定 $x_1,x_2,\cdots ,x_n$。不妨设 $x_1,x_2,\cdots ,x_n$ 不全为零,否则结论显然成立。取 $L=\frac{1}{\sqrt{x_{1}^{2}+x_{2}^{2}+\cdots +x_{n}^{2}}}>0$,考虑 $\left( Lx_1,Lx_2,\cdots ,Lx_n \right) $,则此时 $\left( Lx_1 \right) ^2+\left( Lx_2 \right) ^2+\cdots +\left( Lx_n \right) ^2=1$,因此 $\left( Lx_1,Lx_2,\cdots ,Lx_n \right) \in S$。从而由 \eqref{aaaaproposition7.19-1.1} 式可知
\begin{align*}
\left( \lambda _1Lx_1+\lambda _2Lx_2+\cdots +\lambda _sLx_s \right) ^2\leqslant K.
\end{align*}
于是
\begin{align*}
\left( \lambda _1x_1+\lambda _2x_2+\cdots +\lambda _sx_s \right) ^2\leqslant \frac{K}{L^2}=K\left( x_{1}^{2}+x_{2}^{2}+\cdots +x_{n}^{2} \right) .
\end{align*}
故由 $x_1,x_2,\cdots ,x_n$ 的任意性可得
\begin{align}
\left( \lambda _1x_1+\lambda _2x_2+\cdots +\lambda _sx_s \right) ^2\leqslant K\left( x_{1}^{2}+x_{2}^{2}+\cdots +x_{n}^{2} \right) ,\forall x_1,x_2,\cdots ,x_n\geqslant 0.\label{aaaaproposition7.19-1.2}
\end{align}
因此由 \eqref{aaaaproposition7.19-2.1} \eqref{aaaaproposition7.19-1.2}式可得,对 $\forall x\geqslant 0$,都有
\begin{align*}
g'\left( x \right) &\leqslant e^{-Mx}\left[ \left( 1-M \right) f^2+\left( 2-M \right) \left( f' \right) ^2+\cdots +\left( 2-M \right) \left( f^{\left( s-1 \right)} \right) ^2+\left( \lambda _1\left| f \right|+\lambda _2\left| f' \right|+\cdots +\lambda _s\left| f^{(s-1)} \right| \right) ^2 \right] \\
&\leqslant e^{-Mx}\left[ \left( 1-M \right) f^2+\left( 2-M \right) \left( f' \right) ^2+\cdots +\left( 2-M \right) \left( f^{\left( s-1 \right)} \right) ^2+K\left( f^2+\left( f' \right) ^2+\cdots +\left( f^{\left( s-1 \right)} \right) ^2 \right) \right] \\
&=e^{-Mx}\left[ \left( K+1-M \right) f^2+\left( K+2-M \right) \left( f' \right) ^2+\cdots +\left( K+2-M \right) \left( f^{\left( s-1 \right)} \right) ^2 \right] .
\end{align*}
于是任取 $M>K+2$,利用上式就有 $g'\left( x \right) \leqslant 0,\forall x\geqslant 0$。故 $g\left( x \right) $ 在 $[0,+\infty )$ 上单调递减,因此 $g\left( x \right) \leqslant g\left( 0 \right) =0$。又因为 $g\left( x \right) \geqslant 0,\forall x\geqslant 0$,所以 $g\left( x \right) =0,\forall x\geqslant 0$。故 $f\left( x \right) =f'\left( x \right) =\cdots =f^{\left( s-1 \right)}\left( x \right) =0,\forall x\geqslant 0$.

\end{proof}



\subsection{极值原理}

\begin{example}
设 $f\in C^2[0,1]$ 且 $f(0)=f(1)=0$, 若还有
\begin{align}
f''(x)-g(x)f'(x)=f(x).
\label{equation---12.3445}
\end{align}
证明:
$f(x)=0,\forall x\in[0,1]$.
\end{example}
\begin{proof}
如果 $f$ 在 $(0,1)$ 取得在 $[0,1]$ 上的正的最大值,设最大值点为 $c$ 且 $f(c)>0,f'(c)=0,c\in(0,1)$, 代入\eqref{equation---12.3445}式知 $f''(c)=f(c)>0$. 又由极值的充分条件,我们知道 $c$ 是严格极小值点,这就是一个矛盾!

同样的考虑 $f$ 在 $(0,1)$ 取得在 $[0,1]$ 上的负的最小值,设最小值点为 $c$ 且 $f(c)<0,f'(c)=0,c\in(0,1)$, 代入\eqref{equation---12.3445}式知 $f''(c)=f(c)<0$. 又由极值的充分条件,我们知道 $c$ 是严格极大值点,这就是一个矛盾!

综上,$f$在$(0,1)$上没有正的最大值,也没有负的最小值.即
\begin{align*}
0\leqslant f(x)\leqslant0.
\end{align*}
故$f(x)=0,\forall x\in[0,1]$. 

\end{proof}

\begin{example}
令 $f:[0,1] \to \mathbb{R}$ 为连续的, 满足 $f(0) = f(1) = 0$. 假设 $f''$ 在 $(0,1)$ 内存在, 且具有 $f''+2f'+f \geqslant 0$. 证明对所有 $0 \leqslant x \leqslant 1$, 有 $f(x) \leqslant 0$ 成立.
\end{example}
\begin{proof}
反证,假设$f$存在正的最大值,记
\begin{align*}
f(x_0) = \max_{x \in [0,1]} f(x),
\end{align*}
由$f(0) = f(1) = 0$知$x_0 \in (0,1)$。再记
\begin{align*}
x_1 = \inf \{ x \in (x_0,1] : f(x) = 0 \}.
\end{align*}
由$f \in C[0,1]$知$f(x_1) = 0$。并且
\begin{align*}
f(x) > 0, \forall x \in (x_0, x_1).
\end{align*}
否则,存在$x_2 \in (x_0, x_1)$,使得$f(x_2) = 0$,这与$x_1$的下确界定义矛盾!于是
\begin{align*}
\frac{f(x) - f(x_1)}{x - x_1} = \frac{f(x)}{x - x_1} \leqslant 0, \quad \forall x \in (x_0, x_1).
\end{align*}
令$x \to x_1^-$得
\begin{align}
f'(x_1) = \lim_{x \to x_1^-} \frac{f(x) - f(x_1)}{x - x_1} \leqslant 0. \label{eq:222342rdffg234341.1}
\end{align}
注意到
\begin{align*}
f''(x) + f'(x) \geqslant -(f'(x) + f(x)),
\end{align*}
令$g(x) = f'(x) + f(x)$,则
\begin{align*}
g'(x) + g(x) \geqslant 0.
\end{align*}
再令$C(x) = e^x g(x)$,则
\begin{align*}
C'(x) = e^x [g'(x) + g(x)] \geqslant 0.
\end{align*}
从而$C(x)$递增。由\eqref{eq:222342rdffg234341.1}式知$f'(x_1) \leqslant 0$,故
\begin{align*}
0 < e^{x_0} f(x_0) = C(x_0) \leqslant C(x_1) = e^{x_1} [f'(x_1) + f(x_1)] = e^{x_1} f'(x_1) \leqslant 0
\end{align*}
显然矛盾!

\end{proof}

\begin{example}
设$\alpha > 0$, $f$在$[0,1]$上非负, 有二阶导函数, $f(0) = 0$, 且在$[0,1]$上不恒为零. 求证:存在$\xi \in (0,1)$使得
\begin{align*}
\xi f''(\xi) + (\alpha + 1)f'(\xi) > \alpha f(\xi)
\end{align*}
\end{example}
\begin{note}
要证明的结论是二阶微分不等式无法直接解微分方程构造函数解决,因此考虑局部微分构造.然后通过性态分析来处理.

前面微分不等式的转化可以通过局部解微分方程得到局部构造,然后进行转化.这里的构造和转化方式不唯一,可以随便凑.

证明的关键在于发现矛盾点在$f$的第一个递增区间处.
\end{note}
\begin{proof}
反证,假设
\begin{align*}
xf''(x) + (\alpha + 1)f'(x) \leqslant \alpha f(x), \quad \forall x \in (0,1)
\end{align*}
从而对$\forall x \in (0,1)$,有
\begin{align*}
x^{\alpha + 1}f''(x) + (\alpha + 1)x^{\alpha}f'(x) \leqslant \alpha x^{\alpha}f(x) \Longleftrightarrow \left[ x^{\alpha + 1}f'(x) \right]' \leqslant \alpha x^{\alpha}f(x)
\end{align*}
两边同时积分得,对$\forall x \in (0,1)$,都有
\begin{align}
\int_0^x \left[ t^{\alpha + 1}f'(t) \right]' \mathrm{d}t \leqslant \alpha \int_0^x t^{\alpha}f(t) \mathrm{d}t \Longleftrightarrow x^{\alpha + 1}f'(x) \leqslant \alpha \int_0^x t^{\alpha}f(t) \mathrm{d}t .\label{eq:20rj203f23432r221.1}
\end{align}
由于$f$在$[0,1]$上非负且不恒为0,又$f(0) = 0$,故存在$c \in (0,1]$,使得
\begin{align*}
f(c) = \max_{x \in [0,1]} f(x) > 0,
\end{align*}
令
\begin{align*}
A=\{x\in (0,1]:f' (x)=0\text{且}f(x)>0\}\cup \left\{ c \right\} ,\quad x_1=\mathrm{inf}A.
\end{align*}
由$f$的连续性可知$f(x_1) > 0$。我们断言$f$在$[0,x_1]$上递增。否则,假设存在$0 \leqslant a < b \leqslant x_1$,使得
\begin{align*}
f(a) > f(b) \geqslant 0
\end{align*}
从而$a > 0$。否则,利用上式和$f$非负可得
\begin{align*}
0 \leqslant f(b) < f(0) = 0
\end{align*}
显然矛盾!由$f \in C[a,b]$知,存在$\xi \in [a,b]$,使得
\begin{align*}
f(\xi) = \max_{x \in [a,b]} f(x) \geqslant f(a) > f(b) > 0
\end{align*}
于是$f'(\xi) = 0$且$\xi < b$。从而$\xi \in A$,这与$x_1$的下确界定义矛盾!故$f$在$[0,x_1]$上递增。再结合\eqref{eq:20rj203f23432r221.1}式可得,对$\forall x \in [0,x_1]$,有
\begin{align*}
x^{\alpha + 1}f'(x) \leqslant \alpha \int_0^x t^{\alpha}f(t) \mathrm{d}t \leqslant \alpha \int_0^x x^{\alpha}f(x) \mathrm{d}t = \alpha x^{\alpha + 1}f(x) \Longleftrightarrow f'(x) - \alpha f(x) \leqslant 0
\end{align*}
再令$g(x) = e^{-\alpha x}f(x)$,则
\begin{align*}
g'(x) = e^{-\alpha x} \left[ f'(x) - \alpha f(x) \right] \leqslant 0, \quad \forall x \in [0,x_1]
\end{align*}
故$g$在$[0,x_1]$上递减。从而
\begin{align*}
e^{-x_1}f(x_1) = g(x_1) \leqslant g(0) = 0 \Longrightarrow f(x_1) \leqslant 0
\end{align*}
这与$f(x_1) > 0$矛盾!

\end{proof}












\end{document}
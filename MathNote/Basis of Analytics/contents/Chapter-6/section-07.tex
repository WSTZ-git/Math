\documentclass[../../main.tex]{subfiles}
\graphicspath{{\subfix{../../image/}}} % 指定图片目录,后续可以直接使用图片文件名。

% 例如:
% \begin{figure}[H]
% \centering
% \includegraphics[scale=0.4]{图.png}
% \caption{}
% \label{figure:图}
% \end{figure}
% 注意:上述\label{}一定要放在\caption{}之后,否则引用图片序号会只会显示??.

\begin{document}

\section{更弱定义的导数}

\begin{theorem}[最弱递增条件]\label{theorem:最弱递增条件}
\begin{enumerate}
\item 设 \(f\in C[a,b]\) 满足对任何 \(x_0\in (a,b)\) 都有
\begin{align*}
\varlimsup_{x\rightarrow x_0^+}\frac{f(x)-f(x_0)}{x - x_0}\geqslant slant 0,
\end{align*}
则 \(f\) 在 \([a,b]\) 递增。

\item 设 \(f\in C[a,b]\) 满足对任何 \(x_0\in (a,b)\) 都有
\begin{align}
\varlimsup_{x\rightarrow x_0^+}\frac{f(x)-f(x_0)}{x - x_0}> 0,\label{equation--13.100}
\end{align}
则 \(f\) 在 \([a,b]\) 严格递增。

\item 设 \(f\in C(a,b)\) 满足对任何 \(x\in (a,b)\),都有
\begin{align*}
\varliminf_{h\rightarrow 0^+}\frac{f(x + h)-f(x - h)}{h}> 0.
\end{align*}
证明 \(f\) 在 \((a,b)\) 严格递增。

\item 设 \(f\in C(a,b)\) 满足对任何 \(x\in (a,b)\),都有
\begin{align*}
\varliminf_{h\rightarrow 0^+}\frac{f(x + h)-f(x - h)}{h}\geqslant slant 0.
\end{align*}
证明 \(f\) 在 \((a,b)\) 递增。 
\end{enumerate}
\end{theorem}
\begin{remark}
只需证明 \(f(b) \geqslant  f(a)\)或\(f(b)>f(a)\)的原因:
假设\(f(b) \geqslant  f(a)\)或\(f(b)>f(a)\)已经成立.
任取$c,d\in (a,b)$或$[a,b],$则我们考虑$(c,d)$或$[c,d]$这个区间,并且已知$f$在$(c,d)$或$[c,d]$上连续且满足上述条件,于是由假设可知\(f(d) \geqslant  f(c)\)或\(f(d)>f(c)\).故我们只需证明 \(f(b) \geqslant  f(a)\)或\(f(b)>f(a)\)即可.
\end{remark}
\begin{proof}
\begin{enumerate}
\item 只需证明 \(f(b) \geqslant  f(a)\) . 由$f$的连续性和极限保号性, 我们只需证明对充分小的 \(\varepsilon > 0\) , 有 \(f(b) \geqslant  f(a + \varepsilon)\) . 考虑
\begin{align*}
F(x) = f(x) - f(a + \varepsilon) - \frac{f(b) - f(a + \varepsilon)}{b - a - \varepsilon}(x - a - \varepsilon).
\end{align*}
则
\(F(b) = F(a + \varepsilon) = 0\) , \(\varlimsup_{x \to x_0^+} \frac{F(x) - F(x_0)}{x - x_0} = \varlimsup_{x \to x_0^+} \frac{f(x) - f(x_0)}{x - x_0} - \frac{f(b) - f(a + \varepsilon)}{b - a - \varepsilon}\) , \(\forall x_0 \in [a + \varepsilon, b)\) .
于是设 \(F\) 在 \([a + \varepsilon, b]\) 最大值点为 \(c \) ,

(i)当$c\in [a + \varepsilon, b)$时, 则
\begin{align*}
0 \geqslant  \varlimsup_{x \to c^+} \frac{F(x) - F(c)}{x - c} = \varlimsup_{x \to c^+} \frac{f(x) - f(c)}{x - c} - \frac{f(b) - f(a + \varepsilon)}{b - a - \varepsilon} \geqslant  - \frac{f(b) - f(a + \varepsilon)}{b - a - \varepsilon}
\end{align*}
故 \(f(b) \geqslant  f(a + \varepsilon)\) . 

(ii)当$c=b$时,则对$\forall x\in [a+\varepsilon,b]$,都有$0=F(b)=F(c)\geqslant slant F(x)$.从而
\begin{align*}
&\quad \quad F(x)=f(x)-f(a+\varepsilon )-\frac{f(b)-f(a+\varepsilon )}{b-a-\varepsilon}(x-a-\varepsilon )\leqslant slant 0
\\
&\Rightarrow \frac{f(x)-f(a+\varepsilon )}{x-a-\varepsilon}\leqslant slant \frac{f(b)-f(a+\varepsilon )}{b-a-\varepsilon}
\\
&\Rightarrow \frac{f(b)-f(a+\varepsilon )}{b-a-\varepsilon}\geqslant slant \underset{x\rightarrow \left( a+\varepsilon \right) ^+}{\overline{\lim }}\frac{f(x)-f(a+\varepsilon )}{x-a-\varepsilon}>0
\\
&\Rightarrow f(b)>f(a+\varepsilon )
\end{align*}
证毕.

\item 若 \(f\) 在 \([a, b]\) 不严格增, 则存在 \([c, d] \subset [a, b]\) 使得 \(f(d) = f(c)\) , 注意到由第1问可知 \(f\) 在 \([c, d]\) 递增, 从而只能为常数, 于是$f(x)\equiv f(c)$.不妨设 \([c, d] \subset (a, b)\) , 否则任取 \([a, b]\) 一个子区间即可. 因此
\begin{align*}
\underset{x\rightarrow c^+}{\overline{\lim }}\frac{f(x)-f(c)}{x-c}=0.
\end{align*}
这显然和\eqref{equation--13.100}矛盾! 故我们证明了 \(f\) 在 \([a, b]\) 严格递增.

\item 对 \([c, d] \subset (a, b)\) , 我们断言存在 \(x_1 \in (c, d)\) 使得
\begin{align}
\frac{f(d) - f(c)}{d - c} \geqslant  \varlimsup_{h \to 0^+} \frac{f(x_1 + h) - f(x_1 - h)}{2h} \label{equation--13.101}
\end{align}
现在我们用 \(g(x) = \frac{f(d) - f(c)}{d - c}(x - c) + f(c) - f(x)\) 代替 \(f\) . 于是考虑 \(g \in C^1[c, d]\) , \(g(d) = g(c) = 0\) , 此时要证明\eqref{equation--13.101}, 就只需证明存在 \(x_1 \in (c, d)\) 使得
\begin{align}
\varlimsup_{h \to 0^+} \frac{g(x_1 + h) - g(x_1 - h)}{2h} \geqslant  0 \label{equation--13.102}
\end{align}
若 \(g \equiv 0\) , 已经得到了不等式\eqref{equation--13.102}.

若 \(t \in (a, b)\) 是 \(g\) 的最大值点使得 \(g(t) > 0\) . 取 \(k \in (0, g(t))\) , 则构造非空有界集
\(U = \{x \in [c, t] : g(x) > k\}\) .
记 \(x_1 = \inf U\) , 则存在 \(t_n \in U\) , \(n \in \mathbb{N}\) 使得
\(t_n \geqslant  x_1\) , \(\lim_{n \to \infty} t_n = x_1\) .
注意 \(x_1 \neq c\) , 若 \(g(x_1) > k\) , 则且由函数连续性知 \(x_1\) 左侧仍有 \(g > k\) , 这和 \(x_1\) 是 \(\inf\) 矛盾! 故我们只有 \(x_1 \notin U\) 且 \(g(x_1) = k\) . 注意到
\(\frac{g(x_1 + t_n - x_1) - g(x_1 - (t_n - x_1))}{2(t_n - x_1)} \geqslant  \frac{k - k}{2(t_n - k_1)} = 0\)
这就给出了\eqref{equation--13.102}.

若 \(f\) 有负的最小值 \(g(t) < 0\) . 取 \(k \in (g(c), 0)\) , 构造非空有界集
\(V = \{x \in [t, d] : g(x) < k\}\) .
并取 \(x_1 = \sup V\) , 同样的 \(g(x_1) = k\) 且 \(x_1 \neq d\) . 存在 \(s_n \in V\) 使得 \(\lim_{n \to \infty} s_n = x_1\) . 于是由
\(\frac{g(x_1 + x_1 - s_n) - g(x_1 - (x_1 - s_n))}{2(x_1 - s_n)} \geqslant  \frac{k - k}{2(x_1 - s_n)} = 0\)
知\eqref{equation--13.102}成立.

现在由不等式\eqref{equation--13.101}和题目条件就证明了 \(f(d) > f(c)\) , 从而 \(f\) 严格递增.

\item 注意到 \(f(x) + \varepsilon x\) , \(\varepsilon > 0\) 满足第 3 问要求, 因此
\(f(y) + \varepsilon y > f(x) + \varepsilon x\) , \(\forall b > y > x > a\) , \(\varepsilon > 0\) .
让 \(\varepsilon \to 0^+\) , 我们有 \(f(y) \geqslant  f(x)\) , 这就证明了 \(f\) 在 \((a, b)\) 递增. 
\end{enumerate}
\end{proof}










\end{document}
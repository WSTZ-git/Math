\documentclass[../../main.tex]{subfiles}
\graphicspath{{\subfix{../../image/}}} % 指定图片目录,后续可以直接使用图片文件名。

% 例如:
% \begin{figure}[H]
% \centering
% \includegraphics[scale=0.4]{图.png}
% \caption{}
% \label{figure:图}
% \end{figure}
% 注意:上述\label{}一定要放在\caption{}之后,否则引用图片序号会只会显示??.

\begin{document}

\section{更弱定义的导数}

\begin{theorem}[最弱递增条件]\label{theorem:最弱递增条件}
\begin{enumerate}
\item 设 \(f\in C[a,b]\) 满足对任何 \(x_0\in (a,b)\) 都有
\begin{align*}
\varlimsup_{x\rightarrow x_0^+}\frac{f(x)-f(x_0)}{x - x_0}\geqslant 0,
\end{align*}
则 \(f\) 在 \([a,b]\) 递增。

\item 设 \(f\in C[a,b]\) 满足对任何 \(x_0\in (a,b)\) 都有
\begin{align}
\varlimsup_{x\rightarrow x_0^+}\frac{f(x)-f(x_0)}{x - x_0}> 0,\label{equation--13.100}
\end{align}
则 \(f\) 在 \([a,b]\) 严格递增。

\item 设 \(f\in C(a,b)\) 满足对任何 \(x\in (a,b)\),都有
\begin{align*}
\varliminf_{h\rightarrow 0^+}\frac{f(x + h)-f(x - h)}{h}> 0.
\end{align*}
证明 \(f\) 在 \((a,b)\) 严格递增。

\item 设 \(f\in C(a,b)\) 满足对任何 \(x\in (a,b)\),都有
\begin{align*}
\varliminf_{h\rightarrow 0^+}\frac{f(x + h)-f(x - h)}{h}\geqslant 0.
\end{align*}
证明 \(f\) 在 \((a,b)\) 递增。 
\end{enumerate}
\end{theorem}
\begin{remark}
只需证明 \(f(b) \geqslant  f(a)\)或\(f(b)>f(a)\)的原因:
假设\(f(b) \geqslant  f(a)\)或\(f(b)>f(a)\)已经成立.
任取$c,d\in (a,b)$或$[a,b],$则我们考虑$(c,d)$或$[c,d]$这个区间,并且已知$f$在$(c,d)$或$[c,d]$上连续且满足上述条件,于是由假设可知\(f(d) \geqslant  f(c)\)或\(f(d)>f(c)\).故我们只需证明 \(f(b) \geqslant  f(a)\)或\(f(b)>f(a)\)即可.
\end{remark}
\begin{proof}
\begin{enumerate}
\item 只需证明 \(f(b) \geqslant  f(a)\) . 由$f$的连续性和极限保号性, 我们只需证明对充分小的 \(\varepsilon > 0\) , 有 \(f(b) \geqslant  f(a + \varepsilon)\) . 考虑
\begin{align*}
F(x) = f(x) - f(a + \varepsilon) - \frac{f(b) - f(a + \varepsilon)}{b - a - \varepsilon}(x - a - \varepsilon).
\end{align*}
则
\(F(b) = F(a + \varepsilon) = 0\) , \(\varlimsup_{x \to x_0^+} \frac{F(x) - F(x_0)}{x - x_0} = \varlimsup_{x \to x_0^+} \frac{f(x) - f(x_0)}{x - x_0} - \frac{f(b) - f(a + \varepsilon)}{b - a - \varepsilon}\) , \(\forall x_0 \in [a + \varepsilon, b)\) .
于是设 \(F\) 在 \([a + \varepsilon, b]\) 最大值点为 \(c \) ,

(i)当$c\in [a + \varepsilon, b)$时, 则
\begin{align*}
0 \geqslant  \varlimsup_{x \to c^+} \frac{F(x) - F(c)}{x - c} = \varlimsup_{x \to c^+} \frac{f(x) - f(c)}{x - c} - \frac{f(b) - f(a + \varepsilon)}{b - a - \varepsilon} \geqslant  - \frac{f(b) - f(a + \varepsilon)}{b - a - \varepsilon}
\end{align*}
故 \(f(b) \geqslant  f(a + \varepsilon)\) . 

(ii)当$c=b$时,则对$\forall x\in [a+\varepsilon,b]$,都有$0=F(b)=F(c)\geqslant F(x)$.从而
\begin{align*}
&\quad \quad F(x)=f(x)-f(a+\varepsilon )-\frac{f(b)-f(a+\varepsilon )}{b-a-\varepsilon}(x-a-\varepsilon )\leqslant 0
\\
&\Rightarrow \frac{f(x)-f(a+\varepsilon )}{x-a-\varepsilon}\leqslant \frac{f(b)-f(a+\varepsilon )}{b-a-\varepsilon}
\\
&\Rightarrow \frac{f(b)-f(a+\varepsilon )}{b-a-\varepsilon}\geqslant \underset{x\rightarrow \left( a+\varepsilon \right) ^+}{\overline{\lim }}\frac{f(x)-f(a+\varepsilon )}{x-a-\varepsilon}>0
\\
&\Rightarrow f(b)>f(a+\varepsilon )
\end{align*}
证毕.

\item 若 \(f\) 在 \([a, b]\) 不严格增, 则存在 \([c, d] \subset [a, b]\) 使得 \(f(d) = f(c)\) , 注意到由第1问可知 \(f\) 在 \([c, d]\) 递增, 从而只能为常数, 于是$f(x)\equiv f(c)$.不妨设 \([c, d] \subset (a, b)\) , 否则任取 \([a, b]\) 一个子区间即可. 因此
\begin{align*}
\underset{x\rightarrow c^+}{\overline{\lim }}\frac{f(x)-f(c)}{x-c}=0.
\end{align*}
这显然和\eqref{equation--13.100}矛盾! 故我们证明了 \(f\) 在 \([a, b]\) 严格递增.

\item 对 \([c, d] \subset (a, b)\) , 我们断言存在 \(x_1 \in (c, d)\) 使得
\begin{align}
\frac{f(d) - f(c)}{d - c} \geqslant  \varlimsup_{h \to 0^+} \frac{f(x_1 + h) - f(x_1 - h)}{2h} \label{equation--13.101}
\end{align}
现在我们用 \(g(x) = \frac{f(d) - f(c)}{d - c}(x - c) + f(c) - f(x)\) 代替 \(f\) . 于是考虑 \(g \in C^1[c, d]\) , \(g(d) = g(c) = 0\) , 此时要证明\eqref{equation--13.101}, 就只需证明存在 \(x_1 \in (c, d)\) 使得
\begin{align}
\varlimsup_{h \to 0^+} \frac{g(x_1 + h) - g(x_1 - h)}{2h} \geqslant  0 \label{equation--13.102}
\end{align}
若 \(g \equiv 0\) , 已经得到了不等式\eqref{equation--13.102}.

若 \(t \in (a, b)\) 是 \(g\) 的最大值点使得 \(g(t) > 0\) . 取 \(k \in (0, g(t))\) , 则构造非空有界集
\begin{align*}
U = \{x \in [c, t] : g(x) > k\}.
\end{align*}
记 \(x_1 = \inf U\) , 则存在 \(t_n \in U\) , \(n \in \mathbb{N}\) 使得
\begin{align*}
t_n \geqslant  x_1 , \lim_{n \to \infty} t_n = x_1.
\end{align*}
注意 \(x_1 \neq c\) , 若 \(g(x_1) > k\) , 则且由函数连续性知 \(x_1\) 左侧仍有 \(g > k\) , 这和 \(x_1\) 是 \(\inf\) 矛盾! 故我们只有 \(x_1 \notin U\) 且 \(g(x_1) = k\) . 注意到
\begin{align*}
\frac{g(x_1 + t_n - x_1) - g(x_1 - (t_n - x_1))}{2(t_n - x_1)} \geqslant  \frac{k - k}{2(t_n - k_1)} = 0
\end{align*}
这就给出了\eqref{equation--13.102}.

若 \(f\) 有负的最小值 \(g(t) < 0\) . 取 \(k \in (g(c), 0)\) , 构造非空有界集
\begin{align*}
V = \{x \in [t, d] : g(x) < k\}.
\end{align*}
并取 \(x_1 = \sup V\) , 同样的 \(g(x_1) = k\) 且 \(x_1 \neq d\) . 存在 \(s_n \in V\) 使得 \(\lim_{n \to \infty} s_n = x_1\) . 于是由
\begin{align*}
\frac{g(x_1 + x_1 - s_n) - g(x_1 - (x_1 - s_n))}{2(x_1 - s_n)} \geqslant  \frac{k - k}{2(x_1 - s_n)} = 0
\end{align*}
知\eqref{equation--13.102}成立.

现在由不等式\eqref{equation--13.101}和题目条件就证明了 \(f(d) > f(c)\) , 从而 \(f\) 严格递增.

\item 注意到 \(f(x) + \varepsilon x\) , \(\varepsilon > 0\) 满足第 3 问要求, 因此
\[f(y) + \varepsilon y > f(x) + \varepsilon x ,\forall b > y > x > a , \varepsilon > 0.\]
让 \(\varepsilon \to 0^+\) , 我们有 \(f(y) \geqslant  f(x)\) , 这就证明了 \(f\) 在 \((a, b)\) 递增. 
\end{enumerate}

\end{proof}

\begin{corollary}[右可导函数非负则递增]\label{corollary:右可导函数非负则递增}
设$f$是闭区间$[a,b]$上的连续函数且在开区间$(a,b)$右可导. 若$f'_+(x) \geqslant 0, \forall x \in (a,b)$, 证明$f$在$[a,b]$递增.
\end{corollary}

\begin{theorem}[右导数的Lagrange中值定理]\label{theorem:右导数的Lagrange中值定理}
设$f$在$(a,b)$右可导且在$[a,b]$上连续, 证明存在$x_1,x_2 \in (a,b)$使得
\begin{align}
f'_+(x_2) \geqslant \frac{f(b) - f(a)}{b - a} \geqslant f'_+(x_1). \label{eq:13.2348982334242te3fd938523i23903t3ger34t4108}
\end{align}
\end{theorem}
\begin{note}
类似的, 我们有左导数的版本.
\end{note}
\begin{proof}
不妨设$f(b) = f(a) = 0$, 否则用$f(x) - f(a) - \frac{f(b)-f(a)}{b - a}(x - a)$代替$f$即可.
如果结论不对, 假设$f'_+(x) \geqslant 0$恒成立. 于是由\refcor{corollary:右可导函数非负则递增}我们知道$f$递增, 又$f(a)=f(b)=0$,故$f \equiv 0$, 因此此时仍然有\eqref{eq:13.2348982334242te3fd938523i23903t3ger34t4108}成立, 矛盾! 这就完成了证明.

\end{proof}

\begin{proposition}[右导数连续则原函数可导]\label{proposition:右导数连续则原函数可导}
设$f$在$(a,b)$右可导且$f'_+$在$(a,b)$连续, 证明$f$在$(a,b)$可导且$f'(x) = f'_+(x), \forall x \in (a,b)$.
\end{proposition}
\begin{note}
类似的, 我们有左导数的版本.
\end{note}
\begin{proof}
由\hyperref[theorem:右导数的Lagrange中值定理]{右导数的Lagrange中值定理}, 我们知道对$\forall x_1,x_2\in [a,b]$,都存在$\theta_1,\theta_2$在$x_1,x_2$之间,使得
\begin{align*}
f^S(\theta_2) \geqslant \frac{f(x_2) - f(x_1)}{x_2 - x_1} \geqslant f^S(\theta_1),
\end{align*}
让$x_2 \to x_1$, 由右导数的连续性和夹逼准则即可得
\begin{align*}
f'(x_1) = f^S(x_1).
\end{align*}
这就完成了证明.

\end{proof}

\subsection{Schwarz导数}

\begin{definition}[Schwarz导数]\label{definition:Schwarz导数}
设$f:(a,b) \to \mathbb{R}$, 我们称$f$在$x_0 \in (a,b)$Schwarz可导, 如果存在极限
\begin{align}
f^S(x_0) = \lim_{h \to 0} \frac{f(x_0 + h) - f(x_0 - h)}{2h}. \label{eq:13.2348982334242te3fd938523i23903t3ger34t4234898938523i23903t3ger34t4106}
\end{align}
如果$f$在$(a,b)$处处Schwarz可导, 则称$f$在$(a,b)$Schwarz可导.
\end{definition}
\begin{note}
显然$f$在$x_0$可导则必然在$x_0$Schwarz可导且$f'(x_0) = f^S(x_0)$, 但反之不一定成立.
\end{note}

\begin{theorem}[Schwarz导数的Lagrange中值定理]\label{theorem:Schwarz导数的Lagrange中值定理}
设$f$在$[a,b]$连续且在$(a,b)$Schwarz可导, 证明存在$x_1,x_2 \in (a,b)$, 使得
\begin{align}
f^S(x_1) \geqslant \frac{f(b) - f(a)}{b - a} \geqslant f^S(x_2). \label{eq:13.2348982334242te3fd938523i23903t3ger34t4234898938523i23903t3ger34t4107}
\end{align}
\end{theorem}
\begin{note}
本定理是此类问题的核心定理. 其余结果都是本定理的平凡推论.
\end{note}
\begin{proof}
和证明Lagrange中值定理一样, 我们只需证明Rolle中值定理的情况即可. 即不妨设$f(a) = f(b) = 0$, 否则用$f(x) - f(a) - \frac{f(b)-f(a)}{b - a}(x - a)$代替$f$即可.

若$f \equiv 0$, 则结论是显然的. 若$f$有正的最大值, 则设$c \in (a,b)$是$f$的最大值点使得$f(c) > 0$, 取$k \in (0,f(c))$, 构造非空有界集
\begin{align*}
U = \{x \in [a,c] : f(x) > k\}.
\end{align*}
于是记$x_1 = \inf U$, 就有$t_n \in U$, 使得
\begin{align*}
t_n \geqslant x_1, \lim_{n \to \infty} t_n = x_1.
\end{align*}
注意$x_1 \neq a$且若$f(x_1) > k$, 则且由函数连续性知$x_1$左侧仍有$f > k$, 这和$x_1$是$\inf U$矛盾! 故我们只有$x_1 \notin U$且$f(x_1) = k$.
现在
\begin{align*}
f^S(x_1) = \lim_{h \to 0} \frac{f(x_1 + h) - f(x_1 - h)}{2h} = \lim_{n \to \infty} \frac{f(x_1 + t_n - x_1) - f(x_1 - (t_n - x_1))}{2(t_n - x_1)} \geqslant \lim_{n \to \infty} \frac{k - k}{2(t_n - x_1)} = 0.
\end{align*}
若$f$有负的最小值$f(c) < 0$. 取$k \in (f(c),0)$, 构造非空有界集
\begin{align*}
V = \{x \in [c,b] : f(x) < k\}.
\end{align*}
并取$x_1 = \sup V$, 同样的$f(x_1) = k$且$x_1 \neq b$. 存在$s_n \in V$使得$\lim_{n \to \infty} s_n = x_1$. 于是
\begin{align*}
f^S(x_1) = \lim_{h \to 0} \frac{f(x_1 + h) - f(x_1 - h)}{2h} = \lim_{n \to \infty} \frac{f(x_1 + x_1 - s_n) - f(x_1 - (x_1 - s_n))}{2(x_1 - s_n)} \geqslant \lim_{n \to \infty} \frac{k - k}{2(x_1 - s_n)} = 0.
\end{align*}
考虑$f(a + b - x)$可得$f^S(x_2)\leqslant 0$,这就完成了定理的证明.
\end{proof}

\begin{proposition}
设$f$在$[a,b]$连续且在$(a,b)$Schwarz可导, 若$f^S(x) \geqslant  0, \forall x \in (a,b)$, 则$f$在$[a,b]$递增.
\end{proposition}
\begin{proof}
对$\forall [c,d] \subset [a,b]$, 由\hyperref[theorem:Schwarz导数的Lagrange中值定理]{Schwarz导数的Lagrange中值定理}知存在$\theta \in (c,d)$使得
\begin{align*}
\frac{f(d) - f(c)}{d - c} \geqslant f^S(\theta) \geqslant 0,
\end{align*}
故
\begin{align*}
f(d) \geqslant f(c).
\end{align*}
这就完成了证明.
\end{proof}

\begin{proposition}
若$f$在$[a,b]$连续且在$(a,b)$有连续的Schwarz导数, 则$f$在$(a,b)$可微且
\begin{align*}
f'(x) = f^S(x), \forall x \in (a,b).
\end{align*}
\end{proposition}
\begin{proof}
由\hyperref[theorem:Schwarz导数的Lagrange中值定理]{Schwarz导数的Lagrange中值定理}, 我们知道对$\forall x_1,x_2\in [a,b]$,都存在$\theta_1,\theta_2$在$x_1,x_2$之间,使得
\begin{align*}
f^S(\theta_2) \geqslant \frac{f(x_2) - f(x_1)}{x_2 - x_1} \geqslant f^S(\theta_1),
\end{align*}
让$x_2 \to x_1$, 由Schwarz导数连续性和夹逼准则即可得
\begin{align*}
f'(x_1) = f^S(x_1).
\end{align*}
这就完成了证明.

\end{proof}

\begin{example}
设$f \in C(a,b)$且存在极限:
\begin{align}
f^{[2]}(x) = \lim_{h \to 0^+} \frac{f(x+2h) - 2f(x) + f(x-2h)}{4h^2}. \label{eq:13.109}
\end{align}
\begin{enumerate}
\item 若$f$在$x_0 \in (a,b)$二阶可导, 则$f''(x_0) = f^{[2]}(x_0)$.

\item 若$f^{[2]}(x) < 0, \forall x \in (a,b)$, 证明$f$为$(a,b)$上的严格上凸函数.

\item 若$f^{[2]}(x) < 0, \forall x \in (a,b)$且$f$在$(a,b)$是有下界函数, 证明$\lim_{x \to a^+} f(x)$存在.

\item 若$f^{[2]}(x) = 0, \forall x \in (a,b)$, 则$f(x)$为线性函数.
\end{enumerate}
\end{example}
\begin{proof}
\begin{enumerate}
\item 因为$f$在$x_0 \in (a,b)$二阶可导, 所以$f$在$x_0$邻域一阶可导, 所以
\begin{align*}
&\lim_{h \to 0^+} \frac{f(x_0 + 2h) - 2f(x_0) + f(x_0 - 2h)}{4h^2} \xlongequal{\text{L'Hospital}} \lim_{h \to 0^+} \frac{2f'(x_0 + 2h) - 2f'(x_0 - 2h)}{8h} \\
&= \lim_{h \to 0^+} \frac{2f'(x_0 + 2h) - 2f'(x_0)}{8h} + \lim_{h \to 0^+} \frac{2f'(x_0) - 2f'(x_0 - 2h)}{8h} \\
&= \frac{f''(x_0)}{2} + \frac{f''(x_0)}{2} = f''(x_0).
\end{align*}

\item 对任何$x \in (a,b)$, 存在充分小的$\eta > 0$, 只要$h \in (0,\eta)$, 就有
\begin{align*}
f(x + 2h) - 2f(x) + f(x - 2h) < 0,
\end{align*}
即
\begin{align*}
\frac{f(x + 2h) + f(x - 2h)}{2} < f(x).
\end{align*}
现在对$x \in (a,b), y \in \mathbb{R} \setminus \{0\}, \delta > 0$, 取$0 < h < \min\left\{\delta, \frac{2\delta}{|y|}\right\}$, 就有
\begin{align*}
f(x) > \frac{f(x + hy) + f(x - hy)}{2}.
\end{align*}
由\refthe{theorem:下凸函数的局部定义}知$f$是$(a,b)$上的严格上凸函数.

\item  由\refpro{proposition:开区间的下凸函数有上界的充要条件}和第二问我们知道$\lim_{x \to a^+} f(x)$存在.

\item  \textbf{Method 1} 由第二问我们知道$f$在$(a,b)$即凹又凸. 则由\refpro{proposition:R上的既凸又凹的连续函数是直线}知$f$是线性函数.

\textbf{Method 2} 标准的摄动法, 保持二阶导且不破坏边界条件的最好的振动函数是$-(x - a)(x - b)$.

不妨先一般性, 假设$f \in C[a,b]$, 否则用内闭考虑即可. 
我们用$f(x) - f(a) - \frac{f(b) - f(a)}{b - a}(x - a)$代替$f$, 从而不妨设$f(a) = f(b) = 0$. 若某个$x_0 \in (a,b)$有$f(x_0) > 0$, 考虑
\begin{align*}
f_\varepsilon(x) = f(x) + \varepsilon(x - a)(x - b),
\end{align*}
这里$\varepsilon > 0$, 使得
\begin{align*}
f(x_0) + \varepsilon(x_0 - a)(x_0 - b) > 0.
\end{align*}
不妨设$x_0$是$f_\varepsilon$的最大值点, 现在
\begin{gather*}
f_\varepsilon(a) = f_\varepsilon(b) = 0, f_\varepsilon(x_0) = \max_{x \in [a,b]} f_\varepsilon(x),
\end{gather*}
\begin{align*}
f_{\varepsilon}^{[2]}(x_0)&=\lim_{h\rightarrow 0^+} \frac{f_{\varepsilon}(x_0+2h)-2f_{\varepsilon}(x_0)+f_{\varepsilon}(x_0-2h)}{4h^2}
\\
&=\lim_{h\rightarrow 0^+} \frac{f(x_0+2h)-2f(x_0)+f(x_0-2h)+8\varepsilon h^2}{4h^2}
\\
&=\lim_{h\rightarrow 0^+} \frac{8\varepsilon h^2}{4h^2}=2\varepsilon >0.
\end{align*}
但是
\begin{align*}
f_\varepsilon^{[2]}(x_0) = \lim_{h \to 0^+} \frac{f_\varepsilon(x_0 + 2h) - 2f_\varepsilon(x_0) + f_\varepsilon(x_0 - 2h)}{4h^2} \leqslant 0.
\end{align*}
这就是一个矛盾! 于是我们证明了$f \leqslant 0$. 考虑$-f$可得$f \equiv 0$. 证毕!
\end{enumerate}

\end{proof}

\begin{proposition}[数列内插]\label{proposition:数列内插}
给定实数列$\{x_n\}_{n=1}^\infty$,设
\begin{align*}
\sup_{n \geqslant 0} |x_n| = M, \sup_{n \geqslant 0} |x_{n+2} - 2x_{n+1} + x_n| = K,
\end{align*}
我们断言
\begin{align*}
|x_{n+1} - x_n| \leqslant 2\sqrt{MK}, \forall n \in \mathbb{N}.
\end{align*}
\end{proposition}
\begin{proof}
事实上当$M$或者$K$为0时命题显然成立($K=0$意味着$x_n$是有界的等差数列,必然常数列),因此不妨假设$M,K>0$.注意到对$\forall n \geqslant 2$,我们有
\begin{align*}
M&\geqslant x_n=x_0+\sum_{k=1}^n{\left( x_k-x_{k-1} \right)}=x_0+\left( x_1-x_0 \right) +\sum_{k=2}^n{\left( x_k-x_{k-1} \right)}
\\
&=x_0+\left( x_1-x_0 \right) +\sum_{k=2}^n{\left[ \left( x_1-x_0 \right) +\sum_{j=1}^{k-1}{\left( x_{j+1}-x_j \right) -\left( x_j-x_{j-1} \right)} \right]}
\\
&=x_0+(x_1-x_0)n+\sum_{k=2}^n{\sum_{j=1}^{k-1}{(x_{j+1}}}-2x_j+x_{j-1})
\\
&\geqslant -M+(x_1-x_0)n-\sum_{k=2}^n{\sum_{j=1}^{k-1}{K}}
\\
&=-M+(x_1-x_0)n-\frac{(n-1)n}{2}K,
\end{align*}
因此容易看见
\begin{align*}
x_1 - x_0 \leqslant \frac{2M}{n} + \frac{n-1}{2}K, \forall n \geqslant 1.
\end{align*}
另外一方面对$n \geqslant 2$,我们有
\begin{align*}
-M&\leqslant x_n=x_0+\sum_{k=1}^n{\left( x_k-x_{k-1} \right)}=x_0+\left( x_1-x_0 \right) +\sum_{k=2}^n{\left( x_k-x_{k-1} \right)}
\\
&=x_0+\left( x_1-x_0 \right) +\sum_{k=2}^n{\left[ \left( x_1-x_0 \right) +\sum_{j=1}^{k-1}{\left( x_{j+1}-x_j \right) -\left( x_j-x_{j-1} \right)} \right]}
\\
&=x_0+(x_1-x_0)n+\sum_{k=2}^n{\sum_{j=1}^{k-1}{(x_{j+1}-2x_j+x_{j-1})}}
\\
&\leqslant M+(x_1-x_0)n+\sum_{k=2}^n{\sum_{j=1}^{k-1}{K=M}}+(x_1-x_0)n+\frac{n(n-1)}{2}K,
\end{align*}
因此
\begin{align*}
-\frac{2M}{n} - \frac{n-1}{2}K \leqslant x_1 - x_0, \forall n \geqslant 1.
\end{align*}
所以
\begin{align*}
|x_1 - x_0| \leqslant \frac{2M}{n} + \frac{n-1}{2}K, \forall n \geqslant 1.
\end{align*}
注意到
\begin{align}
\frac{2M}{n} + \frac{n-1}{2}K - 2\sqrt{MK} = \frac{Kn^2 - (4\sqrt{KM} + K)n + 4M}{2n}. \label{eq::928453j3f34tg453234}
\end{align}
记
\begin{align*}
f(x) \triangleq Kx^2 - (4\sqrt{KM} + K)x + 4M,
\end{align*}
则$f(0) = 4M > 0$,$f(x)$的对称轴为$\frac{4\sqrt{MK} + K}{2K} = 2\sqrt{\frac{M}{K}} + \frac{1}{2} > 0$.并且$f(x)$的两个零点之差的绝对值为
\begin{align*}
\frac{2\sqrt{\Delta}}{2K} = \frac{\sqrt{K^2 + 8\sqrt{KM}}}{K} = \sqrt{1 + 8\sqrt{\frac{M}{K}}} > 1.
\end{align*}
故必存在$n \in \mathbb{N}$,使得$f(n) \leqslant 0$.由\eqref{eq::928453j3f34tg453234}式知
\begin{align*}
f(n) \leqslant 0 \Longleftrightarrow \frac{2M}{n} + \frac{n-1}{2}K - 2\sqrt{MK} \leqslant 0 \Longleftrightarrow \frac{2M}{n} + \frac{n-1}{2}K \leqslant 2\sqrt{MK}.
\end{align*}
因此
\begin{align*}
|x_1 - x_0| \leqslant \frac{2M}{n} + \frac{n-1}{2}K \leqslant 2\sqrt{MK}.
\end{align*}
类似可证
\begin{align*}
|x_{n+1} - x_n| \leqslant 2\sqrt{MK}, \forall n \in \mathbb{N}.
\end{align*}

\end{proof}

\begin{example}
若有界函数$f$满足
\begin{align*}
\lim_{h \to 0} \sup_{x \in \mathbb{R}} |f(x + h) - 2f(x) + f(x - h)| = 0,
\end{align*}
证明$f$一致连续.
\end{example}
\begin{proof}
由条件知,对任何$\varepsilon > 0$,存在$\delta > 0$,使得
\begin{align*}
|f(x + h) - 2f(x) + f(x - h)| \leqslant \varepsilon, \forall x \in \mathbb{R}, h \in [-\delta, \delta].
\end{align*}
现在对固定的$h \in [-\delta, \delta]$,考虑$\{f(x + nh)\}_{n=0}^\infty$,我们有
\begin{align*}
\sup_{n \geqslant 0} |f(x + nh)| \leqslant \sup |f|,\quad \sup_{n \geqslant 0} |f(x + (n + 2)h) - 2f(x + (n + 1)h) + f(x + nh)| \leqslant \varepsilon.
\end{align*}
由前面的\hyperref[proposition:数列内插]{数列内插},我们有
\begin{align*}
|f(x + h) - f(x)| \leqslant 2\sqrt{\varepsilon \cdot \sup |f|}, \forall x \in \mathbb{R}, h \in [-\delta, \delta],
\end{align*}
这就证明了$f$在$\mathbb{R}$上一致连续.

\end{proof}

\begin{example}
设 $f \in C(a,b)$ 且
\[
\lim_{h \to 0} \frac{1}{h^3} \left\{ \int_0^h [f(x + u) + f(x - u) - 2f(x)] \, \mathrm{d}u \right\} = 0, \forall x \in (a,b),
\]
证明: $f$ 是线性函数.
\end{example}
\begin{note}
还可以不妨设$a=0,b=1$,否则用$f(a(1-x)+bx)=f((b-a)x+a)$代替$f$即可.这样就可以直接不妨设$f\in C[0,1]$且$f(0)=f(1)=0.$

不妨设$a=0,b=1$的原因:令$g\left( x \right) \triangleq f\left( \left( b-a \right) x+a \right) $,则对$\forall x\in \left( 0,1 \right) $,记$y=\left( b-a \right) x+a\in \left[ a,b \right] $,则
\begin{align*}
&\lim_{h\rightarrow 0} \frac{1}{h^3}\left\{ \int_0^h{[g(x+u)+g(x-u)-2g(x)]\,\mathrm{d}u} \right\} 
\\&=\lim_{h\rightarrow 0} \frac{1}{h^3}\left\{ \int_0^h{[f\left( y+\left( b-a \right) u \right) +f\left( y-\left( b-a \right) u \right) -2f\left( y \right) ]\,\mathrm{d}u} \right\} 
\\&=\lim_{h\rightarrow 0} \frac{1}{\left( b-a \right) h^3}\left\{ \int_0^{\left( b-a \right) h}{[f\left( y+u \right) +f\left( y-u \right) -2f\left( y \right) ]\,\mathrm{d}u} \right\} 
\\&=\left( b-a \right) ^2\lim_{h\rightarrow 0} \frac{1}{\left( b-a \right) ^3h^3}\left\{ \int_0^{\left( b-a \right) h}{[f\left( y+u \right) +f\left( y-u \right) -2f\left( y \right) ]\,\mathrm{d}u} \right\} 
\\&=\left( b-a \right) ^2\lim_{h\rightarrow 0} \frac{1}{h^3}\left\{ \int_0^h{[f\left( y+u \right) +f\left( y-u \right) -2f\left( y \right) ]\,\mathrm{d}u} \right\} 
=0.
\end{align*}
因此$g$仍然满足题目条件.若已证$g\equiv 0$,就有
\[
f\left( \left( b-a \right) x+a \right) =0,\forall x\in \left[ 0,1 \right] \Longleftrightarrow f\left( x \right) =0,\forall x\in \left[ a,b \right] .
\]
故可以不妨设$a=0,b=1.$
\end{note}
\begin{proof}
不妨设 $f \in C[a,b]$, 否则内闭的考虑或修改$f$在端点的值即可. 用 $f(x) - \frac{f(b) - f(a)}{b - a}(x - a) - f(a)$ 代替 $f$ 可以不妨设 $f(a) = f(b) = 0$.此时只需证$f\equiv 0$即可.

若
\[
f(x_0) = \max_{x \in [a,b]} f(x) > 0,
\]
则 $x_0 \in (a,b)$. 取 $\varepsilon > 0$ 使得
\[
f(x_0) + \varepsilon (x_0 - a)(x_0 - b) > 0.
\]
考虑
\[
f_\varepsilon(x) \triangleq f(x) + \varepsilon (x - a)(x - b),
\]
则存在$x_1 \in (a,b)$ ,使得
\[
f_\varepsilon(x_1) = \max_{x \in [a,b]} f_\varepsilon(x) \geqslant f_{\varepsilon}(x_0)> 0.
\]
现在
\begin{align*}
0&=\lim_{h\rightarrow 0} \frac{1}{h^3}\left\{ \int_0^h{[f_{\varepsilon}(x_1)+f_{\varepsilon}(x_1)-2f_{\varepsilon}(x_1)]\,\mathrm{d}u} \right\} 
\\
&\geqslant \lim_{h\rightarrow 0} \frac{1}{h^3}\left\{ \int_0^h{[f_{\varepsilon}(x_1+u)+f_{\varepsilon}(x_1-u)-2f_{\varepsilon}(x_1)]\,\mathrm{d}u} \right\} 
\\
&=\lim_{h\rightarrow 0} \frac{\int_0^h{2\varepsilon u^2\mathrm{d}u}}{h^3}=\lim_{h\rightarrow 0} \frac{\frac{2}{3}\varepsilon h^3}{h^3}=\frac{2}{3}\varepsilon >0,
\end{align*}
这就是一个矛盾! 因此我们有
\[
\max_{x\in [a,b]} f(x)\leqslant 0\Longrightarrow f\left( x \right) \leqslant 0,\forall x\in \left[ a,b \right] .
\]
考虑 $-f$, 令$-f_{\varepsilon}\left( x \right) =-f\left( x \right) -\varepsilon \left( x-a \right) \left( x-b \right) $,同理可得
\[
\max_{x\in [a,b]} (-f(x))\leqslant 0\Longrightarrow -f\left( x \right) \leqslant 0,\forall x\in \left[ a,b \right] \Longrightarrow f\left( x \right) \geqslant 0,\forall x\in \left[ a,b \right] .
\]
现在就有
\[
f(x) = 0, \forall x \in [a,b],
\]
即所求函数 $f$ 为线性函数.

\end{proof}















\end{document}
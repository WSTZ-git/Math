\documentclass[../../main.tex]{subfiles}
\graphicspath{{\subfix{../../image/}}} % 指定图片目录,后续可以直接使用图片文件名。

% 例如:
% \begin{figure}[H]
% \centering
% \includegraphics[scale=0.4]{图.png}
% \caption{}
% \label{figure:图}
% \end{figure}
% 注意:上述\label{}一定要放在\caption{}之后,否则引用图片序号会只会显示??.

\begin{document}

\section{逼近方法}

\subsection{Bernstein多项式}

Bernstein 多项式都能定义在 $[a,b]$ 上,因为只差一个换元法,因此我们不妨设 $[a,b]=[0,1]$。

\begin{definition}[一维 Bernstein 多项式]\label{definition:一维Bernstein多项式}
设 $f \in C[0,1], n \in \mathbb{N}_0$,定义 $f$ 的 \textbf{Bernstein多项式}为
\begin{align*}
B_n(f,x) \triangleq \sum_{k = 0}^{n} f\left(\frac{k}{n}\right) C_{n}^{k} x^{k} (1 - x)^{n - k}.
\end{align*} 

设 $f \in C[a,b], n \in \mathbb{N}_0$,定义 $f$ 的 \textbf{Bernstein多项式}为
\begin{align*}
B_n(f,x)\triangleq \sum_{k=0}^n{f\left( \left( b-a \right) \frac{k}{n}+a \right) C_{n}^{k}\frac{\left( x-a \right) ^k\left( b-x \right) ^{n-k}}{\left( b-a \right) ^n}}.
\end{align*} 
\end{definition}
\begin{remark}
$[a,b]$区间上的Bernstein多项式可由$[0,1]$区间上的Bernstein多项式换元得到.

设 $f \in C[a,b], n \in \mathbb{N}_0$,
令$x=(b-a)y+a$,$\forall x\in [a,b],$则$y\in [0,1]$,并且
\begin{align*}
y=\frac{x-a}{b-a},1-y=\frac{b-x}{b-a}.
\end{align*}
再令$g(y)=f((b-a)y+a)$,则由$f\in C[a,b]$可知$g\in C[0,1]$.于是$g$在$[0,1]$区间上的Bernstein多项式为
\begin{align*}
B_n(g,y)\triangleq \sum_{k=0}^n{g\left( \frac{k}{n} \right) C_{n}^{k}y^k(1}-y)^{n-k}.
\end{align*}
故$[a,b]$区间上$f$的Bernstein多项式可定义为
\begin{align*}
B_n(f,x) \triangleq \sum_{k=0}^n{f\left( \left( b-a \right) \frac{k}{n}+a \right) C_{n}^{k}\frac{\left( x-a \right) ^k\left( b-x \right) ^{n-k}}{\left( b-a \right) ^n}}.
\end{align*}
\end{remark}

\begin{lemma}\label{lemma:一些常见函数的Bernstein多项式}
\begin{enumerate}
\item $B_n(C,x)=C\sum_{k=0}^n{\mathrm{C}_{n}^{k}x^k\left( 1-x \right) ^{n-k}}=C.$

\item $B_n(x,x)=\sum_{k=0}^n{\frac{k}{n}\mathrm{C}_{n}^{k}x^k\left( 1-x \right) ^{n-k}}=x.$

\item $\sum_{k=0}^n{\left( \frac{k}{n}-x \right) \mathrm{C}_{n}^{k}x^k\left( 1-x \right) ^{n-k}}=0.$

\item $\sum_{k=0}^n{\left( \frac{k}{n}-x \right) ^2C_{n}^{k}x^k\left( 1-x \right) ^{n-k}}=\frac{x(1-x)}{n}$.
\end{enumerate}
\end{lemma}
\begin{proof}
\begin{enumerate}
\item 由二项式定理可得$B_n(C,x)=C\sum_{k=0}^n{\mathrm{C}_{n}^{k}x^k\left( 1-x \right) ^{n-k}}=C\left( x+1-x \right) ^n=C.$

\item $B_n(x,x)=\sum_{k=0}^n{\frac{k}{n}\mathrm{C}_{n}^{k}x^k\left( 1-x \right) ^{n-k}}=\frac{\left( 1-x \right) ^n}{n}\sum_{k=0}^n{k\mathrm{C}_{n}^{k}\left( \frac{x}{1-x} \right) ^k}$.
由幂级数可逐项求导得
\begin{align*}
\sum_{k=0}^n{k\mathrm{C}_{n}^{k}y^{k-1}}&=\left( \sum_{k=0}^n{\mathrm{C}_{n}^{k}y^k} \right) ' =\left[ \left( 1+y \right) ^n \right] ' =n\left( 1+y \right) ^{n-1}.
\end{align*}
因此
\begin{align*}
\sum_{k=0}^n{k\mathrm{C}_{n}^{k}y^k}=ny\left( 1+y \right) ^{n-1}.
\end{align*}
故
\begin{align*}
B_n(x,x)&=\frac{\left( 1-x \right) ^n}{n}\sum_{k=0}^n{k\mathrm{C}_{n}^{k}\left( \frac{x}{1-x} \right) ^k}=\frac{\left( 1-x \right) ^n}{n}\frac{nx}{1-x}\left( 1+\frac{x}{1-x} \right) ^{n-1}\\
&=\frac{\left( 1-x \right) ^n}{n}\frac{nx}{1-x}\left( \frac{1}{1-x} \right) ^{n-1}=x.
\end{align*}

\item 由2的结论可直接得到.

\item 首先,展开得到
\begin{align}
&\sum_{k=0}^n{\left( \frac{k}{n}-x \right) ^2\mathrm{C}_{n}^{k}x^k\left( 1-x \right) ^{n-k}}=\left( 1-x \right) ^n\sum_{k=0}^n{\left( x^2-\frac{2xk}{n}+\frac{k^2}{n^2} \right) \mathrm{C}_{n}^{k}\left( \frac{x}{1-x} \right)}
\nonumber
\\
&=x^2\sum_{k=0}^n{\mathrm{C}_{n}^{k}x^k\left( 1-x \right) ^{n-k}}-\frac{2x\left( 1-x \right) ^n}{n}\sum_{k=0}^n{k\mathrm{C}_{n}^{k}\left( \frac{x}{1-x} \right) ^k}+\frac{\left( 1-x \right) ^n}{n^2}\sum_{k=0}^n{k^2\mathrm{C}_{n}^{k}\left( \frac{x}{1-x} \right) ^k}
\nonumber
\\
&=x^2-\frac{2x\left( 1-x \right) ^n}{n}\sum_{k=0}^n{k\mathrm{C}_{n}^{k}\left( \frac{x}{1-x} \right) ^k}+\frac{\left( 1-x \right) ^n}{n^2}\sum_{k=0}^n{k^2\mathrm{C}_{n}^{k}\left( \frac{x}{1-x} \right) ^k}.\label{equation-2388923747}
\end{align}
接着,计算$\sum_{k=0}^n{k\mathrm{C}_{n}^{k}y^k}$和$\sum_{k=0}^n{k^2\mathrm{C}_{n}^{k}y^k}$.由幂级数可逐项求导得
\begin{gather*}
\sum_{k=0}^n{k\mathrm{C}_{n}^{k}y^{k-1}}=\left( \sum_{k=0}^n{\mathrm{C}_{n}^{k}y^k} \right) ' =\left[ \left( 1+y \right) ^n \right] ' =n\left( 1+y \right) ^{n-1}.
\end{gather*}
\begin{align*}
\sum_{k=0}^n{k^2\mathrm{C}_{n}^{k}y^k}&=y\left[ \left( \sum_{k=0}^n{k\mathrm{C}_{n}^{k}y^k} \right) ' \right] =y\left[ \left( y\left( \sum_{k=0}^n{\mathrm{C}_{n}^{k}y^k} \right) ' \right) ' \right] 
\\
&=y\left[ \left( y\left( \left( 1+y \right) ^n \right) ' \right) ' \right] =y\left[ \left( ny\left( 1+y \right) ^{n-1} \right) ' \right] 
\\
&=y\left[ n\left( 1+y \right) ^{n-1}+n\left( n-1 \right) y\left( 1+y \right) ^{n-2} \right] 
\\
&=ny\left( 1+y \right) ^{n-2}\left[ \left( 1+y \right) +\left( n-1 \right) y \right] 
\\
&=ny\left( 1+y \right) ^{n-2}\left( ny+1 \right) .
\end{align*}
令$y=\frac{x}{1-x}$,则由上式可得
\begin{gather*}
\sum_{k=0}^n{k\mathrm{C}_{n}^{k}\left( \frac{x}{1-x} \right) ^k}=n\left( \frac{x}{1-x} \right) \left( 1+\frac{x}{1-x} \right) ^{n-1}=\frac{nx}{1-x}\left( \frac{1}{1-x} \right) ^{n-1}=\frac{nx}{\left( 1-x \right) ^n}.
\\
\sum_{k=0}^n{k^2\mathrm{C}_{n}^{k}\left( \frac{x}{1-x} \right) ^k}=n\left( \frac{x}{1-x} \right) \left( 1+\frac{x}{1-x} \right) ^{n-2}\left( \frac{nx}{1-x}+1 \right) =\frac{nx}{1-x}\left( \frac{1}{1-x} \right) ^{n-2}\frac{\left( n-1 \right) x+1}{1-x}=\frac{nx\left[ \left( n-1 \right) x+1 \right]}{\left( 1-x \right) ^n}.
\end{gather*}
将上式代入\eqref{equation-2388923747}可得
\begin{align*}
\sum_{k=0}^n{\left( \frac{k}{n}-x \right) ^2C_{n}^{k}x^k\left( 1-x \right) ^{n-k}}&=x^2-\frac{2x\left( 1-x \right) ^n}{n}\sum_{k=0}^n{k\mathrm{C}_{n}^{k}\left( \frac{x}{1-x} \right) ^k}+\frac{\left( 1-x \right) ^n}{n^2}\sum_{k=0}^n{k^2\mathrm{C}_{n}^{k}\left( \frac{x}{1-x} \right) ^k}
\\
&=x^2-\frac{2x\left( 1-x \right) ^n}{n}\cdot \frac{nx}{\left( 1-x \right) ^n}+\frac{\left( 1-x \right) ^n}{n^2}\cdot \frac{nx\left[ \left( n-1 \right) x+1 \right]}{\left( 1-x \right) ^n}
\\
&=x^2-2x^2+\frac{\left( n-1 \right) x^2+x}{n}=\frac{x\left( 1-x \right)}{n}.
\end{align*}
\end{enumerate}
\end{proof}


\begin{theorem}[Bernstein多项式的性质]\label{theorem:Bernstein多项式的性质}
\begin{enumerate}[(1)]
\item 设
\(\varphi(x)=n\left[f\left(\frac{n - 1}{n}x+\frac{1}{n}\right)-f\left(\frac{n - 1}{n}x\right)\right]\), \(n = 1,2,3,\cdots\),
则有
\begin{align}\label{equation--Bernstein多项式性质13.115}
B_n^\prime(f,x)=B_{n - 1}(\varphi,x),n\in\mathbb{N}.
\end{align}

\item 若\(f\)递增或者递减,则\(B_n(f,x)\),\(n\in\mathbb{N}_0\)也递增或者递减.

\item 若\(f\)是\([0,1]\)的凸或者凹函数,则对每个\(n\in\mathbb{N}_0\)都有\(B_n(f,x)\)是\([0,1]\)的凸或者凹函数.

\item 设\(f\in C^k[0,1]\),\(k\in\mathbb{N}_0\),则关于\(x\in[0,1]\),一致的有
\begin{align}\label{equation--Bernstein多项式性质13.116}
\lim_{n\rightarrow\infty}B_n(f,x)=f(x),\lim_{n\rightarrow\infty}B_n^\prime(f,x)=f^\prime(x),\cdots,\lim_{n\rightarrow\infty}B_n^{(k)}(f,x)=f^{(k)}(x).
\end{align}
\end{enumerate} 
\end{theorem}
\begin{remark}
\textbf{性质(4)对任意光滑的情况并不成立!}

即当$f\in C^{\infty}[0,1]$时,对 $\forall k\in \mathbb{N}$, 都有 $\lim_{n\rightarrow \infty}B_{n}^{( k )}( f,x ) =f^{( k )}( x )$ \textbf{不成立!}

也即当$f\in C^{\infty}[0,1]$时,对 $\forall k\in \mathbb{N}$, 存在一个 $N$(与 $k$ 无关,公共的$N$),使得 $\forall n>N$, 都有 $B_{n}^{( k )}( f,x ) \rightrightarrows f^{( k )}( x )$ \textbf{不成立!}
\end{remark}
\begin{proof}
\begin{enumerate}[(1)]
\item 对\(n\geqslant1\),直接计算得
\begin{align*}
B_n^\prime(f,x)&=\left[\sum_{k = 0}^n f\left(\frac{k}{n}\right)C_n^k x^k(1 - x)^{n - k}\right]^\prime\\
&=\sum_{k = 1}^n kf\left(\frac{k}{n}\right)C_n^k x^{k - 1}(1 - x)^{n - k}-\sum_{k = 0}^{n - 1}(n - k)f\left(\frac{k}{n}\right)C_n^k x^k(1 - x)^{n - k - 1}\\
&=\sum_{k = 0}^{n - 1}(k + 1)f\left(\frac{k + 1}{n}\right)C_n^{k + 1}x^k(1 - x)^{n - k - 1}-\sum_{k = 0}^{n - 1}(n - k)f\left(\frac{k}{n}\right)C_n^k x^k(1 - x)^{n - k - 1}\\
&=\sum_{k=0}^{n-1}{\left[ \left( k+1 \right) f\left( \frac{k+1}{n} \right) C_{n}^{k+1}-\left( n-k \right) f\left( \frac{k}{n} \right) C_{n}^{k} \right] x^k\left( 1-x \right) ^{n-k-1}}
\\
&=\sum_{k=0}^{n-1}{\left[ \left( k+1 \right) f\left( \frac{k+1}{n} \right) \frac{n!}{\left( k+1 \right) !\left( n-k-1 \right) !}-\left( n-k \right) f\left( \frac{k}{n} \right) \frac{n!}{k!\left( n-k \right) !} \right] x^k\left( 1-x \right) ^{n-k-1}}
\\
&=\sum_{k=0}^{n-1}{\left[ nf\left( \frac{k+1}{n} \right) \frac{\left( n-1 \right) !}{k!\left( n-k-1 \right) !}-nf\left( \frac{k}{n} \right) \frac{\left( n-1 \right) !}{k!\left( n-k-1 \right) !} \right] x^k\left( 1-x \right) ^{n-k-1}}
\\
&=\sum_{k=0}^{n-1}{n\left[ f\left( \frac{k+1}{n} \right) -f\left( \frac{k}{n} \right) \right] \mathrm{C}_{n-1}^{k}x^k\left( 1-x \right) ^{n-k-1}}
\\
&=\sum_{k = 0}^{n - 1}\varphi\left(\frac{k}{n - 1}\right)C_{n - 1}^k x^k(1 - x)^{n - k - 1}\\
&=B_{n - 1}(\varphi,x),
\end{align*}
这就给出了式\eqref{equation--Bernstein多项式性质13.115}. 

\item 如果\(f\)递增,那么就有$\varphi\geqslant 0$,则由\eqref{equation--Bernstein多项式性质13.115}知\(B_n^\prime(f,x)=B_{n - 1}(\varphi,x)\geqslant0\),故\(B_n(f,x)\)递增. 类似可得递减.

\item 如果\(f\)下凸,对\(n = 1\)的情况是否符合条件都可以单独验证,我们略去过程,下设\(n\geqslant2\). 注意继续由\eqref{equation--Bernstein多项式性质13.115}知
\[B_n^{\prime\prime}(f,x)=B_{n - 1}^\prime(\varphi,x)=B_{n - 2}(\psi,x),\psi(x)=(n - 1)\left[\varphi\left(\frac{n - 2}{n-1}x+\frac{1}{n-1}\right)-\varphi\left(\frac{n - 2}{n-1}x\right)\right],\]
从而由$f$的下凸性可得
\begin{align*}
B_{n - 2}(\psi,x)&=\sum_{j=0}^{n-2}{\psi \left( \frac{j}{n-2} \right) \mathrm{C}_{n-2}^{k}(1}-x)^{n-2-j}x^j
\\
&=\sum_{j = 0}^{n - 2}\left[\varphi\left(\frac{j + 1}{n - 1}\right)-\varphi\left(\frac{j}{n - 1}\right)\right]\frac{(n - 1)!}{j!(n - 2 - j)!}(1 - x)^{n - 2 - j}x^j
\\
&=n\sum_{j=0}^{n-2}{\left[ f\left( \frac{j+2}{n} \right) -f\left( \frac{j+1}{n} \right) -f\left( \frac{j+1}{n} \right) +f\left( \frac{j}{n} \right) \right] \frac{(n-1)!}{j!(n-2-j)!}(1}-x)^{n-2-j}x^j
\\
&=2n\sum_{j=0}^{n-2}{\left[ \frac{f\left( \frac{j+2}{n} \right) +f\left( \frac{j}{n} \right)}{2}-f\left( \frac{j+1}{n} \right) \right] \frac{(n-1)!}{j!(n-2-j)!}(1}-x)^{n-2-j}x^j
\\
&=2n\sum_{j = 0}^{n - 2}\left[\frac{f\left(\frac{j + 2}{n}\right)+f\left(\frac{j}{n}\right)}{2}-f\left(\frac{\frac{j + 2}{n}+\frac{j}{n}}{2}\right)\right]\frac{(n - 1)!}{j!(n - 2 - j)!}(1 - x)^{n - 2 - j}x^j
\\
&\geqslant0,
\end{align*}
这就证明了\(B_n(f,x)\)下凸. 类似的可讨论上凸情况.

\item $\mathbf{Step}\mathbf{1}$ 我们证明\(k = 0\)时命题成立. 对任何\(\varepsilon>0\),存在\(\delta>0\),使得只要\(\vert x - y\vert\leqslant\delta\),就有
\[
\vert f(x)-f(y)\vert\leqslant\varepsilon.
\]
注意到
\begin{align*}
\vert B_n(f,x)-f(x)\vert&=\left|\sum_{k = 0}^n\left[f\left(\frac{k}{n}\right)-f(x)\right]C_n^k x^k(1 - x)^{n - k}\right|\\
&\leqslant\sum_{\vert\frac{k}{n}-x\vert\leqslant\delta}\left|\left[f\left(\frac{k}{n}\right)-f(x)\right]C_n^k x^k(1 - x)^{n - k}\right|+\sum_{\vert\frac{k}{n}-x\vert>\delta}\left|\left[f\left(\frac{k}{n}\right)-f(x)\right]C_n^k x^k(1 - x)^{n - k}\right|\\
&\leqslant\varepsilon\sum_{\vert\frac{k}{n}-x\vert\leqslant\delta}\left|C_n^k x^k(1 - x)^{n - k}\right|+2\sup\vert f\vert\sum_{\vert\frac{k}{n}-x\vert>\delta}\left|C_n^k x^k(1 - x)^{n - k}\right|\\
&\overset{\text{类似}Chebeshev\text{不等式的证明}}{\leqslant}\varepsilon\sum_{k = 0}^n C_n^k x^k(1 - x)^{n - k}+\frac{2\sup\vert f\vert}{\delta^2}\sum_{\vert\frac{k}{n}-x\vert>\delta}\left(\frac{k}{n}-x\right)^2C_n^k x^k(1 - x)^{n - k}\\
&\leqslant\varepsilon+\frac{2\sup\vert f\vert}{\delta^2}\sum_{k = 0}^n\left(\frac{k}{n}-x\right)^2C_n^k x^k(1 - x)^{n - k}\\
&\overset{\reflem{lemma:一些常见函数的Bernstein多项式}}{=}\varepsilon+\frac{2\sup\vert f\vert}{n\delta^2}x(1 - x),
\end{align*}
于是从上式立得
\[
\sup_{x\in[0,1]}\vert B_n(f,x)-f(x)\vert\leqslant \varepsilon+\frac{\sup\vert f\vert}{2n\delta^2}.
\]
令$n\to \infty$,得
\begin{align*}
\underset{n\rightarrow \infty}{\overline{\lim }}\mathop {\mathrm{sup}} \limits_{x\in [0,1]}\left| B_n(f,x)-f(x) \right|\leqslant \varepsilon .
\end{align*}
再由$\varepsilon$的任意性可知
\begin{align*}
\underset{n\rightarrow \infty}{\lim}\mathop {\mathrm{sup}} \limits_{x\in [0,1]}\left| B_n(f,x)-f(x) \right|=0.
\end{align*}
故我们得到了\(k = 0\)时,式\eqref{equation--Bernstein多项式性质13.116}成立.

$\mathbf{Step}\mathbf{2}$ \((*)\) 我们定义
\[
T_nf(x)=n\left[f\left(\frac{n - 1}{n}x+\frac{1}{n}\right)-f\left(\frac{n - 1}{n}x\right)\right],n\in\mathbb{N}.
\]
\[
B_n^\prime(f,x)=B_{n - 1}(T_nf,x),\forall n\in\mathbb{N}.
\]
归纳证明
\[
T_{n - j + 1}\cdots T_nf(x)=(n - j + 1)\cdots(n - 1)n\sum_{k = 0}^j(-1)^{k + j}C_j^kf\left(\frac{n - j}{n}x+\frac{k}{n}\right),\forall j\in\mathbb{N}.
\]
事实上,当\(j = 1\),由\eqref{equation--Bernstein多项式性质13.115}可知命题显然成立,假设命题对\(j\in\mathbb{N}\)成立,则
\begin{align*}
&T_{n - j}\cdots T_nf(x)\\
=&T_{n - j}\left((n - j + 1)\cdots(n - 1)n\sum_{k = 0}^j(-1)^{k + j}C_j^kf\left(\frac{n - j}{n}x+\frac{k}{n}\right)\right)\\
=&\frac{n!}{(n - j)!}\sum_{k = 0}^j(-1)^{k + j}C_j^kT_{n - j}\left(f\left(\frac{n - j}{n}x+\frac{k}{n}\right)\right)\\
=&\frac{n!(n - j)}{(n - j)!}\sum_{k = 0}^j(-1)^{k + j}C_j^k\left[f\left(\frac{n - j - 1}{n}x+\frac{k + 1}{n}\right)-f\left(\frac{n - j - 1}{n}x+\frac{k}{n}\right)\right]\\
=&\frac{n!(-1)^j\left[f\left(\frac{n - j - 1}{n}x+\frac{1}{n}\right)-f\left(\frac{n - j - 1}{n}x\right)\right]}{(n - j - 1)!}\\
&+\frac{n!}{(n - j - 1)!}\sum_{k = 1}^j(-1)^{k + j}C_j^kf\left(\frac{n - j - 1}{n}x+\frac{k + 1}{n}\right)\\
&-\frac{n!}{(n - j - 1)!}\sum_{k = 1}^j(-1)^{k - 1 + j}C_j^{k - 1}f\left(\frac{n - j - 1}{n}x+\frac{k}{n}\right)\\
&+\frac{n!}{(n - j - 1)!}\sum_{k = 1}^j(-1)^{k - 1 + j}C_j^{k - 1}f\left(\frac{n - j - 1}{n}x+\frac{k}{n}\right)\\
&-\frac{n!}{(n - j - 1)!}\sum_{k = 1}^j(-1)^{k + j}C_j^kf\left(\frac{n - j - 1}{n}x+\frac{k}{n}\right)\\
=&\frac{n!(-1)^j\left[f\left(\frac{n - j - 1}{n}x+\frac{1}{n}\right)-f\left(\frac{n - j - 1}{n}x\right)\right]}{(n - j - 1)!}\\
&+\frac{n!}{(n - j - 1)!}f\left(\frac{n - j - 1}{n}x+\frac{j + 1}{n}\right)-\frac{n!}{(n - j - 1)!}(-1)^jf\left(\frac{n - j - 1}{n}x+\frac{1}{n}\right)\\
&+\frac{n!}{(n - j - 1)!}\sum_{k = 1}^{j}(-1)^{k + 1 + j}(C_j^{k - 1}+C_j^k)f\left(\frac{n - j - 1}{n}x+\frac{k}{n}\right)\\
=&\frac{n!}{(n - j - 1)!}\sum_{k = 0}^{j + 1}(-1)^{k + 1 + j}C_{j + 1}^kf\left(\frac{n - j - 1}{n}x+\frac{k}{n}\right).
\end{align*}
因此我们证明了对\(j + 1\),结论也成立,因此由数学归纳法,对所有\(j\in\mathbb{N}\),命题都成立.

$\mathbf{Step}\mathbf{3}$ \((*)\) 我们证明一个中值定理的结果.
由\hyperref[theorem:Hermite插值定理]{Hermite插值定理},对\(x\in[0,1]\),存在\(\theta\in[0,1]\),我们有
\[
f(x)=\sum_{k = 1}^j\prod_{s\neq k}\frac{(x-\frac{s + i}{n})}{(\frac{k + i}{n}-\frac{s + i}{n})}f\left(\frac{k + i}{n}\right)+\frac{f^{(j)}(\theta)}{j!}\prod_{s = 1}^j\left(x-\frac{s + i}{n}\right).
\]
特别的存在\(\theta\in[\frac{i}{n},\frac{i + j}{n}]\),使得
\begin{align*}
f\left(\frac{i}{n}\right)&=\sum_{k = 1}^j\prod_{s\neq k}\frac{(\frac{i}{n}-\frac{s + i}{n})}{(\frac{k + i}{n}-\frac{s + i}{n})}\cdot f\left(\frac{k + i}{n}\right)+\frac{f^{(j)}(\theta)}{j!}\prod_{s = 1}^j\left(\frac{i}{n}-\frac{s + i}{n}\right)\\
&=\sum_{k = 1}^j\prod_{s\neq k}\frac{(-\frac{s}{n})}{(\frac{k - s}{n})}\cdot f\left(\frac{k + i}{n}\right)+\frac{(-1)^jf^{(j)}(\theta)}{n^j}\\
&=\sum_{k = 1}^j\prod_{s\neq k}\frac{s}{s - k}\cdot f\left(\frac{k + i}{n}\right)+\frac{(-1)^jf^{(j)}(\theta)}{n^j}\\
&=\sum_{k = 1}^j\frac{j!}{k(j - k)!(k - 1)!}(-1)^{k - 1}f\left(\frac{k + i}{n}\right)+\frac{(-1)^jf^{(j)}(\theta)}{n^j}\\
&=\sum_{k = 1}^j(-1)^{k - 1}C_j^kf\left(\frac{k + i}{n}\right)+\frac{(-1)^jf^{(j)}(\theta)}{n^j},
\end{align*}
从而
\[
\sum_{k = 0}^j(-1)^kC_j^kf\left(\frac{k + i}{n}\right)=\frac{(-1)^jf^{(j)}(\theta)}{n^j}.
\]

$\mathbf{Step}\mathbf{4}$ \((*)\) 注意到
\[
B_n^{(j)}(f,x)=B_{n - j}(T_{n - j + 1}\cdots T_{n - 1}T_nf,x),1\leqslant j\leqslant k,n>k.
\]
于是运用$\mathbf{Step}\mathbf{3}$,我们有
\begin{align*}
&\vert B_n^{(j)}(f,x)-f^{(j)}(x)\vert
\leqslant\vert B_{n - j}(f^{(j)},x)-f^{(j)}(x)\vert+\vert B_{n - j}(f^{(j)},x)-B_{n - j}(T_{n - j + 1}\cdots T_{n - 1}T_nf,x)\vert\\
&\leqslant\vert B_{n - j}(f^{(j)},x)-f^{(j)}(x)\vert
+\sum_{i = 0}^{n - j}\vert f^{(j)}\left(\frac{i}{n - j}\right)-T_{n - j + 1}\cdots T_{n - 1}T_nf\left(\frac{i}{n - j}\right)\vert C_{n - j}^ix^i(1 - x)^{n - j - i}\\
&=\vert B_{n - j}(f^{(j)},x)-f^{(j)}(x)\vert+\sum_{i = 0}^{n - j}\vert f^{(j)}\left(\frac{i}{n - j}\right)-\frac{n!}{(n - j)!}\sum_{k = 0}^j(-1)^{k + j}C_j^kf\left(\frac{k + i}{n}\right)\vert C_{n - j}^ix^i(1 - x)^{n - j - i}\\
&=\vert B_{n - j}(f^{(j)},x)-f^{(j)}(x)\vert+\sum_{i = 0}^{n - j}\vert f^{(j)}\left(\frac{i}{n - j}\right)-\frac{n!f^{(j)}(\theta)}{(n - j)!n^j}\vert C_{n - j}^ix^i(1 - x)^{n - j - i}\\
&\leqslant\vert B_{n - j}(f^{(j)},x)-f^{(j)}(x)\vert+\sum_{i = 0}^{n - j}\vert f^{(j)}\left(\frac{i}{n - j}\right)-f^{(j)}(\theta)\vert C_{n - j}^ix^i(1 - x)^{n - j - i}\\
&\quad +\sum_{i = 0}^{n - j}\left|1-\frac{n!}{(n - j)!n^j}\right|\cdot\vert f^{(j)}(\theta)\vert C_{n - j}^ix^i(1 - x)^{n - j - i}.
\end{align*}

$\mathbf{Step}\mathbf{5}$ \((*)\) $\mathbf{Step}\mathbf{1}$告诉我们关于\(x\in[0,1]\),一致的有
\[
\lim_{n\rightarrow\infty}B_n\left(f^{(j)},x\right)=f^{(j)}(x),j = 0,1,2,\cdots,k.
\]
同时注意到
\[
\lim_{n\rightarrow\infty}\left(1-\frac{n!}{(n - j)!n^j}\right)=0,
\]
以及
\[
\left|\frac{i}{n}-\frac{i}{n - j}\right|=\frac{ji}{n(n - j)}\leqslant\frac{j}{n},\left|\frac{i + j}{n}-\frac{i}{n - j}\right|\leqslant\frac{2j}{n},\forall n>j.
\]
我们同时假设
\[
M_j\triangleq\sup_{x\in[0,1]}\vert f^{(j)}(x)\vert,j = 0,1,2,\cdots,k.
\]
并注意到\(f^{(j)}\)是一致连续的.
现在对任何\(\varepsilon>0\),存在\(N\in\mathbb{N}\)和\(\delta>0\),使得
\[
\left|B_{n - j}\left(f^{(j)},x\right)-f^{(j)}(x)\right|<\frac{\varepsilon}{3},\forall x\in[0,1],
\]
\[
\left|1-\frac{n!}{(n - j)!n^j}\right|<\frac{\varepsilon}{3M_j},\forall n>N,
\]
\[
\left|f^{(j)}(x)-f^{(j)}(y)\right|<\frac{\varepsilon}{3},\forall\vert x - y\vert<\delta,x,y\in[0,1].
\]
因此当正整数\(n>\max\left\{\frac{2j}{\delta},j,N\right\}\),利用$\mathbf{Step}\mathbf{4}$,我们有
\[
\left|B_n^{(j)}(f,x)-f^{(j)}(x)\right|<\varepsilon,\forall x\in[a,b],
\]
这就完成了证明. 
\end{enumerate}
\end{proof}

\begin{example}
设 $f \in C[0,1]$ 使得 
\begin{align*}
\int_{0}^{1} f(x)x^{n}\mathrm{d}x = 0, \forall n = 0,1,2,\cdots.
\end{align*}
证明
\begin{align*}
f(x) = 0, \forall x \in [0,1].
\end{align*} 
\end{example}
\begin{remark}
$p_n(x)$ 的良定义性可由\hyperref[theorem:Bernstein多项式的性质]{Bernstein 多项式的性质 (4)}得到. 实际上, 我们这里取的 $p_n(x)$ 就是 $g$ 的 Bernstein 多项式 $B_n(g,x)$.
\end{remark}
\begin{proof}
由条件可知, 对任意实系数多项式 $p(x)$, 都有
\begin{align*}
\int_0^1 f(x) p(x) \mathrm{d}x = 0, \forall p(x) \in \mathbb{R}[x].
\end{align*}
对 $\forall g \in C[0,1]$, 取 $p_n(x) \in \mathbb{R}[x]$, 使得 $p_n(x) \rightrightarrows g(x)$, 则
\begin{align*}
\int_0^1 f(x) g(x) \mathrm{d}x = \int_0^1 \lim_{n\rightarrow \infty} f(x) p_n(x) \mathrm{d}x = \lim_{n\rightarrow \infty} \int_0^1 f(x) p_n(x) \mathrm{d}x = 0.
\end{align*}
于是
\begin{align*}
\int_0^1 f(x) g(x) \mathrm{d}x = 0, \forall g \in C[a,b].
\end{align*}
再取 $g = f$, 则由上式可得
\begin{align*}
\int_0^1 f^2(x) \mathrm{d}x = 0 \Rightarrow f(x) \equiv 0.
\end{align*}
\end{proof}

\begin{theorem}\label{theorem:多项式逼近有限阶光滑函数}
设 $f(x) \in C^{k}[a,b]$, 这里 $a < b$, $a,b \in \mathbb{R}$, $k \in \mathbb{N}_0$, 那么对任意 $\varepsilon > 0$, 存在多项式 $p(x)$, 使得
\begin{align*}
\left| f^{(s)}(x) - p^{(s)}(x) \right| \leqslant \varepsilon, \forall x \in [a,b], s = 0,1,2,\cdots,k.
\end{align*}
\end{theorem}
\begin{remark}
$q(x)$ 的良定义性可由 $f^{(k)}$ 的连续性和 \hyperref[theorem:Bernstein多项式的性质]{Bernstein 多项式的性质 (4) }直接得到. 实际上, $q(x) = B(f^{(k)},x)$.
\end{remark}
\begin{proof}
由带积分型余项的 Taylor 公式可知
\begin{align*}
f(x) = \sum_{j = 0}^{k - 1} \frac{f^{(j)}(a)}{j!}(x - a)^j + \frac{1}{(k - 1)!} \int_a^x (x - t)^{k - 1} f^{(k)}(t) dt.
\end{align*}
对 $\forall \varepsilon > 0$, 取 $q \in \mathbb{R}[x]$, 使得
\begin{align}
|q(x) - f^{(k)}(x)| \leqslant \varepsilon, \forall x \in [a,b]. \label{theorem-equation-多项式逼近有限光滑函数1.1}
\end{align}
设
\begin{align*}
p(x) = \sum_{j = 0}^{k - 1} \frac{f^{(j)}(a)}{j!}(x - a)^j + \frac{1}{(k - 1)!} \int_a^x (x - t)^{k - 1} q(t) dt,
\end{align*}
则对 $p$ 求导可得, 对 $\forall s \in \mathbb{N}$, 我们有
\begin{align}
p^{(s)}(x) = \sum_{j = 0}^{k - s - 1} \frac{f^{(j + s)}(a)}{j!}(x - a)^j + \frac{1}{(k - s - 1)!} \int_a^x (x - t)^{k - s - 1} q(t) dt. \label{theorem-equation-多项式逼近有限光滑函数1.2}
\end{align}
由带积分型余项的 Taylor 公式可知, 对 $\forall s \in \mathbb{N}$, 我们有
\begin{align}
f^{(s)}(x) = \sum_{j = 0}^{k - s - 1} \frac{f^{(j + s)}(a)}{j!}(x - a)^j + \frac{1}{(k - s - 1)!} \int_a^x (x - t)^{k - s - 1} f^{(k)}(t) dt.\label{theorem-equation-多项式逼近有限光滑函数1.3}
\end{align}
于是利用\eqref{theorem-equation-多项式逼近有限光滑函数1.1}\eqref{theorem-equation-多项式逼近有限光滑函数1.2}\eqref{theorem-equation-多项式逼近有限光滑函数1.3}式可得, 对 $\forall s \in \mathbb{N}$, 我们有
\begin{align*}
|f^{(s)}(x) - p^{(s)}(x)| = \left| \frac{1}{(k - s - 1)!} \int_a^x (x - t)^{k - s - 1} [f^{(k)}(t) - q(t)] dt \right| \leqslant \frac{\varepsilon (b - a)^{k - s}}{(k - s)!}, \forall x \in [a,b].
\end{align*}
故结论得证.
\end{proof}

\begin{example}
设多项式列 $p_n$, $n = 1,2,\cdots$ 在 $\mathbb{R}$ 一致收敛到 $f$, 证明 $f$ 为多项式. 
\end{example}
\begin{proof}
由条件可知, 存在 $N\in \mathbb{N}$, 使得
\begin{align*}
|p_m(x) - p_n(x)| \leqslant 1, \forall m > n \geqslant N, x\in \mathbb{R}.
\end{align*}
由于有界的多项式函数一定是常值函数, 因此 $p_m(x) - p_n(x) = C, \forall m > n \geqslant N, x\in \mathbb{R}$. 其中 $C$ 是一个常数. 故
\begin{align}
p_n(x) = p_N(x) + c_n, \forall n \geqslant N, x\in \mathbb{R}. \label{equation----:::-1.1}
\end{align}
其中 $\{c_n\}$ 是一个常数列. 从而任取 $x_0\in \mathbb{R}$, 结合 $\lim_{n\rightarrow \infty}p_n(x) = f(x)$ 可得
\begin{align*}
\lim_{n\rightarrow \infty}c_n = \lim_{n\rightarrow \infty}[p_n(x_0) - p_N(x_0)] = f(x_0) - p_N(x_0).
\end{align*}
故 $\lim_{n\rightarrow \infty}c_n = c$ 存在. 于是由 $x_0$ 的任意性可得
\begin{align*}
c = \lim_{n\rightarrow \infty}c_n = f(x) - p_N(x), x\in \mathbb{R}.
\end{align*}
即 $f(x) = p_N(x) + c, \forall x\in \mathbb{R}$. 因此结论得证.
或者对\eqref{equation----:::-1.1}式两边同时令 $n\rightarrow \infty$, 也能得到
\begin{align*}
f(x) = p_N(x) + c, \forall x\in \mathbb{R}.
\end{align*} 
\end{proof}

\subsection{可积函数的逼近}

\begin{theorem}[可积被连续函数逼近]\label{theorem:可积被连续函数逼近}
\begin{enumerate}[(1)]
\item 设 $f \in R[a,b]$, 则对任何 $\varepsilon > 0$, 存在 $g \in C[a,b]$, 使得
\begin{align}
\int_{a}^{b} |f(x) - g(x)|\mathrm{d}x < \varepsilon. \label{theorem:可积被连续函数逼近-13.117}
\end{align}

\item 设 $f \in R[a,b]$, 则对任何 $\varepsilon > 0$, 存在 $g \in C_c(a,b)$, 使得
\begin{align*}
\int_{a}^{b} |f(x) - g(x)|\mathrm{d}x < \varepsilon. 
\end{align*}
这里 $g\in C_c(a,b)$ 表示 $g$是有含于 $(a,b)$ 的紧支撑的连续函数.

\item 设 $p \geqslant 1$ 且反常积分 $\int_{-\infty}^{\infty} |f(x)|^{p}\mathrm{d}x < \infty$, 则对任何 $\varepsilon > 0$, 存在 $g \in C_c(\mathbb{R})$, 使得
\begin{align*}
\int_{-\infty}^{\infty} |f(x) - g(x)|^{p}\mathrm{d}x < \varepsilon.
\end{align*}
这里 $g\in C_c(a,b)$ 表示 $g$ 是有含于 $(a,b)$ 的紧支撑的连续函数.
\end{enumerate}
\end{theorem}
\begin{note}
证明的想法即分段线性连接. 紧支撑逼近也叫紧化方法. 第三问对勒贝格积分也是对的.
\end{note}
\begin{proof}
\begin{enumerate}[(1)]
\item 对任何 $\varepsilon > 0$, 因为 $f \in R[a,b]$, 所以存在一个划分 $a = x_0 < x_1 < \cdots < x_n = b$ 使得
\[
\sum_{i = 1}^{n} w_i(f)(x_i - x_{i - 1}) \leqslant \varepsilon, w_i(f) \text{ 表示 } f \text{ 在 } [x_{i - 1},x_i], i = 1,2,\cdots,n \text{ 的振幅}.
\]
连接线段 $(x_{i - 1},f(x_{i - 1})),(x_i,f(x_i)), i = 1,2,\cdots,n$ 得到连续函数 $g$.(不难发现$\underset{x\in \left[ a,b \right]}{\mathrm{sup}}\left| g \right|\leqslant \underset{x\in \left[ a,b \right]}{\mathrm{sup}}\left| f \right|$) 则
\begin{align*}
\int_{a}^{b} |f(x) - g(x)|\mathrm{d}x &= \sum_{i = 1}^{n} \int_{x_{i - 1}}^{x_i} |f(x) - g(x)|\mathrm{d}x \\
&\leqslant \sum_{i = 1}^{n} \int_{x_{i - 1}}^{x_i} |f(x) - f(x_{i - 1})|\mathrm{d}x + \sum_{i = 1}^{n} \int_{x_{i - 1}}^{x_i} |g(x) - f(x_{i - 1})|\mathrm{d}x \\
&\leqslant \sum_{i = 1}^{n} w_i(f)(x_i - x_{i - 1}) + \sum_{i = 1}^{n} \int_{x_{i - 1}}^{x_i} \left| \frac{f(x_i) - f(x_{i - 1})}{x_i - x_{i - 1}} (x - x_{i - 1}) \right|\mathrm{d}x \\
&\leqslant \sum_{i = 1}^{n} w_i(f)(x_i - x_{i - 1}) + \sum_{i = 1}^{n} w_i(f) \int_{x_{i - 1}}^{x_i} \frac{x - x_{i - 1}}{x_i - x_{i - 1}}\mathrm{d}x \\
&= \sum_{i = 1}^{n} w_i(f)(x_i - x_{i - 1}) + \frac{1}{2} \sum_{i = 1}^{n} w_i(f)(x_i - x_{i - 1}) \leqslant \frac{3}{2}\varepsilon,
\end{align*}
这就完成了证明.

\item 对任何 $\varepsilon \in (0,1)$, 由第1问可知,存在 $g \in C[a,b]$ 使得
\[
\int_{a}^{b} |f(x) - g(x)|\mathrm{d}x < \frac{\varepsilon}{4}.
\]
取充分小的 $\delta > 0$, 使得
\[
\int_{a}^{a + \delta} |f(x)|\mathrm{d}x < \frac{\varepsilon}{4}, \int_{b - \delta}^{b} |f(x)|\mathrm{d}x < \frac{\varepsilon}{4}, \int_{a + \delta}^{a + 2\delta} |g(x)|\mathrm{d}x < \frac{\varepsilon}{16}, \int_{b - 2\delta}^{b - \delta} |g(x)|\mathrm{d}x < \frac{\varepsilon}{16}.
\]
再取 $h \in C^{\infty}(\mathbb{R})$ 使得

(a): $0 \leqslant h(x) \leqslant 1, \forall x \in \mathbb{R}$;

(b): $h(x) = 0, \forall x \in (-\infty,a + \delta) \cup (b - \delta,+\infty)$;

(c): $h(x) = 1, \forall x \in [a + 2\delta,b - 2\delta]$.

于是取 $g_1(x) = h(x)g(x) \in C_c(a,b)$,由第1问可知$\underset{x\in \left[ a,b \right]}{\mathrm{sup}}\left| g \right|\leqslant \underset{x\in \left[ a,b \right]}{\mathrm{sup}}\left| f \right|$,从而$\underset{x\in \left[ a,b \right]}{\mathrm{sup}}\left| g_1 \right|\leqslant \underset{x\in \left[ a,b \right]}{\mathrm{sup}}\left| f \right|$.
从而
\begin{align*}
\int_{a}^{b} |f(x) - g_1(x)|\mathrm{d}x &= \int_{a}^{b} |f(x) - g(x)h(x)|\mathrm{d}x \\
&\leqslant \int_{a}^{a + \delta} |f(x)|\mathrm{d}x + \int_{b - \delta}^{b} |f(x)|\mathrm{d}x + \int_{a + \delta}^{b - \delta} |f(x) - h(x)g(x)|\mathrm{d}x \\
&\leqslant \int_{a}^{a + \delta} |f(x)|\mathrm{d}x + \int_{b - \delta}^{b} |f(x)|\mathrm{d}x + \int_{a + \delta}^{a + 2\delta} |f(x) - g(x)|\mathrm{d}x + \int_{a + \delta}^{b - \delta} |g(x) - h(x)g(x)|\mathrm{d}x \\
&\leqslant \int_{a}^{a + \delta} |f(x)|\mathrm{d}x + \int_{b - \delta}^{b} |f(x)|\mathrm{d}x + \int_{a}^{b} |f(x) - g(x)|\mathrm{d}x + \int_{a + \delta}^{b - \delta} |g(x) - h(x)g(x)|\mathrm{d}x \\
&\leqslant \frac{3\varepsilon}{4} + \int_{a + \delta}^{a + 2\delta} |g(x) - h(x)g(x)|\mathrm{d}x + \int_{b - 2\delta}^{b - \delta} |g(x) - h(x)g(x)|\mathrm{d}x \\
&\leqslant \frac{3\varepsilon}{4} + 2 \int_{a + \delta}^{a + 2\delta} |g(x)|\mathrm{d}x + 2 \int_{b - 2\delta}^{b - \delta} |g(x)|\mathrm{d}x \\
&\leqslant \varepsilon.
\end{align*}
这就完成了证明.

\item 证明的想法和第2问类似. 由条件可知,对任何 $\varepsilon > 0$, 存在$X > 0$, 使得
\[
\int_{|x| \geqslant X} |f(x)|^{p}\mathrm{d}x=
\int_{X}^{\infty} |f(x)|^{p}\mathrm{d}x + \int_{-\infty}^{-X} |f(x)|^{p}\mathrm{d}x < \varepsilon.
\]
因为 $f$ 在 $[-X,X]$ 黎曼可积, 所以由第2问, 存在 $g \in C_c(-X,X)$ 使得
\[
\int_{-X}^{X} |f(x) - g(x)|\mathrm{d}x < \frac{\varepsilon}{1 + \sup_{[-X,X]} |2f|^{p - 1}}.
\]
从前两问的构造可以看到
\[
\sup_{[-X,X]} |g| \leqslant \sup_{[-X,X]} |f|,
\]
于是
\begin{align*}
\int_{-\infty}^{\infty} |f(x) - g(x)|^{p}\mathrm{d}x &= \int_{|x| \geqslant X} |f(x)|^{p}\mathrm{d}x + \int_{-X}^{X} |f(x) - g(x)|^{p}\mathrm{d}x \\
&\leqslant \varepsilon + \sup_{[-X,X]} |f - g|^{p - 1} \int_{-X}^{X} |f(x) - g(x)|\mathrm{d}x \\
&\leqslant \varepsilon + \sup_{[-X,X]} (2|f|)^{p - 1} \int_{-X}^{X} |f(x) - g(x)|\mathrm{d}x \\
&< \varepsilon + \varepsilon,
\end{align*}
这就完成了证明. 
\end{enumerate}
\end{proof}

\begin{lemma}\label{lemma:齐次化方法应用(三元p次方)}
证明:若 $A,B,C\in \mathbb{R}$, 则存在 $C_p>0$, 使得
\begin{align*}
|A + B + C|^p \leqslant C_p(|A|^p + |B|^p + |C|^p).
\end{align*}
\end{lemma}
\begin{note}
利用齐次化方法证明齐次不等式的应用.
\end{note}
\begin{proof}
令
\begin{align*}
S \triangleq \left\{ (A,B,C) \mid |A|^p + |B|^p + |C|^p = 1 \right\},
\end{align*}
则 $S$ 是 $\mathbb{R}^3$ 上的有界闭集, 从而 $S$ 是紧集. 于是 $|A + B + C|^p$ 可以看作紧集 $S$ 上关于 $(A,B,C)$ 的连续函数, 故一定存在 $C_p > 0$, 使得
\begin{align}
|A + B + C|^p \leqslant C_p, \forall (A,B,C) \in S. \label{theorem-equation-积分的绝对连续性-0.1}
\end{align}
对 $\forall (A,B,C) \in \mathbb{R}^3$, 固定 $A,B,C$, 不妨设 $A,B,C$ 不全为零, 否则不等式显然成立. 令
\begin{align*}
L = \frac{1}{\sqrt[p]{|A|^p + |B|^p + |C|^p}},
\end{align*}
考虑 $(LA,LB,LC)$, 则此时
\begin{align*}
|LA|^p + |LB|^p + |LC|^p = 1.
\end{align*}
因此 $(LA,LB,LC) \in S$. 从而由\eqref{theorem-equation-积分的绝对连续性-0.1}式可知
\begin{align*}
|LA + LB + LC|^p \leqslant C_p.
\end{align*}
于是
\begin{align*}
|A + B + C|^p \leqslant \frac{C_p}{L^p} = C_p(|A|^p + |B|^p + |C|^p).
\end{align*}
故结论得证. 
\end{proof}

\begin{theorem}[积分的绝对连续性]\label{theorem:积分的绝对连续性}
设 $p \geqslant 1$ 且反常积分 $\int_{-\infty}^{\infty} |f(x)|^{p}\mathrm{d}x < \infty$, 证明
\begin{align}
\lim_{h \to 0} \int_{-\infty}^{\infty} |f(x + h) - f(x)|^{p}\mathrm{d}x = 0. \label{theorem-equation-积分的绝对连续性-13.118}
\end{align}
\end{theorem}
\begin{note}
本结果对勒贝格积分也是正确的, 但我们证明只对黎曼积分进行.
\end{note}
\begin{proof}
$\mathbf{Step}\mathbf{1}$: 当 $f\in C_c(\mathbb{R})$ 时, 则存在 $X>0$, 使得
\begin{align*}
f(x) = 0, \forall |x| \geqslant X.
\end{align*}
从而当 $h\in (-1,1)$ 时, 就有
\begin{align*}
f(x) = 0, \forall |x| \geqslant X + 1.
\end{align*}
又因为 $f\in C[-X - 1,X + 1]$, 所以由 Cantor 定理可知 $f$ 在 $[-X - 1,X + 1]$ 上一致连续. 于是
\begin{align*}
\lim_{h\rightarrow 0} \int_{-\infty}^{+\infty} |f(x + h) - f(x)|^p \mathrm{d}x &= \lim_{h\rightarrow 0} \int_{|x| \leqslant X + 1} |f(x + h) - f(x)|^p \mathrm{d}x \\
&= \int_{|x| \leqslant X + 1} \lim_{h\rightarrow 0} |f(x + h) - f(x)|^p \mathrm{d}x = 0.
\end{align*}
$\mathbf{Step}\mathbf{2}$: 对一般的 $f$, 满足 $\int_{-\infty}^{\infty} |f(x)|^p \mathrm{d}x < \infty$. 对 $\forall \varepsilon > 0$, 由\hyperref[theorem:可积被连续函数逼近]{定理\ref{theorem:可积被连续函数逼近}(3)}可知, 存在 $g\in C_c(\mathbb{R})$, 使得
\begin{align*}
\int_{-\infty}^{+\infty} |f(x) - g(x)|^p \mathrm{d}x < \varepsilon.
\end{align*}
从而
\begin{align*}
\int_{-\infty}^{+\infty} |f(x + h) - f(x)|^p \mathrm{d}x &\leqslant \int_{-\infty}^{+\infty} |f(x + h) - g(x + h) + g(x + h) - g(x) + g(x) - f(x)|^p \mathrm{d}x
\end{align*}
利用齐次化方法得到\hyperref[lemma:齐次化方法应用(三元p次方)]{引理\ref{lemma:齐次化方法应用(三元p次方)}},从而可知 若 $A,B,C\in \mathbb{R}$, 则存在 $C_p>0$, 使得
\begin{align*}
|A + B + C|^p \leqslant C_p(|A|^p + |B|^p + |C|^p).
\end{align*}
故
\begin{align}
\int_{-\infty}^{+\infty} |f(x + h) - f(x)|^p \mathrm{d}x &\leqslant C_p\left( \int_{-\infty}^{+\infty} |f(x + h) - g(x + h)|^p \mathrm{d}x + \int_{-\infty}^{+\infty} |g(x + h) - g(x)|^p \mathrm{d}x + \int_{-\infty}^{+\infty} |f(x) - g(x)|^p \mathrm{d}x \right) \nonumber \\
&\xlongequal[y=x+h]{\text{换元}} 2C_p \int_{-\infty}^{+\infty} |f(x) - g(x)|^p \mathrm{d}x + C_p \int_{-\infty}^{+\infty} |g(x + h) - g(x)|^p \mathrm{d}x \nonumber \\
&\leqslant 2\varepsilon C_p + C_p \int_{-\infty}^{+\infty} |g(x + h) - g(x)|^p \mathrm{d}x. \label{theorem-equation-积分的绝对连续性-2.0}
\end{align}
由 $g\in C_c(\mathbb{R})$, 结合 $\mathbf{Step}\mathbf{1}$ 可知
\begin{align}
\lim_{h\rightarrow 0} \int_{-\infty}^{+\infty} |g(x + h) - g(x)|^p \mathrm{d}x = 0. \label{theorem-equation-积分的绝对连续性-2.1}
\end{align}
于是由\eqref{theorem-equation-积分的绝对连续性-2.0}\eqref{theorem-equation-积分的绝对连续性-2.1} 式可得
\begin{align*}
\underset{h\rightarrow 0}{\overline{\lim }} \int_{-\infty}^{+\infty} |f(x + h) - f(x)|^p \mathrm{d}x \leqslant 2\varepsilon C_p.
\end{align*}
再由 $\varepsilon$ 的任意性得证.
\end{proof}
























\end{document}
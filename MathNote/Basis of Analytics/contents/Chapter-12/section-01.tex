\documentclass[../../main.tex]{subfiles}
\graphicspath{{\subfix{../../image/}}} % 指定图片目录,后续可以直接使用图片文件名。

% 例如:
% \begin{figure}[H]
% \centering
% \includegraphics[scale=0.4]{图.png}
% \caption{}
% \label{figure:图}
% \end{figure}
% 注意:上述\label{}一定要放在\caption{}之后,否则引用图片序号会只会显示??.

\begin{document}

\begin{theorem}[狄利克雷定理]\label{theorem:狄利克雷定理}
对于无理数\(a\),则存在无穷多对互素的整数\(p,q\)使得\(\left|a - \frac{p}{q}\right|\leq\frac{1}{q^2}\),而对有理数\(a\),这样的互素整数对\((p,q)\)只能是有限个.
\end{theorem}
\begin{note}
这通常称为“\textbf{齐次逼近}”,证明利用抽屉原理即可.
\end{note}

\begin{corollary}
对于实数\(a\),则\(a\)为无理数当且仅当任意\(\varepsilon>0\),存在整数\(x,y\)使得\(0 < |ax - y| < \varepsilon\).
\end{corollary}
\begin{proof}
对任意正整数\(N\),将\([0,1]\)均分为\(N\)个闭区间,每一个长度\(\frac{1}{N}\),则\(n + 1\)个数\(0,\{a\},\{2a\},\cdots,\{Na\}\)全部落在\([0,1]\)中,根据抽屉原理必定有两个数落入同一区间,也即存在\(0\leq i < j\leq N\)使得\(\{ia\},\{ja\}\in\left[\frac{k}{N},\frac{k + 1}{N}\right]\)。
注:因为\(a\)是无理数,所以任意\(i\neq j\)都一定有\(\{ia\}\neq\{ja\}\),否则\(ia - [ia]=ja - [ja]\)意味着\(a\)是有理数。
所以
\[|\{ia\} - \{ja\}|=|(j - i)a - M|\leq\frac{1}{N}\Rightarrow\left|a - \frac{M}{j - i}\right|\leq\frac{1}{N(j - i)}\]
这里\(M\)是一个整数,现在不一定有\(M\)与\(j - i\)互素,但是我们可以将其写成既约分数\(M = up,j - i = uq\),其中\((p,q)=1,u\in\mathbb{N}^+\),代入得到:对任意正整数\(N\),都存在互素的整数\(p,q\),其中\(1\leq q\leq N\)是正整数,使得\(\left|a - \frac{p}{q}\right|\leq\frac{1}{Nq}\leq\frac{1}{q^2}\)。
现在还没有说明“无穷多个”,采用反证法,假如使得\(\left|a - \frac{p}{q}\right|\leq\frac{1}{q^2}\)成立的互素的整数\((p,q)\)只有有限对,记为\((p_1,q_1),\cdots,(p_m,q_m)\),那么(在上面证明的结论里面)依次取\(N = 3,4,\cdots\),则每一个\(N\)都能够对应这\(m\)对\((p,q)\)中的某一个,而\(N = 3,4,\cdots\)是无限的,\(m\)是有限的,所以必定有一个\((p_i,q_i)\)对应了无穷多个正整数\(N\)。
不妨设\(i = 1\),换句话说:存在一列正整数\(N_k\)单调递增趋于正无穷,使得\(\left|a - \frac{p_1}{q_1}\right|\leq\frac{1}{N_kq_1}\)恒成立,令\(k\to\infty\)可知\(a = \frac{p}{q}\)是有理数,导致矛盾。

而如果\(a=\frac{m}{n}\)是有理数,但是有无穷个互素的\((p,q)\)使得\(\left|\frac{m}{n}-\frac{p}{q}\right|\leq\frac{1}{q^2}\),则当\(q\)充分大时,所有这些\((p,q)\)中的\(p\)也都会充分大(相当于同时趋于无穷),然而不等式等价于\(\frac{1}{q}\geq\frac{|mq - np|}{n}\),则当\(p,q\)都充分大时\(mq - np\neq0\)(不然会导致\(p|mq\)结合互素有\(p|m\)(对充分大的\(p\)均成立),显然矛盾),于是\(\frac{1}{q}\geq\frac{|mq - np|}{n}\geq\frac{1}{n}\)导致\(q\)有上界,还是矛盾,结论得证。
\end{proof}


\end{document}
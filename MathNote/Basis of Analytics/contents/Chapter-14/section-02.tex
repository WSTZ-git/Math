\documentclass[../../main.tex]{subfiles}
\graphicspath{{\subfix{../../image/}}} % 指定图片目录,后续可以直接使用图片文件名。

% 例如:
% \begin{figure}[h]
% \centering
% \includegraphics{image-01.01}
% \label{fig:image-01.01}
% \caption{图片标题}
% \end{figure}

\begin{document}

\subsection{定积分}

\subsection{建立积分递推}

\begin{example}
计算\(\int_{0}^{\frac{\pi}{2}} \cos^{n}x\sin(nx)dx\),\(n \in \mathbb{N}\)。 
\end{example}
\begin{proof}
利用分部积分和和差化积公式可得
\begin{align*}
I_n&=\int_0^{\frac{\pi}{2}}\cos ^nx\sin(nx) \mathrm{d}x=\int_0^{\frac{\pi}{2}}\cos ^{n - 1}x\cdot\cos x\sin(nx) \mathrm{d}x\\
&=\frac{1}{2}\int_0^{\frac{\pi}{2}}\cos ^{n - 1}x\cdot[\sin((n + 1)x)+\sin((n - 1)x)] \mathrm{d}x\\
&=\frac{I_{n - 1}}{2}+\frac{1}{2}\int_0^{\frac{\pi}{2}}\cos ^{n - 1}x[\sin(nx)\cos x+\cos(nx)\sin x] \mathrm{d}x\\
&=\frac{I_{n - 1}}{2}+\frac{I_n}{2}-\frac{1}{2}\int_0^{\frac{\pi}{2}}\cos ^{n - 1}x\cos(nx) \mathrm{d}\cos x\\
&=\frac{I_{n - 1}+I_n}{2}-\frac{1}{2n}\int_0^{\frac{\pi}{2}}\cos(nx) \mathrm{d}\cos ^nx\\
&=\frac{I_{n - 1}+I_n}{2}+\frac{1}{2n}-\frac{1}{2}\int_0^{\frac{\pi}{2}}\cos ^nx\sin(nx) \mathrm{d}x\\
&=\frac{I_{n - 1}+I_n}{2}+\frac{1}{2n}-\frac{I_n}{2}\\
&=\frac{I_{n - 1}}{2}+\frac{1}{2n}.
\end{align*}
故\(I_n=\frac{I_{n - 1}}{2}+\frac{1}{2n}\),则两边同乘\(2^n\)(强行裂项)
\begin{align*}
2^nI_n = 2^{n - 1}I_{n - 1}+\frac{2^{n - 1}}{n},n = 1,2,\cdots
\end{align*}
又注意到\(I_0 = 0\),从而
\begin{align*}
2^nI_n = 0+\sum_{k = 1}^n\frac{2^{k - 1}}{k}\Rightarrow I_n=\frac{1}{2^n}\sum_{k = 1}^n\frac{2^{k - 1}}{k}.
\end{align*}
\end{proof}

\begin{example}\label{example:经典定积分(必记)}
\begin{enumerate}
\item $\int_0^{\pi}{\frac{\sin \left( nx \right)}{\sin x}\mathrm{d}x}= 
\begin{cases}
0, & n\text{为偶数} \\
\pi, & n\text{为奇数}
\end{cases}.$

\item 
\end{enumerate}
\end{example}
\begin{proof}
\begin{enumerate}
\item 记\(I_n = \int_0^{\pi} \frac{\sin(nx)}{\sin x} \, dx\),则
\begin{align*}
I_{n+2} - I_n = \int_0^{\pi} \frac{\sin((n+2)x) - \sin(nx)}{\sin x} \, dx 
= \int_0^{\pi} \frac{2\cos((n+1)x) \sin x}{\sin x} \, dx 
= 2\int_0^{\pi} \cos((n+1)x) \, dx 
= 0.
\end{align*}
于是
\begin{align*}
\int_0^{\pi} \frac{\sin(nx)}{\sin x} \, dx = I_n = I_{n-2} = \cdots = 
\begin{cases}
I_0, & n\text{为偶数} \\
I_1, & n\text{为奇数}
\end{cases} = 
\begin{cases}
0, & n\text{为偶数} \\
\pi, & n\text{为奇数}
\end{cases}.
\end{align*}

\item 
\end{enumerate}
\end{proof}

\subsection{区间再现}

\begin{theorem}[区间再现恒等式]\label{theorem:区间再现恒等式}
当下述积分有意义时,我们有
\begin{enumerate}
\item \begin{align*}
\int_{a}^{b} f(x)\mathrm{d}x = \int_{a}^{b} f(a + b - x)\mathrm{d}x
=\frac{1}{2}\int_{a}^{b} [f(x)+f(a + b - x)]\mathrm{d}x
=\int_{a}^{\frac{a + b}{2}} [f(x)+f(a + b - x)]\mathrm{d}x.
\end{align*}

\item \begin{align*}
\int_{0}^{\infty} f(x)\mathrm{d}x = \int_{0}^{1} f(x)\mathrm{d}x+\int_{1}^{\infty} f(x)\mathrm{d}x
=\int_{0}^{1} \left[f(x)+\frac{f(\frac{1}{x})}{x^2}\right]\mathrm{d}x.
\end{align*}
\end{enumerate}
\end{theorem}
\begin{note}
注意:倒代换具有将$[0,1]$转化为$[1,+\infty)$的功能.
\end{note}
\begin{proof}
证明是显然.
\end{proof}

\begin{example}\label{example:常见积分1}
证明
\begin{enumerate}
\item \(\int_{0}^{\frac{\pi}{2}} \ln\sin x\mathrm{d}x=-\frac{\pi}{2}\ln 2.\)

\item \(\int_{0}^{\frac{\pi}{2}} \ln\cos x\mathrm{d}x-\frac{\pi}{2}\ln 2.\)

\item \(\int_{0}^{1} \frac{\ln(1 + x)}{1 + x^2}\mathrm{d}x=\frac{\pi}{8}\ln 2.\)
\end{enumerate}
\end{example}
\begin{proof}
\begin{enumerate}
\item \begin{align*}
I&=\int_0^{\frac{\pi}{2}}{\ln\sin x\mathrm{d}x}=\int_0^{\frac{\pi}{4}}{\left[ \ln\cos x+\ln \left( \frac{\pi}{2}-x \right) \right] \mathrm{d}x}
\\
&=\int_0^{\frac{\pi}{4}}{\ln \left( \cos x\sin x \right) \mathrm{d}x}=\int_0^{\frac{\pi}{4}}{\ln \frac{1}{2}\mathrm{d}x}+\int_0^{\frac{\pi}{4}}{\ln \left( \sin 2x \right) \mathrm{d}x}
\\
&=-\frac{\pi}{4}\ln 2+\frac{1}{2}\int_0^{\frac{\pi}{2}}{\ln\sin x\mathrm{d}x}=-\frac{\pi}{4}\ln 2+\frac{1}{2}I
\\
&\Longrightarrow I=\int_0^{\frac{\pi}{2}}{\ln\cos x\mathrm{d}x}=-\frac{\pi}{2}\ln 2.
\end{align*}

\item \begin{align*}
I&=\int_0^{\frac{\pi}{2}}{\ln\cos x\mathrm{d}x}=\int_0^{\frac{\pi}{2}}{\ln\sin x\mathrm{d}x}=\int_0^{\frac{\pi}{4}}{\left[ \ln\cos x+\ln \left( \frac{\pi}{2}-x \right) \right] \mathrm{d}x}
\\
&=\int_0^{\frac{\pi}{4}}{\ln \left( \cos x\sin x \right) \mathrm{d}x}=\int_0^{\frac{\pi}{4}}{\ln \frac{1}{2}\mathrm{d}x}+\int_0^{\frac{\pi}{4}}{\ln \left( \sin 2x \right) \mathrm{d}x}
\\
&=-\frac{\pi}{4}\ln 2+\frac{1}{2}\int_0^{\frac{\pi}{2}}{\ln\sin x\mathrm{d}x}=-\frac{\pi}{4}\ln 2+\frac{1}{2}I
\\
&\Longrightarrow I=\int_0^{\frac{\pi}{2}}{\ln\cos x\mathrm{d}x}=-\frac{\pi}{2}\ln 2.
\end{align*}

\item \begin{align*}
\int_0^1{\frac{\ln \left( 1+x \right)}{1+x^2}\mathrm{d}x}&\xlongequal{x=\tan \theta}\int_0^{\frac{\pi}{4}}{\frac{\ln \left( 1+\tan \theta \right)}{1+\tan \theta ^2}\mathrm{d}\tan \theta}=\int_0^{\frac{\pi}{4}}{\frac{\sec ^2\theta \cdot \ln \left( 1+\tan \theta \right)}{\sec ^2\theta}\mathrm{d}\theta}
\\
&=\int_0^{\frac{\pi}{4}}{\ln \left( 1+\tan \theta \right) \mathrm{d}\theta}=\int_0^{\frac{\pi}{8}}{\left[ \ln \left( 1+\tan \theta \right) +\ln \left( 1+\tan \left( \frac{\pi}{4}-\theta \right) \right) \right] \mathrm{d}\theta}
\\
&=\int_0^{\frac{\pi}{8}}{\left[ \ln \left( 1+\tan \theta \right) +\ln \left( 1+\frac{1-\tan \theta}{1+\tan \theta} \right) \right] \mathrm{d}\theta}
\\
&=\int_0^{\frac{\pi}{8}}{\left[ \ln \left( 1+\tan \theta \right) +\ln \frac{2}{1+\tan \theta} \right] \mathrm{d}\theta}
\\
&=\int_0^{\frac{\pi}{8}}{\ln 2\mathrm{d}\theta}=\frac{\pi}{8}\ln 2.
\end{align*} 
\end{enumerate}
\end{proof}

\begin{example}
计算
\begin{enumerate}
\item \(\int_{0}^{\infty} \frac{\ln x}{x^{2}+a^{2}}\mathrm{d}x\),$a>0$.

\item \(\int_{0}^{\infty} \frac{\ln x}{x^{2}+x + 1}\mathrm{d}x\).

\item\(\int_{0}^{1} \frac{\ln x}{\sqrt{x - x^{2}}}\mathrm{d}x\).
\end{enumerate}
\end{example}
\begin{solution}
\begin{enumerate}
\item 注意到
\begin{align}
\int_0^{+\infty}\frac{\ln x}{x^2+a^2}\mathrm{d}x\xlongequal{x=at}\frac{1}{a}\int_0^{+\infty}\frac{\ln (at)}{1+t^2}\mathrm{d}t
=\frac{1}{a}\int_0^{+\infty}\frac{\ln a}{1+t^2}\mathrm{d}t+\frac{1}{a}\int_0^{+\infty}\frac{\ln t}{1+t^2}\mathrm{d}t
=\frac{\pi \ln a}{2a}+\frac{1}{a}\int_0^{+\infty}\frac{\ln t}{1+t^2}\mathrm{d}t.\label{eq:1.1}
\end{align}
又注意到
\begin{align*}
\int_0^{+\infty}\frac{\ln t}{1+t^2}\mathrm{d}t\xlongequal{t=\frac{1}{x}}\int_0^{+\infty}\frac{\ln \frac{1}{x}}{1+\frac{1}{x^2}}\frac{1}{x^2}\mathrm{d}x=\int_0^{+\infty}\frac{-\ln x}{1+x^2}\mathrm{d}x\Longrightarrow \int_0^{+\infty}\frac{\ln t}{1+t^2}\mathrm{d}t=0.
\end{align*}
于是再结合\eqref{eq:1.1}式可得
\begin{align*}
\int_0^{+\infty}\frac{\ln x}{x^2+a^2}\mathrm{d}x=\frac{\pi \ln a}{2a}.
\end{align*}

\item \begin{align*}
\int_0^{\infty}{\frac{\ln x}{x^2+x+1}\mathrm{d}x}\xlongequal{x=\frac{1}{t}}\int_0^{\infty}{\frac{-\ln t}{1+\frac{1}{t}+\frac{1}{t^2}}\mathrm{d}\frac{1}{t}}=\int_0^{+\infty}{\frac{-\ln t}{1+t+t^2}\mathrm{d}t}\Longrightarrow \int_0^{\infty}{\frac{\ln x}{x^2+x+1}\mathrm{d}x}=0.
\end{align*}

\item\begin{align*}
\int_{0}^{1} \frac{\ln x}{\sqrt{x - x^{2}}}\mathrm{d}x &\xlongequal{x = \sin^{2}y} \int_{0}^{\frac{\pi}{2}} \frac{\ln \sin^{2}y}{\sqrt{\sin^{2}y(1 - \sin^{2}y)}}\mathrm{d}\sin^{2}y\\
&= 4\int_{0}^{\frac{\pi}{2}} \ln \sin y\mathrm{d}y \xlongequal{\text{\refexa{example:常见积分1}}}4\cdot\left(-\frac{\pi}{2}\ln 2\right)= -2\pi\ln 2.
\end{align*} 
\end{enumerate}
\end{solution}

\begin{example}
\begin{enumerate}
\item 对\(n \in \mathbb{N}\),计算\(\int_{-\pi}^{\pi} \frac{\sin(nx)}{(1 + 2^{x})\sin x}dx\)。

\item \(\int_{-\pi}^{\pi} \frac{x\sin x\arctan e^{x}}{1 + \cos^{2}x}dx\)。

\item 对\(n \in \mathbb{N}\),计算\(\int_{0}^{2\pi} \sin(\sin x + nx)dx\).
\end{enumerate}
\end{example}
\begin{solution}
\begin{enumerate}
\item \begin{align*}
&\int_{-\pi}^{\pi}{\frac{\sin \left( nx \right)}{\left( 1+2^x \right) \sin x}\mathrm{d}x}=\int_{-\pi}^0{\left[ \frac{\sin \left( nx \right)}{\left( 1+2^x \right) \sin x}+\frac{\sin \left( nx \right)}{\left( 1+2^{-x} \right) \sin x} \right] \mathrm{d}x}=\int_{-\pi}^0{\frac{\sin \left( nx \right)}{\sin x}\left( \frac{1}{1+2^x}+\frac{1}{1+2^{-x}} \right) \mathrm{d}x}
\\
&=\int_{-\pi}^0{\frac{\sin \left( nx \right)}{\sin x}\cdot \frac{2+2^x+2^{-x}}{2+2^x+2^{-x}}\mathrm{d}x}=\int_{-\pi}^0{\frac{\sin \left( nx \right)}{\sin x}\mathrm{d}x}
=\int_0^{\pi}{\frac{\sin \left( nx \right)}{\sin x}\mathrm{d}x}\xlongequal{\text{\refexa{example:经典定积分(必记)}}}\begin{cases}
0,n\text{为偶数}\\
\pi ,n\text{为奇数}\\
\end{cases}.
\end{align*}

\item \begin{align*}
\int_{-\pi}^{\pi}{\frac{x\sin x\mathrm{arc}\tan e^x}{1+\cos ^2x}\mathrm{d}x}&=\int_{-\pi}^0{\left( \frac{x\sin x\mathrm{arc}\tan e^x}{1+\cos ^2x}+\frac{x\sin x\mathrm{arc}\tan e^{-x}}{1+\cos ^2x} \right) \mathrm{d}x}=\int_{-\pi}^0{\frac{x\sin x}{1+\cos ^2x}\left( \mathrm{arc}\tan e^x+\mathrm{arc}\tan e^{-x} \right) \mathrm{d}x}
\\
&\xlongequal{\text{\nrefpro{proposition:arctan相关等式}{(1)}}}\int_{-\pi}^0{\frac{x\sin x}{1+\cos ^2x}\cdot \frac{\pi}{2}\mathrm{d}x}=\frac{\pi}{2}\int_0^{\pi}{\frac{x\sin x}{1+\cos ^2x}\mathrm{d}x}
\\
&=\frac{\pi}{2}\int_0^{\frac{\pi}{2}}{\left( \frac{x\sin x}{1+\cos ^2x}+\frac{\left( \pi -x \right) \sin x}{1+\cos ^2x} \right) \mathrm{d}x}=\frac{\pi ^2}{2}\int_0^{\frac{\pi}{2}}{\frac{\sin x}{1+\cos ^2x}\mathrm{d}x}
\\
&=\frac{\pi ^2}{2}\mathrm{arctan}\cos x\Big |_{\frac{\pi}{2}}^{0}=\frac{\pi ^2}{2}\cdot \frac{\pi}{4}=\frac{\pi ^3}{8}.
\end{align*}

\item\begin{align*}
\int_0^{2\pi}{\sin \left( \sin x+nx \right) \mathrm{d}x}&=\int_0^{2\pi}{\sin \left[ \sin \left( 2\pi -x \right) +n\left( 2\pi -x \right) \right] \mathrm{d}x}
\\
&=\int_0^{2\pi}{\sin \left( -\sin x-nx \right) \mathrm{d}x}=-\int_0^{2\pi}{\sin \left( \sin x+nx \right) \mathrm{d}x}
\\
&\Longrightarrow \int_0^{2\pi}{\sin \left( \sin x+nx \right) \mathrm{d}x}=0.
\end{align*} 
\end{enumerate}
\end{solution}



\subsection{化成含参积分/多元累次积分(换序)}

\begin{proposition}\label{proposition:重要定积分结果(必记)}
证明:
\begin{enumerate}
\item \(\int_{0}^{\infty} e^{-x^{2}}\mathrm{d}x=\frac{\sqrt{\pi}}{2}\).

\item \(\int_{0}^{\infty} \frac{\sin x}{x}\mathrm{d}x=\frac{\sqrt{\pi}}{2}\).

\item\(\int_{0}^{\infty} \sin x^{2}\mathrm{d}x\), \(\int_{0}^{\infty} \cos x^{2}\mathrm{d}x=\sqrt{\frac{\pi}{8}}\). 
\end{enumerate}
\end{proposition}
\begin{note}
本结果可以直接使用.
\end{note}
\begin{proof}
\begin{enumerate}
\item 注意到
\[\left( \int_0^{+\infty}{e^{-x^2}\mathrm{d}x} \right) ^2=\int_0^{+\infty}{\int_0^{+\infty}{e^{-\left( x^2+y^2 \right)}}}\mathrm{d}x\mathrm{d}y=\int_0^{\frac{\pi}{2}}{\int_0^{+\infty}{re^{-r^2}}}\mathrm{d}r\mathrm{d}\theta =\frac{\pi}{4}\int_0^{+\infty}{e^{-r^2}\mathrm{d}r^2}=\frac{\pi}{4}.\]
故\(\int_{0}^{\infty} e^{-x^{2}}\mathrm{d}x=\frac{\sqrt{\pi}}{2}\).

\item 注意到
\begin{align*}
\int_0^{+\infty}{\sin x e^{-yx}\,\mathrm{d}x}
= \mathrm{Im}\int_0^{+\infty}{e^{\mathrm{i}x - yx}\,\mathrm{d}x} 
= \mathrm{Im}\int_0^{+\infty}{e^{-(y - \mathrm{i})x}\,\mathrm{d}x} 
= \mathrm{Im}\frac{1}{y - \mathrm{i}} 
= \mathrm{Im}\frac{y + \mathrm{i}}{y^2 + 1} 
= \frac{1}{y^2 + 1}.
\end{align*}
因此就有
\begin{align*}
\int_0^{+\infty}{\frac{\sin x}{x}\,\mathrm{d}x}
&= \int_0^{+\infty}{\sin x \left( \int_0^{+\infty}{e^{-yx}\,\mathrm{d}y} \right) \,\mathrm{d}x} 
= \int_0^{+\infty}{\mathrm{d}y} \int_0^{+\infty}{\sin x e^{-yx}\,\mathrm{d}x} \\
&= \int_0^{+\infty}{\mathrm{d}y} \left( \mathrm{Im}\int_0^{+\infty}{e^{\mathrm{i}x - yx}} \right) \,\mathrm{d}x 
= \int_0^{+\infty}{\frac{1}{y^2 + 1}\,\mathrm{d}y} 
= \frac{\pi}{2}.
\end{align*}
当然本题也可以直接利用分部积分计算\(\int_0^{+\infty}{\sin x e^{-yx}\,\mathrm{d}x} = \frac{1}{y^2 + 1}\).

\item 注意到
\begin{align*}
\int_0^{+\infty}{e^{-ax^2}\,\mathrm{d}x} \xlongequal{x=\frac{t}{\sqrt{a}}} \frac{1}{\sqrt{a}}\int_0^{+\infty}{e^{-t^2}\,\mathrm{d}t} = \frac{1}{2}\sqrt{\frac{\pi}{a}}, \quad a > 0.
\end{align*}
并且 \(-\mathrm{i} = e^{-\frac{\pi}{2}\mathrm{i}}\),从而 \(\sqrt{-\mathrm{i}} = e^{-\frac{\pi}{4}\mathrm{i}} = \frac{\sqrt{2}}{2} - \frac{\sqrt{2}}{2}\mathrm{i}\)。于是
\begin{align*}
\int_0^{+\infty}{(\cos x^2 - \mathrm{i}\sin x^2)\,\mathrm{d}x} 
&= \int_0^{+\infty}{e^{-\mathrm{i}x^2}\,\mathrm{d}x} 
= \frac{1}{2}\sqrt{\frac{\pi}{\mathrm{i}}} 
= \frac{1}{2}\sqrt{-\mathrm{i}\pi} \\
&= \frac{\sqrt{\pi}}{2}\left( \frac{\sqrt{2}}{2} - \frac{\sqrt{2}}{2}\mathrm{i} \right) 
= \frac{\sqrt{2\pi}}{4} - \frac{\sqrt{2\pi}}{4}\mathrm{i}.
\end{align*}
故
\begin{align*}
\int_0^{+\infty}{\cos x^2\,\mathrm{d}x} 
= \mathrm{Re}\int_0^{+\infty}{(\cos x^2 - \mathrm{i}\sin x^2)\,\mathrm{d}x} 
= \mathrm{Re}\left( \frac{\sqrt{2\pi}}{4} - \frac{\sqrt{2\pi}}{4}\mathrm{i} \right) = \frac{\sqrt{2\pi}}{4}=\sqrt{\frac{\pi}{8}}, 
\\
\int_0^{+\infty}{\sin x^2\,\mathrm{d}x} = \mathrm{Im}\int_0^{+\infty}{(\cos x^2 - \mathrm{i}\sin x^2)\,\mathrm{d}x} = \mathrm{Im}\left( \frac{\sqrt{2\pi}}{4} - \frac{\sqrt{2\pi}}{4}\mathrm{i} \right) = \frac{\sqrt{2\pi}}{4}=\sqrt{\frac{\pi}{8}}.
\end{align*}
\end{enumerate}
\end{proof}

\begin{example}
计算 \(\int_{0}^{1}\sin\ln\frac{1}{x}\cdot\frac{x^{b}-x^{a}}{\ln x}dx\ (b > a > 0)\)。
\end{example}
\begin{proof}
\begin{align*}
\int_0^1{\mathrm{sin}\ln \frac{1}{x}}\cdot \frac{x^b-x^a}{\ln x}\mathrm{d}x&=\int_0^1{\mathrm{sin}\ln \frac{1}{x}}\left( \int_a^b{x^y}\mathrm{d}y \right) \mathrm{d}x=\int_a^b{\mathrm{d}y}\int_0^1{x^y\mathrm{sin}\ln \frac{1}{x}}\mathrm{d}x
\\
&\xlongequal{x=e^{-t}}\int_a^b{\mathrm{d}y}\int_{+\infty}^0{e^{-ty}\sin t}\mathrm{d}e^{-t}=\int_a^b{\mathrm{d}y}\int_0^{+\infty}{e^{-t\left( y+1 \right)}\sin t}\mathrm{d}t
\\
&\xlongequal{\text{\nrefpro{proposition:重要定积分结果(必记)}{(2)的证明过程}}}\int_a^b{\frac{1}{1+\left( y+1 \right) ^2}\mathrm{d}y}=\mathrm{arc}\tan \left( b+1 \right) -\mathrm{arc}\tan \left( a+1 \right) .
\end{align*} 
\end{proof}


















\end{document}
\documentclass[../../main.tex]{subfiles}
\graphicspath{{\subfix{../../image/}}} % 指定图片目录,后续可以直接使用图片文件名。

% 例如:
% \begin{figure}[H]
% \centering
% \includegraphics[scale=0.4]{image-01.01}
% \caption{图片标题}
% \label{figure:image-01.01}
% \end{figure}
% 注意:上述\label{}一定要放在\caption{}之后,否则引用图片序号会只会显示??.

\begin{document}

\section{定积分}

\subsection{建立积分递推}

\begin{example}
计算\(\int_{0}^{\frac{\pi}{2}} \cos^{n}x\sin(nx)\mathrm{d}x\),\(n \in \mathbb{N}\)。 
\end{example}
\begin{proof}
利用分部积分和和差化积公式可得
\begin{align*}
I_n&=\int_0^{\frac{\pi}{2}}\cos ^nx\sin(nx) \mathrm{d}x=\int_0^{\frac{\pi}{2}}\cos ^{n - 1}x\cdot\cos x\sin(nx) \mathrm{d}x\\
&=\frac{1}{2}\int_0^{\frac{\pi}{2}}\cos ^{n - 1}x\cdot[\sin((n + 1)x)+\sin((n - 1)x)] \mathrm{d}x\\
&=\frac{I_{n - 1}}{2}+\frac{1}{2}\int_0^{\frac{\pi}{2}}\cos ^{n - 1}x[\sin(nx)\cos x+\cos(nx)\sin x] \mathrm{d}x\\
&=\frac{I_{n - 1}}{2}+\frac{I_n}{2}-\frac{1}{2}\int_0^{\frac{\pi}{2}}\cos ^{n - 1}x\cos(nx) \mathrm{d}\cos x\\
&=\frac{I_{n - 1}+I_n}{2}-\frac{1}{2n}\int_0^{\frac{\pi}{2}}\cos(nx) \mathrm{d}\cos ^nx\\
&=\frac{I_{n - 1}+I_n}{2}+\frac{1}{2n}-\frac{1}{2}\int_0^{\frac{\pi}{2}}\cos ^nx\sin(nx) \mathrm{d}x\\
&=\frac{I_{n - 1}+I_n}{2}+\frac{1}{2n}-\frac{I_n}{2}\\
&=\frac{I_{n - 1}}{2}+\frac{1}{2n}.
\end{align*}
故\(I_n=\frac{I_{n - 1}}{2}+\frac{1}{2n}\),则两边同乘\(2^n\)(强行裂项)
\begin{align*}
2^nI_n = 2^{n - 1}I_{n - 1}+\frac{2^{n - 1}}{n},n = 1,2,\cdots
\end{align*}
又注意到\(I_0 = 0\),从而
\begin{align*}
2^nI_n = 0+\sum_{k = 1}^n\frac{2^{k - 1}}{k}\Rightarrow I_n=\frac{1}{2^n}\sum_{k = 1}^n\frac{2^{k - 1}}{k}.
\end{align*}
\end{proof}

\begin{proposition}\label{example:经典定积分(必记)}
证明:
\begin{enumerate}[(1)]
\item $\int_0^{\pi}{\frac{\sin \left( nx \right)}{\sin x}\mathrm{d}x}= 
\begin{cases}
0, & n\text{为偶数} \\
\pi, & n\text{为奇数}
\end{cases}.$

\item $\int_0^{\pi}{\frac{\sin ^2\left( nx \right)}{\sin ^2x}}\mathrm{d}x=n\pi$.

\item $\int_0^{\pi}{\frac{\sin ^2\left( nx \right)}{\sin x}}\mathrm{d}x=\sum_{k=1}^n{\frac{2}{2k-1}}$.
\end{enumerate}
\end{proposition}
\begin{note}
提示:$\sin ^2x-\sin ^2y=\sin \left( x-y \right) \sin \left( x+y \right)$(证明见\refpro{proposition:三角平方差}).
\end{note}
\begin{proof}
\begin{enumerate}[(1)]
\item 记\(I_n = \int_0^{\pi} \frac{\sin(nx)}{\sin x} \, \mathrm{d}x\),则
\begin{align*}
I_{n+2} - I_n = \int_0^{\pi} \frac{\sin((n+2)x) - \sin(nx)}{\sin x} \, \mathrm{d}x 
= \int_0^{\pi} \frac{2\cos((n+1)x) \sin x}{\sin x} \, \mathrm{d}x 
= 2\int_0^{\pi} \cos((n+1)x) \, \mathrm{d}x 
= 0.
\end{align*}
于是
\begin{align*}
\int_0^{\pi} \frac{\sin(nx)}{\sin x} \, \mathrm{d}x = I_n = I_{n-2} = \cdots = 
\begin{cases}
I_0, & n\text{为偶数} \\
I_1, & n\text{为奇数}
\end{cases} = 
\begin{cases}
0, & n\text{为偶数} \\
\pi, & n\text{为奇数}
\end{cases}.
\end{align*}

\item 记 \( I_n = \int_0^\pi \frac{\sin^2(nx)}{\sin^2 x} \, \mathrm{d}x \),则
\begin{align}
I_{n+1} - I_n &= \int_0^\pi \frac{\sin^2((n+1)x) - \sin^2(nx)}{\sin^2 x} \, \mathrm{d}x = \int_0^\pi \frac{\sin x \cdot \sin((2n+1)x)}{\sin^2 x} \, \mathrm{d}x \notag \\
&= \int_0^\pi \frac{\sin((2n+1)x)}{\sin x} \, \mathrm{d}x \xlongequal{\text{\nrefpro{example:经典定积分(必记)}{(1)}}} \pi.
\end{align}
于是
\[
\int_0^\pi \frac{\sin^2(nx)}{\sin^2 x} \, \mathrm{d}x = I_n = \pi + I_{n-1} = \cdots = (n-1)\pi + I_1 = n\pi.
\]

\item 记 \( I_n = \int_0^\pi \frac{\sin^2(nx)}{\sin x} \, \mathrm{d}x \),则
\begin{align}
I_{n+1} - I_n &= \int_0^\pi \frac{\sin^2((n+1)x) - \sin^2(nx)}{\sin x} \, \mathrm{d}x = \int_0^\pi \frac{\sin x \cdot \sin((2n+1)x)}{\sin x} \, \mathrm{d}x \notag \\
&= \int_0^\pi \sin((2n+1)x) \, \mathrm{d}x = \frac{1}{2n+1} \cos((2n+1)x) \bigg|_\pi^0 = \frac{2}{2n+1}.
\end{align}
于是
\[
\int_0^\pi \frac{\sin^2(nx)}{\sin x} \, \mathrm{d}x = I_n = \frac{2}{2n-1} + I_{n-1} = \cdots = \sum_{k=1}^{n-1} \frac{2}{2k+1} + I_1 = \sum_{k=0}^{n-1} \frac{2}{2k+1}=\sum_{k=1}^n{\frac{2}{2k-1}}.
\]
\end{enumerate}
\end{proof}

\subsection{区间再现}

\begin{theorem}[区间再现恒等式]\label{theorem:区间再现恒等式}
当下述积分有意义时,我们有
\begin{enumerate}
\item \begin{align*}
\int_{a}^{b} f(x)\mathrm{d}x = \int_{a}^{b} f(a + b - x)\mathrm{d}x
=\frac{1}{2}\int_{a}^{b} [f(x)+f(a + b - x)]\mathrm{d}x
=\int_{a}^{\frac{a + b}{2}} [f(x)+f(a + b - x)]\mathrm{d}x.
\end{align*}

\item \begin{align*}
\int_{0}^{\infty} f(x)\mathrm{d}x = \int_{0}^{1} f(x)\mathrm{d}x+\int_{1}^{\infty} f(x)\mathrm{d}x
=\int_{0}^{1} \left[f(x)+\frac{f(\frac{1}{x})}{x^2}\right]\mathrm{d}x.
\end{align*}
\end{enumerate}
\end{theorem}
\begin{note}
注意:倒代换具有将$[0,1]$转化为$[1,+\infty)$的功能.
\end{note}
\begin{proof}
证明是显然的.
\end{proof}

\begin{proposition}\label{example:常见积分1}
证明
\begin{enumerate}
\item \(\int_{0}^{\frac{\pi}{2}} \ln\sin x\mathrm{d}x=-\frac{\pi}{2}\ln 2.\)

\item \(\int_{0}^{\frac{\pi}{2}} \ln\cos x\mathrm{d}x-\frac{\pi}{2}\ln 2.\)

\item \(\int_{0}^{1} \frac{\ln(1 + x)}{1 + x^2}\mathrm{d}x=\frac{\pi}{8}\ln 2.\)
\end{enumerate}
\end{proposition}
\begin{proof}
\begin{enumerate}
\item \begin{align*}
I&=\int_0^{\frac{\pi}{2}}{\ln\sin x\mathrm{d}x}=\int_0^{\frac{\pi}{4}}{\left[ \ln\cos x+\ln \left( \frac{\pi}{2}-x \right) \right] \mathrm{d}x}
\\
&=\int_0^{\frac{\pi}{4}}{\ln \left( \cos x\sin x \right) \mathrm{d}x}=\int_0^{\frac{\pi}{4}}{\ln \frac{1}{2}\mathrm{d}x}+\int_0^{\frac{\pi}{4}}{\ln \left( \sin 2x \right) \mathrm{d}x}
\\
&=-\frac{\pi}{4}\ln 2+\frac{1}{2}\int_0^{\frac{\pi}{2}}{\ln\sin x\mathrm{d}x}=-\frac{\pi}{4}\ln 2+\frac{1}{2}I
\\
&\Longrightarrow I=\int_0^{\frac{\pi}{2}}{\ln\cos x\mathrm{d}x}=-\frac{\pi}{2}\ln 2.
\end{align*}

\item \begin{align*}
I&=\int_0^{\frac{\pi}{2}}{\ln\cos x\mathrm{d}x}=\int_0^{\frac{\pi}{2}}{\ln\sin x\mathrm{d}x}=\int_0^{\frac{\pi}{4}}{\left[ \ln\cos x+\ln \left( \frac{\pi}{2}-x \right) \right] \mathrm{d}x}
\\
&=\int_0^{\frac{\pi}{4}}{\ln \left( \cos x\sin x \right) \mathrm{d}x}=\int_0^{\frac{\pi}{4}}{\ln \frac{1}{2}\mathrm{d}x}+\int_0^{\frac{\pi}{4}}{\ln \left( \sin 2x \right) \mathrm{d}x}
\\
&=-\frac{\pi}{4}\ln 2+\frac{1}{2}\int_0^{\frac{\pi}{2}}{\ln\sin x\mathrm{d}x}=-\frac{\pi}{4}\ln 2+\frac{1}{2}I
\\
&\Longrightarrow I=\int_0^{\frac{\pi}{2}}{\ln\cos x\mathrm{d}x}=-\frac{\pi}{2}\ln 2.
\end{align*}

\item \begin{align*}
\int_0^1{\frac{\ln \left( 1+x \right)}{1+x^2}\mathrm{d}x}&\xlongequal{x=\tan \theta}\int_0^{\frac{\pi}{4}}{\frac{\ln \left( 1+\tan \theta \right)}{1+\tan \theta ^2}\mathrm{d}\tan \theta}=\int_0^{\frac{\pi}{4}}{\frac{\sec ^2\theta \cdot \ln \left( 1+\tan \theta \right)}{\sec ^2\theta}\mathrm{d}\theta}
\\
&=\int_0^{\frac{\pi}{4}}{\ln \left( 1+\tan \theta \right) \mathrm{d}\theta}=\int_0^{\frac{\pi}{8}}{\left[ \ln \left( 1+\tan \theta \right) +\ln \left( 1+\tan \left( \frac{\pi}{4}-\theta \right) \right) \right] \mathrm{d}\theta}
\\
&=\int_0^{\frac{\pi}{8}}{\left[ \ln \left( 1+\tan \theta \right) +\ln \left( 1+\frac{1-\tan \theta}{1+\tan \theta} \right) \right] \mathrm{d}\theta}
\\
&=\int_0^{\frac{\pi}{8}}{\left[ \ln \left( 1+\tan \theta \right) +\ln \frac{2}{1+\tan \theta} \right] \mathrm{d}\theta}
\\
&=\int_0^{\frac{\pi}{8}}{\ln 2\mathrm{d}\theta}=\frac{\pi}{8}\ln 2.
\end{align*} 
\end{enumerate}
\end{proof}

\begin{example}
计算
\begin{enumerate}
\item \(\int_{0}^{\infty} \frac{\ln x}{x^{2}+a^{2}}\mathrm{d}x\),$a>0$.

\item \(\int_{0}^{\infty} \frac{\ln x}{x^{2}+x + 1}\mathrm{d}x\).

\item\(\int_{0}^{1} \frac{\ln x}{\sqrt{x - x^{2}}}\mathrm{d}x\).
\end{enumerate}
\end{example}
\begin{solution}
\begin{enumerate}
\item 注意到
\begin{align}
\int_0^{+\infty}\frac{\ln x}{x^2+a^2}\mathrm{d}x\xlongequal{x=at}\frac{1}{a}\int_0^{+\infty}\frac{\ln (at)}{1+t^2}\mathrm{d}t
=\frac{1}{a}\int_0^{+\infty}\frac{\ln a}{1+t^2}\mathrm{d}t+\frac{1}{a}\int_0^{+\infty}\frac{\ln t}{1+t^2}\mathrm{d}t
=\frac{\pi \ln a}{2a}+\frac{1}{a}\int_0^{+\infty}\frac{\ln t}{1+t^2}\mathrm{d}t.\label{eq:1.1}
\end{align}
又注意到
\begin{align*}
\int_0^{+\infty}\frac{\ln t}{1+t^2}\mathrm{d}t\xlongequal{t=\frac{1}{x}}\int_0^{+\infty}\frac{\ln \frac{1}{x}}{1+\frac{1}{x^2}}\frac{1}{x^2}\mathrm{d}x=\int_0^{+\infty}\frac{-\ln x}{1+x^2}\mathrm{d}x\Longrightarrow \int_0^{+\infty}\frac{\ln t}{1+t^2}\mathrm{d}t=0.
\end{align*}
于是再结合\eqref{eq:1.1}式可得
\begin{align*}
\int_0^{+\infty}\frac{\ln x}{x^2+a^2}\mathrm{d}x=\frac{\pi \ln a}{2a}.
\end{align*}

\item \begin{align*}
\int_0^{\infty}{\frac{\ln x}{x^2+x+1}\mathrm{d}x}\xlongequal{x=\frac{1}{t}}\int_0^{\infty}{\frac{-\ln t}{1+\frac{1}{t}+\frac{1}{t^2}}\mathrm{d}\frac{1}{t}}=\int_0^{+\infty}{\frac{-\ln t}{1+t+t^2}\mathrm{d}t}\Longrightarrow \int_0^{\infty}{\frac{\ln x}{x^2+x+1}\mathrm{d}x}=0.
\end{align*}

\item\begin{align*}
\int_{0}^{1} \frac{\ln x}{\sqrt{x - x^{2}}}\mathrm{d}x &\xlongequal{x = \sin^{2}y} \int_{0}^{\frac{\pi}{2}} \frac{\ln \sin^{2}y}{\sqrt{\sin^{2}y(1 - \sin^{2}y)}}\mathrm{d}\sin^{2}y\\
&= 4\int_{0}^{\frac{\pi}{2}} \ln \sin y\mathrm{d}y \xlongequal{\text{\refpro{example:常见积分1}}}4\cdot\left(-\frac{\pi}{2}\ln 2\right)= -2\pi\ln 2.
\end{align*} 
\end{enumerate}
\end{solution}

\begin{example}
\begin{enumerate}
\item 对\(n \in \mathbb{N}\),计算\(\int_{-\pi}^{\pi} \frac{\sin(nx)}{(1 + 2^{x})\sin x}\mathrm{d}x\)。

\item \(\int_{-\pi}^{\pi} \frac{x\sin x\arctan e^{x}}{1 + \cos^{2}x}\mathrm{d}x\)。

\item 对\(n \in \mathbb{N}\),计算\(\int_{0}^{2\pi} \sin(\sin x + nx)\mathrm{d}x\).
\end{enumerate}
\end{example}
\begin{solution}
\begin{enumerate}
\item \begin{align*}
&\int_{-\pi}^{\pi}{\frac{\sin \left( nx \right)}{\left( 1+2^x \right) \sin x}\mathrm{d}x}=\int_{-\pi}^0{\left[ \frac{\sin \left( nx \right)}{\left( 1+2^x \right) \sin x}+\frac{\sin \left( nx \right)}{\left( 1+2^{-x} \right) \sin x} \right] \mathrm{d}x}=\int_{-\pi}^0{\frac{\sin \left( nx \right)}{\sin x}\left( \frac{1}{1+2^x}+\frac{1}{1+2^{-x}} \right) \mathrm{d}x}
\\
&=\int_{-\pi}^0{\frac{\sin \left( nx \right)}{\sin x}\cdot \frac{2+2^x+2^{-x}}{2+2^x+2^{-x}}\mathrm{d}x}=\int_{-\pi}^0{\frac{\sin \left( nx \right)}{\sin x}\mathrm{d}x}
=\int_0^{\pi}{\frac{\sin \left( nx \right)}{\sin x}\mathrm{d}x}\xlongequal{\text{\refexa{example:经典定积分(必记)}}}\begin{cases}
0,n\text{为偶数}\\
\pi ,n\text{为奇数}\\
\end{cases}.
\end{align*}

\item \begin{align*}
\int_{-\pi}^{\pi}{\frac{x\sin x\mathrm{arc}\tan e^x}{1+\cos ^2x}\mathrm{d}x}&=\int_{-\pi}^0{\left( \frac{x\sin x\mathrm{arc}\tan e^x}{1+\cos ^2x}+\frac{x\sin x\mathrm{arc}\tan e^{-x}}{1+\cos ^2x} \right) \mathrm{d}x}=\int_{-\pi}^0{\frac{x\sin x}{1+\cos ^2x}\left( \mathrm{arc}\tan e^x+\mathrm{arc}\tan e^{-x} \right) \mathrm{d}x}
\\
&\xlongequal{\text{\nrefpro{proposition:arctan相关等式}{(1)}}}\int_{-\pi}^0{\frac{x\sin x}{1+\cos ^2x}\cdot \frac{\pi}{2}\mathrm{d}x}=\frac{\pi}{2}\int_0^{\pi}{\frac{x\sin x}{1+\cos ^2x}\mathrm{d}x}
\\
&=\frac{\pi}{2}\int_0^{\frac{\pi}{2}}{\left( \frac{x\sin x}{1+\cos ^2x}+\frac{\left( \pi -x \right) \sin x}{1+\cos ^2x} \right) \mathrm{d}x}=\frac{\pi ^2}{2}\int_0^{\frac{\pi}{2}}{\frac{\sin x}{1+\cos ^2x}\mathrm{d}x}
\\
&=\frac{\pi ^2}{2}\mathrm{arctan}\cos x\Big |_{\frac{\pi}{2}}^{0}=\frac{\pi ^2}{2}\cdot \frac{\pi}{4}=\frac{\pi ^3}{8}.
\end{align*}

\item\begin{align*}
\int_0^{2\pi}{\sin \left( \sin x+nx \right) \mathrm{d}x}&=\int_0^{2\pi}{\sin \left[ \sin \left( 2\pi -x \right) +n\left( 2\pi -x \right) \right] \mathrm{d}x}
\\
&=\int_0^{2\pi}{\sin \left( -\sin x-nx \right) \mathrm{d}x}=-\int_0^{2\pi}{\sin \left( \sin x+nx \right) \mathrm{d}x}
\\
&\Longrightarrow \int_0^{2\pi}{\sin \left( \sin x+nx \right) \mathrm{d}x}=0.
\end{align*} 
\end{enumerate}
\end{solution}



\subsection{化成多元累次积分(换序)}

\begin{proposition}\label{proposition:重要定积分结果(必记)}
证明:
\begin{enumerate}[(1)]
\item \(\int_{0}^{\infty} e^{-x^{2}}\mathrm{d}x=\frac{\sqrt{\pi}}{2}\).

\item \(\int_{0}^{\infty} \frac{\sin x}{x}\mathrm{d}x=\frac{\sqrt{\pi}}{2}\).

\item\(\int_{0}^{\infty} \sin x^{2}\mathrm{d}x\), \(\int_{0}^{\infty} \cos x^{2}\mathrm{d}x=\sqrt{\frac{\pi}{8}}\). 
\end{enumerate}
\end{proposition}
\begin{note}
本结果可以直接使用.
\end{note}
\begin{proof}
\begin{enumerate}[(1)]
\item 注意到
\begin{align*}
\left( \int_0^{+\infty}{e^{-x^2}\mathrm{d}x} \right) ^2&=\left( \int_0^{+\infty}{e^{-x^2}\mathrm{d}x} \right) \left( \int_0^{+\infty}{e^{-y^2}\mathrm{d}y} \right) \xlongequal{\text{把}\int_0^{+\infty}{e^{-y^2}\mathrm{d}y}\text{看作常数}}\int_0^{+\infty}{e^{-x^2}\left( \int_0^{+\infty}{e^{-y^2}}\mathrm{d}x \right) \mathrm{d}y}
\\
&\xlongequal{\text{把}e^{-x^2}\text{看作常数}}\int_0^{+\infty}{\left( \int_0^{+\infty}{e^{-\left( x^2+y^2 \right)}}\mathrm{d}x \right) \mathrm{d}y}\xlongequal{e^{-\left( x^2+y^2 \right)}\text{连续}}\iint_{R^2}{e^{-\left( x^2+y^2 \right)}\mathrm{d}x\mathrm{d}y}
\\
&=\int_0^{\frac{\pi}{2}}{\mathrm{d}\theta}\int_0^{+\infty}{re^{-r^2}\mathrm{d}r}=\frac{\pi}{2}\int_0^{+\infty}{re^{-r^2}\mathrm{d}r}
\\
&=\frac{\pi}{4}\int_0^{+\infty}{e^{-r^2}\mathrm{d}r^2}=\frac{\pi}{4}.
\end{align*}
故\(\int_{0}^{\infty} e^{-x^{2}}\mathrm{d}x=\frac{\sqrt{\pi}}{2}\).

\item 注意到
\begin{align*}
\int_0^{+\infty}{\sin x e^{-yx}\,\mathrm{d}x}
= \mathrm{Im}\int_0^{+\infty}{e^{\mathrm{i}x - yx}\,\mathrm{d}x} 
= \mathrm{Im}\int_0^{+\infty}{e^{-(y - \mathrm{i})x}\,\mathrm{d}x} 
= \mathrm{Im}\frac{1}{y - \mathrm{i}} 
= \mathrm{Im}\frac{y + \mathrm{i}}{y^2 + 1} 
= \frac{1}{y^2 + 1}.
\end{align*}
因此就有
\begin{align*}
\int_0^{+\infty}{\frac{\sin x}{x}\,\mathrm{d}x}
&= \int_0^{+\infty}{\sin x \left( \int_0^{+\infty}{e^{-yx}\,\mathrm{d}y} \right) \,\mathrm{d}x} 
= \int_0^{+\infty}{\mathrm{d}y} \int_0^{+\infty}{\sin x e^{-yx}\,\mathrm{d}x} \\
&= \int_0^{+\infty}{\mathrm{d}y} \left( \mathrm{Im}\int_0^{+\infty}{e^{\mathrm{i}x - yx}} \right) \,\mathrm{d}x 
= \int_0^{+\infty}{\frac{1}{y^2 + 1}\,\mathrm{d}y} 
= \frac{\pi}{2}.
\end{align*}
当然本题也可以直接利用分部积分计算\(\int_0^{+\infty}{\sin x e^{-yx}\,\mathrm{d}x} = \frac{1}{y^2 + 1}\).

\item 注意到
\begin{align*}
\int_0^{+\infty}{e^{-ax^2}\,\mathrm{d}x} \xlongequal{x=\frac{t}{\sqrt{a}}} \frac{1}{\sqrt{a}}\int_0^{+\infty}{e^{-t^2}\,\mathrm{d}t} = \frac{1}{2}\sqrt{\frac{\pi}{a}}, \quad a > 0.
\end{align*}
并且 \(-\mathrm{i} = e^{-\frac{\pi}{2}\mathrm{i}}\),从而 \(\sqrt{-\mathrm{i}} = e^{-\frac{\pi}{4}\mathrm{i}} = \frac{\sqrt{2}}{2} - \frac{\sqrt{2}}{2}\mathrm{i}\)。于是
\begin{align*}
\int_0^{+\infty}{(\cos x^2 - \mathrm{i}\sin x^2)\,\mathrm{d}x} 
&= \int_0^{+\infty}{e^{-\mathrm{i}x^2}\,\mathrm{d}x} 
= \frac{1}{2}\sqrt{\frac{\pi}{\mathrm{i}}} 
= \frac{1}{2}\sqrt{-\mathrm{i}\pi} \\
&= \frac{\sqrt{\pi}}{2}\left( \frac{\sqrt{2}}{2} - \frac{\sqrt{2}}{2}\mathrm{i} \right) 
= \frac{\sqrt{2\pi}}{4} - \frac{\sqrt{2\pi}}{4}\mathrm{i}.
\end{align*}
故
\begin{align*}
\int_0^{+\infty}{\cos x^2\,\mathrm{d}x} 
= \mathrm{Re}\int_0^{+\infty}{(\cos x^2 - \mathrm{i}\sin x^2)\,\mathrm{d}x} 
= \mathrm{Re}\left( \frac{\sqrt{2\pi}}{4} - \frac{\sqrt{2\pi}}{4}\mathrm{i} \right) = \frac{\sqrt{2\pi}}{4}=\sqrt{\frac{\pi}{8}}, 
\\
\int_0^{+\infty}{\sin x^2\,\mathrm{d}x} = \mathrm{Im}\int_0^{+\infty}{(\cos x^2 - \mathrm{i}\sin x^2)\,\mathrm{d}x} = \mathrm{Im}\left( \frac{\sqrt{2\pi}}{4} - \frac{\sqrt{2\pi}}{4}\mathrm{i} \right) = \frac{\sqrt{2\pi}}{4}=\sqrt{\frac{\pi}{8}}.
\end{align*}
\end{enumerate}
\end{proof}

\begin{example}
计算 \(\int_{0}^{1}\sin\ln\frac{1}{x}\cdot\frac{x^{b}-x^{a}}{\ln x}\mathrm{d}x\ (b > a > 0)\)。
\end{example}
\begin{proof}
\begin{align*}
\int_0^1{\mathrm{sin}\ln \frac{1}{x}}\cdot \frac{x^b-x^a}{\ln x}\mathrm{d}x&=\int_0^1{\mathrm{sin}\ln \frac{1}{x}}\left( \int_a^b{x^y}\mathrm{d}y \right) \mathrm{d}x=\int_a^b{\mathrm{d}y}\int_0^1{x^y\mathrm{sin}\ln \frac{1}{x}}\mathrm{d}x
\\
&\xlongequal{x=e^{-t}}\int_a^b{\mathrm{d}y}\int_{+\infty}^0{e^{-ty}\sin t}\mathrm{d}e^{-t}=\int_a^b{\mathrm{d}y}\int_0^{+\infty}{e^{-t\left( y+1 \right)}\sin t}\mathrm{d}t
\\
&\xlongequal{\text{\nrefpro{proposition:重要定积分结果(必记)}{(2)的证明过程}}}\int_a^b{\frac{1}{1+\left( y+1 \right) ^2}\mathrm{d}y}=\mathrm{arc}\tan \left( b+1 \right) -\mathrm{arc}\tan \left( a+1 \right) .
\end{align*} 
\end{proof}


\subsection{化成含参积分(求导)}

\begin{example}
设\(a,b\geqslant 0\)且不全为\(0\),计算\(\int_{0}^{\frac{\pi}{2}} \ln\left( a^{2} \cos^{2}x + b^{2} \sin^{2}x\right) \mathrm{d}x\).
\end{example}
\begin{remark}
实际上,根据$a>b$时得到的结果,可以看出$F(a,b)=\pi \ln\frac{a+b}{2}$对$a,b$有轮换对称性,故这个结果对其他情况显然也成立.
\end{remark}
\begin{proof}
设\(F(a,b)=\int_{0}^{\frac{\pi}{2}} \ln\left( a^{2} \cos^{2}x + b^{2} \sin^{2}x\right) \mathrm{d}x\),当\(a > b\)时,则
\\
\begin{align*}
\frac{\partial}{\partial b}F(a,b) &= \int_{0}^{\frac{\pi}{2}} \frac{\partial}{\partial b} \ln\left( a^{2} \cos^{2}x + b^{2} \sin^{2}x\right) \mathrm{d}x = \int_{0}^{\frac{\pi}{2}} \frac{2b\sin^{2}x}{a^{2} \cos^{2}x + b^{2} \sin^{2}x} \mathrm{d}x \\
&= \int_{0}^{\frac{\pi}{2}} \frac{2b\tan^{2}x}{a^{2} + b^{2}\tan^{2}x} \mathrm{d}x = \int_{0}^{+\infty} \frac{2bt^{2}}{(a^{2} + b^{2}t^{2})(1 + t^{2})} \mathrm{d}t \\
&= \frac{1}{a^{2}-b^{2}} \int_{0}^{+\infty} \left( \frac{2a^{2}b}{a^{2} + b^{2}t^{2}} - \frac{2b}{1 + t^{2}} \right) \mathrm{d}t \\
&= \frac{1}{a^{2}-b^{2}} \int_{0}^{+\infty} \frac{2a^{2}b}{a^{2} + b^{2}t^{2}} \mathrm{d}t - \frac{2b}{a^{2}-b^{2}} \int_{0}^{+\infty} \frac{1}{1 + t^{2}} \mathrm{d}t \\
&= \frac{2b}{a^{2}-b^{2}} \int_{0}^{+\infty} \frac{1}{1 + \left( \frac{b}{a}t \right)^{2}} \mathrm{d}t - \frac{b\pi}{a^{2}-b^{2}} \\
&= \frac{2b}{a^{2}-b^{2}} \cdot \frac{a}{b} \cdot \frac{\pi}{2} - \frac{b\pi}{a^{2}-b^{2}} = \frac{\pi}{a + b}.
\end{align*}
于是
\begin{align*}
F(a,b) &= F(a,0) + \int_{0}^{b} \frac{\partial}{\partial b^\prime} F(a,b^\prime) \mathrm{d}b^\prime = F(a,0) + \int_{0}^{b} \frac{\pi}{a + b^\prime} \mathrm{d}b^\prime \\
&= 2\int_{0}^{\frac{\pi}{2}} \ln(a\cos x) \mathrm{d}x + \pi \ln\frac{a + b}{a} \xlongequal{\text{\refexa{example:常见积分1}}} \pi \ln\frac{a + b}{2}.
\end{align*}
当\(a < b\)时,类似可得\(F(a,b)=\pi \ln\frac{a + b}{2}\)。当\(a = b\)时,\(F(a,b)=\int_{0}^{\frac{\pi}{2}} \ln a^{2} \mathrm{d}x = \pi \ln a = \pi \ln\frac{a + b}{2}\)。

综上,对\(\forall a,b\geqslant 0\),都有\(F(a,b)=\pi \ln\frac{a + b}{2}\)。 
\end{proof}



\subsection{级数展开方法}

积分和求和换序$\sum_{n = 1}^{\infty} \int_{a}^{b} f_n(x) \mathrm{d}x = \int_{a}^{b} \sum_{n = 1}^{\infty} f_n(x) \mathrm{d}x$,等价于
\begin{align*}
\lim_{m \to \infty} \sum_{n = 1}^{m} \int_{a}^{b} f_n(x) \mathrm{d}x &= \int_{a}^{b} \sum_{n = 1}^{\infty} f_n(x) \mathrm{d}x.
\end{align*}
又由于有限和随意交换,因此上式等价于
\begin{align*}
\lim_{m \to \infty} \int_{a}^{b} \sum_{n = 1}^{m} f_n(x) \mathrm{d}x &= \int_{a}^{b} \sum_{n = 1}^{\infty} f_n(x) \mathrm{d}x,
\end{align*}
于是
\begin{align*}
\sum_{n = 1}^{\infty} \int_{a}^{b} f_n(x) \mathrm{d}x = \int_{a}^{b} \sum_{n = 1}^{\infty} f_n(x) \mathrm{d}x \Longleftrightarrow  \lim_{m \to \infty} \int_{a}^{b} \sum_{n = m + 1}^{\infty} f_n(x) \mathrm{d}x = 0.
\end{align*} 

\begin{example}
计算$\int_{0}^{\infty} \frac{x}{1 + e^{x}} \mathrm{d}x$. 
\end{example}
\begin{solution}
由于
\\
\begin{align*}
\frac{1}{1 + x} = \sum_{n = 0}^{\infty} (-1)^n x^n, \quad 0 < x < 1.
\end{align*}
并且\(0 < e^{-x} < 1\),故
\\
\begin{align*}
\int_0^{+\infty} \frac{x}{1 + e^x} \mathrm{d}x &= \int_0^{+\infty} \frac{x e^{-x}}{1 + e^{-x}} \mathrm{d}x = \int_0^{+\infty} x \sum_{n = 0}^{\infty} (-1)^n e^{-(n + 1)x} \mathrm{d}x \\
&\xlongequal{\text{\hyperlink{积分求和换序equation100.39}{换序}}} \sum_{n = 0}^{\infty} (-1)^n \int_0^{+\infty} x e^{-(n + 1)x} \mathrm{d}x = \sum_{n = 0}^{\infty} \frac{(-1)^n}{(n + 1)^2} \label{equation:100.39} \\
&= \sum_{n = 0}^{\infty} \frac{1}{(2n + 1)^2} - \sum_{n = 0}^{\infty} \frac{1}{(2n)^2}.
\end{align*}
又因为\(\sum_{n = 1}^{\infty} \frac{1}{n^2} = \frac{\pi^2}{6}\),所以
\\
\begin{align*}
\sum_{n = 0}^{\infty} \frac{1}{(2n)^2} &= \frac{1}{4} \sum_{n = 1}^{\infty} \frac{1}{n^2} = \frac{\pi^2}{24}, \\
\sum_{n = 0}^{\infty} \frac{1}{(2n + 1)^2} &= \sum_{n = 1}^{\infty} \frac{1}{n^2} - \sum_{n = 0}^{\infty} \frac{1}{(2n)^2} = \frac{\pi^2}{8}.
\end{align*}
故
\\
\begin{align*}
\int_0^{+\infty} \frac{x}{1 + e^x} \mathrm{d}x &= \sum_{n = 0}^{\infty} \frac{1}{(2n + 1)^2} - \sum_{n = 0}^{\infty} \frac{1}{(2n)^2} = \frac{\pi^2}{8} - \frac{\pi^2}{24} = \frac{\pi^2}{12}.
\end{align*}
\hypertarget{积分求和换序equation100.39}{下面证明}\eqref{equation:100.39}式换序成立,等价于证明\(\lim_{m \to +\infty} \int_0^{+\infty} \sum_{n = m}^{\infty} x (-1)^n e^{-(n + 1)x} \mathrm{d}x = 0\).
由\hyperref[theorem:交错级数不等式]{交错级数不等式}及\(x e^{-(n + 1)x}\)关于\(n\)非负递减,对\(\forall m \in \mathbb{N}\),都有
\\
\begin{align*}
\int_0^{+\infty} \left| \sum_{n = m}^{\infty} x (-1)^n e^{-(n + 1)x} \right| \mathrm{d}x &\leqslant \int_0^{+\infty} x e^{-(m + 1)x} \mathrm{d}x = -\frac{x e^{-(m + 1)x}}{m + 1} \big|_0^{+\infty} + \frac{1}{m + 1} \int_0^{+\infty} e^{-(m + 1)x} \mathrm{d}x = \frac{1}{(m + 1)^2}.
\end{align*}
令\(m \to +\infty\),得\(\lim_{m \to +\infty} \int_0^{+\infty} \sum_{n = m}^{\infty} x (-1)^n e^{-(n + 1)x} \mathrm{d}x = 0\). 故\eqref{equation:100.39}式换序成立. 
\end{solution}

\begin{proposition}\label{proposition:常用级数公式1}
证明:
\begin{enumerate}[(1)]
\item \(\sum_{n = 1}^{\infty} \frac{q^{n} \sin(nx)}{n}= \arctan \frac{q \sin x}{1 - q \cos x}, |q| \leqslant 1\).

\item \(\sum_{n = 1}^{\infty} \frac{q^{n} \cos(nx)}{n}= -\frac{1}{2} \ln(1 + q^{2} - 2q \cos x), |q| \leqslant 1\).

\item \(\sum_{n = 1}^{\infty} \frac{q^{n} \cos(nx)}{n!}= e^{q \cos x} \cos(q \sin x) - 1, |q| \leqslant 1, x \in \mathbb{R}\).

\item \(\sum_{n = 1}^{\infty} \frac{q^{n} \sin(nx)}{n!}= e^{q \cos x} \sin(q \sin x), |q| \leqslant 1, x \in \mathbb{R}\).
\end{enumerate}
\end{proposition}
\begin{note}
在\(\mathbb{C}\)上,
\begin{align*}
\text{Ln}z &= \ln|z| + i(\text{arg }z + 2k\pi), k \in \mathbb{Z}.  
\end{align*}
我们定义主值支
\begin{align*}
\ln z &= \ln|z| + i\text{arg }z. 
\end{align*}
本部分内容无需记忆,只需要大概有个可以算的感觉即可,实际做题中可以围绕这种级数给出构造.
\end{note}
\begin{proof}
$\Im$表示取虚部,$\Re$表示取实部.
\begin{enumerate}[(1)]
\item 利用欧拉公式有
\begin{align*}
\sum_{n = 1}^{\infty} \frac{q^{n} \sin(nx)}{n} &= \Im\left( \sum_{n = 1}^{\infty} \frac{q^{n} e^{inx}}{n} \right) = \Im\left( \sum_{n = 1}^{\infty} \frac{(q e^{ix})^{n}}{n} \right) = \Im(-\ln(1 - q e^{ix})) \\
&= -\Im\left( \ln|1 - q e^{ix}| + i \frac{-q \sin x}{1 - q \cos x} \right) = \arctan \frac{q \sin x}{1 - q \cos x}.
\end{align*}

\item 利用欧拉公式有
\begin{align*}
\sum_{n = 1}^{\infty} \frac{q^{n} \cos(nx)}{n} &= -\Re\left( \ln|1 - q e^{ix}| + i \frac{-q \sin x}{1 - q \cos x} \right) = -\frac{1}{2} \ln\left[(1 - q \cos x)^{2} + q^{2} \sin^{2} x\right] \\
&= -\frac{1}{2} \ln(1 + q^{2} - 2q \cos x).
\end{align*}

\item 利用欧拉公式有
\begin{align*}
\sum_{n = 1}^{\infty} \frac{q^{n} \cos(nx)}{n!} &= \Re\left( \sum_{n = 1}^{\infty} \frac{(q e^{ix})^{n}}{n!} \right) = \Re\left( e^{q e^{ix}} - 1 \right) =\Re \left( e^{q\cos x+\mathrm{i}q\sin x}-1 \right) 
\\
&= \Re\left( e^{q \cos x} \cos(q \sin x) - 1 + i e^{q \cos x} \sin(q \sin x) \right) \\
&= e^{q \cos x} \cos(q \sin x) - 1.
\end{align*}

\item 利用(3)有
\begin{align*}
\sum_{n = 1}^{\infty} \frac{q^{n} \sin(nx)}{n!} &= \Im\left( e^{q \cos x} \cos(q \sin x) - 1 + i e^{q \cos x} \sin(q \sin x) \right) \\
&= e^{q \cos x} \sin(q \sin x).
\end{align*} 
\end{enumerate}
\end{proof}

\begin{example}
计算
\begin{enumerate}
\item \(\int_{0}^{2\pi} e^{\cos x} \cos(\sin x) \mathrm{d}x\).

\item \(\int_{0}^{\pi} \ln(1 - 2a\cos x + a^{2}) \mathrm{d}x, a \in (0, +\infty) \setminus \{1\}\). 
\end{enumerate}
\end{example}
\begin{remark}
由1的证明可得
\begin{align*}
e^{\cos x}\cos(\sin x) &= \mathrm{Re}\left( \sum_{n = 0}^{\infty} \frac{(e^{\mathrm{i}x})^n}{n!} \right) = \mathrm{Re}\left( \sum_{n = 0}^{\infty} \frac{e^{\mathrm{i}nx}}{n!} \right) = \sum_{n = 0}^{\infty} \frac{\cos(nx)}{n!}.
\end{align*}
实际上,上式就是\nrefpro{proposition:常用级数公式1}{(3)}的结论. 
\end{remark}
\begin{remark}
第2问也可以用含参积分求导的方法进行计算(这个方法更容易想到).
\end{remark}
\begin{proof}
\begin{enumerate}
\item \begin{align*}
\int_0^{2\pi}{e^{\cos x}\cos \left( \sin x \right) \mathrm{d}x}&=\mathrm{Re}\left( \int_0^{2\pi}{e^{\cos x}e^{\mathrm{i}\sin x}\mathrm{d}x} \right) =\mathrm{Re}\left( \int_0^{2\pi}{e^{\cos x+\mathrm{i}\sin x}\mathrm{d}x} \right) 
\\
&=\mathrm{Re}\left( \int_0^{2\pi}{e^{e^{\mathrm{i}x}}\mathrm{d}x} \right) =\mathrm{Re}\left[ \int_0^{2\pi}{\sum_{n=0}^{+\infty}{\frac{\left( e^{\mathrm{i}x} \right) ^n}{n!}}\mathrm{d}x} \right] =\mathrm{Re}\left[ \sum_{n=0}^{+\infty}{\int_0^{2\pi}{\frac{\left( e^{\mathrm{i}x} \right) ^n}{n!}\mathrm{d}x}} \right] 
\\
&=\mathrm{Re}\left( \sum_{n=0}^{+\infty}{\int_0^{2\pi}{\frac{e^{\mathrm{i}nx}}{n!}\mathrm{d}x}} \right) =\mathrm{Re}\left( \int_0^{2\pi}{\frac{e^{\mathrm{i}\cdot 0\cdot x}}{n!}\mathrm{d}x}+\sum_{n=1}^{+\infty}{\frac{e^{2\pi \mathrm{i}x}-1}{\mathrm{i}n\cdot n!}} \right) 
\\
&=\mathrm{Re}\left( \int_0^{2\pi}{1\mathrm{d}x}+0 \right) =2\pi .
\end{align*}

\item 注意到当\(a\in(0,1)\)时,有
\begin{align*}
\sum_{n = 1}^{\infty} \frac{a^n\cos(nx)}{n} &= \mathrm{Re}\left[ \sum_{n = 1}^{\infty} \frac{(ae^{\mathrm{i}x})^n}{n} \right] = -\mathrm{Re}\left[ \ln(1 - ae^{\mathrm{i}x}) \right] \\
&= -\mathrm{Re}\left[ \ln|1 - ae^{\mathrm{i}x}| + \mathrm{i} \arg(1 - ae^{\mathrm{i}x}) \right] = -\ln|1 - ae^{\mathrm{i}x}| \\
&= -\ln|(1 - a\cos x) + a\mathrm{i}\sin x| = -\frac{1}{2}\ln(1 + a^2 - 2a\cos x).
\end{align*}
于是当\(a\in(0,1)\)时,就有
\begin{align*}
\int_0^{\pi} \ln(1 - 2a\cos x + a^2) \mathrm{d}x &= -\frac{1}{2}\int_0^{\pi} \sum_{n = 1}^{\infty} \frac{a^n\cos(nx)}{n} \mathrm{d}x = 0.
\end{align*}
若\(a > 1\),则\(\frac{1}{a}\in(0,1)\),从而此时我们有
\begin{align*}
\int_0^{\pi} \ln(1 - 2a\cos x + a^2) \mathrm{d}x &= \pi \ln a^2 + \int_0^{\pi} \ln\left( \frac{1}{a^2} - \frac{2}{a}\cos x + 1 \right) \mathrm{d}x = \pi \ln a^2 = 2\pi \ln a.
\end{align*}

又由\(\ln(1 - 2a\cos x + a^2)\)关于\(a\)的偏导存在可知\(\int_0^{\pi} \ln(1 - 2a\cos x + a^2) \mathrm{d}x\)关于\(a\)连续. 于是由
\\
\begin{align*}
\int_0^{\pi} \ln(1 - 2a\cos x + a^2) \mathrm{d}x &= 2\pi \ln a, \quad \forall a > 1.
\end{align*}
可知当\(a = 1\)时,我们有
\\
\begin{align*}
\int_0^{\pi} \ln(2 - 2\cos x) \mathrm{d}x &= \lim_{a \to 1^+} \int_0^{\pi} \ln(1 - 2a\cos x + a^2) \mathrm{d}x = \lim_{a \to 1^+} (2\pi \ln a) = 0.
\end{align*} 
\end{enumerate}
\end{proof}

\begin{definition}[$\mathrm{Li}_2$函数]
定义  
\[
\mathrm{Li}_2(x) \triangleq \sum_{n=1}^{\infty} \frac{x^n}{n^2}, \quad x \in [-1,1].
\] 
\end{definition}

\begin{proposition}\label{proposition:Li_2函数的性质}
\begin{enumerate}[(1)]
\item $\mathrm{Li}_2(x) + \mathrm{Li}_2(1-x) = \frac{\pi^2}{6} - \ln x \cdot \ln(1-x)$,$x \in (0,1)$。  

\item $\mathrm{Li}_2\left( 1 \right) =\sum_{n=1}^{\infty}{\frac{1}{n^2}}=\frac{\pi ^2}{6},\quad \mathrm{Li}_2\left( 0 \right) =0,\quad \mathrm{Li}_2\left(\frac{1}{2}\right) = \frac{\pi^2}{12} - \frac{1}{2} \ln^2 \frac{1}{2}$。
\end{enumerate}
\end{proposition}
\begin{proof}
\begin{enumerate}[(1)]
\item 记 $f(x) \triangleq \mathrm{Li}_2(x)$,$F(x) \triangleq f(x) + f(1-x) + \ln x \ln(1-x)$。则  
\begin{align*}
f'(x) = \sum_{n=1}^{\infty} \frac{x^{n-1}}{n} = -\frac{1}{x} \ln(1-x).
\end{align*}  
于是  
\begin{align*}
F'(x) = -\frac{1}{x} \ln(1-x) + \frac{\ln x}{1-x} - \frac{\ln x}{1-x} + \frac{\ln(1-x)}{x} = 0.
\end{align*}  
故 $F(x) \equiv F(1) = f(0) + f(1) = 0 + \sum_{n=1}^{\infty} \frac{1}{n^2} = \frac{\pi^2}{6}$。  

\item 显然$\mathrm{Li}_2\left( 1 \right) =\sum_{n=1}^{\infty}{\frac{1}{n^2}}=\frac{\pi ^2}{6},\mathrm{Li}_2\left( 0 \right) =0.$由(1)可得  
\begin{align*}
\mathrm{Li}_2\left(\frac{1}{2}\right) + \mathrm{Li}_2\left(\frac{1}{2}\right) = 2\mathrm{Li}_2\left(\frac{1}{2}\right) = \frac{\pi^2}{6} - \ln^2 \frac{1}{2} \implies \mathrm{Li}_2\left(\frac{1}{2}\right) = \frac{\pi^2}{12} - \frac{1}{2} \ln^2 \frac{1}{2}.
\end{align*}  
\end{enumerate}
\end{proof}

\begin{example}
计算$\int_0^{\frac{1}{2}} \frac{\ln x}{1-x}\, \mathrm{d}x$.
\end{example}
\begin{solution}
\begin{align*}
\int_0^{\frac{1}{2}} \frac{\ln x}{1-x} \, \mathrm{d}x &= \int_{\frac{1}{2}}^1 \frac{\ln(1-x)}{x} \, \mathrm{d}x = -\sum_{n=1}^{\infty} \frac{1}{n} \int_{\frac{1}{2}}^1 x^{n-1} \, \mathrm{d}x\\
&= -\sum_{n=1}^{\infty} \frac{1}{n^2} + \sum_{n=1}^{\infty} \frac{1}{2^n n^2} = -\frac{\pi^2}{6} + \mathrm{Li}_2\left(\frac{1}{2}\right)
\\
&\xlongequal{\text{\refpro{proposition:Li_2函数的性质}}}-\frac{\pi^2}{12} - \frac{1}{2} \ln^2 \frac{1}{2}.
\end{align*}
\end{solution}


\subsection{其他}

\begin{example}\label{example:常用定积分公式}
证明积分$\int_{0}^{\infty}e^{-ax^{2}-\frac{b}{x^{2}}}\mathrm{d}x=\frac{1}{2}\sqrt{\frac{\pi}{a}}e^{-2\sqrt{ab}},a,b>0.$
\end{example}
\begin{proof}
当\(a=1\)时,就有
\begin{align*}
\int_0^{+\infty}{e^{-x^2-\frac{b}{x^2}}\mathrm{d}x}&=e^{-2\sqrt{b}}\int_0^{+\infty}{e^{-\left( x-\frac{\sqrt{b}}{x} \right) ^2}\mathrm{d}x}\xlongequal{y=\frac{\sqrt{b}}{x}}e^{-2\sqrt{b}}\int_0^{+\infty}{\frac{\sqrt{b}}{y^2}e^{-\left( \frac{\sqrt{b}}{y}-y \right) ^2}\mathrm{d}y}\notag\\
&=\frac{e^{-2\sqrt{b}}}{2}\int_0^{+\infty}{\left( 1+\frac{\sqrt{b}}{y^2} \right) e^{-\left( y-\frac{\sqrt{b}}{y} \right) ^2}\mathrm{d}y}=\frac{e^{-2\sqrt{b}}}{2}\int_0^{+\infty}{e^{-\left( y-\frac{\sqrt{b}}{y} \right) ^2}\mathrm{d}\left( y-\frac{\sqrt{b}}{y} \right)}\notag\\
&=\frac{e^{-2\sqrt{b}}}{2}\int_{-\infty}^{+\infty}{e^{-t^2}\mathrm{d}t}=\frac{\sqrt{\pi}}{2}e^{-2\sqrt{b}}.
\end{align*}
于是对\(\forall a>0\),就有
\begin{align*}
\int_0^{+\infty}{e^{-ax^2-\frac{b}{x^2}}\mathrm{d}x}=\frac{1}{\sqrt{a}}\int_0^{+\infty}{e^{-x^2-\frac{ab}{x^2}}\mathrm{d}x}=\frac{1}{2}\sqrt{\frac{\pi}{a}}e^{-2\sqrt{ab}}.
\end{align*}
\end{proof}

\begin{example}
计算$\int_{0}^{\infty}\frac{\cos(ax)}{1 + x^{2}}\mathrm{d}x, a\in\mathbb{R}$. 
\end{example}
\begin{remark}
本题可以用复变函数的方法(留数定理)来计算. 但是我们这里用基本的高等数学的方法来计算.

$\int_0^{\infty}{\frac{\sin \left( ax \right)}{1+x^2}\mathrm{d}x}$这个积分没办法算出具体的初等数值.
\end{remark}
\begin{proof}
\begin{align*}
\int_0^{+\infty}{\frac{\cos \left( ax \right)}{1+x^2}\mathrm{d}x}&=\frac{1}{2}\int_{-\infty}^{+\infty}{\frac{\cos \left( ax \right)}{1+x^2}\mathrm{d}x}=\frac{1}{2}\int_{-\infty}^{+\infty}{\cos \left( ax \right) \left( \int_0^{+\infty}{e^{-\left( 1+x^2 \right) y}\mathrm{d}y} \right) \mathrm{d}x}\\
&=\frac{1}{2}\int_{-\infty}^{+\infty}{\left( \int_0^{+\infty}{e^{-\left( 1+x^2 \right) y}\cos \left( ax \right) \mathrm{d}y} \right) \mathrm{d}x}=\frac{1}{2}\int_{-\infty}^{+\infty}{\int_0^{+\infty}{e^{-\left( 1+x^2 \right) y}\cos \left( ax \right) \mathrm{d}y}}\mathrm{d}x\\
&=\frac{1}{2}\int_0^{+\infty}{\left( \int_{-\infty}^{+\infty}{e^{-\left( 1+x^2 \right) y}\cos \left( ax \right) \mathrm{d}x} \right) \mathrm{d}y}=\frac{1}{2}\int_0^{+\infty}{e^{-y}\left( \int_{-\infty}^{+\infty}{e^{-x^2y}\cos \left( ax \right) \mathrm{d}x} \right)}\mathrm{d}y\\
&=\frac{1}{2}\mathrm{Re}\left( \int_0^{+\infty}{e^{-y}\left( \int_{-\infty}^{+\infty}{e^{-x^2y+\mathrm{i}ax}\mathrm{d}x} \right) \mathrm{d}y} \right) =\frac{1}{2}\mathrm{Re}\left( \int_0^{+\infty}{e^{-y}\left( \int_{-\infty}^{+\infty}{e^{-y\left( x^2-\frac{a\mathrm{i}x}{y}+\left( \frac{a\mathrm{i}}{2y} \right) ^2 \right) -\frac{a^2}{4y}}\mathrm{d}x} \right) \mathrm{d}y} \right) \\
&=\frac{1}{2}\mathrm{Re}\left( \int_0^{+\infty}{e^{-y}\left( \int_{-\infty}^{+\infty}{e^{-y\left( x-\frac{a\mathrm{i}}{2y} \right) ^2-\frac{a^2}{4y}}\mathrm{d}x} \right) \mathrm{d}y} \right) =\frac{1}{2}\mathrm{Re}\left( \int_0^{+\infty}{e^{-y}\left( \int_{-\infty}^{+\infty}{e^{-y\left( x+\frac{a}{2\mathrm{i}y} \right) ^2-\frac{a^2}{4y}}\mathrm{d}x} \right) \mathrm{d}y} \right) \\
&=\frac{1}{2}\mathrm{Re}\left( \int_0^{+\infty}{e^{-y}\left( \int_{-\infty}^{+\infty}{e^{-y\left( x+\frac{a}{2\mathrm{i}y} \right) ^2-\frac{a^2}{4y}}\mathrm{d}\left( x+\frac{a}{2\mathrm{i}y} \right)} \right) \mathrm{d}y} \right) =\frac{1}{2}\mathrm{Re}\left( \int_0^{+\infty}{e^{-y}\left( \int_{-\infty}^{+\infty}{e^{-yx^2-\frac{a^2}{4y}}\mathrm{d}x} \right) \mathrm{d}y} \right) \\
&=\frac{1}{2}\int_0^{+\infty}{e^{-y-\frac{a^2}{4y}}\left( \int_{-\infty}^{+\infty}{e^{-yx^2}\mathrm{d}x} \right) \mathrm{d}y}=\frac{\sqrt{\pi}}{2}\int_0^{+\infty}{\frac{1}{\sqrt{y}}e^{-y-\frac{a^2}{4y}}\mathrm{d}y}\\
&\xlongequal{y=t^2}\sqrt{\pi}\int_0^{+\infty}{e^{-t^2-\frac{a^2}{4t^2}}\mathrm{d}t}\xlongequal{\text{\refexa{example:常用定积分公式}}}\sqrt{\pi}\cdot \frac{\sqrt{\pi}}{2}e^{-|a|}=\frac{\pi}{2}e^{-|a|}.
\end{align*}
\end{proof}

\begin{example}
计算$\int_{0}^{\infty}\frac{1}{(1 + x^{8})^2}\mathrm{d}x$. 
\end{example}
\begin{remark}
由\refpro{proposition:分式化积分-Gamma函数相关性质}可知
对$\forall s>0$,都有
\begin{align*}
\frac{1}{t^s}=\frac{1}{\Gamma \left( s \right)}\int_0^{+\infty}{y^{s-1}e^{-ty}\mathrm{d}y},\forall t\in \mathbb{R}.
\end{align*}
本题的核心想法就是利用上式将$\frac{z}{1+x^8}$转化成积分形式.
\end{remark}
\begin{proof}
注意到
\\
\begin{align*}
\int_0^{+\infty}{ye^{-\left( 1+x^8 \right) y}\mathrm{d}y} \xlongequal{y=\frac{z}{1+x^8}} \frac{1}{(1+x^8)^2}\int_0^{+\infty}{ze^{-z}\mathrm{d}z} = \frac{1}{(1+x^8)^2},
\end{align*}
因此
\\
\begin{align*}
\int_0^{+\infty}{\frac{1}{(1+x^8)^2}\mathrm{d}x} &= \int_0^{+\infty}{\left( \int_0^{+\infty}{ye^{-\left( 1+x^8 \right) y}\mathrm{d}y} \right) \mathrm{d}x} 
= \int_0^{+\infty}{\left( \int_0^{+\infty}{ye^{-\left( 1+x^8 \right) y}\mathrm{d}x} \right) \mathrm{d}y} \\
&= \int_0^{+\infty}{ye^{-y}\left( \int_0^{+\infty}{e^{-x^8y}\mathrm{d}x} \right) \mathrm{d}y} \xlongequal{x=y^{-\frac{1}{8}}z^{\frac{1}{8}}} \int_0^{+\infty}{ye^{-y}\left( \int_0^{+\infty}{y^{-\frac{1}{8}}e^{-z}\mathrm{d}z^{\frac{1}{8}}} \right) \mathrm{d}y} \\
&= \frac{1}{8}\int_0^{+\infty}{y^{\frac{7}{8}}e^{-y}\left( \int_0^{+\infty}{z^{-\frac{7}{8}}e^{-z}\mathrm{d}z} \right) \mathrm{d}y} 
= \frac{1}{8}\int_0^{+\infty}{y^{\frac{7}{8}}e^{-y}\Gamma \left( \frac{1}{8} \right) \mathrm{d}y} \\
&= \frac{1}{8}\Gamma \left( \frac{15}{8} \right) \Gamma \left( \frac{1}{8} \right) = \frac{1}{8} \cdot \frac{7}{8} \Gamma \left( \frac{7}{8} \right) \Gamma \left( \frac{1}{8} \right) \\
&\xlongequal{\refthe{theorem:余元公式}} \frac{7\pi}{64 \sin\left( \frac{\pi}{8} \right)} = \frac{7\pi}{32\sqrt{2 - \sqrt{2}}}.
\end{align*}
\end{proof}












\end{document}
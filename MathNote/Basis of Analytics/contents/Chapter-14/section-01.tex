\documentclass[../../main.tex]{subfiles}
\graphicspath{{\subfix{../../image/}}} % 指定图片目录,后续可以直接使用图片文件名。

% 例如:
% \begin{figure}[H]
% \centering
% \includegraphics[scale=0.4]{图.png}
% \caption{}
% \label{figure:图}
% \end{figure}
% 注意:上述\label{}一定要放在\caption{}之后,否则引用图片序号会只会显示??.

\begin{document}

\section{不定积分计算}

\subsection{直接猜原函数}

计算定积分,能直接猜出原函数,就直接写出原函数,然后求导验证即可.

\vspace{0.5cm}

\begin{example}
计算\(\int \frac{e^{-\sin x}\sin(2x)}{(1 - \sin x)^2}\mathrm{d}x\)。
\end{example}
\begin{note}
因为$e^{g(x)}$的原函数一定仍含有$e^{g(x)}$,并且$\frac{1}{1-sinx}$求导后一部分是$\frac{1}{(1-sinx)^2}$,所以我们猜测原函数与$\frac{e^{-\sin x}}{1-\sin x}$有关.因此对其求导进行尝试.
\end{note}
\begin{proof}
注意到
\begin{align*}
\left(\frac{e^{-\sin x}}{1 - \sin x}\right)' = \frac{-\cos x e^{-\sin x}(1 - \sin x)+\cos x e^{-\sin x}}{(1 - \sin x)^2}
=\frac{e^{-\sin x}\cos x\sin x}{(1 - \sin x)^2}.
\end{align*}
故原函数为\(\frac{2e^{-\sin x}}{1 - \sin x}+C\),其中$C$为任意常数.求导验证:
\begin{align*}
\left(\frac{2e^{-\sin x}}{1 - \sin x}\right)'=\frac{e^{-\sin x}(2\cos x\sin x)}{(1 - \sin x)^2}=\frac{e^{-\sin x}\sin 2x}{(1 - \sin x)^2}.
\end{align*} 
\end{proof}

\begin{example}
计算\(\int \frac{1 - \ln x}{(x - \ln x)^2}\mathrm{d}x\)。
\end{example}
\begin{note}
由\((x - \ln x)^2\)知可待定原函数\(\frac{f(x)}{x - \ln x}\),从而猜出答案.
\end{note}
\begin{proof}
注意到
\begin{align*}
\left( \frac{x}{x-\ln x} \right) \prime =\frac{x-\ln x-x\left( 1-\frac{1}{x} \right)}{\left( x-\ln x \right) ^2}=\frac{1-\ln x}{\left( x-\ln x \right) ^2}.
\end{align*}
故原函数为$\frac{x}{x-\ln x}+C$,其中$C$为任意常数.
\end{proof}


\subsection{换元积分}

\begin{example}
设\(y(x - y)^2 = x\),计算\(\int \frac{\mathrm{d}x}{x - 3y}\)。 
\end{example}
\begin{note}
令\(y = tx\),则\(t = \frac{y}{x}\)(这里是猜测过程,\(t\)只是中间变量,不用考虑\(x\)是否取\(0\)),从而由条件可得
\begin{align*}
tx(x - tx)^2 &= x \Rightarrow tx^3(1 - t)^2 = x\\
\Rightarrow x^2&=\frac{1}{t(1 - t)^2} \Rightarrow x = \pm\frac{1}{\sqrt{t}(1 - t)}.
\end{align*}
因为这里是猜测的过程(只要最后能得到一个正确的原函数即可),不需要保证严谨性,所以我们直接取\(x=\frac{1}{\sqrt{t}(1 - t)}\),于是\(\begin{cases}
x=\frac{1}{\sqrt{t}(1 - t)}\\
y=\frac{\sqrt{t}}{1 - t}
\end{cases}\)。
代入不定积分得
\begin{align*}
\int{\frac{\mathrm{d}x}{x - 3y}}&=\int{\frac{\mathrm{d}x}{x - 3y}}=\int{\frac{\mathrm{d}\left(\frac{1}{\sqrt{t}(1 - t)}\right)}{\frac{1}{\sqrt{t}(1 - t)}-\frac{3\sqrt{t}}{1 - t}}}\\
&=\int{\frac{\mathrm{d}t}{2(t^2 - t)}}=\frac{1}{2}\int{\left(\frac{1}{t - 1}-\frac{1}{t}\right)\mathrm{d}t}\\
&=\frac{1}{2}\ln\left|\frac{t - 1}{t}\right| + C=\frac{1}{2}\ln\left|\frac{\frac{y}{x}-1}{\frac{y}{x}}\right| + C\\
&=\frac{1}{2}\ln\left|\frac{y - x}{y}\right| + C,
\end{align*}
其中\(C\)为任意常数。因此我们断言\(\int{\frac{\mathrm{d}x}{x - 3y}}=\frac{1}{2}\ln\left|\frac{y - x}{y}\right| + C\).
\end{note}
\begin{proof}
对原方程两边同时关于\(x\)求导得
\begin{align*}
y'(x - y)^2 + 2y(1 - y')(x - y) &= 1 \Rightarrow y'=\frac{1 - 2y(x - y)}{(x - y)(x - 3y)}.
\end{align*}
于是利用上式经过计算可得
\begin{align*}
\left(\frac{1}{2}\ln\left|\frac{y - x}{y}\right| + C\right)'=\frac{1}{x - 3y}.
\end{align*}
故\(\int{\frac{\mathrm{d}x}{x - 3y}}=\frac{1}{2}\ln\left|\frac{y - x}{y}\right| + C\),其中\(C\)为任意常数。 
\end{proof}





















\end{document}
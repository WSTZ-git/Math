\documentclass[../../main.tex]{subfiles}
\graphicspath{{\subfix{../../image/}}} % 指定图片目录,后续可以直接使用图片文件名。

% 例如:
% \begin{figure}[H]
% \centering
% \includegraphics[scale=0.4]{图.png}
% \caption{}
% \label{figure:图}
% \end{figure}
% 注意:上述\label{}一定要放在\caption{}之后,否则引用图片序号会只会显示??.

\begin{document}

\section{含参量积分}

\begin{definition}[含参量积分]
设 \( f(x,y) \) 是定义在矩形区域 \( R = [a,b] \times [c,d] \) 上的二元函数。当 \( x \) 取 \( [a,b] \) 上某定值时,函数 \( f(x,y) \) 则是定义在 \( [c,d] \) 上以 \( y \) 为自变量的一元函数。倘若这时 \( f(x,y) \) 在 \( [c,d] \) 上可积,则其积分值是 \( x \) 在 \( [a,b] \) 上取值的函数,记它为 \( \varphi(x) \),就有
\begin{align}
\varphi(x) = \int_{c}^{d} f(x,y) \, \mathrm{d}y, \, x \in [a,b]. \label{eq::::::--23u80u11}
\end{align}
一般地,设 \( f(x,y) \) 为定义在区域 \( G = \{ (x,y) \mid c(x) \leqslant y \leqslant d(x), a \leqslant x \leqslant b \} \) 上的二元函数,其中 \( c(x), d(x) \) 为定义在 \( [a,b] \) 上的连续函数,若对于 \( [a,b] \) 上每一固定的 \( x \) 值,\( f(x,y) \) 作为 \( y \) 的函数在闭区间 \( [c(x),d(x)] \) 上可积,则其积分值是 \( x \) 在 \( [a,b] \) 上取值的函数,记作 \( F(x) \) 时,就有
\begin{align}
F(x) = \int_{c(x)}^{d(x)} f(x,y) \, \mathrm{d}y, \, x \in [a,b]. \label{eq::::::--23u80u12}
\end{align}
用积分形式所定义的这两个函数 \(\eqref{eq::::::--23u80u11}\) 与 \(\eqref{eq::::::--23u80u12}\),通称为定义在 \( [a,b] \) 上\textbf{含参量 \( x \) 的(正常)积分},或简称\textbf{含参量积分}。
\end{definition}

\begin{theorem}[连续性]\label{theorem:含参量积分的连续性}
若二元函数 \( f(x,y) \) 在矩形区域 \( R = [a,b] \times [c,d] \) 上连续,则函数
\[
\varphi(x) = \int_{c}^{d} f(x,y) \, \mathrm{d}y,\quad \psi(y) = \int_{a}^{b} f(x,y) \, \mathrm{d}x .
\]
都在 \( [a,b] \) 上连续。
\end{theorem}
\begin{remark}
对于这个定理的结论也可以写成如下的形式:若 \( f(x,y) \) 在矩形区域 \( R \) 上连续,则对任何 \( x_0 \in [a,b] \),都有
\[
\lim_{x \to x_0} \int_{c}^{d} f(x,y) \, \mathrm{d}y = \int_{c}^{d} \lim_{x \to x_0} f(x,y) \, \mathrm{d}y.
\]
这个结论表明,定义在矩形区域上的连续函数,其极限运算与积分运算的顺序是可以交换的。
\end{remark}
\begin{proof}
设 \( x \in [a,b] \),对充分小的 \( \Delta x \),有 \( x + \Delta x \in [a,b] \)(若 \( x \) 为区间的端点,则仅考虑 \( \Delta x > 0 \) 或 \( \Delta x < 0 \)),于是
\begin{align}
\varphi(x + \Delta x) - \varphi(x) = \int_{c}^{d} [f(x + \Delta x,y) - f(x,y)] \, \mathrm{d}y. \label{eq:::--------------jiopewjg3}
\end{align}
由于 \( f(x,y) \) 在有界闭域 \( R \) 上连续,从而一致连续,即对任给的正数 \( \varepsilon \),总存在某个正数 \( \delta \),对 \( R \) 内任意两点 \( (x_1,y_1) \) 与 \( (x_2,y_2) \),只要
\[
|x_1 - x_2| < \delta, \, |y_1 - y_2| < \delta,
\]
就有
\begin{align}
|f(x_1,y_1) - f(x_2,y_2)| < \varepsilon.\label{eq:::--------------jiopewjg4}
\end{align}
所以由 \(\eqref{eq:::--------------jiopewjg3}\),\(\eqref{eq:::--------------jiopewjg4}\) 可推得:当 \( |\Delta x| < \delta \) 时,
\[
|\varphi(x + \Delta x) - \varphi(x)| \leqslant \int_{c}^{d} |f(x + \Delta x,y) - f(x,y)| \, \mathrm{d}y
< \int_{c}^{d} \varepsilon \, \mathrm{d}x = \varepsilon(d - c).
\]
这就证明了 \( \varphi(x) \) 在 \( [a,b] \) 上连续。

同理可证:若 \( f(x,y) \) 在矩形区域 \( R \) 上连续,则含参量 \( y \) 的积分
\[
\psi(y) = \int_{a}^{b} f(x,y) \, \mathrm{d}x .
\]
在 \( [c,d] \) 上连续。

\end{proof}

\begin{theorem}[连续性]\label{theorem:含参量积分的连续性1}
设二元函数 \( f(x,y) \) 在区域
\[
G = \{ (x,y) \mid c(x) \leqslant y \leqslant d(x), a \leqslant x \leqslant b \}
\]
上连续,其中 \( c(x), d(x) \) 为 \( [a,b] \) 上的连续函数,则函数
\begin{align}
F(x) = \int_{c(x)}^{d(x)} f(x,y) \, \mathrm{d}y. \label{eq:::--------------jiopewjg6}
\end{align}
在 \( [a,b] \) 上连续。
\end{theorem}
\begin{proof}
对积分 \(\eqref{eq:::--------------jiopewjg6}\) 用换元积分法,令
\[
y = c(x) + t(d(x) - c(x)).
\]
当 \( y \) 在 \( c(x) \) 与 \( d(x) \) 之间取值时,\( t \) 在 \( [0,1] \) 上取值,且
\[
\mathrm{d}y = (d(x) - c(x)) \, \mathrm{d}t.
\]
所以从 \(\eqref{eq:::--------------jiopewjg6}\) 式可得
\begin{align*}
F(x)  \int_{c(x)}^{d(x)} f(x,y) \, \mathrm{d}y
= \int_{0}^{1} f(x,c(x) + t(d(x) - c(x))) (d(x) - c(x)) \, \mathrm{d}t.
\end{align*}
由于被积函数
\[
f(x,c(x) + t(d(x) - c(x))) (d(x) - c(x))
\]
在矩形区域 \( [a,b] \times [0,1] \) 上连续,由\refthe{theorem:含参量积分的连续性}得积分 \(\eqref{eq:::--------------jiopewjg6}\) 所确定的函数 \( F(x) \) 在 \( [a,b] \) 上连续。

\end{proof}

\begin{theorem}[可微性]\label{theorem:含参量积分的可微性1}
若函数 \( f(x,y) \) 与其偏导数 \( \frac{\partial}{\partial x}f(x,y) \) 都在矩形区域 \( R = [a,b] \times [c,d] \) 上连续,则
\[
\varphi(x) = \int_{c}^{d} f(x,y) \, \mathrm{d}y
\]
在 \( [a,b] \) 上可微,且
\[
\frac{\mathrm{d}}{\mathrm{d}x} \int_{c}^{d} f(x,y) \, \mathrm{d}y = \int_{c}^{d} \frac{\partial}{\partial x} f(x,y) \, \mathrm{d}y.
\]
\end{theorem}
\begin{proof}
对于 \( [a,b] \) 内任一点 \( x \),设 \( x + \Delta x \in [a,b] \)(若 \( x \) 为区间端点,则讨论单侧导数),则
\[
\frac{\varphi(x + \Delta x) - \varphi(x)}{\Delta x} = \int_{c}^{d} \frac{f(x + \Delta x,y) - f(x,y)}{\Delta x} \, \mathrm{d}y.
\]
由微分学的拉格朗日中值定理及 \( f_x(x,y) \) 在有界闭域 \( R \) 上连续(从而一致连续),对任给正数 \( \varepsilon \),存在正数 \( \delta \),只要当 \( |\Delta x| < \delta \) 时,就有
\[
\left| \frac{f(x + \Delta x,y) - f(x,y)}{\Delta x} - f_x(x,y) \right|
= |f_x(x + \theta \Delta x,y) - f_x(x,y)| < \varepsilon,
\]
其中 \( \theta \in (0,1) \)。因此
\begin{align*}
\left| \frac{\Delta \varphi}{\Delta x} - \int_{c}^{d} f_x(x,y) \, \mathrm{d}y \right|
\leqslant \int_{c}^{d} \left| \frac{f(x + \Delta x,y) - f(x,y)}{\Delta x} - f_x(x,y) \right| \, \mathrm{d}y
< \varepsilon(d - c).
\end{align*}
这就证得对一切 \( x \in [a,b] \),有
\[
\frac{\mathrm{d}}{\mathrm{d}x} \varphi(x) = \int_{c}^{d} \frac{\partial}{\partial x} f(x,y) \, \mathrm{d}y.
\]

\end{proof}

\begin{theorem}[可微性]\label{theorem:含参量积分的可微性2}
设 \( f(x,y), f_x(x,y) \) 在 \( R = [a,b] \times [p,q] \) 上连续,\( c(x), d(x) \) 为定义在 \( [a,b] \) 上其值含于 \( [p,q] \) 内的可微函数,则函数
\[
F(x) = \int_{c(x)}^{d(x)} f(x,y) \, \mathrm{d}y
\]
在 \( [a,b] \) 上可微,且
\begin{align}
F'(x) = \int_{c(x)}^{d(x)} f_x(x,y) \, \mathrm{d}y + f(x,d(x)) d'(x) - f(x,c(x)) c'(x). \label{eq::-------6486-7}
\end{align}
\end{theorem} 
\begin{proof}
把 \( F(x) \) 看作复合函数
\[
F(x) = H(x,c,d) = \int_{c}^{d} f(x,y) \, \mathrm{d}y,
c = c(x), \, d = d(x).
\]
由复合函数求导法则及变限积分的求导法则,有
\[
\frac{\mathrm{d}}{\mathrm{d}x} F(x) = \frac{\partial H}{\partial x} + \frac{\partial H}{\partial c} \frac{\mathrm{d}c}{\mathrm{d}x} + \frac{\partial H}{\partial d} \frac{\mathrm{d}d}{\mathrm{d}x}
= \int_{c(x)}^{d(x)} f_x(x,y) \, \mathrm{d}y + f(x,d(x)) d'(x) - f(x,c(x)) c'(x).
\]

\end{proof}

\begin{theorem}[可积性]
若 \( f(x,y) \) 在矩形区域 \( R = [a,b] \times [c,d] \) 上连续,则 \( \varphi(x) \) 和 \( \psi(y) \) 分别在 \( [a,b] \) 和 \( [c,d] \) 上可积。
这就是说:在 \( f(x,y) \) 连续性假设下,同时存在两个求积顺序不同的积分:
\[
\int_{a}^{b} \left[ \int_{c}^{d} f(x,y) \, \mathrm{d}y \right] \, \mathrm{d}x \quad \text{与} \quad \int_{c}^{d} \left[ \int_{a}^{b} f(x,y) \, \mathrm{d}x \right] \, \mathrm{d}y.
\]
为书写简便起见,今后将上述两个积分写作
\[
\int_{a}^{b} \mathrm{d}x \int_{c}^{d} f(x,y) \, \mathrm{d}y \quad \text{与} \quad \int_{c}^{d} \mathrm{d}y \int_{a}^{b} f(x,y) \, \mathrm{d}x,
\]
前者表示 \( f(x,y) \) 先对 \( y \) 求积然后对 \( x \) 求积,后者则求积顺序相反。它们统称为\textbf{累次积分},或更确切地称为\textbf{二次积分}。
\end{theorem}
\begin{proof}


\end{proof}

\begin{theorem}
若 \( f(x,y) \) 在矩形区域 \( R = [a,b] \times [c,d] \) 上连续,则
\begin{align}
\int_{a}^{b} \mathrm{d}x \int_{c}^{d} f(x,y) \, \mathrm{d}y = \int_{c}^{d} \mathrm{d}y \int_{a}^{b} f(x,y) \, \mathrm{d}x. \label{eq:::::::----4618-8}
\end{align}
\end{theorem}
\begin{note}
这个定理指出,在 \( f(x,y) \) 连续性假设下,累次积分与求积顺序无关。
\end{note}
\begin{proof}
记
\[
\varphi_1(u) = \int_{a}^{u} \mathrm{d}x \int_{c}^{d} f(x,y) \, \mathrm{d}y,
\varphi_2(u) = \int_{c}^{d} \mathrm{d}y \int_{a}^{u} f(x,y) \, \mathrm{d}x,
\]
其中 \( u \in [a,b] \),现在分别求 \( \varphi_1(u) \) 与 \( \varphi_2(u) \) 的导数。
\[
\varphi_1'(u) = \frac{\mathrm{d}}{\mathrm{d}u} \int_{a}^{u} \varphi(x) \, \mathrm{d}x = \varphi(u).
\]
对于 \( \varphi_2(u) \),令 \( H(u,y) = \int_{a}^{u} f(x,y) \, \mathrm{d}x \),则有
\[
\varphi_2(u) = \int_{c}^{d} H(u,y) \, \mathrm{d}y.
\]
因为 \( H(u,y) \) 与 \( H_u(u,y) = f(u,y) \) 都在 \( R \) 上连续,由\refthe{theorem:含参量积分的可微性1},
\[
\varphi_2'(u) = \frac{\mathrm{d}}{\mathrm{d}u} \int_{c}^{d} H(u,y) \, \mathrm{d}y = \int_{c}^{d} H_u(u,y) \, \mathrm{d}y
= \int_{c}^{d} f(u,y) \, \mathrm{d}y = \varphi(u).
\]
故得 \( \varphi_1'(u) = \varphi_2'(u) \),因此对一切 \( u \in [a,b] \),有
\[
\varphi_1(u) = \varphi_2(u) + k \quad (k \text{ 为常数}).
\]
当 \( u = a \) 时,\( \varphi_1(a) = \varphi_2(a) = 0 \),于是 \( k = 0 \),即得
\[
\varphi_1(u) = \varphi_2(u), \, u \in [a,b].
\]
取 \( u = b \),就得到所要证明的 \(\eqref{eq:::::::----4618-8}\) 式。

\end{proof}

















\end{document}
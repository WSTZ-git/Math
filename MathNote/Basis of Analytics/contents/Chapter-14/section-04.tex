\documentclass[../../main.tex]{subfiles}
\graphicspath{{\subfix{../../image/}}} % 指定图片目录,后续可以直接使用图片文件名。

% 例如:
% \begin{figure}[H]
% \centering
% \includegraphics[scale=0.4]{图.png}
% \caption{}
% \label{figure:图}
% \end{figure}
% 注意:上述\label{}一定要放在\caption{}之后,否则引用图片序号会只会显示??.

\begin{document}

\section{含参量积分}

\subsection{含参量正常积分}

\begin{definition}[含参量积分]
设 \( f(x,y) \) 是定义在矩形区域 \( R = [a,b] \times [c,d] \) 上的二元函数。当 \( x \) 取 \( [a,b] \) 上某定值时,函数 \( f(x,y) \) 则是定义在 \( [c,d] \) 上以 \( y \) 为自变量的一元函数。倘若这时 \( f(x,y) \) 在 \( [c,d] \) 上可积,则其积分值是 \( x \) 在 \( [a,b] \) 上取值的函数,记它为 \( \varphi(x) \),就有
\begin{align}
\varphi(x) = \int_{c}^{d} f(x,y) \, \mathrm{d}y, \, x \in [a,b]. \label{eq::::::--23u80u11}
\end{align}
一般地,设 \( f(x,y) \) 为定义在区域 \( G = \{ (x,y) \mid c(x) \leqslant y \leqslant d(x), a \leqslant x \leqslant b \} \) 上的二元函数,其中 \( c(x), d(x) \) 为定义在 \( [a,b] \) 上的连续函数,若对于 \( [a,b] \) 上每一固定的 \( x \) 值,\( f(x,y) \) 作为 \( y \) 的函数在闭区间 \( [c(x),d(x)] \) 上可积,则其积分值是 \( x \) 在 \( [a,b] \) 上取值的函数,记作 \( F(x) \) 时,就有
\begin{align}
F(x) = \int_{c(x)}^{d(x)} f(x,y) \, \mathrm{d}y, \, x \in [a,b]. \label{eq::::::--23u80u12}
\end{align}
用积分形式所定义的这两个函数 \(\eqref{eq::::::--23u80u11}\) 与 \(\eqref{eq::::::--23u80u12}\),通称为定义在 \( [a,b] \) 上\textbf{含参量 \( x \) 的(正常)积分},或简称\textbf{含参量积分}。
\end{definition}

\begin{theorem}[连续性]\label{theorem:含参量积分的连续性}
若二元函数 \( f(x,y) \) 在矩形区域 \( R = [a,b] \times [c,d] \) 上连续,则函数
\[
\varphi(x) = \int_{c}^{d} f(x,y) \, \mathrm{d}y,\quad \psi(y) = \int_{a}^{b} f(x,y) \, \mathrm{d}x .
\]
都在 \( [a,b] \) 上连续。
\end{theorem}
\begin{remark}
对于这个定理的结论也可以写成如下的形式:若 \( f(x,y) \) 在矩形区域 \( R \) 上连续,则对任何 \( x_0 \in [a,b] \),都有
\[
\lim_{x \to x_0} \int_{c}^{d} f(x,y) \, \mathrm{d}y = \int_{c}^{d} \lim_{x \to x_0} f(x,y) \, \mathrm{d}y.
\]
这个结论表明,定义在矩形区域上的连续函数,其极限运算与积分运算的顺序是可以交换的。
\end{remark}
\begin{proof}
设 \( x \in [a,b] \),对充分小的 \( \Delta x \),有 \( x + \Delta x \in [a,b] \)(若 \( x \) 为区间的端点,则仅考虑 \( \Delta x > 0 \) 或 \( \Delta x < 0 \)),于是
\begin{align}
\varphi(x + \Delta x) - \varphi(x) = \int_{c}^{d} [f(x + \Delta x,y) - f(x,y)] \, \mathrm{d}y. \label{eq:::--------------jiopewjg3}
\end{align}
由于 \( f(x,y) \) 在有界闭域 \( R \) 上连续,从而一致连续,即对任给的正数 \( \varepsilon \),总存在某个正数 \( \delta \),对 \( R \) 内任意两点 \( (x_1,y_1) \) 与 \( (x_2,y_2) \),只要
\[
|x_1 - x_2| < \delta, \, |y_1 - y_2| < \delta,
\]
就有
\begin{align}
|f(x_1,y_1) - f(x_2,y_2)| < \varepsilon.\label{eq:::--------------jiopewjg4}
\end{align}
所以由 \(\eqref{eq:::--------------jiopewjg3}\),\(\eqref{eq:::--------------jiopewjg4}\) 可推得:当 \( |\Delta x| < \delta \) 时,
\[
|\varphi(x + \Delta x) - \varphi(x)| \leqslant \int_{c}^{d} |f(x + \Delta x,y) - f(x,y)| \, \mathrm{d}y
< \int_{c}^{d} \varepsilon \, \mathrm{d}x = \varepsilon(d - c).
\]
这就证明了 \( \varphi(x) \) 在 \( [a,b] \) 上连续。

同理可证:若 \( f(x,y) \) 在矩形区域 \( R \) 上连续,则含参量 \( y \) 的积分
\[
\psi(y) = \int_{a}^{b} f(x,y) \, \mathrm{d}x .
\]
在 \( [c,d] \) 上连续。

\end{proof}

\begin{theorem}[连续性]\label{theorem:含参量积分的连续性1}
设二元函数 \( f(x,y) \) 在区域
\[
G = \{ (x,y) \mid c(x) \leqslant y \leqslant d(x), a \leqslant x \leqslant b \}
\]
上连续,其中 \( c(x), d(x) \) 为 \( [a,b] \) 上的连续函数,则函数
\begin{align}
F(x) = \int_{c(x)}^{d(x)} f(x,y) \, \mathrm{d}y. \label{eq:::--------------jiopewjg6}
\end{align}
在 \( [a,b] \) 上连续。
\end{theorem}
\begin{proof}
对积分 \(\eqref{eq:::--------------jiopewjg6}\) 用换元积分法,令
\[
y = c(x) + t(d(x) - c(x)).
\]
当 \( y \) 在 \( c(x) \) 与 \( d(x) \) 之间取值时,\( t \) 在 \( [0,1] \) 上取值,且
\[
\mathrm{d}y = (d(x) - c(x)) \, \mathrm{d}t.
\]
所以从 \(\eqref{eq:::--------------jiopewjg6}\) 式可得
\begin{align*}
F(x)  \int_{c(x)}^{d(x)} f(x,y) \, \mathrm{d}y
= \int_{0}^{1} f(x,c(x) + t(d(x) - c(x))) (d(x) - c(x)) \, \mathrm{d}t.
\end{align*}
由于被积函数
\[
f(x,c(x) + t(d(x) - c(x))) (d(x) - c(x))
\]
在矩形区域 \( [a,b] \times [0,1] \) 上连续,由\refthe{theorem:含参量积分的连续性}得积分 \(\eqref{eq:::--------------jiopewjg6}\) 所确定的函数 \( F(x) \) 在 \( [a,b] \) 上连续。

\end{proof}

\begin{theorem}[可微性]\label{theorem:含参量积分的可微性1}
若函数 \( f(x,y) \) 与其偏导数 \( \frac{\partial}{\partial x}f(x,y) \) 都在矩形区域 \( R = [a,b] \times [c,d] \) 上连续,则
\[
\varphi(x) = \int_{c}^{d} f(x,y) \, \mathrm{d}y
\]
在 \( [a,b] \) 上可微,且
\[
\frac{\mathrm{d}}{\mathrm{d}x} \int_{c}^{d} f(x,y) \, \mathrm{d}y = \int_{c}^{d} \frac{\partial}{\partial x} f(x,y) \, \mathrm{d}y.
\]
\end{theorem}
\begin{proof}
对于 \( [a,b] \) 内任一点 \( x \),设 \( x + \Delta x \in [a,b] \)(若 \( x \) 为区间端点,则讨论单侧导数),则
\[
\frac{\varphi(x + \Delta x) - \varphi(x)}{\Delta x} = \int_{c}^{d} \frac{f(x + \Delta x,y) - f(x,y)}{\Delta x} \, \mathrm{d}y.
\]
由微分学的拉格朗日中值定理及 \( f_x(x,y) \) 在有界闭域 \( R \) 上连续(从而一致连续),对任给正数 \( \varepsilon \),存在正数 \( \delta \),只要当 \( |\Delta x| < \delta \) 时,就有
\[
\left| \frac{f(x + \Delta x,y) - f(x,y)}{\Delta x} - f_x(x,y) \right|
= |f_x(x + \theta \Delta x,y) - f_x(x,y)| < \varepsilon,
\]
其中 \( \theta \in (0,1) \)。因此
\begin{align*}
\left| \frac{\Delta \varphi}{\Delta x} - \int_{c}^{d} f_x(x,y) \, \mathrm{d}y \right|
\leqslant \int_{c}^{d} \left| \frac{f(x + \Delta x,y) - f(x,y)}{\Delta x} - f_x(x,y) \right| \, \mathrm{d}y
< \varepsilon(d - c).
\end{align*}
这就证得对一切 \( x \in [a,b] \),有
\[
\frac{\mathrm{d}}{\mathrm{d}x} \varphi(x) = \int_{c}^{d} \frac{\partial}{\partial x} f(x,y) \, \mathrm{d}y.
\]

\end{proof}

\begin{theorem}[可微性]\label{theorem:含参量积分的可微性2}
设 \( f(x,y), f_x(x,y) \) 在 \( R = [a,b] \times [p,q] \) 上连续,\( c(x), d(x) \) 为定义在 \( [a,b] \) 上其值含于 \( [p,q] \) 内的可微函数,则函数
\[
F(x) = \int_{c(x)}^{d(x)} f(x,y) \, \mathrm{d}y
\]
在 \( [a,b] \) 上可微,且
\begin{align}
F'(x) = \int_{c(x)}^{d(x)} f_x(x,y) \, \mathrm{d}y + f(x,d(x)) d'(x) - f(x,c(x)) c'(x). \label{eq::-------6486-7}
\end{align}
\end{theorem} 
\begin{proof}
把 \( F(x) \) 看作复合函数
\[
F(x) = H(x,c,d) = \int_{c}^{d} f(x,y) \, \mathrm{d}y,
c = c(x), \, d = d(x).
\]
由复合函数求导法则及变限积分的求导法则,有
\[
\frac{\mathrm{d}}{\mathrm{d}x} F(x) = \frac{\partial H}{\partial x} + \frac{\partial H}{\partial c} \frac{\mathrm{d}c}{\mathrm{d}x} + \frac{\partial H}{\partial d} \frac{\mathrm{d}d}{\mathrm{d}x}
= \int_{c(x)}^{d(x)} f_x(x,y) \, \mathrm{d}y + f(x,d(x)) d'(x) - f(x,c(x)) c'(x).
\]

\end{proof}

\begin{theorem}[可积性]\label{theorem:含参量正常积分的可积性定理}
若 \( f(x,y) \) 在矩形区域 \( R = [a,b] \times [c,d] \) 上连续,则 \( \varphi(x) \) 和 \( \psi(y) \) 分别在 \( [a,b] \) 和 \( [c,d] \) 上可积。
这就是说:在 \( f(x,y) \) 连续性假设下,同时存在两个求积顺序不同的积分:
\[
\int_{a}^{b} \left[ \int_{c}^{d} f(x,y) \, \mathrm{d}y \right] \, \mathrm{d}x \quad \text{与} \quad \int_{c}^{d} \left[ \int_{a}^{b} f(x,y) \, \mathrm{d}x \right] \, \mathrm{d}y.
\]
为书写简便起见,今后将上述两个积分写作
\[
\int_{a}^{b} \mathrm{d}x \int_{c}^{d} f(x,y) \, \mathrm{d}y \quad \text{与} \quad \int_{c}^{d} \mathrm{d}y \int_{a}^{b} f(x,y) \, \mathrm{d}x,
\]
前者表示 \( f(x,y) \) 先对 \( y \) 求积然后对 \( x \) 求积,后者则求积顺序相反。它们统称为\textbf{累次积分},或更确切地称为\textbf{二次积分}。
\end{theorem}
\begin{proof}


\end{proof}

\begin{theorem}
若 \( f(x,y) \) 在矩形区域 \( R = [a,b] \times [c,d] \) 上连续,则
\begin{align}
\int_{a}^{b} \mathrm{d}x \int_{c}^{d} f(x,y) \, \mathrm{d}y = \int_{c}^{d} \mathrm{d}y \int_{a}^{b} f(x,y) \, \mathrm{d}x. \label{eq:::::::----4618-8}
\end{align}
\end{theorem}
\begin{note}
这个定理指出,在 \( f(x,y) \) 连续性假设下,累次积分与求积顺序无关。
\end{note}
\begin{proof}
记
\[
\varphi_1(u) = \int_{a}^{u} \mathrm{d}x \int_{c}^{d} f(x,y) \, \mathrm{d}y,
\varphi_2(u) = \int_{c}^{d} \mathrm{d}y \int_{a}^{u} f(x,y) \, \mathrm{d}x,
\]
其中 \( u \in [a,b] \),现在分别求 \( \varphi_1(u) \) 与 \( \varphi_2(u) \) 的导数。
\[
\varphi_1'(u) = \frac{\mathrm{d}}{\mathrm{d}u} \int_{a}^{u} \varphi(x) \, \mathrm{d}x = \varphi(u).
\]
对于 \( \varphi_2(u) \),令 \( H(u,y) = \int_{a}^{u} f(x,y) \, \mathrm{d}x \),则有
\[
\varphi_2(u) = \int_{c}^{d} H(u,y) \, \mathrm{d}y.
\]
因为 \( H(u,y) \) 与 \( H_u(u,y) = f(u,y) \) 都在 \( R \) 上连续,由\refthe{theorem:含参量积分的可微性1},
\[
\varphi_2'(u) = \frac{\mathrm{d}}{\mathrm{d}u} \int_{c}^{d} H(u,y) \, \mathrm{d}y = \int_{c}^{d} H_u(u,y) \, \mathrm{d}y
= \int_{c}^{d} f(u,y) \, \mathrm{d}y = \varphi(u).
\]
故得 \( \varphi_1'(u) = \varphi_2'(u) \),因此对一切 \( u \in [a,b] \),有
\[
\varphi_1(u) = \varphi_2(u) + k \quad (k \text{ 为常数}).
\]
当 \( u = a \) 时,\( \varphi_1(a) = \varphi_2(a) = 0 \),于是 \( k = 0 \),即得
\[
\varphi_1(u) = \varphi_2(u), \, u \in [a,b].
\]
取 \( u = b \),就得到所要证明的 \(\eqref{eq:::::::----4618-8}\) 式。

\end{proof}




\subsection{含参量反常积分}

\subsubsection{含参量反常积分的一致收敛性及其判别法}

\begin{definition}[含参量反常积分]
设函数$f(x,y)$定义在无界区域$R=\{(x,y)|x\in I,c\leqslant y<+\infty\}$上,其中$I$为一区间,若对每一个固定的$x\in I$,反常积分
\begin{align}
\int_{c}^{+\infty}f(x,y)\mathrm{d}y \label{eq::::--wefw234fsd3480t34t3532jrwdr24t}
\end{align}
都收敛,则它的值是$x$在$I$上取值的函数,当记这个函数为$\varPhi(x)$时,则有
\begin{align}
\varPhi(x)=\int_{c}^{+\infty}f(x,y)\mathrm{d}y,\ x\in I, \label{eq::::--wefw234fsd3480t34t32jrwdr24t}
\end{align}
称\eqref{eq::::--wefw234fsd3480t34t3532jrwdr24t}式为定义在$I$上的含参量$x$的\textbf{无穷限反常积分},或简称\textbf{含参量反常积分}.
\end{definition}

\begin{definition}
若含参量反常积分\eqref{eq::::--wefw234fsd3480t34t3532jrwdr24t}与函数$\varPhi(x)$对任给的正数$\varepsilon$,总存在某一实数$N>c$,使得当$M>N$时,对一切$x\in I$,都有
\begin{align*}
\left| \int_{c}^{M}f(x,y)\mathrm{d}y - \varPhi(x) \right| < \varepsilon,
\end{align*}
即
\begin{align*}
\left| \int_{M}^{+\infty}f(x,y)\mathrm{d}y \right| < \varepsilon,
\end{align*}
则称含参量反常积分\eqref{eq::::--wefw234fsd3480t34t3532jrwdr24t}在$I$上一致收敛于$\varPhi(x)$,或简单地说含参量反常积分\eqref{eq::::--wefw234fsd3480t34t3532jrwdr24t}在$I$\textbf{上一致收敛}.

若对任意$[a,b]\subset I$,含参量反常积分\eqref{eq::::--wefw234fsd3480t34t3532jrwdr24t}在$[a,b]$上一致收敛,则称\eqref{eq::::--wefw234fsd3480t34t3532jrwdr24t}在$I$上\textbf{内闭一致收敛}(若$I=[a,b]$,则内闭一致收敛即一致收敛).
\end{definition}

\begin{theorem}[一致收敛的柯西准则]
含参量反常积分\eqref{eq::::--wefw234fsd3480t34t3532jrwdr24t}在$I$上一致收敛的充要条件是:对任给正数$\varepsilon$,总存在某一实数$M>c$,使得当$A_1,A_2>M$时,对一切$x\in I$,都有
\begin{align*}
\left| \int_{A_1}^{A_2}f(x,y)\mathrm{d}y \right| < \varepsilon. 
\end{align*}

\end{theorem}
\begin{proof}


\end{proof}

\begin{theorem}
含参量反常积分$\int_{c}^{+\infty}f(x,y)\mathrm{d}y$在$I$上一致收敛的充分必要条件是
\begin{align*}
\lim_{A \to +\infty}F(A)=0,
\end{align*}
其中$F(A)=\sup_{x\in I}\left| \int_{A}^{+\infty}f(x,y)\mathrm{d}y \right|$.
\end{theorem}
\begin{proof}


\end{proof}

\begin{theorem}\label{theorem:数学分析--定理19.9}
含参量反常积分\eqref{eq::::--wefw234fsd3480t34t3532jrwdr24t}在$I$上一致收敛的充要条件是:对任一趋于$+\infty$的递增数列$\{A_n\}$(其中$A_1 = c$),函数项级数
\begin{align}
\sum_{n=1}^{\infty} \int_{A_n}^{A_{n+1}} f(x,y)\mathrm{d}y = \sum_{n=1}^{\infty} u_n(x) \label{eq::::--wefw234fsd34802jrw4t}
\end{align}
在$I$上一致收敛.
\end{theorem}
\begin{proof}
{\heiti 必要性} 由\eqref{eq::::--wefw234fsd3480t34t3532jrwdr24t}在$I$上一致收敛,故对任给$\varepsilon > 0$,必存在$M > c$,使当$A'' > A' > M$时,对一切$x \in I$,总有
\begin{align}
\left| \int_{A'}^{A''} f(x,y)\mathrm{d}y \right| < \varepsilon. \label{eq::::--wefw234fsd34802jrw4f34t}
\end{align}
又由$A_n \to +\infty\ (n \to \infty)$,所以对正数$M$,存在正整数$N$,只要当$m > n > N$时,就有$A_m \geqslant A_n > M$.由\eqref{eq::::--wefw234fsd34802jrw4f34t}对一切$x \in I$,就有
\begin{align*}
|u_n(x) + \cdots + u_m(x)| &= \left| \int_{A_m}^{A_{m+1}} f(x,y)\mathrm{d}y + \cdots + \int_{A_n}^{A_{n+1}} f(x,y)\mathrm{d}y \right| \\
&= \left| \int_{A_n}^{A_{m+1}} f(x,y)\mathrm{d}y \right| < \varepsilon.
\end{align*}
这就证明了级数\eqref{eq::::--wefw234fsd34802jrw4t}在$I$上一致收敛.

{\heiti 充分性:}用反证法.假若\eqref{eq::::--wefw234fsd3480t34t3532jrwdr24t}在$I$上不一致收敛,则存在某个正数$\varepsilon_0$,使得对于任何实数$M > c$,存在相应的$A'' > A' > M$和$x' \in I$,使得
\begin{align*}
\left| \int_{A'}^{A''} f(x',y)\mathrm{d}y \right| \geqslant \varepsilon_0.
\end{align*}
现取$M_1 = \max\{1,c\}$,则存在$A_2 > A_1 > M_1$及$x_1 \in I$,使得
\begin{align*}
\left| \int_{A_1}^{A_2} f(x_1,y)\mathrm{d}y \right| \geqslant \varepsilon_0.
\end{align*}
一般地,取$M_n = \max\{n,A_{2(n-1)}\}\ (n \geqslant 2)$,则有$A_{2n} > A_{2n-1} > M_n$及$x_n \in I$,使得
\begin{align}
\left| \int_{A_{2n-1}}^{A_{2n}} f(x_n,y)\mathrm{d}y \right| \geqslant \varepsilon_0. \label{eq::::--wefw234fsd34802jrwdr24t}
\end{align}
由上述所得到的数列$\{A_n\}$是递增数列,且$\lim\limits_{n \to \infty} A_n = +\infty$.现在考察级数
\begin{align*}
\sum_{n=1}^{\infty} u_n(x) = \sum_{n=1}^{\infty} \int_{A_n}^{A_{n+1}} f(x,y)\mathrm{d}y.
\end{align*}
由\eqref{eq::::--wefw234fsd34802jrwdr24t}式知存在正数$\varepsilon_0$,对任何正整数$N$,只要$n > N$,就有某个$x_n \in I$,使得
\begin{align*}
|u_{2n-1}(x_n)| = \left| \int_{A_{2n-1}}^{A_{2n}} f(x_n,y)\mathrm{d}y \right| \geqslant \varepsilon_0.
\end{align*}
这与级数\eqref{eq::::--wefw234fsd34802jrw4t}在$I$上一致收敛的假设矛盾.故含参量反常积分\eqref{eq::::--wefw234fsd3480t34t3532jrwdr24t}在$I$上一致收敛.

\end{proof}

\begin{theorem}[含参量反常积分一致收敛判别法]
\begin{enumerate}
\item \textbf{魏尔斯特拉斯/M判别法} 

设有函数$g(y)$,使得
\[
|f(x,y)| \leqslant g(y),\ (x,y) \in I \times [c, +\infty).
\]
若$\int_{c}^{+\infty} g(y)\mathrm{d}y$收敛,则$\int_{c}^{+\infty} f(x,y)\mathrm{d}y$在$I$上一致收敛.

\item \textbf{狄利克雷判别法} 

设

(i) 对一切实数$N > c$,含参量正常积分
\[
\int_{c}^{N} f(x,y)\mathrm{d}y
\]
对参量$x$在$I$上一致有界,即存在正数$M$,对一切$N > c$及一切$x \in I$,都有
\[
\left| \int_{c}^{N} f(x,y)\mathrm{d}y \right| \leqslant M.
\]

(ii) 对每一个$x \in I$,函数$g(x,y)$为$y$的单调函数,且当$y \to +\infty$时,对参量$x$,$g(x,y)$一致地收敛于0.

则含参量反常积分
\[
\int_{c}^{+\infty} f(x,y)g(x,y)\mathrm{d}y
\]
在$I$上一致收敛.

\item \textbf{阿贝尔判别法} 

设

(i) $\int_{c}^{+\infty} f(x,y)\mathrm{d}y$在$I$上一致收敛.

(ii) 对每一个$x \in I$,函数$g(x,y)$为$y$的单调函数,且对参量$x$,$g(x,y)$在$I$上一致有界.

则含参量反常积分
\[
\int_{c}^{+\infty} f(x,y)g(x,y)\mathrm{d}y
\]
在$I$上一致收敛.
\end{enumerate}
\end{theorem}
\begin{proof}


\end{proof}


\subsubsection{含参量反常积分的性质}

\begin{theorem}[连续性]\label{theorem:含参量反常积分的连续性定理}
设 \( f(x,y) \) 在 \( I \times [c, +\infty) \) 上连续,若含参量反常积分
\begin{align}\label{12}
\Phi(x) = \int_{c}^{+\infty} f(x,y) \, \mathrm{d}y
\end{align}
在 \( I \) 上一致收敛,则 \( \Phi(x) \) 在 \( I \) 上连续。
\end{theorem}
\begin{note}
这个定理也表明,在一致收敛的条件下,极限运算与无穷积分运算可以交换:
\begin{align}\label{eq::::--34802jrw4f3454t}
\lim_{x \to x_0} \int_{c}^{+\infty} f(x,y) \, \mathrm{d}y = \int_{c}^{+\infty} f(x_0,y) \, \mathrm{d}y = \int_{c}^{+\infty} \lim_{x \to x_0} f(x,y) \, \mathrm{d}y.
\end{align}
\end{note}
\begin{proof}
由\refthe{theorem:数学分析--定理19.9},对任一递增且趋于 \( +\infty \) 的数列 \( \{A_n\} \)(\( A_1 = c \)),函数项级数
\begin{align}\label{eq::::--34802jrw4f34t}
\Phi(x) = \sum_{n=1}^{\infty} \int_{A_n}^{A_{n+1}} f(x,y) \, \mathrm{d}y = \sum_{n=1}^{\infty} u_n(x)
\end{align}
在 \( I \) 上一致收敛。又由于 \( f(x,y) \) 在 \( I \times [c, +\infty) \) 上连续,故每个 \( u_n(x) \) 都在 \( I \) 上连续。根据函数项级数的连续性定理,函数 \( \Phi(x) \) 在 \( I \) 上连续。

\end{proof}

\begin{corollary}
设 \( f(x,y) \) 在 \( I \times [c, +\infty) \) 上连续,若 \( \Phi(x) = \int_{c}^{+\infty} f(x,y) \, \mathrm{d}y \) 在 \( I \) 上内闭一致收敛,则 \( \Phi(x) \) 在 \( I \) 上连续。
\end{corollary}

\begin{theorem}[可微性]
设 \( f(x,y) \) 与 \( f_x(x,y) \) 在区域 \( I \times [c, +\infty) \) 上连续。若 \( \Phi(x) = \int_{c}^{+\infty} f(x,y) \, \mathrm{d}y \) 在 \( I \) 上收敛,\( \int_{c}^{+\infty} f_x(x,y) \, \mathrm{d}y \) 在 \( I \) 上一致收敛,则 \( \Phi(x) \) 在 \( I \) 上可微,且
\begin{align}\label{eq::::--34802jrw4f345634t}
\Phi'(x) = \int_{c}^{+\infty} f_x(x,y) \, \mathrm{d}y.
\end{align}
\end{theorem}
\begin{note}
最后结果表明在定理条件下,求导运算和无穷积分运算可以交换。
\end{note}
\begin{proof}
对任一递增且趋于 \( +\infty \) 的数列 \( \{A_n\} \)(\( A_1 = c \)),令
\[
u_n(x) = \int_{A_n}^{A_{n+1}} f(x,y) \, \mathrm{d}y.
\]
由\refthe{theorem:含参量积分的可微性2}推得
\[
u_n'(x) = \int_{A_n}^{A_{n+1}} f_x(x,y) \, \mathrm{d}y.
\]
由 \( \int_{c}^{+\infty} f_x(x,y) \, \mathrm{d}y \) 在 \( I \) 上一致收敛及\refthe{theorem:数学分析--定理19.9},可得函数项级数
\[
\sum_{n=1}^{\infty} u_n'(x) = \sum_{n=1}^{\infty} \int_{A_n}^{A_{n+1}} f_x(x,y) \, \mathrm{d}y
\]
在 \( I \) 上一致收敛,因此根据函数项级数的逐项求导定理,即得
\[
\Phi'(x) = \sum_{n=1}^{\infty} u_n'(x) = \sum_{n=1}^{\infty} \int_{A_n}^{A_{n+1}} f_x(x,y) \, \mathrm{d}y = \int_{c}^{+\infty} f_x(x,y) \, \mathrm{d}y,
\]
或写作
\[
\frac{\mathrm{d}}{\mathrm{d}x} \int_{c}^{+\infty} f(x,y) \, \mathrm{d}y = \int_{c}^{+\infty} \frac{\partial}{\partial x} f(x,y) \, \mathrm{d}y.
\]

\end{proof}

\begin{corollary}
设 \( f(x,y) \) 和 \( f_x(x,y) \) 在 \( I \times [c, +\infty) \) 上连续,若 \( \Phi(x) = \int_{c}^{+\infty} f(x,y) \, \mathrm{d}y \) 在 \( I \) 上收敛,而 \( \int_{c}^{+\infty} f_x(x,y) \, \mathrm{d}y \) 在 \( I \) 上内闭一致收敛,则 \( \Phi(x) \) 在 \( I \) 上可微,且 \( \Phi'(x) = \int_{c}^{+\infty} f_x(x,y) \, \mathrm{d}y \)。
\end{corollary}

\begin{theorem}[可积性]\label{theorem:含参量反常积分可积性定理}
设 \( f(x,y) \) 在 \( [a,b] \times [c, +\infty) \) 上连续,若 \( \Phi(x) = \int_{c}^{+\infty} f(x,y) \, \mathrm{d}y \) 在 \( [a,b] \) 上一致收敛,则 \( \Phi(x) \) 在 \( [a,b] \) 上可积,且
\begin{align}\label{eq::::--wefw234fsd3480t34t32jrwdr24t948543tdhe634}
\int_{a}^{b} \mathrm{d}x \int_{c}^{+\infty} f(x,y) \, \mathrm{d}y = \int_{c}^{+\infty} \mathrm{d}y \int_{a}^{b} f(x,y) \, \mathrm{d}x.
\end{align}
\end{theorem}
\begin{proof}
由\refthe{theorem:含参量反常积分的连续性定理}知道 \( \Phi(x) \) 在 \( [a,b] \) 上连续,从而 \( \Phi(x) \) 在 \( [a,b] \) 上可积。

又由\refthe{theorem:含参量反常积分的连续性定理}的证明中可以看到,函数项级数 \eqref{eq::::--34802jrw4f34t} 在 \( [a,b] \) 上一致收敛,且各项 \( u_n(x) \) 在 \( [a,b] \) 上连续,因此根据函数项级数逐项求积定理,有
\begin{align}\label{eq::::--wefw234fsd3480t34t32jrwdr24t94854}
\int_{a}^{b} \Phi(x) \, \mathrm{d}x = \sum_{n=1}^{\infty} \int_{a}^{b} u_n(x) \, \mathrm{d}x = \sum_{n=1}^{\infty} \int_{a}^{b} \mathrm{d}x \int_{A_n}^{A_{n+1}} f(x,y) \, \mathrm{d}y = \sum_{n=1}^{\infty} \int_{A_n}^{A_{n+1}} \mathrm{d}y \int_{a}^{b} f(x,y) \, \mathrm{d}x,
\end{align}
这里最后一步是根据\refthe{theorem:含参量正常积分的可积性定理}关于积分顺序的可交换性。\eqref{eq::::--wefw234fsd3480t34t32jrwdr24t94854} 式又可写作
\[
\int_{a}^{b} \Phi(x) \, \mathrm{d}x = \int_{c}^{+\infty} \mathrm{d}y \int_{a}^{b} f(x,y) \, \mathrm{d}x.
\]
这就是\eqref{eq::::--wefw234fsd3480t34t32jrwdr24t948543tdhe634}式。

\end{proof}

\begin{theorem}
设 \( f(x,y) \) 在 \( [a, +\infty) \times [c, +\infty) \) 上连续。若

(i)\( \int_{a}^{+\infty} f(x,y) \, \mathrm{d}x \) 关于 \( y \) 在 \( [c, +\infty) \) 上内闭一致收敛,\( \int_{c}^{+\infty} f(x,y) \, \mathrm{d}y \) 关于 \( x \) 在 \( [a, +\infty) \) 上内闭一致收敛。

(ii)积分
\begin{align}\label{eq::::--wefw234fsd3480t34t32jrwdr24t9485344}
\int_{a}^{+\infty} \mathrm{d}x \int_{c}^{+\infty} |f(x,y)| \, \mathrm{d}y \text{ 与 } \int_{c}^{+\infty} \mathrm{d}y \int_{a}^{+\infty} |f(x,y)| \, \mathrm{d}x
\end{align}
中有一个收敛。
则
\begin{align}\label{eq::::--wefw234fsd3480t34t32jrwdr24t948534t34}
\int_{a}^{+\infty} \mathrm{d}x \int_{c}^{+\infty} f(x,y) \, \mathrm{d}y = \int_{c}^{+\infty} \mathrm{d}y \int_{a}^{+\infty} f(x,y) \, \mathrm{d}x.
\end{align}
\end{theorem}
\begin{proof}
不妨设 \eqref{eq::::--wefw234fsd3480t34t32jrwdr24t9485344} 式中第一个积分收敛,由此推得
\[
\int_{a}^{+\infty} \mathrm{d}x \int_{c}^{+\infty} f(x,y) \, \mathrm{d}y
\]
也收敛。当 \( d > c \) 时,
\begin{align*}
J_d &= \left| \int_{c}^{d} \mathrm{d}y \int_{a}^{+\infty} f(x,y) \, \mathrm{d}x - \int_{a}^{+\infty} \mathrm{d}x \int_{c}^{+\infty} f(x,y) \, \mathrm{d}y \right| \\
&= \left| \int_{c}^{d} \mathrm{d}y \int_{a}^{+\infty} f(x,y) \, \mathrm{d}x - \int_{a}^{+\infty} \mathrm{d}x \int_{c}^{d} f(x,y) \, \mathrm{d}y - \int_{a}^{+\infty} \mathrm{d}x \int_{d}^{+\infty} f(x,y) \, \mathrm{d}y \right|.
\end{align*}
根据条件(i)及\refthe{theorem:含参量反常积分可积性定理},可推得
\begin{align}\label{20}
J_d = \left| \int_{a}^{+\infty} \mathrm{d}x \int_{d}^{+\infty} f(x,y) \, \mathrm{d}y \right| \leqslant \left| \int_{a}^{A} \mathrm{d}x \int_{d}^{+\infty} f(x,y) \, \mathrm{d}y \right| + \int_{A}^{+\infty} \mathrm{d}x \int_{d}^{+\infty} |f(x,y)| \, \mathrm{d}y.
\end{align}
由条件(ii),对于任给的 \( \varepsilon > 0 \),有 \( G > a \),使当 \( A > G \) 时,有
\[
\int_{A}^{+\infty} \mathrm{d}x \int_{d}^{+\infty} |f(x,y)| \, \mathrm{d}y < \frac{\varepsilon}{2}.
\]
选定 \( A \) 后,由 \( \int_{c}^{+\infty} f(x,y) \, \mathrm{d}y \) 的内闭一致收敛性,存在 \( M > c \),使得当 \( d > M \) 时有
\[
\left| \int_{d}^{+\infty} f(x,y) \, \mathrm{d}y \right| < \frac{\varepsilon}{2(A - a)}, \quad \forall x \in [a,A].
\]
把这两个结果应用到 \eqref{20} 式,得到
\[
J_d < \frac{\varepsilon}{2} + \frac{\varepsilon}{2} = \varepsilon,
\]
即 \( \lim_{d \to +\infty} J_d = 0 \),这就证明了 \eqref{eq::::--wefw234fsd3480t34t32jrwdr24t948534t34} 式。

\end{proof}

\begin{definition}
设 \( f(x,y) \) 在区域 \( R = [a,b] \times [c,d) \) 上有定义。若对 \( x \) 的某些值,\( y = d \) 为函数 \( f(x,y) \) 的瑕点,则称
\begin{align}\label{eq::::-23535232-wefw234fsd3480t34t32jrwdr24t}
\int_{c}^{d} f(x,y) \mathrm{d}y
\end{align}
为\textbf{含参量 \( \boldsymbol{x} \) 的无界函数反常积分},或简称为含参量反常积分。若对每一个 \( x \in [a,b] \),积分 \eqref{eq::::-23535232-wefw234fsd3480t34t32jrwdr24t} 都收敛,则其积分值是 \( x \) 在 \( [a,b] \) 上取值的函数。
\end{definition}

\begin{definition}
对任给正数 \( \varepsilon \),总存在某正数 \( \delta < d - c \),使得当 \( 0 < \eta < \delta \) 时,对一切 \( x \in [a,b] \),都有
\begin{align*}
\left| \int_{d - \eta}^{d} f(x,y) \mathrm{d}y \right| < \varepsilon,
\end{align*}
则称含参量反常积分 \eqref{eq::::-23535232-wefw234fsd3480t34t32jrwdr24t} 在 \( [a,b] \) 上一致收敛。
\end{definition}
\begin{note}
可参照含参量无穷限反常积分的办法建立相应的含参量无界函数反常积分的一致收敛性判别法,并讨论它们的性质.
\end{note}












\end{document}
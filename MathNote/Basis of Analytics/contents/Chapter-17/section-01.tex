\documentclass[../../main.tex]{subfiles}
\graphicspath{{\subfix{../../image/}}} % 指定图片目录,后续可以直接使用图片文件名。

% 例如:
% \begin{figure}[H]
% \centering
% \includegraphics[scale=0.4]{image-01.01}
% \caption{图片标题}
% \label{figure:image-01.01}
% \end{figure}
% 注意:上述\label{}一定要放在\caption{}之后,否则引用图片序号会只会显示??.

\begin{document}

\section{Fourier级数基本性质}

\begin{theorem}[Fourier级数的逐项积分定理]\label{theorem:Fourier级数的逐项积分定理}
设$f(x)$在$[-\pi,\pi]$上可积或绝对可积,
\begin{align*}
f(x)\sim\frac{a_0}{2}+\sum_{n = 1}^{\infty}(a_n\cos nx + b_n\sin nx),
\end{align*}
则$f(x)$的 Fourier 级数可以逐项积分,即对于任意$c,x\in[-\pi,\pi]$,
\begin{align*}
\int_{c}^{x}f(t)\mathrm{d}t=\int_{c}^{x}\frac{a_0}{2}\mathrm{d}t+\sum_{n = 1}^{\infty}\int_{c}^{x}(a_n\cos nt + b_n\sin nt)\mathrm{d}t.
\end{align*}
\end{theorem}

\begin{theorem}[Fourier级数的逐项微分定理]\label{theorem:Fourier级数的逐项微分定理}
设$f(x)$在$[-\pi,\pi]$上连续,
\begin{align*}
f(x)\sim\frac{a_0}{2}+\sum_{n = 1}^{\infty}(a_n\cos nx + b_n\sin nx),
\end{align*}
$f(-\pi)=f(\pi)$,且除了有限个点外$f(x)$可导.进一步假设$f^{\prime}(x)$在$[-\pi,\pi]$上可积或绝对可积(注意:$f^{\prime}(x)$在有限个点可能无定义,但这并不影响其可积性). 则$f^{\prime}(x)$的 Fourier 级数可由$f(x)$的 Fourier 级数逐项微分得到,即
\begin{align*}
f^{\prime}(x)\sim\frac{\mathrm{d}}{\mathrm{d}x}\left(\frac{a_0}{2}\right)+\sum_{n = 1}^{\infty}\frac{\mathrm{d}}{\mathrm{d}x}(a_n\cos nx + b_n\sin nx)=\sum_{n = 1}^{\infty}(-a_nn\sin nx + b_nn\cos nx).
\end{align*}
\end{theorem}

\begin{corollary}
$\frac{a_0}{2}+\sum_{n = 1}^{\infty}(a_n\cos nx + b_n\sin nx)$是某个在$[-\pi,\pi]$上可积或绝对可积函数的 Fourier 级数的必要条件是$\sum_{n = 1}^{\infty}\frac{b_n}{n}$收敛. 
\end{corollary}

\begin{theorem}[Bessel不等式]\label{theorem:Bessel不等式}
设$f(x)$在$[-\pi,\pi]$上可积或平方可积,则$f(x)$的 Fourier 系数满足不等式
\begin{align*}
\frac{a_0^2}{2}+\sum_{k = 1}^{\infty}(a_k^2 + b_k^2)\leq\frac{1}{\pi}\int_{-\pi}^{\pi}f^2(x)\mathrm{d}x.
\end{align*}
\end{theorem}
\begin{note}
这表示 Fourier 系数的平方组成了一个收敛的级数。
\end{note}

\begin{theorem}[Parseval恒等式]\label{theorem:Parseval恒等式}
设$f(x)$在$[-\pi,\pi]$上可积或平方可积,则$f(x)$的Fourier系数满足恒等式
\begin{align*}
\frac{a_0^2}{2}+\sum_{k = 1}^{\infty}(a_k^2 + b_k^2)=\frac{1}{\pi}\int_{-\pi}^{\pi}f^2(x)\mathrm{d}x.
\end{align*}
\end{theorem}

\begin{lemma}\label{lemma:Fourier级数与其导函数的系数关系}
设$f$为$[-\pi,\pi]$上的连续可微函数,且$f(-\pi)=f(\pi)$.$a_n,b_n$为$f$的Fourier系数,$a_{n}',b_{n}'$为$f$的导函数$f'$的Fourier系数,证明
$$a_{0}'=0,a_{n}'=nb_n,b_{n}'=-na_{n}'(n=1,2,\cdots).$$
\end{lemma}
\begin{remark}
分部积分的条件,需要$f$的导函数$f'$在积分区域上连续.
\end{remark}
\begin{proof}
由于$f$为$[-\pi,\pi]$上的连续可微函数,因此$f'\in C([-\pi,\pi])$.又$f(\pi)=f(-\pi)$,故
\begin{align*}
a_{0}'&=\frac{1}{\pi}\int_{-\pi}^{\pi}f'(x)\mathrm{d}x=\frac{1}{\pi}f(x)\Big|_{-\pi}^{\pi}=0,\\
a_{n}'&=\frac{1}{\pi}\int_{-\pi}^{\pi}f'(x)\cos nx\mathrm{d}x=\frac{1}{\pi}f(x)\cos nx\Big|_{-\pi}^{\pi}+\frac{n}{\pi}\int_{-\pi}^{\pi}f(x)\sin nx\mathrm{d}x=nb_n(n=1,2,\cdots),\\
b_{n}'&=\frac{1}{\pi}\int_{-\pi}^{\pi}f'(x)\sin nx\mathrm{d}x=\frac{1}{\pi}f(x)\sin nx\Big|_{-\pi}^{\pi}-\frac{n}{\pi}\int_{-\pi}^{\pi}f(x)\cos nx\mathrm{d}x=-na_n(n=1,2,\cdots),
\end{align*}
因此结论得证.
\end{proof}

\begin{example}
设$f$以$2\pi$为周期且具有二阶连续的导函数,证明$f$的Fourier级数在$(-\infty,+\infty)$上一致收敛于$f$.
\end{example}
\begin{proof}
因为$f(x)$是以$2\pi$为周期的具有二阶连续导数的函数,故$f(x)$,$f'(x)$可展开成傅里叶级数,不妨设
\begin{align*}
f(x) &=\frac{a_0}{2}+\sum_{n=1}^{\infty}(a_n\cos nx+b_n\sin nx),\quad f'(x)=\frac{a_{0}'}{2}+\sum_{n=1}^{\infty}(a_{n}'\cos nx+b_{n}'\sin nx).
\end{align*}
先证$\frac{|a_0|}{2}+\sum_{n=1}^{\infty}(|a_n|+|b_n|)$收敛.由\reflem{lemma:Fourier级数与其导函数的系数关系}可知
$$a_{0}'=0,a_{n}'=nb_n,b_{n}'=-nb_n(n=1,2,\cdots),$$
从而
\begin{align}
|a_n|+|b_n|=\frac{|b_{n}'|}{n}+\frac{|a_{n}'|}{n}\leqslant \frac{1}{2}\left[(b_{n}')^2+\frac{1}{n^2}\right]+\frac{1}{2}\left[(a_{n}')^2+\frac{1}{n^2}\right]
=\frac{1}{n^2}+\frac{1}{2}\left[(a_{n}')^2+(b_{n}')^2\right].\label{100.103}
\end{align}
又由\hyperref[theorem:Bessel不等式]{Bessel不等式}可知
$$\frac{a_{0}'}{2}+\sum_{n=1}^{\infty}\left[(a_{n}')^2+(b_{n}')^2\right]\leqslant \frac{1}{\pi}\int_{-\pi}^{\pi}[f'(x)]^2\mathrm{d}x<+\infty.$$
故$\sum_{n=1}^{\infty}\left[(a_{n}')^2+(b_{n}')^2\right]$收敛.再结合$\sum_{n=1}^{\infty}\frac{1}{n^2}$收敛及\eqref{100.103}式可知$\sum_{n=1}^{\infty}(|a_n|+|b_n|)$收敛,进而$\frac{|a_0|}{2}+\sum_{n=1}^{\infty}(|a_n|+|b_n|)$收敛.注意到
$$\left|\frac{a_0}{2}+\sum_{n=1}^{\infty}(a_n\cos nx+b_n\sin nx)\right|\leqslant \frac{|a_0|}{2}+\sum_{n=1}^{\infty}(|a_n|+|b_n|),\forall x\in(-\infty,+\infty).$$
因此由Weierstrass判别法可知,$\frac{a_0}{2}+\sum_{n=1}^{\infty}(a_n\cos nx+b_n\sin nx)$在$(-\infty,+\infty)$上一致收敛,即$f$的Fourier级数在$(-\infty,+\infty)$上一致收敛于$f$.
\end{proof}

























\end{document}
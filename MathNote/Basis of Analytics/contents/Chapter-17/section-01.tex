\documentclass[../../main.tex]{subfiles}
\graphicspath{{\subfix{../../image/}}} % 指定图片目录,后续可以直接使用图片文件名。

% 例如:
% \begin{figure}[H]
% \centering
% \includegraphics[scale=0.4]{图.png}
% \caption{}
% \label{figure:图}
% \end{figure}
% 注意:上述\label{}一定要放在\caption{}之后,否则引用图片序号会只会显示??.

\begin{document}

\section{Fourier级数及基本性质}

我们首先需要熟悉傅立叶级数的现代形式:

\begin{definition}\label{definition:定义18.2}
设 $f$ 是周期 $1$ 的的可积函数, 则定义 $f$ 的傅立叶系数为
\[
\hat{f}(m) = \int_{0}^{1} f(x) e^{-2\pi i m x} \mathrm{d}x, m \in \mathbb{Z}.
\]
$f$ 的傅立叶级数为
\begin{align*}
f(x) \sim \sum_{m = -\infty}^{\infty} \hat{f}(m) e^{2\pi i m x}.
\end{align*}
\end{definition}
\begin{remark}
\refdef{definition:定义18.2}中的傅立叶级数\eqref{eq::;23428798797--18.22}不意味着收敛到 $f$ 或者收敛.
\end{remark}

\begin{definition}\label{definition:定义18.3}
对每个 $N \in \mathbb{N}_0$,
\begin{enumerate}
\item 我们称
\begin{align*}
D_N(x) = \sum_{|m| \leqslant N} e^{2\pi i m x} = \frac{\sin((2N + 1)\pi x)}{\sin(\pi x)}. 
\end{align*}
为$\mathbf{Dirichlet}$\textbf{核}.

\item 我们称
\begin{align*}
F_N(x) &= \frac{1}{N + 1}[D_0(x) + D_1(x) + \cdots + D_N(x)] \\
&= \sum_{j = -N}^{N} \left(1 - \frac{|j|}{N + 1}\right) e^{2\pi i j x} \\
&= \frac{1}{N + 1} \left( \frac{\sin(\pi (N + 1) x)}{\sin(\pi x)} \right)^2
\end{align*}
为$\mathbf{Fejér}$\textbf{核}.
\end{enumerate}
\end{definition}
\begin{remark}
\refdef{definition:定义18.3}中的等式关系都是等比数列求和和欧拉公式, 二重求和换序的应用. 我们略去证明
\end{remark}

下面我们在高数框架下给出Dirichlet核和Fejér核, 为了形式上的统一, 我们定义

\begin{definition}
对每个 $n \in \mathbb{N}_0$,
\begin{enumerate}
\item 我们称
\begin{gather}\label{eq:--------::24--18.25}
D_0(x) = 1, D_n(x) = 1 + 2\sum_{k = 1}^{n} \cos kx = \frac{\sin\left(n + \frac{1}{2}\right)x}{\sin\left(\frac{x}{2}\right)}, n = 1, 2, \cdots. 
\end{gather}
为$\mathbf{Dirichlet}$\textbf{核}.

\item 我们称
\begin{align}
F_n(x) = \frac{1}{n + 1}\sum_{j = 0}^{n} D_j(x) = 1 + \frac{2}{n + 1}\sum_{k = 1}^{n} (n - k + 1)\cos kx = \frac{1}{n + 1} \left[ \frac{\sin\frac{(n + 1)x}{2}}{\sin\frac{x}{2}} \right]^2 \label{eq::;23428798797--18.26}
\end{align}
为$\mathbf{Fejér}$\textbf{核}.
\end{enumerate}
\end{definition}
\begin{proof}
证明的关键是如下结论
\begin{conclusion}[三角函数复合等差数列时, 部分和计算方法]
三角函数复合等差数列时, 部分和计算方法可以通过欧拉公式之后用等比数列求和公式或者乘 $\frac{\sin \frac{nx}{2}}{\sin \frac{x}{2}}$ 之后对分子和差化积得到.
\end{conclusion}
\begin{enumerate}
\item 我们有
\begin{align*}
1 + 2\sum_{k = 1}^{n} \cos kx &= 1 + 2\sum_{k = 1}^{n} \frac{\sin \frac{x}{2} \cos kx}{2\sin \frac{x}{2}} = 1 + \frac{1}{\sin \frac{x}{2}} \sum_{k = 1}^{n} [\sin\left(k + \frac{1}{2}\right)x - \sin\left(k - \frac{1}{2}\right)x] \\
&= 1 + \frac{\sin\left(n + \frac{1}{2}\right)x - \sin \frac{x}{2}}{\sin \frac{x}{2}} = \frac{\sin\left(n + \frac{1}{2}\right)x}{\sin \frac{x}{2}}.
\end{align*}
我们证明了式\eqref{eq:--------::24--18.25}式.

\item 我们有
\begin{align*}
F_n(x) &= \frac{1}{n + 1}\sum_{j = 0}^{n} D_j(x) = \frac{1}{n + 1} \left( 1 + \sum_{j = 1}^{n} \left(1 + 2\sum_{k = 1}^{j} \cos kx \right) \right) \\
&= \frac{1}{n + 1} \left( n + 1 + 2\sum_{j = 1}^{n} \sum_{k = 1}^{j} \cos kx \right) = \frac{1}{n + 1} \left( n + 1 + 2\sum_{k = 1}^{n} \sum_{j = k}^{n} \cos kx \right) \\
&= \frac{1}{n + 1} \left( n + 1 + 2\sum_{k = 1}^{n} (n - k + 1)\cos kx \right) = 1 + \frac{2}{n + 1}\sum_{k = 1}^{n} (n - k + 1)\cos kx,
\end{align*}
以及
\begin{align*}
F_n(x) &= \frac{1}{n + 1}\sum_{j = 0}^{n} D_j(x) = \frac{1}{n + 1}\sum_{j = 0}^{n} \frac{\sin\left(j + \frac{1}{2}\right)x}{\sin \frac{x}{2}} \\
&= \frac{1}{n + 1} \cdot \frac{\sin \frac{x}{2}}{\sin^2 \frac{x}{2}} \sum_{j = 0}^{n} \sin\left(j + \frac{1}{2}\right)x = -\frac{1}{2(n + 1)\sin^2 \frac{x}{2}} \sum_{j = 0}^{n} [\cos(j + 1)x - \cos jx] \\
&= -\frac{\cos(n + 1)x - 1}{2(n + 1)\sin^2 \frac{x}{2}} = \frac{\sin^2 \frac{n + 1}{2}x}{(n + 1)\sin^2 \frac{x}{2}}.
\end{align*}
这就证明了\eqref{eq::;23428798797--18.26}式.
\end{enumerate}

\end{proof}

\begin{theorem}[傅立叶部分和积分表达式]\label{theorem:傅立叶部分和积分表达式}
设 $f$ 是周期 $2\pi$ 的可积函数, 其傅立叶系数为 $a_n, b_n$. 记 $S_0(x) = \sigma_0(x) = \frac{a_0}{2}$ 以及
\[
S_n(x) = \frac{a_0}{2} + \sum_{k = 1}^{n} (a_k \cos kx + b_k \sin kx), \sigma_n(x) = \frac{1}{n + 1}\sum_{k = 0}^{n} S_k(x), n = 1, 2, \cdots.
\]
则我们有

$\mathbf{Dirichlet:}$
\[
S_n(x) = \frac{1}{2\pi}\int_{-\pi}^{\pi} f(x + t) D_n(t) \mathrm{d}t = \frac{1}{2\pi}\int_{-\pi}^{\pi} f(x + t) \frac{\sin\left(n + \frac{1}{2}\right)t}{\sin \frac{t}{2}} \mathrm{d}t, n = 0, 1, \cdots.
\]

$\mathbf{Fejér:}$
\[
\sigma_n(x) = \frac{1}{2\pi}\int_{-\pi}^{\pi} f(x + t) F_n(t) \mathrm{d}t = \frac{1}{2(n + 1)\pi}\int_{-\pi}^{\pi} f(x + t) \frac{\sin^2 \frac{n + 1}{2}t}{\sin^2 \frac{t}{2}} \mathrm{d}t, n = 0, 1, \cdots.
\]
\end{theorem}
\begin{note}
根据经验, 取平均性质会更好一些, 因此 Fejér 是一个好核而 Dirichlet 核性质就相当糟糕, 在后面的证明中我们将充分感受到这一点.
\end{note} 
\begin{proof}
当 $n = 0$, 这个定理显然成立. 当 $n > 0$, 一方面, 我们有
\begin{align*}
S_n(x) &= \frac{a_0}{2} + \sum_{k = 1}^{n} (a_k \cos kx + b_k \sin kx) \\
&= \frac{1}{2\pi}\int_{-\pi}^{\pi} f(y) \mathrm{d}y + \frac{1}{\pi}\sum_{k = 1}^{n} \left( \int_{-\pi}^{\pi} f(y) \cos ky \cos kx \mathrm{d}y + \int_{-\pi}^{\pi} f(y) \sin ky \sin kx \mathrm{d}y \right) \\
&= \frac{1}{2\pi}\int_{-\pi}^{\pi} f(y) \mathrm{d}y + \frac{1}{\pi}\sum_{k = 1}^{n} \left( \int_{-\pi}^{\pi} f(y) \cos k(y - x) \mathrm{d}y \right) \\
&\xlongequal{\text{注意周期}2\pi}\frac{1}{2\pi}\int_{-\pi}^{\pi} f(x + y) \mathrm{d}y + \frac{1}{2\pi}\sum_{k = 1}^{n} \left( \int_{-\pi}^{\pi} f(x + y) 2\cos ky\mathrm{d}y \right) \\
&= \frac{1}{2\pi}\int_{-\pi}^{\pi} f(x + y) \left( 1 + 2\sum_{k = 1}^{n} \cos ky \right) \mathrm{d}y \\
&= \frac{1}{2\pi}\int_{-\pi}^{\pi} f(x + y) D_n(y) \mathrm{d}y.
\end{align*}
另外一方面, 我们有
\begin{align*}
\sigma_n(x) &= \frac{1}{n + 1}\sum_{j = 0}^{n} S_j(x) = \frac{1}{n + 1}\sum_{j = 0}^{n} \frac{1}{2\pi}\int_{-\pi}^{\pi} f(x + y) D_j(y) \mathrm{d}y \\
&= \frac{1}{2\pi}\int_{-\pi}^{\pi} f(x + y) \frac{1}{n + 1}\sum_{j = 0}^{n} D_j(y) \mathrm{d}y = \frac{1}{2\pi}\int_{-\pi}^{\pi} f(x + y) F_n(y) \mathrm{d}y.
\end{align*}
这就证明了这个定理.

\end{proof}

\begin{theorem}[Fourier级数的逐项积分定理]\label{theorem:Fourier级数的逐项积分定理}
设$f(x)$在$[-\pi,\pi]$上可积或绝对可积,
\begin{align*}
f(x)\sim\frac{a_0}{2}+\sum_{n = 1}^{\infty}(a_n\cos nx + b_n\sin nx),
\end{align*}
则$f(x)$的 Fourier 级数可以逐项积分,即对于任意$c,x\in[-\pi,\pi]$,
\begin{align*}
\int_{c}^{x}f(t)\mathrm{d}t=\int_{c}^{x}\frac{a_0}{2}\mathrm{d}t+\sum_{n = 1}^{\infty}\int_{c}^{x}(a_n\cos nt + b_n\sin nt)\mathrm{d}t.
\end{align*}
\end{theorem}

\begin{theorem}[Fourier级数的逐项微分定理]\label{theorem:Fourier级数的逐项微分定理}
设$f(x)$在$[-\pi,\pi]$上连续,
\begin{align*}
f(x)\sim\frac{a_0}{2}+\sum_{n = 1}^{\infty}(a_n\cos nx + b_n\sin nx),
\end{align*}
$f(-\pi)=f(\pi)$,且除了有限个点外$f(x)$可导.进一步假设$f^{\prime}(x)$在$[-\pi,\pi]$上可积或绝对可积(注意:$f^{\prime}(x)$在有限个点可能无定义,但这并不影响其可积性). 则$f^{\prime}(x)$的 Fourier 级数可由$f(x)$的 Fourier 级数逐项微分得到,即
\begin{align*}
f^{\prime}(x)\sim\frac{\mathrm{d}}{\mathrm{d}x}\left(\frac{a_0}{2}\right)+\sum_{n = 1}^{\infty}\frac{\mathrm{d}}{\mathrm{d}x}(a_n\cos nx + b_n\sin nx)=\sum_{n = 1}^{\infty}(-a_nn\sin nx + b_nn\cos nx).
\end{align*}
\end{theorem}

\begin{corollary}
$\frac{a_0}{2}+\sum_{n = 1}^{\infty}(a_n\cos nx + b_n\sin nx)$是某个在$[-\pi,\pi]$上可积或绝对可积函数的 Fourier 级数的必要条件是$\sum_{n = 1}^{\infty}\frac{b_n}{n}$收敛. 
\end{corollary}

\begin{theorem}[Bessel不等式]\label{theorem:Bessel不等式}
设$f(x)$在$[-\pi,\pi]$上可积或平方可积,则$f(x)$的 Fourier 系数满足不等式
\begin{align*}
\frac{a_0^2}{2}+\sum_{k = 1}^{\infty}(a_k^2 + b_k^2)\leqslant \frac{1}{\pi}\int_{-\pi}^{\pi}f^2(x)\mathrm{d}x.
\end{align*}
\end{theorem}
\begin{note}
这表示 Fourier 系数的平方组成了一个收敛的级数。
\end{note}

\begin{theorem}[Parseval恒等式]\label{theorem:Parseval恒等式}
设$f(x)$在$[-\pi,\pi]$上可积或平方可积,则$f(x)$的Fourier系数满足恒等式
\begin{align*}
\frac{a_0^2}{2}+\sum_{k = 1}^{\infty}(a_k^2 + b_k^2)=\frac{1}{\pi}\int_{-\pi}^{\pi}f^2(x)\mathrm{d}x.
\end{align*}
\end{theorem}

\begin{lemma}\label{lemma:Fourier级数与其导函数的系数关系}
设$f$为$[-\pi,\pi]$上的连续可微函数,且$f(-\pi)=f(\pi)$.$a_n,b_n$为$f$的Fourier系数,$a_{n}',b_{n}'$为$f$的导函数$f'$的Fourier系数,证明
$$a_{0}'=0,a_{n}'=nb_n,b_{n}'=-na_{n}'(n=1,2,\cdots).$$
\end{lemma}
\begin{remark}
分部积分的条件,需要$f$的导函数$f'$在积分区域上连续.
\end{remark}
\begin{proof}
由于$f$为$[-\pi,\pi]$上的连续可微函数,因此$f'\in C([-\pi,\pi])$.又$f(\pi)=f(-\pi)$,故
\begin{align*}
a_{0}'&=\frac{1}{\pi}\int_{-\pi}^{\pi}f'(x)\mathrm{d}x=\frac{1}{\pi}f(x)\Big|_{-\pi}^{\pi}=0,\\
a_{n}'&=\frac{1}{\pi}\int_{-\pi}^{\pi}f'(x)\cos nx\mathrm{d}x=\frac{1}{\pi}f(x)\cos nx\Big|_{-\pi}^{\pi}+\frac{n}{\pi}\int_{-\pi}^{\pi}f(x)\sin nx\mathrm{d}x=nb_n(n=1,2,\cdots),\\
b_{n}'&=\frac{1}{\pi}\int_{-\pi}^{\pi}f'(x)\sin nx\mathrm{d}x=\frac{1}{\pi}f(x)\sin nx\Big|_{-\pi}^{\pi}-\frac{n}{\pi}\int_{-\pi}^{\pi}f(x)\cos nx\mathrm{d}x=-na_n(n=1,2,\cdots),
\end{align*}
因此结论得证.

\end{proof}

\begin{example}
设$f$以$2\pi$为周期且具有二阶连续的导函数,证明$f$的Fourier级数在$(-\infty,+\infty)$上一致收敛于$f$.
\end{example}
\begin{proof}
因为$f(x)$是以$2\pi$为周期的具有二阶连续导数的函数,故$f(x)$,$f'(x)$可展开成傅里叶级数,不妨设
\begin{align*}
f(x) &=\frac{a_0}{2}+\sum_{n=1}^{\infty}(a_n\cos nx+b_n\sin nx),\quad f'(x)=\frac{a_{0}'}{2}+\sum_{n=1}^{\infty}(a_{n}'\cos nx+b_{n}'\sin nx).
\end{align*}
先证$\frac{|a_0|}{2}+\sum_{n=1}^{\infty}(|a_n|+|b_n|)$收敛.由\reflem{lemma:Fourier级数与其导函数的系数关系}可知
$$a_{0}'=0,a_{n}'=nb_n,b_{n}'=-nb_n(n=1,2,\cdots),$$
从而
\begin{align}
|a_n|+|b_n|=\frac{|b_{n}'|}{n}+\frac{|a_{n}'|}{n}\leqslant \frac{1}{2}\left[(b_{n}')^2+\frac{1}{n^2}\right]+\frac{1}{2}\left[(a_{n}')^2+\frac{1}{n^2}\right]
=\frac{1}{n^2}+\frac{1}{2}\left[(a_{n}')^2+(b_{n}')^2\right].\label{100.103}
\end{align}
又由\hyperref[theorem:Bessel不等式]{Bessel不等式}可知
$$\frac{a_{0}'}{2}+\sum_{n=1}^{\infty}\left[(a_{n}')^2+(b_{n}')^2\right]\leqslant \frac{1}{\pi}\int_{-\pi}^{\pi}[f'(x)]^2\mathrm{d}x<+\infty.$$
故$\sum_{n=1}^{\infty}\left[(a_{n}')^2+(b_{n}')^2\right]$收敛.再结合$\sum_{n=1}^{\infty}\frac{1}{n^2}$收敛及\eqref{100.103}式可知$\sum_{n=1}^{\infty}(|a_n|+|b_n|)$收敛,进而$\frac{|a_0|}{2}+\sum_{n=1}^{\infty}(|a_n|+|b_n|)$收敛.注意到
$$\left|\frac{a_0}{2}+\sum_{n=1}^{\infty}(a_n\cos nx+b_n\sin nx)\right|\leqslant \frac{|a_0|}{2}+\sum_{n=1}^{\infty}(|a_n|+|b_n|),\forall x\in(-\infty,+\infty).$$
因此由Weierstrass判别法可知,$\frac{a_0}{2}+\sum_{n=1}^{\infty}(a_n\cos nx+b_n\sin nx)$在$(-\infty,+\infty)$上一致收敛,即$f$的Fourier级数在$(-\infty,+\infty)$上一致收敛于$f$.

\end{proof}

























\end{document}
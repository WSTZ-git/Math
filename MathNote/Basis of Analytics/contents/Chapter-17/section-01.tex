\documentclass[../../main.tex]{subfiles}
\graphicspath{{\subfix{../../image/}}} % 指定图片目录,后续可以直接使用图片文件名。

% 例如:
% \begin{figure}[H]
% \centering
% \includegraphics{image-01.01}
% \caption{图片标题}
% \label{figure:image-01.01}
% \end{figure}
% 注意:上述\label{}一定要放在\caption{}之后,否则引用图片序号会只会显示??.

\begin{document}

\section{Fourier级数基本性质}

\begin{theorem}[Fourier级数的逐项积分定理]\label{theorem:Fourier级数的逐项积分定理}
设$f(x)$在$[-\pi,\pi]$上可积或绝对可积,
\begin{align*}
f(x)\sim\frac{a_0}{2}+\sum_{n = 1}^{\infty}(a_n\cos nx + b_n\sin nx),
\end{align*}
则$f(x)$的 Fourier 级数可以逐项积分,即对于任意$c,x\in[-\pi,\pi]$,
\begin{align*}
\int_{c}^{x}f(t)\mathrm{d}t=\int_{c}^{x}\frac{a_0}{2}\mathrm{d}t+\sum_{n = 1}^{\infty}\int_{c}^{x}(a_n\cos nt + b_n\sin nt)\mathrm{d}t.
\end{align*}
\end{theorem}

\begin{theorem}[Fourier级数的逐项微分定理]\label{theorem:Fourier级数的逐项微分定理}
设$f(x)$在$[-\pi,\pi]$上连续,
\begin{align*}
f(x)\sim\frac{a_0}{2}+\sum_{n = 1}^{\infty}(a_n\cos nx + b_n\sin nx),
\end{align*}
$f(-\pi)=f(\pi)$,且除了有限个点外$f(x)$可导.进一步假设$f^{\prime}(x)$在$[-\pi,\pi]$上可积或绝对可积(注意:$f^{\prime}(x)$在有限个点可能无定义,但这并不影响其可积性). 则$f^{\prime}(x)$的 Fourier 级数可由$f(x)$的 Fourier 级数逐项微分得到,即
\begin{align*}
f^{\prime}(x)\sim\frac{\mathrm{d}}{\mathrm{d}x}\left(\frac{a_0}{2}\right)+\sum_{n = 1}^{\infty}\frac{\mathrm{d}}{\mathrm{d}x}(a_n\cos nx + b_n\sin nx)=\sum_{n = 1}^{\infty}(-a_nn\sin nx + b_nn\cos nx).
\end{align*}
\end{theorem}

\begin{corollary}
$\frac{a_0}{2}+\sum_{n = 1}^{\infty}(a_n\cos nx + b_n\sin nx)$是某个在$[-\pi,\pi]$上可积或绝对可积函数的 Fourier 级数的必要条件是$\sum_{n = 1}^{\infty}\frac{b_n}{n}$收敛. 
\end{corollary}

\begin{theorem}[Bessel不等式]\label{theorem:Bessel不等式}
设$f(x)$在$[-\pi,\pi]$上可积或平方可积,则$f(x)$的 Fourier 系数满足不等式
\begin{align*}
\frac{a_0^2}{2}+\sum_{k = 1}^{\infty}(a_k^2 + b_k^2)\leq\frac{1}{\pi}\int_{-\pi}^{\pi}f^2(x)dx.
\end{align*}
\end{theorem}
\begin{note}
这表示 Fourier 系数的平方组成了一个收敛的级数。
\end{note}

\begin{theorem}[Parseval恒等式]\label{theorem:Parseval恒等式}
设$f(x)$在$[-\pi,\pi]$上可积或平方可积,则$f(x)$的Fourier系数满足恒等式
\begin{align*}
\frac{a_0^2}{2}+\sum_{k = 1}^{\infty}(a_k^2 + b_k^2)=\frac{1}{\pi}\int_{-\pi}^{\pi}f^2(x)dx.
\end{align*}
\end{theorem}


























\end{document}
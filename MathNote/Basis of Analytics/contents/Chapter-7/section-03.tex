\documentclass[../../main.tex]{subfiles}
\graphicspath{{\subfix{../../image/}}} % 指定图片目录,后续可以直接使用图片文件名。

% 例如:
% \begin{figure}[H]
% \centering
% \includegraphics[scale=0.4]{image-01.01}
% \caption{图片标题}
% \label{figure:image-01.01}
% \end{figure}
% 注意:上述\label{}一定要放在\caption{}之后,否则引用图片序号会只会显示??.

\begin{document}

\section{基本组合学公式}

\begin{definition}
对 \( \forall m \in \mathbb{R} \),\( k \in \mathbb{N} \),定义
\[
\mathrm{C}_{m}^{k} = \begin{pmatrix} m \\ k \end{pmatrix} \triangleq \frac{m(m - 1) \cdots (m - k + 1)}{k!}.
\]
特别地,\( \mathrm{C}_{m}^{0} \triangleq 1 \)。若 \( m, k \in \mathbb{N} \),则还有
\[
\mathrm{C}_{m}^{k} = \begin{pmatrix} m \\ k \end{pmatrix} = \frac{m!}{k! (m - k)!}.
\]
\end{definition}

\begin{theorem}[二项式定理的推广]\label{theorem:二项式定理的推广}
$\left( a_1+b_1 \right)\cdots \left( a_n+b_n \right) =\sum_{I\subset \left\{ 1,2,\cdots ,n \right\}}{\left( \prod_{i\in I}{a_i}\prod_{j\in \left\{ 1,2,\cdots ,n \right\} -I}{b_j} \right)}.$
\end{theorem}
\begin{proof}
用数学归纳法证明即可.
\end{proof}


\end{document}
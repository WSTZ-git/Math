\documentclass[../../main.tex]{subfiles}
\graphicspath{{\subfix{../../image/}}} % 指定图片目录,后续可以直接使用图片文件名。

% 例如:
% \begin{figure}[H]
% \centering
% \includegraphics[scale=0.4]{图.png}
% \caption{}
% \label{figure:图}
% \end{figure}
% 注意:上述\label{}一定要放在\caption{}之后,否则引用图片序号会只会显示??.

\begin{document}

\section{重积分换元法}

\begin{theorem}[重积分换元法]\label{theorem:重积分换元法}
\begin{enumerate}
\item 考虑换元 $w:\begin{cases}x = x(u,v)\\y = y(u,v)\end{cases}$ ,则有
\begin{align*}
\iint_D f(x,y) \mathrm{d}x\mathrm{d}y= \iint_{w^{-1}(D)} f(x(u,v),y(u,v)) \cdot |\det J| dudv,
\end{align*}
这里
\begin{align*}
J \triangleq \frac{\partial(x,y)}{\partial(u,v)} = {\textstyle \begin{pmatrix}\frac{\partial x}{\partial u}&\frac{\partial x}{\partial v}\\\frac{\partial y}{\partial u}&\frac{\partial y}{\partial v}\end{pmatrix}}.
\end{align*}
\item 考虑换元 $w:\begin{cases}x = x(u,v,w)\\y = y(u,v,w)\\z = z(u,v,w)\end{cases}$ ,则有
\begin{align*}
\iiint_D f(x,y,z) \mathrm{d}x\mathrm{d}y\mathrm{d}z = \iiint_{w^{-1}(D)} f(x(u,v,w),y(u,v,w),z(u,v,w)) \cdot |\det J| \mathrm{d}u\mathrm{d}v\mathrm{d}w,
\end{align*}
这里
\begin{align*}
J \triangleq \frac{\partial(x,y,z)}{\partial(u,v,w)} = \begin{pmatrix}\frac{\partial x}{\partial u}&\frac{\partial x}{\partial v}&\frac{\partial x}{\partial w}\\\frac{\partial y}{\partial u}&\frac{\partial y}{\partial v}&\frac{\partial y}{\partial w}\\\frac{\partial z}{\partial u}&\frac{\partial z}{\partial v}&\frac{\partial z}{\partial w}\end{pmatrix}.
\end{align*}
\end{enumerate}
\end{theorem}
\begin{note}
\begin{enumerate}
\item 记忆结论
\begin{align*}
\left[\frac{\partial(u,v)}{\partial(x,y)}\right]^{-1} = \frac{\partial(x,y)}{\partial(u,v)}, \quad \left[\frac{\partial(u,v,w)}{\partial(x,y,z)}\right]^{-1} = \frac{\partial(x,y,z)}{\partial(u,v,w)},
\end{align*}
于是有
\begin{align*}
\left|\det \frac{\partial(u,v)}{\partial(x,y)}\right| \cdot \left|\det \frac{\partial(x,y)}{\partial(u,v)}\right| = 1,\quad  \left|\det \frac{\partial(u,v,w)}{\partial(x,y,z)}\right| \cdot \left|\det \frac{\partial(x,y,z)}{\partial(u,v,w)}\right| = 1.
\end{align*}

\item 换元法的区域确定重点是:\textbf{边界对应边界}.

\item 极坐标,柱坐标,球坐标等方法是换元法的特例,我们课本上记忆的经典确定上下限的几何直观是他们独有的. 对于一般没有,或者很难有几何意义的换元法,只能把换元后的新变量用直角坐标的方法确定上下限. 特别的,极坐标,柱坐标,球坐标等方法也可以视换元后的新变量用直角坐标的方法确定上下限.
\end{enumerate}
\end{note}






























































































\end{document}
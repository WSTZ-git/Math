\documentclass[../../main.tex]{subfiles}% 注意这里的文件路径不能用 ./main.tex ,否则用latexmk编译子文件会报错
\graphicspath{{\subfix{./image/}}} % 指定图片目录,后续可以直接使用图片文件名
% 注意这里的文件路径不能用 ../../image/ ,否则用latexmk编译子文件会报错

% 例如:
% \begin{figure}[H]
% \centering
% \includegraphics[scale=0.3]{图.png}
% \caption{}
% \label{figure:图}
% \end{figure}
% 注意:上述\label{}一定要放在\caption{}之后,否则引用图片序号会只会显示??.

\begin{document}

\section{隐函数}

\begin{definition}
设$E \subset \mathbf{R}^2$,函数$F:E \to \mathbf{R}$.对于方程
\begin{align}
F(x,y) = 0, \label{eq::---uifn89h821}
\end{align}
如果存在集合$I,J \subset \mathbf{R}$,对任何$x \in I$,有惟一确定的$y \in J$,使得$(x,y) \in E$,且满足方程\eqref{eq::---uifn89h821},则称方程\eqref{eq::---uifn89h821}确定了一个定义在$I$上,值域含于$J$的\textbf{隐函数}.若把它记为
\[
y = f(x) \quad , x \in I, y \in J,
\]
则成立恒等式
\[
F(x,f(x)) \equiv 0,\ x \in I.
\]
\end{definition}

\begin{theorem}[隐函数存在唯一性定理]\label{theorem:隐函数存在唯一性定理}
若函数$F(x,y)$满足下列条件:
\begin{enumerate}[(i)]
\item $F$在以$P_0(x_0,y_0)$为内点的某一区域$D \subseteq \mathbf{R}^2$上连续;
\item $F(x_0,y_0) = 0$(通常称为初始条件);
\item $F$在$D$上存在连续的偏导数$F_y(x,y)$;
\item $F_y(x_0,y_0) \neq 0$.
\end{enumerate}
则

$1^\circ$ 存在点$P_0$的某邻域$U(P_0) \subseteq D$,在$U(P_0)$上方程$F(x,y) = 0$惟一地决定了一个定义在某区间$(x_0-\alpha,x_0+\alpha)$上的(隐)函数$y = f(x)$,使得当$x \in (x_0-\alpha,x_0+\alpha)$时,$(x,f(x)) \in U(P_0)$,且$F(x,f(x)) \equiv 0$,$f(x_0) = y_0$;

$2^\circ$ $f(x)$在$(x_0-\alpha,x_0+\alpha)$上连续.
\end{theorem}

\begin{proof}
先证隐函数$f$的存在性与惟一性.

由条件(iv),不妨设$F_y(x_0,y_0) > 0$(若$F_y(x_0,y_0) < 0$,则可讨论$-F(x,y) = 0$).由条件(iii)$F_y$在$D$上连续,由连续函数的局部保号性,存在点$P_0$的某一闭的方邻域$[x_0-\beta,x_0+\beta] \times [y_0-\beta,y_0+\beta] \subseteq D$,使得在其上每一点都有$F_y(x,y) > 0$.因而,对每个固定的$x \in [x_0-\beta,x_0+\beta]$,$F(x,y)$作为$y$的一元函数,必定在$[y_0-\beta,y_0+\beta]$上严格增且连续.由初始条件(ii)可知
\[
F(x_0,y_0 - \beta) < 0,\ F(x_0,y_0 + \beta) > 0.
\]
再由$F$的连续性条件(i),又可知道$F(x,y_0-\beta)$与$F(x,y_0+\beta)$在$[x_0-\beta,x_0+\beta]$上也是连续的.因此由保号性存在$\alpha > 0$($\alpha \leqslant \beta$),当$x \in (x_0-\alpha,x_0+\alpha)$时恒有
\[
F(x,y_0 - \beta) < 0,\ F(x,y_0 + \beta) > 0.
\]
\begin{figure}[H]
\centering
\includegraphics[scale=0.3]{数分图18-2.png}
\caption{}
\label{figure:数分图18-2}
\end{figure}
如\reffig{figure:数分图18-2}所示,在矩形$ABB'A'$的$AB$边上$F$取负值,在$A'B'$边上$F$取正值.因此对$(x_0-\alpha,x_0+\alpha)$上每个固定值$\bar{x}$,同样有$F(\bar{x},y_0 - \beta) < 0$,$F(\bar{x},y_0 + \beta) > 0$.根据前已指出的$F(\bar{x},y)$在$[y_0-\beta,y_0+\beta]$上严格增且连续,由介值性定理知存在惟一的$\bar{y} \in (y_0-\beta,y_0+\beta)$,满足$F(\bar{x},\bar{y}) = 0$.由$\bar{x}$在$(x_0-\alpha,x_0+\alpha)$中的任意性,这就证明了存在惟一的一个隐函数$y = f(x)$,它的定义域为$(x_0-\alpha,x_0+\alpha)$,值域含于$(y_0-\beta,y_0+\beta)$.若记
\[
U(P_0) = (x_0 - \alpha,x_0 + \alpha) \times (y_0 - \beta,y_0 + \beta),
\]
则$y = f(x)$在$U(P_0)$上满足结论$1^\circ$的各项要求.

再证明$f$的连续性.

对于$(x_0-\alpha,x_0+\alpha)$上的任意点$\bar{x}$,$\bar{y} = f(\bar{x})$.由上述结论可知$y_0-\beta < \bar{y} < y_0+\beta$.任给$\varepsilon > 0$,且$\varepsilon$足够小,使得
\[
y_0 - \beta \leqslant \bar{y} - \varepsilon < \bar{y} < \bar{y} + \varepsilon \leqslant y_0 + \beta.
\]
由$F(\bar{x},\bar{y}) = 0$及$F(x,y)$关于$y$严格递增,可得$F(\bar{x},\bar{y}-\varepsilon) < 0$,$F(\bar{x},\bar{y}+\varepsilon) > 0$.根据保号性,知存在$\bar{x}$的某邻域$(\bar{x}-\delta,\bar{x}+\delta) \subset (x_0-\alpha,x_0+\alpha)$,使得当$x \in (\bar{x}-\delta,\bar{x}+\delta)$时同样有
\[
F(x,\bar{y}-\varepsilon) < 0,\ F(x,\bar{y}+\varepsilon) > 0.
\]
因此存在惟一的$y$,使得$F(x,y) = 0$,即$y = f(x)$,$|y - \bar{y}| < \varepsilon$.这就证明了当$|x - \bar{x}| < \delta$时,$|f(x) - f(\bar{x})| < \varepsilon$,即$f(x)$在$\bar{x}$连续.由$\bar{x}$的任意性,可得$f(x)$在$(x_0-\alpha,x_0+\alpha)$上连续.

\end{proof}

\begin{theorem}[隐函数可微性定理]\label{theorem:隐函数可微性定理}
设函数$F(x,y)$满足下列条件:
\begin{enumerate}[(i)]
\item $F$在以$P_0(x_0,y_0)$为内点的某一区域$D \subseteq \mathbf{R}^2$上连续;
\item $F(x_0,y_0) = 0$(通常称为初始条件);
\item $F$在$D$上存在连续的偏导数$F_y(x,y)$;
\item $F_y(x_0,y_0) \neq 0$.
\end{enumerate}
又设在$D$上还存在连续的偏导数$F_x(x,y)$,则由方程(1)所确定的隐函数$y=f(x)$在其定义域$(x_0-\alpha,x_0+\alpha)$上有连续导函数,且
\begin{align*}
f'(x) = -\frac{F_x(x,y)}{F_y(x,y)}. 
\end{align*}
\end{theorem}
\begin{proof}
设$x$与$x+\Delta x$都属于$(x_0-\alpha,x_0+\alpha)$,它们所对应的函数值$y=f(x)$与$y+\Delta y=f(x+\Delta x)$都含于$(y_0-\beta,y_0+\beta)$内.由于
\begin{align*}
F(x,y) = 0,\ F(x+\Delta x,y+\Delta y) = 0,
\end{align*}
因此由$F_x,F_y$的连续性以及二元函数中值定理,有
\begin{align*}
0 = F(x+\Delta x,y+\Delta y) - F(x,y) = F_x(x+\theta\Delta x,y+\theta\Delta y)\Delta x + F_y(x+\theta\Delta x,y+\theta\Delta y)\Delta y,
\end{align*}
其中$0 < \theta < 1$.因而
\begin{align*}
\frac{\Delta y}{\Delta x} = -\frac{F_x(x+\theta\Delta x,y+\theta\Delta y)}{F_y(x+\theta\Delta x,y+\theta\Delta y)}.
\end{align*}
注意到上式右端是连续函数$F_x(x,y),F_y(x,y)$与$f(x)$的复合函数,而且$F_y(x,y)$在$U(P_0)$上不等于零,故有
\begin{align*}
f'(x) = \lim_{\Delta x \to 0} \frac{\Delta y}{\Delta x} = -\frac{F_x(x,y)}{F_y(x,y)}.
\end{align*}
且$f'(x)$在$(x_0-\alpha,x_0+\alpha)$上连续.


\end{proof}

\begin{theorem}
若
\begin{enumerate}[(i)]
\item 函数$F(x_1,x_2,\cdots,x_n,y)$在以点$P_0(x_1^0,x_2^0,\cdots,x_n^0,y^0)$为内点的区域$D \subseteq \mathbf{R}^{n+1}$上连续;
\item $F(x_1^0,x_2^0,\cdots,x_n^0,y^0) = 0$;
\item 偏导数$F_{x_1},F_{x_2},\cdots,F_{x_n},F_y$在$D$上存在且连续;
\item $F_y(x_1^0,x_2^0,\cdots,x_n^0,y^0) \neq 0$.
\end{enumerate}
则

$1^\circ$ 存在点$P_0$的某邻域$U(P_0) \subseteq D$,在$U(P_0)$上方程$F(x_1,\cdots,x_n,y) = 0$惟一地确定了一个定义在$Q_0(x_1^0,x_2^0,\cdots,x_n^0)$的某邻域$U(Q_0) \subseteq \mathbf{R}^n$上的$n$元连续函数(隐函数)$y = f(x_1,\cdots,x_n)$,使得当$(x_1,x_2,\cdots,x_n) \in U(Q_0)$时,
\[
(x_1,x_2,\cdots,x_n,f(x_1,x_2,\cdots,x_n)) \in U(P_0),
\]
且
\[
F(x_1,\cdots,x_n,f(x_1,\cdots,x_n)) \equiv 0,
\]
\[
y^0 = f(x_1^0,\cdots,x_n^0);
\]

$2^\circ$ $y = f(x_1,\cdots,x_n)$在$U(Q_0)$上有连续偏导数$f_{x_1},f_{x_2},\cdots,f_{x_n}$,而且
\begin{align*}
f_{x_1} = -\frac{F_{x_1}}{F_y},\ f_{x_2} = -\frac{F_{x_2}}{F_y},\ \cdots,\ f_{x_n} = -\frac{F_{x_n}}{F_y}.
\end{align*}
\end{theorem}

\begin{theorem}[反函数的存在性及其导数]
设$y = f(x)$在$x_0$的某邻域上有连续的导函数$f'(x)$,且$f(x_0) = y_0$,则在$y_0$的某邻域$U(y_0)$上的连续可微隐函数$x = g(y)$,并称它为函数$y = f(x)$的\textbf{反函数}.反函数的导数是
\begin{align*}
g'(y) = -\frac{F_y}{F_x} = -\frac{1}{-f'(x)} = \frac{1}{f'(x)}. 
\end{align*}
\end{theorem}
\begin{proof}
考虑方程
\begin{align}
F(x,y) = y - f(x) = 0. \label{eq:::--2890rj231111}
\end{align}
由于
\begin{align*}
F(x_0,y_0) = 0,\ F_y = 1,\ F_x(x_0,y_0) = -f'(x_0),
\end{align*}
所以只要$f'(x_0) \neq 0$,就能满足\hyperref[theorem:隐函数存在唯一性定理]{隐函数存在唯一性定理}的所有条件,这时方程\eqref{eq:::--2890rj231111}能确定出在$y_0$的某邻域$U(y_0)$上的连续可微隐函数$x = g(y)$.再由\hyperref[theorem:隐函数可微性定理]{隐函数可微性定理}可得
\begin{align*}
g'(y) = -\frac{F_y}{F_x} = -\frac{1}{-f'(x)} = \frac{1}{f'(x)}. 
\end{align*}

\end{proof}



\subsection{隐函数组}

\begin{definition}
设有方程组
\begin{align*}
\begin{cases} 
F(x,y,u,v) = 0, \\
G(x,y,u,v) = 0, 
\end{cases} 
\end{align*}
其中$F(x,y,u,v),G(x,y,u,v)$为定义在$V \subseteq \mathbf{R}^4$上的4元函数.若存在平面区域$D$, $E \subseteq \mathbf{R}^2$,对于$D$中每一点$(x,y)$,有惟一的$(u,v) \in E$,使得$(x,y,u,v) \in V$,且满足方程组\eqref{1},则称由方程组\eqref{1}确定了\textbf{隐函数组}
\begin{align*}
\begin{cases} 
u = f(x,y), \\
v = g(x,y), 
\end{cases} \quad (x,y) \in D, \quad (u,v) \in E,
\end{align*}
并在$D$上成立恒等式
\begin{align*}
\begin{cases} 
F(x,y,f(x,y),g(x,y)) \equiv 0, \\
G(x,y,f(x,y),g(x,y)) \equiv 0, 
\end{cases} \quad (x,y) \in D.
\end{align*}
若还有\eqref{1}中的函数$F$与$G$是可微的,而且由\eqref{1}所确定的两个隐函数$u$与$v$也是可微的,则称
\begin{align*}
\begin{vmatrix} 
F_u & F_v \\
G_u & G_v 
\end{vmatrix} 
\end{align*}
为函数$F,G$关于变量$u,v$的\textbf{函数行列式}(或\textbf{雅可比($\mathbf{Jacobi}$)行列式}),亦可记作$\frac{\partial(F,G)}{\partial(u,v)}$.
\end{definition}
\begin{remark}
若\eqref{1}中的函数$F$与$G$是可微的,而且由\eqref{1}所确定的两个隐函数$u$与$v$也是可微的.那么通过对方程组\eqref{1}关于$x,y$分别求偏导数,得到
\begin{align}
\begin{cases} 
F_x + F_u u_x + F_v v_x = 0, \\
G_x + G_u u_x + G_v v_x = 0, 
\end{cases} \label{eq::--ruin2f32jr23}
\end{align}
\begin{align}
\begin{cases} 
F_y + F_u u_y + F_v v_y = 0, \\
G_y + G_u u_y + G_v v_y = 0. 
\end{cases} \label{eq::--fjio2ji}
\end{align}
要想从\eqref{eq::--ruin2f32jr23}解出$u_x$与$v_x$,从\eqref{eq::--fjio2ji}解出$u_y$与$v_y$,其充分条件是它们的系数行列式不为零,即
\begin{align*}
\frac{\partial(F,G)}{\partial(u,v)}=\begin{vmatrix} 
F_u & F_v \\
G_u & G_v 
\end{vmatrix} \neq 0.
\end{align*}
\end{remark}


\begin{theorem}
若
\begin{enumerate}[(i)]
\item $F(x,y,u,v)$与$G(x,y,u,v)$在以点$P_0(x_0,y_0,u_0,v_0)$为内点的区域$V \subseteq \mathbf{R}^4$上连续;
\item $F(x_0,y_0,u_0,v_0) = 0, G(x_0,y_0,u_0,v_0) = 0$(初始条件);
\item 在$V$上$F,G$具有一阶连续偏导数;
\item $J = \frac{\partial(F,G)}{\partial(u,v)}$在点$P_0$不等于零.
\end{enumerate}
则

$1^\circ$ 存在点$P_0$的某一(四维空间)邻域$U(P_0) \subseteq V$,在$U(P_0)$上方程组\eqref{1}惟一地确定了定义在点$Q_0(x_0,y_0)$的某一(二维空间)邻域$U(Q_0)$上的两个二元隐函数
\[
u = f(x,y),\ v = g(x,y),
\]
使得$u_0 = f(x_0,y_0), v_0 = g(x_0,y_0)$,且当$(x,y) \in U(Q_0)$时,
\[
(x,y,f(x,y),g(x,y)) \in U(P_0),
\]
\[
F(x,y,f(x,y),g(x,y)) \equiv 0,
\]
\[
G(x,y,f(x,y),g(x,y)) \equiv 0;
\]

$2^\circ$ $f(x,y),g(x,y)$在$U(Q_0)$上连续;

$3^\circ$ $f(x,y),g(x,y)$在$U(Q_0)$上有一阶连续偏导数,且
\begin{align}
\frac{\partial u}{\partial x} = -\frac{1}{J}\frac{\partial(F,G)}{\partial(x,v)},\ \frac{\partial v}{\partial x} = -\frac{1}{J}\frac{\partial(F,G)}{\partial(u,x)},\quad \label{5} \\
\frac{\partial u}{\partial y} = -\frac{1}{J}\frac{\partial(F,G)}{\partial(y,v)},\ \frac{\partial v}{\partial y} = -\frac{1}{J}\frac{\partial(F,G)}{\partial(u,y)}. \nonumber
\end{align}
\end{theorem}

\begin{definition}
设函数组
\begin{align}
u = u(x,y),\ v = v(x,y) \label{eq::iohdfjio2jr2j1-9}
\end{align}
是定义在$xy$平面点集$B \subset \mathbf{R}^2$上的两个函数,对每一点$P(x,y) \in B$,由方程组\eqref{eq::iohdfjio2jr2j1-9}有$uv$平面上惟一的一点$Q(u,v) \in \mathbf{R}^2$与之对应.我们称方程组\eqref{eq::iohdfjio2jr2j1-9}确定了$B$到$\mathbf{R}^2$的一个\textbf{映射(变换)},记作$T$.这时映射\eqref{eq::iohdfjio2jr2j1-9}可写成如下函数形式:
\[
T: B \to \mathbf{R}^2,
\]
\[
P(x,y) \mapsto Q(u,v)
\]
或写成点函数形式$Q = T(P), P \in B$,并称$Q(u,v)$为映射$T$下$P(x,y)$的\textbf{象},而$P$则是$Q$的\textbf{原象}.记$B$在映射$T$下的象集为$B' = T(B)$.

反过来,若$T$为一一映射(即不仅每一原象只对应一个象,而且不同的原象对应不同的象).这时每一点$Q \in B'$,由方程组\eqref{eq::iohdfjio2jr2j1-9}都有惟一的一点$P \in B$与之相对应.由此所产生的新映射称为映射$T$的\textbf{逆映射(逆变换)},记作$T^{-1}$,即
\[
T^{-1}: B' \to B,
\]
\[
Q \mapsto P
\]
或
\[
P = T^{-1}(Q), Q \in B'.
\]
亦即存在定义在$B'$上的一个函数组
\begin{align}
x = x(u,v),\ y = y(u,v), \label{eq::iohdfjio2jr2j1-119}
\end{align}
把它代入\eqref{eq::iohdfjio2jr2j1-9}而成为恒等式:
\begin{align}
u \equiv u(x(u,v),y(u,v)),\ v \equiv v(x(u,v),y(u,v)), \label{eq::iohdfjio2jr2j1-1129}
\end{align}
这时我们又称函数组\eqref{eq::iohdfjio2jr2j1-119}是函数组\eqref{eq::iohdfjio2jr2j1-9}的\textbf{反函数组}.
\end{definition}

\begin{theorem}\label{theorem:反函数组可微定理}
设函数组
\begin{align}\label{eq::-2r2jfw9}
u = u(x,y),\ v = v(x,y)
\end{align}
及其一阶偏导数在某区域$D \subseteq \mathbf{R}^2$上连续,点$P_0(x_0,y_0)$是$D$的内点,且
\[
u_0 = u(x_0,y_0),\ v_0 = v(x_0,y_0),\ \left. \frac{\partial(u,v)}{\partial(x,y)} \right|_{P_0} \neq 0,
\]
则在点$P'_0(u_0,v_0)$的某一邻域$U(P'_0)$上存在惟一的一组反函数
\begin{align}\label{eq::-2r2jfw91}
x = x(u,v),\ y = y(u,v),
\end{align}
使得$x_0 = x(u_0,v_0), y_0 = y(u_0,v_0)$,且当$(u,v) \in U(P'_0)$时,有
\[
(x(u,v),y(u,v)) \in U(P_0)
\]
以及恒等式
\begin{align*}
u \equiv u(x(u,v),y(u,v)),\ v \equiv v(x(u,v),y(u,v)).
\end{align*}
此外,反函数组\eqref{eq::-2r2jfw91}在$U(P'_0)$上存在连续的一阶偏导数,且
\begin{align}
\frac{\partial x}{\partial u} = \frac{\partial v}{\partial y} \bigg/ \frac{\partial(u,v)}{\partial(x,y)},\ \frac{\partial x}{\partial v} = -\frac{\partial u}{\partial y} \bigg/ \frac{\partial(u,v)}{\partial(x,y)},\quad \nonumber \\
\frac{\partial y}{\partial u} = -\frac{\partial v}{\partial x} \bigg/ \frac{\partial(u,v)}{\partial(x,y)},\ \frac{\partial y}{\partial v} = \frac{\partial u}{\partial x} \bigg/ \frac{\partial(u,v)}{\partial(x,y)}. \nonumber
\end{align}
由上式看到:互为反函数组的\eqref{eq::-2r2jfw9}与\eqref{eq::-2r2jfw91},它们的雅可比行列式互为倒数,即
\[
\frac{\partial(u,v)}{\partial(x,y)} \cdot \frac{\partial(x,y)}{\partial(u,v)} = 1.
\]
\end{theorem}





\end{document}
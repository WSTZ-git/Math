\documentclass[../../main.tex]{subfiles}
\graphicspath{{\subfix{../../image/}}} % 指定图片目录,后续可以直接使用图片文件名。

% 例如:
% \begin{figure}[h]
% \centering
% \includegraphics{image-01.01}
% \caption{图片标题}
% \label{fig:image-01.01}
% \end{figure}
% 注意:上述\label{}一定要放在\caption{}之后,否则引用图片序号会只会显示??.

\begin{document}

\section{重要不等式}

\begin{theorem}[Cauchy不等式]\label{theorem:Cauchy不等式}
对任何$n\in \mathbb{N} ,\left( a_1,a_2,\cdots ,a_n \right) ,\left( b_1,b_2,\cdots ,b_n \right) \in \mathbb{R} ^n,$有
\begin{align}\label{equation:Cauchy不等式}
\sum\limits_{i=1}^n{a_{i}^{2}\cdot}\sum\limits_{i=1}^n{b_{i}^{2}}\geqslant \left( \sum\limits_{i=1}^n{a_ib_i} \right) ^2.
\end{align}
且等号成立条件为$\left( a_1,a_2,\cdots ,a_n \right) ,\left( b_1,b_2,\cdots ,b_n \right)$线性相关.
\end{theorem}
\begin{proof}
(i) 当\(b_i\)全为零时,\eqref{equation:Cauchy不等式}式左右两边均为零,结论显然成立.

(ii) 当\(b_i\)不全为零时,注意到\(\left(\sum\limits_{i = 1}^{n}(a_i + tb_i)\right)^2 \geqslant 0\),\(\forall t \in \mathbb{R}\).等价于

\[
t^2\sum\limits_{i = 1}^{n}b_{i}^{2} + 2t\sum\limits_{i = 1}^{n}a_ib_i + \sum\limits_{i = 1}^{n}a_{i}^{2} \geqslant 0,  \forall t \in \mathbb{R}.
\]

根据一元二次方程根的存在性定理,可知\(\Delta = \left(\sum\limits_{i = 1}^{n}a_ib_i\right)^2 - \sum\limits_{i = 1}^{n}a_{i}^{2} \cdot \sum\limits_{i = 1}^{n}b_{i}^{2} \leqslant 0\).

从而\(\sum\limits_{i = 1}^{n}a_{i}^{2} \cdot \sum\limits_{i = 1}^{n}b_{i}^{2} \geqslant \left(\sum\limits_{i = 1}^{n}a_ib_i\right)^2\).
下证\eqref{equation:Cauchy不等式}式等号成立的充要条件.

若\eqref{equation:Cauchy不等式}式等号成立,则

(i) 当\(b_i\)全为零时,因为零向量与任意向量均线性相关,所以此时\((a_1,a_2,\cdots,a_n)\),\((b_1,b_2,\cdots,b_n)\)线性相关.

(ii) 当\(b_i\)不全为零时,此时我们有\(\Delta = \left(\sum\limits_{i = 1}^{n}a_ib_i\right)^2 - \sum\limits_{i = 1}^{n}a_{i}^{2} \cdot \sum\limits_{i = 1}^{n}b_{i}^{2} = 0\).根据一元二次方程根的存在性定理,可知存在\(t_0 \in \mathbb{R}\),使得

\[
\left(\sum\limits_{i = 1}^{n}(a_i + tb_i)\right)^2 = t_{0}^{2}\sum\limits_{i = 1}^{n}b_{i}^{2} + 2t_0\sum\limits_{i = 1}^{n}a_ib_i + \sum\limits_{i = 1}^{n}a_{i}^{2} = 0.
\]

于是\(a_i + t_0b_i = 0\),\(i = 1,2,\cdots,n\).即\((a_1,a_2,\cdots,a_n)\),\((b_1,b_2,\cdots,b_n)\)线性相关.

反之,若\((a_1,a_2,\cdots,a_n)\),\((b_1,b_2,\cdots,b_n)\)线性相关,则存在不全为零的\(\lambda,\mu \in \mathbb{R}\),使得

\[
\lambda a_i + \mu b_i = 0,  i = 1,2,\cdots,n.
\]

不妨设\(\lambda \neq 0\),则\(a_i = -\frac{\mu}{\lambda}b_i\),\(i = 1,2,\cdots,n\).从而当\(t = \frac{\mu}{\lambda}\)时,\(\left(\sum\limits_{i = 1}^{n}(a_i + tb_i)\right)^2 = 0\).

即一元二次方程\(\left(\sum\limits_{i = 1}^{n}(a_i + tb_i)\right)^2 = t_{0}^{2}\sum\limits_{i = 1}^{n}b_{i}^{2} + 2t_0\sum\limits_{i = 1}^{n}a_ib_i + \sum\limits_{i = 1}^{n}a_{i}^{2} = 0\)有实根\(\frac{\mu}{\lambda}\).

因此\(\Delta = \left(\sum\limits_{i = 1}^{n}a_ib_i\right)^2 - \sum\limits_{i = 1}^{n}a_{i}^{2} \cdot \sum\limits_{i = 1}^{n}b_{i}^{2} = 0\).即\eqref{equation:Cauchy不等式}式等号成立.
\end{proof}

\begin{example}
证明:\begin{align*}
\sum\limits_{i=1}^n{\frac{1}{x_i}}\geqslant \frac{n^2}{\sum\limits_{i=1}^n{x_i}},\forall n\in \mathbb{N} ,x_1,x_2,\cdots ,x_n>0.
\end{align*}
\end{example}
\begin{proof}
对\(\forall n \in \mathbb{N}\),\(x_1,x_2,\cdots,x_n > 0\),由\hyperref[theorem:Cauchy不等式]{Cauchy不等式}可得

\[
\sum\limits_{i = 1}^{n}\frac{1}{x_i} \cdot \sum\limits_{i = 1}^{n}x_i = \sum\limits_{i = 1}^{n}\left(\frac{1}{\sqrt{x_i}}\right)^2 \cdot \sum\limits_{i = 1}^{n}\left(\sqrt{x_i}\right)^2 \geqslant \left(\sum\limits_{i = 1}^{n}\sqrt{x_i} \cdot \frac{1}{\sqrt{x_i}}\right)^2 = n^2.
\]

故\(\sum\limits_{i = 1}^{n}\frac{1}{x_i} \geqslant \frac{n^2}{\sum\limits_{i = 1}^{n}x_i}\),\(\forall n \in \mathbb{N}\),\(x_1,x_2,\cdots,x_n > 0\). 
\end{proof}

\begin{example}
求函数$y=\sqrt{x+27}+\sqrt{13-x}+\sqrt{x}$在定义域内的最大值和最小值.
\end{example}
\begin{note}
首先我们猜测定义域的端点处可能存在最值,然后我们通过简单的放缩就能得到$y(0)$就是最小值.再利用\hyperref[theorem:Cauchy不等式]{Cauchy不等式}我们可以得到函数的最大值.构造Cauchy不等式的思路是:利用待定系数法构造相应的Cauchy不等式.具体步骤如下:

设\(A,B,C > 0\),则由\(Cauchy\)不等式可得

\[
\left(\frac{1}{\sqrt{A}}\sqrt{Ax + 27A} + \frac{1}{\sqrt{B}}\sqrt{13B - Bx} + \frac{1}{\sqrt{C}}\sqrt{Cx}\right)^2 \leqslant \left(\frac{1}{A} + \frac{1}{B} + \frac{1}{C}\right)[(A + C - B)x + 27A + 13B]
\]

并且当且仅当\(\sqrt{A} \cdot \sqrt{Ax + 27A} = \sqrt{B} \cdot \sqrt{13B - Bx} = \sqrt{C} \cdot \sqrt{Cx}\)时,等号成立.

令\(A + C - B = 0\)(因为要求解\(y\)的最大值,我们需要将\(y\)放大成一个不含\(x\)的常数),从而与上式联立得到方程组

\[
\begin{cases}
\sqrt{A} \cdot \sqrt{Ax + 27A} = \sqrt{B} \cdot \sqrt{13B - Bx} = \sqrt{C} \cdot \sqrt{Cx}\\
A + C - B = 0
\end{cases}
\]

解得:\(A = 1\),\(B = 3\),\(C = 2\),\(x = 9\).

从而得到我们需要构造的\(Cauchy\)不等式为

\[
\left(\sqrt{x + 27} + \frac{1}{\sqrt{3}}\sqrt{39 - 3x} + \frac{1}{\sqrt{2}}\sqrt{2x}\right)^2 \leqslant \left(1 + \frac{1}{3} + \frac{1}{2}\right)(x + 27 + 39 - 3x + 2x)
\]

并且当且仅当\(\sqrt{x + 27} = \sqrt{3} \cdot \sqrt{39 - 3x} = \sqrt{2} \cdot \sqrt{2x}\),即\(x = 9\)时,等号成立. 
\end{note}
\begin{solution}
由题可知,函数\(y\)的定义域就是:\(0 \leqslant x \leqslant 13\).而
\begin{align*}
&y(x) = \sqrt{x + 27} + \sqrt{[\sqrt{13 - x} + \sqrt{x}]^2}
\\
&= \sqrt{x + 27} + \sqrt{13 + 2\sqrt{x(13 - x)}}
\\
&\geqslant \sqrt{27} + \sqrt{13} = 3\sqrt{3} + \sqrt{13} = y(0)
\end{align*}
于是\(y\)的最小值为\(3\sqrt{3} + \sqrt{13}\).
由\(Cauchy\)不等式可得
\begin{align*}
y^2(x) &= (\sqrt{x + 27} + \sqrt{13 - x} + \sqrt{x})^2\\
&= (\sqrt{x + 27} + \frac{1}{\sqrt{3}}\sqrt{39 - 3x} + \frac{1}{\sqrt{2}}\sqrt{2x})^2\\
&\leqslant (1 + \frac{1}{3} + \frac{1}{2})(x + 27 + 39 - 3x + 2x)\\
&= 121 = y^2(9)
\end{align*}
即\(y(x) \leqslant y(9) = 11\).并且当且仅当\(\sqrt{x + 27} = \sqrt{3} \cdot \sqrt{39 - 3x} = \sqrt{2} \cdot \sqrt{2x}\),即\(x = 9\)时,等号成立.故\(y\)的最大值为\(11\). 
\end{solution}

\begin{theorem}[均值不等式]\label{theorem:均值不等式}
设$a_1,a_2,\cdots,a_n>0$,则下述函数是连续递增函数
\begin{align}
f\left( r \right) =\begin{cases}
\left( \frac{a_{1}^{r}+a_{2}^{r}+\cdots +a_{n}^{r}}{n} \right) ^{\frac{1}{r}},r\ne 0\\
\sqrt[n]{a_1a_2\cdots a_n},\,\,\,\,\,\,\,\,\,\,\,\,\,\,\,\,\,\,\,\,\,\,\,\,r=0\\
\end{cases}.
\end{align}
其中若$r_1\ne r_2$,则$f(r_1)=f(r_2)$的充要条件是$a_1=a_2=\cdots=a_n$.
\end{theorem}
\begin{note}
均值不等式最重要的特例是下面的\hyperref[theorem:均值不等式常用形式]{均值不等式常用形式}.
\end{note}
\begin{theorem}[均值不等式常用形式]\label{theorem:均值不等式常用形式}
设$a_1,a_2,\cdots,a_n>0$,则
\begin{align*}
\frac{n}{\frac{1}{a_1}+\frac{1}{a_2}+\cdots +\frac{1}{a_n}}\leqslant \sqrt[n]{a_1a_2\cdots a_n}\le \frac{a_1+a_2+\cdots +a_n}{n}\leqslant \sqrt{\frac{a_{1}^{2}+a_{2}^{2}+\cdots +a_{n}^{2}}{n}}.
\end{align*}
\end{theorem}

\begin{example}
设$f(x)=4x(x-1)^2,x\in(0,1)$,求$f$的最大值.
\end{example}
\begin{solution}
由\hyperref[theorem:均值不等式常用形式]{均值不等式常用形式}可得
\begin{align*}
&f\left( x \right) =4x\left( x-1 \right) ^2=2\cdot 2x\left( 1-x \right) \left( 1-x \right) 
\\
&= 2\cdot \left[ \sqrt[3]{2x\left( 1-x \right) \left( 1-x \right)} \right] ^3
\\
&\leqslant 2\cdot \left[ \frac{2x+1-x+1-x}{3} \right] ^3
\\
&=2\cdot \left( \frac{2}{3} \right) ^3=\frac{16}{27}
\end{align*}
并且当且仅当$2x=1-x$,即$x=\frac{1}{3}$时等号成立.
\end{solution}

\begin{theorem}[Bernoulli不等式]\label{theorem:Bernoulli不等式}
设$x_1,x_2,\cdots,x_n\geq -1$且两两同号,则
\begin{align*}
\left( 1+x_1 \right) \left( 1+x_2 \right) \cdots \left( 1+x_n \right) \geqslant 1+x_1+x_2+\cdots +x_n.
\end{align*}
\end{theorem}
\begin{proof}
当$n=1$时,我们有$1+x_1\geq 1+x_1$,结论显然成立.

假设当$n=k$时,结论成立.则当$n=k+1$时,由归纳假设可得
\begin{align*}
&\left( 1+x_1 \right) \left( 1+x_2 \right) \cdots \left( 1+x_{k+1} \right) \geqslant \left( 1+x_1+x_2+\cdots +x_k \right) \left( 1+x_{k+1} \right) 
\\
&=1+x_1+x_2+\cdots +x_k+x_{k+1}+x_1x_{k+1}+x_2x_{k+1}+\cdots +x_kx_{k+1}
\\
&\geqslant 1+x_1+x_2+\cdots +x_k+x_{k+1}
\end{align*}
故由数学归纳法可知,结论成立.
\end{proof}

\begin{theorem}[Bernoulli不等式特殊形式]\label{theorem:Bernoulli不等式特殊形式}
设$x\geq-1$,则
\begin{align*}
(1+x)^n\geqslant1+nx.
\end{align*}
\end{theorem}

\begin{theorem}[Jesen不等式]\label{theorem:Jesen不等式}
设\(\lambda_i \geq 0\),\(i = 1,2,\cdots,n\),\(\sum\limits_{i = 1}^{n} \lambda_i = 1\),则对下凸函数\(f\),有
\[
f\left(\sum\limits_{i = 1}^{n} \lambda_i x_i\right) \leq \sum\limits_{i = 1}^{n} \lambda_i f(x_i).
\]

对上凸函数\(f\),有
\[
f\left(\sum\limits_{i = 1}^{n} \lambda_i x_i\right) \geq \sum\limits_{i = 1}^{n} \lambda_i f(x_i).
\]
\end{theorem}

\begin{theorem}[Young不等式]\label{theorem:Young不等式}
对任何$a,b\geq0,\frac{1}{p}+\frac{1}{q}=1,p>1$有
\begin{align*}
ab\leqslant \frac{a^p}{p}+\frac{b^q}{q}.
\end{align*}
\end{theorem}
\begin{note}
若$\frac{1}{p}+\frac{1}{q}=1$,则我们称$p$与$q$\textbf{共轭}.\label{实数之间的共轭}
\end{note}
\begin{proof}
(i)当$a,b$至少有一个为零时,结论显然成立.

(ii)当$a,b$均不为零时,我们有
\begin{gather*}
ab\leqslant \frac{a^p}{p}+\frac{b^q}{q}
\\
\Leftrightarrow \ln a+\ln b\leqslant \ln \left( \frac{a^p}{p}+\frac{b^q}{q} \right) 
\\
\Leftrightarrow \frac{1}{p}\ln a^p+\frac{1}{q}\ln b^q\leqslant \ln \left( \frac{a^p}{p}+\frac{b^q}{q} \right) 
\end{gather*}
由\hyperref[theorem:Jesen不等式]{Jesen不等式}和$f\left( x \right) =\ln x$函数的上凸性可知,上述不等式成立.故原结论也成立.   
\end{proof}

\begin{theorem}[Hold不等式]\label{theorem:Hold不等式}
设$\frac{1}{p}+\frac{1}{q}=1,p>1,a_1,a_2,\cdots,a_n\geq0,b_1,b_2,\cdots,b_n\geq0$,则有
\begin{align*}
\sum\limits_{k=1}^n{a_kb_k}\le \sqrt[p]{\sum\limits_{k=1}^n{a_{k}^{p}}}\cdot \sqrt[q]{\sum\limits_{k=1}^n{b_{k}^{q}}}.
\end{align*}
\end{theorem}
\begin{proof}
(i)当$a_1,a_2,\cdots,a_n$全为零时,结论显然成立.

(ii)当$a_1,a_2,\cdots,a_n$不全为零时,令
\begin{align*}
a_{k}^{\prime}=\frac{a_k}{\sqrt[p]{\sum\limits_{k=1}^n{a_{k}^{p}}}},b_{k}^{\prime}=\frac{b_k}{\sqrt[q]{\sum\limits_{k=1}^n{b_{k}^{q}}}},k=1,2,\cdots,n.
\end{align*}
从而只需证明$\sum\limits_{k=1}^n{a_{k}^{\prime}b_{k}^{\prime}}\leqslant 1$.由\hyperref[theorem:Young不等式]{Young不等式}可得
\begin{align*}
\sum\limits_{k=1}^n{a_{k}^{\prime}b_{k}^{\prime}}\leqslant &\sum\limits_{k=1}^n{\left[ \frac{\left( a_{k}^{\prime} \right) ^p}{p}+\frac{\left( b_{k}^{\prime} \right) ^q}{q} \right]}=\sum\limits_{k=1}^n{\left( \frac{a_{k}^{p}}{p\sum\limits_{k=1}^n{a_{k}^{p}}}+\frac{b_{k}^{p}}{q\sum\limits_{k=1}^n{b_{k}^{q}}} \right)}
\\
&=\frac{\sum\limits_{k=1}^n{a_{k}^{p}}}{p\sum\limits_{k=1}^n{a_{k}^{p}}}+\frac{\sum\limits_{k=1}^n{b_{k}^{p}}}{q\sum\limits_{k=1}^n{b_{k}^{q}}}=\frac{1}{p}+\frac{1}{q}=1.
\end{align*}
故原结论成立.
\end{proof}

\begin{theorem}[排序和不等式]\label{theorem:排序和不等式}
设\(\{a_1,a_2,\cdots,a_n\} \subset \mathbb{R}\),\(\{b_1,b_2,\cdots,b_n\} \subset \mathbb{R}\)满足
\[
a_1 \leq a_2 \leq \cdots \leq a_n, b_1 \leq b_2 \leq \cdots \leq b_n.
\]

\(\{c_1,c_2,\cdots,c_n\}\)是\(\{b_1,b_2,\cdots,b_n\}\)的一个排列,则有
\[
\sum\limits_{i = 1}^{n} a_i b_{n + 1 - i} \leq \sum\limits_{i = 1}^{n} a_i c_i \leq \sum\limits_{i = 1}^{n} a_i b_i,
\]

且等号成立的充要条件是\(a_i = a_j\),\(1 \leq i < j \leq n\)或者\(b_i = b_j\),\(1 \leq i < j \leq n\).
\end{theorem}
\begin{note}
简单记为倒序和\(\leq\)乱序和\(\leq\)同序和.
\end{note}

\begin{theorem}[Chebeshev不等式]\label{theorem:Chebeshev不等式}
设\(\{a_1,a_2,\cdots,a_n\} \subset \mathbb{R}\),\(\{b_1,b_2,\cdots,b_n\} \subset \mathbb{R}\)满足
\[
a_1 \leq a_2 \leq \cdots \leq a_n, b_1 \leq b_2 \leq \cdots \leq b_n.
\]

\(\{c_1,c_2,\cdots,c_n\}\)是\(\{b_1,b_2,\cdots,b_n\}\)的一个排列,则有
\[
\sum\limits_{i = 1}^{n} a_i b_{n + 1 - i} \leq \frac{1}{n} \sum\limits_{i = 1}^{n} a_i \sum\limits_{i = 1}^{n} b_i \leq \sum\limits_{i = 1}^{n} a_i b_i.
\]

且等号成立的充要条件是\(a_i = a_j\),\(1 \leq i < j \leq n\)或者\(b_i = b_j\),\(1 \leq i < j \leq n\).
\end{theorem}




\end{document}
\documentclass[../../main.tex]{subfiles}
\graphicspath{{\subfix{../../image/}}} % 指定图片目录,后续可以直接使用图片文件名。

% 例如:
% \begin{figure}[H]
% \centering
% \includegraphics[scale=0.4]{图.png}
% \caption{}
% \label{figure:图}
% \end{figure}
% 注意:上述\label{}一定要放在\caption{}之后,否则引用图片序号会只会显示??.

\begin{document}

\section{多元函数的连续性和微分}

我们的极限采用聚点定义,即只需要沿着有定义的地方趋近即可.

\begin{example}
设\(f(x,y)=\begin{cases}\dfrac{x^2 y}{x^4 + y^2},&x^2 + y^2 \neq 0\\0,&x^2 + y^2 = 0\end{cases}\),证明\(f\)沿着每条射线\(\begin{cases}x = t\cos\alpha,&t > 0,\alpha \in [0,2\pi)\\y = t\sin\alpha,&t > 0,\alpha \in [0,2\pi)\end{cases}\)趋于\((0,0)\)时都趋于 0,但是\(f\)在\((0,0)\)不连续.
\end{example}
\begin{note}
本结果表明,使用极坐标求二重极限不一定正确.实际上,我们用极坐标求二重根极限时,都是固定$\alpha$,再令$t\to 0$求极限.因此得到的只是沿着每个过原点的射线(与$x$轴的夹角为$\alpha$)趋于$(0,0)$的极限,比如还可以沿$y=kx^2$这条曲线趋于$(0,0)$.
\end{note}
\begin{proof}
一方面,
\begin{align*}
\lim_{t \to 0^+} f(t\cos\alpha, t\sin\alpha)&=\lim_{t \to 0^+} \frac{t^3 \cos\alpha \sin\alpha}{t^4 \cos^4 \alpha + t^2 \sin^2 \alpha} = \lim_{t \to 0^+} \frac{t \cos\alpha \sin\alpha}{t^2 \cos^4 \alpha + \sin^2 \alpha} = 0,
\end{align*}
另外一方面
\[
\lim_{x \to 0^+} f(x, kx^2)=\lim_{x \to 0^+} \frac{kx^4}{x^4 + k^2 x^4}=\frac{k}{1 + k^2}.
\]
故\(f\)在\((0,0)\)不连续.矛盾!
\end{proof}
\begin{remark}
实际上,使用极坐标变换求极限时,只需要在$t\to 0$的时候让$\alpha$也发生变化(不再固定$\alpha$),再求极限才能得到正确的极限值,但这样反而不方便求极限.

那么什么时候固定$\alpha$后求出来的极限就是原函数的极限呢?实际上只需要极限关于\(\alpha \in [0,2\pi)\)一致,因为你直接考察定义\(\varepsilon - \delta\)语言即可. 实际做题中可以体现为
\[
\lim_{t \to 0^+} \sup_{\alpha \in [0,2\pi)} |f(t\cos\alpha, t\sin\alpha) - A| = 0
\]
更直白的,你需要得到形如
\[
|f(t\cos\alpha, t\sin\alpha) - A| \leq g(t)
\]
的不等式且\(\lim_{t \to 0^+} g(t) = 0\).
\end{remark}

\begin{proposition}\label{proposition:二重极限极坐标换元的条件}
设二元函数 \( f(x, y) \) 在点 \( (a, b) \) 的某个去心邻域内有定义,若对$\forall \alpha \in [0, 2\pi)$,都有
\begin{align*}
\underset{t \rightarrow 0^+}{\lim}f\left( a+t\cos \alpha ,b+t\sin \alpha \right) =A\in \mathbb{R},
\end{align*}
并且
\begin{align*}
\lim_{t \to 0^+} \sup_{\alpha \in [0,2\pi)} |f(t\cos\alpha, t\sin\alpha) - A| = 0,
\end{align*}
或者
存在函数 \(g(t)\) 满足对$\forall \alpha \in [0, 2\pi)$,都有
\[
|f(a + t\cos\alpha, b + t\sin\alpha) - A| \leqslant g(t),\quad \lim_{t \to 0^+} g(t) = 0,
\]
则
\[
\lim_{(x, y) \to (a, b)} f(x, y) = A.
\]
\end{proposition}
\begin{proof}
显然条件$\lim_{t \to 0^+} \sup_{\alpha \in [0,2\pi)} |f(t\cos\alpha, t\sin\alpha) - A| = 0$和条件
存在函数 \(g(t)\) 满足对$\forall \alpha \in [0, 2\pi)$,都有
\[
|f(a + t\cos\alpha, b + t\sin\alpha) - A| \leqslant g(t),\quad \lim_{t \to 0^+} g(t) = 0
\]
等价.
由$\lim\limits_{t\rightarrow 0^+}g\left( t \right) =0$知,$\forall \varepsilon >0$,存在$\delta >0$,使得
\begin{align*}
|g(t)| < \varepsilon,\ \forall t \in (0, \delta).
\end{align*}
对$\forall (x, y)$,满足$0 < \sqrt{(x - a)^2 + (y - b)^2} < \delta$,令
\begin{align*}
t = \sqrt{(x - a)^2 + (y - b)^2},\ \alpha = \arctan\frac{y - b}{x - a},
\end{align*}
则$x = a + t\cos\alpha$,$y = b + t\sin\alpha$。于是
\begin{align*}
|f(x, y) - A| = |f(a + t\cos\alpha, b + t\sin\alpha) - A| \leqslant g(t) < \varepsilon,\ \forall (x, y) \in B\left( (a, b), \delta \right).
\end{align*}
\end{proof}

\begin{example}
计算
\[
\lim_{(x,y) \to (0,0)} \frac{x^3 + y^3}{x + y}
\]
\end{example}
\begin{proof}
考虑
\[
|f(t\cos\alpha, t\sin\alpha)| = t^2 \left| \frac{\cos^3 \alpha + \sin^3 \alpha}{\cos\alpha + \sin\alpha} \right| = t^2 |\cos^2 \alpha + \sin^2 \alpha - \cos\alpha \sin\alpha| \leq 2t^2,
\]
于是
\[
0 \leq \lim_{t \to 0^+} |f(t\cos\alpha, t\sin\alpha)| \leq 2 \lim_{t \to 0^+} t^2 = 0
\]
故我们得到了
\[
\lim_{(x,y) \to (0,0)} \frac{x^3 + y^3}{x + y} = 0.
\]
\end{proof}

\begin{example}
设\(f\)在\((0,0)\)连续且满足
\[
\lim_{(x,y) \to (0,0)} \frac{f(x,y) - xy}{x^2 + y^2} = a > 0
\]
求\(a\)的范围使得\(f\)在\((0,0)\)一定取到极值. 再求\(a\)的范围使得\(f\)在\((0,0)\)一定取不到极值.
\end{example}
\begin{note}
注意到\(\lim_{(x,y) \to (0,0)} \frac{xy}{x^2 + y^2}\)不存在,但是
\[
\left| \frac{xy}{x^2 + y^2} \right| \leq \frac{1}{2}
\]
即猜测\(\frac{1}{2}\)是\(a\)的分界点.
\end{note}
\begin{proof}
由条件容易得到
\begin{align*}
&\quad \quad \,\,\underset{\left( x,y \right) \rightarrow \left( 0,0 \right)}{\lim}\frac{f\left( x,y \right) -xy}{x^2+y^2}=a
\\
&\Longrightarrow \underset{\left( x,y \right) \rightarrow \left( 0,0 \right)}{\lim}f\left( x,y \right) =a\cdot \underset{\left( x,y \right) \rightarrow \left( 0,0 \right)}{\lim}\left( x^2+y^2 \right) +\underset{\left( x,y \right) \rightarrow \left( 0,0 \right)}{\lim}xy=0
\\
&\Longrightarrow f(0,0)=\lim_{(x,y)\rightarrow (0,0)} f(x,y)=0
\end{align*}
以及
\[
f(x,y) = (a + g(x,y))(x^2 + y^2) + xy, \quad \lim_{(x,y) \to (0,0)} g(x,y) = 0.
\]
当\(a > \frac{1}{2}\),当\((x,y)\)足够靠近\(0\)使得\(g(x,y) > \frac{1}{2} - a\). 此时我们有
\[
f(x,y) = (a + g(x,y))(x^2 + y^2) + xy > \frac{1}{2}(x^2 + y^2) + xy = \frac{1}{2}(x + y)^2 \geq 0
\]
故\(a > \frac{1}{2}\)时,$f$在$(0,0)$处取得极小值.

当\(0 < a < \frac{1}{2}\),当\((x,y)\)足够靠近\(0\)使得\(-a < g(x,y) < \frac{1 - 2a}{4}\),则此时当\(x > 0, y > 0\)有\(f(x,y) > 0\). 但是
\[
f(x,y) = (a + g(x,y))(x^2 + y^2) + xy < \frac{1 + 2a}{4}(x^2 + y^2) + xy
\]
又\(\frac{1 + 2a}{4}(x^2 + y^2) + xy\)在\(y = -x\)上有
\[
\frac{1 + 2a}{4}2x^2 - x^2 = \frac{2a - 1}{2}x^2 < 0,
\]
故此时\((0,0)\)不是$f$的极值点.

当\(a = \frac{1}{2}\)问题无法判断,这是因为考虑\(f(x,y) = xy + \frac{1}{2}(x^2 + y^2) + x^2(x^2 + y^2)\),则
\[
f(x,y) = \frac{1}{2}(x + y)^2 + x^2(x^2 + y^2) > 0,
\]
即\((0,0)\)是极值. 但是考虑\(f(x,y) = \frac{1}{2}(x^2 + y^2) + xy - x(x^2 + y^2)\),就有
\[
f(x,-x) = x^2 - x^2 - 2x^3 = -2x^3 < 0, x > 0,
\]
\[
f(x,x) = x^2 + x^2 - 2x^3 = 2x^2 - 2x^3 > 0, 0 < x < 1.
\]
即\((0,0)\)不是极值.
\end{proof}

\begin{theorem}\label{theorem:定理雅克比行列式为0的充要条件}
设\( u = f(x, y), v = g(x, y) \) 在区域 \( D \subset \mathbb{R}^2 \) 上有连续偏导数, 则 \( u \) 与 \( v \) 之间有函数关系当且仅当
\begin{align*}
J = \frac{\partial (u, v)}{\partial (x, y)} = 0.
\end{align*}
\end{theorem}
\begin{proof}
必要性. 假定 \( u, v \) 满足 \( F(u, v) = 0 \), 则由 \( F(u, v) = F[f(x, y), g(x, y)] \) 可知
\[
F'_u \cdot f'_x + F'_v g'_x = 0, \quad F'_u f'_y + F'_v g'_y = 0.
\]
注意到 \( F'_u, F'_v \) 不同时为 0, 故上述方程组存在非零解, 从而有 \( J = \frac{\partial (u, v)}{\partial (x, y)} = 0 \).

充分性. 若 \( u'_x, u'_y, v'_x, v'_y \) 全为 0, 则 \( u, v \) 是常数, 从而有关系 \( u = cv \). 若上述四个值有一个非 0, 例如是 \( v'_y \neq 0 \), 则由隐函数存在定理, 可从 \( v = g(x, y) \) 可确定函数 \( y = \psi(x, v) \). 代入 \( u = f(x, y) \) 可得 \( u = f(x, \psi(x, v)) \), 记为 \( F(x, v) \).
因此, 我们有
\[
0 = J = \begin{vmatrix}
u_x & u_y \\
v_x & v_y
\end{vmatrix} = \begin{vmatrix}
F'_x + F'_v v'_x & F'_v v'_y \\
v'_x & v'_y
\end{vmatrix} = F'_x v'_y.
\]
由此知 \( F'_x = 0 \). 这说明 \( F \) 不是 \( x \) 的函数, 即 \( u = F(v) \).
\end{proof}

\begin{example}
设 \( x f'_x + y f'_y = 0 \),证明 \( f \) 是 \( \frac{y}{x} \) 的函数.
\end{example}
\begin{proof}
注意到
\[
\begin{vmatrix}
f'_x & f'_y \\
-\frac{y}{x^2} & \frac{1}{x}
\end{vmatrix} = \frac{1}{x^2} \left( x f'_x + y f'_y \right) = 0, \frac{\partial \left( \frac{y}{x} \right)}{\partial x} = -\frac{y}{x^2}, \frac{\partial \left( \frac{y}{x} \right)}{\partial y} = \frac{1}{x}
\]
我们由\refthe{theorem:定理雅克比行列式为0的充要条件}知 \( f \) 是 \( \frac{y}{x} \) 的函数.
\end{proof}

\begin{theorem}[用矩阵判定极值]\label{theorem:用矩阵判定极值}
设 \( f \) 是某个区域 \( V \subset \mathbb{R}^n \) 的二阶连续可微函数, 我们定义其 Hess(黑塞) 矩阵为
\[
Hf = \left( \frac{\partial^2 f}{\partial x_i \partial x_j} \right)_{1 \leqslant i, j \leqslant n},
\]
则对 \( \mathbf{x}_0 \in V \) 满足 \( \frac{\partial f}{\partial x_i}(\mathbf{x}_0) = 0, i = 1, 2, \cdots, n \) 有
\begin{enumerate}
\item \( Hf(\mathbf{x}_0) \) 是正定的, 则 \( \mathbf{x}_0 \) 是 \( f \) 严格极小值点;

\item \( Hf(\mathbf{x}_0) \) 是负定的, 则 \( \mathbf{x}_0 \) 是 \( f \) 严格极大值点;

\item \( Hf(\mathbf{x}_0) \) 是不定的(既不是正定,也不是负定), 则 \( \mathbf{x}_0 \) 不是 \( f \) 极值点;

\item 若 \( \mathbf{x}_0 \) 是 \( f \) 极小值点, 则 \( Hf(\mathbf{x}_0) \) 是半正定的;

\item 若 \( \mathbf{x}_0 \) 是 \( f \) 极大值点, 则 \( Hf(\mathbf{x}_0) \) 是半负定的.
\end{enumerate}
\end{theorem}
\begin{proof}

\end{proof}

\begin{definition}
我们称 \( f : \mathbb{R}^2 \to \mathbb{R} \) 为\textbf{齐} \( \boldsymbol{n}( n \in \mathbb{N} )\) \textbf{次函数},如果 \( f \) 满足
\[
f(tx, ty) = t^n f(x, y), \forall x, y \in \mathbb{R}, t > 0.
\]
\end{definition}

\begin{proposition}[齐次函数基本性质]\label{proposition:齐次函数基本性质}
若 \( f \in D^2(\mathbb{R}^2) \), 则 \( f \) 是齐 \( n \) 次函数的充要条件是
\[
x \frac{\partial f}{\partial x} + y \frac{\partial f}{\partial y} = n f.
\]
\end{proposition}
\begin{proof}
若 \( f \) 是齐 \( n \) 次函数, 则
\[
f(tx, ty) = t^n f(x, y), \forall x, y \in \mathbb{R}, t > 0.
\]
两边对 \( t \) 求导得
\[
x \frac{\partial f}{\partial x}(tx, ty) + y \frac{\partial f}{\partial y}(tx, ty) = n t^{n - 1} f(x, y),
\]
于是
\[
tx \frac{\partial f}{\partial x}(tx, ty) + ty \frac{\partial f}{\partial y}(tx, ty) = n t^n f(x, y) = n f(tx, ty),
\]
再令$\mathbf{x}=tx,\mathbf{y}=ty$,即证.

反过来若
\[
x \frac{\partial f}{\partial x} + y \frac{\partial f}{\partial y} = n f
\]
成立, 固定 \( x, y \in \mathbb{R} \) 并考虑 \( g(t) = f(tx, ty) \). 则有
\[
t g'(t) = tx \frac{\partial f}{\partial x}(tx, ty) + ty \frac{\partial f}{\partial y}(tx, ty) = n f(tx, ty) = n g(t).
\]
故解微分方程得 \( g(t) = C t^n \), 从而将$g(1) = f(x, y)$代入得$C=f(x,y),$于是
\[
g(t)=f(tx, ty) = t^n f(x, y).
\]
这就证明了
\[
f(tx, ty) = t^n f(x, y), \forall x, y \in \mathbb{R}, t > 0.
\]
\end{proof}

\begin{example}
设 \( \frac{\partial^2 u}{\partial x \partial y} + \frac{\partial u}{\partial y} = 0 \) 且 \( u(0, y) = y^2, u(x, 1) = \cos x \), 求 \( u \).
\end{example}
\begin{proof}
对 \( \frac{\partial^2 u}{\partial x \partial y} + \frac{\partial u}{\partial y} = 0 \) 两边关于 \( y \) 积分得 \( \frac{\partial u}{\partial x} + u = C(x) \). 由 \( u(x, 1) = \cos x \) 得
\[
\frac{\partial u}{\partial x}(x,1)=-\sin x\Rightarrow \frac{\partial u}{\partial x}\left( x,1 \right) +u\left( x,1 \right) =C(x)=\cos x-\sin x.
\]
又
\[
\frac{\partial (u e^x)}{\partial x} = e^x \left( \frac{\partial u}{\partial x} + u \right) = e^x (\cos x - \sin x) \Rightarrow u e^x = e^x \cos x + C_2(y),
\]
我们有
\[
u(x, y) = \cos x + C_2(y) e^{-x}.
\]
现在
\[
y^2 = u(0, y) = 1 + C_2(y) \Rightarrow C_2(y) = y^2 - 1.
\]
故
\[
u(x, y) = \cos x + (y^2 - 1) e^{-x}.
\]
\end{proof}

\begin{example}
设 \( l_1, l_2 \) 夹角为 \( \varphi \in (0, \pi) \) 且 \( f \) 连续可微, 证明
\[
\left| \frac{\partial f}{\partial x} \right|^2 + \left| \frac{\partial f}{\partial y} \right|^2 \leqslant \frac{2}{\sin^2 \varphi} \left[ \left| \frac{\partial f}{\partial l_1} \right|^2 + \left| \frac{\partial f}{\partial l_2} \right|^2 \right] .
\]
\end{example}
\begin{proof}
由可微时方向导数计算公式有
\[
\frac{\partial f}{\partial l_1} = \cos a \cdot \frac{\partial f}{\partial x} + \sin a \cdot \frac{\partial f}{\partial y},
\]
\[
\frac{\partial f}{\partial l_2} = \cos (a + \varphi) \cdot \frac{\partial f}{\partial x} + \sin (a + \varphi) \cdot \frac{\partial f}{\partial y},
\]
则
\begin{align*}
\begin{large}
\begin{pmatrix}
{\textstyle \frac{\partial f}{\partial l_1} }\\
{\textstyle \frac{\partial f}{\partial l_2}}
\end{pmatrix}
\end{large}
=
\begin{pmatrix}
\cos a & \sin a \\
\cos (a + \varphi) & \sin (a + \varphi)
\end{pmatrix}
\begin{large}
\begin{pmatrix}
{\textstyle \frac{\partial f}{\partial x}} \\
{\textstyle \frac{\partial f}{\partial y}}
\end{pmatrix}
\end{large}.
\end{align*}
于是
\[
\begin{aligned}
&\left| \frac{\partial f}{\partial l_1} \right|^2+\left| \frac{\partial f}{\partial l_2} \right|^2=\left( \frac{\partial f}{\partial l_1}\quad \frac{\partial f}{\partial l_2} \right) \left( \begin{array}{c}
\frac{\partial f}{\partial l_1}\\
\frac{\partial f}{\partial l_2}\\
\end{array} \right) 
\\
&=\left( \frac{\partial f}{\partial x}\quad \frac{\partial f}{\partial y} \right) \left( \begin{matrix}
\cos a&		\sin a\\
\cos\mathrm{(}a+\varphi )&		\sin\mathrm{(}a+\varphi )\\
\end{matrix} \right) ^T\left( \begin{matrix}
\cos a&		\sin a\\
\cos\mathrm{(}a+\varphi )&		\sin\mathrm{(}a+\varphi )\\
\end{matrix} \right)\begin{large}
\begin{pmatrix}
{\textstyle \frac{\partial f}{\partial x}} \\
{\textstyle \frac{\partial f}{\partial y}}
\end{pmatrix}
\end{large}
\\
&=\left( \frac{\partial f}{\partial x}\quad \frac{\partial f}{\partial y} \right) \left( \begin{matrix}
\cos ^2a+\cos ^2(a+\varphi )&		\sin\mathrm{(}a+\varphi )\cos\mathrm{(}a+\varphi )+\sin a\cos a\\
\sin\mathrm{(}a+\varphi )\cos\mathrm{(}a+\varphi )+\sin a\cos a&		\sin ^2a+\sin ^2(a+\varphi )\\
\end{matrix} \right) \begin{large}
\begin{pmatrix}
{\textstyle \frac{\partial f}{\partial x}} \\
{\textstyle \frac{\partial f}{\partial y}}
\end{pmatrix}
\end{large} .
\end{aligned}
\]
利用
\[
\begin{pmatrix}
\cos^2 a + \cos^2 (a + \varphi) & \sin (a + \varphi)\cos (a + \varphi) + \sin a \cos a \\
\sin (a + \varphi)\cos (a + \varphi) + \sin a \cos a & \sin^2 a + \sin^2 (a + \varphi)
\end{pmatrix}
\]
的特征值为 \( 1 \pm \cos \varphi \) 和\hyperref[Basis of Algebra-proposition:Rayleigh-quotient瑞丽商的基本性质]{Rayleigh quotient(瑞丽商)的基本性质}, 我们知道
\[
\left| \frac{\partial f}{\partial x} \right|^2 + \left| \frac{\partial f}{\partial y} \right|^2 \leqslant \frac{1}{1 - |\cos \varphi|} \left[ \left| \frac{\partial f}{\partial l_1} \right|^2 + \left| \frac{\partial f}{\partial l_2} \right|^2 \right] \leqslant \frac{2}{\sin^2 \varphi} \left[ \left| \frac{\partial f}{\partial l_1} \right|^2 + \left| \frac{\partial f}{\partial l_2} \right|^2 \right],
\]
即证.上式最后一个不等式是因为
\begin{align*}
&\quad \quad \,\, \frac{1}{1 - |\cos \varphi|} \geqslant \frac{2}{\sin^2 \varphi} \\
&\Longleftrightarrow 1 - |\cos \varphi| \leqslant \frac{\sin^2 \varphi}{2} \\
&\Longleftrightarrow 2 - 2|\cos \varphi| \leqslant 1 - \cos^2 \varphi \\
&\Longleftrightarrow \cos^2 \varphi - 2|\cos \varphi| + 1 \geqslant 0 \\
&\Longleftrightarrow (|\cos \varphi| - 1)^2 \geqslant 0.
\end{align*}
\end{proof}

\begin{example}
设 \( D \) 为单位圆盘, 考虑 \( f \in C^1(D) \cap C(\overline{D}) \) 且 \( |f| \leqslant 1 \), 证明: 存在 \( D \) 中的一个点 \( (x_0, y_0) \) 使得
\[
\left| \frac{\partial f}{\partial x}(x_0, y_0) \right|^2 + \left| \frac{\partial f}{\partial y}(x_0, y_0) \right|^2 \leqslant 16 .
\]
\end{example}
\begin{note}
摄动想法, 考虑$g(x,y)=f(x,y)+\varepsilon (x^2+y^2)$,其中$\varepsilon$待定.此外很多同学疑惑构造函数咋来的,实际上这是完全没有必要的! 因为大家几乎都是记的. 本题有一些更高端的技术可以加强到最佳系数.
\end{note}
\begin{proof}
考虑 \( g(x, y) = f(x, y) + 2(x^2 + y^2) \), 则由$|f|\leqslant 1$知\( g|_{\partial D} \geqslant 1, g(0, 0)=f(0,0) \leqslant 1 \). 故 \( g \) 最小值在 \( D \) 内取到.又由$g\in C(D)$,从而存在 \( D \) 中的一个 \( g \) 的最小值点 \( (x_0, y_0) \) 使得 \( \frac{\partial g}{\partial x}(x_0, y_0) = 0, \frac{\partial g}{\partial y}(x_0, y_0) = 0 \), 即
\[
\frac{\partial f}{\partial x}(x_0, y_0) = -4x_0, \frac{\partial f}{\partial y}(x_0, y_0) = -4y_0,
\]
这就得到了证明.
\end{proof}

\begin{proposition}\label{proposition:极坐标变换偏导数相关结论}
设$f(x,y)$是$D$上的二元函数且偏导数都存在,令$x=r\cos \theta$,$y=r\sin \theta$,\( g(r, \theta) = f(r \cos \theta, r \sin \theta) \), 则
\begin{gather*}
r \frac{\partial g}{\partial r} = x \frac{\partial f}{\partial x} + y \frac{\partial f}{\partial y},
\\
\frac{\partial g}{\partial \theta} = -y \frac{\partial f}{\partial x} + x \frac{\partial f}{\partial y}.
\end{gather*}
\end{proposition}
\begin{proof}
直接求导得证.
\end{proof}

\begin{example}
设 \( \lim\limits_{r = \sqrt{x^2 + y^2} \to +\infty} \left( x \frac{\partial f}{\partial x} + y \frac{\partial f}{\partial y} \right) = a > 0 \), 证明 \( f \) 在 \( \mathbb{R}^2 \) 取得最小值.
\end{example}
\begin{remark}
本题关键是有一个隐藏条件:条件极限关于角度 \( \theta \) 的一致性.
\end{remark}
\begin{note}
积累想法设 \( g(r, \theta) = f(r \cos \theta, r \sin \theta) \), 则
\[
r \frac{\partial g}{\partial r} = x \frac{\partial f}{\partial x} + y \frac{\partial f}{\partial y}, \frac{\partial g}{\partial \theta} = -y \frac{\partial f}{\partial x} + x \frac{\partial f}{\partial y}.
\]
即联想\refpro{proposition:极坐标变换偏导数相关结论}.
\end{note}
\begin{proof}
注意到
\begin{align*}
\lim\limits_{r \to +\infty} r g'_r = \lim\limits_{r \to +\infty} r \frac{\partial g}{\partial r}= a > 0,
\end{align*}
于是存在 \( r_0 > 0 \) 使得 \( g'_r > 0, \forall r \geqslant r_0 \). 则此时
\[
g(r, \theta) \geqslant g(r_0, \theta), \forall r \geqslant r_0, \theta \in [0, 2\pi),
\]
故$g$的最小值在$D=\left\{ \left( r,\theta \right) :r\in \left[ 0,r_0 \right] ,\theta \in \left[ 0,2\pi \right) \right\}$取到,因此\( \min\limits_{\substack{r \in [0, r_0], \\ \theta \in [0, 2\pi]}} g(r, \theta) \) 为 \( g \) 最小值.
\end{proof}

\begin{example}
设 \( f \) 是 \( \mathbb{R}^2 \) 上的连续可微函数且 \( f(0, 1) = f(1, 0) \), 证明存在单位圆周上两个不同的点使得 \( y \frac{\partial f}{\partial x} = x \frac{\partial f}{\partial y} \).
\end{example}
\begin{note}
联想\refpro{proposition:极坐标变换偏导数相关结论}.
\end{note}
\begin{proof}
令$x=cos\theta ,y=\sin\theta $,考虑 \( g(\theta) = f(\cos \theta, \sin \theta) \), 注意到
\[
g(0) = g\left( \frac{\pi}{2} \right) = g(2\pi).
\]
由Rolle中值定理知,存在$\theta_1\ne \theta_2 \in [0,2\pi)$,记$x_i=\cos\theta_1,y_i=\sin\theta(i=1,2)$,使得
\begin{align*}
&\quad \quad \,\,g' \left( \theta _1 \right) =g' \left( \theta _2 \right) =0
\\
&\Longleftrightarrow \begin{cases}
-\sin \theta _1\frac{\partial f}{\partial x}\left( \cos \theta _1,\sin \theta _1 \right) +\cos \theta _1\frac{\partial f}{\partial y}\left( \cos \theta _1,\sin \theta _1 \right) =0,\\
-\sin \theta _2\frac{\partial f}{\partial x}\left( \cos \theta _2,\sin \theta _2 \right) +\cos \theta _2\frac{\partial f}{\partial y}\left( \cos \theta _2,\sin \theta _2 \right) =0.\\
\end{cases}
\\
&\Longleftrightarrow \begin{cases}
y_1\frac{\partial f}{\partial x}\left( x_1,y_1 \right) =x_1\frac{\partial f}{\partial y}\left( x_1,y_1 \right) ,\\
y_2\frac{\partial f}{\partial x}\left( x_2,y_2 \right) =x_2\frac{\partial f}{\partial y}\left( x_2,y_2 \right).\\
\end{cases}
\end{align*}
即单位圆周上有两个不同的点,使得 \( y \frac{\partial f}{\partial x} = x \frac{\partial f}{\partial y} \).
\end{proof}

\begin{example}
\begin{enumerate}
\item 设 \( f \in C^1(\mathbb{R}^2), f(0,0) = 0 \) 且 \( |\nabla f| \leqslant 1 \), 证明 \( |f(1,2)| \leqslant \sqrt{5} \).

\item 设 \( f \in C^1(\mathbb{R}^2), f(0,0) = 0 \) 且
\[
\left| \frac{\partial f}{\partial x} \right| \leqslant 2|x - y|, \left| \frac{\partial f}{\partial y} \right| \leqslant 2|x - y|,
\]
证明: \( |f(5,4)| \leqslant 1 \).
\end{enumerate}
\end{example}
\begin{remark}
\textbf{梯度}及其\textbf{模}定义为$\nabla f\triangleq \left( \frac{\partial f}{\partial x},\frac{\partial f}{\partial y} \right) ,\left| \nabla f \right|\triangleq \sqrt{\left( \frac{\partial f}{\partial x} \right) ^2+\left( \frac{\partial f}{\partial y} \right) ^2}.$
\end{remark}
\begin{remark}
第二问如果和上一问完全类似, 取积分路径为连接 \( (0,0),(5,4) \) 的线段,那么由第一型曲线积分和第二型曲面积分的联系和\hyperref[theorem:Cauchy不等式]{Cauchy不等式(离散版本)}我们有
\[
\begin{aligned}
|f(5,4)|&=\left| \int_{(0,0)}^{(5,4)}{\left( \frac{\partial f}{\partial x}\mathrm{d}x+\frac{\partial f}{\partial y}\mathrm{d}y \right)} \right|=\left| \int_{(0,0)}^{(5,4)}{\left( \frac{\partial f}{\partial x}\cos \left( \widehat{\boldsymbol{t},x} \right) +\frac{\partial f}{\partial y}\sin \left( \widehat{\boldsymbol{t},x} \right) \right) \mathrm{d}s} \right|
\\
&\leqslant \int_{(0,0)}^{(5,4)}{|\nabla f|\mathrm{d}s}\leqslant \sqrt{8}\int_{(0,0)}^{(5,4)}{|x}-y|\mathrm{d}s
\\
&=\frac{\sqrt{8}}{5}\int_0^5{x\sqrt{1+\left( \frac{4}{5} \right) ^2}\mathrm{d}x}=\sqrt{82}.
\end{aligned}
\]
其中$\left( \cos \left( \widehat{\boldsymbol{t},x} \right) ,\sin \left( \widehat{\boldsymbol{t},y} \right) \right) $为积分路径曲线正切向的方向余弦.
没能成功的原因就是两类曲线积分转换时的损失,没有充分利用题目条件:当$y=x$时,$f$的两个偏导数都为0.上述证明中选取的积分路径与$y=x$这条直线关系不大.
\end{remark}
\begin{proof}
\begin{enumerate}
\item 注意到
\[
f(1,2) - f(0,0) = \int_{(0,0)}^{(1,2)} \left( \frac{\partial f}{\partial x} \mathrm{d}x + \frac{\partial f}{\partial y} \mathrm{d}y \right)
\]
这里积分路径待定. 于是由第一型曲线积分和第二型曲面积分的联系和\hyperref[theorem:Cauchy不等式]{Cauchy不等式(离散版本)}得
\[
\begin{aligned}
|f(1,2)| &= \left| \int_{(0,0)}^{(1,2)} \left( \frac{\partial f}{\partial x} \cos \left( \widehat{\boldsymbol{t},x} \right)  + \frac{\partial f}{\partial y} \sin \left( \widehat{\boldsymbol{t},x} \right)  \right) \mathrm{d}s \right| \\
&\leqslant \int_{(0,0)}^{(1,2)} |\nabla f| \mathrm{d}s \leqslant \int_{(0,0)}^{(1,2)} 1 \mathrm{d}s
\end{aligned},
\]
其中$\left( \cos \left( \widehat{\boldsymbol{t},x} \right) ,\sin \left( \widehat{\boldsymbol{t},y} \right) \right) $为积分路径曲线正切向的方向余弦.为了得到这个方法最佳的估计, 我们取积分路径为连接 \( (0,0),(1,2) \) 的线段, 这恰好给出了 \( |f(1,2)| \leqslant \sqrt{5} \).

\item 先沿着 \( y = x, 0 \leqslant x \leqslant 4 \) 积分, 此时知积分 \( \int_{(0,0)}^{(4,4)} \left( \frac{\partial f}{\partial x} \mathrm{d}x + \frac{\partial f}{\partial y} \mathrm{d}y \right) \) 为 0. 于是
\[
\begin{aligned}
|f(5,4)| &= \left| \int_{(0,0)}^{(5,4)} \left( \frac{\partial f}{\partial x} \mathrm{d}x + \frac{\partial f}{\partial y} \mathrm{d}y \right) \right| = \left| \int_{(4,4)}^{(5,4)} \left( \frac{\partial f}{\partial x} \mathrm{d}x + \frac{\partial f}{\partial y} \mathrm{d}y \right) \right| \\
&= \left| \int_{4}^{5} \frac{\partial f}{\partial x} \mathrm{d}x \right| \leqslant 2 \int_{4}^{5} |x - 4| \mathrm{d}x = 1.
\end{aligned}
\]
\end{enumerate}
\end{proof}










\end{document}
\documentclass[../../main.tex]{subfiles}
\graphicspath{{\subfix{../../image/}}} % 指定图片目录,后续可以直接使用图片文件名。

% 例如:
% \begin{figure}[H]
% \centering
% \includegraphics[scale=0.4]{图.png}
% \caption{}
% \label{figure:图}
% \end{figure}
% 注意:上述\label{}一定要放在\caption{}之后,否则引用图片序号会只会显示??.

\begin{document}

\section{三角函数相关}

\subsection{三角函数}

\begin{proposition}[三角平方差公式]\label{proposition:三角平方差}
$\sin^2 x-\sin^2 y=\sin(x-y)\sin(x+y)=\cos(y-x)\cos(y+x)=\cos^2 y-\cos^2 x.$
\end{proposition}
\begin{proof}
首先,我们有
\begin{align*}
\cos ^2x-\cos ^2y=1-\sin ^2x-\left( 1-\sin ^2y \right) =\sin ^2y-\sin ^2x.
\end{align*}
接着,我们有
\begin{align*}
\sin(x-y)\sin(x+y) &= (\sin x \cos y - \cos x \sin y)(\sin x \cos y + \cos x \sin y) \\
&= \sin^2 x \cos^2 y - \cos^2 x \sin^2 y \\
&= \sin^2 x (1 - \sin^2 y) - (1 - \sin^2 x) \sin^2 y \\
&= \sin^2 x - \sin^2 y;
\end{align*}
\begin{align*}
\cos(y-x)\cos(y+x) &= (\cos x \cos y + \sin x \sin y)(\cos x \cos y - \sin x \sin y) \\
&= \cos^2 x \cos^2 y - \sin^2 x \sin^2 y \\
&= \cos^2 x \cos^2 y - (1 - \cos^2 x)(1 - \cos^2 y) \\
&= \cos^2 x - \cos^2 y.
\end{align*}
故结论得证.
\end{proof}

\subsection{反三角函数}

\begin{proposition}\label{proposition:arctan相关等式}
\begin{enumerate}[(1)]
\item $\arctan x+\arctan \frac{1}{x}=\begin{cases}
\frac{\pi}{2},&x>0\\
-\frac{\pi}{2},&x<0\\
\end{cases}$.

\item $\arctan x - \arctan y = \arctan \frac{x - y}{1 + xy},\forall x,y\in \mathbf{R}.$
\end{enumerate}
\end{proposition}
\begin{proof}
\begin{enumerate}
\item 令 \( f(x)=\arctan x+\arctan\frac{1}{x} \),则
\begin{align*}
f'(x)&=\frac{1}{x^2 + 1}+\frac{1}{(\frac{1}{x})^2 + 1}(-\frac{1}{x^2})=\frac{1}{x^2 + 1}-\frac{1}{x^2 + 1}=0
\end{align*}
故 \( f(x) \) 为常函数,于是就有 \( f(x)=f(1)=\frac{\pi}{2},\forall x>0 \) ;\( f(x)=f(-1)=-\frac{\pi}{2},\forall x<0 \).

\item 
\end{enumerate}
\end{proof}


\subsection{双曲三角函数}

\begin{proposition}
\begin{enumerate}[(1)]
\item $\cosh x=\frac{e^{x}+e^{-x}}{2}\geqslant1,$

\item $\sinh x=\frac{e^{x}-e^{-x}}{2}\geqslant x.$
\end{enumerate}
\end{proposition}
\begin{proof}
可以分别利用均值不等式和求导进行证明. 
\end{proof}





\end{document}
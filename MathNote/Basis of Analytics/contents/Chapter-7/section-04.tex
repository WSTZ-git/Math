\documentclass[../../main.tex]{subfiles}
\graphicspath{{\subfix{../../image/}}} % 指定图片目录,后续可以直接使用图片文件名。

% 例如:
% \begin{figure}[H]
% \centering
% \includegraphics[scale=0.4]{图.png}
% \caption{}
% \label{figure:图}
% \end{figure}
% 注意:上述\label{}一定要放在\caption{}之后,否则引用图片序号会只会显示??.

\begin{document}

\section{三角函数相关}

\subsection{三角函数}

\begin{proposition}[三角平方差公式]\label{proposition:三角平方差}
$\sin^2 x-\sin^2 y=\sin(x-y)\sin(x+y)=\cos(y-x)\cos(y+x)=\cos^2 y-\cos^2 x.$
\end{proposition}
\begin{proof}
首先,我们有
\begin{align*}
\cos ^2x-\cos ^2y=1-\sin ^2x-\left( 1-\sin ^2y \right) =\sin ^2y-\sin ^2x.
\end{align*}
接着,我们有
\begin{align*}
\sin(x-y)\sin(x+y) &= (\sin x \cos y - \cos x \sin y)(\sin x \cos y + \cos x \sin y) \\
&= \sin^2 x \cos^2 y - \cos^2 x \sin^2 y \\
&= \sin^2 x (1 - \sin^2 y) - (1 - \sin^2 x) \sin^2 y \\
&= \sin^2 x - \sin^2 y;
\end{align*}
\begin{align*}
\cos(y-x)\cos(y+x) &= (\cos x \cos y + \sin x \sin y)(\cos x \cos y - \sin x \sin y) \\
&= \cos^2 x \cos^2 y - \sin^2 x \sin^2 y \\
&= \cos^2 x \cos^2 y - (1 - \cos^2 x)(1 - \cos^2 y) \\
&= \cos^2 x - \cos^2 y.
\end{align*}
故结论得证.
\end{proof}

\subsection{反三角函数}

\begin{theorem}[常用反三角函数性质]\label{theorem:常用反三角函数性质}
\begin{enumerate}
\item $\arcsin x+\arcsin y=\begin{cases}
\arcsin \left( x\sqrt{1-y^2}+y\sqrt{1-x^2} \right) &,xy<0\text{或}x^2+y^2\leqslant slant 1\\
\pi -\arcsin \left( x\sqrt{1-y^2}+y\sqrt{1-x^2} \right) &,x>0,y>0,x^2+y^2>1\\
-\pi -\arcsin \left( x\sqrt{1-y^2}+y\sqrt{1-x^2} \right) &,x<0,y<0,x^2+y^2>1\\
\end{cases}.$

\item $\arcsin x-\arcsin y=\begin{cases}
\arcsin \left( x\sqrt{1-y^2}-y\sqrt{1-x^2} \right) &,xy\geqslant slant 0\text{或}x^2+y^2\leqslant slant 1\\
\pi -\arcsin \left( x\sqrt{1-y^2}-y\sqrt{1-x^2} \right) &,x>0,y<0,x^2+y^2>1\\
-\pi -\arcsin \left( x\sqrt{1-y^2}-y\sqrt{1-x^2} \right) &,x<0,y>0,x^2+y^2>1\\
\end{cases}.$

\item $\arccos x+\arccos y=\begin{cases}
\arccos \left( xy-\sqrt{1-x^2}\sqrt{1-y^2} \right) &,x+y\geqslant slant 0\\
2\pi -\arccos \left( xy-\sqrt{1-x^2}\sqrt{1-y^2} \right) &,x+y<0\\
\end{cases}.$

\item $\arccos x-\arccos y=\begin{cases}
-\arccos \left( xy+\sqrt{1-x^2}\sqrt{1-y^2} \right) &,x\geqslant slant y\\
\arccos \left( xy+\sqrt{1-x^2}\sqrt{1-y^2} \right) &,x<y\\
\end{cases}.$

\item $\arctan x+\arctan y=\begin{cases}
\arctan \frac{x+y}{1-xy}&,xy<1\\
\pi +\arctan \frac{x+y}{1-xy},x>0&,xy>1\\
-\pi +\arctan \frac{x+y}{1-xy},x<0&,xy>1\\
\end{cases}.$

\item $\arctan x-\arctan y=\begin{cases}
\arctan \frac{x-y}{1+xy}&,xy>-1\\
\pi +\arctan \frac{x-y}{1+xy}&,x>0,xy<-1\\
-\pi +\arctan \frac{x-y}{1+xy}&,x<0,xy<-1\\
\end{cases}.$

\item $2\arcsin x=\begin{cases}
\arcsin \left( 2x\sqrt{1-x^2} \right) &,\left| x \right|\leqslant slant \frac{\sqrt{2}}{2}\\
\pi -\arcsin \left( 2x\sqrt{1-x^2} \right) &,\frac{\sqrt{2}}{2}<x\leqslant slant 1\\
-\pi -\arcsin \left( 2x\sqrt{1-x^2} \right) &,-1\leqslant slant x<-\frac{\sqrt{2}}{2}\\
\end{cases}.$

\item $2\arccos x=\begin{cases}
\arccos \left( 2x^2-1 \right) &,0\leqslant slant x\leqslant slant 1\\
2\pi -\arccos \left( 2x^2-1 \right) &,-1\leqslant slant x<0\\
\end{cases}.$

\item $2\arctan x=\begin{cases}
\arctan \frac{2x}{1-x^2},\left| x \right|\leqslant slant 1\\
\pi +\arctan \frac{2x}{1-x^2}&,\left| x \right|>1\\
-\pi +\arctan \frac{2x}{1-x^2}&,x<-1\\
\end{cases}.$

\item $\cos \left( n\arccos x \right) =\frac{\left( x+\sqrt{x^2-1} \right) ^n+\left( x-\sqrt{x^2-1} \right) ^n}{2}\left( n\geqslant slant 1 \right) .$
\end{enumerate}
\end{theorem}
\begin{proof}

\end{proof}

\begin{proposition}\label{proposition:arctan相关等式}
$\arctan x+\arctan \frac{1}{x}=\begin{cases}
\frac{\pi}{2},&x>0\\
-\frac{\pi}{2},&x<0\\
\end{cases}$.
\end{proposition}
\begin{proof}
令 \( f(x)=\arctan x+\arctan\frac{1}{x} \),则
\begin{align*}
f'(x)&=\frac{1}{x^2 + 1}+\frac{1}{(\frac{1}{x})^2 + 1}(-\frac{1}{x^2})=\frac{1}{x^2 + 1}-\frac{1}{x^2 + 1}=0
\end{align*}
故 \( f(x) \) 为常函数,于是就有 \( f(x)=f(1)=\frac{\pi}{2},\forall x>0 \) ;\( f(x)=f(-1)=-\frac{\pi}{2},\forall x<0 \).
\end{proof}


\subsection{双曲三角函数}

\begin{proposition}
\begin{enumerate}[(1)]
\item $\cosh x=\frac{e^{x}+e^{-x}}{2}\geqslant slant1,$

\item $\sinh x=\frac{e^{x}-e^{-x}}{2}\geqslant slant x.$
\end{enumerate}
\end{proposition}
\begin{proof}
可以分别利用均值不等式和求导进行证明. 
\end{proof}

\begin{proposition}
\begin{enumerate}
\item $\cosh^2 x-\sinh^2 x=1$.

\item $\cosh(2x)=2\cosh^2 x-1=1-2\sinh^2 x$.

\item $\sinh(2x)=2\sinh x\cosh x$. 
\end{enumerate}
\end{proposition}
\begin{proof}

\end{proof}


\end{document}
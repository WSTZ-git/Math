\documentclass[../../main.tex]{subfiles}
\graphicspath{{\subfix{../../image/}}} % 指定图片目录,后续可以直接使用图片文件名。

% 例如:
% \begin{figure}[H]
% \centering
% \includegraphics{image-01.01}
% \caption{图片标题}
% \label{figure:image-01.01}
% \end{figure}
% 注意:上述\label{}一定要放在\caption{}之后,否则引用图片序号会只会显示??.

\begin{document}

\section{将多项式分式分解为其部分因式的和}

\begin{example}
\begin{enumerate}
\item 分解\(a > 0\),\(\frac{1}{(1 + x^2)(1 + ax)}\).

\item 分解\(\frac{1}{(1 + x^2)(1 + x)^2}\).

\item 分解\(\frac{1}{(1 + x^2)^2(1 + x)}\).

\item 分解\(\frac{1}{(1 + x^2)^2(1 + x)^2}\).
\end{enumerate}
\end{example}
\begin{solution}\label{将多项式分式分解为其部分因式的和常用方法}
\begin{enumerate}
\item 根据代数学知识我们可以设
\begin{align}\label{equation:little tech eq1}
\frac{1}{\left( 1+x^2 \right) \left( 1+ax \right)}=\frac{Ax+B}{1+x^2}+\frac{C}{1+ax}.  
\end{align}
其中$A,B,C$均为常数.

{\color{blue}解法一(待定系数法):}

将\eqref{equation:little tech eq1}式右边通分得到
\begin{align*}
\frac{Ax+B}{1+x^2}+\frac{C}{1+ax}=\frac{\left( Ax+B \right) \left( 1+ax \right) +C\left( 1+x^2 \right)}{\left( 1+x^2 \right) \left( 1+ax \right)}=\frac{\left( Aa+C \right) x^2+\left( A+Ba \right) x+B+C}{\left( 1+x^2 \right) \left( 1+ax \right)}.
\end{align*}
比较上式左右两边分子各项系数可得
\begin{align*}
\begin{cases}
Aa+C=0\\
A+Ba=0\\
B+C=1\\
\end{cases}
\end{align*}
解得:$A=-\frac{a}{1+a^2},B=\frac{1}{1+a^2},C=\frac{a^2}{1+a^2}$.

于是原式可分解为
\begin{align*}
\frac{1}{\left( 1+x^2 \right) \left( 1+ax \right)}=\frac{-\frac{a}{1+a^2}x+\frac{1}{1+a^2}}{1+x^2}+\frac{\frac{a^2}{1+a^2}}{1+ax}.
\end{align*}
{\color{blue}解法二(留数法):}

\eqref{equation:little tech eq1}式两边同时乘$1+ax$,得到$\frac{1}{1+x^2}=\frac{Ax+B}{1+x^2}\cdot \left( 1+ax \right) +C$.再令$x\to-\frac{1}{a}$,得$C=\frac{1}{1+\frac{1}{a^2}}=\frac{a^2}{1+a^2}$.

\eqref{equation:little tech eq1}式两边同时乘$1+x^2$,得到
$\frac{1}{1+ax}=Ax+B+\frac{C}{1+ax}\cdot \left( 1+x^2 \right)$.再分别令$x\to \pm i$,可得
\begin{align*}
\begin{cases}
A\mathrm{i}+B=\frac{1}{1+a\mathrm{i}}\\
-A\mathrm{i}+B=\frac{1}{1-a\mathrm{i}}\\
\end{cases}
\end{align*}
解得:$A=-\frac{a}{1+a^2},B=\frac{1}{1+a^2}$.

于是原式可分解为
\begin{align*}
\frac{1}{\left( 1+x^2 \right) \left( 1+ax \right)}=\frac{-\frac{a}{1+a^2}x+\frac{1}{1+a^2}}{1+x^2}+\frac{\frac{a^2}{1+a^2}}{1+ax}.
\end{align*}
{\color{blue}解法三(留数法+待定系数法):}

\eqref{equation:little tech eq1}式两边同时乘$1+ax$,得到$\frac{1}{1+x^2}=\frac{Ax+B}{1+x^2}\cdot \left( 1+ax \right) +C$.再令$x\to-\frac{1}{a}$,得$C=\frac{1}{1+\frac{1}{a^2}}=\frac{a^2}{1+a^2}$.

容易直接观察出\eqref{equation:little tech eq1}式右边通分后分子的最高次项系数为$Aa+C$,常数项为$B+C$.并将其与\eqref{equation:little tech eq1}式左边的分子对比,可以得到
\begin{align*}
\begin{cases}
Aa+C=0\\
B+C=1\\
\end{cases}
\end{align*}
解得:$A=-\frac{a}{1+a^2},B=\frac{1}{1+a^2}$.

于是原式可分解为
\begin{align*}
\frac{1}{\left( 1+x^2 \right) \left( 1+ax \right)}=\frac{-\frac{a}{1+a^2}x+\frac{1}{1+a^2}}{1+x^2}+\frac{\frac{a^2}{1+a^2}}{1+ax}.
\end{align*}

\item 根据代数学知识我们可以设
\begin{align}\label{equation:little tech eq2}
\frac{1}{\left( 1+x^2 \right) \left( 1+x \right) ^2}=\frac{Ax+B}{1+x^2}+\frac{C}{1+x}+\frac{D}{\left( 1+x \right) ^2}.  
\end{align}
其中$A,B,C,D$均为常数.

\eqref{equation:little tech eq2}式两边同时乘$(1+x)^2$,得到
\begin{align}\label{eq1.1}
\frac{1}{1+x^2}=\frac{Ax+B}{1+x^2}\cdot \left( 1+x \right) ^2+C\left( 1+x \right)+D.
\end{align}
再令$x\to-1$,可得$D=\frac{1}{2}$.对\eqref{eq1.1}式两边同时求导得到
\begin{align*}
\left.\frac{-2x}{\left( 1+x^2 \right) ^2}\right|_{x\rightarrow -1}^{}=\left.\left[ \frac{Ax+B}{1+x^2}\cdot \left( 1+x \right) ^2 \right] ^{\prime}\right|_{x\rightarrow -1}+C=C.
\end{align*}
从而$C=\frac{1}{2}$.
令\eqref{equation:little tech eq2}中的$x=0$,得到$1=B+C+D$,将$C=D=\frac{1}{2}$代入解得:$B=0$.再令\eqref{equation:little tech eq2}中的$x=1$,得到$\frac{1}{8}=\frac{A+B}{2}+\frac{C}{2}+\frac{D}{4}$,将$C=D=\frac{1}{2},B=0$代入解得:$A=-\frac{1}{2}$.

于是原式可分解为
\begin{align*}
\frac{1}{\left( 1+x^2 \right) \left( 1+x \right) ^2}=\frac{-x}{2\left( 1+x^2 \right)}+\frac{1}{2+2x}+\frac{1}{2\left( 1+x \right) ^2}.
\end{align*}

\item 

\item 
\end{enumerate}
\end{solution}

\begin{example}
分解$\frac{1}{1+x^4}$.
\end{example}
\begin{solution}
首先我们注意到
\begin{align*}
\frac{1}{1+x^4}=\frac{1}{\left( 1+x^2 \right) -2x^2}=\frac{1}{\left( x^2-\sqrt{2}x+1 \right) \left( x^2+\sqrt{2}x+1 \right)}.
\end{align*}
然后根据代数学知识我们可以设
\begin{align}\label{equation:little tech eq3}
\frac{1}{1+x^4}=\frac{1}{\left( x^2-\sqrt{2}x+1 \right) \left( x^2+\sqrt{2}x+1 \right)}=\frac{Ax+B}{x^2-\sqrt{2}x+1}+\frac{Cx+D}{x^2+\sqrt{2}x+1}.  
\end{align}
其中$A,B,C,D$均为常数.
将上式右边通分可得
\begin{align*}
\frac{1}{\left( x^2-\sqrt{2}x+1 \right) \left( x^2+\sqrt{2}x+1 \right)}=\frac{\left( Ax+B \right) \left( x^2+\sqrt{2}x+1 \right) +\left( Cx+D \right) \left( x^2-\sqrt{2}x+1 \right)}{\left( x^2-\sqrt{2}x+1 \right) \left( x^2+\sqrt{2}x+1 \right)}.
\end{align*}
比较上式左右两边分子各项系数可得
\begin{align*}
\begin{cases}
B+D=1\\
A+\sqrt{2}B+C-\sqrt{2}D=0\\
A\sqrt{2}+B-C\sqrt{2}+D=0\\
A+C=0\\
\end{cases}
\end{align*}
解得:$A=-\frac{\sqrt{2}}{4},B=\frac{1}{2},C=\frac{\sqrt{2}}{4},D=\frac{1}{2}$.

于是原式可分解为
\begin{align*}
\frac{1}{1+x^4}=\frac{-\frac{\sqrt{2}}{4}x+\frac{1}{2}}{x^2-\sqrt{2}x+1}+\frac{\frac{\sqrt{2}}{4}x+\frac{1}{2}}{x^2+\sqrt{2}x+1}.
\end{align*}
\end{solution}

\begin{example}
分解$\frac{x^4}{(1+x)(1+x^2)}$.
\end{example}
\begin{solution}
先利用多项式除法用$x^4$除以$(1+x)(1+x^2)$得到$x^4=\left( x-1 \right) \left( 1+x \right) \left( 1+x^2 \right) +1$.从而
\begin{align*}
\frac{x^4}{\left( 1+x \right) \left( 1+x^2 \right)}=\frac{\left( x-1 \right) \left( 1+x \right) \left( 1+x^2 \right) +1}{\left( 1+x \right) \left( 1+x^2 \right)}=x-1+\frac{1}{\left( 1+x \right) \left( 1+x^2 \right)}.
\end{align*}
然后再利用多项式分式的分解方法(\hyperref[将多项式分式分解为其部分因式的和常用方法]{待定系数法和留数法})将$\frac{1}{\left( 1+x \right) \left( 1+x^2 \right)}$分解为部分因式的和.最后我们可将原式分解为
\begin{align*}
\frac{x^4}{\left( 1+x \right) \left( 1+x^2 \right)}=x-1+\frac{1}{2+2x}+\frac{-x+1}{2+2x^2}.
\end{align*}
\end{solution}










\end{document}
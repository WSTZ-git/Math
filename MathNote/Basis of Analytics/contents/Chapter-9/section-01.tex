\documentclass[../../main.tex]{subfiles}
\graphicspath{{\subfix{../../image/}}} % 指定图片目录,后续可以直接使用图片文件名。

% 例如:
% \begin{figure}[H]
% \centering
% \includegraphics[scale=0.3]{image-01.01}
% \caption{图片标题}
% \label{figure:image-01.01}
% \end{figure}
% 注意:上述\label{}一定要放在\caption{}之后,否则引用图片序号会只会显示??.

\begin{document}

\section{长除法}

\begin{example}[利用多项式除法计算Taylor级数和Laurent级数]

已知$\sin x=x-\frac{1}{6}x^3+\frac{1}{120}x^5+\cdots$,$\cos x=1-\frac{1}{2}x^2+\frac{1}{24}x^4-\cdots$.

1.求$\tan x$.
\quad \quad
2.求$\frac{1}{\sin^2 x}$.
\end{example}
\begin{note}
实际问题中需要多展开几项,展开得越多,得到的结果也越多.
\end{note}
\begin{solution}
1.根据多项式除法可得
\begin{align*}
\begin{tblr}{colspec = {crcrcrcrcr},
colsep = 3pt,
vline{2} = {2}{}
}
&  x& + & \frac{1}{3}x^3 & +& \frac{2}{15}x^5   & + &\cdots &  &  \\
\cline{2-10}
1-\frac{1}{2}x^2+\frac{1}{24}x^4+\cdots   & x & - & \frac{1}{6}x^3 & + & \frac{1}{120}x^5  & - & \cdots  &  &  \\
& x & - & \frac{1}{2}x^3 & + & \frac{1}{24}x^5 &  +  &\cdots  &  &  \\
\cline{2-10}
&  & - & \frac{1}{3}x^2 & - & \frac{1}{30}x^5 &  +  &\cdots &  &  \\
&  & - & \frac{1}{3}x^2 & - & \frac{1}{12}x^5 &   +&\cdots  &  &  \\
\cline{3-10}
&  &  &  &  & \frac{2}{15}x^5 &  +& \cdots &  &  \\
&  &  &  &  & \frac{2}{15}x^5 & + & \cdots  &  &  \\
\cline{5-10}
&  &  &  &  & 0 & + & \cdots &  &  \\
\end{tblr}
\end{align*}
因此$\tan x=\frac{\sin x}{\cos x}=x+\frac{1}{3}x^3+\frac{2}{15}x^5+\cdots\,\,.$
\\
\\
2.根据多项式乘法可得
\begin{align*}
\sin ^2x=\left( x-\frac{1}{6}x^3+\frac{1}{120}x^5+\cdots \right) \left( x-\frac{1}{6}x^3+\frac{1}{120}x^5+\cdots \right) =x^2-\frac{1}{3}x^4+\cdots 
\end{align*}
再根据多项式除法可得
\begin{align*}
\begin{tblr}{colspec = {crcrcrcrcr},
colsep = 3pt,
vline{2} = {2}{}
}
&  \frac{1}{x^2}& - & \frac{1}{3} & + & \cdots  &  & &  &  \\
\cline{2-10}
x^2-\frac{1}{3}x^4+\cdots & 1 &  &  &  &   &  &   &  &  \\
& 1 & - & \frac{1}{3}x^2 & + & \cdots &  &  &  &  \\
\cline{2-10}
&   &  & \frac{1}{3}x^2 & + & \cdots &  &  &  &  \\
&   &  & \frac{1}{3}x^2 & + & \cdots &  &  &  &  \\
\cline{3-10}
&  &  &   0&  +& \cdots & & &  &  \\
\end{tblr}
\end{align*}
因此$\frac{1}{\sin ^2x}=\frac{1}{x^2}-\frac{1}{3}+\cdots\,\,.$
\end{solution}


\end{document}
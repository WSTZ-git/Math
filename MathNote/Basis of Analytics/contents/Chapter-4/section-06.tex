\documentclass[../../main.tex]{subfiles}
\graphicspath{{\subfix{../../image/}}} % 指定图片目录,后续可以直接使用图片文件名。

% 例如:
% \begin{figure}[H]
% \centering
% \includegraphics{image-01.01}
% \caption{图片标题}
% \label{figure:image-01.01}
% \end{figure}
% 注意:上述\label{}一定要放在\caption{}之后,否则引用图片序号会只会显示??.

\begin{document}

\section{递推数列求极限和估阶}


\subsection{“折线图(蛛网图)”分析法(图未完成,但已学会)}
关于递推数列求极限的问题,可以先画出相应的"折线图",然后根据“折线图(蛛网图)”的性质来判断数列的极限.这种方法可以帮助我们快速得到数列的极限,但是对于数列的估阶问题,这种方法并不适用.

\begin{remark}
这种方法只能用来分析问题,严谨的证明还是需要用单调性分析法或压缩映像法书写.

一般的递推数列问题,我们先画“折线图(蛛网图)”分析,分析出数列(或奇偶子列)的收敛情况,就再用单调分析法或压缩映像法严谨地书写证明.

如果递推函数是单调递增的,则画蛛网图分析起来非常方便,书写证明过程往往用单调有界(单调性分析法)就能解决问题.
\end{remark}

\begin{example}
设\(u_1 = b\),\(u_{n + 1}=u_{n}^{2}+(1 - 2a)u_{n}+a^{2}\),求$a,b$的值使得$a_n$收敛,并求其极限.
\end{example}
\begin{note}
显然递推函数只有一个不动点$x=a$,画蛛网图分析能够快速地得到取不同初值时,$u_n$的收敛情况.但是注意需要严谨地书写证明过程.
\end{note}
\begin{solution}
由条件可得
\[
u_{n + 1}=u_{n}^{2}+(1 - 2a)u_n + a^2=(u_n - a)^2 + u_n\geqslant u_n.
\]
故 \(u_n\) 单调递增。 
\((\mathrm{i})\) 若 \(b > a\),则由 \(u_n\) 单调递增可知,\(u_n > a,\forall n\in \mathbb{N}_+\)。又由单调有界定理可知 \(u_n\) 要么发散到 \(+\infty\),要么收敛到一个有限数。假设 \(u_n\) 收敛,则可设 \(\lim_{n\rightarrow \infty}u_n = u > u_1 > a\)。从而由递推条件可得
\[
u=(u - a)^2 + u\Rightarrow u = a
\]
矛盾。故 \(\lim_{n\rightarrow \infty}u_n = +\infty\)。

\((\mathrm{ii})\) 若 \(b = a\),则由递推条件归纳可得 \(u_n = a,\forall n\in \mathbb{N}_+\)。

\((\mathrm{iii})\) 若 \(b\in [a - 1,a]\),令 \(f(x)=x^2+(1 - 2a)x + a^2\),则
\[
a - 1 < a - \frac{1}{4}=f\left(\frac{2a - 1}{2}\right)\leqslant f(x)\leqslant \max\{f(a - 1),f(a)\}=a,\forall x\in [a - 1,a].
\]
由于 \(u_1 = b\in [a - 1,a]\),假设 \(u_n\in [a - 1,a]\),则
\[
a - 1\leqslant u_{n + 1}=f(u_n)\leqslant a.
\]
由数学归纳法可得 \(u_n\in [a - 1,a],\forall n\in \mathbb{N}_+\)。于是由单调有界定理可知 \(u_n\) 收敛。再对 \(u_{n + 1}=u_{n}^{2}+(1 - 2a)u_n + a^2\) 两边同时取极限,解得 \(\lim_{n\rightarrow \infty}u_n = a\)。

\((\mathrm{iv})\) 若 \(b < a - 1\),则
\[
u_2=(u_1 - a)^2 + u_1 > a\Leftrightarrow (b - a)^2 + b > a\Leftrightarrow (b - a)(b - a + 1) > 0.
\]
由 \(b < a - 1\) 可知上式最后一个不等式显然成立,故 \(u_2 > a\)。于是由 \((\mathrm{i})\) 同理可证 \(\lim_{n\rightarrow \infty}u_n = +\infty\)。

综上,只有当 \(a\in \mathbb{R}\),\(b\in [a - 1,a]\) 时,数列 \(u_n\) 才收敛,极限为 \(a\)。
\end{solution}

\begin{example}
设\(x_1>0,x_1\neq1,x_{n + 1}=\frac{x_n^2}{2(x_n - 1)}\),证明\(x_n\)收敛并求极限。
\end{example}
\begin{note}
显然递推函数有两个个不动点$x=0,2$,画蛛网图分析能够快速地得到取不同初值时,$x_n$的收敛情况.这里利用压缩映像书写过程更加简便.
\end{note}
\begin{solution}
(i) 如果 \(x_1 > 1\),则归纳易证 \(x_n \geq 2,\forall n\geq 2\),所以
\begin{align*}
|x_{n + 1} - 2|&=\left|\frac{x_n^2}{2(x_n - 1)} - 2\right|=\frac{(x_n - 2)^2}{2(x_n - 1)}=|x_n - 2|\left|\frac{x_n - 2}{2(x_n - 1)}\right|\leq\frac{1}{2}|x_n - 2|\leq\cdots\leq\frac{1}{2^n}|x_1 - 2|
\end{align*}
令$n\to \infty$,由此可知$x_n$的极限是 \(2\)。

(ii)如果 \(x_1\in(0,1)\),则归纳易证 \(x_n\leq0,\forall n\geq 2\),所以
\begin{align*}
|x_{n + 1}|&=\left|\frac{x_n^2}{2(x_n - 1)}\right|=|x_n|\left|\frac{x_n}{2(x_n - 1)}\right|\leq\frac{1}{2}|x_n|\leq\cdots\leq\frac{1}{2^n}|x_1|
\end{align*}
令$n\to \infty$,由此可知$x_n$的极限是 \(0\)。
\end{solution}

\begin{example}
设\(S_1 = 1,S_{n + 1}=S_n+\frac{1}{S_n}-\sqrt{2}\),证明:\(\lim_{n\rightarrow\infty}S_n=\frac{1}{\sqrt{2}}\)。
\end{example}
\begin{note}
递推函数性质及例题分析
递推函数递减时候,意味着奇偶两个子列具有相反的单调性,本题没有产生新的不动点,是容易的。

画蛛网图分析表明递推函数(在\((0,1)\)内)是递减的,所以数列不单调,但是奇偶子列分别单调,并且(这一步只能说“似乎”,因为对于不同的递减的递推式,可能结论是不一样的,取决于二次复合有没有新的不动点)奇子列单调递增趋于\(\frac{1}{\sqrt{2}}\),偶子列单调递减趋于\(\frac{1}{\sqrt{2}}\),数列的范围自然是在\([S_1,S_2]\)之间,显然不动点只有\(\frac{1}{\sqrt{2}}\)一个,因此证明单调有界即可解决问题.
\end{note}
\begin{proof}
\(S_1 = 1,S_2 = 2-\sqrt{2}\),先证明\(S_n\in[2 - \sqrt{2},1]\)恒成立,采用归纳法。
\(n = 1,2\)时显然成立,如果\(n\)时成立,则\(n + 1\)时,注意\(f(x)=x+\frac{1}{x}-\sqrt{2}\)在区间\((0,1)\)中单调递减,所以
\[2-\sqrt{2}\leq S_{n + 1}=S_n+\frac{1}{S_n}-\sqrt{2}\leq2-\sqrt{2}+\frac{1}{2 - \sqrt{2}}-\sqrt{2}=2-2\sqrt{2}+\frac{2+\sqrt{2}}{2}=3-\frac{3}{2}\sqrt{2}\leq1\]
这就证明了\(S_n\)是有界数列,且\(S_3\leq S_1,S_4\geq S_2\),下面证明\(S_{2n - 1}\)递减,\(S_{2n}\)递增:
注意函数\(f(x)=x+\frac{1}{x}-\sqrt{2}\)在区间\((0,1)\)中单调递减,所以如果已知\(S_{2n + 1}\leq S_{2n - 1},S_{2n + 2}\geq S_{2n}\),则
\[S_{2n + 3}=f(S_{2n + 2})\leq f(S_{2n})=S_{2n + 1},S_{2n + 4}=f(S_{2n + 3})\geq f(S_{2n + 1})=S_{2n + 2}\]
根据归纳法可得单调性,这说明\(S_{2n - 1},S_{2n}\)都是单调有界的,因此极限存在,设
\[\lim_{n\rightarrow\infty}S_{2n - 1}=a,\lim_{n\rightarrow\infty}S_{2n}=b,a,b\in[2 - \sqrt{2},1]\]
在递推式\(S_{n + 1}=S_n+\frac{1}{S_n}-\sqrt{2}\)中分别让\(n\)取奇数,偶数,然后令\(n\rightarrow\infty\)取极限,可得关于极限\(a,b\)的方程组\(a = b+\frac{1}{b}-\sqrt{2},b = a+\frac{1}{a}-\sqrt{2}\),希望证明\(a = b=\frac{1}{\sqrt{2}}\),为了解这个方程组,三种方法:

{\color{blue}方法一:}直接硬算,将其中一个式子代入到另一个中
\begin{align*}
a=b+\frac{1}{b}-\sqrt{2}=a+\frac{1}{a}-\sqrt{2}+\frac{1}{a+\frac{1}{a}-\sqrt{2}}-\sqrt{2}=\frac{1 - 3\sqrt{2}a + 7a^2-3\sqrt{2}a^3 + a^4}{a(1-\sqrt{2}a + a^2)}\\
1 - 3\sqrt{2}a + 7a^2-3\sqrt{2}a^3 + a^4-a^2(1-\sqrt{2}a + a^2)=-\left(\sqrt{2}a - 1\right)^3=0
\end{align*}
由此可知\(a = b=\frac{1}{\sqrt{2}}\),所以数列\(S_n\)收敛于\(\frac{1}{\sqrt{2}}\)。

{\color{blue}方法二:}上面硬算起来实在太麻烦了,我们可以先对递推式变形化简,减小计算量
\begin{align*}
S_{n + 1}&=S_n+\frac{1}{S_n}-\sqrt{2}=\frac{S_n^2-\sqrt{2}S_n + 1}{S_n}=\frac{\left(S_n-\frac{\sqrt{2}}{2}\right)^2+\frac{1}{2}}{S_n}\\
\Rightarrow S_{n + 1}-\frac{\sqrt{2}}{2}&=\frac{\left(S_n-\frac{\sqrt{2}}{2}\right)^2+\frac{1}{2}-\frac{\sqrt{2}}{2}S_n}{S_n}=\frac{\left(S_n-\frac{\sqrt{2}}{2}\right)(S_n-\sqrt{2})}{S_n}
\end{align*}
然后对奇偶子列(代入递推式)分别取极限可得方程组
\[a-\frac{\sqrt{2}}{2}=\frac{\left(b-\frac{\sqrt{2}}{2}\right)(b - \sqrt{2})}{b},b-\frac{\sqrt{2}}{2}=\frac{\left(a-\frac{\sqrt{2}}{2}\right)(a - \sqrt{2})}{a}\]
如果\(a,b\)之中有一个是\(\frac{1}{\sqrt{2}}\),则另一个也是,显然数列\(S_n\)收敛于\(\frac{1}{\sqrt{2}}\),如果都不是则
\begin{align*}
&a-\frac{\sqrt{2}}{2}=\frac{\left( b-\frac{\sqrt{2}}{2} \right) (b-\sqrt{2})}{b}=\frac{\left( a-\frac{\sqrt{2}}{2} \right) (a-\sqrt{2})(b-\sqrt{2})}{ab}
\\
&\Rightarrow \left( a-\sqrt{2} \right) \left( b-\sqrt{2} \right) -ab=2-\sqrt{2}(a+b)=0\Rightarrow a+b=\sqrt{2}
\\
&\Rightarrow a-\frac{\sqrt{2}}{2}=\frac{\sqrt{2}}{2}-b=\frac{\left( b-\frac{\sqrt{2}}{2} \right) (b-\sqrt{2})}{b}\Rightarrow b-\sqrt{2}=-b,b=\frac{\sqrt{2}}{2}=a.
\end{align*}
导致矛盾。

{\color{blue}方法三:}(最快的方法):如果\(a\neq b\),则根据方程组\(a = b+\frac{1}{b}-\sqrt{2},b = a+\frac{1}{a}-\sqrt{2}\)有
\begin{align*}
ab&=b^2-\sqrt{2}b+1=a^2-\sqrt{2}a+1\Rightarrow a^2-b^2=\sqrt{2}\left( a-b \right) \Rightarrow a+b=\sqrt{2}\\
\Rightarrow &b=a+\frac{1}{a}-\sqrt{2}=\sqrt{2}-a\Rightarrow 2\sqrt{2}=2a+\frac{1}{a}\geq2\sqrt{2a\cdot\frac{1}{a}}=2\sqrt{2}
\end{align*}
最后一个不等式等号成立当且仅当$a=\frac{\sqrt{2}}{2}$,由此可知\(a = b=\frac{1}{\sqrt{2}}\) 矛盾。
\end{proof}
\begin{remark}
一般来说,递推函数递减时候是否收敛完全取决于递推函数二次复合之后在区间内(这个数列的最大,最小值对应的区间)是否会有新的不动点,如果没有就收敛,如果有,则通常奇偶子列收敛到不同极限,于是数列不收敛。
可以看到核心是二次复合后是否有新的不动点,也即解方程\(f(f(x)) = x\),一般不建议硬算,尤其是多项式或者分式类型,往往化为两个方程\(a = f(b),b = f(a)\)然后作差会比较方便,只有出现超越函数时候,才有必要真的把二次复合化简算出来,然后硬解方程,或者求导研究问题,这样“迫不得已”的例子见最后一个练习题。
\end{remark}


\begin{example}
定义数列\(a_0 = x\),\(a_{n + 1}=\frac{a_{n}^{2}+y^{2}}{2}\),\(n = 0,1,2,\cdots\),求\(D\triangleq\{(x,y)\in\mathbb{R}^2:\text{数列}a_n\text{收敛}\}\)的面积.
\end{example}
\begin{solution}

\end{solution}





\subsection{单调性分析法}

\begin{proposition}[不动点]\label{proposition:不动点}
设数列 \(\{x_n\}\) 满足递推公式 \(x_{n+1} = f(x_n), n \in \mathbb{N}_+\)。若有 \(\lim\limits_{n \to \infty} x_n = \xi\), 同时又成立 \(\lim\limits_{n \to \infty} f(x_n) = f(\xi)\) 则极限 \(\xi\) 一定是方程 \(f(x) = x\) 的根 (这时称 \(\xi\) 为函数 \(f\) 的不动点).
\end{proposition}
\begin{proof}
对$x_{n+1} = f(x_n)$两边取极限即得.
\end{proof}

关于递推数列求极限和估阶的问题,单调性分析法只适用于
\[
x_{n + 1} = f(x_n),n \in \mathbb{N}.
\]

\(f\)是递增或者递减的类型,且大多数情况只适用于\(f\)递增情况,其余情况不如压缩映像思想方便快捷.显然递推数列$x_{n + 1} = f(x_n)$确定的\(x_n\)如果收敛于\(x \in \mathbb{R}\),则当\(f\)连续时一定有\(f(x) = x\),此时我们也把这个\(x\)称为\(f\)的不动点.因此\(f(x) = x\)是\(x_n\)收敛于\(x \in \mathbb{R}\)的必要条件.



\begin{proposition}[递增函数递推数列]\label{proposition:递增函数递推数列}
设\(f\)是递增函数,则递推
\begin{align}\label{equation:486565}
x_{n + 1} = f(x_n),n \in \mathbb{N}.
\end{align}
确定的\(x_n\)一定单调,且和不动点大小关系恒定.
\end{proposition}
\begin{note}
本结论表明由递增递推\eqref{equation:486565}确定的数列的单调性和有界性,完全由其\(x_2 - x_1\)和\(x_1\)与不动点$x_0$的大小关系确定.即$x_2>x_1\Rightarrow x_{n+1}>x_n,\forall n\in \mathbb{N} _+.x_1>x_0\Rightarrow x_n>x_0,\forall n\in \mathbb{N} _+$.
\end{note}
\begin{proof}
我们只证一种情况,其余情况是完全类似的.设\(x_0\)是\(f\)的不动点且\(x_1\leq x_0,x_2\geq x_1\),则若\(x_n\leq x_{n + 1},x_n\leq x_0,n\in\mathbb{N}\),运用\(f\)递增性有
\[
x_{n + 1} = f(x_n)\leq f(x_0) = x_0,x_{n + 2} = f(x_{n + 1})\geq f(x_n) = x_{n + 1}.
\]
由数学归纳法即证明了\hyperref[proposition:递增函数递推数列]{命题\ref{proposition:递增函数递推数列}}
\end{proof}

\begin{proposition}[递减函数递推数列]\label{proposition:递减函数递推数列}
设\(f\)是递减函数,则递推
\begin{align}\label{equation:486561}
x_{n + 1} = f(x_n),n \in \mathbb{N}.
\end{align}
确定的\(\{x_n\}\)一定不单调,且和不动点大小关系交错.但$\{x_n\}$的两个奇偶子列$\{x_{2k-1}\}$和$\{x_{2k}\}$分别为单调数列,且具有相反的单调性.
\end{proposition}
\begin{note}
我们注意到\(f\circ f\)递增就能把\(f\)递减转化为递增的情况,本结论无需记忆或证明,只记得思想即可.$x_n$和不动点关系交错,即若$x_0$为数列${x_n}$的不动点,且$x_1\geq x_0,x_2\leq x_0$,则$x_3 \geq x_0,\cdots,x_{2n}\leq x_0,x_{2n-1}\geq x_0,\cdots$;并且$x_2\leq x_1,x_3\geq x_1,x_4\leq x_2,x_5\geq x_3,\cdots,x_{2n}\leq x_{2n-2},x_{2n-1}\geq x_{2n-3},\cdots$.
\end{note}
\begin{proof}
由\hyperref[proposition:递增函数递推数列]{命题\ref{proposition:递增函数递推数列}}类似证明即可.
\end{proof}

\begin{example}[$\,\,$递增/递减递推数列]
\begin{enumerate}
\item 设\(x_1 > - 6,x_{n + 1} = \sqrt{6 + x_n},n = 1,2,\cdots\),计算\(\lim_{n \to \infty} x_n\).
\item 设\(x_1,a > 0,x_{n + 1} = \frac{1}{4}(3x_n+\frac{a}{x_n^3}),n = 1,2,\cdots\),求极限\(\lim_{n \to \infty} x_n\).
\item 设\(x_1 = 2,x_n+(x_n - 4)x_{n - 1} = 3,(n = 2,3,\cdots)\),求极限\(\lim_{n \to \infty} x_n\).
\item 设\(x_1 > 0,x_n e^{x_{n + 1}} = e^{x_n}-1,n = 1,2,\cdots\),求极限\(\lim_{n \to \infty} x_n\).
\end{enumerate}
\end{example}
\begin{note}
\begin{enumerate}
\item 不妨设$x_1\geq 0$的原因:我们只去掉原数列$\{x_n\}$的第一项,得到一个新数列,并且此时新数列是从原数列$\{x_n\}$的第二项$x_2$开始的.对于原数列$\{x_n\}$而言,有$x_{n+1}=\sqrt{6+x_n}\geq 0,\forall n \in \mathbb{N}_+$.故新数列的每一项都大于等于0.将新数列重新记为$\{x_n\}$,则$x_1\geq 0$.若此时能够证得新数列收敛到$x_0$,则由于数列去掉有限项不会影响数列的敛散性以及极限值,可知原数列也收敛到$x_0$.故不妨设$x_1\geq 0$是合理地.

(简单地说,就是原数列用$x_2$代替$x_1$,用$x_{n+1}$代替$x_n$,$\forall n \in \mathbb{N}_+$,而由$x_1>-6$,可知$x_2=\sqrt{6+x_1}\geq 0$.)

\begin{remark}
{\color{blue}这种不妨设的技巧在数列中很常用,能减少一些不必要的讨论.实际上就是去掉数列中有限个有问题的项,而去掉这些项后对数列的极限没有影响.}
\end{remark}
\end{enumerate}
\end{note}
\begin{solution}
\begin{enumerate}
\item 不妨设$x_1\geq 0$,则设\(f(x)=\sqrt{6 + x}\),则\(f(x)\)单调递增.

当\(x_1 < 3\)时,由条件可知
\begin{gather}\label{eqq123}
x_2 - x_1=\sqrt{6 + x_1}-x_1=\frac{(3 - x_1)(2 + x_1)}{\sqrt{6 + x_1}+x_1}.
\end{gather}
从而此时\(x_2 > x_1\).假设当\(n = k\)时,有\(x_k < 3\).则当\(n = k + 1\)时,就有
\[
x_{k + 1}=f(x_k)=\sqrt{6 + x_k}<\sqrt{6 + 3}=3.
\]
故由数学归纳法,可知\(x_n < 3\),\(\forall n\in\mathbb{N}_+\).

假设当\(n = k\)时,有\(x_{k + 1}\geqslant x_k\).则当\(n = k + 1\)时,就有
\[
x_{k + 2}=f(x_{k + 1})\geqslant f(x_k)=x_{k + 1}.
\]
故由数学归纳法,可知\(\{x_n\}\)单调递增.
于是由单调有界定理,可得数列\(\{x_n\}\)收敛.

当\(x_1\geqslant 3\)时,由\eqref{eqq123}式可知,此时\(x_2\leqslant x_1\).假设当\(n = k\)时,有\(x_k\geqslant 3\).则当\(n = k + 1\)时,就有
\[
x_{k + 1}=f(x_k)=\sqrt{6 + x_k}\geqslant\sqrt{6 + 3}=3.
\]
故由数学归纳法,可知\(x_n\geqslant 3\),\(\forall n\in\mathbb{N}_+\).

假设当\(n = k\)时,有\(x_{k + 1}\leqslant x_k\).则当\(n = k + 1\)时,就有
\[
x_{k + 2}=f(x_{k + 1})\leqslant f(x_k)=x_{k + 1}.
\]
故由数学归纳法,可知\(\{x_n\}\)单调递减.
于是由单调有界定理,可得数列\(\{x_n\}\)收敛.

综上,无论\(x_1 > 3\)还是\(x_1\leqslant 3\),都有数列\(\{x_n\}\)收敛.设\(\lim_{n\rightarrow\infty}x_n = a\).则对\(x_{n + 1}=\sqrt{6 + x_n}\)两边同时令\(n\rightarrow\infty\)可得\(a=\sqrt{6 + a}\),解得\(\lim_{n\rightarrow\infty}x_n = a = 3\).

\item 

\item 

\item 
\end{enumerate}
\end{solution}

\begin{example}
设\(c,x_1\in(0,1)\),数列\(\{x_n\}\)满足\(x_{n + 1}=c(1 - x_n^2),x_2\neq x_1\),证明\(x_n\)收敛当且仅当\(c\in\left(0,\frac{\sqrt{3}}{2}\right)\).
\end{example}
\begin{proof}
根据题目显然有\(x_n\in(0,1)\),考虑函数\(f(x)=c(1 - x^2)\),则\(f(x)\)单调递减,并且\(f(x)=x\)在区间\((0,1)\)中有唯一解\(t_0 = \frac{\sqrt{1 + 4c^2}-1}{2c}\),则\(x_1\neq t_0\),不妨设\(x_1\in(0,t_0)\)(若不然\(x_1 > t_0\),则\(x_2 = f(x_1)<f(t_0)=t_0\),从\(x_2\)开始考虑即可),所以\(x_2>t_0,x_3<t_0,\cdots\)也即\(x_{2n - 1}<t_0,x_{2n}>t_0\)恒成立。

为了研究奇偶子列的单调性,考虑二次复合,计算有
\begin{align*}
f(f(x)) - x&=c\left(1 - c^2(1 - x^2)^2\right)-x=(-cx^2 + c - x)(c^2x^2 + cx + 1 - c^2)
\end{align*}
两个因子都是二次函数,前者开口向下,在\((0,1)\)区间中与\(y = x\)的唯一交点(横坐标)是\(t_0 = \frac{\sqrt{1 + 4c^2}-1}{2c}\),后者开口向上,解方程有(形式上)\(x=\frac{-c\pm\sqrt{4c^2 - 3}}{2c}\)。

因此我们应该以\(c = \frac{\sqrt{3}}{2}\)分类,当\(c\in\left(0,\frac{\sqrt{3}}{2}\right)\)时,\(c^2x^2 + cx + 1 - c^2\geq0\)也即当\(x\in(0,t_0)\)时\(f(f(x))\geq x\),\(x\in(t_0,1)\)时\(f(f(x))\leq x\),代入可知
\[x_1\leq x_3\leq x_5\leq\cdots\leq t_0,x_2\geq x_4\geq x_6\geq\cdots\geq t_0\]
也即奇子列单调递增有上界\(t_0\),偶子列单调递减有下界\(t_0\),所以奇偶子列分别都收敛,解方程\(f(f(x)) = x\)可知其在\((0,1)\)中有唯一解\(t_0 = \frac{\sqrt{1 + 4c^2}-1}{2c}\),所以奇偶子列收敛到同一值,数列收敛。

当\(c>\frac{\sqrt{3}}{2}\)时,{\color{blue}方法一:}显然有\(\frac{-c - \sqrt{4c^2 - 3}}{2c}<\frac{\sqrt{1 + 4c^2}-1}{2c}<\frac{-c + \sqrt{4c^2 - 3}}{2c}\),从左至右依次记为\(t_1<t_0<t_2\)。
采用反证法,如果\(x_n\)收敛,则解方程\(f(x)=x\)可知\(x_n\rightarrow t_0\),注意\(x_{2n - 1}\in(0,t_0),x_{2n}\in(0,1)\)并且反证法表明这两个子列也都收敛到\(t_0\),则存在\(N\)使得\(n > N\)时恒有\(x_{2n - 1}\in(t_1,t_0),x_{2n}\in(t_0,t_2)\)。
注意
\[f(f(x)) - x=(-cx^2 + c - x)(c^2x^2 + cx + 1 - c^2)\]
因此在区间\((t_1,t_0)\)中\(f(f(x))<x\),区间\((t_0,t_2)\)中\(f(f(x))>x\),所以\(n > N\)时奇子列单调递减,偶子列单调递增,根据单调有界,只能奇子列收敛到\(t_1\),偶子列收敛到\(t_2\),这与\(x_n\rightarrow t_0\)矛盾。

{\color{blue}方法二:}这个方法可以快速说明\(c>\frac{\sqrt{3}}{2}\)时数列一定不收敛,但是剩下一半似乎用不了。
显然\(f(x)=x\)的解是\(t_0 = \frac{\sqrt{1 + 4c^2}-1}{2c}\),如果\(c>\frac{\sqrt{3}}{2}\),求导有\(f^\prime(x)=-2cx,|f^\prime(t_0)|=\sqrt{1 + 4c^2}-1>1\)。
所以在\(t_0\)附近的一个邻域内都有\(|f^\prime(x)|\geq1+\delta>1\),而如果此时\(x_n\)收敛,则必然收敛到\(t_0\),也就是说存在\(x_N\)落入\(t_0\)附近一个去心邻域内(条件\(x_2\neq x_1\)保证了\(x_n\neq t_0\)恒成立),于是
\[|x_{N + 1}-t_0|=|f(x_N)-f(t_0)|=|f^\prime(\xi)||x_N - t_0|\geq(1 + \delta)|x_N - t_0|\]
以此类推下去,显然\(x_n\)与\(t_0\)的距离只会越来越远,因此不可能收敛到\(t_0\)导致矛盾。
\end{proof}
\begin{remark}
{\color{blue}方法一}是标准方法也是通用的,注意多项式时候一定有整除关系\(f(x) - x\mid f(f(x)) - x\)所以必定能因式分解。
{\color{blue}方法二}则是回忆之前讲过的“极限点处导数大于等于1时候就不可能压缩映射”,利用这个原理我们很快能发现\(c\)的分界线,同时也能快速说明\(c>\frac{\sqrt{3}}{2}\)时数列一定不收敛。
\end{remark}







\subsection{利用上下极限求递推数列极限}

\begin{example}
设\(A,B > 0\),\(a_1 > A\)以及
\(a_{n + 1} = A + \frac{B}{a_n}, n \in \mathbb{N}_+\),
计算\(\lim_{n \to \infty} a_n\).
\end{example}
\begin{proof}
显然$a_n>A>0,\forall n\in\mathbb{N}_+$.从而$a_{n+1}=A+\frac{B}{a_n}\leq A+\frac{B}{A},\forall n\in\mathbb{N}_+$.故数列$\{a_n\}$有界.于是可设$a=\underset{n\rightarrow \infty}{{\varlimsup }}a_n<\infty,b=\underset{n\rightarrow \infty}{{\varliminf }}a_n<\infty$.对等式$a_{n + 1} = A + \frac{B}{a_n}$两边同时关于$n\to+\infty$取上下极限得到
\begin{align*}
&a=\underset{n\rightarrow \infty}{{\varlimsup }}a_{n+1}=A+\underset{n\rightarrow \infty}{{\varlimsup }}\frac{B}{a_n}=A+\frac{B}{\underset{n\rightarrow \infty}{{\varliminf }}a_n}=A+\frac{B}{b},
\\
&b=\underset{n\rightarrow \infty}{{\varliminf }}a_{n+1}=A+\underset{n\rightarrow \infty}{{\varliminf }}\frac{B}{a_n}=A+\frac{B}{\underset{n\rightarrow \infty}{{\varlimsup }}a_n}=A+\frac{B}{a}.
\end{align*}
于是我们有$\begin{cases}
ab=Ab+B\\
ab=Aa+B\\
\end{cases}$,解得$a=b0=\frac{A\pm\sqrt{A^2-4B}}{2}$.又由$a_n>A>0$,可知$a=b=\frac{A+\sqrt{A^2-4B}}{2}$.故$\underset{n\rightarrow \infty}{\lim}a_n=\frac{A+\sqrt{A^2-4B}}{2}$.
\end{proof}

\begin{example}
设\(x_0,y_0 > 0,x_{n + 1}=\frac{1}{x_n^2 + x_ny_n + 2y_n^2 + 1},y_{n + 1}=\frac{1}{2x_n^2 + x_ny_n + y_n^2 + 1}\),证明:数列\(x_n,y_n\)都收敛且极限相同。
\end{example}
\begin{remark}
\hypertarget{均值放缩的思路}{\(1+\frac{3}{4}u^2\) 的放缩思路:}我们希望 \(\frac{x}{(1+\frac{3}{4}x^2)^2}<1\),待定 \(m > 0\),利用均值不等式可知
\begin{align*}
\left(1+\frac{3}{4}x^2\right)^2=\left(\frac{3}{4}x^2+\overbrace{\frac{1}{m}+\frac{1}{m}+\cdots +\frac{1}{m}}^{m\text{个}}\right)^2
\geqslant \left((m + 1)\sqrt[m + 1]{\frac{3}{4}x^2\cdot\frac{1}{m^m}}\right)^2
=\left(\frac{3}{4}\right)^{\frac{2}{m + 1}}\cdot\frac{m + 1}{m^{\frac{2m}{m + 1}}}x^{\frac{4}{m + 1}}.
\end{align*}
从而我们希望 \(x^{\frac{4}{m + 1}} = x\),即 \(m = 3\)。这样就能使得
\[
\frac{x}{(1+\frac{3}{4}x^2)^2}\leqslant \left(\frac{3}{4}\right)^{\frac{2}{m + 1}}\cdot\frac{m + 1}{m^{\frac{2m}{m + 1}}}x^{\frac{4}{m + 1}}=\left(\frac{3}{4}\right)^{\frac{2}{3 + 1}}\cdot\frac{3 + 1}{3^{\frac{2\cdot 3}{3 + 1}}}<1.
\]
故取 \(m = 3\)。
\end{remark}
\begin{proof}
根据条件可知\(x_n,y_n > 0\),并且进一步归纳易证\(x_n,y_n\in[0,1]\),所以上下极限也都在\([0,1]\)之间。
\begin{align*}
x_{n + 1}-y_{n + 1}&=\frac{1}{x_n^2 + x_ny_n + 2y_n^2 + 1}-\frac{1}{2x_n^2 + x_ny_n + y_n^2 + 1}\\
&=\frac{x_n^2 - y_n^2}{(x_n^2 + x_ny_n + 2y_n^2 + 1)(2x_n^2 + x_ny_n + y_n^2 + 1)}
\end{align*}
由均值不等式可得
\[
x^2 + xy + y^2=(x + y)^2 - xy\geqslant (x + y)^2 - \left(\frac{x + y}{2}\right)^2=\frac{3}{4}(x + y)^2.
\]
记 \(u = x_n + y_n\geq 0\),则\hyperlink{均值放缩的思路}{由均值不等式可得}
\[
1+\frac{3}{4}u^2=\frac{3}{4}u^2+\frac{1}{3}+\frac{1}{3}+\frac{1}{3}\geq 4\sqrt[4]{\frac{u^2}{36}}=4\sqrt{\frac{|u|}{6}}\Rightarrow \frac{u}{(1+\frac{3}{4}u^2)^2}\le \frac{8}{3}.
\]
于是
\begin{align*}
|x_{n + 1}-y_{n + 1}|&=\frac{|x_n - y_n|(x_n + y_n)}{(x_{n}^{2}+x_ny_n + 2y_{n}^{2}+1)(2x_{n}^{2}+x_ny_n + y_{n}^{2}+1)}\\
&\le |x_n - y_n|\frac{x_n + y_n}{(x_{n}^{2}+x_ny_n + y_{n}^{2}+1)(x_{n}^{2}+x_ny_n + y_{n}^{2}+1)}\\
&\le |x_n - y_n|\frac{x_n + y_n}{(1+\frac{3}{4}(x_n + y_n)^2)^2}=|x_n - y_n|\frac{u}{(1+\frac{3}{4}u^2)^2}
\end{align*}
故
\[
|x_{n + 1}-y_{n + 1}|\le \frac{3}{8}|x_n - y_n|\le \cdots \le (\frac{3}{8})^n|x_1 - y_1|.
\]
上式两边同时令$n\to \infty$,得到$\underset{n\rightarrow \infty}{\lim}\left( x_n-y_n \right) =0$.因此,设
\(\varlimsup_{n\rightarrow\infty}x_n=\varlimsup_{n\rightarrow\infty}y_n = A,\varliminf_{n\rightarrow\infty}x_n=\varliminf_{n\rightarrow\infty}y_n = B,A,B\in[0,1],A\geq B\)
利用上下极限的基本性质有
\begin{align*}
A&=\varlimsup_{n\rightarrow\infty}x_n=\varlimsup_{n\rightarrow\infty}\frac{1}{x_n^2 + x_ny_n + 2y_n^2 + 1}\leq\frac{1}{4B^2 + 1}\\
B&=\varliminf_{n\rightarrow\infty}x_n=\varliminf_{n\rightarrow\infty}\frac{1}{x_n^2 + x_ny_n + 2y_n^2 + 1}\geq\frac{1}{4A^2 + 1}\\
&\Rightarrow A\leq\frac{1}{4B^2 + 1}\leq\frac{1}{\frac{4}{(4A^2 + 1)^2}+1}=\frac{(4A^2 + 1)^2}{(4A^2 + 1)^2 + 4}
\end{align*}
{\color{blue}方法一:}去分母并化简,因式分解得到(这个方法难算,建议用mma,或者慢慢手动拆)
\[A((4A^2 + 1)^2 + 4)-(4A^2 + 1)^2=(2A - 1)^3(2A^2 + A + 1)\leq0\]
于是\(A\leq\frac{1}{2}\),同理可知\(B\geq\frac{1}{2}\),所以\(A = B=\frac{1}{2}\),因此\(x_n,y_n\)都收敛到\(\frac{1}{2}\)。

{\color{blue}方法二:}最后计算\(A,B\)时候如果采用上述方法硬做有点难算,其实有巧妙一些的选择.
因为\(\lim_{n\rightarrow\infty}(x_n - y_n)=0\),所以\(\lim_{n\rightarrow \infty} (4x_{n}^{2}-(x_{n}^{2}+x_ny_n+2y_{n}^{2}))=\lim_{n\rightarrow \infty} x_n\left( x_n-y_n \right) +2\lim_{n\rightarrow \infty} \left( x_n+y_n \right) \left( x_n-y_n \right) =0\)(有界量乘无穷小量).进而上下极限也有等式
\(\varlimsup_{n\rightarrow\infty}(x_n^2 + x_ny_n + 2y_n^2)=\varlimsup_{n\rightarrow\infty}4x_n^2 = 4A^2,\varliminf_{n\rightarrow\infty}(x_n^2 + x_ny_n + 2y_n^2)=\varliminf_{n\rightarrow\infty}4x_n^2 = 4B^2\)
代入可知
\begin{align*}
A&=\varlimsup_{n\rightarrow\infty}x_n=\varlimsup_{n\rightarrow\infty}\frac{1}{x_n^2 + x_ny_n + 2y_n^2 + 1}=\frac{1}{4B^2 + 1}\\
B&=\varliminf_{n\rightarrow\infty}x_n=\varliminf_{n\rightarrow\infty}\frac{1}{x_n^2 + x_ny_n + 2y_n^2 + 1}=\frac{1}{4A^2 + 1}\\
&\Rightarrow 4AB^2+A = 4A^2B + B = 1,(4AB - 1)(B - A)=0
\end{align*}
所以若\(A = B\)则显然成立,进而由递推条件可得$A=B=\frac{1}{2}$.若\(A\neq B\)则\(AB=\frac{1}{4}\),代入有\(A + B = 1\),显然解出\(A = B=\frac{1}{4}\)矛盾。
\end{proof}
\begin{remark}
有必要先来证明\(x_n - y_n\rightarrow0\)而不是上来直接设\(x_n,y_n\)的上下极限一共四个数字,这样的话根本算不出来(用mma都算不出来),而如果证明了\(x_n - y_n\rightarrow0\),则只有两个变量了。
方法二好做是因为都是等式了,所以可以作差然后简单的因式分解解出来,而方法一那样无脑硬算,就要麻烦。
本题运用的若干上下极限性质都可以在任何一本数学分析教材上面找到证明。只要你记住三点:

1. 逐项(包括加法也包括乘法)取上下极限通常都会成立一个确定方向的不等式。

2. 计算上下极限时候,如果其中某一项极限就是存在的,那么上下极限的不等式将会成为等式。

3. 对于都是正数的问题,取倒数的上下极限运算规则就是你脑海中最自然的那种情况。
这样考试时候就算忘了具体的结论,也可以通过画图和举例快速确定下来。
\end{remark}




\subsection{类递增/类递减递推数列}

\begin{example}[类递增模型]\label{example:类递增模型}
\begin{enumerate}
\item 设\(c_1,c_2 > 0\),\(c_{n + 2} = \sqrt{c_{n + 1}} + \sqrt{c_{n}}\),\(n = 1,2,\cdots\),计算\(\lim_{n \to \infty} c_{n}\).

\item 设\(a_k \in (0,1)\),\(1 \leq k \leq 2021\)且
\((a_{n + 2021})^{2022} = a_{n} + a_{n + 1} + \cdots + a_{n + 2020}\),\(n = 1,2,\cdots\),
这里\(a_{n} > 0\),\(\forall n \in \mathbb{N}\)证明\(\lim_{n \to \infty} a_{n}\)存在.
\end{enumerate}
\end{example}
\begin{note}
解决此类问题一般先定界(即确定$c_n$的上下界的具体数值),再对等式两边同时取上下极限即可.
\end{note}
\begin{remark}
\begin{enumerate}
\item \label{example3.23-1}记\(b\triangleq\max\{c_1,c_2,4\}\)的原因:为了证明数列${c_n}$有界,我们需要先定界(即确定$c_n$的上下界的具体数值),然后再利用数学归纳法证得数列${c_n}$有界.显然${c_n}$有一个下界0,但上界无法直接观察出来.为了确定出数列${c_n}$的一个上界,我们可以先假设${c_n}$有一个上界$b$(此时$b$是待定常数).则$c_{n+1}=\sqrt{c_n}+\sqrt{c_{n - 1}}\leqslant\sqrt{b}+\sqrt{b}=2\sqrt{b}\leqslant b$,由此解得$b\geq 4$.又由数学归纳法的原理,可知需要保证$b$同时也是$c_1,c_2$的上界.故只要取$b\geq 4,c_1,c_2$就一定能归纳出$b$是${c_n}$的一个上界.而我们取$b\triangleq\max\{c_1,c_2,4\}$满足这个条件.

\item \label{example3.23-2}记$M=$的原因:同上一问,假设数列${a_n}$有一个上界$M$(此时$M$是待定常数),则
\begin{align*}
a_{n+2021}=\sqrt[2022]{a_n+a_{n+1}+\cdots +a_{n+2020}}\le \sqrt[2022]{M+M+\cdots +M}=\sqrt[2022]{2021M}\le M.
\end{align*}
由此解得$M\geq (2021)^{\frac{1}{2021}}$.又由数学归纳法的原理,可知需要保证$M$同时也是$a_1,a_2,\cdots,a_{2020}$的上界.故只要取$M\geq \left(2021\right)^{\frac{1}{2021}},a_1,a_2,\cdots,a_{2020}$就一定能归纳出$M$是${a_n}$的一个上界.而我们取$M=\max \left\{ \left( 2021 \right) ^{\frac{1}{2021}},a_1,a_2,\cdots ,a_{2020} \right\}$满足这个条件.
\end{enumerate}
\end{remark}
\begin{solution}
\begin{enumerate}
\item \hyperref[example3.23-1]{记\(b\triangleq\max\{c_1,c_2,4\}\),则\(0 < c_1,c_2\le b\)}.假设\(0 < c_n\le b\),则
\[
0 < c_{n + 1}=\sqrt{c_n}+\sqrt{c_{n - 1}}\leqslant\sqrt{b}+\sqrt{b}=2\sqrt{b}\leqslant b.
\]
由数学归纳法,可知对\(\forall n\in\mathbb{N}_+\),都有\(0 < c_n\le b\)成立.即数列\(\{c_n\}\)有界.

因此可设\(L=\varlimsup_{n\rightarrow\infty}c_n <\infty\),\(l=\varliminf_{n\rightarrow\infty}c_n <\infty\).
令\(c_{n + 1}=\sqrt{c_n}+\sqrt{c_{n - 1}}\)两边同时对\(n\rightarrow\infty\)取上下极限,可得
\[
L=\varlimsup_{n\rightarrow\infty}c_{n + 1}=\varlimsup_{n\rightarrow\infty}(\sqrt{c_n}+\sqrt{c_{n - 1}})\leqslant\varlimsup_{n\rightarrow\infty}\sqrt{c_n}+\varlimsup_{n\rightarrow\infty}\sqrt{c_{n - 1}}=2\sqrt{L}\Rightarrow L\leqslant 4,
\]
\[
l=\varliminf_{n\rightarrow\infty}c_{n + 1}=\varliminf_{n\rightarrow\infty}(\sqrt{c_n}+\sqrt{c_{n - 1}})\geqslant\varliminf_{n\rightarrow\infty}\sqrt{c_n}+\varliminf_{n\rightarrow\infty}\sqrt{c_{n - 1}}=2\sqrt{l}\Rightarrow l\geqslant 4.
\]
又\(l=\varliminf_{n\rightarrow\infty}c_n\leqslant\varlimsup_{n\rightarrow\infty}c_n = L\),故\(L = l = 4\).即\(\lim_{n\rightarrow\infty}c_n = 4\).

\item \hyperref[example3.23-2]{取\(M=\max \left\{ \left( 2021 \right) ^{\frac{1}{2021}},a_1,a_2,\cdots ,a_{2020} \right\}\)},显然$a_n>0$且$a_1,a_2,\cdots ,a_{2020}\leq M$.假设$a_k\leq M$,$k=1,2,\cdots,n$则由条件可得
\begin{align*}
a_{n+1}=\sqrt[2022]{a_{n-2020}+a_{n-2019}+\cdots +a_{n}}\le \sqrt[2022]{M+M+\cdots +M}=\sqrt[2022]{2021M}\le M.
\end{align*}
由数学归纳法,可知$0<a_n\leq M,\forall n\in\mathbb{N}_+$.即数列${a_n}$有界.因此可设$A=\underset{n\rightarrow \infty}{{\varlimsup }}a_n<\infty ,a=\underset{n\rightarrow \infty}{{\varliminf }}a_n<\infty$.由条件可得
\begin{align*}
a_{n+2021}=\sqrt[2022]{a_n+a_{n+1}+\cdots +a_{n+2020}}.
\end{align*}
上式两边同时对$n\to \infty$取上下极限得到
\begin{align*}
&A=\underset{n\rightarrow \infty}{{\varlimsup }}a_{n+2021}=\underset{n\rightarrow \infty}{{\varlimsup }}\sqrt[2022]{a_n+a_{n+1}+\cdots +a_{n+2020}}=\sqrt[2022]{\underset{n\rightarrow \infty}{{\varlimsup }}\left( a_n+a_{n+1}+\cdots +a_{n+2020} \right)}
\\
&\leqslant \sqrt[2022]{\underset{n\rightarrow \infty}{{\varlimsup }}a_n+\underset{n\rightarrow \infty}{{\varlimsup }}a_{n+1}+\cdots +\underset{n\rightarrow \infty}{{\varlimsup }}a_{n+2020}}=\sqrt[2022]{A+A+\cdots +A}\Rightarrow A\leqslant \left( 2021 \right) ^{\frac{1}{2021}},   
\end{align*}
\begin{align*}
&a=\underset{n\rightarrow \infty}{{\varliminf }}a_{n+2021}=\underset{n\rightarrow \infty}{{\varliminf }}\sqrt[2022]{a_n+a_{n+1}+\cdots +a_{n+2020}}=\sqrt[2022]{\underset{n\rightarrow \infty}{{\varliminf }}\left( a_n+a_{n+1}+\cdots +a_{n+2020} \right)}
\\
&\geqslant \sqrt[2022]{\underset{n\rightarrow \infty}{{\varliminf }}a_n+\underset{n\rightarrow \infty}{{\varliminf }}a_{n+1}+\cdots +\underset{n\rightarrow \infty}{{\varliminf }}a_{n+2020}}=\sqrt[2022]{a+a+\cdots +a}\Rightarrow a\geqslant \left( 2021 \right) ^{\frac{1}{2021}}.
\end{align*}
又$a=\underset{n\rightarrow \infty}{{\varliminf }}a_n\leqslant \underset{n\rightarrow \infty}{{\varlimsup }}a_n=A$,故$A=a=\left( 2021 \right) ^{\frac{1}{2021}}$.即$\underset{n\rightarrow \infty}{\lim}a_n=\left( 2021 \right) ^{\frac{1}{2021}}$.
\end{enumerate}
\end{solution}

\begin{example}[类递减模型]\label{example:类递减模型}
\begin{enumerate}
\item 设\(a_{n + 2} = \frac{1}{a_{n + 1}} + \frac{1}{a_{n}}, a_{1}, a_{2} > 0, n = 1,2,\cdots\).证明\(\lim_{n \to \infty} a_{n}\)存在.

\item 设\(x_{1} = a>0, x_{2} = b>0, x_{n + 2} = 3 + \frac{1}{x_{n + 1}^{2}} + \frac{1}{x_{n}^{2}}, n = 1,2,\cdots\).
证明\(\lim_{n \to \infty} x_{n}\)存在.
\end{enumerate}   
\end{example}
\begin{note}
此类问题一定要记住,先定界.
这里我们提供两种方法:

第一题我们使用上下极限,再隔项抽子列的方法.(这里就算我们解不出不动点也能用这个方法证明极限存在.)

第二题我们使用构造二阶差分的线性递推不等式的方法.
(这里也可以设出不动点$x_0$,由条件可知,$x_0=3+\frac{1}{x_0^2}+\frac{1}{x_0^2}$,解出不动点.然后两边减去不动点,类似的去构造一个二阶线性递推数列,然后待定系数放缩一下说明收敛.)

这类题如果不记住做题时会难以想到.与\hyperref[example:类递增模型]{类递增模型}一样,一开始要定界.
\end{note}
\begin{remark}
第二题的极限是一个无理数,特征方程比较难解,因此我们只证明极限的存在性.
\end{remark}
\begin{proof}
\begin{enumerate}
\item \hyperref[example3.24(1)]{取\(a=\min \left\{ a_1,a_2,\frac{2}{a_1},\frac{2}{a_2} \right\}>0\)},则有$0<a\leq a_1,a_2\leq \frac{2}{a}$成立.假设$0<a\leq a_n\leq \frac{2}{a}$,则由条件可得
\begin{align*}
a_{n+2}=\frac{1}{a_{n+1}}+\frac{1}{a_n}\leqslant \frac{1}{a}+\frac{1}{a}=\frac{2}{a}.
\end{align*}
由数学归纳法,可知$0<a\leq a_n\leq \frac{2}{a},\forall n\in\mathbb{N}_+$.即数列${a_n}$有界.于是可设$A=\underset{n\rightarrow \infty}{{\varlimsup }}a_n<\infty ,B=\underset{n\rightarrow \infty}{{\varliminf }}a_n<\infty$.\hyperref[example3.24(2)]{由致密性定理,可知存在一个子列$\{a_{n_k}\}$,使得$\lim_{k\rightarrow \infty} a_{n_k+2}=A,\lim_{k\rightarrow \infty} a_{n_k+1}=l_1<\infty,\lim_{k\rightarrow \infty} a_{n_k}=l_2<\infty,\lim_{k\rightarrow \infty} a_{n_k-1}=l_3<\infty$}.并且根据上下极限的定义,可知$B\leq l_1,l_2,l_3\leq A$.对等式$a_{n + 2} = \frac{1}{a_{n + 1}} + \frac{1}{a_{n}}$两边同时关于$n\to+\infty$取上下极限得到
\begin{align*}
&A=\underset{n\rightarrow \infty}{{\varlimsup }}a_{n+2}=\underset{n\rightarrow \infty}{{\varlimsup }}\left( \frac{1}{a_{n+1}}+\frac{1}{a_n} \right) \leqslant \underset{n\rightarrow \infty}{{\varlimsup }}\frac{1}{a_{n+1}}+\underset{n\rightarrow \infty}{{\varlimsup }}\frac{1}{a_n}
\\
&=\frac{1}{\underset{n\rightarrow \infty}{{\varliminf }}a_{n+1}}+\frac{1}{\underset{n\rightarrow \infty}{{\varliminf }}a_n}=\frac{1}{B}+\frac{1}{B}=\frac{2}{B}\Rightarrow AB\leqslant 2.
\end{align*}
\begin{align*}
&B=\underset{n\rightarrow \infty}{{\varliminf }}a_{n+2}=\underset{n\rightarrow \infty}{{\varliminf }}\left( \frac{1}{a_{n+1}}+\frac{1}{a_n} \right) \geqslant \underset{n\rightarrow \infty}{{\varliminf }}\frac{1}{a_{n+1}}+\underset{n\rightarrow \infty}{{\varliminf }}\frac{1}{a_n}
\\
&=\frac{1}{\underset{n\rightarrow \infty}{{\varlimsup }}a_{n+1}}+\frac{1}{\underset{n\rightarrow \infty}{{\varlimsup }}a_n}=\frac{1}{A}+\frac{1}{A}=\frac{2}{A}\Rightarrow AB\geqslant 2.
\end{align*}
故$AB=2$.因为$\{a_{n_k}\}$是数列${a_n}$的一个子列,所以$\{a_{n_k}\}$也满足$a_{n_k+2}=\frac{1}{a_{n_k+1}}+\frac{1}{a_{n_k}},\forall k\in\mathbb{N}_+$.并且子列$\{a_{n_k-1}\},\{a_{n_k}\},\{a_{n_k+1}\},\{a_{n_k+2}\}$的极限都存在,于是对$a_{n_k+2}=\frac{1}{a_{n_k+1}}+\frac{1}{a_{n_k}}$等式两边同时关于$k\to+\infty$取极限,再结合$B\leq l_1,l_2,l_3\leq A$得到
\begin{align*}
&A=\lim_{k\rightarrow \infty} a_{n_k+2}=\lim_{k\rightarrow \infty} \frac{1}{a_{n_k+1}}+\lim_{k\rightarrow \infty} \frac{1}{a_{n_k}}
\\
&=\frac{1}{l_1}+\frac{1}{l_2}\leqslant \frac{1}{B}+\frac{1}{B}=\frac{2}{B}=A\Rightarrow l_1=l_2=B.
\end{align*}
同理再对$a_{n_k+1}=\frac{1}{a_{n_k}}+\frac{1}{a_{n_k-1}}$等式两边同时关于$k\to+\infty$取极限,再结合$B\leq l_1,l_2,l_3\leq A$得到
\begin{align*}
&B=l_1=\lim_{k\rightarrow \infty} a_{n_k+1}=\lim_{k\rightarrow \infty} \frac{1}{a_{n_k}}+\lim_{k\rightarrow \infty} \frac{1}{a_{n_k-1}}
\\
&=\frac{1}{l_2}+\frac{1}{l_3}\geqslant \frac{1}{A}+\frac{1}{A}=\frac{2}{A}=B\Rightarrow l_2=l_3=A.
\end{align*}
故$A=B=l_1=l_2=l_3$,又由于$AB=2$,因此$\underset{n\rightarrow \infty}{{\varlimsup }}a_n=\underset{n\rightarrow \infty}{{\varliminf }}a_n=A=B=\sqrt{2}$.即$\underset{n\rightarrow \infty}{\lim}a_n=\sqrt{2}$.

\item 根据递推条件显然,\(x_n\geqslant 3,\forall n\geqslant 3\)。从而 \(x_5 = 3+\frac{1}{x_{4}^{2}}+\frac{1}{x_{3}^{2}}\leqslant 3+\frac{1}{9}+\frac{1}{9}<4\)。假设 \(x_n\leqslant 4,\forall n\geqslant 5\),则
\[
x_{n + 1}=3+\frac{1}{x_{n}^{2}}+\frac{1}{x_{n - 1}^{2}}\leqslant 3+\frac{1}{9}+\frac{1}{9}<4.
\]
由数学归纳法可知 \(x_n\in [3,4],\forall n\geqslant 5\)。于是
\begin{align*}
\left| x_{n+2}-x_{n+1} \right|&=\left| \frac{1}{x_{n+1}^{2}}-\frac{1}{x_{n-1}^{2}} \right|\leqslant \left| \frac{1}{x_{n+1}^{2}}-\frac{1}{x_{n}^{2}} \right|+\left| \frac{1}{x_{n}^{2}}-\frac{1}{x_{n-1}^{2}} \right|=\frac{\left| x_{n}^{2}-x_{n+1}^{2} \right|}{x_{n+1}^{2}x_{n}^{2}}+\frac{\left| x_{n-1}^{2}-x_{n}^{2} \right|}{x_{n}^{2}x_{n-1}^{2}}
\\
&=\frac{x_n+x_{n+1}}{x_{n+1}^{2}x_{n}^{2}}\left| x_{n+1}-x_n \right|+\frac{x_n+x_{n-1}}{x_{n}^{2}x_{n-1}^{2}}\left| x_n-x_{n-1} \right|
\\
&=\frac{1}{x_{n+1}x_n}\left( \frac{1}{x_{n+1}}+\frac{1}{x_n} \right) \left| x_{n+1}-x_n \right|+\frac{1}{x_nx_{n-1}}\left( \frac{1}{x_n}+\frac{1}{x_{n-1}} \right) \left| x_n-x_{n-1} \right|
\\
&\leqslant \frac{2}{27}\left| x_{n+1}-x_n \right|+\frac{2}{27}\left| x_n-x_{n-1} \right|,\forall n\geqslant 6.      
\end{align*}
\hyperlink{取q,lambda的原因}{记 \(q = \frac{1}{2}\in(0,1)\),\(\lambda=\frac{1}{3}\)},\(u_n = |x_n - x_{n - 1}|\),则由上式可得
\begin{align*}
&u_{n + 2}\leqslant \frac{2}{27}u_{n + 1}+\frac{2}{27}u_n
\leqslant (q - \lambda)u_{n + 1}+q\lambda u_n,\forall n\geqslant 6.\\
&\Leftrightarrow u_{n + 2}+\lambda u_{n + 1}\leqslant q(u_{n + 1}+\lambda u_n),\forall n\geqslant 6.
\end{align*}
从而对 \(\forall n\geqslant 10\)(\(n\) 大于 \(7\) 就行),我们有
\[
u_n\leqslant u_n+\lambda u_{n - 1}\leqslant q(u_{n - 1}+\lambda u_{n - 2})\leqslant\cdots\leqslant q^{n - 7}(u_7+\lambda u_6).
\]
于是对 \(\forall n\geqslant 10\),我们有
\[
x_n\leqslant \sum_{k = 10}^n|x_{k + 1}-x_k|+x_6=\sum_{k = 10}^n u_k+x_6\leqslant (u_7+\lambda u_6)\sum_{k = 10}^n q^{k - 7}+x_6.
\]
令 \(n\rightarrow\infty\),则由上式右边收敛可知,\(x_n\) 也收敛。
\end{enumerate}    
\end{proof}
\begin{remark}
\begin{enumerate}
\item 
\begin{enumerate}[(1)]
\item  \label{example3.24(1)}
取$a=\min \left\{ a_1,a_2,\frac{2}{a_1},\frac{2}{a_2} \right\}$的原因:为了证明数列${a_n}$有界,我们需要先定界,然后再利用数学归纳法证得数列${a_n}$有界.显然${a_n}$有一个下界0,但上界无法直接观察出来.为了确定出数列${a_n}$的上下界,我们可以先假设$b$为数列${a_n}$的一个上界(此时$b$是待定常数),但是我们根据$a_n>0$和$a_{n+2}=\frac{1}{a_{n+1}}+\frac{1}{a_n}$只能得到$a_{n+2}=\frac{1}{a_{n+1}}+\frac{1}{a_n}<+\infty$,无法归纳法出$a_n\leq b$,故我们无法归纳出$0<a_n<b,\forall n\in\mathbb{N}_+$.因此仅待定一个上界并不够,下界并不能简单的取为0,我们还需要找到一个更接近下确界的大于零的下界,不妨先假设这个下界为$a>0$(此时$a$也是待定常数).利用这个下界和递推式$a_{n+2}=\frac{1}{a_{n+1}}+\frac{1}{a_n}$归纳出$0<a\leq a_n\leq b,\forall n\in\mathbb{N}_+$(此时$a,b$都是待定常数).于是由已知条件可得
\begin{gather*}
a_{n+2}=\frac{1}{a_{n+1}}+\frac{1}{a_n}\leqslant \frac{1}{a}+\frac{1}{a}=\frac{2}{a}\leqslant b\Rightarrow ab\geqslant 2,
\\
a_{n+2}=\frac{1}{a_{n+1}}+\frac{1}{a_n}\geqslant \frac{1}{b}+\frac{1}{b}=\frac{2}{b}\geqslant a\Rightarrow ab\leqslant 2.
\end{gather*}
从而$ab=2$,即$b=\frac{2}{a}$.进而$0<a\leq a_n\leq\frac{2}{a}$.又由数学归纳法的原理,可知我们需要同时保证$0<a\leq a_1,a_2\leq \frac{2}{a}$.因此找到一个合适的$a$,使得$0<a\leq a_1,a_2\leq \frac{2}{a}$成立就一定能归纳出$0<a\leq a_n\leq \frac{2}{a},\forall n\in\mathbb{N}_+$,即数列$\{a_n\}$有界.而当我们取$a=\min \left\{ a_1,a_2,\frac{2}{a_1},\frac{2}{a_2} \right\}$时,有$a_1,a_2\leqslant a,\,\,\frac{2}{a}\ge \frac{2}{\frac{2}{a_1}}=a_1,\,\,\frac{2}{a}\ge \frac{2}{\frac{2}{a_2}}=a_2.$恰好满足这个条件.

\item \label{example3.24(2)}能取到一个子列$a_{n_k}$,使得$\lim_{k\rightarrow \infty} a_{n_k+2}=A,\lim_{k\rightarrow \infty} a_{n_k+1}=l_1<\infty,\lim_{k\rightarrow \infty} a_{n_k}=l_2<\infty,\lim_{k\rightarrow \infty} a_{n_k-1}=l_3<\infty$成立的原因:
由$A=\underset{n\rightarrow \infty}{{\varlimsup}}a_n$和上极限的定义(上极限就是最大的子列极限),可知存在一个子列$\{a_{n_k}\}$,使得$\lim_{k\rightarrow \infty} a_{n_k+2}=A$.因为数列$\{a_{n_k+1}\}$有界(因为数列$\{a_n\}$有界),所以由致密性定理可知$\{a_{n_k+1}\}$一定存在一个收敛的子列$\{a_{n_{k_j}+1}\}$,并记$\lim_{j\rightarrow \infty} a_{n_{k_j}+1}=l_1<\infty$.又因为$\{a_{n_{k_j}+2}\}$是$\{a_{n_{k}+2}\}$的子列,所以$\lim_{k\rightarrow \infty} a_{n_{k_j}+2}=A$.由于$\{a_{n_{k_j}}\}$仍是$\{a_n\}$的一个子列,因此不妨将$\{a_{n_{k_j}}\}$记作$\{a_{n_k}\}$,则此时有$\lim_{k\rightarrow \infty} a_{n_k+2}=A,\lim_{k\rightarrow \infty} a_{n_k+1}=l_1<\infty$.同理由于数列$\{a_{n_k}\}$有界,所以由致密性定理可知$\{a_{n_k}\}$存在一个收敛的子列$\{a_{n_{k_l}}\}$,并记$\lim_{l\rightarrow \infty} a_{n_{k_l}}=l_2$.又因为$\{a_{n_{k_l}+2}\}$是$\{a_{n_{k}+2}\}$的子列,$\{a_{n_{k_l}+1}\}$是$\{a_{n_{k}+1}\}$的子列,所以$\lim_{l\rightarrow \infty} a_{n_{k_l}+2}=A,\lim_{l\rightarrow \infty} a_{n_{k_l}+1}=l_1$.由于$\{a_{n_{k_l}}\}$仍是$\{a_n\}$的一个子列,因此不妨将$\{a_{n_{k_l}}\}$记作$\{a_{n_k}\}$,则此时有$\lim_{k\rightarrow \infty} a_{n_k+2}=A,\lim_{k\rightarrow \infty} a_{n_k+1}=l_1<\infty,\lim_{k\rightarrow \infty} a_{n_k}=l_2<\infty$.再同理由于数列$\{a_{n_k}\}$有界,所以由致密性定理可知$\{a_{n_k}\}$存在一个收敛的子列$\{a_{n_{k_s}}\}$,并记$\lim_{s\rightarrow \infty} a_{n_{k_s}}=l_3$.又因为$\{a_{n_{k_s}+2}\}$是$\{a_{n_{k}+2}\}$的子列,$\{a_{n_{k_s}+1}\}$是$\{a_{n_{k}+1}\}$的子列,$\{a_{n_{k_s}}\}$是$\{a_{n_{k}}\}$的子列,所以$\lim_{s\rightarrow \infty} a_{n_{k_s}+2}=A,\lim_{s\rightarrow \infty} a_{n_{k_s}+1}=l_1,\lim_{s\rightarrow \infty} a_{n_{k_s}}=l_2$.由于$\{a_{n_{k_s}}\}$仍是$\{a_n\}$的一个子列,因此不妨将$\{a_{n_{k_s}}\}$记作$\{a_{n_k}\}$,则此时有$\lim_{k\rightarrow \infty} a_{n_k+2}=A,\lim_{k\rightarrow \infty} a_{n_k+1}=l_1<\infty,\lim_{k\rightarrow \infty} a_{n_k}=l_2<\infty,\lim_{k\rightarrow \infty} a_{n_k-1}=l_3<\infty$.
\end{enumerate}
\item \hypertarget{取q,lambda的原因}{记 \(q = \frac{1}{2}\in(0,1)\),\(\lambda=\frac{1}{3}\)的原因:}记\(u_n = |x_n - x_{n - 1}|\),则\(u_{n + 2}\leq\frac{2}{27}(u_{n + 1}+u_n)\),类比二阶线性递推数列方法,希望找到\(\lambda>0,q\in(0,1)\)使得\(u_{n + 2}+\lambda u_{n + 1}\leq q(u_{n + 1}+\lambda u_n)\)恒成立,这样一直递推下去就有\(u_{n + 2}+\lambda u_{n + 1}\leq Cq^n,C > 0\),说明\(|x_{n + 1}-x_n|\)是以等比数列速度趋于零的,根据级数收敛的比较判别法显然\(x_n\)收敛,结论成立。

而对比已知不等式 \(u_{n + 2}\leq \frac{2}{27}(u_{n + 1}+u_n)\) 和目标不等式 \(u_{n + 2}\leq (q - \lambda)u_{n + 1}+q\lambda u_n\) 可知,只要满足 \(u_{n + 2}\leq \frac{2}{27}(u_{n + 1}+u_n)\leq (q - \lambda)u_{n + 1}+q\lambda u_n,q\in(0,1),\lambda>0\) 即可达到目的。即只需取合适的 \(q,\lambda\) 使其满足 \(q - \lambda\geq \frac{2}{27},q\lambda\geq \frac{2}{27},q\in(0,1),\lambda>0\) 即可.这明显有很多可以的取法,例如\(q=\frac{1}{2},\lambda=\frac{1}{3}\),因此得证。
\end{enumerate}
\end{remark}

\begin{example}
设\(a_1,\cdots,a_k,b_1,\cdots,b_k>0,k\geq2,a_n = \sum_{i = 1}^{k}\frac{b_i}{a_{n - i}},n\geq k + 1\),证明:\(\lim_{n\rightarrow\infty}a_n=\sqrt{\sum_{i = 1}^{k}b_i}\)。
\end{example}
\begin{note}
本题是\hyperref[example:类递减模型]{例题\ref{example:类递减模型}第一题}的推广.核心想法就是\textbf{反复抽收敛子列}.
\end{note}
\begin{proof}
先证明数列是有界的,为此取充分大的正数\(M\)使得
\[a_n\in\left[\frac{b_1 + b_2+\cdots + b_k}{M},M\right],n = 1,2,\cdots,k\]
然后归纳证明对任意\(n\in\mathbb{N}^+\)都有上述不等式成立,若\(n\)时成立,则\(n + 1\)时
\begin{align*}
a_{n + 1}&=\frac{b_1}{a_n}+\frac{b_2}{a_{n - 1}}+\cdots+\frac{b_k}{a_{n - k+1}}\geq\frac{b_1 + b_2+\cdots + b_k}{M}\\
a_{n + 1}&=\frac{b_1}{a_n}+\frac{b_2}{a_{n - 1}}+\cdots+\frac{b_k}{a_{n - k+1}}\leq\frac{b_1}{\frac{b_1 + \cdots + b_k}{M}}+\frac{b_2}{\frac{b_1 + \cdots + b_k}{M}}+\cdots+\frac{b_k}{\frac{b_1 + \cdots + b_k}{M}}=M
\end{align*}
因此\(a_n\)是有界数列,设其上极限为\(L\),下极限为\(l\),则\(L\geq l\)。
在递推式两边取上下极限可知
\begin{align*}
L&=\varlimsup_{n\rightarrow\infty}a_n=\varlimsup_{n\rightarrow\infty}\left(\frac{b_1}{a_{n - 1}}+\frac{b_2}{a_{n - 2}}+\cdots+\frac{b_k}{a_{n - k}}\right)\leq\varlimsup_{n\rightarrow\infty}\frac{b_1}{a_{n - 1}}+\varlimsup_{n\rightarrow\infty}\frac{b_2}{a_{n - 2}}+\cdots+\varlimsup_{n\rightarrow\infty}\frac{b_k}{a_{n - k}}=\frac{b_1 + b_2+\cdots + b_k}{l}\\
l&=\lim_{n\rightarrow\infty}a_n=\lim_{n\rightarrow\infty}\left(\frac{b_1}{a_{n - 1}}+\frac{b_2}{a_{n - 2}}+\cdots+\frac{b_k}{a_{n - k}}\right)\geq\lim_{n\rightarrow\infty}\frac{b_1}{a_{n - 1}}+\lim_{n\rightarrow\infty}\frac{b_2}{a_{n - 2}}+\cdots+\lim_{n\rightarrow\infty}\frac{b_k}{a_{n - k}}=\frac{b_1 + b_2+\cdots + b_k}{L}
\end{align*}
所以\(Ll=b_1 + b_2+\cdots + b_k\),只要证明\(L = l\)便可得到需要的结论。

根据上极限定义,可以取子列\(a_{n_i}\to L\),不妨要求\(n_{i + 1}-n_i>2k + 2\),然后关注各个\(a_{n_i}\)的上一项\(a_{n_i - 1}\)构成的数列,这也是一个有界数列,所以一定存在收敛子列,我们可以将其记为\(a_{n_{i_j}-1},j = 1,2,\cdots\),那么对于这个子列的每一项,它后面的那一项\(a_{n_{i_j}}\)构成的数列,是之前取的数列\(a_{n_i}\to L\)的子列,自然成立\(\lim_{j\rightarrow\infty}a_{n_{i_j}-1}=l_1\in[l,L],\lim_{j\rightarrow\infty}a_{n_{i_j}} = L\),为了方便起见,我们将这两个数列分别记为\(a_{n_{i}-1},a_{n_{i}}\).($n_{i_j}$的指标集是可列集,按对角线或正方形法则排序)

进一步考虑每个\(a_{n_{i}-1}\)的上一项构成的数列,作为有界数列一定存在收敛子列,然后取出这个收敛子列,则对于这个子列,它后面一项构成的数列趋于\(l_1\),它后面第二项构成的数列趋于\(L\)。

以此类推反复操作有限次(可以保证每次取的子列$n_{i+1}-n_i\geq 2$,从而反复取$k+1$次后就有$n_{i+1}-n_i\geq 2(k+1)$,但本题用不上这个条件),最终我们可以得到一列正整数\(n_i\)单调递增趋于无穷,满足
\[a_{n_i}\to L,a_{n_i - 1}\to l_1,a_{n_i - 2}\to l_2,\cdots,a_{n_i - k}\to l_k,a_{n_i - k - 1}\to l_{k + 1},n_{i + 1}-n_i\geq2k + 2,l_1,\cdots,l_{k + 1}\in[l,L]\]
代入到条件递推式中,取极限有
\begin{align*}
L&=\lim_{i\rightarrow\infty}a_{n_i}=\lim_{i\rightarrow\infty}\left(\frac{b_1}{a_{n_i - 1}}+\frac{b_2}{a_{n_i - 2}}+\cdots+\frac{b_k}{a_{n_i - k}}\right)=\frac{b_1}{l_1}+\frac{b_2}{l_2}+\cdots+\frac{b_k}{l_k}\leq\frac{b_1 + b_2+\cdots + b_k}{l}=L\\
\Rightarrow l_1&=l_2=\cdots=l_k = l\\
l_1&=\lim_{i\rightarrow\infty}a_{n_i - 1}=\lim_{i\rightarrow\infty}\left(\frac{b_1}{a_{n_i - 2}}+\frac{b_2}{a_{n_i - 3}}+\cdots+\frac{b_k}{a_{n_i - k - 1}}\right)=\frac{b_1}{l_2}+\frac{b_2}{l_3}+\cdots+\frac{b_k}{l_{k + 1}}\geq\frac{b_1 + b_2+\cdots + b_k}{L}=l_1\\
\Rightarrow l_2&=l_3=\cdots=l_{k + 1}=L
\end{align*}
于是\(L = l_1 = l_2=l\)(这是公共的一个值,注意\(k\geq2\)),结论得证.再对递推条件两边取极限得到极限值.
\end{proof}



\subsection{压缩映像}
我们来看一种重要的处理模型,压缩映像方法,它是我们以后解决基础题的重要方法.其思想内核有两种,一种是找到不动点\(x_0\),然后得到某个\(L\in(0,1)\),使得
\[
|x_n - x_0|\leq L|x_{n - 1} - x_0|\leq\cdots\leq L^{n - 1}|x_1 - x_0|.
\]
还有一种是得到某个\(L\in(0,1)\),使得
\[
|x_n - x_{n - 1}|\leq L|x_{n - 1} - x_{n - 2}|\leq\cdots\leq L^{n - 2}|x_2 - x_1|.
\]
当数列由递推确定时,我们有
\[
|x_n - x_0| = |f(x_{n - 1}) - f(x_0)|,|x_n - x_{n - 1}| = |f(x_{n - 1}) - f(x_{n - 2})|,
\]
因此往往可适用中值定理或者直接放缩法来得到渴望的\(L\in(0,1)\),特别强调\(L = 1\)是不对的.

\begin{note}
常规的递减递推数列求极限问题我们一般使用压缩映像证明.压缩映像的书写过程往往比用递推函数的二次复合和数学归纳法的书写要简便的多.
\end{note}

\begin{remark}
\textbf{当递推函数的不动点/极限点处导数大于等于1的时候,就不可能压缩映射.}
\end{remark}



\begin{example}
\begin{enumerate}
\item 设\(x_1 > - 1,x_{n + 1} = \frac{1}{1 + x_n},n = 1,2,\cdots\),求极限\(\lim_{n \to \infty} x_n\).
\item 求数列\(\sqrt{7},\sqrt{7 - \sqrt{7}},\sqrt{7 - \sqrt{7 + \sqrt{7}}},\cdots\)极限.
\end{enumerate}
\end{example}
\begin{solution}
\begin{enumerate}
\item 
{\color{blue}解法一(\hyperref[proposition:递减函数递推数列]{递减递推归纳法}):}
不妨设$x_1>0$(因为$x_2=\frac{1}{1+x_1}>0$),归纳可知$x_n>0$.由于原递推函数是递减函数,因此考虑递推函数的二次复合$x_{n+2}=\frac{1}{1+\frac{1}{1+x_n}}=\frac{1+x_n}{2+_n}$,这个递推函数一定是单调递增的.进而考虑
\begin{align*}
\frac{1+x}{2+x}-x=\frac{\left( x+\frac{\sqrt{5}+1}{2} \right) \left( \frac{\sqrt{5}-1}{2}-x \right)}{2+x}.
\end{align*}
于是当$x_1\geq \frac{\sqrt{5}-1}{2}$时,有$x_3-x_1=\frac{1+x_1}{2+x_1}-x_1\leqslant 0$,即$x_3\leqslant x_1$.从而由\hyperref[proposition:递增函数递推数列]{递增递推结论}可知,$\{x_{2n-1}\}$单调递减且$x_{2n-1}>\frac{\sqrt{5}-1}{2},\forall n \in \mathbb{N}_+$.此时\(x_2<\frac{\sqrt{5}-1}{2}\)(由\(x = \frac{1}{1 + x}\)以及$x_n>0$可以解得不动点\(x_0=\frac{\sqrt{5}-1}{2}\),又因为原数列是递减递推,所以\(x_n\)与\(x_0\)大小关系交错.而\(x_1\geqslant\frac{\sqrt{5}-1}{2}\),故\(x_2<\frac{\sqrt{5}-1}{2}\)).
于是\(x_4 - x_2=\frac{1 + x_2}{2 + x_2}-x_2>0\),即\(x_4>x_2\).从而由\hyperref[proposition:递增函数递推数列]{递增递推结论}可知,\(\{x_{2n}\}\)单调递增且\(x_{2n}>\frac{\sqrt{5}-1}{2}\),\(\forall n\in\mathbb{N}_+\).

因此由单调有界定理可知,$\{x_{2n}\},\{x_{2n-1}\}$收敛.设$\underset{n\rightarrow \infty}{\lim}x_{2n}=a>0,\underset{n\rightarrow \infty}{\lim}x_{2n-1}=b>0$.又由$x_{2n}=\frac{1}{1+x_{2n}},x_{2n-1}=\frac{1}{1+x_{2n-1}},\forall n\in \mathbb{N}_+$,再令$n\to \infty$,可得$a=\frac{1}{1+a},b=\frac{1}{1+b}$,进而解得$a=b=\frac{\sqrt{5}-1}{2}$.故$\underset{n\rightarrow \infty}{\lim}x_n=\underset{n\rightarrow \infty}{\lim}x_{2n}=\underset{n\rightarrow \infty}{\lim}x_{2n-1}=\frac{\sqrt{5}-1}{2}$.
同理,当$x_1<\frac{\sqrt{5}-1}{2}$时,也有$\underset{n\rightarrow \infty}{\lim}x_n=\frac{\sqrt{5}-1}{2}$.

{\color{blue}解法二(压缩映像):}不妨设$x_1>0$(用$x_2=\frac{1}{1+x_1}>0$代替$x_1$),归纳可知$x_n>0$.设$x=\frac{\sqrt{5}-1}{2}$,则
\begin{align*}
\left| x_{n+1}-x \right|=\left| \frac{1}{1+x_n}-x \right|=\left| \frac{1}{1+x_n}-\frac{1}{1+x} \right|=\frac{\left| x_n-x \right|}{\left( 1+x_n \right) \left( 1+x \right)}\leqslant \frac{1}{1+x}\left| x_n-x \right|.
\end{align*}
从而
\begin{align*}
\left| x_{n+1}-x \right|\leqslant \frac{1}{1+x}\left| x_n-x \right|\leqslant \frac{1}{\left( 1+x \right) ^2}\left| x_{n-1}-x \right|\leqslant \cdots \leqslant \frac{1}{\left( 1+x \right) ^n}\left| x_1-x \right|.
\end{align*}
于是令$n\to\infty$,得到$\underset{n\rightarrow \infty}{\lim}\left| x_{n+1}-x \right|=0$,因此$\underset{n\rightarrow \infty}{\lim}x_n=x=\frac{\sqrt{5}-1}{2}$.

\item 由条件可知,\(x_{n + 2}=\sqrt{7-\sqrt{7 + x_n}}\),\(\forall n\in\mathbb{N}_+\)(由此可解得\(x = 2\)为不动点).于是
\begin{align*}
\vert x_{n + 2}-2\vert&=\vert\sqrt{7-\sqrt{7 + x_n}}-2\vert
=\frac{\vert 3-\sqrt{7 + x_n}\vert}{\sqrt{7-\sqrt{7 + x_n}}+2}\\
&=\frac{\vert 2 - x_n\vert}{(\sqrt{7-\sqrt{7 + x_n}}+2)(3+\sqrt{7 + x_n})}
\leqslant\frac{1}{6}\vert x_n - 2\vert.
\end{align*}
从而对$\forall n \in \mathbb{N}_+$,都有
\begin{align*}
\vert x_{2n}-2\vert&\leqslant\frac{1}{6}\vert x_{2n - 2}-2\vert\leqslant\frac{1}{6^2}\vert x_{2n - 4}-2\vert\leqslant\cdots\leqslant\frac{1}{6^{n - 1}}\vert x_2 - 2\vert;\\
\vert x_{2n + 1}-2\vert&\leqslant\frac{1}{6}\vert x_{2n - 1}-2\vert\leqslant\frac{1}{6^2}\vert x_{2n - 3}-2\vert\leqslant\cdots\leqslant\frac{1}{6^n}\vert x_1 - 2\vert.
\end{align*}
上式两边同时令\(n\rightarrow\infty\),得到\(\lim_{n\rightarrow\infty}\vert x_{2n}-2\vert=\lim_{n\rightarrow\infty}\vert x_{2n + 1}-2\vert = 0\).因此\(\lim_{n\rightarrow\infty}x_n=\lim_{n\rightarrow\infty}x_{2n}=\lim_{n\rightarrow\infty}x_{2n + 1}=2\).
\end{enumerate}
\end{solution}

\begin{example}
设数列\(x_1\in\mathbb{R},x_{n + 1} = \cos x_n,n\in\mathbb{N}\),求\(\lim_{n \to \infty} x_n\).
\end{example}
\begin{solution}
令\(g(x)=x - \cos x\),则\(g^\prime(x)=1 + \sin x\geqslant0\),且\(g^\prime(x)\)不恒等于\(0\).
又\(g(0)= -1 < 0\),\(g(1)=1 - \cos 1 > 0\),因此由零点存在定理可知,\(g\)存在唯一零点\(x_0\in(0,1)\).
不妨设\(x_1\in[-1,1]\)(用\(x_2\)代替\(x_1\)),则\(x_n\in[-1,1]\).再令\(f(x)=\cos x\),则\(f^\prime(x)=-\sin x\).于是记\(C\triangleq \max_{x\in[-1,1]}\vert f^\prime(x)\vert\in(0,1)\).

故由Lagrange中值定理,可得存在\(\theta_n\in(\min\{x_n,x_0\},\max\{x_n,x_0\})\),使得对\(\forall n\in\mathbb{N}_+\),都有
\[
\vert x_{n + 1}-x_0\vert=\vert f(x_n)-f(x_0)\vert=\vert f^\prime(\theta_n)\vert\vert x_n - x_0\vert\leqslant C\vert x_n - x_0\vert.
\]
进而对\(\forall n\in\mathbb{N}_+\),都有
\[
\vert x_{n + 1}-x_0\vert\leqslant C\vert x_n - x_0\vert\leqslant C^2\vert x_{n - 1}-x_0\vert\leqslant\cdots\leqslant C^n\vert x_1 - x_0\vert.
\]
上式两边同时令\(n\rightarrow\infty\),再结合\(C\in(0,1)\),可得\(\lim_{n\rightarrow\infty}\vert x_{n + 1}-x_0\vert = 0\).即\(\lim_{n\rightarrow\infty}x_n = x_0\).
\end{solution}

\begin{proposition}[加强的压缩映像]\label{proposition:加强的压缩映像}
设可微函数\(f:[a,b]\to[a,b]\)满足\(\vert f'(x)\vert<1,\forall x\in[a,b]\).证明:对
\[
x_1\in[a,b],x_{n + 1} = f(x_n),n\in\mathbb{N},
\]
必有\(\lim_{n \to \infty} x_n\)存在.  
\end{proposition}
\begin{remark}
注意到\(f'\)未必是连续函数,所以\(\sup_{x\in[a,b]}\vert f'(x)\vert\)未必可以严格小于\(1\).
\end{remark}
\begin{note}
实际上,用压缩映像证明$\{x_n\}$的极限是$x_0$,也同时蕴含了$x_0$就是这个递推数列的唯一不动点(反证易得).
\end{note}
\begin{proof}
令\(g(x)=x - f(x)\),则\(g(a)=a - f(a)\leqslant0\),\(g(b)=b - f(b)\geqslant0\).由零点存在定理可知,存在\(x_0\in[a,b]\),使得\(x_0 = f(x_0)\).
令\(h(x)=\begin{cases}
\frac{f(x) - f(x_0)}{x - x_0},&x\neq x_0\\
f^\prime(x_0),&x = x_0
\end{cases}\),则由导数定义可知\(h\in C[a,b]\).又由\(\vert f^\prime(x)\vert<1\),\(\forall x\in[a,b]\),可知\(\vert h(x_0)\vert<1\).
对\(\forall x\neq x_0\),由Lagrange中值定理可知
\[
\vert h(x)\vert=\left\vert\frac{f(x) - f(x_0)}{x - x_0}\right\vert=\vert f^\prime(\theta_x)\vert<1,\quad\theta_x\in(\min\{x,x_0\},\max\{x,x_0\})
\]
故\(\vert h(x)\vert<1\),\(\forall x\in[a,b]\).于是记\(L\triangleq\max_{x\in[a,b]}\vert h(x)\vert\in(0,1)\).因此再由\(Lagrange\)中值定理可得,对\(\forall n\in\mathbb{N}_+\),都有
\[
\vert x_{n + 1}-x_0\vert=\vert f(x_n) - f(x_0)\vert=\vert f^\prime(\xi_n)\vert\vert x_n - x_0\vert,\quad\xi_n\in(\min\{x_n,x_0\},\max\{x_n,x_0\})
\]
从而对\(\forall n\in\mathbb{N}_+\),都有
\[
\vert f^\prime(\xi_n)\vert=\left\vert\frac{f(x_n) - f(x_0)}{x_n - x_0}\right\vert=\vert h(x_n)\vert\leqslant L
\]
进而对\(\forall n\in\mathbb{N}_+\),都有
\[
\vert x_{n + 1}-x_0\vert=\vert f^\prime(\xi_n)\vert\vert x_n - x_0\vert\leqslant L\vert x_n - x_0\vert\leqslant L^2\vert x_{n - 1}-x_0\vert\leqslant\cdots\leqslant L^n\vert x_1 - x_0\vert
\]
上式两边同时令\(n\rightarrow\infty\),则\(\lim_{n\rightarrow\infty}\vert x_{n + 1}-x_0\vert = 0\).即\(\lim_{n\rightarrow\infty}x_n = x_0\).
\end{proof}

\begin{proposition}[反向压缩映像]\label{proposition:反向压缩映像}
设\(x_{n + 1} = f(x_n),n\in\mathbb{N}\)满足
\[
\lim_{n \to \infty} x_n = a\in\mathbb{R},x_n\neq a,\forall n\in\mathbb{N},
\]
证明:若\(f\)在\(x = a\)可导,则\(\vert f'(a)\vert\leq 1\).
\end{proposition}
\begin{proof}
(反证法)假设\(\vert f^\prime(a)\vert > 1\),由导数定义及极限保号性可知,存在\(r > 1\),\(\delta > 0\),使得
\[
\left\vert\frac{f(x) - f(a)}{x - a}\right\vert\geqslant r > 1, \quad\forall x\in [a - \delta, a + \delta].
\]
即
\[
\vert f(x) - f(a)\vert\geqslant r\vert x - a\vert, \quad\forall x\in [a - \delta, a + \delta].
\]
因为\(f\)在\(x = a\)可导以及\(\lim_{n\rightarrow\infty}x_n = a\),所以由\(Heine\)归结原则可知\(\lim_{n\rightarrow\infty}f(x_n) = f(a)\).又\(x_{n + 1} = f(x_n)\),\(\forall n\in\mathbb{N}_+\),从而等式两边同时令\(n\rightarrow\infty\),可得\(a = f(a)\).
由于\(\lim_{n\rightarrow\infty}\vert x_n - a\vert = 0\),因此存在\(N\in\mathbb{N}\),使得对\(\forall n\geqslant N\),有
\[
\vert x_{n + 1} - a\vert = \vert f(x_n) - f(a)\vert\geqslant r\vert x_n - a\vert.
\]
故对\(\forall n\geqslant N\),有
\[
\vert x_{n + 1} - a\vert\geqslant r\vert x_n - a\vert\geqslant r^2\vert x_{n - 1} - a\vert\geqslant\cdots\geqslant r^n\vert x_1 - x_0\vert.
\]
上式两边同时令\(n\rightarrow\infty\),得到\(\lim_{n\rightarrow\infty}\vert x_{n + 1} - a\vert = +\infty\),矛盾.
\end{proof}


\subsection{利用不等放缩求递推数列极限}

\begin{example}
对\(x\geq0\),定义\(y_n(x)=\sqrt[n]{[x[x\cdots[x]\cdots]]}\),这里一共\(n\)层取整,求极限\(\lim_{n\rightarrow\infty}y_n(x)\).
\end{example}
\begin{note}
这里求极限运用了递推的想法找关系,如果直接对取整函数用不等式放缩,只能得到\(x - 1<y_n(x)\leq x\),这没什么用处,因为放缩太粗糙了.

实际上,由Stolz定理可知,数列$\frac{1}{n}$次幂的极限与其相邻两项项除的极限近似相等.
\end{note}
\begin{solution}
显然\(x\in[0,1)\)时\(y_n(x)=0\),\(x\in[1,2)\)时\(y_n(x)=1\),这两个式子对任意\(n\)都成立,下面来看\(x\geq2\)时的极限.

令\(u_n(x) = (y_n(x))^n=\overbrace{[x[\cdots [x]\cdots ]]}^{n\text{次复合}}\geqslant 0\),由于单调递增函数的复合仍是单调递增函数,且\([x]\)在\([0, +\infty)\)上单调递增,故\(u_n(x)\)在\([0, +\infty)\)上单调递增。
从而由\(u_n(x)\)的单调性可得
\[
u_n(x) \geqslant u_n(2)=\overbrace{[2[\cdots [2]\cdots ]]}^{n\text{次复合}} = 2^n\rightarrow \infty, n\rightarrow \infty.
\]
再结合\([x]\)的基本不等式:$x-1<\left[ x \right] \leqslant x$可知
\begin{align*}
&xu_{n-1}(x)-1\le u_n(x)=[xu_{n-1}(x)]\le xu_{n-1}(x),\forall x\ge 2.
\\
\Rightarrow &1-\frac{1}{u_{n-1}(x)}\le \frac{u_n(x)}{u_{n-1}(x)}\le x\Rightarrow \lim_{n\rightarrow \infty} \frac{u_n(x)}{u_{n-1}(x)}=x,\forall x\ge 2.
\end{align*}
再根据Stolz公式有
\begin{align*}
\lim_{n\rightarrow \infty} y_n(x)=\lim_{n\rightarrow \infty} u_n(x)^{\frac{1}{n}}=e^{\lim\limits_{n\rightarrow \infty} \frac{\ln u_n(x)}{n}}=e^{\lim\limits_{n\rightarrow \infty} [\ln u_n(x)-\ln u_{n-1}(x)]}=\lim_{n\rightarrow \infty} \frac{u_n(x)}{u_{n-1}(x)}=x.
\end{align*}
因此
\[ 
\lim_{n\rightarrow\infty}y_n(x)=
\begin{cases}
0, & x\in[0,1)\\
1, & x\in[1,2)\\
x, & x\geq2
\end{cases}
\]
\end{solution}


\subsection{可求通项和强求通项}

\subsubsection{三角换元求通项}

先来看能够直接构造出数列通项的例子.这类问题只能靠记忆积累.找不到递推数列通项就很难处理.一般我们可以猜递推数列通项就是三角函数或\href{https://baike.baidu.com/item/%E5%8F%8C%E6%9B%B2%E5%87%BD%E6%95%B0/8704306}{双曲三角函数}的形式,再利用三角函数或\href{https://baike.baidu.com/item/%E5%8F%8C%E6%9B%B2%E5%87%BD%E6%95%B0/8704306}{双曲三角函数}的性质递推归纳.

\begin{example}
设\(a_1\in(0,1)\),\(a_{n + 1}=\sqrt{\frac{1 + a_n}{2}}\),\(n = 1,2,\cdots\),求\(\lim_{n\rightarrow\infty}a_1a_2\cdots a_n\).
\end{example}
\begin{note}
本题是经典的例子,注意此类问题如果不能求出通项就无法求出具体值,本题便是一个能求出通项从而算出极限值的经典例子.
\end{note}
\begin{remark}
这类问题只能靠记忆积累.
\end{remark}
\begin{solution}
利用
\[
\cos\frac{\theta}{2}=\sqrt{\frac{1 + \cos\theta}{2}},\theta\in\mathbb{R},
\]
因为$a_1\in (0,1)$,所以一定存在$\theta\in(0,\frac{\pi}{2})$,使得\(a_1=\cos\theta\).则$\theta=\arccos a_1,\sin\theta=\sqrt{1-a_1^2}$.并且由\(a_{n + 1}=\sqrt{\frac{1 + a_n}{2}}\),\(n = 1,2,\cdots\)可得
\[
a_2=\cos\frac{\theta}{2},a_3=\cos\frac{\theta}{2^2},\cdots,a_n=\cos\frac{\theta}{2^{n - 1}}.
\]
因此
\begin{align*}
\lim_{n\rightarrow\infty}a_1a_2\cdots a_n&=\lim_{n\rightarrow\infty}\prod_{k = 0}^{n - 1}\cos\frac{\theta}{2^k}=\lim_{n\rightarrow\infty}\frac{\sin\frac{\theta}{2^{n - 1}}}{\sin\frac{\theta}{2^{n - 1}}}\prod_{k = 0}^{n - 1}\cos\frac{\theta}{2^k}=\lim_{n\rightarrow\infty}\frac{\sin\frac{\theta}{2^{n - 2}}}{2\sin\frac{\theta}{2^{n - 1}}}\prod_{k = 0}^{n - 2}\cos\frac{\theta}{2^k}\\
&=\cdots=\lim_{n\rightarrow\infty}\frac{\sin2\theta}{2^n\sin\frac{\theta}{2^{n - 1}}}=\frac{\sin2\theta}{2\theta}=\frac{\sin(2\arccos a_1)}{2\arccos a_1}=\frac{a_1\sqrt{1 - a_1^2}}{\arccos a_1}.
\end{align*}
\end{solution}

\begin{example}
设\(x_1 = \sqrt{5}\),\(x_{n + 1}=x_{n}^{2}-2\),计算
\[
\lim_{n\rightarrow\infty}\frac{x_1x_2\cdots x_n}{x_{n + 1}}.
\]
\end{example}
\begin{note}
这类问题只能靠记忆积累.找不到递推数列通项就很难处理.一般我们可以猜递推数列通项就是三角函数/双曲三角函数的形式,再利用三角函数/双曲三角函数的性质递推归纳.
\end{note}
\begin{solution}
注意到\(\cos x=\frac{\sqrt{5}}{2}\)在\(\mathbb{R}\)上无解,因此推测类似的\href{https://baike.baidu.com/item/%E5%8F%8C%E6%9B%B2%E5%87%BD%E6%95%B0/8704306}{双曲三角函数}可以做到.
设\(x_1 = 2\cosh\theta,\theta\in(0,+\infty)\). 利用
\[
\cosh x=2\cosh^{2}\frac{x}{2}-1,\forall x\in\mathbb{R},
\]
我们归纳可证
\[
x_n=2\cosh(2^{n - 1}\theta),n = 1,2,\cdots.
\]
于是利用\(\sinh(2x)=2\sinh x\cosh x,\forall x\in\mathbb{R}\),我们有
\begin{align*}
\lim_{n\rightarrow\infty}\frac{x_1x_2\cdots x_n}{x_{n + 1}}&=\lim_{n\rightarrow\infty}\frac{2^{n}\prod\limits_{k = 0}^{n - 1}\cosh(2^{k}\theta)}{2\cosh(2^{n}\theta)}=\lim_{n\rightarrow\infty}\frac{2^{n}\sinh\theta\prod\limits_{k = 0}^{n - 1}\cosh(2^{k}\theta)}{2\sinh\theta\cosh(2^{n}\theta)}=\lim_{n\rightarrow\infty}\frac{2^{n - 1}\sinh(2\theta)\prod\limits_{k = 1}^{n - 1}\cosh(2^{k}\theta)}{2\sinh\theta\cosh(2^{n}\theta)}\\
&=\lim_{n\rightarrow\infty}\frac{2^{n - 2}\sinh(2^{2}\theta)\prod\limits_{k = 2}^{n - 1}\cosh(2^{k}\theta)}{2\sinh\theta\cosh(2^{n}\theta)}=\lim_{n\rightarrow\infty}\frac{\sinh2^{n}\theta}{2\sinh\theta\cosh(2^{n}\theta)}=\lim_{n\rightarrow\infty}\frac{\tanh2^{n}\theta}{2\sinh\theta}=\frac{1}{2\sinh\theta}=1,
\end{align*}
这里倒数第二个等号来自\(\lim_{x\rightarrow+\infty}\tanh x = 1\).
\end{solution}

\begin{example}
设\(a_1 = 3,a_n=2a_{n - 1}^{2}-1,n = 2,3,\cdots\),则计算
\[
\lim_{n\rightarrow\infty}\frac{a_n}{2^n a_1a_2\cdots a_{n - 1}}.
\]
\end{example}
\begin{remark}
因为双曲三角函数\(\cosh x\)在\((0, +\infty)\)上的值域为\((1, +\infty)\),并且\(\cosh x\)在\((0, +\infty)\)上严格递增,所以一定存在唯一的\(\theta \in (0, +\infty)\),使得\(a_1 = \cosh\theta = 3\).
\end{remark}
\begin{proof}
设\(a_1 = \cosh\theta=3, \theta \in (0, +\infty)\).则利用\(\cosh 2\theta = 2\cosh^2\theta - 1\),再结合条件归纳可得
\[
a_n = 2a_{n - 1}^{2} - 1 = \cosh 2^{n - 1}\theta, \quad n = 2, 3, \cdots.
\]
于是
\begin{align*}
\lim_{n\rightarrow \infty} \frac{a_n}{2^na_1a_2\cdots a_{n - 1}}
&=\lim_{n\rightarrow \infty} \frac{\cosh 2^{n - 1}\theta}{2^n\prod\limits_{k = 1}^{n - 1}{\cosh 2^{k - 1}\theta}}
=\lim_{n\rightarrow \infty} \frac{\sinh\theta\cosh 2^{n - 1}\theta}{2^n\sinh\theta\prod\limits_{k = 1}^{n - 1}{\cosh 2^{k - 1}\theta}}\\
&=\lim_{n\rightarrow \infty} \frac{\sinh\theta\cosh 2^{n - 1}\theta}{2^{n - 1}\sinh 2\theta\prod\limits_{k = 2}^{n - 1}{\cosh 2^{k - 1}\theta}}
=\cdots
=\lim_{n\rightarrow \infty} \frac{\sinh\theta\cosh 2^{n - 1}\theta}{2\sinh 2^{n - 1}\theta}\\
&=\lim_{n\rightarrow \infty} \frac{\sinh\theta}{2\tanh 2^{n - 1}\theta}
\xlongequal{\lim\limits_{n\rightarrow \infty} \tanh 2^{n - 1}\theta = 1}\frac{\sinh\theta}{2}=\frac{\sqrt{\cosh ^2\theta -1}}{2}=\sqrt{2}.
\end{align*}
\end{proof}

\begin{example}
设\(y_0\geq2,y_n=y_{n - 1}^{2}-2,n\in\mathbb{N}\),计算\(\sum_{n = 0}^{\infty}\frac{1}{y_0y_1\cdots y_n}\).
\end{example}
\begin{note}
关于求和的问题,要注意求和的通项能否凑成相邻两项相减的形式,从而就能直接求和消去中间项,进而将求和号去掉.
\end{note}
\begin{remark}
因为双曲三角函数\(2\cosh x\)在\((0, +\infty)\)上的值域为\((1, +\infty)\),并且\(2\cosh x\)在\((0, +\infty)\)上严格递增,所以一定存在唯一的\(\theta \in (0, +\infty)\),使得\(y_0 = 2\cosh\theta \geq 2\).
\end{remark}
\begin{proof}
设\(y_0 = 2\cosh\theta, \theta \in (0, +\infty)\),则利用\(\cosh 2\theta = 2\cosh^2\theta - 1\),再结合条件归纳可得
\begin{align*}
y_1&=y_{0}^{2}-2 = 4\cosh^2\theta - 2 = 2(2\cosh^2\theta - 1) = 2\cosh 2\theta,\\
y_2&=y_{1}^{2}-2 = 4\cosh^22\theta - 2 = 2(2\cosh^22\theta - 1) = 2\cosh 2^2\theta,\\
&\cdots\cdots\\
y_n&=y_{n - 1}^{2}-2 = 4\cosh^22^{n - 1}\theta - 2 = 2(2\cosh^22^{n - 1}\theta - 1) = 2\cosh 2^n\theta,\\
&\cdots\cdots
\end{align*}
于是
\begin{align*}
\sum_{n=0}^{\infty}{\frac{1}{y_0y_1\cdots y_n}}&=\sum_{n=0}^{\infty}{\frac{1}{\prod\limits_{k=0}^n{2^{n+1}\cosh 2^k\theta}}}=\sum_{n=0}^{\infty}{\frac{\sinh \theta}{2^{n+1}\sinh \theta \prod\limits_{k=0}^n{\cosh 2^k\theta}}}
\\
&=\sum_{n=0}^{\infty}{\frac{\sinh \theta}{2^n\sinh 2\theta \prod\limits_{k=1}^n{\cosh 2^k\theta}}}=\cdots =\sum_{n=0}^{\infty}{\frac{\sinh \theta}{\sinh 2^{n+1}\theta}}
\\
&=2\sinh \theta \sum_{n=0}^{\infty}{\frac{1}{e^{2^{n+1}\theta}-e^{-2^{n+1}\theta}}}=2\sinh \theta \sum_{n=0}^{\infty}{\frac{e^{2^{n+1}\theta}}{e^{2^{n+2}\theta}-1}}
\\
&=2\sinh \theta \sum_{n=0}^{\infty}{\left( \frac{1}{e^{2^{n+1}\theta}-1}-\frac{1}{e^{2^{n+2}\theta}-1} \right)}=\frac{2\sinh \theta}{e^{2\theta}-1}
\\
&=\frac{e^{\theta}-e^{-\theta}}{e^{\theta}\left( e^{\theta}-e^{-\theta} \right)}=e^{-\theta}=\cosh \theta -\sinh \theta 
\\
&=\frac{y_0}{2}-\sqrt{\cosh ^2\theta -1}=\frac{y_0}{2}-\sqrt{\frac{y_{0}^{2}}{4}-1}.
\end{align*}
\end{proof}

\subsubsection{凑出可求通项的递推数列}

利用比值换元等方法,可以将原本不能直接求通项的递推数列转化成可三角换元或用高中方法求通项的递推数列.求出通项后,后续问题就很简单了.

\begin{example}
设\(a > b>0\),定义\(a_0 = a\),\(b_0 = b\),\(a_{n + 1}=\frac{a_n + b_n}{2}\),\(b_{n + 1}=\frac{2a_nb_n}{a_n + b_n}\),求\(\lim_{n\rightarrow\infty}a_n\),\(\lim_{n\rightarrow\infty}b_n\)。
\end{example}
\begin{remark}
这是算数-调和平均数数列,与算术-几何平均不同,这个通项以及极限值都可以求出来.
\end{remark}\begin{note}
\(x_{n + 1}=\frac{1}{2}\left(x_n+\frac{1}{x_n}\right)\)是一个经典的可求通项的递推数列(高中学过),处理方法必须掌握.即先求解其特征方程,然后用$x_{n+1}$分别减去两个特征根再作商,再将递推式代入这个分式,反复递推得到一个等比数列,进而得到$x_n$的通项.具体步骤见下述证明.
\end{note}
\begin{proof}
由条件可得
\[
a_{n + 1}=\frac{a_n + b_n}{2},b_{n + 1}=\frac{2a_nb_n}{a_n + b_n}=\frac{a_nb_n}{a_{n + 1}}\Rightarrow a_{n + 1}b_{n + 1}=a_nb_n=\cdots =a_0b_0 = ab.
\]
因此 \(a_{n + 1}=\frac{1}{2}\left(a_n+\frac{ab}{a_n}\right)\)。令 \(a_n=\sqrt{ab}x_n,x_0=\sqrt{\frac{a}{b}}>1\),则 \(x_{n + 1}=\frac{1}{2}\left(x_n+\frac{1}{x_n}\right),\forall n\in \mathbb{N}_+\)。从而
\begin{align*}
\frac{x_{n + 1}-1}{x_{n + 1}+1}=\frac{\frac{x_{n + 1}^2 - 1}{2x_{n + 1}}}{\frac{x_{n + 1}^2 + 1}{2x_{n + 1}}}
=\frac{(x_n - 1)^2}{(x_n + 1)^2}
=\cdots
=\left(\frac{x_0 - 1}{x_0 + 1}\right)^{2^{n + 1}}\Rightarrow \frac{x_n - 1}{x_n + 1}=C^{2^n},C=\frac{x_0 - 1}{x_0 + 1}\in(0,1).
\end{align*}
于是 \(x_n=\frac{1 + C^{2^n}}{1 - C^{2^n}}\)。再由 \(a_n=\sqrt{ab}x_n\) 可得
\[
a_n=\sqrt{ab}\frac{1 + C^{2^n}}{1 - C^{2^n}}\rightarrow\sqrt{ab},n\rightarrow\infty.
\]
\[
b_n=\frac{ab}{a_n}\rightarrow\sqrt{ab},n\rightarrow\infty.
\]
\end{proof}

\begin{example}\label{example-4.54112}
设\(a_{n + 1}=\frac{2a_nb_n}{a_n + b_n}\),\(b_{n + 1}=\sqrt{a_{n + 1}b_n}\),证明:\(a_n,b_n\)收敛到同一极限,并且在\(a_1 = 2\sqrt{3},b_1 = 3\)时,上述极限值为\(\pi\).
\end{example}
\begin{remark}
这是几何 - 调和平均数列,通项也能求出来,自然求极限就没有任何问题.
\end{remark}
\begin{note}
\hypertarget{比值换元的问题}{(1)}因为$a_n,b_n$的递推式都是齐次式,所以我们尝试比值换元,将其转化为可求通项的递推数列.实际上,我们利用的比值换元是$c_n=\frac{b_n}{a_n}$,但是为了避免讨论数列$a_n$能否取0的情况,我们就取$b_n=a_nc_n$.

\hypertarget{三角换元的问题}{(2)}三角换元求通项的一些问题:由递推条件易证$a_n,b_n\geq 0$,其实当$a_n,b_n$中出现为零的项时,由递推条件易知$a_n,b_n$后面的所有项都为零,此时结论平凡.因此我们只需要考虑$a_n,b_n>0$的情况.此时直接设\(\cos x_1 = c_1=\frac{b_1}{a_1}\)似乎不太严谨.因为虽然$c_1>0$,但是$c_1$不一定在$(0,1)$内,所以我们需要对其进行分类讨论.

当$c_1\in (0,1)$时,设\(\cos x_1 = c_1=\frac{b_1}{a_1}\),其中$x_1\in (0,\frac{\pi}{2})$;

当$c_1>1$时,设\(\cosh x_1 = c_1=\frac{b_1}{a_1}\),其中$x_1\in (0,+\infty)$.

实际上,我们直接设\(\cos x_1 = c_1=\frac{b_1}{a_1}\),只要将$x_1$看作一个复数,就可以避免分类讨论.因为由复变函数论可知,$\cos x$在复数域上的性质与极限等结论与在实数域上相同,而且由$c_1>0$可知,一定存在一个复数$x_1$,使得$\cos x_1=c_1$.所以这样做是严谨地.(考试的时候最好还是分类讨论书写)
\end{note}
\begin{proof}
\hyperlink{比值换元的问题}{设\(b_n = a_nc_n\)}代入有
\begin{align}
a_{n + 1}&=\frac{2a_nb_n}{a_n + b_n}=\frac{2a_nc_n}{c_n + 1},a_{n + 1}c_{n + 1}=\sqrt{a_{n + 1}a_nc_n}\Rightarrow\frac{a_{n + 1}}{a_n}=\frac{c_n}{c_{n + 1}^2}=\frac{2c_n}{c_n + 1}\Rightarrow c_{n + 1}=\sqrt{\frac{c_n + 1}{2}}\label{asfasg}
\end{align}
\hyperlink{三角换元的问题}{设\(\cos x_1 = c_1=\frac{b_1}{a_1}\),其中$x_1\in \mathbb{C}$},则由\eqref{asfasg}式归纳可得$c_n=\cos \left( \frac{x_1}{2^{n-1}} \right)$.
代入回去求\(a_n,b_n\)有
\begin{align*}
&c_n=\frac{b_n}{a_n}=\cos\left(\frac{x_1}{2^{n - 1}}\right),b_{n + 1}=\sqrt{a_{n + 1}b_n}\Rightarrow b_{n + 1}^2=a_{n + 1}b_n=\frac{b_{n + 1}b_n}{c_{n + 1}}\Rightarrow\frac{b_{n + 1}}{b_n}=\frac{1}{\cos\left(\frac{x_1}{2^{n}}\right)}\\
&\Rightarrow \frac{b_{n + 1}}{b_1}=\frac{1}{\cos\left(\frac{x_1}{2}\right)\cos\left(\frac{x_1}{2^2}\right)\cdots\cos\left(\frac{x_1}{2^n}\right)}=\frac{2^n\sin\frac{x_1}{2^n}}{\sin x_1}\Rightarrow b_n = b_1\frac{2^{n - 1}\sin\frac{x_1}{2^{n - 1}}}{\sin x_1}\\
&\Rightarrow a_n=\frac{b_n}{c_n}=\frac{b_1\frac{2^{n - 1}\sin\frac{x_1}{2^{n - 1}}}{\sin x_1}}{\cos\frac{x_1}{2^{n - 1}}}=2^{n - 1}\frac{b_1}{\sin x_1}\tan\frac{x_1}{2^{n - 1}},\cos x_1 = c_1=\frac{b_1}{a_1}
\end{align*}
由此可见
\[
\lim_{n\rightarrow\infty}a_n=\lim_{n\rightarrow\infty}b_n=\frac{b_1x_1}{\sin x_1}=\frac{b_1\arccos\frac{b_1}{a_1}}{\sqrt{1 - \frac{b_1^2}{a_1^2}}}=\frac{a_1b_1\arccos\frac{b_1}{a_1}}{\sqrt{a_1^2 - b_1^2}}
\]
所以收敛到同一极限
对于\(a_1 = 2\sqrt{3},b_1 = 3\)的情况有
\[
\cos x_1=\frac{3}{2\sqrt{3}}=\frac{\sqrt{3}}{2},x_1=\frac{\pi}{6},\lim_{n\rightarrow\infty}a_n=\lim_{n\rightarrow\infty}b_n=\frac{b_1x_1}{\sin x_1}=\pi
\]
结论得证.
\end{proof}

\begin{example}
设\(a_n = 2^{n - 1}-3a_{n - 1},n\geq1\),求\(a_0\)的所有可能值,使得\(a_n\)严格单调递增。
\end{example}
\begin{proof}
直接裂项,求通项即可得到
\begin{align*}
&\frac{a_{n+1}}{(-3)^{n+1}}=\frac{a_n}{(-3)^n}+\frac{2^n}{(-3)^{n+1}}\Rightarrow \frac{a_{n+1}}{(-3)^{n+1}}-\frac{a_n}{(-3)^n}=\frac{2^n}{(-3)^{n+1}}
\\
&\Rightarrow \frac{a_{n+1}}{(-3)^{n+1}}=\frac{a_0}{(-3)^0}-\frac{1}{3}\left( 1+\left( -\frac{2}{3} \right) +\cdots +\left( -\frac{2}{3} \right) ^n \right) =a_0-\frac{1}{3}\frac{1-\left( -\frac{2}{3} \right) ^{n+1}}{\frac{5}{3}}
\\
&\Rightarrow \frac{a_n}{(-3)^n}=a_0-\frac{1}{5}\left( 1-\left( -\frac{2}{3} \right) ^n \right) \Rightarrow a_n=\left( a_0-\frac{1}{5} \right) (-3)^n+\frac{1}{5}2^n.
\end{align*}
由此可见\(a_0 = \frac{1}{5}\)是唯一解.
\end{proof}

\begin{example}
设\(x_1 > 0,x_{n + 1}=1+\frac{1}{x_n}\),求极限\(\lim_{n\rightarrow\infty}x_n\).
\end{example}
\begin{proof}
解方程\(x^2 - x - 1 = 0\Rightarrow\lambda_1=\frac{1 + \sqrt{5}}{2},\lambda_2=\frac{1 - \sqrt{5}}{2}\),于是
\begin{align*}
\frac{x_{n + 1}-\lambda_1}{x_{n + 1}-\lambda_2}&=\frac{1+\frac{1}{x_n}-\lambda_1}{1+\frac{1}{x_n}-\lambda_2}=\frac{(1 - \lambda_1)x_n + 1}{(1 - \lambda_2)x_n + 1}=\frac{\lambda_2x_n + 1}{\lambda_1x_n + 1}=\frac{\lambda_2}{\lambda_1}\frac{x_n+\frac{1}{\lambda_2}}{x_n+\frac{1}{\lambda_1}}=\frac{\lambda_2}{\lambda_1}\frac{x_n-\lambda_1}{x_n-\lambda_2}\\
&\Rightarrow\frac{x_{n + 1}-\lambda_1}{x_{n + 1}-\lambda_2}=\left(\frac{\lambda_2}{\lambda_1}\right)^n\frac{x_1-\lambda_1}{x_1-\lambda_2}\to 0,n\to \infty.
\end{align*}
故$\lim_{n\rightarrow\infty}x_n=\lambda_1=\frac{1 + \sqrt{5}}{2}.$
\end{proof}

\begin{example}
设\(a_{n + 1}=\sqrt{a_n + 2}\),求\(a_n\)的通项公式.
\end{example}
\begin{proof}
设\(a_n = 2b_n\)则问题转化为已知$b_{n+1}=\sqrt{\frac{b_n+1}{2}}$,求$b_n$的通项公式.
由\hyperref[example-4.54112]{例题\ref{example-4.54112}},立即得到\[a_n = 2\cos\frac{\theta_1}{2^{n - 1}},\cos\theta_1=\frac{1}{2}a_1.\]
\end{proof}


\subsubsection{直接凑出通项}

\begin{example}
设\(a_1=\frac{1}{2},a_{n + 1}=2a_n^2 + 2a_n\),求\(a_n\)的通项公式.
\end{example}
\begin{proof}
\begin{align*}
&a_{n + 1}=2a_n^2 + 2a_n = 2\left(a_n+\frac{1}{2}\right)^2-\frac{1}{2}\Rightarrow 2a_{n + 1}+1=(2a_n + 1)^2=\cdots=(2a_1 + 1)^{2^n}\\
&\Rightarrow a_n=\frac{(2a_1 + 1)^{2^{n - 1}}-1}{2}=\frac{2^{2^{n - 1}}-1}{2}.
\end{align*}
\end{proof}


\subsubsection{凑裂项}

凑裂项:根据已知的递推式,将需要求解的累乘或求和的通项凑成裂项的形式,使得其相邻两项相乘或相加可以抵消中间项,从而将累乘或求和号去掉.

\begin{example}
设\(a_1 = 1,a_n = n(a_{n - 1}+1),x_n=\prod_{k = 1}^{n}\left(1+\frac{1}{a_k}\right)\),求极限\(\lim_{n\rightarrow\infty}x_n\).
\end{example}
\begin{proof}
由条件可知$a_n+1=\frac{a_{n+1}}{n+1}$,从而
\begin{align*}
x_n=\prod_{k=1}^n{\left( 1+\frac{1}{a_k} \right)}=\prod_{k=1}^n{\frac{a_k+1}{a_k}}=\prod_{k=1}^n{\frac{a_{k+1}}{(k+1)a_k}}=\frac{a_{n+1}}{a_1}\frac{1}{(n+1)!}=\frac{a_{n+1}}{(n+1)!}.
\end{align*}
再根据$a_n=n(a_{n-1}+1)$可得
\begin{align*}
\frac{a_n}{n!}=\frac{a_{n-1}}{(n-1)!}+\frac{1}{(n-1)!}.
\end{align*}
故
\begin{align*}
x_n=\frac{a_{n+1}}{(n+1)!}=\frac{a_n}{n!}+\frac{1}{n!}=\frac{a_{n-1}}{(n-1)!}+\frac{1}{(n-1)!}+\frac{1}{n!}=\cdots =\sum_{k=0}^n{\frac{1}{k!}\rightarrow e.}
\end{align*}
\end{proof}

\begin{example}
设\(y_0>2,y_n=y_{n - 1}^{2}-2,n\in\mathbb{N}\),计算\(\prod_{n = 0}^{\infty}(1 - \frac{1}{y_n})\).  
\end{example}
\begin{note}
关于累乘的问题,要注意累乘的通项能否凑成相邻两项相除的形式,从而就能直接累乘消去中间项,进而将累乘号去掉.

本题是利用已知条件和平方差公式将累乘的通项能否凑成相邻两项相除的形式.
\end{note}
\begin{proof}
一方面
\[
y_n + 1=y_{n - 1}^{2}-1=(y_{n - 1}-1)(y_{n - 1}+1)\Rightarrow y_{n - 1}-1=\frac{y_n + 1}{y_{n - 1}+1}\Rightarrow y_n - 1=\frac{y_{n + 1}+1}{y_n + 1}.
\]
另外一方面
\[
y_n - 2=y_{n - 1}^{2}-4=(y_{n - 1}-2)(y_{n - 1}+2)\Rightarrow y_n - 2=(y_{n - 1}-2)y_{n - 2}^{2}\Rightarrow y_n=\sqrt{\frac{y_{n + 2}-2}{y_{n + 1}-2}}.
\]
于是结合\(\lim_{m\rightarrow\infty}y_m = +\infty\)我们有
\begin{align*}
\prod_{n = 0}^{\infty}\left(1-\frac{1}{y_n}\right)&=\prod_{n=0}^{\infty}{\frac{y_n-1}{y_n}}=\prod_{n = 0}^{\infty}\left(\frac{y_{n + 1}+1}{y_n + 1}\cdot\sqrt{\frac{y_{n + 1}-2}{y_{n + 2}-2}}\right)
=\lim_{m\rightarrow\infty}\prod_{n = 0}^{m}\left(\frac{y_{n + 1}+1}{y_n + 1}\cdot\sqrt{\frac{y_{n + 1}-2}{y_{n + 2}-2}}\right)\\
&=\lim_{m\rightarrow\infty}\frac{y_{m + 1}+1}{y_0 + 1}\cdot\sqrt{\frac{y_1 - 2}{y_{m + 2}-2}}
=\lim_{m\rightarrow\infty}\frac{y_{m + 1}+1}{\sqrt{y_{m + 1}^{2}-4}}\cdot\frac{\sqrt{y_0^{2}-4}}{y_0 + 1}
=\frac{\sqrt{y_0^{2}-4}}{y_0 + 1}.
\end{align*}
\end{proof}





\subsubsection{母函数法求通项}

\begin{example}
设\(a_{n + 1}=a_n+\frac{2}{n + 1}a_{n - 1},n\geq1,a_0>0,a_1>0\),求极限\(\lim_{n\rightarrow\infty}\frac{a_n}{n^2}\)。
\end{example}
\begin{remark}
本题采用单调有界只能证明极限存在,而并不能算出来极限值:
\[
\frac{a_{n + 1}}{(n + 1)^2}-\frac{a_n}{n^2}=\frac{a_n+\frac{2}{n + 1}a_{n - 1}}{(n + 1)^2}-\frac{a_n}{n^2}=\frac{2n^2a_{n - 1}-(2n + 1)(n + 1)a_n}{n^2(n + 1)^3}<0
\]
\end{remark}
\begin{proof}
这类线性递推数列问题采用母函数方法是无敌的,因为能求出来通项公式。
设\(f(x)=\sum_{n = 0}^{\infty}a_nx^n\)则根据条件有
\begin{align*}
f'(x)&=\sum_{n = 1}^{\infty}na_nx^{n - 1}=\sum_{n = 0}^{\infty}(n + 1)a_{n + 1}x^n=a_1+\sum_{n = 1}^{\infty}(n + 1)\left(a_n+\frac{2}{n + 1}a_{n - 1}\right)x^n\\
&=a_1+\sum_{n = 1}^{\infty}na_nx^n+\sum_{n = 1}^{\infty}a_nx^n+2\sum_{n = 1}^{\infty}a_{n - 1}x^n=a_1+xf'(x)+f(x)-a_0+2xf(x)\\
&\Rightarrow f'(x)+\frac{2x + 1}{1 - x}f(x)=\frac{a_1 - a_0}{1 - x},f(0)=a_0,f'(0)=a_1
\end{align*}
这是一阶线性微分方程,容易求出
\[f(x)=\frac{2x^2 - 6x + 5}{(1 - x)^3}\frac{a_1 - a_0}{4}+\frac{e^{-2x}}{(1 - x)^3}\frac{9a_0 - 5a_1}{4}=\sum_{n = 0}^{\infty}a_nx^n\]
然后对左边这两个函数(先不看系数)做泰勒展开,关注\(x^n\)前面的\(n^2\)项系数,就对应极限。
\begin{align*}
\frac{1}{1 - x}&=\sum_{n = 0}^{\infty}x^n\Rightarrow\frac{1}{(1 - x)^2}=\sum_{n = 0}^{\infty}(n + 1)x^n,\frac{1}{(1 - x)^3}=\sum_{n = 0}^{\infty}\frac{(n + 2)(n + 1)}{2}x^n\\
\frac{2x^2 - 6x + 5}{(1 - x)^3}&=(2x^2 - 6x + 5)\sum_{n = 0}^{\infty}\frac{(n + 2)(n + 1)}{2}x^n=\sum_{n = 0}^{\infty}(5b_n - 6b_{n - 1}+2b_{n - 2})x^n\\
b_n&=\frac{(n + 2)(n + 1)}{2}\Rightarrow 5b_n - 6b_{n - 1}+2b_{n - 2}=\frac{1}{2}n^2+O(n)
\end{align*}
由此可见第一部分对应着极限\(\frac{a_1 - a_0}{8}\),然后算第二部分
\begin{align*}
\frac{e^{-2x}}{(1 - x)^3}&=\sum_{m = 0}^{\infty}\frac{(-2)^m}{m!}x^m\sum_{n = 0}^{\infty}\frac{(n + 2)(n + 1)}{2}x^n=\sum_{k = 0}^{\infty}\sum_{m + n = k}\frac{(-2)^m}{m!}\frac{(n + 2)(n + 1)}{2}x^k
\end{align*}
所以每一个\(x^m\)项相应的系数是
\[
\sum_{k = 0}^{m}\frac{(-2)^m}{m!}\frac{(k + 2 - m)(k + 1 - m)}{2}=\frac{1}{2}\sum_{k = 0}^{m}\frac{(-2)^m}{m!}(m - (k - 1))(m - (k - 2))
\]
由Stolz公式和$e^x$的无穷级数展开式可得,对应的极限为
\begin{align*}
\frac{1}{2}\lim_{m\rightarrow\infty}\frac{\sum\limits_{k = 0}^{m}\frac{(-2)^m}{m!}(m^2-(2k - 3)m+(k - 1)(k - 2))}{m^2}
=\frac{1}{2}\lim_{m\rightarrow\infty}\sum_{k = 0}^{m}\frac{(-2)^m}{m!}=\frac{1}{2e^2}
\end{align*}
这是因为括号里面的\(m\)一次项和常数项部分,对应的求和的极限是零,由stolz公式是显然的。
所以第二部分提供了\(\frac{9a_0 - 5a_1}{8e^2}\),最终所求极限为\(\lim_{n\rightarrow\infty}\frac{a_n}{n^2}=\frac{a_1 - a_0}{8}+\frac{9a_0 - 5a_1}{8e^2}\)。
\end{proof}





\subsubsection{强求通项和强行裂项}

若数列\(\{ a_n \}_{n = 0}^{\infty}, \{ b_n \}_{n = 0}^{\infty}, \{ d_n \}_{n = 0}^{\infty}\)满足下列递推条件之一:
\begin{enumerate}
\item \(a_n = d_na_{n - 1} + b_n, n = 1, 2, \cdots\);
\item \(\lim_{n\rightarrow \infty}(a_n - d_na_{n - 1}) = A\).
\end{enumerate}
则我们都可以考虑对\(a_n\)进行强行裂项和强求通项,从而可以将\(a_n\)写成关于\(b_n, d_n\)或\(A, d_n\)的形式,进而将题目条件和要求进行转化.


\begin{proposition}[强求通项和强行裂项]\label{proposition:强求通项和强行裂项}
\begin{enumerate}[(1)]
\item 若数列\(\{ a_n \}_{n = 0}^{\infty}, \{ b_n \}_{n = 0}^{\infty}, \{ d_n \}_{n = 0}^{\infty}\)满足递推条件:
\begin{align}\label{proposition4.8-0.1}
a_n = d_na_{n - 1} + b_n, n = 1, 2, \cdots,  
\end{align}
则令\(c_n = \prod_{k = 1}^n{\frac{1}{d_k}}, n = 0, 1, \cdots\),一定有
\[
a_n = \frac{1}{c_n}\sum_{k = 1}^n{c_kb_k} + a_0, n = 0, 1, \cdots.
\]

\item 若数列\(\{ a_n \}_{n = 0}^{\infty}, \{ d_n \}_{n = 0}^{\infty}\)满足递推条件:
\begin{align}\label{proposition4.8-0.2}
\lim_{n\rightarrow \infty}(a_n - d_na_{n - 1}) = A,  
\end{align}
则令\(c_n = \prod_{k = 1}^n{\frac{1}{d_k}}, n = 0, 1, \cdots\),再令\(b_0 = 1, b_n = a_n - \frac{c_{n - 1}a_{n - 1}}{c_n}, n = 1, 2, \cdots\),一定有
\[
\lim_{n\rightarrow \infty}b_n = A,
\]
\[
a_n = \frac{1}{c_n}\sum_{k = 1}^n{c_kb_k} + a_0, n = 0, 1, \cdots.
\]
\end{enumerate}
\end{proposition}
\begin{remark}
此时\textbf{只能都对\(a_n\)进行强行裂项和强求通项},\(b_n\)和\(d_n\)都无法通过这种方法强行裂项和强求通项!
\end{remark}
\begin{note}
也可以通过观察原数列$a_n$的递推条件直接得到需要构造的数列,从而将$a_n$强行裂项和强求通项.具体可见\hyperref[example:4.341111]{例题\ref{example:4.341111}解法
一}.
(1)的具体应用可见\hyperlink{递推条件是等式形式的待定数列法}{例题\ref{example:4.351111}笔记};
(2)的具体应用可见\hyperlink{递推条件是极限形式的待定数列法}{例题\ref{example:4.341111}笔记}.
\end{note}
\begin{proof}
(强行裂项和强求通项的具体步骤)
\begin{enumerate}[(1)]
\item 若数列\(\{ a_n \}_{n = 0}^{\infty}, \{ b_n \}_{n = 0}^{\infty}, \{ d_n \}_{n = 0}^{\infty}\)满足递推条件\eqref{proposition4.8-0.1}式,则令\(c_0 = 1\),待定\(\{ c_n \}_{n = 0}^{\infty}\),由递推条件\eqref{proposition4.8-0.1}式可得
\begin{align}\label{proposition4.8-1.1}
c_na_n = c_nd_na_{n - 1} + c_nb_n, n = 1, 2, \cdots. 
\end{align}
我们希望\(c_nd_n = c_{n - 1}, n = 2, 3, \cdots\),即\(\frac{c_n}{c_{n - 1}} = \frac{1}{d_n}, n = 2, 3, \cdots\).从而\(c_n = c_0\prod_{k = 1}^n{\frac{c_k}{c_{k - 1}}} = \prod_{k = 1}^n{\frac{1}{d_k}}, n = 1, 2, \cdots\),且该式对\(n = 0\)也成立.因此,令\(c_n = \prod_{k = 1}^n{\frac{1}{d_k}}, n = 0, 1, \cdots\),则由\eqref{proposition4.8-1.1}式可知
\[
c_na_n = c_nd_na_{n - 1} + c_nb_n \Rightarrow c_na_n - c_{n - 1}a_{n - 1} = c_nb_n, n = 1, 2, \cdots.
\]
于是
\begin{align*}
a_n = \frac{1}{c_n}(c_na_n - c_0a_0 + c_0a_0)
= \frac{1}{c_n}\left[ \sum_{k = 1}^n{(c_ka_k - c_{k - 1}a_{k - 1})} + c_0a_0 \right]
= \frac{1}{c_n}\sum_{k = 1}^n{c_kb_k} + a_0, n = 0, 1, \cdots.
\end{align*}
这样就完成了对\(a_n\)的强行裂项和强求通项,并将\(a_n\)写成了关于\(b_n, d_n\)的形式.

\item 若数列\(\{ a_n \}_{n = 0}^{\infty}, \{ d_n \}_{n = 0}^{\infty}\)满足递推条件\eqref{proposition4.8-0.2}式,则令\(c_0 = 1\),待定\(\{ c_n \}_{n = 0}^{\infty}\),由递推条件\eqref{proposition4.8-0.2}式可得
\begin{align}\label{proposition4.8-1.2}
\lim_{n\rightarrow \infty}(a_n - d_na_{n - 1}) = \lim_{n\rightarrow \infty}\frac{c_na_n - c_nd_na_{n - 1}}{c_n} = A. 
\end{align}
我们希望\(c_nd_n = c_{n - 1}, n = 2, 3, \cdots\),即\(\frac{c_n}{c_{n - 1}} = \frac{1}{d_n}, n = 2, 3, \cdots\).从而\(c_n = c_0\prod_{k = 1}^n{\frac{c_k}{c_{k - 1}}} = \prod_{k = 1}^n{\frac{1}{d_k}}, n = 1, 2, \cdots\),且该式对\(n = 0\)也成立.因此,令\(c_n = \prod_{k = 1}^n{\frac{1}{d_k}}, n = 0, 1, \cdots\),则由\eqref{proposition4.8-1.2}式可知
\begin{align}\label{proposition4.8-1.3}
\lim_{n\rightarrow \infty}(a_n - d_na_{n - 1}) = \lim_{n\rightarrow \infty}\frac{c_na_n - c_nd_na_{n - 1}}{c_n} = \lim_{n\rightarrow \infty}\frac{c_na_n - c_{n - 1}a_{n - 1}}{c_n} = A.
\end{align}
于是令\(b_0 = 1\),待定\(\{ b_n \}_{n = 0}^{\infty}\),希望\(b_n\)满足\(c_nb_n = c_na_n - c_{n - 1}a_{n - 1}, n = 1, 2, \cdots\),即\(b_n = \frac{c_na_n - c_{n - 1}a_{n - 1}}{c_n} = a_n - \frac{c_{n - 1}a_{n - 1}}{c_n}, n = 1, 2, \cdots\).
因此,令\(b_0 = 1, b_n = a_n - \frac{c_{n - 1}a_{n - 1}}{c_n}, n = 1, 2, \cdots\),则\(b_n\)满足
\begin{align}\label{proposition4.8-1.4}
c_nb_n = c_na_n - c_{n - 1}a_{n - 1}, n = 1, 2, \cdots.
\end{align}
并且由\eqref{proposition4.8-1.3}式可知
\[
\lim_{n\rightarrow \infty}b_n = \lim_{n\rightarrow \infty}\frac{c_na_n - c_{n - 1}a_{n - 1}}{c_n} = A.
\]
从而由\eqref{proposition4.8-1.4}式可得
\begin{align*}
a_n = \frac{1}{c_n}(c_na_n - c_0a_0 + c_0a_0)
= \frac{1}{c_n}\left[ \sum_{k = 1}^n{(c_ka_k - c_{k - 1}a_{k - 1})} + c_0a_0 \right]
= \frac{1}{c_n}\sum_{k = 1}^n{c_kb_k} + a_0, n = 0, 1, \cdots.
\end{align*}
这样就完成了对\(a_n\)的强行裂项和强求通项.
\end{enumerate}
\end{proof}



\begin{example}\label{example:4.341111}
设\(\{a_n\}_{n = 0}^{\infty}\)满足\(\lim_{n\rightarrow \infty} (a_n-\lambda a_{n-1})=a,|\lambda |<1,\text{计算}\lim_{n\rightarrow \infty} a_n\).
\end{example}
\begin{note}
\hypertarget{递推条件是极限形式的待定数列法}{{\color{blue}解法二}构造数列$c_n,b_n$的思路}:
待定数列$c_n$且$c_0=1$,由条件可得\(\lim_{n\rightarrow \infty}\frac{c_na_n - \lambda c_na_{n - 1}}{c_n}=a\).希望\(c_{n - 1}=\lambda c_n\),即\(\frac{c_n}{c_{n - 1}}=\frac{1}{\lambda}\),等价于\(c_n = c_0\prod_{k = 1}^n{\frac{c_k}{c_{k - 1}}}=\frac{1}{\lambda ^n}\).该式对\(n = 0\)也成立.于是令\(c_n=\frac{1}{\lambda ^n}\),则由条件可知
\[
a=\lim_{n\rightarrow \infty}\frac{c_na_n - \lambda c_na_{n - 1}}{c_n}=\lim_{n\rightarrow \infty}\frac{c_na_n - c_{n - 1}a_{n - 1}}{c_n}
\]
从而待定\(b_n\),希望\(b_n\)满足\(c_nb_n = c_na_n - c_{n - 1}a_{n - 1}\),即\(\frac{b_n}{\lambda ^n}=\frac{a_n}{\lambda ^n}-\frac{a_{n - 1}}{\lambda ^{n - 1}}=\frac{a_n - \lambda a_{n - 1}}{\lambda ^n}\).
于是令\(b_n = a_n - \lambda a_{n - 1}\),则由条件可知\(\lim_{n\rightarrow \infty}b_n=\lim_{n\rightarrow \infty}(a_n - \lambda a_{n - 1}) = a\),\(c_nb_n = c_na_n - c_{n - 1}a_{n-1}\).因此
\begin{align*}
a_n=&\frac{1}{c_n}\left( c_na_n-c_0a_0+c_0a_0 \right) =\frac{1}{c_n}\left[ \sum_{k=1}^n{\left( c_ka_k-c_{k-1}a_{k-1} \right)}+c_0a_0 \right] 
\\
&=\frac{1}{c_n}\left( \sum_{k=1}^n{c_kb_k}+c_0a_0 \right) =\lambda ^n\sum_{k=1}^n{\frac{b_k}{\lambda ^k}}+a_0\lambda ^n.
\end{align*}
这样就完成了对\(a_n\)的强行裂项和强求通项.后续计算极限的方法与{\color{blue}解法一}相同.
\end{note}
\begin{solution}
{\color{blue}解法一(通过观察直接构造出裂项数列$b_n$):}
当\(\lambda = 0\)问题时显然的,当\(\lambda\neq0\),记\(b_n=a_n-\lambda a_{n - 1},n = 1,2,\cdots\),我们有
\[
\frac{b_n}{\lambda^n}=\frac{a_n-\lambda a_{n - 1}}{\lambda^n}=\frac{a_n}{\lambda^n}-\frac{a_{n - 1}}{\lambda^{n - 1}},n = 1,2,\cdots.
\]
上式对\(n = 1,2,\cdots\)求和得
\begin{align}\label{equation:4.341.1}
a_n=\lambda^n\sum_{k = 1}^{n}\frac{b_k}{\lambda^k}+a_0\lambda^n,n = 1,2,\cdots.
\end{align}
由于\(|\lambda|<1\),我们知道\(\lim_{n\rightarrow\infty}a_0\lambda^n = 0\). 于是由Stolz定理,可知
当\(\lambda> 0\)时,我们有
\[
\lim_{n\rightarrow\infty}\frac{\sum\limits_{k = 1}^{n}\frac{b_k}{\lambda^k}}{\frac{1}{\lambda^n}}=\lim_{n\rightarrow\infty}\frac{\frac{b_{n + 1}}{\lambda^{n+1}}}{\frac{1}{\lambda^{n + 1}}-\frac{1}{\lambda^n}}=\lim_{n\rightarrow\infty}\frac{b_{n + 1}}{1 - \lambda}=\frac{a}{1 - \lambda}.
\]
当\(\lambda<0\)时(此时分母$\frac{1}{\lambda^n}$不再严格单调递增趋于$+\infty$,不满足Stolz定理条件.但是不难发现其奇偶子列严格单调递增趋于$+\infty$满足Stolz定理条件,因此需要分奇偶子列讨论),对于\eqref{equation:4.341.1}式的偶子列,由Stolz定理,我们有
\[
\lim_{n\rightarrow\infty}\frac{\sum\limits_{k = 1}^{2n}\frac{b_k}{\lambda^k}}{\frac{1}{\lambda^{2n}}}=\lim_{n\rightarrow\infty}\frac{\sum\limits_{k = 1}^{2n + 2}\frac{b_k}{\lambda^k}-\sum\limits_{k = 1}^{2n}\frac{b_k}{\lambda^k}}{\frac{1}{\lambda^{2n+2}}-\frac{1}{\lambda^{2n}}}=\lim_{n\rightarrow\infty}\frac{\frac{b_{2n+2}}{\lambda^{2n+2}}+\frac{b_{2n+1}}{\lambda^{2n+1}}}{\frac{1}{\lambda^{2n+2}}-\frac{1}{\lambda^{2n}}}=\lim_{n\rightarrow\infty}\frac{b_{2n+2}+\lambda b_{2n+1}}{1 - \lambda^2}=\frac{a+\lambda a}{1 - \lambda^2}=\frac{a}{1 - \lambda}.
\]
对于\eqref{equation:4.341.1}式的奇子列,由Stolz定理,我们有
\[
\lim_{n\rightarrow\infty}\frac{\sum\limits_{k = 1}^{2n - 1}\frac{b_k}{\lambda^k}}{(\frac{1}{\lambda})^{2n - 1}}=\frac{1}{\lambda}\lim_{n\rightarrow\infty}\frac{\sum\limits_{k = 1}^{2n - 1}\frac{b_k}{\lambda^k}}{\frac{1}{\lambda^{2n}}}=\frac{1}{\lambda}\lim_{n\rightarrow\infty}\frac{\sum\limits_{k = 1}^{2n}\frac{b_k}{\lambda^k}}{\frac{1}{\lambda^{2n}}}-\frac{1}{\lambda}\lim_{n\rightarrow\infty}\frac{\frac{b_{2n}}{\lambda^{2n}}}{\frac{1}{\lambda^{2n}}}\xlongequal{\text{因为偶子列的极限}}\frac{a}{\lambda(1 - \lambda)}-\frac{a}{\lambda}=\frac{a}{1 - \lambda}.
\]
因此无论如何我们都有\(\lim_{n\rightarrow\infty}a_n=\frac{a}{1 - \lambda}\).

{\color{blue}解法二(强求通项和强行裂项的标准解法):}
\hyperlink{递推条件是极限形式的待定数列法}{令\(c_n = \frac{1}{\lambda ^n}\),$n=0,1,\cdots$,\(b_n = a_n - \lambda a_{n - 1}\),$n=1,2,\cdots$},则由条件可知\(\lim_{n\rightarrow \infty}b_n=\lim_{n\rightarrow \infty}(a_n - \lambda a_{n - 1}) = a\),\(c_nb_n = c_na_n - c_{n - 1}a_{n-1}\).从而对$\forall n\in \mathbb{N} $,都有
\begin{align}
a_n=&\frac{1}{c_n}\left( c_na_n-c_0a_0+c_0a_0 \right) =\frac{1}{c_n}\left[ \sum_{k=1}^n{\left( c_ka_k-c_{k-1}a_{k-1} \right)}+c_0a_0 \right] 
\nonumber
\\
&=\frac{1}{c_n}\left( \sum_{k=1}^n{c_kb_k}+c_0a_0 \right) =\lambda ^n\sum_{k=1}^n{\frac{b_k}{\lambda ^k}}+a_0\lambda ^n.\label{equation:4.341.2}
\end{align}
由于\(|\lambda|<1\),我们知道\(\lim_{n\rightarrow\infty}a_0\lambda^n = 0\). 于是由Stolz定理,可知
当\(\lambda> 0\)时,我们有
\[
\lim_{n\rightarrow\infty}\frac{\sum\limits_{k = 1}^{n}\frac{b_k}{\lambda^k}}{\frac{1}{\lambda^n}}=\lim_{n\rightarrow\infty}\frac{\frac{b_{n + 1}}{\lambda^{n+1}}}{\frac{1}{\lambda^{n + 1}}-\frac{1}{\lambda^n}}=\lim_{n\rightarrow\infty}\frac{b_{n + 1}}{1 - \lambda}=\frac{a}{1 - \lambda}.
\]
当\(\lambda<0\)时(分母$\frac{1}{\lambda^n}$不再严格单调递增趋于$+\infty$,不满足Stolz定理条件.而我们发现其奇偶子列恰好严格单调递增趋于$+\infty$满足Stolz定理条件,因此需要分奇偶子列讨论),对于\eqref{equation:4.341.2}式的偶子列,由Stolz定理,我们有
\[
\lim_{n\rightarrow\infty}\frac{\sum\limits_{k = 1}^{2n}\frac{b_k}{\lambda^k}}{\frac{1}{\lambda^{2n}}}=\lim_{n\rightarrow\infty}\frac{\sum\limits_{k = 1}^{2n + 2}\frac{b_k}{\lambda^k}-\sum\limits_{k = 1}^{2n}\frac{b_k}{\lambda^k}}{\frac{1}{\lambda^{2n+2}}-\frac{1}{\lambda^{2n}}}=\lim_{n\rightarrow\infty}\frac{\frac{b_{2n+2}}{\lambda^{2n+2}}+\frac{b_{2n+1}}{\lambda^{2n+1}}}{\frac{1}{\lambda^{2n+2}}-\frac{1}{\lambda^{2n}}}=\lim_{n\rightarrow\infty}\frac{b_{2n+2}+\lambda b_{2n+1}}{1 - \lambda^2}=\frac{a+\lambda a}{1 - \lambda^2}=\frac{a}{1 - \lambda}.
\]
对于\eqref{equation:4.341.2}式的奇子列,由Stolz定理,我们有
\[
\lim_{n\rightarrow\infty}\frac{\sum\limits_{k = 1}^{2n - 1}\frac{b_k}{\lambda^k}}{(\frac{1}{\lambda})^{2n - 1}}=\frac{1}{\lambda}\lim_{n\rightarrow\infty}\frac{\sum\limits_{k = 1}^{2n - 1}\frac{b_k}{\lambda^k}}{\frac{1}{\lambda^{2n}}}=\frac{1}{\lambda}\lim_{n\rightarrow\infty}\frac{\sum\limits_{k = 1}^{2n}\frac{b_k}{\lambda^k}}{\frac{1}{\lambda^{2n}}}-\frac{1}{\lambda}\lim_{n\rightarrow\infty}\frac{\frac{b_{2n}}{\lambda^{2n}}}{\frac{1}{\lambda^{2n}}}\xlongequal{\text{因为偶子列的极限}}\frac{a}{\lambda(1 - \lambda)}-\frac{a}{\lambda}=\frac{a}{1 - \lambda}.
\]
因此无论如何我们都有\(\lim_{n\rightarrow\infty}a_n=\frac{a}{1 - \lambda}\).
\end{solution}

\begin{example}\label{example:4.351111}
设\(a_1 = 2,a_n=\frac{1+\frac{1}{n}}{2}a_{n - 1}+\frac{1}{n},n\geq2\),证明:\(\lim_{n\rightarrow\infty}na_n\)存在.
\end{example}
\begin{note}
\hypertarget{递推条件是等式形式的待定数列法}{构造数列$c_n,b_n$的思路}:待定数列$c_n$且$c_1 = 1$,由条件可得$c_na_n = \frac{n + 1}{2n}c_{n}a_{n - 1}+\frac{c_n}{n}$,希望$c_n$满足\(\frac{n + 1}{2n}c_n = c_{n - 1},n = 2,3,\cdots\),即$\frac{c_n}{c_{n-1}}=\frac{n+1}{n}$,等价于\(c_n=\prod_{k = 2}^{n}\frac{2k}{k + 1}=\frac{(2n)!!}{(n + 1)!}\)且该式对\(n = 1\)也成立.于是令$c_n=\frac{(2n)!!}{(n + 1)!}$,则由条件可知
\begin{align*}
c_na_n=\frac{n+1}{2n}c_{n-1}+\frac{c_n}{n}=c_{n-1}a_{n-1}+\frac{c_n}{n},n=2,3,\cdots .
\end{align*}
于是待定$b_n$,希望\(b_n\)满足\(c_nb_n = c_na_n - c_{n - 1}a_{n - 1}\),即$c_nb_n=\frac{1}{n}$.
从而令$b_n=\frac{1}{n}$,则\(c_nb_n = c_na_n - c_{n - 1}a_{n - 1}\).因此对$\forall m\in \mathbb{N}_+ $,都有
\begin{align*}
a_m&=\frac{1}{c_m}\left( c_ma_m-c_1a_1+c_1a_1 \right) =\frac{1}{c_m}\left[ \sum_{n=1}^m{\left( c_na_n-c_{n-1}a_{n-1} \right)}+c_1a_1 \right] 
\\
&=\frac{1}{c_m}\left( \sum_{n=1}^m{c_nb_n}+c_1a_1 \right) =\frac{(m+1)!}{(2m)!!}\left( \sum_{n=1}^m{\frac{(2n)!!}{n(n+1)!}}+2 \right) .
\end{align*}
这样就完成了对\(a_n\)的强行裂项和强求通项.后续再利用Stolz定理计算极限即可.
\end{note}
\begin{proof}
\hyperlink{递推条件是等式形式的待定数列法}{令$c_n=\frac{(2n)!!}{(n + 1)!}$,$b_n=\frac{1}{n}$,$n=1,2,\cdots$},则由条件可知\(c_nb_n = c_na_n - c_{n - 1}a_{n-1}\).从而对$\forall m\in \mathbb{N} $,都有
\[
c_ma_m - 2 = c_ma_m - c_1a_1=\sum_{n = 2}^{m}(c_na_n - c_{n - 1}a_{n - 1})=\sum_{n = 1}^{m}\frac{c_n}{n}=\sum_{n = 1}^{m}\frac{(2n)!!}{n(n + 1)!},
\]
从而
\[
a_m=\frac{1}{c_m}\left(2+\sum_{n = 1}^{m}\frac{(2n)!!}{n(n + 1)!}\right)=\frac{(m + 1)!}{(2m)!!}\left(2+\sum_{n = 1}^{m}\frac{(2n)!!}{n(n + 1)!}\right).
\]
再由Stolz定理可得
\begin{align*}
\lim_{m\rightarrow\infty}ma_m&=\lim_{m\rightarrow\infty}m\frac{(m + 1)!}{(2m)!!}\left(2+\sum_{n = 1}^{m}\frac{(2n)!!}{n(n + 1)!}\right)
=\lim_{m\rightarrow\infty}\frac{2+\sum\limits_{n = 1}^{m}\frac{(2n)!!}{n(n + 1)!}}{\frac{(2m)!!}{m(m + 1)!}}\\
&\xlongequal{\text{Stolz定理}}\lim_{m\rightarrow\infty}\frac{\frac{(2m + 2)!!}{(m + 1)(m + 2)!}}{\frac{(2m+2)!!}{(m + 1)(m + 2)!}-\frac{(2m)!!}{m(m + 1)!}}
=\lim_{m\rightarrow\infty}\frac{\frac{2m + 2}{m + 1}}{\frac{2m+2}{m + 1}-\frac{m + 2}{m}}=\frac{2}{2 - 1}=2.
\end{align*}
\end{proof}

\begin{example}
设\(\lim_{n\rightarrow\infty}b_n = b\)存在,令
\[
a_{n + 1}=b_n-\frac{na_n}{2n + 1},
\]
证明\(\lim_{n\rightarrow\infty}a_n\)存在.
\end{example}
\begin{note}
\hypertarget{构造数列c_n的思路}{构造数列$c_n$的思路:}
令\(c_1 = 1\),待定\(\{ c_n \}_{n = 1}^{+\infty}\),由条件可知\(c_{n + 1}a_{n + 1} = c_{n + 1}b_n - \frac{n}{2n + 1}c_{n + 1}a_n\).
希望\(-\frac{n}{2n + 1}c_{n + 1} = c_n\),则\(\frac{c_{n + 1}}{c_n} = -\frac{2n + 1}{n}\),从而
\[
c_n = \prod_{k = 1}^{n - 1}{\frac{c_{k + 1}}{c_k}} = \prod_{k = 1}^{n - 1}{\left( -\frac{2k + 1}{k} \right)} = (-1)^{n - 1}\frac{(2n - 1)!!}{(n - 1)!}
\]
该式对\(n = 1\)也成立.
因此令\(c_n = (-1)^{n - 1}\frac{(2n - 1)!!}{(n - 1)!}\),则由条件可知
\[
c_{n + 1}a_{n + 1} = c_{n + 1}b_n + c_na_n \Rightarrow c_{n + 1}a_{n + 1} - c_na_n = c_{n + 1}b_n
\]
从而
\[
a_n = \frac{1}{c_n}\left[ \sum_{k = 2}^n{(c_ka_k - c_{k - 1}a_{k - 1})} + c_1a_1 \right] = \frac{1}{c_n}\left[ \sum_{k = 2}^n{c_kb_{k - 1}} + c_1a_1 \right]
\]
这样就完成了对\(a_n\)的强行裂项和强求通项.
\end{note}
\begin{remark}
\hypertarget{example4.44解法的思路分析}{计算$\underset{n\rightarrow \infty}{\lim}a_n$的思路分析:}
如果此时我们将\eqref{example4.44-1.1}中的$\frac{\left( 2n+1 \right) !!}{n!}$看作分母,将$\left( -1 \right) ^n$放到分子上,那么由\hyperref[theorem:Wallis公式]{Wallis公式}可知分母严格单调递增趋于$+\infty$,此时$a_n$满足Stolz定理条件.但是使用一次Stolz定理后我们并不能直接得到结果,并且此时$(-1)^n$仍未消去.因此我们不采用这种处理方式.

如果此时我们将\eqref{example4.44-1.1}中的$\frac{\left( -1 \right) ^n\left( 2n+1 \right) !!}{n!}$看作分母,则由于$\left( -1 \right) ^n$的振荡性,导致这个分母不再严格单调递增趋于$+\infty$,不满足Stolz定理条件.但是不难发现其奇偶子列严格单调递增趋于$+\infty$满足Stolz定理条件,因此我们可以分奇偶子列进行讨论.
\end{remark}
\begin{proof}
\hyperlink{构造数列c_n的思路}{令\(c_n = (-1)^{n - 1}\frac{(2n - 1)!!}{(n - 1)!},n=1,2,\cdots\)},则由条件可知
\[
c_{n + 1}a_{n + 1}=c_{n + 1}b_n-\frac{n}{2n + 1}c_{n + 1}a_n=c_{n + 1}b_n + c_na_n, \quad \forall n\in \mathbb{N}_+.
\]
从而\(c_{n + 1}a_{n + 1}-c_na_n=c_{n + 1}b_n, \quad \forall n\in \mathbb{N}_+\).于是
\begin{align}
a_{n+1}&=\frac{1}{c_{n+1}}\left[ \sum_{k=1}^n{\left( c_{k+1}a_{k+1}-c_ka_k \right)}+c_1a_1 \right] =\frac{1}{c_{n+1}}\left[ \sum_{k=1}^n{c_{k+1}b_k}+c_1a_1 \right] 
\nonumber
\\
&=\frac{1}{c_{n+1}}\left[ \sum_{k=1}^n{c_{k+1}b_k}+a_1 \right] =\frac{\left( -1 \right) ^nn!}{\left( 2n+1 \right) !!}\left[ \sum_{k=1}^n{\left( -1 \right) ^k\frac{\left( 2k+1 \right) !!}{k!}b_k}+a_1 \right] ,n\in \mathbb{N} _+. \label{example4.44-1.1}
\end{align}
\hyperlink{example4.44解法的思路分析}{下面计算$\underset{n\rightarrow \infty}{\lim}a_n$.}

由\hyperref[theorem:Wallis公式]{Wallis公式}可知
\[
\frac{(2n)!!}{(2n - 1)!!}\sim \sqrt{\pi n},  n\rightarrow \infty.
\]
从而我们有
\begin{align}\label{example4.44-1.2}
\frac{n!}{(2n + 1)!!}&=\frac{n!}{(2n + 1)(2n - 1)!!}
=\frac{(2n)!!}{(2n + 1)2^n(2n - 1)!!}
\sim \frac{\sqrt{\pi n}}{n2^{n + 1}}
=\frac{\sqrt{\pi}}{2^{n + 1}\sqrt{n}},n\rightarrow \infty.  
\end{align}
于是由\eqref{example4.44-1.1}\eqref{example4.44-1.2}式以及Stolz定理和\(\lim_{n\rightarrow \infty} b_n = b\)可知,一方面,考虑\(\{ a_n \}\)的奇子列,我们有
\begin{align}
\lim_{n\rightarrow \infty} a_{2n+1}&=\lim_{n\rightarrow \infty} \frac{\left( -1 \right) ^{2n}\left( 2n \right) !}{\left( 4n+1 \right) !!}\left[ \sum\limits_{k=1}^{2n}{\left( -1 \right) ^k\frac{\left( 2k+1 \right) !!}{k!}b_k}+a_1 \right] =\lim_{n\rightarrow \infty} \frac{\sqrt{\pi}\left[ \sum\limits_{k=1}^n{\frac{\left( 4k+1 \right) !!}{\left( 2k \right) !}b_{2k}}-\sum\limits_{k=1}^n{\frac{\left( 4k-1 \right) !!}{\left( 2k-1 \right) !}b_{2k-1}}+a_1 \right]}{2^{2n+1}\sqrt{2n}}
\nonumber
\\
&=\lim_{n\rightarrow \infty} \frac{\sqrt{\pi}\left[ \sum\limits_{k=1}^n{\frac{\left( 4k+1 \right) !!}{\left( 2k \right) !}b_{2k}}-\sum\limits_{k=1}^n{\frac{\left( 4k-1 \right) !!}{\left( 2k-1 \right) !}b_{2k-1}} \right]}{2^{2n+1}\sqrt{2n}}\xlongequal{\text{Stolz定理}}\lim_{n\rightarrow \infty} \frac{\sqrt{\pi}\left[ \frac{\left( 4n+1 \right) !!}{\left( 2n \right) !}b_{2n}-\frac{\left( 4n-1 \right) !!}{\left( 2n-1 \right) !}b_{2n-1} \right]}{2^{2n+1}\sqrt{2n}-2^{2n-1}\sqrt{2n-2}}
\nonumber
\\
&=\frac{\sqrt{\pi}}{\sqrt{2}}\lim_{n\rightarrow \infty} \frac{\frac{\left( 4n-1 \right) !!}{\left( 2n-1 \right) !}\left( \frac{4n+1}{2n}b_{2n}-b_{2n-1} \right)}{2^{2n+1}\sqrt{n}-2^{2n-1}\sqrt{n-1}}=\frac{1}{\sqrt{2}}\lim_{n\rightarrow \infty} \frac{2^{2n}\sqrt{2n-1}\left( \frac{4n+1}{2n}b_{2n}-b_{2n-1} \right)}{2^{2n+1}\sqrt{n}-2^{2n-1}\sqrt{n-1}}
\nonumber
\\
&=\frac{2}{\sqrt{2}}\lim_{n\rightarrow \infty} \frac{\sqrt{2n-1}\left( \frac{4n+1}{2n}b_{2n}-b_{2n-1} \right)}{4\sqrt{n}-\sqrt{n-1}}=\frac{2}{\sqrt{2}}\lim_{n\rightarrow \infty} \frac{\sqrt{2n-1}}{4\sqrt{n}-\sqrt{n-1}}\cdot \lim_{n\rightarrow \infty} \left( \frac{4n+1}{2n}b_{2n}-b_{2n-1} \right) 
\nonumber
\\
&=\frac{2}{\sqrt{2}}\lim_{n\rightarrow \infty} \frac{\sqrt{2-\frac{1}{n}}}{4-\sqrt{1-\frac{1}{n}}}\cdot \left( 2b-b \right) =\frac{2}{\sqrt{2}}\cdot \frac{\sqrt{2}}{3}\cdot b=\frac{2}{3}b.\label{example4.44-1.3}
\end{align}
另一方面,考虑\(\{ a_n \}\)的偶子列,我们有
\begin{align}
\lim_{n\rightarrow \infty} a_{2n}&=\lim_{n\rightarrow \infty} \frac{\left( -1 \right) ^{2n-1}\left( 2n-1 \right) !}{\left( 4n-1 \right) !!}\left[ \sum\limits_{k=1}^{2n-1}{\left( -1 \right) ^k\frac{\left( 2k+1 \right) !!}{k!}b_k}+a_1 \right] =-\lim_{n\rightarrow \infty} \frac{\sqrt{\pi}\left[ \sum\limits_{k=1}^{n-1}{\frac{\left( 4k+1 \right) !!}{\left( 2k \right) !}b_{2k}}-\sum\limits_{k=1}^n{\frac{\left( 4k-1 \right) !!}{\left( 2k-1 \right) !}b_{2k-1}}+a_1 \right]}{2^{2n}\sqrt{2n-1}}
\nonumber
\\
&=-\lim_{n\rightarrow \infty} \frac{\sqrt{\pi}\left[ \sum\limits_{k=1}^{n-1}{\frac{\left( 4k+1 \right) !!}{\left( 2k \right) !}b_{2k}}-\sum\limits_{k=1}^n{\frac{\left( 4k-1 \right) !!}{\left( 2k-1 \right) !}b_{2k-1}} \right]}{2^{2n}\sqrt{2n-1}}\xlongequal{\text{Stolz定理}}-\lim_{n\rightarrow \infty} \frac{\sqrt{\pi}\left[ \frac{\left( 4n-3 \right) !!}{\left( 2n-2 \right) !}b_{2n-2}-\frac{\left( 4n-1 \right) !!}{\left( 2n-1 \right) !}b_{2n-1} \right]}{2^{2n}\sqrt{2n-1}-2^{2n-2}\sqrt{2n-3}}
\nonumber
\\
&=-\sqrt{\pi}\lim_{n\rightarrow \infty} \frac{\frac{\left( 4n-3 \right) !!}{\left( 2n-2 \right) !}\left( b_{2n-2}-\frac{4n-1}{2n-1}b_{2n-1} \right)}{2^{2n}\sqrt{2n-1}-2^{2n-2}\sqrt{2n-3}}=-\lim_{n\rightarrow \infty} \frac{2^{2n-1}\sqrt{2n-2}\left( b_{2n-2}-\frac{4n-1}{2n-1}b_{2n-1} \right)}{2^{2n}\sqrt{2n-1}-2^{2n-2}\sqrt{2n-3}}\nonumber
\\
&=-2\lim_{n\rightarrow \infty} \frac{\sqrt{2n-2}\left( b_{2n-2}-\frac{4n-1}{2n-1}b_{2n-1} \right)}{4\sqrt{2n-1}-\sqrt{2n-3}}=-2\lim_{n\rightarrow \infty} \frac{\sqrt{2n-2}}{4\sqrt{2n-1}-\sqrt{2n-3}}\cdot \lim_{n\rightarrow \infty} \left( b_{2n-2}-\frac{4n-1}{2n-1}b_{2n-1} \right) 
\nonumber
\\
&=-2\lim_{n\rightarrow \infty} \frac{\sqrt{2-\frac{2}{n}}}{4\sqrt{2-\frac{1}{n}}-\sqrt{2-\frac{3}{n}}}=-2\cdot \frac{\sqrt{2}}{3\sqrt{2}}\cdot \left( -b \right) =\frac{2}{3}b.
\label{example4.44-1.4}
\end{align}
故由\eqref{example4.44-1.3}\eqref{example4.44-1.4}式,再结合\hyperref[proposition:子列极限命题]{子列极限命题(b)}可知
\[
\lim_{n\rightarrow \infty} a_n=\lim_{n\rightarrow \infty} a_{2n}=\lim_{n\rightarrow \infty} a_{2n + 1}=\frac{2}{3}b.
\]
\end{proof}

\begin{example}
设\(a_n,b_n>0,a_1 = b_1 = 1,b_n=a_nb_{n - 1}-2,n\geq2\)且\(b_n\)有界,求\(\lim_{n\rightarrow\infty}\sum_{k = 1}^{n}\frac{1}{a_1a_2\cdots a_k}\).
\end{example}
\begin{note}
\hypertarget{example4.45构造数列的思路}{构造数列$c_n$的思路:}观察已知的数列递推条件: \(b_n = a_nb_{n - 1} - 2\),可知我们只能对\(b_n\)进行强行裂项和强求通项.于是令\(c_1 = 1\),待定\(\{ c_n \}_{n = 1}^{+\infty}\),则由条件可知\(c_nb_n = a_nc_nb_{n - 1} - 2c_n, n\geq 2\).
希望\(a_nc_n = c_{n - 1}\),则\(\frac{c_n}{c_{n - 1}} = \frac{1}{a_n}\),从而\(c_n = \prod_{k = 2}^n{\frac{1}{a_k}} = \prod_{k = 1}^n{\frac{1}{a_k}}\).该式对\(n = 1\)也成立.
因此,令\(c_n = \prod_{k = 1}^n{\frac{1}{a_k}}\),则由条件可知
\[
c_nb_n = a_nc_nb_{n - 1} - 2c_n = c_{n - 1}b_{n - 1} - 2c_n, n\geq 2.
\]
于是
\[
c_nb_n - c_{n - 1}b_{n - 1} = -2c_n, n\geq 2.
\]
故
\[
b_{n + 1} = \frac{1}{c_{n + 1}}\left[ \sum_{k = 1}^n{(c_{k + 1}b_{k + 1} - c_kb_k)} + c_1b_1 \right] = \frac{1}{c_n}\left( 1 - 2\sum_{k = 1}^n{c_k} \right).
\]
这样就完成了对\(b_n\)的强行裂项和强求通项,而我们发现\(\sum_{k = 1}^n{c_k} = \sum_{k = 1}^n{\frac{1}{a_1a_2\cdots a_k}}\)恰好就是题目要求的数列极限.
\end{note}
\begin{proof}
\hyperlink{example4.45构造数列的思路}{令\(c_n = \prod_{k = 1}^n{\frac{1}{a_k}}\),则由条件可知\(c_n > 0\)},且
\[
c_nb_n = a_nc_nb_{n - 1} - 2c_n = c_{n - 1}b_{n - 1} - 2c_n, n\geq 2.
\]
于是
\[
c_nb_n - c_{n - 1}b_{n - 1} = -2c_n, n\geq 2.
\]
故
\[
b_{n + 1} = \frac{1}{c_{n + 1}}\left[ \sum_{k = 1}^n{(c_{k + 1}b_{k + 1} - c_kb_k)} + c_1b_1 \right] = \frac{1}{c_n}\left( 1 - 2\sum_{k = 1}^n{c_{k + 1}} \right). \forall n\in \mathbb{N}_+.
\]
由此可得
\begin{align}\label{example4.45-1.1}
\sum_{k = 1}^n{\frac{1}{a_1a_2\cdots a_k}} = \sum_{k = 1}^n{c_k} = 1 + \sum_{k = 1}^n{c_{k + 1}} = 1 + \frac{1 - b_{n + 1}c_n}{2} = \frac{3}{2} - \frac{c_nb_{n + 1}}{2}, \forall n\in \mathbb{N}_+.  
\end{align}
由于\(a_n, b_n, c_n > 0\),再结合\eqref{example4.45-1.1}式,可知\(\sum_{k = 1}^n{\frac{1}{a_1a_2\cdots a_k}}\)单调递增且\(\sum_{k = 1}^n{\frac{1}{a_1a_2\cdots a_k}} = \frac{3}{2} - \frac{c_nb_{n + 1}}{2} \leq \frac{3}{2}\),因此\(\lim_{n\rightarrow \infty} \sum_{k = 1}^n{\frac{1}{a_1a_2\cdots a_k}}\)一定存在.故\(\lim_{n\rightarrow \infty} \frac{1}{a_1a_2\cdots a_n} = \lim_{n\rightarrow \infty} c_n = 0\).从而再结合\eqref{example4.45-1.1}式和\(b_n\)有界可得
\[
\lim_{n\rightarrow \infty} \sum_{k = 1}^n{\frac{1}{a_1a_2\cdots a_k}} = \lim_{n\rightarrow \infty} \left( \frac{3}{2} - \frac{c_nb_{n + 1}}{2} \right) = \frac{3}{2}.
\]
\end{proof}


\subsection{递推数列综合问题}

再次回顾\hyperref[proposition:数列收敛的级数与累乘形式]{命题\ref{proposition:数列收敛的级数与累乘形式}}的想法.这个想法再解决递推数列问题中也很常用.

\begin{example}
设\(a_n,b_n\geq0\)且\(a_{n + 1}<a_n + b_n\),同时\(\sum_{n = 1}^{\infty}b_n\)收敛,证明:\(a_n\)也收敛.
\end{example}
\begin{remark}
\hypertarget{example4.58不妨设的原因}{不妨设 \(m_k > n_k\) 的原因:}由假设 \(a_n\) 不收敛可知,存在 \(\delta > 0\),对 \(\forall N > 0\),都存在 \(m \in \mathbb{N}\),使得 \(\vert a_m - A \vert \geqslant \delta\)。从而
\begin{align*}
&\text{取 }N = n_1 > 0\text{,则存在 }m_1 \in \mathbb{N}\text{,使得 }\vert a_{m_1} - A \vert \geqslant \delta.\\
&\text{取 }N = n_2 > 0\text{,则存在 }m_2 \in \mathbb{N}\text{,使得 }\vert a_{m_2} - A \vert \geqslant \delta.\\
&\cdots\cdots\\
&\text{取 }N = n_k > 0\text{,则存在 }m_k \in \mathbb{N}\text{,使得 }\vert a_{m_k} - A \vert \geqslant \delta.\\
&\cdots\cdots
\end{align*}
这样就得到了一个子列 \(\{ a_{m_k} \}\) 满足对 \(\forall n \in \mathbb{N}_+\),都有 \(m_k > n_k\) 且 \(\vert a_{m_k} - A \vert \geqslant \delta\)。
\end{remark}
\begin{proof}
由 \(a_{n + 1} < a_n + b_n\) 可得
\begin{align}
a_n = a_1 + \sum_{i = 1}^{n - 1} (a_{i + 1} - a_i) < a_1 + \sum_{i = 1}^{n - 1} b_i, \forall n \geqslant 2. \label{example4.58-1.1}
\end{align}
又 \(\sum_{n = 1}^{\infty} b_n\) 收敛,故对 \(\forall n \in \mathbb{N}\),有 \(\sum_{i = 1}^n b_i\) 有界。再结合 \eqref{example4.58-1.1} 式可知,\(a_n\) 也有界。由聚点定理可知,存在一个收敛子列 \(\{ a_{n_k} \}\),设 \(\lim_{k \to \infty} a_{n_k} = A < \infty\)。

(反证)假设 \(a_n\) 不收敛,则存在 \(\delta > 0\) 和一个子列 \(\{ a_{m_k} \}\),使得
\begin{align*}
\vert a_{m_k} - A \vert \geqslant \delta, \forall n \in \mathbb{N}_+. 
\end{align*}
\hyperlink{example4.58不妨设的原因}{不妨设 \(m_k > n_k, \forall n \in \mathbb{N}_+\)}。此时分两种情况讨论。

(i) 如果有无穷多个 \(k\),使得 \(a_{m_k} \geqslant A + \delta\) 成立。再结合条件可得,对这些 \(k\),都有
\begin{align}
a_{m_k} - a_{n_k} &= \sum_{i = n_k}^{m_k - 1} (a_{i + 1} - a_i) < \sum_{i = n_k}^{m_k - 1} b_i, \label{example4.58-1.3}\\
a_{m_k} - a_{n_k} &= (a_{m_k} - A) + (A - a_{n_k}) \geqslant \delta + (A - a_{n_k}). \label{example4.58-1.4}
\end{align}
又因为 \(\sum_{n = 1}^{\infty} b_n\) 收敛和 \(\lim_{k \to \infty} a_{n_k} = A\),所以
\[
\lim_{k \to \infty} \sum_{i = n_k}^{m_k - 1} b_i = \lim_{k \to \infty} (A - a_{n_k}) = 0.
\]
于是对 \eqref{example4.58-1.3}\eqref{example4.58-1.4} 式两边同时令 \(k \to \infty\),得到
\[
0 < \delta \leqslant \lim_{k \to \infty} (a_{m_k} - a_{n_k}) \leqslant \lim_{k \to \infty} \sum_{i = n_k}^{m_k - 1} b_i = 0.
\]
上述不等式矛盾。

(ii) 如果有无穷多个 \(k\),使得 \(a_{m_k} \leqslant A - \delta\) 成立。取 \(\{ a_{n_k} \}\) 的一个子列 \(\{ a_{t_k} \}\),使得 \(t_k > m_k, \forall n \in \mathbb{N}_+\),则 \(\lim_{k \to \infty} a_{t_k} = \lim_{k \to \infty} a_{n_k} = A\)。
再结合条件可得,对这些 \(k\),都有
\begin{align}
a_{t_k} - a_{m_k} &= \sum_{i = m_k}^{t_k - 1} (a_{i + 1} - a_i) < \sum_{i = m_k}^{t_k - 1} b_i, \label{example4.58-1.5}\\
a_{t_k} - a_{m_k} &= (a_{t_k} - A) + (A - a_{m_k}) \geqslant (a_{t_k} - A) + \delta. \label{example4.58-1.6}
\end{align}
又因为 \(\sum_{n = 1}^{\infty} b_n\) 收敛和 \(\lim_{k \to \infty} a_{t_k} = A\),所以
\[
\lim_{k \to \infty} \sum_{i = m_k}^{t_k - 1} b_i = \lim_{k \to \infty} (a_{t_k} - A) = 0.
\]
于是对 \eqref{example4.58-1.5}\eqref{example4.58-1.6}式两边同时令 \(k \to \infty\),得到
\[
0 < \delta \leqslant \lim_{k \to \infty} (a_{t_k} - a_{m_k}) \leqslant \lim_{k \to \infty} \sum_{i = m_k}^{t_k - 1} b_i = 0.
\]
上述不等式矛盾。结论得证。
\end{proof}

\begin{example}
设\(a_{n + 1}=\ln\left(\frac{e^{a_n}-1}{a_n}\right)\),\(a_1 = 1\),证明:极限\(\lim_{n\rightarrow\infty}2^na_n\)存在。
\end{example}
\begin{note}
本题证明的思路分析:

注意到递推函数 \(f(x)=\ln\left(\frac{e^x - 1}{x}\right)\) 在 \((0, +\infty)\) 上单调递增,且 \(a_1 = 1>0\)。因此直接利用单调分析法归纳证明 \(\{ a_n \}\) 单调有界且 \(a_n\in(0, 1]\)。进而得到 \(\lim_{n \to \infty} a_n = 0\)。再利用\hyperref[proposition:数列收敛的级数与累乘形式]{命题\ref{proposition:数列收敛的级数与累乘形式}}将 \(2^n a_n\) 转化为级数的形式。因为递推函数与 \(\ln\) 有关,所以我们考虑作差转换,即
\[
2^{n + 1} a_{n + 1} = \sum_{k = 1}^n (2^{k + 1} a_{k + 1} - 2^k a_k) = \sum_{k = 1}^n 2^{k + 1}\left(\ln\left(\frac{e^{a_k} - 1}{a_k}\right) - \frac{1}{2} a_k\right).
\]
因此我们只需证明级数 \(\sum_{k = 1}^n 2^{k + 1}\left(\ln\left(\frac{e^{a_k} - 1}{a_k}\right) - \frac{1}{2} a_k\right)\) 收敛即可。考虑其通项 \(2^{n + 1}\left(\ln\left(\frac{e^{a_n} - 1}{a_n}\right) - \frac{1}{2} a_n\right)\)。由于 \(\lim_{n \to \infty} a_n = 0\),因此利用 Taylor 公式可得
\begin{align*}
\ln\left(\frac{e^{a_n} - 1}{a_n}\right) - \frac{1}{2} a_n &= \ln\frac{a_n+\frac{a_n^2}{2}+\frac{a_n^3}{6}+o(a_n^3)}{a_n} - \frac{1}{2} a_n
= \ln\left(1 + \frac{a_n}{2}+\frac{a_n^2}{6}+o(a_n^2)\right) - \frac{1}{2} a_n\\
&= \frac{a_n}{2}+\frac{a_n^2}{6}+o(a_n^2) - \left(\frac{a_n}{2}+\frac{a_n^2}{6}+o(a_n^2)\right)^2 + o(a_n^2) - \frac{1}{2} a_n
= \frac{a_n^2}{24}, n \to \infty.
\end{align*}
故当 \(n\) 充分大时,我们有
\[
2^{n + 1}\left(\ln\left(\frac{e^{a_n} - 1}{a_n}\right) - \frac{1}{2} a_n\right) = \frac{1}{24} 2^{n + 1} a_n^2.
\]
于是我们只须证级数 \(\sum_{k = 1}^n \frac{1}{24} 2^{n + 1} a_n^2\) 收敛即可。因此我们需要找到一个收敛级数 \(\sum_{k = 1}^n c_n\),使得 \(2^{n + 1} a_n^2\) 被这个收敛级数的通项 \(c_n\) 控制,即当 \(n\) 充分大时,有
\[
2^{n + 1} a_n^2 \leq c_n.
\]
又题目要证 \(\lim_{n \to \infty} 2^n a_n\) 存在,说明 \(\lim_{n \to \infty} 2^n a_n\) 一定存在,从而一定有
\begin{align}
a_n \sim \frac{c}{2^n}, n \to \infty, \label{example4.58note-1.1}
\end{align}
其中 \(c\) 为常数。虽然无法直接证明 \eqref{example4.58note-1.1} 式,但是 \eqref{example4.58note-1.1}式给我们提供了一种找 \(c_n\) 的想法。\eqref{example4.58note-1.1}式表明 \(a_n\) 与几何级数的通项近似,于是一定存在 \(\lambda \in (0, 1)\),使得 \(a_n \approx \frac{c}{2^n} \leq c_0 \lambda ^n, n \to \infty\)。其中 \(c_0\) 为常数。从而
\[
2^{n + 1} a_n^2 \leq c_0^2 2^{n + 1} \lambda ^{2n} = c_1 (2\lambda ^2)^n, n \to \infty.
\]
故我们只需要保证 \(\sum_{n = 1}^{\infty} (2\lambda ^2)^n\) 收敛,就能由级数的比较判别法推出 \(\sum_{k = 1}^n \frac{1}{24} 2^{n + 1} a_n^2\) 收敛。因此我们待定 \(\lambda \in (0, 1)\),使得 \(\sum_{n = 1}^{\infty} (2\lambda ^2)^n\) 恰好就是一个几何级数。于是 \(2\lambda ^2 < 1 \Rightarrow \lambda < \frac{\sqrt{2}}{2}\)。故我们只要找到一个恰当的 \(\lambda \in \left(0, \frac{\sqrt{2}}{2}\right)\),使得
\begin{align}
a_n \leq c_0 \lambda ^n, n \to \infty. \label{example4.58note-0.1}
\end{align}
其中 \(c_0\) 为常数,即可。我们需要与已知的递推条件联系起来,因此考虑
\begin{align}
a_{n + 1} \leq c_0 \lambda ^{n + 1}, n \to \infty. \label{example4.58note-0.2}
\end{align}
又 \(a_n \in (0, 1]\),显然将\eqref{example4.58note-0.1}与\eqref{example4.58note-0.2} 式作商得到
\[
a_n\leqslant c_0\lambda ^n,n\rightarrow \infty \Leftrightarrow \frac{a_{n+1}}{a_n}\le \lambda ,n\rightarrow \infty \Leftrightarrow \frac{f(a_n)}{a_n}\le \lambda ,n\rightarrow \infty 
\]
又 \(\lim_{n \to \infty} a_n = 0\),故上式等价于
\[
\lim_{x \to 0^+} \frac{f(x)}{x} \leq \lambda \Leftrightarrow \lim_{x \to 0^+} \frac{\ln\left(\frac{e^x - 1}{x}\right)}{x} \leq \lambda 
\]
注意到 \(\lim_{x \to 0^+} \frac{\ln\left(\frac{e^x - 1}{x}\right)}{x} = \lim_{x \to 0^+} \frac{\frac{x}{2}+o(x)}{x} = \frac{1}{2}\),所以任取 \(\lambda \in \left(\frac{1}{2}, \frac{\sqrt{2}}{2}\right)\) 即可。最后根据上述思路严谨地书写证明即可。

(注:也可以利用 \(f(x)\) 的凸性去找 \(\lambda \in \left(\frac{1}{2}, \frac{\sqrt{2}}{2}\right)\),见下述证明过程。 )
\end{note}
\begin{proof}
令 \(f(x)=\ln\left(\frac{e^x - 1}{x}\right)\),注意到对 \(\forall x > 0\),有
\begin{align*}
& \quad f(x) < x \Leftrightarrow \ln\left(\frac{e^x - 1}{x}\right) < x \Leftrightarrow \frac{e^x - 1}{x} < e^x \Leftrightarrow \ln x > 1 - \frac{1}{x}\\
&\Leftrightarrow \ln\frac{1}{t} > 1 - t,\text{ 其中 }t = \frac{1}{x} > 0 \Leftrightarrow \ln t < t - 1,\text{ 其中 }t = \frac{1}{x} > 0.
\end{align*}
上式最后一个不等式显然成立。因此
\begin{align}
f(x)=\ln\left(\frac{e^x - 1}{x}\right) < x,\forall x > 0. \label{1example4.59-1.1}  
\end{align}
由 \(e^x - 1 > x,\forall x \in \mathbb{R}\) 可知
\begin{align}
f(x)=\ln\left(\frac{e^x - 1}{x}\right) > \ln 1 = 0,\forall x > 0. \label{1example4.59-1.2}
\end{align}
从而由 \eqref{1example4.59-1.1}\eqref{1example4.59-1.2} 式及 \(a_1 = 1\),归纳可得 \(\forall n \in \mathbb{N}_+\),都有
\[
a_{n + 1} = f(a_n) < a_n,\quad a_{n + 1} = f(a_n) > 0.
\]
故数列 \(\{ a_n \}\) 单调递减且有下界 \(0\)。于是 \(a_n \in (0, 1]\),并且由单调有界原理可知 \(\lim_{n \to \infty} a_n = A \in [0, 1]\)。
对 \(a_{n + 1} = \ln\left(\frac{e^{a_n} - 1}{a_n}\right)\) 两边同时令 \(n \to \infty\),得到
\[
A = \ln\left(\frac{e^A - 1}{A}\right) \Leftrightarrow Ae^A = e^A - 1 \Leftrightarrow (1 - A)e^A = 1.
\]
显然上述方程只有唯一解:\(A = 0\)。故 \(\lim_{n \to \infty} a_n = 0\)。下面证明 \(\lim_{n \to \infty} 2^n a_n\) 存在。
由 \(a_{n + 1} = \ln\left(\frac{e^{a_n} - 1}{a_n}\right)\) 可得,对 \(\forall n \in \mathbb{N}_+\),都有
\[
2^{n + 1} a_{n + 1} - 2^n a_n = 2^{n + 1}\left[\ln\left(\frac{e^{a_n} - 1}{a_n}\right) - \frac{1}{2} a_n\right].
\]
从而
\[
2^{n + 1} a_{n + 1} = 2a_1 + \sum_{k = 1}^n 2^{k + 1}\left(\ln\left(\frac{e^{a_k} - 1}{a_k}\right) - \frac{1}{2} a_k\right) = 2 + \sum_{k = 1}^n 2^{k + 1}\left(\ln\left(\frac{e^{a_k} - 1}{a_k}\right) - \frac{1}{2} a_k\right),\forall n \in \mathbb{N}_+.
\]
故要证 \(\lim_{n \to \infty} 2^n a_n\) 存在,即证 \(\sum_{k = 1}^n 2^{k + 1}\left(\ln\left(\frac{e^{a_k} - 1}{a_k}\right) - \frac{1}{2} a_k\right)\) 收敛。
注意到
\begin{align*}
\lim_{x\rightarrow 0} \frac{\ln \frac{e^x-1}{x}-\frac{1}{2}x}{x^2}&=\lim_{x\rightarrow 0} \frac{\ln \frac{x+\frac{x^2}{2}+\frac{x^3}{6}+o(x^3)}{x}-\frac{1}{2}x}{x^2}=\lim_{x\rightarrow 0} \frac{\ln \left( 1+\frac{x}{2}+\frac{x^2}{6}+o(x^2) \right) -\frac{1}{2}x}{x^2}
\\
&=\lim_{x\rightarrow 0} \frac{\frac{x}{2}+\frac{x^2}{6}+o(x^2)-\left( \frac{x}{2}+\frac{x^2}{6}+o(x^2) \right) ^2+o(x^2)-\frac{1}{2}x}{x^2}
\\
&=\lim_{x\rightarrow 0} \frac{\frac{x^2}{24}+o(x^2)}{x^2}=\frac{1}{24}<1,
\end{align*}
再结合 \(\lim_{n \to \infty} a_n = 0\) 可得,\(\lim_{n \to \infty} \frac{\ln\left(\frac{e^{a_n} - 1}{a_n}\right) - \frac{1}{2} a_n}{a_n^2} = \frac{1}{24} < 1\)。故存在 \(N \in \mathbb{N}_+\),使得
\begin{align}
\ln\left(\frac{e^{a_n} - 1}{a_n}\right) - \frac{1}{2} a_n < a_n^2,\forall n > N. \label{example4.59-3.1}
\end{align}
由 \(f(x)=\ln\left(\frac{e^x - 1}{x}\right)\) 可知,\(f'(x)=\frac{e^x}{e^x - 1} - \frac{1}{x}\),\(f''(x)=\frac{1}{x^2} - \frac{e^x}{(e^x - 1)^2}\)。注意到对 \(\forall x \in (0, 1]\),都有
\begin{align*}
& \quad \quad f''(x) > 0 \Leftrightarrow \frac{1}{x^2} > \frac{e^x}{(e^x - 1)^2}\\
&\Leftrightarrow \frac{1}{\ln^2 t} > \frac{t}{(t - 1)^2},\text{ 其中 }t = e^x > 1\\
&\Leftrightarrow \ln t < \frac{t - 1}{\sqrt{t}} = \sqrt{t} - \frac{1}{\sqrt{t}},\text{ 其中 }t = e^x > 1
\end{align*}
而上式最后一个不等式显然成立\hyperref[proposition:关于ln的常用不等式2]{(见关于ln的常用不等式\ref{proposition:关于ln的常用不等式2})}。故 \(f''(x) > 0,\forall x \in (0, 1]\)。故 \(f\) 在 \((0, 1]\) 上是下凸函数。从而由下凸函数的性质(切割线放缩)可得,\(\forall x \in (0, 1]\),固定 \(x\),对 \(\forall y \in (0, x)\),都有
\begin{align}
f'(y) x \leq f(x) \leq [f(1) - f(y)] x = [\ln(e - 1) - f(y)] x. \label{example4.59-2.1}
\end{align}
注意到
\begin{align*}
&\lim_{y \to 0^+} f(y) = \lim_{y \to 0^+} \ln\left(\frac{e^y - 1}{y}\right) = \ln\left(\lim_{y \to 0^+} \frac{e^y - 1}{y}\right) = \ln 1 = 0,\\
&\lim_{y \to 0^+} f'(y) = \lim_{y \to 0^+} \left(\frac{e^y}{e^y - 1} - \frac{1}{y}\right) = \lim_{y \to 0^+} \frac{e^y(y - 1) + 1}{y(e^y - 1)}\\
&= \lim_{y \to 0^+} \frac{(1 + y + \frac{1}{2} y^2 + o(y^2))(y - 1) + 1}{y^2} = \lim_{y \to 0^+} \frac{\frac{1}{2} y^2 + o(y^2)}{y^2} = \frac{1}{2}.
\end{align*}
于是令 \eqref{example4.59-2.1} 式 \(y \to 0^+\),得到
\[
\frac{1}{2} x = \lim_{y \to 0^+} f'(y) x \leq f(x) \leq [\ln(e - 1) - \lim_{y \to 0^+} f(y)] x = x \ln(e - 1),\forall x \in (0, 1].
\]
又 \(a_n \in (0, 1]\),故
\[
\frac{1}{2} a_n \leq a_{n + 1} = f(a_n) \leq \ln(e - 1) a_n,\forall n \in \mathbb{N}_+.
\]
从而
\begin{align}
\frac{1}{2} \leq \frac{a_{n + 1}}{a_n} \leq \ln(e - 1) < \frac{\sqrt{2}}{2},\forall n \in \mathbb{N}_+. \label{example4.59-4.1} 
\end{align}
因此
\begin{align}
a_n = a_1 \prod_{k = 1}^{n - 1} \frac{a_{k + 1}}{a_k} \leq [\ln(e - 1)]^{n - 1},\forall n \in \mathbb{N}_+. \label{example4.59-3.2}
\end{align}
于是结合 \eqref{example4.59-3.1}\eqref{example4.59-3.2} 式可得对 \(\forall n > N\),我们有
\[
2^{n + 1}\left(\ln\left(\frac{e^{a_n} - 1}{a_n}\right) - \frac{1}{2} a_n\right) < 2^{n + 1} a_n^2 \leq 2^{n + 1}[\ln(e - 1)]^{2n - 2} = \frac{2}{\ln^2(e - 1)}[2\ln^2(e - 1)]^n.
\]
又由 \eqref{example4.59-4.1}式可知,\(2\ln^2(e - 1) < 2\cdot\left(\frac{\sqrt{2}}{2}\right)^2 = 1\)。故 \(\sum_{k = 1}^n \frac{2}{\ln^2(e - 1)}[2\ln^2(e - 1)]^k\) 收敛。从而由比较判别法知,\(\sum_{k = 1}^n 2^{k + 1}\left(\ln\left(\frac{e^{a_k} - 1}{a_k}\right) - \frac{1}{2} a_k\right)\) 也收敛。结论得证。
\end{proof}


\begin{example}[$\,\,$Herschfeld判别法]\label{example:Herschfeld判别法}
设\(p > 1\),令\(a_n=\sqrt[p]{b_1 + \sqrt[p]{b_2+\cdots+\sqrt[p]{b_n}}}\),\(b_n>0\),证明:数列\(a_n\)收敛等价于数列\(\frac{\ln b_n}{p^n}\)有界。
\end{example}
\begin{remark}
这个很抽象的结果叫做Herschfeld判别法,但是证明起来只需要单调有界。
\end{remark}
\begin{proof}
由条件可知 \(a_2 > a_1\),假设 \(a_n > a_{n - 1}\),则由 \(b_n > 0\) 可得
\[
a_{n + 1}=\sqrt[p]{b_1+\sqrt[p]{b_2+\cdots +\sqrt[p]{b_n+\sqrt[p]{b_{n + 1}}}}}>\sqrt[p]{b_1+\sqrt[p]{b_2+\cdots +\sqrt[p]{b_n}}}=a_n.
\]
由数学归纳法可知 \(\{ a_n \}\) 单调递增。

若 \(a_n\) 收敛,则由单调有界定理可知,\(a_n\) 有上界。即存在 \(M > 0\),使得 \(a_n < M,\forall n\in \mathbb{N}_+\)。从而
\[
M > a_n=\sqrt[p]{b_1+\sqrt[p]{b_2+\cdots +\sqrt[p]{b_n}}}>\sqrt[p]{0+\sqrt[p]{0+\cdots +\sqrt[p]{b_n}}}=b_n^{\frac{1}{p^n}},\forall n\in \mathbb{N}_+.
\]
故
\[
\frac{\ln b_n}{p^n}=\ln b_n^{\frac{1}{p^n}}<\ln M,\forall n\in \mathbb{N}_+.
\]
即 \(\frac{\ln b_n}{p^n}\) 有界。

若 \(\frac{\ln b_n}{p^n}\) 有界,则存在 \(M_1 > 0\),使得
\begin{align}
\frac{\ln b_n}{p^n}<M_1,\forall n\in \mathbb{N}_+. \label{1example4.61-1.1}
\end{align}
记 \(C = e^{M_1}\),则由 \eqref{1example4.61-1.1}式可得
\[
b_n<e^{M_1p^n}=C^{p^n},\forall n\in \mathbb{N}_+.
\]
从而
\begin{align}
a_n=\sqrt[p]{b_1+\sqrt[p]{b_2+\cdots +\sqrt[p]{b_n}}}<\sqrt[p]{C^p+\sqrt[p]{C^{p^2}+\cdots +\sqrt[p]{C^{p^n}}}}=C\sqrt[p]{1+\sqrt[p]{1+\cdots +\sqrt[p]{1}}}. \label{1example4.61-2.1}
\end{align}
考虑数列 \(x_1 = 1,x_{n + 1}=\sqrt[p]{1 + x_n},\forall n\in \mathbb{N}_+\)。显然 \(x_n > 0\),记 \(f(x)=\sqrt[p]{1 + x}\),则
\[
f'(x)=\frac{1}{p}(1 + x)^{\frac{1}{p}-1}<\frac{1}{p}<1,\forall x > 0.
\]
而显然 \(f(x)=x\) 有唯一解 \(a > 1\),从而由 Lagrange 中值定理可得 \(\forall n\in \mathbb{N}_+\),存在 \(\xi_n\in(\min\{ x_n,a \},\max\{ x_n,a \})\),使得
\[
|x_{n + 1}-a|=|f(x_n)-f(a)|=f'(\xi_n)|x_n - a|<\frac{1}{p}|x_n - a|.
\]
于是
\[
|x_{n + 1}-a|<\frac{1}{p}|x_n - a|<\frac{1}{p^2}|x_{n - 1}-a|<\cdots <\frac{1}{p^n}|x_1 - a|\rightarrow 0,n\rightarrow\infty.
\]
故 \(x_n\) 收敛到 \(a\),因此 \(x_n\) 有界,即存在 \(K\),使得 \(x_n < K,\forall n\in \mathbb{N}_+\)。于是结合 \eqref{1example4.61-2.1} 可得
\[
a_n=C\sqrt[p]{1+\sqrt[p]{1+\cdots +\sqrt[p]{1}}}=Cx_n<CK,\forall n\in \mathbb{N}_+.
\]
即 \(a_n\) 有界,又因为 \(\{ a_n \}\) 单调递增,所以由单调有界定理可知,\(a_n\) 收敛。
\end{proof}






\begin{lemma}[有界数列差分极限为0则其闭包一定是闭区间]\label{lemma:有界数列差分极限为0则其闭包一定是闭区间}
有界数列 \( x_n \) 如果满足$\lim_{n \to \infty} (x_{n+1} - x_n) = 0$,
则 \( x_n \) 的全体聚点构成一个闭区间(且这个闭区间的端点就是数列的上下极限).
\end{lemma}
\begin{note}
先根据条件直观地画图分析,分析出大致的思路后,再考虑严谨地书写证明.
\end{note}
\begin{proof}
当数列\(x_n\)收敛时,\(x_n\)的聚点集为单点集,结论显然成立。

当数列\(x_n\)不收敛时,因为数列\(x_n\)有界,所以可设\(\limsup_{n\rightarrow \infty}x_n = L<\infty\),\(\liminf_{n\rightarrow \infty}x_n = l<L\)。假设\(\exists A\in (l,L)\),使得\(A\)不是\(x_n\)的极限点。则\(\exists \delta \in \left( 0,\min \{ L - A,A - l \} \right)\),使得区间\((A - \delta,A + \delta)\subseteq (l,L)\)中只包含了数列\(x_n\)中有限项。因此存在\(N_1\in \mathbb{N}\),使得当\(n > N_1\)时,有\(\vert x_n - A\vert\geqslant \delta\)。即
\begin{align}
\text{当}n>N_1\text{时},\text{要么}x_n\geqslant A+\delta ,\text{要么}x_n\leqslant A-\delta .\label{lemma4.1-1.1}
\end{align}
由\(\lim_{n\rightarrow \infty}(x_{n + 1} - x_n) = 0\)可知,存在\(N_2\in \mathbb{N}\),使得
\begin{align}
\vert x_{n + 1} - x_n\vert<\delta,\forall n > N_2. \label{lemma4.1-1.2}
\end{align}
取\(N = \max \{ N_1,N_2 \}\)。由\(\limsup_{n\rightarrow \infty}x_n = L\)和\(\liminf_{n\rightarrow \infty}x_n = l\)可知,对\(\forall \varepsilon \in \left( 0,\min \{ L - A - \delta,A - l - \delta,\frac{L - l}{2} \} \right)\),存在子列\(\{ x_{n_k} \}\),\(\{ x_{m_k} \}\),使得对\(\forall k\in \mathbb{N}_+\cap (N,+\infty)\),都有
\[
x_{m_k}<l + \varepsilon \leqslant A - \delta <A + \delta \leqslant L - \varepsilon <x_{n_k}.
\]
任取\(K\in \mathbb{N}_+\cap (N,+\infty)\),则\(x_{m_K}<l + \varepsilon \leqslant A - \delta <A + \delta \leqslant L - \varepsilon <x_{n_K}\)。不妨设\(n_K>m_K\),则\(n_K>m_K\geqslant K>N\)。现在考虑\(x_{m_K},x_{m_K + 1},\cdots,x_{n_K - 1},x_{n_K}\)这些项。将其中最后一个小于等于\(A - \delta\)的项记为\(x_s\),显然\(n_K - 1\geqslant s\geqslant m_K\geqslant K>N\),进而\(s + 1\in [m_K + 1,n_K]\),于是\(x_{s + 1}>A - \delta\)。又因为\(s + 1\geqslant m_K + 1>K>N\),所以结合\eqref{lemma4.1-1.1}可知,\(x_{s + 1}\geqslant A + \delta\)。因此\(\vert x_{s + 1} - x_s\vert\geqslant 2\delta\)。这与\eqref{lemma4.1-1.2}式矛盾! 因此\(x_n\)的全体聚点构成一个闭区间$[l,L]$.
\end{proof}

\begin{example}
设连续函数 \( f(x) : [0,1] \to [0,1], x_1 \in [0,1], x_{n+1} = f(x_n) \),证明:数列 \(\{x_n\}\) 收敛的充要条件是
\[
\lim_{n \to \infty} (x_{n+1} - x_n) = 0.
\]
\end{example}
\begin{note}
先根据条件直观地画图分析,分析出大致的思路后,再考虑严谨地书写证明.
\end{note}
\begin{remark}
$x_{n_k} \rightarrow A \Rightarrow x_{n_{k+1}} \rightarrow A$,$k \rightarrow \infty$。但是 $x_{n_k} \rightarrow A \nRightarrow x_{n_k + 1} \rightarrow A$,$k \rightarrow \infty$。
\end{remark}
\begin{proof}
{\heiti 必要性:}如果 \( x_{n} \) 收敛,则显然$\lim _{n \to \infty } ( x_{n+1}-x_{n})=0$.

{\heiti 充分性:}假设数列$x_n$不收敛.设$\underset{n\rightarrow \infty}{\overline{\lim }}x_n=L,\underset{n\rightarrow \infty}{\underline{\lim }}x_n=l$,则由条件可知$l<L$且$\left[ l,L \right] \subseteq \left[ 0,1 \right]$.从而由\hyperref[lemma:有界数列差分极限为0则其闭包一定是闭区间]{引理\ref{lemma:有界数列差分极限为0则其闭包一定是闭区间}}可知,数列\(x_n\)的全体聚点构成一个闭区间$[l,L]$.于是\(\forall A\in [l,L]\),则存在一个子列\(\{ x_{n_k} \}\),使得\(\lim_{k\rightarrow \infty}x_{n_k}=A\)。由\(\lim_{n\rightarrow \infty}(x_{n + 1} - x_n) = 0\)可知,\(\lim_{k\rightarrow \infty}x_{n_k + 1}=\lim_{k\rightarrow \infty}x_{n_k}=A\)。根据\(x_{n + 1}=f(x_n)\)可得\(x_{n_k + 1}=f(x_{n_k})\),令\(k\rightarrow \infty\),再结合\(f\in C[0,1]\)可得
\begin{align}
A = f(A),\forall A\in [l,L]. \label{example5.58-1.1}
\end{align}
因此取\(A = \frac{l + L}{2}\),这也是\(x_n\)的一个极限点,从而令\(\varepsilon_0=\frac{L - l}{2}\)存在\(N\in \mathbb{N}\),使得
\[
l = A - \varepsilon_0<x_N<A + \varepsilon_0 = L.
\]
即\(x_N\in [l,L]\)。于是由\(x_{n + 1}=f(x_n)\)及\eqref{example5.58-1.1}式可得\(x_{N + 1}=f(x_N)=x_N\)。从而归纳可得\(x_n=x_N,\forall n\in \mathbb{N}_+\cap (N,+\infty)\)。显然此时\(x_n\)收敛到\(x_N\),这与\(x_n\)不收敛矛盾! 故数列\(x_n\)收敛。 
\end{proof}

\begin{example}
设\(d\)为正整数,给定\(1 < a\leq\frac{d + 2}{d + 1},x_0,x_1,\cdots,x_d\in(0,a - 1)\),令\(x_{n + 1}=x_n(a - x_{n - d}),n\geq d\),证明:\(\lim_{n\rightarrow\infty}x_n\)存在并求极限。
\end{example}
\begin{proof}
证明见lsz(2024-2025)数学类讲义的不动点与蛛网图方法部分.
\end{proof}



\begin{example}
设\(x_n\)满足当\(|i - j|\leq2\)时总有\(|x_i - x_j|\geq|x_{i + 1}-x_{j + 1}|\),证明:\(\lim_{n\rightarrow\infty}\frac{x_n}{n}\)存在。
\end{example}
\begin{remark}
仅凭\(|x_{n + 1}-x_n|\)单调递减无法保证极限存在,只能说明数列\(\frac{x_n}{n}\)有界,但是完全有可能其聚点集合是一个闭区间,所以\(|x_{n + 2}-x_n|\)的递减性是必要的。
本题其实画图来看走势很直观.
\end{remark}
\begin{proof}
条件等价于\(|x_{n + 1}-x_n|\),\(|x_{n + 2}-x_n|\)这两个数列都是单调递减的,显然非负,所以它们的极限都存在。

(i)如果\(\lim_{n\rightarrow\infty}|x_{n + 1}-x_n| = 0\),则由stolz公式显然\(\lim_{n\rightarrow\infty}\frac{x_n}{n}=\lim_{n\rightarrow\infty}x_{n + 1}-x_n = 0\)。

(ii)如果\(\lim_{n\rightarrow\infty}|x_{n + 2}-x_n| = 0\),则奇偶两个子列分别都有
\[
\lim_{n\rightarrow\infty}\frac{x_{2n}}{2n}=\lim_{n\rightarrow\infty}x_{2n + 2}-x_{2n}=0, 
\lim_{n\rightarrow\infty}\frac{x_{2n + 1}}{2n + 1}=\lim_{n\rightarrow\infty}x_{2n + 1}-x_{2n - 1}=0
\]
所以\(\lim_{n\rightarrow\infty}\frac{x_n}{n}=0\),因此下面只需讨论\(|x_{n + 1}-x_n|\),\(|x_{n + 2}-x_n|\)的极限都非零的情况。

不妨设\(|x_{n + 1}-x_n|\)单调递减趋于\(1\)(如果极限不是\(1\)而是别的正数,考虑\(kx_n\)这样的数列就可以了),由于非负递减数列\(|x_{n + 2}-x_n|\)的极限非零,故存在\(\delta\in(0,1)\)使得\(|x_{n + 2}-x_n|\geq\delta\)恒成立。

(i)如果\(x_n\)是最终单调的,也就是说存在\(N\)使得\(n > N\)时\(x_{n + 1}-x_n\)恒正或者恒负,则\(\lim_{n\rightarrow\infty}x_{n + 1}-x_n = 1\)或者\(\lim_{n\rightarrow\infty}x_{n + 1}-x_n=-1\),再用stolz公式可知极限\(\lim_{n\rightarrow\infty}\frac{x_n}{n}\)存在。

(ii)如果\(x_n\)不是最终单调的,因为\(\lim_{n\rightarrow\infty}|x_{n + 1}-x_n| = 1\),所以存在\(N\)使得\(n > N\)时恒有\(|x_{n + 1}-x_n|\in\left[1,1+\frac{\delta}{2}\right]\),并且\(n > N\)时\(x_n\)不是单调的,故存在\(n > N\)使得以下两种情况之一成立

\((a)\):\(1\leq x_{n + 1}-x_n\leq1+\frac{\delta}{2},1\leq x_{n + 1}-x_{n + 2}\leq1+\frac{\delta}{2}\Rightarrow|x_{n + 2}-x_n|\leq\frac{\delta}{2}\)。

\((b)\):\(1\leq x_n - x_{n + 1}\leq1+\frac{\delta}{2},1\leq x_{n + 2}-x_{n + 1}\leq1+\frac{\delta}{2}\Rightarrow|x_{n + 2}-x_n|\leq\frac{\delta}{2}\)。

可见不论哪种情况成立,都会与\(|x_{n + 2}-x_n|\geq\delta\)恒成立矛盾,结论得证.
\end{proof}

\begin{example}
设四个正数列\(\{a_n\},\{b_n\},\{c_n\},\{t_n\}\)满足
\[t_n\in(0,1),\sum_{n = 1}^{\infty}t_n=+\infty,\sum_{n = 1}^{\infty}b_n<+\infty,\lim_{n\rightarrow\infty}\frac{a_n}{t_n}=0,x_{n + 1}\leq(1 - t_n)x_n+a_n + b_n\]
证明:\(\lim_{n\rightarrow\infty}x_n = 0\)。
\end{example}
\begin{note}
这类问题直接强求通项即可.
\end{note}
\begin{proof}
根据条件有
\begin{align*}
\frac{x_{n + 1}}{(1 - t_n)\cdots(1 - t_1)}&\leq\frac{x_n}{(1 - t_{n-1})\cdots(1 - t_1)}+\frac{a_n + b_n}{(1 - t_n)\cdots(1 - t_1)}\\
\frac{x_{n + 1}}{(1 - t_n)\cdots(1 - t_1)}&\leq x_1+\sum_{k = 1}^{n}\frac{a_k + b_k}{(1 - t_k)\cdots(1 - t_1)}\\
x_{n + 1}&\leq x_1(1 - t_n)\cdots(1 - t_1)+\sum_{k = 1}^{n}(a_k + b_k)(1 - t_{k + 1})\cdots(1 - t_n)
\end{align*}
换元令\(u_n = 1 - t_n\in(0,1)\),则
\[\ln\prod_{n = 1}^{\infty}u_n=\sum_{n = 1}^{\infty}\ln u_n\leq\sum_{n = 1}^{\infty}(u_n - 1)=-\sum_{n = 1}^{\infty}t_n=-\infty\Rightarrow\prod_{n = 1}^{\infty}u_n = 0\]
代入有
\[x_{n + 1}\leq x_1u_1u_2\cdots u_n+\sum_{k = 1}^{n}a_ku_{k + 1}u_{k + 2}\cdots u_n+\sum_{k = 1}^{n}b_ku_{k + 1}u_{k + 2}\cdots u_n\]
显然\(x_1u_1u_2\cdots u_n\rightarrow0\),于是只需要看后面两项。
对于最后一项,我们待定正整数\(N\leq n\),则有
\[\sum_{k = 1}^{n}b_ku_{k + 1}u_{k + 2}\cdots u_n=\sum_{k = 1}^{N}b_ku_{k + 1}u_{k + 2}\cdots u_n+\sum_{k = N + 1}^{n}b_ku_{k + 1}u_{k + 2}\cdots u_n\]
其中\(\sum_{k = N + 1}^{n}b_ku_{k + 1}u_{k + 2}\cdots u_n\leq\sum_{k = N + 1}^{n}b_k<\sum_{k = N}^{\infty}b_k\),于是对任意\(\varepsilon>0\),可以取充分大的\(N\)使得\(\sum_{k = N + 1}^{n}b_ku_{k + 1}u_{k + 2}\cdots u_n<\varepsilon\),现在\(N\)已经取定,再对前面有限项取极限有
\[\varlimsup_{n\rightarrow\infty}\sum_{k = 1}^{n}b_ku_{k + 1}u_{k + 2}\cdots u_n\leq\sum_{k = 1}^{N}b_k\varlimsup_{n\rightarrow\infty}(u_{k + 1}u_{k + 2}\cdots u_n)+\varepsilon=\varepsilon\]
由此可见最后一项的极限是零,最后来看中间一项,记\(s_n=\frac{a_n}{t_n}=\frac{a_n}{1 - u_n}\rightarrow0\),则对任意\(N\)有
\begin{align*}
&\sum_{k = 1}^{n}a_ku_{k + 1}u_{k + 2}\cdots u_n=\sum_{k = 1}^{n}s_k(1 - u_k)u_{k + 1}u_{k + 2}\cdots u_n\\
&=\sum_{k = 1}^{N}s_k(1 - u_k)u_{k + 1}u_{k + 2}\cdots u_n+\sum_{k = N + 1}^{n}s_k(1 - u_k)u_{k + 1}u_{k + 2}\cdots u_n\\
\sum_{k = N + 1}^{n}s_k(1 - u_k)u_{k + 1}u_{k + 2}\cdots u_n&\leq\sup_{k\geq N}s_k\sum_{k = N + 1}^{n}(1 - u_k)u_{k + 1}u_{k + 2}\cdots u_n\leq\sup_{k\geq N}s_k\\
\lim_{n\rightarrow\infty}\sum_{k = 1}^{n}a_ku_{k + 1}u_{k + 2}\cdots u_n&\leq\lim_{n\rightarrow\infty}\sum_{k = 1}^{N}s_k(1 - u_k)u_{k + 1}u_{k + 2}\cdots u_n+\sup_{k\geq N}s_k=\sup_{k\geq N}s_k
\end{align*}
再令$N\to \infty$,由此可见这一部分的极限也是零,结论得证。
\end{proof}


\end{document}
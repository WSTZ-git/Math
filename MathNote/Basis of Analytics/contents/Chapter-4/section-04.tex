\documentclass[../../main.tex]{subfiles}
\graphicspath{{\subfix{../../image/}}} % 指定图片目录,后续可以直接使用图片文件名。

% 例如:
% \begin{figure}[h]
% \centering
% \includegraphics{image-01.01}
% \caption{图片标题}
% \label{fig:image-01.01}
% \end{figure}
% 注意:上述\label{}一定要放在\caption{}之后,否则引用图片序号会只会显示??.

\begin{document}

\section{Abel变换}

\begin{theorem}[Abel变换]\label{theorem:Abel变换}
设\(\{a_n\}_{n = 1}^{N}\),\(\{b_n\}_{n = 1}^{N}\)是数列,则有恒等式
\begin{align*}
\sum\limits_{k = 1}^{N} a_k b_k &= (a_1 - a_2)b_1 + \cdots + (a_{N - 1} - a_N)(b_1 + b_2 + \cdots + b_{N - 1}) + a_N(b_1 + b_2 + \cdots + b_N)
\\
&= \sum\limits_{j = 1}^{N - 1} (a_j - a_{j + 1}) \sum\limits_{i = 1}^{j} b_i + a_N \sum\limits_{i = 1}^{N} b_i.
\end{align*}
\end{theorem}
\begin{note}
\hyperref[theorem:Abel变换]{Abel变换}的证明想法“强行裂项”是一种很重要的思想.
\end{note}
\begin{proof}
为了计算\(\sum\limits_{j = 1}^{N - 1} (a_j - a_{j + 1}) \sum\limits_{i = 1}^{j} b_i + a_N \sum\limits_{i = 1}^{N} b_i\),我们来强行构造裂项,差什么就给他补上去再补回来,即:
\begin{align*}
&\sum\limits_{j = 1}^{N - 1} (a_j - a_{j + 1}) \sum\limits_{i = 1}^{j} b_i + a_N \sum\limits_{i = 1}^{N} b_i = \sum\limits_{j = 1}^{N - 1} \left(a_j \sum\limits_{i = 1}^{j} b_i - a_{j + 1} \sum\limits_{i = 1}^{j} b_i\right) + a_N \sum\limits_{i = 1}^{N} b_i
\\
&= \sum\limits_{j = 1}^{N - 1} \left(a_j \sum\limits_{i =1}^{j} b_i - a_{j + 1} \sum\limits_{i = 1}^{j + 1} b_i\right) + \sum\limits_{j = 1}^{N - 1} \left(a_{j + 1} \sum\limits_{i = 1}^{j + 1} b_i - a_{j + 1} \sum\limits_{i = 1}^{j} b_i\right) + a_N \sum\limits_{i = 1}^{N} b_i
\\
&= a_1 b_1 - a_N \sum\limits_{i = 1}^{N} b_i + \sum\limits_{j = 1}^{N - 1} a_{j + 1} b_{j + 1} + a_N \sum\limits_{i = 1}^{N} b_i
= \sum\limits_{j = 1}^{N} a_j b_j.
\end{align*}
\end{proof}

\begin{proposition}[经典乘积极限结论]\label{proposition:经典乘积极限结论}
设\(a_1 \geqslant a_2 \geqslant \cdots \geqslant a_n \geqslant 0\)且\(\lim_{n \to \infty} a_n = 0\),极限\(\lim_{n \to \infty} \sum\limits_{k = 1}^{n} a_k b_k\)存在.证明
\begin{align*}
\lim_{n \to \infty} (b_1 + b_2 + \cdots + b_n)a_n = 0. 
\end{align*}
\end{proposition}
\begin{note}
为了估计\(\sum\limits_{j = 1}^{n} b_j\),前面的有限项不影响.而要用上极限\(\sum\limits_{n = 1}^{\infty} a_n b_n\)收敛,自然想到\(\sum\limits_{j = 1}^{n} b_j = \sum\limits_{j = 1}^{n} \frac{b_j a_j}{a_j}\)和\hyperref[theorem:Abel变换]{Abel变换}.而\(a_j\)的单调性能用在\hyperref[theorem:Abel变换]{Abel变换}之后去绝对值.
\end{note}
\begin{proof}
不妨设\(a_1 \geqslant a_2 \geqslant \cdots \geqslant a_n > 0\).则由于级数\(\sum\limits_{n = 1}^{\infty} a_n b_n\)收敛,存在\(N \in \mathbb{N}\),使得
\begin{align*}
\left|\sum\limits_{i = N + 1}^{m} a_i b_i\right| \leqslant \varepsilon, \forall m \geqslant N + 1.
\end{align*}
当\(n \geqslant N + 1\),由\hyperref[theorem:Abel变换]{Abel变换},我们有
\begin{align*}
&\left|\sum\limits_{j = N + 1}^{n} b_j\right| = \left|\sum\limits_{j = N + 1}^{n} \frac{a_j b_j}{a_j}\right|
= \left|\sum\limits_{j = N + 1}^{n - 1} \left(\frac{1}{a_j} - \frac{1}{a_{j + 1}}\right) \sum\limits_{i = N + 1}^{j} a_i b_i + \frac{1}{a_n} \sum\limits_{i = N + 1}^{n} a_i b_i\right|
\\
&\leqslant \sum\limits_{j = N + 1}^{n - 1} \left(\left|\frac{1}{a_j} - \frac{1}{a_{j + 1}}\right| \cdot \left|\sum\limits_{i = N + 1}^{j} a_i b_i\right|\right) + \frac{1}{|a_n|} \left|\sum\limits_{i = N + 1}^{n} a_i b_i\right|
\\
&\leqslant \left|\sum\limits_{i = N + 1}^{n} a_i b_i\right|\cdot\sum\limits_{j = N + 1}^{n - 1} \left(\left|\frac{1}{a_j} - \frac{1}{a_{j + 1}}\right| \right) + \frac{1}{|a_n|} \left|\sum\limits_{i = N + 1}^{n} a_i b_i\right|
\\
&\leqslant \varepsilon \left[\sum\limits_{j = N + 1}^{n - 1} \left(\frac{1}{a_{j + 1}} - \frac{1}{a_j}\right) + \frac{1}{a_n}\right]
= \varepsilon \left(\frac{2}{a_n} - \frac{1}{a_{N + 1}}\right).
\end{align*}
因此我们有
\begin{align*}
&\underset{n\rightarrow \infty}{{\varlimsup }} \left|a_n \sum\limits_{j = 1}^{n} b_j\right| \leqslant \underset{n\rightarrow \infty}{{\varlimsup }} \left|a_n \sum\limits_{j = 1}^{N} b_j\right| + \underset{n\rightarrow \infty}{{\varlimsup }} \left|a_n \sum\limits_{j = N + 1}^{n} b_j\right|
\leqslant \underset{n\rightarrow \infty}{{\varlimsup }} \left|a_n \sum\limits_{j = 1}^{N} b_j\right| + \varepsilon \underset{n\rightarrow \infty}{{\varlimsup }} \left(2 - \frac{a_n}{a_{N + 1}}\right)= 2\varepsilon.
\end{align*}
由\(\varepsilon\)任意性即可得
\(\underset{n\rightarrow \infty}{{\varlimsup }} \left|a_n \sum\limits_{j = 1}^{n} b_j\right| = 0\),
于是就证明了
\(\lim_{n \to \infty} (b_1 + b_2 + \cdots + b_n)a_n = 0\).
\end{proof}

\begin{example}
设\(\lim_{n \to \infty} x_n = x\),证明:
\[
\lim_{n \to \infty} \frac{1}{2^n} \sum\limits_{k = 0}^{n} \mathrm{C}_{n}^{k} x_{k} = x.
\]
\end{example}
\begin{note}
可以不妨设$x=0$的原因:
假设当$x=0$时,结论成立,则当$x\ne 0$时,令$y_n=x_n-x$,则$\underset{n\rightarrow +\infty}{\lim}y_n=0$.从而由假设可知
\begin{align*}
0=\underset{n\rightarrow +\infty}{\lim}\frac{1}{2^n}\sum\limits_{k=0}^n{\mathrm{C}_{n}^{k}y_k}=\underset{n\rightarrow +\infty}{\lim}\frac{1}{2^n}\sum\limits_{k=0}^n{\mathrm{C}_{n}^{k}\left( x_k-x \right)}=\underset{n\rightarrow +\infty}{\lim}\frac{1}{2^n}\sum\limits_{k=0}^n{\mathrm{C}_{n}^{k}x_k}-x\underset{n\rightarrow +\infty}{\lim}\frac{1}{2^n}\sum\limits_{k=0}^n{\mathrm{C}_{n}^{k}}=\underset{n\rightarrow +\infty}{\lim}\frac{1}{2^n}\sum\limits_{k=0}^n{\mathrm{C}_{n}^{k}x_k}-x.
\end{align*}
于是$\underset{n\rightarrow +\infty}{\lim}\frac{1}{2^n}\sum\limits_{k=0}^n{\mathrm{C}_{n}^{k}x_k}=x$.
\end{note}
\begin{proof}
不妨设\(x = 0\),则对\(\forall N > 0\),当\(n > N\)时,我们有
\begin{align*}
0 &\leqslant \left|\frac{1}{2^n}\sum\limits_{k = 0}^{n}C_{n}^{k}x_k\right|
= \left|\frac{1}{2^n}\sum\limits_{k = 0}^{N}C_{n}^{k}x_k\right| + \left|\frac{1}{2^n}\sum\limits_{k = N + 1}^{n}C_{n}^{k}x_k\right|\\
&\leqslant \left|\frac{1}{2^n}\sum\limits_{k = 0}^{N}C_{n}^{k}x_k\right| + \frac{1}{2^n}\sum\limits_{k = N + 1}^{n}C_{n}^{k}\sup_{k \geqslant N + 1}|x_k|
\leqslant \left|\frac{1}{2^n}\sum\limits_{k = 0}^{N}C_{n}^{k}x_k\right| + \frac{1}{2^n}\sum\limits_{k = 0}^{n}C_{n}^{k}\sup_{k \geqslant N + 1}|x_k|\\
&= \left|\frac{1}{2^n}\sum\limits_{k = 0}^{N}C_{n}^{k}x_k\right| + \sup_{k \geqslant N + 1}|x_k|
\end{align*}
上式两边同时令\(n \to +\infty\),则结合\(\varlimsup_{n \to +\infty}\left|\frac{1}{2^n}\sum\limits_{k = 0}^{N}C_{n}^{k}x_k\right|\xlongequal{\text{因为分子是关于}n\text{的多项式}}0\),可得

\[
\varlimsup_{n \to +\infty}\left|\frac{1}{2^n}\sum\limits_{k = 0}^{n}C_{n}^{k}x_k\right| \leqslant \sup_{k \geqslant N + 1}|x_k|,  \forall N > 0.
\]

由\(N\)的任意性,上式两边令\(N \to +\infty\),则

\[
\varlimsup_{n \to +\infty}\left|\frac{1}{2^n}\sum\limits_{k = 0}^{n}C_{n}^{k}x_k\right| \leqslant \varlimsup_{N \to +\infty}\sup_{k \geqslant N + 1}|x_k|.
\]

又根据上极限的定义,可知\(\lim_{N \to +\infty}\sup_{k \geqslant N + 1}|x_k| = \varlimsup_{n \to +\infty}|x_n| = \lim_{n \to +\infty}x_n = 0\).

从而

\[
0 \leqslant \varliminf_{n \to +\infty}\left|\frac{1}{2^n}\sum\limits_{k = 0}^{n}C_{n}^{k}x_k\right| \leqslant \varlimsup_{n \to +\infty}\left|\frac{1}{2^n}\sum\limits_{k = 0}^{n}C_{n}^{k}x_k\right| \leqslant 0.
\]

故\(\lim_{n \to +\infty}\frac{1}{2^n}\sum\limits_{k = 0}^{n}C_{n}^{k}x_k = \lim_{n \to +\infty}\left|\frac{1}{2^n}\sum\limits_{k = 0}^{n}C_{n}^{k}x_k\right| = 0\).原命题得证. 
\end{proof}


\end{document}
\documentclass[../../main.tex]{subfiles}
\graphicspath{{\subfix{../../image/}}} % 指定图片目录,后续可以直接使用图片文件名。

% 例如:
% \begin{figure}[H]
% \centering
% \includegraphics[scale=0.4]{图.png}
% \caption{}
% \label{figure:图}
% \end{figure}
% 注意:上述\label{}一定要放在\caption{}之后,否则引用图片序号会只会显示??.

\begin{document}

\section{极限问题综合题}

\begin{example}
设二阶可微函数\(f:[1,+\infty)\to(0,+\infty)\)满足
\[
f''(x)\leqslant0,\lim_{x\rightarrow +\infty}f(x)=+\infty.
\]
求极限
\[
\lim_{s\rightarrow0^{+}}\sum_{n = 1}^{\infty}\frac{(-1)^n}{f^s(n)}.
\]
\end{example}
\begin{note}
本例非常经典,深刻体现了“拉格朗日中值定理”保持阶不变和“和式和积分”转化的思想.
\end{note}
\begin{proof}
由条件$f''(x)\leqslant 0$可知,$f$是上凸函数.而上凸函数只能在递增、递减、先增后减中发生一个.又$\lim_{x\rightarrow +\infty}f(x)=+\infty$,因此$f$一定在$[1,+\infty)$上递增.再结合$f''(x)\leqslant 0$可知$f'\geqslant0$且单调递减.
下面来求极限.

由Lagrange中值定理可得,对\(\forall n\in \mathbb{N}_+\),存在\(\theta_n\in(2n - 1, 2n)\),使得
\begin{align}\label{example4.73-1.0}
\sum_{n = 1}^{\infty}\frac{(-1)^n}{f^s(n)} = \sum_{n = 1}^{\infty}\left[\frac{1}{f^s(2n)} - \frac{1}{f^s(2n - 1)}\right] \xlongequal{\text{Lagrange中值定理}} s\sum_{n = 1}^{\infty}\frac{-f'(\theta_n)}{f^{s + 1}(\theta_n)}.
\end{align}
由于\(\theta_n\in(2n - 1, 2n)\),\(\forall n\in \mathbb{N}_+\)且\(f\geqslant 0\)单调递增,\(f'\geqslant 0\)单调递减,因此
\begin{align}
s\sum_{n = 1}^{\infty}\frac{-f'(2n - 1)}{f^{s + 1}(2n - 1)} \leqslant s\sum_{n = 1}^{\infty}\frac{-f'(\theta_n)}{f^{s + 1}(\theta_n)} \leqslant s\sum_{n = 1}^{\infty}\frac{-f'(2n)}{f^{s + 1}(2n)}. \label{example4.73-1.1}
\end{align}
又因为\(\left[\frac{-f'(x)}{f^{s + 1}(x)}\right]' = \frac{f''(x)f(x) - (s + 1)f'(x)}{f^{s + 2}(x)}\leqslant 0\),所以\(\frac{-f'(x)}{f^{s + 1}(x)}\)单调递减。从而一方面,我们有
\begin{align}
\underset{s\rightarrow 0^+}{\lim}s\sum_{n=1}^{\infty}{\frac{-f'\left( 2n \right)}{f^{s+1}\left( 2n \right)}}&\leqslant -\underset{s\rightarrow 0^+}{\lim}s\sum_{n=1}^{\infty}{\int_{n-1}^n{\frac{f'\left( 2x \right)}{f^{s+1}\left( 2x \right)}\mathrm{d}x}}=-\underset{s\rightarrow 0^+}{\lim}\frac{s}{2}\sum_{n=1}^{\infty}{\int_{2n-1}^{2n}{\frac{f'\left( x \right)}{f^{s+1}\left( x \right)}\mathrm{d}x}}\nonumber
\\
&=-\underset{s\rightarrow 0^+}{\lim}\frac{s}{2}\int_1^{+\infty}{\frac{f'\left( x \right)}{f^{s+1}\left( x \right)}\mathrm{d}x}=-\underset{s\rightarrow 0^+}{\lim}\frac{s}{2}\int_1^{+\infty}{\frac{1}{f^{s+1}\left( x \right)}\mathrm{d}f\left( x \right)}
\nonumber
\\
&=\underset{s\rightarrow 0^+}{\lim}\frac{s}{2}\cdot \frac{1}{sf^s\left( x \right)}\Big|_{1}^{+\infty}=-\underset{s\rightarrow 0^+}{\lim}\left[ \frac{s}{2}\cdot \frac{1}{sf^s\left( 1 \right)} \right] =-\frac{1}{2}. \label{example4.73-1.2}
\end{align}

\begin{align}
\underset{s\rightarrow 0^+}{\lim}s\sum_{n=1}^{\infty}{\frac{-f'\left( 2n \right)}{f^{s+1}\left( 2n \right)}}&\geqslant -\underset{s\rightarrow 0^+}{\lim}s\sum_{n=1}^{\infty}{\int_n^{n+1}{\frac{f'\left( 2x \right)}{f^{s+1}\left( 2x \right)}\mathrm{d}x}}=-\underset{s\rightarrow 0^+}{\lim}\frac{s}{2}\sum_{n=1}^{\infty}{\int_{2n}^{2n+1}{\frac{f'\left( x \right)}{f^{s+1}\left( x \right)}\mathrm{d}x}}
\nonumber
\\
&=-\underset{s\rightarrow 0^+}{\lim}\frac{s}{2}\int_2^{+\infty}{\frac{f'\left( x \right)}{f^{s+1}\left( x \right)}\mathrm{d}x}=-\underset{s\rightarrow 0^+}{\lim}\frac{s}{2}\int_2^{+\infty}{\frac{1}{f^{s+1}\left( x \right)}\mathrm{d}f\left( x \right)}
\nonumber
\\
&=\underset{s\rightarrow 0^+}{\lim}\frac{s}{2}\cdot \frac{1}{sf^s\left( x \right)}\Big|_{2}^{+\infty}=-\underset{s\rightarrow 0^+}{\lim}\left[ \frac{s}{2}\cdot \frac{1}{sf^s\left( 2 \right)} \right] =-\frac{1}{2}. \label{example4.73-1.3}
\end{align}
于是利用\eqref{example4.73-1.2}\eqref{example4.73-1.3}式,由夹逼准则可得
\begin{align}
\lim_{s\rightarrow 0^+}s\sum_{n = 1}^{\infty}\frac{-f'(2n)}{f^{s + 1}(2n)} = -\frac{1}{2}. \label{example4.73-2.1} 
\end{align}
另一方面,我们有
\begin{align}
\underset{s\rightarrow 0^+}{\lim}s\sum_{n=1}^{\infty}{\frac{-f'\left( 2n-1 \right)}{f^{s+1}\left( 2n-1 \right)}}&\leqslant -\underset{s\rightarrow 0^+}{\lim}s\left[ \frac{f'\left( 1 \right)}{f^{s+1}\left( 1 \right)}+\sum_{n=2}^{\infty}{\int_{n-1}^n{\frac{f'\left( 2x-1 \right)}{f^{s+1}\left( 2x-1 \right)}\mathrm{d}x}} \right] =-\underset{s\rightarrow 0^+}{\lim}s\left[ \frac{f'\left( 1 \right)}{f^{s+1}\left( 1 \right)}+\frac{1}{2}\sum_{n=2}^{\infty}{\int_{2n-3}^{2n-1}{\frac{f'\left( x \right)}{f^{s+1}\left( x \right)}\mathrm{d}x}} \right] 
\nonumber
\\
&=-\underset{s\rightarrow 0^+}{\lim}s\left[ \frac{f'\left( 1 \right)}{f^{s+1}\left( 1 \right)}+\frac{1}{2}\int_1^{+\infty}{\frac{f'\left( x \right)}{f^{s+1}\left( x \right)}\mathrm{d}x} \right] =-\underset{s\rightarrow 0^+}{\lim}\frac{s}{2}\int_1^{+\infty}{\frac{f'\left( x \right)}{f^{s+1}\left( x \right)}\mathrm{d}x}
\nonumber
\\
&=-\underset{s\rightarrow 0^+}{\lim}\frac{s}{2}\int_1^{+\infty}{\frac{1}{f^{s+1}\left( x \right)}\mathrm{d}f\left( x \right)}=\underset{s\rightarrow 0^+}{\lim}\frac{s}{2}\cdot \frac{1}{sf^s\left( x \right)}\Big|_{1}^{+\infty}
\nonumber
\\
&=-\underset{s\rightarrow 0^+}{\lim}\left[ \frac{s}{2}\cdot \frac{1}{sf^s\left( 1 \right)} \right] =-\frac{1}{2}.\label{example4.73-1.4}
\end{align}

\begin{align}
\underset{s\rightarrow 0^+}{\lim}s\sum_{n=1}^{\infty}{\frac{-f'\left( 2n-1 \right)}{f^{s+1}\left( 2n-1 \right)}}&\geqslant -\underset{s\rightarrow 0^+}{\lim}\frac{s}{2}\sum_{n=1}^{\infty}{\int_n^{n+1}{\frac{f'\left( x \right)}{f^{s+1}\left( x \right)}\mathrm{d}x}}=-\underset{s\rightarrow 0^+}{\lim}\frac{s}{2}\sum_{n=1}^{\infty}{\int_{2n-1}^{2n+1}{\frac{f'\left( x \right)}{f^{s+1}\left( x \right)}\mathrm{d}x}}
\nonumber
\\
&=-\underset{s\rightarrow 0^+}{\lim}\frac{s}{2}\int_1^{+\infty}{\frac{f'\left( x \right)}{f^{s+1}\left( x \right)}\mathrm{d}x}=-\underset{s\rightarrow 0^+}{\lim}\frac{s}{2}\int_1^{+\infty}{\frac{1}{f^{s+1}\left( x \right)}\mathrm{d}f\left( x \right)}
\nonumber
\\
&=\underset{s\rightarrow 0^+}{\lim}\frac{s}{2}\cdot \frac{1}{sf^s\left( x \right)}\Big|_{1}^{+\infty}=-\underset{s\rightarrow 0^+}{\lim}\left[ \frac{s}{2}\cdot \frac{1}{sf^s\left( 1 \right)} \right] =-\frac{1}{2}.\label{example4.73-1.5}
\end{align}
于是利用\eqref{example4.73-1.4}\eqref{example4.73-1.5}式,由夹逼准则可得
\begin{align}
\lim_{s\rightarrow 0^+}s\sum_{n = 1}^{\infty}\frac{-f'(2n - 1)}{f^{s + 1}(2n - 1)} = -\frac{1}{2}. \label{example4.73-2.2} 
\end{align}
故结合\eqref{example4.73-1.0}\eqref{example4.73-1.1}\eqref{example4.73-2.1}\eqref{example4.73-2.2}式,由夹逼准则可得
\[
\lim_{s\rightarrow 0^+} \sum_{n = 1}^{\infty}\frac{(-1)^n}{f^s(n)} = \lim_{s\rightarrow 0^+} s\sum_{n = 1}^{\infty}\frac{-f'(\theta_n)}{f^{s + 1}(\theta_n)} = -\frac{1}{2}.
\]
\end{proof}

\begin{example}
求极限\(\lim_{n\rightarrow\infty}n\sup_{x\in[0,1]}\sum_{k = 1}^{n - 1}x^{k}(1 - x)^{n - k}\)。
\end{example}
\begin{proof}
根据对称性,不妨设\(x\in\left[0,\frac{1}{2}\right]\),先尝试找到最大值点。
在\(x = 0,\frac{1}{2}\)时代入,很明显对应的极限是零,考虑\(x\in\left(0,\frac{1}{2}\right)\),根据等比数列求和公式有
\[
\sum_{k = 1}^{n - 1}x^{k}(1 - x)^{n - k}=(1 - x)^{n}\sum_{k = 1}^{n - 1}\left(\frac{x}{1 - x}\right)^{k}=\frac{x(1 - x)}{1 - 2x}((1 - x)^{n}-x^{n})
\]
如果\(\delta\in\left(0,\frac{1}{2}\right)\)已经取定,则在区间\(\left[\delta,\frac{1}{2}\right]\)中
\[
n\sum_{k = 1}^{n - 1}x^{k}(1 - x)^{n - k}\leq n\sum_{k = 1}^{n - 1}\left(\frac{1}{2}\right)^{k}(1 - \delta)^{n - k}\leq n(1 - \delta)^{n}\sum_{k = 0}^{\infty}\left(\frac{1}{2(1 - \delta)}\right)^{k}=\frac{n(1 - \delta)^{n}}{1-\frac{1}{2(1 - \delta)}}
\]
右端是指数级趋于零的并且上式不依赖于\(x\),所以函数会一致趋于零。
因此最大值点应该在\(x = 0\)附近,近似的有
\[
n\sum_{k = 1}^{n - 1}x^{k}(1 - x)^{n - k}=\frac{nx(1 - x)}{1 - 2x}((1 - x)^{n}-x^{n})\approx nx(1 - x)^{n}
\]
取\(x = \frac{1}{n}\)显然极限是\(\frac{1}{e}\),我们猜测这就是答案,下面开始证明。
首先取\(x = \frac{1}{n}\)有
\[
\lim_{n\rightarrow\infty}n\sum_{k = 1}^{n - 1}\left(\frac{1}{n}\right)^{k}\left(1-\frac{1}{n}\right)^{n - k}=\lim_{n\rightarrow\infty}\frac{1-\frac{1}{n}}{1-\frac{2}{n}}\left(\left(1-\frac{1}{n}\right)^{n}-\left(\frac{1}{n}\right)^{n}\right)=\frac{1}{e}
\]
由此可知\(\lim_{n\rightarrow\infty}n\sup_{x\in[0,1]}\sum_{k = 1}^{n - 1}x^{k}(1 - x)^{n - k}\geq\frac{1}{e}\),下面估计上极限。
根据对称性,不妨只考虑\(x\in\left[0,\frac{1}{2}\right]\),对任意\(\delta\in\left(0,\frac{1}{2}\right)\)取定,当\(x\in\left[\delta,\frac{1}{2}\right]\)时总有
\[
n\sum_{k = 1}^{n - 1}x^{k}(1 - x)^{n - k}\leq n\sum_{k = 1}^{n - 1}\left(\frac{1}{2}\right)^{k}(1 - \delta)^{n - k}\leq n(1 - \delta)^{n}\sum_{k = 0}^{\infty}\left(\frac{1}{2(1 - \delta)}\right)^{k}=\frac{n(1 - \delta)^{n}}{1-\frac{1}{2(1 - \delta)}}
\]
当\(x\in[0,\delta]\)时,结合均值不等式有
\begin{align*}
n\sum_{k = 1}^{n - 1}x^{k}(1 - x)^{n - k}&=\frac{nx(1 - x)}{1 - 2x}((1 - x)^{n}-x^{n})\approx\frac{nx(1 - x)^{n}}{1 - 2\delta}\leq\frac{\left(1-\frac{1}{n + 1}\right)^{n + 1}}{1 - 2\delta}\leq\frac{1}{e}\frac{1}{1 - 2\delta}
\end{align*}
所以可以取\(n > N\)充分大,使得\(\frac{n(1 - \delta)^{n}}{1-\frac{1}{2(1 - \delta)}}<\frac{1}{e}\),此时便有
\[
n\sup_{x\in[0,1]}\sum_{k = 1}^{n - 1}x^{k}(1 - x)^{n - k}\leq\frac{1}{e}\frac{1}{1 - 2\delta}\Rightarrow\varlimsup_{n\rightarrow\infty}n\sup_{x\in[0,1]}\sum_{k = 1}^{n - 1}x^{k}(1 - x)^{n - k}\leq\frac{1}{e}\frac{1}{1 - 2\delta}
\]
最后,根据\(\delta\)的任意性,可知结论成立。
\end{proof}

\begin{example}
设\(x_n>0\),\(k\)为正整数,证明:\(\varlimsup_{n\rightarrow\infty}\frac{x_1 + x_2+\cdots + x_{n + k}}{x_n}\geq\frac{(k + 1)^{k + 1}}{k^k}\)且常数是最佳的。
\end{example}
\begin{note}
此类问题反证法将会带来一个恒成立的不等式,有很强的效果,所以一般都用反证法,证明的灵感来源于\(k = 1\)时的情况.
\end{note}
\begin{proof}
设\(S_n=x_1 + x_2+\cdots + x_n\),采用反证法,则存在\(N\)使得\(n\geq N\)时恒成立
\[S_{n + k}\leq\lambda(S_n - S_{n - 1}),\lambda\in\left[1,\frac{(k + 1)^{k + 1}}{k^k}\right)\]
显然\(S_n\)是单调递增的,如果\(S_n\)有界,则在不等式两端取极限可知\(S_n\)收敛到零,矛盾,所以\(S_n\)严格单调递增趋于正无穷,因此对任意\(n\geq N\)有\(S_n>S_{n - 1}\)。
如果已经得到了\(S_n>cS_{n - 1}\)对任意\(n\geq N\)恒成立,这里\(c\)是正数,则对任意\(n\geq N\)有
\begin{align*}
S_{n + k}&>cS_{n + k - 1},S_{n + k - 1}>cS_{n + k - 2},\cdots,S_{n + 1}>cS_n\Rightarrow S_{n + k}>c^kS_n\\
0<S_{n + k}-c^kS_n&\leq(\lambda - c^k)S_n-\lambda S_{n - 1}\Rightarrow S_n>\frac{\lambda}{\lambda - c^k}S_{n - 1}
\end{align*}
这样不等式就加强了,记\(c'=\frac{\lambda}{\lambda - c^k}\),我们得到\(S_n>c'S_{n - 1}\)对任意\(n\geq N\)恒成立。
定义数列\(u_n\)为\(u_1 = 1,u_{n + 1}=\frac{\lambda}{\lambda - u_n^k}\),则重复以上过程可知\(S_n>u_mS_{n - 1}\)对任意\(m\)以及\(n\geq N\)都恒成立,所以\(u_m\)这个数列必须是有界的,下面我们就由此导出矛盾。
因为\(u_{n + 1}>u_n\Leftrightarrow(\lambda - u_n^k)u_n<\lambda\Leftrightarrow(\lambda - u_n^k)^ku_n^k<\lambda^k\),由均值不等式有
\[kx^k(\lambda - x^k)^k\leq\left(\frac{k\lambda}{k + 1}\right)^{k + 1}<k\lambda^k\Leftrightarrow\lambda<\frac{(k + 1)^{k + 1}}{k^k}\]
显然成立,所以\(u_m\)单调递增,而如果极限存在,则极限点满足方程\(x=\frac{\lambda}{\lambda - x^k}\Leftrightarrow x(\lambda - x^k)=\lambda\),这与前面均值不等式导出的结果矛盾,所以\(u_m\)单调递增趋于正无穷,又与有界性矛盾。
综上结论得证.
\end{proof}


\begin{example}
设\(x_n>0,x_n\rightarrow0\)且\(\lim_{n\rightarrow\infty}\frac{\ln x_n}{x_1 + x_2+\cdots + x_n}=a<0\),证明:\(\lim_{n\rightarrow\infty}\frac{\ln x_n}{\ln n}=-1\)。
\end{example}
\begin{proof}
不妨设\(a = -1\),否则将\(x_n\)换成\(x_n^k\)即可,取\(k\)将\(a\)变成\(-1\)。

设\(S_n=x_1 + x_2+\cdots + x_n\),则\(S_n>0\)严格单调递增,如果\(S_n\)收敛,则\(\ln x_n\rightarrow-\infty\)与条件矛盾,所以\(S_n\)单调递增趋于正无穷。

因为\(\frac{\ln x_n}{\ln n}=\frac{\ln x_n}{S_n}\frac{S_n}{\ln n}\),\(\frac{\ln x_n}{S_n}\rightarrow -1\),所以等价的只要证明\(\frac{S_n}{\ln n}\rightarrow1\)。

条件为\(\lim_{n\rightarrow\infty}\frac{\ln x_n}{S_n}=-1\),设想作为等式,对应着\(S_n - S_{n - 1}=e^{-S_n}\)是一个隐函数类型的递推式,不方便使用,所以考虑
\[\lim_{n\rightarrow\infty}\frac{\ln x_{n + 1}}{S_n}=\lim_{n\rightarrow\infty}\frac{\ln x_{n + 1}}{S_{n + 1}}\frac{S_{n + 1}}{S_n}=-\lim_{n\rightarrow\infty}\left(1+\frac{x_{n + 1}}{S_n}\right)=-1\]
现在等价的,已知\(S_n\)单调递增趋于无穷且\(\lim_{n\rightarrow\infty}\frac{\ln(S_{n + 1}-S_n)}{S_n}=-1\),要证明\(\lim_{n\rightarrow\infty}\frac{S_n}{\ln n}=1\)。
由极限定义,对任意\(\varepsilon>0\),存在\(N\)使得任意\(n > N\)都有\((-1-\varepsilon)S_n<\ln(S_{n + 1}-S_n)<(-1 + \varepsilon)S_n\)也即
\[\left(\frac{1}{e}-\varepsilon\right)^{S_n}+S_n<S_{n + 1}<\left(\frac{1}{e}+\varepsilon\right)^{S_n}+S_n,\forall n\geq N\]
不妨要求\(S_N>1\),考虑
\[f(x)=\left(\frac{1}{e}+\varepsilon\right)^{x}+x,f'(x)=1+\left(\frac{1}{e}+\varepsilon\right)^{x}\ln\left(\frac{1}{e}+\varepsilon\right)>1-\left(\frac{1}{e}+\varepsilon\right)^{x}>0\]
再定义\(u_N = S_N,u_{n + 1}=\left(\frac{1}{e}+\varepsilon\right)^{u_n}+u_n\),于是若有\(u_n\leq S_n\)则结合单调性可知\(u_{n + 1}=f(u_n)\leq f(S_n)=S_{n + 1}\),这说明\(S_n\leq u_n\)对任意\(n\geq N\)恒成立。
同样考虑
\[g(x)=\left(\frac{1}{e}-\varepsilon\right)^{x}+x,g'(x)=1-\left(\frac{1}{e}-\varepsilon\right)^{x}\ln\left(\frac{1}{e}-\varepsilon\right)\geq1-\left(\frac{1}{e}-\varepsilon\right)\ln\left(\frac{1}{e}-\varepsilon\right)>0\]
再定义\(v_N = S_N,v_{n + 1}=\left(\frac{1}{e}-\varepsilon\right)^{v_n}+v_n\),同样道理\(S_n\geq v_n\)恒成立,于是\(\frac{v_n}{\ln n}\leq\frac{S_n}{\ln n}\leq\frac{u_n}{\ln n},n\geq N\)。

注意\(u_n,v_n\)具备完全一样的形式,所以统一的考虑\(a_1>1,a_{n + 1}=a_n + e^{ca_n}\),其中\(c\)在\(\frac{1}{e}\)附近,显然这个数列是单调递增趋于正无穷的,我们用stolz公式来计算相应的极限,则有
\begin{align*}
\lim_{n\rightarrow\infty}\frac{\ln a_n}{n}&=\lim_{n\rightarrow\infty}\frac{\ln a_{n + 1}-\ln a_n}{1}=\lim_{n\rightarrow\infty}\frac{e^{-ca_n}}{c^{-a_n}-1}=\lim_{n\rightarrow\infty}\frac{1}{c^{-a_{n + 1}}-c^{-a_n}}=\lim_{n\rightarrow\infty}\frac{1}{e^{-ca_n}(c^{-(a_{n + 1}-a_n)}-1)}\\
&=\lim_{n\rightarrow\infty}\frac{e^{ca_n}}{c^{-e^{ca_n}}-1}=\lim_{x\rightarrow+\infty}\frac{e^{cx}}{e^{-x\ln c}-1}=\lim_{x\rightarrow0+}\frac{x}{e^{-x\ln c}-1}=\frac{1}{-\ln c}
\end{align*}
所以
\[\lim_{n\rightarrow\infty}\frac{u_n}{\ln n}=\frac{1}{-\ln(\frac{1}{e}+\varepsilon)}=\frac{1}{1-\ln(1 + e\varepsilon)},\lim_{n\rightarrow\infty}\frac{v_n}{\ln n}=\frac{1}{-\ln(\frac{1}{e}-\varepsilon)}=\frac{1}{1-\ln(1 - e\varepsilon)}\]
这意味着
\[\varlimsup_{n\rightarrow\infty}\frac{S_n}{\ln n}\leq\frac{1}{1-\ln(1 + e\varepsilon)},\varliminf_{n\rightarrow\infty}\frac{S_n}{\ln n}\geq\frac{1}{1-\ln(1 - e\varepsilon)},\forall\varepsilon>0\]
由此可知结论成立。
\end{proof}






\end{document}
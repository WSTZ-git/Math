\documentclass[../../main.tex]{subfiles}
\graphicspath{{\subfix{../../image/}}} % 指定图片目录,后续可以直接使用图片文件名。

% 例如:
% \begin{figure}[H]
% \centering
% \includegraphics[scale=0.4]{图.png}
% \caption{}
% \label{figure:图}
% \end{figure}
% 注意:上述\label{}一定要放在\caption{}之后,否则引用图片序号会只会显示??.

\begin{document}

\section{极限问题综合}

\begin{example}
设二阶可微函数\(f:[1,+\infty)\to(0,+\infty)\)满足
\[
f''(x)\leqslant0,\lim_{x\rightarrow +\infty}f(x)=+\infty.
\]
求极限
\[
\lim_{s\rightarrow0^{+}}\sum_{n = 1}^{\infty}\frac{(-1)^n}{f^s(n)}.
\]
\end{example}
\begin{note}
本例非常经典,深刻体现了“拉格朗日中值定理”保持阶不变和“和式和积分”转化的思想.
\end{note}
\begin{proof}
由条件$f''(x)\leqslant 0$可知,$f$是上凸函数.而上凸函数只能在递增、递减、先增后减中发生一个.又$\lim_{x\rightarrow +\infty}f(x)=+\infty$,因此$f$一定在$[1,+\infty)$上递增.再结合$f''(x)\leqslant 0$可知$f'\geqslant0$且单调递减.
下面来求极限.

由Lagrange中值定理可得,对\(\forall n\in \mathbb{N}\),存在\(\theta_n\in(2n - 1, 2n)\),使得
\begin{align}\label{example4.73-1.0}
\sum_{n = 1}^{\infty}\frac{(-1)^n}{f^s(n)} = \sum_{n = 1}^{\infty}\left[\frac{1}{f^s(2n)} - \frac{1}{f^s(2n - 1)}\right] \xlongequal{\text{Lagrange中值定理}} s\sum_{n = 1}^{\infty}\frac{-f'(\theta_n)}{f^{s + 1}(\theta_n)}.
\end{align}
由于\(\theta_n\in(2n - 1, 2n)\),\(\forall n\in \mathbb{N}\)且\(f\geqslant 0\)单调递增,\(f'\geqslant 0\)单调递减,因此
\begin{align}
s\sum_{n = 1}^{\infty}\frac{-f'(2n - 1)}{f^{s + 1}(2n - 1)} \leqslant s\sum_{n = 1}^{\infty}\frac{-f'(\theta_n)}{f^{s + 1}(\theta_n)} \leqslant s\sum_{n = 1}^{\infty}\frac{-f'(2n)}{f^{s + 1}(2n)}. \label{example4.73-1.1}
\end{align}
又因为\(\left[\frac{-f'(x)}{f^{s + 1}(x)}\right]' = \frac{f''(x)f(x) - (s + 1)f'(x)}{f^{s + 2}(x)}\leqslant 0\),所以\(\frac{-f'(x)}{f^{s + 1}(x)}\)单调递减。从而一方面,我们有
\begin{align}
\underset{s\rightarrow 0^+}{\lim}s\sum_{n=1}^{\infty}{\frac{-f'\left( 2n \right)}{f^{s+1}\left( 2n \right)}}&\leqslant -\underset{s\rightarrow 0^+}{\lim}s\sum_{n=1}^{\infty}{\int_{n-1}^n{\frac{f'\left( 2x \right)}{f^{s+1}\left( 2x \right)}\mathrm{d}x}}=-\underset{s\rightarrow 0^+}{\lim}\frac{s}{2}\sum_{n=1}^{\infty}{\int_{2n-1}^{2n}{\frac{f'\left( x \right)}{f^{s+1}\left( x \right)}\mathrm{d}x}}\nonumber
\\
&=-\underset{s\rightarrow 0^+}{\lim}\frac{s}{2}\int_1^{+\infty}{\frac{f'\left( x \right)}{f^{s+1}\left( x \right)}\mathrm{d}x}=-\underset{s\rightarrow 0^+}{\lim}\frac{s}{2}\int_1^{+\infty}{\frac{1}{f^{s+1}\left( x \right)}\mathrm{d}f\left( x \right)}
\nonumber
\\
&=\underset{s\rightarrow 0^+}{\lim}\frac{s}{2}\cdot \frac{1}{sf^s\left( x \right)}\Big|_{1}^{+\infty}=-\underset{s\rightarrow 0^+}{\lim}\left[ \frac{s}{2}\cdot \frac{1}{sf^s\left( 1 \right)} \right] =-\frac{1}{2}. \label{example4.73-1.2}
\end{align}

\begin{align}
\underset{s\rightarrow 0^+}{\lim}s\sum_{n=1}^{\infty}{\frac{-f'\left( 2n \right)}{f^{s+1}\left( 2n \right)}}&\geqslant -\underset{s\rightarrow 0^+}{\lim}s\sum_{n=1}^{\infty}{\int_n^{n+1}{\frac{f'\left( 2x \right)}{f^{s+1}\left( 2x \right)}\mathrm{d}x}}=-\underset{s\rightarrow 0^+}{\lim}\frac{s}{2}\sum_{n=1}^{\infty}{\int_{2n}^{2n+1}{\frac{f'\left( x \right)}{f^{s+1}\left( x \right)}\mathrm{d}x}}
\nonumber
\\
&=-\underset{s\rightarrow 0^+}{\lim}\frac{s}{2}\int_2^{+\infty}{\frac{f'\left( x \right)}{f^{s+1}\left( x \right)}\mathrm{d}x}=-\underset{s\rightarrow 0^+}{\lim}\frac{s}{2}\int_2^{+\infty}{\frac{1}{f^{s+1}\left( x \right)}\mathrm{d}f\left( x \right)}
\nonumber
\\
&=\underset{s\rightarrow 0^+}{\lim}\frac{s}{2}\cdot \frac{1}{sf^s\left( x \right)}\Big|_{2}^{+\infty}=-\underset{s\rightarrow 0^+}{\lim}\left[ \frac{s}{2}\cdot \frac{1}{sf^s\left( 2 \right)} \right] =-\frac{1}{2}. \label{example4.73-1.3}
\end{align}
于是利用\eqref{example4.73-1.2}\eqref{example4.73-1.3}式,由夹逼准则可得
\begin{align}
\lim_{s\rightarrow 0^+}s\sum_{n = 1}^{\infty}\frac{-f'(2n)}{f^{s + 1}(2n)} = -\frac{1}{2}. \label{example4.73-2.1} 
\end{align}
另一方面,我们有
\begin{align}
\underset{s\rightarrow 0^+}{\lim}s\sum_{n=1}^{\infty}{\frac{-f'\left( 2n-1 \right)}{f^{s+1}\left( 2n-1 \right)}}&\leqslant -\underset{s\rightarrow 0^+}{\lim}s\left[ \frac{f'\left( 1 \right)}{f^{s+1}\left( 1 \right)}+\sum_{n=2}^{\infty}{\int_{n-1}^n{\frac{f'\left( 2x-1 \right)}{f^{s+1}\left( 2x-1 \right)}\mathrm{d}x}} \right] =-\underset{s\rightarrow 0^+}{\lim}s\left[ \frac{f'\left( 1 \right)}{f^{s+1}\left( 1 \right)}+\frac{1}{2}\sum_{n=2}^{\infty}{\int_{2n-3}^{2n-1}{\frac{f'\left( x \right)}{f^{s+1}\left( x \right)}\mathrm{d}x}} \right] 
\nonumber
\\
&=-\underset{s\rightarrow 0^+}{\lim}s\left[ \frac{f'\left( 1 \right)}{f^{s+1}\left( 1 \right)}+\frac{1}{2}\int_1^{+\infty}{\frac{f'\left( x \right)}{f^{s+1}\left( x \right)}\mathrm{d}x} \right] =-\underset{s\rightarrow 0^+}{\lim}\frac{s}{2}\int_1^{+\infty}{\frac{f'\left( x \right)}{f^{s+1}\left( x \right)}\mathrm{d}x}
\nonumber
\\
&=-\underset{s\rightarrow 0^+}{\lim}\frac{s}{2}\int_1^{+\infty}{\frac{1}{f^{s+1}\left( x \right)}\mathrm{d}f\left( x \right)}=\underset{s\rightarrow 0^+}{\lim}\frac{s}{2}\cdot \frac{1}{sf^s\left( x \right)}\Big|_{1}^{+\infty}
\nonumber
\\
&=-\underset{s\rightarrow 0^+}{\lim}\left[ \frac{s}{2}\cdot \frac{1}{sf^s\left( 1 \right)} \right] =-\frac{1}{2}.\label{example4.73-1.4}
\end{align}

\begin{align}
\underset{s\rightarrow 0^+}{\lim}s\sum_{n=1}^{\infty}{\frac{-f'\left( 2n-1 \right)}{f^{s+1}\left( 2n-1 \right)}}&\geqslant -\underset{s\rightarrow 0^+}{\lim}\frac{s}{2}\sum_{n=1}^{\infty}{\int_n^{n+1}{\frac{f'\left( x \right)}{f^{s+1}\left( x \right)}\mathrm{d}x}}=-\underset{s\rightarrow 0^+}{\lim}\frac{s}{2}\sum_{n=1}^{\infty}{\int_{2n-1}^{2n+1}{\frac{f'\left( x \right)}{f^{s+1}\left( x \right)}\mathrm{d}x}}
\nonumber
\\
&=-\underset{s\rightarrow 0^+}{\lim}\frac{s}{2}\int_1^{+\infty}{\frac{f'\left( x \right)}{f^{s+1}\left( x \right)}\mathrm{d}x}=-\underset{s\rightarrow 0^+}{\lim}\frac{s}{2}\int_1^{+\infty}{\frac{1}{f^{s+1}\left( x \right)}\mathrm{d}f\left( x \right)}
\nonumber
\\
&=\underset{s\rightarrow 0^+}{\lim}\frac{s}{2}\cdot \frac{1}{sf^s\left( x \right)}\Big|_{1}^{+\infty}=-\underset{s\rightarrow 0^+}{\lim}\left[ \frac{s}{2}\cdot \frac{1}{sf^s\left( 1 \right)} \right] =-\frac{1}{2}.\label{example4.73-1.5}
\end{align}
于是利用\eqref{example4.73-1.4}\eqref{example4.73-1.5}式,由夹逼准则可得
\begin{align}
\lim_{s\rightarrow 0^+}s\sum_{n = 1}^{\infty}\frac{-f'(2n - 1)}{f^{s + 1}(2n - 1)} = -\frac{1}{2}. \label{example4.73-2.2} 
\end{align}
故结合\eqref{example4.73-1.0}\eqref{example4.73-1.1}\eqref{example4.73-2.1}\eqref{example4.73-2.2}式,由夹逼准则可得
\[
\lim_{s\rightarrow 0^+} \sum_{n = 1}^{\infty}\frac{(-1)^n}{f^s(n)} = \lim_{s\rightarrow 0^+} s\sum_{n = 1}^{\infty}\frac{-f'(\theta_n)}{f^{s + 1}(\theta_n)} = -\frac{1}{2}.
\]

\end{proof}

\begin{example}
求极限\(\lim_{n\rightarrow\infty}n\sup_{x\in[0,1]}\sum_{k = 1}^{n - 1}x^{k}(1 - x)^{n - k}\)。
\end{example}
\begin{proof}
根据对称性,不妨设\(x\in\left[0,\frac{1}{2}\right]\),先尝试找到最大值点。
在\(x = 0,\frac{1}{2}\)时代入,很明显对应的极限是零,考虑\(x\in\left(0,\frac{1}{2}\right)\),根据等比数列求和公式有
\[
\sum_{k = 1}^{n - 1}x^{k}(1 - x)^{n - k}=(1 - x)^{n}\sum_{k = 1}^{n - 1}\left(\frac{x}{1 - x}\right)^{k}=\frac{x(1 - x)}{1 - 2x}((1 - x)^{n}-x^{n})
\]
如果\(\delta\in\left(0,\frac{1}{2}\right)\)已经取定,则在区间\(\left[\delta,\frac{1}{2}\right]\)中
\[
n\sum_{k = 1}^{n - 1}x^{k}(1 - x)^{n - k}\leqslant  n\sum_{k = 1}^{n - 1}\left(\frac{1}{2}\right)^{k}(1 - \delta)^{n - k}\leqslant  n(1 - \delta)^{n}\sum_{k = 0}^{\infty}\left(\frac{1}{2(1 - \delta)}\right)^{k}=\frac{n(1 - \delta)^{n}}{1-\frac{1}{2(1 - \delta)}}
\]
右端是指数级趋于零的并且上式不依赖于\(x\),所以函数会一致趋于零。
因此最大值点应该在\(x = 0\)附近,近似的有
\[
n\sum_{k = 1}^{n - 1}x^{k}(1 - x)^{n - k}=\frac{nx(1 - x)}{1 - 2x}((1 - x)^{n}-x^{n})\approx nx(1 - x)^{n}
\]
取\(x = \frac{1}{n}\)显然极限是\(\frac{1}{e}\),我们猜测这就是答案,下面开始证明。
首先取\(x = \frac{1}{n}\)有
\[
\lim_{n\rightarrow\infty}n\sum_{k = 1}^{n - 1}\left(\frac{1}{n}\right)^{k}\left(1-\frac{1}{n}\right)^{n - k}=\lim_{n\rightarrow\infty}\frac{1-\frac{1}{n}}{1-\frac{2}{n}}\left(\left(1-\frac{1}{n}\right)^{n}-\left(\frac{1}{n}\right)^{n}\right)=\frac{1}{e}
\]
由此可知\(\lim_{n\rightarrow\infty}n\sup_{x\in[0,1]}\sum_{k = 1}^{n - 1}x^{k}(1 - x)^{n - k}\geqslant \frac{1}{e}\),下面估计上极限。
根据对称性,不妨只考虑\(x\in\left[0,\frac{1}{2}\right]\),对任意\(\delta\in\left(0,\frac{1}{2}\right)\)取定,当\(x\in\left[\delta,\frac{1}{2}\right]\)时总有
\[
n\sum_{k = 1}^{n - 1}x^{k}(1 - x)^{n - k}\leqslant  n\sum_{k = 1}^{n - 1}\left(\frac{1}{2}\right)^{k}(1 - \delta)^{n - k}\leqslant  n(1 - \delta)^{n}\sum_{k = 0}^{\infty}\left(\frac{1}{2(1 - \delta)}\right)^{k}=\frac{n(1 - \delta)^{n}}{1-\frac{1}{2(1 - \delta)}}
\]
当\(x\in[0,\delta]\)时,结合均值不等式有
\begin{align*}
n\sum_{k = 1}^{n - 1}x^{k}(1 - x)^{n - k}&=\frac{nx(1 - x)}{1 - 2x}((1 - x)^{n}-x^{n})\approx\frac{nx(1 - x)^{n}}{1 - 2\delta}\leqslant \frac{\left(1-\frac{1}{n + 1}\right)^{n + 1}}{1 - 2\delta}\leqslant \frac{1}{e}\frac{1}{1 - 2\delta}
\end{align*}
所以可以取\(n > N\)充分大,使得\(\frac{n(1 - \delta)^{n}}{1-\frac{1}{2(1 - \delta)}}<\frac{1}{e}\),此时便有
\[
n\sup_{x\in[0,1]}\sum_{k = 1}^{n - 1}x^{k}(1 - x)^{n - k}\leqslant \frac{1}{e}\frac{1}{1 - 2\delta}\Rightarrow\varlimsup_{n\rightarrow\infty}n\sup_{x\in[0,1]}\sum_{k = 1}^{n - 1}x^{k}(1 - x)^{n - k}\leqslant \frac{1}{e}\frac{1}{1 - 2\delta}
\]
最后,根据\(\delta\)的任意性,可知结论成立。

\end{proof}

\begin{example}
设\(x_n>0\),\(k\)为正整数,证明:\(\varlimsup_{n\rightarrow\infty}\frac{x_1 + x_2+\cdots + x_{n + k}}{x_n}\geqslant \frac{(k + 1)^{k + 1}}{k^k}\)且常数是最佳的。
\end{example}
\begin{note}
此类问题反证法将会带来一个恒成立的不等式,有很强的效果,所以一般都用反证法,证明的灵感来源于\(k = 1\)时的情况.
\end{note}
\begin{proof}
设\(S_n=x_1 + x_2+\cdots + x_n\),采用反证法,则存在\(N\)使得\(n\geqslant  N\)时恒成立
\[S_{n + k}\leqslant \lambda(S_n - S_{n - 1}),\lambda\in\left[1,\frac{(k + 1)^{k + 1}}{k^k}\right)\]
显然\(S_n\)是单调递增的,如果\(S_n\)有界,则在不等式两端取极限可知\(S_n\)收敛到零,矛盾,所以\(S_n\)严格单调递增趋于正无穷,因此对任意\(n\geqslant  N\)有\(S_n>S_{n - 1}\)。
如果已经得到了\(S_n>cS_{n - 1}\)对任意\(n\geqslant  N\)恒成立,这里\(c\)是正数,则对任意\(n\geqslant  N\)有
\begin{align*}
S_{n + k}&>cS_{n + k - 1},S_{n + k - 1}>cS_{n + k - 2},\cdots,S_{n + 1}>cS_n\Rightarrow S_{n + k}>c^kS_n\\
0<S_{n + k}-c^kS_n&\leqslant (\lambda - c^k)S_n-\lambda S_{n - 1}\Rightarrow S_n>\frac{\lambda}{\lambda - c^k}S_{n - 1}
\end{align*}
这样不等式就加强了,记\(c'=\frac{\lambda}{\lambda - c^k}\),我们得到\(S_n>c'S_{n - 1}\)对任意\(n\geqslant  N\)恒成立。
定义数列\(u_n\)为\(u_1 = 1,u_{n + 1}=\frac{\lambda}{\lambda - u_n^k}\),则重复以上过程可知\(S_n>u_mS_{n - 1}\)对任意\(m\)以及\(n\geqslant  N\)都恒成立,所以\(u_m\)这个数列必须是有界的,下面我们就由此导出矛盾。
因为\(u_{n + 1}>u_n\Leftrightarrow(\lambda - u_n^k)u_n<\lambda\Leftrightarrow(\lambda - u_n^k)^ku_n^k<\lambda^k\),由均值不等式有
\[kx^k(\lambda - x^k)^k\leqslant \left(\frac{k\lambda}{k + 1}\right)^{k + 1}<k\lambda^k\Leftrightarrow\lambda<\frac{(k + 1)^{k + 1}}{k^k}\]
显然成立,所以\(u_m\)单调递增,而如果极限存在,则极限点满足方程\(x=\frac{\lambda}{\lambda - x^k}\Leftrightarrow x(\lambda - x^k)=\lambda\),这与前面均值不等式导出的结果矛盾,所以\(u_m\)单调递增趋于正无穷,又与有界性矛盾。
综上结论得证.

\end{proof}


\begin{example}
设\(x_n>0,x_n\rightarrow0\)且\(\lim_{n\rightarrow\infty}\frac{\ln x_n}{x_1 + x_2+\cdots + x_n}=a<0\),证明:\(\lim_{n\rightarrow\infty}\frac{\ln x_n}{\ln n}=-1\)。
\end{example}
\begin{proof}
不妨设\(a = -1\),否则将\(x_n\)换成\(x_n^k\)即可,取\(k\)将\(a\)变成\(-1\)。

设\(S_n=x_1 + x_2+\cdots + x_n\),则\(S_n>0\)严格单调递增,如果\(S_n\)收敛,则\(\ln x_n\rightarrow-\infty\)与条件矛盾,所以\(S_n\)单调递增趋于正无穷。

因为\(\frac{\ln x_n}{\ln n}=\frac{\ln x_n}{S_n}\frac{S_n}{\ln n}\),\(\frac{\ln x_n}{S_n}\rightarrow -1\),所以等价的只要证明\(\frac{S_n}{\ln n}\rightarrow1\)。

条件为\(\lim_{n\rightarrow\infty}\frac{\ln x_n}{S_n}=-1\),设想作为等式,对应着\(S_n - S_{n - 1}=e^{-S_n}\)是一个隐函数类型的递推式,不方便使用,所以考虑
\[\lim_{n\rightarrow\infty}\frac{\ln x_{n + 1}}{S_n}=\lim_{n\rightarrow\infty}\frac{\ln x_{n + 1}}{S_{n + 1}}\frac{S_{n + 1}}{S_n}=-\lim_{n\rightarrow\infty}\left(1+\frac{x_{n + 1}}{S_n}\right)=-1\]
现在等价的,已知\(S_n\)单调递增趋于无穷且\(\lim_{n\rightarrow\infty}\frac{\ln(S_{n + 1}-S_n)}{S_n}=-1\),要证明\(\lim_{n\rightarrow\infty}\frac{S_n}{\ln n}=1\)。
由极限定义,对任意\(\varepsilon>0\),存在\(N\)使得任意\(n > N\)都有\((-1-\varepsilon)S_n<\ln(S_{n + 1}-S_n)<(-1 + \varepsilon)S_n\)也即
\[\left(\frac{1}{e}-\varepsilon\right)^{S_n}+S_n<S_{n + 1}<\left(\frac{1}{e}+\varepsilon\right)^{S_n}+S_n,\forall n\geqslant  N\]
不妨要求\(S_N>1\),考虑
\[f(x)=\left(\frac{1}{e}+\varepsilon\right)^{x}+x,f'(x)=1+\left(\frac{1}{e}+\varepsilon\right)^{x}\ln\left(\frac{1}{e}+\varepsilon\right)>1-\left(\frac{1}{e}+\varepsilon\right)^{x}>0\]
再定义\(u_N = S_N,u_{n + 1}=\left(\frac{1}{e}+\varepsilon\right)^{u_n}+u_n\),于是若有\(u_n\leqslant  S_n\)则结合单调性可知\(u_{n + 1}=f(u_n)\leqslant  f(S_n)=S_{n + 1}\),这说明\(S_n\leqslant  u_n\)对任意\(n\geqslant  N\)恒成立。
同样考虑
\[g(x)=\left(\frac{1}{e}-\varepsilon\right)^{x}+x,g'(x)=1-\left(\frac{1}{e}-\varepsilon\right)^{x}\ln\left(\frac{1}{e}-\varepsilon\right)\geqslant 1-\left(\frac{1}{e}-\varepsilon\right)\ln\left(\frac{1}{e}-\varepsilon\right)>0\]
再定义\(v_N = S_N,v_{n + 1}=\left(\frac{1}{e}-\varepsilon\right)^{v_n}+v_n\),同样道理\(S_n\geqslant  v_n\)恒成立,于是\(\frac{v_n}{\ln n}\leqslant \frac{S_n}{\ln n}\leqslant \frac{u_n}{\ln n},n\geqslant  N\)。

注意\(u_n,v_n\)具备完全一样的形式,所以统一的考虑\(a_1>1,a_{n + 1}=a_n + e^{ca_n}\),其中\(c\)在\(\frac{1}{e}\)附近,显然这个数列是单调递增趋于正无穷的,我们用stolz公式来计算相应的极限,则有
\begin{align*}
\lim_{n\rightarrow\infty}\frac{\ln a_n}{n}&=\lim_{n\rightarrow\infty}\frac{\ln a_{n + 1}-\ln a_n}{1}=\lim_{n\rightarrow\infty}\frac{e^{-ca_n}}{c^{-a_n}-1}=\lim_{n\rightarrow\infty}\frac{1}{c^{-a_{n + 1}}-c^{-a_n}}=\lim_{n\rightarrow\infty}\frac{1}{e^{-ca_n}(c^{-(a_{n + 1}-a_n)}-1)}\\
&=\lim_{n\rightarrow\infty}\frac{e^{ca_n}}{c^{-e^{ca_n}}-1}=\lim_{x\rightarrow+\infty}\frac{e^{cx}}{e^{-x\ln c}-1}=\lim_{x\rightarrow0+}\frac{x}{e^{-x\ln c}-1}=\frac{1}{-\ln c}
\end{align*}
所以
\[\lim_{n\rightarrow\infty}\frac{u_n}{\ln n}=\frac{1}{-\ln(\frac{1}{e}+\varepsilon)}=\frac{1}{1-\ln(1 + e\varepsilon)},\lim_{n\rightarrow\infty}\frac{v_n}{\ln n}=\frac{1}{-\ln(\frac{1}{e}-\varepsilon)}=\frac{1}{1-\ln(1 - e\varepsilon)}\]
这意味着
\[\varlimsup_{n\rightarrow\infty}\frac{S_n}{\ln n}\leqslant \frac{1}{1-\ln(1 + e\varepsilon)},\varliminf_{n\rightarrow\infty}\frac{S_n}{\ln n}\geqslant \frac{1}{1-\ln(1 - e\varepsilon)},\forall\varepsilon>0\]
由此可知结论成立。

\end{proof}

\begin{example}
设$n \in \mathbb{N}$, 计算
\begin{align*}
\lim_{x \to 0} \frac{1 - \cos x \sqrt{\cos (2x)} \cdot \sqrt[3]{\cos (3x)} \cdot \cdots \cdot \sqrt[n]{\cos (nx)}}{x^2}.
\end{align*}
\end{example}
\begin{solution}
由 Taylor 公式知
\begin{align*}
\cos x = 1 - \frac{x^2}{2} + o(x^2), x \to 0.
\end{align*}
\begin{align*}
\sqrt[k]{1 + x} = 1 + \frac{x}{k} + o(x), x \to 0.
\end{align*}
于是
\begin{align*}
\sqrt[k]{\cos x} = \sqrt[k]{1 - \frac{x^2}{2} + o(x^2)} = 1 + \frac{-\frac{x^2}{2} + o(x^2)}{k} + o(x^2) = 1 - \frac{k}{2}x^2 + o(x^2), x \to 0.
\end{align*}
从而
\begin{align*}
\prod_{k=1}^n \sqrt[k]{\cos kx} = \prod_{k=1}^n \left( 1 - \frac{k}{2}x^2 + o(x^2) \right) = 1 - \left( \sum_{k=1}^n \frac{k}{2} \right) x^2 + o(x^2), x \to 0.
\end{align*}
故
\begin{align*}
\lim_{x \to 0} \frac{1 - \prod\limits_{k=1}^n \sqrt[k]{\cos kx}}{x^2} = \lim_{x \to 0} \frac{\left( \sum\limits_{k=1}^n \frac{k}{2} \right) x^2 + o(x^2)}{x^2} = \frac{n(n + 1)}{4}.
\end{align*}

\end{solution}

\begin{example}
计算
\begin{align*}
\lim_{n \to \infty} \sqrt{n} \left( 1 - \sum_{i=1}^n \frac{1}{n + \sqrt{i}} \right).
\end{align*}
\end{example}
\begin{note}
注意到
\begin{align*}
\sum_{i=1}^n \frac{1}{n + \sqrt{i}} = \frac{1}{n} \sum_{i=1}^n \frac{1}{1 + \frac{\sqrt{i}}{n}},
\end{align*}
对 $\forall i \in \mathbb{N}$, 都有
\begin{align*}
\frac{\sqrt{i}}{n} \leqslant \frac{1}{\sqrt{n}} \rightarrow 0, n \rightarrow \infty.
\end{align*}
故 $\sum_{i=1}^n \frac{1}{1 + \frac{\sqrt{i}}{n}}$ 中的每一项 $\frac{1}{1 + \frac{\sqrt{i}}{n}}$ 都可以 Taylor 展开.
\end{note}
\begin{solution}
由 Taylor 公式知
\begin{align*}
\sum_{i=1}^n \frac{1}{n + \sqrt{i}} &= \frac{1}{n} \sum_{i=1}^n \frac{1}{1 + \frac{\sqrt{i}}{n}} = \frac{1}{n} \sum_{i=1}^n \left( 1 - \frac{\sqrt{i}}{n} + \frac{i}{n^2} + O\left( \frac{i\sqrt{i}}{n^3} \right) \right) \\
&= \frac{1}{n} \left[ n - \frac{\sum\limits_{i=1}^n \sqrt{i}}{n} + \frac{\sum\limits_{i=1}^n i}{n^2} + nO\left( \frac{1}{n\sqrt{n}} \right) \right] \\
&= 1 - \frac{\sum\limits_{i=1}^n \sqrt{i}}{n^2} + \frac{n + 1}{2n^2} + O\left( \frac{1}{\sqrt{n}} \right) \\
&= 1 - \frac{\sum\limits_{i=1}^n \sqrt{i}}{n^2} + O\left( \frac{1}{n} \right).
\end{align*}
于是
\begin{align*}
\lim_{n \rightarrow \infty} \sqrt{n} \left( 1 - \sum_{i=1}^n \frac{1}{n + \sqrt{i}} \right) &= \lim_{n \rightarrow \infty} \frac{\sum\limits_{i=1}^n \sqrt{i}}{n\sqrt{n}} = \lim_{n \rightarrow \infty} \frac{1}{n} \sum\limits_{i=1}^n \sqrt{\frac{i}{n}} = \int_0^1 \sqrt{x} \mathrm{d}x = \frac{2}{3}.
\end{align*}

\end{solution}

\begin{example}
设$f \in R[0,1]$, 证明
\begin{align*}
\lim_{n \to \infty} \frac{1}{n} \sum_{k=1}^{n} (-1)^k f\left( \frac{k}{n} \right) = 0.
\end{align*}
\end{example}
\begin{note}
注意$\frac{2k}{2n-1},\frac{2k-1}{2n-1}\in \left[ \frac{k}{n},\frac{k+1}{n} \right]$.
\end{note}
\begin{proof}
注意到
\begin{align*}
\frac{1}{2n}\sum_{k=1}^{2n}{\left( -1 \right) ^kf\left( \frac{k}{2n} \right)}&=\frac{1}{2n}\sum_{k=1}^n{f\left( \frac{2k}{2n} \right)}-\frac{1}{2n}\sum_{k=1}^n{f\left( \frac{2k-1}{2n} \right)}\\
&=\frac{1}{2n}\sum_{k=1}^n{f\left( \frac{k}{n} \right)}-\frac{1}{2n}\sum_{k=1}^n{f\left( \frac{k-\frac{1}{2}}{n} \right)}\\
&\rightarrow \frac{1}{2}\int_0^1{f\left( x \right) \mathrm{d}x}-\frac{1}{2}\int_0^1{f\left( x \right) \mathrm{d}x}=0,n\rightarrow \infty .
\end{align*}
\begin{align*}
\frac{1}{2n-1}\sum_{k=1}^{2n-1}{\left( -1 \right) ^kf\left( \frac{k}{2n-1} \right)}&=\frac{1}{2n-1}\sum_{k=1}^{n-1}{f\left( \frac{2k}{2n-1} \right)}-\frac{1}{2n-1}\sum_{k=1}^n{f\left( \frac{2k-1}{2n-1} \right)}\\
&=\frac{n}{2n-1}\cdot \frac{1}{n}\sum_{k=1}^{n-1}{f\left( \frac{2k}{2n-1} \right)}-\frac{n}{2n-1}\cdot \frac{1}{n}\sum_{k=1}^n{f\left( \frac{2k-1}{2n-1} \right)}\\
&\rightarrow \frac{1}{2}\int_0^1{f\left( x \right) \mathrm{d}x}-\frac{1}{2}\int_0^1{f\left( x \right) \mathrm{d}x}=0,n\rightarrow \infty .
\end{align*}
故由\hyperref[proposition:子列极限命题]{子列极限命题(b)}可知
\begin{align*}
\lim_{n\rightarrow \infty} \frac{1}{n}\sum_{k=1}^n{(}-1)^kf\left( \frac{k}{n} \right) =0.
\end{align*}

\end{proof}

\begin{example}
设 $x_{n+1} = x_n - x_n^3$, $x_1 \in \mathbb{R}$, 判断 $\lim_{n \to \infty} x_n$ 收敛性.
\end{example}
\begin{note}
因为递推函数$g(x) = x(1 - x^2)$关于原点对称,而$\{ x_n \}$的敛散性只由$x_1$决定,所以我们只需要考虑$x_1 > 0$的情况即可,由于$g(x)$关于原点对称,故$x_1 < 0$的情况和$x_1 > 0$的情况类似。因此我们可以直接考虑数列$\{ |x_n| \}$。这样能避免很多分类讨论。
注意这个递推函数$g(x)$只有一个不动点$x=0$.

如果不加绝对值,原递推函数的蛛网图会比较杂乱,加上绝对值后讨论会比较清晰。
实际上,通过蛛网图分析,也能得到使得$\{x_n\}$发散的$x_1$的临界点满足$g(x_1)=x_2,g(x_2)=x_1$,即$g(g(x_1))=x_1$.于是就有
\begin{align}
-x_1^6 + 3x_1^4 - 3x_1^2 + 2 = 0.\label{eq:1000.123}
\end{align}
但是当$x_1 = \pm 1, \pm 2$上式不成立,故上述方程没有有理根。令$t = x_1^2$,则上式可化为
\begin{align*}
-t^3 + 3t^2 - 3t + 2 = 0.
\end{align*}
当$t = 2$时,上式成立。故上式可化为
\begin{align*}
(t - 2)(-t^2 + t - 1) = 0.
\end{align*}
因此上式只有一个实根$t = 2$,即\eqref{eq:1000.123}式只有当$x_1^2 = 2$时才有实根。故\eqref{eq:1000.123}式只有两个实根$x_1 = \pm \sqrt{2}$.

考虑$|x_{n+1}| = |x_n - x_n^3| = |x_n| |1 - x_n^2|$,记$f(x) = x |1 - x^2|$,则$f(x)$有两个不动点$x=\pm \sqrt{2}.$
\end{note}
\begin{proof}
考虑$|x_{n+1}| = |x_n - x_n^3| = |x_n| |1 - x_n^2|$,则

(1)当$|x_1| > \sqrt{2}$时,则$|x_{n+1}| = |x_n| |x_n^2 - 1| \geqslant |x_n| > \sqrt{2}$。故此时$\{ |x_n| \}$递增,且有下界$\sqrt{2}$。
而$f$没有大于$\sqrt{2}$的不动点,因此$\lim_{n \to \infty} |x_n| = +\infty$。

(2)当$|x_1| \leqslant \sqrt{2}$时,则$|x_{n+1}| = |x_n| |x_n^2 - 1| \leqslant |x_n| \leqslant \sqrt{2}$。故此时$\{ |x_n| \}$递减,且有下界$\sqrt{2}$。于是$A \triangleq \lim_{n \to \infty} |x_n|$存在。对$|x_{n+1}| = |x_n| |x_n^2 - 1|$两边同时取极限得$A = 0$或$\sqrt{2}$。

(i)若$A = 0$,则由$\lim_{n \to \infty} |x_n| = A = 0$可知$\lim_{n \to \infty} x_n = 0$。

(ii)若$A = \sqrt{2}$,则由$\{ |x_n| \}$递减,且$|x_n| \leqslant \sqrt{2}$知
$\sqrt{2} = \lim_{n \to \infty} |x_n| \leqslant |x_n| \leqslant \sqrt{2} \Rightarrow |x_n| = \sqrt{2}, n = 1, 2, \cdots$。
此时$x_1 = \pm \sqrt{2}$,再代入$x_{n+1} = x_n - x_n^3$得
$x_n = (-1)^n x_1, n = 2, 3, \cdots$。
故此时$\{ x_n \}$发散。

综上
$$\lim_{n \to \infty} x_n = \begin{cases} \text{发散}&, |x_1| \geqslant \sqrt{2} \\ 0&,|x_1| < \sqrt{2} \end{cases}.$$

\end{proof}

\begin{example}
设函数 $f:[a,b] \to [a,b]$ 满足
\begin{align*}
|f(x) - f(y)| \leqslant |x - y|, \forall x,y \in [a,b]
\end{align*}
设递推
\begin{align*}
x_1 \in [a,b], x_{n+1} = \frac{1}{2}(x_n + f(x_n)), n = 1,2,\cdots
\end{align*}
证明 $\lim_{n \to \infty} x_n$ 存在.
\end{example}
\begin{proof}
由于$a \leqslant f(x) \leqslant b$,因此归纳易得$a \leqslant x_n \leqslant b$。令$g(x) = \frac{x + f(x)}{2}$,则
\begin{align*}
g(y) - g(x) = \frac{y - x - [f(y) - f(x)]}{2} \geqslant 0, \forall y \geqslant x.
\end{align*}
由命题可知\hyperref[proposition:递增函数递推数列]{递增递推数列$\{ x_n \}$一定单调},故$\lim_{n \to \infty} x_n$存在。

\end{proof}

\begin{example}
设 $f(x) \in C[0,1]$, $f(x) > 0$, 证明
\begin{align*}
\lim_{n \to \infty} \frac{\int_0^1 f^{n+1}(x) \mathrm{d}x}{\int_0^1 f^n(x) \mathrm{d}x} = \max_{[0,1]} f.
\end{align*}
\end{example}
\begin{note}
回顾\refexa{example-3.31}和
\refpro{proposition:比值极限存在则根值极限等于比值极限}.因此我们只需证明\refpro{proposition:比值极限存在则根值极限等于比值极限}的反向,再结合\refexa{example-3.31}就能得证.但是\hyperref[theorem:反向Stolz定理]{反向Stolz定理}一般不会直接应用,因此我们可以尝试利用单调有界定理证明比值极限存在,再利用\refpro{proposition:比值极限存在则根值极限等于比值极限}就能直接得证.

实际上,只要证明了单调性,就能利用\hyperref[theorem:反向Stolz定理]{反向Stolz定理}证明\refpro{proposition:比值极限存在则根值极限等于比值极限}的反向也成立,再利用\refexa{example-3.31}就能得到结论.
\end{note}
\begin{proof}
注意到
\begin{align}
\frac{\int_0^1 f^{n+2}(x) \mathrm{d}x}{\int_0^1 f^{n+1}(x) \mathrm{d}x} \geqslant \frac{\int_0^1 f^{n+1}(x) \mathrm{d}x}{\int_0^1 f^n(x) \mathrm{d}x} \Longleftrightarrow \int_0^1 f^{n+2}(x) \mathrm{d}x \int_0^1 f^n(x) \mathrm{d}x \geqslant \left( \int_0^1 f^{n+1}(x) \mathrm{d}x \right)^2.\label{eq:103.1}
\end{align}
由Cauchy不等式知
\begin{align*}
\int_0^1 f^{n+2}(x) \mathrm{d}x \int_0^1 f^n(x) \mathrm{d}x \geqslant \left( \int_0^1 f^{\frac{n+2}{2}}(x) f^{\frac{n}{2}}(x) \mathrm{d}x \right)^2 = \left( \int_0^1 f^{n+1}(x) \mathrm{d}x \right)^2.
\end{align*}
故\eqref{eq:103.1}式成立,即$\left\{ \frac{\int_0^1 f^{n+1}(x) \mathrm{d}x}{\int_0^1 f^n(x) \mathrm{d}x} \right\}_{n=0}^{\infty}$单调递增。因此$\lim_{n \to \infty} \frac{\int_0^1 f^{n+1}(x) \mathrm{d}x}{\int_0^1 f^n(x) \mathrm{d}x} \in \mathbb{R} \cup \{ +\infty \}$。由\refexa{example-3.31}可知$$\lim_{n \to \infty} \sqrt[n]{\int_0^1 f^n(x) \mathrm{d}x} = \max_{[0,1]} f.$$再根据\refpro{proposition:比值极限存在则根值极限等于比值极限}可知
\begin{align*}
\lim_{n \to \infty} \frac{\int_0^1 f^{n+1}(x) \mathrm{d}x}{\int_0^1 f^n(x) \mathrm{d}x} = \lim_{n \to \infty} \sqrt[n]{\int_0^1 f^n(x) \mathrm{d}x} = \max_{[0,1]} f.
\end{align*}

\end{proof}

\begin{example}
\begin{enumerate}
\item 设 $\{x_n\}_{n=1}^{\infty} \subset (0,+\infty)$ 满足
\begin{align*}
x_n + \frac{1}{x_{n+1}} < 2,\ n = 1,2,\cdots
\end{align*}
证明: $\lim_{n \to \infty} x_n$ 存在并求极限.
\item 设 $\{a_n\}_{n=1}^{\infty} \subset (0,+\infty)$ 满足
\begin{align*}
a_{n+1} + \frac{4}{a_n} < 4,\ n = 1,2,\cdots
\end{align*}
证明: $\lim_{n \to \infty} a_n$ 存在并求极限.
\item 设 $\{x_n\}_{n=1}^{\infty} \subset (0,+\infty)$ 满足
\begin{align*}
x_n + \frac{4}{x_{n+1}^2} < 3,\ n = 1,2,\cdots
\end{align*}
证明: $\lim_{n \to \infty} x_n$ 存在并求极限.
\item 设 $\{x_n\}_{n=1}^{\infty} \subset (0,+\infty)$ 满足
\begin{align*}
\ln x_n + \frac{1}{x_{n+1}} < 1,\ n = 1,2,\cdots
\end{align*}
证明: $\lim_{n \to \infty} x_n$ 存在并求极限.
\end{enumerate}
\end{example}
\begin{note}
此类问题其实就是把 $x_{n+1}, x_n$ 部分全部换成 $x$, 数字部分往往是 $x$ 部分的一个最值, 从把这个数字用不等式放缩为数列来得到估计.
\end{note}
\begin{proof}
\begin{enumerate}
\item 由均值不等式可知
$$x_n + \frac{1}{x_{n+1}} < 2 \leqslant x_{n+1} + \frac{1}{x_{n+1}} \Rightarrow x_{n+1} \geqslant x_n.$$
并且$x_n < 2 - \frac{1}{x_{n+1}} < 2$,故$\lim_{n \to \infty} x_n \triangleq x$存在。于是
$$2 \leqslant x + \frac{1}{x} = \lim_{n \to \infty} \left( x_n + \frac{1}{x_{n+1}} \right) \leqslant 2 \Rightarrow x + \frac{1}{x} = 2 \Rightarrow x = 1.$$
因此$\lim_{n \to \infty} x_n = 1.$

\item 

\item 

\item 
\end{enumerate}

\end{proof}

\begin{example}
设 $f(x) \in C^1(\mathbb{R}), |f(x)| \leqslant  1, f'(x) > 0$,证明:对任意 $b > a > 0$ 有
\begin{align*}
\lim_{n\rightarrow \infty} \int_a^b{f' \left( nx-\frac{1}{x} \right) \mathrm{d}x}=0.
\end{align*}
\end{example}
\begin{proof}
{\color{blue}证法一:}\begin{align*}
\int_a^b f'\left(nx - \frac{1}{x}\right) \mathrm{d}x &= \int_a^b \frac{1}{n + \frac{1}{x^2}} \left(n + \frac{1}{x^2}\right) f'\left(nx - \frac{1}{x}\right) \mathrm{d}x = \int_a^b \frac{1}{n + \frac{1}{x^2}} \mathrm{d}f\left(nx - \frac{1}{x}\right) \\
&= \frac{f\left(nb - \frac{1}{b}\right)}{n + \frac{1}{b^2}} - \frac{f\left(na - \frac{1}{a}\right)}{n + \frac{1}{a^2}} + \int_a^b f\left(nx - \frac{1}{x}\right) \frac{2}{x^3 \left(n + \frac{1}{x^2}\right)^2} \mathrm{d}x \\
&\leqslant  \frac{1}{n + \frac{1}{b^2}} + \frac{1}{n + \frac{1}{a^2}} + \frac{2}{a^3 \left(n + \frac{1}{b^2}\right)^2} \int_a^b f\left(nx - \frac{1}{x}\right) \mathrm{d}x \\
&\leqslant  \frac{1}{n + \frac{1}{b^2}} + \frac{1}{n + \frac{1}{a^2}} + \frac{2(b - a)}{a^3 \left(n + \frac{1}{b^2}\right)^2} \rightarrow 0, \, n \rightarrow \infty.
\end{align*}

{\color{blue}证法二:}令$y = nx - \frac{1}{x}$,则$x = \frac{y + \sqrt{y^2 + 4n}}{2n} >a> 0$。于是
\begin{align*}
\int_a^b f'\left(nx - \frac{1}{x}\right) \mathrm{d}x &= \int_{na - \frac{1}{a}}^{nb - \frac{1}{b}} f'(y) \frac{1 + \frac{y}{\sqrt{y^2 + 4n}}}{2n} \mathrm{d}y = \int_{na - \frac{1}{a}}^{nb - \frac{1}{b}} \frac{1 + \frac{y}{\sqrt{y^2 + 4n}}}{2n} \mathrm{d}f(y) \\
&= \frac{1 + \frac{nb - \frac{1}{b}}{\sqrt{\left(nb - \frac{1}{b}\right)^2 + 4n}}}{2n} f\left(nb - \frac{1}{b}\right) - \frac{1 + \frac{na - \frac{1}{a}}{\sqrt{\left(na - \frac{1}{a}\right)^2 + 4n}}}{2n} f\left(na - \frac{1}{a}\right) - \int_{na - \frac{1}{a}}^{nb - \frac{1}{b}} f(y) \frac{\sqrt{y^2 + 4n} + \frac{y^2}{\sqrt{y^2 + 4n}}}{4n^2(y^2 + 4n)} \mathrm{d}y \\
&\leqslant  \frac{1 + \frac{nb - \frac{1}{b}}{\sqrt{\left(nb - \frac{1}{b}\right)^2 + 4n}}}{2n} + \frac{1 + \frac{na - \frac{1}{a}}{\sqrt{\left(na - \frac{1}{a}\right)^2 + 4n}}}{2n} + \int_{na - \frac{1}{a}}^{nb - \frac{1}{b}} \frac{\sqrt{\left(nb - \frac{1}{b}\right)^2 + 4n} + \frac{\left(nb - \frac{1}{b}\right)^2}{\sqrt{\left(na - \frac{1}{a}\right)^2 + 4n}}}{4n^2\left(\left(na - \frac{1}{a}\right)^2 + 4n\right)} \mathrm{d}y \\
&\rightarrow 0, \, n \rightarrow +\infty.
\end{align*}

\end{proof}

\begin{example}
求极限
\begin{align*}
\lim_{n \to \infty} \int_0^\infty \frac{\sin nx}{x} e^{-x} \mathrm{d}x.
\end{align*}
\end{example}
\begin{proof}
{\color{blue}证法一:}对$\forall \delta > 0$,我们有
\begin{align*}
\int_{\delta}^{\infty} \frac{e^{-x}}{x} \mathrm{d}x < \frac{1}{\delta} \int_{\delta}^{\infty} \frac{1}{e^x} \mathrm{d}x < +\infty.
\end{align*}
于是由\hyperref[theorem:Riemann引理]{Riemman引理}可知
\begin{align*}
\int_{\delta}^{\infty} \frac{\sin nx}{x} e^{-x} \mathrm{d}x = \frac{1}{2\pi} \int_0^{2\pi} \sin x \mathrm{d}x \int_{\delta}^{\infty} \frac{e^{-x}}{x} \mathrm{d}x = 0.
\end{align*}
注意到
\begin{align*}
\int_0^{\delta} \frac{\sin nx}{x} e^{-x} \mathrm{d}x \sim \int_0^{\delta} \frac{\sin nx}{x} \mathrm{d}x, \, \delta \rightarrow 0^+,
\end{align*}
\begin{align*}
\int_0^{\delta} \frac{\sin nx}{x} \mathrm{d}x = \int_0^{n\delta} \frac{\sin x}{x} \mathrm{d}x \rightarrow \int_0^{\infty} \frac{\sin x}{x} \mathrm{d}x \xlongequal{\text{\nrefpro{proposition:重要定积分结果(必记)}{(2)}}} \frac{\pi}{2}, \, n \rightarrow \infty,
\end{align*}
故
\begin{align*}
\int_0^{\infty} \frac{\sin nx}{x} e^{-x} \mathrm{d}x = \int_0^{\delta} \frac{\sin nx}{x} e^{-x} \mathrm{d}x + \int_{\delta}^{\infty} \frac{\sin nx}{x} e^{-x} \mathrm{d}x = \frac{\pi}{2}.
\end{align*}

{\color{blue}证法二:}记$p(x)=\frac{e^{-x}-1}{x},p(0)=-1$,则$p(x)$可导,并且
\begin{align*}
\int_0^\infty \frac{\sin nx}{x}e^{-x}\mathrm{d}x&=\int_0^\infty \frac{e^{-x}-1}{x}\sin nx\mathrm{d}x+\int_0^\infty \frac{\sin nx}{x}\mathrm{d}x \xlongequal{\text{\nrefpro{proposition:重要定积分结果(必记)}{(2)}}} \int_0^\infty p(x)\sin nx\mathrm{d}x+\frac{\pi}{2}\\
&=\frac{\pi}{2}-\int_0^\infty p(x)d\frac{\cos nx}{n}=\frac{\pi}{2}-\frac{1}{n}\left(1-\int_0^\infty p'(x)\cos nx\mathrm{d}x\right).
\end{align*}
求导有$p'(x)=\frac{1-xe^{-x}-e^{-x}}{x^2},p'(0)=\frac{1}{2}$,所以$\left|\int_0^\infty p'(x)\cos nx\mathrm{d}x\right|\leqslant \int_0^\infty \frac{1-xe^{-x}-e^{-x}}{x^2}\mathrm{d}x<\infty.$
由此可知$\lim\limits_{n\rightarrow \infty}\int_0^\infty \frac{\sin nx}{x}e^{-x}\mathrm{d}x=\frac{\pi}{2}.$

\end{proof}

\begin{example}
设$f(x),g(x)\in C[0,1]$且$\lim\limits_{x\rightarrow 0^+}\frac{g(x)}{x}$为有限数,证明:
\begin{align*}
\lim\limits_{n\rightarrow \infty}n\int_0^1 f(x)g(x^n)\mathrm{d}x=f(1)\int_0^1 \frac{g(x)}{x}\mathrm{d}x.
\end{align*}
\end{example}
\begin{proof}
{\color{blue}证法一:}
注意到
\begin{align*}
n\int_0^1 f(x)g(x^n)\mathrm{d}x&=\int_0^1 f\left(x^{\frac{1}{n}}\right)g(x)x^{\frac{1}{n}-1}\mathrm{d}x=\int_0^1 \frac{g(x)}{x}\cdot x^{\frac{1}{n}}f\left(x^{\frac{1}{n}}\right)\mathrm{d}x.
\end{align*}
令$h(x)\triangleq \begin{cases}
\frac{g(x)}{x},x\ne 0\\
\lim\limits_{x\rightarrow 0^+}\frac{g(x)}{x},x=0
\end{cases}$,则$h\in C[0,1]$. 从而
\begin{align*}
n\int_0^1 f(x)g(x^n)\mathrm{d}x=\int_0^1 h(x)\cdot x^{\frac{1}{n}}f\left(x^{\frac{1}{n}}\right)\mathrm{d}x.
\end{align*}
$\forall \varepsilon >0$,取$\delta =\varepsilon$,对$\forall x\in [\delta,1]$,由$\lim\limits_{n\rightarrow \infty}x^{\frac{1}{n}}=1$及$f\in C[0,1]$可知,存在$N>0$,使得
\begin{align*}
\left|x^{\frac{1}{n}}-1\right|<\varepsilon,\quad \left|f\left(x^{\frac{1}{n}}\right)-f(1)\right|<\varepsilon,\forall n>N.
\end{align*}
设$\left|h(x)\right|,\left|f(x)\right|\leqslant M\in \mathbb{R}$,则
\begin{align*}
&\left|\int_0^1 h(x)\cdot x^{\frac{1}{n}}f\left(x^{\frac{1}{n}}\right)\mathrm{d}x-f(1)\int_0^1 h(x)\mathrm{d}x\right|=\left|\int_0^1 h(x)\left[x^{\frac{1}{n}}f\left(x^{\frac{1}{n}}\right)-f(1)\right]\mathrm{d}x\right|\\
&\leqslant \int_0^{\delta}\left|h(x)\right|\left|x^{\frac{1}{n}}f\left(x^{\frac{1}{n}}\right)-f(1)\right|\mathrm{d}x+\int_{\delta}^1\left|h(x)\right|\left|x^{\frac{1}{n}}f\left(x^{\frac{1}{n}}\right)-f(1)\right|\mathrm{d}x\\
&\leqslant 2M^2\delta+\int_{\delta}^1\left|h(x)\right|\left[\left|x^{\frac{1}{n}}f\left(x^{\frac{1}{n}}\right)-x^{\frac{1}{n}}f(1)\right|+\left|x^{\frac{1}{n}}f(1)-f(1)\right|\right]\mathrm{d}x\\
&=2M^2\delta+\int_{\delta}^1\left|h(x)\right|\left[x^{\frac{1}{n}}\left|f\left(x^{\frac{1}{n}}\right)-f(1)\right|+f(1)\left|x^{\frac{1}{n}}-1\right|\right]\mathrm{d}x\\
&<2M^2\varepsilon+\int_{\varepsilon}^1 M\left[1+f(1)\right]\varepsilon\mathrm{d}x=\left(2M^2+M\left[1+f(1)\right]\left(1-\varepsilon\right)\right)\varepsilon.
\end{align*}
故
\begin{align*}
\lim\limits_{n\rightarrow \infty}\int_0^1 h(x)\cdot x^{\frac{1}{n}}f\left(x^{\frac{1}{n}}\right)\mathrm{d}x=f(1)\int_0^1 h(x)\mathrm{d}x.
\end{align*}
即
\begin{align*}
\lim_{n\rightarrow \infty}n\int_0^1 f(x)g(x^n)\mathrm{d}x=f(1)\int_0^1 \frac{g(x)}{x}\mathrm{d}x.
\end{align*}

{\color{blue}证法二:}因为可以用在两个端点, 插值于 \( f \) 的多项式($f$的Berstein多项式也可以)在 \([0,1]\) 上一致逼近 \( f \), 所以只需对连续可导的函数 \( f \) 证明.

对 \( x \in (0,1] \) 定义 \( G(x) = \int_{0}^{x} \frac{g(t)}{t} \, \mathrm{d}t \), 则 \( G \) 可导, 且 \( G'(x) = \frac{g(x)}{x} \). 因而 \( \left( \frac{1}{n} G(x^n) \right)' = \frac{g(x^n)}{x} \). 用分部积分法, 得
\[
\begin{aligned}
&n \int_{0}^{1} f(x) g(x^n) \, \mathrm{d}x = n \int_{0}^{1} x f(x) \cdot \frac{g(x^n)}{x} \, \mathrm{d}x \\
&= n \left[ x f(x) \cdot \frac{1}{n} G(x^n) \bigg|_{0}^{1} - \int_{0}^{1} \left( f(x) + x f'(x) \right) \frac{1}{n} G(x^n) \, \mathrm{d}x \right] \\
&= f(1) G(1) - \int_{0}^{1} \left( f(x) + x f'(x) \right) G(x^n) \, \mathrm{d}x \\
&= f(1) \int_{0}^{1} \frac{g(x)}{x} \, \mathrm{d}x - \int_{0}^{1} \left( f(x) + x f'(x) \right) G(x^n) \, \mathrm{d}x.
\end{aligned}
\]
因为 \( \lim_{x \to 0^+} \frac{g(x)}{x} \) 收敛, 所以存在 \( M > 0 \), 使得
\[
|f(x) + x f'(x)| \leqslant M, \quad G'(x)=\frac{|g(x)|}{x} \leqslant M \quad (x \in [0,1]).
\]
因此 \( |G(x)| \leqslant M x \).

故
\[
\left| \int_{0}^{1} \left( f(x) + x f'(x) \right) G(x^n) \, \mathrm{d}x \right| \leqslant M^2 \int_{0}^{1} x^n \, \mathrm{d}x = \frac{M^2}{n + 1}\to 0,n\to \infty.
\]
因此
\[
\lim_{n \to +\infty} n \int_{0}^{1} f(x) g(x^n) \, \mathrm{d}x = f(1) \int_{0}^{1} \frac{g(x)}{x} \, \mathrm{d}x.
\]

\end{proof}

\begin{example}
设$f:\mathbb{R}\to\mathbb{R}$,且$f(0)=0$,当$x\neq0$时,$f(x)=\int_0^x\cos\frac{1}{t}\cos\frac{3}{t}\cos\frac{5}{t}\cos\frac{7}{t}\mathrm{d}t$,求证:$f$是可导的,并求$f'(0)$.
\end{example}
\begin{note}
此类问题一般都是利用\hyperref[theorem:Riemann引理]{Riemman引理}解决.
\end{note}
\begin{proof}
由\hyperref[theorem:Riemann引理]{Riemman引理}可知
\begin{align*}
\int_1^{\infty} \frac{\cos nx}{x^2} \mathrm{d}x = \frac{1}{2\pi} \int_0^{2\pi} \cos x \mathrm{d}x \int_1^{\infty} \frac{1}{x^2} \mathrm{d}x = 0, \, \forall n \in \mathbb{N}.
\end{align*}
于是
\begin{align*}
f'(0) &= \lim_{x \to 0} \frac{f(x) - f(0)}{x - 0} = \lim_{x \to 0^+} \frac{1}{x} \int_0^x \cos \frac{1}{t} \cos \frac{3}{t} \cos \frac{5}{t} \cos \frac{7}{t} \mathrm{d}t \\
&\xlongequal{t = \frac{1}{u}} \lim_{x \to 0^+} \frac{1}{x} \int_{\frac{1}{x}}^{\infty} \frac{\cos u \cos 3u \cos 5u \cos 7u}{u^2} \mathrm{d}u = \lim_{\lambda \to +\infty} \lambda \int_{\lambda}^{\infty} \frac{\cos u \cos 3u \cos 5u \cos 7u}{u^2} \mathrm{d}u \\
&\xlongequal{u = \lambda x} \lim_{\lambda \to +\infty} \int_1^{\infty} \frac{\cos(\lambda x) \cos(3\lambda x) \cos(5\lambda x) \cos(7\lambda x)}{x^2} \mathrm{d}x \\
&= \lim_{\lambda \to +\infty} \int_1^{\infty} \frac{\frac{1}{2} \left( \cos(2\lambda x) + \cos(4\lambda x) \right) \cos(5\lambda x) \cos(7\lambda x)}{x^2} \mathrm{d}x \\
&= \lim_{\lambda \to +\infty} \int_1^{\infty} \frac{\frac{1}{4} \left( \cos(3\lambda x) + \cos(7\lambda x) + \cos(9\lambda x) + \cos(\lambda x) \right) \cos(7\lambda x)}{x^2} \mathrm{d}x \\
&= \lim_{\lambda \to +\infty} \int_1^{\infty} \frac{\frac{1}{8} \left[ \cos(16\lambda x) + \cos(14\lambda x) + \cos(10\lambda x) + \cos(6\lambda x) + \cos(4\lambda x) + \cos(2\lambda x) + 1 \right]}{x^2} \mathrm{d}x \\
&= \frac{1}{8} \lim_{\lambda \to +\infty} \int_1^{\infty} \frac{1}{x^2} \mathrm{d}x = \frac{1}{8}.
\end{align*}

\end{proof}

\begin{example}
证明:
\begin{align*}
\lim_{n \to \infty} n \left( n \int_{n\pi}^{2n\pi} \frac{|\sin x|}{x^2} \mathrm{d}x - \frac{1}{\pi^2} \right) = 0.
\end{align*}
\end{example}
\begin{note}
如果需要估计得更精确,就需要利用E-M公式对$\sum_{k=n}^{2n - 1} \frac{1}{(x + k\pi)^2}$进行更精确的估计和计算.
\end{note}
\begin{proof}
注意到
\begin{align*}
\int_{n\pi}^{2n\pi} \frac{|\sin x|}{x^2} \mathrm{d}x &= \int_0^{n\pi} \frac{|\sin x|}{(x + n\pi)^2} \mathrm{d}x = \sum_{k=1}^n \int_0^{k\pi} \frac{|\sin x|}{(x + n\pi)^2} \mathrm{d}x \\
&= \sum_{k=1}^n \int_0^{\pi} \frac{|\sin x|}{(x + (n + k - 1)\pi)^2} \mathrm{d}x = \int_0^{\pi} \sin x \sum_{k=1}^n \frac{1}{(x + (n + k - 1)\pi)^2} \mathrm{d}x \\
&= \int_0^{\pi} \sin x \sum_{k=n}^{2n - 1} \frac{1}{(x + k\pi)^2} \mathrm{d}x.
\end{align*}
对$\forall x \in [0, \pi]$,我们有
\begin{align*}
\sum_{k=n}^{2n - 1} \frac{1}{[(k + 1)\pi]^2} \leqslant  \sum_{k=n}^{2n - 1} \frac{1}{(x + k\pi)^2} \leqslant  \sum_{k=n}^{2n - 1} \frac{1}{(k\pi)^2}.
\end{align*}
又因为
\begin{align*}
\frac{1}{k} - \frac{1}{k + 1} = \frac{1}{k(k + 1)} \leqslant  \frac{1}{k^2} \leqslant  \frac{1}{k(k - 1)} = \frac{1}{k - 1} - \frac{1}{k}, \, \forall k \in \mathbb{N},
\end{align*}
所以一方面,我们有
\begin{align*}
\lim_{n \to \infty} n \int_{n\pi}^{2n\pi} \frac{|\sin x|}{x^2} \mathrm{d}x &= \lim_{n \to \infty} n \int_0^{\pi} \sin x \sum_{k=n}^{2n - 1} \frac{1}{(x + k\pi)^2} \mathrm{d}x \leqslant  \lim_{n \to \infty} n \int_0^{\pi} \sin x \sum_{k=n}^{2n - 1} \frac{1}{(k\pi)^2} \mathrm{d}x \\
&\leqslant  \frac{1}{\pi^2} \lim_{n \to \infty} n \sum_{k=n}^{2n - 1} \left( \frac{1}{k - 1} - \frac{1}{k} \right) \int_0^{\pi} \sin x \mathrm{d}x \\
&= \frac{2}{\pi^2} \lim_{n \to \infty} n \left( \frac{1}{n - 1} - \frac{1}{2n - 1} \right) = \frac{1}{\pi^2}.
\end{align*}
另一方面,我们有
\begin{align*}
\lim_{n \to \infty} n \int_{n\pi}^{2n\pi} \frac{|\sin x|}{x^2} \mathrm{d}x &= \lim_{n \to \infty} n \int_0^{\pi} \sin x \sum_{k=n}^{2n - 1} \frac{1}{(x + k\pi)^2} \mathrm{d}x \geqslant  \lim_{n \to \infty} n \int_0^{\pi} \sin x \sum_{k=n}^{2n - 1} \frac{1}{[(k + 1)\pi]^2} \mathrm{d}x 
\\
&\geqslant  \frac{1}{\pi^2} \lim_{n \to \infty} n \sum_{k=n}^{2n - 1} \left( \frac{1}{k + 1} - \frac{1}{k + 2} \right) \int_0^{\pi} \sin x \mathrm{d}x \\
&= \frac{2}{\pi^2} \lim_{n \to \infty} n \left( \frac{1}{n + 1} - \frac{1}{2n + 1} \right) = \frac{1}{\pi^2}.
\end{align*}
故由夹逼准则可知
\begin{align*}
\lim_{n \to \infty} \left( n \int_{n\pi}^{2n\pi} \frac{|\sin x|}{x^2} \mathrm{d}x - \frac{1}{\pi^2} \right) = 0.
\end{align*}

\end{proof}












\end{document}
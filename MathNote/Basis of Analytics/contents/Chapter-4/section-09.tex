\documentclass[../../main.tex]{subfiles}
\graphicspath{{\subfix{../../image/}}} % 指定图片目录,后续可以直接使用图片文件名。

% 例如:
% \begin{figure}[H]
% \centering
% \includegraphics{image-01.01}
% \caption{图片标题}
% \label{figure:image-01.01}
% \end{figure}
% 注意:上述\label{}一定要放在\caption{}之后,否则引用图片序号会只会显示??.

\begin{document}

\section{Riemann引理}

\begin{lemma}[Riemann引理]\label{lemma:Riemann引理}
设\(E\subset\mathbb{R}\)是区间且\(f\)在\(E\)上绝对可积. \(g\)是定义在\(\mathbb{R}\)的周期\(T > 0\)函数,且在任何有界闭区间上Riemann可积,则我们有
\begin{align}\label{equation:4.1-1.1}
\lim_{x\rightarrow +\infty}\int_{E}f(y)g(xy)dy=\frac{1}{T}\int_{E}f(y)dy\int_{0}^{T}g(y)dy.
\end{align}
\end{lemma}
\begin{remark}
\(f\)在\(E\)上绝对可积包含$f$为反常积分的情况.

考试中,\hyperref[lemma:Riemann引理]{Riemann引理}不能直接使用,需要我们根据具体问题给出证明.具体可见\hyperref[example:4.611315]{例题\ref{example:4.611315}}.
\end{remark}
\begin{note}
\begin{enumerate}[(1)]
\item\label{example-note:不妨设的原因(1)} 不妨设 \(E = \mathbb{R}\) 的原因:若 (1.1) 式在 \(E = \mathbb{R}\) 时已得证明,则当 \(E\subseteq \mathbb{R}\) 时,令 \(\widetilde{f}(y) = f(y)\cdot \mathcal{X}_E\),\(y\in \mathbb{R}\),则由 \(f(y)\) 在 \(E\) 上绝对可积,可得 \(\widetilde{f}(y)\) 在 \(\mathbb{R}\) 上也绝对可积.从而由假设可知
\[
\lim_{x\rightarrow +\infty} \int_{\mathbb{R}}{\widetilde{f}(y)g(xy)dy}=\frac{1}{T}\int_{\mathbb{R}}{\widetilde{f}(y)dy\int_0^T{g(y)dy}}.
\]
于是
\begin{align*}
\lim_{x\rightarrow +\infty} \int_E{f(y)g(xy)dy}=\lim_{x\rightarrow +\infty} \int_{\mathbb{R}}{\widetilde{f}(y)g(xy)dy}
=\frac{1}{T}\int_{\mathbb{R}}{\widetilde{f}(y)dy\int_0^T{g(y)dy}}
=\frac{1}{T}\int_E{f(y)dy\int_0^T{g(y)dy}}
\end{align*}
故可以不妨设 \(E = \mathbb{R}\).

\item\label{example-note:不妨设的原因(2)} 不妨设 \(\sup_{\mathbb{R}}|g| > 0\) 的原因:若 \(\sup_{\mathbb{R}}|g| = 0\),则 \(g(x)\equiv 0\),此时结论显然成立.因此我们只需要考虑当 \(\sup_{\mathbb{R}}|g| > 0\) 时的情况.

\item\label{example-note:不妨设的原因(3)} 不妨设 \(T = 1\) 的原因:若 \eqref{equation:4.1-1.1} 式在 \(T = 1\) 时已得证明,则当 \(T\neq 1\) 时,有
\begin{align}\label{equation:4.1-1.2} 
\frac{1}{T}\int_E{f(y)dy\int_0^T{g(y)dy}}\xlongequal{\text{令}y = Tx}\int_E{f(y)dy\int_0^1{g(Tx)\mathrm{d}x}}=\int_E{f(y)dy\int_0^1{g(Ty)dy}}.
\end{align}
由于 \(g(y)\) 是 \(\mathbb{R}\) 上周期为 \(T\neq 1\) 的函数,因此 \(g(Ty)\) 就是 \(\mathbb{R}\) 上周期为 \(1\) 的函数.从而由假设可知
\begin{align}\label{equation:4.1-1.3} 
\lim_{x\rightarrow +\infty} \int_E{f(y)g(Txy)dy}=\int_E{f(y)dy\int_0^1{g(Ty)dy}}.
\end{align}
又由\eqref{equation:4.1-1.2} 式及 \(T > 0\) 可得
\begin{align*}
\int_E{f(y)dy\int_0^1{g(Ty)dy}}&=\frac{1}{T}\int_E{f(y)dy\int_0^T{g(y)dy}}\\
\lim_{x\rightarrow +\infty} \int_E{f(y)g(Txy)dy}&\xlongequal{\text{令}t = Tx}\lim_{t\rightarrow +\infty} \int_E{f(y)g(ty)dy}=\lim_{x\rightarrow +\infty} \int_E{f(y)g(xy)dy}
\end{align*}
再结合\eqref{equation:4.1-1.3}式可得 \(\lim_{x\rightarrow +\infty} \int_E{f(y)g(xy)dy}=\frac{1}{T}\int_E{f(y)dy\int_0^T{g(y)dy}}\).故可以不妨设 \(T = 1\).

\item\label{example-note:不妨设的原因(4)} 不妨设 \(\int_0^1{g(y)dy} = 0\) 的原因:若 \eqref{equation:4.1-1.1} 式在 \(\int_0^1{g(y)dy} = 0\) 时已得证明,则当 \(\int_0^1{g(y)dy}\neq 0\) 时,
令 \(G(y) = g(y) - \int_0^1{g(t)dt}\),则 \(G(y)\) 是 \(\mathbb{R}\) 上周期为 \(1\) 的函数,并且 \(\int_0^1{G(y)dy} = 0\).于是由假设可知
\begin{align*}
&\,\,\,\,\,\,\,\, \lim_{x\rightarrow +\infty} \int_E{f(y)G(xy)dy}=\int_E{f(y)dy\int_0^1{G(y)dy}}\\
&\Leftrightarrow \lim_{x\rightarrow +\infty} \int_E{f(y)\left[ g(xy) - \int_0^1{g(t)dt} \right] dy}=\int_E{f(y)dy\int_0^1{\left[ g(y) - \int_0^1{g(t)dt} \right] dy}}\\
&\Leftrightarrow \lim_{x\rightarrow +\infty} \left(\int_E{f(y)g(xy)dy}-\int_E{f(y)\int_0^1{g(t)dt}dy}\right)=\int_E{f(y)dy\int_0^1{g(y)dy}}-\int_E{f(y)dy}\int_0^1{g(t)dt}=0\\
&\Leftrightarrow \lim_{x\rightarrow +\infty} \int_E{f(y)g(xy)dy}=\int_E{f(y)\int_0^1{g(t)dt}dy}
\end{align*}
再结合\hyperref[example-note:不妨设的原因(3)]{\ref{example-note:不妨设的原因(2)}}可知,此时原结论成立.故可以不妨设 \(\int_0^1{g(y)dy} = 0\).
\end{enumerate}
\end{note}
\begin{proof}
\hyperref[example-note:不妨设的原因(1)]{不妨设 \(E = \mathbb{R}\)},\hyperref[example-note:不妨设的原因(2)]{\(\sup_{\mathbb{R}}|g| > 0\)},\hyperref[example-note:不妨设的原因(3)]{\(T = 1\)},\hyperref[example-note:不妨设的原因(4)]{再不妨设 \(\int_0^1{g(y)dy} = 0\)}.因此只需证 \(\lim_{x\rightarrow +\infty} \int_{\mathbb{R}}{f(y)g(xy)dy} = 0\).由 \(g\) 的周期为 \(1\) 及 \(\int_0^1{g(y)dy} = 0\) 可得,对 \(\forall n\in \mathbb{N}\),都有
\begin{align*}
&\int_{-n}^0{g(t)dt}\xlongequal{\text{令}x = t + n}\int_0^n{g(x - n)\mathrm{d}x}\xlongequal{g\text{的周期为}1}\int_0^n{g(x)\mathrm{d}x}=\int_0^n{g(t)dt}\\
&=\sum_{k = 0}^{n - 1}{\int_k^{k + 1}{g(t)dt}}\xlongequal{\text{令}y = t - k}\sum_{k = 0}^{n - 1}{\int_0^1{g(y + k)dy}}\xlongequal{g\text{的周期为}1}\sum_{k = 0}^{n - 1}{\int_0^1{g(y)dy}}\\
&=(n - 1)\cdot 0 = 0.
\end{align*}
从而对 \(\forall \beta > \alpha > 0\),我们有
\begin{align*}
\left|\int_{\alpha}^{\beta}{g(t)dt}\right|&=\left|\int_0^{\beta}{g(t)dt}-\int_0^{\alpha}{g(t)dt}\right|
=\left|\int_{-[ \beta ]}^{\beta - [\beta ]}{g(t + [\beta ])dt}-\int_{-[ \alpha ]}^{\alpha - [\alpha ]}{g(t + [\alpha ])dt}\right|\\
&=\left|\int_{-[ \beta ]}^{\beta - [\beta ]}{g(t)dt}-\int_{-[ \alpha ]}^{\alpha - [\alpha ]}{g(t)dt}\right|
=\left|\int_0^{\beta - [\beta ]}{g(t)dt}-\int_0^{\alpha - [\alpha ]}{g(t)dt}\right|\\
&=\left|\int_{\alpha - [\alpha ]}^{\beta - [\beta ]}{g(t)dt}\right|\leqslant \sup_{\mathbb{R}}|g|.
\end{align*}
故
\begin{align}\label{equation:4.1-2.1} 
\left|\int_{\alpha}^{\beta}{g(xy)dy}\right|\xlongequal{\text{令}t = xy}\frac{1}{x}\left|\int_{x\alpha}^{x\beta}{g(t)dt}\right|\leqslant \frac{\sup\limits_{\mathbb{R}}|g|}{x}, \quad \forall x > 0, \forall \beta > \alpha > 0.  
\end{align}
因为 \(f\) 在 \(\mathbb{R}\) 上绝对可积,所以由 Cauchy 收敛准则可知,对 \(\forall \varepsilon > 0\),存在 \(N\in \mathbb{N}\),使得
\begin{align}\label{equation:4.1-2.2} 
\left|\int_{|y| > N}{f(y)dy}\right| < \frac{\varepsilon}{3\sup\limits_{\mathbb{R}}|g|}. 
\end{align}
由于 \(f\) 在 \(\mathbb{R}\) 上绝对可积,从而 \(f\) 在 \(\mathbb{R}\) 上也 Riemann 可积,因此由可积的充要条件可知,存在划分
\[
-N = t_0 < t_1 < t_2 < \cdots<t_n = N,
\]
使得
\begin{align}\label{equation:4.1-2.3} 
\sum_{i = 1}^n{\left(\sup_{[t_{i - 1},t_i]}f - \inf_{[t_{i - 1},t_i]}f\right)(t_i - t_{i - 1})}\leqslant \frac{\varepsilon}{3\sup\limits_{\mathbb{R}}|g|}.
\end{align}
于是当 \(x > \frac{3\sum\limits_{j = 1}^n{|\inf\limits_{[t_{j - 1},t_j]}f|\cdot}\sup\limits_{\mathbb{R}}|g|}{\varepsilon}\) 时,结合\eqref{equation:4.1-2.1}\eqref{equation:4.1-2.2}\eqref{equation:4.1-2.3}可得
\begin{align*}
\left|\int_{-\infty}^{+\infty}{f(y)g(xy)dy}\right|&\leqslant \left|\int_{-N}^N{f(y)g(xy)dy}\right|+\left|\int_{|y| > N}{f(y)g(xy)dy}\right|
\overset{\eqref{equation:4.1-2.2}}{\leqslant}\left|\sum_{j = 1}^n{\int_{t_{j - 1}}^{t_j}{f(y)g(xy)dy}}\right|+\frac{\varepsilon}{3\sup\limits_{\mathbb{R}}|g|}\cdot \sup\limits_{\mathbb{R}}|g|\\
&\leqslant \sum_{j = 1}^n{\left|\int_{t_{j - 1}}^{t_j}{[f(y) - \inf_{[t_{j - 1},t_j]}f]g(xy)dy}\right|}+\sum_{j = 1}^n{\left|\int_{t_{j - 1}}^{t_j}{\inf_{[t_{j - 1},t_j]}f\cdot g(xy)dy}\right|}+\frac{\varepsilon}{3}\\
&\overset{\eqref{equation:4.1-2.1}}{\leqslant}\sum_{j = 1}^n{\int_{t_{j - 1}}^{t_j}{[f(y) - \inf_{[t_{j - 1},t_j]}f]dy}}\cdot \sup\limits_{\mathbb{R}}|g|+\frac{\sup\limits_{\mathbb{R}}|g|}{x}\sum_{j = 1}^n{\int_{t_{j - 1}}^{t_j}{|\inf_{[t_{j - 1},t_j]}f|dy}}+\frac{\varepsilon}{3}\\
&\leqslant \sum_{j = 1}^n{\int_{t_{j - 1}}^{t_j}{(\sup\limits_{[t_{i - 1},t_i]}f - \inf_{[t_{j - 1},t_j]}f)dy}}\cdot \sup\limits_{\mathbb{R}}|g|+\frac{\sup\limits_{\mathbb{R}}|g|}{x}\sum_{j = 1}^n{\int_{t_{j - 1}}^{t_j}{|\inf_{[t_{j - 1},t_j]}f|dy}}+\frac{\varepsilon}{3}\\
&=\sum_{j = 1}^n{(\sup\limits_{[t_{i - 1},t_i]}f - \inf_{[t_{j - 1},t_j]}f)(t_j - t_{j - 1})}\cdot \sup\limits_{\mathbb{R}}|g|+\frac{\sup\limits_{\mathbb{R}}|g|}{x}\sum_{j = 1}^n{\int_{t_{j - 1}}^{t_j}{|\inf_{[t_{j - 1},t_j]}f|dy}}+\frac{\varepsilon}{3}\\
&\overset{\eqref{equation:4.1-2.3}}{<}\frac{\varepsilon}{3\sup\limits_{\mathbb{R}}|g|}\cdot \sup\limits_{\mathbb{R}}|g|+\frac{\sup\limits_{\mathbb{R}}|g|}{x}\sum_{j = 1}^n{\int_{t_{j - 1}}^{t_j}{|\inf_{[t_{j - 1},t_j]}f|dy}}+\frac{\varepsilon}{3}\\
&\overset{x\text{充分大}}{<}\frac{\varepsilon}{3}+\frac{\varepsilon}{3}+\frac{\varepsilon}{3}=\varepsilon
\end{align*}
因此 \(\lim_{x\rightarrow +\infty} \int_{\mathbb{R}}{f(y)g(xy)dy} = 0\).结论得证.
\end{proof}

\begin{example}\label{example:4.611315}
设\(f\in R[0,2\pi]\),不直接使用\hyperref[lemma:Riemann引理]{Riemann引理}计算
\[
\lim_{n\rightarrow\infty}\int_{0}^{2\pi}f(x)|\sin(nx)|\mathrm{d}x.
\]
\end{example}
\begin{proof}
对\(\forall n\in \mathbb{N}_+\),固定\(n\).将\([0, 2\pi]\)等分成\(2n\)段,记这个划分为
\[
T:0 = t_0 < t_1 < \cdots < t_{2n} = 2\pi,
\]
其中\(t_i = \frac{i\pi}{n}, i = 0, 1, \cdots, n\).此时我们有
\begin{align}\label{example4.61-1.1}
\int_{t_{i - 1}}^{t_i}{|\sin(nx)|\mathrm{d}x}=\int_{\frac{(i - 1)\pi}{n}}^{\frac{i\pi}{n}}{|\sin(nx)|\mathrm{d}x}=\frac{1}{n}\int_{(i - 1)\pi}^{i\pi}{|\sin x|\mathrm{d}x}=\frac{2}{n}. 
\end{align}
由\(f\in R[0, 2\pi]\)可知,\(f\)在\([0, 2\pi]\)上有界也内闭有界.从而利用\eqref{example4.61-1.1}式可知,对\(\forall n\in \mathbb{N}_+\),一方面,我们有
\begin{align}
\int_0^{2\pi}{f\left( x \right) \left| \sin \left( nx \right) \right|\mathrm{d}x}&=\sum_{i=1}^{2n}{\int_{t_{i-1}}^{t_i}{f\left( x \right) \left| \sin \left( nx \right) \right|\mathrm{d}x}}\leqslant \sum_{i=1}^{2n}{\int_{t_{i-1}}^{t_i}{\underset{\left[ t_{i-1},t_i \right]}{\mathrm{sup}}f\cdot \left| \sin \left( nx \right) \right|\mathrm{d}x}}\xlongequal{\eqref{example4.61-1.1} \text{式}}\frac{2}{n}\sum_{i=1}^{2n}{\underset{\left[ t_{i-1},t_i \right]}{\mathrm{sup}}f}
\nonumber
\\
&=\frac{2}{n}\sum_{i=1}^{2n}{\underset{\left[ t_{i-1},t_i \right]}{\mathrm{sup}}f}=\frac{2}{\pi}\sum_{i=1}^{2n}{\underset{\left[ t_{i-1},t_i \right]}{\mathrm{sup}}f\cdot \frac{\pi}{n}}=\frac{2}{\pi}\sum_{i=1}^{2n}{\underset{\left[ t_{i-1},t_i \right]}{\mathrm{sup}}f\cdot \left( t_i-t_{i-1} \right)}.  \label{example4.61-1.2}
\end{align}
另一方面,我们有
\begin{align}
\int_0^{2\pi}{f\left( x \right) \left| \sin \left( nx \right) \right|\mathrm{d}x}&=\sum_{i=1}^{2n}{\int_{t_{i-1}}^{t_i}{f\left( x \right) \left| \sin \left( nx \right) \right|\mathrm{d}x}}\geqslant \sum_{i=1}^{2n}{\int_{t_{i-1}}^{t_i}{\underset{\left[ t_{i-1},t_i \right]}{\mathrm{inf}}f\cdot \left| \sin \left( nx \right) \right|\mathrm{d}x}}\xlongequal{\eqref{example4.61-1.1}\text{式}}\frac{2}{n}\sum_{i=1}^{2n}{\underset{\left[ t_{i-1},t_i \right]}{\mathrm{inf}}f}
\nonumber
\\
&=\frac{2}{\pi}\sum_{i=1}^{2n}{\underset{\left[ t_{i-1},t_i \right]}{\mathrm{inf}}f\cdot \frac{\pi}{n}}=\frac{2}{\pi}\sum_{i=1}^{2n}{\underset{\left[ t_{i-1},t_i \right]}{\mathrm{inf}}f\cdot \left( t_i-t_{i-1} \right)}.\label{example4.61-1.3}
\end{align}
由\(f\in R[0, 2\pi]\)和Riemann可积的充要条件可知
\[
\int_0^{2\pi}{f(x)\mathrm{d}x}=\lim_{n\rightarrow \infty}\sum_{i = 1}^{2n}{\sup_{[t_{i - 1}, t_i]}f\cdot(t_i - t_{i - 1})}=\lim_{n\rightarrow \infty}\sum_{i = 1}^{2n}{\inf_{[t_{i - 1}, t_i]}f\cdot(t_i - t_{i - 1})}.
\]
于是对\eqref{example4.61-1.2}\eqref{example4.61-1.3}式两边同时令\(n\rightarrow \infty\),得到
\[
\lim_{n\rightarrow \infty}\int_0^{2\pi}{f(x)|\sin(nx)|\mathrm{d}x}=\frac{2}{\pi}\int_0^{2\pi}{f(x)\mathrm{d}x}.
\]
\end{proof}

\begin{example}
设\(f\)是\(\mathbb{R}\)上周期\(2\pi\)函数且在\([-\pi,\pi]\)上Riemann可积,设
\[
S_n(x)=\frac{1}{\pi}\int_{-\pi}^{\pi}\frac{f(x + t)}{2\sin\frac{t}{2}}\sin\left(\frac{2n + 1}{2}t\right)\mathrm{d}t,n = 1,2,\cdots.
\]
若\(x_0\in(-\pi,\pi)\)是\(f\)在\([-\pi,\pi]\)唯一间断点且存在下述极限
\[
A=\lim_{x\rightarrow x_0^{+}}f(x),B=\lim_{x\rightarrow x_0^{-}}f(x),\lim_{x\rightarrow x_0^{+}}\frac{f(x)-A}{x - x_0},\lim_{x\rightarrow x_0^{-}}\frac{f(x)-B}{x - x_0}.
\]
证明:
\[
\lim\limits_{n\rightarrow\infty}S_n(x_0)=\frac{\lim\limits_{x\rightarrow x_0^{+}}f(x)+\lim\limits_{x\rightarrow x_0^{-}}f(x)}{2}.
\]
\end{example}
\begin{note}
\begin{enumerate}[(1)]
\item 计算 \(I_1=\frac{1}{\pi}\int_0^{\pi}{\frac{f(x_0 + t)}{2\sin \frac{t}{2}}\sin \left( \frac{2n + 1}{2}t \right) \mathrm{d}t}\) 的思路:由于 \(\frac{f(x_0 + t)}{2\sin \frac{t}{2}}\) 在 \([0, \pi]\) 上只可能有奇点 \(t = 0\),因此 \(\frac{f(x_0 + t)}{2\sin \frac{t}{2}}\) 在 \([0, \pi]\) 上不一定绝对可积.从而不能直接利用Riemann引理.于是我们需要将 \(\frac{f(x_0 + t)}{2\sin \frac{t}{2}}\) 转化为在 \([0, \pi]\) 上无奇点的函数(排除 \(t = 0\) 这个奇点,即证明 \(t = 0\) 不再是奇点),只要被积函数在积分区间上无奇点且Riemann可积,就一定绝对可积.进而满足Riemann引理的条件,再利用Riemann引理就能求解出 \(I_1\).具体处理方式见下述证明.

计算 \(I_2=\frac{1}{\pi}\int_0^{\pi}{\frac{f(x_0 - t)}{2\sin \frac{t}{2}}\sin \left( \frac{2n + 1}{2}t \right) \mathrm{d}t}\) 的思路同理,也是要排除 \(t = 0\) 这个可能的奇点,再利用Riemann引理进行求解.具体计算方式见下述证明.

\item 计算 \(\lim_{n\rightarrow \infty} \int_0^{\pi}{\frac{1}{2\sin \frac{t}{2}}\sin \left( \frac{2n + 1}{2}t \right) \mathrm{d}t}\) 的思路:
注意由于 \(\frac{1}{2\sin \frac{t}{2}}\) 在 \([0, \pi]\) 上有一个奇点 \(t = 0\),并且对 \(\forall t\in (0, \pi]\),都有
\[
\left| \frac{1}{2\sin \frac{t}{2}} \right|\geqslant \left| \frac{1}{2\cdot \frac{2}{\pi}\cdot \frac{t}{2}} \right|=\frac{\pi}{2t}>0.
\]
而 \(\int_0^{\pi}{\frac{\pi}{2t}\mathrm{d}t}\) 是发散的,故 \(\int_0^{\pi}{\left| \frac{1}{2\sin \frac{t}{2}} \right|\mathrm{d}t}\) 也发散.因此 \(\frac{1}{2\sin \frac{t}{2}}\) 在 \([0, \pi]\) 上一定不是绝对可积的,从而不能利用Riemann引理计算 \(\lim_{n\rightarrow \infty} \int_0^{\pi}{\frac{1}{2\sin \frac{t}{2}}\sin \left( \frac{2n + 1}{2}t \right) \mathrm{d}t}\).真正能计算 \(\lim_{n\rightarrow \infty} \int_0^{\pi}{\frac{1}{2\sin \frac{t}{2}}\sin \left( \frac{2n + 1}{2}t \right) \mathrm{d}t}\) 的方法有多种,下述证明利用的是\hyperref[强行替换(拟合法)和凑定积分]{强行替换/拟合法}.
\end{enumerate}
\end{note}
\begin{proof}
注意到
\begin{align}
S_n(x_0)&=\frac{1}{\pi}\int_{-\pi}^{\pi}{\frac{f(x_0 + t)}{2\sin \frac{t}{2}}\sin \left( \frac{2n + 1}{2}t \right) \mathrm{d}t}
\nonumber
\\
&=\frac{1}{\pi}\int_0^{\pi}{\frac{f(x_0 + t)}{2\sin \frac{t}{2}}\sin \left( \frac{2n + 1}{2}t \right) \mathrm{d}t}+\frac{1}{\pi}\int_{-\pi}^0{\frac{f(x_0 + t)}{2\sin \frac{t}{2}}\sin \left( \frac{2n + 1}{2}t \right) \mathrm{d}t}
\nonumber
\\
&\xlongequal{\text{令}y = -t}\frac{1}{\pi}\int_0^{\pi}{\frac{f(x_0 + t)}{2\sin \frac{t}{2}}\sin \left( \frac{2n + 1}{2}t \right) \mathrm{d}t}+\frac{1}{\pi}\int_0^{\pi}{\frac{f(x_0 - t)}{2\sin \frac{t}{2}}\sin \left( \frac{2n + 1}{2}t \right) \mathrm{d}t}\label{example4.59-1.0}
\end{align}
记 \(I_1=\frac{1}{\pi}\int_0^{\pi}{\frac{f(x_0 + t)}{2\sin \frac{t}{2}}\sin \left( \frac{2n + 1}{2}t \right) \mathrm{d}t}\),\(I_2=\frac{1}{\pi}\int_0^{\pi}{\frac{f(x_0 - t)}{2\sin \frac{t}{2}}\sin \left( \frac{2n + 1}{2}t \right) \mathrm{d}t}\),则由\eqref{example4.59-1.0}式可得
\begin{align}\label{example4.59-1.1}
\lim_{n\rightarrow \infty} S_n(x_0)=\lim_{n\rightarrow \infty} (I_1 + I_2).
\end{align}
于是
\begin{align}
I_1&=\frac{1}{\pi}\int_0^{\pi}{\frac{f(x_0 + t) - A}{2\sin \frac{t}{2}}\sin \left( \frac{2n + 1}{2}t \right) \mathrm{d}t}+\frac{A}{\pi}\int_0^{\pi}{\frac{1}{2\sin \frac{t}{2}}\sin \left( \frac{2n + 1}{2}t \right) \mathrm{d}t},\label{example4.59-1.2}
\\
I_2&=\frac{1}{\pi}\int_0^{\pi}{\frac{f(x_0 - t) - B}{2\sin \frac{t}{2}}\sin \left( \frac{2n + 1}{2}t \right) \mathrm{d}t}+\frac{B}{\pi}\int_0^{\pi}{\frac{1}{2\sin \frac{t}{2}}\sin \left( \frac{2n + 1}{2}t \right) \mathrm{d}t}.\label{example4.59-1.2-0}
\end{align}
由条件可知 \(\lim_{t\rightarrow 0^+} \frac{f(x_0 + t) - A}{2\sin \frac{t}{2}}=\lim_{t\rightarrow 0^+} \frac{f(x_0 + t) - A}{t}=\lim_{x\rightarrow x_{0}^{+}} \frac{f(x) - A}{x - x_0}\) 存在,\(\lim_{t\rightarrow 0^-} \frac{f(x_0 - t) - B}{2\sin \frac{t}{2}}=\lim_{t\rightarrow 0^-} \frac{f(x_0 - t) - B}{t}=\lim_{x\rightarrow x_{0}^{-}} \frac{f(x) - B}{x - x_0}\) 存在,因此 \(\frac{f(x_0 + t) - A}{2\sin \frac{t}{2}}\),\(\frac{f(x_0 - t) - B}{2\sin \frac{t}{2}}\) 在 \([0, \pi]\) 都没有奇点且Riemann可积,从而 \(\lim_{n\rightarrow \infty} \int_0^{\pi}{\frac{f(x_0 + t) - A}{2\sin \frac{t}{2}}\sin \left( \frac{2n + 1}{2}t \right) \mathrm{d}t}\),\(\lim_{n\rightarrow \infty} \int_0^{\pi}{\frac{f(x_0 - t) - B}{2\sin \frac{t}{2}}\sin \left( \frac{2n + 1}{2}t \right) \mathrm{d}t}\) 都满足Riemann引理的条件.于是由Riemann引理可得
\begin{align}\label{example4.59-1.3}
\lim_{n\rightarrow \infty} \frac{1}{\pi}\int_0^{\pi}{\frac{f(x_0 + t) - A}{2\sin \frac{t}{2}}\sin \left( \frac{2n + 1}{2}t \right) \mathrm{d}t}=0, \quad \lim_{n\rightarrow \infty} \frac{1}{\pi}\int_0^{\pi}{\frac{f(x_0 - t) - B}{2\sin \frac{t}{2}}\sin \left( \frac{2n + 1}{2}t \right) \mathrm{d}t}=0.
\end{align}
下面计算 \(\lim_{n\rightarrow \infty} \int_0^{\pi}{\frac{1}{2\sin \frac{t}{2}}\sin \left( \frac{2n + 1}{2}t \right) \mathrm{d}t}\).
\begin{align}\label{example4.59-1.4}
\left| \int_0^{\pi}{\frac{1}{2\sin \frac{t}{2}}\sin \left( \frac{2n + 1}{2}t \right) \mathrm{d}t}-\int_0^{\pi}{\frac{1}{t}\sin \left( \frac{2n + 1}{2}t \right) \mathrm{d}t} \right|=\left| \int_0^{\pi}{\frac{t - 2\sin \frac{t}{2}}{2t\sin \frac{t}{2}}\sin \left( \frac{2n + 1}{2}t \right) \mathrm{d}t} \right|.
\end{align}
而 \(\lim_{t\rightarrow 0} \frac{t - 2\sin \frac{t}{2}}{2t\sin \frac{t}{2}}=\lim_{t\rightarrow 0} \frac{t - 2\sin \frac{t}{2}}{t^2}\xlongequal{\mathrm{L}'\mathrm{Hospital}'\mathrm{rules}}\lim_{t\rightarrow 0} \frac{1 - \cos \frac{t}{2}}{2t}=0\),因此 \(\frac{t - 2\sin \frac{t}{2}}{2t\sin \frac{t}{2}}\) 在 \([0, \pi]\) 上无奇点且Riemann可积,从而由Riemann引理可知 \(\lim_{n\rightarrow \infty} \int_0^{\pi}{\frac{t - 2\sin \frac{t}{2}}{2t\sin \frac{t}{2}}\sin \left( \frac{2n + 1}{2}t \right) \mathrm{d}t}=0\).于是再结合 \eqref{example4.59-1.4} 式可得
\begin{align}\label{example4.59-1.5}
\lim_{n\rightarrow \infty} \int_0^{\pi}{\frac{1}{2\sin \frac{t}{2}}\sin \left( \frac{2n + 1}{2}t \right) \mathrm{d}t}&=\lim_{n\rightarrow \infty} \int_0^{\pi}{\frac{1}{t}\sin \left( \frac{2n + 1}{2}t \right) \mathrm{d}t}
=\lim_{n\rightarrow \infty} \int_0^{\frac{2n + 1}{2}\pi}{\frac{\sin t}{t}\mathrm{d}t}
=\int_0^{+\infty}{\frac{\sin t}{t}\mathrm{d}t}=\frac{\pi}{2}.
\end{align}
因此,由 \eqref{example4.59-1.2}\eqref{example4.59-1.2-0}\eqref{example4.59-1.3}\eqref{example4.59-1.5}式可得
\begin{align*}
\lim_{n\rightarrow \infty} I_1&=\lim_{n\rightarrow \infty} \frac{1}{\pi}\int_0^{\pi}{\frac{f(x_0 + t) - A}{2\sin \frac{t}{2}}\sin \left( \frac{2n + 1}{2}t \right) \mathrm{d}t}+\lim_{n\rightarrow \infty} \frac{A}{\pi}\int_0^{\pi}{\frac{1}{2\sin \frac{t}{2}}\sin \left( \frac{2n + 1}{2}t \right) \mathrm{d}t}
=0 + \frac{A}{\pi}\cdot \frac{\pi}{2}
=\frac{A}{2},
\end{align*}

\begin{align*}
\lim_{n\rightarrow \infty} I_2&=\lim_{n\rightarrow \infty} \frac{1}{\pi}\int_0^{\pi}{\frac{f(x_0 - t) - B}{2\sin \frac{t}{2}}\sin \left( \frac{2n + 1}{2}t \right) \mathrm{d}t}+\lim_{n\rightarrow \infty} \frac{B}{\pi}\int_0^{\pi}{\frac{1}{2\sin \frac{t}{2}}\sin \left( \frac{2n + 1}{2}t \right) \mathrm{d}t}
=0 + \frac{B}{\pi}\cdot \frac{\pi}{2}
=\frac{B}{2}.
\end{align*}
再结合 \eqref{example4.59-1.1}式可得
\begin{align*}
\lim_{n\rightarrow \infty} S_n(x_0)=\lim_{n\rightarrow \infty} (I_1 + I_2)
=\lim_{n\rightarrow \infty} I_1 + \lim_{n\rightarrow \infty} I_2
=\frac{A + B}{2}.
\end{align*}
\end{proof}

\begin{example}
设\(f\in C^{1}[0,\frac{\pi}{2}],f(0)=0\),计算
\[
\lim_{n\rightarrow\infty}\frac{1}{\ln n}\int_{0}^{\frac{\pi}{2}}\frac{\sin^{2}(nx)}{\sin^{2}x}f(x)\mathrm{d}x.
\]
\end{example}
\begin{remark}
由于\(x = 0\)可能是\(\frac{f(x)}{\sin^2x}\)在\(\left[0, \frac{\pi}{2}\right]\)上的奇点,因此我们需要将其转化为在\(\left[0, \frac{\pi}{2}\right]\)上不含奇点的函数,才能利用\hyperref[lemma:Riemann引理]{Riemann引理}进行计算.
\end{remark}
\begin{proof}
注意到
\begin{align}\label{example4.62-1.1}
\frac{1}{\ln n}\int_0^{\frac{\pi}{2}}{\frac{f(x)}{\sin^2x}\sin^2(nx)\mathrm{d}x}=\frac{1}{\ln n}\int_0^{\frac{\pi}{2}}{\frac{f(x) - f^\prime(0)x}{\sin^2x}\sin^2(nx)\mathrm{d}x}+\frac{1}{\ln n}\int_0^{\frac{\pi}{2}}{\frac{f^\prime(0)x}{\sin^2x}\sin^2(nx)\mathrm{d}x}. 
\end{align}
先计算\(\lim_{n\rightarrow \infty} \frac{1}{\ln n}\int_0^{\frac{\pi}{2}}{\frac{f(x) - f^\prime(0)x}{\sin^2x}\sin^2(nx)\mathrm{d}x}\).
由\(f\in C^1\left[0, \frac{\pi}{2}\right]\)可知,\(f\in D^2\left[0, \frac{\pi}{2}\right]\).从而由L'Hospital法则可知
\[
\lim_{x\rightarrow 0^+} \frac{f(x) - f^\prime(0)x}{\sin^2x}=\lim_{x\rightarrow 0^+} \frac{f^\prime(x) - f^\prime(0)}{2\sin x\cos x}=\frac{1}{2}\lim_{x\rightarrow 0^+} \frac{f^\prime(x) - f^\prime(0)}{x}=\frac{f^{\prime\prime}(0)}{2}.
\]
于是\(\frac{f(x) - f^\prime(0)x}{\sin^2x}\)在\(\left[0, \frac{\pi}{2}\right]\)上无奇点且Riemann可积,从而绝对可积.故由\hyperref[lemma:Riemann引理]{Riemann引理}可得
\begin{align}
\lim_{n\rightarrow \infty} \int_0^{\frac{\pi}{2}}{\frac{f(x) - f^\prime(0)x}{\sin^2x}\sin^2(nx)\mathrm{d}x}&=\lim_{n\rightarrow \infty} \frac{1}{\pi}\int_0^{\frac{\pi}{2}}{\frac{f(x) - f^\prime(0)x}{\sin^2x}\mathrm{d}x}\int_0^{\pi}{\sin^2x\mathrm{d}x}
\nonumber
\\
&=\lim_{n\rightarrow \infty} \frac{1}{2}\int_0^{\frac{\pi}{2}}{\frac{f(x) - f^\prime(0)x}{\sin^2x}\mathrm{d}x}<\infty. \label{example4.62-1.2}
\end{align}
利用\eqref{example4.62-1.2}式可得
\begin{align}\label{example4.62-1.3}
\lim_{n\rightarrow \infty} \frac{1}{\ln n}\int_0^{\frac{\pi}{2}}{\frac{f(x) - f^\prime(0)x}{\sin^2x}\sin^2(nx)\mathrm{d}x}=\lim_{n\rightarrow \infty} \frac{1}{\ln n}\cdot \lim_{n\rightarrow \infty} \int_0^{\frac{\pi}{2}}{\frac{f(x) - f^\prime(0)x}{\sin^2x}\sin^2(nx)\mathrm{d}x}=0.  
\end{align}
下面计算\(\lim_{n\rightarrow \infty} \frac{1}{\ln n}\int_0^{\frac{\pi}{2}}{\frac{f^\prime(0)x}{\sin^2x}\sin^2(nx)\mathrm{d}x}\).
对\(\forall n\in \mathbb{N}_+\),我们有
\begin{align}\label{example4.62-1.4}
\left|\frac{1}{\ln n}\int_0^{\frac{\pi}{2}}{\frac{f^\prime(0)x}{\sin^2x}\sin^2(nx)\mathrm{d}x}-\frac{1}{\ln n}\int_0^{\frac{\pi}{2}}{\frac{f^\prime(0)}{x}\sin^2(nx)\mathrm{d}x}\right|=\left|\frac{f^\prime(0)}{\ln n}\int_0^{\frac{\pi}{2}}{\frac{x^2 - \sin^2x}{x\sin^2x}\cdot \sin^2(nx)\mathrm{d}x}\right|.
\end{align}
又\(\lim_{x\rightarrow 0^+} \frac{x^2 - \sin^2x}{x\sin^2x}=\lim_{x\rightarrow 0^+} \frac{x^2 - \left(x - \frac{x^3}{6} + o(x^3)\right)^2}{x^3}=\lim_{x\rightarrow 0^+} \frac{-\frac{x^3}{3} + o(x^3)}{x^3}=-\frac{1}{3}\),故\(\frac{x^2 - \sin^2x}{x\sin^2x}\)在\(\left[0, \frac{\pi}{2}\right]\)上无奇点且Riemann可积,从而绝对可积.
于是由\hyperref[lemma:Riemann引理]{Riemann引理}可得
\begin{align}\label{example4.62-1.5}
\lim_{n\rightarrow \infty} f^\prime(0) \int_0^{\frac{\pi}{2}}{\frac{x^2 - \sin^2x}{x\sin^2x}\cdot \sin^2(nx)\mathrm{d}x}=\frac{1}{\pi}\int_0^{\frac{\pi}{2}}{\frac{x^2 - \sin^2x}{x\sin^2x}\mathrm{d}x}\int_0^{\pi}{\sin^2x\mathrm{d}x}=\frac{1}{2}\int_0^{\frac{\pi}{2}}{\frac{x^2 - \sin^2x}{x\sin^2x}\mathrm{d}x}<\infty. 
\end{align}
利用\eqref{example4.62-1.5}式可得
\begin{align}\label{example4.62-1.6}
\lim_{n\rightarrow \infty} \frac{1}{\ln n}\int_0^{\frac{\pi}{2}}{\frac{x^2 - \sin^2x}{x\sin^2x}\cdot \sin^2(nx)\mathrm{d}x}=\lim_{n\rightarrow \infty} \frac{1}{\ln n}\cdot \lim_{n\rightarrow \infty} \int_0^{\frac{\pi}{2}}{\frac{x^2 - \sin^2x}{x\sin^2x}\cdot \sin^2(nx)\mathrm{d}x}=0.  
\end{align}
因此,对\eqref{example4.62-1.4}式两边同时令\(n\rightarrow \infty\),利用\eqref{example4.62-1.6}式可得
\begin{align}\label{example4.62-1.7}
\begin{aligned}
&\lim_{n\rightarrow \infty} \frac{1}{\ln n}\int_0^{\frac{\pi}{2}}{\frac{f^\prime(0)x}{\sin^2x}\sin^2(nx)\mathrm{d}x}=\lim_{n\rightarrow \infty} \frac{1}{\ln n}\int_0^{\frac{\pi}{2}}{\frac{f^\prime(0)}{x}\sin^2(nx)\mathrm{d}x}\\
&=\lim_{n\rightarrow \infty} \frac{f^\prime(0)}{\ln n}\int_0^{\frac{n\pi}{2}}{\frac{\sin^2x}{x}\mathrm{d}x}
=\lim_{n\rightarrow \infty} \frac{f^\prime(0) \int_0^{\frac{n\pi}{2}}{\frac{\sin^2x}{x}\mathrm{d}x}}{\ln \frac{n\pi}{2} - \ln \frac{\pi}{2}}.
\end{aligned}
\end{align}
而由\hyperref[theorem:函数Stolz定理]{函数Stolz定理}可知
\begin{align}\label{example4.62-1.8}
\lim_{x\rightarrow \infty} \frac{f^\prime(0) \int_0^x{\frac{\sin^2t}{t}\mathrm{d}t}}{\ln x - \ln \frac{\pi}{2}}=f^\prime(0) \lim_{n\rightarrow \infty} \frac{\int_x^{x + \pi}{\frac{\sin^2t}{t}\mathrm{d}t}}{\ln (x + \pi) - \ln x}=\frac{f^\prime(0)}{\pi}\lim_{n\rightarrow \infty} x\int_x^{x + \pi}{\frac{\sin^2t}{t}\mathrm{d}t}.
\end{align}
由\hyperref[theorem:积分中值定理]{积分中值定理}可知,对\(\forall x > 0\),存在\(\theta_x\in [x, x + \pi]\),使得
\[
\int_x^{x + \pi}{\frac{\sin^2t}{t}\mathrm{d}t}=\frac{1}{\theta_x}\int_x^{x + \pi}{\sin^2t\mathrm{d}t}=\frac{1}{\theta_x}\int_0^{\pi}{\sin^2t\mathrm{d}t}=\frac{\pi}{2\theta_x}.
\]
又由\(\theta_x\in [x, x + \pi]\)可知,\(\theta_x\sim x, x\rightarrow +\infty\).从而\eqref{example4.62-1.8}式可化为
\[
\lim_{x\rightarrow \infty} \frac{f^\prime(0) \int_0^x{\frac{\sin^2t}{t}\mathrm{d}t}}{\ln x - \ln \frac{\pi}{2}}=\frac{f^\prime(0)}{\pi}\lim_{n\rightarrow \infty} x\int_x^{x + \pi}{\frac{\sin^2t}{t}\mathrm{d}t}=\frac{f^\prime(0)}{\pi}\lim_{n\rightarrow \infty} \frac{\pi x}{2\theta_x}=\frac{f^\prime(0)}{2}.
\]
于是由Heine归结原则可得
\begin{align}\label{example4.62-1.9}
\lim_{n\rightarrow \infty} \frac{f^\prime(0) \int_0^{\frac{n\pi}{2}}{\frac{\sin^2x}{x}\mathrm{d}x}}{\ln \frac{n\pi}{2} - \ln \frac{\pi}{2}}=\lim_{x\rightarrow \infty} \frac{f^\prime(0) \int_0^x{\frac{\sin^2t}{t}\mathrm{d}t}}{\ln x - \ln \frac{\pi}{2}}=\frac{f^\prime(0)}{2}. 
\end{align}
利用\eqref{example4.62-1.3}\eqref{example4.62-1.9}式,对\eqref{example4.62-1.1}式两边同时令$n\to \infty$,可得
\begin{align*}
\lim_{n\rightarrow \infty} \frac{1}{\ln n}\int_0^{\frac{\pi}{2}}{\frac{f\left( x \right)}{\sin ^2x}\sin ^2\left( nx \right) \mathrm{d}x}=\lim_{n\rightarrow \infty} \frac{1}{\ln n}\int_0^{\frac{\pi}{2}}{\frac{f\left( x \right) -f'\left( 0 \right) x}{\sin ^2x}\sin ^2\left( nx \right) \mathrm{d}x}+\lim_{n\rightarrow \infty} \frac{1}{\ln n}\int_0^{\frac{\pi}{2}}{\frac{f'\left( 0 \right) x}{\sin ^2x}\sin ^2\left( nx \right) \mathrm{d}x}=\frac{f'\left( 0 \right)}{2}.
\end{align*}
\end{proof}



\end{document}
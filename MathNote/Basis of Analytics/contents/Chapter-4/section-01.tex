\documentclass[../../main.tex]{subfiles}
\graphicspath{{\subfix{../../image/}}} % 指定图片目录,后续可以直接使用图片文件名。

% 例如:
% \begin{figure}[H]
% \centering
% \includegraphics{image-01.01}
% \caption{图片标题}
% \label{figure:image-01.01}
% \end{figure}
% 注意:上述\label{}一定要放在\caption{}之后,否则引用图片序号会只会显示??.

\begin{document}

\section{基本的渐进估计与求极限方法}

\subsection{基本极限计算}

\subsubsection{基本想法}

\textbf{裂项}、\textbf{作差}、\textbf{作商}的想法是解决极限问题的基本想法.

\begin{example}
对正整数\(v\),求极限\(\lim_{n\rightarrow\infty}\sum_{k = 1}^{n}\frac{1}{k(k + 1)\cdots(k + v)}\).
\end{example}
\begin{note}
直接裂项即可.
\end{note}
\begin{solution}
\begin{align*}
\lim_{n\rightarrow \infty} \sum_{k=1}^n{\frac{1}{k(k+1)\cdots (k+v)}}&=\lim_{n\rightarrow \infty} \sum_{k=1}^n{\frac{1}{v}\left[ \frac{1}{k\left( k+1 \right) \cdots \left( k+v-1 \right)}-\frac{1}{\left( k+1 \right) \left( k+2 \right) \cdots \left( k+v \right)} \right]}
\\
&=\lim_{n\rightarrow \infty} \frac{1}{v}\left[ \frac{1}{v!}-\frac{1}{n\left( n+1 \right) \cdots \left( n+v \right)} \right] =\frac{1}{v!v}.
\end{align*}
\end{solution}

\begin{example}
设\(p_0 = 0,0\leq p_j\leq1,j = 1,2,\cdots\)。求\(\sum_{j = 1}^{\infty}\left(p_j\prod_{i = 0}^{j - 1}(1 - p_i)\right)+\prod_{j = 1}^{\infty}(1 - p_j)\)的值.
\end{example}
\begin{note}
遇到求和问题,可以先观察是否存在裂项的结构.
\end{note}
\begin{solution}
记$q_i=1-p_i$,则有
\begin{align*}
\sum_{j=1}^{\infty}{p_j\prod_{i=0}^{j-1}{\left( 1-p_i \right)}}+\prod_{j=1}^{\infty}{\left( 1-p_j \right)}=\sum_{j=1}^n{\left( 1-q_j \right) \prod_{i=0}^{j-1}{q_i}}+\prod_{i=0}^{\infty}{q_i}=\sum_{j=1}^{\infty}{\left( \prod_{i=0}^{j-1}{q_i}-\prod_{i=0}^j{q_i} \right)}+\prod_{i=0}^{\infty}{q_i}=q_0-\prod_{i=0}^{\infty}{q_i}+\prod_{i=0}^{\infty}{q_i}=q_0.
\end{align*}
\end{solution}


\begin{example}
设\(\vert x\vert < 1\),求极限\(\lim_{n\rightarrow\infty}(1 + x)(1 + x^2)\cdots(1 + x^{2^n})\).
\end{example}
\begin{remark}
如果把幂次\(1,2,2^2,\cdots\)改成\(1,2,3,\cdots\),那么显然极限存在,但是并不能求出来,要引入别的特殊函数,省流就是:钓鱼题.
\end{remark}
\begin{note}
平方差公式即可
\end{note}
\begin{solution}
\begin{align*}
\lim_{n\rightarrow \infty} (1+x)(1+x^2)\cdots (1+x^{2^n})&=\lim_{n\rightarrow \infty} \frac{(1-x)(1+x)(1+x^2)\cdots (1+x^{2^n})}{1-x}
\\
&=\lim_{n\rightarrow \infty} \frac{(1-x^2)(1+x^2)\cdots (1+x^{2^n})}{1-x}\\
&=\cdots =\lim_{n\rightarrow \infty} \frac{1-x^{2^{n+1}}}{1-x}=\frac{1}{1-x}.
\end{align*}
\end{solution}

\begin{example}
对正整数\(n\),方程\(\left(1 + \frac{1}{n}\right)^{n + t}=e\)的解记为\(t = t(n)\),证明\(t(n)\)关于\(n\)递增并求极限\((t\to +\infty)\).
\end{example}
\begin{solution}
解方程得到
\begin{align*}
\left(1 + \frac{1}{n}\right)^{n + t}=e\Leftrightarrow(n + t)\ln\left(1 + \frac{1}{n}\right)=1\Leftrightarrow t=\frac{1}{\ln\left(1 + \frac{1}{n}\right)}-n.
\end{align*}
设$f(x)=\frac{1}{\ln\left(1 + \frac{1}{x}\right)}-x,x>0$,则
\begin{align*}
f^\prime(x)=\frac{1}{\ln^{2}\left(1 + \frac{1}{x}\right)}\frac{1}{x^{2}+x}-1>0
\Leftrightarrow\ln^{2}\left(1 + \frac{1}{x}\right)<\frac{1}{x^{2}+x}\Leftrightarrow\ln\mathrm{(}1+t)<\frac{t}{\sqrt{1+t}},t=\frac{1}{x}\in \left( 0,1 \right) .
\end{align*}
最后的不等式由\hyperref[proposition:关于ln的常用不等式1]{关于ln的常用不等式}可知显然成立,于是$f(x)$单调递增,故$t(n)=f(n)$也单调递增.再来求极限
\[
\lim_{n\rightarrow \infty} t\left( n \right) =\lim_{n\rightarrow \infty} \left( \frac{1}{\ln \left( 1+\frac{1}{n} \right)}-n \right) =\lim_{n\rightarrow \infty} \frac{1-n\ln \left( 1+\frac{1}{n} \right)}{\ln \left( 1+\frac{1}{n} \right)}=\lim_{x\rightarrow +\infty} \frac{1-x\ln \left( 1+\frac{1}{x} \right)}{\ln \left( 1+\frac{1}{x} \right)}=\lim_{x\rightarrow +\infty} \frac{1-x\ln \left( 1+\frac{1}{x} \right)}{\frac{1}{x}}=\frac{1}{2}.
\]
\end{solution}

\begin{proposition}
$\prod_{k=1}^n{\left( 1+\frac{1}{k} \right) ^k}=\frac{n!}{\left( n+1 \right) ^n}\sim \frac{\sqrt{2\pi n}}{e^{n+1}}.$
\end{proposition}
\begin{proof}
\begin{align*}
\prod_{k=1}^n{\left( 1+\frac{1}{k} \right) ^k}&=\prod_{k=1}^n{\left( \frac{1+k}{k} \right) ^k}=\left( \frac{2}{1} \right) \left( \frac{3}{2} \right) ^2\left( \frac{4}{3} \right) ^3\cdots \left( \frac{n+1}{n} \right) ^n
\\
&=\frac{n!}{\left( n+1 \right) ^n}\sim \frac{\sqrt{2\pi n}n^n}{\left( n+1 \right) ^ne^n}=\frac{\sqrt{2\pi n}}{\left( 1+\frac{1}{n} \right) ^ne^n}\sim \frac{\sqrt{2\pi n}}{e^{n+1}}.
\end{align*}
\end{proof}

\begin{example}
计算极限 $\lim\limits_{n \to +\infty} \sqrt{n} \prod_{k = 1}^{n} \frac{e^{1 - \frac{1}{k}}}{(1 + \frac{1}{k})^k}$。
\end{example}
\begin{solution}
因为
\begin{align*}
\sqrt{n} \prod_{k = 1}^{n} \frac{e^{1 - \frac{1}{k}}}{(1 + \frac{1}{k})^k} &= \sqrt{n} \frac{e^{n - (1 + \frac{1}{2} + \cdots + \frac{1}{n})}}{(\frac{2}{1})(\frac{3}{2})^2 (\frac{4}{3})^3 \cdots (\frac{n + 1}{n})^n} = \frac{\sqrt{n}n!e^n}{(n + 1)^n e^{1 + \frac{1}{2} + \cdots + \frac{1}{n}}}
\end{align*}
由 Stirling 公式 $n! \sim \sqrt{2\pi n} (\frac{n}{e})^n$ ($n \to +\infty$) 及
\[1 + \frac{1}{2} + \cdots + \frac{1}{n} = \ln n + \gamma + o(1),\quad n \to +\infty\]
得
\begin{align*}
\lim_{n \to +\infty} \sqrt{n} \prod_{k = 1}^{n} \frac{e^{1 - \frac{1}{k}}}{(1 + \frac{1}{k})^k} &= \lim_{n \to +\infty} \frac{\sqrt{2\pi}n}{(1 + 1/n)^n e^{\ln n + \gamma}} = \sqrt{2\pi} e^{-(1 + \gamma)}
\end{align*} 
\end{solution}

\begin{proposition}[数列常见的转型方式]\label{proposition:数列收敛的级数与累乘形式}
数列常见的转型方式:

(1) $\quad a_n = a_1 + \sum_{k=1}^{n-1} \left( a_{k+1} - a_k \right); $

(2)$ \quad a_n = a_1 \prod_{k=1}^{n-1} \frac{a_{k+1}}{a_k}$; 

(3) $\quad a_n = S_n - S_{n-1}, \text{其中} S_n = \sum_{k=1}^n a_k.$

从而我们可以得到
\begin{enumerate}
\item 数列$\left\{ a_n \right\} _{n=1}^{\infty}$收敛的充要条件是$\sum\limits_{n=1}^{\infty}{\left( a_{n+1}-a_n \right)}$收敛.

\item 数列 \(\{ a_n \}_{n = 1}^{\infty}\) (\(a_n\neq 0\))收敛的充要条件是 \(\prod_{n = 1}^{\infty} \frac{a_{n + 1}}{a_n}\) 收敛.
\end{enumerate}
\end{proposition}
\begin{remark}
在关于数列的问题中,\textbf{将原数列的等式或不等式条件转化为相邻两项的差或商的等式或不等式条件}的想法是非常常用的.
\end{remark}
\begin{note}
这个命题给我们证明数列极限的存在性提供了一种想法:
我们可以将数列的收敛性转化为级数的收敛性,或者将数列的收敛性转化为累乘的收敛性.而累乘可以通过取对数的方式转化成级数的形式,这样就可以利用级数的相关理论来证明数列的收敛性.

这种想法的\textbf{具体操作方式:}

(i)先令数列相邻两项作差或作商,将数列的极限写成其相邻两项的差的级数或其相邻两项的商的累乘形式.(如果是累乘的形式,那么可以通过取对数的方式将其转化成级数的形式.)

(ii)若能直接证明累乘或级数收敛,就直接证明即可.若不能,则再利用级数的相关理论来证明上述构造的级数的收敛性,从而得到数列的极限的存在性.
此时,我们一般会考虑这个级数的通项,然后去找一个通项能够控制住所求级数通项的收敛级数(几何级数等),最后利用级数的比较判别法来证明级数收敛
\end{note}
\begin{proof}
\begin{enumerate}
\item 必要性($\Rightarrow$)和充分性($\Leftarrow$)都可由$\underset{n\rightarrow \infty}{\lim}a_n=a_1+\underset{n\rightarrow \infty}{\lim}\sum_{k=1}^{n-1}{\left( a_{k+1}-a_k \right)}$直接得到.

\item 必要性($\Rightarrow$)和充分性($\Leftarrow$)都可由$\underset{n\rightarrow \infty}{\lim}a_n=\underset{n\rightarrow \infty}{\lim}a_1\prod_{k=1}^{n-1}{\frac{a_{k+1}}{a_k}}$直接得到.
\end{enumerate}
\end{proof}

\begin{example}
设\(a_n=\left(\frac{(2n)!!}{(2n - 1)!!}\right)^2\frac{1}{2n + 1}\),证明:数列\(a_n\)收敛到一个正数。
\end{example}
\begin{proof}
由条件可得 \(\forall n\in \mathbb{N}_+\),都有
\begin{align*}
\frac{a_{n+1}}{a_n}=\frac{\left( \frac{(2n+2)!!}{(2n+1)!!} \right) ^2\frac{1}{2n+3}}{\left( \frac{(2n)!!}{(2n-1)!!} \right) ^2\frac{1}{2n+1}}=\frac{\left( 2n+2 \right) ^2}{\left( 2n+1 \right) ^2}\cdot \frac{2n+1}{2n+3}=\frac{\left( 2n+2 \right) ^2}{\left( 2n+1 \right) \left( 2n+3 \right)}=1+\frac{1}{\left( 2n+1 \right) \left( 2n+3 \right)}>1.
\end{align*}
从而 \(\forall n\in \mathbb{N}_+\),都有
\begin{align}
a_n=\prod_{k = 1}^{n - 1}\left[1+\frac{1}{(2k + 1)(2k + 3)}\right]=e^{\sum\limits_{k = 1}^{n - 1}\ln\left[1+\frac{1}{(2k + 1)(2k + 3)}\right]}. \label{example60-1.1}
\end{align}
注意到
\[
\ln\left[1+\frac{1}{(2n + 1)(2n + 3)}\right]\sim\frac{1}{(2n + 1)(2n + 3)},n\rightarrow\infty.
\]
而 \(\sum_{n = 1}^{\infty}\frac{1}{(2n + 1)(2n + 3)}\) 收敛,故 \(\lim_{n\rightarrow\infty}\sum_{k = 1}^{n - 1}\ln\left[1+\frac{1}{(2k + 1)(2k + 3)}\right]\) 存在。于是由 \eqref{example60-1.1}式可知
\[
\lim_{n\rightarrow\infty}a_n=\lim_{n\rightarrow\infty}e^{\sum\limits_{k = 1}^{n - 1}\ln\left[1+\frac{1}{(2k + 1)(2k + 3)}\right]}=e^{\lim\limits_{n\rightarrow\infty}\sum\limits_{k = 1}^{n - 1}\ln\left[1+\frac{1}{(2k + 1)(2k + 3)}\right]}
\]
也存在。
\end{proof}


\subsubsection{带ln的极限计算}
通常,带着一堆\(\ln\)的极限算起来都非常烦人,并不是简单的一个泰勒就秒杀的,比如这种题.这种题不建议用泰勒,很多时候等价无穷小替换、拆项和加一项减一项会方便不少.

\begin{remark}
另外,做这种题一定要严格处理余项,不要想当然.
\end{remark}
\begin{example}
求极限\(\lim_{x\rightarrow +\infty}\left(\frac{(2x^2 + 3x + 1)\ln x}{x\ln(1 + x)}\arctan x-\pi x\right)\)。
\end{example}
\begin{remark}
做这种题一定要严格处理余项,不要想当然,比如下面的做法就是错的(过程和答案都不对)
\[\frac{(2x^2 + 3x + 1)\ln x}{x\ln(1 + x)}\arctan x-\pi x\approx(2x + 3)\frac{\ln x}{\ln(1 + x)}\arctan x-\pi x\approx(2x + 3)\cdot1\cdot\frac{\pi}{2}-\pi x=\frac{3\pi}{2}.\]
\end{remark}
\begin{solution}
根据洛必达法则,显然\(\lim_{x\rightarrow +\infty}\frac{\ln x}{\ln(1 + x)}=\lim_{x\rightarrow +\infty}\frac{\frac{1}{x}}{\frac{1}{1 + x}} = 1\),拆分一下有
\begin{align*}
&\lim_{x\rightarrow +\infty}\left(\frac{(2x^2 + 3x + 1)\ln x}{x\ln(1 + x)}\arctan x-\pi x\right)\\
=&\lim_{x\rightarrow +\infty}\left((2x + 3)\frac{\ln x}{\ln(1 + x)}\arctan x-\pi x\right)+\lim_{x\rightarrow +\infty}\frac{\ln x}{x\ln(1 + x)}\arctan x\\
=&\lim_{x\rightarrow +\infty}\left(\frac{2x\ln x}{\ln(1 + x)}\arctan x-\pi x\right)+3\lim_{x\rightarrow +\infty}\frac{\ln x}{\ln(1 + x)}\arctan x\\
=&2\lim_{x\rightarrow +\infty}x\left(\frac{\ln x}{\ln(1 + x)}\arctan x-\frac{\pi}{2}\right)+\frac{3}{2}\pi\\
=&2\left(\lim_{x\rightarrow +\infty}\frac{x\ln x}{\ln(1 + x)}\left(\arctan x-\frac{\pi}{2}\right)+\frac{\pi}{2}\lim_{x\rightarrow +\infty}x\left(\frac{\ln x}{\ln(1 + x)}-1\right)\right)+\frac{3}{2}\pi\\
=&2\left(\lim_{x\rightarrow +\infty}x\left(\arctan x-\frac{\pi}{2}\right)-\frac{\pi}{2}\lim_{x\rightarrow +\infty}\frac{x\ln(1 + \frac{1}{x})}{\ln(1 + x)}\right)+\frac{3}{2}\pi\\
=&2\left(\lim_{x\rightarrow +\infty}\frac{\arctan x-\frac{\pi}{2}}{\frac{1}{x}}-\frac{\pi}{2}\lim_{x\rightarrow +\infty}\frac{1}{\ln(1 + x)}\right)+\frac{3}{2}\pi\\
=&2\lim_{x\rightarrow +\infty}\frac{\frac{-1}{1 + x^2}}{-\frac{1}{x^2}}+\frac{3}{2}\pi=\frac{3}{2}\pi - 2.
\end{align*}
\end{solution}

\subsubsection{幂指函数的极限问题}
幂指函数的极限问题,一律写成\(e^{\ln}\)形式,并利用等价无穷小替换和加一项减一项去解决,方便.

\begin{remark}
不要用泰勒做这个题,因为你需要分别展开好几项直到余项是高阶无穷小才可以,等价无穷小替换则只需要看Taylor展开的第一项并且是严谨的,泰勒则需要展开好几项,计算量爆炸.  
\end{remark}
\begin{example}
求极限\(\lim_{x\rightarrow0^{+}}\frac{x^{\sin x}-(\sin x)^{x}}{x^{3}\ln x}\)。
\end{example}
\begin{remark}
不要用泰勒做这个题,因为你需要分别展开好几项直到余项是高阶无穷小才可以,等价无穷小替换则只需要看第一项并且是严谨的,泰勒则至少需要展开三项,计算量爆炸,大致如下
\begin{align*}
x^{\sin x}&=e^{\sin x\ln x}=1+\sin x\ln x+\frac{1}{2}\sin^{2}x\ln^{2}x+\frac{1}{6}\sin^{3}x\ln^{3}x+O(x^{4}\ln^{4}x)\\
(\sin x)^{x}&=e^{x\ln\sin x}=1 + x\ln\sin x+\frac{1}{2}x^{2}\ln^{2}\sin x+\frac{1}{6}x^{3}\ln^{3}\sin x+O(x^{4}\ln^{4}\sin x)
\end{align*}
然后你不仅需要看第一项,还要检查并验证平方项,三次方项作差后对应的极限是零,麻烦. 
\end{remark}
\begin{note}
先说明写成\(e^{\ln}\)形式后,指数部分都是趋于零的,然后等价无穷小替换即可.
\end{note}
\begin{solution}
注意到\begin{align*}
\lim_{x\rightarrow0^{+}}\sin x\ln x=\lim_{x\rightarrow0^{+}}x\ln x = 0, \lim_{x\rightarrow0^{+}}x\ln\sin x=\lim_{x\rightarrow0^{+}}\sin x\ln\sin x=\lim_{x\rightarrow0^{+}}x\ln x = 0.
\end{align*}
从而
\begin{align*}
\lim_{x\rightarrow 0^+} (\sin x)^x=\lim_{x\rightarrow 0^+} e^{x\ln\sin x}=1.
\end{align*}
于是我们有
\begin{align*}
&\lim_{x\rightarrow0^{+}}\frac{x^{\sin x}-(\sin x)^{x}}{x^{3}\ln x}=\lim_{x\rightarrow0^{+}}(\sin x)^{x}\frac{e^{\sin x\ln x - x\ln\sin x}-1}{x^{3}\ln x}=\lim_{x\rightarrow0^{+}}\frac{e^{\sin x\ln x - x\ln\sin x}-1}{x^{3}\ln x}\\
=&\lim_{x\rightarrow0^{+}}\frac{\sin x\ln x - x\ln\sin x}{x^{3}\ln x}=\lim_{x\rightarrow0^{+}}\frac{\sin x\ln x - x\ln x + x\ln x - x\ln\sin x}{x^{3}\ln x}\\
=&\lim_{x\rightarrow0^{+}}\frac{\sin x - x}{x^{3}}+\lim_{x\rightarrow0^{+}}\frac{\ln x - \ln\sin x}{x^{2}\ln x}=-\frac{1}{6}-\lim_{x\rightarrow0^{+}}\frac{\ln\frac{\sin x}{x}}{x^{2}\ln x}(\frac{\sin x}{x}\sim1 - \frac{1}{6}x^{2},x\to 0^+)\\
=&-\frac{1}{6}-\lim_{x\rightarrow0^{+}}\frac{\ln(1 + \frac{\sin x - x}{x})}{x^{2}\ln x}=-\frac{1}{6}-\lim_{x\rightarrow0^{+}}\frac{\sin x - x}{x^{3}\ln x}=-\frac{1}{6}+\frac{1}{6}\lim_{x\rightarrow0^{+}}\frac{1}{\ln x}=-\frac{1}{6}.
\end{align*}
\end{solution}

\begin{example}
求极限\(\lim_{x\rightarrow\infty}x^{2}\left(e^{(1 + \frac{1}{x})^{x}} - \left(1 + \frac{1}{x}\right)^{ex}\right)\).
\end{example}
\begin{solution}
注意到
\begin{align*}
\lim_{x\rightarrow\infty}\left(1 + \frac{1}{x}\right)^{x}=e,\lim_{x\rightarrow\infty}ex\ln\left(1 + \frac{1}{x}\right)=e.
\end{align*}
从而
\begin{align*}
\lim_{x\rightarrow 0^+} \left( 1+\frac{1}{x} \right) ^{ex}=\lim_{x\rightarrow 0^+} e^{ex\ln \left( 1+\frac{1}{x} \right)}=e^e.
\end{align*}
于是我们有
\begin{align*}
&\lim_{x\rightarrow\infty}x^{2}\left(e^{(1 + \frac{1}{x})^{x}} - \left(1 + \frac{1}{x}\right)^{ex}\right)=\lim_{x\rightarrow\infty}x^{2}\left(1 + \frac{1}{x}\right)^{ex}\left(e^{(1 + \frac{1}{x})^{x}-ex\ln(1 + \frac{1}{x})}-1\right)\\
&=e^{e}\lim_{x\rightarrow\infty}x^{2}\left(e^{(1 + \frac{1}{x})^{x}-ex\ln(1 + \frac{1}{x})}-1\right)=e^{e}\lim_{x\rightarrow\infty}x^{2}\left(\left(1 + \frac{1}{x}\right)^{x}-ex\ln\left(1 + \frac{1}{x}\right)\right)\\
&=e^{e}\lim_{x\rightarrow\infty}x^{2}\left(e^{x\ln(1 + \frac{1}{x})}-ex\ln\left(1 + \frac{1}{x}\right)\right)=e^{e + 1}\lim_{x\rightarrow\infty}x^{2}\left(e^{x\ln(1 + \frac{1}{x})-1}-x\ln\left(1 + \frac{1}{x}\right)\right)\\
&\xlongequal{Taylor\text{展开}}e^{e + 1}\lim_{x\rightarrow\infty}x^{2}\frac{1}{2}\left(x\ln\left(1 + \frac{1}{x}\right)-1\right)^{2}=\frac{e^{e + 1}}{2}\lim_{x\rightarrow\infty}\left(x^{2}\ln\left(1 + \frac{1}{x}\right)-x\right)^{2}=\frac{e^{e + 1}}{8}
\end{align*}
\end{solution}



\subsubsection{拟合法求极限}

\begin{example}
求极限\(\lim_{n\rightarrow\infty}\frac{\ln^3n}{\sqrt{n}}\sum_{k = 2}^{n - 2}\frac{1}{\ln k\ln(n - k)\ln(n + k)\sqrt{n + k}}\)。
\end{example}
\begin{note}
核心想法是\textbf{拟合法},但是最后的极限估计用到了\textbf{分段估计}的想法.
\end{note}
\begin{proof}
注意到\(\frac{\ln n}{\ln(2n)}\to1\),所以
\[
\lim_{n\rightarrow\infty}\frac{\ln^3n}{\sqrt{n}}\sum_{k = 2}^{n - 2}\frac{1}{\ln k\ln(n - k)\ln(n + k)\sqrt{n + k}}=\lim_{n\rightarrow\infty}\frac{\ln^2n}{\sqrt{n}}\sum_{k = 2}^{n - 2}\frac{1}{\ln k\ln(n - k)\sqrt{n + k}}
\]
显然
\[
\lim_{n\rightarrow\infty}\frac{1}{\sqrt{n}}\sum_{k = 2}^{n - 2}\frac{1}{\sqrt{n + k}}=\lim_{n\rightarrow\infty}\frac{1}{n}\sum_{k = 2}^{n - 2}\frac{1}{\sqrt{1+\frac{k}{n}}}=\lim_{n\rightarrow\infty}\frac{1}{n}\sum_{k = 0}^{n - 1}\frac{1}{\sqrt{1+\frac{k}{n}}}=\int_{0}^{1}\frac{1}{\sqrt{1 + x}}dx=2\sqrt{2}-2
\]
我们用上面的东西来拟合,所以尝试证明
\[
\lim_{n\rightarrow\infty}\frac{1}{\sqrt{n}}\sum_{k = 2}^{n - 2}\frac{1}{\sqrt{n + k}}\left(\frac{\ln^2n}{\ln k\ln(n - k)}-1\right)=\lim_{n\rightarrow\infty}\frac{1}{n}\sum_{k = 2}^{n - 2}\frac{1}{\sqrt{1+\frac{k}{n}}}\left(\frac{\ln^2n}{\ln k\ln(n - k)}-1\right)=0
\]
注意求和里面的每一项都是正的,并且\(\frac{1}{\sqrt{1+\frac{k}{n}}}\in\left[\frac{1}{\sqrt{2}},1\right]\),所以只需证
\[
\lim_{n\rightarrow\infty}\frac{1}{n}\sum_{k = 2}^{n - 2}\left(\frac{\ln^2n}{\ln k\ln(n - k)}-1\right)=0
\]
注意对称性,证明\(\lim_{n\rightarrow\infty}\frac{1}{n}\sum_{k = 2}^{\frac{n}{2}}\left(\frac{\ln^2n}{\ln k\ln(n - k)}-1\right)=0\)即可,待定一个\(m\)来分段放缩。
首先容易看出数列\(\ln k\ln(n - k)\)在\(2\leq k\leq\frac{n}{2}\)时是单调递增的,这是因为
\begin{align*}
f(x)&=\ln x\ln(n - x),f^\prime(x)=\frac{\ln(n - x)}{x}-\frac{\ln x}{n - x}>0\\
&\Leftrightarrow(n - x)\ln(n - x)>x\ln x,\forall x\in\left(2,\frac{n}{2}\right)
\end{align*}
显然成立,所以待定$m\in[2,\frac{n}{2}]$,于是
\begin{align*}
\frac{1}{n}\sum_{k = 2}^{m}\left(\frac{\ln^2n}{\ln k\ln(n - k)}-1\right)&\leq\frac{1}{n}\sum_{k = 2}^{m}\left(\frac{\ln^2n}{\ln 2\ln(n - 2)}-1\right)=\frac{m}{n}\left(\frac{\ln^2n}{\ln 2\ln(n - 2)}-1\right)\\
\frac{1}{n}\sum_{k = m}^{\frac{n}{2}}\left(\frac{\ln^2n}{\ln k\ln(n - k)}-1\right)&\leq\frac{1}{n}\sum_{k = m}^{\frac{n}{2}}\left(\frac{\ln^2n}{\ln m\ln(n - m)}-1\right)\leq\frac{\ln^2n}{\ln m\ln(n - m)}-1
\end{align*}
为了让第一个趋于零,可以取\(m = \frac{n}{2\ln^2n}\),然后代入检查第二个极限
\[
\lim_{n\rightarrow\infty}\frac{\ln^2n}{\ln m\ln(n - m)}-1=\lim_{n\rightarrow\infty}\frac{\ln^2n}{\ln\frac{n}{2\ln^2n}\ln\left(n-\frac{n}{2\ln^2n}\right)}-1 = 0
\]
所以结论得证(过程中严格来讲应补上取整符号,这里方便起见省略了)。
\end{proof}




\subsection{Taylor公式}

\begin{theorem}[带Peano余项的Taylor公式]\label{theorem:带Peano余项的Taylor公式}
设\(f\)在\(x = a\)是\(n\)阶右可微的,则
\begin{gather}
f(x)=\sum\limits_{k = 0}^{n}\frac{f^{(k)}(a)}{k!}(x - a)^{k}+o((x - a)^{n}),x \to a^{+}.
\label{Taylor:eq1}
\\
f(x)=\sum\limits_{k = 0}^{n - 1}\frac{f^{(k)}(a)}{k!}(x - a)^{k}+O((x - a)^{n}),x \to a^{+}.\label{Taylor:eq2}
\end{gather}
\end{theorem}
\begin{note}
用Taylor公式计算极限,如果展开$n$项还是不方便计算,那么就多展开一项或几项即可.
\end{note}
\begin{proof}
(1)要证明\eqref{Taylor:eq1}式等价于证明
\begin{align*}
\underset{x\rightarrow a^+}{\lim}\frac{f\left( x \right) -\sum\limits_{k=0}^n{\frac{f^{(k)}\left( a \right)}{k!}\left( x-a \right) ^k}}{\left( x-a \right) ^n}=0.
\end{align*}
对上式左边反复使用$n-1$次$L'Hospital'rules$,可得
\begin{align*}
&\underset{x\rightarrow a^+}{\lim}\frac{f\left( x \right) -\sum\limits_{k=0}^n{\frac{f^{(k)}\left( a \right)}{k!}\left( x-a \right) ^k}}{\left( x-a \right) ^n}\xlongequal{L'Hospital'rules}\underset{x\rightarrow a^+}{\lim}\frac{f'\left( x \right) -\sum\limits\limits_{k=1}^n{\frac{f^{(k)}\left( a \right)}{\left( k-1 \right) !}\left( x-a \right) ^{k-1}}}{n\left( x-a \right) ^{n-1}}
\\
&\xlongequal{L'Hospital'rules}\underset{x\rightarrow a^+}{\lim}\frac{f''\left( x \right) -\sum\limits_{k=2}^n{\frac{f^{(k)}\left( a \right)}{\left( k-2 \right) !}\left( x-a \right) ^{k-2}}}{n\left( n-1 \right) \left( x-a \right) ^{n-2}}
\\
&\xlongequal{L'Hospital'rules}\cdots \xlongequal{L'Hospital'rules}\underset{x\rightarrow a^+}{\lim}\frac{f^{\left( n-1 \right)}\left( x \right) -f^{\left( n-1 \right)}\left( a \right) -f^{\left( n \right)}\left( a \right) \left( x-a \right)}{n!\left( x-a \right)}
\\
&=\frac{1}{n!}\underset{x\rightarrow a^+}{\lim}\frac{f^{\left( n-1 \right)}\left( x \right) -f^{\left( n-1 \right)}\left( a \right)}{x-a}-\frac{f^{\left( n \right)}\left( a \right)}{n!}\xlongequal{n\text{阶导数定义}}0
\end{align*}
故\eqref{Taylor:eq1}式成立.

(2)要证明\eqref{Taylor:eq2}式等价于证明:存在$C>0$和$\delta>0$,使得
\begin{align*}
\left| \frac{f\left( x \right) -\sum\limits_{k=0}^n{\frac{f^{(k)}\left( a \right)}{k!}\left( x-a \right) ^k}}{\left( x-a \right) ^n} \right|\leqslant C,\forall x\in \left[ a,a+\delta \right].
\end{align*}

\end{proof}


\subsubsection{直接利用Taylor公式计算极限}


\begin{example}
设$\underset{n\rightarrow +\infty}{\lim}\frac{f\left( n \right)}{n}=1$,计算
\begin{align*}
\underset{n\rightarrow +\infty}{\lim}\left( 1+\frac{1}{f\left( n \right)} \right) ^n.
\end{align*}
\end{example}
\begin{note}
由$\frac{f\left( n \right)}{n}=1+o\left( 1 \right) ,n\rightarrow +\infty$,可得$f\left( n \right) =n+o\left( n \right) ,n\rightarrow +\infty$.这个等式的意思是:$f(n)=n+o(n)$对$\forall n\in \mathbb{N}_+$都成立.并且当$n\to +\infty$时,有$\underset{n\rightarrow +\infty}{\lim}\frac{f\left( n \right)}{n}=\underset{n\rightarrow +\infty}{\lim}\frac{n+o\left( n \right)}{n}=1+\underset{n\rightarrow +\infty}{\lim}\frac{o\left( n \right)}{n}=1
$.其中$o(n)$表示一个(类)数列,只不过这个(类)数列具有$\underset{n\rightarrow +\infty}{\lim}\frac{o\left( n \right)}{n}=0$的性质.
\end{note}
\begin{solution}
{\color{blue}解法一(一般解法):}
\begin{align*}
\underset{n\rightarrow +\infty}{\lim}\left( 1+\frac{1}{f\left( n \right)} \right) ^n=\underset{n\rightarrow +\infty}{\lim}e^{n\ln \left( 1+\frac{1}{f\left( n \right)} \right)}=e^{\underset{n\rightarrow +\infty}{\lim}n\ln \left( 1+\frac{1}{f\left( n \right)} \right)}=e^{\underset{n\rightarrow +\infty}{\lim}\frac{n}{f\left( n \right)}}=e.
\end{align*}
{\color{blue}解法二(渐进估计):}

由$\underset{n\rightarrow +\infty}{\lim}\frac{f\left( n \right)}{n}=1$,可知
\begin{align*}
\frac{f\left( n \right)}{n}=1+o\left( 1 \right) ,n\rightarrow +\infty .
\end{align*}
从而
\begin{align*}
\left( 1+\frac{1}{f\left( n \right)} \right) ^n=\left[ 1+\frac{1}{n}\cdot \frac{1}{1+o\left( 1 \right)} \right] ^n=\left[ 1+\frac{1}{n}\left( 1+o\left( 1 \right) \right) \right] ^n=\left[ 1+\frac{1}{n}+o\left( \frac{1}{n} \right) \right] ^n=e^{n\ln \left[ 1+\frac{1}{n}+o\left( \frac{1}{n} \right) \right]},n\to +\infty.
\end{align*}
于是
\begin{align*}
\underset{n\rightarrow +\infty}{\lim}\left( 1+\frac{1}{f\left( n \right)} \right) ^n=\underset{n\rightarrow +\infty}{\lim}e^{n\ln \left[ 1+\frac{1}{n}+o\left( \frac{1}{n} \right) \right]}=\underset{n\rightarrow +\infty}{\lim}e^{n\left[ \frac{1}{n}+o\left( \frac{1}{n} \right) \right]}=\underset{n\rightarrow +\infty}{\lim}e^{1+o\left( 1 \right)}=e.
\end{align*}
\end{solution}

\begin{example}
计算:
\begin{enumerate}
\item \(
\lim_{x \to 0} \frac{\cos \sin x - \cos x}{x^4}.
\)

\item \(
\lim_{x \to +\infty} \left[\left(x^3 - x^2 + \frac{x}{2}\right) e^{\frac{1}{x}} - \sqrt{1 + x^6}\right].
\)
\end{enumerate}
\end{example}
\begin{solution}
\begin{enumerate}
\item 

\item 
\end{enumerate}
\end{solution}

\begin{example}
求极限 $\lim\limits_{n \to +\infty} (\sqrt[n]{n} - 1)^{\frac{1}{(\ln n)^{\alpha}}}$ ($\alpha > 0$)。
\end{example}
\begin{note}
利用Taylor公式即可得到结果.类似$\ln \left( xe^{-x}-1 \right) \sim \ln \left( xe^{-x}+o\left( xe^{-x} \right) \right) \sim \ln \left( xe^{-x} \right)$的等价关系可以直接凭直觉写出,要严谨证明的话,只需要利用L'Hospital法则即可.
\end{note}
\begin{solution}
由
\begin{align*}
(\sqrt[n]{n} - 1)^{\frac{1}{(\ln n)^{\alpha}}} &= e^{\frac{\ln(\sqrt[n]{n} - 1)}{(\ln n)^{\alpha}}}
\end{align*}
从而
\begin{align*}
\lim_{n \to +\infty} \frac{\ln(\sqrt[n]{n} - 1)}{(\ln n)^{\alpha}} &= \lim_{x \to +\infty} \frac{\ln(e^{x e^{-x}} - 1)}{x^{\alpha}} = \lim_{x \to +\infty} \frac{\ln(x e^{-x})}{x^{\alpha}} \\
&= \lim_{x \to +\infty} \frac{\ln x - x}{x^{\alpha}} = \lim_{x \to +\infty} \left(\frac{\ln x}{x^{\alpha}} - \frac{1}{x^{\alpha - 1}}\right) = - \lim_{x \to +\infty} \frac{1}{x^{\alpha - 1}} \\
&= \begin{cases}
0, & \alpha > 1, \\
-1, & \alpha = 1, \\
-\infty, & 0 < \alpha < 1.
\end{cases}
\end{align*}
于是
\begin{align*}
\lim_{n \to +\infty} (\sqrt[n]{n} - 1)^{\frac{1}{(\ln n)^{\alpha}}} &= \begin{cases}
1, & \alpha > 1, \\
e^{-1}, & \alpha = 1, \\
0, & 0 < \alpha < 1.
\end{cases}
\end{align*} 
\end{solution}

\begin{example}
计算$(1+\frac{1}{x})^x,x\to+\infty$的渐进估计.
\end{example}
\begin{solution}
由带$Peano$余项的$Taylor$公式,可得
\begin{align*}
&\left( 1+\frac{1}{x} \right) ^x=e^{x\ln \left( 1+\frac{1}{x} \right)}=e^{x\left[ \frac{1}{x}-\frac{1}{2x^2}+\frac{1}{3x^3}+o\left( \frac{1}{x^3} \right) \right]}=e^{1-\frac{1}{2x}+\frac{1}{3x^2}+o\left( \frac{1}{x^2} \right)}=e\cdot e^{-\frac{1}{2x}+\frac{1}{3x^2}+o\left( \frac{1}{x^2} \right)}
\\
&=e\left[ 1-\frac{1}{2x}+\frac{1}{3x^2}+o\left( \frac{1}{x^2} \right) +\frac{1}{2}\left( -\frac{1}{2x}+\frac{1}{3x^2}+o\left( \frac{1}{x^2} \right) \right) ^2+o\left( -\frac{1}{2x}+\frac{1}{3x^2}+o\left( \frac{1}{x^2} \right) \right) ^2 \right] 
\\
&=e\left[ 1-\frac{1}{2x}+\frac{1}{3x^2}+\frac{1}{8x^2}+o\left( \frac{1}{x^2} \right) \right] 
\\
&e-\frac{e}{2x}+\frac{11e}{24x^2}+o\left( \frac{1}{x^2} \right) 
\end{align*}
故\begin{align*}
\left( 1+\frac{1}{x} \right) ^x=e-\frac{e}{2x}+\frac{11e}{24x^2}+o\left( \frac{1}{x^2} \right) ,x\rightarrow +\infty .
\end{align*}
于是
\begin{align}\label{equation:12345}
\underset{x\rightarrow +\infty}{\lim}x\left[ e-\left( 1+\frac{1}{x} \right) ^x \right] =\frac{e}{2},\underset{x\rightarrow +\infty}{\lim}x\left[ x\left( e-\left( 1+\frac{1}{x} \right) ^x \right) -\frac{e}{2} \right] =-\frac{11e}{24}.
\end{align}
\end{solution}
\begin{remark}
反复利用上述\eqref{equation:12345}式构造极限的方法,再求出相应极限,就能得到$e$的更精确的渐进估计.这也是计算渐进估计的一般方法.
\end{remark}

\begin{example}
计算
\[
\lim_{x \to 0} \frac{1 - \cos x \cos(2x) \cdots \cos(nx)}{x^2}.
\]
\end{example}
\begin{solution}
记$I=\lim_{x \to 0} \frac{1 - \cos x \cos(2x) \cdots \cos(nx)}{x^2}$,则由带$Peano$余项的$Taylor$公式,可得
\begin{align*}
&\cos x \cos(2x) \cdots \cos(nx)
=\left[1 - \frac{1}{2}x^2 + o(x^2)\right]\left[1 - \frac{(2x)^2}{2} + o(x^2)\right] \cdots \left[1 - \frac{(nx)^2}{2} + o(x^2)\right]
\\
&= 1 - \frac{1^2 + 2^2 + \cdots + n^2}{2}x^2 + o(x^2)
= 1 - \frac{n(n + 1)(2n + 1)}{2 \cdot 6}x^2 + o(x^2),x\to0.
\end{align*}
故\(I = \frac{n(n + 1)(2n + 1)}{12}\).
\end{solution}

\begin{example}
计算
\[
\lim_{x\rightarrow 0} \frac{x-\overset{n\text{次复合}}{\overbrace{\sin\sin \cdots \sin x}}}{x^3}.
\]
\end{example}
\begin{solution}
先证明\(\underbrace{\sin(\sin(\sin(\cdots (\sin x))\cdots))}_{n\text{次复合}} = x - \frac{n}{6}x^3 + o(x^3),x\to0\).

当\(n = 1\)时,由$Taylor$公式结论显然成立.假设\(n=k\)时,结论成立.则当\(n=k + 1\)时,我们有
\begin{align*}
&\sin\left(x - \frac{n}{6}x^3 + o(x^3)\right)
\\
&= x - \frac{n}{6}x^3 + o(x^3) - \frac{1}{6}\left(x - \frac{n}{6}x^3 + o(x^3)\right)^3 + o\left(\left(x - \frac{n}{6}x^3 + o(x^3)\right)^3\right)
\\
&= x - \frac{n + 1}{6}x^3 + o(x^3),x\to0.
\end{align*}
由数学归纳法得\(\underbrace{\sin(\sin(\sin(\cdots (\sin x))\cdots))}_{n\text{次复合}} = x - \frac{n}{6}x^3 + o(x^3)\),$x\to0$.
故$\lim_{x\rightarrow 0} \frac{x-\overset{n\text{次复合}}{\overbrace{\sin\sin \cdots \sin x}}}{x^3}=\frac{n}{6}.$
\end{solution}


\begin{example}
计算
\[
\lim_{n\rightarrow \infty}n\sin(2\pi en!).
\]
\end{example}
\begin{solution}
由带Lagrange余项的Taylor展开式可知
\[
e^x = \sum_{k = 0}^{n + 1}\frac{x^k}{k!} + \frac{e^{\theta}x^{n + 2}}{(n + 2)!}, \theta \in (0, x).
\]
从而
\[
e = \sum_{k = 0}^{n + 1}\frac{1}{k!} + \frac{e^{\theta}}{(n + 2)!}, \theta \in (0, 1).
\]
于是
\[
2\pi en! = 2\pi n!\sum_{k = 0}^{n + 1}\frac{1}{k!} + \frac{2\pi n!e^{\theta}}{(n + 2)!}, \theta \in (0, 1).
\]
而\(n!\sum_{k = 0}^n\frac{1}{k!} \in \mathbb{N}\),因此
\begin{align*}
n\sin(2\pi en!)&=n\sin\left(2\pi n!\sum_{k = 0}^{n + 1}\frac{1}{k!} + \frac{2\pi n!e^{\theta}}{(n + 2)!}\right)
=n\sin\left(\frac{2\pi n!}{(n + 1)!} + \frac{2\pi n!e^{\theta}}{(n + 2)!}\right)\\
&=n\sin\left(\frac{2\pi}{n + 1} + \frac{2\pi e^{\theta}}{(n + 1)(n + 2)}\right)
\sim n\left[\frac{2\pi}{n + 1} + \frac{2\pi e^{\theta}}{(n + 1)(n + 2)}\right] \to 2\pi, n \to +\infty.
\end{align*} 
\end{solution}


\subsection{利用Lagrange中值定理求极限}
Lagrange中值定理不会改变原数列或函数的阶,但是可以更加精细地估计原数列或函数的阶.以后利用Lagrange中值定理处理数列或函数的阶的过程都会直接省略.

\begin{example}
计算
\begin{align*}
\lim_{n \to \infty} [\sin(\sqrt{n + 1}) - \sin(\sqrt{n})].
\end{align*}
\end{example}
\begin{solution}
由Lagrange中值定理,可知对$\forall n\in\mathbb{N}_+$,存在$\theta_n \in(\sqrt{n+1},\sqrt{n})$,使得
\begin{align*}
\sin(\sqrt{n+1})-\sin(\sqrt{n})=(\sqrt{n+1}-\sqrt{n})\cos\theta_n=\frac{1}{\sqrt{n+1}+\sqrt{n}}\cdot \cos\theta_n.
\end{align*}
从而当$n\to +\infty$时,有$\theta_n \to +\infty$.于是
\begin{align*}
\lim_{n \to \infty} [\sin(\sqrt{n + 1}) - \sin(\sqrt{n})]=\lim_{n \to \infty} [\frac{1}{\sqrt{n+1}+\sqrt{n}}\cdot \cos\theta_n]=0.
\end{align*}
\end{solution}

\begin{example}
计算
\begin{align*}
\lim_{n \to \infty} n^2 \left(\arctan\frac{2024}{n} - \arctan\frac{2024}{n + 1}\right).
\end{align*}
\end{example}
\begin{proof}
由Lagrange中值定理,可知对$\forall n\in\mathbb{N}$,存在$\theta_n\in(\frac{2024}{n},\frac{2024}{n + 1})$,使得
\begin{align*}
\arctan\frac{2024}{n} - \arctan\frac{2024}{n + 1} = \frac{1}{1+\theta _{n}^{2}}\cdot \left( \frac{2024}{n}-\frac{2024}{n+1} \right).
\end{align*}
并且$\underset{n\rightarrow +\infty}{\lim}\theta _n=0$.故
\begin{align*}
\lim_{n\rightarrow \infty} n^2\left( \mathrm{arc}\tan \frac{2024}{n}-\mathrm{arc}\tan \frac{2024}{n+1} \right) =\lim_{n\rightarrow \infty} \frac{n^2}{1+\theta _{n}^{2}}\cdot \left( \frac{2024}{n}-\frac{2024}{n+1} \right) =2024\lim_{n\rightarrow \infty} \frac{n^2}{n\left( n+1 \right)}=2024.
\end{align*}
\end{proof}

\begin{example}

1. 对\(\alpha \neq 0\),求\((n + 1)^{\alpha} - n^{\alpha}, n \to \infty\)的等价量;

2. 求\(n \ln n - (n - 1) \ln (n - 1), n \to \infty\)的等价量.
\end{example}
\begin{note}
熟练这种利用Lagrange中值定理求极限的方法以后,这类数列或函数的等价量我们应该做到能够快速口算出来.因此,以后利用Lagrange中值定理计算数列或函数的等价量的具体过程我们不再书写,而是直接写出相应的等价量.
\end{note}
\begin{remark}
不难发现利用Lagrange中值定理计算数列或函数的等价量,并不改变原数列或函数的阶.
\end{remark}
\begin{solution}
1.根据Lagrange中值定理,可知对$n\in\mathbb{N}$,都有
\begin{align*}
(n + 1)^{\alpha} - n^{\alpha}=\alpha\cdot\theta _{n}^{\alpha -1},\theta _{n}\in(n,n+1).
\end{align*}
不妨设$\alpha>1$,则有$\alpha n^{\alpha -1}\leqslant \alpha \theta _{n}^{\alpha -1}\leqslant \alpha \left( n+1 \right) ^{\alpha -1}$(若$\alpha\leq1$,则有$\alpha \left( n+1 \right) ^{\alpha -1}\leqslant \alpha \theta _{n}^{\alpha -1}\leqslant \alpha n^{\alpha -1}$).故
\begin{align*}
\alpha =\lim_{n\rightarrow \infty} \frac{\alpha n^{\alpha -1}}{n^{\alpha -1}}\leqslant \lim_{n\rightarrow \infty} \frac{\alpha \theta _{n}^{\alpha -1}}{n^{a-1}}\leqslant \lim_{n\rightarrow \infty} \frac{\alpha (n+1)^{\alpha -1}}{n^{a-1}}=\alpha.
\end{align*}
因此$(n+1)^{\alpha}-n^{\alpha}\sim \alpha n^{\alpha -1},n\rightarrow \infty$.

2.由Lagrange中值定理,可知对$n\in\mathbb{N}$,都有
\begin{align*}
\lim_{n \to \infty} \frac{n \ln n - (n - 1) \ln (n - 1)}{\ln n} = \lim_{n \to \infty} \frac{(n - (n - 1)) \cdot (1 + \ln \theta_n)}{\ln n}= \lim_{n \to \infty} \frac{1}{\ln n} + \lim_{n \to \infty} \frac{\ln \theta_n}{\ln n}=\lim_{n \to \infty} \frac{\ln \theta_n}{\ln n}, n - 1 < \theta_n < n.
\end{align*}
又\(\frac{\ln (n - 1)}{\ln n} < \frac{\ln \theta_n}{\ln n} < \frac{\ln n}{\ln n} = 1\),故\(\lim_{n \to \infty} \frac{\ln \theta_n}{\ln n} = 1\),从而
\begin{align*}
\lim_{n \to \infty} \frac{n \ln n - (n - 1) \ln (n - 1)}{\ln n} = \lim_{n \to \infty} \frac{\ln \theta_n}{\ln n}=1.
\end{align*}
于是\(n \ln n - (n - 1) \ln (n - 1) \sim \ln n,n\to+\infty\).
\end{solution}

\begin{example}
计算
\begin{align*}
\lim_{x \to 0} \frac{\cos(\sin x) - \cos x}{(1 - \cos x)\sin^{2}x}.
\end{align*}
\end{example}
\begin{proof}
由Lagrange中值定理,可知对$\forall x\in U\left( 0 \right)$,都有
\begin{align*}
\cos \left( \sin x \right) -\cos x=\left( x-\sin x \right) \sin \theta ,\theta \in \left( \sin x,x \right) .
\end{align*}
从而
\begin{align*}
\lim_{x\rightarrow 0} \frac{\mathrm{cos(}\sin x)-\cos x}{(1-\cos x)\sin ^2x}=\lim_{x\rightarrow 0} \frac{\left( x-\sin x \right) \sin \theta}{\frac{1}{2}x^2\cdot x^2}=\lim_{x\rightarrow 0} \frac{\frac{1}{6}x^3\cdot \sin \theta}{\frac{1}{2}x^4}=\frac{1}{3}\lim_{x\rightarrow 0} \frac{\sin \theta}{x}.
\end{align*}
又由$\sin x<\theta <x,\forall x\in U\left( 0 \right)$可知
\begin{align*}
1=\lim_{x\rightarrow 0} \frac{\sin x}{x}=\lim_{x\rightarrow 0} \frac{\sin \left( \sin x \right)}{x}<\lim_{x\rightarrow 0} \frac{\sin \theta}{x}\leqslant \lim_{x\rightarrow 0} \frac{\theta}{x}<\lim_{x\rightarrow 0} \frac{x}{x}=1.
\end{align*}
故$\sin \theta \sim \theta \sim x,x\rightarrow 0$.因此$\lim_{x\rightarrow 0} \frac{\mathrm{cos(}\sin x)-\cos x}{(1-\cos x)\sin ^2x}=\frac{1}{3}\lim_{x\rightarrow 0} \frac{\sin \theta}{x}=\frac{1}{3}\lim_{x\rightarrow 0} \frac{x}{x}=\frac{1}{3}$.
\end{proof}



\subsection{L'Hospital'rules}


\begin{theorem}[上下极限L'Hospital法则]\label{theorem:上下极限L'Hospital法则}
\begin{enumerate}
\item 设 \(f, g\) 在 \((a,b)\) 内可微,满足 (i) \(\forall x \in (a,b)\),\(g'(x) \ne 0\)。(ii) \(\lim_{x \to a^+} g(x) = +\infty\)。则
\begin{align}
\varliminf_{x \to a^+} \frac{f'(x)}{g'(x)} \leqslant \varliminf_{x \to a^+} \frac{f(x)}{g(x)} \leqslant \varlimsup_{x \to a^+} \frac{f(x)}{g(x)} \leqslant \varlimsup_{x \to a^+} \frac{f'(x)}{g'(x)}\label{theorem4.2-13.27-1}
\end{align}
且
\begin{align}\label{theorem4.2-13.27}
\underset{x\rightarrow a^+}{\underline{\lim }}\left| \frac{f' \left( x \right)}{g' \left( x \right)} \right|\leqslant \underset{x\rightarrow a^+}{\underline{\lim }}\left| \frac{f\left( x \right)}{g\left( x \right)} \right|\leqslant \underset{x\rightarrow a^+}{\overline{\lim }}\left| \frac{f\left( x \right)}{g\left( x \right)} \right|\leqslant \underset{x\rightarrow a^+}{\overline{\lim }}\left| \frac{f' \left( x \right)}{g' \left( x \right)} \right|.
\end{align}


\item 设 \(f, g\) 在 \((a,b)\) 内可微,满足 (i) \(\forall x \in (a,b)\),\(g'(x) \ne 0\)。(ii) \(\lim_{x \to a^+} f(x) = \lim_{x \to a^+} g(x) = 0\)。则
\begin{align}\label{theorem4.2-13.27-2}
\varliminf_{x \to a^+} \frac{f'(x)}{g'(x)} \leqslant \varliminf_{x \to a^+} \frac{f(x)}{g(x)} \leqslant \varlimsup_{x \to a^+} \frac{f(x)}{g(x)} \leqslant \varlimsup_{x \to a^+} \frac{f'(x)}{g'(x)}
\end{align}
且
\begin{align}\label{theorem4.2-13.281}
\underset{x\rightarrow a^+}{\underline{\lim }}\left| \frac{f' \left( x \right)}{g' \left( x \right)} \right|\leqslant \underset{x\rightarrow a^+}{\underline{\lim }}\left| \frac{f\left( x \right)}{g\left( x \right)} \right|\leqslant \underset{x\rightarrow a^+}{\overline{\lim }}\left| \frac{f\left( x \right)}{g\left( x \right)} \right|\leqslant \underset{x\rightarrow a^+}{\overline{\lim }}\left| \frac{f' \left( x \right)}{g' \left( x \right)} \right|.
\end{align}
\end{enumerate}
\end{theorem}
\begin{note}
此定理第一部分\eqref{theorem4.2-13.27-1}和\eqref{theorem4.2-13.27-2}可以直接使用且以后可以不必再担心分子分母同时求导之后极限不存在而不能使用洛必达法则的情况. 但\eqref{theorem4.2-13.27}和\eqref{theorem4.2-13.281}一般是不能直接用的, 需要给证明.
\end{note}
\begin{proof}
以第一问为例, 事实上, 固定\(x\), 由Cauchy中值定理, 我们有
\[
\frac{f(y)-f(x)}{g(y)-g(x)}=\frac{f'(\xi)}{g'(\xi)},x < \xi < y.
\]
我们断言对\(A \in \mathbb{R} \cup \{+\infty\}\), 必有
\begin{align}\label{theorem4.2-13.28}
\lim_{n\rightarrow\infty}\left|\frac{f(y_n)}{g(y_n)}\right| = A \Leftrightarrow \lim_{n\rightarrow\infty}\left|\frac{f(y_n)-f(x)}{g(y_n)-g(x)}\right| = A.
\end{align}
若\(\lim_{n\rightarrow\infty}\left|\frac{f(y_n)}{g(y_n)}\right| = A\). 首先利用极限的四则运算, 我们有
\[
\lim_{n\rightarrow\infty}\left|\frac{f(y_n)-f(x)}{g(y_n)-g(x)}\right|=\lim_{n\rightarrow\infty}\left|\frac{\frac{f(y_n)}{g(y_n)}-\frac{f(x)}{g(y_n)}}{1 - \frac{g(x)}{g(y_n)}}\right|=\lim_{n\rightarrow\infty}\left|\frac{1}{1 - \frac{g(x)}{g(y_n)}}\right|\cdot\lim_{n\rightarrow\infty}\left|\frac{f(y_n)}{g(y_n)}-\frac{f(x)}{g(y_n)}\right|=\lim_{n\rightarrow\infty}\left|\frac{f(y_n)}{g(y_n)}-\frac{f(x)}{g(y_n)}\right|.
\]
利用
\[
\left|\frac{f(y_n)}{g(y_n)}\right|-\left|\frac{f(x)}{g(y_n)}\right|\leqslant\left|\frac{f(y_n)}{g(y_n)}-\frac{f(x)}{g(y_n)}\right|\leqslant\left|\frac{f(y_n)}{g(y_n)}\right|+\left|\frac{f(x)}{g(y_n)}\right|, \lim_{n\rightarrow\infty} g(y_n) = \infty,
\]
我们知道
\[
\lim_{n\rightarrow\infty}\left|\frac{f(y_n)-f(x)}{g(y_n)-g(x)}\right|=\lim_{n\rightarrow\infty}\left|\frac{f(y_n)}{g(y_n)}-\frac{f(x)}{g(y_n)}\right| = A.
\]
反之设\(\lim_{n\rightarrow\infty}\left|\frac{f(y_n)-f(x)}{g(y_n)-g(x)}\right| = A\), 同样的由四则运算, 我们有
\[
\lim_{n\rightarrow\infty}\left|\frac{f(y_n)}{g(y_n)}-\frac{f(x)}{g(y_n)}\right| = A.
\]
于是由
\[
\left|\frac{f(y_n)}{g(y_n)}-\frac{f(x)}{g(y_n)}\right|-\left|\frac{f(x)}{g(y_n)}\right|\leqslant\left|\frac{f(y_n)}{g(y_n)}\right|\leqslant\left|\frac{f(y_n)}{g(y_n)}-\frac{f(x)}{g(y_n)}\right|+\left|\frac{f(x)}{g(y_n)}\right|, \lim_{n\rightarrow\infty}|g(y_n)| = \infty,
\]
我们知道
\[
\lim_{n\rightarrow\infty}\left|\frac{f(y_n)}{g(y_n)}\right| = A.
\]
现在就证明了\eqref{theorem4.2-13.28}.

于是结合\(x \to +\infty\), 我们容易得到
\begin{align*}
&\varlimsup_{y\rightarrow+\infty}\left|\frac{f(y)}{g(y)}\right|=\varlimsup_{y\rightarrow+\infty}\left|\frac{f(y)-f(x)}{g(y)-g(x)}\right|=\varlimsup_{y\rightarrow+\infty}\left|\frac{f'(\xi)}{g'(\xi)}\right|\leqslant\varlimsup_{y\rightarrow+\infty}\left|\frac{f'(y)}{g'(y)}\right|\\
&\varliminf_{y\rightarrow+\infty}\left|\frac{f(y)}{g(y)}\right|=\varliminf_{y\rightarrow+\infty}\left|\frac{f(y)-f(x)}{g(y)-g(x)}\right|=\varliminf_{y\rightarrow+\infty}\left|\frac{f'(\xi)}{g'(\xi)}\right|\geqslant\varliminf_{y\rightarrow+\infty}\left|\frac{f'(y)}{g'(y)}\right|
\end{align*}
这就完成了证明.
\end{proof}



\begin{example}
若\(f\in D^1[0,+\infty)\).
\begin{enumerate}[(1)]
\item 设
\[
\lim_{x\rightarrow +\infty}[f(x)+f'(x)] = s\in\mathbb{R},
\]
证明\(\lim_{x\rightarrow +\infty}f(x)=s\).

\item 设
\[
\lim_{x\rightarrow +\infty}\left[f'(x)+\frac{2x}{\sqrt[3]{1 + x^{3}}}f(x)\right]=s\in\mathbb{R},
\]
证明\(\lim_{x\rightarrow +\infty}f(x)=\frac{s}{2}\).
\end{enumerate}
\end{example}
\begin{note}
(2)中的构造思路:根据条件构造相应的微分方程,然后求解这个微分方程,再常数变易得到我们需要构造的函数. 具体步骤如下:

构造微分方程:\(y'+\frac{2x}{\sqrt[3]{1 + x^3}}y = 0\),整理可得\(\frac{y'}{y}=-\frac{2x}{\sqrt[3]{1 + x^3}}\),再对其两边同时积分得到\(\ln y = -\int_{0}^{x}\frac{2x}{\sqrt[3]{1 + x^3}}dx + C_0\).从而\(y = Ce^{-\int_{0}^{x}\frac{2x}{\sqrt[3]{1 + x^3}}dx}\),于是\(C = ye^{\int_{0}^{x}\frac{2x}{\sqrt[3]{1 + x^3}}dx}\).
故我们要构造的函数就是\(C(x) = f(x)e^{\int_{0}^{x}\frac{2x}{\sqrt[3]{1 + x^3}}dx}\).并且此时$C(x)$满足$C'\left( x \right) =f'\left( x \right) +\frac{2x}{\sqrt[3]{1+x^3}}f\left( x \right)$.
\end{note}
\begin{proof}
\begin{enumerate}[(1)]
\item $\lim_{x\rightarrow +\infty}f(x) = \lim_{x\rightarrow +\infty}\frac{e^{x}f(x)}{e^{x}}
=\lim_{x\rightarrow +\infty}\frac{e^{x}[f(x)+f'(x)]}{e^{x}}
=\lim_{x\rightarrow +\infty}[f + f']
=s$.

\item 注意到$\lim_{x\rightarrow +\infty} e^{\int_0^x{\frac{2t}{\sqrt[3]{1+t^3}}dt}}=+\infty$,从而由$\mathrm{L}'\mathrm{Hospital}'\mathrm{rules}$可得
\begin{align*}
\lim_{x\rightarrow +\infty} f(x)&=\lim_{x\rightarrow +\infty} \frac{f(x)\cdot e^{\int_0^x{\frac{2t}{\sqrt[3]{1+t^3}}dt}}}{e^{\int_0^x{\frac{2t}{\sqrt[3]{1+t^3}}dt}}}\xlongequal[]{\mathrm{L}'\mathrm{Hospital}'\mathrm{rules}}\lim_{x\rightarrow +\infty} \frac{\left[ f'(x)+\frac{2x}{\sqrt[3]{1+x^3}}f(x) \right] e^{\int_0^x{\frac{2t}{\sqrt[3]{1+t^3}}dt}}}{\frac{2x}{\sqrt[3]{1+x^3}}e^{\int_0^x{\frac{2t}{\sqrt[3]{1+t^3}}dt}}}
\\
&=\lim_{x\rightarrow +\infty} \frac{\sqrt[3]{1+x^3}}{2x}\left[ f(x)+\frac{2x}{\sqrt[3]{1+x^3}}f'(x) \right] =\frac{s}{2}.
\end{align*}
\end{enumerate}
\end{proof}

\begin{example}
设可微函数 \(a, b, f, g : \mathbb{R} \to \mathbb{R}\) 满足
\begin{align*}
f(x) \geqslant 0, g(x) > 0, g'(x) > 0, \frac{f'(x)}{g'(x)} + a(x) \frac{f(x)}{g(x)} = b(x), \quad \forall x \in \mathbb{R}, \\
\lim_{x \to +\infty} a(x) = A > 0, \lim_{x \to +\infty} b(x) = B > 0, \lim_{x \to +\infty} f(x) = \lim_{x \to +\infty} g(x) = +\infty.
\end{align*}
证明:
\begin{align*}
\lim_{x \to +\infty} \frac{f(x)}{g(x)} = \frac{B}{A + 1}.
\end{align*} 
\end{example}
\begin{remark}
如果直接使用 L'Hospital 法则,再结合条件会得到
\begin{align*}
\lim_{x \to +\infty} \frac{f(x)}{g(x)} = \lim_{x \to +\infty} \frac{f'(x)}{g'(x)} = \lim_{x \to +\infty} \left[ b(x) - a(x) \frac{f(x)}{g(x)} \right].
\end{align*}
但是注意这里并不能直接使用极限运算的四则运算法则得到结果,这是因为 \(\lim_{x \to +\infty} \frac{f(x)}{g(x)}\) 不一定存在。 
\end{remark}
\begin{proof}
令 \(p(x) = e^{\int_0^x a(x) \frac{g'(x)}{g(x)} \mathrm{d}x}\),则 \(p'(x) = a(x) \frac{g'(x)}{g(x)}\),进而
\begin{align}
\frac{p'(x)}{p(x)} = \frac{a(x) \frac{g'(x)}{g(x)} e^{\int_0^x a(x) \frac{g'(x)}{g(x)} \mathrm{d}x}}{e^{\int_0^x a(x) \frac{g'(x)}{g(x)} \mathrm{d}x}} = a(x) \frac{g'(x)}{g(x)}. \label{eq:100.14}
\end{align}
于是由条件可得
\begin{align}
f'(x) + a(x) \frac{g'(x)}{g(x)} f(x) = b(x) g'(x) \Longleftrightarrow f'(x) + \frac{p'(x)}{p(x)} f(x) = b(x) g'(x), \quad \forall x \in \mathbb{R}. \label{eq:100.15}
\end{align}
又由 \(\lim_{x \to +\infty} a(x) = A > 0\) 可知,存在 \(M > 0\),使得
\begin{align*}
a(x) \geqslant \frac{A}{2}, \quad \forall x > M.
\end{align*}
从而对 \(\forall x > M\),我们有
\begin{align*}
p(x) &= e^{\int_0^x a(x) \frac{g'(x)}{g(x)} \mathrm{d}x} \geqslant e^{\int_M^x a(x) \frac{g'(x)}{g(x)} \mathrm{d}x} \\
&\geqslant e^{\frac{A}{2} \int_M^x \frac{g'(x)}{g(x)} \mathrm{d}x} = e^{\frac{A}{2} \ln \frac{g(x)}{g(M)}} = \left[ \frac{g(x)}{g(M)} \right]^{\frac{A}{2}}.
\end{align*}
令 \(x \to +\infty\),则由 \(\lim_{x \to +\infty} g(x) = +\infty\) 可知 \(\lim_{x \to +\infty} p(x) = +\infty\)。因此,利用 L'Hospital 法则,再结合 \eqref{eq:100.14} 和 \eqref{eq:100.15} 式,可得
\begin{align*}
\lim_{x \to +\infty} \frac{f(x)}{g(x)} &= \lim_{x \to +\infty} \frac{f(x) p(x)}{g(x) p(x)} \xlongequal{\text{L'Hospital}} \lim_{x \to +\infty} \frac{f'(x) p(x) + f(x) p'(x)}{g'(x) p(x) + g(x) p'(x)} \\
&= \lim_{x \to +\infty} \frac{f'(x) + \frac{p'(x)}{p(x)} f(x)}{g'(x) + \frac{p'(x)}{p(x)} g(x)} = \lim_{x \to +\infty} \frac{b(x) g'(x)}{g'(x) + a(x) \frac{g'(x)}{g(x)} g(x)} \\
&= \lim_{x \to +\infty} \frac{b(x)}{1 + a(x)} = \frac{B}{A + 1}.
\end{align*}
\end{proof}

\begin{proposition}[L'Hospital法则(复变函数版本)]\label{proposition:L'Hospital法则(复变函数版本)}
设 \( f(x) = u(x) + \mathrm{i} v(x) \),\( g(x) \) 为实值函数,且 \( \lim_{x \to a^+} g(x) = +\infty \),若 \( \lim_{x \to a^+} \frac{f'(x)}{g'(x)} = z_0 \),则 \( \lim_{x \to a^+} \frac{f(x)}{g(x)} = z_0 \)。
\end{proposition}
\begin{proof}
由实数 L’Hospital 法则可得
\begin{align*}
\lim_{x \to a^+} \left( \frac{u'(x)}{g'(x)} + \mathrm{i} \frac{v'(x)}{g'(x)} \right) = z_0 &\Rightarrow \lim_{x \to a^+} \frac{u(x)}{g(x)} = \mathrm{Re} z_0,\ \lim_{x \to a^+} \frac{v(x)}{g(x)} = \mathrm{Im} z_0 \\
&\Rightarrow \lim_{x \to a^+} \frac{f(x)}{g(x)} = \lim_{x \to a^+} \frac{u(x) + \mathrm{i} v(x)}{g(x)} = z_0.
\end{align*} 
\end{proof}

\begin{example}
设 \(f(x)\) 在 \(\mathbb{R}\) 上二阶可微且 \(a, b \in \mathbb{R}\),满足 \(a > 0\),\(a^2 - 4b < 0\) 或者 \(a > 0\),\(b > 0\),\(a^2 - 4b > 0\) 且有 \(\lim_{x \to +\infty} (f''(x) + af'(x) + bf(x)) = \ell \in \mathbb{R}\),证明:\(\lim_{x \to +\infty} f(x) = \frac{\ell}{b}\),\(\lim_{x \to +\infty} f'(x) = \lim_{x \to +\infty} f''(x) = 0\)。 
\end{example}
\begin{note}
对于二阶微分方程而言,一般考虑降阶.本题利用L'Hospital法则实现降阶.
\end{note}
\begin{proof}
不妨设 \( l=0 \),否则用 \( f(x) - \frac{l}{b} \) 代替 \( f(x) \) 即可。

\one 当 \( a>0 \)、\( b>0 \)、\( a^2-4b>0 \) 时,考虑二次方程 \( x^2+ax+b=0 \),则此时该方程必有两负根。设这两个负根分别为 \( \lambda_1,\lambda_2<0 \),则 \( x^2+ax+b = x^2 + (\lambda_1+\lambda_2)x + \lambda_1\lambda_2 \)。注意到
\begin{align*}
\left[ e^{-\lambda_2 x} \left( f'(x) - \lambda_1 f(x) \right) \right]' = e^{\lambda_2 x} \left[ f''(x) + (\lambda_1+\lambda_2) f'(x) + \lambda_1\lambda_2 f(x) \right] = e^{\lambda_2 x} \left[ f''(x) + a f'(x) + b f(x) \right],
\end{align*}
因此由条件可得
\[
\frac{\left[ e^{-\lambda_2 x} \left( f'(x) - \lambda_1 f(x) \right) \right]'}{\left( e^{-\lambda_2 x} \right)'} = \frac{f''(x) + a f'(x) + b f(x)}{-\lambda_2} \to 0,\ x \to +\infty.
\]
从而利用 L’Hospital 法则可得
\[
\lim_{x \to +\infty} \left[ f'(x) - \lambda_1 f(x) \right] = \lim_{x \to +\infty} \frac{e^{-\lambda_2 x} \left( f'(x) - \lambda_1 f(x) \right)}{e^{-\lambda_2 x}} = \lim_{x \to +\infty} \frac{\left[ e^{-\lambda_2 x} \left( f'(x) - \lambda_1 f(x) \right) \right]'}{\left( e^{-\lambda_2 x} \right)'} = 0.
\]
又注意到
\[
\left[ e^{-\lambda_1 x} f(x) \right]' = e^{-\lambda_1 x} \left[ f'(x) - \lambda_1 f(x) \right],
\]
因此
\[
\frac{\left[ e^{-\lambda_1 x} f(x) \right]'}{\left( e^{-\lambda_1 x} \right)'} = \frac{f'(x) - \lambda_1 f(x)}{-\lambda_1} \to 0,\ x \to +\infty.
\]
于是再利用 L’Hospital 法则可得
\[
\lim_{x \to +\infty} f(x) = \lim_{x \to +\infty} \frac{e^{-\lambda_1 x} f(x)}{e^{-\lambda_1 x}} = \lim_{x \to +\infty} \frac{\left[ e^{-\lambda_1 x} f(x) \right]'}{\left( e^{-\lambda_1 x} \right)'} = 0.
\]
故由 \( \lim_{x \to +\infty} \left[ f'(x) - \lambda_1 f(x) \right] = 0 \) 和极限的四则运算法则可得
\[
\lim_{x \to +\infty} f'(x) = \lim_{x \to +\infty} \left[ f'(x) - \lambda_1 f(x) \right] + \lambda_1 \lim_{x \to +\infty} f(x) = 0.
\]
进而再由 \( \lim_{x \to +\infty} \left[ f''(x) + a f'(x) + b f(x) \right] = 0 \) 可得
\[
\lim_{x \to +\infty} f''(x) = \lim_{x \to +\infty} \left[ f''(x) + a f'(x) + b f(x) \right] - a \lim_{x \to +\infty} f'(x) - b \lim_{x \to +\infty} f(x) = 0.
\]

\two 当 \( a>0 \)、\( a^2-4b<0 \) 时,考虑二次方程 \( x^2+ax+b=0 \),则此时该方程必有两复根,并且 \( \lambda_1+\lambda_2=a<0 \)。

于是设这两个复根分别为 \( \lambda_1=-u+v\mathrm{i} \)、\( \lambda_2=-u-v\mathrm{i} \)(\( u>0 \),\( v \in \mathbb{R} \setminus \{0\} \)),则 \( x^2+ax+b = x^2 + (\lambda_1+\lambda_2)x + \lambda_1\lambda_2 \)。
从而由 L’Hospital 法则可得
\begin{align*}
\lim_{x \to +\infty} e^{\mathrm{i} v x} \left[ f'(x) - \lambda_1 f(x) \right] &= \lim_{x \to +\infty} \frac{e^{(u+\mathrm{i} v)x} \left[ f'(x) - \lambda_1 f(x) \right]}{e^{u x}} \xlongequal{\text{\hyperref[proposition:L'Hospital法则(复变函数版本)]{L’Hospital 法则(复变函数版本)}}} \lim_{x \to +\infty} \frac{\left[ e^{(u+\mathrm{i} v)x} \left( f'(x) - \lambda_1 f(x) \right) \right]'}{\left( e^{u x} \right)'} \\
&= \lim_{x \to +\infty} \frac{\left[ e^{-\lambda_2 x} \left( f'(x) - \lambda_1 f(x) \right) \right]'}{\left( e^{u x} \right)'} = \lim_{x \to +\infty} \frac{e^{-\lambda_2 x} \left[ f''(x) + (\lambda_1+\lambda_2) f'(x) + \lambda_1\lambda_2 f(x) \right]}{u e^{u x}} \\
&= \lim_{x \to +\infty} \frac{e^{(u+\mathrm{i} v)x} \left[ f''(x) + a f'(x) + b f(x) \right]}{u e^{u x}} = \lim_{x \to +\infty} \frac{e^{\mathrm{i} v x} \left[ f''(x) + a f'(x) + b f(x) \right]}{u} = 0,
\end{align*}
因此
\[
0 = \lim_{x \to +\infty} \left[ f'(x) - \lambda_1 f(x) \right] = \lim_{x \to +\infty} \left[ f'(x) + u f(x) + \mathrm{i} v f(x) \right].
\]
于是
\[
\lim_{x \to +\infty} \left[ f'(x) + u f(x) \right] = \lim_{x \to +\infty} v f(x) = 0.
\]
又因为 \( u>0 \)、\( v \ne 0 \),所以 \( \lim_{x \to +\infty} f(x) = 0 \),进而 \( \lim_{x \to +\infty} f'(x) = 0 \)。再由 \( \lim_{x \to +\infty} \left[ f''(x) + a f'(x) + b f(x) \right] = 0 \) 可得 \( \lim_{x \to +\infty} f''(x) = 0 \)。

综上,结论得证。
\end{proof}
\begin{remark}
第\two 中情况中不使用\hyperref[proposition:L'Hospital法则(复变函数版本)]{L'Hospital 法则(复变函数版本)}的方法:
考虑
\[
e^{-\lambda_2 x} \left[ f'(x) - \lambda_1 f(x) \right] = e^{(u+\mathrm{i} v)x} \left[ f'(x) - (-u+\mathrm{i} v) f(x) \right] = e^{u x} (\cos v x + \mathrm{i} \sin v x) \left[ f'(x) + u f(x) - \mathrm{i} v f(x) \right].
\]
则上述复变函数实部和虚部分别为
\[
\text{实部} : e^{u x} \left[ (f'(x) + u f(x)) \cos v x + v f(x) \sin v x \right] ;
\]
\[
\text{虚部} : e^{u x} \left[ (f'(x) + u f(x)) \sin v x - v f(x) \cos v x \right].
\]
于是利用 L’Hospital 法则可得
\begin{align*}
\lim_{x \to +\infty} \left[ (f'(x) + u f(x)) \cos v x + v f(x) \sin v x \right] &= \lim_{x \to +\infty} \frac{e^{u x} \left[ (f'(x) + u f(x)) \cos v x + v f(x) \sin v x \right]}{e^{u x}} \\
&= \lim_{x \to +\infty} \frac{\left[ e^{u x} \left( (f'(x) + u f(x)) \cos v x + v f(x) \sin v x \right) \right]'}{\left( e^{u x} \right)'} \\
&= \lim_{x \to +\infty} \frac{e^{u x} \cos v x \left[ f''(x) + a f'(x) + b f(x) \right]}{u e^{u x}} 
\\
&= \lim_{x \to +\infty} \frac{\cos v x \left[ f''(x) + a f'(x) + b f(x) \right]}{u} = 0.
\end{align*}
同理利用 L’Hospital 法则可得
\[
\lim_{x \to +\infty} \left[ (f'(x) + u f(x)) \sin v x - v f(x) \cos v x \right] = 0.
\]
因此当 \( x \to +\infty \) 时,\( e^{-\lambda_2 x} \left[ f'(x) - \lambda_1 f(x) \right] \) 的实部和虚部都趋于 0,故
\[
\lim_{x \to +\infty} e^{-\lambda_2 x} \left[ f'(x) - \lambda_1 f(x) \right] = 0.
\]
从而 \( \lim_{x \to +\infty} \left[ f'(x) - \lambda_1 f(x) \right] = 0 \),后续同理可证 \( \lim_{x \to +\infty} f'(x) = \lim_{x \to +\infty} f''(x) = 0 \)。
\end{remark}

\vspace{0.5cm}

\begin{example}
给定正整数 \( n \),设 \( f(x) \in C^n[-1,1] \),\( |f(x)| \leq 1 \),证明:存在与 \( f(x) \) 无关的常数 \( C \),使得只要 \( |f'(0)| \geq C \),\( f^{(n)}(x) \) 在 \( (-1,1) \) 中就会有至少 \( n - 1 \) 个不同的根。
\end{example}
\begin{proof}
证明见豌豆(2024-2025竞赛班下数学类讲义洛必达法则部分),本题证明直观上定性分析比较容易,但是要严谨地书写过程比较繁琐(证明太麻烦没看).
\end{proof}

\begin{example}
设 \( f(x) \) 在 \( (0,1) \) 中任意阶可导且各阶导数均非负,证明:\( f(x) \) 是实解析函数.(伯恩斯坦定理) 
类似的,如果 \( (-1)^n f^{(n)}(x) \geq 0 \) 恒成立,则 \( f(x) \) 也是实解析的. 
\end{example}
\begin{proof}
对 \(\forall x \in (0,1)\),固定 \(x\),则任取 \(h>0\),使得 \(x+2h \in (0,1)\)。于是由 Taylor 定理可知
\begin{align*}
f(x+2h) = f(x) + f'(x) \cdot 2h + \cdots + \frac{f^{(n)}(x)}{n!} (2h)^n + \frac{1}{n!} \int_x^{x+2h} f^{(n+1)}(t) (x+2h-t)^n \, \mathrm{d}t.
\end{align*}
又由于 \(f\) 任意阶导数均非负,故 \(f\) 的任意阶导数都是单调递增函数。从而
\begin{align*}
\frac{1}{n!} \int_x^{x+h} f^{(n+1)}(t) (x+h-t)^n \, \mathrm{d}t &\leqslant \frac{1}{n!} \int_x^{x+h} f^{(n+1)}(2t-x) (x+h-t)^n \, \mathrm{d}t \\
&\xlongequal{u=2t-x} \frac{1}{2^{n+1} n!} \int_x^{x+2h} f^{(n+1)}(u) (x+2h-u)^n \, \mathrm{d}u \\
&= \frac{1}{2^{n+1}} \left[ f(x+2h) - \left( f(x) + f'(x) \cdot 2h + \cdots + \frac{f^{(n)}(x)}{n!} (2h)^n \right) \right] \\
&\leqslant \frac{f(x+2h) - f(x)}{2^{n+1}} \to 0, \quad n \to \infty.
\end{align*}
因此 \(f\) 可以在 \(x\) 的任意右邻域展开成幂级数(因为余项趋于0),即
\[
f(y) = f(x) + f'(x) (y-x) + \cdots + \frac{f^{(n)}(x)}{n!} (y-x)^n + \cdots = \sum_{n=0}^{+\infty} \frac{f^{(n)}(x)}{n!} (y-x)^n, \quad \forall y \in U_+(x).
\]
但是同样的方法对于\(x < 0\)时似乎难以处理,因为单调性对不上,所以换个方法(可以一次解决问题,直接对高阶导数进行估计,由此说明余项趋于零,也无需讨论正负)

设\(\vert f(x)\vert\)在\([-\frac{3}{4},\frac{3}{4}]\)中的最大值为\(M\),对任意\(\vert x\vert<\frac{1}{4}\)有
\begin{align*}
f\left(x + \frac{1}{2}\right)=f(x)+\sum_{k = 1}^{n - 1}\frac{f^{(k)}(x)}{k!}\frac{1}{2^{k}}+\frac{f^{(n)}(x)}{n!}\frac{1}{2^{n}}+\frac{f^{(n + 1)}(\xi)}{(n + 1)!}\frac{1}{2^{n + 1}}\geq f(x)+\frac{f^{(n)}(x)}{n!}\frac{1}{2^{n}}
\end{align*}
由此得到
\begin{align*}
0\leq\frac{f^{(n)}(x)}{n!}\leq 2^{n}\left(f\left(x+\frac{1}{2}\right)-f(x)\right)\leq 2^{n + 1}M,\forall x\in\left(-\frac{1}{4},\frac{1}{4}\right)
\end{align*}
进而
\begin{align*}
\left\vert f(x)-\sum_{k = 0}^{n}\frac{f^{(k)}(0)}{k!}x^{k}\right\vert=\frac{\vert f^{(n + 1)}(\xi)\vert}{(n + 1)!}\vert x\vert^{n + 1}\leq 2^{n + 2}M\vert x\vert^{n + 1}\leq 2^{n + 2}M\frac{1}{4^{n + 1}}\to 0,n\to\infty
\end{align*}
所以\(f(x)=\sum_{n = 0}^{\infty}\frac{f^{(n)}(0)}{n!}x^{n},\forall x\in\left(-\frac{1}{4},\frac{1}{4}\right)\),这就证明了实解析

对于第二问,考虑函数\(f(-x)\)即可 .
\end{proof}

\vspace{0.5cm}

\begin{example}
设 \( g(x) \) 是 \( (0,+\infty) \) 中恒正的连续函数,\( a>0 \) 使得 \( \lim_{x \to +\infty} \frac{g(x)}{x^{1 + a}} = +\infty \),若 \( f(x) \) 在 \( (0,+\infty) \) 中恒正且二阶可导,满足 \( f''(x) + f'(x) > g(f(x)) \) 恒成立,证明:\( \lim_{x \to +\infty} f(x) = 0 \)。
\end{example}
\begin{proof}
由 \( f''(x) + f'(x) > g(f(x)) > 0, \forall x \in (0,+\infty) \) 可得
\begin{align*}
\left( e^x f'(x) \right)' = e^x \left( f''(x) + f'(x) \right) > 0, \forall x \in (0,+\infty).
\end{align*}
从而 \( e^x f'(x) \) 在 \( (0,+\infty) \) 上严格递增。

(i) 若 \( e^x f'(x) \) 在 \( (0,+\infty) \) 上无零点,则
\[
e^x f'(x) > 0 \Rightarrow f'(x) > 0, \forall x \in (0,+\infty).
\]

(ii) 若 \( e^x f'(x) \) 在 \( (0,+\infty) \) 上有零点,则由其严格递增性可知,存在唯一的 \( a > 0 \),使得 \( e^a f'(a) = 0 \)。于是
\[
e^x f'(x) > e^a f'(a) = 0, \forall x \in (a,+\infty).
\]
故一定存在 \( X > 0 \),使得 \( f'(x) > 0, \forall x \in (X,+\infty) \)。从而 \( f(x) \) 在 \( (X,+\infty) \) 上严格递增。

由 \( f''(x) + f'(x) > g(f(x)) > 0, \forall x \in (0,+\infty) \) 还可以得到
\begin{align*}
\left[ f'(x) + f(x) \right]' = f''(x) + f'(x) > 0, \forall x \in (0,+\infty).
\end{align*}
于是 \( f'(x) + f(x) \) 在 \( (0,+\infty) \) 上严格递增。从而 \( \lim_{x \to +\infty} \left[ f'(x) + f(x) \right] = L \) 为有限数或 \( +\infty \)(广义存在)。

由 L'Hospital 法则可得
\begin{align*}
\lim_{x \to +\infty} f(x) = \lim_{x \to +\infty} \frac{e^x f(x)}{e^x} = \lim_{x \to +\infty} \frac{e^x \left[ f'(x) + f(x) \right]}{e^x} = \lim_{x \to +\infty} \left[ f'(x) + f(x) \right] = L.
\end{align*}
又由 \( f \) 恒正可知 \( L \geq 0 \)。反证,假设 \( L \neq 0 \),则

\one 当 \( L \in (0,+\infty) \) 时,此时,由 \( \lim_{x \to +\infty} \left[ f'(x) + f(x) \right] = \lim_{x \to +\infty} f(x) = L \) 可得 \( \lim_{x \to +\infty} f'(x) = 0 \)。

再对 \( f''(x) + f'(x) > g(f(x)) \) 两边同时令 \( x \to +\infty \),可得
\begin{align*}
\liminf_{x \to +\infty} f''(x) + \lim_{x \to +\infty} f'(x) \geq \lim_{x \to +\infty} g(f(x)) = g\left( \lim_{x \to +\infty} f(x) \right).
\end{align*}
即 \( \liminf_{x \to +\infty} f''(x) \geq g(L) \)。于是由 Lagrange 中值定理可知,存在 \( c > X + 1 \),使得
\begin{align*}
f'(x) = f'(X + 1) + f''(c) (x - X - 1) \geq f'(X + 1) + g(L) (x - X - 1), \forall x > X + 1.
\end{align*}
令 \( x \to +\infty \),得 \( \lim_{x \to +\infty} f'(x) = +\infty \),这与 \( \lim_{x \to +\infty} f'(x) = 0 \) 矛盾!

\two 当 \( L = +\infty \) 时,此时 \( \lim_{x \to +\infty} \left[ f'(x) + f(x) \right] = \lim_{x \to +\infty} f(x) = +\infty \)。由 \( f''(x) + f'(x) > g(f(x)) \) 可得
\begin{align*}
\frac{f''(x) + f'(x)}{\left( f(x) \right)^{1 + a}} > \frac{g(f(x))}{\left( f(x) \right)^{1 + a}} = \frac{g(x)}{x^{1 + a}}.
\end{align*}
从而由 \( \lim_{x \to +\infty} \frac{g(x)}{x^{1 + a}} = +\infty \) 可得 \( \lim_{x \to +\infty} \frac{f''(x) + f'(x)}{\left( f(x) \right)^{1 + a}} = +\infty \)。

由 \( f(x) > 0, \forall x > X \) 可得,对 \( \forall x > X \),我们有
\begin{align*}
\frac{f''(x) + f'(x)}{\left( f(x) \right)^{1 + a}} = \frac{\left( f''(x) + f'(x) \right) \left( f'(x) + f(x) \right)}{\left( f(x) \right)^{1 + a} \left( f'(x) + f(x) \right)} < \frac{\left( f''(x) + f'(x) \right) \left( f'(x) + f(x) \right)}{\left( f(x) \right)^{1 + a} f'(x)}.
\end{align*}
令 \( x \to +\infty \),得 \( \lim_{x \to +\infty} \frac{\left( f''(x) + f'(x) \right) \left( f'(x) + f(x) \right)}{\left( f(x) \right)^{1 + a} f'(x)} = +\infty \)。

又由 L'Hospital 法则可知
\begin{align*}
\lim_{x \to +\infty} \frac{\left( f'(x) + f(x) \right)^2}{\left( f(x) \right)^{2 + a}} = \lim_{x \to +\infty} \frac{\left[ \left( f'(x) + f(x) \right)^2 \right]'}{\left[ \left( f(x) \right)^{2 + a} \right]'} = \lim_{x \to +\infty} \frac{\left( f''(x) + f'(x) \right) \left( f'(x) + f(x) \right)}{(2 + a) \left( f(x) \right)^{1 + a} f'(x)} = +\infty.
\end{align*}
于是 \( \lim_{x \to +\infty} \frac{f'(x) + f(x)}{\left( f(x) \right)^{1 + \frac{a}{2}}} = +\infty \)。又由 \( \lim_{x \to +\infty} f(x) = +\infty \) 可得 \( \lim_{x \to +\infty} \frac{f(x)}{\left( f(x) \right)^{1 + \frac{a}{2}}} = 0 \)。因此 \( \lim_{x \to +\infty} \frac{f'(x)}{\left( f(x) \right)^{1 + \frac{a}{2}}} = +\infty \)。
故存在 \( M > X + 1 \),使得
\begin{align*}
\frac{f'(x)}{\left( f(x) \right)^{1 + \frac{a}{2}}} > 1, \forall x > M.
\end{align*}
两边同时积分可得
\begin{align*}
\int_M^{+\infty} \frac{1}{x^{1 + \frac{a}{2}}} \, \mathrm{d}x = \int_M^{+\infty} \frac{1}{\left( f(x) \right)^{1 + \frac{a}{2}}} \, \mathrm{d}f(x) = \int_M^{+\infty} \frac{f'(x)}{\left( f(x) \right)^{1 + \frac{a}{2}}} \, \mathrm{d}x \geqslant  \int_M^{+\infty} \, \mathrm{d}x = +\infty.
\end{align*}
而 \( \int_M^{+\infty} \frac{1}{x^{1 + \frac{a}{2}}} \, \mathrm{d}x \) 收敛,矛盾!
\end{proof}

\vspace{0.5cm}

\begin{example}
设 \( f(x) \) 非负且二阶可导,\( \lim_{x \to +\infty} \frac{f''(x)}{f(x)(1 + f'^2(x))^2} = +\infty \),证明:
\[
\lim_{x \to +\infty} \int_0^x \frac{\sqrt{1 + f'^2(t)}}{f(t)} \, \mathrm{d}t \int_x^{+\infty} f(t) \sqrt{1 + f'^2(t)} \, \mathrm{d}t = 0.
\] 
\end{example}
\begin{proof}
由条件可知 存在 \( X > 0 \) 使得
\begin{align*}
\frac{f''(x)}{f(x)(1 + f'^2(x))^2} > 0 \quad ,\forall x > X \implies f''(x) > 0 \quad ,\forall x > X.
\end{align*}
从而 \( f(x) \) 在 \( (X, +\infty) \) 上下凸 \( f'(x) \) 在 \( (X, +\infty) \) 上递增 于是由下凸函数的单调性可知 \( f \) 在 \( (X, +\infty) \) 上的单调性只有三种情况 递减、递增、先递减再递增 若 \( f(x) \) 在 \( (X, +\infty) \) 上递增或者先递减再递增 则一定存在 \( X_2 > X \) 使得 \( f(x) \) 在 \( (X_2, +\infty) \) 上递增 现在只在 \( (X_2, +\infty) \) 上进行考虑 由 \( f \) 递增且非负可知 \( \lim_{x \to +\infty} f(x) \triangleq A_1 \) 为正数或 \( +\infty \) 假设 \( A_1 \) 为某个正数 则
\begin{align*}
+\infty = \lim_{x \to +\infty} \frac{f''(x)}{f(x)(1 + f'^2(x))^2} = \frac{1}{A_1} \lim_{x \to +\infty} \frac{f''(x)}{(1 + f'^2(x))^2}.
\end{align*}
从而 \( \lim_{x \to +\infty} f''(x) = +\infty \) 于是由 Lagrange 中值定理可知 存在 \( \eta > X_2 + 1 \) 使得
\[
f'(x) = f'(X_2 + 1) + f''(\eta)(x - X_2 - 1) \quad, \forall x > X_2 + 1.
\]
令 \( x \to +\infty \) 则 \( \lim_{x \to +\infty} f'(x) = +\infty \) 再利用 Lagrange 中值定理同理可得 \( \lim_{x \to +\infty} f(x) = +\infty \) 这与 \( \lim_{x \to +\infty} f(x) \triangleq A_1 \) 为某个正数矛盾 故 \( \lim_{x \to +\infty} f(x) = +\infty \) 再利用 L’Hospital 法则可得
\begin{align*}
\lim_{x \to +\infty} \frac{-\frac{1}{1 + (f'(x))^2}}{f^2(x)} &= \lim_{x \to +\infty} \frac{\left[ -\frac{1}{1 + (f'(x))^2} \right]'}{\left[ f^2(x) \right]'} = \lim_{x \to +\infty} \frac{\frac{2f''(x)f'(x)}{(1 + f'^2(x))^2}}{2f(x)f'(x)} \\
&= \lim_{x \to +\infty} \frac{f''(x)}{f(x)(1 + f'^2(x))^2} = +\infty.
\end{align*}
而 \( \lim_{x \to +\infty} \frac{-\frac{1}{1 + (f'(x))^2}}{f^2(x)} \leqslant 0 \) 矛盾 故 \( f(x) \) 在 \( (X, +\infty) \) 上必然单调递减 则 \( f'(x) < 0 \quad \forall x > X \) 又 \( f(x) > 0 \) 故 \( \lim_{x \to +\infty} f(x) \triangleq A \geqslant 0 \) 由 \( f'(x) \) 在 \( (X, +\infty) \) 上递增可知 \( \lim_{x \to +\infty} f'(x) \) 存在且 \( \lim_{x \to +\infty} f'(x) \leqslant 0 \) 假设 \( \lim_{x \to +\infty} f'(x) \triangleq A' < 0 \) 则存在 \( X_1 > X \) 使得
\[
f'(x) < \frac{A'}{2} < 0 \quad \forall x > X_1.
\]
于是由 Lagrange 中值定理可知 存在 \( \xi > X_1 + 1 \) 使得
\[
f(x) = f(X_1 + 1) + f'(\xi)(x - X_1 - 1) < f(X_1 + 1) + \frac{A'}{2}(x - X_1 - 1) \quad \forall x > X_1 + 1.
\]
令 \( x \to +\infty \) 得 \( \lim_{x \to +\infty} f(x) = -\infty \) 这与 \( \lim_{x \to +\infty} f(x) \geqslant 0 \) 矛盾 故 \( \lim_{x \to +\infty} f'(x) = 0 \) 从而再由条件可得
\[
\lim_{x \to +\infty} \frac{f''(x)}{f(x)(1 + f'^2(x))^2} = \lim_{x \to +\infty} \frac{f''(x)}{f(x)} = +\infty.
\]
再考虑 \( \lim_{x \to +\infty} f(x) \) 已知 \( \lim_{x \to +\infty} f(x) \triangleq A \geqslant 0 \) 假设 \( A > 0 \) 则由 \( \lim_{x \to +\infty} f'(x) = 0 \) 及条件可得
\[
+\infty = \lim_{x \to +\infty} \frac{f''(x)}{f(x)(1 + f'^2(x))^2} = \frac{1}{A} \lim_{x \to +\infty} f''(x) \implies \lim_{x \to +\infty} f''(x) = +\infty.
\]
于是存在 \( M > X_1 + 1 \) 使得
\[
f''(x) > 1 \quad \forall x > M.
\]
从而由 Lagrange 中值定理可知 存在 \( \xi_1 > M + 1 \) 使得
\[
f'(x) = f'(M + 1) + f''(\xi_1)(x - M - 1) > f'(M + 1) + (x - M - 1) \quad \forall x > M + 1.
\]
令 \( x \to +\infty \) 则 \( \lim_{x \to +\infty} f'(x) = +\infty \) 再利用 Lagrange 中值定理同理可得 \( \lim_{x \to +\infty} f(x) = +\infty \) 这与 \( \lim_{x \to +\infty} f(x) = A > 0 \) 矛盾 故 \( \lim_{x \to +\infty} f(x) = 0 \) 综上可知 \( f(x) \) 在 \( (X, +\infty) \) 上递减 进而 \( f'(x) \leqslant 0 \) 并且 \( f'(x) \) 在 \( (X, +\infty) \) 上递增 还有
\[
\lim_{x \to +\infty} f(x) = \lim_{x \to +\infty} f'(x) = 0, \quad \lim_{x \to +\infty} \frac{f''(x)}{f(x)} = +\infty.
\]
于是显然 \( f(x) \geqslant 0 \) 从而存在 \( X' > X \) 使得
\begin{align}
f'(x) \leqslant 1 \quad \forall x > X' \label{100.21}.
\end{align}
又因为 \( f \in D^2(\mathbb{R}) \) 所以 \( f, f' \) 都连续 从而在 \( [0, X'] \) 上都有界 即存在 \( L > 0 \) 使得
\begin{align}
|f(x)|, |f'(x)| < L \quad \forall x \in [0, X'] \label{100.20}.
\end{align}
由 \( \lim_{x \to +\infty} \frac{f''(x)}{f(x)} = +\infty \) 可知 存在 \( X'' > X' \) 使得
\[
f''(x) > f(x) \quad \forall x > X''.
\]
从而结合 \( \lim_{x \to +\infty} f'(x) = 0 \) 可得
\begin{align}
\int_x^{+\infty} f(t) \, \mathrm{d}t < \int_x^{+\infty} f''(t) \, \mathrm{d}t = f'(+\infty) - f'(x) = -f'(x) \quad \forall x > X'' \label{100.18}.
\end{align}
于是由 \( \lim_{x \to +\infty} f'(x) = 0 \) 可得
\begin{align}
\lim_{x \to +\infty} \int_x^{+\infty} f(t) \, \mathrm{d}t = 0 \label{100.22}.
\end{align}
利用 L’Hospital 法则可得
\[
\lim_{x \to +\infty} \frac{(f'(x))^2}{(f(x))^2} = \lim_{x \to +\infty} \frac{[(f'(x))^2]'}{[(f(x))^2]'} = \lim_{x \to +\infty} \frac{f''(x)f'(x)}{f(x)f'(x)} = \lim_{x \to +\infty} \frac{f''(x)}{f(x)} = +\infty.
\]
又因为 \( f'(x) \leqslant 0 \) \( f(x) \geqslant 0 \) 所以
\[
\lim_{x \to +\infty} \frac{|f'(x)|}{|f(x)|} = \lim_{x \to +\infty} \frac{-f'(x)}{f(x)} = +\infty \implies \lim_{x \to +\infty} \frac{f(x)}{f'(x)} = 0.
\]
再结合 \eqref{100.22} 式及 \( \lim_{x \to +\infty} f(x) = 0 \) 利用 L’Hospital 法则可得
\begin{align}
\lim_{x \to +\infty} \frac{\int_x^{+\infty} f(t) \, \mathrm{d}t}{f(x)} = \lim_{x \to +\infty} \frac{f(x)}{f'(x)} = 0 \label{100.23}.
\end{align}
令 \( g(x) = \frac{1}{f(x)} \) 则由 \( \lim_{x \to +\infty} f(x) = 0 \) 可知 \( \lim_{x \to +\infty} g(x) = +\infty \) 并且
\[
0 = -\lim_{x \to +\infty} \frac{f(x)}{f'(x)} = -\lim_{x \to +\infty} \frac{\frac{1}{g(x)}}{-\frac{g'(x)}{g^2(x)}} = \lim_{x \to +\infty} \frac{g(x)}{g'(x)}.
\]
由L’Hospital 法则可得
\begin{align}
\lim_{x \to +\infty} f(x) \int_0^x \frac{1}{f(t)} \, \mathrm{d}t = \lim_{x \to +\infty} \frac{\int_0^x g(t) \, \mathrm{d}t}{g(x)} = \lim_{x \to +\infty} \frac{g(x)}{g'(x)} = 0 \label{100.19}.
\end{align}
于是由 \eqref{100.21} \eqref{100.20} \eqref{100.18} 式可得
\begin{align*}
\int_0^x \frac{\sqrt{1 + f'^2(t)}}{f(t)} \, \mathrm{d}t \int_x^{+\infty} f(t)\sqrt{1 + f'^2(t)} \, \mathrm{d}t 
&\leqslant \left( \int_0^{X''} \frac{\sqrt{1 + f'^2(t)}}{f(t)} \, \mathrm{d}t + \int_{X''}^x \frac{1}{f(t)} \, \mathrm{d}t \right) \sqrt{2} \int_x^{+\infty} f(t) \, \mathrm{d}t \\
&\leqslant \sqrt{2} \left( \int_0^{X''} \frac{\sqrt{1 + L^2}}{-L} \, \mathrm{d}t + \int_0^x \frac{1}{f(t)} \, \mathrm{d}t \right) \int_x^{+\infty} f(t) \, \mathrm{d}t \\
&\leqslant -\sqrt{2} \left( \frac{X'' \sqrt{1 + L^2}}{-L} + \int_0^x \frac{1}{f(t)} \, \mathrm{d}t \right) \int_x^{+\infty} f(t) \, \mathrm{d}t \\
&= \frac{\sqrt{2} X'' \sqrt{1 + L^2}}{L} \int_x^{+\infty} f(t) \, \mathrm{d}t - \sqrt{2} \int_0^x \frac{1}{f(t)} \, \mathrm{d}t \int_x^{+\infty} f(t) \, \mathrm{d}t \\
&\leqslant \frac{\sqrt{2} X'' \sqrt{1 + L^2}}{L} f'(x) - \sqrt{2} f(x) \int_0^x \frac{1}{f(t)} \, \mathrm{d}t \frac{\int_x^{+\infty} f(t) \, \mathrm{d}t}{f(x)} \quad ,\forall x > X''.
\end{align*}
令 \( x \to +\infty \) 则由 \eqref{100.19} \eqref{100.23} 式和 \( \lim_{x \to +\infty} f'(x) = 0 \) 可得
\[
\limsup_{x \to +\infty} \int_0^x \frac{\sqrt{1 + f'^2(t)}}{f(t)} \, \mathrm{d}t \int_x^{+\infty} f(t)\sqrt{1 + f'^2(t)} \, \mathrm{d}t \leqslant 0.
\]
故结论得证.
\end{proof}










\subsection{与方程的根有关的渐近估计}

\subsubsection{可以解出n的类型}

\begin{example}
设\(x^{2n + 1}+e^{x}=0\)的根记为\(x_n\),计算
\[
\lim_{n\rightarrow\infty}x_n,\lim_{n\rightarrow\infty}n(1 + x_n).
\]
\end{example}
\begin{solution}
注意到\(0^{2n + 1}+e^{0}>0,(-1)^{2n + 1}+e^{-1}<0\)且\(x^{2n + 1}+e^{x}\)严格单调递增,所以由零点存在定理可知,对每个\(n\in\mathbb{N}\),存在唯一的\(x_n\in(-1,0)\),使得
\[
x_n^{2n + 1}+e^{x_n}=0\Rightarrow\frac{x_n}{\ln(-x_n)} = 2n + 1\rightarrow +\infty,n\to +\infty.
\]
任取$\{x_n\}$的一个收敛子列$\{x_{n_k}\}$,又$x_n\in(-1,0)$,因此可设\(\lim_{k\rightarrow\infty}x_{n_k}=c\in[-1,0]\),则$\underset{k\rightarrow +\infty}{\lim}\frac{x_{n_k}}{\ln \left( -x_{n_k} \right)}=\frac{c}{\ln \left( -c \right)}$.又因为$\underset{n\rightarrow +\infty}{\lim}\frac{x_n}{\ln \left( -x_n \right)}=+\infty$,所以由Heine归结原则可知$\underset{k\rightarrow +\infty}{\lim}\frac{x_{n_k}}{\ln \left( -x_{n_k} \right)}=+\infty$.从而
\[
\underset{k\rightarrow +\infty}{\lim}\frac{x_{n_k}}{\ln \left( -x_{n_k} \right)}=\frac{c}{\ln(-c)}=+\infty,
\]
故$c = - 1$.
于是由\hyperref[proposition:子列极限命题]{子列极限命题(a)}知\(\lim_{n\rightarrow\infty}x_n=-1\).因此
\begin{align*}
\lim_{n\rightarrow\infty}n(1 + x_n)=\frac{1}{2}\lim_{n\rightarrow\infty}(2n + 1)(1 + x_n)
=\frac{1}{2}\lim_{n\rightarrow\infty}\frac{x_n(1 + x_n)}{\ln(-x_n)}
=\frac{1}{2}\lim_{x\rightarrow - 1^+}\frac{x(1 + x)}{\ln(-x)}
=\frac{1}{2}.
\end{align*}
\end{solution}

\begin{example}
设\(a_n\in(0,1)\)是\(x^n + x = 1\)的根,证明
\[
a_n=1-\frac{\ln n}{n}+o\left(\frac{\ln n}{n}\right).
\]
\end{example}
\begin{proof}
注意到\(0^n + 0 - 1 < 0\),\(1^n + 1 - 1 > 0\),且\(x^n + x - 1\)在\((0, 1)\)上严格单调递增,所以由零点存在定理可知,对\(\forall n\in \mathbb{N}_+\),存在唯一的\(a_n\in (0, 1)\),使得
\begin{align}\label{example4.16-1.1}
a_{n}^{n} + a_n = 1 \Rightarrow \frac{\ln(1 - a_n)}{\ln a_n} = n \rightarrow +\infty, n \rightarrow +\infty.  
\end{align}
任取\(\{ a_n \}\)的一个收敛子列\(\{ a_{n_k} \}\),又\(a_n\in (0, 1)\),因此可设\(\lim_{k\rightarrow +\infty}a_{n_k} = c\in [0, 1]\),则\(\lim_{k\rightarrow +\infty}\frac{\ln(1 - a_{n_k})}{\ln a_{n_k}} = \frac{\ln(1 - c)}{\ln c}\).又由\((1.1)\)式可知\(\lim_{n\rightarrow +\infty}\frac{\ln(1 - a_n)}{\ln a_n} = +\infty\),所以由Heine归结原则可知\(\lim_{k\rightarrow +\infty}\frac{\ln(1 - a_{n_k})}{\ln a_{n_k}} = +\infty\).从而
\[
\lim_{k\rightarrow +\infty}\frac{\ln(1 - a_{n_k})}{\ln a_{n_k}} = \frac{\ln(1 - c)}{\ln c} = +\infty.
\]
故\(c = 1\),于是由\hyperref[proposition:子列极限命题]{子列极限命题(a)}可知
\begin{align}\label{example4.16-1.2}
\lim_{n\rightarrow +\infty}a_n = c = 1.  
\end{align}
而要证\(a_n = 1 - \frac{\ln n}{n} + o\left(\frac{\ln n}{n}\right), n \rightarrow +\infty\),等价于证明\(\lim_{n\rightarrow +\infty}\frac{a_n - 1 + \frac{\ln n}{n}}{\frac{\ln n}{n}} = \lim_{n\rightarrow +\infty}\frac{na_n - n + \ln n}{\ln n} = 0\).利用\eqref{example4.16-1.1}\eqref{example4.16-1.2}式可得
\begin{align}
\underset{n\rightarrow +\infty}{\lim}\frac{na_n-n+\ln n}{\ln n}&=\underset{n\rightarrow +\infty}{\lim}\left[ \frac{\frac{\ln \left( 1-a_n \right)}{\ln a_n}\cdot a_n-\frac{\ln \left( 1-a_n \right)}{\ln a_n}}{\ln \frac{\ln \left( 1-a_n \right)}{\ln a_n}}+1 \right] =\underset{n\rightarrow +\infty}{\lim}\left[ \frac{\left( a_n-1 \right) \ln \left( 1-a_n \right)}{\ln a_n\left( \ln \frac{\ln \left( 1-a_n \right)}{\ln a_n} \right)}+1 \right] \nonumber
\\
&=\underset{x\rightarrow 1^-}{\lim}\left[ \frac{\left( x-1 \right) \ln \left( 1-x \right)}{\ln x\left( \ln \frac{\ln \left( 1-x \right)}{\ln x} \right)}+1 \right] =\underset{x\rightarrow 0^-}{\lim}\left[ \frac{x\ln \left( -x \right)}{\ln \left( 1+x \right) \left( \ln \frac{\ln \left( -x \right)}{\ln \left( 1+x \right)} \right)}+1 \right] .
\label{example4.16-1.3}
\end{align}
由L'Hospital's rules可得
\begin{align}
\underset{x\rightarrow 0^-}{\lim}\frac{x\ln \left( -x \right)}{\ln \left( 1+x \right) \left( \ln \frac{\ln \left( -x \right)}{\ln \left( 1+x \right)} \right)}&=\underset{x\rightarrow 0^-}{\lim}\frac{\ln \left( -x \right)}{\ln \frac{\ln \left( -x \right)}{\ln \left( 1+x \right)}}\xlongequal{\text{L'Hospital's rules}}\underset{x\rightarrow 0^-}{\lim}\frac{\frac{1}{x}}{\frac{\ln \left( 1+x \right)}{\ln \left( -x \right)}\cdot \frac{\frac{1}{x}\ln \left( 1+x \right) -\frac{1}{1+x}\ln \left( -x \right)}{\ln ^2\left( 1+x \right)}}\nonumber
\\
&=\underset{x\rightarrow 0^-}{\lim}\frac{\ln \left( -x \right) \cdot \ln \left( 1+x \right)}{\ln \left( 1+x \right) -\frac{x}{1+x}\ln \left( -x \right)}=\underset{x\rightarrow 0^-}{\lim}\frac{x\ln \left( -x \right)}{\ln \left( 1+x \right) -\frac{x}{1+x}\ln \left( -x \right)}\nonumber
\\
&=\underset{x\rightarrow 0^-}{\lim}\frac{x}{\frac{\ln \left( 1+x \right)}{\ln \left( -x \right)}-\frac{x}{1+x}}=\underset{x\rightarrow 0^-}{\lim}\frac{x}{-\frac{x}{1+x}}=-1.
\label{example4.16-1.4}
\end{align}
于是结合\eqref{example4.16-1.3}\eqref{example4.16-1.4}式可得
\[
\lim_{n\rightarrow +\infty}\frac{na_n - n + \ln n}{\ln n} = \lim_{x\rightarrow 0^-}\left[ \frac{x\ln(-x)}{\ln(1 + x)\left(\ln \frac{\ln(-x)}{\ln(1 + x)}\right)} + 1 \right] = -1 + 1 = 0.
\]
故\(a_n = 1 - \frac{\ln n}{n} + o\left(\frac{\ln n}{n}\right), n \rightarrow +\infty\).
\end{proof}

\begin{example}
设\(f_n(x)=x + x^2+\cdots+x^n,n\in\mathbb{N}\),\(f_n(x)=1\)在\([0,1]\)的根为\(x_n\).求\(\lim_{n\rightarrow\infty}x_n\).
\end{example}
\begin{solution}
注意到\(f_n(x) - 1\)严格单调递增,且\(f_n(0) - 1 = -1 < 0\),\(f_n(1) - 1 = n - 1 > 0\),\(\forall n\geqslant 2\)。故由零点存在定理可知,当\(n\geqslant 2\)时,存在唯一的\(x_n\in(0, 1)\),使得\(f_n(x_n) = 1\)。
从而
\begin{align}\label{example4.17-1.1}
f_n(x_n)=\frac{x_n - x_{n}^{n + 1}}{1 - x_n}=1\Rightarrow x_n - x_{n}^{n + 1}=1 - x_n\Rightarrow x_{n}^{n + 1}=2x_n - 1\Rightarrow n + 1=\frac{\ln(2x_n - 1)}{\ln x_n}.  
\end{align}
由上式\eqref{example4.17-1.1}可知\(x_{n}^{n + 1}=2x_n - 1\)且\(x_n\in(0, 1)\),因此
\[
0\leqslant x_{n}^{n + 1}=2x_n - 1\leqslant 1\Rightarrow x_n\in\left(\frac{1}{2}, 1\right).
\]
任取\(\{x_n\}\)的收敛子列\(\{x_{n_k}\}\),设\(\lim_{k\rightarrow +\infty}x_{n_k}=a\in\left[\frac{1}{2}, 1\right]\),则由\((1.1)\)式和Heine归结原则可知
\[
\lim_{k\rightarrow +\infty}\frac{\ln(2x_{n_k} - 1)}{\ln x_{n_k}}=\frac{\ln(2a - 1)}{\ln a}=+\infty.
\]
故\(a = \frac{1}{2}\),再由\hyperref[proposition:子列极限命题]{子列极限命题(a)}可知\(\lim_{n\rightarrow +\infty}x_n=a=\frac{1}{2}\)。
\end{solution}

\subsubsection{迭代方法}

\begin{example}
设\(x_n\)是\(x = \tan x\)从小到大排列的全部正根,设
\[
\lim_{n\rightarrow\infty}n(x_n - An - B)=C,
\]
求\(A,B,C\)。
\end{example}
\begin{note}
主要想法是结合$\arctan x$的性质:\(\arctan x + \arctan\frac{1}{x} = \frac{\pi}{2}\),\(x > 0\),再利用迭代法计算渐近展开.
\end{note}
\begin{solution}
令\(f(x)=\tan x - x\),\(x\in(n\pi, n\pi + \frac{\pi}{2})\),\(n = 1, 2, \cdots\),则\(f^\prime(x)=\tan^2 x > 0\),\(\forall x\in(n\pi, n\pi + \frac{\pi}{2})\),\(n = 1, 2, \cdots\)。因此\(f(x)\)在\((n\pi, n\pi + \frac{\pi}{2})\)上严格单调递增,其中\(n = 1, 2, \cdots\)。又注意到\(\lim_{x\rightarrow (n\pi)^+}(\tan x - x)= -n\pi < 0\),\(\lim_{x\rightarrow (n\pi + \frac{\pi}{2})^+}(\tan x - x)= +\infty > 0\)。故由零点存在定理可知,存在唯一的\(x_n\in(n\pi, n\pi + \frac{\pi}{2})\),\(n = 1, 2, \cdots\),使得
\[
\tan x_n = x_n.
\]
从而\(x_n - n\pi\in(0, \frac{\pi}{2})\),于是
\begin{align}\label{example4.19-1.1}
x_n = \tan x_n = \tan(x_n - n\pi) \Rightarrow x_n = \arctan x_n + n\pi.  
\end{align}
又因为\(x_n\in(n\pi, n\pi + \frac{\pi}{2})\),\(n = 1, 2, \cdots\),所以当\(n\rightarrow +\infty\)时,有\(x_n\rightarrow +\infty\)。再结合\eqref{example4.19-1.1}式可得
\begin{align}\label{example4.19-1.2}
x_n = \arctan x_n + n\pi = n\pi + \frac{\pi}{2} + o(1), n\rightarrow +\infty.
\end{align}
注意到\(\arctan x + \arctan\frac{1}{x} = \frac{\pi}{2}\),\(x > 0\),从而\(\arctan x = \frac{\pi}{2} - \arctan\frac{1}{x}\)。于是利用\eqref{example4.19-1.2}式可得
\begin{align*}
x_n&=\mathrm{arc}\tan x_n+n\pi =\frac{\pi}{2}+n\pi -\mathrm{arc}\tan \frac{1}{x_n}=\frac{\pi}{2}+n\pi -\mathrm{arc}\tan \frac{1}{n\pi +\frac{\pi}{2}+o(1)}
\\
&=\frac{\pi}{2}+n\pi -\mathrm{arc}\tan \left( \frac{1}{n\pi}\frac{1}{1+\frac{1}{2n}+o(\frac{1}{n})} \right) =\frac{\pi}{2}+n\pi -\mathrm{arc}\tan \left[ \frac{1}{n\pi}\left( 1+O(\frac{1}{n}) \right) \right] 
\\
&=\frac{\pi}{2}+n\pi -\mathrm{arc}\tan \left[ \frac{1}{n\pi}+O(\frac{1}{n^2}) \right] =\frac{\pi}{2}+n\pi -\frac{1}{n\pi}+O(\frac{1}{n^2}),n\rightarrow +\infty .
\end{align*}
因此\(\lim_{n\rightarrow +\infty}n\left(x_n - \frac{\pi}{2} - n\pi\right)= -\frac{1}{\pi}\)。
\end{solution}


\end{document}
\documentclass[../../main.tex]{subfiles}
\graphicspath{{\subfix{../../image/}}} % 指定图片目录,后续可以直接使用图片文件名。

% 例如:
% \begin{figure}[H]
% \centering
% \includegraphics[scale=0.4]{图.png}
% \caption{}
% \label{figure:图}
% \end{figure}
% 注意:上述\label{}一定要放在\caption{}之后,否则引用图片序号会只会显示??.

\begin{document}

\section{Stirling公式}

对于阶乘问题,最好用的估计工具就是Stirling公式.与组合数相关的极限问题,都可以尝试将其全部转化为阶乘然后估计大小.

\begin{theorem}[Stirling公式]\label{theorem:Stirling公式}
\[
n!\sim\sqrt{2\pi n}\left(\frac{n}{e}\right)^{n},n\to\infty.
\]
\end{theorem}
\begin{note}
提示:用欧拉麦克劳林公式估计\(\sum_{k = 1}^{n}\ln k,n\to\infty\)的渐近展开式, 以此结合 \hyperref[theorem:Wallis公式]{Wallis 公式}:\(\frac{(2n)!!}{(2n - 1)!!}\sim\sqrt{\pi n},n\to\infty\)证明.
\end{note}
\begin{proof}
由\hyperref[proposition:0阶欧拉麦克劳林公式(0阶E-M公式)]{E-M公式}可知,对\(\forall n\in\mathbb{N}\),都有
\begin{align}\label{equation:4.4911}
\sum_{k = 1}^n\ln k&=\frac{\ln n}{2}+\int_1^n\ln x\mathrm{d}x+\int_1^n\left(x - [x] - \frac{1}{2}\right)\frac{1}{x}\mathrm{d}x
=\frac{\ln n}{2}+n\ln n - n + 1+\int_1^n\left(x - [x] - \frac{1}{2}\right)\frac{1}{x}\mathrm{d}x.
\end{align}
由Dirichlet判别法可知,\(\int_1^{+\infty}\left(x - [x] - \frac{1}{2}\right)\frac{1}{x}\mathrm{d}x\)收敛.则可设\(\lim_{n\rightarrow\infty}\int_1^n\left(x - [x] - \frac{1}{2}\right)\frac{1}{x}\mathrm{d}x=\int_1^{+\infty}\left(x - [x] - \frac{1}{2}\right)\frac{1}{x}\mathrm{d}x\triangleq C_0<\infty\).
记\(b_1(x)=x - [x] - \frac{1}{2}\),再令\(b_2(x)=\frac{1}{2}(x - [x])^2 - \frac{1}{2}(x - [x]) + \frac{1}{12}\),\(x\in\mathbb{R}\).则\hyperlink{b2性质}{不难发现}\(b_2(x)\)在\(\mathbb{R}\)上连续且周期为\(1\),并且
\[
b_2(x)=\int_0^x{b_1(y)dy}, \quad |b_2(x)|\leqslant\frac{1}{12}, \forall x\in\mathbb{R}.
\]
从而对\eqref{equation:4.4911}式使用分部积分可得
\begin{align*}
\sum_{k=1}^n{\ln k}&=\frac{\ln n}{2}+n\ln n-n+1+\int_1^n{\frac{b_1\left( x \right)}{x}\mathrm{d}x}=\frac{\ln n}{2}+n\ln n-n+1+\int_1^{+\infty}{\frac{b_1\left( x \right)}{x}\mathrm{d}x}-\int_n^{+\infty}{\frac{b_1\left( x \right)}{x}\mathrm{d}x}
\\
&=\frac{\ln n}{2}+n\ln n-n+1+C_0-\int_n^{+\infty}{\frac{1}{x}db_2\left( x \right)}=\frac{\ln n}{2}+n\ln n-n+1+C_0-\frac{b_2\left( x \right)}{x}\Big|_{n}^{+\infty}-\int_n^{+\infty}{\frac{b_2\left( x \right)}{x^2}\mathrm{d}x}
\\
&=\left( n+\frac{1}{2} \right) \ln n-n+1+C_0+\frac{b_2\left( n \right)}{n}-\int_n^{+\infty}{\frac{b_2\left( x \right)}{x^2}\mathrm{d}x},\forall n\in\mathbb{N}.
\end{align*}
又因为\(|b_2(x)|\leqslant\frac{1}{12}\),\(\forall x\in\mathbb{R}\).所以对\(\forall n\in\mathbb{N}\),我们有
\[
\left|\frac{b_2(n)}{n} - \int_n^{+\infty}\frac{b_2(x)}{x^2}\mathrm{d}x\right|\leqslant\frac{1}{12}\left(\frac{1}{n}+\int_n^{+\infty}\frac{1}{x^2}\mathrm{d}x\right)=\frac{1}{6n}.
\]
故\(\frac{b_2(n)}{n} - \int_n^{+\infty}\frac{b_2(x)}{x^2}\mathrm{d}x = O\left(\frac{1}{n}\right)\),\(\forall n\in\mathbb{N}\).于是再记\(C = 1 + C_0\),则
\begin{align}\label{equation:4491}
\sum_{k = 1}^n\ln k=\left(n + \frac{1}{2}\right)\ln n - n + C + O\left(\frac{1}{n}\right),\forall n\in\mathbb{N} .
\end{align}
注意到
\begin{align}\label{equation:4492}
(2n)!! = 2^n n!, n = 0,1,2,\cdots.
\end{align}
于是由\(Wallis\)公式:\(\frac{(2n)!!}{(2n - 1)!!}\sim\sqrt{\pi n}\),\(n\rightarrow\infty\).再结合\eqref{equation:4491}\eqref{equation:4492}可得
\begin{align*}
\sqrt{\pi}&=\underset{n\rightarrow \infty}{\lim}\frac{\left( 2n \right) !!}{\left( 2n-1 \right) !!\sqrt{n}}=\underset{n\rightarrow \infty}{\lim}\frac{\left[ \left( 2n \right) !! \right] ^2}{\left( 2n \right) !\sqrt{n}}=\underset{n\rightarrow \infty}{\lim}\frac{\left( 2^nn! \right) ^2}{\left( 2n \right) !\sqrt{n}}=\underset{n\rightarrow \infty}{\lim}\frac{4^nn!\cdot n!}{\left( 2n \right) !\sqrt{n}}
\\
&=\underset{n\rightarrow \infty}{\lim}\frac{4^nn!\prod\limits_{k=1}^n{k}}{\sqrt{n}\prod\limits_{k=n+1}^{2n}{k}}=\underset{n\rightarrow \infty}{\lim}\frac{4^nn!e^{\sum\limits_{k=1}^n{\ln k}}}{\sqrt{n}e^{\sum\limits_{k=1}^{2n}{\ln k}}}=\underset{n\rightarrow \infty}{\lim}\frac{4^nn!e^{\left( n+\frac{1}{2} \right) \ln n-n+C+O\left( \frac{1}{n} \right)}}{\sqrt{n}e^{\left( 2n+\frac{1}{2} \right) \ln 2n-2n+C+O\left( \frac{1}{n} \right)}}
\\
&=\underset{n\rightarrow \infty}{\lim}\frac{4^nn!e^{\left( n+\frac{1}{2} \right) \ln n-n+C+O\left( \frac{1}{n} \right) -\left[ \left( 2n+\frac{1}{2} \right) \ln 2n-2n+C+O\left( \frac{1}{n} \right) \right]}}{\sqrt{n}}=\underset{n\rightarrow \infty}{\lim}\frac{4^nn!e^{-n\ln n+n-\left( 2n+\frac{1}{2} \right) \ln 2+O\left( \frac{1}{n} \right)}}{\sqrt{n}}
\\
&=\underset{n\rightarrow \infty}{\lim}\frac{4^nn!2^{-2n-\frac{1}{2}}e^n}{n^n\sqrt{n}}e^{O\left( \frac{1}{n} \right)}=\underset{n\rightarrow \infty}{\lim}\frac{n!e^n}{n^n\sqrt{2n}}e^{O\left( \frac{1}{n} \right)}.
\end{align*}
从而\(\lim_{n\rightarrow\infty}\frac{n!e^n}{n^n\sqrt{2n}}=\frac{\sqrt{\pi}}{\lim\limits_{n\rightarrow\infty}e^{O\left(\frac{1}{n}\right)}}=\sqrt{\pi}\).因此\(\lim_{n\rightarrow\infty}\frac{n!}{\sqrt{n}\left(\frac{n}{e}\right)^n}=\lim_{n\rightarrow\infty}\frac{n!e^n}{n^n\sqrt{n}}=\sqrt{2\pi}\).故\(n!\sim\sqrt{2\pi n}\left(\frac{n}{e}\right)^n\),\(n\rightarrow\infty\).
\end{proof}

\begin{example}
设\(n,v\)为正整数且\(1 < v < n\),满足\(\lim_{n\rightarrow\infty}\frac{v-\frac{n}{2}}{\sqrt{n}}=\lambda>0\),证明:\(\lim_{n\rightarrow\infty}\frac{\sqrt{n}}{2^n}C_n^v=\sqrt{\frac{2}{\pi}}e^{-2\lambda^2}\)。
\end{example}
\begin{proof}
根据条件,显然在\(n\rightarrow\infty\)时\(v\)也会趋于无穷,设\(v = \frac{n}{2}+w\sqrt{n}\),则$w=\frac{v-\frac{n}{2}}{\sqrt{n}}$,从而\(\lim_{n\rightarrow\infty}w=\lambda>0\),则有
\begin{align*}
\frac{\sqrt{n}}{2^n}C_n^v=\frac{\sqrt{n}}{2^n}\frac{n!}{v!(n - v)!},n!\sim\sqrt{2\pi n}\left(\frac{n}{e}\right)^n,n\to \infty.
\end{align*}
从而
\begin{align*}
\lim_{n\rightarrow \infty} \frac{\sqrt{n}}{2^n}C_{n}^{v}&=\underset{n\rightarrow \infty}{\lim}\frac{\sqrt{n}}{2^n}\frac{n!}{v!\left( n-v \right) !}=\lim_{n\rightarrow \infty} \frac{\sqrt{n}}{2^n}\frac{\sqrt{2\pi n}\left( \frac{n}{e} \right) ^n}{\sqrt{2\pi v}\left( \frac{v}{e} \right) ^v\sqrt{2\pi (n-v)}\left( \frac{n-v}{e} \right) ^{n-v}}
\\
&=\frac{1}{\sqrt{2\pi}}\lim_{n\rightarrow \infty} \frac{n^n}{2^nv^v(n-v)^{n-v}}\frac{n}{\sqrt{v(n-v)}}=\sqrt{\frac{2}{\pi}}e^{-2\lambda ^2}
\\
\Leftrightarrow &\lim_{n\rightarrow \infty} \frac{n^n}{2^n\left( \frac{n}{2}+w\sqrt{n} \right) ^v\left( \frac{n}{2}-w\sqrt{n} \right) ^{n-v}}\frac{n}{2\sqrt{v(n-v)}}=e^{-2\lambda ^2}.
\end{align*}
又
\begin{align*}
\lim_{n\rightarrow \infty} \frac{n}{2\sqrt{v(n-v)}}=\lim_{n\rightarrow \infty} \frac{n}{2\sqrt{\left( \frac{n}{2}+w\sqrt{n} \right) \left( \frac{n}{2}-w\sqrt{n} \right)}}=\lim_{n\rightarrow \infty} \frac{1}{\sqrt{1-\frac{4w^2}{\sqrt{n}}}}=1,
\end{align*}
故
\begin{align}
&\lim_{n\rightarrow \infty} \frac{n^n}{2^n\left( \frac{n}{2}+w\sqrt{n} \right) ^v\left( \frac{n}{2}-w\sqrt{n} \right) ^{n-v}}\frac{n}{2\sqrt{v(n-v)}}=e^{-2\lambda ^2}
\nonumber
\\
&\Leftrightarrow \lim_{n\rightarrow \infty} \frac{n^{\left( \frac{n}{2}+w\sqrt{n} \right) +\left( \frac{n}{2}-w\sqrt{n} \right)}}{2^{\left( \frac{n}{2}+w\sqrt{n} \right) +\left( \frac{n}{2}-w\sqrt{n} \right)}\left( \frac{n}{2}+w\sqrt{n} \right) ^{\frac{n}{2}+w\sqrt{n}}\left( \frac{n}{2}-w\sqrt{n} \right) ^{\frac{n}{2}-w\sqrt{n}}}=e^{-2\lambda ^2}
\nonumber
\\
&\Leftrightarrow \lim_{n\rightarrow \infty} \frac{n^{\left( \frac{n}{2}+w\sqrt{n} \right) +\left( \frac{n}{2}-w\sqrt{n} \right)}}{(n+2w\sqrt{n})^{\frac{n}{2}+w\sqrt{n}}(n-2w\sqrt{n})^{\frac{n}{2}-w\sqrt{n}}}=e^{-2\lambda ^2}
\nonumber
\\
&\Leftrightarrow \lim_{n\rightarrow \infty} \frac{1}{\left( 1+\frac{2w}{\sqrt{n}} \right) ^{\frac{n}{2}+w\sqrt{n}}\left( 1-\frac{2w}{\sqrt{n}} \right) ^{\frac{n}{2}-w\sqrt{n}}}=e^{-2\lambda ^2}
\nonumber
\\
&\Leftrightarrow \lim_{n\rightarrow \infty} \left[ \left( \frac{n}{2}+w\sqrt{n} \right) \ln \left( 1+\frac{2w}{\sqrt{n}} \right) +\left( \frac{n}{2}-w\sqrt{n} \right) \ln \left( 1-\frac{2w}{\sqrt{n}} \right) \right] =2\lambda ^2.\label{15641891651}
\end{align}
又由Taylor公式可得
\begin{align*}
&\left( \frac{n}{2}+w\sqrt{n} \right) \ln \left( 1+\frac{2w}{\sqrt{n}} \right) +\left( \frac{n}{2}-w\sqrt{n} \right) \ln \left( 1-\frac{2w}{\sqrt{n}} \right) 
\\
&=\left( \frac{n}{2}+w\sqrt{n} \right) \left( \frac{2w}{\sqrt{n}}-\frac{2w^2}{n}+O\left( \frac{1}{n\sqrt{n}} \right) \right) +\left( \frac{n}{2}-w\sqrt{n} \right) \left( -\frac{2w}{\sqrt{n}}-\frac{2w^2}{n}+O\left( \frac{1}{n\sqrt{n}} \right) \right) 
\\
&=w\sqrt{n}+w^2+O\left( \frac{1}{\sqrt{n}} \right) -w\sqrt{n}+w^2+O\left( \frac{1}{\sqrt{n}} \right) =2w^2+O\left( \frac{1}{\sqrt{n}} \right) ,n\to \infty.
\end{align*}
再结合$\underset{n\rightarrow \infty}{\lim}w=\lambda$可知\eqref{15641891651}式成立,因此结论得证.
\end{proof}


\end{document}
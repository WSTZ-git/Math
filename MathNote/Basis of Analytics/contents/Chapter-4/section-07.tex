\documentclass[../../main.tex]{subfiles}
\graphicspath{{\subfix{../../image/}}} % 指定图片目录,后续可以直接使用图片文件名。

% 例如:
% \begin{figure}[H]
% \centering
% \includegraphics[scale=0.3]{image-01.01}
% \caption{图片标题}
% \label{figure:image-01.01}
% \end{figure}
% 注意:上述\label{}一定要放在\caption{}之后,否则引用图片序号会只会显示??.

\begin{document}

\section{分部积分}

分析学里流传着一句话:“遇事不决分部积分”.

分部积分在渐近分析中的用法:
\begin{enumerate}[(1)]
\item\label{分部积分在渐近分析中的用法(1)} 有时候分部积分不能计算出某一积分的具体值,但是我们可以利用分部积分去估计原积分(或原含参积分)的范围.并且我们可以通过不断分部积分来提高估计的精确程度.

\item\label{分部积分在渐近分析中的用法(2)} 分部积分也可以转移被积函数的导数.

\item\label{分部积分在渐近分析中的用法(3)} 分部积分可以改善阶.通过分部积分提高分母的次方从而增加收敛速度方便估计.并且可以通过反复分部积分得到更加精细的估计.
\end{enumerate}

\begin{example}
\begin{align*}
f\left( x \right) =\int_x^{x+1}{\sin \left( t^2 \right) dt}.
\end{align*}
证明$\left| f\left( x \right) \right|\le \frac{1}{x},x>0$.
\end{example}
\begin{note}
证明的想法是利用\hyperref[分部积分在渐近分析中的用法(1)]{分部积分在渐近分析中的用法(1)}.
\end{note}
\begin{proof}
由分部积分可得,对$\forall x>0$,都有
\begin{align*}
&\left| f\left( x \right) \right|=\left| \int_x^{x+1}{\sin \left( t^2 \right) dt} \right|=\left| \int_{x^2}^{\left( x+1 \right) ^2}{\frac{\sin u}{2\sqrt{u}}du} \right|=\left| -\frac{1}{4}\int_{x^2}^{\left( x+1 \right) ^2}{u^{-\frac{3}{2}}\cos udu}-\frac{\cos u}{2\sqrt{u}}\Big|_{x^2}^{\left( x+1 \right) ^2} \right|
\\
&\leqslant \left| \frac{1}{4}\int_{x^2}^{\left( x+1 \right) ^2}{u^{-\frac{3}{2}}du} \right|+\left| \frac{\cos x}{2x}-\frac{\cos \left( x+1 \right)}{2\left( x+1 \right)} \right|=\frac{1}{2}\left| \frac{1}{x}-\frac{1}{x+1} \right|+\frac{1}{2}\left| \frac{\cos x}{x}-\frac{\cos \left( x+1 \right)}{\left( x+1 \right)} \right|
\\
&=\frac{1}{2x\left( x+1 \right)}+\frac{x\left[ \cos x-\cos \left( x+1 \right) \right] +\cos x}{2x\left( x+1 \right)}=\frac{1}{2x\left( x+1 \right)}+\frac{2\sin \frac{1}{2}x\sin \frac{2x+1}{2}+\cos x}{2x\left( x+1 \right)}
\\
&\le \frac{1}{2x\left( x+1 \right)}+\frac{x+1}{2x\left( x+1 \right)}=\frac{1}{2x\left( x+1 \right)}+\frac{1}{2x}\leqslant \frac{1}{x}.
\end{align*}
\end{proof}

\begin{example}
设$f\left( x \right) =\int_0^x{\sin \frac{1}{y}dy}$,求$f'\left( 0 \right) $.
\end{example}
\begin{note}
证明的想法是利用\hyperref[分部积分在渐近分析中的用法(3)]{分部积分在渐近分析中的用法(3)}.
\end{note}
\begin{solution}
注意到
\begin{align}
f_{+}^{\prime}\left( 0 \right) =\underset{x\rightarrow 0^+}{\lim}\frac{\int_0^x{\sin \frac{1}{y}dy}}{x}=\underset{x\rightarrow 0^+}{\lim}\frac{\int_{+\infty}^{\frac{1}{x}}{\sin yd\frac{1}{y}}}{x}=\underset{x\rightarrow 0^+}{\lim}\frac{\int_{\frac{1}{x}}^{+\infty}{\frac{\sin y}{y^2}dy}}{x}\xlongequal{\text{令}t=\frac{1}{x}}\underset{t\rightarrow +\infty}{\lim}t\int_t^{+\infty}{\frac{\sin y}{y^2}dy},\left( 1.1 \right) \label{example4.48-1.1}
\\
f_{-}^{\prime}\left( 0 \right) =\underset{x\rightarrow 0^-}{\lim}\frac{\int_0^x{\sin \frac{1}{y}dy}}{x}=\underset{x\rightarrow 0^-}{\lim}\frac{\int_{+\infty}^{\frac{1}{x}}{\sin yd\frac{1}{y}}}{x}=\underset{x\rightarrow 0^-}{\lim}\frac{\int_{\frac{1}{x}}^{+\infty}{\frac{\sin y}{y^2}dy}}{x}\xlongequal{\text{令}t=\frac{1}{x}}\underset{t\rightarrow -\infty}{\lim}t\int_t^{-\infty}{\frac{\sin y}{y^2}dy}.\left( 1.2 \right) \label{example4.48-1.2}
\end{align}
由分部积分可得
\begin{align*}
\int_t^{+\infty}{\frac{\sin y}{y^2}dy}=-\int_t^{+\infty}{\frac{1}{y^2}d\cos y}
=\frac{\cos y}{y^2}\big|_{+\infty}^{t}+\int_t^{+\infty}{\cos yd\frac{1}{y^2}}
=\frac{\cos t}{t^2}-2\int_t^{+\infty}{\frac{\cos y}{y^3}dy}.
\end{align*}
故对\(\forall t>0\),我们有
\begin{align*}
\left|\int_t^{+\infty}{\frac{\sin y}{y^2}dy}\right|=\left|\frac{\cos t}{t^2}-2\int_t^{+\infty}{\frac{\cos y}{y^3}dy}\right|
\leqslant \frac{1}{t^2}+2\int_t^{+\infty}{\frac{1}{y^3}dy}\
=\frac{2}{t^2}.
\end{align*}
即\(\int_t^{+\infty}{\frac{\sin y}{y^2}dy}=O\left(\frac{1}{t^2}\right), \forall t>0\)。再结合\eqref{example4.48-1.1}式可知
\[
f_{+}^{\prime}(0)=\lim_{t\rightarrow +\infty}t\int_t^{+\infty}{\frac{\sin y}{y^2}dy}=0.
\]
同理可得\(f_{-}^{\prime}(0)=\lim_{t\rightarrow -\infty}t\int_t^{-\infty}{\frac{\sin y}{y^2}dy}=0\)。故\(f^{\prime}(0)=f_{+}^{\prime}(0)=f_{-}^{\prime}(0)=0\)。
\end{solution}


\end{document}
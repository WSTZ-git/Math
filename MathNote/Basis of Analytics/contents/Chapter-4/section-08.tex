\documentclass[../../main.tex]{subfiles}
\graphicspath{{\subfix{../../image/}}} % 指定图片目录,后续可以直接使用图片文件名。

% 例如:
% \begin{figure}[H]
% \centering
% \includegraphics[scale=0.4]{图.png}
% \caption{}
% \label{figure:图}
% \end{figure}
% 注意:上述\label{}一定要放在\caption{}之后,否则引用图片序号会只会显示??.

\begin{document}

\section{Laplace方法}

Laplace方法适用于估计形如$\int_a^b{\left[ f\left( x \right) \right] ^ng\left( x \right) \mathrm{d}x},n\rightarrow \infty $的渐近展开式,其中$f,g\in C[a,b]$且$g$在[a,b]上有界;或者$\int_a^b{e^{f\left( x,y \right)}g(y)\mathrm{d}y},x\rightarrow +\infty $的渐近展开式,其中$f,g\in C[a,b]$且$g$在[a,b]上有界.实际上,若要估计的是前者,我们可以将其转化为后者的形式如下:
\begin{align*}
\int_a^b{\left[ f\left( x \right) \right] ^ng\left( x \right) \mathrm{d}x}=\int_a^b{e^{n\ln f\left( x \right)}g\left( x \right) \mathrm{d}x}.
\end{align*}
若参变量$n,x$在积分区间上,或者估计的不是$n,x\to +\infty$处的渐近展开式,而是其他点处($x\to x_0$)处的渐近展开式.我们都可以通过积分换元将其转化为标准形式$\int_a^b{e^{f\left( x,y \right)}g(y)\mathrm{d}y},x\rightarrow +\infty $,其中$f,g\in C[a,b]$.

思路分析:
首先,由含参量积分的计算规律(若被积函数含有$e^{f(x)}$,则积分得到的结果中一定仍含有$e^{f(x)}$),我们可以大致估计积分$\int_a^b{e^{f\left( x,y \right)}g(y)\mathrm{d}y},x\rightarrow +\infty $的结果是$C_1h_1\left( x \right) e^{f\left( x,b \right)}-C_2h_2\left( x \right) e^{f\left( x,b \right)}e^{f\left( x,a \right)} $,其中$C$为常数.因为指数函数的阶远大于一般初等函数的阶,这个结果的阶的主体部分就是$e^{f\left( x,b \right)}$和$e^{f\left( x,a \right)}$.而我们注意到到改变指数函数$e^{px+q}$的幂指数部分的常数$p$会对这个指数函数的阶$(x\to +\infty)$产生较大影响,而改变$q$不会影响这个指数函数的阶.比如,$e^{2x}$比$e^x$高阶$(x\to+\infty)$.由此我们可以发现$e^{f\left( x,b \right)}$和$e^{f\left( x,a \right)}$中的幂指数部分中$f(x,a),f(x,b)$中除常数项外的含$x$项的系数(暂时叫作指数系数)对这个函数的阶影响较大.然而这些系数都是由被积函数中的$f(x,y)$和积分区间决定的,但是在实际问题中$f(x,y)$的形式已经确定,因此这些系数仅仅由积分区间决定.于是当我们只计算某些不同点附近(充分小的邻域内)的含参量积分时,得到的这些系数一般不同,从而导致这些积分的阶不同.故我们可以断言这类问题的含参量积分在每一小段上的阶都是不同的.因此我们只要找到这些不同的阶中最大的阶(此时最大阶就是主体部分)就相当于估计出了积分在整个区间$[a,b]$上的阶.由定积分的几何意义,我们不难发现当参变量$x$固定时,并且当积分区间为某一点$y_0$附近时,只要被积函数的$e^{f(x,y)}$在$y_0$处(关于$y$)的取值越大,积分后得到的(值/充分小邻域内函数与x轴围成的面积)指数系数就会越大,从而在$y_0$附近的积分的阶也就越大.综上所述,当参变量$x$固定时,$f(x,y)$(关于$y$)的最大值点附近的积分就是原积分的主体部分,在其他区间上的积分全都是余项部分.

然后,我们将原积分按照上述的积分区间分段,划分为主体部分和余项部分.我们知道余项部分一定可以通过放缩、取上下极限等操作变成0(余项部分的放缩一般需要结合具体问题,并使用一些放缩技巧来实现.但是我们其实只要心里清楚余项部分一定能够通过放缩、取上下极限变成0即可),关键是估计主体部分的阶.我们注意到主体部分的积分区间都包含在某一点的邻域内,而一般估计在某个点附近的函数的阶,我们都会想到利用$Taylor$定理将其在这个点附近展开.因此我们利用$Taylor$定理将主体部分的被积函数的指数部分$f(x,y)$在最大值点附近(关于$y$)展开(注意:此时最多展开到$x^2$项,如果展开项的次数超过二次,那么后续要么就无法计算积分,要么计算就无法得到有效结果,比如最后积分、取极限得到$\infty+\infty$或$0\cdot \infty$等这一类无效的结果).
$Taylor$展开之后,我们只需要利用欧拉积分和定积分,直接计算得到结果即可.

事实上,若原积分中的有界连续函数$g(x)$在$f$的极值点处不为$0$,则$g(x)$只会影响渐进展开式中的系数,对整体的阶并不造成影响.在实际估计中处理$g(x)$的方法:(i)在余项部分,直接将$g(x)$放缩成其在相应区间上的上界或下界即可.(ii)在主体部分,因为主体部分都包含在$f(x,y)$(关于$y$)的某些最大值点$y_i$的邻域内,所以结合$g(x)$的连续性,直接将$g(x)$用$g(y_i)$代替即可(将$g(x)$放缩成$g(y_i)\pm \varepsilon$即可).即相应的主体部分($y_i$点附近)乘以$g(x)$相应的函数值$g(y_i)$.具体例题见\hyperref[Laplace方法例题4]{例题\ref{Laplace方法例题4}}.也可以采取拟合法处理$g(x)$,具体例题见\hyperref[example4544166848]{例题\ref{example4544166848}}.

若原积分中的有界连续函数$g(x)$在$f$的极值点处为$0$,则在估计积分的阶的时候就要将$g(x)$考虑进去.需要结合$g(x)$的具体结构、性质进行分析.

严谨的证明过程最好用上下极限和$\varepsilon-\delta$语言书写.具体严谨的证明书写见例题:\hyperref[Laplace方法例题1]{例题\ref{Laplace方法例题1}},\hyperref[Laplace方法例题2]{例题\ref{Laplace方法例题2}},\hyperref[Laplace方法例题3]{例题\ref{Laplace方法例题3}},\hyperref[Laplace方法例题4]{例题\ref{Laplace方法例题4}}.
\begin{note}
$Laplace$方法的思路蕴含了一些常用的想法:\textbf{分段估计}、\textbf{Taylor定理估阶}.而严谨的证明书写也使用一些常用方法:\textbf{上下极限}、\textbf{$\varepsilon-\delta$语言}、\textbf{拟合法}.
\end{note}
\begin{remark}
上述Laplace方法得到的渐近估计其实比较粗糙,想要得到更加精细的渐近估计需要用到更加深刻的想法和技巧(比如Puiseux级数展开(见清疏讲义)等).
\end{remark}


\begin{example}
设\(a_1,a_2,\cdots,a_m > 0,m \in \mathbb{N}\),则
\[
\lim_{n \to \infty} \sqrt[n]{a_1^n + a_2^n + \cdots + a_m^n} = \max_{1\leqslant  j\leqslant  m} a_j.
\]
\end{example} 
\begin{remark}
熟知,极限蕴含在\(a_1,a_2,\cdots,a_m\)的最大值中.
\end{remark}
\begin{proof}
显然
\begin{align}\label{equation:9.2131}
\max_{1\leqslant  j\leqslant  m} a_j = \lim_{n \to \infty} \sqrt[n]{\max_{1\leqslant  j\leqslant  m} a_j^n} \leqslant  \lim_{n \to \infty} \sqrt[n]{a_1^n + a_2^n + \cdots + a_m^n} \leqslant  \max_{1\leqslant  j\leqslant  m} a_j \cdot \lim_{n \to \infty} \sqrt[n]{m} = \max_{1\leqslant  j\leqslant  m} a_j,
\end{align}
从而我们证明了\eqref{equation:9.2131}.  

\end{proof}

\begin{example}\label{example-3.31}
设非负函数\(f \in C[a,b]\),则
\[
\lim_{n \to \infty} \sqrt[n]{\int_{a}^{b} f^n(x)\mathrm{d}x} = \max_{x\in[a,b]} f(x).
\]
\end{example}
\begin{remark}
熟知,极限蕴含在\(f\)的最大值中. 
\end{remark}
\begin{note}
这两个基本例子也暗示了离散和连续之间有时候存在某种类似的联系.
\end{note}
\begin{proof}
事实上记\(f(x_0) = \max_{x\in[a,b]} f(x), x_0 \in [a,b]\),不失一般性我们假设\(x_0 \in (a,b)\).那么对充分大的\(n \in \mathbb{N}\),我们由积分中值定理知道存在\(\theta_n \in (x_0 - \frac{1}{2n}, x_0 + \frac{1}{2n})\),使得
\begin{align}\label{equation:9.223423}
f(\theta_n) \sqrt[n]{\frac{1}{n}} = \sqrt[n]{\int_{x_0 - \frac{1}{2n}}^{x_0 + \frac{1}{2n}} f^n(x)\mathrm{d}x} \leqslant  \sqrt[n]{\int_{a}^{b} f^n(x)\mathrm{d}x} \leqslant  \sqrt[n]{\int_{a}^{b} f^n(x_0)\mathrm{d}x} = f(x_0) \sqrt[n]{b - a}.   
\end{align}
两边取极限即得\eqref{equation:9.223423}.

\end{proof}

\begin{example}
设非负严格递增函数\(f \in C[a,b]\),由积分中值定理我们知道存在\(x_n \in [a,b]\),使得
\[
f^n(x_n) = \frac{1}{b - a} \int_{a}^{b} f^n(x)\mathrm{d}x.
\]
计算\(\lim_{n \to \infty} x_n\).
\end{example}
\begin{proof}
由\hyperref[example-3.31]{(上一题)例题\ref{example-3.31}},我们知道
\[
\lim_{n \to \infty} f(x_n) = \lim_{n \to \infty} \sqrt[n]{\frac{1}{b - a}} \cdot \lim_{n \to \infty} \sqrt[n]{\int_{a}^{b} f^n(x)\mathrm{d}x} = f(b).
\]
注意到\(\{x_n\}_{n = 1}^{\infty} \subset [a,b]\),我们知道对任何\(\lim_{k \to \infty} x_{n_k} = c \in [a,b]\),都有\(\lim_{k \to \infty} f(x_{n_k}) = f(c) = f(b)\).又由于$f$为严格递增函数,因此只能有\(c = b\),利用\hyperref[proposition:子列极限命题]{命题\ref{proposition:子列极限命题}的(a)(Heine归结原理)},我们知道\(\lim_{n \to \infty} x_n = b\).证毕!

\end{proof}

\begin{theorem}[Wallis公式]\label{theorem:Wallis公式}
\begin{align}\label{theorem:Wallis公式-equation}
\frac{(2n)!!}{(2n - 1)!!}=\sqrt{\pi n}+\frac{\sqrt{\pi}}{8}\cdot\frac{1}{\sqrt{n}}+o\left(\frac{1}{\sqrt{n}}\right).
\end{align}
\end{theorem}
\begin{remark}
我们只需要记住$\frac{(2n)!!}{(2n-1)!!}\sim \sqrt{\pi n},n\rightarrow +\infty$及其证明即可,更精细的渐近表达式一般用不到.
\end{remark}
\begin{note}
\eqref{theorem:Wallis公式-equation}式等价于
\begin{align}\label{theorem4.6-0.0}
\lim_{n\rightarrow\infty}\sqrt{n}\left[\frac{(2n)!!}{(2n - 1)!!}-\sqrt{\pi n}\right]=\frac{\sqrt{\pi}}{8}.
\end{align}
证明的想法是把\eqref{theorem4.6-0.0}式用积分表示并运用Laplace方法进行估计.
\end{note}
\begin{proof}
我们只证明$\frac{(2n)!!}{(2n-1)!!}\sim \sqrt{\pi n},n\rightarrow +\infty$,更精细的渐近表达式一般不会被考察,故在此不给出证明.(更精细的渐近表达式的证明可见清疏讲义)

注意到经典积分公式
\begin{align}\label{theorem4.6-1.1}
\int_{0}^{\frac{\pi}{2}}\sin^{2n}x \mathrm{d}x=\frac{\pi}{2}\cdot\frac{(2n - 1)!!}{(2n)!!}.
\end{align}
利用Taylor公式的Peano余项,我们知道
\begin{align}\label{theorem4.6-9.19}
\ln\sin^{2}x=-\left(x - \frac{\pi}{2}\right)^{2}+o\left[\left(x - \frac{\pi}{2}\right)^{2}\right],
\end{align}
即\(\lim_{x\rightarrow \frac{\pi}{2}}\frac{\ln\sin^{2}x}{-(x - \frac{\pi}{2})^{2}}=-1\).于是利用\eqref{theorem4.6-9.19},对任何\(\varepsilon\in(0,1)\),我们知道存在\(\delta\in(0,1)\),使得对任何\(x\in[\frac{\pi}{2}-\delta,\frac{\pi}{2}]\),都有
\begin{align}\label{theorem4.6-9.20}
-(1 + \varepsilon)\left(x - \frac{\pi}{2}\right)^{2}\leqslant\ln\sin^{2}x\leqslant-(1 - \varepsilon)\left(x - \frac{\pi}{2}\right)^{2}.
\end{align}
利用\eqref{theorem4.6-9.20}式,现在一方面,我们有
\begin{align*}
\int_{0}^{\frac{\pi}{2}}\sin^{2n}x \mathrm{d}x&=\int_{0}^{\frac{\pi}{2}}e^{n\ln\sin^{2}x}\mathrm{d}x
\leqslant\int_{0}^{\frac{\pi}{2}-\delta}e^{n\ln\sin^{2}(\frac{\pi}{2}-\delta)}\mathrm{d}x+\int_{\frac{\pi}{2}-\delta}^{\frac{\pi}{2}}e^{-n(1 - \varepsilon)(x - \frac{\pi}{2})^{2}}\mathrm{d}x\\
&=(\frac{\pi}{2}-\delta)\sin^{2n}(\frac{\pi}{2}-\delta)+\int_{0}^{\delta}e^{-n(1 - \varepsilon)y^{2}}\mathrm{d}y\\
&=(\frac{\pi}{2}-\delta)\sin^{2n}(\frac{\pi}{2}-\delta)+\frac{1}{\sqrt{(1 - \varepsilon)n}}\int_{0}^{\delta\sqrt{(1 - \varepsilon)n}}e^{-z^{2}}\mathrm{d}z\\
&\leqslant(\frac{\pi}{2}-\delta)\sin^{2n}(\frac{\pi}{2}-\delta)+\frac{1}{\sqrt{(1 - \varepsilon)n}}\int_{0}^{\infty}e^{-z^{2}}\mathrm{d}z.
\end{align*}
另外一方面,我们有
\begin{align*}
\int_{0}^{\frac{\pi}{2}}\sin^{2n}x \mathrm{d}x\geqslant\int_{\frac{\pi}{2}-\delta}^{\frac{\pi}{2}}e^{-n(1 + \varepsilon)(x - \frac{\pi}{2})^{2}}\mathrm{d}x
=\int_{0}^{\delta}e^{-n(1 + \varepsilon)y^{2}}\mathrm{d}y
=\frac{1}{\sqrt{n(1 + \varepsilon)}}\int_{0}^{\delta\sqrt{n(1 + \varepsilon)}}e^{-z^{2}}\mathrm{d}z.
\end{align*}
因此我们有
\[
\frac{1}{\sqrt{1 + \varepsilon}}\int_{0}^{\infty}e^{-z^{2}}\mathrm{d}z\leqslant\lim_{n\rightarrow\infty}\sqrt{n}\int_{0}^{\frac{\pi}{2}}\sin^{2n}x \mathrm{d}x\leqslant\frac{1}{\sqrt{1 - \varepsilon}}\int_{0}^{\infty}e^{-z^{2}}\mathrm{d}z,
\]
由\(\varepsilon\)任意性即可得
\[
\lim_{n\rightarrow\infty}\sqrt{n}\int_{0}^{\frac{\pi}{2}}\sin^{2n}x \mathrm{d}x=\int_{0}^{\infty}e^{-z^{2}}\mathrm{d}z=\frac{\sqrt{\pi}}{2}.
\]
再结合\eqref{theorem4.6-1.1}式可得
\begin{align*}
\lim_{n\rightarrow \infty} \frac{\pi \sqrt{n}}{2}\frac{(2n-1)!!}{(2n)!!}=\frac{\sqrt{\pi}}{2}.
\end{align*}
即
\begin{align*}
\lim_{n\rightarrow \infty} \sqrt{\pi n}\cdot \frac{(2n-1)!!}{(2n)!!}=1.
\end{align*}
故$\frac{(2n)!!}{(2n-1)!!}\sim \sqrt{\pi n},n\rightarrow +\infty $.

\end{proof}

\begin{example}\label{Laplace方法例题1}
求\(\int_{0}^{\infty} \frac{1}{(2 + x^2)^n}\mathrm{d}x, n \to \infty\)的等价无穷小.
\end{example}
\begin{solution}
由Taylor定理可知,对\(\forall\varepsilon \in(0,1)\),存在\(\delta > 0\),使得当\(x\in [0,\delta]\)时,有
\[
\frac{x^2}{2}-\varepsilon x^2\leqslant\ln\left(1+\frac{x^2}{2}\right)\leqslant\frac{x^2}{2}+\varepsilon x^2.
\]
现在,一方面我们有
\begin{align*}
\int_0^{\infty}\frac{1}{(2 + x^2)^n}\mathrm{d}x&=\frac{1}{2^n}\int_0^{\infty}\frac{1}{\left(1+\frac{x^2}{2}\right)^n}\mathrm{d}x=\frac{1}{2^n}\left(\int_0^{\delta}\frac{1}{\left(1+\frac{x^2}{2}\right)^n}\mathrm{d}x+\int_{\delta}^{\infty}\frac{1}{\left(1+\frac{x^2}{2}\right)^n}\mathrm{d}x\right)\\
&=\frac{1}{2^n}\left(\int_0^{\delta}e^{-n\ln\left(1+\frac{x^2}{2}\right)}\mathrm{d}x+\int_{\delta}^{\infty}\frac{1}{\left(1+\frac{x^2}{2}\right)^n}\mathrm{d}x\right)\\
&\leqslant\frac{1}{2^n}\left(\int_0^{\delta}e^{-n\left(\frac{x^2}{2}-\varepsilon x^2\right)}\mathrm{d}x+\int_{\delta}^{\infty}\frac{1}{1+\frac{x^2}{2}}\cdot\frac{1}{\left(1+\frac{\delta^2}{2}\right)^{n - 1}}\mathrm{d}x\right)\\
&\xlongequal{\text{令}y = x\sqrt{n\left(\frac{1}{2}-\varepsilon\right)}}\frac{1}{2^n}\left(\frac{1}{\sqrt{n\left(\frac{1}{2}-\varepsilon\right)}}\int_0^{\delta\sqrt{n\left(\frac{1}{2}-\varepsilon\right)}}e^{-y^2}\mathrm{d}y+\frac{\sqrt{2}}{\left(1+\frac{\delta^2}{2}\right)^{n - 1}}\left(\frac{\pi}{2}-\arctan\frac{\delta}{\sqrt{2}}\right)\right)\\
&\leqslant\frac{1}{2^n}\left(\frac{1}{\sqrt{n\left(\frac{1}{2}-\varepsilon\right)}}\int_0^{\infty}e^{-y^2}\mathrm{d}y+\frac{\pi\sqrt{2}}{2\left(1+\frac{\delta^2}{2}\right)^{n - 1}}\right)=\frac{1}{2^n}\left(\frac{\sqrt{\pi}}{2\sqrt{n\left(\frac{1}{2}-\varepsilon\right)}}+\frac{\pi\sqrt{2}}{2\left(1+\frac{\delta^2}{2}\right)^{n - 1}}\right).
\end{align*}
于是
\[
\int_0^{\infty}\frac{2^n\sqrt{n}}{(2 + x^2)^n}\mathrm{d}x\leqslant\frac{\sqrt{\pi}}{2\sqrt{\left(\frac{1}{2}-\varepsilon\right)}}+\frac{\pi\sqrt{2n}}{2\left(1+\frac{\delta^2}{2}\right)^{n - 1}}.
\]
上式两边同时令\(n\rightarrow\infty\)并取上极限得到
\[
\varlimsup_{n\rightarrow\infty}\int_0^{\infty}\frac{2^n\sqrt{n}}{(2 + x^2)^n}\mathrm{d}x\leqslant\varlimsup_{n\rightarrow\infty}\left(\frac{\sqrt{\pi}}{2\sqrt{\left(\frac{1}{2}-\varepsilon\right)}}+\frac{\pi\sqrt{2n}}{2\left(1+\frac{\delta^2}{2}\right)^{n - 1}}\right)=\frac{\sqrt{\pi}}{2\sqrt{\left(\frac{1}{2}-\varepsilon\right)}}.
\]
再由\(\varepsilon\)的任意性可得$
\varlimsup_{n\rightarrow\infty}\int_0^{\infty}\frac{2^n\sqrt{n}}{(2 + x^2)^n}\mathrm{d}x\leqslant\frac{\sqrt{\pi}}{2\sqrt{\frac{1}{2}}}=\sqrt{\frac{\pi}{2}}.$

另外一方面,我们有
\begin{align*}
\int_0^{\infty}\frac{1}{(2 + x^2)^n}\mathrm{d}x&=\frac{1}{2^n}\int_0^{\infty}\frac{1}{\left(1+\frac{x^2}{2}\right)^n}\mathrm{d}x\geqslant\frac{1}{2^n}\int_0^{\delta}\frac{1}{\left(1+\frac{x^2}{2}\right)^n}\mathrm{d}x\\
&=\frac{1}{2^n}\int_0^{\delta}e^{-n\ln\left(1+\frac{x^2}{2}\right)}\mathrm{d}x\geqslant\frac{1}{2^n}\int_0^{\delta}e^{-n\left(\frac{x^2}{2}+\varepsilon x^2\right)}\mathrm{d}x\\
&\xlongequal{\text{令}y = x\sqrt{n\left(\frac{1}{2}+\varepsilon\right)}}\frac{1}{2^n\sqrt{n\left(\frac{1}{2}+\varepsilon\right)}}\int_0^{\delta\sqrt{n\left(\frac{1}{2}+\varepsilon\right)}}e^{-y^2}\mathrm{d}y.
\end{align*}
于是
\[
\int_0^{\infty}\frac{2^n\sqrt{n}}{(2 + x^2)^n}\mathrm{d}x\geqslant\frac{1}{\sqrt{\left(\frac{1}{2}+\varepsilon\right)}}\int_0^{\delta\sqrt{n\left(\frac{1}{2}+\varepsilon\right)}}e^{-y^2}\mathrm{d}y.
\]
上式两边同时令\(n\rightarrow\infty\)并取下极限得到
\[
\varliminf_{n\rightarrow\infty}\int_0^{\infty}\frac{2^n\sqrt{n}}{(2 + x^2}^n)\mathrm{d}x\geqslant\varliminf_{n\rightarrow\infty}\frac{1}{\sqrt{\left(\frac{1}{2}+\varepsilon\right)}}\int_0^{\delta\sqrt{n\left(\frac{1}{2}+\varepsilon\right)}}e^{-y^2}\mathrm{d}y=\varliminf_{n\rightarrow\infty}\frac{1}{\sqrt{\left(\frac{1}{2}+\varepsilon\right)}}\int_0^{\infty}e^{-y^2}\mathrm{d}y=\frac{\sqrt{\pi}}{2\sqrt{\left(\frac{1}{2}+\varepsilon\right)}}.
\]
再由\(\varepsilon\)的任意性可得$\varliminf_{n\rightarrow\infty}\int_0^{\infty}\frac{2^n\sqrt{n}}{(2 + x^2)^n}\mathrm{d}x\geqslant\frac{\sqrt{\pi}}{2\sqrt{\frac{1}{2}}}=\sqrt{\frac{\pi}{2}}.$

因此,再结合\(\varliminf_{n\rightarrow\infty}\int_0^{\infty}\frac{2^n\sqrt{n}}{(2 + x^2)^n}\mathrm{d}x\leqslant\varlimsup_{n\rightarrow\infty}\int_0^{\infty}\frac{2^n\sqrt{n}}{(2 + x^2)^n}\mathrm{d}x\),我们就有
\[
\sqrt{\frac{\pi}{2}}\leqslant\varliminf_{n\rightarrow\infty}\int_0^{\infty}\frac{2^n\sqrt{n}}{(2 + x^2)^n}\mathrm{d}x\leqslant\varlimsup_{n\rightarrow\infty}\int_0^{\infty}\frac{2^n\sqrt{n}}{(2 + x^2)^n}\mathrm{d}x\leqslant\sqrt{\frac{\pi}{2}}.
\]
故\(\lim_{n\rightarrow\infty}\int_0^{\infty}\frac{2^n\sqrt{n}}{(2 + x^2)^n}\mathrm{d}x=\sqrt{\frac{\pi}{2}}\).即\(\int_0^{\infty}\frac{1}{(2 + x^2)^n}\mathrm{d}x=\frac{\sqrt{\pi}}{2^n\sqrt{2n}}+o\left(\frac{1}{2^n\sqrt{n}}\right),n\rightarrow\infty\).

\end{solution}

\begin{example}\label{Laplace方法例题2}
求\(\int_{0}^{x} e^{-y^2} \mathrm{d}y, x \to +\infty\)的渐近估计(仅两项).
\end{example}
\begin{note}
因为\(\lim_{x\rightarrow +\infty}\int_0^x{e^{-y^2}\mathrm{d}y}=\frac{\sqrt{\pi}}{2}\),所以实际上只需要估计
\begin{align*}
\frac{\sqrt{\pi}}{2}-\int_0^x{e^{-y^2}\mathrm{d}y}=\int_0^{\infty}{e^{-y^2}\mathrm{d}y}-\int_0^x{e^{-y^2}\mathrm{d}y}=\int_x^{\infty}{e^{-y^2}\mathrm{d}y},x\rightarrow+\infty .
\end{align*}
\end{note}
\begin{solution}
由\(Taylor\)定理可知,对\(\forall\varepsilon > 0\),存在\(\delta > 0\),使得当\(x\in[0,\delta]\)时,有
\[
2x-\varepsilon x\leqslant x^2 + 2x\leqslant 2x+\varepsilon x.
\]
现在,一方面我们有
\begin{align*}
\int_x^{\infty}{e^{-y^2}\mathrm{d}y}&\xlongequal{\text{令}y = xu} x\int_1^{\infty}{e^{-(xu)^2}\mathrm{d}u}\xlongequal{\text{令}t = u - 1} x\int_0^{\infty}{e^{-(xt + x)^2}\mathrm{d}t}\\
&=x\int_0^{\infty}{e^{-(xt)^2 - 2x^2t - x^2}\mathrm{d}t}=xe^{-x^2}\int_0^{\infty}{e^{-x^2(t^2 + 2t)}\mathrm{d}t}\\
&=xe^{-x^2}\left(\int_0^{\delta}{e^{-x^2(t^2 + 2t)}\mathrm{d}t}+\int_{\delta}^{\infty}{e^{-x^2(t^2 + 2t)}\mathrm{d}t}\right)\\
&\leqslant xe^{-x^2}\left(\int_0^{\delta}{e^{-x^2(2t+\varepsilon t)}\mathrm{d}t}+\int_{\delta}^{\infty}{e^{-x^2(t + 2)}e^{-x^2\delta}\mathrm{d}t}\right)\\
&=xe^{-x^2}\left(\frac{1 - e^{-(2+\varepsilon)x^2\delta}}{(2+\varepsilon)x^2}+\frac{e^{-2x^2(\delta + 1)}}{x^2}\right)\\
&=\frac{e^{-x^2}}{x}\left(\frac{1 - e^{-(2+\varepsilon)x^2\delta}}{2+\varepsilon}+e^{-2x^2(\delta + 1)}\right).
\end{align*}
于是就有
\begin{align*}
xe^{x^2}\int_x^{\infty}{e^{-y^2}\mathrm{d}y}\leqslant\frac{1 - e^{-(2+\varepsilon)x^2\delta}}{2+\varepsilon}+e^{-2x^2(\delta + 1)} .
\end{align*}
上式两边同时令\(x\rightarrow+\infty\)并取上极限得到
\[
\varlimsup_{x\rightarrow+\infty}xe^{x^2}\int_x^{\infty}{e^{-y^2}\mathrm{d}y}\leqslant\varlimsup_{x\rightarrow+\infty}\left(\frac{1 - e^{-(2+\varepsilon)x^2\delta}}{2+\varepsilon}+e^{-2x^2(\delta + 1)}\right)=\frac{1}{2+\varepsilon}.
\]
再由\(\varepsilon\)的任意性可得\(\varlimsup_{x\rightarrow+\infty}xe^{x^2}\int_x^{\infty}{e^{-y^2}\mathrm{d}y}\leqslant\frac{1}{2}\).

另外一方面,我们有
\begin{align*}
\int_x^{\infty}{e^{-y^2}\mathrm{d}y}&\xlongequal{\text{令}y = xu} x\int_1^{\infty}{e^{-(xu)^2}\mathrm{d}u}\xlongequal{\text{令}t = u - 1} x\int_0^{\infty}{e^{-(xt + x)^2}\mathrm{d}t}\\
&=x\int_0^{\infty}{e^{-(xt)^2 - 2x^2t - x^2}\mathrm{d}t}=xe^{-x^2}\int_0^{\infty}{e^{-x^2(t^2 + 2t)}\mathrm{d}t}\\
&\geqslant xe^{-x^2}\int_0^{\delta}{e^{-x^2(t^2 + 2t)}\mathrm{d}t}\geqslant xe^{-x^2}\int_0^{\delta}{e^{-x^2(2t-\varepsilon t)}\mathrm{d}t}\\
&=xe^{-x^2}\cdot\frac{1 - e^{-(2-\varepsilon)x^2\delta}}{(2-\varepsilon)x^2}.
\end{align*}
于是就有\begin{align*}
xe^{x^2}\int_x^{\infty}{e^{-y^2}\mathrm{d}y}\geqslant\frac{1 - e^{-(2-\varepsilon)x^2\delta}}{(2-\varepsilon)x^2}.
\end{align*}
上式两边同时令\(x\rightarrow+\infty\)并取下极限得到
\[
\varliminf_{x\rightarrow +\infty}xe^{x^2}\int_x^{\infty}{e^{-y^2}\mathrm{d}y}\geqslant\varliminf_{x\rightarrow +\infty}\frac{1 - e^{-(2-\varepsilon)x^2\delta}}{(2-\varepsilon)x^2}=\frac{1}{2-\varepsilon}.
\]
再由\(\varepsilon\)的任意性可得\(\varliminf_{x\rightarrow +\infty}xe^{x^2}\int_x^{\infty}{e^{-y^2}\mathrm{d}y}\geqslant\frac{1}{2}\).

因此,再结合\(\varliminf_{x\rightarrow +\infty}xe^{x^2}\int_x^{\infty}{e^{-y^2}\mathrm{d}y}\leqslant\varlimsup_{x\rightarrow +\infty}xe^{x^2}\int_x^{\infty}{e^{-y^2}\mathrm{d}y}\),我们就有
\[
\frac{1}{2}\leqslant\varliminf_{x\rightarrow +\infty}xe^{x^2}\int_x^{\infty}{e^{-y^2}\mathrm{d}y}\leqslant\varlimsup_{x\rightarrow+\infty}xe^{x^2}\int_x^{\infty}{e^{-y^2}\mathrm{d}y}\leqslant\frac{1}{2}.
\]
故\(\lim_{x\rightarrow +\infty}xe^{x^2}\int_x^{\infty}{e^{-y^2}\mathrm{d}y}=\frac{1}{2}\),即\(\int_x^{\infty}{e^{-y^2}\mathrm{d}y}=\frac{e^{-x^2}}{2x}+o\left(\frac{e^{-x^2}}{x}\right),x\rightarrow+\infty\).

因此\(\int_0^x{e^{-y^2}\mathrm{d}y}=\frac{\sqrt{\pi}}{2}-\int_x^{\infty}{e^{-y^2}\mathrm{d}y}=\frac{\sqrt{\pi}}{2}-\frac{e^{-x^2}}{2x}+o\left(\frac{e^{-x^2}}{x}\right),x\rightarrow+\infty\).

\end{solution}

\begin{example}\label{Laplace方法例题3}
计算$\lim_{n \to \infty} \int_{0}^{10n} \left(1 - \left|\sin \left(\frac{x}{n}\right)\right|\right)^n \mathrm{d}x.$
\end{example}
\begin{solution}
由Taylor定理可知,对\(\forall \varepsilon \in (0,1)\),存在\(\delta \in (0,\frac{\pi}{4})\),使得当\(x\in [0,\delta]\)时,有
\[
-t-\varepsilon t\leqslant\ln(1-\sin t)\leqslant -t+\varepsilon t.
\]
此时,我们有
\begin{align}
&\int_0^{10n}{(1-\vert\sin(\frac{x}{n})\vert)^n}\mathrm{d}x\xlongequal{\text{令}x = nt} n\int_0^{10}{(1-\vert\sin t\vert)^n}\mathrm{d}t=n\int_0^{10}{e^{n\ln(1-\vert\sin t\vert)}}\mathrm{d}t
\nonumber
\\
&=n\int_0^{\delta}{e^{n\ln(1-\vert\sin t\vert)}}\mathrm{d}t + n\int_{\delta}^{\pi -\delta}{e^{n\ln(1-\vert\sin t\vert)}}\mathrm{d}t + n\int_{\pi -\delta}^{\pi +\delta}{e^{n\ln(1-\vert\sin t\vert)}}\mathrm{d}t + n\int_{\pi +\delta}^{2\pi -\delta}{e^{n\ln(1-\vert\sin t\vert)}}\mathrm{d}t
\nonumber
\\
&\quad +n\int_{2\pi -\delta}^{2\pi +\delta}{e^{n\ln(1-\vert\sin t\vert)}}\mathrm{d}t + n\int_{2\pi +\delta}^{3\pi -\delta}{e^{n\ln(1-\vert\sin t\vert)}}\mathrm{d}t + n\int_{3\pi -\delta}^{3\pi +\delta}{e^{n\ln(1-\vert\sin t\vert)}}\mathrm{d}t+n\int_{3\pi +\delta}^{10}{e^{n\ln\mathrm{(}1-\left| \sin t \right|)}\mathrm{d}t}
\nonumber
\\
&=n\int_0^{\delta}{e^{n\ln(1-\sin t)}}\mathrm{d}t + n\int_{\delta}^{\pi -\delta}{e^{n\ln(1-\sin t)}}\mathrm{d}t + n\int_{\pi -\delta}^{\pi +\delta}{e^{n\ln(1-\vert\sin t\vert)}}\mathrm{d}t + n\int_{\pi +\delta}^{2\pi -\delta}{e^{n\ln(1+\sin t)}}\mathrm{d}t
\nonumber
\\
&\quad +n\int_{2\pi -\delta}^{2\pi +\delta}{e^{n\ln(1-\vert\sin t\vert)}}\mathrm{d}t + n\int_{2\pi +\delta}^{3\pi -\delta}{e^{n\ln(1-\sin t)}}\mathrm{d}t + n\int_{3\pi -\delta}^{3\pi +\delta}{e^{n\ln(1-\vert\sin t\vert)}}\mathrm{d}t+n\int_{3\pi +\delta}^{10}{e^{n\ln \left( 1-\sin t \right)}\mathrm{d}t}.\label{equation:3.366}
\end{align}
由积分换元可得
\begin{gather*}
n\int_{\pi -\delta}^{\pi}{e^{n\ln(1-\sin t)}}\mathrm{d}t\xlongequal{\text{令}u=\pi -t} -n\int_{\delta}^0{e^{n\ln(1-\sin(\pi -u))}}\mathrm{d}u=n\int_0^{\delta}{e^{n\ln(1-\sin u)}}\mathrm{d}u,\\
n\int_{\pi}^{\pi +\delta}{e^{n\ln(1+\sin t)}}\mathrm{d}t\xlongequal{\text{令}u = t-\pi} n\int_0^{\delta}{e^{n\ln(1+\sin(\pi +u))}}\mathrm{d}u=n\int_0^{\delta}{e^{n\ln(1-\sin u)}}\mathrm{d}u,\\
n\int_{\pi +\delta}^{2\pi -\delta}{e^{n\ln(1+\sin t)}}\mathrm{d}t\xlongequal{\text{令}u = t-\pi} \int_{\delta}^{\pi -\delta}{e^{n\ln(1+\sin(\pi +u))}}\mathrm{d}u=\int_{\delta}^{\pi -\delta}{e^{n\ln(1-\sin u)}}\mathrm{d}u,\\
n\int_{2\pi +\delta}^{3\pi -\delta}{e^{n\ln(1-\sin t)}}\mathrm{d}t\xlongequal{\text{令}u = t - 2\pi} \int_{\delta}^{\pi -\delta}{e^{n\ln(1-\sin(2\pi +u))}}\mathrm{d}u=\int_{\delta}^{\pi -\delta}{e^{n\ln(1+\sin u)}}\mathrm{d}u.
\end{gather*}
从而
\[
n\int_{\pi -\delta}^{\pi +\delta}{e^{n\ln(1-\vert\sin t\vert)}}\mathrm{d}t=n\int_{\pi -\delta}^{\pi}{e^{n\ln(1-\sin t)}}\mathrm{d}t + n\int_{\pi}^{\pi +\delta}{e^{n\ln(1-\sin t)}}\mathrm{d}t = 2n\int_0^{\delta}{e^{n\ln(1+\sin t)}}\mathrm{d}t.
\]
同理
\begin{align*}
n\int_{2\pi -\delta}^{2\pi +\delta}{e^{n\ln(1-\vert\sin t\vert)}}\mathrm{d}t = n\int_{3\pi -\delta}^{3\pi +\delta}{e^{n\ln(1-\vert\sin t\vert)}}\mathrm{d}t = 2n\int_0^{\delta}{e^{n\ln(1-\sin t)}}\mathrm{d}t.
\end{align*}
于是原积分\eqref{equation:3.366}式可化为
\[
\int_0^{10n}{(1-\vert\sin(\frac{x}{n})\vert)^n}\mathrm{d}x = 7n\int_0^{\delta}{e^{n\ln(1-\sin t)}}\mathrm{d}t + 3\int_{\delta}^{\pi -\delta}{e^{n\ln(1-\sin t)}}\mathrm{d}t+n\int_{3\pi +\delta}^{10}{e^{n\ln \left( 1+\sin t \right)}\mathrm{d}t}.
\]
进而,一方面我们有
\begin{align*}
\int_0^{10n}{(1-\vert\sin(\frac{x}{n})\vert)^n}\mathrm{d}x&=7n\int_0^{\delta}{e^{n\ln(1-\sin t)}}\mathrm{d}t + 3n\int_{\delta}^{\pi -\delta}{e^{n\ln(1-\sin t)}}\mathrm{d}t+n\int_{3\pi +\delta}^{10}{e^{n\ln \left( 1+\sin t \right)}\mathrm{d}t}\\
&= 7n\int_0^{\delta}{e^{n(-t+\varepsilon t)}\mathrm{d}t}+3n\int_{\delta}^{\pi -\delta}{e^{n\ln\mathrm{(}1-\sin \delta )}\mathrm{d}t}+n\int_{\delta}^{10-3\pi}{e^{n\ln \left( 1-\sin t \right)}\mathrm{d}t}
\\
&\leqslant 7n\int_0^{\delta}{e^{n(-t+\varepsilon t)}}\mathrm{d}t + 4n\int_{\delta}^{\pi -\delta}{e^{n\ln(1-\sin \delta)}}\mathrm{d}t\\
&=7\cdot\frac{e^{(\varepsilon -1)n\delta}-1}{\varepsilon -1}+4ne^{n\ln(1+\sin \delta)}(\pi -2\delta).
\end{align*}
上式两边同时令\(n\rightarrow\infty\)并取上极限得到
\[
\varlimsup_{n\rightarrow\infty}\int_0^{10n}{(1-\vert\sin(\frac{x}{n})\vert)^n}\mathrm{d}x\leqslant\varlimsup_{n\rightarrow\infty}\left[7\cdot\frac{e^{(\varepsilon -1)n\delta}-1}{\varepsilon -1}+4ne^{n\ln(1-\sin \delta)}(\pi -2\delta)\right]=\frac{7}{1-\varepsilon}.
\]
再由\(\varepsilon\)的任意性可得\(\varlimsup_{n\rightarrow\infty}\int_0^{10n}{(1-\vert\sin(\frac{x}{n})\vert)^n}\mathrm{d}x\leqslant 7\).

另外一方面,我们有
\begin{align*}
\int_0^{10n}{(1-\vert\sin(\frac{x}{n})\vert)^n}\mathrm{d}x&=7n\int_0^{\delta}{e^{n\ln(1-\sin t)}}\mathrm{d}t + 3\int_{\delta}^{\pi -\delta}{e^{n\ln(1-\sin t)}}\mathrm{d}t\\
&\geqslant 7n\int_0^{\delta}{e^{n\ln(1-\sin t)}}\mathrm{d}t\geqslant 7n\int_0^{\delta}{e^{n(-t-\varepsilon t)}}\mathrm{d}t\\
&=7\cdot\frac{1 - e^{-(\varepsilon +1)n\delta}}{\varepsilon +1}
\end{align*}
上式两边同时令\(n\rightarrow\infty\)并取下极限得到
\[
\varliminf_{n\rightarrow\infty}\int_0^{10n}{(1-\vert\sin(\frac{x}{n})\vert)^n}\mathrm{d}x\geqslant\varliminf_{n\rightarrow\infty}7\cdot\frac{1 - e^{-(\varepsilon +1)n\delta}}{\varepsilon +1}=\frac{7}{\varepsilon +1}.
\]
再由\(\varepsilon\)的任意性可得\(\varliminf_{n\rightarrow\infty}\int_0^{10n}{(1-\vert\sin(\frac{x}{n})\vert)^n}\mathrm{d}x\geqslant\frac{7}{\varepsilon +1}\).

因此,再结合\(\varliminf_{n\rightarrow\infty}\int_0^{10n}{(1-\vert\sin(\frac{x}{n})\vert)^n}\mathrm{d}x\leqslant\varlimsup_{n\rightarrow\infty}\int_0^{10n}{(1-\vert\sin(\frac{x}{n})\vert)^n}\mathrm{d}x\),我们就有
\[
7\leqslant\varliminf_{n\rightarrow\infty}\int_0^{10n}{(1-\vert\sin(\frac{x}{n})\vert)^n}\mathrm{d}x\leqslant\varlimsup_{n\rightarrow\infty}\int_0^{10n}{(1-\vert\sin(\frac{x}{n})\vert)^n}\mathrm{d}x\leqslant 7.
\]
故\(\lim_{n\rightarrow\infty}\int_0^{10n}{(1-\vert\sin(\frac{x}{n})\vert)^n}\mathrm{d}x = 7\).

\end{solution}

\begin{example}
求极限 $\lim_{n\rightarrow \infty} \frac{\int_0^1{\left( 1-\frac{x}{2} \right) ^n\left( 1-\frac{x}{4} \right) ^n\mathrm{d}x}}{\int_0^1{\left( 1-\frac{x}{2} \right) ^n\mathrm{d}x}}$。
\end{example}
\begin{proof}
首先注意到
\begin{align}
\int_0^1{\left( 1-\frac{x}{2} \right) ^n\mathrm{d}x}&=\frac{2}{n+1}\left( 1-\frac{x}{2} \right) ^{n+1}\Big|_{1}^{0}=\frac{2}{n+1}\left( 1-\frac{1}{2^{n+1}} \right) .\label{example-Laplace0.7-0.0}
\end{align}
接着,由Taylor定理可知
\[
\ln \left( 1-\frac{3}{4}x+\frac{x^2}{8} \right) =-\frac{3}{4}x+o\left( x \right) .
\]
从而对$\forall \varepsilon \in \left( 0,\frac{1}{4} \right)$,都存在$\delta \in \left( 0,1 \right)$,使得
\[
-\frac{3}{4}x-\varepsilon x\leqslant \ln \left( 1-\frac{3}{4}x+\frac{x^2}{8} \right) \leqslant -\frac{3}{4}x+\varepsilon x,\forall x\in \left[ -\delta ,\delta \right] .
\]
于是一方面,我们有
\begin{align*}
\int_0^1{\left( 1-\frac{x}{2} \right) ^n\left( 1-\frac{x}{4} \right) ^n\mathrm{d}x}&=\int_0^1{e^{n\ln \left( 1-\frac{3}{4}x+\frac{x^2}{8} \right)}\mathrm{d}x}=\int_0^{\delta}{e^{n\ln \left( 1-\frac{3}{4}x+\frac{x^2}{8} \right)}\mathrm{d}x}+\int_{\delta}^1{e^{n\ln \left( 1-\frac{3}{4}x+\frac{x^2}{8} \right)}\mathrm{d}x}\\
&\leqslant \int_0^{\delta}{e^{n\left( -\frac{3}{4}x+\varepsilon x \right)}\mathrm{d}x}+\int_{\delta}^1{e^{n\left( -\frac{3}{4}x+\varepsilon x \right)}\mathrm{d}x}\leqslant \frac{1}{n}\int_0^{n\delta}{e^{\left( -\frac{3}{4}+\varepsilon \right) x}\mathrm{d}x}+\int_{\delta}^1{e^{n\left( -\frac{3}{4}+\varepsilon \right) \delta}\mathrm{d}x}\\
&\leqslant \frac{1}{n}\int_0^{\infty}{e^{\left( -\frac{3}{4}+\varepsilon \right) x}\mathrm{d}x}+e^{n\left( -\frac{3}{4}+\varepsilon \right) \delta}\left( 1-\delta \right) =-\frac{1}{\left( -\frac{3}{4}+\varepsilon \right) n}+e^{n\left( -\frac{3}{4}+\varepsilon \right) \delta}\left( 1-\delta \right) .
\end{align*}
另一方面,我们有
\begin{align*}
\int_0^1{\left( 1-\frac{x}{2} \right) ^n\left( 1-\frac{x}{4} \right) ^n\mathrm{d}x}&=\int_0^1{e^{n\ln \left( 1-\frac{3}{4}x+\frac{x^2}{8} \right)}\mathrm{d}x}=\int_0^{\delta}{e^{n\ln \left( 1-\frac{3}{4}x+\frac{x^2}{8} \right)}\mathrm{d}x}+\int_{\delta}^1{e^{n\ln \left( 1-\frac{3}{4}x+\frac{x^2}{8} \right)}\mathrm{d}x}\\
&\geqslant \int_0^{\delta}{e^{n\left( -\frac{3}{4}x-\varepsilon x \right)}\mathrm{d}x}+\int_{\delta}^1{e^{n\left( -\frac{3}{4}x-\varepsilon x \right)}\mathrm{d}x}\geqslant \frac{1}{n}\int_0^{n\delta}{e^{\left( -\frac{3}{4}-\varepsilon \right) x}\mathrm{d}x}+\int_{\delta}^1{e^{n\left( -\frac{3}{4}-\varepsilon \right)}\mathrm{d}x}\\
&=\frac{e^{\left( -\frac{3}{4}-\varepsilon \right) n\delta}-1}{\left( -\frac{3}{4}-\varepsilon \right) n}+e^{n\left( -\frac{3}{4}-\varepsilon \right)}\left( 1-\delta \right) .
\end{align*}
因此
\begin{align*}
\frac{e^{\left( -\frac{3}{4}-\varepsilon \right) n\delta}-1}{-\frac{3}{4}-\varepsilon}+ne^{n\left( -\frac{3}{4}-\varepsilon \right)}\left( 1-\delta \right) \leqslant n\int_0^1{\left( 1-\frac{x}{2} \right) ^n\left( 1-\frac{x}{4} \right) ^n\mathrm{d}x}\leqslant -\frac{1}{-\frac{3}{4}+\varepsilon}+ne^{n\left( -\frac{3}{4}+\varepsilon \right) \delta}\left( 1-\delta \right) .
\end{align*}
上式两边同时令$n\rightarrow \infty$,得
\[
-\frac{1}{-\frac{3}{4}-\varepsilon}\leqslant \underset{n\rightarrow \infty}{\underline{\lim }}n\int_0^1{\left( 1-\frac{x}{2} \right) ^n\left( 1-\frac{x}{4} \right) ^n\mathrm{d}x}\leqslant \underset{n\rightarrow \infty}{\overline{\lim }}n\int_0^1{\left( 1-\frac{x}{2} \right) ^n\left( 1-\frac{x}{4} \right) ^n\mathrm{d}x}\leqslant -\frac{1}{-\frac{3}{4}+\varepsilon}.
\]
再令$\varepsilon \rightarrow 0^+$,得
\[
\lim\limits_{n\rightarrow \infty}n\int_0^1{\left( 1-\frac{x}{2} \right) ^n\left( 1-\frac{x}{4} \right) ^n\mathrm{d}x}=\frac{4}{3}.
\]
故$\int_0^1{\left( 1-\frac{x}{2} \right) ^n\left( 1-\frac{x}{4} \right) ^n\mathrm{d}x}=\frac{4}{3n}+o\left( \frac{1}{n} \right) .$于是再结合\eqref{example-Laplace0.7-0.0}式可得
\[
\lim_{n\rightarrow \infty} \frac{\int_0^1{\left( 1-\frac{x}{2} \right) ^n\left( 1-\frac{x}{4} \right) ^n\mathrm{d}x}}{\int_0^1{\left( 1-\frac{x}{2} \right) ^n\mathrm{d}x}}=\lim_{n\rightarrow \infty} \frac{\frac{4}{3n}+o\left( \frac{1}{n} \right)}{\frac{2}{n+1}\left( 1-\frac{1}{2^{n+1}} \right)}=\frac{2}{3}.
\]

\end{proof}

\begin{example}
证明极限 $\lim\limits_{n \to +\infty} \frac{\int_{0}^{1} \ln^n(1 + x)x^{-n} \mathrm{d}x}{\int_{0}^{1} \frac{\sin^n x}{x^{n - 1}} \mathrm{d}x}$ 存在并求其值。 
\end{example}
\begin{note}
原式可写成$\frac{\int_0^1{\left[ \frac{\ln \left( 1+x \right)}{x} \right] ^n\mathrm{d}x}}{\int_0^1{x\left( \frac{\sin x}{x} \right) ^n\mathrm{d}x}}$,求导可知$\frac{\sin x}{x}$和$\frac{\ln \left( 1+x \right)}{x}$在$(0,1]$上单调递增,故原式分子和分母的阶都集中在$x = 0$处。
因为分母积分的被积函数除指数部分外,$x$在$0$处取值也为$0$,所以我们在估阶的时候需要将$x$也考虑进去。
利用Laplace方法估计分子、分母的阶,但是此时$\frac{\sin x}{x}$和$\frac{\ln \left( 1+x \right)}{x}$在极值点$x = 0$处间断,
故我们需要先对$\frac{\sin x}{x}$和$\frac{\ln \left( 1+x \right)}{x}$补充定义,使相关函数光滑,才能进行Taylor展开。
\end{note}
\begin{proof}
由Taylor公式可知
\begin{align*}
\ln \left( \frac{\sin x}{x} \right) &=\ln \frac{x-\frac{1}{6}x^3+o\left( x^3 \right)}{x}=\ln \left( 1-\frac{x^2}{6}+o\left( x^2 \right) \right) =-\frac{x^2}{6}+o\left( x^2 \right) ,\\
\ln \left( \frac{\ln \left( 1+x \right)}{x} \right) &=\ln \frac{x-\frac{x^2}{2}+\frac{x^3}{3}+o\left( x^3 \right)}{x}=\ln \left( 1-\frac{x}{2}+\frac{x^2}{3}+o\left( x^2 \right) \right) =-\frac{x}{2}+\frac{x^2}{3}+o\left( x^2 \right) .
\end{align*}
从而对$\forall \varepsilon \in \left( 0,\frac{1}{6} \right)$,都存在$\delta \in (0,1)$,使得
\begin{align*}
-\frac{x^2}{6}-\varepsilon x^2&\leqslant \ln \left( \frac{\sin x}{x} \right) \leqslant -\frac{x^2}{6}+\varepsilon x^2,\forall x\in [-\delta ,\delta ] ,\\
-\frac{x}{2}-\varepsilon x&\leqslant \ln \left( \frac{\ln \left( 1+x \right)}{x} \right) \leqslant -\frac{x}{2}+\varepsilon x,\forall x\in [-\delta ,\delta ] .
\end{align*}
于是一方面,我们有
\begin{align*}
\int_0^1{x\left( \frac{\sin x}{x} \right) ^n\mathrm{d}x}&=\int_0^1{xe^{n\ln \left( \frac{\sin x}{x} \right)}\mathrm{d}x}=\int_0^{\delta}{xe^{n\ln \left( \frac{\sin x}{x} \right)}\mathrm{d}x}+\int_{\delta}^1{x\left( \frac{\sin x}{x} \right) ^n\mathrm{d}x}\\
&\leqslant \int_0^{\delta}{xe^{n\left( -\frac{x^2}{6}+\varepsilon x^2 \right)}\mathrm{d}x}+\sin ^n1\int_{\delta}^1{\frac{1}{x^{n-1}}\mathrm{d}x}\leqslant \frac{1}{n}\int_0^{\sqrt{n}\delta}{xe^{\left( -\frac{1}{6}+\varepsilon \right) x^2}\mathrm{d}x}+\sin ^n1\left( \frac{n}{\delta ^n}-1 \right)\\
&\leqslant \frac{1}{n}\int_0^{\infty}{xe^{\left( -\frac{1}{6}+\varepsilon \right) x^2}\mathrm{d}x}+\sin ^n1\left( \frac{n}{\delta ^n}-1 \right) =-\frac{1}{2\left( -\frac{1}{6}+\varepsilon \right) n}+\sin ^n1\left( \frac{n}{\delta ^n}-1 \right) .
\end{align*}
\begin{align*}
\int_0^1{\left[ \frac{\ln \left( 1+x \right)}{x} \right] ^n\mathrm{d}x}&=\int_0^1{e^{n\ln \left( \frac{\ln \left( 1+x \right)}{x} \right)}\mathrm{d}x}=\int_0^{\delta}{e^{n\ln \left( \frac{\ln \left( 1+x \right)}{x} \right)}\mathrm{d}x}+\int_{\delta}^1{\left[ \frac{\ln \left( 1+x \right)}{x} \right] ^n\mathrm{d}x}\\
&\leqslant \int_0^{\delta}{e^{n\left( -\frac{x}{2}+\varepsilon x \right)}\mathrm{d}x}+\left( \ln 2 \right) ^n\int_{\delta}^1{\frac{1}{x^n}\mathrm{d}x}\leqslant \frac{1}{n}\int_0^{n\delta}{e^{\left( -\frac{1}{2}+\varepsilon \right) x}\mathrm{d}x}+\left( \ln 2 \right) ^n\left( \frac{n+1}{\delta ^{n+1}}-1 \right)\\
&\leqslant \frac{1}{n}\int_0^{\infty}{e^{\left( -\frac{1}{2}+\varepsilon \right) x}\mathrm{d}x}+\left( \ln 2 \right) ^n\left( \frac{n+1}{\delta ^{n+1}}-1 \right) =-\frac{1}{\left( -\frac{1}{2}+\varepsilon \right) n}+\left( \ln 2 \right) ^n\left( \frac{n+1}{\delta ^{n+1}}-1 \right) .
\end{align*}
另一方面,我们有
\begin{align*}
\int_0^1{x\left( \frac{\sin x}{x} \right) ^n\mathrm{d}x}&=\int_0^1{xe^{n\ln \left( \frac{\sin x}{x} \right)}\mathrm{d}x}\geqslant \int_0^{\delta}{xe^{n\ln \left( \frac{\sin x}{x} \right)}\mathrm{d}x}\\
&\geqslant \int_0^{\delta}{xe^{n\left( -\frac{x^2}{6}-\varepsilon x^2 \right)}\mathrm{d}x}\geqslant \frac{1}{n}\int_0^{\sqrt{n}\delta}{xe^{\left( -\frac{1}{6}-\varepsilon \right) x^2}\mathrm{d}x}\\
&\geqslant \frac{1}{n}\int_0^{\infty}{xe^{\left( -\frac{1}{6}-\varepsilon \right) x^2}\mathrm{d}x}=-\frac{1}{2\left( -\frac{1}{6}-\varepsilon \right) n}.
\end{align*}
\begin{align*}
\int_0^1{\left[ \frac{\ln \left( 1+x \right)}{x} \right] ^n\mathrm{d}x}&=\int_0^1{e^{n\ln \left( \frac{\ln \left( 1+x \right)}{x} \right)}\mathrm{d}x}\geqslant \int_0^{\delta}{e^{n\ln \left( \frac{\ln \left( 1+x \right)}{x} \right)}\mathrm{d}x}\\
&\geqslant \int_0^{\delta}{e^{n\left( -\frac{x}{2}-\varepsilon x \right)}\mathrm{d}x}\geqslant \frac{1}{n}\int_0^{n\delta}{e^{\left( -\frac{1}{2}-\varepsilon \right) x}\mathrm{d}x}\\
&\geqslant \frac{1}{n}\int_0^{\infty}{e^{\left( -\frac{1}{2}-\varepsilon \right) x}\mathrm{d}x}=-\frac{1}{\left( -\frac{1}{2}-\varepsilon \right) n}.
\end{align*}
因此,我们就有
\begin{gather*}
-\frac{1}{2\left( -\frac{1}{6}-\varepsilon \right)}\leqslant n\int_0^1{x\left( \frac{\sin x}{x} \right) ^n\mathrm{d}x}\leqslant -\frac{1}{2\left( -\frac{1}{6}+\varepsilon \right)}+\sin ^n1\left( \frac{n}{\delta ^n}-1 \right) ,\\
\frac{1}{\frac{1}{2}+\varepsilon}\leqslant n\int_0^1{\left[ \frac{\ln \left( 1+x \right)}{x} \right] ^n\mathrm{d}x}\leqslant -\frac{1}{-\frac{1}{2}+\varepsilon}+\left( \ln 2 \right) ^n\left( \frac{n+1}{\delta ^{n+1}}-1 \right) .
\end{gather*}
令$n\rightarrow \infty$,得
\begin{align*}
-\frac{1}{2\left( -\frac{1}{6}-\varepsilon \right)}&\leqslant \varliminf_{n\to \infty}n\int_0^1{x\left( \frac{\sin x}{x} \right) ^n\mathrm{d}x}\leqslant \varlimsup_{n\to \infty}n\int_0^1{x\left( \frac{\sin x}{x} \right) ^n\mathrm{d}x}\leqslant -\frac{1}{2\left( -\frac{1}{6}+\varepsilon \right)}\\
\frac{1}{\frac{1}{2}+\varepsilon}&\leqslant \varliminf_{n\to \infty}n\int_0^1{\left[ \frac{\ln \left( 1+x \right)}{x} \right] ^n\mathrm{d}x}\leqslant \varlimsup_{n\to \infty}n\int_0^1{\left[ \frac{\ln \left( 1+x \right)}{x} \right] ^n\mathrm{d}x}\leqslant -\frac{1}{-\frac{1}{2}+\varepsilon}
\end{align*}
再令$\varepsilon \rightarrow 0^+$,得
\[
\lim\limits_{n\rightarrow \infty}n\int_0^1{x\left( \frac{\sin x}{x} \right) ^n\mathrm{d}x}=\frac{1}{3},\quad \lim\limits_{n\rightarrow \infty}n\int_0^1{\left[ \frac{\ln \left( 1+x \right)}{x} \right] ^n\mathrm{d}x}=2.
\]
故
\[
\int_0^1{x\left( \frac{\sin x}{x} \right) ^n\mathrm{d}x}=\frac{1}{3n}+o\left( \frac{1}{n} \right) ,\quad \int_0^1{\left[ \frac{\ln \left( 1+x \right)}{x} \right] ^n\mathrm{d}x}=\frac{2}{n}+o\left( \frac{1}{n} \right) .
\]
进而
\[
\lim\limits_{n\rightarrow \infty}\frac{\int_0^1{\ln ^n(1 + x)x^{-n}\mathrm{d}x}}{\int_0^1{\frac{\sin ^nx}{x^{n-1}}\mathrm{d}x}}=\lim\limits_{n\rightarrow \infty}\frac{\int_0^1{\left[ \frac{\ln \left( 1+x \right)}{x} \right] ^n\mathrm{d}x}}{\int_0^1{x\left( \frac{\sin x}{x} \right) ^n\mathrm{d}x}}=\lim\limits_{n\rightarrow \infty}\frac{\frac{2}{n}+o\left( \frac{1}{n} \right)}{\frac{1}{3n}+o\left( \frac{1}{n} \right)}=\frac{2}{3}.
\]

\end{proof}

\begin{example}\label{Laplace方法例题4}
计算$\lim_{n\rightarrow \infty} \frac{\int_0^1{\left( 1-x^2+x^3 \right) ^n\ln \left( x+2 \right) \mathrm{d}x}}{\int_0^1{\left( 1-x^2+x^3 \right) ^n\mathrm{d}x}}.$
\end{example}
\begin{note}
我们首先可以求解出被积函数带$n$次幂部分的最大值点即$1-x^2+x^3$的最大值点为$x=0,1$.于是被积函数的阶一定集中在这两个最大值点附近.
\end{note}
\begin{remark}
注意由$\ln(1 - x^2 + x^3) = x - 1 + o(x - 1),  x\rightarrow 1$.
得到的是$\ln(1 - x^2 + x^3) = x - 1 + o(x - 1),  x\rightarrow 1$.而不是.
\end{remark}
\begin{proof}
由Taylor定理可知,
\begin{align*}
\ln(1 - x^2 + x^3) &= -x^2 + o(x^2),  x\rightarrow 0;\\
\ln(1 - x^2 + x^3) &= x - 1 + o(x - 1),  x\rightarrow 1.
\end{align*}
从而对\(\forall \varepsilon \in (0,\frac{1}{2})\),存在\(\delta_1 \in (0,\frac{1}{10})\),使得
\begin{align*}
-x^2 - \varepsilon x^2&\leqslant\ln(1 - x^2 + x^3)\leqslant -x^2 + \varepsilon x^2, \forall x\in (0,\delta_1);\\
x - 1 - \varepsilon(x - 1)&\leqslant\ln(1 - x^2 + x^3)\leqslant x - 1 + \varepsilon(x - 1), \forall x\in (1 - \delta_1,1).
\end{align*}
设\(f\in C[0,1]\)且$f(0),f(1)>0$,则由连续函数最大值、最小值定理可知,\(f\)在闭区间\([0,\frac{1}{2}]\)和$[\frac{1}{2},1]$上都存在最大值和最小值.设\(M_1 = \sup_{x\in [0,\frac{1}{2}]}f(x)\),\(M_2 = \sup_{x\in [\frac{1}{2},1]}f(x)\).又由连续性可知,对上述\(\varepsilon\),存在\(\delta_2>0\),使得
\begin{align*}
f(0) - \varepsilon&< f(x) < f(0) + \varepsilon, \forall x\in [0,\delta_2];\\
f(1) - \varepsilon&< f(x) < f(1) + \varepsilon, \forall x\in [1 - \delta_2,1].
\end{align*}
取\(\delta = \min\{\delta_1,\delta_2\}\),则一方面我们有
\begin{align*}
\int_0^{\frac{1}{2}}{(1 - x^2 + x^3)^nf(x)\mathrm{d}x}&=\int_0^{\delta}{(1 - x^2 + x^3)^nf(x)\mathrm{d}x}+\int_{\delta}^{\frac{1}{2}}{(1 - x^2 + x^3)^nf(x)\mathrm{d}x}\\
&=\int_0^{\delta}{e^{n\ln(1 - x^2 + x^3)}f(x)\mathrm{d}x}+\int_{\delta}^{\frac{1}{2}}{(1 - x^2 + x^3)^nf(x)\mathrm{d}x}\\
&\leqslant (f(0) + \varepsilon)\int_0^{\delta}{e^{n(-x^2 + \varepsilon x^2)}\mathrm{d}x}+\int_{\delta}^{\frac{1}{2}}{M_1\left(\frac{7}{8}-\delta^2\right)^n\mathrm{d}x}\\
&=\frac{f(0) + \varepsilon}{\sqrt{n(1 - \varepsilon)}}\int_0^{\delta\sqrt{n(1 - \varepsilon)}}{e^{-y^2}\mathrm{d}y}+M_1\left(\frac{7}{8}-\delta^2\right)^n\left(\frac{1}{2}-\delta\right),
\end{align*}
又易知$1-x^2+x^3$在$[0,\frac{2}{3}]$上单调递减,在$(\frac{2}{3},1]$上单调递增.再结合$\delta<\frac{1}{10}$可知,$1-(\frac{1}{2})^2+(\frac{1}{2})^3<1-(\frac{1}{10})^2+(\frac{1}{10})^3<1-(1-\delta)^2+(1-\delta)^3$.从而当$x\in (\frac{1}{2},1-\delta)$时,我们就有$1-x^2+x^3<1-(1-\delta)^2+(1-\delta)^3<1$.进而可得
\begin{align*}
\int_{\frac{1}{2}}^1{(1 - x^2 + x^3)^nf(x)\mathrm{d}x}&=\int_{\frac{1}{2}}^{1 - \delta}{(1 - x^2 + x^3)^nf(x)\mathrm{d}x}+\int_{1 - \delta}^1{(1 - x^2 + x^3)^nf(x)\mathrm{d}x}\\
&=\int_{\frac{1}{2}}^{1 - \delta}{(1 - x^2 + x^3)^nf(x)\mathrm{d}x}+\int_{1 - \delta}^1{e^{n\ln(1 - x^2 + x^3)}f(x)\mathrm{d}x}\\
&\leqslant \int_{\frac{1}{2}}^{1 - \delta}{M_2\left(1-(1-\delta)^2+(1 - \delta)^3\right)^n\mathrm{d}x}+(f(1) + \varepsilon)\int_{1 - \delta}^1{e^{n[x - 1 + \varepsilon(x - 1)]}\mathrm{d}x}\\
&=M_2\left(1-(1-\delta)^2+(1 - \delta)^3\right)^n\left(\frac{1}{2}-\delta\right)+\frac{f(1) + \varepsilon}{n(1 + \varepsilon)}\left(1 - e^{-n\delta(1 + \varepsilon)}\right).
\end{align*}
于是就有
\begin{align*}
\sqrt{n}\int_0^{\frac{1}{2}}{(1 - x^2 + x^3)^nf(x)\mathrm{d}x}&\leqslant \frac{f(0) + \varepsilon}{\sqrt{1 - \varepsilon}}\int_0^{\delta\sqrt{n(1 - \varepsilon)}}{e^{-y^2}\mathrm{d}y}+\sqrt{n} M_1\left(\frac{7}{8}-\delta^2\right)^n\left(\frac{1}{2}-\delta\right),\\
n\int_{\frac{1}{2}}^1{(1 - x^2 + x^3)^nf(x)\mathrm{d}x}&\leqslant n M_2\left(\frac{3}{4}+(1 - \delta)^3\right)^n\left(\frac{1}{2}-\delta\right)+\frac{f(1) + \varepsilon}{1 + \varepsilon}\left(1 - e^{-n\delta(1 + \varepsilon)}\right).
\end{align*}
上式两边同时令\(n\rightarrow\infty\)并取上极限得到
\begin{gather*}
\varlimsup_{n\rightarrow\infty}\sqrt{n}\int_0^{\frac{1}{2}}{(1 - x^2 + x^3)^nf(x)\mathrm{d}x}\leqslant \frac{f(0) + \varepsilon}{\sqrt{1 - \varepsilon}}\int_0^{\infty}{e^{-y^2}\mathrm{d}y}=\frac{\sqrt{\pi}}{2\sqrt{1 - \varepsilon}}(f(0) + \varepsilon),\\
\varlimsup_{n\rightarrow\infty}n\int_{\frac{1}{2}}^1{(1 - x^2 + x^3)^nf(x)\mathrm{d}x}\leqslant \frac{f(1) + \varepsilon}{1 + \varepsilon}.
\end{gather*}
再由\(\varepsilon\)的任意性可得\(\varlimsup_{n\rightarrow\infty}\sqrt{n}\int_0^{\frac{1}{2}}{(1 - x^2 + x^3)^nf(x)\mathrm{d}x}\leqslant \frac{\sqrt{\pi}}{2}f(0)\),\(\varlimsup_{n\rightarrow\infty}n\int_{\frac{1}{2}}^1{(1 - x^2 + x^3)^nf(x)\mathrm{d}x}\leqslant f(1)\).

另外一方面,我们有
\begin{align*}
\int_0^{\frac{1}{2}}{(1 - x^2 + x^3)^nf(x)\mathrm{d}x}&\geqslant \int_0^{\delta}{(1 - x^2 + x^3)^nf(x)\mathrm{d}x}=\int_0^{\delta}{e^{n\ln(1 - x^2 + x^3)}f(x)\mathrm{d}x}\\
&\geqslant (f(0) - \varepsilon)\int_0^{\delta}{e^{n(-x^2 - \varepsilon x^2)}\mathrm{d}x}=\frac{f(0) - \varepsilon}{\sqrt{n(1 + \varepsilon)}}\int_0^{\delta\sqrt{n(1 + \varepsilon)}}{e^{-y^2}\mathrm{d}y},
\end{align*}
\begin{align*}
\int_{\frac{1}{2}}^1{(1 - x^2 + x^3)^nf(x)\mathrm{d}x}&\geqslant \int_{1 - \delta}^1{(1 - x^2 + x^3)^nf(x)\mathrm{d}x}=\int_{1 - \delta}^1{e^{n\ln(1 - x^2 + x^3)}f(x)\mathrm{d}x}\\
&\geqslant (f(1) - \varepsilon)\int_{1 - \delta}^1{e^{n[x - 1 - \varepsilon(x - 1)]}\mathrm{d}x}=\frac{f(1) - \varepsilon}{n(1 + \varepsilon)}\left(1 - e^{-n\delta(1 - \varepsilon)}\right).
\end{align*}
于是就有
\begin{align*}
\sqrt{n}\int_0^{\frac{1}{2}}{(1 - x^2 + x^3)^nf(x)\mathrm{d}x}&\geqslant \frac{f(0) - \varepsilon}{\sqrt{1 + \varepsilon}}\int_0^{\delta\sqrt{n(1 + \varepsilon)}}{e^{-y^2}\mathrm{d}y},\\
n\int_{\frac{1}{2}}^1{(1 - x^2 + x^3)^nf(x)\mathrm{d}x}&\geqslant \frac{f(1) - \varepsilon}{1 + \varepsilon}\left(1 - e^{-n\delta(1 - \varepsilon)}\right).
\end{align*}
上式两边同时令\(n\rightarrow\infty\)并取下极限得到
\begin{align*}
\varliminf_{n\rightarrow\infty}\sqrt{n}\int_0^{\frac{1}{2}}{(1 - x^2 + x^3)^nf(x)\mathrm{d}x}&\geqslant \frac{f(0) - \varepsilon}{\sqrt{1 + \varepsilon}}\int_0^{\infty}{e^{-y^2}\mathrm{d}y}=\frac{\sqrt{\pi}}{2\sqrt{1 + \varepsilon}}(f(0) - \varepsilon),\\
\varliminf_{n\rightarrow\infty}n\int_{\frac{1}{2}}^1{(1 - x^2 + x^3)^nf(x)\mathrm{d}x}&\geqslant \frac{f(1) - \varepsilon}{1 + \varepsilon}.
\end{align*}
再由\(\varepsilon\)的任意性可得\(\varliminf_{n\rightarrow\infty}\sqrt{n}\int_0^{\frac{1}{2}}{(1 - x^2 + x^3)^nf(x)\mathrm{d}x}\geqslant \frac{\sqrt{\pi}}{2}f(0)\),\(\varliminf_{n\rightarrow\infty}n\int_{\frac{1}{2}}^1{(1 - x^2 + x^3)^nf(x)\mathrm{d}x}\geqslant f(1)\).

因此,我们就有
\begin{align*}
\frac{\sqrt{\pi}}{2}f(0)&\leqslant \varliminf_{n\rightarrow\infty}\sqrt{n}\int_0^{\frac{1}{2}}{(1 - x^2 + x^3)^nf(x)\mathrm{d}x}\leqslant \varlimsup_{n\rightarrow\infty}\sqrt{n}\int_0^{\frac{1}{2}}{(1 - x^2 + x^3)^nf(x)\mathrm{d}x}\leqslant \frac{\sqrt{\pi}}{2}f(0),\\
f(1)&\leqslant \varliminf_{n\rightarrow\infty}n\int_{\frac{1}{2}}^1{(1 - x^2 + x^3)^nf(x)\mathrm{d}x}\leqslant \varlimsup_{n\rightarrow\infty}n\int_{\frac{1}{2}}^1{(1 - x^2 + x^3)^nf(x)\mathrm{d}x}\leqslant f(1).
\end{align*}
故\(\lim_{n\rightarrow\infty}\sqrt{n}\int_0^{\frac{1}{2}}{(1 - x^2 + x^3)^nf(x)\mathrm{d}x}=\frac{\sqrt{\pi}}{2}f(0)\),\(\lim_{n\rightarrow\infty}n\int_{\frac{1}{2}}^1{(1 - x^2 + x^3)^nf(x)\mathrm{d}x}=f(1)\).从而
\begin{align*}
\int_0^{\frac{1}{2}}{(1 - x^2 + x^3)^nf(x)\mathrm{d}x}&=\frac{f(0)\sqrt{\pi}}{2\sqrt{n}}+o\left(\frac{1}{\sqrt{n}}\right),  n\rightarrow\infty;\\
\int_{\frac{1}{2}}^1{(1 - x^2 + x^3)^nf(x)\mathrm{d}x}&=\frac{f(1)}{n}+o\left(\frac{1}{n}\right),  n\rightarrow\infty.
\end{align*}
故\(\int_0^1{(1 - x^2 + x^3)^nf(x)\mathrm{d}x}=\int_0^{\frac{1}{2}}{(1 - x^2 + x^3)^nf(x)\mathrm{d}x}+\int_{\frac{1}{2}}^1{(1 - x^2 + x^3)^nf(x)\mathrm{d}x}=\frac{f(0)\sqrt{\pi}}{2\sqrt{n}}+\frac{f(1)}{n}+o\left(\frac{1}{n}\right),  n\rightarrow\infty\).
从而当\(f\equiv 1\)时,上式等价于\(\int_0^1{(1 - x^2 + x^3)^n\mathrm{d}x}=\frac{\sqrt{\pi}}{2\sqrt{n}}+\frac{1}{n}+o\left(\frac{1}{n}\right),  n\rightarrow\infty\);当\(f(x) = \ln(x + 2)\)时,上式等价于\(\int_0^1{(1 - x^2 + x^3)^n\ln(x + 2)\mathrm{d}x}=\frac{\sqrt{\pi}\ln 2}{2\sqrt{n}}+\frac{\ln 3}{n}+o\left(\frac{1}{n}\right),  n\rightarrow\infty\).于是
\[
\lim_{n\rightarrow\infty}\frac{\int_0^1{(1 - x^2 + x^3)^n\ln(x + 2)\mathrm{d}x}}{\int_0^1{(1 - x^2 + x^3)^n\mathrm{d}x}}=\lim_{n\rightarrow\infty}\frac{\frac{\sqrt{\pi}\ln 2}{2\sqrt{n}}+\frac{\ln 3}{n}+o\left(\frac{1}{n}\right)}{\frac{\sqrt{\pi}}{2\sqrt{n}}+\frac{1}{n}+o\left(\frac{1}{n}\right)}=\ln 2.
\]

\end{proof}

\begin{example}\label{example4544166848}
设\(f\in R[0,1]\)且\(f\)在\(x = 1\)连续,证明
\[
\lim_{n \to \infty} n\int_{0}^{1} f(x)x^n \mathrm{d}x = f(1).
\]
\end{example}
\begin{note}
这种运用Laplace方法估阶的题目,也可以用拟合法进行证明.
\end{note}
\begin{proof}
由于\(f\in R[0,1]\),因此存在\(M > 0\),使得\(\vert f(x)\vert\leqslant M\),\(\forall x\in [0,1]\).
于是对\(\forall n\in\mathbb{N}\),\(\forall\delta\in(0,1)\),有
\begin{align*}
&\left\vert n\int_0^1 f(x)x^n \mathrm{d}x - n\int_0^1 f(1)x^n \mathrm{d}x\right\vert=\left\vert n\int_0^1 [f(x) - f(1)]x^n \mathrm{d}x\right\vert\\
&\leqslant n\int_0^1 \vert [f(x) - f(1)]x^n\vert \mathrm{d}x
=n\int_0^{\delta} \vert f(x) - f(1)\vert x^n \mathrm{d}x + n\int_{\delta}^1 \vert f(x) - f(1)\vert x^n \mathrm{d}x\\
&\leqslant n\int_0^{\delta} \vert M + f(1)\vert\delta^n \mathrm{d}x + n\sup_{x\in[\delta,1]}\vert f(x) - f(1)\vert\int_{\delta}^1 x^n \mathrm{d}x\\
&\leqslant n\vert M + f(1)\vert\delta^{n + 1} + n\sup_{x\in[\delta,1]}\vert f(x) - f(1)\vert\int_0^1 x^n \mathrm{d}x\\
&=n\vert M + f(1)\vert\delta^{n + 1} + \frac{n}{n + 1}\sup_{x\in[\delta,1]}\vert f(x) - f(1)\vert.
\end{align*}
上式两边同时令\(n\rightarrow\infty\),并取上极限可得
\[
\varlimsup_{n\rightarrow\infty}\left\vert n\int_0^1 f(x)x^n \mathrm{d}x - n\int_0^1 f(1)x^n \mathrm{d}x\right\vert\leqslant\sup_{x\in[\delta,1]}\vert f(x) - f(1)\vert, \quad\forall\delta\in(0,1).
\]
再根据\(\delta\)的任意性,令\(\delta\rightarrow 1^-\)可得
\[
\varlimsup_{n\rightarrow\infty}\left\vert n\int_0^1 f(x)x^n \mathrm{d}x - n\int_0^1 f(1)x^n \mathrm{d}x\right\vert\leqslant\lim_{\delta\rightarrow 1^-}\sup_{x\in[\delta,1]}\vert f(x) - f(1)\vert=\varlimsup_{\delta\rightarrow 1^-}\vert f(x) - f(1)\vert.
\]
又因为\(f\)在\(x = 1\)处连续,所以\(\varlimsup_{\delta\rightarrow 1^-}\vert f(x) - f(1)\vert = 0\).故
\[
0\leqslant\varliminf_{n\rightarrow\infty}\left\vert n\int_0^1 f(x)x^n \mathrm{d}x - n\int_0^1 f(1)x^n \mathrm{d}x\right\vert\leqslant\varlimsup_{n\rightarrow\infty}\left\vert n\int_0^1 f(x)x^n \mathrm{d}x - n\int_0^1 f(1)x^n \mathrm{d}x\right\vert\leqslant 0.
\]
因此$\lim_{n\rightarrow\infty}n\int_0^1 f(x)x^n \mathrm{d}x=\lim_{n\rightarrow\infty}n\int_0^1 f(1)x^n \mathrm{d}x = f(1)\lim_{n\rightarrow\infty}\frac{n}{n + 1}=f(1)$.

\end{proof}

\begin{example}
$f$是$[0,1]$上Riemann可积的函数,且在$x=1$处存在导数,$f(1)=0$,$f'(1)=-1$,证明$$\lim\limits_{n \to \infty} n^2 \int_0^1 x^n f(x) \mathrm{d}x = 1.$$
\end{example}
\begin{note}
本题也可以类似\refexa{example4544166848}用拟合法进行证明.
\end{note}
\begin{proof}
由Taylor定理可知,存在$\delta \in (0,1)$,对$\forall x\in [\delta,1]$,存在$\theta \in (x,1)$,使得
\begin{align*}
f(x) = f'(1)(x-1) + \frac{f''(\theta)}{2}(x-1)^2 = 1 - x + \frac{f''(\theta)}{2}(x-1)^2.
\end{align*}
记$M \triangleq \sup_{[0,1]}f$,$m \triangleq \inf_{[0,1]}f$,则一方面,我们有
\begin{align*}
n^2\int_0^1 x^n f(x) \mathrm{d}x &= n^2\int_0^\delta x^n f(x) \mathrm{d}x + n^2\int_\delta^1 x^n f(x) \mathrm{d}x \leqslant Mn^2\delta^n + n^2\int_\delta^1 x^n \left[1 - x + \frac{f''(\theta)}{2}(x - 1)^2\right] \mathrm{d}x \\
&\leqslant Mn^2\delta^n + n^2\int_0^1 \left[x^n - x^{n+1} + \frac{f''(\theta)}{2}(x^{n+2} - 2x^{n+1} + x^n)\right] \mathrm{d}x \\
&= Mn^2\delta^n + \frac{n^2}{(n+1)(n+2)} + \frac{n^2 f''(\theta)}{(n+1)(n+2)(n+3)}.
\end{align*}
令$n \to \infty$得
\begin{align*}
\varlimsup_{n \to \infty} n^2\int_0^1 x^n f(x) \mathrm{d}x \leqslant 1.
\end{align*}
另一方面,我们有
\begin{align*}
n^2\int_0^1 x^n f(x) \mathrm{d}x &= n^2\int_0^\delta x^n f(x) \mathrm{d}x + n^2\int_\delta^1 x^n f(x) \mathrm{d}x \geqslant mn^2\int_0^{\delta}{x^n\mathrm{d}x} + n^2\int_\delta^1 x^n \left[1 - x + \frac{f''(\theta)}{2}(x - 1)^2\right] \mathrm{d}x \\
&= \frac{mn^2}{n+1}\delta ^{n+1} + \frac{n^2}{(n+1)(n+2)} - n^2\left(\frac{\delta^{n+1}}{n+1} - \frac{\delta^{n+2}}{n+2}\right) + \frac{n^2 f''(\theta)}{(n+1)(n+2)(n+3)} - n^2\left(\frac{\delta^{n+3}}{n+3} - \frac{2\delta^{n+2}}{n+2} + \frac{\delta^{n+1}}{n+1}\right).
\end{align*}
令$n \to \infty$得
\begin{align*}
\varliminf_{n \to \infty} n^2\int_0^1 x^n f(x) \mathrm{d}x \geqslant 1.
\end{align*}
故
\begin{align*}
\lim_{n \to \infty} n^2\int_0^1 x^n f(x) \mathrm{d}x = 1.
\end{align*}

\end{proof}

\begin{example}[$\,\,$Possion核]\label{example:Possion核}
设\(f\in R[0,1]\)且\(f\)在\(x = 0\)连续,证明
\[
\lim_{t\rightarrow 0^+} \int_0^1{\frac{t}{x^2+t^2}f(x)\mathrm{d}x}=\frac{\pi}{2}f(0).
\]
\end{example}
\begin{proof}
因为\(f\in R[0,1]\),所以存在\(M > 0\),使得\(\vert f(x)\vert\leqslant M\),\(\forall x\in [0,1]\).于是对\(\forall\delta\in(0,1)\),固定\(\delta\),再对\(\forall t > 0\),我们有
\begin{align*}
&\left|\int_0^1\frac{t}{x^2 + t^2}f(x)\mathrm{d}x - \int_0^1\frac{t}{x^2 + t^2}f(0)\mathrm{d}x\right| \leqslant \int_0^1\frac{t}{x^2 + t^2}\vert f(x) - f(0)\vert \mathrm{d}x\\
&= \int_0^{\delta}\frac{t}{x^2 + t^2}\vert f(x) - f(0)\vert \mathrm{d}x + \int_{\delta}^1\frac{t}{x^2 + t^2}\vert f(x) - f(0)\vert \mathrm{d}x\\
&\leqslant \sup_{x\in[0,\delta]}\vert f(x) - f(0)\vert\int_0^{\delta}\frac{t}{x^2 + t^2}\mathrm{d}x + \int_0^1\frac{t}{\delta^2 + t^2}\vert M + f(0)\vert \mathrm{d}x\\
&= \sup_{x\in[0,\delta]}\vert f(x) - f(0)\vert\left.\arctan\frac{x}{t}\right|_{0}^{\delta} + \frac{t}{\delta^2 + t^2}\vert M + f(0)\vert\\
&= \sup_{x\in[0,\delta]}\vert f(x) - f(0)\vert\cdot\arctan\frac{\delta}{t} + \frac{t}{\delta^2 + t^2}\vert M + f(0)\vert.
\end{align*}
上式两边同时令\(t\rightarrow 0^+\)并取上极限,可得
\[
{\varlimsup_{t\rightarrow 0^+}}\left|\int_0^1\frac{t}{x^2 + t^2}f(x)\mathrm{d}x - \int_0^1\frac{t}{x^2 + t^2}f(0)\mathrm{d}x\right| \leqslant \frac{\pi}{2}\sup_{x\in[0,\delta]}\vert f(x) - f(0)\vert, \forall\delta\in(0,1).
\]
再根据\(\delta\)的任意性,令\(\delta\rightarrow 0^+\)可得
\[
{\varlimsup_{t\rightarrow 0^+}}\left|\int_0^1\frac{t}{x^2 + t^2}f(x)\mathrm{d}x - \int_0^1\frac{t}{x^2 + t^2}f(0)\mathrm{d}x\right| \leqslant \frac{\pi}{2}\lim_{\delta\rightarrow 0^+}\sup_{x\in[0,\delta]}\vert f(x) - f(0)\vert = \frac{\pi}{2}{\varlimsup_{x\rightarrow 0^+}}\vert f(x) - f(0)\vert.
\]
又由于\(f\)在\(x = 0\)处连续,从而\({\varlimsup_{x\rightarrow 0^+}}\vert f(x) - f(0)\vert = 0\).故
\[
0\leqslant \underset{t\rightarrow 0^+}{{\varliminf }}\left| \int_0^1{\frac{t}{x^2+t^2}f(x)\mathrm{d}x}-\int_0^1{\frac{t}{x^2+t^2}f(0)\mathrm{d}x} \right|\leqslant {\varlimsup_{t\rightarrow 0^+} }\left| \int_0^1{\frac{t}{x^2+t^2}f(x)\mathrm{d}x}-\int_0^1{\frac{t}{x^2+t^2}f(0)\mathrm{d}x} \right|\leqslant 0.
\]
因此\(\lim_{t\rightarrow 0^+}\int_0^1\frac{t}{x^2 + t^2}f(x)\mathrm{d}x = \lim_{t\rightarrow 0^+}\int_0^1\frac{t}{x^2 + t^2}f(0)\mathrm{d}x = f(0)\lim_{t\rightarrow 0^+}\arctan\frac{1}{t} = \frac{\pi}{2}f(0)\).

\end{proof}

\begin{example}[$\,\,$Fejer核]\label{example:Fejer核}
设\(f\)在\(x = 0\)连续且在\([-\frac{1}{2},\frac{1}{2}]\)可积,则
\[
\lim_{N\rightarrow +\infty} \int_{-\frac{1}{2}}^{\frac{1}{2}}{\frac{1}{N}\frac{\sin ^2\left( \pi Nx \right)}{\sin ^2\left( \pi x \right)}f\left( x \right) \mathrm{d}x}=f\left( 0 \right).
\]
\end{example}
\begin{proof}
因为\(f\in R\left[-\frac{1}{2},\frac{1}{2}\right]\),所以存在\(M > 0\),使得\(\vert f(x)\vert\leqslant M\),\(\forall x\in\left[-\frac{1}{2},\frac{1}{2}\right]\).又因为\(\sin x\sim x\),\(x\rightarrow 0\),所以对\(\forall\varepsilon\in(0,1)\),存在\(\delta_0 > 0\),使得当\(\vert x\vert\leqslant\delta_0\)时,有\(\sin x\geqslant(1 - \varepsilon)x\).于是对\(\forall\delta\in (0,\min\left\{\frac{1}{2},\delta_0\right\})\),我们有
\begin{align*}
&\left|\int_{-\frac{1}{2}}^{\frac{1}{2}}\frac{1}{N}\frac{\sin^2(\pi Nx)}{\sin^2(\pi x)}[f(x) - f(0)]\mathrm{d}x\right|
\leqslant\int_{-\frac{1}{2}}^{\frac{1}{2}}\frac{1}{N}\frac{\sin^2(\pi Nx)}{\sin^2(\pi x)}\vert f(x) - f(0)\vert \mathrm{d}x\\
&=\int_{\vert x\vert\leqslant\delta}\frac{1}{N}\frac{\sin^2(\pi Nx)}{\sin^2(\pi x)}\vert f(x) - f(0)\vert \mathrm{d}x + \int_{\delta\leqslant\vert x\vert\leqslant\frac{1}{2}}\frac{1}{N}\frac{\sin^2(\pi Nx)}{\sin^2(\pi x)}\vert f(x) - f(0)\vert \mathrm{d}x\\
&\leqslant\sup_{\vert x\vert\leqslant\delta}\vert f(x) - f(0)\vert\int_{\vert x\vert\leqslant\delta}\frac{1}{N}\frac{\sin^2(\pi Nx)}{\sin^2(\pi x)}\mathrm{d}x + \int_{\delta\leqslant\vert x\vert\leqslant\frac{1}{2}}\frac{1}{N}\frac{1}{\sin^2(\pi\delta)}\vert M + f(0)\vert \mathrm{d}x\\
&\leqslant\frac{\sup_{\vert x\vert\leqslant\delta}\vert f(x) - f(0)\vert}{1 - \varepsilon}\int_{\vert x\vert\leqslant\delta}\frac{1}{N}\frac{\sin^2(\pi Nx)}{(\pi x)^2}\mathrm{d}x + \frac{1}{N}\int_{\delta\leqslant\vert x\vert\leqslant\frac{1}{2}}\frac{\vert M + f(0)\vert}{\sin^2(\pi\delta)}\mathrm{d}x\\
&\xlongequal{\text{令}y = Nx}\frac{\sup_{\vert x\vert\leqslant\delta}\vert f(x) - f(0)\vert}{1 - \varepsilon}\int_{\vert y\vert\leqslant N\delta}\frac{\sin^2(\pi y)}{(\pi y)^2}\mathrm{d}y + \frac{1}{N}\int_{\delta\leqslant\vert x\vert\leqslant\frac{1}{2}}\frac{\vert M + f(0)\vert}{\sin^2(\pi\delta)}\mathrm{d}x\\
&\leqslant \frac{\sup_{\vert x\vert\leqslant\delta}\vert f(x) - f(0)\vert}{1 - \varepsilon}\int_{-\infty}^{+\infty}\frac{\sin^2(\pi y)}{(\pi y)^2}\mathrm{d}y + \frac{1}{N}\int_{\delta\leqslant\vert x\vert\leqslant\frac{1}{2}}\frac{\vert M + f(0)\vert}{\sin^2(\pi\delta)}\mathrm{d}x.
\end{align*}
上式两边同时令\(N\rightarrow +\infty\)并取上极限,得到
\[
{\varlimsup_{N\rightarrow +\infty}}\left|\int_{-\frac{1}{2}}^{\frac{1}{2}}\frac{1}{N}\frac{\sin^2(\pi Nx)}{\sin^2(\pi x)}[f(x) - f(0)]\mathrm{d}x\right|\leqslant\frac{\sup_{\vert x\vert\leqslant\delta}\vert f(x) - f(0)\vert}{1 - \varepsilon}\int_{-\infty}^{+\infty}\frac{\sin^2(\pi y)}{(\pi y)^2}\mathrm{d}y.
\]
由
\begin{align*}
\left| \int_{-\infty}^{+\infty}{\frac{\sin ^2(\pi y)}{(\pi y)^2}\mathrm{d}y} \right|\leqslant \int_{-\infty}^{+\infty}{\frac{1}{(\pi y)^2}\mathrm{d}y}<+\infty .
\end{align*}
可知\(\int_{-\infty}^{+\infty}\frac{\sin^2(\pi y)}{(\pi y)^2}\mathrm{d}y\)收敛.从而根据\(\delta\)的任意性,上式两边同时令\(\delta\rightarrow 0^+\),再结合\(f\)在\(x = 0\)处连续,可得
\begin{align*}
&{\varlimsup_{N\rightarrow +\infty}}\left|\int_{-\frac{1}{2}}^{\frac{1}{2}}\frac{1}{N}\frac{\sin^2(\pi Nx)}{\sin^2(\pi x)}[f(x) - f(0)]\mathrm{d}x\right|\\
&\leqslant\lim_{\delta\rightarrow 0^+}\frac{\sup_{\vert x\vert\leqslant\delta}\vert f(x) - f(0)\vert}{1 - \varepsilon}\int_{-\infty}^{+\infty}\frac{\sin^2(\pi y)}{(\pi y)^2}\mathrm{d}y\\
&=\frac{\int_{-\infty}^{+\infty}\frac{\sin^2(\pi y)}{(\pi y)^2}\mathrm{d}y}{1 - \varepsilon}\lim_{x\rightarrow 0^+}\vert f(x) - f(0)\vert = 0.
\end{align*}
从而
\[
0\leqslant{\varliminf_{N\rightarrow +\infty}}\left|\int_{-\frac{1}{2}}^{\frac{1}{2}}\frac{1}{N}\frac{\sin^2(\pi Nx)}{\sin^2(\pi x)}[f(x) - f(0)]\mathrm{d}x\right|\leqslant{\varlimsup_{N\rightarrow +\infty}}\left|\int_{-\frac{1}{2}}^{\frac{1}{2}}\frac{1}{N}\frac{\sin^2(\pi Nx)}{\sin^2(\pi x)}[f(x) - f(0)]\mathrm{d}x\right|\leqslant 0.
\]
故\(\lim_{N\rightarrow +\infty}\left|\int_{-\frac{1}{2}}^{\frac{1}{2}}\frac{1}{N}\frac{\sin^2(\pi Nx)}{\sin^2(\pi x)}[f(x) - f(0)]\mathrm{d}x\right| = 0\).而一方面,我们有
\begin{align*}
&\lim_{N\rightarrow +\infty}\int_{-\frac{1}{2}}^{\frac{1}{2}}\frac{1}{N}\frac{\sin^2(\pi Nx)}{\sin^2(\pi x)}f(0)\mathrm{d}x
\geqslant\lim_{N\rightarrow +\infty}\int_{-\frac{1}{2}}^{\frac{1}{2}}\frac{1}{N}\frac{\sin^2(\pi Nx)}{(\pi x)^2}f(0)\mathrm{d}x\\
&\xlongequal{\text{令}y = Nx}\lim_{N\rightarrow +\infty}\int_{-\frac{N}{2}}^{\frac{N}{2}}\frac{\sin^2(\pi y)}{(\pi y)^2}f(0)\mathrm{d}y
=\int_{-\infty}^{+\infty}\frac{\sin^2(\pi y)}{(\pi y)^2}f(0)\mathrm{d}y 
\\
&=\frac{2}{\pi}\int_0^{+\infty}{\frac{\sin ^2x}{x^2}f\left( 0 \right) \mathrm{d}x} \xlongequal{\text{\refpro{proposition:重要定积分结果(必记)}}} f(0).
\end{align*}
另一方面,对\(\forall\varepsilon\in(0,1)\)我们有
\begin{align*}
&\lim_{N\rightarrow +\infty}\int_{-\frac{1}{2}}^{\frac{1}{2}}\frac{1}{N}\frac{\sin^2(\pi Nx)}{\sin^2(\pi x)}f(0)\mathrm{d}x
=\lim_{N\rightarrow +\infty}\int_{\vert x\vert\leqslant\delta}\frac{1}{N}\frac{\sin^2(\pi Nx)}{\sin^2(\pi x)}f(0)\mathrm{d}x+\lim_{N\rightarrow +\infty}\int_{\delta\leqslant\vert x\vert\leqslant\frac{1}{2}}\frac{1}{N}\frac{\sin^2(\pi Nx)}{\sin^2(\pi x)}f(0)\mathrm{d}x\\
&\leqslant f(0)\lim_{N\rightarrow +\infty}\int_{\vert x\vert\leqslant\delta}\frac{1}{N}\frac{\sin^2(\pi Nx)}{\sin^2(\pi x)}\mathrm{d}x+\lim_{N\rightarrow +\infty}\int_{\delta\leqslant\vert x\vert\leqslant\frac{1}{2}}\frac{1}{N}\frac{1}{\sin^2(\pi\delta)}f(0)\mathrm{d}x
\leqslant\frac{f(0)}{1 - \varepsilon}\lim_{N\rightarrow +\infty}\int_{\vert x\vert\leqslant\delta}\frac{1}{N}\frac{\sin^2(\pi Nx)}{(\pi x)^2}\mathrm{d}x\\
&\xlongequal{\text{令}y = Nx}\frac{f(0)}{1 - \varepsilon}\lim_{N\rightarrow +\infty}\int_{\vert y\vert\leqslant N\delta}\frac{\sin^2(\pi y)}{(\pi y)^2}\mathrm{d}y
=\frac{f(0)}{1 - \varepsilon}\int_{-\infty}^{+\infty}\frac{\sin^2(\pi y)}{(\pi y)^2}\mathrm{d}y=\frac{2}{\pi}\int_0^{+\infty}{\frac{\sin ^2x}{x^2} \mathrm{d}x}\frac{f(0)}{1 - \varepsilon} \xlongequal{\text{\refpro{proposition:重要定积分结果(必记)}}} \frac{f(0)}{1 - \varepsilon}.
\end{align*}
再根据\(\varepsilon\)的任意性,可知
\[
\lim_{N\rightarrow +\infty}\int_{-\frac{1}{2}}^{\frac{1}{2}}\frac{1}{N}\frac{\sin^2(\pi Nx)}{\sin^2(\pi x)}f(0)\mathrm{d}x\leqslant f(0).
\]
因此,由夹逼准则,可知\(\lim_{N\rightarrow +\infty}\int_{-\frac{1}{2}}^{\frac{1}{2}}\frac{1}{N}\frac{\sin^2(\pi Nx)}{\sin^2(\pi x)}f(0)\mathrm{d}x = f(0).
\) 

\end{proof}

\begin{example}
设\(\varphi_n(x)=\frac{n}{\sqrt{\pi}}e^{-n^2x^2},n = 1,2,\cdots\),\(f\)是\(\mathbb{R}\)上的有界实值连续函数,证明:
\[
\lim_{n \to \infty} \int_{-\infty}^{\infty} f(y)\varphi_n(x - y) \mathrm{d}y = f(x).
\]
\end{example}
\begin{proof}
由条件可知,存在\(M > 0\),使得\(\vert f(x)\vert\leqslant M\),\(\forall x\in\mathbb{R}\).于是对\(\forall x\in\mathbb{R}\),固定\(x\),再对\(\forall\delta > 0\),我们有
\begin{align*}
&{\varlimsup_{n\rightarrow\infty}}\left|\int_{-\infty}^{\infty}\vert f(y) - f(x)\vert\frac{n}{\sqrt{\pi}}e^{-n^2(x - y)^2}\mathrm{d}y\right|
\leqslant{\varlimsup_{n\rightarrow\infty}}\int_{-\infty}^{\infty}\vert f(y) - f(x)\vert\frac{n}{\sqrt{\pi}}e^{-n^2(x - y)^2}\mathrm{d}y\\
&\leqslant{\varlimsup_{n\rightarrow\infty}}\int_{\vert x - y\vert\leqslant\delta}\vert f(y) - f(x)\vert\frac{n}{\sqrt{\pi}}e^{-n^2(x - y)^2}\mathrm{d}y + {\varlimsup_{n\rightarrow\infty}}\int_{\vert x - y\vert\geqslant\delta}\vert f(y) - f(x)\vert\frac{n}{\sqrt{\pi}}e^{-n^2(x - y)^2}\mathrm{d}y\\
&\leqslant\sup_{\vert x - y\vert\leqslant\delta}\vert f(y) - f(x)\vert{\varlimsup_{n\rightarrow\infty}}\int_{\vert x - y\vert\leqslant\delta}\frac{n}{\sqrt{\pi}}e^{-n^2(x - y)^2}\mathrm{d}y + {\varlimsup_{n\rightarrow\infty}}\int_{\vert x - y\vert\geqslant\delta}2M\frac{n}{\sqrt{\pi}}e^{-n^2\delta^2}\mathrm{d}y\\
&\xlongequal{\text{令}z = n(x - y)}\sup_{\vert x - y\vert\leqslant\delta}\vert f(y) - f(x)\vert{\varlimsup_{n\rightarrow\infty}}\int_{\vert z\vert\leqslant n\delta}\frac{1}{\sqrt{\pi}}e^{-z^2}\mathrm{d}z\\
&=\sup_{\vert x - y\vert\leqslant\delta}\vert f(y) - f(x)\vert\int_{-\infty}^{+\infty}\frac{1}{\sqrt{\pi}}e^{-z^2}\mathrm{d}z
=\sup_{\vert x - y\vert\leqslant\delta}\vert f(y) - f(x)\vert.
\end{align*}
令\(\delta\rightarrow 0^+\),再结合\(f\)在\(\forall x\in\mathbb{R}\)上连续,可得
\[
{\varlimsup_{n\rightarrow\infty}}\left|\int_{-\infty}^{\infty}\vert f(y) - f(x)\vert\frac{n}{\sqrt{\pi}}e^{-n^2(x - y)^2}\mathrm{d}y\right|\leqslant\lim_{\delta\rightarrow 0^+}\sup_{\vert x - y\vert\leqslant\delta}\vert f(y) - f(x)\vert=\lim_{y\rightarrow x}\vert f(y) - f(x)\vert = 0.
\]
故
\begin{align*}
&\lim_{n\rightarrow\infty}\int_{-\infty}^{\infty}f(y)\frac{n}{\sqrt{\pi}}e^{-n^2(x - y)^2}\mathrm{d}y
=\lim_{n\rightarrow\infty}\int_{-\infty}^{\infty}f(x)\frac{n}{\sqrt{\pi}}e^{-n^2(x - y)^2}\mathrm{d}y\\
&=f(x)\lim_{n\rightarrow\infty}\int_{-\infty}^{\infty}\frac{n}{\sqrt{\pi}}e^{-n^2(x - y)^2}\mathrm{d}y
\xlongequal{\text{令}z = n(x - y)}f(x)\lim_{n\rightarrow\infty}\int_{\vert z\vert\leqslant n\delta}\frac{1}{\sqrt{\pi}}e^{-z^2}\mathrm{d}z\\
&=f(x)\int_{-\infty}^{+\infty}\frac{1}{\sqrt{\pi}}e^{-z^2}\mathrm{d}z
=f(x).
\end{align*}

\end{proof}

\begin{example}
设\(f(x)\in C[0,1],f'(0)\)存在,证明:对任意正整数\(m\),在\(n\rightarrow\infty\)时有
\[
\int_{0}^{1}f(x^{n})\mathrm{d}x = f(0)+\sum_{k = 0}^{m - 1}\frac{1}{n^{k + 1}}\int_{0}^{1}\frac{f(x)-f(0)}{x}\frac{\ln^{k}x}{k!}\mathrm{d}x + O\left(\frac{1}{n^{m+1}}\right).
\]
\end{example}
\begin{remark}
这里积分换元之后,再Taylor展开,但是后续的积分与求和的换序以及余项的估计并不好处理.
\end{remark}
\begin{note}
估计抽象函数的渐近展开一般考虑拟合和分段.如果考虑积分与求和换序的话并不好处理,一般只有估计具体函数的渐近才会考虑换序.

这里分段的想法也是将原积分分成主体部分和余项部分.容易观察(直观地分析一下即可)到这里积分的阶的主体部分集中在0附近.
\end{note}
\begin{proof}
记 \(g(x)=\frac{f(x)-f(0)}{x}\),则由条件可知,\(g\in C[0,1]\),从而
\begin{align}
|g(x)|\leqslant C,\forall x\in[0,1]. \label{example110-1.1}
\end{align}
于是
\begin{align*}
\int_0^1{f\left( x^n \right) \mathrm{d}x}-f\left( 0 \right) &=\int_0^1{\left[ f\left( x^n \right) -f\left( 0 \right) \right] \mathrm{d}x}\xlongequal{\text{令}y=x^n}\int_0^1{\frac{x^{\frac{1}{n}-1}}{n}\left[ f\left( x \right) -f\left( 0 \right) \right] \mathrm{d}x}
\\
&=\frac{1}{n}\int_0^1{e^{\frac{\ln x}{n}}\frac{f\left( x \right) -f\left( 0 \right)}{x}\mathrm{d}x}=\frac{1}{n}\int_0^1{e^{\frac{\ln x}{n}}g\left( x \right) \mathrm{d}x}.
\end{align*}
因此原问题等价于证明对 \(\forall m\in\mathbb{N}\),当 \(n\rightarrow\infty\) 时,都有
\[
\frac{1}{n}\int_{0}^{1}e^{\frac{\ln x}{n}}g(x)\mathrm{d}x=\sum_{k = 0}^{m - 1}\frac{1}{n^{k + 1}}\int_{0}^{1}\frac{\ln^k x}{k!}g(x)\mathrm{d}x+O\left(\frac{1}{n^{m + 1}}\right).
\]
由 Taylor 公式可知,\(\forall x\in[\delta,1]\),对 \(\forall m\in\mathbb{N}\),都有
\[
e^{\frac{\ln x}{n}}=\sum_{k = 0}^{m - 1}\frac{\ln^k x}{k!n^k}+O\left(\frac{1}{n^m}\right),n\rightarrow\infty.
\]
即存在 \(M>0\),使得 \(\forall x\in[\delta,1]\),对 \(\forall m\in\mathbb{N}\),存在 \(N>0\),使得 \(\forall n > N\),都有
\begin{align}
\left|e^{\frac{\ln x}{n}}-\sum_{k = 0}^{m - 1}\frac{\ln^k x}{k!n^k}\right|\leqslant\frac{M}{n^m}. \label{example110-1.2}
\end{align}
取 \(\delta=\frac{1}{n^{2m}}\in(0,1)\),则对 \(\forall m\in\mathbb{N}\),当 \(n > N\) 时,结合 \eqref{example110-1.1}\eqref{example110-1.2} 式,我们有
\begin{align}
&\left| \frac{1}{n}\int_0^1{e^{\frac{\ln x}{n}}g\left( x \right) \mathrm{d}x}-\sum_{k=0}^{m-1}{\frac{1}{n^{k+1}}\int_0^1{\frac{\ln ^kx}{k!}g\left( x \right) \mathrm{d}x}} \right|
=\left| \frac{1}{n}\int_0^1{e^{\frac{\ln x}{n}}g\left( x \right) \mathrm{d}x}-\frac{1}{n}\sum_{k=0}^{m-1}{\int_0^1{\frac{\ln ^kx}{k!n^k}g\left( x \right) \mathrm{d}x}} \right|
\\
&=\left| \frac{1}{n}\int_0^1{e^{\frac{\ln x}{n}}g\left( x \right) \mathrm{d}x}-\frac{1}{n}\int_0^1{\sum_{k=0}^{m-1}{\frac{\ln ^kx}{k!n^k}}g\left( x \right) \mathrm{d}x} \right|
\nonumber
=\left| \frac{1}{n}\int_0^1{\left( e^{\frac{\ln x}{n}}-\sum_{k=0}^{m-1}{\frac{\ln ^kx}{k!n^k}} \right) g\left( x \right) \mathrm{d}x} \right|
\\
&\leqslant \frac{1}{n}\int_0^{\delta}{\left| e^{\frac{\ln x}{n}}-\sum_{k=0}^{m-1}{\frac{\ln ^kx}{k!n^k}} \right|g\left( x \right) \mathrm{d}x}+\frac{1}{n}\int_{\delta}^1{\left| e^{\frac{\ln x}{n}}-\sum_{k=0}^{m-1}{\frac{\ln ^kx}{k!n^k}} \right|g\left( x \right) \mathrm{d}x}
\nonumber
\\
&\leqslant \frac{C}{n}\int_0^{\delta}{\left( x^{\frac{1}{n}}+\sum_{k=0}^{m-1}{\frac{\left| \ln x \right|^k}{k!n^k}} \right) \mathrm{d}x}+\frac{C}{n}\int_{\delta}^1{\left| e^{\frac{\ln x}{n}}-\sum_{k=0}^{m-1}{\frac{\ln ^kx}{k!n^k}} \right|\mathrm{d}x}\leqslant \frac{C}{n}\int_0^{\delta}{\left( 1+\sum_{k=0}^{m-1}{\left| \ln x \right|^k} \right) \mathrm{d}x}+\frac{C}{n}\int_0^1{\frac{M}{n^m}\mathrm{d}x}
\nonumber
\\
&\leqslant \frac{C}{n}\int_0^{\delta}{\left( 1+m\left| \ln x \right|^{m-1} \right) \mathrm{d}x}+\frac{MC}{n^{m+1}}=\frac{C}{n}\int_0^{\frac{1}{n^{2m}}}{\left( 1-m\ln ^{m-1}x \right) \mathrm{d}x}+\frac{MC}{n^{m+1}}
\nonumber
\\
&=\frac{C}{n^{2m+1}}-\frac{mC}{n}\int_0^{\frac{1}{n^{2m}}}{\ln ^{m-1}x\mathrm{d}x}+\frac{MC}{n^{m+1}}\leqslant \frac{MC+C}{n^{m+1}}+\frac{mC}{n}\left| \int_0^{\frac{1}{n^{2m}}}{\ln ^{m-1}x\mathrm{d}x} \right|.\label{example110-2.0}
\end{align}
注意到
\[
\int\ln^n x\mathrm{d}x=x\left(a_0 + a_1\ln x+\cdots + a_n\ln^n x\right)+c=x\left(a_0+\sum_{k = 1}^{n}a_k\ln k\right)+c,
\]
其中 \(a_0,a_1,\cdots,a_n,c\) 都是常数。又因为对 \(\forall n\in\mathbb{N}\),都成立 \(\lim_{x\rightarrow +\infty}\frac{\ln^n x}{x}=0\),所以一定存在 \(N'>0\),使得当 \(n > N'\) 时,我们有
\begin{align}
\left| \int_0^{\frac{1}{n^{2m}}}{\ln ^{m-1}x\mathrm{d}x} \right|&=\left| x\left( b_0+b_1\ln x+\cdots +b_{m-1}\ln ^{m-1}x \right) \mid_{0}^{\frac{1}{n^{2m}}} \right|=\left| \frac{1}{n^{2m}}\left( b_0+b_1\ln \frac{1}{n^{2m}}+\cdots +b_{m-1}\ln ^{m-1}\frac{1}{n^{2m}} \right) \right|
\nonumber
\\
&\leqslant \frac{mB}{n^{2m}}\left| \ln ^{m-1}\frac{1}{n^{2m}} \right|=\frac{2m^2B\ln ^{m-1}n}{n^{2m}}\leqslant \frac{2m^2B}{n^{2m-1}}\leqslant \frac{2m^2B}{n^m},\label{example110-2.1}
\end{align}
其中 \(b_0,b_1,\cdots,b_{m - 1}\) 都是常数,\(B = \max\{b_0,b_1,\cdots,b_{m - 1}\}\)。因此由 \eqref{example110-2.0}\eqref{example110-2.1} 式可得,对 \(\forall m\in\mathbb{N}\),当 \(n>\max\{N,N'\}\) 时,我们有
\begin{align*}
&\left|\frac{1}{n}\int_{0}^{1}e^{\frac{\ln x}{n}}g(x)\mathrm{d}x-\sum_{k = 0}^{m - 1}\frac{1}{n^{k + 1}}\int_{0}^{1}\frac{\ln^k x}{k!}g(x)\mathrm{d}x\right|
\leqslant\frac{MC + C}{n^{m + 1}}+\frac{mC}{n}\left|\int_{0}^{\frac{1}{n^{2m}}}\ln^{m - 1}x\mathrm{d}x\right|
\\
&\leqslant\frac{MC + C}{n^{m + 1}}+\frac{mC}{n}\cdot\frac{2m^2B}{n^m}
=\frac{MC + C-2m^3BC}{n^{m + 1}}.
\end{align*}
即 \(\frac{1}{n}\int_{0}^{1}e^{\frac{\ln x}{n}}g(x)\mathrm{d}x-\sum_{k = 0}^{m - 1}\frac{1}{n^{k + 1}}\int_{0}^{1}\frac{\ln^k x}{k!}g(x)\mathrm{d}x=O\left(\frac{1}{n^{m + 1}}\right),n\rightarrow\infty\)。结论得证。

\end{proof}


\end{document}
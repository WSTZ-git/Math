\documentclass[../../main.tex]{subfiles}
\graphicspath{{\subfix{../../image/}}} % 指定图片目录,后续可以直接使用图片文件名。

% 例如:
% \begin{figure}[H]
% \centering
% \includegraphics[scale=0.4]{图.png}
% \caption{}
% \label{figure:图}
% \end{figure}
% 注意:上述\label{}一定要放在\caption{}之后,否则引用图片序号会只会显示??.

\begin{document}

\section{微分学计算}

\subsection{单变量微分学计算}

\begin{example}
\begin{enumerate}[(1)]
\item 设 \( f(x) = \prod_{k=0}^{n} (x - k) \)。对整数 \( 0 \leqslant j \leqslant n \),求导数 \( f'(j) \)。

\item 设 \( g(x) = \prod_{k=0}^{n} (e^x - k) \),求 \( g'(\ln j) \),\( j = 0, 1, 2, \cdots, n \)。
\end{enumerate}
\end{example}
\begin{solution}
\begin{enumerate}[(1)]
\item {\color{blue} 解法一:}注意到 \( f'(x) = \sum_{i=0}^{n} \prod_{\substack{k=0 \\ k \neq i}}^{n} (x - k) \),故
\begin{align*}
f'(j) &= \sum_{i=0}^{n} \prod_{\substack{k=0 \\ k \neq i}}^{n} (j - k) = \prod_{\substack{k=0 \\ k \neq j}}^{n} (j - k) + \sum_{\substack{i=0 \\ i \neq j}}^{n} \prod_{\substack{k=0 \\ k \neq i}}^{n} (j - k) \\
&= (-1)^{n - j} j! (n - j)! + \sum_{\substack{i=0 \\ i \neq j}}^{n} (j - j) \prod_{\substack{k=0 \\ k \neq i, j}}^{n} (j - k) \\
&= (-1)^{n - j} j! (n - j)!
\end{align*}

{\color{blue} 解法二:}\begin{align*}
f'(j) &= \lim_{x \to j} \frac{f(x) - f(j)}{x - j} = \lim_{x \to j} \frac{\prod\limits_{k=0}^{n} (x - k) - \prod\limits_{k=0}^{n} (j - k)}{x - j} \\
&= \prod_{\substack{k=0 \\ k \neq j}}^{n} (j - k) + \lim_{x \to j} \frac{(j - j) \prod\limits_{\substack{k=0 \\ k \neq j}}^{n} (j - k)}{x - j} \\
&= \prod_{\substack{k=0 \\ k \neq j}}^{n} (j - k) = (-1)^{n - j} j! (n - j)!
\end{align*}

\item 记 \( f(x) = \prod_{i=0}^{n} (x - k) \),则 \( g(x) = f(e^x) \)。从而 \( g'(x) = e^x f'(e^x) \),于是由 (1) 可知
\begin{align*}
g'(\ln j) = j f'(j) = j \cdot (-1)^{n - j} j! (n - j)!
\end{align*}
\end{enumerate}
\end{solution}

\begin{example}
对 \( n \in \mathbb{N} \),
\begin{enumerate}[(1)]
\item 设 \( f(x) = \sin(ax) \),\( a \in \mathbb{R} \),求 \( f^{(n)} \)。

\item 设 \( f(x) = e^x \cos x \),求 \( f^{(n)} \)。

\item 设 \( f(x) = \frac{\ln x}{x} \),求 \( f^{(n)} \)。

\item 设 \( f(x) = \frac{1}{1 - x^2} \),求 \( f^{(n)} \)。

\item 设 \( f(x) = \arctan x \),\( x > 0 \),求 \( f^{(n)} \)。
\end{enumerate}
\end{example}
\begin{solution}
\begin{enumerate}[(1)]
\item 我们断言
\begin{align}\label{100.2412424234----2-4-:::::--24..116}
f^{(n)}(x) = a^n \sin\left(ax + \frac{n}{2}\pi\right), \quad \forall n \in \mathbb{N}. 
\end{align}
当 \( n = 0 \) 时,上式显然成立。假设当 \( n = k \) 时上式成立,则
\[
f^{(k+1)}(x) = a^{k+1} \cos\left(ax + \frac{k}{2}\pi\right) = a^{k+1} \sin\left(ax + \frac{k+1}{2}\pi\right).
\]
故由数学归纳法可知 \eqref{100.2412424234----2-4-:::::--24..116} 式成立。

\item 由Euler公式可知,\(\cos x = \mathrm{Re}(e^{\mathrm{i}x})\),从而\(f(x) = \mathrm{Re}[e^{(1+\mathrm{i})x}]\)。于是
\[
f^{(n)}(x) = \mathrm{Re}[(1+\mathrm{i})^n e^{(1+\mathrm{i})x}], \quad \forall n \in \mathbb{N}.
\]
注意到
\[
1+\mathrm{i} = \sqrt{2}\left(\frac{\sqrt{2}}{2} + \frac{\sqrt{2}}{2}\mathrm{i}\right) = \sqrt{2}e^{\frac{\pi}{4}\mathrm{i}},
\]
进而\((1+\mathrm{i})^n = 2^{\frac{n}{2}}e^{\frac{n\pi}{4}\mathrm{i}}\)。故
\[
f^{(n)}(x) = \mathrm{Re}\left[2^{\frac{n}{2}}e^{\frac{n\pi}{4}\mathrm{i} + (1+\mathrm{i})x}\right] = 2^{\frac{n}{2}}e^x \mathrm{Re}\left[e^{\left(x + \frac{n\pi}{4}\right)\mathrm{i}}\right] = 2^{\frac{n}{2}}e^x \cos\left(x + \frac{n\pi}{4}\right).
\]

\item 令 \( y = f(x) = \frac{\ln x}{x} \),则 \( \ln x = xy \)。对 \( \forall n \in \mathbb{N} \),两边同时对 \( x \) 求 \( n \) 阶导,得
\[
(\ln x)^{(n)} = (xy)^{(n)} \Longleftrightarrow \frac{(-1)^{n - 1}(n - 1)!}{x^n} = \sum_{k=0}^{n} x^{(k)} y^{(n - k)} = xy^{(n)} + n y^{(n - 1)}.
\]
从而对 \( \forall n \in \mathbb{N} \),都有
\begin{align*}
&\quad \quad xy^{\left( n \right)}+ny^{\left( n-1 \right)}=\frac{\left( -1 \right) ^{n-1}\left( n-1 \right) !}{x^n}
\\
&\Longleftrightarrow \left( -1 \right) ^nx^{n+1}y^{\left( n \right)}-\left( -1 \right) ^{n-1}nx^ny^{\left( n-1 \right)}=-\left( n-1 \right) !
\\
&\Longleftrightarrow \frac{\left( -1 \right) ^nx^{n+1}y^{\left( n \right)}}{n!}-\frac{\left( -1 \right) ^{n-1}x^ny^{\left( n-1 \right)}}{\left( n-1 \right) !}=-\frac{1}{n}.
\end{align*}
于是
\[
\frac{(-1)^n x^{n + 1} y^{(n)}}{n!} - xy = \sum_{k=1}^{n} \left( -\frac{1}{k} \right).
\]
故
\[
f^{(n)}(x) = y^{(n)} = \frac{(-1)^n n!}{x^{n + 1}} \left( \sum_{k=1}^{n} \left( -\frac{1}{k} \right) - \ln x \right).
\]

\item 注意到$f\left( x \right) =\frac{1}{2}\left( \frac{1}{1-x}+\frac{1}{1+x} \right) ,$则$f^{\left( n \right)}\left( x \right) =\frac{n!}{2}\left( \frac{1}{\left( 1-x \right) ^{n+1}}+\frac{\left( -1 \right) ^n}{\left( 1+x \right) ^{n+1}} \right) .$

\item 注意到$f' \left( x \right) =\frac{1}{1+x^2}=\frac{1}{2\mathrm{i}}\left( \frac{1}{x-\mathrm{i}}-\frac{1}{x+\mathrm{i}} \right) ,$故
\begin{align*}
f^{\left( n \right)}\left( x \right) &=\left( \frac{1}{1+x^2} \right) ^{\left( n-1 \right)}=\frac{\left( -1 \right) ^{n-1}\left( n-1 \right) !}{2\mathrm{i}}\left[ \frac{1}{\left( x-\mathrm{i} \right) ^n}-\frac{1}{\left( x+\mathrm{i} \right) ^{\mathrm{n}}} \right] =\frac{\left( -1 \right) ^{n-1}\left( n-1 \right) !}{2\mathrm{i}\left( x^2+1 \right) ^{\mathrm{n}}}\left[ \left( x+\mathrm{i} \right) ^{\mathrm{n}}-\left( x-\mathrm{i} \right) ^n \right] 
\\
&=\frac{\left( -1 \right) ^{n-1}\left( n-1 \right) !}{2\mathrm{i}\left( x^2+1 \right) ^{\mathrm{n}}}\left[ \left( \sqrt{1+x^2}e^{\mathrm{iarc}\tan \frac{1}{x}} \right) ^{\mathrm{n}}-\left( \sqrt{1+x^2}e^{-\mathrm{iarc}\tan \frac{1}{x}} \right) ^n \right] 
\\
&=\frac{\left( -1 \right) ^{n-1}\left( n-1 \right) !}{2\mathrm{i}\left( x^2+1 \right) ^{\frac{n}{2}}}\left( e^{\mathrm{i}n\mathrm{arc}\tan \frac{1}{x}}-e^{-\mathrm{i}n\mathrm{arc}\tan \frac{1}{x}} \right) =\frac{\left( -1 \right) ^{n-1}\left( n-1 \right) !}{2\mathrm{i}\left( x^2+1 \right) ^{\frac{n}{2}}}\cdot 2\mathrm{i}\cdot \sin \left( n\mathrm{arc}\tan \frac{1}{x} \right) 
\\
&=\frac{\left( -1 \right) ^{n-1}\left( n-1 \right) !}{\left( x^2+1 \right) ^{\frac{n}{2}}}\sin \left( n\mathrm{arc}\tan \frac{1}{x} \right) .
\end{align*}
\end{enumerate}
\end{solution}

\begin{example}
设 \( f(x) = x^2 \ln(x + \sqrt{1 + x^2}) \),计算 \( f^{(n)}(0) \),\( n \in \mathbb{N} \)。
\end{example}
\begin{note}
此类问题都是通过背Taylor公式之后通过拼凑来得到 \( f^{(n)}(0) \),这是因为
\[
f(x) \sim \sum_{n=0}^{\infty} \frac{f^{(n)}(0)}{n!} x^n.
\]
\end{note}
\begin{solution}
注意到
\begin{align*}
\left[ \ln\left(x + \sqrt{1 + x^2}\right) \right]' &= \left[ \mathrm{arc}\sinh x \right]' = \frac{1}{\sqrt{1 + x^2}} = \left(1 + x^2\right)^{-\frac{1}{2}} \\
&\xlongequal{\text{广义二项式定理}} \sum_{n=0}^{\infty} \mathrm{C}_{-\frac{1}{2}}^{n} x^{2n},
\end{align*}
于是
\begin{align*}
\ln\left(x + \sqrt{1 + x^2}\right) &= \sum_{n=0}^{\infty} \frac{\mathrm{C}_{-\frac{1}{2}}^{n}}{2n + 1} x^{2n + 1} = x + \sum_{n=0}^{\infty} \frac{\mathrm{C}_{-\frac{1}{2}}^{n}}{2n + 1} x^{2n + 1} \\
&= x + \sum_{n=0}^{\infty} \frac{\left(-\frac{1}{2}\right)\left(-\frac{1}{2} - 1\right) \cdots \left(-\frac{1}{2} - n + 1\right)}{(2n + 1) \cdot n!} x^{2n + 1} \\
&= x + \sum_{n=0}^{\infty} \frac{(-1)^n (2n - 1)!!}{(2n + 1) \cdot n!} x^{2n + 1}.
\end{align*}
从而 \( f(x) = x^3 + \sum_{n=0}^{\infty} \frac{(-1)^n (2n - 1)!!}{(2n + 1) \cdot n!} x^{2n + 3} \),因此
\[
f^{(n)}(0) = \begin{cases}
6, & n = 3 \\
\frac{(-1)^m (2m - 1)!! (2m + 3)!}{m! \cdot 2^m (2m + 1)}, & n = 2m + 3, \, m = 1, 2, \cdots \\
0, & \text{其他}
\end{cases}
\]
\end{solution}

\begin{proposition}\label{proposition:一个额外记忆的级数}
\[
\arcsin^2 x = \sum_{n=1}^{\infty} \frac{2^{2n - 1} ((n - 1)!)^2}{(2n)!} x^{2n}, \, x \in (-1, 1).
\]
\end{proposition}

\begin{example}[生成级数或者建立递推法求解高阶导数值]
对 \( n \in \mathbb{N}_0 \),
\begin{enumerate}[(1)]
\item 设 \( f(x) = \arcsin^2 x \),求 \( f^{(n)}(0) \)。

\item 设 \( f(x) = \arcsin x \cdot \arccos x \),求 \( f^{(n)}(0) \)。

\item 设 \( f(x) = (x + \sqrt{x^2 + 1})^m \),\( m \in \mathbb{N} \),求 \( f^{(n)}(0) \)。

\item 设 \( f(x) = \arctan^2 x \),求 \( f^{(n)}(0) \)。
\end{enumerate}
\end{example}
\begin{note}
此类问一般是先建立函数满足的微分方程,然后用乘积求导法则或者形式幂级数对比系数来得到导数的递推,从而完成了证明。
\end{note}
\begin{solution}
\begin{enumerate}[(1)]
\item {\color{blue}解法一:}注意到
\[
f'(x) = \frac{2\arcsin x}{\sqrt{1 - x^2}} \Longleftrightarrow \sqrt{1 - x^2} f' = 2\arcsin x,
\]
令 \( y = f(x) \),则对上式两边同时求导得
\[
-\frac{x}{\sqrt{1 - x^2}} f' + \sqrt{1 - x^2} f'' = \frac{2}{\sqrt{1 - x^2}} \Longleftrightarrow -x y' + (1 - x^2) y'' = 2.
\]
再对上式两边同时对 \( x \) 求 \( n \ (n \geqslant 2) \) 阶导,得
\[
\left[ -x y' + (1 - x^2) y'' \right]^{(n)} = 2^{(n)}
\]
\[
\Longleftrightarrow -\left[ n y^{(n)} + x y^{(n + 1)} \right] + \left[ \dbinom{n}{2} \cdot (-2) y^{(n)} + \dbinom{n}{1} (-2x) y^{(n + 1)} + (1 - x^2) y^{(n + 2)} \right] = 0
\]
将 \( x = 0 \) 代入上式得
\begin{align}\label{100.2412424234----2-4-:::::--24..117}
f^{(n + 2)}(0) = n^2 f^{(n)}(0), \forall n \geqslant 2. 
\end{align}
显然上式对 \( n = 1 \) 也成立。又注意到 \( f''(0) = 2 \),因此对 \( \forall n \in \mathbb{N}_1 \),由\eqref{100.2412424234----2-4-:::::--24..117} 式可得
\[
\frac{f^{(2n + 2)}(0)}{f^{(2n)}(0)} = 4n^2 \Rightarrow \frac{f^{(2n + 2)}(0)}{f^{(2)}(0)} = \prod_{i = 1}^n 4i^2 \Rightarrow f^{(2n + 2)}(0) = 2^{2n + 1} (n!)^2.
\]
显然上式对 \( n = 0 \) 也成立。故
\[
f^{(2n + 2)}(0) = 2^{2n + 1} (n!)^2, \forall n \in \mathbb{N}_0.
\]
又 \( f'''(0) = 0 \),故由\eqref{100.2412424234----2-4-:::::--24..117}式可得
\[
f^{(2n - 1)}(0) = (2n - 1)^2 f^{(2n - 3)}(0) = \cdots = \left[ (2n - 1)!! \right]^2 f^{(3)}(0) = 0, \forall n \in \mathbb{N}_1.
\]

{\color{blue}解法二:}注意到
\[
f'(x) = \frac{2\arcsin x}{\sqrt{1 - x^2}} \Longleftrightarrow \sqrt{1 - x^2} f' = 2\arcsin x,
\]
令 \( y = f(x) \),则对上式两边同时求导得
\begin{align}\label{100.2412424234----2-4-:::::--24..118}
-\frac{x}{\sqrt{1 - x^2}} f' + \sqrt{1 - x^2} f'' = \frac{2}{\sqrt{1 - x^2}} \Longleftrightarrow -x y' + (1 - x^2) y'' = 2.
\end{align}
因为 \( f \in C^{\infty}(\mathbb{R}) \),所以由 Taylor 公式可知
\[
y = \sum_{n=0}^{\infty} a_n x^n, \quad y' = \sum_{n=1}^{\infty} n a_n x^{n - 1}, \quad y'' = \sum_{n=2}^{\infty} n(n - 1) a_n x^{n - 2},
\]
其中 \( a_n = \frac{f^{(n)}(0)}{n!}, n \in \mathbb{N}_0 \)。再将上式代入\eqref{100.2412424234----2-4-:::::--24..118} 式可得
\begin{align*}
2 &= -\sum_{n=1}^{\infty} n a_n x^n + \sum_{n=2}^{\infty} n(n - 1) a_n x^{n - 2} - \sum_{n=2}^{\infty} n(n - 1) a_n x^n
\\
&= -\sum_{n=0}^{\infty} n a_n x^n + \sum_{n=0}^{\infty} (n + 2)(n + 1) a_{n + 2} x^n - \sum_{n=0}^{\infty} n(n - 1) a_n x^n
\\
&= \sum_{n=0}^{\infty} \left[ (n + 2)(n + 1) a_{n + 2} - n a_n - n(n - 1) a_n \right] x^n.
\end{align*}
比较上式两边系数,得对 \( \forall n \in \mathbb{N}_1 \),都有
\begin{align}
&\quad \quad \left( n+2 \right) \left( n+1 \right) a_{n+2}-na_n-n\left( n-1 \right) a_n=0
\nonumber
\\
&\Longleftrightarrow \left( n+2 \right) \left( n+1 \right) \cdot \frac{f^{\left( n+2 \right)}\left( 0 \right)}{\left( n+2 \right) !}-n\cdot \frac{f^{\left( n \right)}\left( 0 \right)}{n!}-n\left( n-1 \right) \cdot \frac{f^{\left( n \right)}\left( 0 \right)}{n!}=0
\nonumber
\\
&\Longleftrightarrow f^{\left( n+2 \right)}\left( 0 \right) =n^2f^{\left( n \right)}\left( 0 \right) .\label{100.2412424234----2-4-:::::--24..119}
\end{align}
又 \( f''(0) = 2 \),因此对 \( \forall n \in \mathbb{N}_1 \),由 \eqref{100.2412424234----2-4-:::::--24..119}式可得
\[
\frac{f^{(2n + 2)}(0)}{f^{(2n)}(0)} = 4n^2 \Rightarrow \frac{f^{(2n + 2)}(0)}{f^{(2)}(0)} = \prod_{i = 1}^n 4i^2 \Rightarrow f^{(2n + 2)}(0) = 2^{2n + 1} (n!)^2.
\]
显然上式对 \( n = 0 \) 也成立。故
\[
f^{(2n + 2)}(0) = 2^{2n + 1} (n!)^2, \forall n \in \mathbb{N}_0.
\]
又 \( f'''(0) = 0 \),故由\eqref{100.2412424234----2-4-:::::--24..119}式可得
\[
f^{(2n - 1)}(0) = (2n - 1)^2 f^{(2n - 3)}(0) = \cdots = \left[ (2n - 1)!! \right]^2 f^{(3)}(0) = 0, \forall n \in \mathbb{N}_1.
\]

\item 

\item 

\item 
\end{enumerate}
\end{solution}

\begin{proposition}\label{proposition:n阶导数极限的计算}
设 \( f \) 在 \( a \) 处 \( n + 1 \) 阶连续可导的,证明:
\[
\lim_{x \to a} \frac{\mathrm{d}^n}{\mathrm{d}x^n} \left[ \frac{f(x) - f(a)}{x - a} \right] = \frac{f^{(n + 1)}(a)}{n + 1}.
\]
\end{proposition}
\begin{remark}
不妨设$a=0$,$f(a)=0$的原因:先证不妨设$f(a)=0$成立.假设$f(a)=0$时结论成立,则当$f(a)\ne0$时,令$g(x)=f(x)-f(a)$,则$g(a)=0$,从而由假设可知
\begin{align*}
\lim_{x\rightarrow a}\frac{\mathrm{d}^n}{\mathrm{d}x^n}\left[ \frac{f(x)-f(a)}{x-a} \right]&=\lim_{x\rightarrow a}\frac{\mathrm{d}^n}{\mathrm{d}x^n}\left[ \frac{g(x)}{x-a} \right]=\frac{g^{(n+1)}(a)}{n+1}=\frac{f^{(n+1)}(a)}{n+1}.
\end{align*}
故可以不妨设$f(a)=0$.

再证不妨设$a=0$成立.假设$a=0$时结论成立,则当$a\ne0$时,
令$g(x)=f(x+a)$,则由假设可知
\begin{align*}
\lim_{x\rightarrow 0}\frac{\mathrm{d}^n}{\mathrm{d}x^n}\left[ \frac{g(x)}{x} \right]=\frac{g^{(n+1)}(0)}{n+1}.
\end{align*}
从而
\begin{align*}
\lim_{x\rightarrow a}\frac{\mathrm{d}^n}{\mathrm{d}x^n}\left[ \frac{f(x)}{x-a} \right]&=\lim_{x\rightarrow 0}\frac{\mathrm{d}^n}{\mathrm{d}x^n}\left[ \frac{f(x+a)}{x} \right]=\lim_{x\rightarrow 0}\frac{\mathrm{d}^n}{\mathrm{d}x^n}\left[ \frac{g(x)}{x} \right]\\
&=\frac{g^{(n+1)}(0)}{n+1}=\frac{f^{(n+1)}(a)}{n+1}.
\end{align*}
故可以不妨设$a=0$.
\end{remark}
\begin{proof}
不妨设$a=0$,$f(a)=0$,从而
\begin{align*}
\frac{\mathrm{d}^n}{\mathrm{d}x^n}\left[ \frac{f(x)}{x} \right]&=\sum_{k=0}^n{\mathrm{C}_{n}^{k}f^{(k)}(x) \frac{(-1)^{n-k}(n-k)!}{x^{n-k+1}}}=\frac{n!(-1)^n}{x^{n+1}}\sum_{k=0}^n{\frac{1}{k!}f^{(k)}(x)(-x)^k}.
\end{align*}
于是由L'Hospital法则可得
\begin{align*}
&\lim_{x\rightarrow 0}\frac{\mathrm{d}^n}{\mathrm{d}x^n}\left[ \frac{f(x)}{x} \right]=n!(-1)^n\lim_{x\rightarrow 0}\frac{\sum\limits_{k=0}^n{\frac{1}{k!}f^{(k)}(x)(-x)^k}}{x^{n+1}}\\
&\xlongequal{\text{L'Hospital法则}}n!(-1)^n\lim_{x\rightarrow 0}\frac{\sum\limits_{k=0}^n{\frac{1}{k!}f^{(k+1)}(x)(-x)^k}-\sum\limits_{k=1}^n{\frac{1}{(k-1)!}f^{(k)}(x)(-x)^{k-1}}}{(n+1)x^n}\\
&=n!(-1)^n\lim_{x\rightarrow 0}\frac{\sum\limits_{k=0}^n{\frac{1}{k!}f^{(k+1)}(x)(-x)^k}-\sum\limits_{k=0}^{n-1}{\frac{1}{(k)!}f^{(k+1)}(x)(-x)^k}}{(n+1)x^n}\\
&=n!(-1)^n\lim_{x\rightarrow 0}\frac{\frac{1}{n!}f^{(n+1)}(x)(-x)^n}{(n+1)x^n}\\
&=\frac{f^{(n+1)}(0)}{n+1}.
\end{align*}
\end{proof}

\begin{example}
设 $f \in C^\infty(\mathbb{R}), n \in \mathbb{N}$ 满足
\[
f^{(j)}(0) = 0, j = 0, 1, 2, \cdots, n - 1, f^{(n)}(0) \neq 0.
\]
证明: \( g(x) = \begin{cases} 
\dfrac{f(x)}{x^n}, & x \neq 0 \\
\dfrac{f^{(n)}(0)}{n!}, & x = 0 
\end{cases} \) 在 $\mathbb{R}$ 上无穷次可微.
\end{example}
\begin{note}
本题不能对 Taylor 公式的 peano 余项求导来说明 $g$ 可微分性, 这是不严格的.
\end{note}
\begin{proof}
当$n=0$时,$g=f\in C^{\infty}(\mathbb{R})$显然成立.假设命题对$n\in \mathbb{N}$成立,考虑$n+1$的情形.令$h(x)=\frac{f(x)}{x}$,则
\begin{align}
\frac{f(x)}{x^{n+1}}=\frac{\frac{f(x)}{x}}{x^n}=\frac{h(x)}{x^n}. \tag{100.2412424234----2-4-:::::--24..119}
\end{align}
对$\forall k\in \mathbb{N}$,由\refpro{proposition:n阶导数极限的计算}可知
\begin{align*}
\lim_{x\rightarrow 0}h^{(k)}(x)=\lim_{x\rightarrow 0}\left[ \frac{f(x)}{x} \right]^{(k)}=\frac{f^{(k+1)}(0)}{k+1}.
\end{align*}
于是由\hyperref[theorem:单侧导数极限定理]{导数极限定理}可知$h^{(k)}(0)=\frac{f^{(k+1)}(0)}{k+1},\forall k\in \mathbb{N}$,故$h$在$x=0$处无穷次可微.又由$f\in C^{\infty}(\mathbb{R})$,从而$h\in C^{\infty}(\mathbb{R}\backslash \{0\})$,故$h\in C^{\infty}(\mathbb{R})$.于是
\begin{gather}
h^{(j)}(0)=\lim_{x\rightarrow 0}h^{(j)}(x)=\frac{f^{(j+1)}(0)}{j+1}=0,\quad 0\leqslant j\leqslant n-1,\nonumber \\
h^{(n)}(0)=\lim_{x\rightarrow 0}h^{(n)}(x)=\frac{f^{(n+1)}(0)}{n+1}\ne 0. \label{100.2412424234----2-4-:::::--24..120}
\end{gather}
因此$h(x)$满足归纳假设条件,进而由归纳假设及\eqref{100.2412424234----2-4-:::::--24..119}\eqref{100.2412424234----2-4-:::::--24..120}式可知
\begin{align*}
g(x)=\begin{cases}
\frac{f(x)}{x^{n+1}}&,x\ne 0\\
\frac{f^{(n+1)}(0)}{(n+1)!}&,x=0\\
\end{cases}=\begin{cases}
\frac{h(x)}{x^n}&,x\ne 0\\
\frac{h^{(n)}(0)}{n!}&,x=0\\
\end{cases}\in C^{\infty}(\mathbb{R}).
\end{align*}
因此由数学归纳法可知,结论成立.
\end{proof}










\end{document}
\documentclass[../../main.tex]{subfiles}
\graphicspath{{\subfix{../../image/}}} % 指定图片目录,后续可以直接使用图片文件名。

% 例如:
% \begin{figure}[H]
% \centering
% \includegraphics[scale=0.4]{图.png}
% \caption{}
% \label{figure:图}
% \end{figure}
% 注意:上述\label{}一定要放在\caption{}之后,否则引用图片序号会只会显示??.

\begin{document}

\section{$\mathbb{R}^n$中点集的距离}

\subsection{点集距离概念及性质}

\begin{definition}
设 $A \subset \mathbb{R}^n$ 为非空集, 则 $A$ 的直径定义为
\begin{align*}
\mathrm{diam}(A) = \sup\{d(x, y) : x, y \in A\}
\end{align*}
若 $\mathrm{diam}(A) < \infty$, 则称 $A$ 为有界集. 容易验证: $A$ 有界, 当且仅当存在 $M \geqslant slant 0$ 使得 $d(\mathbf{0}, x) \leqslant slant M$, $\forall x \in A$.

设 $x \in \mathbb{R}^n$, 则点 $x$ 到集合 $A$ 的距离定义为
\begin{align*}
d(x, A) = \inf\{d(x, y) : y \in A\}
\end{align*}

设 $A, B \subset \mathbb{R}^n$ 为非空集, 则集合 $A$ 到 $B$ 的距离定义为
\begin{align*}
d(A, B) = \inf\{d(x, y) : x \in A, y \in B\}
\end{align*}
\end{definition}

\begin{proposition}\label{proposition:点与集合之间的距离函数在R^n上一致连续}
设 $E$ 为 $\mathbb{R}^n$ 中的非空点集, 则 $d(x, E)$ 作为 $x$ 的函数在 $\mathbb{R}^n$ 上是一致连续的.
\end{proposition}
\begin{proof}
设 $x, y \in \mathbb{R}^n$, 由 $d(y, E)$ 的定义, 对 $\forall \varepsilon > 0$, 存在 $z \in E$ 使得 $d(y, z) < d(y, E) + \varepsilon$. 从而
\begin{align*}
d(x, E) \leqslant slant d(x, z) \leqslant slant d(x, y) + d(y, z) < d(x, y) + d(y, E) + \varepsilon
\end{align*}
由 $\varepsilon$ 的任意性, $d(x, E) - d(y, E) \leqslant slant d(x, y)$. 同理可证 $d(y, E) - d(x, E) \leqslant slant d(x, y)$. 从而
\begin{align*}
|d(x, E) - d(y, E)| \leqslant slant d(x, y), \quad \forall x, y \in \mathbb{R}^n
\end{align*}
因此, $d(x, E)$ 在 $\mathbb{R}^n$ 上是一致连续.
\end{proof}

\begin{proposition}
设 $F_1, F_2$ 为 $\mathbb{R}^n$ 中互不相交的非空闭集, 则存在 $\mathbb{R}^n$ 上的连续函数 $f(x)$ 满足:
\begin{enumerate}[(i)]
\item $0 \leqslant slant f(x) \leqslant slant 1$, $\forall x \in \mathbb{R}^n$;
\item $F_1 = \{x \in \mathbb{R}^n : f(x) = 1\}$, $F_2 = \{x \in \mathbb{R}^n : f(x) = 0\}$.
\end{enumerate}
\end{proposition}
\begin{proof}
定义函数
\begin{align*}
f(x) = \frac{d(x, F_2)}{d(x, F_1) + d(x, F_2)}, \quad x \in \mathbb{R}^n
\end{align*}
则由\refpro{proposition:点与集合之间的距离函数在R^n上一致连续}以及连续函数四则运算封闭性知 $f(x)$ 连续. 容易验证 $f(x)$ 满足条件 (i), (ii).
\end{proof}
  
\begin{proposition}
$\mathbb{R}^n$ 中每个闭集可表示为可列个开集的交集; 每个开集可表示为可列个闭集的并集.
\end{proposition}
\begin{proof}
首先证明: 若 $A \subset \mathbb{R}^n$ 为非空集, $r > 0$, 则 $G = \{x \in \mathbb{R}^n : d(x, A) < r\}$ 是开集. 实际上, 任取 $x \in G$, 则 $d(x, A) < r$. 令 $h = r - d(x, A)$, 则对 $\forall y \in U(x, h/2)$, 有
\begin{align*}
d(y, A) \leqslant slant d(y, x) + d(x, A) < \frac{h}{2} + d(x, A) = \frac{r}{2} + \frac{d(x, A)}{2} < r
\end{align*}
故 $U(x, h/2) \subset G$. 从而 $x$ 是 $G$ 的内点, 因此, $G$ 是开集.

(1) 设 $F \subset \mathbb{R}^n$ 为闭集, 令
\begin{align*}
G_n = \left\{x \in \mathbb{R}^n : d(x, F) < \frac{1}{n}\right\}
\end{align*}
则 $G$ 是开集. 下面证 $F = \bigcap_{n = 1}^{\infty} G_n$.

设 $x \in F$, 则 $d(x, F) = 0 < 1/n$, $n = 1, 2, \cdots$, 故 $x \in G_n$, $n = 1, 2, \cdots$, 于是 $x \in \bigcap_{n = 1}^{\infty} G_n$. 从而 $F \subset \bigcap_{n = 1}^{\infty} G_n$.

设 $x \in \bigcap_{n = 1}^{\infty} G_n$, 则 $x \in G_n$, $n = 1, 2, \cdots$, 故 $d(x, F) < 1/n$, $n = 1, 2, \cdots$. 令 $n \to \infty$, 可得 $d(x, F) = 0$. 又 $F$ 是闭集, 故 $x \in F$. 从而 $\bigcap_{n = 1}^{\infty} G_n \subset F$.

(2) 设 $G \subset \mathbb{R}^n$ 为开集, 则 $G^c$ 是闭集. 由 (1) 知 $G^c = \bigcap_{n = 1}^{\infty} G_n$, 其中, $G_n$ 为开集. 从而 $G = \left(\bigcap_{n = 1}^{\infty} G_n\right)^c = \bigcup_{n = 1}^{\infty} G_n^c$, 其中 $G_n^c$ 为闭集.

特别地, 对于一维的闭区间与开区间, 有
\begin{align*}
[a, b] = \bigcap_{n = 1}^{\infty}\left(a - \frac{1}{n}, b + \frac{1}{n}\right), \quad (a, b) = \bigcup_{n = 1}^{\infty}\left[a + \frac{1}{n}, b - \frac{1}{n}\right]
\end{align*}
\end{proof}

\subsection{$\mathbb{R}^n$ 上的完备性定理及应用}

我们知道, 数学分析中学过的实数集完备性的基本定理是构建极限理论和数学分析的基础. 实际上这些结果在 $\mathbb{R}^n$ 中仍然成立.

\begin{theorem}[致密性定理]\label{theorem:致密性定理}
$\mathbb{R}^n$ 中任意有界点列都存在收敛子列.
\end{theorem}
\begin{proof}
设 $\{x_k\} \subset \mathbb{R}^n$ 为有界点列, 记
\begin{align*}
x_k = (x_1^{(k)}, x_2^{(k)}, \cdots, x_n^{(k)}), \quad k = 1, 2, \cdots
\end{align*}
注意到 $|x_i^{(k)}| \leqslant slant \left(\sum_{i = 1}^{n}|x_i^{(k)}|^2\right)^{1/2} = d(x_k, \mathbf{0})$, 则 $\{x_1^{(k)}\}$ 是 $\mathbb{R}$ 中的有界数列, 由 $\mathbb{R}$ 上的致密性定理, 存在 $\{x_1^{(k)}\}$ 的子列 $\{x_1^{(k_i)}\}$ 满足 $x_1^{(k_i)} \to x_1$; 同理, 实数列 $\{x_2^{(k_i)}\}$ 进而 $\{x_2^{(k_{i_j})}\}$ 有界, 由 $\mathbb{R}$ 上的致密性定理, 存在 $\{k_{i_j}\}$ 的子列 $\{k_{i_{j_l}}\}$ 使得 $x_2^{(k_{i_{j_l}})} \to x_2$; 以此类推 $\cdots\cdots$; 存在 $\{k_{n - 1}\}$ 的子列 $\{k_n\}$ 使得 $x_n^{(k_n)} \to x_n$. 令 $x = (x_1, x_2, \cdots, x_n)$, 则
\begin{align*}
d(x_{k_n}, x) = \left(\sum_{i = 1}^{n}(x_i^{(k_n)} - x_i)^2\right)^{\frac{1}{2}} \to 0
\end{align*}
因此, $x_{k_n} \to x$, $k_n \to \infty$.
\end{proof}

\begin{theorem}[闭集套定理]\label{theorem:闭集套定理}
设 $\{A_n\}$ 为 $\mathbb{R}^n$ 中一列单调递减的非空闭集列, 则 $\bigcap_{n = 1}^{\infty} A_n$ 为非空有界闭集. 若 $\lim_{n \to \infty} \mathrm{diam}(A_n) = 0$, 则 $\bigcap_{n = 1}^{\infty} A_n$ 为单点集.
\end{theorem}
\begin{proof}
$A_n \neq \varnothing$, 取 $x_n \in A_n$, $n \in \mathbb{N}$, 则 $\{x_n\}$ 有界. 由致密性定理, 存在子列 $\{x_{n_k}\} \subset \{x_n\}$ 满足 $x_{n_k} \to x_0$, $k \to \infty$.

由于 $\{A_n\}$ 单调递减, 则对 $\forall k \in \mathbb{N}$, 有 $x_{n_k}, x_{n_{k + 1}}, \cdots \in A_{n_k}$. 注意到 $x_{n_k} \to x_0$ 以及 $A_{n_k}$ 是闭集, 可得 $x_0 \in A_{n_k} \subset A_k$ (因为 $n_k \geqslant slant k$). 因此, $x_0 \in \bigcap_{k = 1}^{\infty} A_k$.

设 $\lim_{n \to \infty} \mathrm{diam}(A_n) = 0$. 若存在 $y_0 \in \bigcap_{n = 1}^{\infty} A_n$ 且 $y_0 \neq x_0$, 则
\begin{align*}
\mathrm{diam}(A_n) \geqslant slant \mathrm{diam}\left(\bigcap_{n = 1}^{\infty} A_n\right) \geqslant slant d(x_0, y_0)
\end{align*}
这与 $\mathrm{diam}(A_n) \to 0$ 矛盾. 
\end{proof}

\begin{theorem}[有限覆盖定理]\label{theorem:有限覆盖定理}
设 $A$ 为 $\mathbb{R}^n$ 中的有界闭集, 若 $A \subset \bigcup_{\alpha \in \varGamma} G_\alpha$, 其中 $G_\alpha$ 为开集, 则存在 $G_1, G_2, \cdots, G_n$ 使得 $A \subset \bigcup_{i = 1}^{n} G_i$.
\end{theorem}
\begin{proof}
令 $E_n = \bigcup_{i = 1}^{n} G_i$, $F_n = E_n^c \cap A$. 只需证存在 $n_0 \in \mathbb{N}$, 使得 $F_{n_0} = \varnothing$.

若对 $\forall n \in \mathbb{N}$, 都有 $F_n \neq \varnothing$. 注意到 $\{E_n\}$ 是单调递增的开集列, $A$ 是有界闭集, 则 $\{F_n\}$ 是有界单调递减的非空闭集列. 由闭集套定理知 $\bigcap_{n = 1}^{\infty} F_n \neq \varnothing$, 则存在 $x_0 \in F_n$, $\forall n \in \mathbb{N}$. 又 $F_n = E_n^c \cap A$, 从而 $x_0 \in A$, 且 $x_0 \notin E_n$, $\forall n \in \mathbb{N}$, 故 $x_0 \notin \bigcup_{i = 1}^{\infty} G_i$. 这与 $x_0 \in A \subset \bigcup_{i = 1}^{\infty} G_i$ 矛盾.
\end{proof}

\begin{definition}[紧集]
若 $X$ 的任一开覆盖均含有一个有限子覆盖, 则称 $X$ 为\textbf{紧集}.
\end{definition}

\begin{theorem}\label{theorem:mathbb{R}^n中的紧集等价于有家闭集.}
$\mathbb{R}^n$中的紧集等价于有家闭集.
\end{theorem}
\begin{proof}
见豌豆讲义.
\end{proof}

\begin{theorem}
设 $A \subset \mathbb{R}^n$ 为非空闭集, $x \in \mathbb{R}^n$, 则存在 $y_0 \in A$ 使得
\begin{align*}
d(x, A) = d(x, y_0).
\end{align*}
\end{theorem}
\begin{proof}
由下确界的定义, 存在点列 $\{y_n\} \subset A$ 满足
\begin{align*}
d(x, A) = \lim_{n \to \infty} d(x, y_n)
\end{align*}
故存在 $M > 0$ 使得 $d(x, y_n) \leqslant slant M$. 再由
\begin{align*}
d(\mathbf{0}, y_n) \leqslant slant d(\mathbf{0}, x) + d(x, y_n) \leqslant slant d(\mathbf{0}, x) + M
\end{align*}
知 $\{y_n\}$ 有界, 利用致密性定理, 存在子列 $\{y_{n_k}\} \subset \{y_n\}$ 满足 $y_{n_k} \to y_0$, $k \to \infty$. 又 $A$ 是闭集, 故 $y_0 \in A$, 从而
\begin{align*}
d(x, y_0) = \lim_{k \to \infty} d(x, y_{n_k}) = d(x, A)
\end{align*}
\end{proof}

\begin{theorem}
设 $A, B$ 为 $\mathbb{R}^n$ 中的非空闭集, 且其中之一有界, 则存在 $x_0 \in A$, $y_0 \in B$ 使得 $d(A, B) = d(x_0, y_0)$. 若 $A \cap B = \varnothing$, 则 $d(A, B) > 0$.
\end{theorem}
\begin{remark}
若这个定理中的 $A, B$ 均为无界闭集, 则上述结论不一定成立. 例如
\begin{align*}
A = \{n\}, \quad B = \{n + 1/2n\}
\end{align*}
则 $A' = B' = \varnothing$, 从而 $A, B$ 是闭集. 显然, $d(A, B) = 0$, 而 $A \cap B = \varnothing$, 故不存在 $x_0 \in A$, $y_0 \in B$ 使得 $d(x_0, y_0) = 0$.
\end{remark}
\begin{proof}
由 $d(A, B)$ 的定义, 存在 $\{x_n\} \subset A$ 与 $\{y_n\} \subset B$ 满足
\begin{align*}
d(A, B) = \lim_{n \to \infty} d(x_n, y_n)
\end{align*}
故存在 $M \geqslant slant 0$ 使得 $d(x_n, y_n) \leqslant slant M$.

不妨设 $A$ 有界, 则 $\{x_n\}$ 有界. 由致密性定理, 存在子列 $\{x_{n_k}\} \subset \{x_n\}$ 满足 $x_{n_k} \to x_0$, $k \to \infty$. 又 $A$ 是闭集, 故 $x_0 \in A$. 注意到
\begin{align*}
d(\mathbf{0}, y_n) \leqslant slant d(\mathbf{0}, x_n) + d(x_n, y_n) \leqslant slant d(\mathbf{0}, x_n) + M
\end{align*}
以及 $\{x_n\}$ 的有界性, 则有 $\{y_n\}$ 有界. 再由致密性定理, 存在子列 $\{y_{n_{k'}}\} \subset \{y_n\}$ 满足 $y_{n_{k'}} \to y_0$, $k' \to \infty$. 同样, $B$ 是闭集保证了 $y_0 \in B$. 于是
\begin{align*}
d(x_0, y_0) = \lim_{k' \to \infty} d(x_{n_{k'}}, y_{n_{k'}}) = d(A, B)
\end{align*}

若 $A \cap B = \varnothing$, 则 $x_0 \neq y_0$, 从而 $d(A, B) = d(x_0, y_0) > 0$.
\end{proof}

\begin{theorem}\label{theorem:距离大于零的两个集合中一定存在两个无交的子集}
设 $F_1, F_2 \subset \mathbb{R}^n$, 若 $d(F_1, F_2) > 0$, 则存在开集 $G_1, G_2$ 满足 $G_1 \supset F_1$, $G_2 \supset F_2$ 且 $G_1 \cap G_2 = \varnothing$.
\end{theorem}
\begin{proof}
由于 $d(F_1, F_2) > 0$, 则对 $\forall x \in F_1$, 都有 $d(x, F_2) \geqslant slant d(F_1, F_2) > 0$. 同理, 对 $\forall y \in F_2$, 都有 $d(y, F_1) \geqslant slant d(F_1, F_2) > 0$. 令
\begin{align*}
G_1 = \bigcup_{x \in F_1} U\left(x, \frac{d(x, F_2)}{2}\right), \quad G_2 = \bigcup_{y \in F_2} U\left(y, \frac{d(y, F_1)}{2}\right)
\end{align*}
则 $G_1, G_2$ 是开集, 且 $G_1 \supset F_1$, $G_2 \supset F_2$. 下面证明 $G_1 \cap G_2 = \varnothing$.

若存在 $z \in G_1 \cap G_2$, 即 $z \in G_1$ 且 $z \in G_2$, 则存在 $x_0 \in F_1$, $y_0 \in F_2$ 使得
\begin{align*}
z \in U\left(x_0, \frac{d(x_0, F_2)}{2}\right), \quad z \in U\left(y_0, \frac{d(y_0, F_1)}{2}\right)
\end{align*}
不妨设 $d(y_0, F_1) \leqslant slant d(x_0, F_2)$, 则有
\begin{align*}
d(x_0, F_2) &\leqslant slant d(x_0, y_0) \leqslant slant d(x_0, z) + d(z, y_0)\\
&< \frac{d(x_0, F_2)}{2} + \frac{d(y_0, F_1)}{2}\\
&\leqslant slant d(x_0, F_2)
\end{align*}
矛盾.
\end{proof}

\begin{theorem}[分离定理]\label{theorem:分离定理}
设 $F_1, F_2$ 为 $\mathbb{R}^n$ 中的闭集, 且 $F_1 \cap F_2 = \varnothing$, 则存在开集 $G_1, G_2$ 满足 $G_1 \supset F_1$, $G_2 \supset F_2$ 且 $G_1 \cap G_2 = \varnothing$.
\end{theorem}
\begin{proof}
由于 $F_1, F_2$ 是闭集, 且 $F_1 \cap F_2 = \varnothing$, 则 $\forall x \in F_1$ 都有 $d(x, F_2) > 0$; $\forall y \in F_2$ 都有 $d(y, F_1) > 0$. 由\refthe{theorem:距离大于零的两个集合中一定存在两个无交的子集}知结论成立.
\end{proof}

\begin{theorem}[连续函数延拓定理(Tietze扩张定理)]\label{theorem:连续函数延拓定理(Tietze扩张定理)}
设 $F$ 是 $\mathbb{R}^n$ 中的闭集, $f(x)$ 是定义在 $F$ 上的连续函数, 且 $|f(x)| \leqslant slant M$, $\forall x \in F$, 则存在 $\mathbb{R}^n$ 上的连续函数 $g(x)$ 满足:
\begin{enumerate}[(i)]
\item $g(x) = f(x)$, $\forall x \in F$;
\item $|g(x)| \leqslant slant M$, $\forall x \in \mathbb{R}^n$.
\end{enumerate}
\end{theorem}
\begin{remark}
$f(x)$ 无界时, 定理中连续延拓的结论仍成立. 
\end{remark}
\begin{proof}
(1) 把 $F$ 分为三个点集
\begin{align*}
F_1 &= \{x \in F : M/3 \leqslant slant f(x) \leqslant slant M\}\\
F_2 &= \{x \in F : -M \leqslant slant f(x) \leqslant slant -M/3\}\\
F_3 &= \{x \in F : -M/3 < f(x) < M/3\}
\end{align*}
则 $F_1, F_2$ 是互不相交的闭集, 作 $\mathbb{R}^n$ 上的函数
\begin{align*}
g_1(x) = \frac{M}{3} \cdot \frac{d(x, F_2) - d(x, F_1)}{d(x, F_2) + d(x, F_1)}, \quad x \in \mathbb{R}^n
\end{align*}
由\refpro{proposition:点与集合之间的距离函数在R^n上一致连续}以及连续函数四则运算封闭性, 知 $g_1(x)$ 在 $\mathbb{R}^n$ 上连续. 又容易验证
\begin{align*}
|g_1(x)| &\leqslant slant \frac{M}{3}, \quad \forall x \in \mathbb{R}^n\\
|f(x) - g_1(x)| &\leqslant slant \frac{2M}{3}, \quad \forall x \in F
\end{align*}

(2) 用 (1) 的方法在 $F$ 上考察函数 $f(x) - g_1(x)$(相当于上述的 $f(x)$). 由于 $|f(x) - g_1(x)| \leqslant slant 2M/3$, 故得到 $\mathbb{R}^n$ 上的函数 $g_2(x)$ 满足
\begin{align*}
|g_1(x)| &\leqslant slant \frac{1}{3} \cdot \frac{2M}{3}, \quad \forall x \in \mathbb{R}^n\\
|f(x) - g_1(x) - g_2(x)| &\leqslant slant \frac{2}{3} \cdot \frac{2M}{3} = \left(\frac{2}{3}\right)^2 M, \quad \forall x \in F
\end{align*}

(3) 依次做下去, 得到一列 $\mathbb{R}^n$ 上的函数 $\{g_k(x)\}$ 满足
\begin{gather}
|g_k(x)| \leqslant slant \frac{1}{3} \times \left(\frac{2}{3}\right)^{k - 1} M, \quad \forall x \in \mathbb{R}^n, \quad k = 1, 2, \cdots
\label{theorem1864891614-1.1}
\\
\left|f(x) - \sum_{i = 1}^{k} g_i(x)\right| \leqslant slant \left(\frac{2}{3}\right)^k M, \quad \forall x \in F, \quad k = 1, 2, \cdots
\label{theorem1864891614-1.2}
\end{gather}
式\eqref{theorem1864891614-1.1}表明函数项级数 $\sum_{k = 1}^{\infty} g_k(x)$ 一致收敛, 记其和函数为 $g(x)$, 则 $g(x)$ 连续 (连续函数项级数一致收敛的和函数是连续函数). 令式 \eqref{theorem1864891614-1.2}中的 $k \to \infty$ 得
\begin{align*}
f(x) = \sum_{k = 1}^{\infty} g_k(x) = g(x), \quad \forall x \in F
\end{align*}

(4) 对 $\forall x \in \mathbb{R}^n$, 有
\begin{align*}
|g(x)| \leqslant slant \sum_{k = 1}^{\infty}|g_k(x)| \leqslant slant \frac{M}{3}\left(1 + \frac{2}{3} + \left(\frac{2}{3}\right)^2 + \cdots\right) = M
\end{align*}
定理得证.
\end{proof}











\end{document}
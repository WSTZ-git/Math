\documentclass[lang=cn,newtx,10pt,scheme=chinese]{../Template/elegantbook}

\title{阶的估计}
\subtitle{\,\,}

\author{邹文杰}
\institute{无}
\date{2024/10/25}
\version{ElegantBook-4.5}
\bioinfo{自定义}{信息}

\extrainfo{宠辱不惊,闲看庭前花开花落;
\\
去留无意,漫随天外云卷云舒.}


\setcounter{tocdepth}{3}

\logo{logo-blue.png}
\cover{cover.png}

% 本文档额外使用的宏包和命令
\usepackage{../Styles/mystyle-elegantbook}

\begin{document}

\maketitle
\frontmatter

\tableofcontents

\mainmatter% 将行为改回预期版本,并重置页码

\chapter{阶的概念及$O$与$o$的运算}

\section{关于$O$与$o$的基本定理及应用}

\begin{theorem}[$O$与$o$的基本运算法则]\label{$O$与$o$的基本运算法则}
    \textbf{法则1:}若\(f(x)\)是无穷大量,\(x\to x_0\),并且\(\varphi(x)=O(1)\),则
\(\varphi(x)=o(f(x))\),\(x\to x_0\).

\textbf{法则2:}若\(f(x)=O(\rho)\),\(\rho = O(\psi)\),则
\(f(x)=O(\psi)\).

\textbf{法则3:}若\(f(x)=O(\rho)\),\(\rho = o(\psi)\),则
\(f(x)=o(\psi)\).

\textbf{法则4:}\(O(f)+O(g)=O(f + g)\).

\textbf{法则5:}\(O(f)O(g)=O(fg)\).

\textbf{法则6:}\(o(1)O(f)=o(f)\).

\textbf{法则7:}\(O(1)o(f)=o(f)\).

\textbf{法则8:}\(O(f)+o(f)=O(f)\).

\textbf{法则9:}\(o(f)+o(g)=o(|f|+|g|)\).

\textbf{法则10:}\(o(f)\cdot o(g)=o(fg)\).

\textbf{法则11:}\(\{O(f)\}^k = O(f^k)\),\(k\)是自然数.一般地,“大\(O\)常数”与\(k\)有关.

\textbf{法则12:}\(\{o(f)\}^k = o(f^k)\).

\textbf{法则13:}若\(f\sim g\),\(g\sim\varphi\),则\(f\sim\varphi\).

\textbf{法则14:}若\(f = o(g)\),\(g\sim\varphi\),则\(g\sim\varphi\pm f\).

\textbf{法则15:}若\(f(x)\)与\(g(x)\)都是正值函数,\(f(x)=o(g(x))\),\(x\to\infty\),则
\[
\int_{A}^{B}f(x)dx = o\left(\int_{A}^{B}g(x)dx\right),\quad B > A\to\infty.
\]
特别地,若存在\(A_0\)使得\(\int_{A_0}^{\infty}g(x)dx<\infty\),则
\[
\int_{A}^{\infty}f(x)dx = o\left(\int_{A}^{\infty}g(x)dx\right),\quad A\to\infty.
\]

\textbf{法则16:}若\(a_n\)与\(b_n\)都取正值\((n = 1,2,3,\cdots)\),\(a_n=o(b_n)\),\(n\to\infty\),则
\[
\sum_{n = N}^{M}a_n = o\left(\sum_{n = N}^{M}b_n\right),\quad M > N\to\infty.
\]
特别地,若\(\sum_{n = 1}^{\infty}b_n<\infty\),则
\[
\sum_{n = N}^{\infty}a_n = o\left(\sum_{n = N}^{\infty}b_n\right),\quad N\to\infty.
\]
\end{theorem}
\begin{note}
这些性质不需要死记硬背,需要使用的时候直接利用$o,O$的定义验证即可.
\end{note}
\begin{proof}
    {\color[RGB]{128, 128, 0} \text{法则}6\text{的证明}:}设\(\lim_{x\to x_0}\varphi(x) = 0\),且
\[
|g(x)|\leq Mf(x),\quad x\in(a,b).
\]
则
\[
0\leq\lim_{x\to x_0}\left|\frac{\varphi(x)g(x)}{f(x)}\right|\leq M\lim_{x\to x_0}|\varphi(x)| = 0,
\]
于是
\[
\varphi(x)g(x)=o(f(x)).
\]

{\color[RGB]{128, 128, 0} \text{法则}11\text{的证明}:}设\(|g(x)|\leq Mf(x),\quad x\in(a,b)\).则
\[
|g(x)|^k\leq M^k f^k(x),\quad x\in(a,b),
\]
即
\[
(g(x))^k = O_k(f^k(x)),\quad x\in(a,b).
\]
\end{proof}

\begin{proposition}[极限的等价定义]\label{proposition:极限的等价定义(用o余项定义极限)}
\begin{enumerate}
\item $\underset{n\rightarrow \infty}{\lim}a_n=a$
等价于$a_n=a+o\left( 1 \right) ,n\rightarrow \infty$.

\item $\underset{x\rightarrow x_0}{\lim}f\left( x \right) =A$
等价于$f\left( x \right) =A+o\left( 1 \right) ,x\rightarrow x_0$.
\end{enumerate}

\end{proposition}

\begin{theorem}\label{theorem:发散级数的求和号与o可交换}
    设\(b_n>0\),\(\sum_{n = 1}^{\infty}b_n=\infty\),且\(a_n = o(b_n),n\to\infty\),则
\[
\sum_{n = 1}^{N}a_n=o\left(\sum_{n = 1}^{N}b_n\right),N\to\infty.
\]
\end{theorem}
\begin{proof}
    根据$Stolz$定理,再结合$a_n=o\left( b_n \right)$可知
    \begin{align*}
        \underset{n\rightarrow +\infty}{\lim}\frac{\sum\limits_{i=1}^n{a_i}}{\sum\limits_{i=1}^n{b_i}}=\underset{n\rightarrow +\infty}{\lim}\frac{a_n}{b_n}=0.
    \end{align*}得证.
\end{proof}

\begin{theorem}\label{theorem:发散积分的积分号与o可交换}
   设\(g(x)>0\),\(f(x)=o(g(x))\),\(x\to\infty\),且\(\int_{a}^{\infty}g(x)dx\)发散,则
\[
\int_{a}^{x}f(t)dt = o\left(\int_{a}^{x}g(t)dt\right),\quad x\to\infty.
\]
\end{theorem}
\begin{proof}
    根据$L'Hospital'rule$,再结合$f(x)=o(g(x))$可知
    \begin{align*}
        \underset{x\rightarrow +\infty}{\lim}\frac{\int_a^x{f\left( t \right) dt}}{\int_a^x{g\left( t \right) dt}}=\underset{x\rightarrow +\infty}{\lim}\frac{f\left( t \right)}{g\left( t \right)}=0.
    \end{align*}得证.
\end{proof}

\begin{theorem}[Abel极限定理]\label{theorem:Abel极限定理}
    设\(\{b_n\}\)是正数列,\(a_n = o(b_n)\),
\[
f(x)=\sum_{n = 0}^{\infty}a_nx^n,\quad g(x)=\sum_{n = 0}^{\infty}b_nx^n.
\]
又设当\(0\leq x<1\)时,级数
\[
\sum_{n = 0}^{\infty}b_nx^n<\infty,
\]
并且
\[
\sum_{n = 0}^{\infty}b_nx^n\to\infty,\quad x\to 1^-,
\]
则
\[
f(x)=o(g(x)),\quad x\to 1^-.
\]
\end{theorem}
\begin{proof}
    根据$Stolz$定理,再结合$a_n=o(b_n)$可知
    \begin{align*}
        \underset{x\rightarrow 1^-}{\lim}\frac{f\left( x \right)}{g\left( x \right)}=\underset{x\rightarrow 1^-}{\lim}\frac{\sum\limits_{n=1}^{\infty}{a_ix^n}}{\sum\limits_{n=1}^{\infty}{b_ix^n}}=\underset{x\rightarrow 1^-}{\lim}\frac{a_nx^n}{b_nx^n}=\underset{x\rightarrow 1^-}{\lim}\frac{a_n}{b_n}=0.
    \end{align*}
    得证.
\end{proof}













\begin{example}
    设\(a_n = O(b_n)\),\(n\geq1\),并且级数\(\sum_{n = 1}^{\infty}b_n<\infty\),则存在常数\(C\),使得
\[
\sum_{k = 1}^{n}a_k = C + o\left(\sum_{k = n + 1}^{\infty}b_k\right),\quad n\to\infty.
\]
\end{example}
\begin{proof}
    显然,级数\(\sum_{k = 1}^{\infty}a_k\)收敛,因此存在常数\(C\)使得\(\sum_{k = 1}^{\infty}a_k = C\),于是
\[
\sum_{k = 1}^{n}a_k=\sum_{k = 1}^{\infty}a_k-\sum_{k = n + 1}^{\infty}a_k = C + o\left(\sum_{k = n + 1}^{\infty}b_k\right),\quad n\to\infty. 
\]
得证.
\end{proof}























\end{document}

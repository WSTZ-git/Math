\documentclass[aspectratio=169]{beamer} 
% aspectratio=169 表示使用 16:9 的宽屏比例

% --- 中文支持 ---
\usepackage{ctex} 

% --- 主题设置 ---
% 推荐使用 metropolis 主题,如果没有安装,可以改用 Madrid
\usetheme{metropolis} 
% \usetheme{Madrid} % 如果不喜欢极简风,取消这行注释

% --- 颜色微调 (可选) ---
\definecolor{mycolor}{RGB}{0, 80, 160} % 自定义颜色
\usecolortheme[named=mycolor]{structure}

% --- 基础信息 ---
\title{这是一个 LaTeX Beamer 演示}
\subtitle{这里是副标题}
\author{你的名字}
\institute{你的单位/大学}
\date{\today}

\begin{document}

% 1. 标题页
\begin{frame}
    \titlepage
\end{frame}

% 2. 目录页
\begin{frame}{目录}
    \tableofcontents
\end{frame}

% 3. 正文页面
\section{介绍}
\begin{frame}{第一部分:介绍}
    这里是正文内容。
    
    \begin{itemize}
        \item 第一点:LaTeX 做 PPT 很整洁
        \item 第二点:数学公式支持完美:$E = mc^2$
        \item 第三点:**Metropolis** 主题非常现代
    \end{itemize}
    
    \vspace{0.5cm} % 增加垂直间距
    
    \begin{block}{重要结论}
        使用 block 环境可以突出显示重要内容。
    \end{block}
\end{frame}

\section{数学部分}
\begin{frame}{数学公式展示}
    Beamer 非常适合展示复杂公式:
    
    $$
    \int_{-\infty}^{\infty} e^{-x^2} dx = \sqrt{\pi}
    $$
    
    可以分步显示内容(按回车出现):
    \pause
    \begin{enumerate}
        \item 第一步...
        \pause
        \item 第二步...
    \end{enumerate}
\end{frame}

\begin{frame}
    \centering
    \Huge 谢谢聆听!
\end{frame}

\end{document}
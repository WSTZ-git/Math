\documentclass[lang=cn,newtx,10pt,scheme=chinese]{elegantbook}

\title{数学技巧积累}
\subtitle{数学技巧积累}

\author{邹文杰}
\institute{无}
\date{2024/10/25}
\version{ElegantBook-4.5}
\bioinfo{自定义}{信息}

\extrainfo{宠辱不惊,闲看庭前花开花落;
\\
去留无意,漫随天外云卷云舒.}

\setcounter{tocdepth}{3}

\logo{logo-blue.png}
\cover{cover.png}

% 本文档额外使用的宏包和命令
\usepackage{mystyle-cn}

\begin{document}

\maketitle
\frontmatter

\tableofcontents

\mainmatter
\everymath{\displaystyle} % 让全文的行内公式都显示行间公式效果

\chapter{群论I——Group Theorey I}

\section{幺半群}

\begin{definition}[代数运算/二元运算定义]\label{definition:代数运算/二元运算定义}
设\(A\)是一个非空集合,若对\(A\)中任意两个元素\(a,b\),通过某个法则“\(\cdot\)”,有\(A\)中唯一确定的元素\(c\)与之对应,则称法则“\(\cdot\)”为集合\(A\)上的一个\textbf{代数运算(algebraic operation)或二元运算}.元素\(c\)是\(a,b\)通过运算“\(\cdot\)”作用的结果,将此结果记为\(a \cdot b = c\).
\end{definition}

\begin{definition}[(交换)半群定义]\label{definition:(交换)半群定义}
非空集合\(S\)和\(S\)上满足结合律的二元运算\(\cdot\)所形成的代数结构叫做\textbf{半群}.这个半群记成\((S,\cdot)\)或者简记成\(S\),运算\(x\cdot y\)也常常简写成\(xy\).此外,如果半群\((S,\cdot)\)中的运算“$\cdot$”又满足交换律,则\((S,\cdot)\)叫做\textbf{交换半群}.
\end{definition}
\begin{remark}
    像通常那样令\(x^2 = x\cdot x\),\(x^{n + 1} = x^n\cdot x( = x\cdot x^n, n\geq1)\).
\end{remark}

\begin{definition}[幺元素定义]\label{definition:幺元素定义}
    设\(S\)是半群,元素\(e\in S\)叫做半群\(S\)的\textbf{幺元素(也叫单位元(unit element)或恒等元(identity))},是指对每个\(x\in S\),\(xe = ex = x\).
\end{definition}
\begin{note}
    \textbf{如果半群$\boldsymbol{S}$中有幺元素,则幺元素一定唯一.}因若$e'$也是幺元素,则$e'=e'e=e$.我们将半群$S$中这个唯一的幺元素(如果存在的话)通常记作$\boldsymbol{1_S}$\textbf{或者1}.
\end{note}

\begin{definition}[(交换)含幺半群定义]\label{definition:(交换)幺半群定义}
如果半群\((S,\cdot)\)含有幺元素,则\((S,\cdot)\)叫做\textbf{含幺半群}.此外,如果幺半群\((S,\cdot)\)中的运算“$\cdot$”又满足交换律,则\((S,\cdot)\)叫做\textbf{交换幺半群}.
\end{definition}

\begin{definition}
设\((S,\cdot)\)是含幺半群. 元素\(y\in S\)叫做元素\(x\in S\)的\textbf{逆元素},是指\(xy = yx = 1\).
\end{definition}
\begin{note}
\textbf{如果\(\boldsymbol{x}\)有逆元素,则它一定唯一.} 因为若\(y'\)也是\(x\)的逆元素,则\(xy' = y'x = 1\).于是$y = y\cdot 1 = y(xy') = (yx)y' = 1\cdot y' = y'.$所以,若\(x\)具有逆元素,我们把这个唯一的逆元素记作\(\boldsymbol{x^{-1}}\),则\(xx^{-1} = x^{-1}x = 1\).
\end{note}

\begin{definition}
如果含幺半群\((G,\cdot)\)的每个元素均可逆,则\((G,\cdot)\)叫做\textbf{群}. 此外,如果群\((G,\cdot)\)中的运算“$\cdot$”又满足交换律,则\(G\)叫做\textbf{交换群}或叫\textbf{阿贝尔(Abel)群}.
\end{definition}




\begin{note}
    容易验证$\left( M_n\left( \mathbb{R} \right) ,\cdot \right)$是(加法)交换幺半群,其中单位元是零矩阵.
\end{note}

\begin{example}
$\left( M_n\left( \mathbb{R} \right) ,\cdot \right)$是一个含幺(乘法)半群.
\end{example}
\begin{proof}
\(\forall A,B,C\in (M_n(\mathbb{R}),\cdot)\),则不妨设\(A=(a_{ij})_{n\times n}\),\(B=(b_{ij})_{n\times n}\),\(C=(c_{ij})_{n\times n}\).再设\(A\cdot B=(d_{ij})_{n\times n}\),\(B\cdot C=(e_{ij})_{n\times n}\),\((A\cdot B)\cdot C=(f_{ij})_{n\times n}\),\(A\cdot (B\cdot C)=(g_{ij})_{n\times n}\).于是
\[
d_{ij}=\sum_{k = 1}^n{a_{ik}b_{kl}},e_{ij}=\sum_{k = 1}^n{b_{ik}c_{kl}}.
\]
其中\(i,j = 1,2,\cdots,n\).

从而
\[
f_{ij}=\sum_{l = 1}^n{d_{il}c_{lj}}=\sum_{l = 1}^n{\left(\sum_{k = 1}^n{a_{ik}b_{kl}}\right)\cdot c_{lj}}=\sum_{l = 1}^n{\sum_{k = 1}^n{a_{ik}b_{kl}c_{lj}}},
\]
\[
g_{ij}=\sum_{k = 1}^n{a_{ik}e_{kj}}=\sum_{k = 1}^n{a_{ik}\cdot\left(\sum_{l = 1}^n{b_{kl}c_{lj}}\right)}=\sum_{k = 1}^n{\sum_{l = 1}^n{a_{ik}b_{kl}c_{lj}}}.
\]
由二重求和号的可交换性,可知\(f_{ij}=g_{ij}\),\(\forall i,j\in \{1,2,\cdots,n\}\).故\((A\cdot B)\cdot C = A\cdot (B\cdot C)\).

记\(I_n=\begin{pmatrix}
    1 & & & \\
    & 1 & & \\
    & & \ddots & \\
    & & & 1\\
    \end{pmatrix}\in M_n(\mathbb{R})\),于是\(\forall X\in M_n(\mathbb{R})\),则不妨设\(X=(x_{ij})_{n\times n}\),\(I_n = (\delta_{ij})_{n\times n}\).其中\(\delta_{ij}=\begin{cases}
    1,\text{当 }i = j\text{ 时},\\
    0,\text{当 }i\neq j\text{ 时}
    \end{cases}\).
    再设\(I_n\cdot X=(x_{ij}')_{n\times n}\),\(X\cdot I_n=(x_{ij}'')_{n\times n}\),于是由矩阵乘法的定义可知
    \[
    x_{ij}'=\sum_{k = 1}^n{x_{ik}\delta_{kj}}=x_{ij}\delta_{jj}=x_{ij}.
    \]
    \[
    x_{ij}''=\sum_{k = 1}^n{\delta_{ik}x_{kj}}=\delta_{ii}x_{ij}=x_{ij}.
    \]
    故\(x_{ij}'=x_{ij}''=x_{ij}\),\(\forall i,j\in \{1,2,\cdots,n\}\).从而\(X = I_n\cdot X = X\cdot I_n\).因此\(I_n\)是\((M_n(\mathbb{R}),\cdot)\)的单位元.综上所述,$\left( M_n\left( \mathbb{R} \right) ,\cdot \right)$是一个含幺(乘法)半群.
\end{proof}


\begin{definition}
    
\end{definition}



\end{document}

\documentclass[lang=cn,newtx,10pt,scheme=chinese]{../Template/elegantbook}

\title{数学分析和高等代数杂题}
\subtitle{数学分析、高等代数习题}

\author{邹文杰}
\institute{无}
\date{2024/10/25}
\version{ElegantBook-4.5}
\bioinfo{自定义}{信息}

\extrainfo{宠辱不惊,闲看庭前花开花落;
\\
去留无意,漫随天外云卷云舒.}


\setcounter{tocdepth}{3}

\logo{logo-blue.png}
\cover{cover.png}

% 本文档额外使用的宏包和命令
\usepackage{../Styles/mystyle-elegantbook}

\begin{document}

\maketitle
\frontmatter

\tableofcontents

\mainmatter% 将行为改回预期版本,并重置页码


\chapter{数学分析习题}

\begin{lemma}[Riemann 引理] \label{thm:Riemann}
若$f\in R\left[ a,b \right] $,$g$以$T$为周期且在$\left[ 0,T \right] $上可积,则有
\begin{equation}
\underset{p\rightarrow +\infty}{\lim}\int_a^b{f\left( x \right) g\left( px \right) dx=\frac{1}{T}}\int_0^T{g\left( x \right) dx}\int_a^b{f\left( x \right) dx}
\end{equation}
\end{lemma}

\begin{theorem}[积分第一中值定理的推广] \label{thm:JFDYZZ}
若$f,g\in R\left[ a,b \right] $,
其中$f$在$\left[ a,b \right] $上有原函数,
$g$在$\left[ a,b \right] $上不变号,
则存在$\xi \in \left( a,b \right) $,使
\begin{equation}
\int_a^b{f\left( x \right) g\left( x \right) dx}=f\left( \xi \right) \int_a^b{g\left( x \right) dx}
\end{equation}
\end{theorem}

\begin{proposition} \label{pro:YZLXHS}
若$f$在$\left( -\infty ,+\infty \right) $上一致连续,则存在非负常数$a$和$b$,
使得成立$\left| f\left( x \right) \right|\leqslant a\left| x \right|+b$.
\end{proposition}

\begin{proposition} \label{pro:HSYZLXCYTJ}
函数$f$在区间$I$上一致连续的充分必要条件是:
对任何满足条件$\underset{n\rightarrow +\infty}{\lim}\left( x_n-y_n \right) =0$的
$\left\{ x_n \right\} \subset I$和$\left\{ y_n \right\} \subset I$,
都有$\underset{n\rightarrow +\infty}{\lim}\left[ f\left( x_n \right) -f\left( y_n \right) \right] =0$.
\end{proposition}


\newpage


\begin{example}
设$f$在$\left[ 0,+\infty \right)$的任意闭区间上Riemann可积.对于$x\geqslant 0$,定义$F\left( x \right) =\int_0^x{t^{\alpha}}f\left( t+x \right) dt$.
\begin{enumerate}[(1)]
\item 若$\alpha \in \left( -1,0 \right)$ 且$\lim_{x\rightarrow +\infty} f\left( x \right) =A$,证明:$F$在$\left[ 0,+\infty \right) $上一致连续.

\item 若$\alpha \in \left( 0,1 \right)$,$f$以$T>0$为周期,$\int_0^3{f\left( t \right) dt}=2022$.证明:$F$在$\left[ 0,+\infty \right) $上非一致连续.
\end{enumerate}
\end{example}
\begin{note}
本题(1)中的$\lim_{x\rightarrow +\infty} f\left( x \right) =A$可以削弱为$\exists M>0,\left| f\left( x \right) \right|\leqslant M,x\in \left[ 0,+\infty \right) $.
\end{note}

\begin{proof}
(1)
由于$\lim_{x\rightarrow +\infty} f\left( x \right) =A$,所以
$\exists M>0,\left| f\left( x \right) \right|\leqslant M,x\in \left[ 0,+\infty \right) $.

取$\delta =\left[ \frac{\left( \alpha +1 \right) \varepsilon}{3M} \right] ^{\frac{1}{\alpha +1}}$
,任取$y>x \geqslant 0$,且$\left| y-x \right|<\delta $有
\begin{align}
&\left| F\left( y \right) -F\left( x \right) \right|=\left| \int_0^y{t^{\alpha}}f\left( t+y \right) dt-\int_0^x{t^{\alpha}}f\left( t+x \right) dt \right| \nonumber
\\
&=\left| \int_y^{2y}{\left( t-y \right) ^{\alpha}}f\left( t \right) dt-\int_x^{2x}{\left( t-x \right) ^{\alpha}}f\left( t \right) dt \right| \nonumber
\\
&\leqslant \left| \int_y^{2y}{\left( t-y \right) ^{\alpha}}f\left( t \right) dt-\int_y^{2y}{\left( t-x \right) ^{\alpha}}f\left( t \right) dt+\int_y^{2y}{\left( t-x \right) ^{\alpha}}f\left( t \right) dt-\int_x^{2x}{\left( t-x \right) ^{\alpha}}f\left( t \right) dt \right| \nonumber
\\
&\leqslant \left| \int_y^{2y}{\left( t-y \right) ^{\alpha}}f\left( t \right) dt-\int_y^{2y}{\left( t-x \right) ^{\alpha}}f\left( t \right) dt \right|+\left| \int_y^{2y}{\left( t-x \right) ^{\alpha}}f\left( t \right) dt-\int_x^{2x}{\left( t-x \right) ^{\alpha}}f\left( t \right) dt \right| \nonumber
\\
&\leqslant M\int_y^{2y}{\left| \left( t-y \right) ^{\alpha}-\left( t-x \right) ^{\alpha} \right|}dt+\left| \int_{2x}^{2y}{\left( t-x \right) ^{\alpha}}f\left( t \right) dt-\int_x^y{\left( t-x \right) ^{\alpha}}f\left( t \right) dt \right| \nonumber
\\
&\leqslant M\int_y^{2y}{\left[ \left( t-y \right) ^{\alpha}-\left( t-x \right) ^{\alpha} \right]}dt+M\left| \int_{2x}^{2y}{\left( t-x \right) ^{\alpha}}dt \right|+M\left| \int_x^y{\left( t-x \right) ^{\alpha}}dt \right| \nonumber
\\
&\leqslant M\int_y^{2y}{\left[ \left( t-y \right) ^{\alpha}-\left( t-x \right) ^{\alpha} \right]}dt+M\int_x^{2y-x}{t^{\alpha}}dt+M\int_0^{y-x}{t^{\alpha}}dt \nonumber
\\
&=M\left[ \int_y^{2y}{\left[ \left( t-y \right) ^{\alpha}-\left( t-x \right) ^{\alpha} \right]}dt+\int_x^{2y-x}{t^{\alpha}}dt+\int_0^{y-x}{t^{\alpha}}dt \right]  \nonumber
\\
&=M\left[ \frac{y^{\alpha +1}-\left( 2y-x \right) ^{\alpha +1}+\left( y-x \right) ^{\alpha +1}}{\alpha +1}+\int_x^{2y-x}{t^{\alpha}}dt+\int_0^{y-x}{t^{\alpha}}dt \right]  \nonumber
\\
&=M\left[ \int_0^{y-x}{t^{\alpha}}dt-\int_y^{2y-x}{t^{\alpha}}dt+\int_x^{2y-x}{t^{\alpha}}dt+\int_0^{y-x}{t^{\alpha}}dt \right]  \nonumber
\\
&=M\left[ 2\int_0^{y-x}{t^{\alpha}}dt+\int_x^y{t^{\alpha}}dt \right]  \nonumber
\\
&=M\left[ 2\int_0^{y-x}{t^{\alpha}}dt+\int_0^{y-x}{\left( t-x \right) ^{\alpha}}dt \right]  \nonumber
\\
&=\frac{M}{\alpha +1}\left[ 2\left( y-x \right) ^{\alpha +1}+\left( y-2x \right) ^{\alpha +1} \right]  \nonumber
\\
&\leqslant \frac{3M}{\alpha +1}\left( y-x \right) ^{\alpha +1}<\frac{3M}{\alpha +1}\delta ^{\alpha +1}<\varepsilon  \nonumber
\end{align}


因此,$F$在$\left[ 0,+\infty \right) $上一致连续.

(2)
假设$F(x)$在$\left[ 0,+\infty \right) $上一致连续.
那么存在$a,b>0$,使得$F\left( x \right) <a\left| x \right|+b$.(见命题\ref{pro:YZLXHS})

从而$\left| \frac{F\left( x \right)}{x^{\alpha +1}} \right|<\frac{a\left| x \right|+b}{\left| x \right|^{\alpha +1}}$,
进而$\lim_{x\rightarrow +\infty} \frac{F\left( x \right)}{x^{\alpha +1}}=0$.

于是
\begin{equation}
\begin{split}
\lim_{x\rightarrow +\infty} \frac{F\left( x \right)}{x^{\alpha +1}}&=\frac{\int_0^x{t^{\alpha}}f\left( t+x \right) dt}{x^{\alpha +1}}\xlongequal{\text{换元}}\frac{\int_x^{2x}{\left( t-x \right) ^{\alpha}}f\left( t \right) dt}{x^{\alpha +1}}\xlongequal{\text{换元}}\frac{\int_1^2{x^{\alpha +1}\left( t-1 \right) ^{\alpha}}f\left( tx \right) dt}{x^{\alpha +1}}
\\
&\xlongequal{\text{黎曼定理\ref{thm:Riemann}}}\int_1^2{\left( t-1 \right) ^{\alpha}}f\left( tx \right) dt=\frac{1}{T}\int_0^T{f\left( x \right)}dt\int_1^2{\left( t-1 \right) ^{\alpha}}dt=0    
\end{split}
\nonumber
\end{equation}
再结合$\int_1^2{\left( t-1 \right) ^{\alpha}}dt>0$,
知$\int_0^T{f\left( x \right)}dt=0$.

现在有
\begin{equation}
\begin{split}
F\left( x \right) &=\int_0^x{t^{\alpha}f\left( t+x \right)}dt=\int_0^x{t^{\alpha}d}\left[ \int_0^{x+t}{f\left( y \right) dy} \right] 
\\
&\xlongequal{\text{分部积分}}x^{\alpha}\int_0^{2x}{f\left( y \right) dy}-\alpha \int_0^x{t^{\alpha -1}\left[ \int_0^{x+t}{f\left( y \right) dy} \right]}dt
\\
&=x^{\alpha}\int_0^{2x}{f\left( y \right) dy}-\alpha \int_0^x{t^{\alpha -1}F\left( x+t \right)}dt
\end{split}
\nonumber
\end{equation}

设$G(x)=\int_0^x{f\left( x \right)}dt$,则由$f$在$\left[ 0,+\infty \right)$的任意闭区间上Riemann可积知,
$G\in C\left[ 0,+\infty \right) $.
又由$\int_0^T{f\left( x \right)}dt=0$,得
\begin{equation}
\begin{split}
G\left( x+T \right) -G\left( x \right) =\int_0^x{f\left( x+T \right)}dt-\int_0^x{f\left( x \right)}dt=\int_x^{x+T}{f\left( x \right)}dt=\int_0^T{f\left( x \right)}dt=0
\end{split}
\nonumber
\end{equation}
因为连续的周期函数必有界,所以$G(x)$有界.

又$\alpha -1\in \left( -1,0 \right) $,
故由(1)可得,$-\alpha \int_0^x{t^{\alpha -1}F\left( x+t \right)}dt$
在$\left[ 0,+\infty \right) $上一致连续.

下面证明$x^{\alpha}\int_0^{2x}{f\left( y \right) dy}$
不一致连续.

由于$G(2x)$在$\left[ 0,\frac{T}{2} \right] $上连续,所以由连续函数最大、最小值定理知

记$M=\underset{x\in \left[ 0,\frac{T}{2} \right]}{\max}G\left( 2x \right) $,则存在$x_2\in \left[ 0,\frac{T}{2} \right] $,
使得$M=G\left( 2x_2 \right) \geqslant G\left( 2x \right) ,x\in \left[ 0,\frac{T}{2} \right] $.

又因为$G(3)=\int_0^3{f\left( t \right) dt}=2022$,且$G(2x)$以$\frac{T}{2}$为周期
,所以存在$x_1\in \left[ 0,\frac{T}{2} \right] $,使得$G(2x_1)=G(3)>0$.

因此,$M=G{\left( 2x_2 \right)}\geqslant G(2x_1)=G\left( 3 \right) =\int_0^3{f\left( t \right) dt}>0$.

构造数集$E=\left\{ x'\in \left[ 0,\frac{T}{2} \right] \mid G\left( 2x^{\prime} \right) =M \right\} $,
由$x_2\in E$知,
$E\ne \varnothing $.
又因为$E$有界,所以由确界存在定理知,$E$必有上确界,取$x_0=supE$.

假设$x_0\notin E$,取$\varepsilon _0=\frac{1}{2}\left| G\left( 2x_0 \right) -M \right|$,
则$\varepsilon _0>0$,否则$x_0\in E$矛盾.

从而$\forall \delta ^{\prime}>0$,$\exists x_{\delta'}\in E$,
使得$ x_0-\delta ^{\prime}<x_{\delta'}<x_0$,
都有$\left| G\left( 2x_0 \right) -G\left( 2x'_{\delta'} \right) \right|\geqslant \varepsilon _0$.

这与G(2x)在闭区间$\left[ 0,\frac{T}{2} \right]$上连续,进而一致连续矛盾.
故$x_0\in E$.

任取$\delta \in \left( 0,\frac{1}{2}\left( \frac{T}{2}-x_0 \right) \right) $,
则$G\left( 2x_0+\delta \right) <M=G\left( 2x_0 \right) $,
否则与$x_0=supE$矛盾.

进而$\left| \int_{2x_0}^{2x_0+\delta}{f\left( y \right) dy} \right|=\left| G\left( 2x_0+\delta \right) -G\left( 2x_0 \right) \right|>0$.

从而当$n>\left( \frac{2}{\delta} \right) ^{\frac{2}{\alpha}}$时,由积分中值定理,得

存在$\xi _n\in \left( 2x_0,2x_0+\frac{2}{n^{\frac{\alpha}{2}}} \right) $,使得
\begin{equation}\label{A}
\left| \int_{2x_0}^{2x_0+\frac{2}{n^{\frac{\alpha}{2}}}}{f\left( y \right) dy} \right|=\frac{2}{n^{\frac{\alpha}{2}}}\left| f\left( \xi _n \right) \right|>0
\end{equation}

又因为$f$在$\left[ 0,+\infty \right)$的任意闭区间上Riemann可积,
所以$f$在$\left( 2x_0,2x_0+\frac{2}{n^{\frac{\alpha}{2}}} \right) $
上有界.

于是存在$K,L> 0$,使得
\begin{equation}\label{B}
K\leqslant \left| f\left( \xi _n \right) \right|\leqslant L
\end{equation}

取数列$\left\{ x_n \right\} \text{、}\left\{ y_n \right\} $,
其中$x_n=x_0+n\frac{T}{2},y_n=x_0+n\frac{T}{2}+\frac{2}{n^{\frac{\alpha}{2}}},n\in \mathbb{N} _+$.
并且$\underset{n\rightarrow +\infty}{\lim}\left( x_n-y_n \right) =\underset{n\rightarrow +\infty}{\lim}\left( \frac{2}{n^{\frac{\alpha}{2}}} \right) =0$.

由拉格朗日中值定理,得
对$\forall n\in \mathbb{N} _+$,

存在$\xi _n\in \left( x_0+n\frac{T}{2},x_0+n\frac{T}{2}+\frac{2}{n^{\frac{\alpha}{2}}} \right) $,使得
$\left( x_0+n\frac{T}{2} \right) ^{\alpha}-\left( x_0+n\frac{T}{2}+\frac{2}{n^{\frac{\alpha}{2}}} \right) ^{\alpha}=\frac{2\alpha}{n^{\frac{\alpha}{2}}}{\xi _n}^{\alpha -1}$

从而
\begin{equation}
\begin{split}
\frac{2\alpha}{n^{\frac{\alpha}{2}}}\left( x_0+n\frac{T}{2} \right) ^{\alpha -1}\leqslant \frac{2\alpha}{n^{\frac{\alpha}{2}}}{\xi _n}^{\alpha -1}\leqslant \frac{2\alpha}{n^{\frac{\alpha}{2}}}\left( x_0+n\frac{T}{2}+\frac{2}{n^{\frac{\alpha}{2}}} \right) ^{\alpha -1}
\end{split}
\nonumber
\end{equation}
令$n\rightarrow +\infty$,有$\lim_{n\rightarrow +\infty} \left[ \left( x_0+n\frac{T}{2} \right) ^{\alpha}-\left( x_0+n\frac{T}{2}+\frac{2}{n^{\frac{\alpha}{2}}} \right) ^{\alpha} \right] =\lim_{n\rightarrow +\infty} \frac{2\alpha}{n^{\frac{\alpha}{2}}}{\xi _n}^{\alpha -1}=0$.

于是存在$N>0$,使得$\forall n>N$,有
\begin{equation}\label{C}
\left( x_0+n\frac{T}{2} \right) ^{\alpha}-\left( x_0+n\frac{T}{2}+\frac{2}{n^{\frac{\alpha}{2}}} \right) ^{\alpha}<\frac{\varepsilon}{\int_0^{2x_0}{f\left( y \right) dy}}
\end{equation}

现在,当$n>\max \left\{ N,\left( \frac{2}{\delta} \right) ^{\frac{2}{\alpha}} \right\} $时,
结合\eqref{A}\eqref{B}\eqref{C},我们有
\begin{align*}
&\left| {x_n}^{\alpha}\int_0^{2x_n}{f\left( y \right) dy}-{y_n}^{\alpha}\int_0^{2y_n}{f\left( y \right) dy} \right|
\\
&=\left| \left( x_0+n\frac{T}{2} \right) ^{\alpha}\int_0^{2\left( x_0+n\frac{T}{2} \right)}{f\left( y \right) dy}-\left( x_0+n\frac{T}{2}+\frac{2}{n^{\frac{\alpha}{2}}} \right) ^{\alpha}\int_0^{2\left( x_0+n\frac{T}{2}+\frac{2}{n^{\frac{\alpha}{2}}} \right)}{f\left( y \right) dy} \right|
\\
&=\left| \left( x_0+n\frac{T}{2} \right) ^{\alpha}\int_0^{2\left( x_0+n\frac{T}{2} \right)}{f\left( y \right) dy}-\left[ \left( x_0+n\frac{T}{2}+\frac{2}{n^{\frac{\alpha}{2}}} \right) ^{\alpha}\int_0^{2\left( x_0+n\frac{T}{2} \right)}{f\left( y \right) dy}+\left( x_0+n\frac{T}{2}+\frac{2}{n^{\frac{\alpha}{2}}} \right) ^{\alpha}\int_{2\left( x_0+n\frac{T}{2} \right)}^{2\left( x_0+n\frac{T}{2}+\frac{2}{n^{\frac{\alpha}{2}}} \right)}{f\left( y \right) dy} \right] \right|
\\
&=\left| \left[ \left( x_0+n\frac{T}{2} \right) ^{\alpha}-\left( x_0+n\frac{T}{2}+\frac{2}{n^{\frac{\alpha}{2}}} \right) ^{\alpha} \right] \int_0^{2\left( x_0+n\frac{T}{2} \right)}{f\left( y \right) dy}-\left( x_0+n\frac{T}{2}+\frac{2}{n^{\frac{\alpha}{2}}} \right) ^{\alpha}\int_{2\left( x_0+n\frac{T}{2} \right)}^{2\left( x_0+n\frac{T}{2}+\frac{2}{n^{\frac{\alpha}{2}}} \right)}{f\left( y \right) dy} \right|
\\
&\geqslant \left| \left( x_0+n\frac{T}{2}+\frac{2}{n^{\frac{\alpha}{2}}} \right) ^{\alpha}\int_{2\left( x_0+n\frac{T}{2} \right)}^{2\left( x_0+n\frac{T}{2}+\frac{2}{n^{\frac{\alpha}{2}}} \right)}{f\left( y \right) dy} \right|-\left| \left[ \left( x_0+n\frac{T}{2} \right) ^{\alpha}-\left( x_0+n\frac{T}{2}+\frac{2}{n^{\frac{\alpha}{2}}} \right) ^{\alpha} \right] \int_0^{2\left( x_0+n\frac{T}{2} \right)}{f\left( y \right) dy} \right|
\\
&=\left| \left( x_0+n\frac{T}{2}+\frac{2}{n^{\frac{\alpha}{2}}} \right) ^{\alpha}\int_{2x_0}^{2x_0+\frac{2}{n^{\frac{\alpha}{2}}}}{f\left( y \right) dy} \right|-\left| \left[ \left( x_0+n\frac{T}{2} \right) ^{\alpha}-\left( x_0+n\frac{T}{2}+\frac{2}{n^{\frac{\alpha}{2}}} \right) ^{\alpha} \right] \int_0^{2x_0}{f\left( y \right) dy} \right|
\\
&=\left| \left( x_0+n\frac{T}{2}+\frac{2}{n^{\frac{\alpha}{2}}} \right) ^{\alpha}\right|\cdot\left| \frac{2}{n^{\frac{\alpha}{2}}}f\left( \xi _n \right) \right|-\left| \left[ \left( x_0+n\frac{T}{2} \right) ^{\alpha}-\left( x_0+n\frac{T}{2}+\frac{2}{n^{\frac{\alpha}{2}}} \right) ^{\alpha} \right] \int_0^{2x_0}{f\left( y \right) dy} \right|
\\
&\geqslant 2\left( \frac{T}{2} \right) ^{\alpha}\left|f\left( \xi _n \right)\right| \cdot n^{\frac{\alpha}{2}}-\varepsilon 
\\
&\geqslant 2\left( \frac{T}{2} \right) ^{\alpha}K\cdot n^{\frac{\alpha}{2}}-\varepsilon 
\\
&\,\,\,\,\,\,\,\,\,\,\,\, \text{令}n\rightarrow +\infty ,\text{有}\underset{n\rightarrow +\infty}{\lim}\left( {x_n}^{\alpha}\int_0^{2x_n}{f\left( y \right) dy}-{y_n}^{\alpha}\int_0^{2y_n}{f\left( y \right) dy} \right) =+\infty .  
\text{故}x^{\alpha}\int_0^{2x}{f\left( y \right) dy}\text{在}\left[ 0,+\infty \right)\text{上非一致连续.}
\\
&\text{这与$F(x)$在$\left[ 0,+\infty \right) $上一致连续矛盾.因此,}F\text{在}\left[ 0,+\infty \right)\text{上非一致连续.}
\nonumber
\end{align*}
\end{proof}
\begin{remark}
最后一步证明非一致连续利用的是函数一致连续的充要条件\ref{pro:HSYZLXCYTJ}
\end{remark}

\begin{example}
证明:Riemann函数$R(x)$处处不可导.
\end{example}

\begin{proof}
因为$R(x)$在有理点处均不连续,所以$R(x)$在有理数点均不可导.

$\forall x_0\in \mathbb{R} /\mathbb{Q} ,\forall q\in \mathbb{N} _+,\exists p_q\in \mathbb{Z} ,$
使得$\frac{p_q}{q}<x_0<\frac{p_q+1}{q}$.

取有理数列${r_q},{s_q}$,其中$r_q=\frac{p_q}{q},s_q=\frac{p_q+1}{q}$,
$
\text{则}0<x_0-r_q<\frac{1}{q},0<s_q-x_0<\frac{1}{q}.\text{从而}\underset{q\rightarrow +\infty}{\lim}r_q=\underset{q\rightarrow +\infty}{\lim}s_q=x_0.
$

$
\text{假设}R\left( x \right) \text{在}x_0\text{处可导},\text{则由}Heine\text{归结原理及导数的定义知}
$
\begin{equation}
\begin{split}
\underset{p\rightarrow +\infty}{\lim}\frac{R\left( r_q \right) -R\left( x_0 \right)}{r_q-x_0}=\underset{p\rightarrow +\infty}{\lim}\frac{R\left( s_q \right) -R\left( x_0 \right)}{s_q-x_0}
\\
\text{即}\underset{p\rightarrow +\infty}{\lim}\left[ \frac{R\left( s_q \right) -R\left( x_0 \right)}{s_q-x_0}-\frac{R\left( r_q \right) -R\left( x_0 \right)}{r_q-x_0} \right] =0
\end{split}
\nonumber
\end{equation}
$
\text{又由}\forall q\in \mathbb{N} _+,r_q<x_0<s_q\text{得}
$
\begin{equation}
\frac{R\left( s_q \right) -R\left( x_0 \right)}{s_q-x_0}=\frac{\frac{1}{q}}{s_q-x_0}>\frac{\frac{1}{q}}{s_q-r_q}=\mathbf{1},\frac{R\left( r_q \right) -R\left( x_0 \right)}{r_q-x_0}=\frac{\frac{1}{q}}{r_q-x_0}<\frac{\frac{1}{q}}{r_q-s_q}=-1
\nonumber
\end{equation}
$
\text{于是}\forall q\in \mathbb{N} _+,\text{有}
$
\begin{equation}
\frac{R\left( s_q \right) -R\left( x_0 \right)}{s_q-x_0}-\frac{R\left( r_q \right) -R\left( x_0 \right)}{r_q-x_0}>1-\left( -1 \right) =2
\nonumber
\end{equation}
$
\text{这与}\underset{p\rightarrow +\infty}{\lim}\left[ \frac{R\left( s_q \right) -R\left( x_0 \right)}{s_q-x_0}-\frac{R\left( r_q \right) -R\left( x_0 \right)}{r_q-x_0} \right] =0\text{矛盾}.
\text{故}R\left( x \right) \text{在任意无理点处均不可导}.
\text{综上},\text{函数}R\left( x \right) \text{处处不可导}.
$

\end{proof}

\begin{example}
$f$在$\left[ 0,+\infty \right) $上单调递减趋于0,
$\int_0^{+\infty}{\sqrt{\frac{f(x)}{x}}dx}$收敛.

证明:$\int_0^{+\infty}{f\left( x \right) dx}$收敛
且$\int_0^{\infty}{f\left( x \right) dx}\leqslant \frac{\left( \int_0^{+\infty}{\sqrt{\frac{f\left( x \right)}{x}}dx} \right) ^2}{2}.$
\begin{proof}
不妨设$f(x)>0,x\in \left[ 0,+\infty \right)$.
否则存在$x_0\in \left[ 0,+\infty \right)$,
使得$f\left( x_0 \right) =0$.
由$f$在$\left[ 0,+\infty \right) $上单调递减趋于0知,
$ \forall x\in [0,+\infty)$,有$f(x)=0$.
从而$\int_0^{+\infty}{f\left( x \right) dx}=\int_0^{x_0}{f\left( x \right) dx}$,
于是$\int_0^{+\infty}{f\left( x \right) dx}$收敛.

由$\int_0^{+\infty}{\sqrt{\frac{f(x)}{x}}dx}$收敛及柯西收敛准则知,
$\forall \varepsilon>0,\exists M\geqslant 0,s.t. \forall A>4M,\text{有}
\int_{\frac{A}{4}}^A{\sqrt{\frac{f(x)}{x}}dx}<\varepsilon$

结合$f$在$\left[ 0,+\infty \right) $上单调递减趋于0知,
当$\forall A>4M$时,有
\begin{equation}
\begin{split}
0<\sqrt{Af(A)}=\sqrt{f(A)}\int_{\frac{A}{4}}^A{\frac{1}{\sqrt{x}}dx<}\int_{\frac{A}{4}}^A{\sqrt{\frac{f(x)}{x}}dx<}\varepsilon 
\end{split}
\nonumber
\end{equation}
由迫敛性知,$\underset{x\rightarrow +\infty}{\lim}\sqrt{xf(x)}=0$.
又因为$\underset{x\rightarrow +\infty}{\lim}\frac{f(x)}{\sqrt{\frac{f(x)}{x}}}=\underset{x\rightarrow +\infty}{\lim}\sqrt{xf(x)}=0$,
所以根据比较原则知,$\int_0^{+\infty}{f\left( x \right) dx}$收敛.

利用$f$的单调性知:$\forall x \geqslant 0$有
\begin{equation}
\begin{split}
\int_0^{+\infty}{\sqrt{\frac{f(x)}{x}}dx}\geqslant \int_0^x{\sqrt{\frac{f(t)}{t}}dt}\geqslant \sqrt{\frac{f(x)}{x}}\int_0^x{1dt}=\sqrt{xf\left( x \right)}
\end{split}
\nonumber
\end{equation}

从而
\begin{equation}
\begin{split}
&\int_0^{+\infty}{f\left( x \right) dx=}\int_0^{+\infty}{\sqrt{xf\left( x \right)}\cdot \sqrt{\frac{f\left( x \right)}{x}}dx}\leqslant \int_0^{+\infty}{\left( \int_0^x{\sqrt{\frac{f\left( t \right)}{t}}dt} \right) \sqrt{\frac{f\left( x \right)}{x}}dx}
\\
&=\int_0^{+\infty}{\left( \int_0^x{\sqrt{\frac{f\left( t \right)}{t}}dt} \right) d\left( \int_0^x{\sqrt{\frac{f\left( t \right)}{t}}dt} \right)}=\frac{\left( \int_0^x{\sqrt{\frac{f\left( t \right)}{t}}dt} \right) ^2}{2}\Bigg|_{0}^{+\infty}=\frac{\left( \int_0^{+\infty}{\sqrt{\frac{f\left( t \right)}{t}}dt} \right) ^2}{2}
\end{split}
\nonumber
\end{equation}
\end{proof}
\end{example}

\begin{example}
设可导函数$f(x)$定义域为$R$,$F(0)=0$,
并且当$\left| f\left( x \right) \right|\in \left( 0,\frac{1}{2} \right) $
时总是成立
\begin{equation}
\left| f^{\prime}\left( x \right) \right|\leqslant \left| f\left( x \right) \right|\left| \ln f\left( x \right) \right|
\nonumber
\end{equation}
证明:$f(x)$恒为零.
\end{example}
\begin{note}
一道不常规的函数性态分析题
\end{note}
\begin{proof}
(反证法)假设存在一点$x_0\ne 0$,
使得$f(x_0)\ne 0$.
不妨设$x_0> 0$,且$f(x_0)>0$.
由$f$的连续性及$f(0)=0$知,存在$t_0\in(0,x_0)$,
使得$\forall x\in(t_0,x_0)$,
有$f(x)>0$.

构造数集$E=\left\{ t\in [0,x_0)\mid f\left( x \right) >0,x\in \left( t,x_0 \right) \right\} $,
又因为$t_0\in E$,所以$E \ne \varnothing $.

从而由确界存在定理知,$E$存在下确界,
设$t_1=inf E$,则$\forall x\in (t_1,x_0)$,
有$f(x)>0$且$f(t_1)=0$.

若$f(t_1)\ne 0$,则当$f(t_1)>0$时,由$f$的连续性可得,
存在$t_{\varepsilon_1}<t_1$,使得$f(t_{\varepsilon_1})>0$.
与$t_1=inf E$矛盾.
当$f(t_1)<0$时,由$f$的连续性可知,
存在$t_{\varepsilon_2}>t_1$,
使得$\forall t\in(t_1,t_{\varepsilon_2}),f(t)<0$.
由$t_1=inf E$可得,
存在$t_1^{\prime}\in (t_1,t_{\varepsilon_2})$,
使得$f(t_1^{\prime})>0$.这与$\forall t\in(t_1,t_{\varepsilon_2}),f(t)<0$矛盾.
故$f(t_1)=0$.

根据$f$的连续性,一定存在$t_2\in(t_1,x_0)$且$\left| t_1-t_2 \right|<1$,
使得$\forall x\in (t_1,t_2),f(x)\in(0,\frac{1}{2})$.

现在我们在开区间$(t_1,t_2)$中考虑问题.

设$g(x)=\ln f(x)$,则有
\begin{equation}
g^{\prime}(x)=\frac{f^{\prime}(x)}{f(x)}
,\lim_{x\rightarrow {t_1}^{+}} g\left( x \right) =-\infty ,g\left( x \right) \in \left( -\infty ,-\ln 2 \right) 
\nonumber
\end{equation}
根据已知条件,有
\begin{equation}\label{1111}
\begin{split}
\left| g^{\prime}\left( x \right) \right|=\left| \frac{f^{\prime}\left( x \right)}{f\left( x \right)} \right|<\left| \ln f\left( x \right) \right|=\left| g\left( x \right) \right|
\end{split}
\end{equation}
再设$h(x)=\ln g^2(x)$,则
\begin{equation}
\lim_{x\rightarrow {t_1}^{+}} h\left( x \right) =+\infty
,h^{\prime}(x)=2\frac{g^{\prime}(x)}{g(x)}
\nonumber
\end{equation}
再结合\eqref{1111},对$\forall x\in(t_1,t_2)$,有
\begin{equation}
\left| h^{\prime}(x) \right|=2\left| \frac{g^{\prime}(x)}{g(x)} \right|<2
\nonumber
\end{equation}
任取$x_1\in(t_1,t_2)$,
又由$\lim_{x\rightarrow {t_1}^{+}} h\left( x \right) =+\infty$知,
存在$x_2\in (t_1,t_2)$,使得$h(x_2)-h(x_1)>3$.

但是根据拉格朗日中值定理可知,
存在$\xi \in \left( \min \left\{ x_1,x_2 \right\} ,\max \left\{ x_1,x_2 \right\} \right)$,
使得
\begin{equation}
\left| h\left( x_2 \right) -h\left( x_1 \right) \right|=\left| h^{\prime}\left( \xi \right) \right|\left| x_2-x_1 \right|\le \left| h^{\prime}\left( \xi \right) \right|\le 2
\nonumber
\end{equation}
这与$h(x_2)-h(x_1)>3$矛盾.
故$f(x)$恒为零.
\end{proof}

\begin{example}
设\(f(x)\)在闭区间\([0,1]\)上二阶可导,且\(f(0)=1\),\(f(1)=1 + e\),\(f''(0)=1\),证明:存在\(\xi\in(0,1)\),使得
\begin{align*}
f''(\xi)-2f'(\xi)+f(\xi)=1.
\nonumber
\end{align*}
\end{example}
\begin{note}
考虑微分方程$y''-2y'+y=1$,利用欧拉待定指数函数法求解得:$y=(C_1+C_2x)e^x+1$.
从而$[(y-1)e^{-x}]''=(C_1+C_2x)''=0$.于是我们构造辅助函数$g(x)=(f(x)-1)e^{-x}$.
再结合中值定理并利用题目条件就能得到证明.
\end{note}
\begin{proof}
令$g(x)=(f(x)-1)e^{-x}$,则$g(0)=0,g(1)=1$.
从而由$Lagrange$中值定理可知,存在$\eta\in(0,1)$,使得
\begin{align*}
g'(\eta)=\frac{g(1)-g(0)}{1-0}=1.
\nonumber
\end{align*}
又因为$g'(x)=\frac{f'\left( x \right) -f\left( x \right) +1}{e^x}$,所以$g'(0)=1$.
因此,根据$Rolle$中值定理可知,存在$\xi\in(0,1)$,使得
\begin{align*}
g''\left( \xi \right) =\frac{f''\left( \xi \right) -2f'\left( \xi \right) +f\left( \xi \right)-1}{e^{\xi}}=0.
\nonumber
\end{align*}
故原结论得到证.
\end{proof}
\begin{conclusion}
设$f(x)\in D^n\left[ a,b \right]$,且已知$f(x)$在某些特殊点处的函数值及导数值.证明:存在$\xi\in(a,b)$,使得
\begin{align*}
\varphi \left( C,\xi ,f\left( \xi \right) ,f'\left( \xi \right) ,\cdots ,f^{\left( n \right)}\left( \xi \right) \right) =0,\text{其中}C\text{为常数}.
\nonumber
\end{align*}
解决这类问题的常用方法是:
先通过解微分方程来找到需要构造的辅助函数,再结合中值定理并利用题目已知$f(x)$在某些特殊点处的函数值及导数值就能得到证明.

\hypertarget{通过解微分方程来找到需要构造的辅助函数的步骤}{\textbf{通过解微分方程来找到需要构造的辅助函数的步骤}}:

\textbf{Step1:}考虑微分方程$\varphi \left( C,x,y,y',\cdots ,y^{\left( n \right)} \right) =0$,利用求解常微分方程的方法求解$\varphi \left( C,x,y,y',\cdots ,y^{\left( n \right)} \right) =0$的通解.解得:$y=f\left( x,C_1,\cdots ,C_k \right)$,其中$C_1,\cdots,C_k$均为任意常数.

\textbf{Step2:}从上述通解$y=f\left( x,C_1,\cdots ,C_k \right)$中,通过移项化简找出$g(x)$使得$g\left( x \right) =h\left( x,C_1,\cdots ,C_k \right)$,并且$g^{\left( n \right)}\left( x \right) =h^{\left( n \right)}h\left( x,C_1,\cdots ,C_k \right) =0$.
(一般题目中都可以得到$h\left( x,C_1,\cdots ,C_k \right)\in P_{n-1}[x]$,从而$h$自然满足$h^n(x,C_1,\cdots ,C_k)=0$).

\textbf{Step3:}上述得到的$g(x)$就是我们需要的辅助函数.
\end{conclusion}
\begin{remark}
这种通过解微分方程来找到需要构造的辅助函数的方法基本上都能解决这类问题.在这类问题中,难题的难点一般就在于如何解出微分方程.
\end{remark}

\begin{example}
$f$是$[0,1]$上非负递增连续函数对$0 < \alpha < \beta < 1,$证明:
\begin{gather}
\int_{0}^{1} f(x) dx \geq \frac{1 - \alpha}{\beta - \alpha} \int_{\alpha}^{\beta} f(x) dx.
\nonumber
\end{gather}
\end{example}
\begin{proof}
令$\chi _{\left[ \alpha ,1 \right]}\left( x \right) =\begin{cases}
1,x\in \left[ \alpha ,\beta \right]\\
0,x\in \left( \beta ,1 \right]\\
\end{cases}$,则$\chi _{\left[ \alpha ,1 \right]}\left( x \right)$在$[\alpha,1]$单调递减.由\hyperref[Basis of Analytics-Chebeshev不等式积分形式]{Chebeshev不等式积分形式}可得
\begin{align*}
\left( \beta -\alpha \right) \int_{\alpha}^1{f(x)dx}=\int_{\alpha}^1{f(x)dx\int_{\alpha}^1{\chi _{\left[ \alpha ,1 \right]}\left( x \right) dx}}\geqslant \left( 1-\alpha \right) \int_{\alpha}^1{f\left( x \right) \chi _{\left[ \alpha ,1 \right]}\left( x \right) dx}=\left( 1-\alpha \right) \int_{\alpha}^{\beta}{f\left( x \right) dx}.
\nonumber
\end{align*}
从而
\begin{align*}
\int_{\alpha}^1{f(x)dx}\geqslant \frac{1-\alpha}{\beta -\alpha}\int_{\alpha}^{\beta}{f\left( x \right) dx}.
\nonumber
\end{align*}
于是
\begin{align*}
\int_0^1{f(x)dx}\geqslant \int_{\alpha}^1{f(x)dx}\geqslant \frac{1-\alpha}{\beta -\alpha}\int_{\alpha}^{\beta}{f\left( x \right) dx}.
\nonumber
\end{align*}
\end{proof}
\begin{remark}
实际上,$\frac{1-\alpha}{\beta -\alpha}$已经是本题不等式的最佳系数.证明如下:

令$f_n\left( x \right) =\begin{cases}
1,x\in \left[ a+\frac{1}{n},1 \right]\\
n\left( x-a \right) ,x\in \left( a,a+\frac{1}{n} \right)\\
0,x\in \left[ 0,a \right]\\
\end{cases}$,则显然$f_n(x)\in C[0,1]$.

从而
\begin{align*}
\int_0^1{f_n\left( x \right) dx}=\frac{1}{2n}+\left( 1-\alpha -\frac{1}{n} \right) \rightarrow 1-\alpha \left( n\rightarrow +\infty \right) ,
\\
\int_{\alpha}^{\beta}{f_n\left( x \right) dx}=\frac{1}{2n}+\left( \beta -\alpha -\frac{1}{n} \right) \rightarrow \beta -\alpha \left( n\rightarrow +\infty \right) .
\nonumber
\end{align*}
于是当$n$充分大时,取$f(x)=f_n(x)$,则此时不等式的等号成立.

故不等式系数$\frac{1-\alpha}{\beta -\alpha}$不可改进.
\end{remark}

\begin{example}
证明:$\sum_{n=0}^{\infty}{\int_0^x{t^n\sin \left( \pi t \right) dt}}$在$\left[ 0,1 \right] $上一致收敛.
\end{example}
\begin{note}
{\color{blue}\text{证法一思路分析:}}
我们首先想到利用$Weierstrass$判别进行放缩证明,运用常规的放缩想法,得到初步放缩的结果
\begin{align*}
\left| \int_0^x{t^n\sin \left( \pi t \right) dt} \right|=\int_0^x{t^n\sin \left( \pi t \right) dt}\leqslant \int_0^1{t^n\sin \left( \pi t \right) dt}\leqslant \int_0^1{t^ndt}=\frac{1}{n+1}.
\end{align*}
对$\forall t\in [0,1],n\in \mathbb{N}$,固定$t,n$.有$\int_0^x{t^n\sin \left( \pi t \right) dt}$关于$x$单调递增,故$\int_0^1{t^n\sin \left( \pi t \right) dt}$是$\int_0^x{t^n\sin \left( \pi t \right) dt}$的上确界,因此第一个不等式已经放缩到最精细的程度.
根据$Weierstrass$判别法可知,我们只需要证明第一个不等号右边式子作为通项的级数$\sum_{n=0}^{\infty}{\int_0^1{t^n\sin \left( \pi t \right) dt}}$收敛即可.而我们知道级数的敛散性是由其通项的阶决定的,于是原命题可转化为估计$\int_0^1{t^n\sin \left( \pi t \right) dt}$的阶.由上述初步放缩得到的不等式可知$\int_0^1{t^n\sin \left( \pi t \right) dt}$一定大于等于一阶,但这样的初步放缩并不能直接由比较判别法得到原级数收敛.因此我们需要更加精确的估计$\int_0^1{t^n\sin \left( \pi t \right) dt}$的阶.

现在我们来估计$\int_0^1{t^n\sin \left( \pi t \right) dt}$的阶.
(注意这里并不是严谨的证明!只是$laplace$估阶的大致思路框架.)

对这类积分估阶我们想到\hyperlink{Laplace估阶方法}{Laplace估阶方法}.取充分小的$\delta_1,\delta_2$(注意:在严谨的证明中,这里的$\delta_1,\delta_2$是待定的,需要我们根据后续的放缩、$Taylor$公式以及其他处理去确定其存在性),然后对$\int_0^1{t^n\sin \left( \pi t \right) dt}$的积分区间进行分段估阶得到
\begin{align}\label{eq:分段估计积分的阶1.7}
\int_0^1{t^n\sin \left( \pi t \right) dt}=\int_0^{\delta _1}{t^n\sin \left( \pi t \right) dt}+\int_{\delta _1}^{1-\delta _2}{t^n\sin \left( \pi t \right) dt}+\int_{1-\delta _2}^1{t^n\sin \left( \pi t \right) dt}=\int_{1-\delta _2}^1{t^n\sin \left( \pi t \right) dt}+O\left( \int_{1-\delta _2}^1{t^n\sin \left( \pi t \right) dt} \right). 
\end{align}
其中第二个等号是因为:从直觉上来说,我们可以认为
\begin{align*}
\int_0^{\delta _1}{t^n\sin \left( \pi t \right) dt}+\int_{\delta _1}^{1-\delta _2}{t^n\sin \left( \pi t \right) dt}+\int_{1-\delta _2}^1{t^n\sin \left( \pi t \right) dt}\approx \int_0^{\delta _1}{0^n\cdot 0dt}+\int_{\delta _1}^{1-\delta _2}{c^n\cdot kdt}+\int_{1-\delta _2}^1{1^n\cdot 0dt}.
\end{align*}
其中$c,k\in(0,1)$.
而$0^n\cdot 0<c^n\cdot k<1^n\cdot 0$.

于是我们根据直觉断言$\int_0^{\delta _1}{t^n\sin \left( \pi t \right) dt}+\int_{\delta _1}^{1-\delta _2}{t^n\sin \left( \pi t \right) dt}=O\left( \int_{1-\delta _2}^1{t^n\sin \left( \pi t \right) dt} \right)$.

从而问题转化为估计$\int_{1-\delta _2}^1{t^n\sin \left( \pi t \right) dt}$的阶.因为被积函数只有$\sin(\pi t)$不是幂函数,所以我们只需要处理$\sin(\pi t)$即可(即用幂函数估计$\sin(\pi t)$的阶,自然联想到$Taylor$公式).
又因为$\sin(\pi t)$可以在$t=1$附近$Taylor$展开(根据题意可知展开一项即可),
所以得到
\begin{align*}
\int_0^1{t^n\sin \left( \pi t \right) dt}&=\int_0^{\delta _1}{t^n\sin \left( \pi t \right) dt}+\int_{\delta _1}^{1-\delta _2}{t^n\sin \left( \pi t \right) dt}+\int_{1-\delta _2}^1{t^n\sin \left( \pi t \right) dt}
\\
&=\int_{\delta _2}^1{t^n\sin \left( \pi t \right) dt}+O\left( \int_{1-\delta _2}^1{t^n\sin \left( \pi t \right) dt} \right) 
\\
&=\int_{\delta _2}^1{t^n\sin \left( \pi t \right) dt}+O\left( \int_{1-\delta _2}^1{t^n\sin \left( \pi t \right) dt} \right) 
\\
&=\int_{1-\delta _2}^1{t^n\left[ \pi \left( 1-t \right) +o\left( 1-t \right) \right] dt}+O\left( \int_{1-\delta _2}^1{t^n\sin \left( \pi t \right) dt} \right) 
\\
&=\int_{1-\delta _2}^1{t^n\pi \left( 1-t \right) dt}+\int_{1-\delta _2}^1{o\left( t^n\left( 1-t \right) \right) dt}+O\left( \int_{1-\delta _2}^1{t^n\sin \left( \pi t \right) dt} \right) 
\\
&=\frac{4\pi \delta _2}{\left( n+1 \right) \left( n+2 \right)}+o\left( \frac{4\delta _2}{\left( n+1 \right) \left( n+2 \right)} \right) +O\left( \int_{1-\delta _2}^1{t^n\sin \left( \pi t \right) dt} \right) 
\\
&=\frac{4\pi \delta _2}{\left( n+1 \right) \left( n+2 \right)}+o\left( \frac{4\delta _2}{\left( n+1 \right) \left( n+2 \right)} \right) +O\left( \frac{4\pi \delta _2}{\left( n+1 \right) \left( n+2 \right)}+o\left( \frac{4\delta _2}{\left( n+1 \right) \left( n+2 \right)} \right) \right) 
\\
&=\frac{4\pi \delta _2}{\left( n+1 \right) \left( n+2 \right)}+O\left( \frac{4\pi \delta _2}{\left( n+1 \right) \left( n+2 \right)} \right) \sim \frac{1}{n^2}.
\end{align*}
虽然通过上述$Laplace$估阶方法能够准确的得到$\int_0^1{t^n\sin \left( \pi t \right) dt}$的阶.但是具体过程较为繁琐.于是我们思考能不能通过一种简单的方式,直接估计出$\int_0^1{t^n\sin \left( \pi t \right) dt}$的阶.接下来我们尝试找到一种更加简便的方式去估计$\int_0^1{t^n\sin \left( \pi t \right) dt}$的阶.

通过式\ref{eq:分段估计积分的阶1.7}的讨论我们知道,$\int_0^1{t^n\sin \left( \pi t \right) dt}$的阶是由$\int_{1-\delta_2}^1{t^n\sin \left( \pi t \right) dt}$的阶
决定的.因此无论我们怎么估阶都不能避开估计$\int_{1-\delta_2}^1{t^n\sin \left( \pi t \right) dt}$的阶,故要想简化估阶只能不对原有积分进行分段估计.而我们知道估计这个积分阶的关键就是估计$\int_{1-\delta_2}^1{t^n\sin \left( \pi t \right) dt}$的阶,原积分的其他部分忽略后并不影响积分的阶.又因为估计这个积分的阶我们只需要处理被积分函数中的$\sin(\pi t)$,于是我们就想到$Taylor$公式用幂函数去逼近$\sin(\pi t)$,从而将$\sin(\pi t)$在$t=1$处$Taylor$展开得到$\sin \left( \pi t \right) =\pi\left( 1-t \right) +o\left( 1-t \right)$$(t\to 1)$.但这个式子只在$(1-\delta_2,1)$上满足,不能保证在$(0,1)$上都满足,而在不同点处$Taylor$展开后再积分得到的函数的阶是不同的,但是我们知道我们只需要估计原积分在$t=1$附近的阶即可(因为只有在$(1-\delta_2,1)$上的积分才是原积分的主体部分,在其他积分区间上的积分全都是余项部分).因此我们只需要考虑$\sin(\pi t)$在$t=1$处的$Taylor$展开就可以了.只要再找到一个合理的放缩、构造等方法将余项部分合并进主体部分当中或直接变成常数就能实现简化解答步骤的目的.

对于本题我们有如下方式简化估阶过程:首先我们根据这个$Taylor$展开式,构造函数$g\left( x \right) =t^n\left( 1-t \right) \frac{\sin \left( \pi t \right)}{1-t}=\left( t^n-t^{n+1} \right) \cdot \frac{\sin \left( \pi t \right)}{1-t}$.这样就可以将$\sin(\pi t)$的主体部分给暴露出来,然后将$\frac{\sin(\pi t)}{1-t}$放缩成一个常数.又由于$g(x)$只去掉原被积函数在$t=1$处的函数值,所以$\int_0^1{t^n\sin \left( \pi t \right) dt}=\int_0^1{\left( t^n-t^{n+1} \right) \cdot \frac{\sin \left( \pi t \right)}{1-t}dt}$.自然原积分的阶也与$g(x)$相同.于是问题转化为了估计$g(x)$的阶.而$g(x)$的阶通过放缩很容易得到,具体证明见下述证法一.
\end{note}
\begin{remark}
$\sin(\pi t)$在$t=1$处的$Taylor$展开式系数可通过求极限(直接用$Taylor$公式求导比较麻烦)得到,设$\sin \left( \pi t \right) =a\left( 1-t \right) +o\left( 1-t \right)$,则由
\begin{align*}
\underset{t\rightarrow 1}{\lim}\frac{\sin \left( \pi t \right)}{1-t}\xlongequal[]{L'Hoptial's\,\,rule}\underset{t\rightarrow 1}{\lim}\frac{\pi \cos \left( \pi t \right)}{-1}=\pi.
\end{align*} 
可得$a=\pi$.于是$\sin \left( \pi t \right) =\pi\left( 1-t \right) +o\left( 1-t \right)$.
\end{remark}
\begin{proof}
{\color{blue}\text{证法一:}}
注意到\begin{align*}
\int_0^1{t^n\sin \left( \pi t \right) dt}=\int_0^1{ \left( t^n-t^{n+1} \right) \cdot \frac{\sin \left( \pi t \right)}{1-t}  dt}\leqslant M\int_0^1{\left( t^n-t^{n+1} \right)}dt=M\left( \frac{1}{n+1}-\frac{1}{n+2} \right) =\frac{M}{\left( n+1 \right) \left( n+2 \right)}
\end{align*}
其中$M=\underset{t\in \left[ 0,1 \right]}{\mathrm{sup}}\frac{\sin \left( \pi t \right)}{1-t}$.

从而对$\forall x\in[0,1]$,有\begin{align*}
\left| \int_0^x{t^n\sin \left( \pi t \right) dt} \right|=\int_0^x{t^n\sin \left( \pi t \right) dt}\leqslant \int_0^1{t^n\sin \left( \pi t \right) dt}\leqslant \frac{M}{\left( n+1 \right) \left( n+2 \right)}.
\end{align*}



又因为$\sum_{n=0}^{\infty}{\frac{M}{\left( n+1 \right) \left( n+2 \right)}}$收敛,
所以由$Weierstrass$判别可知,$\sum_{n=0}^{\infty}{\int_0^x{t^n\sin \left( \pi t \right) dt}}$在$\left[ 0,1 \right] $上一致收敛.
\end{proof}
\begin{remark}
能取$M=\underset{t\in \left[ 0,1 \right]}{\mathrm{sup}}\frac{\sin \left( \pi t \right)}{1-t}$是因为:
$\frac{\sin \left( \pi t \right)}{1-t}\in C[0,1)$,并且
\begin{align*}
\underset{t\rightarrow 1}{\lim}\frac{\sin \left( \pi t \right)}{1-t}\xlongequal{L'Hopital's\,\,rule}\underset{t\rightarrow 1}{\lim}\left[ -\pi \cos \left( \pi t \right) \right] =\pi .
\end{align*}
于是由\hyperlink{label}{推广的连续函数最大、最小值定理}可知,$\frac{\sin \left( \pi t \right)}{1-t}$在$[0,1)$上有界.从而存在上界$M=\underset{t\in \left[ 0,1 \right]}{\mathrm{sup}}\frac{\sin \left( \pi t \right)}{1-t}$.
\end{remark}
\begin{remark}
解决这类问题虽然实际上我们的想法是估阶,但是为了使解答过程简便,我们不需要用思路分析里这种从头到尾把那些余项都写出来的方式去估阶.

在解答过程中,我们只需要在保证不改变阶的前提下,将那些余项全部放缩成常数、或通过放缩将其合并到主体部分中、或通过构造一个与原函数同阶但更易估阶的函数再将原函数放缩成这个函数(本题采取的就是这个方式)等其他技巧.
即我们可以将不影响阶的部分(就是那些比主体部分还要高阶的部分和常数项)全部放缩掉,最终放缩得到的式子中只含有主体部分.
然后我们只需要估计放缩得到的式子的阶就可以通过迫敛性得到原函数的阶.(这里估计放缩得到的式子的阶的方式与我们在思路分析中估计主体部分的阶的方法一致).
\end{remark}
\begin{conclusion}\label{简化估阶过程的核心想法}
\textbf{简化估阶过程的核心想法就是:先确定原函数的主体部分(若原函数的主体部分并不明显就需要利用$Taylor$公式将其主体部分彻底暴露出来再进行估阶),然后通过放缩、构造等方式将余项部分合并进主体部分中或者放缩成常数,最后估计主体部分的阶即可.}
\end{conclusion}

\begin{example}
讨论级数
\begin{align*}
\sum_{n=1}^{+\infty}{\frac{\cos \left( \ln n \right)}{n^p}}.
\end{align*}
的敛散性.
\end{example}
\begin{solution}
\(\sum_{n = 1}^{+\infty}\frac{\cos(\ln n)}{n^p}\),当\(p > 1\)时,有\(\left|\frac{\cos(\ln n)}{n^p}\right| \leq \frac{1}{n^p}\),而此时\(\sum_{n = 1}^{+\infty}\frac{1}{n^p}\)是绝对收敛的.故此时\(\sum_{n = 1}^{+\infty}\frac{\cos(\ln n)}{n^p}\)也绝对收敛.

当\(p \leq 0\)时,注意到原级数的通项并不趋于零,于是此时\(\sum_{n = 1}^{+\infty}\frac{\cos(\ln n)}{n^p}\)一定发散.

以下设\(p \in (0,1]\),我们来证明此时级数都是发散的.

对\(\forall N > 0\),任取\(k > \max\{N,10\}\),则\([e^{2k\pi + \frac{\pi}{4}}] > N\),从而我们有:
\begin{align*}
&\sum_{n = 1}^{[e^{2k\pi + \frac{\pi}{4}}]}\frac{\cos(\ln n)}{n^p} \geq \sum_{n = [e^{2k\pi - \frac{\pi}{4}}] + 1}^{[e^{2k\pi + \frac{\pi}{4}}]}\frac{\cos(\ln n)}{n^p}\geq \sum_{n = [e^{2k\pi - \frac{\pi}{4}}] + 1}^{[e^{2k\pi + \frac{\pi}{4}}]}\frac{\cos(\ln n)}{n}\geq \frac{\sqrt{2}}{2}\sum_{n = [e^{2k\pi - \frac{\pi}{4}}] + 1}^{[e^{2k\pi + \frac{\pi}{4}}]}\frac{1}{n}
\geq \frac{\sqrt{2}}{2}\sum_{n = [e^{2k\pi - \frac{\pi}{4}}] + 1}^{[e^{2k\pi + \frac{\pi}{4}}]}\frac{1}{[e^{2k\pi + \frac{\pi}{4}}]}
\\
&=\frac{\sqrt{2}}{2}\frac{[e^{2k\pi + \frac{\pi}{4}}] - [e^{2k\pi - \frac{\pi}{4}}]}{[e^{2k\pi + \frac{\pi}{4}}]}=\frac{\sqrt{2}}{2}\left(1 - \frac{[e^{2k\pi - \frac{\pi}{4}}]}{[e^{2k\pi + \frac{\pi}{4}}]}\right)\geq \frac{\sqrt{2}}{2}\left(1 - \frac{e^{2k\pi - \frac{\pi}{4}}}{e^{2k\pi + \frac{\pi}{4}} - 1}\right) > \frac{1}{2}.
\end{align*}
于是由\(Cauchy\)收敛准则,可知此时级数发散.

综上,\(\sum_{n = 1}^{+\infty}\frac{\cos(\ln n)}{n^p}\)在\(p > 1\)时绝对收敛,在\(p \leq 0\)以及\(p \in (0,1]\)时均发散. 
\end{solution}

\begin{example}
判断级数$\sum_{n=1}^{+\infty}{\sin \left[ \pi \left( 3+\sqrt{7} \right) ^n \right]}$的敛散性.
\end{example}
\begin{note}
这类问题的核心想法就是考虑共轭式.
\end{note}
\begin{proof}
注意到对\(\forall n \in \mathbb{N}_+\),都存在\(A_n,B_n \in \mathbb{N}_+\),使得

\[
(3 + \sqrt{7})^n = A_n + B_n\sqrt{7}, \quad (3 - \sqrt{7})^n = A_n - B_n\sqrt{7}.
\]

从而
\begin{align*}
&\sin[\pi(3 + \sqrt{7})^n] = \sin(\pi A_n + \pi B_n\sqrt{7})
= \sin(\pi A_n + \pi B_n\sqrt{7} - 2\pi A_n)\\
&= -\sin(\pi A_n - \pi B_n\sqrt{7})= -\sin[\pi(3 - \sqrt{7})^n].
\end{align*}
因此结合\(0 < 3 - \sqrt{7} < 1\),可知

\[
\sum_{n = 1}^{+\infty} \sin[\pi(3 + \sqrt{7})^n] = -\sum_{n = 1}^{+\infty} \sin[\pi(3 - \sqrt{7})^n] \sim -\sum_{n = 1}^{+\infty} \pi(3 - \sqrt{7})^n \text{收敛}.
\]
\end{proof}

\begin{example}
若数列${na_n}$单调,正项级数$\sum_{n=1}^{\infty}{a_n}$收敛,证明:
\begin{align*}
\lim_{n\rightarrow \infty} na_n\ln n=0.
\end{align*}
\end{example}
\begin{proof}
若数列\(\{ na_n\}\)单调递增,则对\(\forall n \in \mathbb{N}_+\),都有\(na_n \geqslant a_1\).从而
\(a_n \geqslant \frac{a_1}{n}\),\(\forall n \in \mathbb{N}_+\).
而\(\sum_{n = 1}^{\infty}\frac{a_1}{n}\)发散,于是由比较判别法,可知\(\sum_{n = 1}^{\infty}a_n\)也发散.这与题设矛盾.故数列\(\{ na_n\}\)单调递减.
又由\(\sum_{n = 1}^{\infty}a_n\)收敛和\(Cauchy\)收敛准则,可知\(\forall \varepsilon > 0\),\(\exists N \in \mathbb{N}_+\),使得当\(n > m > N\)时,都有\(\sum_{k = m}^{n}a_k < \varepsilon\).

因此对\(\forall n > N + 3\)(充分大的\(n\)),取\(m = [\sqrt{n}] - 1\),我们都有
\begin{align*}
&0 \leqslant na_n\ln n < na_n\ln\frac{n}{\sqrt{n}} < na_n\int_{m}^{n}\frac{1}{x}dx<na_n\ln \frac{n}{m} \\
&= na_n\sum_{k = m}^{n}\int_{k}^{k + 1}\frac{1}{x}dx \leqslant na_n\sum_{k = m}^{n}\int_{k}^{k + 1}\frac{1}{k}dx\\
&= na_n\sum_{k = m}^{n}\frac{1}{k} \leqslant \sum_{k = m}^{n}ka_k\cdot\frac{1}{k} = \sum_{k = m}^{n}a_k < \varepsilon
\end{align*}
由\(\varepsilon\)的任意性,令\(\varepsilon \to 0^+\),即得\(\lim_{n \to \infty}na_n\ln n = 0\). 
\end{proof}

\begin{example}
证明:
\begin{align*}
I(a)=\int_0^{\frac{\pi}{2}}{\ln \frac{1+a\cos x}{1-a\cos x}}\cdot \frac{1}{\cos x}dx=\pi \mathrm{arc}\sin a,|a|<1.
\end{align*}
\end{example}
\begin{note}
这题显然用含参积分求导即可,但多想一想,如果题目没告诉你参数$a$,直接让你计算具体对应的定积分,你该如何思考,如何引入参数?
\end{note}
\begin{proof}
\begin{align*}
I^{\prime}(a)&=\frac{d}{da}\int_{0}^{\frac{\pi}{2}}\ln\frac{1 + a\cos x}{1 - a\cos x}\cdot\frac{1}{\cos x}dx=\int_{0}^{\frac{\pi}{2}}\frac{\partial}{\partial a}\left(\ln\frac{1 + a\cos x}{1 - a\cos x}\cdot\frac{1}{\cos x}\right)dx
\\
&=\int_{0}^{\frac{\pi}{2}}\frac{2}{1 - a^{2}\cos^{2}x}dx=\int_{0}^{\frac{\pi}{2}}\frac{2}{1 - a^{2}\cdot\frac{1 + \cos 2x}{2}}dx
\\
&\xlongequal[\text{万能公式换元}]{\text{令}t=\tan x}\int_{0}^{+\infty}\frac{2}{1 - a^{2}\cdot\frac{1 + \frac{1 - t^{2}}{1 + t^{2}}}{2}}\cdot\frac{1}{1 + t^{2}}dt
\\
&=\int_{0}^{+\infty}\frac{2}{t^{2} + 1 - a^{2}}dt=\frac{2}{\sqrt{1 - a^{2}}}\arctan\frac{t}{\sqrt{1 - a^{2}}}\Big|_{0}^{+\infty}
\\
&=\frac{\pi}{\sqrt{1 - a^{2}}}
\end{align*}
又因为\(I(0) = 0\),所以\(I(a)=\int_{0}^{a}I^{\prime}(a)da=\int_{0}^{a}\frac{\pi}{\sqrt{1 - a^{2}}}da=\pi\arcsin a\). 
\end{proof}

\begin{example}
设$f\in C^1\left[ 0,1 \right] $,使得$\int_0^1{f\left( x \right) dx}=0$且有
\begin{align*}
f'\left( x \right) +f\left( x \right) \tan x=\int_0^x{f\left( y \right) dy,\forall x\in \left[ 0,1 \right] .}
\end{align*}
证明:$f$在$[0,1]$上恒为$0$.
\end{example}
\begin{note}
核心想法是:利用\hyperref[conclusion:在有界闭区间上连续的函数恒为$0$的充要条件:在有界闭区间上没有正的最大值和负的最小值]{在有界闭区间上连续的函数恒为$0$的充要条件:在有界闭区间上没有正的最大值和负的最小值}.然后假设函数的最值在区间内部取到,再比较等式两边符号给出矛盾.
\end{note}
\begin{proof}
记\(F(x) = \int_{0}^{x}f(y)dy\),则\(F \in D^1[0,1]\),\(F(0) = F(1) = 0\)并且
\begin{align*}
F^{\prime\prime}(x) + F^{\prime}(x)\tan x = F(x),\forall x \in [0,1].
\end{align*}
因为\(F \in D^1[0,1]\),所以\(F \in C^1[0,1]\).从而由连续函数最大、最小值定理,可知\(F(x)\)在\([0,1]\)上存在最大值和最小值.
若\(F\)在\([0,1]\)取得正的最大值,最大值点为\(a\),则由\(F(0) = F(1) = 0\),可知\(a \in (0,1)\).并且\(F^{\prime}(a) = 0\),\(F^{\prime\prime}(a) \leqslant 0\).于是
\begin{align*}
0 < F(a) = F^{\prime\prime}(a) + F^{\prime}(a)\tan a = F^{\prime\prime}(a) \leqslant 0.
\end{align*}
这就是矛盾!从而\(F\)在\([0,1]\)没有正的最大值.类似的可以讨论最小值,得到\(F\)在\([0,1]\)没有负的最小值.因此\(0 \leqslant F(x) \leqslant 0\),\(\forall x \in [0,1]\).即\(F(x) \equiv 0\),\(\forall x \in [0,1]\).故\(f\)在\([0,1]\)上恒为\(0\). 
\end{proof}
\begin{conclusion}\label{conclusion:在有界闭区间上连续的函数恒为$0$的充要条件:在有界闭区间上没有正的最大值和负的最小值}
若$f\in C[a,b]$,则$f$在$[a,b]$上恒为$0$的充要条件是:$f$在$[a,b]$上没有正的最大值和负的最小值(只有非正的最大值和非负的最小值).
\begin{proof}
必要性是显然的.我们只证明充分性.

已知$f\in C[a,b]$,则根据连续函数的最大、最小值定理,可知$f$在$[a,b]$上存在最大值和最小值.不妨设最大值点为$M$,最小值点为$m$,则$M,m\in[a,b]$.
又因为$f$在$[a,b]$上没有正的最大值和负的最小值,所以
\begin{align*}
0\leq f(m)\leq f(x)\leq f(M)\leq 0,\forall x\in[a,b].
\end{align*}
故$f$在$[a,b]$上恒为$0$.
\end{proof}
\end{conclusion}

\begin{lemma}[Fekete 次可加性引理]\label{proposition:Fekete次可加性引理}
设非负数列\(\{a_{n}\}\)满足对任意正整数\(n,m\)有
\[a_{m + n}\leq a_{m}+a_{n},\]
则
\begin{align*}
\underset{n\rightarrow \infty}{\lim}\frac{a_n}{n}=\mathrm{inf}\left\{ \frac{a_n}{n} \right\} .
\end{align*}
\end{lemma}
\begin{proof}
对 $\forall k\in \mathbb{N} _+$,固定 $k$,则由带余除法可知,存在 $q,r\in \mathbb{N} _+$,使得 $n=kq+r$。从而由条件可得
\begin{align*}
\frac{a_n}{n}=\frac{a_{kq+r}}{n}\leqslant \frac{a_{kq}+a_r}{n}\leqslant \frac{a_{k\left( q-1 \right)}+a_k}{kq+r}+\frac{a_r}{n}\leqslant \cdots \leqslant \frac{qa_k}{kq+r}+\frac{a_r}{n}\leqslant \frac{a_k}{k}+\frac{a_r}{n},\forall k\in \mathbb{N} _+.
\end{align*}
令 $n\rightarrow \infty$ 并取上极限得到
\begin{align}
\underset{n\rightarrow \infty}{\overline{\lim }}\frac{a_n}{n}\leqslant \frac{a_k}{k}<\infty ,\forall k\in \mathbb{N} _+.\label{lemma1.2--1.1}
\end{align}
再令 $k\rightarrow \infty$ 并取下极限可得
\begin{align*}
\underset{n\rightarrow \infty}{\overline{\lim }}\frac{a_n}{n}\leqslant \underset{n\rightarrow \infty}{\underline{\lim }}\frac{a_k}{k}=\underset{n\rightarrow \infty}{\underline{\lim }}\frac{a_n}{n}\leqslant \underset{n\rightarrow \infty}{\overline{\lim }}\frac{a_n}{n}.
\end{align*}
故 $\left\{ \frac{a_n}{n} \right\}$ 收敛。注意到$\{a_n\}$非负,则$\{\frac{a_n}{n}\}$一定有下界0,从而一定存在下确界$\mathrm{inf}\left\{ \frac{a_n}{n} \right\}$.于是我们有
\begin{align*}
\mathrm{inf}\left\{ \frac{a_n}{n} \right\} \leqslant \frac{a_n}{n},\forall n\in \mathbb{N} _+.
\end{align*}
令 $n\rightarrow \infty$ 并取上极限,再结合 $\left\{ \frac{a_n}{n} \right\}$ 收敛和\eqref{lemma1.2--1.1}式可得
\begin{align*}
\mathrm{inf}\left\{ \frac{a_n}{n} \right\} \leqslant \underset{n\rightarrow \infty}{\overline{\lim }}\frac{a_n}{n}=\underset{n\rightarrow \infty}{\lim}\frac{a_n}{n}\leqslant \frac{a_k}{k},\forall k\in \mathbb{N} _+.
\end{align*}
再对 $k$ 取下确界即得
\begin{align*}
\mathrm{inf}\left\{ \frac{a_n}{n} \right\} \leqslant \underset{n\rightarrow \infty}{\lim}\frac{a_n}{n}\leqslant \mathrm{inf}\left\{ \frac{a_k}{k} \right\} =\mathrm{inf}\left\{ \frac{a_n}{n} \right\} .
\end{align*}
故 $\underset{n\rightarrow \infty}{\lim}\frac{a_n}{n}=\mathrm{inf}\left\{ \frac{a_n}{n} \right\}$。
\end{proof}

\begin{corollary}
设$f(x)$在$[0,+\infty)$上连续非负,且对任意的$x,y\geq 0$,有$f(x+y)\leq f(x)+f(y)$.

证明:$\lim_{x\rightarrow +\infty}\frac{f(x)}{x}$
存在且有限.
\end{corollary}
\begin{remark}
$[x]$:表示$x$的整数部分;$\{x\}$:表示$x$的小数部分.
\end{remark}
\begin{proof}
由条件可知,对$\forall n,m \in \mathbb{N}$,都有$f(n+m)\leq f(n)+f(m)$.从而
由\hyperref[proposition:Fekete次可加性引理]{Fekete次可加性引理}可得\[\underset{n\rightarrow \infty}{\lim}\frac{f\left( n \right)}{n}=\mathrm{inf} \left\{ \frac{f\left( n \right)}{n} \right\}.\]
对$\forall x>1$,都存在$n=\left[ x \right]$,使得$x=n+\left\{ x \right\}$.由条件可知
\begin{align*}
&f\left( x \right) =f\left( n+\left\{ x \right\} \right) \leqslant f\left( n \right) +f\left( \left\{ x \right\} \right),\\
&f\left( n+1 \right) =f\left( \left[ x \right] +1 \right) =f\left( x+1-\left\{ x \right\} \right) \leqslant f\left( x \right) +f\left( 1-\left\{ x \right\} \right) \Rightarrow f\left( x \right) \geqslant f\left( n+1 \right) -f\left( 1-\left\{ x \right\} \right).
\end{align*}
从而对$\forall x>1,n=\left[ x \right]$,我们都有
\begin{align*}
\frac{f\left( x \right)}{x}&\leqslant \frac{f\left( n \right) +f\left( \left\{ x \right\} \right)}{x}=\frac{f\left( n \right)}{n+\left\{ x \right\}}+\frac{f\left( \left\{ x \right\} \right)}{x}=\frac{f\left( n \right)}{n}+\frac{f\left( \left\{ x \right\} \right)}{x},\\
\frac{f\left( x \right)}{x}&\geqslant \frac{f\left( n+1 \right) -f\left( 1-\left\{ x \right\} \right)}{x}=\frac{f\left( n+1 \right)}{n+\left\{ x \right\}}+\frac{f\left( 1-\left\{ x \right\} \right)}{x}\geqslant \frac{f\left( n+1 \right)}{n+1}+\frac{f\left( 1-\left\{ x \right\} \right)}{x}.
\end{align*}
即
\begin{align}
\frac{f\left( n+1 \right)}{n+1}+\frac{f\left( 1-\left\{ x \right\} \right)}{x}\leqslant \frac{f\left( x \right)}{x}\leqslant \frac{f\left( n \right)}{n}+\frac{f\left( \left\{ x \right\} \right)}{x},\forall x>1,n=\left[ x \right].\label{corollary1.1-1.1}
\end{align}
又由$f\in C\left[ 0,+\infty \right)$,$\left\{ x \right\} \in \left[ 0,1 \right]$,因此$f\left( \left\{ x \right\} \right)$,$f\left( 1-\left\{ x \right\} \right)$都有界.
对\eqref{corollary1.1-1.1}两边令$x\rightarrow +\infty$,则同时有$n\rightarrow \infty$,再分别取上、下极限得到
\begin{align*}
\mathrm{inf}\left\{ \frac{f\left( n \right)}{n} \right\} =\underset{n\rightarrow \infty}{\lim}\frac{f\left( n+1 \right)}{n+1}\leqslant \underset{x\rightarrow +\infty}{\underline{\lim }}\frac{f\left( x \right)}{x}\leqslant \underset{x\rightarrow +\infty}{\overline{\lim }}\frac{f\left( x \right)}{x}\leqslant \underset{n\rightarrow \infty}{\lim}\frac{f\left( n \right)}{n}=\mathrm{inf}\left\{ \frac{f\left( n \right)}{n} \right\}.
\end{align*}
故$\underset{x\rightarrow +\infty}{\lim}\frac{f\left( x \right)}{x}=\mathrm{inf}\left\{ \frac{f\left( n \right)}{n} \right\}$.
\end{proof}

\begin{example}
设 $f(x)$ 在 $[0,1]$ 上二阶连续可微,$f(0)<0<f(1)$,且 $f'(x)>0$,$f''(x)>0$。
\begin{enumerate}[(1)]
\item 证明:存在唯一的 $\xi \in (0,1)$,满足 $f(\xi) = 0$。
\item 记 $x_1 = 1$,$x_{n+1} = x_n - \frac{f(x_n)}{f'(x_n)}$,证明:$\lim_{n \to \infty} x_n = \xi$。
\item 求极限 $\lim_{n \to \infty} \frac{x_{n+1} - \xi}{(x_n - \xi)^2}$.
\end{enumerate}
\end{example}
\begin{proof}
\begin{enumerate}[(1)]
\item 由条件可知 $f \in C[0,1]$ 其 $f(0) < 0 < f(1)$。从而由连续函数的介值定理可知,存在 $\xi \in (0,1)$,使得 $f(\xi) = 0$。又由 $f'(x) > 0$ 可知,$f$ 在 $[0,1]$ 上严格单调递增,于是满足条件的 $\xi$ 是存在且唯一的。

\item 由条件可知 $f'(x) > 0, \forall x \in [0,1]$,从而 $f$ 在 $[0,1]$ 上严格单调递增。注意到 $x_1 \in (\xi, 1]$,假设 $x_k \in (\xi, 1]$,则由 $f, f'$ 严格递增可知
\begin{align*}
0 = f(\xi) < f(x_k) \leqslant f(1), \quad f'(\xi) < f'(x_k) \leqslant f'(1).
\end{align*}
又由 $f''(x) > 0, \forall x \in [0,1]$ 可知,$f'$ 在 $[0,1]$ 上严格递增且 $f$ 是在 $[0,1]$ 上的严格下凸函数。从而由\hyperref[Basis of Analytics-可微的下凸函数恒在切线上方]{可微的下凸函数恒在切线上方}可得
\begin{align*}
0 = f(\xi) > f'(x_k)(\xi - x_k) + f(x_k).
\end{align*}
从而 $\frac{f(x_k)}{f'(\xi)} < x_k - \xi$。于是
\begin{align*}
\xi = x_k - (x_k - \xi) < x_{k+1} = x_k - \frac{f(x_k)}{f'(x_k)} < x_k \leqslant 1.
\end{align*}
故由数学归纳法可知 $x_n \in (\xi, 1], \forall n \in \mathbb{N}$。注意到 $x_2 = x_1 - \frac{f(1)}{f'(1)} < x_1$。假设 $x_n < x_{n-1}$,则由 $x_n \in (\xi, 1]$ 及 $f$ 严格递增可得 $f(x_n) > f(\xi) = 0$。从而再结合 $f'(x_n) > 0$ 可得
\begin{align*}
x_{n+1} = x_n - \frac{f(x_n)}{f'(x_n)} < x_n。
\end{align*}
故由数学归纳法可知 $\{x_n\}$ 单调递减。由单调有界定理可知,$\{x_n\}$ 收敛。设 $\lim_{n \to \infty} x_n = x$,则对 $x_{n+1} = x_n - \frac{f(x_n)}{f'(x_n)}$ 两边取极限得到
\begin{align*}
x = x - \frac{f(x)}{f'(x)} \Rightarrow f(x) = 0。
\end{align*}
再由 (1) 可知 $\lim_{n \to \infty} x_n = x = \xi$。

\item 由 (2) 及 $f \in D^2[0,1]$,再利用归结原则和 L'Hospital 法则可得
\begin{align*}
\lim_{n \to \infty} \frac{x_{n+1} - \xi}{(x_n - \xi)^2} &= \lim_{n \to \infty} \frac{x_n - \frac{f(x_n)}{f'(x_n)} - \xi}{(x_n - \xi)^2} = \lim_{x \to \xi^+} \frac{x - \frac{f(x)}{f'(x)} - \xi}{(x - \xi)^2} 
= \lim_{x \to \xi^+} \frac{(x - \xi) f'(x) - f(x)}{(x - \xi)^2 f'(x)}\\ 
&= \frac{1}{f'(\xi)} \lim_{x \to \xi^+} \frac{(x - \xi) f'(x) - f(x)}{(x - \xi)^2} 
\xlongequal{\mathrm{L'Hospital 法则}} \frac{1}{f'(\xi)} \lim_{x \to \xi^+} \frac{f''(x)}{2} = \frac{f''(\xi)}{2 f'(\xi)}.
\end{align*}
\end{enumerate}
\end{proof}

\begin{example}
求极限 $\lim_{n \to \infty} \left( \frac{2}{\pi} \arctan n \right)^n$.
\end{example}
\begin{proof}
由归结原则及 L'Hospital 法则可得
\begin{align*}
\lim_{n \to \infty} n \left( \ln \arctan n + \ln \frac{\pi}{2} \right) &= \lim_{n \to \infty} x \left( \ln \arctan x + \ln \frac{\pi}{2} \right) = \lim_{n \to \infty} \frac{\ln \arctan x + \ln \frac{\pi}{2}}{\frac{1}{x}} \\
&= \lim_{n \to \infty} \frac{\frac{1}{\left( 1 + x^2 \right) \arctan x}}{-\frac{1}{x^2}} = -\frac{2}{\pi} \lim_{n \to \infty} \frac{x^2}{1 + x^2} = -\frac{2}{\pi}.
\end{align*}
于是
\begin{align*}
\lim_{n \to \infty} \left( \frac{2}{\pi} \arctan n \right)^n = \lim_{n \to \infty} e^{n \left( \ln \arctan n + \ln \frac{\pi}{2} \right)} = e^{\lim\limits_{n \to \infty} n \left( \ln \arctan n + \ln \frac{\pi}{2} \right)} = e^{-\frac{2}{\pi}}.
\end{align*}
\end{proof}

\begin{example}
求极限 $\lim_{x \to +\infty} x \left[ \frac{1}{e} - \left( \frac{x}{1 + x} \right)^x \right]$.
\end{example}
\begin{proof}

\end{proof}

\begin{example}
设 $f(x)$ 在 $[0, +\infty)$ 上可导, 且 $f'(x)$ 在 $[0, +\infty)$ 上连续, 若
\begin{align*}
\lim_{x \to +\infty} [2f(x) + f'(x)] = 0,
\end{align*}
证明: $\lim_{x \to +\infty} f(x) = 0$。
\end{example}
\begin{proof}
注意到
\begin{align*}
\lim_{x \to +\infty} x \ln \frac{x}{1 + x} &= -\lim_{x \to +\infty} x \ln \left( 1 + \frac{1}{x} \right) = -1.
\end{align*}
从而 $\lim_{x \to +\infty} \left( x \ln \frac{x}{1 + x} + 1 \right) = 0$。于是
\begin{align*}
\lim_{x \to +\infty} x \left[ \frac{1}{e} - \left( \frac{x}{1 + x} \right)^x \right] &= \frac{1}{e} \lim_{x \to +\infty} x \left( 1 - e^{x \ln \frac{x}{1 + x} + 1} \right) = -\frac{1}{e} \lim_{x \to +\infty} x \left( x \ln \frac{x}{1 + x} + 1 \right) \\
&= -\frac{1}{e} \lim_{x \to +\infty} \left[ -x^2 \ln \left( 1 + \frac{1}{x} \right) + x \right] = -\frac{1}{e} \lim_{x \to +\infty} \left[ -x^2 \left( \frac{1}{x} - \frac{1}{2x^2} + o\left( \frac{1}{x^2} \right) \right) + x \right] \\
&= -\frac{1}{e} \lim_{x \to +\infty} \left( \frac{1}{2} + o\left( 1 \right) \right) = -\frac{1}{2e}.
\end{align*}
\end{proof}

\begin{example}
求极限 $\lim_{x \to 0} \left( \frac{\sin x}{x} \right)^{\frac{1}{1 - \cos x}}$.
\end{example}
\begin{proof}
由等价无穷小替换与洛必达法则可知
\begin{align*}
\lim_{x \to 0} \ln \left( \left( \frac{\sin x}{x} \right)^{\frac{1}{1 - \cos x}} \right) &= \lim_{x \to 0} \frac{\ln \left| \sin x \right| - \ln \left| x \right|}{1 - \cos x} = \lim_{x \to 0} \frac{\cot x - \frac{1}{x}}{\frac{x^2}{2}} = \lim_{x \to 0} \frac{x \cos x - \sin x}{x^2 \sin x} \\
&= \lim_{x \to 0} \frac{x \cos x - \sin x}{x^3} = \lim_{x \to 0} \frac{-x \sin x}{3x^2} = -\frac{1}{3}.
\end{align*}
因此 $\lim_{x \to 0} \left( \frac{\sin x}{x} \right)^{\frac{1}{1 - \cos x}} = e^{-\frac{1}{3}}$.
\end{proof}

\begin{example}
设 $a > 0, b > 0$, 且 \[a_1 = a, a_{n+1} = \frac{1}{2} \left( a_n + \frac{b}{a_n} \right), n \in \mathbb{N}_+\].
证明数列 $\{a_n\}$ 收敛并计算 $\lim_{n \to \infty} a_n$.
\end{example}
\begin{proof}
注意到 $a_1 = a > 0$,假设 $a_k > 0$,则
\begin{align*}
a_{k+1} &= \frac{1}{2} \left( a_k + \frac{b}{a_k} \right) > 0.
\end{align*}
故由数学归纳法可知 $a_n > 0, \forall n \in \mathbb{N}_+$。从而
\begin{align*}
\left| a_{n+1} - \sqrt{b} \right| &= \left| \frac{1}{2} \left( a_n + \frac{b}{a_n} \right) - \sqrt{b} \right| = \left| \frac{a_n^2 - 2\sqrt{b}a_n + b}{2a_n} \right| \\
&= \left| \frac{a_n - \sqrt{b}}{2a_n} \right| \left| a_n - \sqrt{b} \right| = \left| \frac{1}{2} - \frac{\sqrt{b}}{a_n} \right| \left| a_n - \sqrt{b} \right| \\
&\leqslant \frac{1}{2} \left| a_n - \sqrt{b} \right|, \forall n \in \mathbb{N}_+.
\end{align*}
于是
\begin{align*}
\left| a_n - \sqrt{b} \right| &\leqslant \frac{1}{2} \left| a_{n-1} - \sqrt{b} \right| \leqslant \cdots \leqslant \frac{1}{2^{n-1}} \left| a_1 - \sqrt{b} \right|, \forall n \in \mathbb{N}_+.
\end{align*}
令 $n \rightarrow \infty$,得 $\underset{n \to \infty}{\lim} \left| a_n - \sqrt{b} \right| = 0$,故 $\underset{n \to \infty}{\lim} a_n = \sqrt{b}$.
\end{proof}

\begin{example}
设 $f : (a, b) \to (a, b)$ 满足对任意的 $x, y \in (a, b)$, 当 $x \neq y$ 时, 有 $\left| f(x) - f(y) \right| < \left| x - y \right|$. 任取 $x_1 \in (a, b)$, 令 $x_{n+1} = f(x_n), n = 1, 2, \cdots$, 证明: 数列 $\{x_n\}_{n=1}^{\infty}$ 收敛.
\end{example}
\begin{remark}
\begin{enumerate}[(1)]
\item \hypertarget{找相邻子列满足条件的方法证明}{存在子列} $\{x_{n_k}\}$ 满足 $\lim_{k \to \infty} x_{n_k} = \xi - A, \lim_{k \to \infty} x_{n_k+1} = \xi + A$ 的原因:
记 $X = \xi - A, Y = \xi + A$,假设存在 $k_0 \in \mathbb{N}$,对 $\forall n > k_0$,有
\begin{align}
x_n \notin (X - \frac{1}{k_0}, X + \frac{1}{k_0}) 或 x_{n+1} \notin (Y - \frac{1}{k_0}, Y + \frac{1}{k_0})。 \label{example1.19-0.1}
\end{align}
因为 $\{x_n\}$ 有且仅有两个聚点 $X$ 和 $Y$,所以对上述 $\varepsilon$,$\{x_n\}$ 中都有无穷多项落在 $(X - \frac{1}{k_0}, X + \frac{1}{k_0}) \cup (Y - \frac{1}{k_0}, Y + \frac{1}{k_0})$ 内。
从而存在 $N \in \mathbb{N}$,使得 $\forall n > N$,有
\begin{align}
x_n \in (X - \frac{1}{k_0}, X + \frac{1}{k_0}) \cup (Y - \frac{1}{k_0}, Y + \frac{1}{k_0})。 \label{example1.19-0.2}
\end{align}
于是由\eqref{example1.19-0.1}\eqref{example1.19-0.2}式可得,当 $n > \max\{N, k_0\}$ 时,我们有
\begin{align*}
x_n \in (Y - \frac{1}{k_0}, Y + \frac{1}{k_0}) 或 x_{n+1} \in (X - \frac{1}{k_0}, X + \frac{1}{k_0})。
\end{align*}
因此若 $x_n \in (Y - \frac{1}{k_0}, Y + \frac{1}{k_0}), \forall n > \max\{N, k_0\}$,则 $\{x_n\}$ 最多只有有限项落在 $(X - \frac{1}{k_0}, X + \frac{1}{k_0})$ 内,这与 $X$ 是 $\{x_n\}$ 的一个聚点矛盾。若 $x_{n+1} \in (X - \frac{1}{k_0}, X + \frac{1}{k_0}), \forall n > \max\{N, k_0\}$,则 $\{x_n\}$ 最多只有有限项落在 $(Y - \frac{1}{k_0}, Y + \frac{1}{k_0})$ 内,这与 $Y$ 是 $\{x_n\}$ 的一个聚点矛盾。
故假设不成立,从而对 $\forall k \in \mathbb{N}$,都存在 $n_k > k$,使得
\begin{align*}
x_{n_k} \in (X - \frac{1}{k}, X + \frac{1}{k}) 且 x_{n_k+1} \in (Y - \frac{1}{k}, Y + \frac{1}{k})。
\end{align*}
于是根据 $k$ 的任意性可知 $\lim_{k \to \infty} x_{n_k} = X = \xi - A, \lim_{k \to \infty} x_{n_k+1} = Y = \xi + A$。

\item \hypertarget{xi-A,xi+A在(a,b)中的原因证明}{$\xi - A, \xi + A \in (a, b)$ 的原因}:一定存在 $k_1, k_2 \in \mathbb{N}$,使得
\begin{align}
x_{n_{k_1}} < \xi, x_{n_{k_2}} > \xi。 \label{example1.19-1.1}
\end{align}
否则,对 $\forall k_1, k_2 \in \mathbb{N}$,都有
\begin{align*}
x_{n_{k_1}} \geqslant \xi, \quad x_{n_{k_2}} \leqslant \xi。
\end{align*}
令 $k_1, k_2 \to \infty$,再结合 $\lim_{k \to \infty} x_{n_k} = \xi - A, \lim_{k \to \infty} x_{n_k+1} = \xi + A$ 得到
\begin{align*}
\xi - A = \lim_{k_1 \to \infty} x_{n_{k_1}} \geqslant \xi > \xi - A, \quad \xi + A = \lim_{k_2 \to \infty} x_{n_{k_2}} \leqslant \xi < \xi + A。
\end{align*}
显然矛盾!又因为 $\{|x_n - \xi|\}$ 单调递减趋于 $A$,所以
\begin{align*}
|x_n - A| \geqslant A, \forall n \in \mathbb{N}。
\end{align*}
从而由 $x_n \in (a, b)$ 及\eqref{example1.19-1.1}式可得
\begin{align*}
A \leqslant |x_{n_{k_1}} - \xi| = \xi - x_{n_{k_1}} < \xi - a \Rightarrow \xi - A > a,
\\
A \leqslant |x_{n_{k_2}} - \xi| = x_{n_{k_2}} - \xi < b - \xi \Rightarrow \xi + A < b。
\end{align*}
因此 $\xi - A, \xi + A \in (a, b)$。
\end{enumerate}
\end{remark}
\begin{proof}
注意到 $x_1 \in (a, b)$,假设 $x_k \in (a, b)$,则 $x_{k+1} = f(x_k) \in (a, b)$,故由数学归纳法可知 $x_n \in (a, b), \forall n \in \mathbb{N}$。
又由条件可知,对 $\forall \varepsilon > 0$,令 $\delta = \varepsilon > 0$,当 $x, y \in (a, b)$ 且 $0 < |x - y| < \delta$ 时,有
\begin{align*}
|f(x) - f(y)| < |x - y| < \delta = \varepsilon。
\end{align*}
故 $f$ 在 $(a, b)$ 上一致连续。从而 $f \in C(a, b)$,令 $F(x) = f(x) - x$,则 $F \in C(a, b)$。下面我们对 $F$ 进行分类讨论。

\begin{enumerate}[(1)]
\item 若 $F$ 在 $(a, b)$ 上不变号,则由 $F \in C(a, b)$ 及命题可知,$F$ 要么恒大于零,要么恒小于零。不妨设 $F$ 在 $(a, b)$ 上恒大于零,即 $f(x) > x, \forall x \in (a, b)$。从而
\begin{align*}
x_{n+1} = f(x_n) > x_n, \forall n \in \mathbb{N}。
\end{align*}
即 $\{x_n\}$ 单调递增。又因为 $x_n \in (a, b), \forall n \in \mathbb{N}$,所以由单调有界定理可知 $\{x_n\}$ 收敛。

\item 若 $F$ 在 $(a, b)$ 上变号,则由 $F \in C(a, b)$ 及介值定理可得,存在 $\xi \in (a, b)$,使得 $f(\xi) = \xi$。
若存在 $\xi' \in (a, b)$ 且 $\xi' \neq \xi$,使得 $f(\xi') = \xi'$,则由条件可得到
\begin{align*}
|\xi - \xi'| = |f(\xi) - f(\xi')| < |\xi - \xi'|。
\end{align*}
显然矛盾!因此存在唯一的 $\xi \in (a, b)$,使得 $f(\xi) = \xi$。从而
\begin{align*}
|x_{n+1} - \xi| = |f(x_n) - f(\xi)| < |x_n - \xi|, \forall n \in \mathbb{N}。
\end{align*}
于是 $\{|x_n - \xi|\}$ 单调递减且有下界 $0$,故由单调有界定理可知 $\lim_{n \to \infty} |x_n - \xi| = A \geqslant 0$ 存在。
\begin{enumerate}[(i)]
\item 当 $A = 0$ 时,则由 $\lim_{n \to \infty} |x_n - \xi| = A = 0$ 可得 $\lim_{n \to \infty} x_n = \xi$。
\item 当 $A > 0$ 时,若 $\{x_n\}$ 收敛,则结论已经成立。若 $\{x_n\}$ 发散,则由 $x_n \in (a, b), \forall n \in \mathbb{N}$ 及聚点定理可知,$\{x_n\}$ 至少有一个聚点。若 $\{x_n\}$ 只有一个聚点,则 $\{x_n\}$ 收敛与假设矛盾!因此 $\{x_n\}$ 至少有两个聚点。任取收敛子列 $\{x_{n_k}\} \subset \{x_n\}$,设 $\lim_{k \to \infty} x_{n_k} = B$,则
\begin{align*}
A = \lim_{n \to \infty} |x_n - \xi| = |B - \xi|。
\end{align*}
从而 $B = \xi - A$ 或 $\xi + A$。因此 $\{x_n\}$ 最多有两个聚点 $\xi - A$ 和 $\xi + A$。又因为 $\{x_n\}$ 至少有两个聚点,所以 $\{x_n\}$ 有且仅有两个聚点 $\xi - A$ 和 $\xi + A$。进而\hyperlink{找相邻子列满足条件的方法证明}{一定存在收敛子列 $\{x_{n_k}\} \subset \{x_n\}$},使得
\begin{align*}
\lim_{k \to \infty} x_{n_k} = \xi - A, \lim_{k \to \infty} x_{n_k+1} = \xi + A \text{并且} \hyperlink{xi-A,xi+A在(a,b)中的原因证明}{\xi - A, \xi + A \in (a, b)}.
\end{align*}
从而由条件可知
\begin{align*}
x_{n_k+1} = f(x_{n_k})。
\end{align*}
令 $k \to \infty$,由 $f \in C(a, b)$及归结原则可得
\begin{align*}
\xi + A = f(\xi - A)。
\end{align*}
再结合 $\xi = f(\xi),\xi - A, \xi + A \in (a, b)$ 及条件可得
\begin{align*}
A = |\xi - (\xi + A)| = |f(\xi) - f(\xi - A)| < |\xi - (\xi - A)| = A。
\end{align*}
显然矛盾!故 $A > 0$ 不成立,于是 $A = 0$。再由 (1) 可得 $\lim_{n \to \infty} x_n = \xi$,即 $\{x_n\}$ 收敛,与假设$\{x_n\}$发散矛盾!
\end{enumerate}
\end{enumerate}
\end{proof}

\begin{example}
设
\begin{align*}
x_1 = 1, 
x_2 = \sqrt{\frac{1}{2} + 1}, 
x_3 = \sqrt{\frac{1}{3} + \sqrt{\frac{1}{2} + 1}}, 
x_4 = \sqrt{\frac{1}{4} + \sqrt{\frac{1}{3} + \sqrt{\frac{1}{2} + 1}}}, 
\cdots ,
x_{n+1} = \sqrt{\frac{1}{n+ 1}  + x_n}, \cdots
\end{align*}
证明数列 $\{x_n\}$ 收敛并求出极限值。
\end{example}
\begin{proof}
由条件可得  
\[
x_{n+1}=\sqrt{\frac{1}{n+1}+x_n},\forall n\in \mathbb{N}.
\]
注意到 $x_1\in [1,2)$,假设 $x_k\in [1,2)$,其中 $k\in \mathbb{N}$,则  
\[
1\leqslant \sqrt{\frac{1}{k+1}+1}\leqslant x_{k+1}=\sqrt{\frac{1}{k+1}+x_k}<\sqrt{1+2}<2.
\]
故由数学归纳法可知,$x_n\in [1,2)$,$\forall n\in \mathbb{N}$。于是可设  
\[
\underset{n\rightarrow \infty}{\overline{\lim }}x_n=L\in [1,2],\quad \underset{n\rightarrow \infty}{\underline{\lim }}x_n=l\in [1,2],
\]
又由 $x_{n+1}=\sqrt{\frac{1}{n+1}+x_n},\forall n\in \mathbb{N}$ 可得  
\[
x_{n+1}^{2}=\frac{1}{n+1}+x_n,\forall n\in \mathbb{N}.
\]
两边同时取上、下极限得到  
\[
L^2=\underset{n\rightarrow \infty}{\lim}\frac{1}{n+1}+\underset{n\rightarrow \infty}{\overline{\lim }}x_n=L\Rightarrow L=0\text{ 或 }1,
\]
\[
l^2=\underset{n\rightarrow \infty}{\lim}\frac{1}{n+1}+\underset{n\rightarrow \infty}{\underline{\lim }}x_n=l\Rightarrow l=0\text{ 或 }1.
\]
再结合 $l,L\in [1,2]$ 可得 $L=l=1$。故  
\[
\underset{n\rightarrow \infty}{\lim}x_n=1.
\]
\end{proof}

\begin{example}
已知数列 $\{a_n\}$: $a_1 = 0$, $a_{2m} = \frac{a_{2m-1}}{2}$, $a_{2m+1} = \frac{1}{2} + a_{2m}$, 求 $\{a_n\}$ 的上下极限。
\end{example}
\begin{proof}
由条件可得  
\[
a_{2m+1}=\frac{1}{2}+a_{2m}=\frac{1}{2}+\frac{a_{2m-1}}{2},\quad a_{2m}=\frac{a_{2m-1}}{2}=\frac{1}{2}\left( \frac{1}{2}+a_{2m-2} \right),\forall m\in \mathbb{N}.
\]
即  
\[
a_{2m+1}-\frac{1}{2}a_{2m-1}=\frac{1}{2},\quad a_{2m}-\frac{1}{2}a_{2m-2}=\frac{1}{4},\forall m\in \mathbb{N}.
\]
注意到 $a_1=0,a_2=0$,于是对 $\forall m\in \mathbb{N}$,我们有  
\begin{align*}
a_{2m+1}&=\left( a_{2m+1}-\frac{1}{2}a_{2m-1} \right) +\frac{1}{2}\left( a_{2m-1}-\frac{1}{2}a_{2m-3} \right) +\cdots +\frac{1}{2^{m-2}}\left( a_3-\frac{1}{2}a_1 \right) +\frac{1}{2^{m-1}}a_1 \\
&=\sum_{k=1}^m{\frac{1}{2^{m-k}}\left( a_{2k+1}-\frac{1}{2}a_{2k-1} \right)}+\frac{1}{2^{m-1}}a_1=\sum_{k=1}^m{\frac{1}{2^{m-k+1}}} \\
&=\sum_{k=1}^m{\frac{1}{2^k}}=\frac{\frac{1}{2}-\frac{1}{2^{m+1}}}{1-\frac{1}{2}}\rightarrow 1,\quad m\rightarrow \infty.
\end{align*}

\begin{align*}
a_{2m}&=\left( a_{2m}-\frac{1}{2}a_{2m-2} \right) +\frac{1}{2}\left( a_{2m-2}-\frac{1}{2}a_{2m-4} \right) +\cdots +\frac{1}{2^{m-2}}\left( a_4-\frac{1}{2}a_2 \right) +\frac{1}{2^{m-1}}a_2 \\
&=\sum_{k=2}^m{\frac{1}{2^{m-k}}\left( a_{2k}-\frac{1}{2}a_{2k-2} \right)}+\frac{1}{2^{m-1}}a_2=\sum_{k=2}^m{\frac{1}{2^{m-k+2}}} \\
&=\sum_{k=2}^m{\frac{1}{2^k}}=\frac{\frac{1}{4}-\frac{1}{2^{m+2}}}{1-\frac{1}{2}}\rightarrow \frac{1}{2},\quad m\rightarrow \infty.
\end{align*}
因此  
$\underset{m\rightarrow \infty}{\lim}a_{2m}=\frac{1}{2},\quad \underset{m\rightarrow \infty}{\lim}a_{2m+1}=1$.故$\underset{n\rightarrow \infty}{\overline{\lim }}a_n=1,\underset{n\rightarrow \infty}{\underline{\lim }}a_n=\frac{1}{2}.$
\end{proof}
\begin{remark}
由 $\lim_{m\rightarrow \infty}a_{2m}=\frac{1}{2}, \lim_{m\rightarrow \infty}a_{2m+1}=1$ 可知,数列 $\{ a_n \}$ 的任何由奇数项组成的子列都收敛到 $1$,任何由偶数项组成的子列都收敛到 $\frac{1}{2}$。设收敛子列 $\{ a_{n_k} \} \subset \{ a_n \}$,则

(i)若 $\{ a_{n_k} \}$ 中只含有无穷多奇数项,不含无穷多偶数项,那么一定存在 $K\in \mathbb{N}$,使得对 $\forall k>K$,都存在 $m\in \mathbb{N}$,使得 $n_k = 2m + 1$,即 $a_{n_k} = a_{2m + 1}$。从而 $a_{n_k} = a_{2m + 1} \rightarrow 1, k\rightarrow \infty$。

(ii)若 $\{ a_{n_k} \}$ 中只含有无穷多偶数项,不含无穷多奇数项,那么同理可得 $a_{n_k} \rightarrow \frac{1}{2}, k\rightarrow \infty$。

(iii)若 $\{ a_{n_k} \}$ 中既含有无穷多偶数项,又含有无穷多奇数项,那么一定存在奇偶子列 $\{ a_{2m + 1} \}, \{ a_{2m} \} \subset \{ a_{n_k} \}$,但是 $\lim_{m\rightarrow \infty}a_{2m} = \frac{1}{2} \neq \lim_{m\rightarrow \infty}a_{2m + 1} = 1$。因此 $\{ a_{n_k} \}$ 发散,这与 $\{ a_{n_k} \}$ 收敛矛盾!

故 $\{ a_n \}$ 的任何收敛子列要么收敛到 $1$,要么收敛到 $\frac{1}{2}$,即 $\{ a_n \}$ 有且仅有两个聚点 $1$ 和 $\frac{1}{2}$,因此 $\varliminf_{n\rightarrow \infty}a_n = \frac{1}{2}, \varlimsup_{n\rightarrow \infty}a_n = 1$。
\end{remark}

\begin{example}
设 $f(x)$ 在 $[0, +\infty)$ 上连续且有界, 若 $\forall r \in (-\infty, +\infty)$, $f(x) = r$ 在 $[0, +\infty)$ 上只有有限个根或无根, 证明: $\lim_{x \to +\infty} f(x)$ 存在。
\end{example}
\begin{proof}
反证,假设 $\underset{x\rightarrow +\infty}{\lim}f\left( x \right)$ 不存在。由于 $f\in C\left[ 0,+\infty \right)$ 且在 $\left[ 0,+\infty \right)$ 上有界,因此可设 $\underset{x\rightarrow +\infty}{\overline{\lim }}f\left( x \right) =L<\infty ,\underset{x\rightarrow +\infty}{\underline{\lim }}f\left( x \right) =l<\infty$。又因为 $\underset{x\rightarrow +\infty}{\lim}f\left( x \right)$ 不存在,所以 $l<L$。

任取 $r\in \left( l,L \right)$,则由 $\underset{x\rightarrow +\infty}{\overline{\lim }}f\left( x \right) =L,\underset{x\rightarrow +\infty}{\underline{\lim }}f\left( x \right) =l$ 可知,存在严格递增趋于 $+\infty$ 的非负子列 $\left\{ x_{n_{l_k}} \right\},\left\{ x_{n_{s_k}} \right\}$,使得 $\underset{k\rightarrow \infty}{\lim}f\left( x_{n_k} \right) =L,\underset{k\rightarrow \infty}{\lim}f\left( x_{m_k} \right) =l$。
不妨设 $\left\{ x_{n_{l_k}} \right\},\left\{ x_{n_{s_k}} \right\}$ 满足  
\begin{align}
0<x_{m_k}<x_{n_k}<x_{m_{k+1}}<x_{n_{k+1}},\forall k\in \mathbb{N}。 \label{example1.22-1.1}
\end{align}
于是根据极限的保号性可知,存在 $K\in \mathbb{N}$,使得对 $\forall k>K$,都有  
\[
f\left( x_{n_k} \right) >r>f\left( x_{m_k} \right)。
\]
由 $f\in C\left[ 0,+\infty \right)$ 及连续函数介值定理可知,对 $\forall k>K$,都存在 $\xi _k\in \left( x_{m_k},x_{n_k} \right)$,使得 $f\left( \xi _k \right) =r$。同时由\eqref{example1.22-1.1}式可知  
\[
0<x_{m_k}<\xi _k<x_{n_k}<x_{m_{k+1}}<\xi _{k+1}<x_{n_{k+1}},\forall k\in \mathbb{N}。
\]
即存在各项互异的非负数列 $\left\{ \xi _k \right\} _{k=1}^{\infty}$,使得 $f\left( \xi _k \right) =r$。这与 $f\left( x \right) =r$ 在 $\left[ 0,+\infty \right)$ 上至多只有有限个根矛盾!故 $\underset{x\rightarrow +\infty}{\lim}f\left( x \right)$ 存在。
\end{proof}

\begin{example}
设
\begin{align*}
a_{n+1} = \lambda a_n + \frac{1}{a_n}, a_1 > 0, \lambda \in (0,1),
\end{align*}
求 $\lim_{n \to \infty} a_n$.
\end{example}
\begin{note}
单调性分析法也证明,不过较为繁琐.下述证明利用的是\hyperref[Basis of Analytics-example:类递减模型]{类递减模型}的证明想法.数列下界显然,只需待定上界,形式计算,确定数列上界,然后隔项抽子列,再利用上下极限即可得证.
\end{note}
\begin{proof}
取 $m=\min \left\{ a_1,2\sqrt{\lambda} \right\}$, 再令 $M=\max \left\{ a_1,a_2,\frac{1}{\left( 1-\lambda \right) m} \right\}$。注意到 $a_1\geqslant m$, 假设 $a_k\geqslant m$, 其中 $k\in \mathbb{N}$, 则由均值不等式可得
\begin{align*}
a_{k+1}=\lambda a_k+\frac{1}{a_k}\geqslant 2\sqrt{\lambda}\geqslant m。
\end{align*}
故由数学归纳法可得 $a_n\geqslant m,\forall n\in \mathbb{N}$。又注意到 $a_2\leqslant M$, 假设 $a_l\leqslant M$, 由 $M=\max \left\{ a_1,a_2,\frac{1}{\left( 1-\lambda \right) m} \right\}$ 可知
\begin{align*}
M\geqslant \frac{1}{\left( 1-\lambda \right) m}\Rightarrow \frac{1}{m}\leqslant \left( 1-\lambda \right) M。
\end{align*}
于是
\begin{align*}
a_{l+1}=\lambda a_l+\frac{1}{a_l}\leqslant \lambda M+\frac{1}{m}\leqslant \lambda M+\left( 1-\lambda \right) =M。
\end{align*}
因此由数学归纳法可知 $a_n\leqslant M,\forall n\in \mathbb{N}$。故 $\left\{ a_n \right\}$ 有界。从而可设 $\underset{n\rightarrow \infty}{\overline{\lim }}a_n=L\in \left[ m,M \right] ,\underset{n\rightarrow \infty}{\underline{\lim }}a_n=l\in \left[ m,M \right]$。又因为 $a_{n+1}=\lambda a_n+\frac{1}{a_n},\forall n\in \mathbb{N}$, 所以令 $n\rightarrow \infty$ 并分别取上、下极限可得
\begin{align*}
L\leqslant \lambda L+\frac{1}{l},l\geqslant \lambda l+\frac{1}{L}\Rightarrow Ll\leqslant \frac{1}{1-\lambda},Ll\ge \frac{1}{1-\lambda}\Rightarrow Ll=\frac{1}{1-\lambda}。
\end{align*}
由 $\underset{n\rightarrow \infty}{\overline{\lim }}a_n=L$ 可知, 一定存在 $\left\{ a_{n_k} \right\} \subset \left\{ a_n \right\}$, 使得
\begin{align*}
\underset{n\rightarrow \infty}{\lim}a_{n_k+1}=L,\underset{n\rightarrow \infty}{\lim}a_{n_k}=s\in \left[ l,L \right]。
\end{align*}
又由条件可知 $a_{n_k+1}=\lambda a_{n_k}+\frac{1}{a_{n_k}},\forall k\in \mathbb{N}$。于是令 $k\rightarrow \infty$, 再结合上式可得
\begin{align*}
L=\lambda s+\frac{1}{s}\leqslant \lambda L+\frac{1}{l}=\lambda L+\left( 1-\lambda \right) L=L。
\end{align*}
因此 $L=s=l$, 故 $\underset{n\rightarrow \infty}{\lim}a_n$ 存在, 设 $\underset{n\rightarrow \infty}{\lim}a_n=a\geqslant m>0$, 则对 $a_{n+1}=\lambda a_n+\frac{1}{a_n}$ 两边同时取极限得到
\begin{align*}
a=\lambda a+\frac{1}{a}\Rightarrow \underset{n\rightarrow \infty}{\lim}a_n=a=\frac{1}{\sqrt{1-\lambda}}.
\end{align*}
\end{proof}

\begin{example}

\end{example}
\begin{proof}

\end{proof}

\begin{example}

\end{example}
\begin{proof}

\end{proof}

\begin{example}

\end{example}
\begin{proof}

\end{proof}

\begin{example}

\end{example}
\begin{proof}

\end{proof}

\begin{example}

\end{example}
\begin{proof}

\end{proof}

\begin{example}

\end{example}
\begin{proof}

\end{proof}

\begin{example}

\end{example}
\begin{proof}

\end{proof}

\begin{example}

\end{example}
\begin{proof}

\end{proof}

\begin{example}

\end{example}
\begin{proof}

\end{proof}

\begin{example}

\end{example}
\begin{proof}

\end{proof}

\begin{example}

\end{example}
\begin{proof}

\end{proof}

\begin{example}

\end{example}
\begin{proof}

\end{proof}

\begin{example}

\end{example}
\begin{proof}

\end{proof}

\begin{example}

\end{example}
\begin{proof}

\end{proof}

\begin{example}

\end{example}
\begin{proof}

\end{proof}

\begin{example}

\end{example}
\begin{proof}

\end{proof}

\begin{example}

\end{example}
\begin{proof}

\end{proof}







\chapter{高等代数习题}


























\end{document}

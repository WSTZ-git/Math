\documentclass[11pt]{elegantbook}

\title{Real and Complex Analysis}
\subtitle{\,\,}

\author{Wenjie\,\,Zou}
\institute{Elegant\LaTeX{} Program}
\date{Oct. 23, 2024}
\version{4.5}
\bioinfo{Bio}{Information}


\extrainfo{\,\,}


\setcounter{tocdepth}{3}

\logo{logo-blue.png}
\cover{cover.png}

% extract package and command
\usepackage{mystyle-en}

\begin{document}

\maketitle

\frontmatter
\tableofcontents

\mainmatter

\everymath{\displaystyle} % 让全文的行内公式都显示行间公式效果


\chapter{Prologue:The Exponential Function}

\begin{definition}[Exponential Function]\label{definition:Exponential Function}
  Exponential function is defined , for every complex number z , by the formula
  \begin{align}\label{equation:1.1}
    \exp \left( z \right) =\sum_{n=0}^{\infty}{\frac{z^n}{n!}} .
  \end{align}
  We define the number $e$ to be $\exp(1)$,and shall usually replace $\exp(z)$ by the customary shorter expression $e^z$.Note that $e^0=\exp(0)=1$,by\eqref{equation:1.1}.
\end{definition}

\begin{property}\label{The properties of exponential function}

\textbf{(a)} The series \eqref{equation:1.1} converges absoultely for every z and converges uniformly on every bounded subset of the complex plane . Thus $\exp$ is a contionuous function .

\textbf{(b)} For all complex numbers a and b , we have
\begin{align}
  \exp(a)\exp(b)=\exp(a+b) .
\end{align}
\end{property}
\begin{proof}
  (a) is obvious . Next proof (b) .
  
  The absoulte convergence of \eqref{equation:1.1} and Cauchy's theorem shows that computation 
\begin{align*}
  &\exp \left( a \right) \exp \left( b \right) =\sum_{k=0}^{\infty}{\frac{a^k}{k!}}\sum_{m=0}^{\infty}{\frac{b^m}{m!}}\xlongequal[]{\text{Add them in square order}}\sum_{k=0}^{\infty}{\sum_{m=0}^{\infty}{\frac{a^kb^m}{k!m!}}}
\\
&\xlongequal[\text{Add them up in diagonal order}]{Cauchy's\,\,theorem\left( \text{华师大数分下册}P20 \right)}\sum_{n=0}^{\infty}{\sum_{\substack{
	m+k=n\\
	m,k\in \mathbb{N}\\
}}{\frac{a^kb^m}{k!m!}}}=\sum_{n=0}^{\infty}{\sum_{k=0}^n{\frac{a^kb^{n-k}}{k!\left( n-k \right) !}}}
\\
&=\sum_{n=0}^{\infty}{\frac{1}{n!}\sum_{k=0}^n{\frac{n!a^kb^{n-k}}{k!\left( n-k \right) !}}}=\sum_{n=0}^{\infty}{\frac{1}{n!}\sum_{k=0}^n{\mathrm{C}_{n}^{k}a^kb^{n-k}}}
\\
&=\sum_{n=0}^{\infty}{\frac{\left( a+b \right) ^n}{n!}}=\exp \left( a+b \right) .
\end{align*}

\end{proof}

\begin{theorem}[The conclusions of exponential function]\label{theorem:The conclusions of exponential function}
  (a) For every complex z we have $e^z \ne 0$.

  (b) $\exp$ is its own derivative : $\exp'(z)=\exp(z)$.

  (c) The resticton of $\exp$ to the real axis is a monotonically increasing positive function , and 
  \begin{align*}
    e^x\rightarrow \infty\,\,as\,\, x\rightarrow \infty\,\,,\,\, e^x\rightarrow 0\,\,as\,\, x\rightarrow -\infty .
  \end{align*}

  (d)There exists a positive number $\pi$ such that $e^{\frac{\pi}{2}}=i$ and such that $e^z=1$ if and only if $z/(2\pi i)$ is an integer .

  (e) $\exp$ is a periodic function , with period $2\pi i$ .

  (f) The mapping $t\to e^{iz}$ maps tha the real axis onto unit circle .

  (g) If $w$ is a complex number and $w\ne 0$ , then $w=e^z$ for some z .
\end{theorem}
\begin{proof}
  
\end{proof}









\end{document}

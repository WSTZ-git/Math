\documentclass[lang=cn,newtx,10pt,scheme=chinese]{elegantbook}

\title{分析学技巧积累}
\subtitle{\,\,}

\author{邹文杰}
\institute{无}
\date{2024/10/25}
\version{ElegantBook-4.5}
\bioinfo{自定义}{信息}

\extrainfo{宠辱不惊,闲看庭前花开花落;
\\
去留无意,漫随天外云卷云舒.}

\setcounter{tocdepth}{3}

\logo{logo-blue.png}
\cover{cover.png}

% 本文档额外使用的宏包和命令
\usepackage{mystyle-cn}

\begin{document}

\maketitle
\frontmatter

\tableofcontents

\mainmatter
\everymath{\displaystyle} % 让全文的行内公式都显示行间公式效果


\part{一元实变函数的Lebesgue积分}

\chapter{集合、映射与关系的预备知识}

\section{集合的基本概念}

\begin{definition}[集合的基本概念]\label{definition:集合的基本概念}
\begin{enumerate}
  \item 对于集合\(A\),元素\(x\)是\(A\)的成员关系记为\(x\in A\),而\(x\)不是\(A\)的成员关系记为\(x\notin A\)。我们常说\(A\)的一个成员属于\(A\)且称\(A\)的成员是\(A\)中的一个点。通常集合用花括号表示,因此\(\{x|关于x的陈述\}\)是使得关于\(x\)的陈述成立的所有元素\(x\)的集合。若两个集合有相同的成员,我们说它们相同.

  \item 令\(A\)和\(B\)为集合。若\(A\)的每个成员也是\(B\)的成员,我们称\(A\)为\(B\)的\textbf{子集},记之为\(A\subseteq B\),也说\(A\)包含于\(B\)或\(B\)包含\(A\)。\(B\)的子集\(A\)称为\(B\)的\textbf{真子集}.

  \item 若\(A\neq B\)。\(A\)和\(B\)的\textbf{并},记为\(A\cup B\),是所有或者属于\(A\)或者属于\(B\)的点的集合,即\(A\cup B = \{x|x\in A或x\in B\}\)。

  \item \(A\)和\(B\)的\textbf{交},记为\(A\cap B\),是所有同时属于\(A\)和\(B\)的点的集合,即\(A\cap B = \{x|x\in A且x\in B\}\)。

  \item \(A\)在\(B\)中的\textbf{补},记为\(B - A\),是\(B\)中那些不在\(A\)中的点的集合,即\(B - A=\{x\in B且x\notin A\}\)。若在特别的讨论中所有的集合是参考集\(X\)的子集,我们常简单地称\(X - A\)为\(A\)的补。

  \item 没有任何成员的集合称为\textbf{空集},记为\(\varnothing\)。不等于空集的集合称为非空的。

  \item 我们称只有一个成员的集合为\textbf{单点集}。

  \item 给定集合\(X\),\(X\)的所有子集的集合记为\(\mathcal{P}(X)\)或\(2^X\),称之为\(X\)的\textbf{幂集}。
\end{enumerate}
\end{definition}
\begin{remark}
  为了避免考虑集合的集合时可能产生混淆,我们常用词“族”或“簇”作为“集”的同义词。我通常称集合的集合为\textbf{集族}或\textbf{集簇}.
\end{remark}

\begin{definition}[集族的并和交]\label{definition:集族的并和交}
  令\(\mathcal{F}\)为集族.
  \begin{enumerate}
    \item \(\mathcal{F}\)的并,记为\(\bigcup_{F\in\mathcal{F}}F\),定义为属于\(\mathcal{F}\)中的至少一个集合的点的集合。

    \item \(\mathcal{F}\)的交,记为\(\bigcap_{F\in\mathcal{F}}F\),定义为属于\(\mathcal{F}\)中的每个集合的点的集合。
    
    \item 若集族\(\mathcal{F}\)中的任何两个集合的交是空的,集族\(\mathcal{F}\)称为是\textbf{不交的}.

    \item 若集族\(\mathcal{F}\)是不交的,则\(\mathcal{F}\)的并称为是\textbf{无交并}或\textbf{没交并},记为$\bigsqcup_{F\in \mathcal{F}}{F}$.
  \end{enumerate}
\end{definition}

\begin{theorem}[De Morgan等式]\label{theorem:De Morgan等式}
  令$X$为集合,\(\mathcal{F}\)为集族,则一定有
\[
X-\left[\bigcup_{F\in\mathcal{F}}F\right]=\bigcap_{F\in\mathcal{F}}[X- F],\quad X-\left[\bigcap_{F\in\mathcal{F}}F\right]=\bigcup_{F\in\mathcal{F}}[X- F]
\]
即并的补是补的交,且交的补是补的并.
\end{theorem}

\begin{definition}[指标集]\label{definition:指标集}
  对于集合\(\Lambda\),假定对每个\(\lambda\in\Lambda\),存在已定义的\(E_{\lambda}\)。令\(\mathcal{F}\)为集族\(\{E_{\lambda}|\lambda\in\Lambda\}\)。我们写作\(\mathcal{F}=\{E_{\lambda}\}_{\lambda\in\Lambda}\)且称$\Lambda$中的元素为\(\mathcal{F}\)的用\textbf{指标集}(或\textbf{参数集})\(\Lambda\)标记的\textbf{指标}(或\textbf{参数化}).
\end{definition}


\section{集合之间的映射}

\begin{definition}[映射的基本概念]\label{definition:映射的基本概念}
  给定两个集合\(A\)和\(B\),从\(A\)到\(B\)的映射或函数意味着对\(A\)的每个成员指派\(B\)的一个成员给它。在\(B\)是实数集的情形下,我们总是用“函数”这个词。一般我们记这样的映射为\(f:A\to B\),而对\(A\)的每个成员\(x\),我们记\(f(x)\)为\(B\)中指派给\(x\)的成员。
  \begin{enumerate}
    \item  对于\(A\)的子集\(A'\),我们定义\(f(A') = \{b|b = f(a), a\text{为}A'\text{的某个成员}\}\):\(f(A')\)称为\(A'\)在\(f\)下的\textbf{象}.

    \item 我们称集合\(A\)为函数\(f\)的\textbf{定义域}.

    \item 我们称\(f(A)\)为\(f\)的\textbf{象或值域}。
  \end{enumerate}
  
\end{definition}

\begin{definition}[满射、单射和双射]\label{definition:满射、单射和双射}
\begin{enumerate}
  \item 若\(f(A)=B\),函数\(f\)称为是\textbf{映上的}或\textbf{满射}.

  \item 若对\(f(A)\)的每个成员\(b\)恰有\(A\)的一个成员\(a\)使得\(b = f(a)\),函数\(f\)称为是\textbf{一对一的}或\textbf{单射}.

  \item 既是一对一又是映上的映射\(f:A\to B\)称为是\textbf{可逆的}或\textbf{双射},我们说该映射建立了集合\(A\)与\(B\)之间的一一对应.
\end{enumerate}
\end{definition}  

\begin{definition}[可逆映射的逆]\label{definition:可逆映射的逆}
  给定一个可逆映射\(f:A\to B\),对\(B\)中的每个点\(b\),恰好存在\(A\)中的一个成员\(a\)使得\(f(a)=b\),它被记为\(f^{-1}(b)\)。这个指派定义了映射\(f^{-1}:B\to A\),称之为\(f\)的\textbf{逆}.
\end{definition}

\begin{definition}[对等的集合]\label{definition:对等的集合}
  两个集合\(A\)和\(B\)称为是\textbf{对等的},记为$A\sim B$,若存在从\(A\)映到\(B\)的可逆映射.
\end{definition}
\begin{remark}
  从集合论的观点看,对等的两个集合是不可区分的.
\end{remark}

\begin{proposition}[可逆映射的复合是可逆的]\label{proposition:可逆映射的复合是可逆的}
  给定两个映射\(f:A\to B\)和\(g:C\to D\)使得\(f(A)\subseteq C\),则复合\(g\circ f:A\to D\)定义为对每个\(x\in A\),\([g\circ f](x)=g(f(x))\)。不难看出\textbf{可逆映射的复合是可逆的}。
\end{proposition}

\begin{definition}[恒等映射]\label{definition:恒等映射}
  对于集合\(D\),定义\textbf{恒等映射}\(\text{id}_D:D\to D\)为对所有\(x\in D\),\(\text{id}_D(x)=x\)。
\end{definition}

\begin{proposition}[可逆映射的充要条件]\label{proposition:可逆映射的充要条件1}
映射\(f:A\to B\)是可逆的,当且仅当存在映射\(g:B\to A\)使得
\[
g\circ f = \text{id}_A且f\circ g = \text{id}_B.
\]
\end{proposition}

\begin{definition}[原象]\label{definition:原象的定义}
  即便映射\(f:A\to B\)不是可逆的,对于集合\(E\),我们定义\(f^{-1}(E)\)为集合\(\{a\in A|f(a)\in E\}\),称之为\(E\)在\(f\)下的\textbf{原象}.
\end{definition}

\begin{proposition}[原像的性质]\label{proposition:原像的性质}
  我们有下面有用的性质:对于任何两个集合\(E_1\)和\(E_2\),
\[
f^{-1}(E_1\cup E_2)=f^{-1}(E_1)\cup f^{-1}(E_2),\quad f^{-1}(E_1\cap E_2)=f^{-1}(E_1)\cap f^{-1}(E_2)
\]
与
\[
f^{-1}(E_1- E_2)=f^{-1}(E_1)- f^{-1}(E_2).
\]
\end{proposition}

\begin{definition}[映射的限制]\label{definition:映射的限制}
  对于映射\(f:A\to B\)和它的定义域\(A\)的一个子集\(A'\),\(f\)在\(A'\)上的\textbf{限制},记为\(f|_{A'}\),是从\(A'\)到\(B\)的映射,它将\(f(x)\)指派给每个\(x\in A'\).
\end{definition}



\section{等价关系、选择公理以及Zorn引理}

\begin{definition}[笛卡尔积]\label{}
  给定两个非空集\(A\)和\(B\),\(A\)和\(B\)的\textbf{笛卡尔积},记为\(A\times B\),定义为所有有序对\((a, b)\)的族,其中\(a\in A\)而\(b\in B\),且我们考虑\((a, b)=(a', b')\)当且仅当\(a = a'\)且\(b = b'\).
\end{definition}

\begin{definition}[关系及其自反性、对称性、传递性]\label{definition:关系及其自反性、对称性、传递性}
  对于非空集合\(X\),我们称\(X\times X\)的子集\(R\)为\(X\)上的一个\textbf{关系},且写作\(xRx'\).
  \begin{enumerate}
    \item 若\((x, x')\)属于\(R\).关系\(R\)称为\textbf{自反的},若对所有\(x\in X\)有\(xRx\);

    \item 若\((x, x')\)属于\(R\).关系\(R\)称为\textbf{对称的},若\(x'Rx\)则\(xRx'\);
    
    \item 若\((x, x')\)属于\(R\).关系\(R\)称为\textbf{传递的},若\(xRx'\)且\(x'Rx''\)则\(xRx''\).  
  \end{enumerate}
\end{definition}

\begin{definition}[等价关系]\label{definition:等价关系}
集合\(X\)上的关系\(R\)称为\textbf{等价关系},若它是自反的、对称的和传递的。
\end{definition}

\begin{definition}[等价类]\label{definition:等价类}
给定集合\(X\)上的等价关系\(R\),对每个\(x\in X\),集合\(R_x = \{x'|x'\in X, xRx'\}\)称为\(x\)(关于\(R\))的\textbf{等价类}.

集合$X$中所有元素(关于$R$)的等价类构成的集合称为$X$(关于$R$)的\textbf{等价类族},记为\(X/R\).
\end{definition}

\begin{proposition}[等价类的性质]\label{proposition:等价类的性质}
  给定集合\(X\)上的等价关系\(R\),
\begin{enumerate}[(1)]
  \item\label{等价类的性质1} \(R_x = R_{x'}\)当且仅当\(xRx'\).

  \item\label{等价类的性质2} 等价类族\(X/R\)是不交的.

  \item\label{等价类的性质3} \(X/R\)是\(X\)的非空子集的不交族,其并是\(X\).即$X=\bigsqcup_{x\in X}{R_x}=\bigsqcup_{F\in X/R}{F}$.

  \item (反过来)给定\(X\)的非空子集的不交族\(\mathcal{F}\),其并是\(X\),属于\(\mathcal{F}\)中的同一个集的关系是\(X\)上使得\(\mathcal{F}=X/R\)的等价关系\(R\).
\end{enumerate}
\end{proposition}
\begin{proof}
\begin{enumerate}[(1)]
  \item 由\(R\)是对称的和传递的\hyperref[等价类的性质1]{等价类的性质\ref{等价类的性质1}}容易验证.

  \item 由关系\(R\)是自反的容易验证.

  \item 
\end{enumerate}
\end{proof}


\begin{definition}[集合的势]\label{definition:集合的势}
  给定集合\(X\),对等关系是\(X\)的所有子集组成的族\(2^X\)上的等价关系.一个集合关于对等关系的等价类称为该集合的\textbf{势}或\textbf{基数}.

  换句话说,设集合$A$和$B$,若$A\sim B$,就可以称$A$与$B$具有相同\textbf{势}或\textbf{基数}.
\end{definition}

\begin{definition}[选择函数]\label{definition:选择函数}
  令\(\mathcal{F}\)为非空集的非空簇。\(\mathcal{F}\)上的一个\textbf{选择函数}\(f\)是从\(\mathcal{F}\)到\(\bigcup_{F\in\mathcal{F}}F\)的函数,它具有以下性质:对\(\mathcal{F}\)中的每个集合\(F\),\(f(F)\)是\(F\)的一个成员.
\end{definition}

\begin{axiom}[Zermelo选择公理]\label{axiom:Zermelo选择公理}
  令\(\mathcal{F}\)为非空集的非空族,则\(\mathcal{F}\)上存在选择函数.
\end{axiom}

\begin{definition}[序关系]\label{definition:序关系}
  非空集合\(X\)上的关系\(R\)称为\textbf{偏序},若它是自反的、传递的,且对\(X\)中的\(x\),\(x'\) 
若\(xRx'\)且\(x'Rx\),则\(x = x'\).

\(X\)的子集\(E\)称为是\textbf{全序的},若对\(E\)中的\(x\),\(x'\),或者\(xRx'\)或者\(x'Rx\).
\begin{enumerate}
  \item \(X\)的成员\(x\)称为是\(X\)的子集\(E\)的一个\textbf{上界},若对所有\(x'\in E\),都有\(x'Rx\);
  
  \item \(X\)的成员\(x\)称为\textbf{最大的},若\(X\)中使得\(xRx'\)的唯一成员是\(x' = x\).
\end{enumerate}
\end{definition}
\begin{note}
对于集簇\(\mathcal{F}\)和\(A\),\(B\in\mathcal{F}\),定义\(ARB\),若\(A\subseteq B\)。\textbf{集合的被包含关系}是\(\mathcal{F}\)的偏序。观察到\(\mathcal{F}\)中的集合\(F\)是\(\mathcal{F}\)的子簇\(\mathcal{F}'\)的一个上界,若\(\mathcal{F}'\)中的每个集合是\(F\)的子集;而\(\mathcal{F}\)中的集合\(F\)是最大的,若它不是\(\mathcal{F}\)中任何集合的真子集。

类似地,给定集簇\(\mathcal{F}\)和\(A\),\(B\in\mathcal{F}\),定义\(ARB\),若\(B\subseteq A\)。\textbf{集合的包含关系}是\(\mathcal{F}\)的偏序。观察到\(\mathcal{F}\)中的集合\(F\)是\(\mathcal{F}\)的子簇\(\mathcal{F}'\)的一个上界,若\(\mathcal{F}'\)的每个集合包含\(F\);而\(\mathcal{F}\)中的集合\(F\)是最大的,若它不真包含\(\mathcal{F}\)中的任何集合。
\end{note}



\begin{lemma}[Zorn引理]\label{theorem:Zorn引理}
  令\(X\)为偏序集.它的每个全序子集有一个上界.则\(X\)有一个最大元.
\end{lemma}

我们已定义了两个集合的笛卡尔积。对一般的参数化集族定义笛卡尔积是有用的。对于由集合\(\Lambda\)参数化的集族\(\{E_{\lambda}\}_{\lambda\in\Lambda}\)的笛卡尔积,记为\(\prod_{\lambda\in\Lambda}E_{\lambda}\),定义为从\(\Lambda\)到\(\bigcup_{\lambda\in\Lambda}E_{\lambda}\)使得对每个\(\lambda\in\Lambda\),\(f(\lambda)\)属于\(E_{\lambda}\)的函数\(f\)的集合。显然选择公理等价于非空集的非空簇的笛卡尔积是非空的这一断言。注意到笛卡尔积是对参数化的集簇定义的,而相同的簇的两个不同的参数化将有不同的笛卡尔积。笛卡尔积的这个一般定义与对两个集合给出的定义一致。事实上,考虑两个非空集\(A\)和\(B\)。定义\(\Lambda=\{\lambda_1,\lambda_2\}\),其中\(\lambda_1\neq\lambda_2\),接着定义\(E_{\lambda_1}=A\)与\(E_{\lambda_2}=B\)。该映射将有序对\((f(\lambda_1),f(\lambda_2))\)指派给函数\(f\in\prod_{\lambda\in\Lambda}E_{\lambda}\)是一个将笛卡尔积\(\prod_{\lambda\in\Lambda}E_{\lambda}\)映到有序对族\(A\times B\)的可逆映射,因此这两个集合是对等的。对于两个集合\(E\)和\(\Lambda\),对所有\(\lambda\in\Lambda\)定义\(E_{\lambda}=E\),则笛卡尔积\(\prod_{\lambda\in\Lambda}E_{\lambda}\)等于由所有从\(\Lambda\)到\(E\)的映射组成的集合且记为\(E^{\Lambda}\)。







\chapter{实数集:集合、序列与函数}

\section{域、正性以及完备性公理}

假设给定实数集\(\mathbb{R}\),使得对于每对实数\(a\)和\(b\),存在有定义的实数\(a + b\)和\(ab\),分别称为\(a\)和\(b\)的和与积。它们满足以下的域公理、正性公理与完备性公理。

\begin{axiom}[域公理]\label{axiom:域公理}
  \textbf{加法的交换性}:对所有实数\(a\)和\(b\),
\[a + b = b + a\]
\textbf{加法的结合性}:对所有实数\(a\),\(b\)和\(c\),
\[(a + b)+c = a+(b + c)\]
\textbf{加法的单位元} :存在实数,记为\(0\),使得对所有实数\(a\),
\[0 + a = a + 0 = a\]
\textbf{加法的逆元} :对每个实数\(a\),存在实数\(b\)使得
\[a + b = 0\]
\textbf{乘法的交换性} :对所有实数\(a\)和\(b\),
\[ab = ba\]
\textbf{乘法的结合性} :对所有实数\(a\),\(b\)和\(c\),
\[(ab)c = a(bc)\]
\textbf{乘法的单位元} :存在实数,记为\(1\),使得
对所有实数\(a\),\(1a = a1 = a\)
 乘法的逆元 :对每个实数\(a\neq0\),存在实数\(b\)使得
\[ab = 1\]
\textbf{分配性} :对所有实数\(a\),\(b\)和\(c\),
\[a(b + c)=ab + ac\]
\textbf{非平凡性假设}:
\[1\neq0\]

满足上述公理的任何集合称为\textbf{域}.从加法的交换性可以得出加法的单位元\(0\)是唯一的,从乘法的交换性得出乘法的单位元\(1\)也是唯一的。加法的逆元和乘法的逆元也是唯一的。我们记\(a\)的加法的逆为\(-a\),且若\(a\neq0\),记它的乘法逆为\(a^{-1}\)或\(1/a\).
\end{axiom}
\begin{remark}
  若有一个域,我们能实施所有初等代数的运算,包括解线性方程组。我们不加声明地使用这些公理的多种推论.
\end{remark}

\begin{axiom}[正性公理]\label{axiom:正性公理}
存在称为正数的实数集,记为\(\mathcal{P}\)。它有以下两个性质:
\begin{enumerate}[(1)]
  \item 若\(a\)和\(b\)是正的,则\(ab\)和\(a + b\)也是正的。

  \item 对于实数\(a\),以下三种情况恰有一种成立:
\[a\text{是正的},\quad -a\text{是正的},\quad a = 0.\]
\end{enumerate}
\end{axiom}

\begin{definition}[实数的序]\label{definition:实数的序}
对于实数\(a\)和\(b\),
\begin{enumerate}
  \item 定义\(a>b\)意味着\(a - b\)是正的.
  \item 定义\(a\geqslant b\)意味着\(a>b\)或\(a = b\).
  \item 定义\(a<b\)意味着\(b>a\).
  \item 定义\(a\leqslant b\)意味着\(b\geqslant a\).
\end{enumerate}
\end{definition}
\begin{remark}
  实数的序的定义是根据\hyperref[axiom:正性公理]{实数的正性公理}给出的.
\end{remark}

\begin{definition}[实数的区间]\label{definition:实数的区间}
  给定实数\(a\)和\(b\)满足\(a<b\),我们定义\((a, b)=\{x|a<x<b\}\),且说\((a, b)\)的点落在\(a\)与\(b\)之间.

  我们称非空实数集\(I\)为\textbf{区间},若对\(I\)中任意两点,所有落在这两点之间的点也属于\(I\)。当然,集合\((a, b)\)是区间。以下集合也是区间:
  \begin{align}\label{equation:有界区间}
    (a, b)=\{x|a<x<b\};[a,b] = \{x|a\leqslant x\leqslant b\};[a,b) = \{x|a\leqslant x<b\};(a,b] = \{x|a<x\leqslant b\}.
  \end{align}
\end{definition}
\begin{note}
  所有有界区间都是\eqref{equation:有界区间}式列出的形式.
\end{note}

\begin{definition}[上界和下界]\label{definition:上界和下界}
\begin{enumerate}
  \item 非空实数集\(E\)称为\textbf{有上界},若存在实数\(b\),使得对所有\(x\in E\),\(x\leqslant b\):数\(b\)称为\(E\)的\textbf{上界}.
  \item 非空实数集\(E\)称为\textbf{有下界},若存在实数\(b\),使得对所有\(x\in E\),\(x\geqslant b\):数\(b\)称为\(E\)的\textbf{下界}.
\end{enumerate}
\end{definition}

\begin{axiom}[完备性公理]\label{axiom:完备性公理}
  令\(E\)为有上界的非空实数集。则在\(E\)的上界的集合中有一个最小的上界。
\end{axiom}
\begin{note}
  有上界的集合未必有最大的成员。但完备性公理断言它一定有一个最小的上界.
\end{note}

\begin{definition}[上下确界]\label{definition:上下确界}
\begin{enumerate}
  \item 有上界的非空实数集\(E\)有\textbf{最小上界},记为\(\text{l.u.b. }E\)。\(E\)的最小上界通常称为\(E\)的\textbf{上确界}且记为\(\sup E\)。

  \item 有下界的非空实数集\(E\)有\textbf{最大下界},记为\(\text{g.l.b. }E\)。\(E\)的最大下界通常称为\(E\)的\textbf{下确界}且记为\(\inf E\)。

  \item 一个非空实数集称为\textbf{有界的},若它既有下界又有上界。
\end{enumerate}
\end{definition}
\begin{remark}
  有上界的非空实数集\(E\)的最小上界和有下界的非空实数集\(E\)的最大下界的存在性由\hyperref[axiom:完备性公理]{完备性公理}保证,因此这个定义是良定义.
\end{remark}

\begin{definition}[实数的绝对值]\label{definition:绝对值}
  定义实数\(x\)的绝对值\(\vert x\vert\)为:若\(x\geqslant0\)则等于\(x\),若\(x<0\)则等于\(-x\)。
\end{definition}

\begin{theorem}[三角不等式]\label{theorem:三角不等式}
对任何实数对\(a\)和\(b\),都有
  \[
    \left| a+b \right|\leqslant \left| a \right|+\left| b \right|.
  \]
\end{theorem}

\begin{definition}[扩充的实数]\label{definition:扩充的实数}
  引入符号\(\infty\)和\(-\infty\)并对所有实数\(x\)写\(-\infty<x<\infty\)是方便的。我们称集合\(\mathbb{R}\cup\{\pm\infty\}\)为\textbf{扩充的实数}.
\end{definition}
\begin{note}
  我们将定义实数序列的极限,而允许极限是扩充的实数是方便的.
\end{note}

\begin{definition}[扩充的实数的上下确界]\label{definition:扩充的实数的上下确界}
\begin{enumerate}
  \item 若非空实数集\(E\)没有上界,我们定义它的上确界为\(\infty\)或$+\infty$。定义空集的上确界为\(-\infty\).
  \item 若非空实数集\(E\)没有下界,我们定义它的下确界为$-\infty$.定义空集的下确界为\(+\infty\).
\end{enumerate}
\end{definition}
\begin{note}
  因此每个实数集有一个属于扩充的实数的上确界和下确界.
\end{note}

\begin{proposition}[扩充的实数关于和与积的性质]\label{proposition:扩充的实数关于和与积的性质}
\begin{enumerate}
  \item \(\infty+\infty=\infty\),\(-\infty - \infty=-\infty\).
  \item 对每个实数\(x\),\(x+\infty=\infty\)而\(x - \infty=-\infty\).
  \item 若\(x>0\),\(x\cdot\infty=\infty\)而\(x\cdot(-\infty)=-\infty\).
  \item 若\(x<0\),\(x\cdot\infty=-\infty\)而\(x\cdot(-\infty)=\infty\)。
\end{enumerate}
\end{proposition}
\begin{remark}
  注意到收敛到实数的实数序列的许多性质在极限是\(\pm\infty\)时继续成立,例如,和的极限是极限的和且积的极限是极限的积.因此我们容易验证这些扩充的实数关于和与积的性质.
\end{remark}

\begin{definition}[无界区间]\label{definition:无界区间}
  定义\((-\infty,\infty)=\mathbb{R}\)。对于\(a\),\(b\in\mathbb{R}\),定义
\[
(a,\infty)=\{x\in\mathbb{R}|a < x\},\quad (-\infty,b)=\{x\in\mathbb{R}|x < b\}
\]
与
\[
[a,\infty)=\{x\in\mathbb{R}|a\leqslant x\},\quad (-\infty,b]=\{x\in\mathbb{R}|x\leqslant b\}.
\]
\end{definition}
\begin{note}
  上面形式的集合是无界区间。从\(\mathbb{R}\)的完备性可以推出所有无界区间是上述形式的一种,而所有有界区间都是\eqref{equation:有界区间}式列出的形式.
\end{note}

\begin{example}\label{example:1.12311}
令\(a\)和\(b\)为实数.
\begin{enumerate}[(i)]
  \item\label{example:1.12311-i} 证明:若\(ab = 0\),则\(a = 0\)或\(b = 0\).
  \item\label{example:1.12311-ii}  验证\(a^{2}-b^{2}=(a - b)(a + b)\),并从(i)部分推出:若\(a^{2}=b^{2}\),则\(a = b\)或\(a=-b\).
  \item\label{example:1.12311-iii} 令\(c\)为正实数. 定义\(E = \{x\in\mathbb{R}|x^{2}<c\}\). 验证\(E\)是非空的且有上界. 定义\(x_0=\sup E\). 证明\(x_0^{2}=c\). 用(ii)部分证明存在唯一的\(x>0\)使得\(x^{2}=c\). 记之为\(\sqrt{c}\).
\end{enumerate}
\end{example}
\begin{proof}
  
\end{proof}




\section{自然数与有理数}

\begin{definition}[归纳集]\label{definition:归纳集}
实数集\(E\)称为是\textbf{归纳的},若它包含\(1\),且若实数\(x\)属于\(E\),则数\(x + 1\)也属于\(E\)。
\end{definition}
\begin{note}
  显然全体实数集\(\mathbb{R}\)是归纳的。从不等式\(1>0\)我们容易推出集合\(\{x\in\mathbb{R}|x\geqslant0\}\)和\(\{x\in\mathbb{R}|x\geqslant1\}\)是归纳的。
\end{note}

\begin{definition}[自然数集]\label{definition:自然数集}
  \textbf{自然数集},记为\(\mathbb{N}\),定义为\(\mathbb{R}\)的所有归纳子集的交,即包含数1的最小归纳集.
\end{definition}
\begin{remark}
  集合论中的自然数集一般是从0开始,但这里自然数集是从0开始的,也就是说$0\notin \mathbb{N}$.
\end{remark}

\begin{proposition}[自然数集是归纳的]\label{proposition:自然数集是归纳的}
\(\mathbb{N}\)是归纳的.
\end{proposition}
\begin{proof}
  观察到数\(1\)属于\(\mathbb{N}\),这是由于\(1\)属于每个归纳集。此外,若数\(k\)属于\(\mathbb{N}\),则\(k\)属于每个归纳集。因此,由归纳集的定义可知,\(k + 1\)属于每个归纳集,所以\(k + 1\)属于\(\mathbb{N}\).
\end{proof}

\begin{theorem}[数学归纳法原理]\label{theorem:数学归纳法原理}
  对每个自然数\(n\),令\(S(n)\)为某个数学断言。假定\(S(1)\)成立。也假定每当\(k\)是使得\(S(k)\)成立的自然数,则\(S(k + 1)\)也成立。那么,对每个自然数\(n\),\(S(n)\)成立。
\end{theorem}
\begin{proof}
  定义\(A = \{k\in\mathbb{N}|S(k)\)成立\(\}\)。假设恰好意味着\(A\)是一个归纳集。于是\(\mathbb{N}\subseteq A\)。因此对每个自然数\(n\),\(S(n)\)成立。
\end{proof}

\begin{theorem}\label{theorem:每个非空自然数集有一个最小成员.}
  每个非空自然数集有一个最小成员.
\end{theorem}
\begin{proof}
  令\(E\)为自然数的非空集。由于集合\(\{x\in\mathbb{R}|x\geqslant1\}\)是归纳的,自然数有下界\(1\)。因此\(E\)有下界\(1\)。作为完备性公理的一个推论,\(E\)有下确界,定义\(c=\inf E\)。由于\(c + 1\)不是\(E\)的下界,存在\(m\in E\)使得\(m < c + 1\)。我们宣称\(m\)是\(E\)的最小成员。否则,存在\(n\in E\)使得\(n < m\)。由于\(n\in E\),\(c\leqslant n\)。于是\(c\leqslant n < m < c + 1\),且因此\(m - n < 1\)。因此自然数\(m\)属于区间\((n, n + 1)\)。\hyperref[example-1111]{例题\ref{example-1111}}表明对每个自然数\(n\),\((n, n + 1)\cap\mathbb{N}=\varnothing\).这个矛盾证明了\(m\)是\(E\)的最小成员.
\end{proof}

\begin{theorem}[实数的Archimedeas性质]\label{theorem:实数的Archimedeas性质}
  对于每对正实数\(a\)和\(b\),存在自然数\(n\)使得\(na > b\)。
\end{theorem}
\begin{note}
  我们经常重述\(\mathbb{R}\)的Archimedeas性质:对每个正实数\(\varepsilon\),存在自然数\(n\)使得\(1/n<\varepsilon\)。$^{\ominus}$.
\end{note}
\begin{proof}
  定义\(c = b/a>0\)。我们用反证法证明。若定理是错的,则\(c\)是自然数的一个上界。根据完备性公理,自然数有一个上确界,定义\(c_0=\sup\mathbb{N}\)。则\(c_0 - 1\)不是自然数的上界。选取自然数\(n\)使得\(n>c_0 - 1\)。因此\(n + 1>c_0\)。但自然数集是归纳的,因此\(n + 1\)是自然数。由于\(n + 1>c_0\),而\(c_0\)不是自然数集的上界。这个矛盾完成了证明。
\end{proof}

\begin{definition}[整数集、有理数集和无理数]\label{definition:整数集、有理数集和无理数}
\begin{enumerate}
  \item 定义\textbf{整数集}(记为\(\mathbb{Z}\))为由自然数、它们的相反数和数\(0\)组成的数集。
  \item \textbf{有理数集},记为\(\mathbb{Q}\),定义为整数的商的集合,即形如\(x = m/n\)的数\(x\),其中\(m\)和\(n\)是整数且\(n\neq0\).
  \item 若一个实数不是有理的就称它为无理数.
\end{enumerate}
\end{definition}

\begin{example}
  正如我们在\hyperref[example:1.12311-iii]{例题\ref{example:1.12311}\ref{example:1.12311-iii}}证明的,存在唯一的正数\(x\)使得\(x^2 = 2\),记之为\(\sqrt{2}\)。证明:\(\sqrt{2}\)这个数不是有理的.
\end{example}
\begin{proof}
  事实上,假定\(p\)和\(q\)是自然数使得\((p/q)^2 = 2\),则\(p^2 = 2q^2\)。素数分解定理$^{\ominus}$告诉我们\(2\)除\(p^2\)的次数正好是它除\(p\)的次数的两倍。因此\(2\)除\(p^2\)偶数次。类似地,\(2\)除\(2q^2\)奇数次。于是\(p^2\neq2q^2\),且因此\(\sqrt{2}\)是无理的。
\end{proof}

\begin{definition}[稠密]\label{definition:稠密}
  实数的集合\(E\)称为在\(\mathbb{R}\)中\textbf{稠密},若任何两个实数之间有\(E\)的成员.
\end{definition}

\begin{theorem}[有理数的稠密性]\label{theorem:有理数的稠密性}
  有理数在\(\mathbb{R}\)中稠密.
\end{theorem}
\begin{proof}
  令\(a\)和\(b\)为实数,满足\(a < b\)。首先假定\(a>0\)。根据 \hyperref[theorem:实数的Archimedeas性质]{\(\mathbb{R}\)的Archimedeas性质}可知,存在自然数\(q\)使得\((1/q)<b - a\)。再一次利用 \hyperref[theorem:实数的Archimedeas性质]{\(\mathbb{R}\)的Archimedeas性质}可知,自然数集\(S=\{n\in\mathbb{N}|n/q\geqslant b\}\)非空。根据\hyperref[theorem:每个非空自然数集有一个最小成员.]{定理\ref{theorem:每个非空自然数集有一个最小成员.}}可知,\(S\)具有最小成员\(p\)。观察到\(1/q<b - a<b\),于是\(p>1\)。因此\(p - 1\)是自然数(见\hyperref[example-9]{例题\ref{example-9}}),因而根据\(p\)的选取的最小性,\((p - 1)/q<b\)。我们也有
\[
a = b-(b - a)<(p/q)-(1/q)=(p - 1)/q
\]
因此有理数\(r=(p - 1)/q\)落在\(a\)与\(b\)之间。若\(a<0\),根据\hyperref[theorem:实数的Archimedeas性质]{\(\mathbb{R}\)的Archimedeas性质}可知,存在自然数\(n\)使得\(n>-a\)。我们从考虑过的第一种情形推出:存在有理数\(r\)落在\(n + a\)与\(n + b\)之间。因此有理数\(r - n\)落在\(a\)与\(b\)之间。
\end{proof}

\begin{example}\label{example-1111}
  用归纳法证明:对每个自然数\(n\),区间\((n, n + 1)\)不含任何自然数.
\end{example}
\begin{proof}
  
\end{proof}

\begin{example}\label{example-9}
用归纳法证明:若\(n > 1\)是自然数,则\(n - 1\)也是一个自然数。接着用归纳法证明:若\(m\)和\(n\)是满足\(n>m\)的自然数,则\(n - m\)是自然数。
\end{example}
\begin{proof}
  
\end{proof}





\section{可数集与不可数集}

\begin{axiom}[良序原理]\label{theorem:良序原理}
  自然数集的每个非空子集都有一个最小元素,即自然数在其标准的大小关系$<$下构成一良序集.
\end{axiom}
\begin{note}
  \hyperref[theorem:良序原理]{良序原理}等价于\hyperref[axiom:Zermelo选择公理]{选择公理}.
\end{note}

\begin{proposition}\label{proposition:集合之间的对等是一个等价关系}
  对等在集合间定义了一个等价关系,即它是自反的、对称的与传递的.
\end{proposition}
\begin{proof}
  
\end{proof}

\begin{definition}[自然数]\label{definition:自然数}
  定义自然数\(\{k\in\mathbb{N}|1\leqslant k\leqslant n\}\)为\(\{1, \cdots, n\}\).
\end{definition}

\begin{theorem}[鸽笼原理]\label{theorem:鸽笼原理}
  对任何自然数\(n\)和\(m\),集合\(\{1, \cdots, n + m\}\)与集合\(\{1, \cdots, n\}\)不对等.
\end{theorem}
\begin{proof}
  归纳可证.
\end{proof}

\begin{definition}[可数集与不可数集]\label{definition:可数集与不可数集}
\begin{enumerate}
  \item 集合\(E\)称为是\textbf{有限的}或\textbf{有限集}.若它或者是空集,或者存在自然数\(n\)使得\(E\)与\(\{1, \cdots, n\}\)对等.
  \item 我们说\(E\)是\textbf{可数无穷的},若\(E\)与自然数集\(\mathbb{N}\)对等.
  \item 有限或可数无穷的集合称为\textbf{可数集}.不是可数的集合称为\textbf{不可数集}.
\end{enumerate}
\end{definition}

\begin{proposition}\label{proposition:与可数集对等的集合是可数集}
  若一个集与可数集对等,则它是可数的.
\end{proposition}
\begin{proof}
  
\end{proof}

\begin{theorem}\label{theorem:可数集的子集是可数的}
  可数集的子集是可数的。特别是,每个自然数集是可数的.
\end{theorem}
\begin{proof}
  令\(B\)为可数集而\(A\)是\(B\)的一个非空子集。首先考虑\(B\)是有限的情形。令\(f\)为\(\{1, \cdots, n\}\)与\(B\)之间的一一对应。定义\(g(1)\)为第一个使得\(f(j)\)属于\(A\)的自然数\(j\),\(1\leqslant j\leqslant n\)。由于\(f\circ g\)是\(\{1\}\)与\(A\)之间的一一对应,若\(A = \{f(g(1))\}\),证明完成。否则,定义\(g(2)\)为使得\(f(j)\)属于\(A-\{f(g(1))\}\)的第一个自然数\(j\),\(1\leqslant j\leqslant n\)。\hyperref[theorem:鸽笼原理]{鸽笼原理}告诉我们至多\(N\)步后该归纳选择过程终止,其中\(N\leqslant n\)。因此\(f\circ g\)是\(\{1, \cdots, N\}\)与\(A\)之间的一一对应。于是\(A\)有限。

现在考虑\(B\)是可数无穷的情形。令\(f\)为\(\mathbb{N}\)与\(B\)之间的一一对应。定义\(g(1)\)为第一个使得\(f(j)\)属于\(A\)的自然数\(j\)。如同第一种情形的证明,我们看到若该选择过程终止,则\(A\)是有限的。否则,该选择过程不终止而\(g\)在所有的\(\mathbb{N}\)上恰当定义。显然\(f\circ g\)是一一映射,其中定义域是\(\mathbb{N}\)而象包含于\(A\)中。归纳论证表明对所有\(j\),\(g(j)\geqslant j\)。对每个\(x\in A\),存在某个\(k\)使得\(x = f(k)\)。因此\(x\)属于集合\(\{f(g(1)), \cdots, f(g(k))\}\)。因此\(f\circ g\)的象是\(A\)。因此\(A\)是可数无穷。
\end{proof}

\begin{corollary}\label{corollary:自然数集的有限积是可数的、有理数集也是可数的}
\begin{enumerate}[(i)]
  \item 对每个自然数\(n\),笛卡尔积\(\underbrace{\mathbb{N}\times\cdots\times\mathbb{N}}_{n\text{次}}\)是可数无穷的.即自然数集与其自身的有限次笛卡尔积是可数无穷的.

  \item\label{corollary:有理数集也是可数的} 有理数集\(\mathbb{Q}\)是可数无穷的.
\end{enumerate}
\end{corollary}
\begin{proof}
\begin{enumerate}[(i)]
  \item 我们对\(n = 2\)证明(i),而一般情形留作归纳法的练习。定义从\(\mathbb{N}\times\mathbb{N}\)到\(\mathbb{N}\)的映射\(g\)为\(g(m, n)=(m + n)^2 + n\)。映射\(g\)是一对一的。事实上,若\(g(m, n)=g(m', n')\),则\((m + n)^2-(m' + n')^2=n' - n\),因此
\[|m + n + m' + n'|\cdot|m + n - m' - n'|=|n' - n|\]
若\(n\neq n'\),则自然数\(m + n + m' + n'\)大于自然数\(\vert n' - n\vert\),这是不可能的。于是\(n = n'\),因而\(m = m'\)。因此\(\mathbb{N}\times\mathbb{N}\)与可数集\(\mathbb{N}\)的子集\(g(\mathbb{N}\times\mathbb{N})\)对等。我们从\hyperref[theorem:可数集的子集是可数的]{定理\ref{theorem:可数集的子集是可数的}}推出\(\mathbb{N}\times\mathbb{N}\)是可数的。

  \item 为证明\(\mathbb{Q}\)的可数性,我们首先从素数分解定理推出每个正有理数\(x\)可唯一写成\(x = p/q\),其中\(p\)和\(q\)是互素的自然数。对\(x = p/q>0\)定义从\(\mathbb{Q}\)到\(\mathbb{N}\)的映射\(g\)为\(g(x)=2((p + q)^2 + q)\),其中\(p\)和\(q\)是互素的自然数,\(g(0)=1\),而对\(x<0\),\(g(x)=g(-x)+1\)。我们将证明\(g\)是一对一的留作练习。于是\(\mathbb{Q}\)与\(\mathbb{N}\)的一个子集对等,因此根据\hyperref[theorem:可数集的子集是可数的]{定理\ref{theorem:可数集的子集是可数的}},是可数的。我们将用\hyperref[theorem:鸽笼原理]{鸽笼原理}证明\(\mathbb{N}\times\mathbb{N}\)和\(\mathbb{Q}\)都不是有限的留作练习。
\end{enumerate}
\end{proof}

\begin{definition}[可数无穷集的列举]\label{definition:可数无穷集的列举}
  对于可数无穷集\(X\),我们说\(\{x_n|n\in\mathbb{N}\}\)是\(X\)的一个\textbf{列举},若
\[X = \{x_n|n\in\mathbb{N}\}, \ x_n\neq x_m(\text{若 }n\neq m).\]
\end{definition}

\begin{theorem}\label{theorem:可数集关于映射的充要条件}
  非空集是可数的当且仅当它是某个定义域为非空可数集的函数的象。
\end{theorem}
\begin{proof}
  令\(A\)为非空可数集,而\(f\)为将\(A\)映上\(B\)的映射。假定\(A\)是可数无穷的,而将有限的情形留作练习。通过\(A\)与\(\mathbb{N}\)之间的一一对应的复合,我们可以假定\(A = \mathbb{N}\)。定义\(A\)中的两点\(x\),\(x'\)为等价的,若\(f(x)=f(x')\)。这是一个等价关系,即它是自反的、对称的与传递的。令\(E\)为\(A\)的子集,它由每个等价类的一个成员组成。则\(f\)在\(E\)的限制是\(E\)与\(B\)之间的一一对应。但\(E\)是\(\mathbb{N}\)的子集,因此,根据\hyperref[theorem:可数集的子集是可数的]{定理\ref{theorem:可数集的子集是可数的}},是可数的。集合\(B\)与\(E\)对等,因此\(B\)是可数的。逆断言是显然的,若\(B\)是非空可数集,则它或者与自然数的一个初始部分对等,或者与自然数全体对等。
\end{proof}

\begin{corollary}\label{corollary:可数集的可数并是可数的}
  可数集的可数族的并是可数的。
\end{corollary}
\begin{proof}
  令\(\Lambda\)为可数集且对每个\(\lambda\in\Lambda\),令\(E_{\lambda}\)为可数集。我们将证明并\(E=\bigcup_{\lambda\in\Lambda}E_{\lambda}\)是可数的。若\(E\)是空集,则它是可数的。因此我们假设\(E\neq\varnothing\)。我们考虑\(\Lambda\)是可数无穷的情形,而将有限的情形留作练习。令\(\{\lambda_n|n\in\mathbb{N}\}\)为\(\Lambda\)的一个列举。固定\(n\in\mathbb{N}\)。若\(E_{\lambda_n}\)是有限且非空的,选取自然数\(N(n)\)与将\(\{1, \cdots, N(n)\}\)映上\(E_{\lambda_n}\)的一一映射\(f_n\);若\(E_{\lambda_n}\)是可数无穷的,选取\(\mathbb{N}\)映上\(E_{\lambda_n}\)的一一映射\(f_n\)。定义
\[E' = \{(n,k)\in\mathbb{N}\times\mathbb{N}|E_{\lambda_n}\text{ 是非空的,且若 }E_{\lambda_n}\text{ 也是有限的,}1\leqslant k\leqslant N(n)\}\]
定义\(E'\)到\(E\)的映射\(f\)为\(f(n, k)=f_n(k)\)。则\(f\)是\(E'\)映上\(E\)的映射。然而,\(E'\)是可数集\(\mathbb{N}\times\mathbb{N}\)的子集,因此,根据\hyperref[theorem:可数集的子集是可数的]{定理\ref{theorem:可数集的子集是可数的}},是可数的。定理5告诉我们\(E\)也是可数的。
\end{proof}

\begin{definition}[退化的区间]\label{definition:退化的区间}
  我们称实数的区间为退化的,若它是空的或包含一个单独的成员.
\end{definition}

\begin{theorem}\label{theorem:非退化的实数区间是不可数的}
  一个非退化实数区间是不可数的。
\end{theorem}
\begin{proof}
  令\(I\)为实数的非退化区间。显然\(I\)不是有限的。我们用反证法证明\(I\)是不可数的。假定\(I\)是可数无穷的。令\(\{x_n|n\in\mathbb{N}\}\)为\(I\)的一个列举。令\([a_1, b_1]\)为\(I\)的不包含\(x_1\)的非退化的闭有界子区间。接着令\([a_2, b_2]\)为\([a_1, b_1]\)的非退化的闭有界子区间,它不包含\(x_2\)。我们归纳地选取非退化闭有界区间的可数族\(\{[a_n, b_n]\}_{n = 1}^{\infty}\),对每个\(n\),\([a_{n + 1}, b_{n + 1}]\subseteq[a_n, b_n]\),并使得对每个\(n\),\(x_n\notin[a_n, b_n]\)。非空集\(E = \{a_n|n\in\mathbb{N}\}\)有上界\(b_1\)。\hyperref[axiom:完备性公理]{完备性公理}告诉我们\(E\)有上确界。定义\(x^*=\sup E\)。由于\(x^*\)是\(E\)的一个上界,对所有\(n\),\(a_n\leqslant x^*\)。另一方面,由于\(\{[a_n, b_n]\}_{n = 1}^{\infty}\)是下降的,对每个\(n\),\(b_n\)是\(E\)的上界。于是,对每个\(n\),\(x^*\leqslant b_n\)。因此对每个\(n\),\(x^*\)属于\([a_n, b_n]\)。但\(x^*\)属于\([a_1, b_1]\subseteq I\),因此存在自然数\(n_0\)使得\(x^* = x_{n_0}\)。由于\(x^* = x_{n_0}\)不属于\([a_{n_0}, b_{n_0}]\),我们得到矛盾。因此,\(I\)是不可数的。
\end{proof}









\section{实数的开集、闭集和Borel集}

\begin{definition}[实数的开集]\label{definition:实数的开集}
一个实数的集合\(\mathcal{O}\)称为\textbf{开的},若对每个\(x\in\mathcal{O}\),存在\(r > 0\)使得区间\((x - r, x + r)\)包含于\(\mathcal{O}\).
\end{definition}


\begin{theorem}\label{theorem: 实数的开集就是开区间}
  实数的开集就是开区间.
\end{theorem}

\begin{proposition}[实数的开区间]\label{proposition:实数的开区间}
\begin{enumerate}[(1)]
  \item 对于\(a < b\),区间\((a, b)\)是一个开集,且每个开有界区间(有界开集)都是这种形式。

  \item 对于\(a\),\(b\in\mathbb{R}\),区间
\((a,\infty),(-\infty,b), (-\infty,\infty)\)都是开集,且每个开无界区间(无界开集)都是这三中形式之一.
\end{enumerate}
\end{proposition}
\begin{proof}
\begin{enumerate}[(1)]
  \item 事实上,令\(x\)属于\((a, b)\)。定义\(r=\min\{b - x, x - a\}\)。观察到\((x - r, x + r)\)包含于\((a, b)\)。因此\((a, b)\)是开有界区间.又因为实数的有界开集等价于有界开区间,而实数的有界开区间都是这种形式,所以每个开有界区间(有界开集)都是这种形式。

  \item 观察到每个这样的集合是一个开区间。此外,不难看出,由于每个实数集在扩充实数集中有下确界与上确界,因此每个开无界区间(无界开集)都是这三中形式之一.
\end{enumerate}
\end{proof}

\begin{proposition}[实数集的开集的性质]\label{proposition:实数集的开集的性质}
实数集\(\mathbb{R}\)和空集\(\varnothing\)是开的,任何开集的有限族的交是开的,任何开集族的并是开的。
\end{proposition}
\begin{remark}
  然而,任何开集族的交是开的不成立。例如,对每个自然数\(n\),令\(\mathcal{O}_n\)为开区间\((-1/n, 1/n)\)。则根据\hyperref[theorem:实数的Archimedeas性质]{实数的Archimedeas性质}可知,\(\bigcap_{n = 1}^{\infty}\mathcal{O}_n=\{0\}\),而\(\{0\}\)不是一个开集.
\end{remark}
\begin{proof}显然\(\mathbb{R}\)和\(\varnothing\)是开的,而任何开集族的并是开的。令\(\{\mathcal{O}_k\}_{k = 1}^{n}\)为\(\mathbb{R}\)的开子集的有限族。若该族的交是空的,则交是空集,因此是开的。否则,令\(x\)属于\(\bigcap_{k = 1}^{n}\mathcal{O}_k\)。对于\(1\leqslant k\leqslant n\),选取\(r_k>0\)使得\((x - r_k, x + r_k)\subseteq\mathcal{O}_k\)。定义\(r = \min\{r_1, \cdots, r_n\}\)。则\(r>0\)且\((x - r, x + r)\subseteq\bigcap_{k = 1}^{n}\mathcal{O}_k\)。因此\(\bigcap_{k = 1}^{n}\mathcal{O}_k\)是开的
\end{proof}

\begin{proposition}\label{proposition:开集可以写成不交开区间族的并}
每个非空开集是可数个不交开区间族的并。
\end{proposition}
\begin{proof}
  令\(\mathcal{O}\)为\(\mathbb{R}\)的非空开子集。令\(x\)属于\(\mathcal{O}\)。存在\(y > x\)使得\((x, y)\subseteq\mathcal{O}\),且存在\(z < x\)使得\((z, x)\subseteq\mathcal{O}\)。定义扩充的实数\(a_x\)和\(b_x\)为
\[a_x=\inf\{z|(z,x)\subseteq\mathcal{O}\}\text{与}b_x=\sup\{y|(x,y)\subseteq\mathcal{O}\}\]
则\(I_x=(a_x, b_x)\)是包含\(x\)的开区间。我们宣称
\begin{align}\label{proposition2.7-equation:(2)}
  I_x\subseteq\mathcal{O}\text{ 但 }a_x\notin\mathcal{O},\ b_x\notin\mathcal{O}.
\end{align}
事实上,令\(w\)属于\(I_x\),比如\(x < w < b_x\)。根据\(b_x\)的定义,存在数\(y > w\)使得\((x, y)\subseteq\mathcal{O}\),因而\(w\in\mathcal{O}\)。此外,\(b_x\notin\mathcal{O}\),因为若\(b_x\in\mathcal{O}\),则对某个\(r > 0\)我们有\((b_x - r, b_x + r)\subseteq\mathcal{O}\)。因此\((x, b_x + r)\subseteq\mathcal{O}\),与\(b_x\)的定义矛盾。类似地,\(a_x\notin\mathcal{O}\),考虑开区间族\(\{I_x\}_{x\in\mathcal{O}}\)。由于\(\mathcal{O}\)中的每个\(x\)是\(I_x\)的成员,而每个\(I_x\)包含于\(\mathcal{O}\),我们有\(\mathcal{O}=\bigcup_{x\in\mathcal{O}}I_x\)。我们从\eqref{proposition2.7-equation:(2)}推出\(\{I_x\}_{x\in\mathcal{O}}\)是不交的。因此\(\mathcal{O}\)是不交的开区间族的并。剩下来要证明该族是可数的。根据\hyperref[theorem:有理数的稠密性]{有理数的稠密性},这些开区间的每一个包含一个有理数。这建立了开区间族与有理数子集之间的一一对应。我们从\hyperref[theorem:可数集的子集是可数的]{定理\ref{theorem:可数集的子集是可数的}}和\hyperref[corollary:有理数集也是可数的]{推论\ref{corollary:自然数集的有限积是可数的、有理数集也是可数的}\ref{corollary:有理数集也是可数的}}推出任何有理数集是可数的。因此\(\mathcal{O}\)是可数个不交开区间族的并。
\end{proof}

\begin{definition}[实数的闭包]\label{definition:实数的闭包}
  对于实数集\(E\),\(x\)称为\(E\)的\textbf{闭包点},若每个包含\(x\)的开区间也包含\(E\)的点。\(E\)的全体闭包点称为\(E\)的\textbf{闭包}且记为\(\overline{E}\)。
\end{definition}

\begin{proposition}\label{proposition:集合一定包含于其闭包}
  对于实数集\(E\),我们总是有\(E\subseteq\overline{E}\).
\end{proposition}

\begin{definition}[实数的闭集]\label{definition:实数的闭集}
  若\(E\)包含它的所有闭包点,即\(E = \overline{E}\),则集合\(E\)称为\textbf{闭的}或\textbf{闭集}.
\end{definition}

\begin{proposition}\label{proposition:闭包是最小的闭集}
  对于实数集\(E\),它的闭包\(\overline{E}\)是闭的。此外,\(\overline{E}\)在以下意义下是包含\(E\)的最小闭集:若\(F\)是闭的且\(E\subseteq F\),则\(\overline{E}\subseteq F\).
\end{proposition}
\begin{proof}
  集合\(\overline{E}\)是闭的,若它包含所有闭包点。令\(x\)为\(\overline{E}\)的闭包点。考虑包含\(x\)的开区间\(I_x\)。存在一个点\(x'\in\overline{E}\cap I_x\)。由于\(x'\)是\(E\)的闭包点,且开区间\(I_x\)包含\(x'\),存在点\(x''\in E\cap I_x\)。因此每个包含\(x\)的开区间也包含\(E\)的点,且因此\(x\in\overline{E}\)。所以集合\(\overline{E}\)是闭的。显然,若\(A\subseteq B\),则\(\overline{A}\subseteq\overline{B}\),因此,若\(F\)是闭的且包含\(E\),则\(\overline{E}\subseteq\overline{F}=F\)。
\end{proof}

\begin{proposition}\label{proposition:开集的补集是闭集}
  实数集是开的当且仅当它在\(\mathbb{R}\)中的补是闭的。
\end{proposition}
\begin{proof}
  首先假定\(E\)是\(\mathbb{R}\)的一个开子集。令\(x\)为\(\mathbb{R}- E\)的闭包点。则\(x\)不属于\(E\),因为否则就会有一个包含\(x\)且包含于\(E\)的开区间,因而与\(\mathbb{R}- E\)不交。于是\(x\)属于\(\mathbb{R}- E\)且因此\(\mathbb{R}- E\)是闭的。现在假定\(\mathbb{R}- E\)是闭的。令\(x\)属于\(E\)。则必有包含\(x\)且包含于\(E\)的开区间,否则每个包含\(x\)的开区间包含\(\mathbb{R}- E\)的点,且因此\(x\)是\(\mathbb{R}- E\)的闭包点。由于\(\mathbb{R}- E\)是闭的,\(x\)也属于\(\mathbb{R}- E\)。这是一个矛盾。
\end{proof}

\begin{proposition}[实数集的闭集的性质]\label{proposition:实数集的闭集的性质}
  空集\(\varnothing\)和\(\mathbb{R}\)是闭的,任何闭集的有限族的并是闭的,任何闭集族的交是闭的.
\end{proposition}
\begin{remark}
  空集\(\varnothing\)和\(\mathbb{R}\)是既开又闭的.
\end{remark}
\begin{proof}
  由于\(\mathbb{R}-[\mathbb{R}-E]=E\),从\hyperref[proposition:闭包是最小的闭集]{命题\ref{proposition:闭包是最小的闭集}}得出一个集合是闭的当且仅当它的补是开的。因此,根据\hyperref[theorem:De Morgan等式]{De Morgan等式}和\hyperref[proposition:实数集的开集的性质]{命题\ref{proposition:实数集的开集的性质}}立得.
\end{proof}

\begin{definition}[覆盖]\label{definition:覆盖}
\begin{enumerate}
  \item 集族\(\{E_{\lambda}\}_{\lambda\in\Lambda}\)称为是集合\(E\)的\textbf{覆盖},若\(E\subseteq\bigcup_{\lambda\in\Lambda}E_{\lambda}\)。
  
  \item 若\(E\)的覆盖的子族自身也是\(E\)的一个覆盖,我们称为\(E\)的覆盖的\textbf{子覆盖}。
  
  \item 若覆盖中的每个集合\(E_{\lambda}\)是开的,我们称\(\{E_{\lambda}\}_{\lambda\in\Lambda}\)为\(E\)的一个\textbf{开覆盖}。
  
  \item 若覆盖\(\{E_{\lambda}\}_{\lambda\in\Lambda}\)仅包含有限个集合,我们称它为\textbf{有限覆盖}.
\end{enumerate} 
\end{definition}
\begin{note}
  该术语是不一致的:“开覆盖”中的“开”指的是该覆盖的集合;“有限覆盖”中的“有限”指的是族而不是隐含该族中的集合是有限集。因此,术语“开覆盖”是语言的误用,而恰当的说法应该是“用开集覆盖”。遗憾的是,前一个术语已在数学中广泛使用.
\end{note}

\begin{theorem}[Heine - Borel定理]\label{theorem:Heine - Borel定理}
  令\(F\)为闭有界实数集。则\(F\)的每个开覆盖有一个有限子覆盖。
\end{theorem}
\begin{proof}
  我们首先考虑\(F\)是闭有界区间\([a, b]\)的情形。令\(\mathcal{F}\)为\([a, b]\)的开覆盖。定义\(E\)为具有如下性质的区间\([a, x]\),即可被\(\mathcal{F}\)的有限个集合覆盖的数\(x\in[a, b]\)的集合。由于\(a\in E\),\(E\)是非空的。由于\(E\)有上界\(b\),根据\(\mathbb{R}\)的完备性,\(E\)有上确界。定义\(c = \sup E\)。由于\(c\)属于\([a, b]\),存在\(\mathcal{O}\in\mathcal{F}\)包含\(c\)。由于\(\mathcal{O}\)是开的,存在\(\varepsilon>0\),使得区间\((c - \varepsilon, c+\varepsilon)\)包含于\(\mathcal{O}\)。现在\(c - \varepsilon\)不是\(E\)的上界,因而必有\(x\in E\)满足\(x>c - \varepsilon\)。由于\(x\in E\),存在覆盖\([a, x]\)的\(\mathcal{F}\)中的集合的有限族\(\{\mathcal{O}_1, \cdots, \mathcal{O}_k\}\)。因此,有限族\(\{\mathcal{O}_1, \cdots, \mathcal{O}_k, \mathcal{O}\}\)覆盖区间\([a, c+\varepsilon)\)。于是\(c = b\),否则\(c < b\)且\(c\)不是\(E\)的上界。因此\([a, b]\)可被\(\mathcal{F}\)中的有限个集合覆盖,这证明了我们考虑的特殊情形。

现在令\(F\)为任何闭有界集,而\(\mathcal{F}\)是\(F\)的一个开覆盖。由于\(F\)是有界的,它包含于某个有界闭区间\([a, b]\)。\hyperref[proposition:开集的补集是闭集]{命题\ref{proposition:开集的补集是闭集}}告诉我们集合\(\mathcal{O}=\mathbb{R}- F\)是开的,因为\(F\)是闭的。令\(\mathcal{F}^*\)为添加\(\mathcal{O}\)到\(\mathcal{F}\)后得到的开集族,即\(\mathcal{F}^*=\mathcal{F}\cup\mathcal{O}\)。由于\(\mathcal{F}\)覆盖\(F\),\(\mathcal{F}^*\)覆盖\([a, b]\)。根据我们刚考虑的情形,存在\(\mathcal{F}^*\)的有限子族覆盖\([a, b]\),因此也覆盖\(F\)。通过从\(F\)的这个有限子覆盖去掉\(\mathcal{O}\),若\(\mathcal{O}\)属于该有限子覆盖,我们得到\(\mathcal{F}\)中覆盖\(F\)的有限族。
\end{proof}

\begin{definition}[集族的下降与上升]\label{definition:集族的下降与上升}
\begin{enumerate}
  \item 我们说集合的可数族\(\{E_n\}_{n = 1}^{\infty}\)是\textbf{下降的},若对每个自然数\(n\),\(E_{n + 1}\subseteq E_n\)。

  \item 我们说集合的可数族\(\{E_n\}_{n = 1}^{\infty}\)是\textbf{上升的},若对每个自然数\(n\),\(E_n\subseteq E_{n + 1}\)。
\end{enumerate}
\end{definition}

\begin{theorem}[集套定理]\label{theorem:集套定理}
  令\(\{F_n\}_{n = 1}^{\infty}\)为下降的非空闭实数集的可数族,其中\(F_1\)有界。则
\[\bigcap_{n = 1}^{\infty}F_n\neq\varnothing.\]
\end{theorem}
\begin{proof}
  我们用反证法。假定交集是空的。则对每个实数\(x\),存在自然数\(n\)使得\(x\notin F_n\),即\(x\in\mathcal{O}_n=\mathbb{R}- F_n\)。因此\(\bigcup_{n = 1}^{\infty}\mathcal{O}_n=\mathbb{R}\)。根据\hyperref[proposition:开集的补集是闭集]{命题\ref{proposition:开集的补集是闭集}},由于每个\(F_n\)是闭的,每个\(\mathcal{O}_n\)是开的。因此\(\{\mathcal{O}_n\}_{n = 1}^{\infty}\)是\(\mathbb{R}\)的一个开覆盖,从而也是\(F_1\)的开覆盖。\hyperref[theorem:Heine - Borel定理]{Heine - Borel定理}告诉我们存在自然数\(N\)使得\(F_1\subseteq\bigcup_{n = 1}^{N}\mathcal{O}_n\)。由于\(\{F_n\}_{n = 1}^{\infty}\)是下降的,补集族\(\{\mathcal{O}_n\}_{n = 1}^{\infty}\)是上升的。因此\(\bigcup_{n = 1}^{N}\mathcal{O}_n=\mathcal{O}_N=\mathbb{R}- F_N\)。因此\(F_1\subseteq\mathbb{R}- F_N\),这与\(F_N\)是\(F_1\)的非空子集的假设矛盾。
\end{proof}

\begin{definition}[$\sigma$代数]\label{definition: sigma 代数}
  给定集合\(X\),\(X\)的子集族\(\mathcal{A}\)称为(\(X\)的子集的)\textbf{\(\sigma\)代数},若:
  \begin{enumerate}[(i)]
    \item 空集\(\varnothing\)属于\(\mathcal{A}\);

    \item \(\mathcal{A}\)中的集合在\(X\)中的补也属于\(\mathcal{A}\);

    \item \(\mathcal{A}\)中集合的可数族的并也属于\(\mathcal{A}\)。
  \end{enumerate}
\end{definition}
\begin{note}
\begin{enumerate}[(1)]
  \item 给定集合\(X\),族\(\{\varnothing, X\}\)是一个\(\sigma\)代数,它有两个成员且它包含于每个\(X\)的子集的\(\sigma\)代数。

  \item 另一个极端情形是\(X\)的所有子集组成的集族且包含每个\(X\)的子集的\(\sigma\)代数\(2^X\)。
\end{enumerate}
\end{note}

\begin{proposition}[$\sigma$ 代数的基本性质]\label{proposition:sigma 代数的基本性质}
 对任何\(\sigma\)代数\(\mathcal{A}\),
\begin{enumerate}[(1)]
  \item \(\mathcal{A}\)关于属于\(\mathcal{A}\)的集合的可数族的交封闭。
  
  \item\(\mathcal{A}\)关于属于\(\mathcal{A}\)的集合的有限并与有限交封闭。
  
  \item\(\mathcal{A}\)关于属于\(\mathcal{A}\)的集合的相对补封闭,即若\(A_1\)和\(A_2\)属于\(\mathcal{A}\),则\(A_1- A_2\)也属于\(\mathcal{A}\)。
\end{enumerate}
\end{proposition}
\begin{proof}
\begin{enumerate}[(1)]
  \item 从De Morgan等式容易推出.

  \item 由空集属于\(\mathcal{A}\)易得.

  \item 由\(A_1- A_2=A_1\cap[X- A_2]\)易得.
\end{enumerate}
\end{proof}

\begin{proposition}\label{proposition:一个集合的最小的子集的sigma代数}
  令\(\mathcal{F}\)为集合\(X\)的子集族。则所有包含\(\mathcal{F}\)的\(X\)的子集的\(\sigma\)代数的交\(\mathcal{A}\)是一个包含\(\mathcal{F}\)的\(\sigma\)代数。此外,在任何包含\(\mathcal{F}\)的\(\sigma\)代数也包含\(\mathcal{A}\)的意义下,\(\mathcal{A}\)是包含\(\mathcal{F}\)的最小的\(X\)的子集的\(\sigma\)代数。
\end{proposition}
\begin{proof}
  这个命题的证明可直接从\hyperref[definition: sigma 代数]{\(\sigma\)代数的定义}得到。
\end{proof}

\begin{proposition}\label{proposition:集合的上下极限也包含于sigma代数}
  令\(\{A_n\}_{n = 1}^{\infty}\)为属于\(\sigma\)代数\(\mathcal{A}\)的集合的可数族。以下两个集合属于\(\mathcal{A}\):
\[\limsup\{A_n\}_{n = 1}^{\infty}=\bigcap_{k = 1}^{\infty}\left[\bigcup_{n = k}^{\infty}A_n\right] \text{与} \liminf\{A_n\}_{n = 1}^{\infty}=\bigcup_{k = 1}^{\infty}\left[\bigcap_{n = k}^{\infty}A_n\right]\]

集合\(\limsup\{A_n\}_{n = 1}^{\infty}\)是对可数无穷多个指标\(n\)属于\(A_n\)的点的集合,而集合\(\liminf\{A_n\}_{n = 1}^{\infty}\)是除指定至多有限多个指标\(n\)外属于\(A_n\)的点的集合。
\end{proposition}
\begin{proof}
  由\(\mathcal{A}\)关于可数交与并封闭立得.
\end{proof}

\begin{definition}[实数的Borel集]\label{definition:实数的Borel集}
  实数的Borel集族\(\mathcal{B}\)是包含所有实数的开集的实数集的最小\(\sigma\)代数.
\end{definition}

\begin{definition}[\(G_{\delta}\)集与\(F_{\sigma}\)集]\label{definition:G_{delta}集与F_{sigma}集}
  开集的可数交称为\(G_{\delta}\)集。闭集的可数并称为\(F_{\sigma}\)集。
\end{definition}

\begin{proposition}[Borel集的基本性质]\label{proposition:Borel集的基本性质}
\begin{enumerate}[(1)]
  \item 每个开集和闭集都是Borel集.

  \item 每个单点集都是Borel集.

  \item 每个可数集都是Borel集.

  \item 每个\(G_{\delta}\)集和每个\(F_{\sigma}\)集是Borel集。

  \item 每个开的或者闭的实数集的可数族的\(\liminf\)和\(\limsup\)都是Borel集.
\end{enumerate}
\end{proposition}
\begin{proof}
\begin{enumerate}[(1)]
  \item 显然每个开集都是Borel集,由于\(\sigma\)代数关于补是封闭的,我们从\hyperref[proposition:开集的补集是闭集]{命题\ref{proposition:开集的补集是闭集}}推出每个闭集是Borel集.

  \item 由每个单点集是闭的结合(1)立得.

  \item 
  
  \item 由\(\sigma\)代数关于可数并与可数交封闭立得.

  \item 
\end{enumerate}
\end{proof}







\section{实数序列}

\begin{definition}[实数序列/实数列]\label{definition:实数序列/实数列}
  实数序列是一个实值函数,其定义域是自然数集。习惯上我们不用标准的函数记号如\(f:\mathbb{N}\to\mathbb{R}\)表示序列,而用下标\(a_n\)代替\(f(n)\),将一个序列记为\(\{a_n\}\)。自然数\(n\)称为该序列的指标,对应于指标\(n\)的数\(a_n\)称为序列的第\(n\)项。
  
  正如同我们说实值函数是有界的,若它的象是有界实数集;我们说序列是有界的,若存在某个\(c\geqslant0\)使得对所有\(n\),\(\vert a_n\vert\leqslant c\)。
  
  若对所有\(n\),\(a_n\leqslant a_{n + 1}\),序列\(\{a_n\}\)称为是递增的;若\(\{-a_n\}\)是\textbf{递增的},序列\(\{a_n\}\)称为是\textbf{递减的};若它是递增的或者递减的,序列\(\{a_n\}\)则称为是\textbf{单调的}。
\end{definition}

\begin{definition}
  我们说序列\(\{a_n\}\)收敛到数\(a\),若对每个\(\varepsilon>0\),存在指标\(N\),使得
当\(n\geqslant N\)时,有\[\vert a - a_n\vert<\varepsilon.\]
我们称\(a\)为序列的极限且用
\[\{a_n\}\to a\text{ 或}\lim_{n\to\infty}a_n = a.\]
表示\(\{a_n\}\)的收敛性。
\end{definition}

\begin{proposition}[收敛的实数列的性质]\label{proposition:收敛的实数列的性质}
  令实数序列\(\{a_n\}\)收敛到实数\(a\)。则极限是唯一的,该序列是有界的,且对实数\(c\),
若对所有\(n\), \(a_n\leqslant c\), 则\(a\leqslant c\).
\end{proposition}

\begin{theorem}[实数序列的单调收敛准则]\label{theorem:实数序列的单调收敛准则}
单调的实数序列收敛当且仅当它是有界的。
\end{theorem}
\begin{proof}
  令\(\{a_n\}\)为递增序列。若该序列收敛,则根据前一个命题,它是有界的。现在假设\(\{a_n\}\)是有界的,根据完备性公理,集合\(S = \{a_n|n\in N\}\)有上确界:定义\(a = \sup S\)。我们宣称\(\{a_n\}\to a\)。事实上,令\(\varepsilon>0\)。由于\(a\)是\(S\)的上界,对所有\(n\),\(a_n\leqslant a\)。由于\(a - \varepsilon\)不是\(S\)的上界,存在指标\(N\),使得\(a_N>a - \varepsilon\)。由于该序列是递增的,对所有\(n\geqslant N\),\(a_n>a - \varepsilon\)。因此,若\(n\geqslant N\),则\(\vert a - a_n\vert<\varepsilon\)。因此\(\{a_n\}\to a\)。序列递减情形的证明是相同的。
\end{proof}

\begin{definition}[子序列]\label{definition:子序列}
  对于序列\(\{a_n\}\)和严格递增的自然数序列\(\{n_k\}\),序列\(\{a_{n_k}\}\)的第\(k\)项是\(a_{n_k}\)并被称为\(\{a_n\}\)的一个\textbf{子序列}.
\end{definition}

\begin{theorem}[Bolzano - Weierstrass定理]\label{theorem:Bolzano - Weierstrass定理}
  每个有界实数序列有一个收敛的子序列。
\end{theorem}
\begin{proof}
  令\(\{a_n\}\)为有界实数序列。选取\(M\geqslant0\)使得对所有\(n\),\(\vert a_n\vert\leqslant M\)。令\(n\)为自然数。定义\(E_n=\overline{\{a_j|j\geqslant n\}}\)。则\(E_n\subseteq[-M,M]\)且\(E_n\)是闭的,因为它是集合的闭包。 因此,\(\{E_n\}\)是下降的\(\mathbb{R}\)的非空闭有界子集序列。集套定理告诉我们\(\bigcap_{n = 1}^{\infty}E_n\neq\varnothing\),选取\(a\in\bigcap_{n = 1}^{\infty}E_n\)。对于每个自然数\(k\),\(a\)是\(\{a_j|j\geqslant k\}\)的闭包点。 因此,对于无穷多个指标\(j\geqslant n\),\(a_j\)属于\((a - 1/k,a + 1/k)\)。根据归纳法,选取严格递增的自然数序列\(\{n_k\}\)使得对所有\(k\),\(\vert a - a_{n_k}\vert<1/k\)。我们从\(\mathbb{R}\)的Archimedeas性质推出子序列\(\{a_{n_k}\}\)收敛到\(a\)。
\end{proof}

\begin{definition}
  实数序列\(\{a_n\}\)称为是\textbf{Cauchy的}或\textbf{Cauchy列},若对每个\(\varepsilon>0\),存在指标\(N\)使得
  当\(n,m\geqslant N\)时,有\(\vert a_m - a_n\vert<\varepsilon\).
\end{definition}

\begin{theorem}[实数序列的Cauchy收敛准则]\label{theorem:实数序列的Cauchy收敛准则}
  实数序列收敛当且仅当它是Cauchy的。
\end{theorem}
\begin{proof}
  首先假定\(\{a_n\}\to a\)。观察到对所有自然数\(n\)和\(m\),
  \begin{align}\label{equation:(5)}
    \vert a_n - a_m\vert=\vert(a_n - a)+(a - a_m)\vert\leqslant\vert a_n - a\vert+\vert a_m - a\vert
  \end{align}
  令\(\varepsilon>0\)。由于\(\{a_n\}\to a\),我们可以选取一个自然数\(N\)使得若\(n\geqslant N\),则\(\vert a_n - a\vert<\varepsilon/2\)。我们从\eqref{equation:(5)}推出若\(n\),\(m\geqslant N\),则\(\vert a_m - a_n\vert<\varepsilon\)。因此序列\(\{a_n\}\)是Cauchy的。
  
  为证明反命题,令\(\{a_n\}\)为Cauchy序列。我们宣称它是有界的。事实上,对\(\varepsilon = 1\),选取\(N\)使得若\(n\),\(m\geqslant N\),则\(\vert a_m - a_n\vert<1\)。因此,对所有\(n\geqslant N\),
  \[\vert a_n\vert=\vert(a_n - a_N)+a_N\vert\leqslant\vert a_n - a_N\vert+\vert a_N\vert\leqslant1+\vert a_N\vert.\]
  定义\(M = 1+\max\{\vert a_1\vert, \cdots, \vert a_N\vert\}\)。则对所有\(n\),\(\vert a_n\vert\leqslant M\)。因此\(\{a_n\}\)是有界的。Bolzano - Weierstrass定理告诉我们存在收敛于\(a\)的子序列\(\{a_{n_k}\}\)。我们宣称整个序列收敛于\(a\)。事实上,令\(\varepsilon>0\)。由于\(\{a_n\}\)是Cauchy的,我们可以选取自然数\(N\),使得
  若\(n,m\geqslant N\), 则\(\vert a_n - a_m\vert<\varepsilon/2\)
  另外,由于\(\{a_{n_k}\}\to a\),我们可以选取自然数\(n_k\),使得对\(n_k\geqslant N\),\(\vert a - a_{n_k}\vert<\varepsilon/2\)。因此,对所有\(n\geqslant N\),
  \[\vert a_n - a\vert=\vert(a_n - a_{n_k})+(a_{n_k}-a)\vert\leqslant\vert a_n - a_{n_k}\vert+\vert a - a_{n_k}\vert<\varepsilon.\]
\end{proof}

\begin{theorem}[实序列收敛的线性与单调性]\label{theorem:实序列收敛的线性与单调性}
  令\(\{a_n\}\)和\(\{b_n\}\)为收敛的实数序列。则对每对实数\(\alpha\)和\(\beta\),序列\(\{\alpha\cdot a_n+\beta\cdot b_n\}\)收敛且
\begin{align}
  \lim_{n \to \infty}[\alpha \cdot a_n + \beta \cdot b_n]=\alpha \cdot \lim_{n \to \infty}a_n+\beta \cdot \lim_{n \to \infty}b_n \label{equation(6)}
\end{align}
此外,
若对所有\(n,a_n \leqslant b_n\), 则\(\lim_{n \to \infty}a_n \leqslant \lim_{n \to \infty}b_n \)
\end{theorem}
\begin{proof}
  设
\[\lim_{n \to \infty}a_n = a\text{ 与}\lim_{n \to \infty}b_n = b\]
观察到对所有\(n\),
\begin{align}
  |[\alpha \cdot a_n+\beta \cdot b_n]-[\alpha \cdot a+\beta \cdot b]|\leqslant|\alpha|\cdot|a_n - a|+|\beta|\cdot|b_n - b| \label{equation(8)}
\end{align}
令\(\varepsilon>0\)。选取自然数\(N\)使得对所有\(n\geqslant N\),
\[|a_n - a|<\varepsilon/[2 + 2|\alpha|]\text{ 且 }|b_n - b|<\varepsilon/[2 + 2|\beta|]\]
我们从\eqref{equation(8)}推出对所有\(n\geqslant N\),
\[|[\alpha \cdot a_n+\beta \cdot b_n]-[\alpha \cdot a+\beta \cdot b]|<\varepsilon\]
因此\eqref{equation(6)}成立。为了验证\(\lim_{n \to \infty}a_n \leqslant \lim_{n \to \infty}b_n \),对所有\(n\),设\(c_n = b_n - a_n\)与\(c = b - a\)。则对所有\(n\),\(c_n\geqslant0\),根据收敛的线性,\(\{c_n\}\to c\)。我们必须证明\(c\geqslant0\)。令\(\varepsilon>0\)。存在\(N\)使得对所有\(n\geqslant N\),
\[-\varepsilon < c - c_n < \varepsilon\]
特别地,\(0\leqslant c_N < c+\varepsilon\)。由于对每个正数\(\varepsilon\),\(c>-\varepsilon\),所以\(c\geqslant0\)。
\end{proof}

\begin{definition}[实数列扩充的收敛]\label{definition:实数列扩充的收敛}
  对每个实数\(c\),存在指标\(N\)使得\(n\geqslant N\)时有\(a_n\geqslant c\),我们就说序列\(\{a_n\}\)\textbf{收敛到无穷},称\(\infty\)为\(\{a_n\}\)的极限且记作\(\lim_{n \to \infty}\{a_n\}=\infty\)。收敛到\(-\infty\)可做出类似的定义。
\end{definition}
\begin{note}
  有了这个扩充的收敛的概念,我们可以断言任何单调实数序列\(\{a_n\}\)(有界或无界)且一定会收敛到某个扩充的实数,且因此\(\lim_{n \to \infty}a_n\)是良定义的。
\end{note}
  
\begin{definition}[实数序列的上下极限]\label{definition:实数序列的上下极限}
  令\(\{a_n\}\)为实数序列。\(\{a_n\}\)的\textbf{上极限},记为\(\limsup\{a_n\}\),定义为
\[\limsup\{a_n\}=\lim_{n \to \infty}[\sup\{a_k|k \geqslant n\}]\]
\(\{a_n\}\)的\textbf{下极限},记为\(\liminf\{a_n\}\),定义为
\[\liminf\{a_n\}=\lim_{n \to \infty}[\inf\{a_k|k \geqslant n\}]\]
\end{definition}
  
\begin{proposition}[实数序列的上下极限的等价命题]\label{definition:实数序列的上下极限的等价命题}
  令\(\{a_n\}\)和\(\{b_n\}\)为实数序列。
\begin{enumerate}[(i)]
  \item  \(\liminf\{a_n\}=\ell\in\mathbb{R}\)当且仅当对每个\(\varepsilon>0\),存在无穷多个指标\(n\)使得\(a_n>\ell - \varepsilon\),且仅有有限多个指标\(n\)使得\(a_n>\ell+\varepsilon\)。

  \item \(\limsup\{a_n\}=\infty\)当且仅当\(\{a_n\}\)没有上界。

  \item \(\limsup\{a_n\}=-\liminf\{-a_n\}.\)

  \item 实数序列\(\{a_n\}\)收敛到扩充的实数\(a\)当且仅当
\[\liminf\{a_n\}=\limsup\{a_n\}=a\]

  \item 若对所有\(n\),\(a_n\leqslant b_n\),则
\[\limsup\{a_n\}\leqslant\limsup\{b_n\}\]
\end{enumerate}
\end{proposition}

\begin{definition}[级数的部分和与可和]\label{definition:级数的部分和与可和}
  对每个实数序列\(\{a_k\}\),对每个指标\(n\)对应着定义为\(s_n = \sum_{k = 1}^{n}a_k\)的\textbf{部分和序列}\(\{s_n\}\)。我们说级数\(\sum_{k = 1}^{\infty}a_k\)\textbf{可和于实数\(s\)},若\(\{s_n\}\to s\)且写作\(s = \sum_{k = 1}^{\infty}a_k\)。
\end{definition}

\begin{proposition}[级数收敛/可和的充要条件]\label{proposition:级数收敛/可和的充要条件}
  令\(\{a_n\}\)为实数序列。
\begin{enumerate}[(i)]
  \item 级数\(\sum_{k = 1}^{\infty}a_k\)可和当且仅当对每个\(\varepsilon>0\),存在指标\(N\)使得对\(n\geqslant N\)和任何自然数\(m\),
\[\left|\sum_{k = n}^{n + m}a_k\right|<\varepsilon\]

  \item 若级数\(\sum_{k = 1}^{\infty}|a_k|\)可和,则\(\sum_{k = 1}^{\infty}a_k\)也是可和的。

  \item 若每项\(a_k\)非负,则级数\(\sum_{k = 1}^{\infty}a_k\)可和当且仅当部分和序列是有界的。
\end{enumerate}
\end{proposition}
\begin{proof}
  
\end{proof}







\section{实变量的连续函数实值函数}

\begin{definition}[实值函数在一点连续]\label{definition:实值函数在一点连续}
  令\(f\)为定义在实数集\(E\)上的实值函数。我们说\(f\)在\(E\)中的点\(x\)\textbf{连续},若对每个\(\varepsilon>0\),存在\(\delta>0\),使得
若\(x'\in E\)且\(\vert x' - x\vert<\delta\),则\(\vert f(x') - f(x)\vert<\varepsilon\)
\end{definition}

\begin{definition}[实值函数在定义域上连续]\label{definition:实值函数在定义域上连续}
  称函数\(f\)(在\(E\)上)连续,若它在其定义域\(E\)的每一点是连续的。
\end{definition}

\begin{definition}[Lipschitz连续]\label{definition:Lipschitz连续}
  函数\(f\)称为是\textbf{Lipschitz的}或\textbf{Lipschitz连续}或\textbf{Lipschitz函数},若存在\(c\geqslant0\),使得
对所有\(x'\),\(x\in E\),\(\vert f(x') - f(x)\vert\leqslant c\vert x' - x\vert\)
\end{definition}

\begin{proposition}\label{proposition:一个Lipschitz函数是连续的}
  一个Lipschitz函数是连续的。
\end{proposition}
\begin{note}
  这个命题反过来是不对的,不是所有连续函数都是Lipschitz的。例如,若对于\(0\leqslant x\leqslant1\),\(f(x)=\sqrt{x}\),则\(f\)在\([0, 1]\)上是连续的,但不是Lipschitz的。
\end{note}
\begin{proof}
  事实上,对于数\(x\in E\)和任何\(\varepsilon>0\),\(\delta = \varepsilon/c\)对应关于\(f\)在\(x\)连续的准则的\(\varepsilon\)条件。
\end{proof}

\begin{proposition}[序列的收敛性对函数在一个点的连续性的刻画]\label{proposition:序列的收敛性对函数在一个点的连续性的刻画}
  定义在实数集\(E\)上的实值函数\(f\)在点\(x_*\in E\)连续,当且仅当\(E\)中的序列\(\{x_n\}\)收敛到\(x_*\),它的象序列\(\{f(x_n)\}\)收敛到\(f(x_*)\)。
\end{proposition}


\begin{proposition}[函数在其定义域上连续的刻画]\label{proposition;函数在其定义域上连续的刻画}
  令\(f\)为定义在实数集\(E\)上的实值函数。则\(f\)在\(E\)上连续当且仅当对每个开集\(\mathcal{O}\),
\begin{align}
  f^{-1}(\mathcal{O}) = E\cap\mathcal{U},\text{ 其中 }\mathcal{U}\text{ 是开集}.\label{equation(9)}
\end{align}
\end{proposition}
\begin{proof}
  首先假设任何开集在\(f\)的原象是定义域与一个开集的交。令\(x\)属于\(E\)。为证明\(f\)在\(x\)连续,令\(\varepsilon>0\)。区间\(I=(f(x)-\varepsilon, f(x)+\varepsilon)\)是一个开集。因此,存在开集\(\mathcal{U}\)使得
\[f^{-1}(I)=\{x'\in E|f(x)-\varepsilon<f(x')<f(x)+\varepsilon\}=E\cap\mathcal{U}\]
特别地,\(f(E\cap\mathcal{U})\subseteq I\)且\(x\)属于\(E\cap\mathcal{U}\)。由于\(\mathcal{U}\)是开的,存在\(\delta>0\)使得\((x - \delta, x+\delta)\subseteq\mathcal{U}\)。于是,若\(x'\in E\)且\(\vert x' - x\vert<\delta\),则\(\vert f(x') - f(x)\vert\leqslant\varepsilon\)。因此\(f\)在\(x\)连续。

假定现在\(f\)是连续的。令\(\mathcal{O}\)为开集而\(x\)属于\(f^{-1}(\mathcal{O})\)。则\(f(x)\)属于开集\(\mathcal{O}\),使得存在\(\varepsilon>0\),满足\((f(x)-\varepsilon, f(x)+\varepsilon)\subseteq\mathcal{O}\)。由于\(f\)在\(x\)连续,存在\(\delta>0\)使得若\(x'\)属于\(E\)且\(\vert x' - x\vert<\delta\),则\(\vert f(x') - f(x)\vert<\varepsilon\)。定义\(I_x=(x - \delta, x+\delta)\)。则\(f(E\cap I_x)\subseteq\mathcal{O}\)。定义
\[\mathcal{U}=\bigcup_{x\in f^{-1}(\mathcal{O})}I_x\]
由于\(\mathcal{U}\)是开集的并,它是开的。它已被构造使得\eqref{equation(9)}成立。
\end{proof}

\begin{theorem}[极值定理]\label{theorem:极值定理}
  在非空闭有界实数集上的连续实值函数一定能取得最小值与最大值。
\end{theorem}
\begin{proof}
  令\(f\)为非空闭有界实数集\(E\)上的连续实值函数。我们首先证明\(f\)在\(E\)上有界,即存在实数\(M\),使得
对所有\(x\in E\),都有
\begin{align}
  \vert f(x)\vert\leqslant M\label{equation(10)}
\end{align}
令\(x\)属于\(E\)。令\(\delta>0\)对应关于\(f\)在\(x\)连续的准则的\(\varepsilon = 1\)挑战。定义\(I_x=(x - \delta, x+\delta)\)。因此,若\(x'\)属于\(E\cap I_x\),则\(\vert f(x') - f(x)\vert<1\),因而\(\vert f(x')\vert\leqslant\vert f(x)\vert + 1\)。集族\(\{I_{x}\}_{x\in E}\)是\(E\)的开覆盖。Heine - Borel定理告诉我们\(E\)中存在有限个点\(\{x_1, \cdots, x_n\}\)使得\(\{I_{x_k}\}_{k = 1}^{n}\)也覆盖\(E\)。定义\(M = 1+\max\{\vert f(x_1)\vert, \cdots, \vert f(x_n)\vert\}\)。我们宣称\eqref{equation(10)}对\(E\)的这个选取成立。事实上,令\(x\)属于\(E\)。存在指标\(k\)使得\(x\)属于\(I_{x_k}\),因此\(\vert f(x)\vert\leqslant1+\vert f(x_k)\vert\leqslant M\)。为看到\(f\)在\(E\)上取到最大值,定义\(m = \sup f(E)\)。若\(f\)在\(E\)上取不到值\(m\),则函数\(x\mapsto1/(f(x)-m)(x\in E)\)是\(E\)上的无界连续函数。这与我们刚证明的矛盾。因此,\(f\)取到\(E\)的最大值。由于\(-f\)是连续的,\(-f\)取得最大值,即\(f\)在\(E\)上取到最小值。
\end{proof}

\begin{theorem}[介值定理]\label{theorem:介值定理}
  令\(f\)为闭有界区间\([a, b]\)上的连续实值函数,使得\(f(a)<c<f(b)\)。则存在\((a, b)\)中的点\(x_0\)使得\(f(x_0)=c\)。
\end{theorem}
\begin{proof}
  我们将归纳地定义一个下降的闭区间的可数族\(\{[a_n, b_n]\}_{n = 1}^{\infty}\),其交由单点\(x_0\in(a, b)\)构成,在该点\(f(x_0)=c\)。定义\(a_1 = a\)与\(b_1 = b\)。考虑\([a_1, b_1]\)的中点\(m_1\)。若\(c<f(m_1)\),定义\(a_2 = a_1\)与\(b_2 = m_1\)。若\(f(m_1)\geqslant c\),定义\(a_2 = m_1\)与\(b_2 = b_1\)。因此\(f(a_2)\leqslant c\leqslant f(b_2)\)且\(b_2 - a_2=[b_1 - a_1]/2\)。我们归纳地继续这个二分过程,以得到一个下降的闭区间族\(\{[a_n, b_n]\}_{n = 1}^{\infty}\),使得对所有\(n\)
\[f(a_n)\leqslant c\leqslant f(b_n)\text{ 且 }b_n - a_n=[b - a]/2^{n - 1}\]
根据集套定理,\(\bigcap_{n = 1}^{\infty}[a_n, b_n]\)是非空的。选取\(x_0\)属于\(\bigcap_{n = 1}^{\infty}[a_n, b_n]\)。观察到对所有\(n\),
\[\vert a_n - x_0\vert\leqslant b_n - a_n=[b - a]/2^{n - 1}\]
因此\(\{a_n\}\to x_0\)。根据\(f\)在\(x_0\)的连续性,\(\{f(a_n)\}\to f(x_0)\)。由于对所有\(n\),\(f(a_n)\leqslant c\),且集合\((-\infty, c]\)是闭的,\(f(x_0)\leqslant c\)。用类似的方法,\(f(x_0)\geqslant c\)。因此\(f(x_0)=c\)。
\end{proof}

\begin{definition}[一致连续]\label{definition:一致连续}
  定义在实数集\(E\)上的实值函数\(f\)称为是\textbf{一致连续的},若对每个\(\varepsilon>0\),存在\(\delta>0\)使得对\(E\)中的所有\(x\),\(x'\),
若\(\vert x - x'\vert<\delta\), 则\(\vert f(x) - f(x')\vert<\varepsilon.\)
\end{definition}

\begin{theorem}\label{theorem:闭有界实数集上的连续实值函数是一致连续的。}
  闭有界实数集上的连续实值函数是一致连续的。
\end{theorem}
\begin{proof}
  令\(f\)为闭有界实数集\(E\)上的连续实值函数。令\(\varepsilon>0\)。对每个\(x\in E\),存在\(\delta_x>0\)使得若\(x'\in E\)且\(\vert x' - x\vert<\delta_x\),则\(\vert f(x') - f(x)\vert<\varepsilon/2\)。定义\(I_x\)为开区间\((x - \delta_x/2, x+\delta_x/2)\)。则\(\{I_x\}_{x\in E}\)是\(E\)的开覆盖。根据Heine - Borel定理,存在覆盖\(E\)的有限子族\(\{I_{x_1}, \cdots, I_{x_n}\}\)。定义
\[
\delta=\frac{1}{2}\min\{\delta_{x_1},\cdots,\delta_{x_n}\}
\]
我们宣称该\(\delta>0\)对应关于\(f\)在\(E\)上一致连续的准则的\(\varepsilon>0\)挑战。事实上,令\(x\)和\(x'\)属于\(E\)满足\(\vert x - x'\vert<\delta\)。由于\(\{I_{x_1}, \cdots, I_{x_n}\}\)覆盖\(E\),存在指标\(k\)使得\(\vert x - x_k\vert<\delta_{x_k}/2\)。由于\(\vert x - x'\vert<\delta\leqslant\delta_{x_k}/2\),因此
\[
\vert x' - x_k\vert\leqslant\vert x' - x\vert+\vert x - x_k\vert<\delta_{x_k}/2+\delta_{x_k}/2=\delta_{x_k}
\]
根据\(\delta_{x_k}\)的定义,由于\(\vert x - x_k\vert<\delta_{x_k}\)且\(\vert x' - x_k\vert<\delta_{x_k}\),我们有\(\vert f(x) - f(x_k)\vert<\varepsilon/2\)与\(\vert f(x') - f(x_k)\vert<\varepsilon/2\)。因此
\[
\vert f(x) - f(x')\vert\leqslant\vert f(x) - f(x_k)\vert+\vert f(x') - f(x_k)\vert<\varepsilon/2+\varepsilon/2=\varepsilon.
\]
\end{proof}

\begin{definition}[实值函数的单调性]\label{definition:实值函数的单调性}
  定义在实数集\(E\)上的实值函数\(f\)称为是\textbf{递增的},若\(x\),\(x'\)属于\(E\)且\(x\leqslant x'\)时,\(f(x)\leqslant f(x')\);称为是\textbf{递减的},若\(-f\)是递增的;称为是\textbf{单调的},若它是递增的或递减的。
\end{definition}















\chapter{Lebesgue测度}

\section{Lebesgue外测度}







\section{Lebesgue可测集的$\sigma $代数}





\section{Lebesgue可测集的外逼近和内逼近}





\section{可数可加性、连续性以及Borel-Cantelli引理}




\section{不可测集}





\section{Cantor集和Cantor-Lebesgue函数}







\chapter{Lebesgue可测函数}

\section{和、积与复合}

\section{序列的逐点连续与简单逼近}

\section{Littlewood的三个原理、Egoroff定理以及Lusin定理}









\chapter{Lebesgue积分}

\section{Riemann积分}

\section{有限测度集上的有界可测函数的Lebesgue积分}

\section{非负可测函数的Lebesgue积分}

\section{一般的Lebesgue积分}

\section{积分的可数可加性与连续性}

\section{一致可积性:Vitali收敛定理}

\section{一致可积性和紧性:一般的Vitali收敛定理}

\section{依测度收敛}

\section{Riemann可积与Lebesgue可积的刻画}






\chapter{微分与积分}


\section{单调函数的连续性}

\section{单调函数的可微性:Lebesgue定理}

\section{有界变差函数:Jordan定理}

\section{绝对连续函数}

\section{导数的积分:微分不定积分}

\section{凸函数}






\chapter{$L^p$空间:完备性与逼近}

\section{赋范线性空间}

\section{Young、$\mathrm{H}\ddot{\mathrm{o}}\mathrm{lder}$与Minkowski不等式}

\section{$L^p$是完备的:Riesz-Fischer定理}

\section{逼近与可分性}





\chapter{$L^p$空间:对偶与弱收敛}

\section{关于$L^p(1\leqslant p<\infty)$的对偶的Riesz表示定理}

\section{$L^p$中的弱序列收敛}

\section{弱序列紧性}

\section{凸泛函的最小化}




















\end{document}
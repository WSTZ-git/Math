\documentclass[../../main.tex]{subfiles}
\graphicspath{{\subfix{../../image/}}} % 指定图片目录,后续可以直接使用图片文件名。

% 例如:
% \begin{figure}[h]
% \centering
% \includegraphics{image-01.01}
% \label{fig:image-01.01}
% \caption{图片标题}
% \end{figure}

\begin{document}

\section{中值定理及推广}

\begin{theorem}[积分中值定理]\label{theorem:积分中值定理}
\begin{enumerate}[(1)]
\item \label{theorem:积分中值定理(1)}\(f(x)\in R[a,b],g(x)\)是\([a,b]\)上的非负递减函数, 则存在\(\zeta\in[a,b]\), 使得
\begin{align*}
\int_{a}^{b}f(x)g(x)dx = g(a)\int_{a}^{\zeta}f(x)dx.
\end{align*}

\item \label{theorem:积分中值定理(2)} \(f(x)\in R[a,b],g(x)\)是\([a,b]\)上的非负递增函数, 则存在\(\zeta\in[a,b]\), 使得
\begin{align*}
\int_{a}^{b}f(x)g(x)dx = g(b)\int_{\zeta}^{b}f(x)dx.
\end{align*}

\item \label{theorem:积分中值定理(3)}\(f(x)\in R[a,b],g(x)\)是\([a,b]\)上的单调函数, 则存在\(\zeta\in[a,b]\), 使得
\begin{align*}
\int_{a}^{b}f(x)g(x)dx = g(a)\int_{a}^{\zeta}f(x)dx + g(b)\int_{\zeta}^{b}f(x)dx.
\end{align*}

\item \label{theorem:积分中值定理(4)}\(f(x)\in R[a,b]\)且不变号,\(g(x)\in R[a,b]\), 则存在\(\eta\)介于\(g(x)\)上下确界之间, 使得
\begin{align*}
\int_{a}^{b}f(x)g(x)dx = \eta\int_{a}^{b}f(x)dx.
\end{align*}

\item \label{theorem:积分中值定理(5)}\(f(x)\in R[a,b]\)且不变号,\(g(x)\in C[a,b]\), 则存在\(\zeta\in(a,b)\), 使得
\begin{align*}
\int_{a}^{b}f(x)g(x)dx = g(\zeta)\int_{a}^{b}f(x)dx.
\end{align*}

\item 若(1),(2),(3)中再加入条件\(g(x)\)在\((a,b)\)中不为常数, 则结论可以加强到\(\zeta\in(a,b)\).
\end{enumerate}
\end{theorem}

\begin{theorem}[Hadamard不等式]\label{theorem:Hadamard不等式}
\(f\)是\([a,b]\)上的下凸函数,则
\begin{align*}
\frac{f(a)+f(b)}{2} \geqslant \frac{1}{b - a}\int_{a}^{b}f(x)dx \geqslant f\left(\frac{a + b}{2}\right).
\end{align*} 
\end{theorem}
\begin{note}
一句话积累证明:一边是区间再现,一边是换元到区间\([0,1]\)。
\end{note}
\begin{proof}
由$f$在$[a,b]$上下凸,一方面,我们有
\begin{align*}
\frac{1}{b - a}\int_{a}^{b}f(x)dx = \int_{0}^{1}f(a(1 - t)+bt)dt \leqslant \int_{0}^{1}[(1 - t)f(a)+tf(b)]dt = \frac{f(a)+f(b)}{2}.
\end{align*}
另一方面,我们有
\begin{align*}
\frac{1}{b - a}\int_{a}^{b}f(x)dx &= \frac{1}{b - a}\int_{a}^{b}f(a + b - x)dx
= \frac{1}{2(b - a)}\int_{a}^{b}[f(a + b - x)+f(x)]dx\\
&\geqslant \frac{1}{b - a}\int_{a}^{b}f\left(\frac{a + b}{2}\right)dx
= f\left(\frac{a + b}{2}\right).
\end{align*}
故结论成立.
\end{proof}


\begin{theorem}[达布中值定理/导数介值定理]\label{theorem:导数介值定理}
设\(f \in D[a,b]\),对任何介于\(f'(a),f'(b)\)之间的\(\eta\),存在\(c \in [a,b]\)使得\(f'(c)=\eta\).
\end{theorem}
\begin{proof}
和连续函数介值定理一样,我们只需证明导数满足零点定理。即不妨设\(f'(a) < 0 < f'(b)\),去找\(c \in [a,b]\)使得\(f'(c)=0\)。
事实上由极限保号性和
\begin{align*}
\lim_{x \to a^{+}}\frac{f(x)-f(a)}{x - a}=f'(a)<0,\lim_{x \to b^{-}}\frac{f(x)-f(b)}{x - b}=f'(b)>0,
\end{align*}
我们知道存在\(\delta>0\),使得
\[f(x)<f(a),\forall x \in (a,a + \delta],f(x)<f(b),\forall x \in [b - \delta,b).\]
因此\(f\)最小值在内部取到,此时由费马引理知最小值的导数为\(0\),从而证毕!
\end{proof}

\begin{theorem}[加强的Rolle中值定理]\label{theorem:加强的Rolle中值定理}
\((a)\):设\(f \in D(a,b)\)且在\([a,b]\)上\(f\)有介值性,则若\(f(a)=f(b)\),必然存在\(\theta \in (a,b)\),使得\(f'(\theta)=0\)。
\((b)\):设\(f \in C[a,+\infty)\cap D^{1}(a,+\infty)\)满足\(f(a)=\lim_{x \to +\infty}f(x)\),则存在\(\theta \in (a,+\infty)\)使得\(f'(\theta)=0\)。
\end{theorem}
\begin{note}
一旦罗尔成立,所有中值定理和插值定理都会有类似的结果,可以具体情况具体分析。
\end{note}
\begin{proof}
对于\((a)\):不妨设\(f\)不恒为常数,则可取\(x_0 \in (a,b)\),使得\(f(x_0)\neq f(a)\),不妨设\(f(x_0)>f(a)\),则由\(f\)的介值性,我们知道存在\(x_1 \in (a,x_0),x_2 \in (x_0,b)\),使得
\begin{align*}
f(x_1)=\frac{f(a)+f(x_0)}{2},f(x_2)=\frac{f(b)+f(x_0)}{2}.
\end{align*}
因为\(f(a)=f(b)\),我们知道\(f(x_1)=f(x_2)\),由罗尔中值定理可知道存在\(\theta \in (a,b)\),使得\(f'(\theta)=0\)。这就完成了\((a)\)的证明。
对于\((b)\):若对任何\(x \in (0,+\infty)\)使得\(f'(x)\neq 0\),则\(f\)在\([0,+\infty)\)严格单调,不妨设为递增。现在
\[f(x)\geqslant f(a + 1)>f(a),\forall x\geqslant a + 1,\]
于是
\[f(a)=\lim_{x \to +\infty}f(x)\geqslant f(a + 1)>f(a),\]
这就是一个矛盾!因此我们证明了存在\(\theta \in (a,+\infty)\)使得\(f'(\theta)=0\)。    
\end{proof}








\end{document}
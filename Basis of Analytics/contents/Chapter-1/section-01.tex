\documentclass[../../main.tex]{subfiles}
\graphicspath{{\subfix{../../image/}}} % 指定图片目录,后续可以直接使用图片文件名。

% 例如:
% \begin{figure}[h]
% \centering
% \includegraphics{image-01.01}
% \label{fig:image-01.01}
% \caption{图片标题}
% \end{figure}

\begin{document}

\section{分段估计}

\begin{conclusion}
\textbf{分段估计和式}

分段的方式:将和式分成两部分,一部分是和式的前充分多项(前有限项/前N项),另一部分是余项(从N+1项开始包括后面的所有项).(黎曼积分本质就是和式的极限,直接细分成每一小段,估计每一小段的被积函数值,进而区分积分(和式)的主体部分和余项部分)
\end{conclusion}
\begin{note}
如果和式的极限存在,则由Cauchy收敛准则,可知和式的余项的极限一般会趋于0.
\end{note}


\end{document}
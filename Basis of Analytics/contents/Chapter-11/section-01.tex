\documentclass[../../main.tex]{subfiles}
\graphicspath{{\subfix{../../image/}}} % 指定图片目录,后续可以直接使用图片文件名。

% 例如:
% \begin{figure}[h]
% \centering
% \includegraphics{image-01.01}
% \label{fig:image-01.01}
% \caption{图片标题}
% \end{figure}

\begin{document}

\section{基本定理}

常见的反例:$f(x)=x^msin\frac{1}{x^n}.$

\begin{theorem}[Leibniz公式]\label{theorem:Leibniz公式}
$(f(x)g(x))^{(n)} = \sum_{k = 0}^{n} C_{n}^{k}f^{(n - k)}(x)g^{(k)}(x).$
\end{theorem}

\begin{example}
设\(f(x)\)定义在\([0,1]\)中且\(\lim_{x\rightarrow0^{+}}f\left(x\left(\frac{1}{x}-\left[\frac{1}{x}\right]\right)\right)=0\),证明:\(\lim_{x\rightarrow0^{+}}f(x)=0\)。
\end{example}
\begin{note}
将极限定义中的$\varepsilon、\delta$适当地替换成$\frac{1}{n}、\frac{1}{N}$往往更方便我们分析问题和书写过程.
\end{note}
\begin{proof}
用\(\{x\}\)表示\(x\)的小数部分,则\(x\left(\frac{1}{x}-\left[\frac{1}{x}\right]\right)=x\left\{\frac{1}{x}\right\}\)。

对任意\(\varepsilon>0\),依据极限定义,存在\(\delta>0\)使得任意\(x\in(0,\delta)\)都有\(\left|f\left(x\left\{\frac{1}{x}\right\}\right)\right|<\varepsilon\)。

取充分大的正整数\(N\)使得\(\frac{1}{N}<\delta\),则任意\(x\in\left(\frac{1}{N + 1},\frac{1}{N}\right)\)都有\(\left|f\left(x\left\{\frac{1}{x}\right\}\right)\right|<\varepsilon\)。

考虑函数\(x\left\{\frac{1}{x}\right\}\)在区间\(\left(\frac{1}{N + 1},\frac{1}{N}\right)\)中的值域,也就是连续函数
\[g(u)=\frac{u - [u]}{u}=\frac{u - N}{u},u\in(N,N + 1)\]
的值域,考虑端点处的极限可知\(g(u)\)的值域是\(\left(0,\frac{1}{N + 1}\right)\),且严格单调递增.所以对任意\(y\in\left(0,\frac{1}{N + 1}\right)\),都存在\(x\in\left(\frac{1}{N + 1},\frac{1}{N}\right)\subset(0,\delta)\)使得$\frac{1}{x}=g^{-1}(y)\in(N,N+1)$,即\(y =g(\frac{1}{x})= x\left\{\frac{1}{x}\right\}\),故\(|f(y)|=\left|f\left(x\left\{\frac{1}{x}\right\}\right)\right|<\varepsilon\).

也就是说,任意\(\varepsilon>0\),存在正整数\(N\),使得任意\(y\in\left(0,\frac{1}{N + 1}\right)\),都有\(|f(y)|<\varepsilon\),结论得证。
\end{proof}

\begin{example}

\end{example}
\begin{proof}

\end{proof}



\end{document}
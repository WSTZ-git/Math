\documentclass[../../main.tex]{subfiles}
\graphicspath{{\subfix{../../image/}}} % 指定图片目录,后续可以直接使用图片文件名。

% 例如:
% \begin{figure}[h]
% \centering
% \includegraphics{image-01.01}
% \label{fig:image-01.01}
% \caption{图片标题}
% \end{figure}

\begin{document}

\section{极小多项式与Cayley-Hamilton定理}

\begin{proposition}\label{proposition:矩阵一定适合一个非零多项式}
数域$\mathbb{K}$上的$n$阶矩阵$A$一定适合数域$\mathbb{K}$上的一个非零多项式.
\end{proposition}
\begin{proof}
我们已经知道, 数域$\mathbb{K}$上的$n$阶矩阵全体组成了$\mathbb{K}$上的线性空间, 其维数等于$n^2$. 因此对任一$n$阶矩阵$A$, 下列$n^2 + 1$个矩阵必线性相关:
$A^{n^2}, A^{n^2-1}, \cdots, A, I_n$.

也就是说, 存在$\mathbb{K}$中不全为零的数$c_i (i = 0, 1, 2, \cdots, c_{n^2})$, 使
\begin{align*}
c_{n^2}A^{n^2} + c_{n^2-1}A^{n^2-1} + \cdots + c_1A + c_0I_n = O.
\end{align*}
这表明矩阵$A$适合数域$\mathbb{K}$上的一个非零多项式.
\end{proof}

\begin{definition}[矩阵的极小多项式]
若$n$阶矩阵$A$ (或$n$维线性空间$V$上的线性变换$\varphi$) 适合一个非零首一多项式$m(x)$, 且$m(x)$是$A$ (或$\varphi$) 所适合的非零多项式中次数最小者, 则称$m(x)$是$A$ (或$\varphi$) 的一个极小多项式或最小多项式.
\end{definition}
\begin{remark}
由\hyperref[proposition:矩阵一定适合一个非零多项式]{命题\ref{proposition:矩阵一定适合一个非零多项式}}可知矩阵$A$的极小多项式$m(x)$一定存在,故极小多项式是良定义的.
\end{remark}

\begin{lemma}[矩阵极小多项式的基本性质]\label{lemma:矩阵极小多项式的基本性质}
若$f(x)$是$A$适合的一个多项式, 则$A$的极小多项式$m(x)$整除$f(x)$.
\end{lemma}
\begin{proof}
由多项式的带余除法知道
\begin{align*}
f(x) = m(x)q(x) + r(x),
\end{align*}
且$\deg r(x) < \deg m(x)$. 将$x = A$代入上式得$r(A) = O$, 若$r(x) \neq 0$, 则$A$适合一个比$m(x)$次数更小的非零多项式, 矛盾. 故$r(x) = 0$, 即$m(x) \mid f(x)$.
\end{proof}

\begin{proposition}[矩阵的极小多项式必唯一]\label{proposition:矩阵的极小多项式必唯一}
任一$n$阶矩阵的极小多项式必唯一.
\end{proposition}
\begin{proof}
若$m(x), g(x)$都是矩阵$A$的极小多项式, 则由\hyperref[lemma:矩阵极小多项式的基本性质]{矩阵极小多项式的基本性质}知道$m(x)$能够整除$g(x), g(x)$也能够整除$m(x)$. 因此$m(x)$与$g(x)$只差一个常数因子, 又极小多项式必须首项系数为1, 故$g(x) = m(x)$.
\end{proof}

\begin{proposition}[相似的矩阵具有相同的极小多项式]\label{proposition:相似的矩阵具有相同的极小多项式}
相似的矩阵具有相同的极小多项式.
\end{proposition}
\begin{proof}
设矩阵$A$和$B$相似, 即存在可逆矩阵$P$, 使$B = P^{-1}AP$. 设$A, B$的极小多项式分别为$m(x), g(x)$, 注意到
\begin{align*}
m(B) = m(P^{-1}AP) = P^{-1}m(A)P = O,
\end{align*}
因此$g(x) \mid m(x)$. 同理, $m(x) \mid g(x)$, 故$m(x) = g(x)$.
\end{proof}

\begin{proposition}
设$A$是一个分块对角阵
\begin{align*}
A = \begin{pmatrix}
A_1 & & \\
& A_2 & \\
& & \ddots & \\
& & & A_k
\end{pmatrix},
\end{align*}
其中$A_i$都是方阵, 则$A$的极小多项式等于诸$A_i$的极小多项式之最小公倍式.
\end{proposition}
\begin{proof}
设$A$的极小多项式为$m(x)$, $A_i$的极小多项式为$m_i(x)$, 诸$m_i(x)$的最小公倍式为$g(x)$, 则$g(A_i) = O$, 于是
\begin{align*}
g(A) = \begin{pmatrix}
g(A_1) & & \\
& g(A_2) & \\
& & \ddots & \\
& & & g(A_k)
\end{pmatrix} = O,
\end{align*}
从而$m(x) \mid g(x)$. 又因为
\begin{align*}
m(A) = \begin{pmatrix}
m(A_1) & & \\
& m(A_2) & \\
& & \ddots & \\
& & & m(A_k)
\end{pmatrix} = O,
\end{align*}
从而$m(x) \mid g(x)$. 又因为
\begin{align*}
m(A) = \begin{pmatrix}
m(A_1) & & \\
& m(A_2) & \\
& & \ddots & \\
& & & m(A_k)
\end{pmatrix} = O,
\end{align*}
所以对每个$i$有$m(A_i) = O$, 从而$m_i(x) \mid m(x)$,即$m(x)$是$m_i(x)$的公倍式.又$g(x)$是诸$m_i(x)$的最小公倍式, 故$g(x) \mid m(x)$.综上所述,$m(x) = g(x)$.
\end{proof}

\begin{proposition}\label{proposition:矩阵的特征值一定是其极小多项式的根}
设$m(x)$是$n$阶矩阵$A$的极小多项式, $\lambda_0$是$A$的特征值, 则$(x - \lambda_0) \mid m(x)$.
\end{proposition}
\begin{note}
这个命题告诉我们:矩阵的特征值一定是其极小多项式的根.
\end{note}
\begin{proof}
由$m(A) = O$及
\hyperref[proposition:矩阵适合的多项式其特征值也适合]{命题\ref{proposition:矩阵适合的多项式其特征值也适合}}可得$m(\lambda_0) = 0$, 故结论成立.
\end{proof}

\begin{theorem}[Cayley-Hamilton定理]\label{theorem:Cayley-Hamilton定理}
\begin{enumerate}
\item \textbf{代数形式:}设$A$是数域$\mathbb{K}$上的$n$阶矩阵, $f(x)$是$A$的特征多项式, 则$f(A) = O$.

\item \textbf{几何形式:}设$\varphi$是$n$维线性空间$V$上的线性变换, $f(x)$是$\varphi$的特征多项式, 则$f(\varphi) = O$.
\end{enumerate}
\end{theorem}
\begin{proof}
\begin{enumerate}
\item {\heiti 代数形式:}因为复数域是最大数域,所以可将$A$看作一个复矩阵.由\hyperref[theorem:复方阵必相似于上三角阵]{复方阵必相似于上三角阵}知$A$复相似于一个上三角阵, 也就是说存在的可逆矩阵$P$, 使$P^{-1}AP = B$是一个上三角阵, 其中$P$与$B$都是复矩阵,由\hyperref[theorem:相似矩阵有相同的特征多项式与特征值]{相似矩阵有相同特征多项式}可知$A$与$B$有相同的特征多项式$f(x)$. 记
\begin{align*}
f(x) = x^n + a_1 x^{n-1} + \cdots + a_n,
\end{align*}
则$f(B) = O$. 而
\begin{align*}
f(A) &= A^n + a_1 A^{n-1} + \cdots + a_n I_n \\
&= (PBP^{-1})^n + a_1 (PBP^{-1})^{n-1} + \cdots + a_n I_n \\
&= PB^n P^{-1} + a_1 PB^{n-1} P^{-1} + \cdots + a_n I_n \\
&= P(B^n + a_1 B^{n-1} + \cdots + a_n I_n)P^{-1} \\
&= P f(B) P^{-1} = O.
\end{align*}

\item {\heiti 几何形式:}设 $\{ e_1,e_2,\cdots ,e_n \}$ 是 $V$ 的一组标准基,$\varphi$ 在这组基下的矩阵为 $A$,  
则由 $f(x)$ 是 $\varphi$ 的特征多项式可知,$f(x)$ 也是 $A$ 的特征多项式。  
从而由代数形式的结论可知 $f(A) = 0$。  
于是对 $\forall \alpha \in V$,都存在 $k_1,k_2,\cdots ,k_n$,使得  
\begin{align*}
\alpha = k_1e_1+k_2e_2+\cdots +k_ne_n 
= \left( e_1,e_2,\cdots ,e_n \right) \begin{pmatrix}
k_1\\
k_2\\
\vdots\\
k_n
\end{pmatrix}.
\end{align*}
两边同时作用 $\varphi$ 得到  
\begin{align*}
&\varphi \left( \alpha \right) =k_1\varphi \left( e_1 \right) +k_2\varphi \left( e_2 \right) +\cdots +k_n\varphi \left( e_n \right) =\left( \varphi \left( e_1 \right) ,\varphi \left( e_2 \right) ,\cdots ,\varphi \left( e_n \right) \right) \left( \begin{array}{c}
k_1\\
k_2\\
\vdots\\
k_n\\
\end{array} \right) 
\\
&=\left( e_1,e_2,\cdots ,e_n \right) A\left( \begin{array}{c}
k_1\\
k_2\\
\vdots\\
k_n\\
\end{array} \right) =A\left( \begin{array}{c}
k_1\\
k_2\\
\vdots\\
k_n\\
\end{array} \right) =A\left( e_1,e_2,\cdots ,e_n \right) \left( \begin{array}{c}
k_1\\
k_2\\
\vdots\\
k_n\\
\end{array} \right) =A\alpha .
\end{align*}
因此 $f(\varphi)(\alpha) = f(A)(\alpha) = 0$。故由 $\alpha$ 的任意性可知 $f(\varphi) = O$.
\end{enumerate}
\end{proof}

\begin{corollary}\label{corollary:矩阵的极小多项式是其特征多项式的因式}
$n$阶矩阵$A$的极小多项式是其特征多项式的因式. 特别, $A$的极小多项式的次数不超过$n$.
\end{corollary}
\begin{proof}
由\hyperref[theorem:Cayley-Hamilton定理]{Cayley-Hamilton定理}及\hyperref[lemma:矩阵极小多项式的基本性质]{矩阵极小多项式的基本性质}即得结论.
\end{proof}

\begin{corollary}
$n$阶矩阵$A$的极小多项式和特征多项式有相同的根(不计重数).
\end{corollary}
\begin{proof}
由\hyperref[proposition:矩阵的特征值一定是其极小多项式的根]{命题\ref{proposition:矩阵的特征值一定是其极小多项式的根}}和\hyperref[corollary:矩阵的极小多项式是其特征多项式的因式]{推论\ref{corollary:矩阵的极小多项式是其特征多项式的因式}}即得结论.
\end{proof}


\end{document}
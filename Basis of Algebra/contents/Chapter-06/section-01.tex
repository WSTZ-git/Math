\documentclass[../../main.tex]{subfiles}
\graphicspath{{\subfix{../../image/}}} % 指定图片目录,后续可以直接使用图片文件名。

% 例如:
% \begin{figure}[h]
% \centering
% \includegraphics{image-01.01}
% \label{fig:image-01.01}
% \caption{图片标题}
% \end{figure}

\begin{document}

\section{对角化}

\begin{definition}[可对角化线性变换]
若 $n$ 维线性空间 $V$ 上的线性变换 $\varphi$ 在某组基 $\{e_1, e_2, \cdots, e_n\}$ 下的表示矩阵为对角阵:
\[
\begin{pmatrix}
\lambda_1 & & & \\
& \lambda_2 & & \\
& & \ddots & \\
& & & \lambda_n
\end{pmatrix},
\]
则称 $\varphi$ 为\textbf{可对角化线性变换}.
\end{definition}

\begin{theorem}[线性变换可对角化的充要条件]\label{theorem:线性变换可对角化的充要条件}
设 $\varphi$ 是 $n$ 维线性空间 $V$ 上的线性变换,则 $\varphi$ 可对角化的充分必要条件是 $\varphi$ 有 $n$ 个线性无关的特征向量。
\end{theorem}
\begin{proof}
若$\varphi$ 是$V$上可对角化线性变换,则可设$\varphi$ 在某组基 $\{e_1, e_2, \cdots, e_n\}$ 下的表示矩阵为对角阵:
\[
\begin{pmatrix}
\lambda_1 & & & \\
& \lambda_2 & & \\
& & \ddots & \\
& & & \lambda_n
\end{pmatrix},
\]
此时 $\varphi(e_i) = \lambda_i e_i$,即 $e_1, e_2, \cdots, e_n$ 是 $\varphi$ 的特征向量,于是 $\varphi$ 有 $n$ 个线性无关的特征向量。

反过来,若 $n$ 维线性空间 $V$ 上的线性变换 $\varphi$ 有 $n$ 个线性无关的特征向量 $e_1, e_2, \cdots, e_n$,则这组向量构成了 $V$ 的一组基,且 $\varphi$ 在这组基下的表示矩阵显然是一个对角阵。
\end{proof}

\begin{definition}[可对角化矩阵]
设 $A$ 是 $n$ 阶矩阵, 若 $A$ 相似于对角阵, 即存在可逆阵 $P$, 使 $P^{-1}AP$ 为对角阵, 则称 $A$ 为\textbf{可对角化矩阵}.
\end{definition}

\begin{lemma}\label{lemma:线性变换与矩阵的可对角化等价,特征值相同}
设$A$是$n$阶矩阵,$\varphi$ 是线性空间 $V$ 上由矩阵 $A$ 乘法诱导的线性变换,即 $\varphi(\alpha) = A\alpha, \forall \alpha \in V$。设 $\{e_1, e_2, \cdots, e_n\}$ 是 $V$ 的一组标准基,则 $\varphi$ 在这组基下的矩阵就是 $A$。证明:
\begin{enumerate}[(1)]
\item 矩阵 $A$ 与线性变换 $\varphi$ 的特征值相同;

\item 矩阵 $A$ 可对角化等价于线性变换 $\varphi$ 可对角化。
\end{enumerate}
\end{lemma}
\begin{proof}
\begin{enumerate}[(1)]
\item 若 $\lambda$ 为矩阵 $A$ 的特征值,则存在 $\xi \in V$,使得 $\varphi(\xi) = A\xi = \lambda \xi$,因此矩阵 $A$ 的特征值也是线性变换 $\varphi$ 的特征值。

若 $\lambda$ 为线性变换 $\varphi$ 的特征值,则存在 $\eta \in V$,使得 $\varphi(\eta) = A\eta = \lambda \eta$,因此线性变换 $\varphi$ 的特征值也是矩阵 $A$ 的特征值。

故矩阵 $A$ 与线性变换 $\varphi$ 的特征值相同。

\item 若矩阵 $A$ 可对角化,则存在可逆矩阵 $P$,使得 $P^{-1}AP$ 为对角矩阵。

从而 $(e_1, e_2, \cdots, e_n)P$ 的列向量也是 $V$ 的一组基,于是由\hyperref[proposition:线性映射与过渡矩阵]{命题\ref{proposition:线性映射与过渡矩阵}} 可知 $\varphi$ 在这组基下的矩阵为 $P^{-1}AP$ 是对角矩阵,故 $\varphi$ 也可对角化。

若线性变换 $\varphi$ 可对角化,则存在 $V$ 的一组基 $\{f_1, f_2, \cdots, f_n\}$,使得 $\varphi$ 在这组基下的矩阵 $B$ 为对角矩阵。设基 $\{e_1, e_2, \cdots, e_n\}$ 到基 $\{f_1, f_2, \cdots, f_n\}$ 的过渡矩阵为 $G$,则由\hyperref[proposition:线性映射与过渡矩阵]{命题\ref{proposition:线性映射与过渡矩阵}}可知 $B = G^{-1}AG$。因此矩阵 $A$ 也可对角化。

故矩阵 $A$ 可对角化等价于线性变换 $\varphi$ 可对角化。
\end{enumerate}
\end{proof}

\begin{theorem}[矩阵可对角化的充要条件]\label{theorem:矩阵可对角化的充要条件}
设 $A$ 是 $n$ 阶矩阵, 则 $A$ 可对角化的充分必要条件是 $A$ 有 $n$ 个线性无关的特征向量.
\end{theorem}
\begin{proof}
设 $\varphi$ 是线性空间 $V$ 上由矩阵 $A$ 乘法诱导的线性变换.

若矩阵 $A$ 有 $n$ 个线性无关的特征值,则由\hyperref[lemma:线性变换与矩阵的可对角化等价,特征值相同]{引理\ref{lemma:线性变换与矩阵的可对角化等价,特征值相同}(1)}可知线性变换 $\varphi$ 也有相同的 $n$ 个线性无关的特征值,于是由\hyperref[theorem:线性变换可对角化的充要条件]{定理\ref{theorem:线性变换可对角化的充要条件}}可知线性变换 $\varphi$ 可对角化,从而再由\hyperref[lemma:线性变换与矩阵的可对角化等价,特征值相同]{引理\ref{lemma:线性变换与矩阵的可对角化等价,特征值相同}(2)} 可知矩阵 $A$ 也可对角化.

若矩阵 $A$ 可对角化,则由\hyperref[lemma:线性变换与矩阵的可对角化等价,特征值相同]{引理\ref{lemma:线性变换与矩阵的可对角化等价,特征值相同}(2)}可知线性变换 $\varphi$ 也可对角化,从而由\hyperref[theorem:线性变换可对角化的充要条件]{定理\ref{theorem:线性变换可对角化的充要条件}}可知 $\varphi$ 有 $n$ 个线性无关的特征值,于是由\hyperref[lemma:线性变换与矩阵的可对角化等价,特征值相同]{引理\ref{lemma:线性变换与矩阵的可对角化等价,特征值相同}(1)}可知矩阵 $A$ 也有相同的 $n$ 个线性无关的特征值。
\end{proof}

\begin{theorem}\label{theorem:不同特征值的特征子空间的和必是直和}
若 $\lambda_1, \lambda_2, \cdots, \lambda_k$ 为 $n$ 维线性空间 $V$ 上的线性变换 $\varphi$ 的不同的特征值,记$\lambda_i$的特征子空间为$V_i(1\leq i\leq k)$,则
\[
V_1 + V_2 + \cdots + V_k = V_1 \oplus V_2 \oplus \cdots \oplus V_k.
\]
\end{theorem}
\begin{proof}
对 $k$ 用数学归纳法. 若 $k = 1$, 结论显然成立. 现设对 $k - 1$ 个不同的特征值 $\lambda_1, \lambda_2, \cdots, \lambda_{k-1}$, 它们相应的特征子空间 $V_1, V_2, \cdots, V_{k-1}$ 之和是直和. 我们要证明 $V_1, V_2, \cdots, V_{k-1}, V_k$ 之和为直和, 这只需证明:
\begin{align}\label{theorem6.6-0.1}
V_k \cap (V_1 + V_2 + \cdots + V_{k-1}) = 0.
\end{align}
即可. 设 $v \in V_k \cap (V_1 + V_2 + \cdots + V_{k-1})$, 则
\begin{align}\label{theorem6.6-1.1}
v = v_1 + v_2 + \cdots + v_{k-1},
\end{align}
其中 $v_i \in V_i (i = 1, 2,\cdots, k - 1)$. 在\eqref{theorem6.6-1.1}式两边作用 $\varphi$,得
\begin{align}\label{theorem6.6-1.2}
\varphi(v) = \varphi(v_1) + \varphi(v_2) + \cdots + \varphi(v_{k-1}).   
\end{align}
但 $v, v_1, v_2, \cdots, v_{k-1}$ 都是 $\varphi$ 的特征向量或零向量, 因此
\begin{align}\label{theorem6.6-1.3}
\lambda_k v = \lambda_1 v_1 + \lambda_2 v_2 + \cdots + \lambda_{k-1} v_{k-1}.    
\end{align}
在\eqref{theorem6.6-1.2}式两边乘以 $\lambda_k$ 减去\eqref{theorem6.6-1.3}式得
\[
0 = (\lambda_k - \lambda_1) v_1 + (\lambda_k - \lambda_2) v_2 + \cdots + (\lambda_k - \lambda_{k-1}) v_{k-1}.
\]
由于 $v_1, v_2, \cdots, v_{k-1}$ 是直和, 因此 $(\lambda_k - \lambda_i) v_i = 0$, 而 $\lambda_k - \lambda_i \neq 0$, 从而 $v_i = 0 (i = 1, 2, \cdots, k - 1)$. 这就证明了\eqref{theorem6.6-0.1}式.
\end{proof}

\begin{corollary}\label{corollary:线性变换属于不同特征值的特征向量必线性无关}
线性变换 $\varphi$ 属于不同特征值的特征向量必线性无关.
\end{corollary}
\begin{proof}
设 $\lambda_1, \lambda_2, \cdots, \lambda_k$ 是线性变换 $\varphi$ 的 $k$ 个不同特征值,由\hyperref[theorem:不同特征值的特征子空间的和必是直和]{定理\ref{theorem:不同特征值的特征子空间的和必是直和}}可知$V_{\lambda_1} \oplus V_{\lambda_2} \oplus \cdots \oplus V_{\lambda_k}$。于是任取 $\alpha_i \in V_{\lambda_i} \ (1 \leqslant i \leqslant k)$ 且 $\alpha_i \ne 0$,假设 $\alpha_1, \alpha_2, \cdots, \alpha_k$ 线性无关,则存在一组不全为零的数 $b_1, b_2, \cdots, b_k$,使得
\begin{align*}
b_1 \alpha_1 + b_2 \alpha_2 + \cdots + b_k \alpha_k = 0.
\end{align*}
不妨设 $b_1 \ne 0$,则
\begin{align*}
\alpha_1 = \frac{b_2}{b_1} \alpha_2 + \frac{b_3}{b_1} \alpha_3 + \cdots + \frac{b_k}{b_1} \alpha_k \in V_{\lambda_1} \cap \left( V_{\lambda_2} \oplus \cdots \oplus V_{\lambda_k} \right).
\end{align*}
又由 $V_{\lambda_1} \oplus V_{\lambda_2} \oplus \cdots \oplus V_{\lambda_k}$ 及直和的等价条件可知,
\begin{align*}
V_{\lambda_1} \cap \left( V_{\lambda_2} \oplus \cdots \oplus V_{\lambda_k} \right) = \{ 0 \},
\end{align*}
从而 $\alpha_1 = 0$,这与 $\alpha_i \ne 0 \ (1 \leqslant i \leqslant k)$ 矛盾!
\end{proof}

\begin{corollary}
若$n$维线性空间$V$上的线性变换 $\varphi$有$n$个不同的特征值,则 $\varphi$ 必可对角化.
\end{corollary}
\begin{note}
注意这个推论只是可对角化的充分条件而非必要条件, 比如说纯量变换 $\varphi = cI_V$ 当然可对角化,但 $\varphi$ 的 $n$ 个特征值都是$c$.
\end{note}
\begin{proof}
设 $\lambda_1, \lambda_2, \cdots, \lambda_n$ 是线性变换 $\varphi$ 的 $n$ 个不同特征值,则任取 $\alpha_i \in V_{\lambda_i} \ (1 \leqslant i \leqslant n)$,由\hyperref[corollary:线性变换属于不同特征值的特征向量必线性无关]{推论\ref{corollary:线性变换属于不同特征值的特征向量必线性无关}}可知 $\alpha_1, \alpha_2, \cdots, \alpha_n$ 一定线性无关。从而由\hyperref[theorem:线性变换可对角化的充要条件]{定理\ref{theorem:线性变换可对角化的充要条件}}可知,$\varphi$ 一定可对角化。
\end{proof}

\begin{theorem}[线性变换可对角化的充要条件]\label{theorem:线性变换可对角化的充要条件2}
设 $\varphi$ 是 $n$ 维线性空间 $V$ 上的线性变换, $\lambda_1, \lambda_2, \cdots, \lambda_k$ 是 $\varphi$ 的全部不同的特征值, $V_i (i = 1, 2, \cdots, k)$ 是特征值 $\lambda_i$ 的特征子空间, 则 $\varphi$ 可对角化的充要条件是
\begin{align*}
V = V_1 \oplus V_2 \oplus \cdots \oplus V_k.
\end{align*}
\end{theorem}
\begin{proof}
先证充分性. 设
\begin{align*}
V = V_1 \oplus V_2 \oplus \cdots \oplus V_k,
\end{align*}
分别取 $V_i$ 的一组基 $\{e_{i1}, e_{i2}, \cdots, e_{it_i}\} (i = 1, 2, \cdots, k)$, 则由\hyperref[theorem:直和的等价条件]{直和的等价条件(4)} 知这些向量拼成了 $V$ 的一组基, 并且它们都是 $\varphi$ 的特征向量. 因此 $\varphi$ 有 $n$ 个线性无关的特征向量, 从而\hyperref[theorem:线性变换可对角化的充要条件]{定理\ref{theorem:线性变换可对角化的充要条件}}可知$\varphi$可对角化.

再证必要性. 设 $\varphi$ 可对角化, 则由\hyperref[theorem:线性变换可对角化的充要条件]{定理\ref{theorem:线性变换可对角化的充要条件}}可知$\varphi$ 有 $n$ 个线性无关的特征向量 $\{e_1, e_2, \cdots, e_n\}$, 它们构成了 $V$ 的一组基. 不失一般性, 可设这组基中前 $t_1$ 个是关于特征值 $\lambda_1$ 的特征向量; 接下去的 $t_2$ 个是关于特征值 $\lambda_2$ 的特征向量; $\cdots$; 最后 $t_k$ 个是关于特征值 $\lambda_k$ 的特征向量. 对任一 $\alpha \in V$, 设 $\alpha = a_1 e_1 + a_2 e_2 + \cdots + a_n e_n$, 则 $\alpha$ 可写成 $V_1, V_2, \cdots, V_k$ 中向量之和, 因此由\hyperref[theorem:不同特征值的特征子空间的和必是直和]{定理\ref{theorem:不同特征值的特征子空间的和必是直和}}可知
\begin{align*}
V = V_1 + V_2 + \cdots + V_k = V_1 \oplus V_2 \oplus \cdots \oplus V_k.
\end{align*}
\end{proof}


\begin{definition}[线性变换的几何重数与代数重数]
设 $\lambda_0$ 是 $n$ 维线性空间 $V$ 上的线性变换 $\varphi$ 的一个特征值, $V_0$ 是属于 $\lambda_0$ 的特征子空间, 称 $\dim V_0$ 为 $\lambda_0$ 的\textbf{度数或几何重数}. $\lambda_0$ 作为 $\varphi$ 的特征多项式根的重数称为 $\lambda_0$ 的\textbf{重数或代数重数}.
\end{definition}
\begin{note}
由线性映射的维数公式可知,特征值$\lambda_0$的度数$\dim V_0 = \dim \operatorname{Ker}(\lambda_0 I_V - \varphi) = n - r(\lambda_0 I_V - \varphi)$,而特征值$\lambda_0$的重数则由特征多项式$|\lambda I_V - \varphi|$的因式分解决定.
\end{note}

\begin{definition}[矩阵的几何重数与代数重数]
设 $\lambda_0$ 是$n$阶方阵的$A$的一个特征值, $V_0$ 是属于 $\lambda_0$ 的特征子空间, 称 $\dim V_0$ 为 $\lambda_0$ 的\textbf{度数或几何重数}.$\lambda_0$ 作为 $A$ 的特征多项式根的重数称为 $\lambda_0$ 的\textbf{重数或代数重数}.
\end{definition}
\begin{note}
由线性方程组的理论可知,特征值$\lambda_0$的度数$\dim V_0  = n - r(\lambda_0 I_n - A)$,若将$A$看作由矩阵$A$乘法诱导的$V$上的线性变换,则由线性变换的维数公式可知$\dim V_0  =\dim \operatorname{Ker}(\lambda_0 I_V - A)= n - r(\lambda_0 I_V - A)$.
而特征值$\lambda_0$的重数则由特征多项式$|\lambda I_n - A|$的因式分解决定.
\end{note}

\begin{lemma}\label{lemma:特征值的度数总小于重数}
设$\varphi$是$n$维线性空间$V$上的线性变换, $\lambda_0$是$\varphi$的一个特征值, 则$\lambda_0$的度数总是小于等于$\lambda_0$的重数.
\end{lemma}
\begin{proof}
设特征值$\lambda_0$的重数为$m$, 度数为$t$, 又$V_0$是属于$\lambda_0$的特征子空间, 则$\dim V_0 = t$. 设$\{e_1, \cdots, e_t\}$是$V_0$的一组基. 由于$V_0$中的非零向量都是$\varphi$关于$\lambda_0$的特征向量, 故
\begin{align*}
\varphi(e_i) = \lambda_0 e_i, \quad i = 1, \cdots, t.
\end{align*}
将$\{e_1, \cdots, e_t\}$扩充为$V$的一组基, 记为$\{e_1, \cdots, e_t, e_{t+1}, \cdots, e_n\}$, 则$\varphi$在这组基下的表示矩阵为
\begin{align*}
A = \begin{pmatrix}
\lambda_0 I_t & * \\
O & B
\end{pmatrix},
\end{align*}
其中$B$是一个$n-t$阶方阵. 因此, 线性变换$\varphi$的特征多项式具有如下形式:
\begin{align*}
|\lambda I_V - \varphi| = |\lambda I_n - A| = (\lambda - \lambda_0)^t |\lambda I_{n-t} - B|,
\end{align*}
这表明$\lambda_0$的重数至少为$t$, 即$t \leq m$.
\end{proof}

\begin{definition}[完全的特征向量系]
设 $\lambda_0$ 是 $\varphi$ (或 $A$) 的 $m$ 重特征值, 即它是 $\varphi$ (或 $A$) 的特征多项式的 $m$ 重根.此时若有 $m = \dim V_{\lambda_0}$, 即 $\lambda_0$ 的代数重数和几何重数相等, 则称  \textbf{$\lambda_0$有完全的特征向量系}。若对 $\varphi$ (或 $A$) 的任一特征值, 其代数重数和几何重数都相等, 则称  \textbf{$\varphi$ (或 $A$) 有完全的特征向量系}.
\end{definition}

\begin{theorem}[线性变换可对角化的充要条件]\label{theorem:线性变换可对角化的充要条件3}
设$\varphi$是$n$维线性空间$V$上的线性变换, 则$\varphi$可对角化的充分必要条件是$\varphi$有完全的特征向量系.
\end{theorem}
\begin{proof}
设$\lambda_1, \lambda_2, \cdots, \lambda_k$是$\varphi$的全部不同的特征值, 它们对应的特征子空间、重数和度数分别记为$V_i, m_i, t_i (i = 1, 2, \cdots, k)$. 由重数的定义以及\hyperref[lemma:特征值的度数总小于重数]{引理\ref{lemma:特征值的度数总小于重数}}可知
$m_1 + m_2 + \cdots + m_k = n, t_i \leq m_i, i = 1, 2, \cdots, k$.

由\hyperref[theorem:线性变换可对角化的充要条件2]{定理\ref{theorem:线性变换可对角化的充要条件2}}可知,我们只要证明$\varphi$有完全的特征向量系当且仅当$V = V_1 \oplus V_2 \oplus \cdots \oplus V_k$.

若$V = V_1 \oplus V_2 \oplus \cdots \oplus V_k$, 则
\begin{align*}
n &= \dim V = \dim(V_1 \oplus V_2 \oplus \cdots \oplus V_k) \\
&= \dim V_1 + \dim V_2 + \cdots + \dim V_k \\
&= \sum_{i=1}^{k} t_i \leq \sum_{i=1}^{k} m_i = n,
\end{align*}
因此$t_i = m_i (i = 1, 2, \cdots, k)$, 即$\varphi$有完全的特征向量系. 反过来, 若$\varphi$有完全的特征向量系, 则
\begin{align*}
\dim(V_1 \oplus V_2 \oplus \cdots \oplus V_k) = \sum_{i=1}^{k} t_i = \sum_{i=1}^{k} m_i = n = \dim V,
\end{align*}
又$V_1\oplus V_2\oplus \cdots \oplus V_k\subset V$,故$V = V_1 \oplus V_2 \oplus \cdots \oplus V_k$成立.
\end{proof}

\begin{theorem}[线性变换可对角化的充要条件汇总]\label{theorem:线性变换可对角化的充要条件汇总}
\textbf{几何形式:}
\begin{enumerate}[(1)]
\item 设 $\varphi$ 是 $n$ 维线性空间 $V$ 上的线性变换,则 $\varphi$ 可对角化的充分必要条件是 $\varphi$ 有 $n$ 个线性无关的特征向量.

\item 设 $\varphi$ 是 $n$ 维线性空间 $V$ 上的线性变换, $\lambda_1, \lambda_2, \cdots, \lambda_k$ 是 $\varphi$ 的全部不同的特征值, $V_i (i = 1, 2, \cdots, k)$ 是特征值 $\lambda_i$ 的特征子空间, 则 $\varphi$ 可对角化的充要条件是
\begin{align*}
V = V_1 \oplus V_2 \oplus \cdots \oplus V_k.
\end{align*}

\item 设$\varphi$是$n$维线性空间$V$上的线性变换, 则$\varphi$可对角化的充分必要条件是$\varphi$有完全的特征向量系.
\end{enumerate}
\textbf{代数形式:}
\begin{enumerate}[(1)]
\item 设 $A$ 是 $n$ 阶矩阵, 则 $A$ 可对角化的充分必要条件是 $A$ 有 $n$ 个线性无关的特征向量.

\item 设$A$是$n$阶矩阵, $\lambda_1, \lambda_2, \cdots, \lambda_k$ 是 $A$ 的全部不同的特征值, $V_i (i = 1, 2, \cdots, k)$ 是特征值 $\lambda_i$ 的特征子空间, 则 $A$ 可对角化的充要条件是
\begin{align*}
V = V_1 \oplus V_2 \oplus \cdots \oplus V_k.
\end{align*}

\item 设$A$是$n$阶矩阵, 则$\varphi$可对角化的充分必要条件是$A$有完全的特征向量系.
\end{enumerate}
\end{theorem}
\begin{proof}
{\heiti 几何形式:}
\begin{enumerate}[(1)]
\item 证明见\hyperref[theorem:线性变换可对角化的充要条件]{定理\ref{theorem:线性变换可对角化的充要条件}}.

\item 证明见\hyperref[theorem:线性变换可对角化的充要条件2]{定理\ref{theorem:线性变换可对角化的充要条件2}}.

\item 证明见\hyperref[theorem:线性变换可对角化的充要条件3]{定理\ref{theorem:线性变换可对角化的充要条件3}}.
\end{enumerate}
{\heiti 代数形式:}
由几何形式的结论及\hyperref[lemma:线性变换与矩阵的可对角化等价,特征值相同]{引理\ref{lemma:线性变换与矩阵的可对角化等价,特征值相同}}立即得到证明.
\end{proof}


\end{document}
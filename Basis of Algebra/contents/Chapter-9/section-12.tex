\documentclass[../../main.tex]{subfiles}
\graphicspath{{\subfix{../../image/}}} % 指定图片目录,后续可以直接使用图片文件名。

% 例如:
% \begin{figure}[h]
% \centering
% \includegraphics{image-01.01}
% \label{fig:image-01.01}
% \caption{图片标题}
% \end{figure}

\begin{document}

\section{实正规矩阵的正交相似标准型}









\begin{proposition}\label{proposition:两正交矩阵的和的秩与n的差为奇数则行列式和为0}
设\(\boldsymbol{A},\boldsymbol{B}\)为\(n\)阶正交矩阵, 求证: \(\vert\boldsymbol{A}\vert+\vert\boldsymbol{B}\vert = 0\)当且仅当\(n-\mathrm{r}(\boldsymbol{A}+\boldsymbol{B})\)为奇数.
\end{proposition}
\begin{remark}
这个命题的直接推论是: 若正交矩阵\(\boldsymbol{A},\boldsymbol{B}\)满足\(\vert\boldsymbol{A}\vert+\vert\boldsymbol{B}\vert = 0\), 则\(\vert\boldsymbol{A}+\boldsymbol{B}\vert = 0\). 这一结论也可由第2章矩阵的技巧 (类似于例2.19的讨论) 来得到. 又因为正交矩阵行列式的值等于\(1\)或\(-1\), 故例9.119的等价命题为: 设\(\boldsymbol{A},\boldsymbol{B}\)为\(n\)阶正交矩阵, 则\(\vert\boldsymbol{A}\vert = \vert\boldsymbol{B}\vert\)当且仅当\(n-\mathrm{r}(\boldsymbol{A}+\boldsymbol{B})\)为偶数.
\end{remark}
\begin{proof}
因为正交矩阵的逆矩阵以及正交矩阵的乘积都是正交矩阵, 故\(\boldsymbol{AB}^{-1}\)还是正交矩阵. \(\vert\boldsymbol{A}\vert+\vert\boldsymbol{B}\vert = 0\)等价于\(\vert\boldsymbol{AB}^{-1}\vert = -1\), 又\(\mathrm{r}(\boldsymbol{A}+\boldsymbol{B})=\mathrm{r}(\boldsymbol{AB}^{-1}+\boldsymbol{I}_{n})\), 故只要证明: 若\(\boldsymbol{A}\)是\(n\)阶正交矩阵, 则\(\vert\boldsymbol{A}\vert = -1\)当且仅当\(n-\mathrm{r}(\boldsymbol{A}+\boldsymbol{I}_{n})\)为奇数即可. 下面给出两种证法.

{\color{blue}证法一:}设\(\boldsymbol{P}\)是正交矩阵, 使得
    \begin{align*}
    \boldsymbol{P}^{\prime}\boldsymbol{AP}=\mathrm{diag}\left\{\begin{pmatrix}\cos\theta_{1}&-\sin\theta_{1}\\\sin\theta_{1}&\cos\theta_{1}\end{pmatrix},\cdots,\begin{pmatrix}\cos\theta_{r}&-\sin\theta_{r}\\\sin\theta_{r}&\cos\theta_{r}\end{pmatrix},1,\cdots,1, -1,\cdots,-1\right\},
    \end{align*}
    其中\(\sin\theta_{i}\neq 0(1\leq i\leq r)\), 且有\(s\)个\(1\), \(t\)个\(-1\). 于是\(\vert\boldsymbol{A}\vert = (-1)^{t}\), 并且
    \begin{align*}
    \boldsymbol{P}^{\prime}(\boldsymbol{A}+\boldsymbol{I}_{n})\boldsymbol{P}&=\mathrm{diag}\left\{\begin{pmatrix}1 + \cos\theta_{1}&-\sin\theta_{1}\\\sin\theta_{1}&1 + \cos\theta_{1}\end{pmatrix},\cdots,\begin{pmatrix}1 + \cos\theta_{r}&-\sin\theta_{r}\\\sin\theta_{r}&1 + \cos\theta_{r}\end{pmatrix},2,\cdots,2,0,\cdots,0\right\},
    \end{align*}
    从而\(\mathrm{r}(\boldsymbol{A}+\boldsymbol{I}_{n}) = n - t\). 因此\(\vert\boldsymbol{A}\vert = -1\)当且仅当\(t\)为奇数, 即当且仅当\(n-\mathrm{r}(\boldsymbol{A}+\boldsymbol{I}_{n})\)为奇数.

{\color{blue}证法二:}由于正交矩阵\(\boldsymbol{A}\)也是复正规矩阵, 从而酉相似于对角矩阵, 特别地, \(\boldsymbol{A}\)可复对角化. 注意到\(\boldsymbol{A}\)的特征值是模长等于\(1\)的复数, 故或者是模长等于\(1\)的共轭虚特征值, 或者是\(\pm1\). 设\(\boldsymbol{A}\)的特征值\(-1\)的几何重数\(n-\mathrm{r}(\boldsymbol{A}+\boldsymbol{I}_{n}) = t\), 则其代数重数也为\(t\), 于是\(\vert\boldsymbol{A}\vert = (-1)^{t} = -1\)当且仅当\(n-\mathrm{r}(\boldsymbol{A}+\boldsymbol{I}_{n}) = t\)为奇数.

\end{proof}



\end{document}
\documentclass[../../main.tex]{subfiles}
\graphicspath{{\subfix{../../image/}}} % 指定图片目录,后续可以直接使用图片文件名。

% 例如:
% \begin{figure}[h]
% \centering
% \includegraphics{image-01.01}
% \label{fig:image-01.01}
% \caption{图片标题}
% \end{figure}

\begin{document}

\section{Vandermode行列式相关}

本节我们用\(V_n(x_1,x_2,\cdots,x_n)\)表示\(n\)阶Vandermonde行列式.
\begin{definition}
对\(1\leqslant i\leqslant n,V_n^{(i)}(x_1,x_2,\cdots,x_n)\)表示删除\(V_n(x_1,x_2,\cdots,x_n)\)的第\(i\)行\((x_1^{i - 1},x_2^{i - 1},\cdots,x_n^{i - 1})\)之后新添第\(n\)行\((x_1^{n},x_2^{n},\cdots,x_n^{n})\)所得\(n\)阶行列式.
\end{definition}

\begin{definition}
\(\Delta_n(x_1,x_2,\cdots,x_n)\)表示将\(V_n(x_1,x_2,\cdots,x_n)\)的第\(n\)行换成\((x_1^{n + 1},x_2^{n + 1},\cdots,x_n^{n + 1})\)所得\(n\)阶行列式.
\end{definition}


\begin{example}
设初等对称多项式
\begin{align}\label{23.137}
\sigma_j=\sum_{1\leqslant k_1<k_2<\cdots<k_j\leqslant n}x_{k_1}x_{k_2}\cdots x_{k_j},j = 1,2,\cdots,n,
\end{align}
我们有
\begin{align}
V_n^{(i)}(x_1,x_2,\cdots,x_n)=\sigma_{n - i + 1}V_n(x_1,x_2,\cdots,x_n),i = 1,2,\cdots,n.\label{23.138}
\end{align}
\end{example}
\begin{proof}
(加边法)不妨设\(x_i,1\leqslant i\leqslant n\)互不相同. 设
\begin{align*}
D_n(x)\triangleq\begin{vmatrix}
1&1&1&\cdots&1\\
-x&x_1&x_2&\cdots&x_n\\
(-x)^2&x_1^2&x_2^2&\cdots&x_n^2\\
\vdots&\vdots&\vdots&\ddots&\vdots\\
(-x)^n&x_1^n&x_2^n&\cdots&x_n^n
\end{vmatrix}.
\end{align*}
由行列式性质我们知道\(D_n\)是\(n\)次多项式且有\(n\)个根\(-x_1,-x_2,\cdots,-x_n\). 于是我们有
\begin{align}\label{eqeqeq1}
D_n(x)=c(x + x_1)(x + x_2)\cdots(x + x_n).
\end{align}
把\(D_n(x)\)按第一列展开得
\begin{align}\label{eqeqeq2}
D_n(x)=\sum_{i = 1}^{n}V_n^{(i)}(x_1,x_2,\cdots,x_n)x^{i - 1}+V_n(x_1,x_2,\cdots,x_n)x^n.
\end{align}
于是比较\eqref{eqeqeq1}式和\eqref{eqeqeq2}式最高次项系数,我们有\(c = V_n(x_1,x_2,\cdots,x_n)\). 定义\(\sigma_0 = 1\),利用根和系数的关系(Vieta定理),结合\eqref{eqeqeq1}式和\eqref{eqeqeq2}式得
\begin{align*}
D_n(x)=\sum_{i = 1}^{n + 1}\sigma_{n - i + 1}V_n(x_1,x_2,\cdots,x_n)x^{i - 1}=\sum_{i = 1}^{n}V_n^{(i)}(x_1,x_2,\cdots,x_n)x^{i - 1}+V_n(x_1,x_2,\cdots,x_n)x^n,
\end{align*}
比较上式等号两边$x^i(1\leq i\leq n)$的系数就能得到\eqref{23.138}. 
\end{proof}

\begin{example}
证明:
\begin{align}
\Delta_n(x_1,x_2,\cdots,x_n)=\left(\sum_{k = 1}^n x_k^2+\sum_{1\leq i<j\leq n}x_ix_j\right)V_n(x_1,x_2,\cdots,x_n) \label{23.139}
\end{align} 
\end{example}
\begin{proof}
不妨设\(x_i,1\leq i\leq n\)互不相同。设\(n + 1\)次多项式
\begin{align*}
P_{n + 1}(x)\triangleq\begin{vmatrix}
1&1&1&\cdots&1\\
-x&x_1&x_2&\cdots&x_n\\
(-x)^2&x_1^2&x_2^2&\cdots&x_n^2\\
\vdots&\vdots&\vdots&\ddots&\vdots\\
(-x)^{n - 1}&x_1^{n - 1}&x_2^{n - 1}&\cdots&x_n^{n - 1}\\
(-x)^{n + 1}&x_1^{n + 1}&x_2^{n + 1}&\cdots&x_n^{n + 1}
\end{vmatrix}
\end{align*}
注意到有\(n\)个根\(-x_1,-x_2,\cdots,-x_n\)。我们用\(-x_{n + 1}\)表示\(P_{n + 1}\)第\(n + 1\)个根。于是我们有
\begin{align}
P_{n + 1}(x)=c(x + x_1)(x + x_2)\cdots(x + x_n)(x + x_{n + 1}).\label{23.111}
\end{align}
将\(P_{n + 1}(x)\)按第一列展开得
\begin{align}\label{23.140}
P_{n + 1}(x)=-V_n(x_1,x_2,\cdots,x_n)x^{n + 1}+\Delta_n(x_1,x_2,\cdots,x_n)x^{n - 1}+a_{n - 2}x^{n - 2}+\cdots+a_0
\end{align}
其中\(a_{n - 2},\cdots,a_0\)是某些与\(x_j\)有关的\(n\)阶行列式。
比较\eqref{23.111}和\eqref{23.140}式的系数可知\(c = -V_n(x_1,x_2,\cdots,x_n)\).于是结合\eqref{23.111}式,并利用Vieta定理得
\begin{align}\label{23.141}
P_{n + 1}(x)=-V_n(x_1,x_2,\cdots,x_n)(x^{n + 1}+\delta_1x^n+\delta_2x^{n - 1}+\cdots+\delta_{n - 1})
\end{align}
这里\(\delta_j\)类似\eqref{23.137}式定义是\(x_1,x_2,\cdots,x_n,x_{n + 1}\)的初等对称多项式。
比较\eqref{23.140}\eqref{23.141}式的\(x^{n - 1}\)系数可得\(\Delta_n(x_1,x_2,\cdots,x_n)=-\delta_2V_n(x_1,x_2,\cdots,x_n)\)。
因为\(P_{n + 1}(x)\)没有\(x^n\)的项,所以
\begin{align*}
\delta _1=x_1+x_2+\cdots +x_{n+1}=0\Rightarrow x_{n+1}=-\left( x_1+x_2+\cdots +x_n \right) .
\end{align*}
从而
\begin{align*}
\delta _2&=\sum_{1\le i<j\le n+1}{x_ix_j}=\sum_{1\le i<j\le n}{x_ix_j}+x_{n+1}\sum_{i=1}^n{x_i}
\\
&=\sum_{1\le i<j\le n}{x_ix_j}-\left( x_1+x_2+\cdots +x_n \right) \sum_{i=1}^n{x_i}
\\
&=\sum_{1\le i<j\le n}{x_ix_j}-\left( \sum_{i=1}^n{x_i} \right) ^2=-\sum_{i=1}^n{x_{i}^{2}}-\sum_{1\le i<j\le n}{x_ix_j}.
\end{align*}
现在就有\eqref{23.139}成立。 
\end{proof}

\begin{proposition}\label{proposition:将矩阵拆分成Vandermode矩阵的形式}
设\(A = (a_{ij})_{n\times n},f_i(x)=a_{i1}+a_{i2}x+\cdots+a_{in}x^{n - 1}(i = 1,2,\cdots,n)\),证明:对任何复数\(x_1,x_2,\cdots,x_n\),都有
\begin{align*}
\begin{vmatrix}
f_1(x_1)&f_1(x_2)&\cdots&f_1(x_n)\\
f_2(x_1)&f_2(x_2)&\cdots&f_2(x_n)\\
\vdots&\vdots&\ddots&\vdots\\
f_n(x_1)&f_n(x_2)&\cdots&f_n(x_n)
\end{vmatrix}=|A|\cdot V_n(x_1,x_2,\cdots,x_n)
\end{align*}
这里\(V_n(x_1,x_2,\cdots,x_n)\)表示\(x_1,x_2,\cdots,x_n\)的Vandermonde行列式。
\end{proposition}
\begin{note}
关键是利用\hyperref[proposition:一些能写成两个向量乘积的矩阵]{命题\ref{proposition:一些能写成两个向量乘积的矩阵}}.
\end{note}
\begin{proof}
直接由矩阵乘法观察知显然。 
\end{proof}

\begin{example}
计算
\begin{align*}
\begin{vmatrix}
1&1&1&\cdots&1\\
x_1 + 1&x_2 + 1&x_3 + 1&\cdots&x_n + 1\\
x_1^2 + x_1&x_2^2 + x_2&x_3^2 + x_3&\cdots&x_n^2 + x_n\\
\vdots&\vdots&\vdots&\ddots&\vdots\\
x_1^{n - 1} + x_1^{n - 2}&x_2^{n - 1} + x_2^{n - 2}&x_3^{n - 1} + x_3^{n - 2}&\cdots&x_n^{n - 1} + x_n^{n - 2}
\end{vmatrix}
\end{align*}
\end{example}
\begin{proof}
由\hyperref[proposition:将矩阵拆分成Vandermode矩阵的形式]{命题\ref{proposition:将矩阵拆分成Vandermode矩阵的形式}}我们知道
\begin{align*}
&\left| \begin{matrix}
1&		1&		1&		\cdots&		1\\
x_1+1&		x_2+1&		x_3+1&		\cdots&		x_n+1\\
x_{1}^{2}+x_1&		x_{2}^{2}+x_2&		x_{3}^{2}+x_3&		\cdots&		x_{n}^{2}+x_n\\
\vdots&		\vdots&		\vdots&		\ddots&		\vdots\\
x_{1}^{n-1}+x_{1}^{n-2}&		x_{2}^{n-1}+x_{2}^{n-2}&		x_{3}^{n-1}+x_{3}^{n-2}&		\cdots&		x_{n}^{n-1}+x_{n}^{n-2}\\
\end{matrix} \right|=V_n\left( x_1,x_2,\cdots ,x_n \right) \cdot \left| \begin{matrix}
1&		0&		0&		\cdots&		0&		0&		0\\
1&		1&		0&		\cdots&		0&		0&		0\\
0&		1&		1&		\cdots&		0&		0&		0\\
\vdots&		\vdots&		\vdots&		\ddots&		\vdots&		\vdots&		\vdots\\
0&		0&		0&		\cdots&		1&		1&		0\\
0&		0&		0&		\cdots&		0&		1&		1\\
\end{matrix} \right|
\\
&=\prod_{1\le i<j\le n}{(x_j}-x_i)\cdot \left| \begin{matrix}
1&		0&		0&		\cdots&		0&		0&		0\\
1&		1&		0&		\cdots&		0&		0&		0\\
0&		1&		1&		\cdots&		0&		0&		0\\
\vdots&		\vdots&		\vdots&		\ddots&		\vdots&		\vdots&		\vdots\\
0&		0&		0&		\cdots&		1&		1&		0\\
0&		0&		0&		\cdots&		0&		1&		1\\
\end{matrix} \right|=\prod_{1\le i<j\le n}{\left( x_j-x_i \right)}.
\end{align*} 
\end{proof}













\end{document}
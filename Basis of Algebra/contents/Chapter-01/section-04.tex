\documentclass[../../main.tex]{subfiles}
\graphicspath{{\subfix{../../image/}}} % 指定图片目录,后续可以直接使用图片文件名。

% 例如:
% \begin{figure}[h]
% \centering
% \includegraphics{image-01.01}
% \label{fig:image-01.01}
% \caption{图片标题}
% \end{figure}

\begin{document}

\section{Vandermode行列式相关}

本节我们用\(V_n(x_1,x_2,\cdots,x_n)\)表示\(n\)阶Vandermonde行列式.

对\(1\leqslant i\leqslant n,V_n^{(i)}(x_1,x_2,\cdots,x_n)\)表示删除\(V_n(x_1,x_2,\cdots,x_n)\)的第\(i\)行\((x_1^{i - 1},x_2^{i - 1},\cdots,x_n^{i - 1})\)之后新添第\(n\)行\((x_1^{n},x_2^{n},\cdots,x_n^{n})\)所得\(n\)阶行列式.

\(\Delta_n(x_1,x_2,\cdots,x_n)\)表示将\(V_n(x_1,x_2,\cdots,x_n)\)的第\(n\)行换成\((x_1^{n + 1},x_2^{n + 1},\cdots,x_n^{n + 1})\)所得\(n\)阶行列式.

\begin{proposition}\label{proposition:初等对称多项式与Vandermode行列式}
设初等对称多项式
\begin{align*}
\sigma_j=\sum_{1\leqslant k_1<k_2<\cdots<k_j\leqslant n}x_{k_1}x_{k_2}\cdots x_{k_j},j = 1,2,\cdots,n,
\end{align*}
我们有
\begin{align}
V_n^{(i)}(x_1,x_2,\cdots,x_n)=\sigma_{n - i + 1}V_n(x_1,x_2,\cdots,x_n),i = 1,2,\cdots,n.\label{23.138}
\end{align}
\end{proposition}
\begin{proof}
(加边法)不妨设\(x_i,1\leqslant i\leqslant n\)互不相同. 设
\begin{align*}
D_n(x)\triangleq\begin{vmatrix}
1&1&1&\cdots&1\\
-x&x_1&x_2&\cdots&x_n\\
(-x)^2&x_1^2&x_2^2&\cdots&x_n^2\\
\vdots&\vdots&\vdots&\ddots&\vdots\\
(-x)^n&x_1^n&x_2^n&\cdots&x_n^n
\end{vmatrix}.
\end{align*}
由行列式性质我们知道\(D_n\)是\(n\)次多项式且有\(n\)个根\(-x_1,-x_2,\cdots,-x_n\). 于是我们有
\begin{align}\label{eqeqeq1}
D_n(x)=c(x + x_1)(x + x_2)\cdots(x + x_n).
\end{align}
把\(D_n(x)\)按第一列展开得
\begin{align}\label{eqeqeq2}
D_n(x)=\sum_{i = 1}^{n}V_n^{(i)}(x_1,x_2,\cdots,x_n)x^{i - 1}+V_n(x_1,x_2,\cdots,x_n)x^n.
\end{align}
于是比较\eqref{eqeqeq1}式和\eqref{eqeqeq2}式最高次项系数,我们有\(c = V_n(x_1,x_2,\cdots,x_n)\). 定义\(\sigma_0 = 1\),利用根和系数的关系(Vieta定理),结合\eqref{eqeqeq1}式和\eqref{eqeqeq2}式得
\begin{align*}
D_n(x)=\sum_{i = 1}^{n + 1}\sigma_{n - i + 1}V_n(x_1,x_2,\cdots,x_n)x^{i - 1}=\sum_{i = 1}^{n}V_n^{(i)}(x_1,x_2,\cdots,x_n)x^{i - 1}+V_n(x_1,x_2,\cdots,x_n)x^n,
\end{align*}
比较上式等号两边$x^i(1\leq i\leq n)$的系数就能得到\eqref{23.138}. 
\end{proof}





\end{document}
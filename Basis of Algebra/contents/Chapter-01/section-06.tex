\documentclass[../../main.tex]{subfiles}
\graphicspath{{\subfix{../../image/}}} % 指定图片目录,后续可以直接使用图片文件名。

% 例如:
% \begin{figure}[h]
% \centering
% \includegraphics{image-01.01}
% \label{fig:image-01.01}
% \caption{图片标题}
% \end{figure}

\begin{document}

\section{求和法}

\begin{example}
设$x_1,x_2,x_3$是方程$x^3+px+q=0$的3个根,求下列行列式的值:
\begin{gather}
\left| \boldsymbol{A} \right|=\left| \begin{matrix}
x_1&		x_2&		x_3\\
x_2&		x_3&		x_1\\
x_3&		x_1&		x_2\\
\end{matrix} \right|.
\nonumber
\end{gather}
\end{example}
\begin{solution}
由$Vieta$定理可知,$x_1+x_2+x_3=0$.因此,我们有
\begin{equation}
\begin{split}
\left| \boldsymbol{A} \right|=\left| \begin{matrix}
x_1&		x_2&		x_3\\
x_2&		x_3&		x_1\\
x_3&		x_1&		x_2\\
\end{matrix} \right|\xlongequal[i=2,3]{r_i+r_1}\left| \begin{matrix}
0&		0&		0\\
x_2&		x_3&		x_1\\
x_3&		x_1&		x_2\\
\end{matrix} \right|=0.
\end{split}
\nonumber
\end{equation}
\end{solution}

\begin{example}
设$b_{ij}=\left( a_{i1}+a_{i2}+\cdots +a_{in} \right) -a_{ij}$,求证:
\begin{equation}
\left| \begin{matrix}
b_{11}&		\cdots&		b_{1n}\\
\vdots&		&		\vdots\\
b_{n1}&		\cdots&		b_{nn}\\
\end{matrix} \right|=(-1)^{n-1}(n-1)\left| \begin{matrix}
a_{11}&		\cdots&		a_{1n}\\
\vdots&		&		\vdots\\
a_{n1}&		\cdots&		a_{nn}\\
\end{matrix} \right|.
\nonumber
\end{equation}
\begin{solution}
\begin{align*}
&\left| \begin{matrix}
b_{11}&		b_{12}&		\cdots&		b_{1n}\\
b_{21}&		b_{22}&		\cdots&		b_{2n}\\
\vdots&		\vdots&		&		\vdots\\
b_{n1}&		b_{n2}&		\cdots&		b_{nn}\\
\end{matrix} \right|=\left| \begin{matrix}
\left( a_{11}+a_{12}+\cdots +a_{1n} \right) -a_{11}&		\left( a_{11}+a_{12}+\cdots +a_{1n} \right) -a_{12}&		\cdots&		\left( a_{11}+a_{12}+\cdots +a_{1n} \right) -a_{1n}\\
\left( a_{21}+a_{22}+\cdots +a_{2n} \right) -a_{21}&		\left( a_{21}+a_{22}+\cdots +a_{2n} \right) -a_{22}&		\cdots&		\left( a_{21}+a_{22}+\cdots +a_{2n} \right) -a_{2n}\\
\vdots&		\vdots&		&		\vdots\\
\left( a_{n1}+a_{n2}+\cdots +a_{nn} \right) -a_{n1}&		\left( a_{n1}+a_{n2}+\cdots +a_{nn} \right) -a_{n2}&		\cdots&		\left( a_{n1}+a_{n2}+\cdots +a_{nn} \right) -a_{nn}\\
\end{matrix} \right|
\\
&\xlongequal[i=2,\cdots ,n]{j_i+j_1}\left| \begin{matrix}
\left( n-1 \right) \left( a_{11}+a_{12}+\cdots +a_{1n} \right)&		\left( a_{11}+a_{12}+\cdots +a_{1n} \right) -a_{12}&		\cdots&		\left( a_{11}+a_{12}+\cdots +a_{1n} \right) -a_{1n}\\
\left( n-1 \right) \left( a_{21}+a_{22}+\cdots +a_{2n} \right)&		\left( a_{21}+a_{22}+\cdots +a_{2n} \right) -a_{22}&		\cdots&		\left( a_{21}+a_{22}+\cdots +a_{2n} \right) -a_{2n}\\
\vdots&		\vdots&		&		\vdots\\
\left( n-1 \right) \left( a_{n1}+a_{n2}+\cdots +a_{nn} \right)&		\left( a_{n1}+a_{n2}+\cdots +a_{nn} \right) -a_{n2}&		\cdots&		\left( a_{n1}+a_{n2}+\cdots +a_{nn} \right) -a_{nn}\\
\end{matrix} \right|
\\
&=\left( n-1 \right) \left| \begin{matrix}
\left( a_{11}+a_{12}+\cdots +a_{1n} \right)&		\left( a_{11}+a_{12}+\cdots +a_{1n} \right) -a_{12}&		\cdots&		\left( a_{11}+a_{12}+\cdots +a_{1n} \right) -a_{1n}\\
\left( a_{21}+a_{22}+\cdots +a_{2n} \right)&		\left( a_{21}+a_{22}+\cdots +a_{2n} \right) -a_{22}&		\cdots&		\left( a_{21}+a_{22}+\cdots +a_{2n} \right) -a_{2n}\\
\vdots&		\vdots&		&		\vdots\\
\left( a_{n1}+a_{n2}+\cdots +a_{nn} \right)&		\left( a_{n1}+a_{n2}+\cdots +a_{nn} \right) -a_{n2}&		\cdots&		\left( a_{n1}+a_{n2}+\cdots +a_{nn} \right) -a_{nn}\\
\end{matrix} \right|
\\
&\xlongequal[i=2,\cdots ,n]{\left( -1 \right) j_1+j_i}\left( n-1 \right) \left| \begin{matrix}
\left( a_{11}+a_{12}+\cdots +a_{1n} \right)&		-a_{12}&		\cdots&		-a_{1n}\\
\left( a_{21}+a_{22}+\cdots +a_{2n} \right)&		-a_{22}&		\cdots&		-a_{2n}\\
\vdots&		\vdots&		&		\vdots\\
\left( a_{n1}+a_{n2}+\cdots +a_{nn} \right)&		-a_{n2}&		\cdots&		-a_{nn}\\
\end{matrix} \right|
\\
&\xlongequal[i=2,\cdots ,n]{j_i+j_1}\left( n-1 \right) \left| \begin{matrix}
a_{11}&		-a_{12}&		\cdots&		-a_{1n}\\
a_{21}&		-a_{22}&		\cdots&		-a_{2n}\\
\vdots&		\vdots&		&		\vdots\\
a_{n1}&		-a_{n2}&		\cdots&		-a_{nn}\\
\end{matrix} \right|
\\
&=(-1)^{n-1}(n-1)\left| \begin{matrix}
a_{11}&		-a_{12}&		\cdots&		-a_{1n}\\
a_{21}&		-a_{22}&		\cdots&		-a_{2n}\\
\vdots&		\vdots&		&		\vdots\\
a_{n1}&		-a_{n2}&		\cdots&		-a_{nn}\\
\end{matrix} \right|.
\end{align*}
\end{solution}
\end{example}
\begin{conclusion}\label{行列式计算:求和法}
第二个等号是行列式计算中的一个常用方法\hypertarget{行列式计算:求和法}{\textbf{求和法}}:

将除第一列外的其余列全部加到第一列上(或将除第一行外的其余行全部加到第一行上),
使第一列(或列)一样或者具有相同形式.
然后根据具体情况将第一列(或行)的倍数加到其余列(或行)上,
从而将行列式化为我们熟悉的形式.

应用该方法的一般情形:

1.行列式每行(或列)和相等时;

2.行列式每行(或列)和有一定规律时.

\end{conclusion}

\begin{example}
计算$n$阶行列式:
\begin{equation}
|\boldsymbol{A}|=\left| \begin{matrix}
0&		1&		\cdots&		1&		1\\
1&		0&		\cdots&		1&		1\\
\vdots&		\vdots&		&		\vdots&		\vdots\\
1&		1&		\cdots&		0&		1\\
1&		1&		\cdots&		1&		0\\
\end{matrix} \right|.
\nonumber
\end{equation}
\begin{solution}
\begin{equation}
\begin{split}
&|\boldsymbol{A}|=\left| \begin{matrix}
0&		1&		\cdots&		1&		1\\
1&		0&		\cdots&		1&		1\\
\vdots&		\vdots&		&		\vdots&		\vdots\\
1&		1&		\cdots&		0&		1\\
1&		1&		\cdots&		1&		0\\
\end{matrix} \right|\xlongequal[i=2,\cdots ,n]{j_i+j_1}\left| \begin{matrix}
n-1&		1&		\cdots&		1&		1\\
n-1&		0&		\cdots&		1&		1\\
\vdots&		\vdots&		&		\vdots&		\vdots\\
n-1&		1&		\cdots&		0&		1\\
n-1&		1&		\cdots&		1&		0\\
\end{matrix} \right|=\left( n-1 \right) \left| \begin{matrix}
1&		1&		\cdots&		1&		1\\
1&		0&		\cdots&		1&		1\\
\vdots&		\vdots&		&		\vdots&		\vdots\\
1&		1&		\cdots&		0&		1\\
1&		1&		\cdots&		1&		0\\
\end{matrix} \right|
\\
&\xlongequal[i=2,\cdots ,n]{\left( -1 \right) r_1+r_i}\left( n-1 \right) \left| \begin{matrix}
1&		1&		\cdots&		1&		1\\
0&		-1&		\cdots&		0&		0\\
\vdots&		\vdots&		&		\vdots&		\vdots\\
0&		0&		\cdots&		-1&		0\\
0&		0&		\cdots&		0&		-1\\
\end{matrix} \right|=\left( -1 \right) ^{n-1}\left( n-1 \right) .
\end{split}
\nonumber
\end{equation}
\end{solution}
\begin{remark}
因为$|\boldsymbol{A}|$除对角元素外,每行都一样,
所以本题也可以看成命题\ref{"爪"型行列式的推广}的应用,利用命题\ref{"爪"型行列式的推广}的计算方法直接得到结果.
\begin{equation}
|\boldsymbol{A}|=\left| \begin{matrix}
0&		1&		\cdots&		1&		1\\
1&		0&		\cdots&		1&		1\\
\vdots&		\vdots&		&		\vdots&		\vdots\\
1&		1&		\cdots&		0&		1\\
1&		1&		\cdots&		1&		0\\
\end{matrix} \right|\xlongequal[i=2,\cdots ,n]{\left( -1 \right) r_1+r_i}\left| \begin{matrix}
0&		1&		\cdots&		1&		1\\
1&		-1&		\cdots&		0&		0\\
\vdots&		\vdots&		&		\vdots&		\vdots\\
1&		0&		\cdots&		-1&		0\\
1&		0&		\cdots&		0&		-1\\
\end{matrix} \right|\xlongequal[]{\text{命题}1.2}-\sum_{i=2}^n{\left( -1 \right) ^{n-2}}=\left( -1 \right) ^{n-1}\left( n-1 \right) .            
\nonumber
\end{equation}
\end{remark}
\end{example}

\begin{example}
计算$n$阶行列式:
\begin{equation}
\begin{split}
|\boldsymbol{A}|=\left| \begin{matrix}
a_1+b&		a_2&		a_3&		\cdots&		a_n\\
a_1&		a_2+b&		a_3&		\cdots&		a_n\\
a_1&		a_2&		a_3+b&		\cdots&		a_n\\
\vdots&		\vdots&		\vdots&		&		\vdots\\
a_1&		a_2&		a_3&		\cdots&		a_n+b\\
\end{matrix} \right|.
\end{split}
\nonumber
\end{equation}
\begin{note}
既可以将$|\boldsymbol{A}|$看作命题\ref{"爪"型行列式的推广}的应用,
利用命题\ref{"爪"型行列式的推广}的计算方法直接得到结果.即下述解法一.

也可以利用\hyperlink{行列式计算:求和法}{求和法}将$|\boldsymbol{A}|$
化为上三角形行列式.即下述解法二.
\end{note}
\begin{solution}
{\color{blue} \text{解法一}}:
\begin{equation}
\begin{split}
&|\boldsymbol{A}|=\left| \begin{matrix}
a_1+b&		a_2&		a_3&		\cdots&		a_n\\
a_1&		a_2+b&		a_3&		\cdots&		a_n\\
a_1&		a_2&		a_3+b&		\cdots&		a_n\\
\vdots&		\vdots&		\vdots&		&		\vdots\\
a_1&		a_2&		a_3&		\cdots&		a_n+b\\
\end{matrix} \right|
\xlongequal[i=2,\cdots ,n]{-r_1+r_i}\left| \begin{matrix}
a_1+b&		a_2&		a_3&		\cdots&		a_n\\
-b&		b&		0&		\cdots&		0\\
-b&		0&		b&		\cdots&		0\\
\vdots&		\vdots&		\vdots&		&		\vdots\\
-b&		0&		0&		\cdots&		b\\
\end{matrix} \right|
\\
&\xlongequal[]{\text{命题}\ref{"爪"型行列式}}\left( a_1+b \right) b^{n-1}-\sum_{i=2}^n{b^{n-2}a_i\left( -b \right)}
=b^{n-1}\left[ \left( a_1+b \right) +\sum_{i=2}^n{a_i} \right] 
=\left(( b+\sum_{i=1}^n{a_i} \right) b^{n-1}.
\end{split}
\nonumber
\end{equation}
{\color{blue} \text{解法二}}:
\begin{equation}
\begin{split}
&|\boldsymbol{A}|=\left| \begin{matrix}
a_1+b&		a_2&		a_3&		\cdots&		a_n\\
a_1&		a_2+b&		a_3&		\cdots&		a_n\\
a_1&		a_2&		a_3+b&		\cdots&		a_n\\
\vdots&		\vdots&		\vdots&		&		\vdots\\
a_1&		a_2&		a_3&		\cdots&		a_n+b\\
\end{matrix} \right|
\xlongequal[i=2,\cdots ,n]{j_i+j_1}(b+\sum_{i=1}^n{a_i)\left| \begin{matrix}
1&		a_2&		a_3&		\cdots&		a_n\\
1&		a_2+b&		a_3&		\cdots&		a_n\\
1&		a_2&		a_3+b&		\cdots&		a_n\\
\vdots&		\vdots&		\vdots&		&		\vdots\\
1&		a_2&		a_3&		\cdots&		a_n+b\\
\end{matrix} \right|}
\\
&\xlongequal[i=2,\cdots ,n]{-a_i\cdot j_1+j_i}(b+\sum_{i=1}^n{a_i)\left| \begin{matrix}
1&		0&		0&		\cdots&		0\\
1&		b&		0&		\cdots&		0\\
1&		0&		b&		\cdots&		0\\
\vdots&		\vdots&		\vdots&		&		\vdots\\
1&		0&		0&		\cdots&		b\\
\end{matrix} \right|}
=(b+\sum_{i=1}^n{a_i)b^{n-1}}.
\end{split}
\nonumber
\end{equation}
\end{solution}
\end{example}

\begin{example}
计算$n$阶行列式:
\begin{equation}
\begin{split}
|\boldsymbol{A}|=\left| \begin{matrix}
1&		2&		3&		\cdots&		n-1&		n\\
n&		1&		2&		\cdots&		n-2&		n-1\\
n-1&		n&		1&		\cdots&		n-3&		n-2\\
\vdots&		\vdots&		\vdots&		&		\vdots&		\vdots\\
3&		4&		5&		\cdots&		1&		2\\
2&		3&		4&		\cdots&		n&		1\\
\end{matrix}\right|.
\end{split}
\nonumber
\end{equation}
\end{example}
\begin{note}
\hyperlink{行列式计算:求和法}{求和法}的经典应用.
\end{note}
\begin{solution}
\begin{equation}
\begin{split}
&|\boldsymbol{A}|=\left| \begin{matrix}
1&		2&		3&		\cdots&		n-1&		n\\
n&		1&		2&		\cdots&		n-2&		n-1\\
n-1&		n&		1&		\cdots&		n-3&		n-2\\
\vdots&		\vdots&		\vdots&		&		\vdots&		\vdots\\
3&		4&		5&		\cdots&		1&		2\\
2&		3&		4&		\cdots&		n&		1\\
\end{matrix} \right|\xlongequal[i=2,\cdots ,n]{j_i+j_1}\frac{n\left( n+1 \right)}{2}\left| \begin{matrix}
1&		2&		3&		\cdots&		n-1&		n\\
1&		1&		2&		\cdots&		n-2&		n-1\\
1&		n&		1&		\cdots&		n-3&		n-2\\
\vdots&		\vdots&		\vdots&		&		\vdots&		\vdots\\
1&		4&		5&		\cdots&		1&		2\\
1&		3&		4&		\cdots&		n&		1\\
\end{matrix} \right|
\\
&\xlongequal[i=2,\cdots ,n]{-r_1+r_i}\frac{n\left( n+1 \right)}{2}\left| \begin{matrix}
1&		2&		3&		\cdots&		n-1&		n\\
0&		-1&		-1&		\cdots&		-1&		-1\\
0&		n-2&		-2&		\cdots&		-2&		-2\\
\vdots&		\vdots&		\vdots&		&		\vdots&		\vdots\\
0&		2&		2&		\cdots&		2-n&		2-n\\
0&		1&		1&		\cdots&		1&		1-n\\
\end{matrix} \right|\xlongequal[]{\text{按第一列展开}}\frac{n\left( n+1 \right)}{2}\left| \begin{matrix}
-1&		-1&		\cdots&		-1&		-1\\
n-2&		-2&		\cdots&		-2&		-2\\
\vdots&		\vdots&		&		\vdots&		\vdots\\
2&		2&		\cdots&		2-n&		2-n\\
1&		1&		\cdots&		1&		1-n\\
\end{matrix} \right|
\\
&\xlongequal[i=2,\cdots ,n]{-j_1+j_i}\frac{n\left( n+1 \right)}{2}\left| \begin{matrix}
-1&		0&		\cdots&		0&		0\\
n-2&		-n&		\cdots&		-n&		-n\\
\vdots&		\vdots&		&		\vdots&		\vdots\\
2&		0&		\cdots&		-n&		-n\\
1&		0&		\cdots&		0&		-n\\
\end{matrix} \right|\xlongequal[]{\text{按第一行展开}}-\frac{n\left( n+1 \right)}{2}\left| \begin{matrix}
-n&		\cdots&		-n&		-n\\
\vdots&		&		\vdots&		\vdots\\
0&		\cdots&		-n&		-n\\
0&		\cdots&		0&		-n\\
\end{matrix} \right|
\\
&=-\frac{n\left( n+1 \right)}{2}\left( -n \right) ^{n-2}=\left( -1 \right) ^{n-1}\frac{n+1}{2}n^{n-1}.            
\end{split}
\nonumber
\end{equation}
\end{solution}

















\end{document}
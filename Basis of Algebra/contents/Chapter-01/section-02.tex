% contents/Chapter-01/section-02.tex 第一章第二节
\documentclass[../../main.tex]{subfiles}
\graphicspath{{\subfix{../../image/}}} % 指定图片目录,后续可以直接使用图片文件名。

% 例如:
% \begin{figure}[h]
% \centering
% \includegraphics{image-01.01}
% \label{fig:image-01.01}
% \caption{图片标题}
% \end{figure}

\begin{document}

\section{练习}

\begin{exercise}
计算$n$阶行列式:
\begin{equation}
|\boldsymbol{A}|=\left| \begin{matrix}
1&		1&		\cdots&		1\\
1&		\mathrm{C}_{2}^{1}&		\cdots&		\mathrm{C}_{n}^{1}\\
1&		\mathrm{C}_{3}^{2}&		\cdots&		\mathrm{C}_{n+1}^{2}\\
\vdots&		\vdots&		&		\vdots\\
1&		\mathrm{C}_{n}^{n-1}&		\cdots&		\mathrm{C}_{2n-2}^{n-1}\\
\end{matrix} \right|.
\nonumber
\end{equation}
\begin{note}
组合数公式:
$\mathrm{C}_{m}^{k-1}+\mathrm{C}_{m}^{k}=\mathrm{C}_{m+1}^{k}$.

于是有
\begin{gather}
\mathrm{C}_{m}^{k}=\mathrm{C}_{m+1}^{k}-\mathrm{C}_{m}^{k-1}
\nonumber
\\
\mathrm{C}_{m}^{k-1}=\mathrm{C}_{m+1}^{k}-\mathrm{C}_{m}^{k}
\nonumber
\end{gather}
\end{note}
\begin{solution}
\begin{equation}
\begin{split}
|\boldsymbol{A}|&=\left| \begin{matrix}
1&		1&		\cdots&		1\\
1&		\mathrm{C}_{2}^{1}&		\cdots&		\mathrm{C}_{n}^{1}\\
1&		\mathrm{C}_{3}^{2}&		\cdots&		\mathrm{C}_{n+1}^{2}\\
\vdots&		\vdots&		&		\vdots\\
1&		\mathrm{C}_{n}^{n-1}&		\cdots&		\mathrm{C}_{2n-2}^{n-1}\\
\end{matrix} \right|\xlongequal[i=n,\cdots ,2]{\left( -1 \right) \cdot r_{i-1}+r_i}\left| \begin{matrix}
\mathrm{C}_{0}^{0}&		\mathrm{C}_{1}^{0}&		\cdots&		\mathrm{C}_{\mathrm{n}-1}^{0}\\
0&		\mathrm{C}_{2}^{1}-\mathrm{C}_{1}^{0}&		\cdots&		\mathrm{C}_{n}^{1}-\mathrm{C}_{\mathrm{n}-1}^{0}\\
0&		\mathrm{C}_{3}^{2}-\mathrm{C}_{2}^{1}&		\cdots&		\mathrm{C}_{n+1}^{2}-\mathrm{C}_{n}^{1}\\
\vdots&		\vdots&		&		\vdots\\
0&		\mathrm{C}_{n}^{n-1}-\mathrm{C}_{\mathrm{n}-1}^{\mathrm{n}-2}&		\cdots&		\mathrm{C}_{2n-2}^{n-1}-\mathrm{C}_{2\mathrm{n}-3}^{\mathrm{n}-2}\\
\end{matrix} \right|
\\
&=\left| \begin{matrix}
\mathrm{C}_{0}^{0}&		\mathrm{C}_{1}^{0}&		\cdots&		\mathrm{C}_{\mathrm{n}-1}^{0}\\
0&		\mathrm{C}_{1}^{1}&		\cdots&		\mathrm{C}_{\mathrm{n}-1}^{1}\\
0&		\mathrm{C}_{2}^{2}&		\cdots&		\mathrm{C}_{n}^{2}\\
\vdots&		\vdots&		&		\vdots\\
0&		\mathrm{C}_{\mathrm{n}-1}^{\mathrm{n}-1}&		\cdots&		\mathrm{C}_{2\mathrm{n}-3}^{\mathrm{n}-1}\\
\end{matrix} \right|\xlongequal{\text{按第一列展开}}\left| \begin{matrix}
\mathrm{C}_{1}^{1}&		\mathrm{C}_{2}^{1}&		\cdots&		\mathrm{C}_{\mathrm{n}-1}^{1}\\
\mathrm{C}_{2}^{2}&		\mathrm{C}_{3}^{2}&		\cdots&		\mathrm{C}_{n}^{2}\\
\vdots&		\vdots&		&		\vdots\\
\mathrm{C}_{\mathrm{n}-1}^{\mathrm{n}-1}&		\mathrm{C}_{\mathrm{n}}^{\mathrm{n}-1}&		\cdots&		\mathrm{C}_{2\mathrm{n}-3}^{\mathrm{n}-1}\\
\end{matrix} \right|
\\
&\xlongequal[i=n,\cdots ,2]{\left( -1 \right) \cdot j_{i-1}+j_i}\left| \begin{matrix}
\mathrm{C}_{1}^{1}&		\mathrm{C}_{2}^{1}-\mathrm{C}_{1}^{1}&		\cdots&		\mathrm{C}_{\mathrm{n}-1}^{1}-\mathrm{C}_{\mathrm{n}-2}^{1}\\
\mathrm{C}_{2}^{2}&		\mathrm{C}_{3}^{2}-\mathrm{C}_{2}^{2}&		\cdots&		\mathrm{C}_{n}^{2}-\mathrm{C}_{\mathrm{n}-1}^{2}\\
\vdots&		\vdots&		&		\vdots\\
\mathrm{C}_{\mathrm{n}-1}^{\mathrm{n}-1}&		\mathrm{C}_{\mathrm{n}}^{\mathrm{n}-1}-\mathrm{C}_{\mathrm{n}-1}^{\mathrm{n}-1}&		\cdots&		\mathrm{C}_{2\mathrm{n}-3}^{\mathrm{n}-1}-\mathrm{C}_{2\mathrm{n}-4}^{\mathrm{n}-1}\\
\end{matrix} \right|=\left| \begin{matrix}
\mathrm{C}_{1}^{1}&		\mathrm{C}_{1}^{0}&		\cdots&		\mathrm{C}_{\mathrm{n}-2}^{0}\\
\mathrm{C}_{2}^{2}&		\mathrm{C}_{2}^{1}&		\cdots&		\mathrm{C}_{\mathrm{n}-1}^{1}\\
\vdots&		\vdots&		&		\vdots\\
\mathrm{C}_{\mathrm{n}-1}^{\mathrm{n}-1}&		\mathrm{C}_{\mathrm{n}-1}^{\mathrm{n}-2}&		\cdots&		\mathrm{C}_{2\mathrm{n}-4}^{\mathrm{n}-2}\\
\end{matrix} \right|
\\
&=\left| \begin{matrix}
1&		1&		\cdots&		1\\
1&		\mathrm{C}_{2}^{1}&		\cdots&		\mathrm{C}_{\mathrm{n}-1}^{1}\\
\vdots&		\vdots&		&		\vdots\\
1&		\mathrm{C}_{\mathrm{n}-1}^{\mathrm{n}-2}&		\cdots&		\mathrm{C}_{2\mathrm{n}-4}^{\mathrm{n}-2}\\
\end{matrix} \right|
\end{split}.
\nonumber
\end{equation}
此时得到的行列式恰好是原行列式的左上角部分,并具有相同的规律.
不断这样做下去,最后可得$|\boldsymbol{A}|=1$
\end{solution}
\end{exercise}

\begin{exercise}
计算$n$阶行列式:
\begin{gather}
|\boldsymbol{A}|=\left| \begin{matrix}
1&		2&		3&		\cdots&		n\\
-1&		0&		3&		\cdots&		n\\
-1&		-2&		0&		\cdots&		n\\
\vdots&		\vdots&		\vdots&		&		\vdots\\
-1&		-2&		-3&		\cdots&		0\\
\end{matrix} \right|
\nonumber
\end{gather}
\begin{solution}
\begin{equation}
\begin{split}
|\boldsymbol{A}|=\left| \begin{matrix}
1&		2&		3&		\cdots&		n\\
-1&		0&		3&		\cdots&		n\\
-1&		-2&		0&		\cdots&		n\\
\vdots&		\vdots&		\vdots&		&		\vdots\\
-1&		-2&		-3&		\cdots&		0\\
\end{matrix} \right|
\xlongequal[i=2,\cdots ,n]{r_1+r_i}\left| \begin{matrix}
1&		2&		3&		\cdots&		n\\
0&		2&		*&		\cdots&		*\\
0&		0&		3&		\cdots&		*\\
\vdots&		\vdots&		\vdots&		&		\vdots\\
0&		0&		0&		\cdots&		n\\
\end{matrix} \right|=n!
\end{split}
\nonumber
\end{equation}
\end{solution}
\end{exercise}

\begin{exercise}
计算$n$阶行列式:
\begin{gather}
|\boldsymbol{A}|=\left| \begin{matrix}
a_1b_1&		a_1b_2&		a_1b_3&		\cdots&		a_1b_n\\
a_1b_2&		a_2b_2&		a_2b_3&		\cdots&		a_2b_n\\
a_1b_3&		a_2b_3&		a_3b_3&		\cdots&		a_3b_n\\
\vdots&		\vdots&		\vdots&		&		\vdots\\
a_1b_n&		a_2b_n&		a_3b_n&		\cdots&		a_nb_n\\
\end{matrix} \right|.
\nonumber
\end{gather}
\begin{solution}
\begin{equation}
\begin{split}
|\boldsymbol{A}|&=\left| \begin{matrix}
a_1b_1&		a_1b_2&		a_1b_3&		\cdots&		a_1b_n\\
a_1b_2&		a_2b_2&		a_2b_3&		\cdots&		a_2b_n\\
a_1b_3&		a_2b_3&		a_3b_3&		\cdots&		a_3b_n\\
\vdots&		\vdots&		\vdots&		&		\vdots\\
a_1b_n&		a_2b_n&		a_3b_n&		\cdots&		a_nb_n\\
\end{matrix} \right|=a_1\left| \begin{matrix}
b_1&		b_2&		b_3&		\cdots&		b_n\\
a_1b_2&		a_2b_2&		a_2b_3&		\cdots&		a_2b_n\\
a_1b_3&		a_2b_3&		a_3b_3&		\cdots&		a_3b_n\\
\vdots&		\vdots&		\vdots&		&		\vdots\\
a_1b_n&		a_2b_n&		a_3b_n&		\cdots&		a_nb_n\\
\end{matrix} \right|
\\
&\xlongequal[i=2,\cdots ,n]{\left( -a_i \right) r_1+r_i}a_1\left| \begin{matrix}
b_1&		b_2&		b_3&		\cdots&		b_n\\
a_1b_2-a_2b_1&		0&		0&		\cdots&		0\\
a_1b_3-a_3b_1&		a_2b_3-a_3b_2&		0&		\cdots&		0\\
\vdots&		\vdots&		\vdots&		&		\vdots\\
a_1b_n-a_nb_1&		a_2b_n-a_nb_2&		a_3b_n-a_nb_3&		\cdots&		0\\
\end{matrix} \right|
\\
&\xlongequal{\text{按第}n\text{列展开}}\left( -1 \right) ^{n+1}a_1b_n\left| \begin{matrix}
a_1b_2-a_2b_1&		0&		\cdots&		0\\
a_1b_3-a_3b_1&		a_2b_3-a_3b_2&		\cdots&		0\\
\vdots&		\vdots&		&		\vdots\\
a_1b_n-a_nb_1&		a_2b_n-a_nb_2&		\cdots&		a_{n-1}b_n-a_nb_{n-1}\\
\end{matrix} \right|
\\
&=\left( -1 \right) ^{n-1}a_1b_n\prod\limits_{i=1}^{n-1}{\left( a_ib_{i+1}-a_{i+1}b_i \right)}
\\
&=a_1b_n\prod\limits_{i=1}^{n-1}{\left( a_{i+1}b_i-a_ib_{i+1} \right)}.
\end{split}
\nonumber
\end{equation}
\end{solution}
\end{exercise}

\begin{exercise}
计算$n$阶行列式:
\begin{equation}
|\boldsymbol{A}|=\left| \begin{matrix}
a&		0&		\cdots&		0&		1\\
0&		a&		\cdots&		0&		0\\
\vdots&		\vdots&		&		\vdots&		\vdots\\
0&		0&		\cdots&		a&		0\\
1&		0&		\cdots&		0&		a\\
\end{matrix} \right|.
\nonumber
\end{equation}    
\begin{solution}
\begin{equation}
\begin{split}
|\boldsymbol{A}|=\left| \begin{matrix}
a&		0&		\cdots&		0&		1\\
0&		a&		\cdots&		0&		0\\
\vdots&		\vdots&		&		\vdots&		\vdots\\
0&		0&		\cdots&		a&		0\\
1&		0&		\cdots&		0&		a\\
\end{matrix} \right|\xlongequal{\text{按第一列展开}}a^n+\left( -1 \right) ^{n+1}\left| \begin{matrix}
0&		0&		\cdots&		0&		1\\
a&		0&		\cdots&		0&		0\\
\vdots&		\vdots&		&		\vdots&		\vdots\\
0&		0&		\cdots&		a&		0\\
\end{matrix} \right|=a^n+\left( -1 \right) ^{n+1+n}a^{n-2}=a^n-a^{n-2}.
\end{split}
\nonumber
\end{equation}
\end{solution}
\begin{remark}
本题也可由命题\ref{"爪"型行列式}直接得到,$|\boldsymbol{A}|=a^n-a^{n-2}$.
\end{remark}
\end{exercise}

\begin{exercise}
设$x_1,x_2,x_3$是方程$x^3+px+q=0$的3个根,求下列行列式的值:
\begin{gather}
\left| \boldsymbol{A} \right|=\left| \begin{matrix}
x_1&		x_2&		x_3\\
x_2&		x_3&		x_1\\
x_3&		x_1&		x_2\\
\end{matrix} \right|.
\nonumber
\end{gather}
\begin{solution}
由$Vieta$定理可知,$x_1+x_2+x_3=0$.因此,我们有
\begin{equation}
\begin{split}
\left| \boldsymbol{A} \right|=\left| \begin{matrix}
x_1&		x_2&		x_3\\
x_2&		x_3&		x_1\\
x_3&		x_1&		x_2\\
\end{matrix} \right|\xlongequal[i=2,3]{r_i+r_1}\left| \begin{matrix}
0&		0&		0\\
x_2&		x_3&		x_1\\
x_3&		x_1&		x_2\\
\end{matrix} \right|=0.
\end{split}
\nonumber
\end{equation}
\end{solution}
\end{exercise}

\begin{exercise}
设$b_{ij}=\left( a_{i1}+a_{i2}+\cdots +a_{in} \right) -a_{ij}$,求证:
\begin{equation}
\left| \begin{matrix}
b_{11}&		\cdots&		b_{1n}\\
\vdots&		&		\vdots\\
b_{n1}&		\cdots&		b_{nn}\\
\end{matrix} \right|=(-1)^{n-1}(n-1)\left| \begin{matrix}
a_{11}&		\cdots&		a_{1n}\\
\vdots&		&		\vdots\\
a_{n1}&		\cdots&		a_{nn}\\
\end{matrix} \right|.
\nonumber
\end{equation}
\begin{solution}
\begin{align*}
&\left| \begin{matrix}
b_{11}&		b_{12}&		\cdots&		b_{1n}\\
b_{21}&		b_{22}&		\cdots&		b_{2n}\\
\vdots&		\vdots&		&		\vdots\\
b_{n1}&		b_{n2}&		\cdots&		b_{nn}\\
\end{matrix} \right|=\left| \begin{matrix}
\left( a_{11}+a_{12}+\cdots +a_{1n} \right) -a_{11}&		\left( a_{11}+a_{12}+\cdots +a_{1n} \right) -a_{12}&		\cdots&		\left( a_{11}+a_{12}+\cdots +a_{1n} \right) -a_{1n}\\
\left( a_{21}+a_{22}+\cdots +a_{2n} \right) -a_{21}&		\left( a_{21}+a_{22}+\cdots +a_{2n} \right) -a_{22}&		\cdots&		\left( a_{21}+a_{22}+\cdots +a_{2n} \right) -a_{2n}\\
\vdots&		\vdots&		&		\vdots\\
\left( a_{n1}+a_{n2}+\cdots +a_{nn} \right) -a_{n1}&		\left( a_{n1}+a_{n2}+\cdots +a_{nn} \right) -a_{n2}&		\cdots&		\left( a_{n1}+a_{n2}+\cdots +a_{nn} \right) -a_{nn}\\
\end{matrix} \right|
\\
&\xlongequal[i=2,\cdots ,n]{j_i+j_1}\left| \begin{matrix}
\left( n-1 \right) \left( a_{11}+a_{12}+\cdots +a_{1n} \right)&		\left( a_{11}+a_{12}+\cdots +a_{1n} \right) -a_{12}&		\cdots&		\left( a_{11}+a_{12}+\cdots +a_{1n} \right) -a_{1n}\\
\left( n-1 \right) \left( a_{21}+a_{22}+\cdots +a_{2n} \right)&		\left( a_{21}+a_{22}+\cdots +a_{2n} \right) -a_{22}&		\cdots&		\left( a_{21}+a_{22}+\cdots +a_{2n} \right) -a_{2n}\\
\vdots&		\vdots&		&		\vdots\\
\left( n-1 \right) \left( a_{n1}+a_{n2}+\cdots +a_{nn} \right)&		\left( a_{n1}+a_{n2}+\cdots +a_{nn} \right) -a_{n2}&		\cdots&		\left( a_{n1}+a_{n2}+\cdots +a_{nn} \right) -a_{nn}\\
\end{matrix} \right|
\\
&=\left( n-1 \right) \left| \begin{matrix}
\left( a_{11}+a_{12}+\cdots +a_{1n} \right)&		\left( a_{11}+a_{12}+\cdots +a_{1n} \right) -a_{12}&		\cdots&		\left( a_{11}+a_{12}+\cdots +a_{1n} \right) -a_{1n}\\
\left( a_{21}+a_{22}+\cdots +a_{2n} \right)&		\left( a_{21}+a_{22}+\cdots +a_{2n} \right) -a_{22}&		\cdots&		\left( a_{21}+a_{22}+\cdots +a_{2n} \right) -a_{2n}\\
\vdots&		\vdots&		&		\vdots\\
\left( a_{n1}+a_{n2}+\cdots +a_{nn} \right)&		\left( a_{n1}+a_{n2}+\cdots +a_{nn} \right) -a_{n2}&		\cdots&		\left( a_{n1}+a_{n2}+\cdots +a_{nn} \right) -a_{nn}\\
\end{matrix} \right|
\\
&\xlongequal[i=2,\cdots ,n]{\left( -1 \right) j_1+j_i}\left( n-1 \right) \left| \begin{matrix}
\left( a_{11}+a_{12}+\cdots +a_{1n} \right)&		-a_{12}&		\cdots&		-a_{1n}\\
\left( a_{21}+a_{22}+\cdots +a_{2n} \right)&		-a_{22}&		\cdots&		-a_{2n}\\
\vdots&		\vdots&		&		\vdots\\
\left( a_{n1}+a_{n2}+\cdots +a_{nn} \right)&		-a_{n2}&		\cdots&		-a_{nn}\\
\end{matrix} \right|
\\
&\xlongequal[i=2,\cdots ,n]{j_i+j_1}\left( n-1 \right) \left| \begin{matrix}
a_{11}&		-a_{12}&		\cdots&		-a_{1n}\\
a_{21}&		-a_{22}&		\cdots&		-a_{2n}\\
\vdots&		\vdots&		&		\vdots\\
a_{n1}&		-a_{n2}&		\cdots&		-a_{nn}\\
\end{matrix} \right|
\\
&=(-1)^{n-1}(n-1)\left| \begin{matrix}
a_{11}&		-a_{12}&		\cdots&		-a_{1n}\\
a_{21}&		-a_{22}&		\cdots&		-a_{2n}\\
\vdots&		\vdots&		&		\vdots\\
a_{n1}&		-a_{n2}&		\cdots&		-a_{nn}\\
\end{matrix} \right|.
\end{align*}
\end{solution}
\end{exercise}
\begin{conclusion}\label{行列式计算:求和法}
第二个等号是行列式计算中的一个常用方法\hypertarget{行列式计算:求和法}{\textbf{求和法}}:

将除第一列外的其余列全部加到第一列上(或将除第一行外的其余行全部加到第一行上),
使第一列(或列)一样或者具有相同形式.
然后根据具体情况将第一列(或行)的倍数加到其余列(或行)上,
从而将行列式化为我们熟悉的形式.

应用该方法的一般情形:

1.行列式每行(或列)和相等时;

2.行列式每行(或列)和有一定规律时.

\end{conclusion}

\begin{exercise}
计算$n$阶行列式:
\begin{equation}
|\boldsymbol{A}|=\left| \begin{matrix}
0&		1&		\cdots&		1&		1\\
1&		0&		\cdots&		1&		1\\
\vdots&		\vdots&		&		\vdots&		\vdots\\
1&		1&		\cdots&		0&		1\\
1&		1&		\cdots&		1&		0\\
\end{matrix} \right|.
\nonumber
\end{equation}
\begin{solution}
\begin{equation}
\begin{split}
&|\boldsymbol{A}|=\left| \begin{matrix}
0&		1&		\cdots&		1&		1\\
1&		0&		\cdots&		1&		1\\
\vdots&		\vdots&		&		\vdots&		\vdots\\
1&		1&		\cdots&		0&		1\\
1&		1&		\cdots&		1&		0\\
\end{matrix} \right|\xlongequal[i=2,\cdots ,n]{j_i+j_1}\left| \begin{matrix}
n-1&		1&		\cdots&		1&		1\\
n-1&		0&		\cdots&		1&		1\\
\vdots&		\vdots&		&		\vdots&		\vdots\\
n-1&		1&		\cdots&		0&		1\\
n-1&		1&		\cdots&		1&		0\\
\end{matrix} \right|=\left( n-1 \right) \left| \begin{matrix}
1&		1&		\cdots&		1&		1\\
1&		0&		\cdots&		1&		1\\
\vdots&		\vdots&		&		\vdots&		\vdots\\
1&		1&		\cdots&		0&		1\\
1&		1&		\cdots&		1&		0\\
\end{matrix} \right|
\\
&\xlongequal[i=2,\cdots ,n]{\left( -1 \right) r_1+r_i}\left( n-1 \right) \left| \begin{matrix}
1&		1&		\cdots&		1&		1\\
0&		-1&		\cdots&		0&		0\\
\vdots&		\vdots&		&		\vdots&		\vdots\\
0&		0&		\cdots&		-1&		0\\
0&		0&		\cdots&		0&		-1\\
\end{matrix} \right|=\left( -1 \right) ^{n-1}\left( n-1 \right) .
\end{split}
\nonumber
\end{equation}
\end{solution}
\begin{remark}
因为$|\boldsymbol{A}|$除对角元素外,每行都一样,
所以本题也可以看成命题\ref{"爪"型行列式的推广}的应用,利用命题\ref{"爪"型行列式的推广}的计算方法直接得到结果.
\begin{equation}
|\boldsymbol{A}|=\left| \begin{matrix}
0&		1&		\cdots&		1&		1\\
1&		0&		\cdots&		1&		1\\
\vdots&		\vdots&		&		\vdots&		\vdots\\
1&		1&		\cdots&		0&		1\\
1&		1&		\cdots&		1&		0\\
\end{matrix} \right|\xlongequal[i=2,\cdots ,n]{\left( -1 \right) r_1+r_i}\left| \begin{matrix}
0&		1&		\cdots&		1&		1\\
1&		-1&		\cdots&		0&		0\\
\vdots&		\vdots&		&		\vdots&		\vdots\\
1&		0&		\cdots&		-1&		0\\
1&		0&		\cdots&		0&		-1\\
\end{matrix} \right|\xlongequal[]{\text{命题}1.2}-\sum_{i=2}^n{\left( -1 \right) ^{n-2}}=\left( -1 \right) ^{n-1}\left( n-1 \right) .            
\nonumber
\end{equation}
\end{remark}
\end{exercise}

\begin{exercise}
计算$n$阶行列式:
\begin{equation}
\begin{split}
|\boldsymbol{A}|=\left| \begin{matrix}
a_1+b&		a_2&		a_3&		\cdots&		a_n\\
a_1&		a_2+b&		a_3&		\cdots&		a_n\\
a_1&		a_2&		a_3+b&		\cdots&		a_n\\
\vdots&		\vdots&		\vdots&		&		\vdots\\
a_1&		a_2&		a_3&		\cdots&		a_n+b\\
\end{matrix} \right|.
\end{split}
\nonumber
\end{equation}
\begin{note}
既可以将$|\boldsymbol{A}|$看作命题\ref{"爪"型行列式的推广}的应用,
利用命题\ref{"爪"型行列式的推广}的计算方法直接得到结果.即下述解法一.

也可以利用\hyperlink{行列式计算:求和法}{求和法}将$|\boldsymbol{A}|$
化为上三角形行列式.即下述解法二.
\end{note}
\begin{solution}
{\color{blue} \text{解法一}}:
\begin{equation}
\begin{split}
&|\boldsymbol{A}|=\left| \begin{matrix}
a_1+b&		a_2&		a_3&		\cdots&		a_n\\
a_1&		a_2+b&		a_3&		\cdots&		a_n\\
a_1&		a_2&		a_3+b&		\cdots&		a_n\\
\vdots&		\vdots&		\vdots&		&		\vdots\\
a_1&		a_2&		a_3&		\cdots&		a_n+b\\
\end{matrix} \right|
\xlongequal[i=2,\cdots ,n]{-r_1+r_i}\left| \begin{matrix}
a_1+b&		a_2&		a_3&		\cdots&		a_n\\
-b&		b&		0&		\cdots&		0\\
-b&		0&		b&		\cdots&		0\\
\vdots&		\vdots&		\vdots&		&		\vdots\\
-b&		0&		0&		\cdots&		b\\
\end{matrix} \right|
\\
&\xlongequal[]{\text{命题}\ref{"爪"型行列式}}\left( a_1+b \right) b^{n-1}-\sum_{i=2}^n{b^{n-2}a_i\left( -b \right)}
=b^{n-1}\left[ \left( a_1+b \right) +\sum_{i=2}^n{a_i} \right] 
=\left(( b+\sum_{i=1}^n{a_i} \right) b^{n-1}.
\end{split}
\nonumber
\end{equation}
{\color{blue} \text{解法二}}:
\begin{equation}
\begin{split}
&|\boldsymbol{A}|=\left| \begin{matrix}
a_1+b&		a_2&		a_3&		\cdots&		a_n\\
a_1&		a_2+b&		a_3&		\cdots&		a_n\\
a_1&		a_2&		a_3+b&		\cdots&		a_n\\
\vdots&		\vdots&		\vdots&		&		\vdots\\
a_1&		a_2&		a_3&		\cdots&		a_n+b\\
\end{matrix} \right|
\xlongequal[i=2,\cdots ,n]{j_i+j_1}(b+\sum_{i=1}^n{a_i)\left| \begin{matrix}
1&		a_2&		a_3&		\cdots&		a_n\\
1&		a_2+b&		a_3&		\cdots&		a_n\\
1&		a_2&		a_3+b&		\cdots&		a_n\\
\vdots&		\vdots&		\vdots&		&		\vdots\\
1&		a_2&		a_3&		\cdots&		a_n+b\\
\end{matrix} \right|}
\\
&\xlongequal[i=2,\cdots ,n]{-a_i\cdot j_1+j_i}(b+\sum_{i=1}^n{a_i)\left| \begin{matrix}
1&		0&		0&		\cdots&		0\\
1&		b&		0&		\cdots&		0\\
1&		0&		b&		\cdots&		0\\
\vdots&		\vdots&		\vdots&		&		\vdots\\
1&		0&		0&		\cdots&		b\\
\end{matrix} \right|}
=(b+\sum_{i=1}^n{a_i)b^{n-1}}.
\end{split}
\nonumber
\end{equation}
\end{solution}
\end{exercise}

\begin{exercise}
计算$n$阶行列式:
\begin{equation}
\begin{split}
|\boldsymbol{A}|=\left| \begin{matrix}
1&		2&		3&		\cdots&		n-1&		n\\
n&		1&		2&		\cdots&		n-2&		n-1\\
n-1&		n&		1&		\cdots&		n-3&		n-2\\
\vdots&		\vdots&		\vdots&		&		\vdots&		\vdots\\
3&		4&		5&		\cdots&		1&		2\\
2&		3&		4&		\cdots&		n&		1\\
\end{matrix}\right|.
\end{split}
\nonumber
\end{equation}
\end{exercise}
\begin{note}
\hyperlink{行列式计算:求和法}{求和法}的经典应用.
\end{note}
\begin{solution}
\begin{equation}
\begin{split}
&|\boldsymbol{A}|=\left| \begin{matrix}
1&		2&		3&		\cdots&		n-1&		n\\
n&		1&		2&		\cdots&		n-2&		n-1\\
n-1&		n&		1&		\cdots&		n-3&		n-2\\
\vdots&		\vdots&		\vdots&		&		\vdots&		\vdots\\
3&		4&		5&		\cdots&		1&		2\\
2&		3&		4&		\cdots&		n&		1\\
\end{matrix} \right|\xlongequal[i=2,\cdots ,n]{j_i+j_1}\frac{n\left( n+1 \right)}{2}\left| \begin{matrix}
1&		2&		3&		\cdots&		n-1&		n\\
1&		1&		2&		\cdots&		n-2&		n-1\\
1&		n&		1&		\cdots&		n-3&		n-2\\
\vdots&		\vdots&		\vdots&		&		\vdots&		\vdots\\
1&		4&		5&		\cdots&		1&		2\\
1&		3&		4&		\cdots&		n&		1\\
\end{matrix} \right|
\\
&\xlongequal[i=2,\cdots ,n]{-r_1+r_i}\frac{n\left( n+1 \right)}{2}\left| \begin{matrix}
1&		2&		3&		\cdots&		n-1&		n\\
0&		-1&		-1&		\cdots&		-1&		-1\\
0&		n-2&		-2&		\cdots&		-2&		-2\\
\vdots&		\vdots&		\vdots&		&		\vdots&		\vdots\\
0&		2&		2&		\cdots&		2-n&		2-n\\
0&		1&		1&		\cdots&		1&		1-n\\
\end{matrix} \right|\xlongequal[]{\text{按第一列展开}}\frac{n\left( n+1 \right)}{2}\left| \begin{matrix}
-1&		-1&		\cdots&		-1&		-1\\
n-2&		-2&		\cdots&		-2&		-2\\
\vdots&		\vdots&		&		\vdots&		\vdots\\
2&		2&		\cdots&		2-n&		2-n\\
1&		1&		\cdots&		1&		1-n\\
\end{matrix} \right|
\\
&\xlongequal[i=2,\cdots ,n]{-j_1+j_i}\frac{n\left( n+1 \right)}{2}\left| \begin{matrix}
-1&		0&		\cdots&		0&		0\\
n-2&		-n&		\cdots&		-n&		-n\\
\vdots&		\vdots&		&		\vdots&		\vdots\\
2&		0&		\cdots&		-n&		-n\\
1&		0&		\cdots&		0&		-n\\
\end{matrix} \right|\xlongequal[]{\text{按第一行展开}}-\frac{n\left( n+1 \right)}{2}\left| \begin{matrix}
-n&		\cdots&		-n&		-n\\
\vdots&		&		\vdots&		\vdots\\
0&		\cdots&		-n&		-n\\
0&		\cdots&		0&		-n\\
\end{matrix} \right|
\\
&=-\frac{n\left( n+1 \right)}{2}\left( -n \right) ^{n-2}=\left( -1 \right) ^{n-1}\frac{n+1}{2}n^{n-1}.            
\end{split}
\nonumber
\end{equation}
\end{solution}

\begin{exercise}
计算$D_{n+1}=\left| \begin{matrix}
\left( a_0+b_0 \right) ^n&		\left( a_0+b_1 \right) ^n&		\cdots&		\left( a_0+b_n \right) ^n\\
\left( a_1+b_0 \right) ^n&		\left( a_1+b_1 \right) ^n&		\cdots&		\left( a_1+b_n \right) ^n\\
\vdots&		\vdots&		\vdots&		\\
\left( a_n+b_0 \right) ^n&		\left( a_n+b_1 \right) ^n&		\cdots&		\left( a_n+b_n \right) ^n\\
\end{matrix} \right|$.
\end{exercise}
\begin{solution}
由二项式定理可知
\begin{align*}
\left( a_i+b_j \right) ^n={a_i}^n+\mathrm{C}_{n}^{1}{a_i}^{n-1}b_j+\cdots +\mathrm{C}_{n}^{n-1}a_i{b_j}^{n-1}+{b_j}^n,\text{其中}i,j=0,1,\cdots ,n.
\nonumber
\end{align*}
从而
\begin{align*}
D_{n+1}&=\left| \begin{matrix}
{a_0}^n+\mathrm{C}_{n}^{1}{a_0}^{n-1}b_0+\cdots +\mathrm{C}_{n}^{n-1}a_0{b_0}^{n-1}+{b_0}^n&		\cdots&		{a_0}^n+\mathrm{C}_{n}^{1}{a_0}^{n-1}b_n+\cdots +\mathrm{C}_{n}^{n-1}a_0{b_n}^{n-1}+{b_n}^n\\
{a_1}^n+\mathrm{C}_{n}^{1}{a_1}^{n-1}b_0+\cdots +\mathrm{C}_{n}^{n-1}a_1{b_0}^{n-1}+{b_0}^n&		\cdots&		{a_1}^n+\mathrm{C}_{n}^{1}{a_1}^{n-1}b_n+\cdots +\mathrm{C}_{n}^{n-1}a_1{b_n}^{n-1}+{b_n}^n\\
\vdots&		&		\vdots\\
{a_{n-1}}^n+\mathrm{C}_{n}^{1}{a_{n\-1}}^{n-1}b_0+\cdots +\mathrm{C}_{n}^{n-1}a_{n-1}{b_0}^{n-1}+{b_0}^n&		\cdots&		{a_{n-1}}^n+\mathrm{C}_{n}^{1}{a_{n-1}}^{n-1}b_n+\cdots +\mathrm{C}_{n}^{n-1}a_{n-1}{b_n}^{n-1}+{b_n}^n\\
{a_n}^n+\mathrm{C}_{n}^{1}{a_n}^{n-1}b_0+\cdots +\mathrm{C}_{n}^{n-1}a_n{b_0}^{n-1}+{b_0}^n&		\cdots&		{a_n}^n+\mathrm{C}_{n}^{1}{a_n}^{n-1}b_n+\cdots +\mathrm{C}_{n}^{n-1}a_n{b_n}^{n-1}+{b_n}^n\\
\end{matrix} \right|
\\
&=\left| \begin{matrix}
{a_0}^n&		{a_0}^{n-1}&		\cdots&		a_0&		1\\
{a_1}^n&		{a_1}^{n-1}&		\cdots&		a_1&		1\\
\vdots&		\vdots&		&		\vdots&		\vdots\\
{a_{n-1}}^n&		{a_{n-1}}^{n-1}&		\cdots&		a_{n-1}&		1\\
{a_n}^n&		{a_n}^{n-1}&		\cdots&		a_n&		1\\
\end{matrix} \right|\cdot \left| \begin{matrix}
1&		1&		\cdots&		1&		1\\
\mathrm{C}_{n}^{1}b_0&		\mathrm{C}_{n}^{1}b_1&		\cdots&		\mathrm{C}_{n}^{1}b_{n-1}&		\mathrm{C}_{n}^{1}b_n\\
\vdots&		\vdots&		&		\vdots&		\vdots\\
\mathrm{C}_{n}^{n-1}{b_0}^{n-1}&		\mathrm{C}_{n}^{n-1}{b_1}^{n-1}&		\cdots&		\mathrm{C}_{n}^{n-1}{b_{n-1}}^{n-1}&		\mathrm{C}_{n}^{n-1}{b_n}^{n-1}\\
{b_0}^n&		{b_1}^n&		\cdots&		{b_{n-1}}^n&		{b_n}^n\\
\end{matrix} \right|
\\
&\xlongequal{\hyperlink{行列式计算常识}{\text{列倒排}}}\left( -1 \right) ^{\frac{n\left( n+1 \right)}{2}}\left| \begin{matrix}
1&		a_0&		\cdots&		{a_0}^{n-1}&		{a_0}^n\\
1&		a_1&		\cdots&		{a_1}^{n-1}&		{a_1}^n\\
\vdots&		\vdots&		&		\vdots&		\vdots\\
1&		a_n&		\cdots&		{a_n}^{n-1}&		{a_n}^n\\
\end{matrix} \right|\cdot \prod_{i=1}^{n-1}{\mathrm{C}_{n}^{i}\left| \begin{matrix}
1&		1&		\cdots&		1&		1\\
b_0&		b_1&		\cdots&		b_{n-1}&		b_n\\
\vdots&		\vdots&		&		\vdots&		\vdots\\
{b_0}^{n-1}&		{b_1}^{n-1}&		\cdots&		{b_{n-1}}^{n-1}&		{b_n}^{n-1}\\
{b_0}^n&		{b_1}^n&		\cdots&		{b_{n-1}}^n&		{b_n}^n\\
\end{matrix} \right|}
\\
&=\left( -1 \right) ^{\frac{n\left( n+1 \right)}{2}}\prod_{0\le j<i\le n}{\left( a_i-a_j \right)}\prod_{i=1}^{n-1}{\mathrm{C}_{n}^{i}\prod_{0\le j<i\le n}{\left( b_i-b_j \right)}}
= =\prod_{i=1}^{n-1}{\mathrm{C}_{n}^{i}\prod_{0\le j<i\le n}{\left( a_j-a_i \right) \left( b_i-b_j \right)}}.
\end{align*}
\end{solution}

\begin{exercise}\label{三对角行列式例题1}
求证:$n$阶行列式
\begin{equation}
|\boldsymbol{A}|=\left| \begin{matrix}
\cos x&		1&		0&		0&		\cdots&		0&		0&		0\\
1&		2\cos x&		1&		0&		\cdots&		0&		0&		0\\
0&		1&		2\cos x&		1&		\cdots&		0&		0&		0\\
\vdots&		\vdots&		\vdots&		\vdots&		&		\vdots&		\vdots&		\vdots\\
0&		0&		0&		0&		\cdots&		1&		2\cos x&		1\\
0&		0&		0&		0&		\cdots&		0&		1&		2\cos x\\
\end{matrix} \right|=\cos nx.
\nonumber
\end{equation}
\end{exercise}
\begin{solution}
{\color{blue} \text{解法一:}}

设$|\boldsymbol{A}|=D_n$,其中$n$表示$|\boldsymbol{A}|$的阶数$(n\ge0)$.易知$D_0=1,D_1=\cos x$.

从而$|\boldsymbol{A}|=D_n\xlongequal[\text{命题}\ref{三对角行列式}]{\text{按最后一列展开}}2\cos xD_{n-1}-D_{n-2}\left( n\ge 2 \right)$.

其对应的特征方程为$\lambda ^2=2\cos x\lambda -1$,解得$\lambda _1=\cos x+i\sin x,\lambda _2=\cos x-i\sin x$.

于是当$n\ge2$时,我们有$D_n=\left( \lambda _1+\lambda _2 \right) D_{n-1}+\lambda _1\lambda _2D_{n-2}$.

进而
\begin{equation}
\label{eq:递推式1.1}
\begin{split}
&D_n-\lambda _1D_{n-1}=\lambda _2\left( D_n-\lambda _1D_{n-1} \right),
\\
&D_n-\lambda _2D_{n-1}=\lambda _1\left( D_n-\lambda _2D_{n-1} \right).
\end{split}
\end{equation}
由此可得
\begin{gather}
D_n-\lambda _1D_{n-1}={\lambda _2}^{n-1}\left( D_1-\lambda _1D_0 \right) =-i\sin x\cdot {\lambda _2}^{n-1},
\nonumber\\
D_n-\lambda _2D_{n-1}={\lambda _1}^{n-1}\left( D_1-\lambda _2D_0 \right) =i\sin x\cdot {\lambda _1}^{n-1}.
\nonumber
\end{gather}
若$x\ne k\pi(k\in\mathbb{Z})$,则联立上面两式,解得
\begin{equation}
\begin{split}
D_n&=\frac{i\sin x\cdot {\lambda _1}^n+i\sin x\cdot {\lambda _2}^n}{\lambda _1-\lambda _2}=\frac{i\sin x\cdot \left( \cos x+i\sin x \right) ^n+i\sin x\cdot \left( \cos x-i\sin x \right) ^n}{2i\sin x}
\\
&\xlongequal[e^{ix}=\cos x+i\sin x,e^{-ix}=\cos x-i\sin x.]{Euler\text{公式}}\frac{i\sin x\cdot e^{nxi}+i\sin x\cdot e^{-nxi}}{2i\sin x}=\frac{i\sin x\cdot \left( \cos nx+i\sin nx \right) +i\sin x\cdot \left( \cos nx-i\sin nx \right)}{2i\sin x}
\\
&=\frac{2i\sin x\cdot \cos nx}{2i\sin x}=\cos nx.
\end{split}
\nonumber
\end{equation}
若$x=k\pi(k\in\mathbb{Z})$,则$\lambda _1=\lambda _2=\cos k\pi$.
从而由\eqref{eq:递推式1.1}式可得,$D_n-\cos k\pi D_{n-1}=-i\sin x\cdot \left( \cos k\pi \right) =0.$

于是
\begin{align*}
D_n=\cos k\pi D_{n-1}=\left( \cos k\pi \right) ^2D_{n-2}=\cdots =\left( \cos k\pi \right) ^nD_0=\left( \cos k\pi \right) ^n=\left( -1 \right) ^{kn}=\cos \left( nk\pi \right) =\cos nx.
\nonumber
\end{align*}
{\color{blue} \text{解法二:}}仿照练习\ref{使用数学归纳法计算行列式例题1}中的数学归纳法证明.
\end{solution}

\begin{exercise}\label{三对角行列式例题2}
求下列$n$阶行列式的值:
\begin{gather}
D_{n}=\begin{vmatrix}1-a_{1}&a_{2}&0&0&\cdots&0&0\\ -1&1-a_{2}&a_{3}&0&\cdots&0&0\\ 0&-1&1-a_{3}&a_{4}&\cdots&0&0\\ \vdots&\vdots&\vdots&\vdots&&\vdots&\vdots\\ 0&0&0&0&\cdots&-1&1-a_{n}\end{vmatrix}.
\nonumber
\end{gather}
\end{exercise}
\begin{note}
观察原行列式我们可以得到,$D_n$的每列和有一定的规律,即除了第一列和最后一列,中间每列和均为0.并且$D_n$是三对角行列式.
因此,我们既可以直接应用三对角行列式的结论(即命题\ref{三对角行列式}),又可以使用求和法进行求解.
如果我们直接应用三对角行列式的结论(即命题\ref{三对角行列式}),按照对一般的三对角行列式展开的方法能得到相应递推式,但是这样得到的递推式并不是相邻两项之间的递推,后续求解通项并不简便.
又因为使用求和法计算行列式后续计算一般比较简便所以我们先采用求和法进行尝试.
\end{note}
\begin{solution}
{\color{blue} \text{解法一:}}
当$n\ge1$时,我们有
\begin{align}
D_n&=\left| \begin{matrix}
1-a_1&		a_2&		0&		0&		\cdots&		0&		0\\
-1&		1-a_2&		a_3&		0&		\cdots&		0&		0\\
0&		-1&		1-a_3&		a_4&		\cdots&		0&		0\\
\vdots&		\vdots&		\vdots&		\vdots&		&		\vdots&		\vdots\\
0&		0&		0&		0&		\cdots&		-1&		1-a_n\\
\end{matrix} \right|\xlongequal[i=2,\cdots ,n]{r_i+r_1}\left| \begin{matrix}
-a_1&		0&		0&		0&		\cdots&		0&		1\\
-1&		1-a_2&		a_3&		0&		\cdots&		0&		0\\
0&		-1&		1-a_3&		a_4&		\cdots&		0&		0\\
\vdots&		\vdots&		\vdots&		\vdots&		&		\vdots&		\vdots\\
0&		0&		0&		0&		\cdots&		-1&		1-a_n\\
\end{matrix} \right|
\nonumber\\\nonumber
&\xlongequal[]{\text{按第一行展开}}-a_1D_{n-1}+\left( -1 \right) ^{n+1}\left| \begin{matrix}
-1&		1-a_2&		a_3&		0&		\cdots&		0\\
0&		-1&		1-a_3&		a_4&		\cdots&		0\\
\vdots&		\vdots&		\vdots&		\vdots&		&		\vdots\\
0&		0&		0&		0&		\cdots&		-1\\
\end{matrix} \right|
\\\nonumber
&=-a_1D_{n-1}+\left( -1 \right) ^{n+1}\left( -1 \right) ^{n-1}
\\\nonumber
&=1-a_1D_{n-1}.
\end{align}
其中$D_{n-i}$表示$D_{n-i+1}$去掉第一行和第一列得到的$n-i$阶行列式,$i=1,2,\cdots,n-1$.
(或者称$D_{n-i}$表示以$a_{i+1},\cdots,a_n$为未定元的$n-i$阶行列式,$i=1,2,\cdots,n-1$)

由递推不难得到
\begin{align*}
D_n=1-a_1\left( 1-a_2D_{n-2} \right) =1-a_1+a_1a_2D_{n-2}=\cdots =1-a_1+a_1a_2-a_1a_2a_3+\cdots +\left( -1 \right) ^na_1a_2\cdots a_n.
\nonumber
\end{align*}
{\color{blue} \text{解法二:}}仿照练习\ref{使用数学归纳法计算行列式例题1}中的数学归纳法证明.
\end{solution}

\begin{exercise}\label{使用数学归纳法计算行列式例题1}
设$n$阶行列式
\begin{align}
A_n=\left| \begin{matrix}
a_0+a_1&		a_1&		0&		0&		\cdots&		0&		0\\
a_1&		a_1+a_2&		a_2&		0&		\cdots&		0&		0\\
0&		a_2&		a_2+a_3&		a_3&		\cdots&		0&		0\\
\vdots&		\vdots&		\vdots&		\vdots&		&		\vdots&		\vdots\\
0&		0&		0&		0&		\cdots&		a_{n-1}&		a_{n-1}+a_n\\
\end{matrix} \right|,
\nonumber
\end{align}
求证:
\begin{align}
A_n=a_0a_1\cdots a_n\left( \frac{1}{a_0}+\frac{1}{a_1}+\cdots +\frac{1}{a_n} \right) .
\nonumber
\end{align}
\end{exercise}
\begin{note}
用\hypertarget{用数学归纳法与行列式有关的结论}{\textbf{数学归纳法}}证明与行列式有关的结论.

练习\ref{三对角行列式例题1}和练习\ref{三对角行列式例题2}都可同理使用用数学归纳法证明(对阶数$n$进行归纳即可).
\end{note}
\begin{proof}
(数学归纳法)对阶数$n$进行归纳.当$n=1,2$时,结论显然成立.假设阶数小于$n$结论成立.

现证明$n$阶的情形.注意到
\begin{align*}
A_n=\left| \begin{matrix}
a_0+a_1&		a_1&		0&		0&		\cdots&		0	&0\\
a_1&		a_1+a_2&		a_2&		0&		\cdots&		0&		0\\
0&		a_2&		a_2+a_3&		a_3&		\cdots&		0&		0\\
\vdots&		\vdots&		\vdots&		\vdots&		&		\vdots&		\vdots\\
0&		0&		0&		0&		\cdots&		a_{n-1}&		a_{n-1}+a_n\\
\end{matrix} \right|=\left( a_{n-1}+a_n \right) A_{n-1}-a_{n-1}^{2}A_{n-2}.
\nonumber
\end{align*}
将归纳假设代入上面的式子中得
\begin{align*}
A_n&=\left( a_{n-1}+a_n \right) A_{n-1}-a_{n-1}^{2}A_{n-2}
\\
&=\left( a_{n-1}+a_n \right) a_0a_1\cdots a_{n-1}\left( \frac{1}{a_0}+\frac{1}{a_1}+\cdots +\frac{1}{a_{n-1}} \right) -a_{n-1}^{2}a_0a_1\cdots a_{n-2}\left( \frac{1}{a_0}+\frac{1}{a_1}+\cdots +\frac{1}{a_{n-2}} \right) 
\\
&=a_0a_1\cdots a_n\left( \frac{1}{a_0}+\frac{1}{a_1}+\cdots +\frac{1}{a_{n-1}} \right) +a_0a_1\cdots a_{n-2}a_{n-1}^{2}\frac{1}{a_{n-1}}
\\
&=a_0a_1\cdots a_{n-1}\left[ a_n\left( \frac{1}{a_0}+\frac{1}{a_1}+\cdots +\frac{1}{a_{n-1}} \right) +1 \right] 
\\
&=a_0a_1\cdots a_{n-1}a_n\left( \frac{1}{a_0}+\frac{1}{a_1}+\cdots +\frac{1}{a_{n-1}}+\frac{1}{a_n} \right) .
\nonumber
\end{align*}
故由数学归纳法可知,结论对任意正整数$n$都成立.
\end{proof}

\begin{exercise}
设\(n(n > 2)\)阶行列式\(\vert A \vert\)的所有元素或为\(1\)或为\(-1\),求证:\(\vert A \vert\)的绝对值小于等于\(\frac{2}{3}n!\).
\end{exercise}
\begin{solution}
对阶数$n$进行归纳.当$n=3$时,将$\left| A \right|$的第一列元素为-1的行都乘以-1,再将$\left| A \right|$的第一行元素为1的列都乘以-1,$\left| A \right|$的绝对值不改变.

因此不妨设$\left| A \right|=\left| \begin{matrix}
1&		-1&		-1\\
1&		a_0&		b_0\\
1&		c_0&		d_0\\
\end{matrix} \right|,\text{其中}a_0,b_0,c_0,d_0=1\text{或}-1.$

从而
\begin{align*}
\left| A \right|=\left| \begin{matrix}
1&		-1&		-1\\
1&		a_0&		b_0\\
1&		c_0&		d_0\\
\end{matrix} \right|\xlongequal[i=2,3]{j_1+j_i}\left| \begin{matrix}
1&		0&		0\\
1&		a&		b\\
1&		c&		d\\
\end{matrix} \right|,\text{其中}a,b,c,d=0\text{或}2.
\nonumber
\end{align*}
于是
\begin{align*}
abs \left( \left| A \right| \right) =abs \left( \left| \begin{matrix}
1&		0&		0\\
1&		a&		b\\
1&		c&		d\\
\end{matrix} \right| \right) =abs \left( ad-bc \right) \leqslant 4=\frac{2}{3}\cdot 3!
\nonumber
\end{align*}
假设n-1阶时结论成立,现证$n$阶的情形.将$\left| A \right|$按第一行展开得
\begin{align*}
\left| A \right|=a_{11}A_{11}+a_{12}A_{12}+\cdots +a_{1n}A_{1n},\text{其中}a_{1i}=1\text{或}-1\left( i=1,2\cdots ,n \right) .
\nonumber
\end{align*}
从而由归纳假设可得
\begin{align*}
abs \left( \left| A \right| \right) &=abs \left( a_{11}A_{11}+a_{12}A_{12}+\cdots +a_{1n}A_{1n} \right) \leqslant abs \left( A_{11} \right) +abs \left( A_{12} \right) +\cdots +abs \left( A_{1n} \right) 
\\
&\leqslant \frac{2}{3}\left( n-1 \right) !+\frac{2}{3}\left( n-1 \right) !+\cdots +\frac{2}{3}\left( n-1 \right) !
\\
&=n\cdot \frac{2}{3}\left( n-1 \right) !=\frac{2}{3}n!.
\nonumber
\end{align*}
故由数学归纳法可知结论对任意正整数都成立.
\end{solution}

\begin{exercise}\label{大/小拆分法例题1}
计算$n$阶行列式:
\begin{align*}
|\boldsymbol{A}|=\left| \begin{matrix}
a&		b&		\cdots&		b\\
b&		a&		\cdots&		b\\
\vdots&		\vdots&		&		\vdots\\
b&		b&		\cdots&		a\\
\end{matrix} \right|.
\nonumber
\end{align*}
\end{exercise}
\begin{note}
{\color{blue}\text{解法一(\hyperlink{大拆分法}{大拆分法}):}}
注意到
\begin{align*}
|\boldsymbol{A}|&=\left| \begin{matrix}
a&		b&		\cdots&		b\\
b&		a&		\cdots&		b\\
\vdots&		\vdots&		&		\vdots\\
b&		b&		\cdots&		a\\
\end{matrix} \right|=\left| \begin{matrix}
b+\left( a-b \right)&		b+0&		\cdots&		b+0\\
b+0&		b+\left( a-b \right)&		\cdots&		b+0\\
\vdots&		\vdots&		&		\vdots\\
b+0&		b+0&		\cdots&		b+\left( a-b \right)\\
\end{matrix} \right|
\\
&=\left| \begin{matrix}
a-b&		0&		\cdots&		0\\
0&		a-b&		\cdots&		0\\
\vdots&		\vdots&		&		\vdots\\
0&		0&		\cdots&		a-b\\
\end{matrix} \right|+\sum_{i=1}^n{A_i}=\left( a-b \right) ^n+\sum_{i=1}^n{A_i}.
\end{align*}
其中$A_i$是第$i$行元素全为$b$,主对角元素除了$( i,i )$元外都为$a-b$,其他元素都为0的$n$阶行列式.

又因为
\begin{align*}
A_i=\begin{array}{l}
1\\
\vdots\\
i\\
\vdots\\
n\\
\end{array}\left| \begin{matrix}
a-b&		&		&		&		\\
&		\ddots&		&		&		\\
b&		\cdots&		b&		\cdots&		b\\
&		&		&		\ddots&		\\
&		&		&		&		a-b\\
\end{matrix} \right|=b\left( a-b \right) ^{n-1},i=1,2,\cdots ,n. 
\end{align*}
所以
\begin{align*}
|\boldsymbol{A}|=\left( a-b \right) ^n+\sum_{i=1}^n{A_i}=\left( a-b \right) ^n+nb\left( a-b \right) ^{n-1}=\left[ a+\left( n-1 \right) b \right] \left( a-b \right) ^{n-1}.
\end{align*}
{\color{blue}\text{解法二(\hyperlink{小拆分法}{小拆分法}):}}
记原行列式为$D_n$,其中$n$为原行列式的阶数.则将原行列式按第一列拆开为两个行列式得
\begin{align*}
D_n&=\left| \begin{matrix}
a&		b&		\cdots&		b\\
b&		a&		\cdots&		b\\
\vdots&		\vdots&		&		\vdots\\
b&		b&		\cdots&		a\\
\end{matrix} \right|=\left| \begin{matrix}
b+\left( a-b \right)&		b&		\cdots&		b\\
b+0&		a&		\cdots&		b\\
\vdots&		\vdots&		&		\vdots\\
b+0&		b&		\cdots&		a\\
\end{matrix} \right|=\left| \begin{matrix}
b&		b&		\cdots&		b\\
b&		a&		\cdots&		b\\
\vdots&		\vdots&		&		\vdots\\
b&		b&		\cdots&		a\\
\end{matrix} \right|+\left| \begin{matrix}
a-b&		b&		\cdots&		b\\
0&		a&		\cdots&		b\\
\vdots&		\vdots&		&		\vdots\\
0&		b&		\cdots&		a\\
\end{matrix} \right|
\\
&=\left| \begin{matrix}
b&		b&		\cdots&		b\\
0&		a-b&		\cdots&		0\\
\vdots&		\vdots&		&		\vdots\\
0&		0&		\cdots&		a-b\\
\end{matrix} \right|+\left( a-b \right) D_{n-1}=b\left( a-b \right) ^{n-1}+\left( a-b \right) D_{n-1}.
(n\ge2)
\end{align*}
从而由上式递推可得
\begin{align*}
D_n&=b\left( a-b \right) ^{n-1}+\left( a-b \right) D_{n-1}
\\
&=b\left( a-b \right) ^{n-1}+\left( a-b \right) \left[ b\left( a-b \right) ^{n-2}+\left( a-b \right) D_{n-2} \right] 
=2b\left( a-b \right) ^{n-1}+\left( a-b \right) ^2D_{n-2}
\\
&=\cdots =\left( n-1 \right) b\left( a-b \right) ^{n-1}+\left( a-b \right) ^{n-1}D_1
\\
&=\left( n-1 \right) b\left( a-b \right) ^{n-1}+\left( a-b \right) ^{n-1}a
\\
&=\left[ a+\left( n-1 \right) b \right] \left( a-b \right) ^{n-1}.
\end{align*}
{\color{blue}\text{解法三(\hyperlink{行列式计算:求和法}{求和法}):}}
\begin{align*}
|\boldsymbol{A}|&=\left| \begin{matrix}
a&		b&		\cdots&		b\\
b&		a&		\cdots&		b\\
\vdots&		\vdots&		&		\vdots\\
b&		b&		\cdots&		a\\
\end{matrix} \right|\xlongequal[i=2,3,\cdots ,n]{j_i+j_1}\left| \begin{matrix}
a+\left( n-1 \right) b&		b&		\cdots&		b\\
a+\left( n-1 \right) b&		a&		\cdots&		b\\
\vdots&		\vdots&		&		\vdots\\
a+\left( n-1 \right) b&		b&		\cdots&		a\\
\end{matrix} \right|=\left[ a+\left( n-1 \right) b \right] \left| \begin{matrix}
1&		b&		\cdots&		b\\
1&		a&		\cdots&		b\\
\vdots&		\vdots&		&		\vdots\\
1&		b&		\cdots&		a\\
\end{matrix} \right|
\\
&\xlongequal[i=2,3,\cdots ,n]{-r_1+r_i}\left[ a+\left( n-1 \right) b \right] \left| \begin{matrix}
1&		b&		\cdots&		b\\
0&		a-b&		\cdots&		0\\
\vdots&		\vdots&		&		\vdots\\
0&		0&		\cdots&		a-b\\
\end{matrix} \right|=\left[ a+\left( n-1 \right) b \right] \left( a-b \right) ^{n-1}.
\end{align*}
{\color{blue}\text{解法四(\hyperlink{"爪"型行列式的推广}{"爪"型行列式的推广}):}}
\begin{align*}
|\boldsymbol{A}|&=\left| \begin{matrix}
a&		b&		\cdots&		b\\
b&		a&		\cdots&		b\\
\vdots&		\vdots&		&		\vdots\\
b&		b&		\cdots&		a\\
\end{matrix} \right|\xlongequal[i=2,3,\cdots ,n]{-r_1+r_i}\left| \begin{matrix}
a&		b&		\cdots&		b\\
b-a&		a-b&		\cdots&		0\\
\vdots&		\vdots&		&		\vdots\\
b-a&		0&		\cdots&		a-b\\
\end{matrix} \right|
\\
&\xlongequal[i=2,3,\cdots ,n]{-j_i+j_1}\left| \begin{matrix}
a-\left( n-1 \right) b&		b&		\cdots&		b\\
0&		a-b&		\cdots&		0\\
\vdots&		\vdots&		&		\vdots\\
0&		0&		\cdots&		a-b\\
\end{matrix} \right|=\left[ a-\left( n-1 \right) b \right] \left( a-b \right) ^{n-1}.
\end{align*}
\end{note}

\begin{exercise}
计算$n$阶行列式:
\begin{align*}
|\boldsymbol{A}|=\left| \begin{matrix}
a&		b&		\cdots&		b\\
c&		a&		\cdots&		b\\
\vdots&		\vdots&		&		\vdots\\
c&		c&		\cdots&		a\\
\end{matrix} \right|.
\nonumber
\end{align*}
\end{exercise}
\begin{solution}
{\color{blue}\text{解法一(\hyperlink{大拆分法}{大拆分法}):}}
令
\begin{align*}
|\boldsymbol{A}(t)|=\left| \begin{matrix}
a+t&		b+t&		\cdots&		b+t\\
c+t&		a+t&		\cdots&		b+t\\
\vdots&		\vdots&		&		\vdots\\
c+t&		c+t&		\cdots&		a+t\\
\end{matrix} \right|=|\boldsymbol{A}|+tu,  u=\sum_{i,j=1}^n{A_{ij}.}
\nonumber
\end{align*}
当$t=-b$时,可得
\begin{align*}
|\boldsymbol{A}(-b)|=\left| \begin{matrix}
a-b&		0&		\cdots&		0\\
c-b&		a-b&		\cdots&		0\\
\vdots&		\vdots&		&		\vdots\\
c-b&		c-b&		\cdots&		a-b\\
\end{matrix} \right|=|\boldsymbol{A}|-bu=(a-b)^n.
\end{align*}
当$t=-c$时,可得
\begin{align*}
|\boldsymbol{A}(-c)|=\left| \begin{matrix}
a-c&		b-c&		\cdots&		b-c\\
0&		a-c&		\cdots&		b-c\\
\vdots&		\vdots&		&		\vdots\\
0&		0&		\cdots&		a-c\\
\end{matrix} \right|=|\boldsymbol{A}|-cu=(a-c)^n.
\end{align*}
若$b\ne c$,则联立上面两式可得
\begin{align*}
\left| \boldsymbol{A} \right|=\frac{b\left( a-c \right) ^n-c\left( a-b \right) ^n}{b-c}.
\nonumber
\end{align*}
若$b=c$,则由练习\ref{大/小拆分法例题1}可知
\begin{align*}
|\boldsymbol{A}|=\left[ a+\left( n-1 \right) b \right] \left( a-b \right) ^{n-1}.
\nonumber
\end{align*}
{\color{blue}\text{解法二(\hyperlink{小拆分法}{小拆分法}):}}
记原行列式为$D_n$,其中$n$为原行列式的阶数.则将原行列式分别按第一行、第一列拆开为两个行列式得
\begin{align*}
D_n&=\left| \begin{matrix}
a&		b&		\cdots&		b\\
c&		a&		\cdots&		b\\
\vdots&		\vdots&		&		\vdots\\
c&		c&		\cdots&		a\\
\end{matrix} \right|=\left| \begin{matrix}
b+\left( a-b \right)&		b+0&		\cdots&		b+0\\
c&		a&		\cdots&		b\\
\vdots&		\vdots&		&		\vdots\\
c&		c&		\cdots&		a\\
\end{matrix} \right|=\left| \begin{matrix}
b&		b&		\cdots&		b\\
c&		a&		\cdots&		b\\
\vdots&		\vdots&		&		\vdots\\
c&		c&		\cdots&		a\\
\end{matrix} \right|+\left| \begin{matrix}
a-b&		0&		\cdots&		0\\
c&		a&		\cdots&		b\\
\vdots&		\vdots&		&		\vdots\\
c&		c&		\cdots&		a\\
\end{matrix} \right|
\\
&=b\left| \begin{matrix}
1&		1&		\cdots&		1\\
c&		a&		\cdots&		b\\
\vdots&		\vdots&		&		\vdots\\
c&		c&		\cdots&		a\\
\end{matrix} \right|+\left( a-b \right) D_{n-1}=b\left| \begin{matrix}
1&		1&		\cdots&		1\\
0&		a-c&		\cdots&		b-c\\
\vdots&		\vdots&		&		\vdots\\
0&		0&		\cdots&		a-c\\
\end{matrix} \right|+\left( a-b \right) D_{n-1}
\\
&=b\left( a-c \right) ^{n-1}++\left( a-b \right) D_{n-1}.\left( n\ge 2 \right) 
\end{align*}
\begin{align*}
D_n&=\left| \begin{matrix}
a&		b&		\cdots&		b\\
c&		a&		\cdots&		b\\
\vdots&		\vdots&		&		\vdots\\
c&		c&		\cdots&		a\\
\end{matrix} \right|=\left| \begin{matrix}
c+\left( a-c \right)&		b&		\cdots&		b\\
c+0&		a&		\cdots&		b\\
\vdots&		\vdots&		&		\vdots\\
c+0&		c&		\cdots&		a\\
\end{matrix} \right|=\left| \begin{matrix}
c&		b&		\cdots&		b\\
c&		a&		\cdots&		b\\
\vdots&		\vdots&		&		\vdots\\
c&		c&		\cdots&		a\\
\end{matrix} \right|+\left| \begin{matrix}
a-c&		b&		\cdots&		b\\
0&		a&		\cdots&		b\\
\vdots&		\vdots&		&		\vdots\\
0&		c&		\cdots&		a\\
\end{matrix} \right|
\\
&=c\left| \begin{matrix}
1&		b&		\cdots&		b\\
1&		a&		\cdots&		b\\
\vdots&		\vdots&		&		\vdots\\
1&		c&		\cdots&		a\\
\end{matrix} \right|+\left( a-c \right) D_{n-1}=c\left| \begin{matrix}
1&		0&		\cdots&		0\\
1&		a-b&		\cdots&		0\\
\vdots&		\vdots&		&		\vdots\\
1&		c-b&		\cdots&		a-b\\
\end{matrix} \right|+\left( a-c \right) D_{n-1}
\\
&=c\left( a-b \right) ^{n-1}++\left( a-c \right) D_{n-1}.\left( n\ge 2 \right) 
\end{align*}
若$b\ne c$,则联立上面两式可得
\begin{align*}
\left| \boldsymbol{A} \right|=D_n=\frac{b\left( a-c \right) ^n-c\left( a-b \right) ^n}{b-c}.
\nonumber
\end{align*}
若$b=c$,则由上面式子递推可得
\begin{align*}
\left| \boldsymbol{A} \right|=D_n&=b\left( a-b \right) ^{n-1}+\left( a-b \right) D_{n-1}
\\
&=b\left( a-b \right) ^{n-1}+\left( a-b \right) \left[ b\left( a-b \right) ^{n-2}+\left( a-b \right) D_{n-2} \right] 
=2b\left( a-b \right) ^{n-1}+\left( a-b \right) ^2D_{n-2}
\\
&=\cdots =\left( n-1 \right) b\left( a-b \right) ^{n-1}+\left( a-b \right) ^{n-1}D_1
\\
&=\left( n-1 \right) b\left( a-b \right) ^{n-1}+\left( a-b \right) ^{n-1}a
\\
&=\left[ a+\left( n-1 \right) b \right] \left( a-b \right) ^{n-1}.
\end{align*}

当$b=c$时,也可以由练习\ref{大/小拆分法例题1}可知
\begin{align*}
|\boldsymbol{A}|=\left[ a+\left( n-1 \right) b \right] \left( a-b \right) ^{n-1}.
\nonumber
\end{align*}
\end{solution}

\begin{exercise}
设\(f_1(x), f_2(x), \cdots, f_n(x)\)是次数不超过\(n - 2\)的多项式,求证:对任意\(n\)个数\(a_1, a_2, \cdots, a_n\),均有
\begin{align*}
\begin{vmatrix}
f_1(a_1) & f_2(a_1) & \cdots & f_n(a_1) \\
f_1(a_2) & f_2(a_2) & \cdots & f_n(a_2) \\
\vdots & \vdots & \ddots & \vdots \\
f_1(a_n) & f_2(a_n) & \cdots & f_n(a_n)
\end{vmatrix} = 0.
\end{align*}
\end{exercise}
\begin{proof}
{\color{blue}\text{证法一(\hyperlink{大拆分法}{大拆分法}):}}
因为\(f_k(x)(1 \leq k \leq n)\)的次数不超过\(n - 2\),所以它们都是单项式\(1,x,\cdots,x^{n - 2}\)的线性组合.将原行列式中每一列的多项式都按这\(n - 1\)个单项式进行拆分,最后得到至多$(n-1)!$个简单行列式之和,这些行列式中每一列的多项式只是单项式.由于每个简单行列式都有\(n\)列,根据抽屉原理,每个简单行列式中至少有两列是共用同一个单项式(可能相差一个常系数),于是这两列成比例,从而所有这样的简单行列式都等于零,因此原行列式也等于零.

{\color{blue}\text{证法二(\hyperlink{多项式根的有限性}{多项式根的有限性}):}}
令$f\left( x \right) =\left| \begin{matrix}
f_1(x)&		f_2(a_1)&		\cdots&		f_n(a_1)\\
f_1(x)&		f_2(a_2)&		\cdots&		f_n(a_2)\\
\vdots&		\vdots&		\ddots&		\vdots\\
f_1(x)&		f_2(a_n)&		\cdots&		f_n(a_n)\\
\end{matrix} \right|$,则将$f(x)$按第一列展开得到
\begin{align*}
f\left( x \right) =k_1f_1\left( x \right) +k_2f_2\left( x \right) +\cdots +k_nf_n\left( x \right) .
\end{align*}
其中$k_i$为行列式$f\left( x \right)$的第$\left( i,1 \right)$元素的代数余子式,$i=1,2,\cdots ,n$.

注意$k_i$与$x$无关,均为常数.若$f(x)$不恒为0,则又因为\(f_k(x)(1 \leq k \leq n)\)的次数不超过\(n - 2\),所以$degf(x)\le n-2$.
但是,注意到$f(a_2)=f(a_3)=\cdots=f(a_n)=0$,即$f(x)$有$n-1$个根.于是由\hyperlink{余数定理}{余数定理}可知,$\left( x-a_2 \right) \cdots \left( x-a_n \right) |f\left( x \right)$.从而$n-1=deg\left( x-a_2 \right) \cdots \left( x-a_n \right) \ge degf\left( x \right)$.这与$degf(x)\le n-2$矛盾.故$f(x)\equiv 0$,当然也有$f(a_1)=0$.

{\color{blue}证法三:}

设多项式
\[
f_k(x)=c_{k,n - 2}x^{n - 2}+\cdots+c_{k1}x + c_{k0},1\leq k\leq n.
\]
则有如下的矩阵分解:
\[
\begin{pmatrix}
f_1(a_1) & f_2(a_1) & \cdots & f_n(a_1)\\
f_1(a_2) & f_2(a_2) & \cdots & f_n(a_2)\\
\vdots & \vdots & & \vdots\\
f_1(a_n) & f_2(a_n) & \cdots & f_n(a_n)
\end{pmatrix}
=
\begin{pmatrix}
1 & a_1 & \cdots & a_1^{n - 2}\\
1 & a_2 & \cdots & a_2^{n - 2}\\
\vdots & \vdots & & \vdots\\
1 & a_n & \cdots & a_n^{n - 2}
\end{pmatrix}
\begin{pmatrix}
c_{10} & c_{20} & \cdots & c_{n0}\\
c_{11} & c_{21} & \cdots & c_{n1}\\
\vdots & \vdots & & \vdots\\
c_{1,n - 2} & c_{2,n - 2} & \cdots & c_{n,n - 2}
\end{pmatrix}.
\]
注意到上式右边的两个矩阵分别是\(n\times(n - 1)\)和\((n - 1)\times n\)矩阵,故由\hyperref[theorem:Cauchy-Binet公式]{Cauchy - Binet公式}马上得到左边矩阵的行列式值等于零.
\end{proof}

\begin{proposition}\label{proposition:小拆分法经典例题}
计算$n$阶行列式:
\begin{align}
D_n=\left| \begin{matrix}
x_1&		y&		y&		\cdots&		y&		y\\
z&		x_2&		y&		\cdots&		y&		y\\
z&		z&		x_3&		\cdots&		y&		y\\
\vdots&		\vdots&		\vdots&		&		\vdots&		\vdots\\
z&		z&		z&		\cdots&		x_{n-1}&		y\\
z&		z&		z&		\cdots&		z&		x_n\\
\end{matrix} \right|.
\nonumber
\end{align}
\end{proposition}
\begin{solution}(\hyperref[小拆分法]{小拆分法})
对第$n$列进行拆分即可得到递推式:
(对第1或n行(或列)拆分都可以得到相同结果)
\begin{align}
&D_n=\left| \begin{matrix}
x_1&		y&		y&		\cdots&		y&		y+0\\
z&		x_2&		y&		\cdots&		y&		y+0\\
z&		z&		x_3&		\cdots&		y&		y+0\\
\vdots&		\vdots&		\vdots&		&		\vdots&		\vdots\\
z&		z&		z&		\cdots&		x_{n-1}&		y+0\\
z&		z&		z&		\cdots&		z&		y+x_n-y\\
\end{matrix} \right|=\left| \begin{matrix}
x_1&		y&		y&		\cdots&		y&		y\\
z&		x_2&		y&		\cdots&		y&		y\\
z&		z&		x_3&		\cdots&		y&		y\\
\vdots&		\vdots&		\vdots&		&		\vdots&		\vdots\\
z&		z&		z&		\cdots&		x_{n-1}&		y\\
z&		z&		z&		\cdots&		z&		y\\
\end{matrix} \right|+\left| \begin{matrix}
x_1&		y&		y&		\cdots&		y&		0\\
z&		x_2&		y&		\cdots&		y&		0\\
z&		z&		x_3&		\cdots&		y&		0\\
\vdots&		\vdots&		\vdots&		&		\vdots&		\vdots\\
z&		z&		z&		\cdots&		x_{n-1}&		0\\
z&		z&		z&		\cdots&		z&		x_n-y\\
\end{matrix} \right|
\nonumber\\
&=\left| \begin{matrix}
x_1-z&		0&		0&		\cdots&		0&		0\\
0&		x_2-z&		0&		\cdots&		0&		0\\
0&		0&		x_3-z&		\cdots&		0&		0\\
\vdots&		\vdots&		\vdots&		&		\vdots&		\vdots\\
0&		0&		0&		\cdots&		x_{n-1}-z&		0\\
z&		z&		z&		\cdots&		z&		y\\
\end{matrix} \right|+\left( x_n-y \right) D_{n-1}=y\prod\limits_{i=1}^{n-1}{\left( x_i-z \right)}+\left( x_n-y \right) D_{n-1}.
\label{eq:递推式1.2}
\end{align}
将原行列式转置后,同理可得
\begin{align}
&D_n=D_{n}^{T}=\left| \begin{matrix}
x_1&		z&		z&		\cdots&		z&		z+0\\
y&		x_2&		z&		\cdots&		z&		z+0\\
y&		y&		x_3&		\cdots&		z&		z+0\\
\vdots&		\vdots&		\vdots&		&		\vdots&		\vdots\\
y&		y&		y&		\cdots&		x_{n-1}&		z+0\\
y&		y&		y&		\cdots&		y&		z+x_n-z\\
\end{matrix} \right|=\left| \begin{matrix}
x_1&		z&		z&		\cdots&		z&		z\\
y&		x_2&		z&		\cdots&		z&		z\\
y&		y&		x_3&		\cdots&		z&		z\\
\vdots&		\vdots&		\vdots&		&		\vdots&		\vdots\\
y&		y&		y&		\cdots&		x_{n-1}&		z\\
y&		y&		y&		\cdots&		y&		z\\
\end{matrix} \right|+\left| \begin{matrix}
x_1&		z&		z&		\cdots&		z&		0\\
y&		x_2&		z&		\cdots&		z&		0\\
y&		y&		x_3&		\cdots&		z&		0\\
\vdots&		\vdots&		\vdots&		&		\vdots&		\vdots\\
y&		y&		y&		\cdots&		x_{n-1}&		0\\
y&		y&		y&		\cdots&		y&		x_n-z\\
\end{matrix} \right|
\nonumber\\
&=\left| \begin{matrix}
x_1-y&		0&		0&		\cdots&		0&		0\\
0&		x_2-y&		0&		\cdots&		0&		0\\
0&		0&		x_3-y&		\cdots&		0&		0\\
\vdots&		\vdots&		\vdots&		&		\vdots&		\vdots\\
0&		0&		0&		\cdots&		x_{n-1}-y&		0\\
y&		y&		y&		\cdots&		y&		z\\
\end{matrix} \right|+\left( x_n-z \right) D_{n-1}^{T}=z\prod\limits_{i=1}^{n-1}{\left( x_i-y \right)}+\left( x_n-z \right) D_{n-1}.
\label{eq:递推式1.3}
\end{align}
若$z\ne y$,则联立\eqref{eq:递推式1.2}\eqref{eq:递推式1.3}式,解得
\begin{equation}
D_n=\frac{1}{z-y}\biggl[ z\prod\limits_{i=1}^n{(x_i}-y)-y\prod\limits_{i=1}^n{(x_i}-z) \biggr];
\nonumber
\end{equation}
若$z= y$,则由\eqref{eq:递推式1.2}式递推可得
\begin{equation}
\begin{split}
D_n&=y\prod\limits_{i=1}^{n-1}{\left( x_i-y \right)}+\left( x_n-y \right) D_{n-1}
\\
&=y\prod\limits_{i=1}^{n-1}{\left( x_i-y \right)}+\left( x_n-y \right) \left( y\prod\limits_{i=1}^{n-2}{\left( x_i-y \right)}+\left( x_{n-1}-y \right) D_{n-2} \right) 
\\
&=y\prod\limits_{j\ne n}^{}{\left( x_i-y \right)}+y\prod\limits_{j\ne n-1}^{}{\left( x_i-y \right)}+\left( x_n-y \right) \left( x_{n-1}-y \right) D_{n-2}
\\
&=\cdots =y\sum_{i=1}^n{\prod\limits_{j\ne i}{(x_j}}-y)+\prod\limits_{i=1}^n{(x_i}-y)D_0
\\
&=y\sum_{i=1}^n{\prod\limits_{j\ne i}{(x_j}}-y)+\prod\limits_{i=1}^n{(x_i}-y).
\end{split}
\nonumber
\end{equation}
\end{solution}

\begin{exercise}
求下列$n$阶行列式的值:
\begin{align*}
D_n = 
\begin{vmatrix}
1 + a_1^2 & a_1 a_2 & \cdots & a_1 a_n \\
a_2 a_1 & 1 + a_2^2 & \cdots & a_2 a_n \\
\vdots & \vdots & \ddots & \vdots \\
a_n a_1 & a_n a_2 & \cdots & 1 + a_n^2
\end{vmatrix}
\end{align*}
\end{exercise}
\begin{note}
本题行列式每行或每列求和后得到的结果不具备明显的规律性,故不适合使用\hyperref[行列式计算:求和法]{求和法}.

本题行列式难以找到合适的$t$对其进行\hyperref[大拆分法]{大拆分},故也不适合使用大拆分法.(并且因为难以找到合适的$t_i$,所以\hyperref[大拆分法的推广]{推广的大拆分}也不行)
\end{note}
\begin{solution}
(\hyperlink{小拆分法}{小拆分法})
将$D_n$最后一列拆成两列得
\begin{align*}
D_n&=\left| \begin{matrix}
1+a_{1}^{2}&		a_1a_2&		\cdots&		a_1a_n\\
a_2a_1&		1+a_{2}^{2}&		\cdots&		a_2a_n\\
\vdots&		\vdots&		\ddots&		\vdots\\
a_na_1&		a_na_2&		\cdots&		1+a_{n}^{2}\\
\end{matrix} \right|=\left| \begin{matrix}
1+a_{1}^{2}&		a_1a_2&		\cdots&		a_1a_n\\
a_2a_1&		1+a_{2}^{2}&		\cdots&		a_2a_n\\
\vdots&		\vdots&		\ddots&		\vdots\\
a_na_1&		a_na_2&		\cdots&		a_{n}^{2}\\
\end{matrix} \right|+\left| \begin{matrix}
1+a_{1}^{2}&		a_1a_2&		\cdots&		0\\
a_2a_1&		1+a_{2}^{2}&		\cdots&		0\\
\vdots&		\vdots&		\ddots&		\vdots\\
a_na_1&		a_na_2&		\cdots&		1\\
\end{matrix} \right|
\\
&=\left| \begin{matrix}
1+a_{1}^{2}&		a_1a_2&		\cdots&		a_1a_n\\
a_2a_1&		1+a_{2}^{2}&		\cdots&		a_2a_n\\
\vdots&		\vdots&		\ddots&		\vdots\\
a_na_1&		a_na_2&		\cdots&		a_{n}^{2}\\
\end{matrix} \right|+D_{n-1}.
\end{align*}
若$a_n\ne0$,则由上式可得
\begin{align*}
D_n=a_n\left| \begin{matrix}
1+a_{1}^{2}&		a_1a_2&		\cdots&		a_1\\
a_2a_1&		1+a_{2}^{2}&		\cdots&		a_2\\
\vdots&		\vdots&		\ddots&		\vdots\\
a_na_1&		a_na_2&		\cdots&		a_n\\
\end{matrix} \right|+D_{n-1}\xlongequal[i=1,2,\cdots ,n]{\text{对第一个行列式}:-a_ij_n+j_i}a_n\left| \begin{matrix}
1&		0&		\cdots&		a_1\\
0&		1&		\cdots&		a_2\\
\vdots&		\vdots&		\ddots&		\vdots\\
0&		0&		\cdots&		a_n\\
\end{matrix} \right|+D_{n-1}=a_{n}^{2}+D_{n-1}.\left( n\ge 2 \right) 
\end{align*}
若$a_n=0$,则上面第一个行列式等于0,进而$D_n=D_{n-1}(n\ge0)$.仍然满足上述递推式.

从而由上式递推可得
\begin{align*}
D_n=a_{n}^{2}+D_{n-1}=a_{n}^{2}+\left( a_{n-1}^{2}+D_{n-2} \right) =\cdots =\sum_{i=2}^n{a_{i}^{2}}+D_1=1+\sum_{i=1}^n{a_{i}^{2}}.
\end{align*}
\end{solution}

\begin{exercise}\label{Vandermode行列式三角函数例题}
求下列行列式的值:
\begin{align*}
|\boldsymbol{A}|=\left| \begin{matrix}
1&		\cos \theta _1&		\cos 2\theta _1&		\cdots&		\cos\mathrm{(}n-1)\theta _1\\
1&		\cos \theta _2&		\cos 2\theta _2&		\cdots&		\cos\mathrm{(}n-1)\theta _2\\
\vdots&		\vdots&		\vdots&		&		\vdots\\
1&		\cos \theta _n&		\cos 2\theta _n&		\cdots&		\cos\mathrm{(}n-1)\theta _n\\
\end{matrix} \right|.
\end{align*}
\end{exercise}
\begin{solution}
由De Moivre公式及二项式定理,可得
\begin{align*}
&\cos k\theta +\mathrm{i}\sin k\theta =(\cos \theta +\mathrm{i}\sin \theta )^k
\\
&=\cos ^k\theta +\mathrm{iC}_{k}^{1}\cos ^{k-1}\theta \sin \theta -\mathrm{C}_{k}^{2}\cos ^{k-2}\theta \sin ^2\theta +\mathrm{iC}_{k}^{3}\cos ^{k-3}\theta \sin ^3\theta -\cdots 
\\
&=\cos ^k\theta +\mathrm{iC}_{k}^{1}\cos ^{k-1}\theta \sin \theta -\mathrm{C}_{k}^{2}\cos ^{k-2}\theta \left( 1-\cos ^2\theta \right) +\mathrm{iC}_{k}^{3}\cos ^{k-3}\theta \sin ^3\theta -\cdots 
\end{align*}
比较实部可得
\begin{align*}
\cos k\theta& =\cos ^k\theta \left( 1+\mathrm{C}_{k}^{2}+\mathrm{C}_{k}^{4}+\cdots \right) -\mathrm{C}_{k}^{2}\cos ^{k-2}+\mathrm{C}_{k}^{4}\cos ^{k-4}-\cdots 
\\
&\hyperlink{组合式计算常用公式}{=}2^{k-1}\cos ^k\theta -\mathrm{C}_{k}^{2}\cos ^{k-2}+\mathrm{C}_{k}^{4}\cos ^{k-4}-\cdots 
\end{align*}

利用这个事实,依次将原行列式各列表示成\(\cos \theta _j\)(\(j = 2,3,\cdots,n\))的多项式.

再利用行列式的性质,可依次将第\(3,4,\cdots,n\)列消去除最高次项外的其他项,从而得到
\begin{align*}
|\boldsymbol{A}|&=\left| \begin{matrix}
1&		\cos \theta _1&		2\cos ^2\theta _1&		\cdots&		2^{n-2}\cos ^{n-1}\theta _1\\
1&		\cos \theta _2&		2\cos ^2\theta _2&		\cdots&		2^{n-2}\cos ^{n-1}\theta _2\\
\vdots&		\vdots&		\vdots&		&		\vdots\\
1&		\cos \theta _n&		2\cos ^2\theta _n&		\cdots&		2^{n-2}\cos ^{n-1}\theta _n\\
\end{matrix} \right|=2^{\frac{1}{2}(n-1)(n-2)}\left| \begin{matrix}
1&		\cos \theta _1&		\cos ^2\theta _1&		\cdots&		\cos ^{n-1}\theta _1\\
1&		\cos \theta _2&		\cos ^2\theta _2&		\cdots&		\cos ^{n-1}\theta _2\\
\vdots&		\vdots&		\vdots&		&		\vdots\\
1&		\cos \theta _n&		\cos ^2\theta _n&		\cdots&		\cos ^{n-1}\theta _n\\
\end{matrix} \right|
\\
&=2^{\frac{1}{2}(n-1)(n-2)}\prod_{1\le i<j\le n}{\left( \cos \theta _j-\cos \theta _i \right)}.
\end{align*}
\end{solution}
\begin{conclusion}
\hypertarget{组合式计算常用公式}{组合式计算常用公式:}

(1)$\mathrm{C}_{n}^{m}=\mathrm{C}_{n-1}^{m}+\mathrm{C}_{n-1}^{m-1}$

(2)$\mathrm{C}_{n}^{0}+\mathrm{C}_{n}^{2}+\cdots =\mathrm{C}_{n}^{1}+\mathrm{C}_{n}^{3}+\cdots =2^{n-1}$

证明:(1)\begin{align*}
\mathrm{C}_{n}^{m}&=\frac{n!}{m!\left( n-m \right) !}=\frac{\left( n-1 \right) !\left( n-m+m \right)}{m!\left( n-m \right) !}=\frac{\left( n-1 \right) !\left( n-m \right)}{m!\left( n-m \right) !}+\frac{\left( n-1 \right) !m}{m!\left( n-m \right) !}
\\
&=\frac{\left( n-1 \right) !}{m!\left( n-m-1 \right) !}+\frac{\left( n-1 \right) !}{\left( m-1 \right) !\left( n-m \right) !}=\mathrm{C}_{n-1}^{m}+\mathrm{C}_{n-1}^{m-1}
\end{align*}
(2)(i)当\(n\)为奇数时,由\(C_{n}^{m}=C_{n - 1}^{m - 1} + C_{n - 1}^{m}\),可得
\begin{align*}
&C_{n}^{0} + C_{n}^{2} + C_{n}^{4} \cdots + C_{n}^{n - 1} = C_{n - 1}^{0} + C_{n - 1}^{1} + C_{n - 1}^{2} + C_{n - 1}^{3} + C_{n - 1}^{4} \cdots + C_{n - 1}^{n - 2} + C_{n - 1}^{n - 1}
\\
&C_{n}^{1} + C_{n}^{3} + C_{n}^{5} \cdots + C_{n}^{n} = C_{n - 1}^{0} + C_{n - 1}^{1} + C_{n - 1}^{2} + C_{n - 1}^{3} + C_{n - 1}^{4} + C_{n - 1}^{5} + \cdots + C_{n - 1}^{n - 1} + C_{n - 1}^{n}
\end{align*}
由于\(C_{n - 1}^{n} = 0\),再对比上面两式每一项可知,上面两式相等.

而上面两式相加,得$
C_{n}^{0} + C_{n}^{1} + C_{n}^{2} \cdots + C_{n}^{n - 1} + C_{n}^{n} = (1 + 1)^n = 2^n.$

故\(C_{n}^{0} + C_{n}^{2} + C_{n}^{4} \cdots + C_{n}^{n - 1} = C_{n}^{1} + C_{n}^{3} + C_{n}^{5} \cdots + C_{n}^{n} = 2^{n - 1}\).

(ii)当\(n\)为偶数时,由\(C_{n}^{m} = C_{n - 1}^{m - 1} + C_{n - 1}^{m}\),可得
\begin{align*}
&C_{n}^{0} + C_{n}^{2} + C_{n}^{4} \cdots + C_{n}^{n} = C_{n - 1}^{0} + C_{n - 1}^{1} + C_{n - 1}^{2} + C_{n - 1}^{3} + C_{n - 1}^{4} \cdots + C_{n - 1}^{n - 1} + C_{n - 1}^{n} 
\\
&C_{n}^{1} + C_{n}^{3} + C_{n}^{5} \cdots + C_{n}^{n - 1} = C_{n - 1}^{0} + C_{n - 1}^{1} + C_{n - 1}^{2} + C_{n - 1}^{3} + C_{n - 1}^{4} + C_{n - 1}^{5} + \cdots + C_{n - 1}^{n - 2} + C_{n - 1}^{n - 1}
\end{align*}
由于\(C_{n - 1}^{n} = 0\),再对比上面两式每一项可知,上面两式相等.

而上面两式相加,得
$C_{n}^{0} + C_{n}^{1} + C_{n}^{2} \cdots + C_{n}^{n - 1} + C_{n}^{n} = (1 + 1)^n = 2^n.$

故\(C_{n}^{0} + C_{n}^{2} + C_{n}^{4} \cdots + C_{n}^{n - 1} = C_{n}^{1} + C_{n}^{3} + C_{n}^{5} \cdots + C_{n}^{n} = 2^{n - 1}\).

综上所述,\(C_{n}^{0} + C_{n}^{2} + \cdots = C_{n}^{1} + C_{n}^{3} + \cdots = 2^{n - 1}\). 
\end{conclusion}

\begin{exercise}
求下列行列式式的值:
\begin{align*}
|\boldsymbol{A}|=\left| \begin{matrix}
\sin \theta _1&		\sin 2\theta _1&		\cdots&		\sin n\theta _1\\
\sin \theta _2&		\sin 2\theta _2&		\cdots&		\sin n\theta _2\\
\vdots&		\vdots&		&		\vdots\\
\sin \theta _n&		\sin 2\theta _n&		\cdots&		\sin n\theta _n\\
\end{matrix} \right|.
\end{align*}
\end{exercise}
\begin{note}
可以利用\hyperref[Vandermode行列式三角函数例题]{上一题}类似的方法求解.但我们给出另外一种解法,目的是直接利用\hyperref[Vandermode行列式三角函数例题]{上一题}的结论.
\end{note}
\begin{solution}
根据和差化积公式,可得
\begin{align*}
\sin k\theta -\sin \left( k-2 \right) \theta =2\sin \theta \cos \left( k-1 \right) \theta ,k=2,3,\cdots ,n.
\end{align*}
再结合上一题结论,可得
\begin{align*}
|\boldsymbol{A}|&=\left| \begin{matrix}
\sin \theta _1&		\sin 2\theta _1&		\cdots&		\sin n\theta _1\\
\sin \theta _2&		\sin 2\theta _2&		\cdots&		\sin n\theta _2\\
\vdots&		\vdots&		&		\vdots\\
\sin \theta _n&		\sin 2\theta _n&		\cdots&		\sin n\theta _n\\
\end{matrix} \right|=\left| \begin{matrix}
\sin \theta _1&		2\sin \theta _1\cos \theta _1&		\cdots&		2\sin \theta _1\cos \left( n-1 \right) \theta _1\\
\sin \theta _2&		2\sin \theta _2\cos \theta _2&		\cdots&		2\sin \theta _2\cos \left( n-1 \right) \theta _2\\
\vdots&		\vdots&		&		\vdots\\
\sin \theta _n&		2\sin \theta _n\cos \theta _n&		\cdots&		2\sin \theta _n\cos \left( n-1 \right) \theta _n\\
\end{matrix} \right|
\\
&=2^{n-1}\prod_{i=1}^n{\sin \theta _i}\left| \begin{matrix}
\cos \theta _1&		\cos 2\theta _1&		\cdots&		\cos (n-1)\theta _1\\
\cos \theta _2&		\cos 2\theta _2&		\cdots&		\cos (n-1)\theta _2\\
\vdots&		\vdots&		&		\vdots\\
\cos \theta _n&		\cos 2\theta _n&		\cdots&		\cos (n-1)\theta _n\\
\end{matrix} \right|=2^{\frac{1}{2}\left( n-2 \right) \left( n-1 \right) +n-1}\prod_{i=1}^n{\sin \theta _i}\prod_{1\le i<j\le n}{\left( \cos \theta _j-\cos \theta _i \right)}
\\
&=2^{\frac{1}{2}n\left( n-1 \right)}\prod_{i=1}^n{\sin \theta _i}\prod_{1\le i<j\le n}{\left( \cos \theta _j-\cos \theta _i \right)}.
\end{align*}
\end{solution}

\begin{exercise}\label{升阶法的应用(1)例题}
计算$n$阶行列式:
\begin{align*}
|\boldsymbol{A}|=\left| \begin{matrix}
1+x_1&		1+x_2&		\cdots&		1+x_{1}^{n}\\1+x_{1}^{2}
&		1+x_{2}^{2}&		\cdots&		1+x_{2}^{n}\\
\vdots&		\vdots&		&		\vdots\\
1+x_n&		1+x_{n}^{2}&		\cdots&		1+x_{n}^{n}\\
\end{matrix} \right|.
\end{align*}
\end{exercise}
\begin{note}
本题也可以使用\hyperref[大拆分法]{大拆分法}进行求解.但我们以本题为例介绍利用\textbf{升阶法}计算行列式.
\end{note}
\begin{solution}
{\color{blue}解法一\hyperref[行列式计算:升阶法]{升阶法}:}
\begin{align*}
|\boldsymbol{A}|&=\left| \begin{matrix}
1&		0&		0&		\cdots&		0\\
1&		1+x_1&		1+x_{1}^{2}&		\cdots&		1+x_{1}^{n}\\
1&		1+x_2&		1+x_{2}^{2}&		\cdots&		1+x_{2}^{n}\\
\vdots&		\vdots&		\vdots&		&		\vdots\\
1&		1+x_n&		1+x_{n}^{2}&		\cdots&		1+x_{n}^{n}\\
\end{matrix} \right|=\left| \begin{matrix}
1&		-1&		-1&		\cdots&		-1\\
1&		x_1&		x_{1}^{2}&		\cdots&		x_{1}^{n}\\
1&		x_2&		x_{2}^{2}&		\cdots&		x_{2}^{n}\\
\vdots&		\vdots&		\vdots&		&		\vdots\\
1&		x_n&		x_{n}^{2}&		\cdots&		x_{n}^{n}\\
\end{matrix} \right|
\\
&\xlongequal{\hyperlink{小拆分法}{\text{小拆分法}}}\left| \begin{matrix}
2&		0&		0&		\cdots&		0\\
1&		x_1&		x_{1}^{2}&		\cdots&		x_{1}^{n}\\
1&		x_2&		x_{2}^{2}&		\cdots&		x_{2}^{n}\\
\vdots&		\vdots&		\vdots&		&		\vdots\\
1&		x_n&		x_{n}^{2}&		\cdots&		x_{n}^{n}\\
\end{matrix} \right|+\left| \begin{matrix}
-1&		-1&		-1&		\cdots&		-1\\
1&		x_1&		x_{1}^{2}&		\cdots&		x_{1}^{n}\\
1&		x_2&		x_{2}^{2}&		\cdots&		x_{2}^{n}\\
\vdots&		\vdots&		\vdots&		&		\vdots\\
1&		x_n&		x_{n}^{2}&		\cdots&		x_{n}^{n}\\
\end{matrix} \right|
\\
&=2\left| \begin{matrix}
x_1&		x_{1}^{2}&		\cdots&		x_{1}^{n}\\
x_2&		x_{2}^{2}&		\cdots&		x_{2}^{n}\\
\vdots&		\vdots&		&		\vdots\\
x_n&		x_{n}^{2}&		\cdots&		x_{n}^{n}\\
\end{matrix} \right|-\left| \begin{matrix}
1&		1&		1&		\cdots&		1\\
1&		x_1&		x_{1}^{2}&		\cdots&		x_{1}^{n}\\
1&		x_2&		x_{2}^{2}&		\cdots&		x_{2}^{n}\\
\vdots&		\vdots&		\vdots&		&		\vdots\\
1&		x_n&		x_{n}^{2}&		\cdots&		x_{n}^{n}\\
\end{matrix} \right|
\\
&=2x_1x_2\cdots x_n\left| \begin{matrix}
1&		x_1&		\cdots&		x_{1}^{n-1}\\
1&		x_2&		\cdots&		x_{2}^{n-1}\\
\vdots&		\vdots&		&		\vdots\\
1&		x_n&		\cdots&		x_{n}^{n-1}\\
\end{matrix} \right|-\left( x_1-1 \right) \left( x_2-1 \right) \cdots \left( x_n-1 \right) \prod_{1\le i<j\le n}{\left( x_j-x_i \right)}
\\
&=2x_1x_2\cdots x_n\prod_{1\le i<j\le n}{\left( x_j-x_i \right)}-\left( x_1-1 \right) \left( x_2-1 \right) \cdots \left( x_n-1 \right) \prod_{1\le i<j\le n}{\left( x_j-x_i \right)}
\\
&=\left[ 2x_1x_2\cdots x_n-\left( x_1-1 \right) \left( x_2-1 \right) \cdots \left( x_n-1 \right) \right] \prod_{1\le i<j\le n}{\left( x_j-x_i \right)}.
\end{align*}

{\color{blue}解法二(\hyperref[大拆分法]{大拆分法}):}
设\(\vert\boldsymbol{B}(t)\vert=\left|\begin{matrix}
x_1 + t & x_{1}^{2} + t & \cdots & x_{1}^{n} + t\\
x_2 + t & x_{2}^{2} + t & \cdots & x_{2}^{n} + t\\
\vdots & \vdots &  & \vdots\\
x_n + t & x_{n}^{2} + t & \cdots & x_{n}^{n} + t
\end{matrix}\right|\),且\(B_{ij}\)是\(\vert\boldsymbol{B}(0)\vert\)的第\((i,j)\)元素的代数余子式.

根据行列式的性质将\(\vert\boldsymbol{A}\vert\)每一列都拆分成两列,然后按\(t\)所在的列展开得到
\begin{gather*}
\vert\boldsymbol{A}\vert=\vert\boldsymbol{B}(1)\vert=\vert\boldsymbol{B}(0)\vert+\sum_{i,j = 1}^{n}B_{ij},
\\
\vert\boldsymbol{B}(-1)\vert=\vert\boldsymbol{B}(0)\vert-\sum_{i,j = 1}^{n}B_{ij}.
\end{gather*}
于是\(\vert\boldsymbol{A}\vert = 2\vert\boldsymbol{B}(0)\vert - \vert\boldsymbol{B}(-1)\vert\).注意到
\begin{align*}
\vert\boldsymbol{B}(0)\vert=\left|\begin{matrix}
x_1 & x_{1}^{2} & \cdots & x_{1}^{n}\\
x_2 & x_{2}^{2} & \cdots & x_{2}^{n}\\
\vdots & \vdots &  & \vdots\\
x_n & x_{n}^{2} & \cdots & x_{n}^{n}
\end{matrix}\right|=x_1x_2\cdots x_n\left|\begin{matrix}
1 & x_1 & \cdots & x_{1}^{n}\\
1 & x_2 & \cdots & x_{2}^{n}\\
\vdots & \vdots &  & \vdots\\
1 & x_n & \cdots & x_{n}^{n}
\end{matrix}\right|=x_1x_2\cdots x_n\prod_{1\leqslant i < j\leqslant n}(x_j - x_i). 
\end{align*}
又由行列式性质可得
\begin{align*}
\vert\boldsymbol{B}(-1)\vert&=\left|\begin{matrix}
x_1 - 1 & x_{1}^{2} - 1 & \cdots & x_{1}^{n} - 1\\
x_2 - 1 & x_{2}^{2} - 1 & \cdots & x_{2}^{n} - 1\\
\vdots & \vdots &  & \vdots\\
x_n - 1 & x_{n}^{2} - 1 & \cdots & x_{n}^{n} - 1
\end{matrix}\right|
=(x_1 - 1)(x_2 - 1)\cdots (x_n - 1)\left|\begin{matrix}
1 & x_1 + 1 & \cdots & x_{1}^{n - 1} + x_{1}^{n - 2}\cdots + x_1 + 1\\
1 & x_2 + 1 & \cdots & x_{2}^{n - 1} + x_{2}^{n - 2}\cdots + x_2 + 1\\
\vdots & \vdots &  & \vdots\\
1 & x_n + 1 & \cdots & x_{n}^{n - 1} + x_{n}^{n - 2}\cdots + x_n + 1
\end{matrix}\right|
\\
&=(x_1 - 1)(x_2 - 1)\cdots (x_n - 1)\left|\begin{matrix}
1 & x_1 & \cdots & x_{1}^{n - 1}\\
1 & x_2 & \cdots & x_{2}^{n - 1}\\
\vdots & \vdots &  & \vdots\\
1 & x_n & \cdots & x_{n}^{n - 1}
\end{matrix}\right|
=(x_1 - 1)(x_2 - 1)\cdots (x_n - 1)\prod_{1\leqslant i < j\leqslant n}(x_j - x_i).
\end{align*}
故可得
\begin{align*}
\vert\boldsymbol{A}\vert &= 2\vert\boldsymbol{B}(0)\vert - \vert\boldsymbol{B}(-1)\vert
=2x_1x_2\cdots x_n\prod_{1\leqslant i < j\leqslant n}(x_j - x_i)-(x_1 - 1)(x_2 - 1)\cdots (x_n - 1)\prod_{1\leqslant i < j\leqslant n}(x_j - x_i)
\\
&=\left[2x_1x_2\cdots x_n-(x_1 - 1)(x_2 - 1)\cdots (x_n - 1)\right]\prod_{1\leqslant i < j\leqslant n}(x_j - x_i).
\end{align*}
\end{solution}
\begin{conclusion}\label{行列式计算:升阶法}
\hypertarget{行列式计算:升阶法}{\textbf{升阶法:}}
将原行列式加上一行和一列使得到到新行列式的阶数比原行列式要高一阶.

\textbf{升阶法的应用:}

(1)当原行列式每一行具有相同的结构时,我们可以在原行列式的基础上加上一行和一列,新加上的一列和一行需要满足:新的一列除了与新的一行交叉位置的元素为1外其余全为0(这样才能保证新的行列式按新的一行或一列展开后与原行列式相同),并且新加上的一行除1以外其他位置的元素就取原行列式中每一行所具有的相同结构(这样可以利用行列式的性质将每一行中的相同的结构减去,进而达到简化原行列式的目的).具体例子见练习\ref{升阶法的应用(1)例题}.

(2)当原行列式是我们由熟悉的行列式去掉某一行、或某一列、或某一行和一列得到的,我们可以在原行列式的基础上补充上缺少的那一行和一列,再进行计算得到新行列式的式子.再将新行列式按照新添加的一行或一列展开得到的对应元素乘与其对应的代数余子式,而新添加的一行和一列交叉位置的元素对应的余子式就是原行列式,最后两边式子比较系数一般就能得到原行列式的值.
具体例子见练习\ref{升阶法的应用(2)例题}.
\end{conclusion}

\begin{exercise}\label{升阶法的应用(2)例题}
求下列$n$阶行列式的值($1\le i\le n-1$):
\begin{align*}
|\boldsymbol{A}|=\left| \begin{matrix}
1&		x_1&		\cdots&		x_{1}^{i-1}&		x_{1}^{i+1}&		\cdots&		x_{1}^{n}\\
1&		x_2&		\cdots&		x_{2}^{i-1}&		x_{2}^{i+1}&		\cdots&		x_{2}^{n}\\
\vdots&		\vdots&		&		\vdots&		\vdots&		&		\vdots\\
1&		x_n&		\cdots&		x_{n}^{i-1}&		x_{n}^{i+1}&		\cdots&		x_{n}^{n}\\
\end{matrix} \right|.
\end{align*}
\end{exercise}
\begin{solution}
令
\begin{align}\label{eq:1.4(Vandermode行列式升阶法)}
|\boldsymbol{B}|=\left|\begin{matrix}
1 & x_1 & \cdots & x_{1}^{i - 1} & x_{1}^{i} & x_{1}^{i + 1} & \cdots & x_{1}^{n}\\
1 & x_2 & \cdots & x_{2}^{i - 1} & x_{2}^{i} & x_{2}^{i + 1} & \cdots & x_{2}^{n}\\
\vdots & \vdots &  & \vdots & \vdots & \vdots &  & \vdots\\
1 & x_n & \cdots & x_{n}^{i - 1} & x_{n}^{i} & x_{n}^{i + 1} & \cdots & x_{n}^{n}\\
1 & y & \cdots & y^{i - 1} & y^i & y^{i + 1} & \cdots & y^n
\end{matrix}\right|=(y - x_1)(y - x_2)\cdots (y - x_n)\prod_{1\leqslant i < j\leqslant n}(x_j - x_i).
\end{align}
而上式右边是关于\(y\)的\(n\)次多项式,并且其\(y^i\)前的系数是
\begin{align*}
\sum_{1\leqslant k_1 < k_2 < \cdots < k_{n - i}\leqslant n}(-1)^{n - i}x_{k_1}x_{k_2}\cdots x_{k_{n - i}}\prod_{1\leqslant i < j\leqslant n}(x_j - x_i).
\end{align*}
将\(|\boldsymbol{B}|\)按最后一行展开,得
\begin{align*}
|\boldsymbol{B}|=A_{n1} + A_{n2}y + \cdots + A_{ni}y^i + \cdots + A_{nn}y^n, 
\end{align*}
其中\(A_{nk}\)为\(|\boldsymbol{B}|\)的\((n,k)\)位置元素的代数余子式,\(k = 1,2,\cdots,n\).

注意到\(A_{nk}\)均与\(y\)无关.因此\(|\boldsymbol{B}|\)作为关于\(y\)的\(n\)次多项式,其\(y^i\)前的系数是
\begin{align*}
A_{ni}=(-1)^{n + 1 + i + 1}|\boldsymbol{A}|=(-1)^{n + i}|\boldsymbol{A}|.
\end{align*}
再结合\eqref{eq:1.4(Vandermode行列式升阶法)}式,可知
\begin{align*}
(-1)^{n + i}|\boldsymbol{A}|=\sum_{1\leqslant k_1 < k_2 < \cdots < k_{n - i}\leqslant n}(-1)^{n - i}x_{k_1}x_{k_2}\cdots x_{k_{n - i}}\prod_{1\leqslant i < j\leqslant n}(x_j - x_i). 
\end{align*}
故\(|\boldsymbol{A}|=x_{k_1}x_{k_2}\cdots x_{k_{n - i}}\prod_{1\leqslant i < j\leqslant n}(x_j - x_i)\).
\end{solution}

\begin{exercise}
求下列$n$阶行列式的值,其中$a_i\ne 0(1\le i\le n)$:
\begin{align*}
|\boldsymbol{A}|=\left| \begin{matrix}
0&		a_1+a_2&		\cdots&		a_1+a_{n-1}&		a_1+a_n\\
a_2+a_1&		0&		\cdots&		a_2+a_{n-1}&		a_2+a_n\\
\vdots&		\vdots&		&		\vdots&		\vdots\\
a_{n-1}+a_1&		a_{n-1}+a_2&		\cdots&		0&		a_{n-1}+a_n\\
a_n+a_1&		a_n+a_2&		\cdots&		a_n+a_{n-1}&		0\\
\end{matrix} \right|.
\end{align*}
\end{exercise}
\begin{note}
{\color{blue}解法一}中不仅使用了\hyperlink{行列式计算:升阶法}{升阶法}还使用了\hyperref[proposition:分块"爪"型行列式]{分块"爪"型行列式的计算方法}.观察到各行各列有不同的公共项,因此可以利用升阶法将各行各列的公共项消去.
\end{note}
\begin{solution}
{\color{blue}解法一(\hyperlink{行列式计算:升阶法}{升阶法}):}
\begin{align*}
&|\boldsymbol{A}|\xlongequal[]{\text{升阶}}\left| \begin{matrix}
1&		-a_1&		-a_2&		\cdots&		-a_{n-1}&		-a_n\\
0&		0&		a_1+a_2&		\cdots&		a_1+a_{n-1}&		a_1+a_n\\
0&		a_2+a_1&		0&		\cdots&		a_2+a_{n-1}&		a_2+a_n\\
\vdots&		\vdots&		\vdots&		&		\vdots&		\vdots\\
0&		a_{n-1}+a_1&		a_{n-1}+a_2&		\cdots&		0&		a_{n-1}+a_n\\
0&		a_n+a_1&		a_n+a_2&		\cdots&		a_n+a_{n-1}&		0\\
\end{matrix} \right|
\\
&\xlongequal[i=1,2,\cdots ,n+1]{r_1+r_i}\left| \begin{matrix}
1&		-a_1&		-a_2&		\cdots&		-a_{n-1}&		-a_n\\
1&		-a_1&		a_1&		\cdots&		a_1&		a_1\\
1&		a_2&		-a_2&		\cdots&		a_2&		a_2\\
\vdots&		\vdots&		\vdots&		&		\vdots&		\vdots\\
1&		a_{n-1}&		a_{n-1}&		\cdots&		-a_{n-1}&		a_{n-1}\\
1&		a_n&		a_n&		\cdots&		a_n&		-a_n\\
\end{matrix} \right|\xlongequal[]{\text{升阶}}\left| \begin{matrix}
1&		0&		0&		0&		\cdots&		0&		0\\
0&		1&		-a_1&		-a_2&		\cdots&		-a_{n-1}&		-a_n\\
-a_1&		1&		-a_1&		a_1&		\cdots&		a_1&		a_1\\
-a_2&		1&		a_2&		-a_2&		\cdots&		a_2&		a_2\\
\vdots&		\vdots&		\vdots&		\vdots&		&		\vdots&		\vdots\\
-a_{n-1}&		1&		a_{n-1}&		a_{n-1}&		\cdots&		-a_{n-1}&		a_{n-1}\\
-a_n&		1&		a_n&		a_n&		\cdots&		a_n&		-a_n\\
\end{matrix} \right|
\\
&\xlongequal[i=1,3,4\cdots ,n+2]{j_1+j_i}\left| \begin{matrix}
1&		0&		1&		1&		\cdots&		1&		1\\
0&		1&		-a_1&		-a_2&		\cdots&		-a_{n-1}&		-a_n\\
-a_1&		1&		-2a_1&		0&		\cdots&		0&		0\\
-a_2&		1&		0&		-2a_2&		\cdots&		0&		0_2\\
\vdots&		\vdots&		\vdots&		\vdots&		&		\vdots&		\vdots\\
-a_{n-1}&		1&		0&		0&		\cdots&		-2a_{n-1}&		0\\
-a_n&		1&		0&		0&		\cdots&		0&		-2a_n\\
\end{matrix} \right|
\\
&\xlongequal[i=3,4\cdots ,n+2]{\begin{array}{c}
-\frac{1}{2}j_i+j_1\\
\frac{1}{2a_{i-2}}j_i+j_2\\
\end{array}}\left| \begin{matrix}
1-\frac{n}{2}&		\frac{S}{2}&		1&		1&		\cdots&		1&		1\\
\frac{T}{2}&		1-\frac{n}{2}&		-a_1&		-a_2&		\cdots&		-a_{n-1}&		-a_n\\
0&		0&		-2a_1&		0&		\cdots&		0&		0\\
0&		0&		0&		-2a_2&		\cdots&		0&		0_2\\
\vdots&		\vdots&		\vdots&		\vdots&		&		\vdots&		\vdots\\
0&		0&		0&		0&		\cdots&		-2a_{n-1}&		0\\
0&		0&		0&		0&		\cdots&		0&		-2a_n\\
\end{matrix} \right|. 
\end{align*}
其中\(S = a_1 + a_2 + \cdots + a_n\),\(T = \frac{1}{a_1} + \frac{1}{a_2} + \cdots + \frac{1}{a_n}\).注意到上述行列式是分块上三角行列式,从而可得
\begin{align*}
\vert\boldsymbol{A}\vert = (-2)^n\prod_{i = 1}^n{a_i} \cdot \frac{(n - 2)^2 - ST}{4} = (-2)^{n - 2}\prod_{i = 1}^n{a_i}[(n - 2)^2 - (\sum_{i = 1}^n{a_i})(\sum_{i = 1}^n{\frac{1}{a_i}})].
\end{align*}

{\color{blue}解法二(\hyperref[proposition:直接计算两个矩阵和的行列式]{直接计算两个矩阵和的行列式}):}

设\(\boldsymbol{B}=\left(\begin{matrix}
2a_1 & a_1 + a_2 & \cdots & a_1 + a_n\\
a_2 + a_1 & 2a_2 & \cdots & a_2 + a_n\\
\vdots & \vdots &  & \vdots\\
a_n + a_1 & a_n + a_2 & \cdots & 2a_n
\end{matrix}\right)\),\(\boldsymbol{C}=\left(\begin{matrix}
-2a_1 &  &  & \\
& -2a_2 &  & \\
&  & \ddots & \\
&  &  & -2a_n
\end{matrix}\right)\),则\(|\boldsymbol{A}| = |\boldsymbol{B} + \boldsymbol{C}|\).

从而利用\hyperref[proposition:直接计算两个矩阵和的行列式]{直接计算两个矩阵和的行列式}的结论得到
\begin{align}\label{eq(行列式):1.5式}
&|\boldsymbol{A}| = |\boldsymbol{B}| + |\boldsymbol{C}| + \sum_{1\leqslant k\leqslant n - 1}\left(\sum_{\begin{array}{c}
1\leqslant i_1 < i_2 < \cdots < i_k\leqslant n\\
1\leqslant j_1 < j_2 < \cdots < j_k\leqslant n
\end{array}}\boldsymbol{B}\left(\begin{matrix}
i_1 & i_2 & \cdots & i_k\\
j_1 & j_2 & \cdots & j_k
\end{matrix}\right)\widehat{\boldsymbol{C}}\left(\begin{matrix}
i_1 & i_2 & \cdots & i_k\\
j_1 & j_2 & \cdots & j_k
\end{matrix}\right)\right)
\end{align}
其中\(\widehat{\boldsymbol{C}}\left(\begin{matrix}
i_1 & i_2 & \cdots & i_k\\
j_1 & j_2 & \cdots & j_k
\end{matrix}\right)\)是\(\boldsymbol{C}\left(\begin{matrix}
i_1 & i_2 & \cdots & i_k\\
j_1 & j_2 & \cdots & j_k
\end{matrix}\right)\)的代数余子式.

我们先来计算\(\boldsymbol{B}\left(\begin{matrix}
i_1 & i_2 & \cdots & i_k\\
j_1 & j_2 & \cdots & j_k
\end{matrix}\right)\),\(k = 1,2,\cdots,n\).拆分\(\boldsymbol{B}\left(\begin{matrix}
i_1 & i_2 & \cdots & i_k\\
j_1 & j_2 & \cdots & j_k
\end{matrix}\right)\)的第一列得到
\begin{align*}
&\boldsymbol{B}\left(\begin{matrix}
i_1 & i_2 & \cdots & i_k\\
j_1 & j_2 & \cdots & j_k
\end{matrix}\right) = \left|\begin{matrix}
a_{i_1} + a_{j_1} & a_{i_1} + a_{j_2} & \cdots & a_{i_1} + a_{j_k}\\
a_{i_2} + a_{j_1} & a_{i_2} + a_{j_2} & \cdots & a_{i_2} + a_{j_k}\\
\vdots & \vdots &  & \vdots\\
a_{i_k} + a_{j_1} & a_{i_k} + a_{j_2} & \cdots & a_{i_k} + a_{j_k}
\end{matrix}\right|
\\
&=\left|\begin{matrix}
a_{i_1} & a_{i_1} + a_{j_2} & \cdots & a_{i_1} + a_{j_k}\\
a_{i_2} & a_{i_2} + a_{j_2} & \cdots & a_{i_2} + a_{j_k}\\
\vdots & \vdots &  & \vdots\\
a_{i_k} & a_{i_k} + a_{j_2} & \cdots & a_{i_k} + a_{j_k}
\end{matrix}\right| + \left|\begin{matrix}
a_{j_1} & a_{i_1} + a_{j_2} & \cdots & a_{i_1} + a_{j_k}\\
a_{j_1} & a_{i_2} + a_{j_2} & \cdots & a_{i_2} + a_{j_k}\\
\vdots & \vdots &  & \vdots\\
a_{j_1} & a_{i_k} + a_{j_2} & \cdots & a_{i_k} + a_{j_k}
\end{matrix}\right|
\\
&=\left|\begin{matrix}
a_{i_1} & a_{j_2} & \cdots & a_{j_k}\\
a_{i_2} & a_{j_2} & \cdots & a_{j_k}\\
\vdots & \vdots &  & \vdots\\
a_{i_k} & a_{j_2} & \cdots & a_{j_k}
\end{matrix}\right| + \left|\begin{matrix}
a_{j_1} & a_{i_1} & \cdots & a_{i_1}\\
a_{j_1} & a_{i_2} & \cdots & a_{i_2}\\
\vdots & \vdots &  & \vdots\\
a_{j_1} & a_{i_k} & \cdots & a_{i_k}
\end{matrix}\right|
\end{align*}
因此当\(k\geqslant 3\)时,\(\boldsymbol{B}\left(\begin{matrix}
i_1 & i_2 & \cdots & i_k\\
j_1 & j_2 & \cdots & j_k
\end{matrix}\right) = 0\);
当\(k = 2\)时,\(\boldsymbol{B}\left(\begin{matrix}
i_1 & i_2 & \cdots & i_k\\
j_1 & j_2 & \cdots & j_k
\end{matrix}\right) = \boldsymbol{B}\left(\begin{matrix}
i_1 & i_2\\
j_1 & j_2
\end{matrix}\right) = \left|\begin{matrix}
a_{i_1} & a_{j_2}\\
a_{i_2} & a_{j_2}
\end{matrix}\right| + \left|\begin{matrix}
a_{j_1} & a_{i_1}\\
a_{j_1} & a_{i_2}
\end{matrix}\right| = (a_{i_1}a_{j_2} - a_{i_2}a_{j_2})(a_{i_2}a_{j_1} - a_{i_1}a_{j_1})\);
当\(k = 1\)时,\(\boldsymbol{B}\left(\begin{matrix}
i_1 & i_2 & \cdots & i_k\\
j_1 & j_2 & \cdots & j_k
\end{matrix}\right) = \boldsymbol{B}\left(\begin{array}{c}
i_1\\
j_1
\end{array}\right) = a_{i_1} + a_{j_1}\).

又注意到\(|\boldsymbol{C}|\)只有主子式非零,而其主子式\(\boldsymbol{C}\left(\begin{matrix}
i_1 & i_2 & \cdots & i_k\\
i_1 & i_2 & \cdots & i_k
\end{matrix}\right) = (-2)^ka_{i_1}a_{i_2}\cdots a_{i_k}\).
于是当\(\exists m\in \{1,2,\cdots,k\}\),使得\(i_m\neq j_m\)时,\(\widehat{\boldsymbol{C}}\left(\begin{matrix}
i_1 & i_2 & \cdots & i_k\\
j_1 & j_2 & \cdots & j_k
\end{matrix}\right) = 0\);
当\(i_m\neq j_m\),\(m = 1,2,\cdots,k\)时,\(\widehat{\boldsymbol{C}}\left(\begin{matrix}
i_1 & i_2 & \cdots & i_k\\
j_1 & j_2 & \cdots & j_k
\end{matrix}\right) = \widehat{\boldsymbol{C}}\left(\begin{matrix}
i_1 & i_2 & \cdots & i_k\\
i_1 & i_2 & \cdots & i_k
\end{matrix}\right) = (-2)^{n - k}a_1\cdots \hat{a}_{i_1}\cdots \hat{a}_{i_2}\cdots \hat{a}_{i_k}\cdots a_n\).

故当\(n\geqslant 3\)时,\eqref{eq(行列式):1.5式}式可化为
\begin{align*}
&|\boldsymbol{A}| = |\boldsymbol{B}| + |\boldsymbol{C}| + \sum_{1\leqslant k\leqslant n - 1}\left(\sum_{\substack{
1\leqslant i_1 < i_2 < \cdots < i_k\leqslant n\\
1\leqslant j_1 < j_2 < \cdots < j_k\leqslant n
}}\boldsymbol{B}\left(\begin{matrix}
i_1 & i_2 & \cdots & i_k\\
j_1 & j_2 & \cdots & j_k
\end{matrix}\right)\widehat{\boldsymbol{C}}\left(\begin{matrix}
i_1 & i_2 & \cdots & i_k\\
j_1 & j_2 & \cdots & j_k
\end{matrix}\right)\right) 
\\
&= |\boldsymbol{C}| + \sum_{\substack{
1\leqslant i_1\leqslant n\\
1\leqslant j_1\leqslant n
}}\boldsymbol{B}\left(\substack{
i_1\\
j_1
}\right)\widehat{\boldsymbol{C}}\left(\begin{array}{c}
i_1\\
j_1
\end{array}\right) + \sum_{\substack{
1\leqslant i_1 < i_2\leqslant n\\
1\leqslant j_1 < j_2\leqslant n
}}\boldsymbol{B}\left(\begin{matrix}
i_1 & i_2\\
j_1 & j_2
\end{matrix}\right)\widehat{\boldsymbol{C}}\left(\begin{matrix}
i_1 & i_2\\
j_1 & j_2
\end{matrix}\right)
\\
&= |\boldsymbol{C}| + \sum_{1\leqslant i_1\leqslant n}\boldsymbol{B}\left(\begin{array}{c}
i_1\\
i_1
\end{array}\right)\widehat{\boldsymbol{C}}\left(\begin{array}{c}
i_1\\
i_1
\end{array}\right) + \sum_{1\leqslant i_1 < i_2\leqslant n}\boldsymbol{B}\left(\begin{matrix}
i_1 & i_2\\
i_1 & i_2
\end{matrix}\right)\widehat{\boldsymbol{C}}\left(\begin{matrix}
i_1 & i_2\\
i_1 & i_2
\end{matrix}\right)
= |\boldsymbol{C}| + \sum_{1\leqslant i\leqslant n}\boldsymbol{B}\left(\begin{matrix}
i\\
i
\end{matrix}
\right)\widehat{\boldsymbol{C}}\left(\begin{matrix}
i\\
i
\end{matrix}
\right) + \sum_{1\leqslant i < j\leqslant n}\boldsymbol{B}\left(\begin{matrix}
i & j\\
i & j
\end{matrix}\right)\widehat{\boldsymbol{C}}\left(\begin{matrix}
i & j\\
i & j
\end{matrix}\right)
\\
&= (-2)^na_1a_2\cdots a_n + \sum_{1\leqslant i\leqslant n}2a_i(-2)^{n - 1}a_1\cdots \hat{a}_i\cdots a_n
+\sum_{1\leqslant i < j\leqslant n}[(a_ia_j - a_{j}^{2})(a_ia_j - a_{i}^{2})(-2)^{n - 2}a_1\cdots \hat{a}_i\cdots \hat{a}_j\cdots a_n]
\\
&= (-2)^na_1a_2\cdots a_n - (-2)^n\sum_{1\leqslant i\leqslant n}a_1a_2\cdots \cdots a_n
+ (-2)^{n - 2}\sum_{1\leqslant i < j\leqslant n}[-(a_i - a_j)^2a_1\cdots \hat{a}_i\cdots \hat{a}_j\cdots a_n]
\\
&= (-2)^na_1a_2\cdots a_n - (-2)^nna_1a_2\cdots \cdots a_n
- (-2)^{n - 2}\sum_{1\leqslant i < j\leqslant n}[(a_i - a_j)^2a_1\cdots \hat{a}_i\cdots \hat{a}_j\cdots a_n]
\\
&= (-2)^n\prod_{i = 1}^n{a_i}(1 - n) - (-2)^{n - 2}\prod_{i = 1}^n{a_i}\sum_{1\leqslant i < j\leqslant n}\frac{(a_i - a_j)^2}{a_{i}a_{j}}
\\
& = (-2)^{n - 2}\prod_{i = 1}^n{a_i}[(n - 2)^2 - (\sum_{i = 1}^n{a_i})(\sum_{i = 1}^n{\frac{1}{a_i}})]
\\
&=(-2)^n\prod_{i=1}^n{a_i(1}-n)-(-2)^{n-2}\prod_{i=1}^n{a_i\sum_{1\leqslant i<j\leqslant n}{\frac{(a_i-a_j)^2}{a_ia_j}}}
\\
&=\left( -2 \right) ^{n-2}\prod_{i=1}^n{a_i}\left[ 4-4n-\sum_{1\leqslant i<j\leqslant n}{\frac{(a_i-a_j)^2}{a_ia_j}} \right] 
\\
&=\left( -2 \right) ^{n-2}\prod_{i=1}^n{a_i}\left[ 4-4n-\sum_{1\leqslant i<j\leqslant n}{\left( \frac{a_j}{a_i}+\frac{a_i}{a_j}-2 \right)} \right] 
\\
&=\left( -2 \right) ^{n-2}\prod_{i=1}^n{a_i}\left[ 4-4n-\sum_{\substack{
1\leqslant i,j\leqslant n\\
i\ne j\\
}}{\frac{a_i}{a_j}}+\sum_{1\leqslant i<j\leqslant n}{2} \right] 
\\
&=\left( -2 \right) ^{n-2}\prod_{i=1}^n{a_i}\left[ 4-4n-\left( \sum_{1\leqslant i,j\leqslant n}{\frac{a_i}{a_j}}-\sum_{i=1}^n{\frac{a_i}{a_i}} \right) +\sum_{i=1}^{n-1}{\sum_{j=i+1}^n{2}} \right] 
\\
&=\left( -2 \right) ^{n-2}\prod_{i=1}^n{a_i}\left[ 4-4n-\left( \sum_{1\leqslant i,j\leqslant n}{\frac{a_i}{a_j}}-n \right) +2\sum_{i=1}^{n-1}{\left( n-i \right)} \right] 
\\
&=\left( -2 \right) ^{n-2}\prod_{i=1}^n{a_i}\left[ 4-4n+n+n\left( n-1 \right) -\sum_{i=1}^n{\sum_{j=1}^n{\frac{a_i}{a_j}}} \right] 
\\
&=\left( -2 \right) ^{n-2}\prod_{i=1}^n{a_i}\left[ n^2-4n+4-\sum_{i=1}^n{a_i\sum_{j=1}^n{\frac{1}{a_j}}} \right] 
\\
&=\left( -2 \right) ^{n-2}\prod_{i=1}^n{a_i[(n}-2)^2-(\sum_{i=1}^n{a_i)(\sum_{i=1}^n{\frac{1}{a_i})]}}.
\end{align*}
{\color{blue}解法三:}
令$\varLambda=\left( \begin{matrix}
a_1&		1\\
a_2&		1\\
\vdots&		\vdots\\
a_n&		1\\
\end{matrix} \right) ,B=\left( \begin{matrix}
-2a_1&		&		&		\\
&		-2a_2&		&		\\
&		&		\ddots&		\\
&		&		&		-2a_n\\
\end{matrix} \right) $,则
\begin{align*}
A=\left( \begin{matrix}
-2a_1&		&		&		\\
&		-2a_2&		&		\\
&		&		\ddots&		\\
&		&		&		-2a_n\\
\end{matrix} \right) +\left( \begin{matrix}
a_1&		1\\
a_2&		1\\
\vdots&		\vdots\\
a_n&		1\\
\end{matrix} \right) I_{2}^{-1}\left( \begin{matrix}
1&		1&		\cdots&		1\\
a_1&		a_2&		\cdots&		a_n\\
\end{matrix} \right) =B+\varLambda I_{2}^{-1}\varLambda '.
\end{align*}
于是由降价公式(打洞原理)我们有
\begin{align*}
|A|&=|I|\left|B + \Lambda I_{2}^{-1}\Lambda '\right|=\left|\begin{matrix}
I_2 & \Lambda '\\
\Lambda & B
\end{matrix}\right|=|B|\left|I_2 - \Lambda 'B^{-1}\Lambda\right|\\
&=\left|\begin{matrix}
-2a_1 & & & \\
& -2a_2 & & \\
& & \ddots & \\
& & & -2a_n
\end{matrix}\right|\cdot\left|I_2 - \left(\begin{matrix}
1 & 1 & \cdots & 1\\
a_1 & a_2 & \cdots & a_n
\end{matrix}\right)\left(\begin{matrix}
-\frac{1}{2a_1} & & & \\
& -\frac{1}{2a_2} & & \\
& & \ddots & \\
& & & -\frac{1}{2a_n}
\end{matrix}\right)\left(\begin{matrix}
a_1 & 1\\
a_2 & 1\\
\vdots & \vdots\\
a_n & 1
\end{matrix}\right)\right|\\
&=(-2)^n\prod_{i = 1}^n a_i\left|I_2 - \left(\begin{matrix}
-\frac{1}{2a_1} & -\frac{1}{2a_2} & \cdots & -\frac{1}{2a_n}\\
-\frac{1}{2} & -\frac{1}{2} & \cdots & -\frac{1}{2}
\end{matrix}\right)\left(\begin{matrix}
a_1 & 1\\
a_2 & 1\\
\vdots & \vdots\\
a_n & 1
\end{matrix}\right)\right|\\
&=(-2)^n\prod_{i = 1}^n a_i\left|I_2 - \left(\begin{matrix}
-\frac{n}{2} & -\frac{1}{2}\sum_{i = 1}^n\frac{1}{a_i}\\
-\frac{1}{2}\sum_{i = 1}^n a_i & -\frac{n}{2}
\end{matrix}\right)\right|=(-2)^n\prod_{i = 1}^n a_i\left|\begin{matrix}
\frac{n + 2}{2} & \frac{1}{2}\sum_{i = 1}^n\frac{1}{a_i}\\
\frac{1}{2}\sum_{i = 1}^n a_i & \frac{n + 2}{2}
\end{matrix}\right|\\
&=(-2)^{n - 2}\prod_{i = 1}^n a_i\left[(n + 2)^2 - \left(\sum_{i = 1}^n a_i\right)\left(\sum_{i = 1}^n\frac{1}{a_i}\right)\right].
\end{align*}
\end{solution}
\begin{conclusion}\label{对角矩阵行列式的子式和余子式}
\hypertarget{对角矩阵行列式的子式和余子式}{\textbf{对角矩阵行列式的子式和余子式:}}

设\(|\boldsymbol{A}|=\left|\begin{matrix}
a_1 & 0 & \cdots & 0\\
0 & a_2 & \cdots & 0\\
\vdots & \vdots & \ddots & \vdots\\
0 & 0 & \cdots & a_n
\end{matrix}\right|\),则其\(k\)阶子式\(\boldsymbol{A}\left(\begin{matrix}
i_1 & i_2 & \cdots & i_k\\
j_1 & j_2 & \cdots & j_k
\end{matrix}\right)\)除\(k\)阶主子式\(\boldsymbol{A}\left(\begin{matrix}
i_1 & i_2 & \cdots & i_k\\
i_1 & i_2 & \cdots & i_k
\end{matrix}\right)\)外都为零,其中\(k = 1,2,\cdots,n\).

记\(\widehat{\boldsymbol{A}}\left(\begin{matrix}
i_1 & i_2 & \cdots & i_k\\
j_1 & j_2 & \cdots & j_k
\end{matrix}\right)\)为\(\boldsymbol{A}\left(\begin{matrix}
i_1 & i_2 & \cdots & i_k\\
j_1 & j_2 & \cdots & j_k
\end{matrix}\right)\)的代数余子式(\(n - k\)阶).于是\(\widehat{\boldsymbol{A}}\left(\begin{matrix}
i_1 & i_2 & \cdots & i_k\\
j_1 & j_2 & \cdots & j_k
\end{matrix}\right)\)除\(\widehat{\boldsymbol{A}}\left(\begin{matrix}
i_1 & i_2 & \cdots & i_k\\
i_1 & i_2 & \cdots & i_k
\end{matrix}\right)\)外也都为零,其中\(k = 1,2,\cdots,n\).

并且
\begin{align*}
&\boldsymbol{A}\left(\begin{matrix}
i_1 & i_2 & \cdots & i_k\\
i_1 & i_2 & \cdots & i_k
\end{matrix}\right) = a_{i_1}a_{i_2}\cdots a_{i_k},
\\
&\widehat{\boldsymbol{A}}\left(\begin{matrix}
i_1 & i_2 & \cdots & i_k\\
i_1 & i_2 & \cdots & i_k
\end{matrix}\right) = a_1\cdots \hat{a}_{i_1}\cdots \hat{a}_{i_2}\cdots \hat{a}_{i_k}\cdots a_n\,
\end{align*}
其中\(k = 1,2,\cdots,n\).
\end{conclusion}

\begin{exercise}
若\(n\)阶行列式\(\vert A\vert\)中零元素的个数超过\(n^2 - n\)个,证明:\(\vert A\vert = 0\).
\end{exercise}
\begin{solution}
证明 由行列式的组合定义可得
\[
|A|=\sum_{1\leq k_1k_2\cdots k_n\leq n}(-1)^{\tau (k_1,k_2,\cdots,k_n)}a_{k_{11}}a_{k_{22}}\cdots a_{k_{nn}}
\]
由于\(|A|\)中零元素的个数超过\(n^2 - n\)个,故\(a_{k_{11}},a_{k_{22}},\cdots,a_{k_{nn}}\)中至少有一个为零,从而\(a_{k_{11}}a_{k_{22}}\cdots a_{k_{nn}} = 0\),因此\(|A| = 0\).如直接利用行列式的性质,也可以这样来证明:因为\(|A|\)中零元素的个数超过\(n^2 - n\)个,由抽屉原理可知,\(|A|\)至少有一列其零元素的个数大于等于\(\left\lfloor\frac{n^2 - n}{n}\right\rfloor+ 1=n\),即\(|A|\)至少有一列其元素全为零,因此\(|A| = 0\).
\end{solution}

\begin{exercise}
设\(A=(a_{ij})\)是\(n(n\geq2)\)阶非异整数方阵,满足对任意的\(i,j\),\(\vert A\vert\)均可整除\(a_{ij}\),证明:\(\vert A\vert=\pm1\).
\end{exercise}
\begin{solution}
\(\vert A\vert\)可整除每个元素\(a_{i,j}\),故由行列式的组合定义
\[
\sum_{1\le k_1,k_2,\cdots ,k_n\le n}{\left( -1 \right) ^{\tau \left( k_1k_2\cdots k_n \right)}a_{k_{11}}a_{k_{22}}\cdots a_{k_{nn}}}
\]
可知\(\vert A\vert^n\)可整除\(\vert A\vert\)中每个单项\(a_{k_{11}}a_{k_{22}}\cdots a_{k_{nn}}\),从而\(\vert A\vert^n\)可整除\(\vert A\vert\),即有\(\vert A\vert^{n - 1}\)可整除\(1\),于是\(\vert A\vert^{n - 1}=\pm1\).又由行列式的组合定义可知\(\vert A\vert\)是整数,从而只能是\(\vert A\vert=\pm1\).
\end{solution}

\begin{exercise}
利用行列式的$Laplace$定理证明恒等式:
\[
(ab' - a'b)(cd' - c'd)-(ac' - a'c)(bd' - b'd)+(ad' - a'd)(bc' - b'c)=0.
\]
\end{exercise}
\begin{solution}
显然下列行列式的值为零:
\begin{align*}
\left| \begin{matrix}
a&		a^{\prime}&		a&		a^{\prime}\\
b&		b^{\prime}&		b&		b^{\prime}\\
c&		c^{\prime}&		c&		c^{\prime}\\
d&		d^{\prime}&		d&		d^{\prime}\\
\end{matrix} \right|.
\end{align*}
利用$Laplace$定理按第一、二列展开得
\begin{align*}
\left| \begin{matrix}
a&		a^{\prime}&		a&		a^{\prime}\\
b&		b^{\prime}&		b&		b^{\prime}\\
c&		c^{\prime}&		c&		c^{\prime}\\
d&		d^{\prime}&		d&		d^{\prime}\\
\end{matrix} \right|&=\left( -1 \right) ^{1+2+1+2}\left| \begin{matrix}
a&		a^{\prime}\\
b&		b^{\prime}\\
\end{matrix} \right|\left| \begin{matrix}
c&		c^{\prime}\\
d&		d^{\prime}\\
\end{matrix} \right|+\left( -1 \right) ^{1+2+1+3}\left| \begin{matrix}
a&		a^{\prime}\\
c&		c^{\prime}\\
\end{matrix} \right|\left| \begin{matrix}
b&		b^{\prime}\\
d&		d^{\prime}\\
\end{matrix} \right|+\left( -1 \right) ^{1+2+1+4}\left| \begin{matrix}
a&		a^{\prime}\\
d&		d^{\prime}\\
\end{matrix} \right|\left| \begin{matrix}
b&		b^{\prime}\\
c&		c^{\prime}\\
\end{matrix} \right|
\\
&\quad+\left( -1 \right) ^{1+2+2+3}\left| \begin{matrix}
b&		b^{\prime}\\
c&		c^{\prime}\\
\end{matrix} \right|\left| \begin{matrix}
a&		a^{\prime}\\
d&		d^{\prime}\\
\end{matrix} \right|+\left( -1 \right) ^{1+2+2+4}\left| \begin{matrix}
b&		b^{\prime}\\
d&		d^{\prime}\\
\end{matrix} \right|\left| \begin{matrix}
a&		a^{\prime}\\
c&		c^{\prime}\\
\end{matrix} \right|
\\
&\quad+\left( -1 \right) ^{1+2+3+4}\left| \begin{matrix}
c&		c^{\prime}\\
d&		d^{\prime}\\
\end{matrix} \right|\left| \begin{matrix}
a&		a^{\prime}\\
b&		b^{\prime}\\
\end{matrix} \right|
\\
&=2\left| \begin{matrix}
a&		a^{\prime}\\
b&		b^{\prime}\\
\end{matrix} \right|\left| \begin{matrix}
c&		c^{\prime}\\
d&		d^{\prime}\\
\end{matrix} \right|-2\left| \begin{matrix}
a&		a^{\prime}\\
c&		c^{\prime}\\
\end{matrix} \right|\left| \begin{matrix}
b&		b^{\prime}\\
d&		d^{\prime}\\
\end{matrix} \right|+2\left| \begin{matrix}
a&		a^{\prime}\\
d&		d^{\prime}\\
\end{matrix} \right|\left| \begin{matrix}
b&		b^{\prime}\\
c&		c^{\prime}\\
\end{matrix} \right|=0.
\end{align*}
上式等价于
\begin{align*}
\left| \begin{matrix}
a&		a^{\prime}\\
b&		b^{\prime}\\
\end{matrix} \right|\left| \begin{matrix}
c&		c^{\prime}\\
d&		d^{\prime}\\
\end{matrix} \right|-\left| \begin{matrix}
a&		a^{\prime}\\
c&		c^{\prime}\\
\end{matrix} \right|\left| \begin{matrix}
b&		b^{\prime}\\
d&		d^{\prime}\\
\end{matrix} \right|+\left| \begin{matrix}
a&		a^{\prime}\\
d&		d^{\prime}\\
\end{matrix} \right|\left| \begin{matrix}
b&		b^{\prime}\\
c&		c^{\prime}\\
\end{matrix} \right|=0.
\end{align*}
整理可得
\begin{align*}
(ab' - a'b)(cd' - c'd)-(ac' - a'c)(bd' - b'd)+(ad' - a'd)(bc' - b'c)=0.
\end{align*}
\end{solution}

\begin{exercise}
求\(2n\)阶行列式的值(空缺处都是零):
\begin{align*}
\left| \begin{matrix}
a&		&		&		&		&		b\\
&		\ddots&		&		&		\begin{turn}{80}$\ddots$\end{turn}&		\\
&		&		a&		b&		&		\\
&		&		b&		a&		&		\\
&		\begin{turn}{80}$\ddots$\end{turn}&		&		&		\ddots&		\\
b&		&		&		&		&		a\\
\end{matrix} \right|.
\end{align*}
\end{exercise}
\begin{solution}
设原行列式为$D_{2n}$,其中$2n$为行列式的阶数.
不断用$Laplace$定理按第一行及最后一行展开,可得
\begin{align*}
D_{2n}=\left| \begin{matrix}
a&		&		&		&		&		b\\
&		\ddots&		&		&		\begin{turn}{80}$\ddots$\end{turn}&		\\
&		&		a&		b&		&		\\
&		&		b&		a&		&		\\
&		\begin{turn}{80}$\ddots$\end{turn}&		&		&		\ddots&		\\
b&		&		&		&		&		a\\
\end{matrix} \right|\xlongequal[]{\text{按第一行及最后一行展开}}\left| \begin{matrix}
a&		b\\
b&		a\\
\end{matrix} \right|D_{2n-2}=\left( a^2-b^2 \right) D_{2\left( n-1 \right)}.
\end{align*}
进而,由上述递推式可得
\begin{align*}
D_{2n}&=\left( a^2-b^2 \right) D_{2\left( n-1 \right)}=\left( a^2-b^2 \right) ^2D_{2\left( n-2 \right)}=\cdots =\left( a^2-b^2 \right) ^{n-1}D_2
\\
&=\left( a^2-b^2 \right) ^{n-1}\left| \begin{matrix}
a&		b\\
b&		a\\
\end{matrix} \right|=\left( a^2-b^2 \right) ^n.
\end{align*}
\end{solution}

\begin{exercise}
求下列$n$阶行列式的值:
\[
\left| A \right|=\begin{vmatrix}
(x - a_1)^2 & a_2^2 & \cdots & a_n^2 \\
a_1^2 & (x - a_2)^2 & \cdots & a_n^2 \\
\vdots & \vdots & \ddots & \vdots \\
a_1^2 & a_2^2 & \cdots & (x - a_n)^2
\end{vmatrix}.
\]
\end{exercise}
\begin{note}
注意到这个行列式每行元素除了主对角元素外,其余位置元素都相同.因此这个行列式是\hyperref["爪"型行列式的推广]{推广的"爪"型行列式}.
\end{note}
\begin{solution}
\begin{align*}
&\left| A \right|=\left| \begin{matrix}
(x-a_1)^2&		a_{2}^{2}&		\cdots&		a_{n}^{2}\\
a_{1}^{2}&		(x-a_2)^2&		\cdots&		a_{n}^{2}\\
\vdots&		\vdots&		\ddots&		\vdots\\
a_{1}^{2}&		a_{2}^{2}&		\cdots&		(x-a_n)^2\\
\end{matrix} \right|=\left| \begin{matrix}
(x-a_1)^2&		a_{2}^{2}&		\cdots&		a_{n}^{2}\\
2a_1x-x^2&		x^2-2a_2x&		\cdots&		0\\
\vdots&		\vdots&		\ddots&		\vdots\\
2a_1x-x^2&		0&		\cdots&		x^2-2a_nx\\
\end{matrix} \right|
\\
&\xlongequal{\hyperref["爪"型行列式]{\text{"爪"型行列式}}}(x-a_1)^2\prod_{i=2}^n{\left( x^2-2a_ix \right)}-\sum_{i=2}^n{a_{i}^{2}\left( 2a_1x-x^2 \right) \left( x^2-2a_2x \right) \cdots \widehat{\left( x^2-2a_ix \right) }\cdots}\left( x^2-2a_nx \right) 
\\
&=(x-a_1)^2\prod_{i=2}^n{\left( x^2-2a_ix \right)}+\sum_{i=2}^n{a_{i}^{2}\left( x^2-2a_1x \right) \left( x^2-2a_2x \right) \cdots \widehat{\left( x^2-2a_ix \right) }\cdots}\left( x^2-2a_nx \right) 
\\
&=(x-a_1)^2\prod_{i=2}^n{\left( x^2-2a_ix \right)}+\sum_{i=2}^n{\left( x^2-2a_1x \right) \cdots \left( x^2-2a_{i-1}x \right) a_{i}^{2}\left( x^2-2a_{i+1}x \right) \cdots}\left( x^2-2a_nx \right) 
\\
&=\left[ \left( x^2-2a_1x \right) +a_{1}^{2} \right] \prod_{i=2}^n{\left( x^2-2a_ix \right)}+\sum_{i=2}^n{\left( x^2-2a_1x \right) \cdots \left( x^2-2a_{i-1}x \right) a_{i}^{2}\left( x^2-2a_{i+1}x \right) \cdots}\left( x^2-2a_nx \right) 
\\
&=\prod_{i=1}^n{\left( x^2-2a_ix \right)}+\sum_{i=1}^n{\left( x^2-2a_1x \right) \cdots \left( x^2-2a_{i-1}x \right) a_{i}^{2}\left( x^2-2a_{i+1}x \right) \cdots}\left( x^2-2a_nx \right).
\end{align*}
\end{solution}

\begin{exercise}
求下列行列式式的值:
\[
\left| \boldsymbol{A} \right|=\begin{vmatrix}
(a + b)^2 & c^2 & c^2 \\
a^2 & (b + c)^2 & a^2 \\
b^2 & b^2 & (c + a)^2
\end{vmatrix}.
\]
\end{exercise}
\begin{solution}
{\color{blue}解法一:}
\begin{align*}
&\left| \boldsymbol{A} \right|=\left| \begin{matrix}
(a+b)^2&		c^2&		c^2\\
a^2&		(b+c)^2&		a^2\\
b^2&		b^2&		(c+a)^2\\
\end{matrix} \right|\xlongequal[i=1,2]{-j_1+j_i}\left| \begin{matrix}
(a+b)^2-c^2&		c^2&		0\\
a^2-(b+c)^2&		(b+c)^2&		a^2-(b+c)^2\\
0&		b^2&		(c+a)^2-b^2\\
\end{matrix} \right|
\\
&=(a+b+c)^2\left| \begin{matrix}
a+b-c&		c^2&		0\\
a-b-c&		(b+c)^2&		a-b-c\\
0&		b^2&		a+c-b\\
\end{matrix} \right|\xlongequal[i=1,2]{-r_i+r_2}(a+b+c)^2\left| \begin{matrix}
a+b-c&		c^2&		0\\
-2b&		2bc&		-2c\\
0&		b^2&		a+c-b\\
\end{matrix} \right|
\\
&\xlongequal[\frac{b}{2}j_3+j_2]{\frac{c}{2}j_1+j_2}(a+b+c)^2\left| \begin{matrix}
a+b-c&		\frac{c}{2}\left( a+b+c \right)&		0\\
-2b&		0&		-2c\\
0&		\frac{b}{2}\left( a+b+c \right)&		a+c-b\\
\end{matrix} \right|=(a+b+c)^3\left| \begin{matrix}
a+b-c&		\frac{c}{2}&		0\\
-2b&		0&		-2c\\
0&		\frac{b}{2}&		a+c-b\\
\end{matrix} \right|
\\
&=2abc(a+b+c)^3.
\end{align*}

{\color{blue}解法二(求根法):}
\end{solution}

\begin{exercise}
证明:若一个\(n(n>1)\)阶行列式中元素或为\(1\)或为\(-1\),则其值必为偶数.
\end{exercise}
\begin{proof}
将该行列式的任意一行加到另一行上去得到的行列式有一行元素全是偶数(注意:零也是偶数),由行列式的基本性质知道,可将因子2提出,剩下的行列式的元素都是整数,其值也是整数,乘以2后必是偶数.
\end{proof}

\begin{exercise}
\(n\) 阶行列式\(\vert \boldsymbol{A}\vert\)的值为\(c\),若从第二列开始每一列加上它前面的一列,同时对第一列加上\(\vert \boldsymbol{A}\vert\)的第\(n\)列,求得到的新行列式\(\vert \boldsymbol{B}\vert\)的值.
\end{exercise}
\begin{solution}
\begin{align*}
\left| \boldsymbol{B} \right|&=\left| \boldsymbol{\alpha }_1+\boldsymbol{\alpha }_n,\boldsymbol{\alpha }_2+\boldsymbol{\alpha }_1,\cdots ,\boldsymbol{\alpha }_n+\boldsymbol{\alpha }_{n-1} \right|
\\
&=\left| \boldsymbol{\alpha }_1,\boldsymbol{\alpha }_2,\cdots ,\boldsymbol{\alpha }_n \right|+\left| \boldsymbol{\alpha }_n,\boldsymbol{\alpha }_1,\cdots ,\boldsymbol{\alpha }_{n-1} \right|
+\sum_{1\leqslant k\leqslant n-2}{\sum_{2\leq j_1\leq j_2\leq \cdots\leq j_k\leq n}{\begin{array}{c}
\begin{array}{c@{}c@{}c@{}c@{}c@{}c@{}c@{}c@{}c@{}c@{}c@{}}
& 1 & \cdots & j_1 &\cdots &j_2 &\cdots &j_k &\cdots &n \\
\left.\right|
&\boldsymbol{\alpha }_n,&\cdots ,&\boldsymbol{\alpha }_{j_1+1},&\cdots ,&\boldsymbol{\alpha }_{j_2+1},&\cdots ,&\boldsymbol{\alpha }_{j_k+1},&\cdots ,&\boldsymbol{\alpha }_{n-1}& \left|\right.
\end{array}\\
\\
\end{array}}}
\\
&\quad +\sum_{1\leqslant k\leqslant n-2}{\sum_{2\leq j_1\leq j_2\leq \cdots\leq j_k\leq n}{\begin{array}{c}
\begin{array}{c@{}c@{}c@{}c@{}c@{}c@{}c@{}c@{}c@{}c@{}c@{}}
& 1 & \cdots & j_1 &\cdots &j_2 &\cdots &j_k &\cdots &n \\
\left.\right|
&\boldsymbol{\alpha }_1,&\cdots ,&\boldsymbol{\alpha }_{j_1+1},&\cdots ,&\boldsymbol{\alpha }_{j_2+1},&\cdots ,&\boldsymbol{\alpha }_{j_k+1},&\cdots ,&\boldsymbol{\alpha }_{n-1}& \left|\right.
\end{array}\\
\\
\end{array}}}.
\\
&=\left| \boldsymbol{\alpha }_1,\boldsymbol{\alpha }_2,\cdots ,\boldsymbol{\alpha }_n \right|+\left| \boldsymbol{\alpha }_n,\boldsymbol{\alpha }_1,\cdots ,\boldsymbol{\alpha }_{n-1} \right|
\\
&=c+\left( -1 \right) ^{n-1}\left| \boldsymbol{\alpha }_1,\boldsymbol{\alpha }_2,\cdots ,\boldsymbol{\alpha }_n \right|
\\
&=c+\left( -1 \right) ^{n-1}c
\\
&=\begin{cases}
0 \,\,,n\text{为偶数}\\
2c,n\text{为奇数}\\
\end{cases}
\end{align*}
\end{solution}

\begin{exercise}
令
\[
\left( a_{1} a_{2} \cdots a_{n} \right) = 
\begin{vmatrix}
a_{1} & 1 &   &   &   \\
-1 & a_{2} & 1 &   &   \\
& -1 & a_{3} & \ddots &   \\
&   & \ddots & \ddots & 1 \\
&   &   & -1 & a_{n}
\end{vmatrix},
\]
证明关于连分数的如下等式成立:
\[
a_{1} + \frac{1}{a_{2} + \frac{1}{a_{3} + \cdots + \frac{1}{a_{n - 1} + \frac{1}{a_{n}}}}} = \frac{\left( a_{1} a_{2} \cdots a_{n} \right)}{\left( a_{2} a_{3} \cdots a_{n} \right)}.
\]
\end{exercise}
\begin{solution}
假设等式对$\forall n\leq k-1,k\in \mathbb{N}_+$都成立.则当$n=k$时,将行列式$(a_1a_2,\cdots,a_k)$按第一列展开得
\begin{align*}
\left( a_1a_2\cdots a_k \right) &=\left| \begin{matrix}
a_1&		1&		&		&		\\
-1&		a_2&		1&		&		\\
&		-1&		a_3&		\ddots&		\\
&		&		\ddots&		\ddots&		1\\
&		&		&		-1&		a_k\\
\end{matrix} \right|=a_1\left| \begin{matrix}
a_2&		1&		&		\\
-1&		a_3&		\ddots&		\\
&		\ddots&		\ddots&		1\\
&		&		-1&		a_k\\
\end{matrix} \right|+\left| \begin{matrix}
a_3&		1&		&		\\
-1&		a_4&		\ddots&		\\
&		\ddots&		\ddots&		1\\
&		&		-1&		a_k\\
\end{matrix} \right|
\\
&=a_1\left( a_2a_3\cdots a_k \right) +\left( a_3a_4\cdots a_k \right).
\end{align*}
从而
\begin{align*}
\frac{\left( a_1a_2\cdots a_k \right)}{\left( a_2a_3\cdots a_k \right)}=a_1+\frac{\left( a_3a_4\cdots a_k \right)}{\left( a_2a_3\cdots a_k \right)}=a_1+\frac{1}{\frac{\left( a_2a_3\cdots a_k \right)}{\left( a_3a_4\cdots a_k \right)}}.
\end{align*}
于是由归纳假设可知
\begin{align*}
\frac{\left( a_1a_2\cdots a_k \right)}{\left( a_2a_3\cdots a_k \right)}=a_1+\frac{1}{\frac{\left( a_2a_3\cdots a_k \right)}{\left( a_3a_4\cdots a_k \right)}}=a_1+\frac{1}{a_2+\frac{1}{a_3+\cdots +\frac{1}{a_{n-1}+\frac{1}{a_n}}}}.
\end{align*}
故由数学归纳法可知结论成立.
\end{solution}

\begin{exercise}
设\(\vert A\vert\)是\(n\)阶行列式,\(\vert A\vert\)的第\((i,j)\)元素\(a_{ij}=\max\{i,j\}\),试求\(\vert A\vert\)的值.
\end{exercise}
\begin{solution}
\begin{align*}
\left| \boldsymbol{A} \right|=\left| \begin{matrix}
1&		2&		3&		\cdots&		n\\
2&		2&		3&		\cdots&		n\\
3&		3&		3&		\cdots&		n\\
\vdots&		\vdots&		\vdots&		&		\vdots\\
n&		n&		n&		\cdots&		n\\
\end{matrix} \right|\xlongequal[i=n,n-1,\cdots ,2]{-r_i+r_{i-1}}\left| \begin{matrix}
-1&		0&		0&		\cdots&		0\\
2&		-1&		0&		\cdots&		0\\
3&		3&		-1&		\cdots&		0\\
\vdots&		\vdots&		\vdots&		&		\vdots\\
n&		n&		n&		\cdots&		n\\
\end{matrix} \right|=\left( -1 \right) ^{n-1}n.
\end{align*}
\end{solution}

\begin{exercise}
设\(\vert A\vert\)是\(n\)阶行列式,\(\vert A\vert\)的第\((i,j)\)元素\(a_{ij}=\vert i - j\vert\),试求\(\vert A\vert\)的值.
\end{exercise}
\begin{note}
注意:这只是一个\textbf{对称行列式},不是循环行列式.
类似这种每行、每列元素有一定的等差递进关系的行列式,都可以先尝试用每一列减去前面一列.
\end{note}
\begin{solution}
\begin{align*}
\left| \boldsymbol{A} \right|&=\left| \begin{matrix}
0&		1&		2&		\cdots&		n-2&		n-1\\
1&		0&		1&		\cdots&		n-3&		n-2\\
2&		1&		0&		\cdots&		n-4&		n-3\\
\vdots&		\vdots&		\vdots&		&		\vdots&		\vdots\\
n-1&		n-2&		n-3&		\cdots&		1&		0\\
\end{matrix} \right|\xlongequal[i=n,n-1,\cdots ,2]{-j_{i-1}+j_i}\left| \begin{matrix}
0&		1&		1&		\cdots&		1&		1\\
1&		-1&		1&		\cdots&		1&		1\\
2&		-1&		-1&		\cdots&		1&		1\\
\vdots&		\vdots&		\vdots&		&		\vdots&		\vdots\\
n-1&		-1&		-1&		\cdots&		-1&		-1\\
\end{matrix} \right|
\\
&\xlongequal[i=n-1,n-2,\cdots ,1]{r_n+r_i}\left| \begin{matrix}
n-1&		0&		0&		\cdots&		0&		0\\
n&		-2&		0&		\cdots&		0&		0\\
n+1&		-2&		-2&		\cdots&		0&		0\\
\vdots&		\vdots&		\vdots&		&		\vdots&		\vdots\\
n-1&		-1&		-1&		\cdots&		-1&		-1\\
\end{matrix} \right|=\left( -2 \right) ^{n-2}\left( n-1 \right) .
\end{align*}
\end{solution}

\begin{exercise}
求下列\(n\)阶行列式的值:
\[
\left| \boldsymbol{A} \right| = 
\begin{vmatrix}
1 & x_1(x_1 - a) & x_1^2(x_1 - a) & \cdots & x_1^{n - 1}(x_1 - a)\\
1 & x_2(x_2 - a) & x_2^2(x_2 - a) & \cdots & x_2^{n - 1}(x_2 - a)\\
\vdots & \vdots & \vdots & \ddots & \vdots\\
1 & x_n(x_n - a) & x_n^2(x_n - a) & \cdots & x_n^{n - 1}(x_n - a)
\end{vmatrix}.
\]
\end{exercise}
\begin{note}
当行列式的行或列有一定的规律性时,但是由于缺少一行或一列导致这个行列式行或列的规律性并不完整.此时我们可以尝试\hyperlink{行列式计算:升阶法}{升阶法}补全这个行列式行或列的规律,再对行列式进行化简.

本题若直接使用\hyperref[大拆分法]{大拆分法}会得到比较多的行列式,而且每个行列式并不是完整的$Vandermode$行列式.后续求解很繁琐,因此不采取\hyperref[大拆分法]{大拆分法}.
\end{note}
\begin{solution}
(\hyperlink{行列式计算:升阶法}{升阶法})考虑$n+1$阶行列式\(|\boldsymbol{B}|=\left|\begin{matrix}
1 & x_1 - a & x_1(x_1 - a) & x_{1}^{2}(x_1 - a) & \cdots & x_{1}^{n - 1}(x_1 - a)\\
1 & x_2 - a & x_2(x_2 - a) & x_{2}^{2}(x_2 - a) & \cdots & x_{2}^{n - 1}(x_2 - a)\\
\vdots & \vdots & \vdots & \vdots &  & \vdots\\
1 & x_n - a & x_n(x_n - a) & x_{n}^{2}(x_n - a) & \cdots & x_{n}^{n - 1}(x_n - a)\\
1 & y - a & y(y - a) & y^2(y - a) & \cdots & y^{n - 1}(y - a)
\end{matrix}\right|\),则
\begin{align*}
|\boldsymbol{B}|=\left|\begin{matrix}
1 & x_1 & x_{1}^{2} & x_{1}^{3} & \cdots & x_{1}^{n}\\
1 & x_2 & x_{2}^{2} & x_{2}^{3} & \cdots & x_{2}^{n}\\
\vdots & \vdots & \vdots & \vdots &  & \vdots\\
1 & x_n & x_{n}^{2} & x_{n}^{3} & \cdots & x_{n}^{n}\\
1 & y & y^2 & y^3 & \cdots & y^n
\end{matrix}\right|=\prod_{k = 1}^{n}(y - x_k)\prod_{1\leqslant i < j\leqslant n}(x_j - x_i).
\end{align*}
由上式可知,\(|\boldsymbol{B}|\)可以看作一个关于\(y\)的\(n\)次多项式.
将\(|\boldsymbol{B}|\)按最后一行展开得到
\begin{align*}
|\boldsymbol{B}|=\sum_{i = 1}^{n + 1}(-1)^{n + i}B_{n + 1,i}y^{i - 1},\text{其中}B_{ni}\text{是}|\boldsymbol{B}|\text{的第}(n + 1,i)\text{元的余子式},i = 1,2,\cdots,n + 1.
\end{align*}
从而
\begin{align}\label{eq:两多项式相等1.1}
|\boldsymbol{B}|=(-1)^{n + 2}B_{n + 1,1}+\sum_{i = 2}^{n + 1}(-1)^{n + i + 1}B_{n + 1,i}y^{i - 2}(y - a)=\prod_{k = 1}^{n}(y - x_k)\prod_{1\leqslant i < j\leqslant n}(x_j - x_i).
\end{align}
又易知\(B_{n + 1,2}=|\boldsymbol{A}|\),而当\(a = 0\)时,由等式\eqref{eq:两多项式相等1.1}可知,\(|\boldsymbol{B}|\)中\(y\)前面的系数只有\(B_{n + 1,2}\).比较等式\eqref{eq:两多项式相等1.1}两边\(y\)的系数可得
\begin{align*}
(-1)^{n + 3}|\boldsymbol{A}|=(-1)^{n + 3}B_{n + 1,2}=\prod_{1\leqslant i < j\leqslant n}(x_j - x_i)\left(\sum_{i = 1}^{n}(-x_1)\cdots (-x_{i - 1})(-x_{i + 1})\cdots (-x_n)\right).
\end{align*}
于是\(|\boldsymbol{A}|=(-1)^{n + 3}(-1)^{n - 1}\prod_{1\leqslant i < j\leqslant n}(x_j - x_i)\left(\sum_{i = 1}^{n}x_1\cdots x_{i - 1}x_{i + 1}\cdots x_n\right)=\prod_{1\leqslant i < j\leqslant n}(x_j - x_i)\left(\sum_{i = 1}^{n}x_1\cdots x_{i - 1}x_{i + 1}\cdots x_n\right)\).

当\(a\neq 0\)时,由等式\((1.1)\)可知,\(|\boldsymbol{B}|\)中\(y\)前面的系数不只有\(B_{n + 1,2}\),但是,我们比较等式\eqref{eq:两多项式相等1.1}两边的常数项可得
\begin{align}\label{eq:等式1.2}
(-1)^{n + 2}B_{n + 1,1}-a(-1)^{n + 3}B_{n + 1,2}=\prod_{1\leqslant i < j\leqslant n}(x_j - x_i)\prod_{k = 1}^{n}(-x_k).
\end{align}
又因为
\begin{align*}
B_{n + 1,1}&=\left|\begin{matrix}
x_1 - a & x_1(x_1 - a) & x_{1}^{2}(x_1 - a) & \cdots & x_{1}^{n - 1}(x_1 - a)\\
x_2 - a & x_2(x_2 - a) & x_{2}^{2}(x_2 - a) & \cdots & x_{2}^{n - 1}(x_2 - a)\\
\vdots & \vdots & \vdots &  & \vdots\\
x_n - a & x_n(x_n - a) & x_{n}^{2}(x_n - a) & \cdots & x_{n}^{n - 1}(x_n - a)
\end{matrix}\right|
\\
&=\prod_{i = 1}^{n}(x_i - a)\left|\begin{matrix}
1 & x_1 & x_{1}^{2} & x_{1}^{3} & \cdots & x_{1}^{n - 1}\\
1 & x_2 & x_{2}^{2} & x_{2}^{3} & \cdots & x_{2}^{n - 1}\\
\vdots & \vdots & \vdots & \vdots &  & \vdots\\
1 & x_n & x_{n}^{2} & x_{n}^{3} & \cdots & x_{n}^{n - 1}
\end{matrix}\right|=\prod_{i = 1}^{n}(x_i - a)\prod_{1\leqslant i < j\leqslant n}(x_j - x_i).
\end{align*}
所以再结合等式\eqref{eq:等式1.2}可得
\begin{align*}
-a(-1)^{n + 3}|\boldsymbol{A}|&=-a(-1)^{n + 3}B_{n + 1,2}=\prod_{1\leqslant i < j\leqslant n}(x_j - x_i)\prod_{k = 1}^{n}(-x_k)-(-1)^{n + 2}B_{n + 1,1}
\\
&=(-1)^n\prod_{k = 1}^{n}x_k\prod_{1\leqslant i < j\leqslant n}(x_j - x_i)+(-1)^{n + 1}\prod_{i = 1}^{n}(x_i - a)\prod_{1\leqslant i < j\leqslant n}(x_j - x_i)
\\
&=(-1)^n\prod_{1\leqslant i < j\leqslant n}(x_j - x_i)\left[\prod_{k = 1}^{n}x_k-\prod_{i = 1}^{n}(x_i - a)\right].
\end{align*}
故此时\(|\boldsymbol{A}|=\prod_{1\leqslant i < j\leqslant n}(x_j - x_i)\left(\prod_{k = 1}^{n}x_k-\prod_{i = 1}^{n}(x_i - a)\right)\).
\end{solution}

\begin{exercise}
求下列行列式式的值($n$为偶数)
\begin{align*}
I=\left| \begin{matrix}
1&		1&		\cdots&		1&		1\\
2&		2^2&		\cdots&		2^n&		2^{n+1}\\
\vdots&		\vdots&		\ddots&		\vdots&		\vdots\\
n&		n^2&		\cdots&		n^n&		n^{n+1}\\
\frac{n}{2}&		\frac{n^2}{3}&		\cdots&		\frac{n^n}{n+1}&		\frac{n^{n+1}}{n+2}\\
\end{matrix} \right|.
\end{align*}
\end{exercise}
\begin{note}
应用\hyperref[proposition:行列式的求导运算]{行列式函数求导求行列式}的值.
\end{note}
\begin{solution}
令\(G(x)=\left|\begin{matrix}
1 & 1 & \cdots & 1 & 1\\
2 & 2^2 & \cdots & 2^n & 2^{n + 1}\\
\vdots & \vdots & \ddots & \vdots & \vdots\\
n & n^2 & \cdots & n^n & n^{n + 1}\\
\frac{x^2}{2} & \frac{x^3}{3} & \cdots & \frac{x^{n + 1}}{n + 1} & \frac{x^{n + 2}}{n + 2}
\end{matrix}\right|\),则\(I = \frac{G(n)}{n}\)且\(G(0) = 0\).      
利用行列式求导公式,可得
\begin{align*}
G'(x)&=\left|\begin{matrix}
1 & 1 & \cdots & 1 & 1\\
2 & 2^2 & \cdots & 2^n & 2^{n + 1}\\
\vdots & \vdots & \ddots & \vdots & \vdots\\
n & n^2 & \cdots & n^n & n^{n + 1}\\
x & x^2 & \cdots & x^n & x^{n + 1}
\end{matrix}\right|
= n!x\left|\begin{matrix}
1 & 1 & \cdots & 1 & 1\\
1 & 2 & \cdots & 2^{n - 1} & 2^n\\
\vdots & \vdots & \ddots & \vdots & \vdots\\
1 & n & \cdots & n^{n - 1} & n^n\\
1 & x & \cdots & x^{n - 1} & x^n
\end{matrix}\right|
= n!\prod_{1\leqslant i < j\leqslant n}(j - i)\prod_{k = 0}^{n}(x - k).
\end{align*}
因此
\begin{align*}
I &= \frac{G(n)}{n}=\frac{\int_{0}^{n}G'(x)dx}{n}=(n - 1)!\prod_{1\leqslant i < j\leqslant n}(j - i)\int_{0}^{n}\prod_{k = 0}^{n}(x - k)dx
\\
&\stackrel{\text{区间再现}}{=}(n - 1)!\prod_{1\leqslant i < j\leqslant n}(j - i)\int_{0}^{n}\prod_{k = 0}^{n}(n - k - x)dx
\\
&= (-1)^{n + 1}(n - 1)!\prod_{1\leqslant i < j\leqslant n}(j - i)\int_{0}^{n}\prod_{k = 0}^{n}(x - k)dx
\\
&= (-1)^{n + 1}I.
\end{align*}      
由于\(n\)为偶数,所以\((-1)^{n + 1} = -1\).于是\(I = -I\).故\(I = 0\). 
\end{solution}



\end{document}
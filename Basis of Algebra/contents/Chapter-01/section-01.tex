% contents/Chapter-01/section-01.tex 第一章第一节
\documentclass[../../main.tex]{subfiles}
\graphicspath{{\subfix{../../image/}}} % 指定图片目录,后续可以直接使用图片文件名。

% 例如:
% \begin{figure}[h]
% \centering
% \includegraphics{image-01.01}
% \label{fig:image-01.01}
% \caption{图片标题}
% \end{figure}

\begin{document}
\section{定义、定理和命题}

\begin{proposition}[\hypertarget{行列式计算常识}{行列式计算常识}]\label{pro:行列式计算常识}
(1)$\left| \begin{matrix}
&		&		&		a_n\\
&		&		\begin{turn}{80}$\ddots$\end{turn}&		\\
&		a_2&		&		\\
a_1&		&		&		\\
\end{matrix} \right|=\left( -1 \right) ^{\frac{n\left( n-1 \right)}{2}}a_1a_2\cdots a_n$
;\,\,$\left| \begin{matrix}
a_1&		&		&		&		\\
&		\ddots&		&		&		\\
b_1&		\cdots&		a_i&		\cdots&		b_n\\
&		&		&		\ddots&		\\
&		&		&		&		a_n\\
\end{matrix} \right|=a_1a_2\cdots a_n$.

(2)设$n$阶行列式$D=\det(a_{ij})$,把$D$上下翻转(\textbf{行倒排})、或左右翻转(\textbf{列倒排})分别得到$D_1$、$D_2$;把$D$\textbf{逆时针旋转$90^{\circ}$}、或\textbf{顺时针旋转$90^{\circ}$}分别得到$D_3$、$D_4$;把$D$\textbf{依副对角线翻转}、或\textbf{依主对角线翻转}分别得到$D_5$、$D_6$.易知
\begin{align*}
D_1=\left| \begin{matrix}
a_{n1}&		\cdots&		a_{nn}\\
\vdots&		&		\vdots\\
a_{11}&		\cdots&		a_{1n}\\
\end{matrix} \right|,D_2=\left| \begin{matrix}
a_{1n}&		\cdots&		a_{11}\\
\vdots&		&		\vdots\\
a_{nn}&		\cdots&		a_{n1}\\
\end{matrix} \right|,D_3=\left| \begin{matrix}
a_{1n}&		\cdots&		a_{nn}\\
\vdots&		&		\vdots\\
a_{11}&		\cdots&		a_{n1}\\
\end{matrix} \right|,
\\
D_4=\left| \begin{matrix}
a_{n1}&		\cdots&		a_{11}\\
\vdots&		&		\vdots\\
a_{nn}&		\cdots&		a_{1n}\\
\end{matrix} \right|,D_5=\left| \begin{matrix}
a_{nn}&		\cdots&		a_{1n}\\
\vdots&		&		\vdots\\
a_{n1}&		\cdots&		a_{11}\\
\end{matrix} \right|,D_6=\left| \begin{matrix}
a_{nn}&		\cdots&		a_{n1}\\
\vdots&		&		\vdots\\
a_{1n}&		\cdots&		a_{11}\\
\end{matrix} \right|.
\nonumber
\end{align*}
则一定有
\begin{gather*}
D_1=D_2=D_3=D_4=\left( -1 \right) ^{\frac{n\left( n-1 \right)}{2}}D,
\\
D_5=D_6=D.
\nonumber
\end{gather*}
(3)设\(A=(a_{i,j})\)为\(n\)阶复矩阵,则一定有\(\vert A\vert=\overline{\vert A\vert}\).

(4)若\(\vert A\vert\)是\(n\)阶行列式,\(\vert B\vert\)是\(m\)阶行列式,它们的值都不为零,则
\begin{align*}
\left| \left. \begin{matrix}
\boldsymbol{A}&		\boldsymbol{O}\\
\boldsymbol{O}&		\boldsymbol{B}\\
\end{matrix} \right. \right|=\left( -1 \right) ^{mn}\left. \left| \begin{matrix}
\boldsymbol{O}&		\boldsymbol{A}\\
\boldsymbol{B}&		\boldsymbol{O}\\
\end{matrix} \right| \right. .
\end{align*}
\end{proposition}
\begin{proof}
(1)运用行列式的定义即可得到结论.
\begin{align*}
(2)\,\,D_1&=\left| \begin{matrix}
a_{n1}&		\cdots&		a_{nn}\\
\vdots&		&		\vdots\\
a_{11}&		\cdots&		a_{1n}\\
\end{matrix} \right|\xlongequal[i=1,2,\cdots ,n-1]{r_i\longleftrightarrow r_{i+1}}\left( -1 \right) ^{n-1}\left| \begin{matrix}
a_{n-1,1}&		\cdots&		a_{n-1,n}\\
\vdots&		&		\vdots\\
a_{n1}&		\cdots&		a_{nn}\\
\end{matrix} \right|\xlongequal[i=1,2,\cdots ,n-2]{r_i\longleftrightarrow r_{i+1}}\left( -1 \right) ^{n-1+n-2}\left| \begin{matrix}
a_{n-2,1}&		\cdots&		a_{n-2,n}\\
\vdots&		&		\vdots\\
a_{n1}&		\cdots&		a_{nn}\\
\end{matrix} \right|
\\
&=\cdots =\left( -1 \right) ^{n-1+n-2+\cdots +1}\left| \begin{matrix}
a_{11}&		\cdots&		a_{1n}\\
\vdots&		&		\vdots\\
a_{n1}&		\cdots&		a_{nn}\\
\end{matrix} \right|=\left( -1 \right) ^{\frac{n\left( n-1 \right)}{2}}\left| \begin{matrix}
a_{11}&		\cdots&		a_{1n}\\
\vdots&		&		\vdots\\
a_{n1}&		\cdots&		a_{nn}\\
\end{matrix} \right|=\left( -1 \right) ^{\frac{n\left( n-1 \right)}{2}}D.
\end{align*}
\begin{align*}
D_2&=\left| \begin{matrix}
a_{1n}&		\cdots&		a_{11}\\
\vdots&		&		\vdots\\
a_{nn}&		\cdots&		a_{n1}\\
\end{matrix} \right|\xlongequal[i=1,2,\cdots ,n-1]{j_i\longleftrightarrow j_{i+1}}\left( -1 \right) ^{n-1}\left| \begin{matrix}
a_{1,n-1}&		\cdots&		a_{1n}\\
\vdots&		&		\vdots\\
a_{n,n-1}&		\cdots&		a_{nn}\\
\end{matrix} \right|\xlongequal[i=1,2,\cdots ,n-2]{j_i\longleftrightarrow j_{i+1}}\left( -1 \right) ^{n-1+n-2}\left| \begin{matrix}
a_{1,n-2}&		\cdots&		a_{1n}\\
\vdots&		&		\vdots\\
a_{n,n-2}&		\cdots&		a_{nn}\\
\end{matrix} \right|
\\
&=\cdots =\left( -1 \right) ^{n-1+n-2+\cdots +1}\left| \begin{matrix}
a_{11}&		\cdots&		a_{1n}\\
\vdots&		&		\vdots\\
a_{n1}&		\cdots&		a_{nn}\\
\end{matrix} \right|=\left( -1 \right) ^{\frac{n\left( n-1 \right)}{2}}\left| \begin{matrix}
a_{11}&		\cdots&		a_{1n}\\
\vdots&		&		\vdots\\
a_{n1}&		\cdots&		a_{nn}\\
\end{matrix} \right|=\left( -1 \right) ^{\frac{n\left( n-1 \right)}{2}}D.
\nonumber
\end{align*}
\begin{gather*}
D_3=\left| \begin{matrix}
a_{1n}&		\cdots&		a_{nn}\\
\vdots&		&		\vdots\\
a_{11}&		\cdots&		a_{n1}\\
\end{matrix} \right|\xlongequal{\text{行倒排}}\left( -1 \right) ^{\frac{n\left( n-1 \right)}{2}}\left| \begin{matrix}
a_{11}&		\cdots&		a_{n1}\\
\vdots&		&		\vdots\\
a_{1n}&		\cdots&		a_{nn}\\
\end{matrix} \right|=\left( -1 \right) ^{\frac{n\left( n-1 \right)}{2}}D^T=\left( -1 \right) ^{\frac{n\left( n-1 \right)}{2}}D.
\\
D_4=\left| \begin{matrix}
a_{n1}&		\cdots&		a_{11}\\
\vdots&		&		\vdots\\
a_{nn}&		\cdots&		a_{1n}\\
\end{matrix} \right|\xlongequal{\text{列倒排}}\left( -1 \right) ^{\frac{n\left( n-1 \right)}{2}}\left| \begin{matrix}
a_{11}&		\cdots&		a_{n1}\\
\vdots&		&		\vdots\\
a_{1n}&		\cdots&		a_{nn}\\
\end{matrix} \right|=\left( -1 \right) ^{\frac{n\left( n-1 \right)}{2}}D^T=\left( -1 \right) ^{\frac{n\left( n-1 \right)}{2}}D.
\\
D_5=\left| \begin{matrix}
a_{nn}&		\cdots&		a_{1n}\\
\vdots&		&		\vdots\\
a_{n1}&		\cdots&		a_{11}\\
\end{matrix} \right|\xlongequal[]{\text{逆时针旋转}90^{\circ}}\left( -1 \right) ^{\frac{n\left( n-1 \right)}{2}}\left| \begin{matrix}
a_{1n}&		\cdots&		a_{11}\\
\vdots&		&		\vdots\\
a_{nn}&		\cdots&		a_{n1}\\
\end{matrix} \right|\xlongequal[]{\text{列倒排}}\left( -1 \right) ^{\frac{n\left( n-1 \right)}{2}}\cdot \left( -1 \right) ^{\frac{n\left( n-1 \right)}{2}}\left| \begin{matrix}
a_{11}&		\cdots&		a_{1n}\\
\vdots&		&		\vdots\\
a_{n1}&		\cdots&		a_{nn}\\
\end{matrix} \right|=D.
\\
D_6=\left| \begin{matrix}
a_{nn}&		\cdots&		a_{n1}\\
\vdots&		&		\vdots\\
a_{1n}&		\cdots&		a_{11}\\
\end{matrix} \right|\xlongequal[]{\text{顺时针旋转}90^{\circ}}\left( -1 \right) ^{\frac{n\left( n-1 \right)}{2}}\left| \begin{matrix}
a_{1n}&		\cdots&		a_{nn}\\
\vdots&		&		\vdots\\
a_{11}&		\cdots&		a_{n1}\\
\end{matrix} \right|\xlongequal[]{\text{行倒排}}\left( -1 \right) ^{\frac{n\left( n-1 \right)}{2}}\cdot \left( -1 \right) ^{\frac{n\left( n-1 \right)}{2}}\left| \begin{matrix}
a_{11}&		\cdots&		a_{1n}\\
\vdots&		&		\vdots\\
a_{n1}&		\cdots&		a_{nn}\\
\end{matrix} \right|=D.
\nonumber
\end{gather*}
(3)复数的共轭保持加法和乘法:\(\overline{z_1 + z_2}=\overline{z_1}+\overline{z_2}\),\(\overline{z_1\cdot z_2}=\overline{z_1}\cdot\overline{z_2}\),故由行列式的组合定义可得
\begin{align*}
|A|&=\sum_{1\le k_1,k_2,\cdots ,k_n\le n}{\left( -1 \right) ^{\tau (k_1k_2\cdots k_n)}a_{k_{11}}a_{k_{22}}\cdots a_{k_{nn}}}
\\
&=\sum_{1\le k_1,k_2,\cdots ,k_n\le n}{\left( -1 \right) ^{\tau (k_1k_2\cdots k_n)}\overline{a_{k_{11}}}\cdot\overline{a_{k_{22}}}\cdots \overline{a_{k_{nn}}}}=|\overline{A}|.
\end{align*}

(4)将\(\vert A\vert\)的第一列依次和\(\vert B\vert\)的第\(m\)列,第\(m - 1\)列,…,第一列对换,共换了\(m\)次;再将\(\vert A\vert\)的第二列依次和\(\vert B\vert\)的第\(m\)列,第\(m - 1\)列,…,第一列对换,又换了\(m\)次;$\cdots$.依次类推,经过\(mn\)次对换可将第二个行列式变为第一个行列式.因此\(\vert D\vert=(-1)^{mn}\vert C\vert\),于是
由行列式的基本性质可得
\begin{align*}
\left| \left. \begin{matrix}
\boldsymbol{A}&		\boldsymbol{O}\\
\boldsymbol{O}&		\boldsymbol{B}\\
\end{matrix} \right. \right|=\left( -1 \right) ^{mn}\left. \left| \begin{matrix}
\boldsymbol{O}&		\boldsymbol{A}\\
\boldsymbol{B}&		\boldsymbol{O}\\
\end{matrix} \right| \right. .
\end{align*}
\end{proof}

\begin{proposition}[奇数阶反对称行列式的值等于零]\label{proposition:奇数阶反对称行列式的值等于零}
如果\(n\)阶行列式\(\vert A\vert\)的元素满足\(a_{ij}=-a_{ji}(1\leq i,j\leq n)\),则称为反对称行列式.求证:奇数阶反对称行列式的值等于零.
\end{proposition}
\begin{note}
{\color{blue}证法二}的想法是将行列式按组合的定义写成(n-1)!个单项的和.然后将其两两分组再求和(因为一共有(n-1)!个单项,即和式中共有偶数个单项,所以只要使用合适的分组方式就一定能够将其两两分组再求和),最后发现每组的和均为0.

构造的这个映射$\varphi$的目的是为了更加准确、严谨地说明分组的方式.证明这个映射$\varphi$是一个双射是为了保证原来的和式中的每一个单项都能与和式中另一个单项一一对应.\CJKunderline*{然后利用反证法证明了这两个一一对应的单项一定互不相同}\textbf{(注:我认为这步有些多余.这里应该只需要说明这两个一一对应的单项是原和式中不同的单项即可,即这两个单项的角标不完全相同就行,其实,这个在我们定义映射$\varphi$的时候就已经满足了.满足这个条件就足以说明原和式可以按照这种方式进行分组.并且利用反对称行列式的性质也能够证明这两个单项不仅互不相同,还能进一步得到这两个单项互为相反数)}.于是我们就可以将原和式中的每一个单项与其在双射$\varphi$作用下的像看成一组,按照这种方式就可以将原和式进行分组再求和.
\end{note}
\begin{proof}
{\color{blue}证法一(行列式的性质):}
由反对称行列式的定义可知,\(\vert A\vert\)的转置\(\vert A^{\prime}\vert\)与\(\vert A\vert\)的每个元素都相差一个符号,将\(\vert A^{\prime}\vert\)的每一行都提出公因子\(-1\)可得\(\vert A\vert=\vert A^{\prime}\vert=(-1)^{n}\vert A\vert=-\vert A\vert\),从而\(\vert A\vert = 0\).

{\color{blue}证法二(行列式的组合定义):}
由于\(\vert A\vert\)的主对角元全为0,故由组合定义,只需考虑下列单项:
\[
T = \{a_{k_11}a_{k_22}\cdots a_{k_{nn}} \mid k_i\neq i(1\leq i\leq n)\}
\]
定义映射\(\varphi:T\to T\),\(a_{k_11}a_{k_22}\cdots a_{k_{nn}}\mapsto a_{1k_1}a_{2k_2}\cdots a_{nk_n}\).显然\(\varphi^2 = \text{Id}_T\),于是\(\varphi\)是一个双射.我们断言:\(a_{k_11}a_{k_22}\cdots a_{k_{nn}}\)和\(a_{1k_1}a_{2k_2}\cdots a_{nk_n}\)作为\(\vert A\vert\)的单项不相同,否则\(\{1,2,\cdots,n\}\)必可分成若干对\((i_1,j_1),\cdots,(i_t,j_t)\),使得\(a_{k_11}a_{k_22}\cdots a_{k_{nn}}=a_{i_1j_1}a_{j_1i_1}\cdots a_{i_tj_t}a_{j_ti_t}\),这与\(n\)为奇数矛盾.将上述两个单项看成一组,则它们在\(\vert A\vert\)中符号均为\((-1)^{\tau(k_1k_2\cdots k_n)}\).由于\(\vert A\vert\)反对称,故
\[
a_{1k_1}a_{2k_2}\cdots a_{nk_n}=(-1)^n a_{k_11}a_{k_22}\cdots a_{k_{nn}}=-a_{k_11}a_{k_22}\cdots a_{k_{nn}}
\]
从而每组和为0,于是\(\vert A\vert = 0\).
\end{proof}

\begin{proposition}[\hypertarget{"爪"型行列式}{"爪"型行列式}]\label{"爪"型行列式}
证明$n$阶行列式:
\begin{gather}
|\boldsymbol{A}|=\left| \begin{matrix}
a_1&		b_2&		\cdots&		b_n\\
c_2&		a_2&		&		\\
\vdots&		&		\ddots&		\\
c_n&		&		&		a_n\\
\end{matrix} \right|
=a_1a_2\cdots a_n-\sum_{i=2}^n{a_2}\cdots \widehat{a_i}\cdots a_nb_ic_i.
\nonumber
\end{gather}
\end{proposition}
\begin{note}
记忆"爪"型行列式的计算方法和结论.
\end{note}
\begin{proof}
当$a_i\ne 0\left( \forall i\in \left[ 2,n \right] \cap \mathbb{N}  \right)$时,我们有
\begin{align*}
&|\boldsymbol{A}|=\left| \begin{matrix}
a_1&		b_2&		\cdots&		b_n\\
c_2&		a_2&		&		\\
\vdots&		&		\ddots&		\\
c_n&		&		&		a_n\\
\end{matrix} \right|\xlongequal[i=2,\cdots ,n]{\left( -\frac{c_i}{a_i} \right) j_i+j_1}\left| \begin{matrix}
a_1-\sum_{i=2}^n{\frac{b_ic_i}{a_i}}&		b_2&		\cdots&		b_n\\
0&		a_2&		&		\\
\vdots&		&		\ddots&		\\
0&		&		&		a_n\\
\end{matrix} \right|
\\
&=\left( a_1-\sum_{i=2}^n{\frac{b_ic_i}{a_i}} \right) \prod\limits_{i=2}^n{a_i}
=a_1a_2\cdots a_n-\sum_{i=2}^n{a_2}\cdots \widehat{a_i}\cdots a_nb_ic_i.
\end{align*}
当$\exists i\in \left[ 2,n \right] \cap \mathbb{N} \,\,s.t. \,\,a_i=0$时,则
$a_1a_2\cdots a_n-\sum_{i=2}^n{a_2}\cdots \widehat{a_i}\cdots a_nb_ic_i=-a_2\cdots \widehat{a_i}\cdots a_nb_ic_i$
.此时,我们有
\begin{align*}
|\boldsymbol{A}| &= \left| \begin{matrix}
a_1 & b_2 & \cdots & b_{i-1} & b_i & b_{i+1} & \cdots & b_n \\
c_2 & a_2 & & & & & & \\
\vdots & & \ddots & & & & & \\
c_{i-1} & & & a_{i-1} & & & & \\
c_i & & & & 0 & & & \\
c_{i+1} & & & & & a_{i+1} & & \\
\vdots & & & & & & \ddots & \\
c_n & & & & & & & a_n \\
\end{matrix} \right|
\xlongequal[(\text{按}c_i\text{所在行展开})]{\text{按第}i\text{行展开}} (-1)^{i+1}c_i \left| \begin{matrix}
b_2 & \cdots & b_{i-1} & b_i & b_{i+1} & \cdots & b_n \\
a_2 & & & & & & & \\
& \ddots & & & & & & \\
& & a_{i-1} & 0 & 0 & & & \\
& & 0 & 0 & a_{i+1} & & & \\
& & & & & \ddots & & \\
& & & & & & a_n \\
\end{matrix} \right| \\
&\xlongequal[(\text{按}b_i\text{所在列展开})]{\text{按第}i-1\text{列展开}} (-1)^{i+1}(-1)^{i}b_ic_i \left| \begin{matrix}
a_2 & & & & & \\
& \ddots & & & & \\
& & a_{i-1} & & & \\
& & & a_{i+1} & & \\
& & & & \ddots & \\
& & & & & a_n \\
\end{matrix} \right|      
= -a_2 \cdots \widehat{a_i} \cdots a_nb_ic_i.
\end{align*}
综上所述,原命题得证.
\end{proof}

\begin{proposition}[分块"爪"型行列式]\label{proposition:分块"爪"型行列式}
计算$n$阶行列式($a_{ii}\ne 0,i=k+1,k+2,\cdots,n$):
\begin{align*}
|\boldsymbol{A}|=\left| \begin{matrix}
a_{11}&		\cdots&		a_{1k}&		a_{1,k+1}&		\cdots&		a_{1n}\\
\vdots&		&		\vdots&		\vdots&		&		\vdots\\
a_{k1}&		\cdots&		a_{kk}&		a_{k,k+1}&		\cdots&		a_{kn}\\
a_{k+1,1}&		\cdots&		a_{k+1,k}&		a_{k+1,k+1}&		&		\\
\vdots&		&		\vdots&		&		\ddots&		\\
a_{n1}&		\cdots&		a_{nk}&		&		&		a_{nn}\\
\end{matrix} \right|.
\end{align*}
\end{proposition}
\begin{note}
记忆分块"爪"型行列式的计算方法即可,计算方法和"爪"型行列式的计算方法类似.
\end{note}
\begin{solution}
\begin{align*}
&|\boldsymbol{A}|=\left| \begin{matrix}
a_{11}&		\cdots&		a_{1k}&		a_{1,k+1}&		\cdots&		a_{1n}\\
\vdots&		&		\vdots&		\vdots&		&		\vdots\\
a_{k1}&		\cdots&		a_{kk}&		a_{k,k+1}&		\cdots&		a_{kn}\\
a_{k+1,1}&		\cdots&		a_{k+1,k}&		a_{k+1,k+1}&		&		\\
\vdots&		&		\vdots&		&		\ddots&		\\
a_{n1}&		\cdots&		a_{nk}&		&		&		a_{nn}\\
\end{matrix} \right|
\\
&\xlongequal[i=k+1,k+2,\cdots ,n]{-\frac{a_{i1}}{a_{ii}}j_i+j_1,-\frac{a_{i2}}{a_{ii}}j_i+j_2,\cdots ,-\frac{a_{in}}{a_{ii}}j_i+j_k}\left| \begin{matrix}
c_{11}&		\cdots&		c_{1k}&		a_{1,k+1}&		\cdots&		a_{1n}\\
\vdots&		&		\vdots&		\vdots&		&		\vdots\\
c_{k1}&		\cdots&		c_{kk}&		a_{k,k+1}&		\cdots&		a_{kn}\\
0&		\cdots&		0&		a_{k+1,k+1}&		&		\\
\vdots&		&		\vdots&		&		\ddots&		\\
0&		\cdots&		0&		&		&		a_{nn}\\
\end{matrix} \right|
\\
&=\left| \begin{matrix}
C&		B\\
O&		\Lambda\\
\end{matrix} \right|=|C|\cdot |\Lambda |=|C|\prod_{i=k+1}^n{a_{ii}}.
\end{align*}
其中$C=\left( \begin{matrix}
c_{11}&		\cdots&		c_{1k}\\
\vdots&		&		\vdots\\
c_{k1}&		\cdots&		c_{kk}\\
\end{matrix} \right) ,B=\left( \begin{matrix}
a_{1,k+1}&		\cdots&		a_{1n}\\
\vdots&		&		\vdots\\
a_{k,k+1}&		\cdots&		a_{kn}\\
\end{matrix} \right) ,\Lambda =\left( \begin{matrix}
a_{k+1}&		&		\\
&		\ddots&		\\
&		&		a_n\\
\end{matrix} \right).$
并且$c_{pq}=a_{pq}-\sum_{i=k+1}^n{\frac{a_{iq}a_{pi}}{a_{ii}}},p,q=1,2,\cdots ,n$.
\end{solution}

\begin{corollary}[\hypertarget{"爪"型行列式的推广}{"爪"型行列式的推广}]\label{"爪"型行列式的推广}
计算$n$阶行列式:
\begin{equation}
|\boldsymbol{A}|=\left| \begin{matrix}
x_1-a_1&		x_2&		x_3&		\cdots&		x_n\\
x_1&		x_2-a_2&		x_3&		\cdots&		x_n\\
x_1&		x_2&		x_3-a_3&		\cdots&		x_n\\
\vdots&		\vdots&		\vdots&		&		\vdots\\
x_1&		x_2&		x_3&		\cdots&		x_n-a_n\\
\end{matrix} \right|.
\nonumber
\end{equation}
\end{corollary}
\begin{note}
这是一个有用的模板(即\textbf{行列式除了主对角元素外,每行都一样}).

记忆该命题的计算方法即可.即先化为"爪"型行列式,再利用"爪"型行列式的计算结果.
\end{note}
\begin{solution}
当$a_i\ne 0\left( \forall i\in \left[ 2,n \right] \cap \mathbb{N}  \right)$时,我们有
\begin{equation}
\begin{split}
|\boldsymbol{A}|&=\left| \begin{matrix}
x_1-a_1&		x_2&		x_3&		\cdots&		x_n\\
x_1&		x_2-a_2&		x_3&		\cdots&		x_n\\
x_1&		x_2&		x_3-a_3&		\cdots&		x_n\\
\vdots&		\vdots&		\vdots&		&		\vdots\\
x_1&		x_2&		x_3&		\cdots&		x_n-a_n\\
\end{matrix} \right|\xlongequal[i=2,\cdots ,n]{\left( -1 \right) r_1+r_i}\left| \begin{matrix}
x_1-a_1&		x_2&		x_3&		\cdots&		x_n\\
a_1&		-a_2&		0&		\cdots&		0\\
a_1&		0&		-a_3&		\cdots&		0\\
\vdots&		\vdots&		\vdots&		&		\vdots\\
a_1&		0&		0&		\cdots&		-a_n\\
\end{matrix} \right|
\\
&\xlongequal{\text{命题}\ref{"爪"型行列式}}\left[ \left( x_1-a_1 \right) +\sum_{i=2}^n{\frac{a_1x_i}{a_i}} \right] \prod\limits_{i=2}^n{\left( -a_i \right)}=\left( -1 \right) ^{n-1}\left[ \left( x_1-a_1 \right) +\sum_{i=2}^n{\frac{a_1x_i}{a_i}} \right] \prod\limits_{i=2}^n{a_i}
\\
&=\left( -1 \right) ^{n-1}\left[ \left( x_1-a_1 \right) \prod\limits_{i=2}^n{a_i}+\sum_{i=2}^n{a_1a_2\cdots \widehat{a_i}\cdots a_n}x_i \right] .
\end{split}
\nonumber
\end{equation}

当$\exists i\in \left[ 2,n \right] \cap \mathbb{N}\,\,s.t.\,\, a_i=0$时,我们有
\begin{equation}
\begin{split}
|\boldsymbol{A}|&=\left| \begin{matrix}
x_1-a_1&		x_2&		x_3&		\cdots&		x_n\\
x_1&		x_2-a_2&		x_3&		\cdots&		x_n\\
x_1&		x_2&		x_3-a_3&		\cdots&		x_n\\
\vdots&		\vdots&		\vdots&		&		\vdots\\
x_1&		x_2&		x_3&		\cdots&		x_n-a_n\\
\end{matrix} \right|\xlongequal[i=2,\cdots ,n]{\left( -1 \right) r_1+r_i}\left| \begin{matrix}
x_1-a_1&		x_2&		x_3&		\cdots&		x_n\\
a_1&		-a_2&		0&		\cdots&		0\\
a_1&		0&		-a_3&		\cdots&		0\\
\vdots&		\vdots&		\vdots&		&		\vdots\\
a_1&		0&		0&		\cdots&		-a_n\\
\end{matrix} \right|
\\
&\xlongequal{\text{命题}\ref{"爪"型行列式}}\left( x_1-a_1 \right) \left( -a_2 \right) \left( -a_3 \right) \cdots \left( -a_n \right) -\sum_{i=2}^n{\left( -a_2 \right) \cdots \widehat{\left( -a_i \right) }\cdots \left( -a_n \right)}a_1x_i
\\
&=\left( -1 \right) ^{n-1}\left( x_1-a_1 \right) \prod\limits_{i=2}^n{a_i}+\left( -1 \right) ^{n-1}\sum_{i=2}^n{a_1a_2\cdots \widehat{a_i}\cdots a_n}x_i
\\
&=\left( -1 \right) ^{n-1}\left[ \left( x_1-a_1 \right) \prod\limits_{i=2}^n{a_i}+\sum_{i=2}^n{a_1a_2\cdots \widehat{a_i}\cdots a_n}x_i \right] .            
\end{split}
\nonumber
\end{equation}
综上所述,$|\boldsymbol{A}|=\left( -1 \right) ^{n-1}\left[ \left( x_1-a_1 \right) \prod\limits_{i=2}^n{a_i}+\sum_{i=2}^n{a_1a_2\cdots \widehat{a_i}\cdots a_nx_i} \right]$.
\end{solution}

\begin{proposition}\label{根据行列式代数余子式构造行列式}
设\(\vert A\vert=\vert a_{i}\vert\)是一个\(n\)阶行列式,\(A_{ij}\)是它的第\((i,j)\)元素的代数余子式,求证:
\begin{gather}
\left| \begin{matrix}
a_{11}&		a_{12}&		\cdots&		a_{1n}&		x_1\\
a_{21}&		a_{22}&		\cdots&		a_{2n}&		x_2\\
\vdots&		\vdots&		&		\vdots&		\vdots\\
a_{n1}&		a_{n2}&		\cdots&		a_{nn}&		x_n\\
y_1&		y_2&		\cdots&		y_n&		z\\
\end{matrix} \right|=z|\boldsymbol{A}|-\sum_{i=1}^n{\sum_{j=1}^n{A_{ij}x_iy_j.}}
\nonumber
\end{gather}
\end{proposition}
\begin{note}\label{关于行列式|A|所有代数余子式求和的构造}
根据这个命题可以得到一个\textbf{关于行列式$|\boldsymbol{A}|$的所有代数余子式求和的构造}:

\begin{align*}
-\sum_{i,j=1}^n{A_{ij}}=\left| \begin{matrix}
\boldsymbol{A}&		\mathbf{1}\\
\mathbf{1}'&		0\\
\end{matrix} \right|=\left| \begin{matrix}
\boldsymbol{\alpha }_{\mathbf{1}}&		\boldsymbol{\alpha }_{\mathbf{2}}&		\cdots&		\boldsymbol{\alpha }_{\boldsymbol{n}}&		\mathbf{1}\\
1&		1&		\cdots&		1&		0\\
\end{matrix} \right|=\left| \begin{matrix}
\boldsymbol{\beta }_{\mathbf{1}}&		1\\
\boldsymbol{\beta }_{\mathbf{2}}&		1\\
\vdots&		\vdots\\
\boldsymbol{\beta }_{\boldsymbol{n}}&		1\\
\mathbf{1}'&		0\\
\end{matrix} \right|.
\end{align*}
其中$|\boldsymbol{A}|$的列向量依次为$\boldsymbol{\alpha }_{\mathbf{1}},\boldsymbol{\alpha }_{\mathbf{2}},\cdots ,\boldsymbol{\alpha }_{\boldsymbol{n}}$,$|\boldsymbol{A}|$的行向量依次为$\boldsymbol{\beta }_{\mathbf{1}},\boldsymbol{\beta }_{\mathbf{2}},\cdots ,\boldsymbol{\beta }_{\boldsymbol{n}}$.并且$\mathbf{1}$表示元素均为1的列向量,$\mathbf{1}'$表示$\mathbf{1}$的转置.
(令上述命题中的$z=0,x_i=y_i=1,i=1,2,\cdots,n$即可得到.)
\end{note}
\begin{remark}
如果需要证明的是矩阵的代数余子式的相关命题,我们可以考虑一下这种构造,即令上述命题中的$z=0$并且待定/任取$x_i,y_i$.
\end{remark}
\begin{proof}
{\color{blue}证法一:}
将上述行列式先按最后一列展开,展开式的第一项为
\begin{equation}
\begin{split}
\left( -1 \right) ^{n+2}x_1\left| \begin{matrix}
a_{21}&		a_{22}&		\cdots&		a_{2n}\\
\vdots&		\vdots&		&		\vdots\\
a_{n1}&		a_{n2}&		\cdots&		a_{nn}\\
y_1&		y_2&		\cdots&		y_n\\
\end{matrix} \right|.
\end{split}
\nonumber
\end{equation}
再将上式按最后一行展开得到
\begin{equation}
\begin{split}
&\left( -1 \right) ^{n+2}x_1\left[ \left( -1 \right) ^{n+1}\left( -1 \right) ^{1+1}y_1A_{11}+\left( -1 \right) ^{n+2}\left( -1 \right) ^{1+2}y_2A_{12}+\cdots +\left( -1 \right) ^{n+n}\left( -1 \right) ^{1+n}y_nA_{1n} \right]
\\
&=\left( -1 \right) ^{n+2}x_1\left( -1 \right) ^{n+1}\left[ \left( -1 \right) ^2y_1A_{11}+\left( -1 \right) ^4y_2A_{12}+\cdots +\left( -1 \right) ^{2n}y_nA_{1n} \right] 
\\
&=-x_1\left( y_1A_{11}+y_2A_{12}+\cdots +y_nA_{1n} \right)
\\
&=-x_1\sum_{j=1}^n{y_jA_{1j}}.            
\end{split}
\nonumber
\end{equation}
同理可得原行列式展开式的第$i(i=1,2,\cdots,n-1)$项为
\begin{equation}
\begin{split}
\left( -1 \right) ^{n+1+i}x_i\left| \begin{matrix}
a_{11}&		a_{12}&		\cdots&		a_{1n}\\
\vdots&		\vdots&		&		\vdots\\
a_{i-1,1}&		a_{i-1,2}&		\cdots&		a_{i-1,n}\\
a_{i+1,1}&		a_{i+1,2}&		\cdots&		a_{i+1,n}\\
\vdots&		\vdots&		&		\vdots\\
a_{n1}&		a_{n2}&		\cdots&		a_{nn}\\
y_1&		y_2&		\cdots&		y_n\\
\end{matrix} \right|.
\end{split}
\nonumber
\end{equation}
将上式按最后一行展开得到$z\left|\boldsymbol{A}\right|$.
\begin{equation}
\begin{split}
&\left( -1 \right) ^{n+1+i}x_i\left[ \left( -1 \right) ^{n+1}\left( -1 \right) ^{i+1}y_1A_{i1}+\left( -1 \right) ^{n+2}\left( -1 \right) ^{i+2}y_2A_{i2}+\cdots +\left( -1 \right) ^{n+n}\left( -1 \right) ^{i+n}y_nA_{in} \right] 
\\
&=\left( -1 \right) ^{n+1+i}x_i\left( -1 \right) ^{n+1}\left[ \left( -1 \right) ^{i+1}y_1A_{i1}+\left( -1 \right) ^{i+2+1}y_2A_{i2}+\cdots +\left( -1 \right) ^{i+n+n-1}y_nA_{in} \right] 
\\
&=\left( -1 \right) ^{2i+1}y_1A_{i1}+\left( -1 \right) ^{2i+3}y_2A_{i2}+\cdots +\left( -1 \right) ^{2i+2n-1}y_nA_{in}
\\
&=-x_i\left( y_1A_{i1}+y_2A_{i2}+\cdots +y_nA_{in} \right) 
\\
&=-x_i\sum_{j=1}^n{y_jA_{ij}.}
\end{split}
\nonumber
\end{equation}
而展开式的最后一项为$z\left|\boldsymbol{A}\right|$.

因此,原行列式的值为
\begin{equation}
z|\boldsymbol{A}|-\sum_{i=1}^n{\sum_{j=1}^n{A_{ij}x_iy_j.}}
\nonumber
\end{equation}

{\color{blue}证法二:}设\(\boldsymbol{x}=(x_1,x_2,\cdots,x_n)',\boldsymbol{y}=(y_1,y_2,\cdots,y_n)'\). 若\(A\)是非异阵,则由降阶公式可得
\[
\begin{vmatrix}
A & \boldsymbol{x}\\
\boldsymbol{y}' & z
\end{vmatrix}=|A|(z - \boldsymbol{y}'A^{-1}\boldsymbol{x})=z|A| - \boldsymbol{y}'A^*\boldsymbol{x}.
\]

对于一般的方阵\(A\),可取到一列有理数\(t_k\rightarrow0\),使得\(t_kI_n + A\)为非异阵. 由非异阵情形的证明可得
\[
\begin{vmatrix}
t_kI_n + A & \boldsymbol{x}\\
\boldsymbol{y}' & z
\end{vmatrix}=z|t_kI_n + A| - \boldsymbol{y}'(t_kI_n + A)^*\boldsymbol{x}.
\]

注意到上式两边都是关于\(t_k\)的多项式,从而关于\(t_k\)连续. 上式两边同时取极限,令\(t_k\rightarrow0\),即有
\[
\begin{vmatrix}
A & \boldsymbol{x}\\
\boldsymbol{y}' & z
\end{vmatrix}=z|A| - \boldsymbol{y}'A^*\boldsymbol{x}=z|A|-\sum_{i = 1}^{n}\sum_{j = 1}^{n}A_{ij}x_iy_j.
\]
\end{proof}

\begin{example}\label{example:求矩阵代数余子式和的方法1}
设\(n\)阶行列式\(\vert \boldsymbol{A} \vert=\vert a_{ij}\vert\),\(A_{ij}\)是元素\(a_{ij}\)的代数余子式,求证:
\[
\vert B \vert = 
\begin{vmatrix}
a_{11}-a_{12} & a_{12}-a_{13} & \cdots & a_{1,n - 1}-a_{1n} & 1\\
a_{21}-a_{22} & a_{22}-a_{23} & \cdots & a_{2,n - 1}-a_{2n} & 1\\
a_{31}-a_{32} & a_{32}-a_{33} & \cdots & a_{3,n - 1}-a_{3n} & 1\\
\vdots & \vdots & \ddots & \vdots & \vdots\\
a_{n1}-a_{n2} & a_{n2}-a_{n3} & \cdots & a_{n,n - 1}-a_{nn} & 1
\end{vmatrix}
= \sum_{i,j = 1}^{n}A_{ij}.
\]
\end{example}
\begin{proof}
{\color{blue}证法一:}设\(|\boldsymbol{A}|\)的列向量依次为\(\boldsymbol{\alpha }_{\mathbf{1}},\boldsymbol{\alpha }_{\mathbf{2}},\cdots ,\boldsymbol{\alpha }_{\boldsymbol{n}}\),并且\(\mathbf{1}\)表示元素均为\(1\)的列向量.则
\begin{align*}
|\boldsymbol{B}|=|\boldsymbol{\alpha }_{\mathbf{1}}-\boldsymbol{\alpha }_{\mathbf{2}},\boldsymbol{\alpha }_{\mathbf{2}}-\boldsymbol{\alpha }_{\mathbf{3}},\cdots ,\boldsymbol{\alpha }_{\boldsymbol{n}-\mathbf{1}}-\boldsymbol{\alpha }_{\boldsymbol{n}},1|\xlongequal[i=n-1,n-2,\cdots ,2]{j_i+j_{i-1}}|\boldsymbol{\alpha }_{\mathbf{1}}-\boldsymbol{\alpha }_{\boldsymbol{n}},\boldsymbol{\alpha }_{\mathbf{2}}-\boldsymbol{\alpha }_{\boldsymbol{n}},\cdots ,\boldsymbol{\alpha }_{\boldsymbol{n}-\mathbf{1}}-\boldsymbol{\alpha }_{\boldsymbol{n}},1|.        
\end{align*}
将最后一列写成\((\boldsymbol{\alpha}_{\boldsymbol{n}} + \mathbf{1}) - \boldsymbol{\alpha}_{\boldsymbol{n}}\),进行拆分可得
\begin{align*}
&|\boldsymbol{B}| = |\boldsymbol{\alpha}_{\boldsymbol{1}} - \boldsymbol{\alpha}_{\boldsymbol{n}},\boldsymbol{\alpha}_{\boldsymbol{2}} - \boldsymbol{\alpha}_{\boldsymbol{n}},\cdots,\boldsymbol{\alpha}_{\boldsymbol{n - 1}} - \boldsymbol{\alpha}_n,(\boldsymbol{\alpha}_{\boldsymbol{n}} + \mathbf{1}) - \boldsymbol{\alpha}_{\boldsymbol{n}}|
\\
&= |\boldsymbol{\alpha}_{\boldsymbol{1}} - \boldsymbol{\alpha}_{\boldsymbol{n}},\boldsymbol{\alpha}_{\boldsymbol{2}} - \boldsymbol{\alpha}_{\boldsymbol{n}},\cdots,\boldsymbol{\alpha}_{\boldsymbol{n - 1}} - \boldsymbol{\alpha}_{\boldsymbol{n}},\boldsymbol{\alpha}_{\boldsymbol{n}} + \mathbf{1}| - |\boldsymbol{\alpha}_{\boldsymbol{1}} - \boldsymbol{\alpha}_{\boldsymbol{n}},\boldsymbol{\alpha}_{\boldsymbol{2}} - \boldsymbol{\alpha}_{\boldsymbol{n}},\cdots,\boldsymbol{\alpha}_{\boldsymbol{n - 1}} - \boldsymbol{\alpha}_{\boldsymbol{n}},\boldsymbol{\alpha}_{\boldsymbol{n}}|
\\
&= |\boldsymbol{\alpha}_{\boldsymbol{1}} + \mathbf{1},\boldsymbol{\alpha}_{\boldsymbol{2}} + \mathbf{1},\cdots,\boldsymbol{\alpha}_{\boldsymbol{n - 1}} + \mathbf{1},\boldsymbol{\alpha}_{\boldsymbol{n}} + \mathbf{1}| - |\boldsymbol{\alpha}_{\boldsymbol{1}},\boldsymbol{\alpha}_{\boldsymbol{2}},\cdots,\boldsymbol{\alpha}_{\boldsymbol{n-1}},\boldsymbol{\alpha}_{\boldsymbol{n}}|.
\end{align*}
根据行列式的性质将\(|\boldsymbol{\alpha}_{\boldsymbol{1}} + \mathbf{1},\boldsymbol{\alpha}_{\boldsymbol{2}} + \mathbf{1},\cdots,\boldsymbol{\alpha}_{\boldsymbol{n-1}} + \mathbf{1},\boldsymbol{\alpha}_{\boldsymbol{n}} + \mathbf{1}|\)每一列都拆分成两列,然后按\(1\)所在的列展开得到
\begin{align*}
&|\boldsymbol{B}| = |\boldsymbol{\alpha}_{\boldsymbol{1}} + \mathbf{1},\boldsymbol{\alpha}_{\boldsymbol{2}} + \mathbf{1},\cdots,\boldsymbol{\alpha}_{\boldsymbol{n-1}} + \mathbf{1},\boldsymbol{\alpha}_{\boldsymbol{n}} + \mathbf{1}| - |\boldsymbol{\alpha}_{\boldsymbol{1}},\boldsymbol{\alpha}_{\boldsymbol{2}},\cdots,\boldsymbol{\alpha}_{\boldsymbol{n-1}},\boldsymbol{\alpha}_{\boldsymbol{n}}|
\\
&= |\boldsymbol{\alpha}_{\boldsymbol{1}},\boldsymbol{\alpha}_{\boldsymbol{2}},\cdots,\boldsymbol{\alpha}_{\boldsymbol{n-1}},\boldsymbol{\alpha}_{\boldsymbol{n}}| + \sum_{i,j = 1}^{n}A_{ij} - |\boldsymbol{\alpha}_{\boldsymbol{1}},\boldsymbol{\alpha}_2,\cdots,\boldsymbol{\alpha}_{\boldsymbol{n-1}},\boldsymbol{\alpha}_{\boldsymbol{n}}| = \sum_{i,j = 1}^{n}A_{ij}.
\end{align*}

{\color{blue}证法二:}设\(|\boldsymbol{A}|\)的列向量依次为\(\boldsymbol{\alpha}_{\boldsymbol{1}},\boldsymbol{\alpha}_{\boldsymbol{2}},\cdots,\boldsymbol{\alpha}_{\boldsymbol{n}}\),并且\(\mathbf{1}\)表示元素均为\(1\)的列向量.\hyperref[关于行列式|A|所有代数余子式求和的构造]{注意到}
\begin{align*}
-\sum_{i,j=1}^n{A_{ij}}=\left| \begin{matrix}
\boldsymbol{\alpha }_{\mathbf{1}}&		\boldsymbol{\alpha }_{\mathbf{2}}&		\cdots&		\boldsymbol{\alpha }_{\boldsymbol{n}}&		\mathbf{1}\\
1&		1&		\cdots&		1&		0\\
\end{matrix} \right|.
\end{align*}
依次将第$i$列乘以$-1$加到第$i-1$列上去$(i=2,3,\cdots,n)$,再按第$n+1$行展开可得
\begin{align*}
-\sum_{i,j=1}^n{A_{ij}=\left| \begin{matrix}
\boldsymbol{\alpha }_{\mathbf{1}}-\boldsymbol{\alpha }_{\mathbf{2}}&		\boldsymbol{\alpha }_{\mathbf{2}}-\boldsymbol{\alpha }_{\mathbf{3}}&		\cdots&		\boldsymbol{\alpha }_{\boldsymbol{n}-\mathbf{1}}-\boldsymbol{\alpha }_{\boldsymbol{n}}&		\boldsymbol{\alpha }_{\boldsymbol{n}}&		1\\
0&		0&		\cdots&		0&		1&		0\\
\end{matrix} \right|}
\\
=-|\boldsymbol{\alpha }_{\mathbf{1}}-\boldsymbol{\alpha }_{\mathbf{2}},\boldsymbol{\alpha }_{\mathbf{2}}-\boldsymbol{\alpha }_{\mathbf{3}},\cdots ,\boldsymbol{\alpha }_{\boldsymbol{n}-\mathbf{1}}-\boldsymbol{\alpha }_{\boldsymbol{n}},1|=-|\boldsymbol{B}|.
\end{align*}
结论得证.
\end{proof}

\begin{example}
设\(n\)阶矩阵\(A\)的每一行、每一列的元素之和都为零,证明:\(A\)的每个元素的代数余子式都相等.
\end{example}
\begin{proof}
{\color{blue}证法一:}设\(A=(a_{ij})\),\(\boldsymbol{x}=(x_1,x_2,\cdots,x_n)'\),\(\boldsymbol{y}=(y_1,y_2,\cdots,y_n)'\),不妨设$x_iy_j$均不相同,$i,j=1,2,\cdots,n$.考虑如下\(n + 1\)阶矩阵的行列式求值:
\[
B=\begin{pmatrix}
A & \boldsymbol{x}\\
\boldsymbol{y}' & 0
\end{pmatrix}
\]
一方面,由\hyperref[根据行列式代数余子式构造行列式]{命题\ref{根据行列式代数余子式构造行列式}}可得\(|B|=-\sum_{i = 1}^{n}\sum_{j = 1}^{n}A_{ij}x_iy_j\). 另一方面,先把行列式\(|B|\)的第二行,\(\cdots\),第\(n\)行全部加到第一行上;再将第二列,\(\cdots\),第\(n\)列全部加到第一列上,可得
\[
\begin{vmatrix}
a_{11}&a_{12}&\cdots&a_{1n}&x_1\\
a_{21}&a_{22}&\cdots&a_{2n}&x_2\\
\vdots&\vdots&&\vdots&\vdots\\
a_{n1}&a_{n2}&\cdots&a_{nn}&x_n\\
y_1&y_2&\cdots&y_n&0
\end{vmatrix}=
\begin{vmatrix}
0&0&\cdots&0&\sum_{i = 1}^{n}x_i\\
a_{21}&a_{22}&\cdots&a_{2n}&x_2\\
\vdots&\vdots&&\vdots&\vdots\\
a_{n1}&a_{n2}&\cdots&a_{nn}&x_n\\
y_1&y_2&\cdots&y_n&0
\end{vmatrix}=
\begin{vmatrix}
0&0&\cdots&0&\sum_{i = 1}^{n}x_i\\
0&a_{22}&\cdots&a_{2n}&x_2\\
\vdots&\vdots&&\vdots&\vdots\\
0&a_{n2}&\cdots&a_{nn}&x_n\\
\sum_{j = 1}^{n}y_j&y_2&\cdots&y_n&0
\end{vmatrix}
\]
依次按照第一行和第一列进行展开,可得\(|B|=-A_{11}\sum_{i = 1}^{n}\sum_{j = 1}^{n}x_iy_j\). 比较上述两个结果,又由于$x_iy_j$均不
相同,因此可得\(A\)的所有代数余子式都相等.

{\color{blue}证法二:}由假设可知$\left| \boldsymbol{A} \right|=0$(每行元素全部加到第一行即得),从而\(\boldsymbol{A}\)是奇异矩阵. 若\(\boldsymbol{A}\)的秩小于\(n - 1\),则\(\boldsymbol{A}\)的任意一个代数余子式\(A_{ij}\)都等于零,结论显然成立. 若\(\boldsymbol{A}\)的秩等于\(n - 1\),则线性方程组\(\boldsymbol{A}\boldsymbol{x}=\boldsymbol{0}\)的基础解系只含一个向量. 又因为\(\boldsymbol{A}\)的每一行元素之和都等于零,所以由\hyperref[proposition:对矩阵行和和列和的一种刻画]{命题\ref{proposition:对矩阵行和和列和的一种刻画}}可知,我们可以选取\(\boldsymbol{\alpha}=(1,1,\cdots,1)'\)作为\(\boldsymbol{A}\boldsymbol{x}=\boldsymbol{0}\)的基础解系. 由\hyperref[proposition:奇异系数矩阵Ax=0的解空间]{命题\ref{proposition:奇异系数矩阵Ax=0的解空间}的证明}可知\(\boldsymbol{A}^*\)的每一列都是$\boldsymbol{A}\boldsymbol{x}=\boldsymbol{0}$的解,从而\(\boldsymbol{A}^*\)的每一列与\(\boldsymbol{\alpha}\)成比例,特别地,\(\boldsymbol{A}^*\)的每一行都相等. 对\(\boldsymbol{A}'\)重复上面的讨论,可得\((\boldsymbol{A}')^*\)的每一行都相等.注意到\((\boldsymbol{A}')^*=(\boldsymbol{A}^*)'\),从而\(\boldsymbol{A}^*\)的每一列都相等,于是\(\boldsymbol{A}\)的所有代数余子式\(A_{ij}\)都相等.
\end{proof}

\begin{proposition}[\hypertarget{三对角行列式}{三对角行列式}]\label{三对角行列式}
求下列行列式的递推关系式(空白处均为0):
\begin{equation}
\begin{split}
D_n=\left| \begin{matrix}
a_1&		b_1&		&		&		&		\\
c_1&		a_2&		b_2&		&		&		\\
&		c_2&		a_3&		\ddots&		&		\\
&		&		\ddots&		\ddots&		\ddots&		\\
&		&		&		\ddots&		a_{n-1}&		b_{n-1}\\
&		&		&		&		c_{n-1}&		a_n\\
\end{matrix} \right|.
\end{split}
\nonumber
\end{equation}
\end{proposition}
\begin{note}
记忆三对角行列式的计算方法和结果:
$\boldsymbol{D}_{\boldsymbol{n}}=\boldsymbol{a}_{\boldsymbol{n}}\boldsymbol{D}_{\boldsymbol{n}-\boldsymbol{1}}-\boldsymbol{b}_{\boldsymbol{n}-\boldsymbol{1}}\boldsymbol{c}_{\boldsymbol{n}-\boldsymbol{1}}\boldsymbol{D}_{\boldsymbol{n}-\boldsymbol{2}}\boldsymbol{(n}\ge \boldsymbol{2)}$,

即按最后一列(或行)展开得到递推公式.
\end{note}
\begin{solution}
显然$D_0=1,D_1=a_1$.当$n\ge2$时,我们有
\begin{align*}
D_n&=\left| \begin{matrix}
a_1&		b_1&		&		&		&		\\
c_1&		a_2&		b_2&		&		&		\\
&		c_2&		a_3&		\ddots&		&		\\
&		&		\ddots&		\ddots&		\ddots&		\\
&		&		&		\ddots&		a_{n-1}&		b_{n-1}\\
&		&		&		&		c_{n-1}&		a_n\\
\end{matrix} \right|=\left| \begin{matrix}
a_1&		b_1&		&		&		&		&		\\
c_1&		a_2&		b_2&		&		&		&		\\
&		c_2&		a_3&		\ddots&		&		&		\\
&		&		\ddots&		\ddots&		\ddots&		&		\\
&		&		&		\ddots&		a_{n-2}&		b_{n-2}&		\\
&		&		&		&		c_{n-2}&		a_{n-1}&		b_{n-1}\\
&		&		&		&		&		c_{n-1}&		a_n\\
\end{matrix} \right|
\\
&\xlongequal[]{\text{按最后一列展开}}a_n\left| \begin{matrix}
a_1&		b_1&		&		&		&		\\
c_1&		a_2&		b_2&		&		&		\\
&		c_2&		a_3&		\ddots&		&		\\
&		&		\ddots&		\ddots&		\ddots&		\\
&		&		&		\ddots&		a_{n-2}&		b_{n-2}\\
&		&		&		&		c_{n-2}&		a_{n-1}\\
\end{matrix} \right|-b_{n-1}\left| \begin{matrix}
a_1&		b_1&		&		&		&		&		\\
c_1&		a_2&		b_2&		&		&		&		\\
&		c_2&		a_3&		\ddots&		&		&		\\
&		&		\ddots&		\ddots&		\ddots&		&		\\
&		&		&		\ddots&		a_{n-3}&		b_{n-3}&		\\
&		&		&		&		c_{n-3}&		a_{n-2}&		b_{n-2}\\
&		&		&		&		&		0&		c_{n-1}\\
\end{matrix} \right|
\\
&\xlongequal[]{\text{第二项按最后}\mathbf{一行}\text{展开}}a_n\left| \begin{matrix}
a_1&		b_1&		&		&		&		\\
c_1&		a_2&		b_2&		&		&		\\
&		c_2&		a_3&		\ddots&		&		\\
&		&		\ddots&		\ddots&		\ddots&		\\
&		&		&		\ddots&		a_{n-2}&		b_{n-2}\\
&		&		&		&		c_{n-2}&		a_{n-1}\\
\end{matrix} \right|-b_{n-1}c_{n-1}\left| \begin{matrix}
a_1&		b_1&		&		&		&		\\
c_1&		a_2&		b_2&		&		&		\\
&		c_2&		a_3&		\ddots&		&		\\
&		&		\ddots&		\ddots&		\ddots&		\\
&		&		&		\ddots&		a_{n-3}&		b_{n-3}\\
&		&		&		&		c_{n-3}&		a_{n-2}\\
\end{matrix} \right|
\\
&=a_nD_{n-1}-b_{n-1}c_{n-1}D_{n-2}.
\nonumber
\end{align*}
\end{solution}

\begin{proposition}[\hypertarget{大拆分法}{大拆分法}]\label{大拆分法}
设\(t\)是一个参数,
\begin{align*}
|A(t)| = 
\begin{vmatrix}
a_{11}+t & a_{12}+t & \cdots & a_{1n}+t \\
a_{21}+t & a_{22}+t & \cdots & a_{2n}+t \\
\vdots & \vdots & \ddots & \vdots \\
a_{n1}+t & a_{n2}+t & \cdots & a_{nn}+t
\end{vmatrix}
\nonumber
\end{align*}

求证:
\begin{align*}
|A(t)| = |A(0)| + t \sum_{i,j = 1}^{n} A_{ij},
\nonumber
\end{align*}
其中\(A_{ij}\)是\(a_{ij}\)在\(|A(0)|\)中的代数余子式.
\end{proposition}
\begin{note}
大拆分法的想法:
\textbf{将行列式的每一行/列拆分成两行/列},得到
\begin{align*}
|\boldsymbol{A}(t)|=|\boldsymbol{A}(0)|+t\sum_{j=1}^n{|A_j|}.
\text{其中}A_j=\bordermatrix{%
&1 &	\cdots	&	i&	\cdots	&n		\cr
& a_{11}&		\cdots&		t&		\cdots&		a_{1n}\cr
&a_{21}&		\cdots&		t&		\cdots&		a_{2n}\cr
&\vdots&		&		\vdots&		&		\vdots\cr
&a_{n1}&		\cdots&		t&		\cdots&		a_{nn}
},j=1,2,\cdots ,n.
\end{align*}
大拆分法的关键是\textbf{拆分},根据行列式的性质将原行列式拆分成$2^n$个行列式.(不一定需要公共的$t$).不仅要熟悉大拆分法的想法还要记住大拆分法的这个命题.
\end{note}
\begin{remark}
大拆分法后续计算不一定要按行/列展开,拆分的方式一般比较多,只要拆分的方式方便后续计算即可.
\end{remark}
\begin{proof}
将行列式第一列拆成两列再展开得到
\begin{align*}
|\boldsymbol{A}(t)|=\left| \begin{matrix}
a_{11}&		a_{12}+t&		\cdots&		a_{1n}+t\\
a_{21}&		a_{22}+t&		\cdots&		a_{2n}+t\\
\vdots&		\vdots&		&		\vdots\\
a_{n1}&		a_{n2}+t&		\cdots&		a_{nn}+t\\
\end{matrix} \right|+\left| \begin{matrix}
t&		a_{12}+t&		\cdots&		a_{1n}+t\\
t&		a_{22}+t&		\cdots&		a_{2n}+t\\
\vdots&		\vdots&		&		\vdots\\
t&		a_{n2}+t&		\cdots&		a_{nn}+t\\
\end{matrix} \right|.
\nonumber
\end{align*}
将上式右边第二个行列式的第一列乘-1加到后面每一列上,得到
\begin{align*}
\left| \boldsymbol{A} \right|=\left| \begin{matrix}
a_{11}&		a_{12}+t&		\cdots&		a_{1n}+t\\
a_{21}&		a_{22}+t&		\cdots&		a_{2n}+t\\
\vdots&		\vdots&		&		\vdots\\
a_{n1}&		a_{n2}+t&		\cdots&		a_{nn}+t\\
\end{matrix} \right|+\left| \begin{matrix}
t&		a_{12}&		\cdots&		a_{1n}\\
t&		a_{22}&		\cdots&		a_{2n}\\
\vdots&		\vdots&		&		\vdots\\
t&		a_{n2}&		\cdots&		a_{nn}\\
\end{matrix} \right| .
\nonumber
\end{align*}
再对上式右边第一个行列式的第二列拆成两列展开,不断这样做下去就可得到
\begin{gather*}
|\boldsymbol{A}(t)|=\left| \begin{matrix}
a_{11}&		a_{12}&		\cdots&		a_{1n}\\
a_{21}&		a_{22}&		\cdots&		a_{2n}\\
\vdots&		\vdots&		&		\vdots\\
a_{n1}&		a_{n2}&		\cdots&		a_{nn}\\
\end{matrix} \right|+\left| \begin{matrix}
t&		a_{12}&		\cdots&		a_{1n}\\
t&		a_{22}&		\cdots&		a_{2n}\\
\vdots&		\vdots&		&		\vdots\\
t&		a_{n2}&		\cdots&		a_{nn}\\
\end{matrix} \right|+\cdots +\left| \begin{matrix}
a_{11}&		a_{1n}&		\cdots&		t\\
a_{21}&		a_{2n}&		\cdots&		t\\
\vdots&		\vdots&		&		\vdots\\
a_{n1}&		a_{nn}&		\cdots&		t\\
\end{matrix} \right|=|\boldsymbol{A}(0)|+\sum_{j=1}^n{|A_j|}.
\end{gather*}
其中$A_j=\bordermatrix{%
&1 &	\cdots	&	i&	\cdots	&n		\cr
& a_{11}&		\cdots&		t&		\cdots&		a_{1n}\cr
&a_{21}&		\cdots&		t&		\cdots&		a_{2n}\cr
&\vdots&		&		\vdots&		&		\vdots\cr
&a_{n1}&		\cdots&		t&		\cdots&		a_{nn}
}$,$j=1,2,\cdots ,n.$
将$A_j$按第$j$列展开可得
\begin{align*}
A_j=\left| \begin{matrix}
a_{11}&		\cdots&		t&		\cdots&		a_{1n}\\
a_{21}&		\cdots&		t&		\cdots&		a_{2n}\\
\vdots&		&		\vdots&		&		\vdots\\
a_{n1}&		\cdots&		t&		\cdots&		a_{nn}\\
\end{matrix} \right|=t\left( A_{1j}+A_{2j}+\cdots +A_{nj} \right) =t\sum_{i=1}^n{A_{ij}}.
\nonumber
\end{align*}
从而
\begin{align*}
|\boldsymbol{A}(t)|=|\boldsymbol{A}(0)|+\sum_{i=1}^n{A_i}=|\boldsymbol{A}(0)|+t\sum_{i=1}^n{\sum_{i=1}^n{A_{ij}}}=|\boldsymbol{A}(0)|+t\sum_{i,j=1}^n{A_{ij}}.
\end{align*}
\end{proof}

\begin{corollary}[\hypertarget{大拆分法的推广}{推广的大拆分法}]\label{大拆分法的推广}
设
\begin{align*}
|A| = 
\begin{vmatrix}
a_{11} & a_{12} & \cdots & a_{1n} \\
a_{21} & a_{22} & \cdots & a_{2n} \\
\vdots & \vdots & \ddots & \vdots \\
a_{n1} & a_{n2} & \cdots & a_{nn}
\end{vmatrix},
\nonumber
\end{align*}
则
\begin{align*}
|A(t_1,t_2,\cdots,t_n)| = 
\begin{vmatrix}
a_{11}+t_1 & a_{12}+t_2 & \cdots & a_{1n}+t_n \\
a_{21}+t_1 & a_{22}+t_2 & \cdots & a_{2n}+t_n \\
\vdots & \vdots & \ddots & \vdots \\
a_{n1}+t_1 & a_{n2}+t_2 & \cdots & a_{nn}+t_n
\end{vmatrix}
= |A| + \sum_{j = 1}^{n} \left( t_j \sum_{i = 1}^{n} A_{ij} \right).
\nonumber
\end{align*}
\end{corollary}
\begin{note}
记忆这种推广的大拆分法的想法(即\textbf{将行列式的每一行/列拆分成两行/列}).

这里推广的大拆分法的关键也是\textbf{要找到合适的$t_1,t_2,\cdots,t_n$}进行拆分将原行列式拆分成更好处理的形式.
\end{note}
\begin{remark}
大拆分法后续计算不一定要按行/列展开,拆分的方式一般比较多,只要拆分的方式方便后续计算即可.
\end{remark}
\begin{proof}
运用\hyperlink{大拆分法}{大拆分法}的证明方法不难得到.
\end{proof}

\begin{proposition}[\hypertarget{小拆分法}{小拆分法}]\label{小拆分法}
设
\begin{align*}
|A| = 
\begin{vmatrix}
a_{11} & a_{12} & \cdots & a_{1n} \\
a_{21} & a_{22} & \cdots & a_{2n} \\
\vdots & \vdots & \ddots & \vdots \\
a_{n1} & a_{n2} & \cdots & a_{nn}
\end{vmatrix},
\nonumber
\end{align*}
并且$a_{in}$可以拆分成$b_{in}+c_{in}$,$\,\,i=1,2,\cdots,n.$

则
\begin{align*}
\left| \boldsymbol{A} \right|=\left| \begin{matrix}
a_{11}&		a_{12}&		\cdots&		a_{1n}\\
a_{21}&		a_{22}&		\cdots&		a_{2n}\\
\vdots&		\vdots&		&		\vdots\\
a_{n1}&		a_{n2}&		\cdots&		a_{nn}\\
\end{matrix} \right|=\left| \begin{matrix}
a_{11}&		a_{12}&		\cdots&		b_{1n}+c_{1n}\\
a_{21}&		a_{22}&		\cdots&		b_{2n}+c_{2n}\\
\vdots&		\vdots&		&		\vdots\\
a_{n1}&		a_{n2}&		\cdots&		b_{nn}+c_{nn}\\
\end{matrix} \right|=\left| \begin{matrix}
a_{11}&		a_{12}&		\cdots&		b_{1n}\\
a_{21}&		a_{22}&		\cdots&		b_{2n}\\
\vdots&		\vdots&		&		\vdots\\
a_{n1}&		a_{n2}&		\cdots&		b_{nn}\\
\end{matrix} \right|+\left| \begin{matrix}
a_{11}&		a_{12}&		\cdots&		c_{1n}\\
a_{21}&		a_{22}&		\cdots&		c_{2n}\\
\vdots&		\vdots&		&		\vdots\\
a_{n1}&		a_{n2}&		\cdots&		c_{nn}\\
\end{matrix} \right|.
\end{align*}
\end{proposition}
\begin{note}
记忆小拆分法的想法(即\textbf{拆边列/行,再展开得到递推式}).
\end{note}
\begin{remark}
若已知的拆分不是最后一列而是其他的某一行或某一列,则可以通过\hyperref[pro:行列式计算常识]{倒排、旋转、翻转、两行或两列对换}的方法将这一行或一列变成最后一列,再按照上述方法进行拆分即可.

小拆分法后续计算也不一定要按行/列展开,拆分的方式一般比较多,只要拆分的方式方便后续计算即可. 
\end{remark}
\begin{proof}
由行列式的性质可直接得到结论.
\end{proof}

\begin{proposition}[行列式的求导运算]\label{proposition:行列式的求导运算}
设\(f_{ij}(t)\)是可微函数,
\begin{align*}
F(t) = 
\left| \begin{matrix}
f_{11}(t) & f_{12}(t) & \cdots & f_{1n}(t) \\
f_{21}(t) & f_{22}(t) & \cdots & f_{2n}(t) \\
\vdots & \vdots &  & \vdots \\
f_{n1}(t) & f_{n2}(t) & \cdots & f_{nn}(t)
\end{matrix} \right| 
\nonumber
\end{align*}
求证:$\frac{d}{dt}F\left( t \right) =\sum_{j=1}^n{F_j\left( t \right)}$,其中
\begin{align*}
F_{j}(t) = 
\left| \begin{matrix}
f_{11}(t)&		f_{12}(t)&		\cdots&		\frac{d}{dt}f_{1j}(t)&		\cdots&		f_{1n}(t)\\
f_{21}(t)&		f_{22}(t)&		\cdots&		\frac{d}{dt}f_{2j}(t)&		\cdots&		f_{2n}(t)\\
\vdots&		\vdots&		&		\vdots&		&		\vdots\\
f_{n1}(t)&		f_{n2}(t)&		\cdots&		\frac{d}{dt}f_{nj}(t)&		\cdots&		f_{nn}(t)\\
\end{matrix} \right| 
\nonumber
\end{align*}
\end{proposition}
\begin{proof}
{\color{blue}证法一(数学归纳法):}对阶数$n$进行归纳.当$n=1$时结论显然成立.假设$n-1$阶时结论成立,现证$n$阶的情形.

将$F(t)$按第一列展开得
\begin{align*}
F\left( t \right) =f_{11}\left( t \right) A_{11}\left( t \right) +f_{21}\left( t \right) A_{21}\left( t \right) +\cdots +f_{n1}\left( t \right) A_{n1}\left( t \right) .
\nonumber
\end{align*}
其中$A_{i1}(t)$是元素$f_{i1}(t)$的代数余子式.($i=1,2,\cdots,n$)

从而由归纳假设可得
\begin{gather*}
A_{i1}^{\prime}\left( t \right) =\frac{d}{dt}A_{i1}\left( t \right)=\sum_{k=2}^{n}{A_{i1}^{k}(t),i=1,2,\cdots ,n}. 
\\
\text{其中}A_{i1}^{k}(t)=\left| \begin{matrix}
f_{12}\left( t \right)&		\cdots&		\frac{d}{dt}f_{1k}\left( t \right)&		\cdots&		f_{1n}\left( t \right)\\
\vdots&		&		\vdots&		&		\vdots\\
f_{i-1,2}(t)&		\cdots&		\frac{d}{dt}f_{i-1,k}\left( t \right)&		\cdots&		f_{i-1,n}\left( t \right)\\
f_{i+1,2}\left( t \right)&		\cdots&		\frac{d}{dt}f_{i+1,k}(t)&		\cdots&		f_{i+1,n}\left( t \right)\\
\vdots&		&		\vdots&		&		\vdots\\
f_{n2}\left( t \right)&		\cdots&		\frac{d}{dt}f_{nk}\left( t \right)&		\cdots&		f_{nn}\left( t \right)\\
\end{matrix} \right|,k=2,3,\cdots ,n.
\nonumber
\end{gather*}
于是,我们就有
\begin{align*}
\frac{d}{dt}F\left( t \right) &=\frac{d}{dt}\left[ f_{11}\left( t \right) A_{11}\left( t \right) +f_{21}\left( t \right) A_{21}\left( t \right) +\cdots +f_{n1}\left( t \right) A_{n1}\left( t \right) \right] 
\\
&=f_{11}^{\prime}\left( t \right) A_{11}\left( t \right) +f_{21}^{\prime}\left( t \right) A_{21}\left( t \right) +\cdots +f_{n1}^{\prime}\left( t \right) A_{n1}\left( t \right) +f_{11}\left( t \right) A_{11}^{\prime}\left( t \right) +f_{21}\left( t \right) A_{21}^{\prime}\left( t \right) +\cdots +f_{n1}\left( t \right) A_{n1}^{\prime}\left( t \right) 
\\
&=\sum_{i=1}^n{f_{i1}^{\prime}\left( t \right) A_{i1}\left( t \right)}+f_{11}\left( t \right) \sum_{k=2}^{n}{A_{11}^{k}(t)}+f_{21}\left( t \right) \sum_{k=2}^{n}{A_{21}^{k}(t)}+\cdots +f_{n1}\left( t \right) \sum_{k=2}^{n}{A_{n1}^{k}(t)}
\\
&=\sum_{i=1}^n{f_{i1}^{\prime}\left( t \right) A_{i1}\left( t \right)}+\sum_{i=1}^n{\left( f_{i1}\left( t \right) \sum_{k=2}^n{A_{i1}^{k}\left( t \right)} \right)}
\\
&=\sum_{i=1}^n{f_{i1}^{\prime}\left( t \right) A_{i1}\left( t \right)}+\sum_{i=1}^n{f_{i1}\left( t \right) \left( A_{i1}^{2}+A_{i1}^{3}+\cdots +A_{i1}^{n} \right)}
\\
&=\sum_{i=1}^n{f_{i1}^{\prime}\left( t \right) A_{i1}\left( t \right)}+\sum_{i=1}^n{f_{i1}\left( t \right) A_{i1}^{2}}+\sum_{i=1}^n{f_{i1}\left( t \right) A_{i1}^{3}}+\cdots +\sum_{i=1}^n{f_{i1}\left( t \right) A_{i1}^{n}}
\\
&=F_1\left( t \right) +F_2\left( t \right) +F_3\left( t \right) +\cdots +F_n\left( t \right) 
\\
&=\sum_{j=1}^n{F_j\left( t \right)}.
\end{align*}
故由数学归纳法可知结论对任意正整数都成立.

{\color{blue}证法二(行列式的组合定义):}由行列式的组合定义可得
\begin{align*}
F(t)=\sum_{1\le k_1,k_2,\cdots ,k_n\le n}{(}-1)^{\tau (k_1k_2\cdots k_n)}f_{k_11}(t)f_{k_22}(t)\cdots f_{k_nn}(t).
\end{align*}
因此
\begin{align*}
\frac{d}{dt}F(t)&=\sum_{1\le k_1,k_2,\cdots ,k_n\le n}{(}-1)^{\tau (k_1k_2\cdots k_n)}f_{k_{11}}(t)f_{k_{22}}(t)\cdots f_{k_{nn}}(t)
\\
&\quad+\sum_{1\le k_1,k_2,\cdots ,k_n\le n}{(}-1)^{\tau (k_1k_2\cdots k_n)}f_{k_{11}}(t)f\prime_{k_{22}}(t)\cdots f_{k_{nn}}(t)
\\
&\quad+\cdots +\sum_{1\le k_1,k_2,\cdots ,k_n\le n}{(}-1)^{\tau (k_1k_2\cdots k_n)}f_{k_{11}}(t)f_{k_{22}}(t)\cdots f\prime_{k_{nn}}(t)
\\
&=F_1(t)+F_2(t)+\cdots +F_n(t).
\end{align*}
\end{proof}

\begin{proposition}[直接计算两个矩阵和的行列式]\label{proposition:直接计算两个矩阵和的行列式}
设\(A,B\)都是\(n\)阶矩阵,求证:
\begin{align*}
|\boldsymbol{A}+\boldsymbol{B}|=|\boldsymbol{A}|+|\boldsymbol{B}|+\sum_{1\le k\le n-1}{\left( \sum_{\substack{1\le i_1<i_2<\cdots <i_k\le n\\1\le j_1<j_2<\cdots <j_k\le n}}{\boldsymbol{A}\left( \begin{matrix}
i_1&		i_2&		\cdots&		i_k\\
j_1&		j_2&		\cdots&		j_k\\
\end{matrix} \right) \widehat{\boldsymbol{B}}\left( \begin{matrix}
i_1&		i_2&		\cdots&		i_k\\
j_1&		j_2&		\cdots&		j_k\\
\end{matrix} \right)} \right)}.
\end{align*}
其中$\widehat{\boldsymbol{B}}\left( \begin{matrix}
i_1&		i_2&		\cdots&		i_k\\
j_1&		j_2&		\cdots&		j_k\\
\end{matrix} \right)$是$|\boldsymbol{B}|$的$k$阶子式$\boldsymbol{B}\left( \begin{matrix}
i_1&		i_2&		\cdots&		i_k\\
j_1&		j_2&		\cdots&		j_k\\
\end{matrix} \right)$的代数余子式.
\end{proposition}
\begin{note}
当\(\boldsymbol{A}\),\(\boldsymbol{B}\)之一是比较简单的矩阵(例如对角矩阵或秩较小的矩阵)时,可利用这个命题计算$|\boldsymbol{A}+\boldsymbol{B}|$.
\end{note}
\begin{solution}
设\(|\boldsymbol{A}| = |\alpha_1,\alpha_2,\cdots,\alpha_n|\),\(|\boldsymbol{B}| = |\beta_1,\beta_2,\cdots,\beta_n|\),其中\(\alpha_j,\beta_j\)(\(j = 1,2,\cdots,n\))分别是\(\boldsymbol{A}\)和\(\boldsymbol{B}\)的列向量.注意到
\begin{align*}
|\boldsymbol{A} + \boldsymbol{B}| = |\alpha_1 + \beta_1,\alpha_2 + \beta_2,\cdots,\alpha_n + \beta_n|.
\end{align*}
对\(|\boldsymbol{A} + \boldsymbol{B}|\),按列用行列式的性质展开,使每个行列式的每一列或者只含有\(\alpha_j\),或者只含有\(\beta_j\)(即利用大拆分法按列向量将行列式完全拆分开),
则\(|\boldsymbol{A} + \boldsymbol{B}|\)可以表示为\(2^n\)个这样的行列式之和.即(并且单独把\(k = 0,n\)的项分离出来,即将\(|\boldsymbol{A}|\)、\(|\boldsymbol{B}|\)分离出来)
\begin{align*}
&|\boldsymbol{A} + \boldsymbol{B}| = |\alpha_1 + \beta_1,\alpha_2 + \beta_2,\cdots,\alpha_n + \beta_n| 
\\
&=|\boldsymbol{A}|+|\boldsymbol{B}|+\sum_{1\leqslant k\leqslant n-1}{\sum_{1\leq j_1\leq j_2\leq \cdots\leq j_k\leq n}{\begin{array}{c}
\begin{array}{c@{}c@{}c@{}c@{}c@{}c@{}c@{}c@{}c@{}c@{}c@{}}
& 1 & \cdots & j_1 &\cdots &j_2 &\cdots &j_k &\cdots &n \\
\left.\right|
&\beta _1,&\cdots ,&\alpha _{j_1},&\cdots ,&\alpha_{j_2},&\cdots ,&\alpha_{j_k},&\cdots ,&\beta_n& \left|\right.
\end{array}\\
\\
\end{array}}}.
\end{align*}
再对上式右边除\(|\boldsymbol{A}|\)、\(|\boldsymbol{B}|\)外的每个行列式用\(Laplace\)定理按含有\(\boldsymbol{A}\)的列向量的那些列展开得到
\begin{align*}
&|\boldsymbol{A} + \boldsymbol{B}| =|\boldsymbol{A}|+|\boldsymbol{B}|+\sum_{1\leqslant k\leqslant n-1}{\sum_{1\leq j_1\leq j_2\leq \cdots\leq j_k\leq n}{\begin{array}{c}
\begin{array}{c@{}c@{}c@{}c@{}c@{}c@{}c@{}c@{}c@{}c@{}c@{}}
& 1 & \cdots & j_1 &\cdots &j_2 &\cdots &j_k &\cdots &n \\
\left.\right|
&\beta _1,&\cdots ,&\alpha _{j_1},&\cdots ,&\alpha_{j_2},&\cdots ,&\alpha_{j_k},&\cdots ,&\beta_n& \left|\right.
\end{array}\\
\\
\end{array}}}
\\
&= |\boldsymbol{A}| + |\boldsymbol{B}| + \sum_{1\leqslant k\leqslant n - 1}\sum_{1\leqslant j_1,j_2,\cdots,j_k\leqslant n}\sum_{1\leqslant i_1,i_2,\cdots,i_k\leqslant n}\boldsymbol{A}\left(\begin{matrix}
i_1 & i_2 & \cdots & i_k\\
j_1 & j_2 & \cdots & j_k
\end{matrix}\right)\widehat{\boldsymbol{B}}\left(\begin{matrix}
i_1 & i_2 & \cdots & i_k\\
j_1 & j_2 & \cdots & j_k
\end{matrix}\right)
\\
&=|\boldsymbol{A}|+|\boldsymbol{B}|+\sum_{1\le k\le n-1}{\left( \sum_{\substack{1\le i_1<i_2<\cdots <i_k\le n\\1\le j_1<j_2<\cdots <j_k\le n}}{\boldsymbol{A}\left( \begin{matrix}
i_1&		i_2&		\cdots&		i_k\\
j_1&		j_2&		\cdots&		j_k\\
\end{matrix} \right) \widehat{\boldsymbol{B}}\left( \begin{matrix}
i_1&		i_2&		\cdots&		i_k\\
j_1&		j_2&		\cdots&		j_k\\
\end{matrix} \right)} \right)}.
\end{align*}
\end{solution}

\begin{example}\label{example:特征行列式写成多项式形式的系数}
设
\[
f(x)=\left| \begin{matrix}
x-a_{11}&		-a_{12}&		\cdots&		-a_{1n}\\
-a_{21}&		x-a_{22}&		\cdots&		-a_{2n}\\
\vdots&		\vdots&		\ddots&		\vdots\\
-a_{n1}&		-a_{n2}&		\cdots&		x-a_{nn}\\
\end{matrix} \right|,
\]
其中\(x\)是未定元,\(a_{ij}\)是常数.证明:\(f(x)\)是一个最高次项系数为\(1\)的\(n\)次多项式,且其\(n - 1\)次项的系数等于\(-(a_{11}+a_{22}+\cdots + a_{nn})\).
\end{example}
\begin{note}
注意$f(x)$的每行每列除主对角元素外,其他元素均不相同.因此$f(x)$并不是\hyperref["爪"型行列式的推广]{推广的"爪"型行列式}.
\end{note}
\begin{solution}
由行列式的组合定义可知,\(f(x)\)的最高次项出现在组合定义展开式中的单项\((x - a_{11})(x - a_{22})\cdots(x - a_{nn})\)中,且展开式中的其他单项作为\(x\)的多项式其次数小于等于\(n - 2\).因此\(f(x)\)是一个最高次项系数为\(1\)的\(n\)次多项式,且其\(n - 1\)次项的系数等于\(-(a_{11}+a_{22}+\cdots + a_{nn})\).
\end{solution}
\begin{remark}
将这个例题进行推广再结合\hyperref[proposition:直接计算两个矩阵和的行列式]{直接计算两个矩阵和的行列式的结论}可以得到下述推论.
\end{remark}

\begin{corollary}
设\(A=(a_{ij})\)为\(n\)阶方阵,\(x\)为未定元,
\[
f(x)=\vert xI_n - A\vert = 
\begin{vmatrix}
x - a_{11} & -a_{12} & \cdots & -a_{1n} \\
-a_{21} & x - a_{22} & \cdots & -a_{2n} \\
\vdots & \vdots & \ddots & \vdots \\
-a_{n1} & -a_{n2} & \cdots & x - a_{nn}
\end{vmatrix}
\]

证明:\(f(x)=x^n + a_1x^{n - 1}+ \cdots + a_{n - 1}x + a_n\),其中
\[
a_k=(-1)^k \sum_{1\leq i_1 < i_2<\cdots <i_k\leq n} A
\begin{pmatrix}
i_1 & i_2 & \cdots & i_k \\
i_1 & i_2 & \cdots & i_k
\end{pmatrix}, 1\leq k\leq n.
\]
\end{corollary}
\begin{note}
需要注意上述推论中$a_1=-(a_{11}+a_{22}+\cdots+a_{nn}),a_n=\left( -1 \right) ^n\left| \boldsymbol{A} \right|.$
\end{note}
\begin{proof}
注意到 \(xI_{n}\) 非零的 \(n - k\) 阶子式只有 \(n - k\) 阶主子式,并且其值为 \(x^{n - k}\),其余$n-k$阶子式均为零.
记$\widehat{x\boldsymbol{I}_n}\left( \begin{matrix}
i_1&		i_2&		\cdots&		i_k\\
j_1&		j_2&		\cdots&		j_k\\
\end{matrix} \right)$是$x\boldsymbol{I}_n\left( \begin{matrix}
i_1&		i_2&		\cdots&		i_k\\
j_1&		j_2&		\cdots&		j_k\\
\end{matrix} \right)$的代数余子式,则$\widehat{x\boldsymbol{I}_n}\left( \begin{matrix}
i_1&		i_2&		\cdots&		i_k\\
j_1&		j_2&		\cdots&		j_k\\
\end{matrix} \right)$是\(xI_{n}\)非零的 \(n - k\) 阶子式.于是我们有
\begin{align*}
\widehat{x\boldsymbol{I}_n}\left( \begin{matrix}
i_1&		i_2&		\cdots&		i_k\\
j_1&		j_2&		\cdots&		j_k\\
\end{matrix} \right) =x^{n-k}.
\end{align*}
再利用\hyperref[proposition:直接计算两个矩阵和的行列式]{直接计算两个矩阵和的行列式的结论}就可以得到
\begin{align*}
&f(x)=|x\boldsymbol{I}_n-\boldsymbol{A}|=\left| \begin{matrix}
x-a_{11}&		-a_{12}&		\cdots&		-a_{1n}\\
-a_{21}&		x-a_{22}&		\cdots&		-a_{2n}\\
\vdots&		\vdots&		&		\vdots\\
-a_{n1}&		-a_{n2}&		\cdots&		x-a_{nn}\\
\end{matrix} \right|=\left| \left( \begin{matrix}
-a_{11}&		-a_{12}&		\cdots&		-a_{1n}\\
-a_{21}&		-a_{22}&		\cdots&		-a_{2n}\\
\vdots&		\vdots&		&		\vdots\\
-a_{n1}&		-a_{n2}&		\cdots&		-a_{nn}\\
\end{matrix} \right) +\left( \begin{matrix}
x&		0&		\cdots&		0\\
0&		x&		\cdots&		0\\
\vdots&		\vdots&		&		\vdots\\
0&		0&		\cdots&		x\\
\end{matrix} \right) \right|
\\
&=\left| \begin{matrix}
-a_{11}&		-a_{12}&		\cdots&		-a_{1n}\\
-a_{21}&		-a_{22}&		\cdots&		-a_{2n}\\
\vdots&		\vdots&		&		\vdots\\
-a_{n1}&		-a_{n2}&		\cdots&		-a_{nn}\\
\end{matrix} \right|+\left| \begin{matrix}
x&		0&		\cdots&		0\\
0&		x&		\cdots&		0\\
\vdots&		\vdots&		&		\vdots\\
0&		0&		\cdots&		x\\
\end{matrix} \right|+\sum_{1\le k\le n-1}{\sum_{\substack{1\le i_1,i_2,\cdots ,i_k\le n\\
1\le j_1,j_2,\cdots ,j_k\le n\\}
}{\left( -\boldsymbol{A} \right) \left( \begin{matrix}
i_1&		i_2&		\cdots&		i_k\\
j_1&		j_2&		\cdots&		j_k\\
\end{matrix} \right) \widehat{x\boldsymbol{I}_n}\left( \begin{matrix}
i_1&		i_2&		\cdots&		i_k\\
j_1&		j_2&		\cdots&		j_k\\
\end{matrix} \right)}}
\\
&=\left( -1 \right) ^n\left| \boldsymbol{A} \right|+x^n+\sum_{1\le k\le n-1}{\sum_{1\le i_1,i_2,\cdots ,i_k\le n}{\left( -1 \right) ^k\boldsymbol{A}\left( \begin{matrix}
i_1&		i_2&		\cdots&		i_k\\
i_1&		i_2&		\cdots&		i_k\\
\end{matrix} \right) \widehat{x\boldsymbol{I}_n}\left( \begin{matrix}
i_1&		i_2&		\cdots&		i_k\\
i_1&		i_2&		\cdots&		i_k\\
\end{matrix} \right)}}
\\
&=x^n+\sum_{1\le k\le n-1}{\left( -1 \right) ^k\sum_{1\le i_1,i_2,\cdots ,i_k\le n}{\boldsymbol{A}\left( \begin{matrix}
i_1&		i_2&		\cdots&		i_k\\
i_1&		i_2&		\cdots&		i_k\\
\end{matrix} \right) \cdot x^{n-k}}}+\left( -1 \right) ^n\left| \boldsymbol{A} \right|
\\
&=x^n+\sum_{1\le k\le n-1}{x^{n-k}\left( -1 \right) ^k\sum_{1\le i_1,i_2,\cdots ,i_k\le n}{\boldsymbol{A}\left( \begin{matrix}
i_1&		i_2&		\cdots&		i_k\\
i_1&		i_2&		\cdots&		i_k\\
\end{matrix} \right)}}+\left( -1 \right) ^n\left| \boldsymbol{A} \right|.
\end{align*}
因此
\(f(x)=x^n + a_1x^{n - 1}+ \cdots + a_{n - 1}x + a_n\),其中
\[
a_k=(-1)^k \sum_{1\leq i_1 < i_2<\cdots <i_k\leq n} A
\begin{pmatrix}
i_1 & i_2 & \cdots & i_k \\
i_1 & i_2 & \cdots & i_k
\end{pmatrix}, 1\leq k\leq n.
\]
\end{proof}

\begin{proposition}
设$f_k\left( x \right) =x^k+a_{k1}x^{k-1}+a_{k2}x^{k-2}+\cdots +a_{kk}
$,求下列行列式的值:
\begin{align*}
\left| \begin{matrix}
1&		f_1(x_1)&		f_2(x_1)&		\cdots&		f_{n-1}(x_1)\\
1&		f_1(x_2)&		f_2(x_2)&		\cdots&		f_{n-1}(x_2)\\
\vdots&		\vdots&		\vdots&		&		\vdots\\
1&		f_1(x_n)&		f_2(x_n)&		\cdots&		f_{n-1}(x_n)\\
\end{matrix} \right|.
\end{align*}
\end{proposition}
\begin{note}
知道这类行列式化简的操作即可.以后这种行列式化简操作不再作额外说明.
\end{note}
\begin{solution}
利用行列式的性质可得
\begin{align*}
&\left| \begin{matrix}
1&		f_1(x_1)&		f_2(x_1)&		\cdots&		f_{n-1}(x_1)\\
1&		f_1(x_2)&		f_2(x_2)&		\cdots&		f_{n-1}(x_2)\\
\vdots&		\vdots&		\vdots&		&		\vdots\\
1&		f_1(x_n)&		f_2(x_n)&		\cdots&		f_{n-1}(x_n)\\
\end{matrix} \right|
=\left| \begin{matrix}
1&		x_1+a_{11}&		x_{1}^{2}+a_{21}x_1+a_{22}&		\cdots&		x_{1}^{n-1}+a_{n-1,1}x_{1}^{n-2}+\cdots +a_{n-1,n-2}x_1+a_{n-1,n-1}\\
1&		x_2+a_{11}&		x_{2}^{2}+a_{21}x_2+a_{22}&		\cdots&		x_{2}^{n-1}+a_{n-1,1}x_{2}^{n-2}+\cdots +a_{n-1,n-2}x_2+a_{n-1,n-1}\\
\vdots&		\vdots&		\vdots&		&		\vdots\\
1&		x_n+a_{11}&		x_{n}^{2}+a_{21}x_n+a_{22}&		\cdots&		x_{n}^{n-1}+a_{n-1,1}x_{n}^{n-2}+\cdots +a_{n-1,n-2}x_n+a_{n-1,n-1}\\
\end{matrix} \right|
\\
&\xlongequal[\begin{array}{c}
\cdots\\
-a_{i,i-\left( n-3 \right)}j_{n-2}+j_{i+1},i=n-2,n-1\\
-a_{n-1,1}j_{n-1}+j_n\\
\end{array}]{\begin{array}{c}
-a_{ii}j_1+j_{i+1},i=1,2,\cdots n-1\\
-a_{i,i-1}j_2+j_{i+1},i=2,3,\cdots ,n-1\\
\end{array}}\left| \begin{matrix}
1&		x_1&		x_{1}^{2}&		\cdots&		x_{1}^{n-1}\\
1&		x_2&		x_{2}^{2}&		\cdots&		x_{2}^{n-1}\\
\vdots&		\vdots&		\vdots&		&		\vdots\\
1&		x_n&		x_{n}^{2}&		\cdots&		x_{n}^{n-1}\\
\end{matrix} \right|=\prod_{1\le i<j\le n}{\left( x_j-x_i \right) .}
\end{align*}
\end{solution}

\begin{proposition}[\hypertarget{多项式根的有限性}{多项式根的有限性}]\label{proposition:多项式根的有限性}
设多项式
\[
f(x)=a_nx^n + a_{n - 1}x^{n - 1}+\cdots+a_1x + a_0
\]
若\(f(x)\)有\(n + 1\)个不同的根\(b_1,b_2,\cdots,b_{n+1}\),即\(f(b_1)=f(b_2)=\cdots=f(b_{n+1})=0\),
求证:\(f(x)\)是零多项式,即\(a_n=a_{n - 1}=\cdots=a_1=a_0 = 0\).
\end{proposition}
\begin{proof}
由\(f(b_1)=f(b_2)=\cdots=f(b_{n+1})=0\),可知$x_0=a_0,x_1=a_1,\cdots ,x_{n-1}=a_{n-1},x_n=a_n$是下列线性方程组的解:
\begin{align*}
\left\{ \begin{aligned}
&x_0+b_1x_1+\cdots +b_{1}^{n-1}x_{n-1}+b_{1}^{n}x_n=0,\\
&x_0+b_2x_1+\cdots +b_{2}^{n-1}x_{n-1}+b_{2}^{n}x_n=0,\\
&\qquad \qquad \qquad \cdots \cdots \cdots \cdots\\
&x_0+b_{n+1}x_1+\cdots +b_{n+1}^{n-1}x_{n-1}+b_{n+1}^{n}x_n=0.\\
\end{aligned} \right. 
\end{align*}
上述线性方程组的系数行列式是一个Vandermode行列式,由于$b_1,b_2,\cdots,b_{n+1}$互不相同,所以系数行列式不等于零.由Crammer法则可知上述方程组只有零解.即有$a_n=a_{n - 1}=\cdots=a_1=a_0 = 0$.
\end{proof}

\begin{proposition}[\hypertarget{Cauchy行列式}{Cauchy行列式}]\label{Cauchy行列式}
计算$n$阶行列式:
\begin{gather}
|\boldsymbol{A}|=\left| \begin{matrix}
(a_1+b_1)^{-1}&		(a_1+b_2)^{-1}&		\cdots&		(a_1+b_n)^{-1}\\
(a_2+b_1)^{-1}&		(a_2+b_2)^{-1}&		\cdots&		(a_2+b_n)^{-1}\\
\vdots&		\vdots&		&		\vdots\\
(a_n+b_1)^{-1}&		(a_n+b_2)^{-1}&		\cdots&		(a_n+b_n)^{-1}\\
\end{matrix} \right|.
\nonumber
\end{gather}
\end{proposition}
\begin{note}
需要记忆Cauchy行列式的计算方法.

1.分式分母有公共部分可以作差,得到的分子会变得相对简便.

2.行列式内行列做加减一般都是加减同一行(或列).但是在\hyperlink{循环行列式}{循环行列式}中,我们一般采取相邻两行(或列)相加减的方法.
\end{note}
\begin{solution}
\begin{align*}
&|\boldsymbol{A}|=\left| \begin{matrix}
\frac{1}{a_1+b_1}&		\frac{1}{a_1+b_2}&		\cdots&		\frac{1}{a_1+b_n}\\
\frac{1}{a_2+b_1}&		\frac{1}{a_2+b_2}&		\cdots&		\frac{1}{a_2+b_n}\\
\vdots&		\vdots&		&		\vdots\\
\frac{1}{a_n+b_1}&		\frac{1}{a_n+b_2}&		\cdots&		\frac{1}{a_n+b_n}\\
\end{matrix} \right|
\\
&\xlongequal[i=n-1,\cdots ,1]{-j_n+j_i}\left| \begin{matrix}
\frac{b_n-b_1}{\left( a_1+b_1 \right) \left( a_1+b_n \right)}&		\frac{b_n-b_2}{\left( a_1+b_2 \right) \left( a_1+b_n \right)}&		\cdots&		\frac{b_n-b_{n-1}}{\left( a_1+b_{n-1} \right) \left( a_1+b_n \right)}&		\frac{1}{a_1+b_n}\\
\frac{b_n-b_1}{\left( a_2+b_1 \right) \left( a_2+b_n \right)}&		\frac{b_n-b_2}{\left( a_2+b_2 \right) \left( a_2+b_n \right)}&		\cdots&		\frac{b_n-b_{n-1}}{\left( a_1+b_{n-1} \right) \left( a_2+b_n \right)}&		\frac{1}{a_2+b_n}\\
\vdots&		\vdots&		&		\vdots&		\vdots\\
\frac{b_n-b_1}{\left( a_n+b_1 \right) \left( a_n+b_n \right)}&		\frac{b_n-b_2}{\left( a_n+b_2 \right) \left( a_n+b_n \right)}&		\cdots&		\frac{b_n-b_{n-1}}{\left( a_1+b_{n-1} \right) \left( a_n+b_n \right)}&		\frac{1}{a_n+b_n}\\
\end{matrix} \right|
\\
&=\frac{\prod\limits_{i=1}^{n-1}{\left( b_n-b_i \right)}}{\prod\limits_{j=1}^n{\left( a_j+b_n \right)}}\left| \begin{matrix}
\frac{1}{a_1+b_1}&		\frac{1}{a_1+b_2}&		\cdots&		\frac{1}{a_1+b_{n-1}}&		1\\
\frac{1}{a_2+b_1}&		\frac{1}{a_2+b_2}&		\cdots&		\frac{1}{a_2+b_{n-1}}&		1\\
\vdots&		\vdots&		&		\vdots&		\vdots\\
\frac{1}{a_n+b_1}&		\frac{1}{a_n+b_2}&		\cdots&		\frac{1}{a_n+b_{n-1}}&		1\\
\end{matrix} \right|
\\
&\xlongequal[i=n-1,\cdots ,1]{-r_n+r_i}\frac{\prod\limits_{i=1}^{n-1}{\left( b_n-b_i \right)}}{\prod\limits_{j=1}^n{\left( a_j+b_n \right)}}\left| \begin{matrix}
\frac{a_n-a_1}{\left( a_1+b_1 \right) \left( a_n+b_1 \right)}&		\frac{a_n-a_1}{\left( a_1+b_2 \right) \left( a_n+b_2 \right)}&		\cdots&		\frac{a_n-a_1}{\left( a_1+b_{n-1} \right) \left( a_n+b_{n-1} \right)}&		0\\
\frac{a_n-a_2}{\left( a_2+b_1 \right) \left( a_n+b_1 \right)}&		\frac{a_n-a_2}{\left( a_1+b_2 \right) \left( a_n+b_2 \right)}&		\cdots&		\frac{a_n-a_2}{\left( a_1+b_{n-1} \right) \left( a_n+b_{n-1} \right)}&		0\\
\vdots&		\vdots&		&		\vdots&		\vdots\\
\frac{a_n-a_{n-1}}{\left( a_{n-1}+b_1 \right) \left( a_n+b_1 \right)}&		\frac{a_n-a_{n-1}}{\left( a_{n-1}+b_2 \right) \left( a_n+b_2 \right)}&		\cdots&		\frac{a_n-a_{n-1}}{\left( a_{n-1}+b_{n-1} \right) \left( a_n+b_{n-1} \right)}&		0\\
\frac{1}{a_n+b_1}&		\frac{1}{a_n+b_2}&		\cdots&		\frac{1}{a_n+b_{n-1}}&		1\\
\end{matrix} \right|
\\
&=\frac{\prod\limits_{i=1}^{n-1}{\left( b_n-b_i \right)}}{\prod\limits_{j=1}^n{\left( a_j+b_n \right)}}\cdot \frac{\prod\limits_{i=1}^{n-1}{\left( a_n-a_i \right)}}{\prod\limits_{k=1}^{n-1}{\left( a_n+b_k \right)}}\left| \begin{matrix}
\frac{1}{a_1+b_1}&		\frac{1}{a_1+b_2}&		\cdots&		\frac{1}{a_1+b_{n-1}}&		0\\
\frac{1}{a_2+b_1}&		\frac{1}{a_2+b_2}&		\cdots&		\frac{1}{a_2+b_{n-1}}&		0\\
\vdots&		\vdots&		&		\vdots&		\vdots\\
\frac{1}{a_{n-1}+b_1}&		\frac{1}{a_{n-1}+b_2}&		\cdots&		\frac{1}{a_{n-1}+b_{n-1}}&		0\\
1&		1&		\cdots&		1&		1\\
\end{matrix} \right|
\\
&\xlongequal[]{\text{按最后一列展开}}\frac{\prod\limits_{i=1}^{n-1}{\left( b_n-b_i \right) \left( a_n-a_i \right)}}{\prod\limits_{j=1}^n{\left( a_j+b_n \right) \prod\limits_{k=1}^{n-1}{\left( a_n+b_k \right)}}}\left| \begin{matrix}
\frac{1}{a_1+b_1}&		\frac{1}{a_1+b_2}&		\cdots&		\frac{1}{a_1+b_{n-1}}\\
\frac{1}{a_2+b_1}&		\frac{1}{a_2+b_2}&		\cdots&		\frac{1}{a_2+b_{n-1}}\\
\vdots&		\vdots&		&		\vdots\\
\frac{1}{a_{n-1}+b_1}&		\frac{1}{a_{n-1}+b_2}&		\cdots&		\frac{1}{a_{n-1}+b_{n-1}}\\
\end{matrix} \right|
\\
&=\frac{\prod\limits_{i=1}^{n-1}{\left( b_n-b_i \right) \left( a_n-a_i \right)}}{\prod\limits_{j=1}^n{\left( a_j+b_n \right) \prod\limits_{k=1}^{n-1}{\left( a_n+b_k \right)}}}\cdot D_{n-1}.
\nonumber
\end{align*}
不断递推下去即得
\begin{align*}
&D_n=\frac{\prod\limits_{i=1}^{n-1}{\left( b_n-b_i \right) \left( a_n-a_i \right)}}{\prod\limits_{j=1}^n{\left( a_j+b_n \right) \prod\limits_{k=1}^{n-1}{\left( a_n+b_k \right)}}}\cdot D_{n-1}=\frac{\prod\limits_{i=1}^{n-1}{\left( b_n-b_i \right) \left( a_n-a_i \right)}}{\prod\limits_{j=1}^n{\left( a_j+b_n \right) \prod\limits_{k=1}^{n-1}{\left( a_n+b_k \right)}}}\cdot \frac{\prod\limits_{i=1}^{n-2}{\left( b_{n-1}-b_i \right) \left( a_{n-1}-a_i \right)}}{\prod\limits_{j=1}^{n-1}{\left( a_j+b_{n-1} \right) \prod\limits_{k=1}^{n-2}{\left( a_{n-1}+b_k \right)}}}\cdot D_{n-2}
\\
&=\cdots =\frac{\prod\limits_{i=1}^{n-1}{\left( b_n-b_i \right) \left( a_n-a_i \right)}}{\prod\limits_{j=1}^n{\left( a_j+b_n \right) \prod\limits_{k=1}^{n-1}{\left( a_n+b_k \right)}}}\cdot \frac{\prod\limits_{i=1}^{n-2}{\left( b_{n-1}-b_i \right) \left( a_{n-1}-a_i \right)}}{\prod\limits_{j=1}^{n-1}{\left( a_j+b_{n-1} \right) \prod\limits_{k=1}^{n-2}{\left( a_{n-1}+b_k \right)}}}\cdots \cdots \frac{\prod\limits_{i=1}^2{\left( b_3-b_i \right) \left( a_3-a_i \right)}}{\prod\limits_{j=1}^3{\left( a_j+b_3 \right) \prod\limits_{k=1}^2{\left( a_3+b_k \right)}}}\cdot D_2
\\
&=\frac{\prod\limits_{i=1}^{n-1}{\left( b_n-b_i \right) \left( a_n-a_i \right)}}{\prod\limits_{j=1}^n{\left( a_j+b_n \right) \prod\limits_{k=1}^{n-1}{\left( a_n+b_k \right)}}}\cdot \frac{\prod\limits_{i=1}^{n-2}{\left( b_{n-1}-b_i \right) \left( a_{n-1}-a_i \right)}}{\prod\limits_{j=1}^{n-1}{\left( a_j+b_{n-1} \right) \prod\limits_{k=1}^{n-2}{\left( a_{n-1}+b_k \right)}}}\cdots 
\frac{\prod\limits_{i=1}^2{\left( b_3-b_i \right) \left( a_3-a_i \right)}}{\prod\limits_{j=1}^3{\left( a_j+b_3 \right) \prod\limits_{k=1}^2{\left( a_3+b_k \right)}}}\cdot \frac{\left( b_2-b_1 \right) \left( a_2-a_1 \right)}{\prod\limits_{j=1}^2{\left( a_j+b_2 \right) \left( a_2+b_1 \right)}}\cdot D_1
\\
&=\frac{\prod\limits_{i=1}^{n-1}{\left( b_n-b_i \right) \left( a_n-a_i \right)}}{\prod\limits_{j=1}^n{\left( a_j+b_n \right) \prod\limits_{k=1}^{n-1}{\left( a_n+b_k \right)}}}\cdot \frac{\prod\limits_{i=1}^{n-2}{\left( b_{n-1}-b_i \right) \left( a_{n-1}-a_i \right)}}{\prod\limits_{j=1}^{n-1}{\left( a_j+b_{n-1} \right) \prod\limits_{k=1}^{n-2}{\left( a_{n-1}+b_k \right)}}}\cdots 
\frac{\prod\limits_{i=1}^2{\left( b_3-b_i \right) \left( a_3-a_i \right)}}{\prod\limits_{j=1}^3{\left( a_j+b_3 \right) \prod\limits_{k=1}^2{\left( a_3+b_k \right)}}}\cdot \frac{\left( b_2-b_1 \right) \left( a_2-a_1 \right)}{\prod\limits_{j=1}^2{\left( a_j+b_2 \right) \left( a_2+b_1 \right)}}\cdot \frac{1}{a_1+b_1}
\\
&=\frac{\prod\limits_{1\le i<j\le n}{(a_j}-a_i)(b_j-b_i)}{\prod\limits_{1\le i\le j\le n}{(a_i}+b_j)\prod\limits_{1\le j<i\le n}{(a_i}+b_j)}=\frac{\prod\limits_{1\le i<j\le n}{(a_j}-a_i)(b_j-b_i)}{\prod\limits_{i,j=1}^n{(a_i}+b_j)}.
\nonumber
\end{align*}
\end{solution}

\begin{proposition}\label{proposition:Vandermode行列式的"卷积"形式}
计算下列行列式的值:

\[
|\boldsymbol{A}|=\left| \begin{matrix}
a_{1}^{n-1}&		a_{1}^{n-2}b_1&		\cdots&		a_1b_{1}^{n-2}&		b_{1}^{n-1}\\
a_{2}^{n-1}&		a_{2}^{n-2}b_2&		\cdots&		a_2b_{2}^{n-2}&		b_{2}^{n-1}\\
\vdots&		\vdots&		&		\vdots&		\vdots\\
a_{n}^{n-1}&		a_{n}^{n-2}b_n&		\cdots&		a_nb_{n}^{n-2}&		b_{n}^{n-1}\\
\end{matrix} \right|.
\]
\end{proposition}
\begin{solution}
若所有的$a_i(i=1,2,\cdots,n)$都不为0,则有
\begin{align*}
|\boldsymbol{A}|&=\left| \begin{matrix}
a_{1}^{n-1}&		a_{1}^{n-2}b_1&		\cdots&		a_1b_{1}^{n-2}&		b_{1}^{n-1}\\
a_{2}^{n-1}&		a_{2}^{n-2}b_2&		\cdots&		a_2b_{2}^{n-2}&		b_{2}^{n-1}\\
\vdots&		\vdots&		&		\vdots&		\vdots\\
a_{n}^{n-1}&		a_{n}^{n-2}b_n&		\cdots&		a_nb_{n}^{n-2}&		b_{n}^{n-1}\\
\end{matrix} \right|=\prod_{i=1}^n{a_{i}^{n-1}}\left| \begin{matrix}
1&		\frac{b_1}{a_1}&		\cdots&		\frac{b_{1}^{n-2}}{a_{1}^{n-2}}&		\frac{b_{1}^{n-1}}{a_{1}^{n-1}}\\
1&		\frac{b_2}{a_2}&		\cdots&		\frac{b_{2}^{n-2}}{a_{2}^{n-2}}&		\frac{b_{2}^{n-1}}{a_{2}^{n-1}}\\
\vdots&		\vdots&		&		\vdots&		\vdots\\
1&		\frac{b_n}{a_n}&		\cdots&		\frac{b_{n}^{n-2}}{a_{n}^{n-2}}&		\frac{b_{n}^{n-2}}{a_{n}^{n-2}}\\
\end{matrix} \right|
\\
&=\prod_{i=1}^n{a_{i}^{n-1}}\prod_{1\le i<j\le n}{\left( \frac{b_j}{a_j}-\frac{b_i}{a_i} \right)}=\prod_{i=1}^n{a_{i}^{n-1}}\prod_{1\le i<j\le n}{\frac{a_ib_j-a_jb_i}{a_ja_i}}\hyperlink{连乘号计算小结论(1)}{=}\prod_{1\le i<j\le n}{(a_ib_j-a_jb_i)}.
\end{align*}
若只有一个$a_i$为0,则将原行列式按第$i$行展开得到具有相同类型的$n-1$阶行列式
\begin{align*}
|\boldsymbol{A}|&=\left| \begin{matrix}
a_{1}^{n-1}&		a_{1}^{n-2}b_1&		\cdots&		a_1b_{1}^{n-2}&		b_{1}^{n-1}\\
a_{2}^{n-1}&		a_{2}^{n-2}b_2&		\cdots&		a_2b_{2}^{n-2}&		b_{2}^{n-1}\\
\vdots&		\vdots&		&		\vdots&		\vdots\\
a_{i}^{n-1}&		a_{i}^{n-2}b_i&		\cdots&		a_ib_{i}^{n-2}&		b_{i}^{n-1}\\
\vdots&		\vdots&		&		\vdots&		\vdots\\
a_{n}^{n-1}&		a_{n}^{n-2}b_n&		\cdots&		a_nb_{n}^{n-2}&		b_{n}^{n-1}\\
\end{matrix} \right|=\left| \begin{matrix}
a_{1}^{n-1}&		a_{1}^{n-2}b_1&		\cdots&		a_1b_{1}^{n-2}&		b_{1}^{n-1}\\
a_{2}^{n-1}&		a_{2}^{n-2}b_2&		\cdots&		a_2b_{2}^{n-2}&		b_{2}^{n-1}\\
\vdots&		\vdots&		&		\vdots&		\vdots\\
0&		0&		\cdots&		0&		b_{i}^{n-1}\\
\vdots&		\vdots&		&		\vdots&		\vdots\\
a_{n}^{n-1}&		a_{n}^{n-2}b_n&		\cdots&		a_nb_{n}^{n-2}&		b_{n}^{n-1}\\
\end{matrix} \right|
\\
&\xlongequal{\text{按第}i\text{行展开}}\left( -1 \right) ^{n+i}b_{i}^{n-1}\left| \begin{matrix}
a_{1}^{n-1}&		a_{1}^{n-2}b_1&		\cdots&		a_1b_{1}^{n-2}\\
a_{2}^{n-1}&		a_{2}^{n-2}b_2&		\cdots&		a_2b_{2}^{n-2}\\
\vdots&		\vdots&		&		\vdots\\
a_{i-1}^{n-1}&		a_{i-1}^{n-2}b_{i-1}&		\cdots&		a_{i-1}b_{i-1}^{n-2}\\
a_{i+!}^{n-1}&		a_{i+1}^{n-2}b_{i+1}&		\cdots&		a_{i+1}b_{i+1}^{n-2}\\
\vdots&		\vdots&		&		\vdots\\
a_{n}^{n-1}&		a_{n}^{n-2}b_n&		\cdots&		a_nb_{n}^{n-2}\\
\end{matrix} \right|.
\end{align*}
此时同理可得
\begin{align*}
&|\boldsymbol{A}|=\left( -1 \right) ^{n+i}b_{i}^{n-1}\left| \begin{matrix}
a_{1}^{n-1}&		a_{1}^{n-2}b_1&		\cdots&		a_1b_{1}^{n-2}\\
a_{2}^{n-1}&		a_{2}^{n-2}b_2&		\cdots&		a_2b_{2}^{n-2}\\
\vdots&		\vdots&		&		\vdots\\
a_{i-1}^{n-1}&		a_{i-1}^{n-2}b_{i-1}&		\cdots&		a_{i-1}b_{i-1}^{n-2}\\
a_{i+1}^{n-1}&		a_{i+1}^{n-2}b_{i+1}&		\cdots&		a_{i+1}b_{i+1}^{n-2}\\
\vdots&		\vdots&		&		\vdots\\
a_{n}^{n-1}&		a_{n}^{n-2}b_n&		\cdots&		a_nb_{n}^{n-2}\\
\end{matrix} \right|=\left( -1 \right) ^{n+i}b_{i}^{n-1}\prod_{\substack{1\le k\le n\\
k\ne i\\}}{a_{k}^{n-1}}\left| \begin{matrix}
1&		\frac{b_1}{a_1}&		\cdots&		\frac{b_{1}^{n-2}}{a_{1}^{n-2}}\\
1&		\frac{b_2}{a_2}&		\cdots&		\frac{b_{2}^{n-2}}{a_{2}^{n-2}}\\
\vdots&		\vdots&		&		\vdots\\
1&		\frac{b_{i-1}}{a_{i-1}}&		\cdots&		\frac{b_{i-1}^{n-2}}{a_{i-1}^{n-2}}\\
1&		\frac{b_{i+1}}{a_{i+1}}&		\cdots&		\frac{b_{i+1}^{n-2}}{a_{i+1}^{n-2}}\\
\vdots&		\vdots&		&		\vdots\\
1&		\frac{b_n}{a_n}&		\cdots&		\frac{b_{n}^{n-2}}{a_{n}^{n-2}}\\
\end{matrix} \right|
\\
&=\left( -1 \right) ^{n+i}b_{i}^{n-1}\prod_{\substack{1\le k\le n\\
k\ne i\\}}{a_{k}^{n-1}}\prod_{\substack{1\le k<l\le n\\
k,l\ne i\\}}{\left( \frac{b_l}{a_l}-\frac{b_k}{a_k} \right)}=\left( -1 \right) ^{n+i}b_{i}^{n-1}\prod_{\substack{
1\le k\le n\\
k\ne i\\
}}{a_{k}^{n-1}}\prod_{\substack{
1\le k<l\le n\\
k,l\ne i\\
}}{\frac{a_kb_l-a_lb_k}{a_ka_l}}
\\
&\hyperlink{连乘号计算小结论(2)}{=}\left( -1 \right) ^{n+i}b_{i}^{n-1}\prod_{\substack{
1\le k\le n\\
k\ne i\\
}}{a_k}\cdot \prod_{\substack{
1\le k<l\le n\\
k,l\ne i\\
}}{\left( a_kb_l-a_lb_k \right)}=\left( -1 \right) ^{n-i}b_{i}^{n-1}\prod_{\substack{
1\le k\le n\\
k\ne i\\
}}{a_k}\cdot \prod_{\substack{
1\le k<l\le n\\
k,l\ne i\\
}}{\left( a_kb_l-a_lb_k \right)}
\\
&=\prod_{1\le k<i}{a_kb_i}\prod_{i<l\le n}{\left( -a_lb_i \right)}\cdot \prod_{\substack{
1\le k<l\le n\\
k,l\ne i\\
}}{\left( a_kb_l-a_lb_k \right)}
\\
&=\prod_{1\le k<l\le n}{\left( a_kb_l-a_lb_k \right)}.\left( a_i=0 \right).
\end{align*}
若至少有两个$a_i=a_j=0$,则第$i$行与第$j$行成比例,因此行列式的值等于0.经过计算发现,后面两种情形的答案都可以统一到第一种情形的答案.

综上所述,$|\boldsymbol{A}|=\prod_{1\le i<j\le n}{(a_ib_j-a_jb_i)}.$

\end{solution}
\begin{conclusion}\label{连乘号计算小技巧1}
\textbf{连乘号计算小结论:}

\hypertarget{连乘号计算小结论(1)}{(1)}$\prod_{1\le i<j\le n}{a_ia_j}=\prod_{i=1}^n{a_{i}^{n-1}}.$
\begin{align*}
\text{证明:}&\prod_{1\le i<j\le n}{a_ia_j}=\underset{n-1\text{组}}{\underbrace{a_2a_1\cdot a_3a_2a_3a_1\cdot a_4a_3a_4a_2a_4a_1\cdots \cdots \overset{k-1\text{对}}{\overbrace{a_ka_{k-1}a_ka_{k-2}\cdots a_ka_1}}\cdots \cdots \overset{n-1\text{对}}{\overbrace{a_na_{n-1}a_na_{n-2}\cdots a_na_1}}}}
\\
&\xlongequal{\text{从左往右按组计数}}a_{1}^{n-1}a_{2}^{1+n-2}a_{3}^{2+n-3}a_{4}^{3+n-4}\cdots a_{k}^{k-1+n-k}\cdots a_{n}^{n-1}=\prod_{i=1}^n{a_{i}^{n-1}}.
\end{align*}
\hypertarget{连乘号计算小结论(2)}{(2)}$\prod_{\substack{1\le i<j\le n\\i,j\ne k}}{a_ia_j}=\prod_{\substack{
1\le i\le n\\
i\ne k\\}}{a_{i}^{n-2}}$,其中$k\in [1,n]\cap \mathbb{N_+}$.

\begin{align*}
\text{证明:}&\prod_{\substack{
1\le i<j\le n\\
i,j\ne k\\
}}{a_ia_j}=\underset{n-2\text{组}}{\underbrace{a_2a_1\cdot a_3a_2a_3a_1\cdots \cdots \overset{k-2\text{对}}{\overbrace{a_{k-1}a_{k-2}\cdots a_{k-1}a_1}}\cdot \overset{k-1\text{对}}{\overbrace{a_{k+1}a_{k-1}\cdots a_{k+1}a_1}}\cdots \cdots \overset{n-2\text{对}}{\overbrace{a_na_{n-1}\cdots a_na_{k+1}a_na_{k-1}\cdots a_na_1}}}}
\\
&\xlongequal{\text{从左往右按组计数}}a_{1}^{n-2}a_{2}^{1+n-3}a_{3}^{2+n-4}a_{4}^{3+n-4}\cdots a_{k-1}^{k-2+n-k}a_{k+1}^{k-1+n-k-1}\cdots a_{n}^{n-2}=\prod_{\substack{
1\le i\le n\\
i\ne k\\}}{a_{i}^{n-2}}.
\end{align*}
注意:从第$k-1$组开始,后面每组都比原来少一对(后面每组均缺少原本含$a_k$的那一对).
\end{conclusion}

\begin{proposition}[行列式的刻画]\label{proposition:行列式的刻画}
设$f$为从$n$阶方阵全体构成的集合到数集上的映射,使得对任意的$n$阶方阵$\boldsymbol{A}$,任意的指标$1\leq i\leq n$,以及任意的常数$c$,满足下列条件:

(1) 设$\boldsymbol{A}$的第$i$列是方阵$\boldsymbol{B}$和$\boldsymbol{C}$的第$i$列之和,且$\boldsymbol{A}$的其余列与$\boldsymbol{B}$和$\boldsymbol{C}$的对应列完全相同,则$f(\boldsymbol{A})=f(\boldsymbol{B})+f(\boldsymbol{C})$;

(2) 将$\boldsymbol{A}$的第$i$列乘以常数$c$得到方阵$\boldsymbol{B}$,则$f(\boldsymbol{B})=cf(\boldsymbol{A})$;

(3) 对换$\boldsymbol{A}$的任意两列得到方阵$\boldsymbol{B}$,则$f(\boldsymbol{B})= - f(\boldsymbol{A})$;

(4) $f(\boldsymbol{I}_n)=1$,其中$\boldsymbol{I}_n$是$n$阶单位阵.

求证:$f(\boldsymbol{A})=\vert \boldsymbol{A}\vert$.
\end{proposition}
\begin{note}
这个命题给出了\textbf{行列式的刻画}:在方阵\(n\)个列向量上的多重线性和反对称性,以及正规性(即单位矩阵处的取值为\(1\)),唯一确定了行列式这个函数.
\end{note}
\begin{proof}
设\(\boldsymbol{A} = (\boldsymbol{\alpha}_1,\boldsymbol{\alpha}_2,\cdots,\boldsymbol{\alpha}_n)\),其中\(\boldsymbol{\alpha}_k\)为\(\boldsymbol{A}\)的第\(k\)列,\(\boldsymbol{e}_1,\boldsymbol{e}_2,\cdots,\boldsymbol{e}_n\)为标准单位列向量,则
\begin{align*}
\boldsymbol{\alpha}_j = a_{1j}\boldsymbol{e}_1 + a_{2j}\boldsymbol{e}_2 + \cdots + a_{nj}\boldsymbol{e}_n = \sum_{k = 1}^{n}a_{kj}\boldsymbol{e}_k,j = 1,2,\cdots,n.
\end{align*}
从而由条件\((1)\)和\((2)\)可得
\begin{align*}
&f\left( \boldsymbol{A} \right) =f\left( \boldsymbol{\alpha }_1,\boldsymbol{\alpha }_2,\cdots ,\boldsymbol{\alpha }_n \right) =f\left( \sum_{k_1=1}^n{a_{k_11}\boldsymbol{e}_k},\boldsymbol{\alpha }_2,\cdots ,\boldsymbol{\alpha }_n \right) 
\\
&=a_{11}f\left( \boldsymbol{e}_1,\boldsymbol{\alpha }_2,\cdots ,\boldsymbol{\alpha }_n \right) +a_{21}f\left( \boldsymbol{e}_2,\boldsymbol{\alpha }_2,\cdots ,\boldsymbol{\alpha }_n \right) +\cdots +a_{n1}f\left( \boldsymbol{e}_n,\boldsymbol{\alpha }_2,\cdots ,\boldsymbol{\alpha }_n \right) 
\\
&=\sum_{k_1=1}^n{a_{k_11}f\left( \boldsymbol{e}_{k_1},\boldsymbol{\alpha }_2,\cdots ,\boldsymbol{\alpha }_n \right)}=\sum_{k_1=1}^n{a_{k_11}f\left( \boldsymbol{e}_{k_1},\sum_{k_2=1}^n{a_{k_22}\boldsymbol{e}_{k_2}},\cdots ,\boldsymbol{\alpha }_n \right)}
\\
&=\sum_{k_1=1}^n{a_{k_11}\left[ a_{12}f\left( \boldsymbol{e}_{k_1},\boldsymbol{e}_1,\cdots ,\boldsymbol{\alpha }_n \right) +a_{22}f\left( \boldsymbol{e}_{k_1},\boldsymbol{e}_2,\cdots ,\boldsymbol{\alpha }_n \right) +\cdots +a_{n2}f\left( \boldsymbol{e}_{k_1},\boldsymbol{e}_n,\cdots ,\boldsymbol{\alpha }_n \right) \right]}
\\
&=\sum_{k_1=1}^n{a_{k_11}\sum_{k_2=1}^n{a_{k_22}}f\left( \boldsymbol{e}_{k_1},\boldsymbol{e}_{k_2},\cdots ,\boldsymbol{\alpha }_n \right)}=\cdots =\sum_{k_1=1}^n{a_{k1}\sum_{k_2=1}^n{a_{k_22}}\cdots \sum_{k_n=1}^n{a_{k_nn}f\left( \boldsymbol{e}_{k_1},\boldsymbol{e}_{k_2},\cdots ,\boldsymbol{e}_{k_n} \right)}}
\\
&=\sum_{k_1=1}^n{\sum_{k_2=1}^n{\cdots \sum_{k_n=1}^n{a_{k1}a_{k_22}\cdots a_{k_nn}f\left( \boldsymbol{e}_{k_1},\boldsymbol{e}_{k_2},\cdots ,\boldsymbol{e}_{k_n} \right)}}}=\sum_{\left( k_1,k_2,\cdots ,k_n \right)}{a_{k_11}a_{k_22}\cdots a_{k_nn}f\left( \boldsymbol{e}_{k_1},\boldsymbol{e}_{k_2},\cdots ,\boldsymbol{e}_{k_n} \right)}.
\end{align*}
若\(k_i = k_j\),则\((\boldsymbol{e}_{k_1},\boldsymbol{e}_{k_2},\cdots,\boldsymbol{e}_{k_n})\)的第\(i\)列和第\(j\)列对换后仍然是\((\boldsymbol{e}_{k_1},\boldsymbol{e}_{k_2},\cdots,\boldsymbol{e}_{k_n})\).由条件\((3)\)可知,\(f(\boldsymbol{e}_{k_1},\boldsymbol{e}_{k_2},\cdots,\boldsymbol{e}_{k_n}) = -f(\boldsymbol{e}_{k_1},\boldsymbol{e}_{k_2},\cdots,\boldsymbol{e}_{k_n})\),于是\(f(\boldsymbol{e}_{k_1},\boldsymbol{e}_{k_2},\cdots,\boldsymbol{e}_{k_n}) = 0\).
因此在\(f(\boldsymbol{A})\)的表示式中,只剩下\(k_i\)(\(i = 1,2,\cdots,n\))互不相同的项.
通过\(\tau(k_1k_2\cdots k_n)\)次相邻对换可将\((\boldsymbol{e}_{k_1},\boldsymbol{e}_{k_2},\cdots,\boldsymbol{e}_{k_n})\)变成\((\boldsymbol{e}_1,\boldsymbol{e}_2,\cdots,\boldsymbol{e}_n) = \boldsymbol{I}_n\),
故由条件\((3)\)和\((4)\)可得
\begin{align*}
f(\boldsymbol{e}_{k_1},\boldsymbol{e}_{k_2},\cdots,\boldsymbol{e}_{k_n}) = (-1)^{\tau(k_1k_2\cdots k_n)}f(\boldsymbol{I}_n) = (-1)^{\tau(k_1k_2\cdots k_n)}.
\end{align*}
于是由行列式的组合定义可知
\begin{align*}
f(\boldsymbol{A}) = \sum_{(k_1,k_2,\cdots,k_n)}a_{k_11}a_{k_22}\cdots a_{k_nn}f(\boldsymbol{e}_{k_1},\boldsymbol{e}_{k_2},\cdots,\boldsymbol{e}_{k_n}) = \sum_{(k_1,k_2,\cdots,k_n)}(-1)^{\tau(k_1k_2\cdots k_n)}a_{k_11}a_{k_22}\cdots a_{k_nn} = |\boldsymbol{A}|.
\end{align*}
\end{proof}




\end{document}
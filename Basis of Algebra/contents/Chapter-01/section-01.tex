% contents/Chapter-01/section-01.tex 第一章第一节
\documentclass[../../main.tex]{subfiles}
\graphicspath{{\subfix{../../image/}}} % 指定图片目录,后续可以直接使用图片文件名。

% 例如:
% \begin{figure}[h]
% \centering
% \includegraphics{image-01.01}
% \label{fig:image-01.01}
% \caption{图片标题}
% \end{figure}

\begin{document}
\section{行列式基本性质}

\begin{proposition}[\hypertarget{行列式计算常识}{行列式计算常识}]\label{pro:行列式计算常识}
(1)$\left| \begin{matrix}
&		&		&		a_n\\
&		&		\begin{turn}{80}$\ddots$\end{turn}&		\\
&		a_2&		&		\\
a_1&		&		&		\\
\end{matrix} \right|=\left( -1 \right) ^{\frac{n\left( n-1 \right)}{2}}a_1a_2\cdots a_n$
;\,\,$\left| \begin{matrix}
a_1&		&		&		&		\\
&		\ddots&		&		&		\\
b_1&		\cdots&		a_i&		\cdots&		b_n\\
&		&		&		\ddots&		\\
&		&		&		&		a_n\\
\end{matrix} \right|=a_1a_2\cdots a_n$.

(2)设$n$阶行列式$D=\det(a_{ij})$,把$D$上下翻转(\textbf{行倒排})、或左右翻转(\textbf{列倒排})分别得到$D_1$、$D_2$;把$D$\textbf{逆时针旋转$90^{\circ}$}、或\textbf{顺时针旋转$90^{\circ}$}分别得到$D_3$、$D_4$;把$D$\textbf{依副对角线翻转}、或\textbf{依主对角线翻转}分别得到$D_5$、$D_6$.易知
\begin{align*}
D_1=\left| \begin{matrix}
a_{n1}&		\cdots&		a_{nn}\\
\vdots&		&		\vdots\\
a_{11}&		\cdots&		a_{1n}\\
\end{matrix} \right|,D_2=\left| \begin{matrix}
a_{1n}&		\cdots&		a_{11}\\
\vdots&		&		\vdots\\
a_{nn}&		\cdots&		a_{n1}\\
\end{matrix} \right|,D_3=\left| \begin{matrix}
a_{1n}&		\cdots&		a_{nn}\\
\vdots&		&		\vdots\\
a_{11}&		\cdots&		a_{n1}\\
\end{matrix} \right|,
\\
D_4=\left| \begin{matrix}
a_{n1}&		\cdots&		a_{11}\\
\vdots&		&		\vdots\\
a_{nn}&		\cdots&		a_{1n}\\
\end{matrix} \right|,D_5=\left| \begin{matrix}
a_{nn}&		\cdots&		a_{1n}\\
\vdots&		&		\vdots\\
a_{n1}&		\cdots&		a_{11}\\
\end{matrix} \right|,D_6=\left| \begin{matrix}
a_{nn}&		\cdots&		a_{n1}\\
\vdots&		&		\vdots\\
a_{1n}&		\cdots&		a_{11}\\
\end{matrix} \right|.
\nonumber
\end{align*}
则一定有
\begin{gather*}
D_1=D_2=D_3=D_4=\left( -1 \right) ^{\frac{n\left( n-1 \right)}{2}}D,
\\
D_5=D_6=D.
\nonumber
\end{gather*}
(3)设\(A=(a_{i,j})\)为\(n\)阶复矩阵,则一定有\(\vert A\vert=\overline{\vert A\vert}\).

(4)若\(\vert A\vert\)是\(n\)阶行列式,\(\vert B\vert\)是\(m\)阶行列式,它们的值都不为零,则
\begin{align*}
\left| \left. \begin{matrix}
\boldsymbol{A}&		\boldsymbol{O}\\
\boldsymbol{O}&		\boldsymbol{B}\\
\end{matrix} \right. \right|=\left( -1 \right) ^{mn}\left. \left| \begin{matrix}
\boldsymbol{O}&		\boldsymbol{A}\\
\boldsymbol{B}&		\boldsymbol{O}\\
\end{matrix} \right| \right. .
\end{align*}
\end{proposition}
\begin{proof}
(1)运用行列式的定义即可得到结论.
\begin{align*}
(2)\,\,D_1&=\left| \begin{matrix}
a_{n1}&		\cdots&		a_{nn}\\
\vdots&		&		\vdots\\
a_{11}&		\cdots&		a_{1n}\\
\end{matrix} \right|\xlongequal[i=1,2,\cdots ,n-1]{r_i\longleftrightarrow r_{i+1}}\left( -1 \right) ^{n-1}\left| \begin{matrix}
a_{n-1,1}&		\cdots&		a_{n-1,n}\\
\vdots&		&		\vdots\\
a_{n1}&		\cdots&		a_{nn}\\
\end{matrix} \right|\xlongequal[i=1,2,\cdots ,n-2]{r_i\longleftrightarrow r_{i+1}}\left( -1 \right) ^{n-1+n-2}\left| \begin{matrix}
a_{n-2,1}&		\cdots&		a_{n-2,n}\\
\vdots&		&		\vdots\\
a_{n1}&		\cdots&		a_{nn}\\
\end{matrix} \right|
\\
&=\cdots =\left( -1 \right) ^{n-1+n-2+\cdots +1}\left| \begin{matrix}
a_{11}&		\cdots&		a_{1n}\\
\vdots&		&		\vdots\\
a_{n1}&		\cdots&		a_{nn}\\
\end{matrix} \right|=\left( -1 \right) ^{\frac{n\left( n-1 \right)}{2}}\left| \begin{matrix}
a_{11}&		\cdots&		a_{1n}\\
\vdots&		&		\vdots\\
a_{n1}&		\cdots&		a_{nn}\\
\end{matrix} \right|=\left( -1 \right) ^{\frac{n\left( n-1 \right)}{2}}D.
\end{align*}
\begin{align*}
D_2&=\left| \begin{matrix}
a_{1n}&		\cdots&		a_{11}\\
\vdots&		&		\vdots\\
a_{nn}&		\cdots&		a_{n1}\\
\end{matrix} \right|\xlongequal[i=1,2,\cdots ,n-1]{j_i\longleftrightarrow j_{i+1}}\left( -1 \right) ^{n-1}\left| \begin{matrix}
a_{1,n-1}&		\cdots&		a_{1n}\\
\vdots&		&		\vdots\\
a_{n,n-1}&		\cdots&		a_{nn}\\
\end{matrix} \right|\xlongequal[i=1,2,\cdots ,n-2]{j_i\longleftrightarrow j_{i+1}}\left( -1 \right) ^{n-1+n-2}\left| \begin{matrix}
a_{1,n-2}&		\cdots&		a_{1n}\\
\vdots&		&		\vdots\\
a_{n,n-2}&		\cdots&		a_{nn}\\
\end{matrix} \right|
\\
&=\cdots =\left( -1 \right) ^{n-1+n-2+\cdots +1}\left| \begin{matrix}
a_{11}&		\cdots&		a_{1n}\\
\vdots&		&		\vdots\\
a_{n1}&		\cdots&		a_{nn}\\
\end{matrix} \right|=\left( -1 \right) ^{\frac{n\left( n-1 \right)}{2}}\left| \begin{matrix}
a_{11}&		\cdots&		a_{1n}\\
\vdots&		&		\vdots\\
a_{n1}&		\cdots&		a_{nn}\\
\end{matrix} \right|=\left( -1 \right) ^{\frac{n\left( n-1 \right)}{2}}D.
\nonumber
\end{align*}
\begin{gather*}
D_3=\left| \begin{matrix}
a_{1n}&		\cdots&		a_{nn}\\
\vdots&		&		\vdots\\
a_{11}&		\cdots&		a_{n1}\\
\end{matrix} \right|\xlongequal{\text{行倒排}}\left( -1 \right) ^{\frac{n\left( n-1 \right)}{2}}\left| \begin{matrix}
a_{11}&		\cdots&		a_{n1}\\
\vdots&		&		\vdots\\
a_{1n}&		\cdots&		a_{nn}\\
\end{matrix} \right|=\left( -1 \right) ^{\frac{n\left( n-1 \right)}{2}}D^T=\left( -1 \right) ^{\frac{n\left( n-1 \right)}{2}}D.
\\
D_4=\left| \begin{matrix}
a_{n1}&		\cdots&		a_{11}\\
\vdots&		&		\vdots\\
a_{nn}&		\cdots&		a_{1n}\\
\end{matrix} \right|\xlongequal{\text{列倒排}}\left( -1 \right) ^{\frac{n\left( n-1 \right)}{2}}\left| \begin{matrix}
a_{11}&		\cdots&		a_{n1}\\
\vdots&		&		\vdots\\
a_{1n}&		\cdots&		a_{nn}\\
\end{matrix} \right|=\left( -1 \right) ^{\frac{n\left( n-1 \right)}{2}}D^T=\left( -1 \right) ^{\frac{n\left( n-1 \right)}{2}}D.
\\
D_5=\left| \begin{matrix}
a_{nn}&		\cdots&		a_{1n}\\
\vdots&		&		\vdots\\
a_{n1}&		\cdots&		a_{11}\\
\end{matrix} \right|\xlongequal[]{\text{逆时针旋转}90^{\circ}}\left( -1 \right) ^{\frac{n\left( n-1 \right)}{2}}\left| \begin{matrix}
a_{1n}&		\cdots&		a_{11}\\
\vdots&		&		\vdots\\
a_{nn}&		\cdots&		a_{n1}\\
\end{matrix} \right|\xlongequal[]{\text{列倒排}}\left( -1 \right) ^{\frac{n\left( n-1 \right)}{2}}\cdot \left( -1 \right) ^{\frac{n\left( n-1 \right)}{2}}\left| \begin{matrix}
a_{11}&		\cdots&		a_{1n}\\
\vdots&		&		\vdots\\
a_{n1}&		\cdots&		a_{nn}\\
\end{matrix} \right|=D.
\\
D_6=\left| \begin{matrix}
a_{nn}&		\cdots&		a_{n1}\\
\vdots&		&		\vdots\\
a_{1n}&		\cdots&		a_{11}\\
\end{matrix} \right|\xlongequal[]{\text{顺时针旋转}90^{\circ}}\left( -1 \right) ^{\frac{n\left( n-1 \right)}{2}}\left| \begin{matrix}
a_{1n}&		\cdots&		a_{nn}\\
\vdots&		&		\vdots\\
a_{11}&		\cdots&		a_{n1}\\
\end{matrix} \right|\xlongequal[]{\text{行倒排}}\left( -1 \right) ^{\frac{n\left( n-1 \right)}{2}}\cdot \left( -1 \right) ^{\frac{n\left( n-1 \right)}{2}}\left| \begin{matrix}
a_{11}&		\cdots&		a_{1n}\\
\vdots&		&		\vdots\\
a_{n1}&		\cdots&		a_{nn}\\
\end{matrix} \right|=D.
\nonumber
\end{gather*}
(3)复数的共轭保持加法和乘法:\(\overline{z_1 + z_2}=\overline{z_1}+\overline{z_2}\),\(\overline{z_1\cdot z_2}=\overline{z_1}\cdot\overline{z_2}\),故由行列式的组合定义可得
\begin{align*}
|A|&=\sum_{1\le k_1,k_2,\cdots ,k_n\le n}{\left( -1 \right) ^{\tau (k_1k_2\cdots k_n)}a_{k_{11}}a_{k_{22}}\cdots a_{k_{nn}}}
\\
&=\sum_{1\le k_1,k_2,\cdots ,k_n\le n}{\left( -1 \right) ^{\tau (k_1k_2\cdots k_n)}\overline{a_{k_{11}}}\cdot\overline{a_{k_{22}}}\cdots \overline{a_{k_{nn}}}}=|\overline{A}|.
\end{align*}

(4)将\(\vert A\vert\)的第一列依次和\(\vert B\vert\)的第\(m\)列,第\(m - 1\)列,…,第一列对换,共换了\(m\)次;再将\(\vert A\vert\)的第二列依次和\(\vert B\vert\)的第\(m\)列,第\(m - 1\)列,…,第一列对换,又换了\(m\)次;$\cdots$.依次类推,经过\(mn\)次对换可将第二个行列式变为第一个行列式.因此\(\vert D\vert=(-1)^{mn}\vert C\vert\),于是
由行列式的基本性质可得
\begin{align*}
\left| \left. \begin{matrix}
\boldsymbol{A}&		\boldsymbol{O}\\
\boldsymbol{O}&		\boldsymbol{B}\\
\end{matrix} \right. \right|=\left( -1 \right) ^{mn}\left. \left| \begin{matrix}
\boldsymbol{O}&		\boldsymbol{A}\\
\boldsymbol{B}&		\boldsymbol{O}\\
\end{matrix} \right| \right. .
\end{align*}
\end{proof}

\begin{proposition}[行列式的刻画]\label{proposition:行列式的刻画}
设$f$为从$n$阶方阵全体构成的集合到数集上的映射,使得对任意的$n$阶方阵$\boldsymbol{A}$,任意的指标$1\leq i\leq n$,以及任意的常数$c$,满足下列条件:

(1) 设$\boldsymbol{A}$的第$i$列是方阵$\boldsymbol{B}$和$\boldsymbol{C}$的第$i$列之和,且$\boldsymbol{A}$的其余列与$\boldsymbol{B}$和$\boldsymbol{C}$的对应列完全相同,则$f(\boldsymbol{A})=f(\boldsymbol{B})+f(\boldsymbol{C})$;

(2) 将$\boldsymbol{A}$的第$i$列乘以常数$c$得到方阵$\boldsymbol{B}$,则$f(\boldsymbol{B})=cf(\boldsymbol{A})$;

(3) 对换$\boldsymbol{A}$的任意两列得到方阵$\boldsymbol{B}$,则$f(\boldsymbol{B})= - f(\boldsymbol{A})$;

(4) $f(\boldsymbol{I}_n)=1$,其中$\boldsymbol{I}_n$是$n$阶单位阵.

求证:$f(\boldsymbol{A})=\vert \boldsymbol{A}\vert$.
\end{proposition}
\begin{note}
这个命题给出了\textbf{行列式的刻画}:在方阵\(n\)个列向量上的多重线性和反对称性,以及正规性(即单位矩阵处的取值为\(1\)),唯一确定了行列式这个函数.
\end{note}
\begin{proof}
设\(\boldsymbol{A} = (\boldsymbol{\alpha}_1,\boldsymbol{\alpha}_2,\cdots,\boldsymbol{\alpha}_n)\),其中\(\boldsymbol{\alpha}_k\)为\(\boldsymbol{A}\)的第\(k\)列,\(\boldsymbol{e}_1,\boldsymbol{e}_2,\cdots,\boldsymbol{e}_n\)为标准单位列向量,则
\begin{align*}
\boldsymbol{\alpha}_j = a_{1j}\boldsymbol{e}_1 + a_{2j}\boldsymbol{e}_2 + \cdots + a_{nj}\boldsymbol{e}_n = \sum_{k = 1}^{n}a_{kj}\boldsymbol{e}_k,j = 1,2,\cdots,n.
\end{align*}
从而由条件\((1)\)和\((2)\)可得
\begin{align*}
&f\left( \boldsymbol{A} \right) =f\left( \boldsymbol{\alpha }_1,\boldsymbol{\alpha }_2,\cdots ,\boldsymbol{\alpha }_n \right) =f\left( \sum_{k_1=1}^n{a_{k_11}\boldsymbol{e}_k},\boldsymbol{\alpha }_2,\cdots ,\boldsymbol{\alpha }_n \right) 
\\
&=a_{11}f\left( \boldsymbol{e}_1,\boldsymbol{\alpha }_2,\cdots ,\boldsymbol{\alpha }_n \right) +a_{21}f\left( \boldsymbol{e}_2,\boldsymbol{\alpha }_2,\cdots ,\boldsymbol{\alpha }_n \right) +\cdots +a_{n1}f\left( \boldsymbol{e}_n,\boldsymbol{\alpha }_2,\cdots ,\boldsymbol{\alpha }_n \right) 
\\
&=\sum_{k_1=1}^n{a_{k_11}f\left( \boldsymbol{e}_{k_1},\boldsymbol{\alpha }_2,\cdots ,\boldsymbol{\alpha }_n \right)}=\sum_{k_1=1}^n{a_{k_11}f\left( \boldsymbol{e}_{k_1},\sum_{k_2=1}^n{a_{k_22}\boldsymbol{e}_{k_2}},\cdots ,\boldsymbol{\alpha }_n \right)}
\\
&=\sum_{k_1=1}^n{a_{k_11}\left[ a_{12}f\left( \boldsymbol{e}_{k_1},\boldsymbol{e}_1,\cdots ,\boldsymbol{\alpha }_n \right) +a_{22}f\left( \boldsymbol{e}_{k_1},\boldsymbol{e}_2,\cdots ,\boldsymbol{\alpha }_n \right) +\cdots +a_{n2}f\left( \boldsymbol{e}_{k_1},\boldsymbol{e}_n,\cdots ,\boldsymbol{\alpha }_n \right) \right]}
\\
&=\sum_{k_1=1}^n{a_{k_11}\sum_{k_2=1}^n{a_{k_22}}f\left( \boldsymbol{e}_{k_1},\boldsymbol{e}_{k_2},\cdots ,\boldsymbol{\alpha }_n \right)}=\cdots =\sum_{k_1=1}^n{a_{k1}\sum_{k_2=1}^n{a_{k_22}}\cdots \sum_{k_n=1}^n{a_{k_nn}f\left( \boldsymbol{e}_{k_1},\boldsymbol{e}_{k_2},\cdots ,\boldsymbol{e}_{k_n} \right)}}
\\
&=\sum_{k_1=1}^n{\sum_{k_2=1}^n{\cdots \sum_{k_n=1}^n{a_{k1}a_{k_22}\cdots a_{k_nn}f\left( \boldsymbol{e}_{k_1},\boldsymbol{e}_{k_2},\cdots ,\boldsymbol{e}_{k_n} \right)}}}=\sum_{\left( k_1,k_2,\cdots ,k_n \right)}{a_{k_11}a_{k_22}\cdots a_{k_nn}f\left( \boldsymbol{e}_{k_1},\boldsymbol{e}_{k_2},\cdots ,\boldsymbol{e}_{k_n} \right)}.
\end{align*}
若\(k_i = k_j\),则\((\boldsymbol{e}_{k_1},\boldsymbol{e}_{k_2},\cdots,\boldsymbol{e}_{k_n})\)的第\(i\)列和第\(j\)列对换后仍然是\((\boldsymbol{e}_{k_1},\boldsymbol{e}_{k_2},\cdots,\boldsymbol{e}_{k_n})\).由条件\((3)\)可知,\(f(\boldsymbol{e}_{k_1},\boldsymbol{e}_{k_2},\cdots,\boldsymbol{e}_{k_n}) = -f(\boldsymbol{e}_{k_1},\boldsymbol{e}_{k_2},\cdots,\boldsymbol{e}_{k_n})\),于是\(f(\boldsymbol{e}_{k_1},\boldsymbol{e}_{k_2},\cdots,\boldsymbol{e}_{k_n}) = 0\).
因此在\(f(\boldsymbol{A})\)的表示式中,只剩下\(k_i\)(\(i = 1,2,\cdots,n\))互不相同的项.
通过\(\tau(k_1k_2\cdots k_n)\)次相邻对换可将\((\boldsymbol{e}_{k_1},\boldsymbol{e}_{k_2},\cdots,\boldsymbol{e}_{k_n})\)变成\((\boldsymbol{e}_1,\boldsymbol{e}_2,\cdots,\boldsymbol{e}_n) = \boldsymbol{I}_n\),
故由条件\((3)\)和\((4)\)可得
\begin{align*}
f(\boldsymbol{e}_{k_1},\boldsymbol{e}_{k_2},\cdots,\boldsymbol{e}_{k_n}) = (-1)^{\tau(k_1k_2\cdots k_n)}f(\boldsymbol{I}_n) = (-1)^{\tau(k_1k_2\cdots k_n)}.
\end{align*}
于是由行列式的组合定义可知
\begin{align*}
f(\boldsymbol{A}) = \sum_{(k_1,k_2,\cdots,k_n)}a_{k_11}a_{k_22}\cdots a_{k_nn}f(\boldsymbol{e}_{k_1},\boldsymbol{e}_{k_2},\cdots,\boldsymbol{e}_{k_n}) = \sum_{(k_1,k_2,\cdots,k_n)}(-1)^{\tau(k_1k_2\cdots k_n)}a_{k_11}a_{k_22}\cdots a_{k_nn} = |\boldsymbol{A}|.
\end{align*}
\end{proof}




\end{document}
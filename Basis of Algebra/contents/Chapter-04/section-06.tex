\documentclass[../../main.tex]{subfiles}
\graphicspath{{\subfix{../../image/}}} % 指定图片目录,后续可以直接使用图片文件名。

% 例如:
% \begin{figure}[h]
% \centering
% \includegraphics{image-01.01}
% \label{fig:image-01.01}
% \caption{图片标题}
% \end{figure}

\begin{document}

\section{不变子空间}

\begin{example}
设线性空间\(V\)上的线性变换\(\varphi\)在基\(\{\boldsymbol{e}_1,\boldsymbol{e}_2,\boldsymbol{e}_3,\boldsymbol{e}_4\}\)下的表示矩阵为
\[
\boldsymbol{A}=\begin{pmatrix}
1&0&2&-1\\
0&1&4&-2\\
2&-1&0&1\\
2&-1&-1&2
\end{pmatrix},
\]
求证:\(U = L(\boldsymbol{e}_1 + 2\boldsymbol{e}_2,\boldsymbol{e}_3 + \boldsymbol{e}_4,\boldsymbol{e}_1 + \boldsymbol{e}_2)\)和\(W = L(\boldsymbol{e}_2 + \boldsymbol{e}_3 + 2\boldsymbol{e}_4)\)都是\(\varphi\)的不变子空间.
\end{example}
\begin{proof}
要证明由若干个向量生成的子空间是某个线性变换的不变子空间,通常只需证明这些向量在线性变换的作用下仍在这个子空间中即可.因此只需证明这些子空间的一组基在线性变换的作用下仍在这个子空间中即可.注意到\(\varphi(\boldsymbol{e}_1 + 2\boldsymbol{e}_2)\)的坐标向量为
\[
\begin{pmatrix}
1&0&2&-1\\
0&1&4&-2\\
2&-1&0&1\\
2&-1&-1&2
\end{pmatrix}
\begin{pmatrix}
1\\
2\\
0\\
0
\end{pmatrix}=
\begin{pmatrix}
1\\
2\\
0\\
0
\end{pmatrix},
\]
即\(\varphi(\boldsymbol{e}_1 + 2\boldsymbol{e}_2)=\boldsymbol{e}_1 + 2\boldsymbol{e}_2\in U\). 同理可计算出
\begin{align*}
\varphi(\boldsymbol{e}_3 + \boldsymbol{e}_4)&=(\boldsymbol{e}_1 + 2\boldsymbol{e}_2)+(\boldsymbol{e}_3 + \boldsymbol{e}_4)\in U,\\
\varphi(\boldsymbol{e}_1 + \boldsymbol{e}_2)&=(\boldsymbol{e}_1 + \boldsymbol{e}_2)+(\boldsymbol{e}_3 + \boldsymbol{e}_4)\in U,\\
\varphi(\boldsymbol{e}_2 + \boldsymbol{e}_3 + 2\boldsymbol{e}_4)&=\boldsymbol{e}_2 + \boldsymbol{e}_3 + 2\boldsymbol{e}_4\in W,
\end{align*}
因此结论成立. 
\end{proof}

\begin{proposition}\label{proposition:不变子空间的子空间、交与和仍是不变子空间}
设\(V_1,V_2\)是\(V\)上线性变换\(\varphi\)的不变子空间,任取$V_0\subset V$,求证:,$V_0$,\(V_1\cap V_2\),\(V_1 + V_2\)也是\(\varphi\)的不变子空间.
\end{proposition}
\begin{proof}
$V_0$是\(\varphi -\)不变子空间是显然的.

任取\(\boldsymbol{v}\in V_1\cap V_2\),则由\(\boldsymbol{v}\in V_i\)可得\(\varphi(\boldsymbol{v})\in V_i(i = 1,2)\),于是\(\varphi(\boldsymbol{v})\in V_1\cap V_2\),从而\(V_1\cap V_2\)是\(\varphi -\)不变子空间.

任取\(\boldsymbol{v}\in V_1 + V_2\),则\(\boldsymbol{v}=\boldsymbol{v}_1+\boldsymbol{v}_2\),其中\(\boldsymbol{v}_i\in V_i\),故\(\varphi(\boldsymbol{v}_i)\in V_i(i = 1,2)\),于是\(\varphi(\boldsymbol{v})=\varphi(\boldsymbol{v}_1)+\varphi(\boldsymbol{v}_2)\in V_1 + V_2\),从而\(V_1 + V_2\)是\(\varphi -\)不变子空间.
\end{proof}

\begin{proposition}\label{proposition:纯量变换关于不变子空间的等价条件}
设\(\varphi\)是\(n(n\geq 2)\)维线性空间\(V\)上的线性变换,证明以下\(n+1\)个结论等价:

(1) \(V\)的任一\(1\)维子空间都是\(\varphi -\)不变子空间;

\(\cdots\cdots\)

(r) \(V\)的任一\(r\)维子空间都是\(\varphi -\)不变子空间;

\(\cdots\cdots\)

(n - 1) \(V\)的任一\(n - 1\)维子空间都是\(\varphi -\)不变子空间;

(n)\(V\)本身就是\(\varphi -\)不变子空间;

(n+1) \(\varphi\)是纯量变换.
\end{proposition}
\begin{proof}
(n+1) \(\Rightarrow\) (n)是显然的.
注意到当\(1\leq i\leq n - 2\)时,任一\(i\)维子空间\(V_0\)都可表示为两个\(i + 1\)维子空间\(V_1,V_2\)的交;而$V$的任意$n-1$维子空间都是$V$的子空间.于是由\hyperref[proposition:不变子空间的子空间、交与和仍是不变子空间]{命题\ref{proposition:不变子空间的子空间、交与和仍是不变子空间}}可知:(n) \(\Rightarrow\) (n - 1) \(\Rightarrow\) (n - 2) \(\Rightarrow\cdots\Rightarrow\) (1) 显然成立,剩下只要证明 (1) \(\Rightarrow\) (n+1) 即可. 取\(V\)的一组基\(\{\boldsymbol{e}_1,\boldsymbol{e}_2,\cdots,\boldsymbol{e}_n\}\),由 (1) 可知$L\left( \boldsymbol{e}_1 \right) ,L\left( \boldsymbol{e}_2 \right) ,\cdots ,L\left( \boldsymbol{e}_n \right)$都是\(\varphi -\)不变子空间,设\(\varphi(\boldsymbol{e}_i)=\lambda_i\boldsymbol{e}_i(1\leq i\leq n)\). 只要证明\(\lambda_1=\lambda_2=\cdots=\lambda_n\)即可得到\(\varphi\)为纯量变换. 用反证法,不妨设\(\lambda_1\neq\lambda_2\),则由\(L(\boldsymbol{e}_1+\boldsymbol{e}_2)\)也是\(\varphi -\)不变子空间可设\(\varphi(\boldsymbol{e}_1+\boldsymbol{e}_2)=\lambda_0(\boldsymbol{e}_1+\boldsymbol{e}_2)\),于是\((\lambda_1 - \lambda_0)\boldsymbol{e}_1+(\lambda_2 - \lambda_0)\boldsymbol{e}_2=\boldsymbol{0}\),从而由$\boldsymbol{e}_1,\boldsymbol{e}_2$线性无关可知\(\lambda_1=\lambda_2=\lambda_0\),矛盾.
\end{proof}

\begin{proposition}\label{proposition:乘法可交换的线性变换值域和核互为不变子空间}
设\(\varphi,\psi\)是线性空间\(V\)上的线性变换且\(\varphi\psi = \psi\varphi\),求证:\(\text{Im}\varphi\)及\(\text{Ker}\varphi\)都是\(\psi\)的不变子空间.同理,\(\text{Im}\psi\)及\(\text{Ker}\psi\)也都是\(\varphi\)的不变子空间.
\end{proposition}
\begin{note}
显然\(\text{Im}\varphi\)及\(\text{Ker}\varphi\)都是\(\varphi\)自身的不变子空间,\(\text{Im}\psi\)及\(\text{Ker}\psi\)也都是\(\psi\)自身的不变子空间.
\end{note}
\begin{proof}
任取\(\boldsymbol{v}\in\text{Im}\varphi\),即\(\boldsymbol{v}=\varphi(\boldsymbol{u})\),则\(\psi(\boldsymbol{v})=\psi\varphi(\boldsymbol{u})=\varphi\psi(\boldsymbol{u})\in\text{Im}\varphi\),即\(\text{Im}\varphi\)是\(\psi\)的不变子空间.

任取\(\boldsymbol{v}\in\text{Ker}\varphi\),即\(\varphi(\boldsymbol{v}) = 0\),则\(\varphi\psi(\boldsymbol{v})=\psi\varphi(\boldsymbol{v}) = 0\). 因此,\(\psi(\boldsymbol{v})\in\text{Ker}\varphi\),即\(\text{Ker}\varphi\)是\(\psi\)的不变子空间.
\end{proof}

\begin{proposition}\label{proposition:线性变换在其不变子空间下的限制}
设\(\varphi\)是\(n\)维线性空间\(V\)上的线性变换,\(W\)为\(\varphi -\)不变子空间,\(\varphi\)在\(W\)上的限制为\(\varphi|_W\),
则\(\varphi|_W\)的像集与原像集相同且均为\(W\),对\(\forall k\in \mathbb{N}\),\((\varphi|_W)^k\)有意义并且\(\varphi^k|_W = (\varphi|_W)^k\).    
\end{proposition}
\begin{proof}
因为\(W\)为\(\varphi -\)不变子空间,所以\(\varphi|_W\)的像集与原像集相同且均为\(W\)是显然的.
并且对\(\forall \alpha \in W\),有\(\varphi|_W(\alpha) \in W\).因此对\(\forall k \in \mathbb{N}\),有\((\varphi|_W)^k(\alpha) \in W\).故\((\varphi|_W)^k\)有意义.下证\(\varphi^k|_W = (\varphi|_W)^k\).

对\(\forall k\in \mathbb{N}\),显然\(\varphi^k|_W\)和\((\varphi|_W)^k\)的定义域都是\(W\).从而对\(\forall a\in W\),有
\[
\varphi^k|_W(a)=\varphi^k(a),
\]
\[
(\varphi|_W)^k(a)=(\varphi|_W)^{k - 1}\varphi|_W(a)=(\varphi|_W)^{k - 1}\varphi(a)=\cdots=\varphi^k(a).
\]
因此\(\varphi^k|_W(a)=(\varphi|_W)^k(a)=\varphi^k(a)\).故\(\varphi^k|_W = (\varphi|_W)^k\).
\end{proof}

\begin{proposition}\label{proposition:自同构在其不变子空间下的限制}
设\(\varphi\)是\(n\)维线性空间\(V\)上的自同构,\(W\)为\(\varphi -\)不变子空间,\(\varphi\)在\(W\)上的限制为\(\varphi|_W\),
则\(\varphi|_W\)的像集与原像集相同且均为\(W\),\(\varphi|_W\)是\(W\)上的自同构并且\((\varphi|_W)^{-1} = \varphi^{-1}|_W\).
\end{proposition}
\begin{proof}

\end{proof}


\begin{example}
设\(\boldsymbol{A}\)为数域\(\mathbb{K}\)上的\(n\)阶幂零阵,\(\boldsymbol{B}\)为\(n\)阶方阵,满足\(\boldsymbol{A}\boldsymbol{B}=\boldsymbol{B}\boldsymbol{A}\)且\(\text{r}(\boldsymbol{A}\boldsymbol{B})=\text{r}(\boldsymbol{B})\). 求证:\(\boldsymbol{B}=\boldsymbol{O}\).
\end{example}
\begin{remark}
因为$\text{Im}\boldsymbol{B}$是$\boldsymbol{A}-$不变子空间,所以\(\boldsymbol{A}|_{\text{Im}\boldsymbol{B}}(\text{Im}\boldsymbol{B})\in \text{Im}\boldsymbol{B}\).因此$(\boldsymbol{A}|_{\text{Im}\boldsymbol{B}})^k $有意义.
对于一般的限制$W$,$(\boldsymbol{A}|_{W})^k $不一定有意义.见\hyperref[proposition:线性变换在其不变子空间下的限制]{命题\ref{proposition:线性变换在其不变子空间下的限制}}.
\end{remark}
\begin{proof}
将\(\boldsymbol{A},\boldsymbol{B}\)都看成是\(\mathbb{K}^n\)上(由矩阵\(\boldsymbol{A},\boldsymbol{B}\)乘法诱导)的线性变换,设\(\boldsymbol{A}^k = \boldsymbol{O}\),其中\(k\)为正整数. 由\(\boldsymbol{A}\boldsymbol{B}=\boldsymbol{B}\boldsymbol{A}\)以及\hyperref[proposition:乘法可交换的线性变换值域和核互为不变子空间]{命题\ref{proposition:乘法可交换的线性变换值域和核互为不变子空间}}可知\(\text{Im}\boldsymbol{B}\)是\(\boldsymbol{A}-\)不变子空间. 考虑\(\boldsymbol{A}\)在\(\text{Im}\boldsymbol{B}\)上的限制\(\boldsymbol{A}|_{\text{Im}\boldsymbol{B}}\),其像空间的维数\(\dim\boldsymbol{A}\boldsymbol{B}(\mathbb{K}^n)=\text{r}(\boldsymbol{A}\boldsymbol{B})=\text{r}(\boldsymbol{B})=\dim\text{Im}\boldsymbol{B}\),故\(\boldsymbol{A}|_{\text{Im}\boldsymbol{B}}\)是\(\text{Im}\boldsymbol{B}\)上的满线性变换. 于是由\hyperref[proposition:线性变换在其不变子空间下的限制]{命题\ref{proposition:线性变换在其不变子空间下的限制}}和\hyperref[proposition:满射的复合仍是满射]{满射的复合仍是满射}可知\((\boldsymbol{A}|_{\text{Im}\boldsymbol{B}})^k = \boldsymbol{A}^k|_{\text{Im}\boldsymbol{B}}=\boldsymbol{O}|_{\text{Im}\boldsymbol{B}}\)也是\(\text{Im}\boldsymbol{B}\)上的满线性变换,从而只能是\(\text{Im}\boldsymbol{B}=0\),即\(\boldsymbol{B}=\boldsymbol{O}\). 
\end{proof}

\begin{proposition}\label{proposition:可逆线性变换的不变子空间仍是其逆的不变子空间}
设\(\varphi\)是\(n\)维线性空间\(V\)上的自同构,若\(W\)是\(\varphi\)的不变子空间,求证:\(W\)也是\(\varphi^{-1}\)的不变子空间.
\end{proposition}
\begin{proof}
将\(\varphi\)限制在\(W\)上,得到$\varphi :W\rightarrow W$.它是\(W\)上的线性变换. 由于\(\varphi\)是单映射,故它在\(W\)上的限制也是单映射,从而由\hyperref[corollary:线性变换自同构的充要条件]{推论\ref{corollary:线性变换自同构的充要条件}}可知,\(\varphi\)在\(W\)上的限制也是满映射,即它是\(W\)上的自同构,于是结合\hyperref[proposition:自同构在其不变子空间下的限制]{命题\ref{proposition:自同构在其不变子空间下的限制}}可知\(W=\varphi |_W\left( W \right) =\varphi \left( W \right) \),对其两边同时取$\varphi ^{-1}$可得$\varphi ^{-1}\left( W \right) =W$.故结论得证.
\end{proof}
\begin{remark}
如果\(V\)是无限维线性空间,则这个命题的结论一般并不成立. 例如,\(V = \mathbb{K}[x^{-1},x]\)是由数域\(\mathbb{K}\)上的Laurent多项式\(f(x)=\sum_{i = -m}^{n}a_ix^i(m,n\in\mathbb{N})\)构成的线性空间,\(V\)上的线性变换\(\varphi,\psi\)定义为\(\varphi(f(x)) = xf(x)\),\(\psi(f(x)) = x^{-1}f(x)\). 显然,\(\varphi,\psi\)互为逆映射,从而都是自同构. 注意到\(W = \mathbb{K}[x]\)是\(V\)的\(\varphi -\)不变子空间,但\(W\)显然不是\(\varphi^{-1}-\)不变子空间.
\end{remark}

\begin{example}
设\(V\)是次数小于\(n\)的实系数多项式组成的线性空间,\(D\)是\(V\)上的求导变换. 求证:\(D\)的任一\(k(k\geq 1)\)维不变子空间必是由\(\{1,x,\cdots,x^{k - 1}\}\)生成的子空间. 特别地,向量\(1\)包含在\(D\)的任一非零不变子空间中.
\end{example}
\begin{proof}
任取\(D\)的一个\(k(k\geq 1)\)维不变子空间\(V_0\),再取出\(V_0\)中次数最高的一个多项式(不唯一)\(f(x)=a_lx^l + a_{l - 1}x^{l - 1}+\cdots + a_1x + a_0\),其中\(a_l\neq 0\). 注意到\(V_0\)是\(D -\)不变子空间,由\(D^lf(x)=a_ll!\in V_0\)可得\(1\in V_0\);由\(D^{l - 1}f(x)=a_ll!x + a_{l - 1}(l - 1)!\in V_0\)可得\(x\in V_0\);\(\cdots\);由\(Df(x)=a_llx^{l - 1}+a_{l - 1}(l - 1)x^{l - 2}+\cdots + a_1\in V_0\)可得\(x^{l - 1}\in V_0\);最后由\(f(x)\in V_0\)可得\(x^l\in V_0\). 因为\(V_0\)中所有多项式的次数都小于等于\(l\),所以\(\{1,x,\cdots,x^l\}\)构成了\(V_0\)的一组基,于是\(k = \dim V_0 = l + 1\),即\(l = k - 1\),从而结论得证.
\end{proof}

\begin{example}
设\(\varphi\)是\(n\)维线性空间\(V\)上的线性变换,\(\varphi\)在\(V\)的一组基下的表示矩阵为对角阵且主对角线上的元素互不相同,求\(\varphi\)的所有不变子空间.
\end{example}
\begin{proof}
{\color{blue}证法一:}
设\(\varphi\)在基\(\boldsymbol{e}_1,\boldsymbol{e}_2,\cdots,\boldsymbol{e}_n\)下的表示矩阵为\(\text{diag}\{d_1,d_2,\cdots,d_n\}\),其中\(d_1,d_2,\cdots,d_n\)互不相同,则\(\varphi(\boldsymbol{e}_i)=d_i\boldsymbol{e}_i\). 对任意的指标集\(1\leq i_1 < i_2 < \cdots < i_r\leq n\),容易验证\(U = L(\boldsymbol{e}_{i_1},\boldsymbol{e}_{i_2},\cdots,\boldsymbol{e}_{i_r})\)是\(\varphi\)的不变子空间. 注意到\(1,2,\cdots,n\)的子集共有\(2^n\)个(空集对应于零子空间),故上述形式的\(\varphi -\)不变子空间共有\(2^n\)个. 下面我们证明\(\varphi\)的任一不变子空间都是上述不变子空间之一.

任取\(\varphi\)的非零不变子空间\(U\),设指标集
\[
I = \{i\in[1,n]|\text{存在某个}\boldsymbol{\alpha}\in U,\text{使得}\boldsymbol{\alpha}=c\boldsymbol{e}_i + \cdots,\text{其中}c\neq 0\}.
\]
因为\(U\neq 0\),故\(I\neq\varnothing\),不妨设\(I = \{i_1,i_2,\cdots,i_r\}\). 由指标集\(I\)的定义可知,\(U\subseteq L(\boldsymbol{e}_{i_1},\boldsymbol{e}_{i_2},\cdots,\boldsymbol{e}_{i_r})\). 下面我们证明\(\boldsymbol{e}_{i_j}\in U(j = 1,2,\cdots,r)\)成立. 不失一般性,我们只需证明\(\boldsymbol{e}_{i_1}\in U\)即可. 由指标集\(I\)的定义可知,存在\(\boldsymbol{\alpha}\in U\),使得
\[
\boldsymbol{\alpha}=c_1\boldsymbol{e}_{i_1}+c_2\boldsymbol{e}_{i_2}+\cdots + c_k\boldsymbol{e}_{i_k},
\]
其中\(c_1,c_2,\cdots,c_k\)都是非零常数. 将上式作用\(\varphi^l\),可得
\[
\varphi^l(\boldsymbol{\alpha})=c_1d_{i_1}^l\boldsymbol{e}_{i_1}+c_2d_{i_2}^l\boldsymbol{e}_{i_2}+\cdots + c_kd_{i_k}^l\boldsymbol{e}_{i_k},l = 1,2,\cdots,k - 1.
\]
因此,我们有
\[
(\boldsymbol{\alpha},\varphi(\boldsymbol{\alpha}),\cdots,\varphi^{k - 1}(\boldsymbol{\alpha}))=(\boldsymbol{e}_{i_1},\boldsymbol{e}_{i_2},\cdots,\boldsymbol{e}_{i_k})
\begin{pmatrix}
c_1&c_1d_{i_1}&\cdots&c_1d_{i_1}^{k - 1}\\
c_2&c_2d_{i_2}&\cdots&c_2d_{i_2}^{k - 1}\\
\vdots&\vdots&&\vdots\\
c_k&c_kd_{i_k}&\cdots&c_kd_{i_k}^{k - 1}
\end{pmatrix}.
\]
上式右边的矩阵记为\(\boldsymbol{A}\),由于\(|\boldsymbol{A}| = c_1c_2\cdots c_k\prod_{1\leq r < s\leq k}(d_{i_s}-d_{i_r})\neq 0\),故\(\boldsymbol{A}\)为可逆矩阵,从而
\[
(\boldsymbol{e}_{i_1},\boldsymbol{e}_{i_2},\cdots,\boldsymbol{e}_{i_k})=(\boldsymbol{\alpha},\varphi(\boldsymbol{\alpha}),\cdots,\varphi^{k - 1}(\boldsymbol{\alpha}))\boldsymbol{A}^{-1},
\]
特别地,\(\boldsymbol{e}_{i_1}\)可以表示为\(\boldsymbol{\alpha},\varphi(\boldsymbol{\alpha}),\cdots,\varphi^{k - 1}(\boldsymbol{\alpha})\)的线性组合. 因为\(U\)是\(\varphi\)的不变子空间,故\(\boldsymbol{\alpha},\varphi(\boldsymbol{\alpha}),\cdots,\varphi^{k - 1}(\boldsymbol{\alpha})\)都是\(U\)中的向量,从而\(\boldsymbol{e}_{i_1}\in U\),因此\(U = L(\boldsymbol{e}_{i_1},\boldsymbol{e}_{i_2},\cdots,\boldsymbol{e}_{i_r})\).

又因为任取$j_1,j_2,\cdots,j_m\in\{1,2,\cdots,n\}$,都有
\begin{align*}
L(\boldsymbol{e}_{j_1},\boldsymbol{e}_{j_2},\cdots ,\boldsymbol{e}_{j_r})=V_{j_1}\oplus V_{j_2}\oplus \cdots \oplus V_{j_m}.
\end{align*}
而特征子空间$V_{j_k}(k=1,2,\cdots,m)$都是\(\varphi\)的不变子空间,\(\varphi\)的不变子空间的直和仍是不变子空间,所以$L(\boldsymbol{e}_{j_1},\boldsymbol{e}_{j_2},\cdots ,\boldsymbol{e}_{j_r})$也是\(\varphi\)的不变子空间.

综上所述,\(\varphi\)的不变子空间共有\(2^n\)个.

{\color{blue}证法二:}
设线性变换\(\boldsymbol{\varphi}\)在\(V\)的一组基\(\{\boldsymbol{e}_1,\boldsymbol{e}_2,\cdots,\boldsymbol{e}_n\}\)下的表示矩阵是对角矩阵\(\mathrm{diag}\{\lambda_1,\lambda_2,\cdots,\lambda_n\}\), 且\(\lambda_i\)互不相同, 因此\(\boldsymbol{\varphi}\)可对角化, \(\boldsymbol{\varphi}\)有\(n\)个不同的特征值\(\lambda_1,\lambda_2,\cdots,\lambda_n\), 且\(\boldsymbol{\varphi}(\boldsymbol{e}_i) = \lambda_i\boldsymbol{e}_i(1\leq i\leq n)\). 此时, 特征值\(\lambda_i\)的特征子空间\(V_i = L(\boldsymbol{e}_i)\),并且\(V = V_1\oplus V_2\oplus\cdots\oplus V_n\). 

任取\(\boldsymbol{\varphi}\)的非零不变子空间\(U\)以及\(U\)的一组基,并将这组基扩张为\(V\)的一组基,则\(\boldsymbol{\varphi}\)在这组基下的表示矩阵是分块上三角矩阵\(\begin{pmatrix}
\boldsymbol{A}&\boldsymbol{C}\\
\boldsymbol{O}&\boldsymbol{B}
\end{pmatrix}\),其中\(\boldsymbol{A}\)是\(\boldsymbol{\varphi}|_U\)的表示矩阵, 不妨设为\(r\)阶矩阵. 考虑到
\begin{align*}
|\lambda\boldsymbol{I}_V - \boldsymbol{\varphi}| = |\lambda\boldsymbol{I} - \boldsymbol{A}||\lambda\boldsymbol{I} - \boldsymbol{B}| = (\lambda - \lambda_1)(\lambda - \lambda_2)\cdots(\lambda - \lambda_n),
\end{align*}
故\(\boldsymbol{A}\)或\(\boldsymbol{\varphi}|_U\)有\(r\)个不同的特征值, 设为\(\lambda_{i_1},\lambda_{i_2},\cdots,\lambda_{i_r}\). 考虑\(\boldsymbol{\varphi}|_U\)关于特征值\(\lambda_{i_j}\)的特征子空间\(U_{i_j} = \{\boldsymbol{u}\in U|\boldsymbol{\varphi}(\boldsymbol{u}) = \lambda_{i_j}\boldsymbol{u}\}\), 由于\(U_{i_j} = U\cap V_{i_j}\)且\(\dim V_{i_j} = 1\), 故只能是\(U_{i_j} = V_{i_j} = L(\boldsymbol{e}_{i_j})(1\leq j\leq r)\). 因为\(\boldsymbol{\varphi}|_U\)有\(r\)个不同的特征值, 所以\(\boldsymbol{\varphi}|_U\)可对角化, 于是
\begin{align*}
U = U_{i_1}\oplus U_{i_2}\oplus\cdots\oplus U_{i_r} = L(\boldsymbol{e}_{i_1},\boldsymbol{e}_{i_2},\cdots,\boldsymbol{e}_{i_r}).
\end{align*}
又因为任取$j_1,j_2,\cdots,j_m\in\{1,2,\cdots,n\}$,都有
\begin{align*}
L(\boldsymbol{e}_{j_1},\boldsymbol{e}_{j_2},\cdots ,\boldsymbol{e}_{j_r})=V_{j_1}\oplus V_{j_2}\oplus \cdots \oplus V_{j_m}.
\end{align*}
而特征子空间$V_{j_k}(k=1,2,\cdots,m)$都是\(\varphi\)的不变子空间,\(\varphi\)的不变子空间的直和仍是不变子空间,所以$L(\boldsymbol{e}_{j_1},\boldsymbol{e}_{j_2},\cdots ,\boldsymbol{e}_{j_r})$也是\(\varphi\)的不变子空间.

综上所述,\(\varphi\)的不变子空间共有\(2^n\)个. 
\end{proof}

\begin{theorem}\label{theorem:在不变子空间基下的矩阵}
设\(\varphi\)是数域\(\mathbb{F}\)上向量空间\(V\)上的线性变换,\(W\)是\(\varphi\)的不变子空间. 若取\(W\)的一组基\(\{e_1,\cdots,e_r\}\),再扩张为\(V\)的一组基\(\{e_1,\cdots,e_r,e_{r + 1},\cdots,e_n\}\),则\(\varphi\)在这组基下的表示矩阵具有下列分块上三角矩阵的形状:
\[
\begin{pmatrix}
\boldsymbol{A}_{11}&\boldsymbol{A}_{12}\\
\boldsymbol{O}&\boldsymbol{A}_{22}
\end{pmatrix},
\]
其中\(\boldsymbol{A}_{11}\)是一个\(r\)阶矩阵.
\end{theorem}

\begin{theorem}\label{theorem:在直和的基下的矩阵}
设\(\varphi\)是数域\(\mathbb{F}\)上向量空间\(V\)上的线性变换,\(V_1,V_2,\cdots,V_m\)是\(\varphi\)的不变子空间且\(V = V_1\oplus V_2\oplus\cdots\oplus V_m\). 若取\(V_i\)的基拼成\(V\)的一组基\(\{e_1,e_2,\cdots,e_n\}\),则\(\varphi\)在这组基下的表示矩阵具有下列分块对角矩阵的形状:
\[
\begin{pmatrix}
\boldsymbol{A}_{11}&&&\\
&\boldsymbol{A}_{22}&&\\
&&\ddots&\\
&&&\boldsymbol{A}_{mm}
\end{pmatrix}.
\]
\end{theorem}

\begin{proposition}\label{proposition:商空间下的线性变换对应表示矩阵}
设\(\varphi\)是\(n\)维线性空间\(V\)上的线性变换,\(U\)是\(r\)维\(\varphi -\)不变子空间. 取\(U\)的一组基\(\{\boldsymbol{e}_1,\cdots,\boldsymbol{e}_r\}\),并扩张为\(V\)的一组基\(\{\boldsymbol{e}_1,\cdots,\boldsymbol{e}_r,\boldsymbol{e}_{r + 1},\cdots,\boldsymbol{e}_n\}\). 设\(\varphi\)在这组基下的表示矩阵\(\boldsymbol{A}=(a_{ij})=\begin{pmatrix}\boldsymbol{A}_{11}&\boldsymbol{A}_{12}\\ \boldsymbol{O}&\boldsymbol{A}_{22}\end{pmatrix}\)为分块上三角阵,其中\(\boldsymbol{A}_{11}\)是\(\varphi\)在不变子空间\(U\)上的限制\(\varphi|_U\)在基\(\{\boldsymbol{e}_1,\cdots,\boldsymbol{e}_r\}\)下的表示矩阵. 证明:\(\varphi\)诱导的变换\(\overline{\varphi}(\boldsymbol{v}+U)=\varphi(\boldsymbol{v})+U\)是商空间\(V/U\)上的线性变换,并且在\(V/U\)的一组基\(\{\boldsymbol{e}_{r + 1}+U,\cdots,\boldsymbol{e}_n+U\}\)下的表示矩阵为\(\boldsymbol{A}_{22}\).
\end{proposition}
\begin{proof}
由\(U\)是\(\varphi -\)不变子空间容易验证\(\overline{\varphi}\)的定义不依赖于\(U -\)陪集代表元的选取,从而是定义好的变换. \(\overline{\varphi}\)的线性由\(\varphi\)的线性即得. 由\(\varphi\)的表示矩阵为\(\boldsymbol{A}\),再结合$\boldsymbol{e}_1,\boldsymbol{e}_2\cdots ,\boldsymbol{e}_r\in U$及\hyperref[proposition:U-陪集的性质]{U-陪集的性质(2)}和\hyperref[definition:U-陪集与商空间]{商空间的加法和数乘的定义}可得
\begin{align*}
&\begin{aligned}
\overline{\varphi }\left( \boldsymbol{e}_{r+1}+U \right) &=\varphi \left( \boldsymbol{e}_{r+1} \right) +U=a_{1,r+1}\boldsymbol{e}_1+\cdots +a_{r,r+1}\boldsymbol{e}_r+a_{r+1,r+1}\boldsymbol{e}_{r+1}+\cdots +a_{n,r+1}\boldsymbol{e}_n+U
\\
&=a_{r+1,r+1}\boldsymbol{e}_{r+1}+\cdots +a_{n,r+1}\boldsymbol{e}_n+U=a_{r+1,r+1}\left( \boldsymbol{e}_{r+1}+U \right) +\cdots +a_{n,r+1}\left( \boldsymbol{e}_n+U \right) ,
\end{aligned}
\\
&\cdots \cdots \cdots \cdots 
\\
&\overline{\varphi }(\boldsymbol{e}_n+U)=a_{r+1,n}(\boldsymbol{e}_{r+1}+U)+\cdots +a_{n,n}(\boldsymbol{e}_n+U),
\end{align*}
故\(\overline{\varphi}\)在基\(\{\boldsymbol{e}_{r + 1}+U,\cdots,\boldsymbol{e}_n+U\}\)下的表示矩阵为\(\boldsymbol{A}_{22}\). 
\end{proof}


\end{document}
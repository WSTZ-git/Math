% contents/chapter-02/section-05.tex 第二章第三节
\documentclass[../../main.tex]{subfiles}
\graphicspath{{\subfix{../../image/}}} % 指定图片目录,后续可以直接使用图片文件名。

% 例如:
% \begin{figure}[h]
% \centering
% \includegraphics{image-01.01}
% \label{fig:image-01.01}
% \caption{图片标题}
% \end{figure}

\begin{document}

\section{矩阵乘法与行列式计算}

\begin{proposition}[可以写成两个矩阵(向量)乘积的矩阵]\label{proposition:可以写成两个矩阵(向量)乘积的矩阵}
若已知矩阵\begin{align*}
A=\left( \begin{matrix}
a_{11}b_{11}+a_{12}b_{12}+\cdots +a_{1n}b_{1n}&		a_{21}b_{11}+a_{22}b_{12}+\cdots +a_{2n}b_{1n}&		\cdots&		a_{n1}b_{11}+a_{n2}b_{12}+\cdots +a_{nn}b_{1n}\\
a_{11}b_{21}+a_{12}b_{22}+\cdots +a_{1n}b_{2n}&		a_{21}b_{21}+a_{22}b_{22}+\cdots +a_{2n}b_{2n}&		\cdots&		a_{n1}b_{21}+a_{n2}b_{22}+\cdots +a_{nn}b_{2n}\\
\vdots&		\vdots&		&		\vdots\\
a_{11}b_{n1}+a_{12}b_{n2}+\cdots +a_{1n}b_{nn}&		a_{21}b_{n1}+a_{22}b_{n2}+\cdots +a_{2n}b_{nn}&		\cdots&		a_{n1}b_{n1}+a_{n2}b_{n2}+\cdots +a_{nn}b_{nn}\\
\end{matrix} \right) .
\end{align*}
则矩阵$A$可以写成$BC$,其中
\begin{align*}
B=\left( \begin{matrix}
b_{11}&		b_{12}&		\cdots&		b_{1n}\\
b_{21}&		b_{22}&		\cdots&		b_{2n}\\
\vdots&		\vdots&		&		\vdots\\
b_{n1}&		b_{n2}&		\cdots&		b_{nn}\\
\end{matrix} \right),
C=\left( \begin{matrix}
a_{11}&		a_{21}&		\cdots&		a_{n1}\\
a_{12}&		a_{22}&		\cdots&		a_{n2}\\
\vdots&		\vdots&		&		\vdots\\
a_{1n}&		a_{2n}&		\cdots&		a_{nn}\\
\end{matrix} \right) .
\end{align*}
即\begin{align*}
A=\left( \begin{matrix}
b_{11}&		b_{12}&		\cdots&		b_{1n}\\
b_{21}&		b_{22}&		\cdots&		b_{2n}\\
\vdots&		\vdots&		&		\vdots\\
b_{n1}&		b_{n2}&		\cdots&		b_{nn}\\
\end{matrix} \right) \left( \begin{matrix}
a_{11}&		a_{21}&		\cdots&		a_{n1}\\
a_{12}&		a_{22}&		\cdots&		a_{n2}\\
\vdots&		\vdots&		&		\vdots\\
a_{1n}&		a_{2n}&		\cdots&		a_{nn}\\
\end{matrix} \right) =BC.
\end{align*}
特别地,若矩阵$A$的行/列向量成比例,不妨设
\begin{align*}
A=\left( \begin{matrix}
a_1b_1&		a_2b_1&		\cdots&		a_nb_1\\
a_1b_2&		a_2b_2&		\cdots&		a_nb_2\\
\vdots&		\vdots&		&		\vdots\\
a_1b_n&		a_2b_n&		\cdots&		a_nb_n\\
\end{matrix} \right) .
\end{align*}
则令$\alpha =\left( a_1,a_2,\cdots ,a_n \right) ,\beta =\left( b_1,b_2,\cdots ,b_n \right)$,就有$A=\beta'\alpha=\left( \begin{array}{c}
b_1\\
b_2\\
\vdots\\
b_n\\
\end{array} \right) \left( \begin{matrix}
a_1&		a_2&		\cdots&		a_n\\
\end{matrix} \right)$.
\end{proposition}
\begin{remark}
若矩阵的列向量成比例,则行向量也一定成比例.反之也成立.
\end{remark}
\begin{note}
观察原矩阵$A$不难发现:\textbf{矩阵$A$的每一行沿行方向只有$a_{ij}$(的角标)改变,而$b_{kl}$(的角标)并不改变;而矩阵$A$的每一列沿列方向只有$b_{kl}$(的角标)改变,$a_{ij}$(的角标)并不改变.}因此具有这种性质的矩阵,都可以按照\hyperref[proposition:可以写成两个矩阵乘积的矩阵]{这个命题}将其写成两个矩阵的乘积.特别地,\textbf{若矩阵的行/列向量成比例,则一定可以将其写成两个向量的乘积.}

记忆小技巧:只需要记住矩阵$B$的形式(沿行方向不变的项写在前面作为矩阵$B$的元素),然后结合原矩阵,利用矩阵乘法就能写出矩阵$C$.即\textbf{按行变化的项写左边(作为矩阵$B$的元素),按列变化的项写右边(作为矩阵$C$的元素).}
\end{note}
\begin{remark}
拆分后的矩阵$B$的行数与原矩阵$A$相同,矩阵$C$的列数与原矩阵$A$相同.但是矩阵$B$的列数与矩阵$C$的行数可以任意选取,只要满足$BC=A$即可.
\end{remark}
\begin{proof}
利用矩阵乘法容易得到证明.
\end{proof}

\begin{proposition}[一些能写成两个向量乘积的矩阵]\label{proposition:一些能写成两个向量乘积的矩阵}
1.$\quad$$\left( \begin{matrix}
1&		1&		\cdots&		1\\
1&		1&		\cdots&		1\\
\vdots&		\vdots&		&		\vdots\\
1&		1&		\cdots&		1\\
\end{matrix} \right) =\alpha\alpha'$,其中$\alpha =\left( 1,1,\cdots ,1 \right)' $.

2.若矩阵$A$的行/列向量成比例,不妨设
\begin{align*}
A=\left( \begin{matrix}
a_1b_1&		a_1b_2&		\cdots&		a_1b_n\\
a_2b_1&		a_2b_2&		\cdots&		a_2b_n\\
\vdots&		\vdots&		&		\vdots\\
a_nb_1&		a_nb_2&		\cdots&		a_nb_n\\
\end{matrix} \right) .
\end{align*}
则有$A =\left( \begin{array}{c}
a_1\\
a_2\\
\vdots\\
a_n\\
\end{array} \right) \left( \begin{matrix}
b_1&		b_2&		\cdots&		b_n\\
\end{matrix} \right) $.
\end{proposition}
\begin{note}
这里的$a_i$可以是行向量,$b_i$可以是列向量.此时矩阵$A$的元素就是$a_ib_i$仍然是一个数.并且此时矩阵$A$能够分解的条件应该改为\textbf{矩阵$A$的每一行都有公共的行向量$a_i$,每一列都有公共的列向量$b_i$.}

\begin{remark}
若$a_i,b_i$是上述向量,则根据矩阵乘法,可知$a_i$的列数可以任意选取,$b_i$的行数可以任意选取.此时只要确定每个向量$a_i,b_i$就可以确定矩阵$A$的分解式.
\end{remark}
\end{note}

\begin{example}
设\(s_k = x_1^k + x_2^k+\cdots+x_n^k(k\geq1)\),\(s_0 = n\),
\[
S = 
\begin{pmatrix}
s_0 & s_1 & s_2 & \cdots & s_{n - 1}\\
s_1 & s_2 & s_3 & \cdots & s_{n}\\
s_2 & s_3 & s_4 & \cdots & s_{n + 1}\\
\vdots & \vdots & \vdots & & \vdots\\
s_{n - 1} & s_{n} & s_{n + 1} & \cdots & s_{2n - 2}
\end{pmatrix}.
\]
求\(|S|\)的值并证明若\(x_i\)是实数,则\(|S|\geq0\).
\end{example}
\begin{solution}
设
\[
V = 
\begin{pmatrix}
1 & 1 & 1 & \cdots & 1\\
x_1 & x_2 & x_3 & \cdots & x_n\\
x_1^2 & x_2^2 & x_3^2 & \cdots & x_n^2\\
\vdots & \vdots & \vdots & & \vdots\\
x_1^{n - 1} & x_2^{n - 1} & x_3^{n - 1} & \cdots & x_n^{n - 1}
\end{pmatrix}
\]
则\(S = VV'\),因此
\[
|S| = |V|^2=\prod_{1\leq i<j\leq n}(x_j - x_i)^2\geq0.
\]
\end{solution}

\begin{example}
设\(s_k = x_1^k + x_2^k+\cdots + x_n^k(k\geq1)\),\(s_0 = n\),计算矩阵\(A\)的行列式的值:
\[
A = 
\begin{pmatrix}
s_0 & s_1 & \cdots & s_{n - 1} & 1\\
s_1 & s_2 & \cdots & s_n & x\\
\vdots & \vdots & & \vdots & \vdots\\
s_n & s_{n + 1} & \cdots & s_{2n - 1} & x^n
\end{pmatrix}.
\]
\end{example}
\begin{solution}
将矩阵\(A\)分解为两个矩阵的乘积:
\[
A = 
\begin{pmatrix}
1 & 1 & \cdots & 1 & 1\\
x_1 & x_2 & \cdots & x_n & x\\
\vdots & \vdots & & \vdots & \vdots\\
x_1^{n - 1} & x_2^{n - 1} & \cdots & x_n^{n - 1} & x^{n - 1}\\
x_1^n & x_2^n & \cdots & x_n^n & x^n
\end{pmatrix}
\begin{pmatrix}
1 & x_1 & \cdots & x_1^{n - 1} & 0\\
1 & x_2 & \cdots & x_2^{n - 1} & 0\\
\vdots & \vdots & & \vdots & \vdots\\
1 & x_n & \cdots & x_n^{n - 1} & 0\\
0 & 0 & \cdots & 0 & 1
\end{pmatrix}
\]
因此\(|A|=(x - x_1)(x - x_2)\cdots(x - x_n)\prod_{1\leq i < j\leq n}(x_j - x_i)^2\).
\end{solution}


\begin{example}
计算下列矩阵\(A\)的行列式的值:
\[
A = 
\begin{pmatrix}
x & -y & -z & -w\\
y & x & -w & z\\
z & w & x & -y\\
w & -z & y & x
\end{pmatrix}.
\]
\end{example}
\begin{solution}
{\color{blue}解法一:}
注意到
\[
AA' = 
\begin{pmatrix}
x & -y & -z & -w\\
y & x & -w & z\\
z & w & x & -y\\
w & -z & y & x
\end{pmatrix}
\begin{pmatrix}
x & y & z & w\\
-y & x & w & -z\\
-z & -w & x & y\\
-w & z & -y & x
\end{pmatrix}
=
\begin{pmatrix}
u & 0 & 0 & 0\\
0 & u & 0 & 0\\
0 & 0 & u & 0\\
0 & 0 & 0 & u
\end{pmatrix}
\]
其中\(u = x^2 + y^2 + z^2 + w^2\),因此
\[
|A|^2=(x^2 + y^2 + z^2 + w^2)^4.
\]
故
\[
|A|=(x^2 + y^2 + z^2 + w^2)^2.
\]
在矩阵\(A\)中令\(x = 1,y = z = w = 0\),显然\(|A| = 1\).

{\color{blue}解法二:}
令
\[
B = 
\begin{pmatrix}
x & -y\\
y & x
\end{pmatrix},
C = 
\begin{pmatrix}
z & w\\
w & -z
\end{pmatrix},
\]
则\(|A| = 
\begin{vmatrix}
B & -C\\
C & B
\end{vmatrix}\).由\hyperref[proposition:对角相同的复分块矩阵行列式计算]{命题\ref{proposition:对角相同的复分块矩阵行列式计算}}可得
\begin{align*}
|A|&=|B + \mathrm{i}C||B - \mathrm{i}C|
=\begin{vmatrix}
x + \mathrm{i}z & -y + \mathrm{i}w\\
y + \mathrm{i}w & x - \mathrm{i}z
\end{vmatrix}\begin{vmatrix}
x - \mathrm{i}z & -y - \mathrm{i}w\\
y - \mathrm{i}w & x + \mathrm{i}z
\end{vmatrix}
=(x^2 + y^2 + z^2 + w^2)^2.
\end{align*}
\end{solution}

\begin{example}
计算下列矩阵\(A\)的行列式的值:
\[
A = 
\begin{pmatrix}
\cos\theta & \cos2\theta & \cos3\theta & \cdots & \cos n\theta\\
\cos n\theta & \cos\theta & \cos2\theta & \cdots & \cos(n - 1)\theta\\
\cos(n - 1)\theta & \cos n\theta & \cos\theta & \cdots & \cos(n - 2)\theta\\
\vdots & \vdots & \vdots & & \vdots\\
\cos2\theta & \cos3\theta & \cos4\theta & \cdots & \cos\theta
\end{pmatrix}.
\]
\end{example}
\begin{solution}
解 由\hyperref[proposition:循环行列式计算公式]{上面的结论}可知
\[
|A| = f(\varepsilon_1)f(\varepsilon_2)\cdots f(\varepsilon_n),
\]
其中\(\varepsilon_1,\varepsilon_2,\cdots,\varepsilon_n\)是\(1\)的所有\(n\)次方根,\(f(x)=\cos\theta + x\cos2\theta+\cdots+x^{n - 1}\cos n\theta\).令
\[
g(x)=\sin\theta + x\sin2\theta+\cdots+x^{n - 1}\sin n\theta,
\]
则由De Moivre公式可得
\begin{align*}
&f(x)+\mathrm{i}g(x)=\left( \cos \theta +\mathrm{i}\sin \theta \right) +x\left( \cos \theta +\mathrm{i}\sin \theta \right) ^2+\cdots +x^{n-1}\left( \cos \theta +\mathrm{i}\sin \theta \right) ^n
\\
&=\frac{\left( \cos \theta +\mathrm{i}\sin \theta \right) \left[ 1-x^n\left( \cos \theta +\mathrm{i}\sin \theta \right) ^n \right]}{1-x\left( \cos \theta +\mathrm{i}\sin \theta \right)}=\frac{1-x^n\left( \cos \theta +\mathrm{i}\sin \theta \right) ^n}{\cos \theta -x-\mathrm{i}\sin \theta}=\frac{\left( 1-x^n\cos n\theta +\mathrm{i}x^n\sin n\theta \right) \left( \cos \theta -x+\mathrm{i}\sin \theta \right)}{\left[ \left( \cos \theta -\mathrm{i}\sin \theta \right) -x \right] \left[ \left( \cos \theta +\mathrm{i}\sin \theta \right) -x \right]}
\\
&=\frac{\left( 1-x^n\cos n\theta +\mathrm{i}x^n\sin n\theta \right) \left( \cos \theta -x+\mathrm{i}\sin \theta \right)}{\left( \cos \theta -x \right) ^2+\sin ^2\theta}=\frac{\left( 1-x^n\cos n\theta +\mathrm{i}x^n\sin n\theta \right) \left( \cos \theta -x+\mathrm{i}\sin \theta \right)}{x^2-2x\cos \theta +1}
\\
&=\frac{\left[ x^{n+1}\cos n\theta -x^n\cos \left( n+1 \right) \theta -x+\cos \theta \right] +\mathrm{i}\left[ x^{n+1}\sin n\theta +x^n\sin \left( n-1 \right) \theta +\sin \theta \right]}{x^2-2x\cos \theta +1}.
\end{align*}
再比较实部,可得
\[
f(x)=\frac{\cos n\theta\cdot x^{n + 1}-\cos(n + 1)\theta\cdot x^n - x+\cos\theta}{x^2 - 2\cos\theta\cdot x + 1}.
\]
对任意的\(\varepsilon_i\),经计算并化简,可得
\[
f(\varepsilon _i)=\frac{\left( \cos \theta -\cos \left( n+1 \right) \theta \right) -\varepsilon _i\left( 1-\cos n\theta \right)}{\left[ \left( \cos \theta +\mathrm{i}\sin \theta \right) -\varepsilon _i \right] \left[ \left( \cos \theta -\mathrm{i}\sin \theta \right) -\varepsilon _i \right]}.
\]
注意到对任意的\(a,b\),有\(a^n - b^n=(a - \varepsilon_1b)(a - \varepsilon_2b)\cdots(a - \varepsilon_nb)\),因此
\begin{align*}
&\left| A \right|=\prod_{i=1}^n{f(\varepsilon _i)}=\frac{(\cos \theta -\cos\mathrm{(}n+1)\theta )^n-(1-\cos n\theta )^n}{(\cos n\theta +\mathrm{i}\sin n\theta -1)(\cos n\theta -\mathrm{i}\sin n\theta -1)}
\\
&=\frac{(\cos \theta -\cos\mathrm{(}n+1)\theta )^n-(1-\cos n\theta )^n}{2(1-\cos n\theta )}
\\
&=\frac{2^n\sin ^n\frac{n\theta}{2}\sin ^n\frac{\left( n-2 \right) \theta}{2}-2^n\sin ^{2n}\frac{n\theta}{2}}{4\sin ^2\frac{n\theta}{2}}
\\
&=2^{n-2}\sin ^{n-2}\frac{n\theta}{2}\left( \sin ^n\frac{(n+2)\theta}{2}-\sin ^n\frac{n\theta}{2} \right) .
\end{align*}
\end{solution}



\end{document}
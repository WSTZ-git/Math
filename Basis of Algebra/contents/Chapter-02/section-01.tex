% contents/chapter-02/section-01.tex 第二章第一节
\documentclass[../../main.tex]{subfiles}
\graphicspath{{\subfix{../../image/}}} % 指定图片目录,后续可以直接使用图片文件名。

% 例如:
% \begin{figure}[h]
% \centering
% \includegraphics{image-01.01}
% \label{fig:image-01.01}
% \caption{图片标题}
% \end{figure}

\begin{document}


\section{矩阵的运算}

\begin{proposition}[标准单位向量和基础矩阵]\label{proposition:标准单位向量和基础矩阵}
\large{\textbf{1.标准单位向量}}

\(n\)维标准单位列向量是指下列\(n\)个\(n\)维列向量:
\[
\boldsymbol{e}_{1}=\left(\begin{array}{c}
1 \\
0 \\
\vdots \\
0
\end{array}\right), \quad \boldsymbol{e}_{2}=\left(\begin{array}{c}
0 \\
1 \\
\vdots \\
0
\end{array}\right), \quad \cdots, \quad \boldsymbol{e}_{n}=\left(\begin{array}{c}
0 \\
0 \\
\vdots \\
1
\end{array}\right)
\]

向量组\(\boldsymbol{e}_{1}', \boldsymbol{e}_{2}', \cdots, \boldsymbol{e}_{n}'\)则被称为\(n\)维标准单位行向量,容易验证标准单位向量有下列基本性质:
\begin{enumerate}
\item 若\(i \neq j\),则\(\boldsymbol{e}_{i}' \boldsymbol{e}_{j}=0\),而\(\boldsymbol{e}_{i}' \boldsymbol{e}_{i}=1\);

\item 若\(A=(a_{ij})\)是\(m\times n\)矩阵,则\(A\boldsymbol{e}_{i}\)是\(A\)的第\(i\)个列向量;\(\boldsymbol{e}_{i}'A\)是\(A\)的第\(i\)个行向量;

\item 若\(A=(a_{ij})\)是\(m\times n\)矩阵,则\(\boldsymbol{e}_{i}'A\boldsymbol{e}_{j}=a_{ij}\);

\item \label{矩阵相等的判定准则}\hypertarget{proposition:矩阵相等的判定法则}{\textbf{判定准则:}}设\(A,B\)都是\(m\times n\)矩阵,则\(A = B\)当且仅当\(A\boldsymbol{e}_{i}=B\boldsymbol{e}_{i}(1\leq i\leq n)\)成立,也当且仅当\(\boldsymbol{e}_{i}'A=\boldsymbol{e}_{i}'B(1\leq i\leq m)\)成立.
\end{enumerate}

\large{\textbf{2.基础矩阵}}

\(n\)阶基础矩阵(又称初级矩阵)是指\(n^{2}\)个\(n\)阶矩阵\(\{E_{ij}, 1\leq i,j\leq n\}\).这里\(E_{ij}\)是一个\(n\)阶矩阵,它的第\((i,j)\)元素等于\(1\),其他元素全为\(0\).基础矩阵也可以看成是标准单位向量的积:\(E_{ij}=\boldsymbol{e}_{i}\boldsymbol{e}_{j}^{T}\).由此不难证明基础矩阵的下列性质:
\begin{enumerate}
\item 若\(j\neq k\),则\(E_{ij}E_{kl} = 0\);

\item  若\(j = k\),则\(E_{ij}E_{kl}=E_{il}\);

\item  若\(A\)是\(n\)阶矩阵且\(A=(a_{ij})\),则\(A=\sum_{i = 1}^{n}\sum_{j = 1}^{n}a_{ij}E_{ij}\);

\item  若\(A\)是\(n\)阶矩阵且\(A=(a_{ij})\),则\(E_{ij}A\)的第\(i\)行是\(A\)的第\(j\)行,\(E_{ij}A\)的其他行全为零;

\item  若\(A\)是\(n\)阶矩阵且\(A=(a_{ij})\),则\(AE_{ij}\)的第\(j\)列是\(A\)的第\(i\)列,\(AE_{ij}\)的其他列全为零;

\item  若\(A\)是\(n\)阶矩阵且\(A=(a_{ij})\),则\(E_{ij}AE_{kl}=a_{jk}E_{il}\).
\end{enumerate}
\end{proposition}
\begin{note}
标准单位向量和基础矩阵虽然很简单,但如能灵活应用就可以得到意外的结果.我们在今后将经常应用它们,因此请读者熟记这些结论.

一些常见的想法:

\textbf{1.可以将一般的矩阵写成标准单位列向量或基础矩阵的形式(这个形式可以是和式的形式,也可以是分块的形式)}.

\textbf{2.如果要证明两个矩阵相等,那么我们就可以考虑\hyperlink{proposition:矩阵相等的判定法则}{判定法则}}.

\textbf{3.如果某种等价关系蕴含了一种递减的规律(项数减少,阶数降低等),那么我们就可以考虑数学归纳法,去尝试根据这个规律得到一些结论}.
\end{note}

\begin{definition}[循环矩阵]\label{definition:循环矩阵}
1. 下列形状的$n$阶矩阵称为\(n\)阶基础循环矩阵:
\[ 
\boldsymbol{J}=
\left( \begin{matrix}
0&		1&		0&		\cdots&		0\\
0&		0&		1&		\cdots&		0\\
\vdots&		\vdots&		\vdots&	\ddots&		\vdots\\
0&		0&		0&		\cdots&		1\\
1&		0&		0&		\cdots&		0\\
\end{matrix} \right)=\left( \begin{matrix}
	O&		I_{n-1}\\
	1&		O\\
\end{matrix} \right) .
\]

2. 下列形状的矩阵称为循环矩阵:
\[
\begin{pmatrix}
a_1 & a_2 & a_3 & \cdots & a_n \\
a_n & a_1 & a_2 & \cdots & a_{n - 1} \\
a_{n - 1} & a_n & a_1 & \cdots & a_{n - 2} \\
\vdots & \vdots & \vdots & & \vdots \\
a_2 & a_3 & a_4 & \cdots & a_1
\end{pmatrix}.
\]
\end{definition}
\begin{note}
记\(C_n(\mathbb{K})\)为\(\mathbb{K}\)上所有\(n\)阶循环矩阵构成的集合.
\end{note}

\begin{proposition}[循环矩阵的性质]\label{proposition:循环矩阵的性质}
\begin{enumerate}
\item 若$\boldsymbol{J}$为$n$阶基础循环矩阵,
则
\[
\boldsymbol{J}^{k} = 
\begin{pmatrix}
O & I_{n - k} \\
I_{k} & O
\end{pmatrix},  1 \leq k \leq n.
\]
\item 若$A$是循环矩阵,即\begin{align*}
A=\begin{pmatrix}
a_1 & a_2 & a_3 & \cdots & a_n \\
a_n & a_1 & a_2 & \cdots & a_{n - 1} \\
a_{n - 1} & a_n & a_1 & \cdots & a_{n - 2} \\
\vdots & \vdots & \vdots & & \vdots \\
a_2 & a_3 & a_4 & \cdots & a_1
\end{pmatrix},
\end{align*}
则循环矩阵$A$可以表示为基础循环矩阵$J$的多项式:
\begin{align*}
A=a_1I_n+a_2J+a_3J^2+\cdots+a_nJ^{n-1}.
\end{align*}
反之,若一个矩阵能表示为基础循环矩阵$J$的多项式,则它必是循环矩阵.
\item \label{example:iten546641856}同阶循环矩阵之积仍是循环矩阵.
\item 基础循环矩阵$\boldsymbol{J}=
\begin{pmatrix}
O & I_{n - k} \\
I_{k} & O
\end{pmatrix} (1 \leq k \leq n)$的逆仍是循环矩阵,并且
\begin{align*}
\boldsymbol{J}^{-1} =\begin{pmatrix}
O & I_{k} \\
I_{n-k} & O
\end{pmatrix},1 \leq k \leq n.
\end{align*}
\end{enumerate}
\end{proposition}
\begin{note}
循环矩阵的性质及应用详见\href{https://www.cnblogs.com/torsor/p/8848641.html}{谢启鸿博客}.
\end{note}
\begin{proof}
\begin{enumerate}
\item 将$\boldsymbol{J}$写作$(e_n,e_1,\cdots,e_{n-1})$,其中$e_i$是标准单位列向量($i=1,2,\cdots,n$).由分块矩阵乘法并注意到$\boldsymbol{J}e_i$就是$\boldsymbol{J}$的第$i$列,可得
\begin{align*}
\boldsymbol{J}^2=\boldsymbol{J}\left( e_{n,}e_1,\cdots ,e_{n-1} \right) =\left( \boldsymbol{J}e_{n,}\boldsymbol{J}e_1,\cdots ,\boldsymbol{J}e_{n-1} \right) =\left( e_{n-1},e_n,\cdots ,e_{n-2} \right) .
\end{align*}
不断这样做下去就可以得到结论.
\item 由\hyperref[definition:循环矩阵]{循环矩阵和基础循环矩阵的定义}和\hyperref[proposition:循环矩阵的性质]{循环矩阵的性质1}容易得到证明.
\item 由\hyperref[proposition:循环矩阵的性质]{循环矩阵的性质2}可知两个循环矩阵之积可写为基础循环矩阵$J$的两个多项式之积.又由\hyperref[proposition:循环矩阵的性质]{循环矩阵的性质1},可知$J^n=I_n$.因此两个循环矩阵之积可以表示为基础循环矩阵$J$的多项式,故由\hyperref[proposition:循环矩阵的性质]{循环矩阵的性质1}即得结论.
\item 利用矩阵初等行变换可得
\begin{align*}
\left( \begin{matrix}
O&		\boldsymbol{I}_{n-k}&		\boldsymbol{I}_{n-k}&		O\\
\boldsymbol{I}_k&		O&		O&		\boldsymbol{I}_k\\
\end{matrix} \right) \rightarrow \left( \begin{matrix}
\boldsymbol{I}_k&		O&		O&		\boldsymbol{I}_k\\
O&		\boldsymbol{I}_{n-k}&		\boldsymbol{I}_{n-k}&		O\\
\end{matrix} \right) ,1\le k\le n.
\end{align*}
从而$\boldsymbol{J}^{-1}=\left( \begin{matrix}
O&		\boldsymbol{I}_k\\
\boldsymbol{I}_{n-k}&		O\\
\end{matrix} \right) ,1\le k\le n.$
\end{enumerate}
\end{proof}

\begin{proposition}[循环行列式关于$n$次方根的计算公式]\label{proposition:循环行列式计算公式}
已知下列循环矩阵\(A\):
\[
A = 
\begin{pmatrix}
a_1 & a_2 & a_3 & \cdots & a_n\\
a_n & a_1 & a_2 & \cdots & a_{n - 1}\\
a_{n - 1} & a_n & a_1 & \cdots & a_{n - 2}\\
\vdots & \vdots & \vdots & & \vdots\\
a_2 & a_3 & a_4 & \cdots & a_1
\end{pmatrix}.
\]
则\(A\)的行列式的值为:
\[
|A|| = f(\varepsilon_1)f(\varepsilon_2)\cdots f(\varepsilon_n).
\]
其中\(f(x)=a_1 + a_2x + a_3x^2+\cdots+a_nx^{n - 1}\),\(\varepsilon_1,\varepsilon_2,\cdots,\varepsilon_n\)是\(1\)的所有\(n\)次方根.
\end{proposition}
\begin{note}
关键是要注意到
\begin{align*}
AV = 
\begin{pmatrix}
f(\varepsilon_1) & f(\varepsilon_2) & f(\varepsilon_3) & \cdots & f(\varepsilon_n)\\
\varepsilon_1f(\varepsilon_1) & \varepsilon_2f(\varepsilon_2) & \varepsilon_3f(\varepsilon_3) & \cdots & \varepsilon_nf(\varepsilon_n)\\
\varepsilon_1^2f(\varepsilon_1) & \varepsilon_2^2f(\varepsilon_2) & \varepsilon_3^2f(\varepsilon_3) & \cdots & \varepsilon_n^2f(\varepsilon_n)\\
\vdots & \vdots & \vdots & & \vdots\\
\varepsilon_1^{n - 1}f(\varepsilon_1) & \varepsilon_2^{n - 1}f(\varepsilon_2) & \varepsilon_3^{n - 1}f(\varepsilon_3) & \cdots & \varepsilon_n^{n - 1}f(\varepsilon_n)
\end{pmatrix}.
\end{align*}
然后再利用\hyperref[proposition:将矩阵拆分成Vandermode矩阵的形式]{命题\ref{proposition:将矩阵拆分成Vandermode矩阵的形式}}就能得到分解$AV = V\Lambda$.
\end{note}
\begin{proof}
作多项式\(f(x)=a_1 + a_2x + a_3x^2+\cdots+a_nx^{n - 1}\),令\(\varepsilon_1,\varepsilon_2,\cdots,\varepsilon_n\)是\(1\)的所有\(n\)次方根.又令
\[
V = 
\begin{pmatrix}
1 & 1 & 1 & \cdots & 1\\
\varepsilon_1 & \varepsilon_2 & \varepsilon_3 & \cdots & \varepsilon_n\\
\varepsilon_1^2 & \varepsilon_2^2 & \varepsilon_3^2 & \cdots & \varepsilon_n^2\\
\vdots & \vdots & \vdots & & \vdots\\
\varepsilon_1^{n - 1} & \varepsilon_2^{n - 1} & \varepsilon_3^{n - 1} & \cdots & \varepsilon_n^{n - 1}
\end{pmatrix},
\Lambda = 
\begin{pmatrix}
f(\varepsilon_1) & 0 & 0 & \cdots & 0\\
0 & f(\varepsilon_2) & 0 & \cdots & 0\\
0 & 0 & f(\varepsilon_3) & \cdots & 0\\
\vdots & \vdots & \vdots & & \vdots\\
0 & 0 & 0 & \cdots & f(\varepsilon_n)
\end{pmatrix}
\]
则
\[
AV = 
\begin{pmatrix}
f(\varepsilon_1) & f(\varepsilon_2) & f(\varepsilon_3) & \cdots & f(\varepsilon_n)\\
\varepsilon_1f(\varepsilon_1) & \varepsilon_2f(\varepsilon_2) & \varepsilon_3f(\varepsilon_3) & \cdots & \varepsilon_nf(\varepsilon_n)\\
\varepsilon_1^2f(\varepsilon_1) & \varepsilon_2^2f(\varepsilon_2) & \varepsilon_3^2f(\varepsilon_3) & \cdots & \varepsilon_n^2f(\varepsilon_n)\\
\vdots & \vdots & \vdots & & \vdots\\
\varepsilon_1^{n - 1}f(\varepsilon_1) & \varepsilon_2^{n - 1}f(\varepsilon_2) & \varepsilon_3^{n - 1}f(\varepsilon_3) & \cdots & \varepsilon_n^{n - 1}f(\varepsilon_n)
\end{pmatrix}
= V\Lambda
\]
从而$V^{-1}AV=\Lambda$,
又因为\(\varepsilon_i\)互不相同,所以\(|V|\neq0\),
故
\[
|A|=|V^{-1}AV| = |\Lambda| = f(\varepsilon_1)f(\varepsilon_2)\cdots f(\varepsilon_n).
\]
\end{proof}

\begin{proposition}[b-循环矩阵]\label{proposition:b-循环矩阵}
设\(b\)为非零常数,下列形状的矩阵称为b -循环矩阵:
\[
A = 
\begin{pmatrix}
a_1 & a_2 & a_3 & \cdots & a_n\\
ba_n & a_1 & a_2 & \cdots & a_{n - 1}\\
ba_{n - 1} & ba_n & a_1 & \cdots & a_{n - 2}\\
\vdots & \vdots & \vdots & & \vdots\\
ba_2 & ba_3 & ba_4 & \cdots & a_1
\end{pmatrix}
\]
\begin{enumerate}[(1)]
\item 证明:同阶b -循环矩阵的乘积仍然是b-循环矩阵;
\item 求上述b-循环矩阵\(A\)的行列式的值.
\end{enumerate}
\end{proposition}
\begin{proof}
\begin{enumerate}[(1)]
\item (证明类似于\hyperref[example:iten546641856]{循环矩阵的性质\ref{example:iten546641856}})设\(J_b=\begin{pmatrix}
O & I_{n - 1}\\
b & O
\end{pmatrix}\),则$J_{b}^{k}=\left( \begin{matrix}
O&		I_{n-k}\\
bI_k&		O\\
\end{matrix} \right) ,0\le k\le n-1$.从而\(J_b^n = bI_n\)且\(A = a_1I_n + a_2J_b + a_3J_b^2+\cdots+ a_nJ_b^{n - 1}\). 因此同阶\(b -\)循环阵的乘积仍然可以写成$J_b$的$n-1$次多项式,故同阶\(b -\)循环阵的乘积仍然是\(b -\)循环矩阵.

\item (证明完全类似\hyperref[proposition:循环行列式计算公式]{循环行列式计算公式的证明})作多项式\(f(x)=a_1 + a_2x + a_3x^2+\cdots + a_nx^{n - 1}\),令\(\varepsilon_1,\varepsilon_2,\cdots,\varepsilon_n\)是\(b\)的所有\(n\)次方根.又令
\[
V = 
\begin{pmatrix}
1 & 1 & 1 & \cdots & 1\\
\varepsilon_1 & \varepsilon_2 & \varepsilon_3 & \cdots & \varepsilon_n\\
\varepsilon_1^2 & \varepsilon_2^2 & \varepsilon_3^2 & \cdots & \varepsilon_n^2\\
\vdots & \vdots & \vdots & & \vdots\\
\varepsilon_1^{n - 1} & \varepsilon_2^{n - 1} & \varepsilon_3^{n - 1} & \cdots & \varepsilon_n^{n - 1}
\end{pmatrix},
\Lambda = 
\begin{pmatrix}
f(\varepsilon_1) & 0 & 0 & \cdots & 0\\
0 & f(\varepsilon_2) & 0 & \cdots & 0\\
0 & 0 & f(\varepsilon_3) & \cdots & 0\\
\vdots & \vdots & \vdots & & \vdots\\
0 & 0 & 0 & \cdots & f(\varepsilon_n)
\end{pmatrix}
\]
则
\[
AV = 
\begin{pmatrix}
f(\varepsilon_1) & f(\varepsilon_2) & f(\varepsilon_3) & \cdots & f(\varepsilon_n)\\
\varepsilon_1f(\varepsilon_1) & \varepsilon_2f(\varepsilon_2) & \varepsilon_3f(\varepsilon_3) & \cdots & \varepsilon_nf(\varepsilon_n)\\
\varepsilon_1^2f(\varepsilon_1) & \varepsilon_2^2f(\varepsilon_2) & \varepsilon_3^2f(\varepsilon_3) & \cdots & \varepsilon_n^2f(\varepsilon_n)\\
\vdots & \vdots & \vdots & & \vdots\\
\varepsilon_1^{n - 1}f(\varepsilon_1) & \varepsilon_2^{n - 1}f(\varepsilon_2) & \varepsilon_3^{n - 1}f(\varepsilon_3) & \cdots & \varepsilon_n^{n - 1}f(\varepsilon_n)
\end{pmatrix}
= V\Lambda
\]
从而$V^{-1}AV=\Lambda$,
又因为\(\varepsilon_i\)互不相同,所以\(|V|\neq0\),
故
\[
|A|=|V^{-1}AV| = |\Lambda| = f(\varepsilon_1)f(\varepsilon_2)\cdots f(\varepsilon_n).
\]
\end{enumerate}
\end{proof}


\begin{proposition}[幂零Jordan块]\label{proposition:幂零Jordan块}
设\(n\)阶幂零Jordan块
\[
\boldsymbol{A}=\left(\begin{array}{ccccc}
0 & 1 & 0 & \cdots & 0 \\
0 & 0 & 1 & \cdots & 0 \\
\vdots & \vdots & \vdots & \ddots & \vdots \\
0 & 0 & 0 & \cdots & 1 \\
0 & 0 & 0 & \cdots & 0
\end{array}\right)
\]
则

\[
\boldsymbol{A}^{k}=\left(\begin{array}{cc}
O & I_{n - k} \\
O & O
\end{array}\right),1 \leq k \leq n.
\]
\end{proposition}
\begin{proof}
将\(\boldsymbol{A}\)写为\(\boldsymbol{A}=(0,\mathbf{e}_{1},\mathbf{e}_{2},\cdots,\mathbf{e}_{n - 1})\),其中\(\mathbf{e}_{i}\)是标准单位列向量.由分块矩阵乘法并注意\(\boldsymbol{A}\mathbf{e}_{i}\)就是\(\boldsymbol{A}\)的第\(i\)列,因此
\[
\boldsymbol{A}^{2}=(0,\boldsymbol{A}\mathbf{e}_{1},\boldsymbol{A}\mathbf{e}_{2},\cdots,\boldsymbol{A}\mathbf{e}_{n - 1})=(0,0,\mathbf{e}_{1},\cdots,\mathbf{e}_{n - 2})
\]
不断这样做下去就可得到结论.
\end{proof}

\begin{example}\label{example:56471456}
设\(A\)是\(n\)阶矩阵,\(A\)适合\(A^n = O\)时,\(I_n - A\)必是可逆矩阵.
\end{example}
\begin{proof}
注意到
\begin{align*}
I_n=I_n-A^n=\left( I_n-A \right) \left( I_n+A+A^2+\cdots +A^{n-1} \right) .
\end{align*}
故此时\(I_n - A\)必是可逆矩阵.
\end{proof}

\begin{example}
设\(A\)是\(n\)阶矩阵,\(A\)适合\(AB = B(I_n - A)\)对任意\(n\)阶矩阵\(B\)成立,那么\(B = O\).
\end{example}
\begin{note}
若已知矩阵乘法的相关等式,可以尝试得到一些递推等式.
\end{note}
\begin{proof}
假设\(A^k = O\),其中\(k\)为某个正整数. 由条件可得\(AB = B(I_n - A)\),于是\(O = A^kB = B(I_n - A)^k\). 由\hyperref[example:56471456]{上一题}知\(I_n - A\)是可逆矩阵,从而\(B = O\).
\end{proof}


\begin{proposition}[多项式的友矩和Frobenius块]\label{proposition:多项式的友矩和Frobenius块}
设首一多项式\(f(x)=x^{n}+a_{1}x^{n - 1}+\cdots+a_{n - 1}x + a_{n}\),\(f(x)\)的友阵
\[
\boldsymbol{C}(f(x))=\left(\begin{array}{cccccc}
0 & 0 & \cdots & 0 & 0 & -a_{n} \\
1 & 0 & \cdots & 0 & 0 & -a_{n - 1} \\
0 & 1 & \cdots & 0 & 0 & -a_{n - 2} \\
\vdots & \vdots & \ddots & \vdots & \vdots & \vdots \\
0 & 0 & \cdots & 1 & 0 & -a_{2} \\
0 & 0 & \cdots & 0 & 1 & -a_{1}
\end{array}\right),
\]

则
\(\vert x\boldsymbol{I}_{n}-\boldsymbol{C}(f(x))\vert=f(x)\).

\(\boldsymbol{C}(f(x))\)的转置\(F(f(x))\)称为\(f(x)\)的Frobenius块.即
\begin{align*}
\boldsymbol{C}^T(f(x))=\left( \begin{matrix}
0&		1&		0&		\cdots&		0&		0\\
0&		0&		1&		\cdots&		0&		0\\
\vdots&		\vdots&		\vdots&		\ddots&		\vdots&		\vdots\\
0&		0&		0&		\cdots&		1&		0\\
0&		0&		0&		\cdots&		0&		1\\
-a_n&		-a_{n-1}&		-a_{n-2}&		\cdots&		-a_2&		-a_1\\
\end{matrix} \right).
\end{align*}
并且容易验证$\boldsymbol{C}(f(x))$具有以下性质,其中$\boldsymbol{e}_i$是标准单位列向量($i=1,2,\cdots,n$):
\begin{align*}
C(f(x))\boldsymbol{e}_i = \boldsymbol{e}_{i + 1} \ (1 \leq i \leq n - 1), \ C(f(x))\boldsymbol{e}_n = - \sum_{i = 1}^{n} a_{n - i + 1} \boldsymbol{e}_i .
\end{align*}
\end{proposition}
\begin{proof}
\(\vert x\boldsymbol{I}_{n}-\boldsymbol{C}(f(x))\vert=f(x)\)的证明见\hyperref[pro:友矩阵的特征多项式/行列式]{友矩阵的特征多项式/行列式}.
\end{proof}

\begin{example}
求下列矩阵的逆矩阵\((a_n\neq0)\):
\[
F = 
\begin{pmatrix}
0 & 0 & \cdots & 0 & -a_n\\
1 & 0 & \cdots & 0 & -a_{n - 1}\\
0 & 1 & \cdots & 0 & -a_{n - 2}\\
\vdots & \vdots & & \vdots & \vdots\\
0 & 0 & \cdots & 1 & -a_1
\end{pmatrix}.
\]
\end{example}
\begin{solution}
用初等变换法不难求得
\[
F^{-1} = 
\begin{pmatrix}
-\frac{a_{n - 1}}{a_n} & 1 & 0 & \cdots & 0\\
-\frac{a_{n - 2}}{a_n} & 0 & 1 & \cdots & 0\\
-\frac{a_{n - 3}}{a_n} & 0 & 0 & \cdots & 0\\
\vdots & \vdots & \vdots & & \vdots\\
-\frac{1}{a_n} & 0 & 0 & \cdots & 0
\end{pmatrix}.
\]
\end{solution}


\begin{proposition}\label{proposition:与对角矩阵可交换的矩阵必是对角阵}
和所有$n$阶对角矩阵乘法可交换的矩阵必是对角矩阵.
\end{proposition}
\begin{proof}
由矩阵乘法易得.
\end{proof}

\begin{proposition}[纯量矩阵的刻画]\label{proposition:纯量阵的刻画}
\begin{enumerate}[(1)]
\item 和所有\(n\)阶奇异阵乘法可交换的矩阵必是纯量阵\(kI_{n}\).

\item 和所有\(n\)阶非奇异阵乘法可交换的矩阵必是纯量阵\(kI_{n}\).

\item 和所有\(n\)阶正交阵乘法可交换的矩阵必是纯量阵\(kI_{n}\).

\item 和所有\(n\)阶矩阵乘法可交换的矩阵必是纯量阵\(kI_{n}\).
\end{enumerate}

\end{proposition}
\begin{proof}
首先设$A=(a_{ij})_{n\times n}$.
\begin{enumerate}
\item 设\(E_{ij}(1\leq i\neq j\leq n)\)为基础矩阵,因为基础矩阵都是奇异阵,所以由条件可知\(E_{ij}A = AE_{ij}\).注意到\(E_{ij}A\)是将\(A\)的第\(j\)行变为第\(i\)行而其他行都是零的\(n\)阶矩阵,\(AE_{ij}\)是将\(A\)的第\(i\)列变为第\(j\)列而其他列都是零的\(n\)阶矩阵,于是我们有
\begin{align*}
\bordermatrix{%
&    &		&		&		j&		&
\cr
&    &		&		&		&		&		\cr
&   &		&		&		&		&		\cr
i&    a_{j1}&		a_{j2}&		\cdots&		a_{jj}&		\cdots&		a_{jn}
\cr
&    &		&		&		&		&		\cr
&    &		&		&		&		&		\cr
} \quad= \quad \bordermatrix{%
&    &       &             j&     &
\cr
&    &		&		a_{1i}&		&		\cr
&    &		&		a_{2i}&		&		\cr
&    &		&		\vdots&		&		\cr
i&    &		&		a_{ii}&		&		\cr
&    &		&		\vdots&		&		\cr
&    &		&		a_{ni}&		&		\cr
}.
\end{align*}
从而比较上述等式两边矩阵的每个元素可得\(a_{ij}=0(i\neq j)\),\(a_{ii}=a_{jj}(1\leq i\neq j\leq n)\),因此\(A\)是纯量阵.

\item 设\(D=\text{diag}\{1,2,\cdots,n\}\)为对角阵,因为$D$为非奇异阵,所以由条件可知\(AD = DA\).进而
\begin{gather*}
AD=DA
\\
\Leftrightarrow A\left( \boldsymbol{e}_1,2\boldsymbol{e}_2,\cdots ,n\boldsymbol{e}_n \right) =\left( \boldsymbol{e}_1,2\boldsymbol{e}_2,\cdots ,n\boldsymbol{e}_n \right) A
\\
\Leftrightarrow \left( A\boldsymbol{e}_1,2A\boldsymbol{e}_2,\cdots ,nA\boldsymbol{e}_n \right) =\left( \boldsymbol{e}_1A,2\boldsymbol{e}_2A,\cdots ,n\boldsymbol{e}_nA \right) 
\\
\Leftrightarrow \left( \begin{matrix}
a_{11}&		2a_{12}&		\cdots&		na_{1n}\\
a_{21}&		2a_{22}&		\cdots&		na_{2n}\\
\vdots&		\vdots&		\ddots&		\vdots\\
a_{n1}&		2a_{n2}&		\cdots&		na_{nn}\\
\end{matrix} \right) =\left( \begin{matrix}
a_{11}&		a_{12}&		\cdots&		a_{1n}\\
2a_{21}&		2a_{22}&		\cdots&		2a_{2n}\\
\vdots&		\vdots&		\ddots&		\vdots\\
na_{n1}&		na_{n2}&		\cdots&		na_{nn}\\
\end{matrix} \right).
\end{gather*}
比较上述等式两边矩阵的每个元素可得$ja_{ij}=ia_{ij}\left( i\ne j \right)$,从而$\left( i-j \right) a_{ij}=0\left( i\ne j \right)$,于是$a_{ij}=0\left( i\ne j \right) $.故\(A=\text{diag}\{a_{11},a_{22},\cdots,a_{nn}\}\)也为对角阵.

设\(P_{ij}(1\leq i\neq j\leq n)\)为第一类初等阵,因为第一类初等阵均为非奇异阵,所以由条件可知\(AP_{ij}=P_{ij}A\).进而可得
\begin{align*}
\bordermatrix{%
&    &		&		i&		&		j&		&		\cr
&    a_{11}&		&		&		&		&		&		\cr
&    &		\ddots&		&		&		&		&		\cr
i&    &		&		0&		\cdots&		a_{jj}&		&		\cr
&   &		&		\vdots&		\ddots&		\vdots&		&		\cr
j&    &		&		a_{ii}&		\cdots&		0&		&		\cr
&    &		&		&		&		&		\ddots&		\cr
&    &		&		&		&		&		&		a_{nn}\cr
} \quad =\quad \bordermatrix{%
&    &		&		i&		&		j&		&		\cr
&    a_{11}&		&		&		&		&		&		\cr
&    &		\ddots&		&		&		&		&		\cr
i&    &		&		0&		\cdots&		a_{ii}&		&		\cr
&    &		&		\vdots&		\ddots&		\vdots&		&		\cr
j&   &		&		a_{jj}&		\cdots&		0&		&		\cr
&    &		&		&		&		&		\ddots&		\cr
&    &		&		&		&		&		&		a_{nn}\cr
} .
\end{align*}
从而比较上述等式两边矩阵的每个元素可得\(a_{ii}=a_{jj}(1\leq i\neq j\leq n)\),于是\(A\)为纯量阵.

\item 设第二类初等阵\(P_{i}(-1)(1\leq i\leq n)\),因为$P_{i}(-1)(1\leq i\leq n)$都是正交阵,所以由条件可知$P_{i}(-1)A=AP_{i}(-1)$.进而可得
\begin{align*}
\left( \begin{matrix}
a_{11}&		a_{12}&		\cdots&		a_{1i}&		\cdots&		a_{1n}\\
\vdots&		\vdots&		&		\vdots&		&		\vdots\\
-a_{i1}&		-a_{i2}&		\cdots&		-a_{ii}&		\cdots&		-a_{in}\\
\vdots&		\vdots&		&		\vdots&		&		\vdots\\
a_{n1}&		a_{n2}&		\cdots&		a_{ni}&		\cdots&		a_{nn}\\
\end{matrix} \right) =\left( \begin{matrix}
a_{11}&		\cdots&		-a_{1i}&		\cdots&		a_{1n}\\
a_{21}&		\cdots&		-a_{2i}&		\cdots&		a_{2n}\\
\vdots&		&		\vdots&		&		\vdots\\
a_{i1}&		\cdots&		-a_{ii}&		\cdots&		a_{in}\\
\vdots&		&		\vdots&		&		\vdots\\
a_{n1}&		\cdots&		-a_{ni}&		\cdots&		a_{nn}\\
\end{matrix} \right) .
\end{align*}
比较上述等式两边矩阵的每个元素可得$a_{ij}=-a_{ij}\left( i\ne j \right) $,从而$a_{ij}=0\left( i\ne j \right) $.于是\(A=\text{diag}\{a_{11},a_{22},\cdots,a_{nn}\}\)为对角阵.

设\(P_{ij}(1\leq i\neq j\leq n)\)为第一类初等阵,因为第一类初等阵均为正交阵,所以由条件可知\(AP_{ij}=P_{ij}A\).进而可得
\begin{align*}
\bordermatrix{%
&    &		&		i&		&		j&		&		\cr
&    a_{11}&		&		&		&		&		&		\cr
&    &		\ddots&		&		&		&		&		\cr
i&    &		&		0&		\cdots&		a_{jj}&		&		\cr
&   &		&		\vdots&		\ddots&		\vdots&		&		\cr
j&    &		&		a_{ii}&		\cdots&		0&		&		\cr
&    &		&		&		&		&		\ddots&		\cr
&    &		&		&		&		&		&		a_{nn}\cr
} \quad =\quad \bordermatrix{%
&    &		&		i&		&		j&		&		\cr
&    a_{11}&		&		&		&		&		&		\cr
&    &		\ddots&		&		&		&		&		\cr
i&    &		&		0&		\cdots&		a_{ii}&		&		\cr
&    &		&		\vdots&		\ddots&		\vdots&		&		\cr
j&   &		&		a_{jj}&		\cdots&		0&		&		\cr
&    &		&		&		&		&		\ddots&		\cr
&    &		&		&		&		&		&		a_{nn}\cr
} .
\end{align*}
从而比较上述等式两边矩阵的每个元素可得\(a_{ii}=a_{jj}(1\leq i\neq j\leq n)\),于是\(A\)为纯量阵.

\item 可以由上面(1)(2)(3)中任意一个证明得到.注意如果此时用(3)的证明方法,那么我们可以先考虑$A$与第一类初等矩阵\(P_{i}(c)(c\ne 1,1\leq i\leq n)\)的乘法交换性.而不是像(3)中只能考虑\(P_{i}(-1)(1\leq i\leq n)\).
\end{enumerate}
\end{proof}

\begin{proposition}[零矩阵的充要条件]\label{proposition:零矩阵的充要条件}
\begin{enumerate}
\item \(m\times n\)实矩阵\(A = O\)的充要条件是适合条件\(AA' = O\)或\(\mathrm{tr}(AA')\geq0\),等号成立;

\item \(m\times n\)复矩阵\(A = O\)的充要条件是适合条件\(A\overline{A}'= O\)或\(\mathrm{tr}(A\overline{A}')\geq0\),等号成立.
\end{enumerate}
\end{proposition}
\begin{proof}
\begin{enumerate}
\item (1) 设\(A=(a_{ij})_{m\times n}\),则\(AA'\)的第\((i,i)\)元素等于零,即
\begin{align*}
a_{i1}^{2}+a_{i2}^{2}+\cdots +a_{in}^{2}=0,i=1,2,\cdots,m.
\end{align*}
又因为\(a_{ij}\)都是实数,所以必有\(a_{ij}=0,i=1,2,\cdots,m,j=1,2,\cdots,n\).故$A=O$.

(2)设\(A=(a_{ij})\)为\(m\times n\)实矩阵,则通过计算可得
\[
\mathrm{tr}(AA')=\sum_{i = 1}^{m}\sum_{j = 1}^{n}a_{ij}^2\geq0,
\]
等号成立当且仅当\(a_{ij}=0(1\leq i\leq m,1\leq j\leq n)\),即\(A = O\).

\item (1)设\(A=(a_{ij})_{m\times n}\),则\(A\overline{A'}\)的第\((i,i)\)元素等于零,即
\begin{align*}
\left| a_{i1} \right|^2+\left| a_{i2} \right|^2+\cdots +\left| a_{in} \right|^2=0,i=1,2,\cdots ,m.
\end{align*}
又因为\(a_{ij}\)都是复数,所以可设$a_{ij}=b_{ij}+\mathrm{i}c_{ij}$,其中$b_{ij},c_{ij}\in \mathbb{R},i=1,2,\cdots,m,j=1,2,\cdots,n$.于是
\begin{align*}
b_{i1}^{2}+c_{i1}^{2}+b_{i2}^{2}+c_{i2}^{2}+\cdots +b_{in}^{2}+c_{in}^{2}=0,i=1,2,\cdots ,m.
\end{align*}
再结合$b_{ij},c_{ij}\in \mathbb{R}$,可知\(b_{ij}=c_{ij}=0\).即\(a_{ij}=0,i=1,2,\cdots,m,j=1,2,\cdots,n\).故$A=O$.

(2)设\(A=(a_{ij})\)为\(m\times n\)复矩阵,则通过计算可得
\[
\mathrm{tr}(A\overline{A}')=\sum_{i = 1}^{m}\sum_{j = 1}^{n}|a_{ij}|^2\geq0,
\]
等号成立当且仅当\(a_{ij}=0(1\leq i\leq m,1\leq j\leq n)\),即\(A = O\).
\end{enumerate}



\end{proof}

\begin{proposition}[对称阵是零矩阵的充要条件]\label{proposition:对称阵是零矩阵的充要条件}
设\(\boldsymbol{A}\)为\(n\)阶对称阵,则\(\boldsymbol{A}\)是零矩阵的充要条件是对任意的\(n\)维列向量\(\boldsymbol{\alpha}\),有
\[
\boldsymbol{\alpha'} \boldsymbol{A} \boldsymbol{\alpha} = 0.
\]
\end{proposition}
\begin{proof}
只要证明充分性.设\(\boldsymbol{A}=(a_{ij})\),令\(\alpha = \boldsymbol{e}_{i}\),是第\(i\)个标准单位列向量.因为\(\boldsymbol{e}_{i}'\boldsymbol{A}\boldsymbol{e}_{i}\)是\(\boldsymbol{A}\)的第\((i, i)\)元素,故\(a_{ii}=0\).又令\(\boldsymbol{\alpha}=\boldsymbol{e}_{i}+\boldsymbol{e}_{j}(i \neq j)\),则
\[
0 = (\boldsymbol{e}_{i}+\boldsymbol{e}_{j})'\boldsymbol{A}(\boldsymbol{e}_{i}+\boldsymbol{e}_{j})=a_{ii}+a_{jj}+a_{ij}+a_{ji}.
\]
由于\(\boldsymbol{A}\)是对称阵,故\(a_{ij}=a_{ji}\),又上面已经证明\(a_{ii}=a_{jj}=0\),从而\(a_{ij}=0\),这就证明了\(\boldsymbol{A} = \boldsymbol{O}\).
\end{proof}

\begin{proposition}[反对称阵的刻画]\label{proposition:反对称阵的刻画}
设\(\boldsymbol{A}\)为\(n\)阶方阵,则\(\boldsymbol{A}\)是反称阵的充要条件是对任意的\(n\)维列向量\(\boldsymbol{\alpha}\),有
\[
\boldsymbol{\alpha'} \boldsymbol{A} \boldsymbol{\alpha} = 0.
\]
\end{proposition}
\begin{proof}
必要性($\Rightarrow $):若\(\boldsymbol{A}\)是反称阵,则对任意的\(n\)维列向量\(\boldsymbol{\alpha}\),有\((\boldsymbol{\alpha' A \alpha})' = - \boldsymbol{\alpha' A \alpha}\).而\(\boldsymbol{\alpha' A \alpha}\)是数,因此\((\boldsymbol{\alpha' A \alpha})' = \boldsymbol{\alpha' A \alpha}\).比较上面两个式子便有\(\boldsymbol{\alpha' A \alpha} = 0\).

充分性($\Leftarrow $):若上式对任意的\(n\)维列向量\(\boldsymbol{\alpha}\)成立,则由\(\boldsymbol{\alpha' A \alpha}\)是数,可知$\boldsymbol{\alpha }'\boldsymbol{A\alpha }=\left( \boldsymbol{\alpha }'\boldsymbol{A\alpha } \right) '=\boldsymbol{\alpha }'\boldsymbol{A}'\boldsymbol{\alpha }=0$,故\(\boldsymbol{\alpha'}(\boldsymbol{A} + \boldsymbol{A'})\boldsymbol{\alpha} = 0\).因为矩阵\(\boldsymbol{A} + \boldsymbol{A'}\)是对称阵,故由\hyperref[proposition:对称阵是零矩阵的充要条件]{对称阵是零矩阵的充要条件}可得\(\boldsymbol{A} + \boldsymbol{A'} = \boldsymbol{O}\),即\(\boldsymbol{A'} = - \boldsymbol{A}\),\(\boldsymbol{A}\)是反称阵.
\end{proof}

\begin{proposition}\label{proposition:任一阶方阵可表示为对称阵与反对称阵之和}
任一\(n\)阶方阵均可表示为一个对称阵与一个反对称阵之和.
\end{proposition}
\begin{note}
构造思路:设$A=B+C$,且$B$为对称矩阵,$C$为反称矩阵.则两边取转置可得
\begin{align*}
\begin{cases}
A=B+C\\
A'=\left( B+C \right)'=B-C\\
\end{cases}
\end{align*}
解得:$B=\frac{1}{2}(\boldsymbol{A}+\boldsymbol{A}'),C=\frac{1}{2}(\boldsymbol{A}-\boldsymbol{A}')$.
\end{note}
\begin{proof}
设\(\boldsymbol{A}\)是\(n\)阶方阵,则\(\boldsymbol{A}+\boldsymbol{A}'\)是对称阵,\(\boldsymbol{A}-\boldsymbol{A}'\)是反对称阵,并且
\[
\boldsymbol{A}=\frac{1}{2}(\boldsymbol{A}+\boldsymbol{A}')+\frac{1}{2}(\boldsymbol{A}-\boldsymbol{A}').
\]
\end{proof}
\begin{remark}
上例中的\(\frac{1}{2}(\boldsymbol{A}+\boldsymbol{A}')\)称为\(\boldsymbol{A}\)的对称化,\(\frac{1}{2}(\boldsymbol{A}-\boldsymbol{A}')\)称为\(\boldsymbol{A}\)的反对称化.
\end{remark}

\begin{proposition}[上三角阵性质]\label{proposition:上三角阵性质}
(1) 设\(\boldsymbol{A}\)是\(n\)阶上三角阵且主对角线上元素全为零,则\(\boldsymbol{A}^{n}=O\).

\hypertarget{proposition:上三角阵的性质第2条性质}{(2)}
设\(\boldsymbol{A}\)是\(n(n\geq2)\)阶上三角阵,若\(i < j\),则\(A_{ij}=M_{ij}=0\).

(3)上(下)三角阵的加减、数乘、乘积(幂)、多项式、伴随和求逆仍然是上(下)三角阵,并且所得上(下)三角阵的主对角元是原上(下)三角阵对应主对角元的加减、数乘、乘积(幂)、多项式、伴随和求逆.
\end{proposition}
\begin{proof}
(1) {\color{blue}证法一(抽屉原理):}
设\(A=(a_{ij})\),当\(i\geq j\)时,\(a_{ij} = 0\).将\(A\)表示为基础矩阵\(E_{ij}\)之和:
\[
A=\sum_{i>j}a_{ij}E_{ij}
\]
因为当\(j\neq k\)时,\(E_{ij}E_{kl}=\boldsymbol{O}\),故在\(A^{n}\)的乘法展开式中,可能非零的项只能是具有形式\(E_{i_{1}j_{1}}E_{i_{2}j_{2}}\cdots E_{i_{n - 1}j_{n - 1}}\),但足标必须满足条件\(1\leq i_{1}<j_{1}<i_{2}<j_{2}<\cdots <j_{n - 1}\leq n\).根据\hyperlink{抽屉原理}可知,这样的项也不存在,因此\(A^{n}=\boldsymbol{O}\).

{\color{blue}证法二(数学归纳法):} 
由假设\(Ae_{i}=a_{i1}e_{1}+\cdots +a_{i, i - 1}e_{i - 1}(1\leq i\leq n)\),我们只要用归纳法证明:\(A^{k}e_{k}=0\)对任意的\(1\leq k\leq n\)都成立,则\(A^{n}e_{i}=A^{n-i}\cdot A^ie_i=A^{n-i}\cdot 0=0\)对任意的\(1\leq i\leq n\)都成立,从而由\hyperlink{proposition:矩阵相等的判定法则}{判定法则}可知\(A^{n}=O\)成立.显然,\(Ae_{1}=0\)成立.假设\(A^{k}e_{k}=0\)对任意的\(1 < k < n\)都成立,则
\[
A^{k}e_{k}=A^{k - 1}(Ae_{k})=A^{k - 1}(a_{k1}e_{1}+\cdots +a_{k, k - 1}e_{k - 1})
\]
\[
=a_{k1}A^{k - 1}e_{1}+\cdots +a_{k, k - 1}A^{k - 1}e_{k - 1}=0.
\]

(2) 根据条件可设$\left| \boldsymbol{A} \right|=\left| \begin{matrix}
a_{11}&		a_{12}&		\cdots&		a_{1n}\\
0&		a_{22}&		\cdots&		a_{2n}\\
\vdots&		\vdots&		&		\vdots\\
0&		0&		\cdots&		a_{nn}\\
\end{matrix} \right|$,则当$i<j$时,有
\begin{align*}
& M_{ij}=\left| \begin{matrix}
a_{11}&		a_{12}&		\cdots&		a_{1i}&		a_{1,i+1}&		\cdots&		\MyTikzmark{topA}{a_{1j}}&		\cdots&		a_{1n}\\
0&		a_{22}&		\cdots&		a_{2i}&		a_{2,i+1}&		\cdots&		a_{2j}&		\cdots&		a_{2n}\\
\vdots&		\vdots&		&		\vdots&		\vdots&		&		\vdots&		&		\vdots\\
\MyTikzmark{leftA}{0}&		0&		\cdots&		a_{ii}&		a_{i,i+1}&		\cdots&		a_{ij}&		\cdots&		\MyTikzmark{rightA}{a_{in}}\\
0&		0&		\cdots&		0&		a_{i+1,i+1}&		\cdots&		a_{i+1,j}&		\cdots&		a_{i+1,n}\\
\vdots&		\vdots&		&		\vdots&		\vdots&		&		\vdots&		&		\vdots\\
\vdots&		\vdots&		&		\vdots&		\vdots&		&		\vdots&		&		\vdots\\
0&		0&		\cdots&		0&		0&		\cdots&		\MyTikzmark{bottomA}{0}&		\cdots&		a_{nn}\\
\end{matrix} \right|
\text{\DrawVLine[red, thick, opacity=0.5]{topA}{bottomA}
\DrawHLine[red, thick, opacity=0.5]{leftA}{rightA}}
\\
&=\left| \begin{matrix}
a_{11}&		a_{12}&		\cdots&		a_{1i}&		a_{1,i+1}&		\cdots&		a_{1n}\\
0&		a_{22}&		\cdots&		a_{2i}&		a_{2,i+1}&		\cdots&		a_{2n}\\
\vdots&		\vdots&		&		\vdots&		\vdots&		&		\vdots\\
0&		0&		\cdots&		0&		a_{i+1,i+1}&		\cdots&		a_{i+1,n}\\
\vdots&		\vdots&		&		\vdots&		\vdots&		&		\vdots\\
\vdots&		\vdots&		&		\vdots&		\vdots&		&		\vdots\\
0&		0&		\cdots&		0&		0&		\cdots&		a_{nn}\\
\end{matrix} \right|=\left| \begin{matrix}
a_{11}&		a_{12}&		\cdots&		a_{1i}&		\cdots&		a_{1n}\\
&		a_{22}&		\cdots&		a_{2i}&		\cdots&		a_{2n}\\
&		&		\ddots&		\vdots&		&		\vdots\\
&		&		&		0&		\cdots&		a_{i+1,n}\\
&		&		&		&		\ddots&		\vdots\\
&		&		&		&		&		a_{nn}\\
\end{matrix} \right|=0.
\end{align*}
故\(A_{ij}=M_{ij}=0\).

(3)只证上三角阵的情形,下三角阵的情形完全类似.上三角阵的加减、数乘、乘积(幂)以及多项式结论的证明是显然的.下面我们来证明伴随和求逆的结论.设$\boldsymbol{A}=(a_{ij})$为$n$阶上三角阵,即满足$a_{ij}=0,(\forall i>j)$.由\hyperlink{proposition:上三角阵的性质第2条性质}{(2)}可知$\boldsymbol{A}$的代数余子式$A_{ij}=0,\forall i<j$.于是
\begin{align*}
\boldsymbol{A}^*=\left( \begin{matrix}
A_{11}&		A_{21}&		\cdots&		A_{n1}\\
0&		A_{22}&		\cdots&		A_{n2}\\
\vdots&		\vdots&		\ddots&		\vdots\\
0&		0&		\cdots&		A_{nn}\\
\end{matrix} \right) .
\end{align*}
故$\boldsymbol{A}^*$也是上三角阵.而对$\forall i\in [1,n]\cap N$,有

我们又\textbf{将$A_{ii}=a_{11}\cdots \widehat{a_{ii}}\cdots a_{nn}$这个数称为$a_{ii}$的伴随}.这就完成了$\boldsymbol{A}^*$结论的证明.
\begin{align*}
&A_{ii}=\left( -1 \right) ^{2i}M_{ii}=M_{ii}=\left| \begin{matrix}
a_{11}&		a_{12}&		\cdots&		a_{1,i-1}&		\MyTikzmark{topB}{a_{1i}}&		a_{1,i+1}&		\cdots&		a_{1n}\\
0&		a_{22}&		\cdots&		a_{2,i-1}&		a_{2i}&		a_{2,i+1}&		\cdots&		a_{2n}\\
\vdots&		\vdots&		\ddots&		\vdots&		\vdots&		\vdots&		&		\vdots\\
0&		0&		\cdots&		a_{i-1,i-1}&		a_{i-1,i}&		a_{i-1,i+1}&		\cdots&		a_{i-1,n}\\
\MyTikzmark{leftB}{0}&		0&		\cdots&		0&		a_{ii}&		a_{i,i+1}&		\cdots&		\MyTikzmark{rightB}{a_{in}}\\
0&		0&		\cdots&		0&		0&		a_{i+1,i+1}&		\cdots&		a_{i+1,n}\\
\vdots&		\vdots&		&		\vdots&		\vdots&		\vdots&		\ddots&		\vdots\\
0&		0&		\cdots&		0&		\MyTikzmark{bottomB}{0}&		0&		\cdots&		a_{nn}\\
\end{matrix} \right|\
\intertext{\DrawVLine[red, thick, opacity=0.5]{topB}{bottomB}
\DrawHLine[red, thick, opacity=0.5]{leftB}{rightB}}
\\
&=\left| \begin{matrix}
a_{11}&		a_{12}&		\cdots&		a_{1,i-1}&		a_{1,i+1}&		\cdots&		a_{1n}\\
0&		a_{22}&		\cdots&		a_{2,i-1}&		a_{2,i+1}&		\cdots&		a_{2n}\\
\vdots&		\vdots&		\ddots&		\vdots&		\vdots&		&		\vdots\\
0&		0&		\cdots&		a_{i-1,i-1}&		a_{i-1,i+1}&		\cdots&		a_{i-1,n}\\
0&		0&		\cdots&		0&		a_{i+1,i+1}&		\cdots&		a_{i+1,n}\\
\vdots&		\vdots&		&		\vdots&		\vdots&		\ddots&		\vdots\\
0&		0&		\cdots&		0&		0&		\cdots&		a_{nn}\\
\end{matrix} \right|=a_{11}\cdots \widehat{a_{ii}}\cdots a_{nn}.
\end{align*}

由于当$\left| \boldsymbol{A} \right|\ne 0$时,我们有$\boldsymbol{A}^{-1}=\frac{1}{\left| \boldsymbol{A} \right|}\boldsymbol{A}^*$,故由上三角阵的数乘结论可知,$\boldsymbol{A}^{-1}$也是上三角阵,其主对角元为$\frac{1}{\left| \boldsymbol{A} \right|}A_{ii}=a_{ii}^{-1}$.结论得证.
\end{proof}

\begin{proposition}
若\(\boldsymbol{A}\),\(\boldsymbol{B}\)都是由非负实数组成的矩阵且\(AB\)有一行等于零,则或者\(\boldsymbol{A}\)有一行为零,或者\(\boldsymbol{B}\)有一行为零.
\end{proposition}
\begin{proof}
设\(\boldsymbol{A} = (a_{ij})_{n\times m}\),\(\boldsymbol{B} = (b_{ij})_{m\times s}\).假设\(\boldsymbol{C} = \boldsymbol{AB}\),\(\boldsymbol{C} = (c_{ij})_{n\times s}\)的第\(i\)行全为零.
则对\(\forall j\in [1,s]\cap N\),都有
\begin{align*}
c_{ij} = a_{i1}b_{1j} + a_{i2}b_{2j} + \cdots + a_{im}b_{nj} = 0 .
\end{align*}
已知对\(\forall i\in [1,n]\cap N\),\(j\in [1,m]\cap N\),有\(a_{ij} \geq 0\);对\(\forall i\in [1,m]\cap N\),\(j\in [1,s]\cap N\),有\(b_{ij} \geq 0\).从而
\begin{align*}
a_{i1}b_{1j} = a_{i2}b_{2j} = \cdots = a_{im}b_{nj} = 0,\forall j\in [1,s]\cap N.
\end{align*}
若\(\boldsymbol{A}\)的第\(i\)行不全为零,不妨设\(a_{ik} \neq 0\),\(k\in [1,m]\cap N\),则由\(a_{ik}b_{kj} = 0\),\(\forall j\in [1,s]\cap N\)可得\(b_{kj} = 0\),对\(\forall j\in [1,s]\cap N\)都成立,即\(\boldsymbol{B}\)的第\(k\)行全为零. 
\end{proof}

\begin{proposition}[矩阵行和和列和的一种刻画]\label{proposition:对矩阵行和和列和的一种刻画}
\begin{enumerate}[(1)]
\item $n$阶矩阵$A$第$i$行元素之和为$a_i(i=1,2,\cdots,n)$当且仅当
\begin{align*}
A\left( \begin{array}{c}
1\\
1\\
\vdots\\
1\\
\end{array} \right) =\left( \begin{array}{c}
a_1\\
a_2\\
\vdots\\
a_n\\
\end{array} \right)
.
\end{align*}
特别地,$n$阶矩阵$A$的每一行元素之和等于$c$当且仅当$A\alpha=c\cdot\alpha$,其中$\alpha=(1,1,\cdots,1)'$.
\item $n$阶矩阵$A$第$i$列元素之和为$a_i(i=1,2,\cdots,n)$当且仅当
\begin{align*}
\left( \begin{matrix}
1&		1&		\cdots&		1\\
\end{matrix} \right) A=\left( \begin{matrix}
a_1&		a_2&		\cdots&		a_n\\
\end{matrix} \right). 
\end{align*}
特别地,$n$阶矩阵$A$的每一列元素之和等于$c$当且仅当$\alpha A=c\cdot\alpha$,其中$\alpha=(1,1,\cdots,1)$.
\end{enumerate}
\end{proposition}
\begin{proof}
由矩阵乘法容易得到证明.
\end{proof}

\begin{example}
设\(n\)阶方阵\(\boldsymbol{A}\)的每一行元素之和等于常数\(c\),求证:

(1) 对任意的正整数\(k\),\(\boldsymbol{A}^{k}\)的每一行元素之和等于\(c^{k}\);

(2) 若\(\boldsymbol{A}\)为可逆阵,则\(c \neq 0\)并且\(\boldsymbol{A}^{-1}\)的每一行元素之和等于\(c^{-1}\).
\end{example}
\begin{note}
核心想法是利用\hyperref[proposition:对矩阵行和和列和的一种刻画]{命题\ref{proposition:对矩阵行和和列和的一种刻画}}.
\end{note}
\begin{proof}
设\(\boldsymbol{\alpha} = (1,1,\cdots,1)'\),则由矩阵乘法可知,\(\boldsymbol{A}\)的每一行元素之和等于\(c\)当且仅当\(\boldsymbol{A}\boldsymbol{\alpha} = c \cdot \boldsymbol{\alpha}\)成立.

(1) 由\(\boldsymbol{A}\boldsymbol{\alpha} = c \cdot \boldsymbol{\alpha}\)不断递推可得\(\boldsymbol{A}^{k}\boldsymbol{\alpha} = c^{k} \cdot \boldsymbol{\alpha}\),故结论成立.

(2) 若\(c = 0\),则由\(\boldsymbol{A}\)可逆以及\(\boldsymbol{A}\boldsymbol{\alpha} = \boldsymbol{0}\)可得\(\boldsymbol{\alpha} = \boldsymbol{0}\),矛盾.在\(\boldsymbol{A}\boldsymbol{\alpha} = c \cdot \boldsymbol{\alpha}\)的两边同时左乘\(c^{-1}\boldsymbol{A}^{-1}\),可得\(\boldsymbol{A}^{-1}\boldsymbol{\alpha} = c^{-1} \cdot \boldsymbol{\alpha}\),由此即得结论.
\end{proof}

\begin{proposition}[矩阵可逆的等价命题]\label{proposition:矩阵可逆的等价命题}
(1)\(n\)阶方阵\(\boldsymbol{A}\)可逆.

(2)存在矩阵$\boldsymbol{B}$,使得$\boldsymbol{AB}=\boldsymbol{BA}=\boldsymbol{I}_n$(这个等式同时也说明$\boldsymbol{B}$可逆).

(3)\(\boldsymbol{A}\)的行列式\(\vert \boldsymbol{A}\vert\neq0\).

(4)\(\boldsymbol{A}\)等价(相抵)于\(n\)阶单位矩阵.

(5)\(\boldsymbol{A}\)可以表示为有限个初等矩阵的积.

(6)\(\boldsymbol{A}\)的\(n\)个行向量(列向量)线性无关.
\end{proposition}

\begin{proposition}\label{proposition:由矩阵的零化多项式构造其逆矩阵}
(1)若已知\(\lambda_1\boldsymbol{A}^2+\lambda_2\boldsymbol{A}+\lambda_3\boldsymbol{I}_n=\boldsymbol{O}\),其中\(\lambda_1,\lambda_2,\lambda_3\in\mathbb{R}\),$\lambda_1\ne 0$,并且$\lambda_1x^2+\lambda_2x+\lambda_3=0$无实根(即原等式左边不可因式分解成$\left( a_1\boldsymbol{I}_n+a_2\boldsymbol{A} \right) \left( b_1\boldsymbol{I}_n+b_2\boldsymbol{A} \right)$),
则对任何$c,d\in\mathbb{R}$,都有\(c\boldsymbol{A}+d\boldsymbol{I}_n\)可逆.

(2)若已知\(\lambda_1\boldsymbol{A}^2+\lambda_2\boldsymbol{A}+\lambda_3\boldsymbol{I}_n=(a_1\boldsymbol{A}+b_1\boldsymbol{I}_n)(a_2\boldsymbol{A}+b_2\boldsymbol{I}_n)=\boldsymbol{O}\),其中\(a_1,a_2,b_1,b_2\in\mathbb{R}\),且\(\lambda_1=a_1a_2,\lambda_2=a_1b_2+a_2b_1,\lambda_3=b_1b_2,a_1\ne 0,a_2\ne 0\).
则对任何实数对\(\left( c,d \right) \ne \left( a_1,b_1 \right) ,\left( a_2,b_2 \right)\),都有\(c\boldsymbol{A}+d\boldsymbol{I}_n\)可逆.
\end{proposition}
\begin{note}
\textbf{构造逆矩阵的方法:}不妨设\(k(c\boldsymbol{A}+d\boldsymbol{I}_n)^{-1}=(p\boldsymbol{A}+q\boldsymbol{I}_n)\),其中\(k,p,q\)为待定系数.则
\begin{align*}
(c\boldsymbol{A}+d\boldsymbol{I}_n)\cdot k(c\boldsymbol{A}+d\boldsymbol{I}_n)^{-1}=(c\boldsymbol{A}+d\boldsymbol{I}_n)(p\boldsymbol{A}+q\boldsymbol{I}_n)=pc\boldsymbol{A}^2+(cq + dp)\boldsymbol{A}+dq\boldsymbol{I}_n=k\boldsymbol{I}_n.
\end{align*}
令\(pc=\lambda_1,cq + dp=\lambda_2\),则\(p=\frac{\lambda_1}{c},q=\frac{\lambda_2}{c}-\frac{\lambda_1d}{c^2}\).于是由已知条件可得
\begin{align*}
(c\boldsymbol{A}+d\boldsymbol{I}_n)(p\boldsymbol{A}+q\boldsymbol{I}_n)=(c\boldsymbol{A}+d\boldsymbol{I}_n)\left(\frac{\lambda_1}{c}\boldsymbol{A}+\left(\frac{\lambda_2}{c}-\frac{\lambda_1d}{c^2}\right)\boldsymbol{I}_n\right)=\lambda_1\boldsymbol{A}^2+\lambda_2\boldsymbol{A}+d\left(\frac{\lambda_2}{c}-\frac{\lambda_1d}{c^2}\right)\boldsymbol{I}_n=\left(\frac{\lambda_2d}{c}-\frac{\lambda_1d^2}{c^2}-\lambda_3\right)\boldsymbol{I}_n.
\end{align*}
从而\(k=\frac{\lambda_2d}{c}-\frac{\lambda_1d^2}{c^2}-\lambda_3\).因此\((c\boldsymbol{A}+d\boldsymbol{I}_n)^{-1}=\frac{1}{k}(p\boldsymbol{A}+q\boldsymbol{I}_n)=\frac{1}{\frac{\lambda_2d}{c}-\frac{\lambda_1d^2}{c^2}-\lambda_3}\left(\frac{\lambda_1}{c}\boldsymbol{A}+\left(\frac{\lambda_2}{c}-\frac{\lambda_1d}{c^2}\right)\boldsymbol{I}_n\right)\).

{\color{blue}实际做题中只需要先设\(k(c\boldsymbol{A}+d\boldsymbol{I}_n)^{-1}=(p\boldsymbol{A}+q\boldsymbol{I}_n)\),其中\(k,p,q\)为待定系数.则有\((c\boldsymbol{A}+d\boldsymbol{I}_n)(p\boldsymbol{A}+q\boldsymbol{I}_n)=k\boldsymbol{I}_n\).然后通过比较二次项和一次项的系数得到方程组\(\begin{cases}
pc=\lambda_1\\
cq + dp=\lambda_2\\
\end{cases}\)(即要凑出合适的p,q,使得$(c\boldsymbol{A}+d\boldsymbol{I}_n)(p\boldsymbol{A}+q\boldsymbol{I}_n)$与$\lambda_1\boldsymbol{A}^2+\lambda_2\boldsymbol{A}+\lambda_3\boldsymbol{I}_n$的二次项和一次项的系数相等),解出\(p,q\)的值.最后将已知条件\(\lambda_1\boldsymbol{A}^2+\lambda_2\boldsymbol{A}+\lambda_3\boldsymbol{I}_n=\boldsymbol{O}\)代入\((c\boldsymbol{A}+d\boldsymbol{I}_n)(p\boldsymbol{A}+q\boldsymbol{I}_n)=k\boldsymbol{I}_n\)即可得到\(k\)的值.}

熟悉这种方式之后就能快速构造出我们需要的逆矩阵.
\end{note}
\begin{proof}
(1)和(2)的证明相同.如下(这里我们是利用了上述构造逆矩阵的方法直接构造出逆矩阵,再根据逆矩阵的定义直接得到证明):

当\(c = 0\)时,\(c\boldsymbol{A}+d\boldsymbol{I}_n=d\boldsymbol{I}_n\)显然可逆.

当\(c\neq 0\)时,注意到\((c\boldsymbol{A}+d\boldsymbol{I}_n)\left(\frac{\lambda_1}{c}\boldsymbol{A}+\left(\frac{\lambda_2}{c}-\frac{\lambda_1d}{c^2}\right)\boldsymbol{I}_n\right)=\left(\frac{\lambda_2d}{c}-\frac{\lambda_1d^2}{c^2}-\lambda_3\right)\boldsymbol{I}_n\),故\(c\boldsymbol{A}+d\boldsymbol{I}_n\)可逆.
\end{proof}

\begin{example}
设\(n\)阶方阵\(\boldsymbol{A}\)适合等式\(\boldsymbol{A}^{2}-3\boldsymbol{A}+2\boldsymbol{I}_{n}=\boldsymbol{O}\),求证:\(\boldsymbol{A}\)和\(\boldsymbol{A}+\boldsymbol{I}_{n}\)都是可逆阵,而若\(\boldsymbol{A}\neq\boldsymbol{I}_{n}\),则\(\boldsymbol{A}-2\boldsymbol{I}_{n}\)必不是可逆阵.
\end{example}
\begin{note}
这里构造逆矩阵利用了\hyperref[proposition:由矩阵的零化多项式构造其逆矩阵]{命题\ref{proposition:由矩阵的零化多项式构造其逆矩阵}}.
\end{note}
\begin{proof}
由已知得\(\boldsymbol{A}(\boldsymbol{A}-3\boldsymbol{I}_{n})=-2\boldsymbol{I}_{n}\),因此\(\boldsymbol{A}\)是可逆阵.又\(\boldsymbol{A}^{2}-3\boldsymbol{A}-4\boldsymbol{I}_{n}=-6\boldsymbol{I}_{n}\),于是\((\boldsymbol{A}+\boldsymbol{I}_{n})(\boldsymbol{A}-4\boldsymbol{I}_{n})=-6\boldsymbol{I}_{n}\),故\(\boldsymbol{A}+\boldsymbol{I}_{n}\)也是可逆阵.

另一方面,由已知等式可得\((\boldsymbol{A}-\boldsymbol{I}_{n})(\boldsymbol{A}-2\boldsymbol{I}_{n})=\boldsymbol{O}\),如果\(\boldsymbol{A}-2\boldsymbol{I}_{n}\)可逆,则\(\boldsymbol{A}-\boldsymbol{I}_{n}=\boldsymbol{O}\),\(\boldsymbol{A}=\boldsymbol{I}_{n}\)和假设不合,因此\(\boldsymbol{A}-2\boldsymbol{I}_{n}\)不是可逆阵.
\end{proof}

\begin{proposition}\label{proposition:由矩阵等式构造逆矩阵}
(1)若已知\(\lambda _1\boldsymbol{AB}+\lambda _2\boldsymbol{A}+\lambda _3\boldsymbol{B}+\lambda _4\boldsymbol{I}_n=\boldsymbol{O}\),其中\(\lambda_1,\lambda_2,\lambda_3,\lambda_4\in\mathbb{R}\),$\lambda_1\ne 0$,并且$\lambda_1x^2+(\lambda_2+\lambda_3)x+\lambda_4=0$无实根(即原等式左边不可因式分解成$\left( a_1\boldsymbol{I}_n+a_2\boldsymbol{A} \right) \left( b_1\boldsymbol{I}_n+b_2\boldsymbol{B} \right)$),
则对任何$c,d\in\mathbb{R}$,都有\(a\boldsymbol{I}_n+b\boldsymbol{A},c\boldsymbol{I}_n+d\boldsymbol{B}\)可逆.

(2)若已知\(\lambda _1\boldsymbol{AB}+\lambda _2\boldsymbol{A}+\lambda _3\boldsymbol{B}+\lambda _4\boldsymbol{I}_n=\left( a_1\boldsymbol{I}_n+b_1\boldsymbol{A} \right) \left( a_2\boldsymbol{I}_n+b_2\boldsymbol{B} \right) =\boldsymbol{O}\),其中\(a_1,a_2,b_1,b_2\in\mathbb{R}\),且\(\lambda _1=b_1b_2,\lambda _2=a_2b_1,\lambda _3=a_1b_2,\lambda _4=a_1a_2,b_1\ne 0,b_2\ne 0\).
则对任何实数对\(\left( a,b \right),\left( c,d \right) \ne \left( a_1,b_1 \right) ,\left( a_2,b_2 \right)\),都有\(a\boldsymbol{I}_n+b\boldsymbol{A},c\boldsymbol{I}_n+d\boldsymbol{B}\)可逆.
\end{proposition}
\begin{proof}
证明方法与\hyperref[proposition:由矩阵的零化多项式构造其逆矩阵]{命题\ref{proposition:由矩阵的零化多项式构造其逆矩阵}}类似,构造逆矩阵的方法也与其类似.这里不再赘述.
\end{proof}

\begin{example}
\begin{enumerate}
\item 求证:不存在\(n\)阶奇异矩阵\(A\),适合条件\(A^2 + A+I_n = O\).
\item 设\(A\)是\(n\)阶矩阵,且\(A^2 = A\),求证:\(I_n - 2A\)是可逆矩阵.
\item \label{example:item3}若\(A\)是\(n\)阶矩阵,且\(2A(A - I_n)=A^3\),求证:\(I_n - A\)可逆.
\end{enumerate}
\end{example}
\begin{note}
这类问题构造逆矩阵的方法(以\ref{example:item3}为例):已知条件等价于$A^3-2A^2+2A=O$,设$(I_n-A)^{-1}=aA^2+bA+cI_n$,其中$a,b,c$为待定系数,使得\begin{align*}
\left( I_n-A \right) \left( aA^2+bA+cI_n \right) =A^3-2A^2+2A+kI_n=kI_n,k\text{为待定常数}.
\end{align*}
比较等式两边系数可得
\begin{align*}
\begin{cases}
-a=1\\
a-b=-2\\
b-c=2\\
c=k\\
\end{cases}\Rightarrow \begin{cases}
a=-1\\
b=1\\
k=c=-1\\
\end{cases}
\end{align*}
于是$\left( I_n-A \right) \left( -A^2+A-I_n \right) =-I_n.$从而$\left( I_n-A \right) ^{-1}=A^2-A+I_n.$
\end{note}
\begin{proof}
\begin{enumerate}
\item 由已知\(A^2 + A+I_n = O\),则\((A - I_n)(A^2 + A+I_n)=A^3 - I_n = O\),即\(A^3 = I_n\),于是\(A\)是可逆矩阵.
\item 因为\((I_n - 2A)^2 = I_n - 4A + 4A^2 = I_n\),故\(I_n - 2A\)是可逆矩阵.
\item 由已知\(A^3 - 2A^2 + 2A - I_n=-I_n\),即\((A - I_n)(A^2 - A + I_n)=-I_n\),于是\((I_n - A)^{-1}=A^2 - A + I_n\).
\end{enumerate}
\end{proof}

\begin{example}
设\(n\)阶方阵\(\boldsymbol{A}\)和\(\boldsymbol{B}\)满足\(\boldsymbol{A}+\boldsymbol{B}=\boldsymbol{A}\boldsymbol{B}\),求证:\(\boldsymbol{I}_{n}-\boldsymbol{A}\)是可逆阵且\(\boldsymbol{A}\boldsymbol{B}=\boldsymbol{B}\boldsymbol{A}\).
\end{example}
\begin{proof}
因为
\[
(\boldsymbol{I}_{n}-\boldsymbol{A})(\boldsymbol{I}_{n}-\boldsymbol{B})=\boldsymbol{I}_{n}-\boldsymbol{A}-\boldsymbol{B}+\boldsymbol{A}\boldsymbol{B}=\boldsymbol{I}_{n},
\]
所以\(\boldsymbol{I}_{n}-\boldsymbol{A}\)是可逆阵.另一方面,由上式可得\((\boldsymbol{I}_{n}-\boldsymbol{A})^{-1}=(\boldsymbol{I}_{n}-\boldsymbol{B})\),故
\[
\boldsymbol{I}_{n}=(\boldsymbol{I}_{n}-\boldsymbol{B})(\boldsymbol{I}_{n}-\boldsymbol{A})=\boldsymbol{I}_{n}-\boldsymbol{B}-\boldsymbol{A}+\boldsymbol{B}\boldsymbol{A},
\]
从而\(\boldsymbol{B}\boldsymbol{A}=\boldsymbol{A}+\boldsymbol{B}=\boldsymbol{A}\boldsymbol{B}\).
\end{proof}

\begin{proposition}[矩阵转置的性质]\label{proposition:矩阵转置的性质}
设矩阵$A,B$,则有
\begin{enumerate}
\item $(A')'=A$;
\item $(A+B)'=A'+B'$;
\item $(kA)'=kA'$;
\item $(AB)'=B'A'$.
\end{enumerate}
\end{proposition}
\begin{proof}
由矩阵的性质易证.
\end{proof}

\begin{proposition}[矩阵的逆运算]\label{proposition:矩阵的逆运算}
设矩阵$\boldsymbol{A},\boldsymbol{B},\boldsymbol{C}$可逆,则有

常规逆运算:

\begin{enumerate}
\item $\left( \boldsymbol{AB} \right) ^{-1}=\boldsymbol{B}^{-1}\boldsymbol{A}^{-1}$.

\item $\left( \boldsymbol{AC}+\boldsymbol{BC} \right) ^{-1}=\boldsymbol{C}^{-1}\left( \boldsymbol{A}+\boldsymbol{B} \right) ^{-1}$.

\item $\left( \boldsymbol{A}+\boldsymbol{B} \right) ^{-1}\boldsymbol{C}=\left( \boldsymbol{C}^{-1}\boldsymbol{A}+\boldsymbol{C}^{-1}\boldsymbol{B} \right) ^{-1}$.

\item $\boldsymbol{C}\left( \boldsymbol{A}+\boldsymbol{B} \right) ^{-1}=\left( \boldsymbol{AC}^{-1}+\boldsymbol{BC}^{-1} \right) ^{-1}$.
\end{enumerate}
\noindent{\textbf{凑因子:}}
\begin{enumerate}
\item $\boldsymbol{A}=\left( \boldsymbol{AB}^{-1} \right) \boldsymbol{B}=\left( \boldsymbol{AB} \right) \boldsymbol{B}^{-1}=\boldsymbol{B}\left( \boldsymbol{B}^{-1}\boldsymbol{A} \right) =\boldsymbol{B}^{-1}\left( \boldsymbol{BA} \right)$ .

\item $\boldsymbol{A}+\boldsymbol{B}=\left( \boldsymbol{AC}^{-1}+\boldsymbol{BC}^{-1} \right) \boldsymbol{C}=\left( \boldsymbol{AC}+\boldsymbol{BC} \right) \boldsymbol{C}^{-1}=\boldsymbol{C}\left( \boldsymbol{C}^{-1}\boldsymbol{A}+\boldsymbol{C}^{-1}\boldsymbol{B} \right) =\boldsymbol{C}^{-1}\left( \boldsymbol{CA}+\boldsymbol{CB} \right)$ .
\end{enumerate}
\end{proposition}
\begin{note}
无需额外记忆这些公式,只需要知道\textbf{凑因子}的想法,即\textbf{在矩阵可逆的条件下,我们可以利用矩阵$\boldsymbol{I}_n=\boldsymbol{AA}^{-1}=\boldsymbol{A}^{-1}\boldsymbol{A}$的性质,将原本矩阵没有的因子凑出来,然后提取我们需要的矩阵因子到矩阵逆的外面或将其乘入矩阵逆的内部,从而达到化简原矩阵的目的.}
\end{note}
\begin{proof}
由矩阵的运算性质不难证明.
\end{proof}
\begin{remark}
\textbf{凑因子想法}的应用:\hyperref[example:凑因子例1]{例题\ref{example:凑因子例1}},\hyperref[example:凑因子例2]{例题\ref{example:凑因子例2}},\hyperref[Sherman-Morrison-Woodbury公式]{例题\ref{Sherman-Morrison-Woodbury公式}}.
\end{remark}

\begin{example}
设\(A,B,A - B\)都是\(n\)阶可逆阵,证明:
\[
B^{-1} - A^{-1} = (B + B(A - B)^{-1}B)^{-1}.
\]    
\end{example}
\begin{note}
直接运用逆矩阵的定义验证即可.
\end{note}
\begin{proof}
\begin{align*}
&\left( B^{-1}-A^{-1} \right) \left( B+B\left( A-B \right) ^{-1}B \right) 
\\
&=I_n+\left( A-B \right) ^{-1}B-A^{-1}B-A^{-1}B\left( A-B \right) -1B
\\
&=I_n+\left( A-B \right) ^{-1}B-A^{-1}B\left( I_n+\left( A-B \right) -1B \right) 
\\
&=\left( I_n-A^{-1}B \right) \left( I_n+\left( A-B \right) ^{-1}B \right) 
\\
&=A^{-1}\left( A-B \right) \left[ \left( A-B \right) ^{-1}\left( A-+B \right) \right] 
\\
&=A^{-1}\left( A-B \right) \left( A-B \right) ^{-1}A=I_n.
\end{align*}
\end{proof}

\begin{example}[$\,\,$Sherman-Morrison公式]\label{Sherman-Morrison公式}
设\(A\)是\(n\)阶可逆阵,\(\alpha,\beta\)是\(n\)维列向量,且\(1 + \beta'A^{-1}\alpha \neq 0\).求证:
\[
(A + \alpha\beta')^{-1} = A^{-1} - \frac{1}{1 + \beta'A^{-1}\alpha}A^{-1}\alpha\beta'A^{-1}.
\]
\end{example}
\begin{note}
直接运用逆矩阵的定义验证即可,注意$\beta'A^{-1}\alpha $是一个数可以提出来.
\end{note}
\begin{proof}
\begin{align*}
&\left( A+\alpha \beta ' \right) \left( A^{-1}-\frac{1}{1+\beta 'A^{-1}\alpha}A^{-1}\alpha \beta 'A^{-1} \right) 
\\
&=I_n-\frac{1}{1+\beta 'A^{-1}\alpha}\alpha \beta 'A^{-1}+\alpha \beta 'A^{-1}-\frac{1}{1+\beta 'A^{-1}\alpha}\alpha \left( \beta 'A^{-1}\alpha \right) \beta 'A^{-1}
\\
&=I_n+\alpha \beta 'A^{-1}-\frac{1}{1+\beta 'A^{-1}\alpha}\alpha \beta 'A^{-1}-\frac{\beta 'A^{-1}\alpha}{1+\beta 'A^{-1}\alpha}\alpha \beta 'A^{-1}
\\
&=I_n+\alpha \beta 'A^{-1}-\frac{1+\beta 'A^{-1}\alpha}{1+\beta 'A^{-1}\alpha}\alpha \beta 'A^{-1}=I_n.
\end{align*}
\end{proof}


\begin{proposition}[一些矩阵等式]\label{proposition:一些矩阵等式}
\begin{enumerate}
\item 设$\boldsymbol{A}$为$m\times n$矩阵,$\boldsymbol{B}$为$n\times m$矩阵.则有$\boldsymbol{A}\left( \boldsymbol{I}_{\boldsymbol{n}}+\boldsymbol{BA} \right) =\left( \boldsymbol{I}_{\boldsymbol{m}}+\boldsymbol{AB} \right) \boldsymbol{A}$.

\item 设$\boldsymbol{A},\boldsymbol{B}$均为$n$阶可逆矩阵,则有$\boldsymbol{A}+\boldsymbol{B}=\boldsymbol{A}(\boldsymbol{A}^{-1}+\boldsymbol{B}^{-1})\boldsymbol{B}$.

\item 若$n$阶矩阵$A,B$满足$A^2=B^2$,则$A(A+B)=A^2+AB=B^2+AB=(A+B)B$.
\end{enumerate}
\end{proposition}
\begin{note}
这是一些常见的矩阵等式.可以通过反复\hyperref[proposition:矩阵的逆运算]{凑因子}得到.
\end{note}
\begin{proof}
由矩阵的运算性质不难证明.
\end{proof}

\begin{example}\label{example:凑因子例1}
设\(\boldsymbol{A},\boldsymbol{B},\boldsymbol{AB}-\boldsymbol{I}_{n}\)都是\(n\)阶可逆阵,证明:\(\boldsymbol{A}-\boldsymbol{B}^{-1}\)与\((\boldsymbol{A}-\boldsymbol{B}^{-1})^{-1}-\boldsymbol{A}^{-1}\)均可逆,并求它们的逆矩阵.
\end{example}
\begin{note}
核心想法是利用\hyperref[proposition:矩阵的逆运算]{命题\ref{proposition:矩阵的逆运算}}和\hyperref[proposition:一些矩阵等式]{命题\ref{proposition:一些矩阵等式}}.
\end{note}
\begin{proof}
注意到\(\boldsymbol{A}-\boldsymbol{B}^{-1}=(\boldsymbol{AB}-\boldsymbol{I}_{n})\boldsymbol{B}^{-1}\),故\(\boldsymbol{A}-\boldsymbol{B}^{-1}\)是可逆矩阵,并且\((\boldsymbol{A}-\boldsymbol{B}^{-1})^{-1}=\boldsymbol{B}(\boldsymbol{AB}-\boldsymbol{I}_{n})^{-1}\).注意到如下变形:
\begin{align*}
&(\boldsymbol{A}-\boldsymbol{B}^{-1})^{-1}-\boldsymbol{A}^{-1}\\
=&\boldsymbol{B}(\boldsymbol{AB}-\boldsymbol{I}_{n})^{-1}-\boldsymbol{A}^{-1}=\boldsymbol{A}^{-1}(\boldsymbol{AB}(\boldsymbol{AB}-\boldsymbol{I}_{n})^{-1}-\boldsymbol{I}_{n})\\
=&\boldsymbol{A}^{-1}(\boldsymbol{AB}-(\boldsymbol{AB}-\boldsymbol{I}_{n}))(\boldsymbol{AB}-\boldsymbol{I}_{n})^{-1}=\boldsymbol{A}^{-1}(\boldsymbol{AB}-\boldsymbol{I}_{n})^{-1}.
\end{align*}
故\((\boldsymbol{A}-\boldsymbol{B}^{-1})^{-1}-\boldsymbol{A}^{-1}\)可逆,并且\((( \boldsymbol{A}-\boldsymbol{B}^{-1})^{-1}-\boldsymbol{A}^{-1})^{-1}=(\boldsymbol{AB}-\boldsymbol{I}_{n})\boldsymbol{A}\).
\end{proof}

\begin{proposition}\label{proposition:矩阵可逆的重要结论1}
设\(\boldsymbol{A}\)为\(m\times n\)矩阵,\(\boldsymbol{B}\)为\(n\times m\)矩阵,使得\(\boldsymbol{I}_{m}+\boldsymbol{AB}\)可逆,则\(\boldsymbol{I}_{n}+\boldsymbol{BA}\)也可逆,并且\((\boldsymbol{I}_{n}+\boldsymbol{BA})^{-1}=\boldsymbol{I}_{n}-\boldsymbol{B}(\boldsymbol{I}_{m}+\boldsymbol{AB})^{-1}\boldsymbol{A}\).
\end{proposition}
\begin{note}
\hyperref[proposition:矩阵可逆的重要结论1]{命题\ref{proposition:矩阵可逆的重要结论1}}的应用:一般对于求只含有两项的矩阵和式的逆矩阵,我们可以利用\hyperref[proposition:矩阵的逆运算]{矩阵的逆运算(凑因子)}的方法将原矩阵和式转化为$\boldsymbol{C}\left( \boldsymbol{I}_n+\boldsymbol{AB} \right)$或$\left( \boldsymbol{I}_n+\boldsymbol{AB} \right) \boldsymbol{C}$的形式,再利用\hyperref[proposition:矩阵可逆的重要结论1]{这个命题}求得原矩阵的逆.
\end{note}
\begin{remark}
证法一只能得到\(\boldsymbol{I}_{n}+\boldsymbol{BA}\)可逆,并不能得到具体的逆矩阵.而证法二可以求出\((\boldsymbol{I}_{n}+\boldsymbol{BA})^{-1}=\boldsymbol{I}_{n}-\boldsymbol{B}(\boldsymbol{I}_{m}+\boldsymbol{AB})^{-1}\boldsymbol{A}\).
\end{remark}
\begin{proof}
{\color{blue}证法一(\hyperref[proposition:打洞原理]{打洞原理}):}根据分块矩阵的初等变换可得\begin{align*}
\left| \begin{matrix}
\boldsymbol{I}_m&		-\boldsymbol{A}\\
\boldsymbol{B}&		\boldsymbol{I}_n\\
\end{matrix} \right|=\left| \begin{matrix}
\boldsymbol{I}_m&		\boldsymbol{O}\\
-\boldsymbol{B}&		\boldsymbol{I}_n\\
\end{matrix} \right|\left| \begin{matrix}
\boldsymbol{I}_m&		-\boldsymbol{A}\\
\boldsymbol{B}&		\boldsymbol{I}_n\\
\end{matrix} \right|=\left| \left( \begin{matrix}
\boldsymbol{I}_m&		\boldsymbol{O}\\
-\boldsymbol{B}&		\boldsymbol{I}_n\\
\end{matrix} \right) \left( \begin{matrix}
\boldsymbol{I}_m&		-\boldsymbol{A}\\
\boldsymbol{B}&		\boldsymbol{I}_n\\
\end{matrix} \right) \right|=\left| \begin{matrix}
\boldsymbol{I}_m&		-\boldsymbol{A}\\
\boldsymbol{O}&		\boldsymbol{I}_n+\boldsymbol{BA}\\
\end{matrix} \right|=\left| \boldsymbol{I}_n+\boldsymbol{BA} \right|.
\\
\left| \begin{matrix}
\boldsymbol{I}_m&		-\boldsymbol{A}\\
\boldsymbol{B}&		\boldsymbol{I}_n\\
\end{matrix} \right|=\left| \begin{matrix}
\boldsymbol{I}_m&		\boldsymbol{A}\\
\boldsymbol{O}&		\boldsymbol{I}_n\\
\end{matrix} \right|\left| \begin{matrix}
\boldsymbol{I}_m&		-\boldsymbol{A}\\
\boldsymbol{B}&		\boldsymbol{I}_n\\
\end{matrix} \right|=\left| \left( \begin{matrix}
\boldsymbol{I}_m&		\boldsymbol{A}\\
\boldsymbol{O}&		\boldsymbol{I}_n\\
\end{matrix} \right) \left( \begin{matrix}
\boldsymbol{I}_m&		-\boldsymbol{A}\\
\boldsymbol{B}&		\boldsymbol{I}_n\\
\end{matrix} \right) \right|=\left| \begin{matrix}
\boldsymbol{I}_m+\boldsymbol{AB}&		\boldsymbol{O}\\
\boldsymbol{B}&		\boldsymbol{I}_n\\
\end{matrix} \right|=\left| \boldsymbol{I}_m+\boldsymbol{AB} \right|.
\end{align*}
故$\left| \begin{matrix}
\boldsymbol{I}_m&		-\boldsymbol{A}\\
\boldsymbol{B}&		\boldsymbol{I}_n\\
\end{matrix} \right|=\left| \boldsymbol{I}_m+\boldsymbol{AB} \right|=\left| \boldsymbol{I}_n+\boldsymbol{BA} \right|$.又因为$\boldsymbol{I}_m+\boldsymbol{AB}$可逆,所以$\left| \boldsymbol{I}_n+\boldsymbol{BA} \right|=\left| \boldsymbol{I}_m+\boldsymbol{AB} \right|\ne0$.因此$\boldsymbol{I}_n+\boldsymbol{BA}$也可逆.

{\color{blue}证法二(\hyperref[proposition:矩阵的逆运算]{矩阵的逆运算}):}
注意到\(\boldsymbol{A}(\boldsymbol{I}_{n}+\boldsymbol{BA})=(\boldsymbol{I}_{m}+\boldsymbol{AB})\boldsymbol{A}\),故\((\boldsymbol{I}_{m}+\boldsymbol{AB})^{-1}\boldsymbol{A}(\boldsymbol{I}_{n}+\boldsymbol{BA})=\boldsymbol{A}\),
于是\(\boldsymbol{B}(\boldsymbol{I}_{m}+\boldsymbol{AB})^{-1}\boldsymbol{A}(\boldsymbol{I}_{n}+\boldsymbol{BA})=\boldsymbol{BA}\),从而
\begin{align*}
\boldsymbol{I}_{n}&=\boldsymbol{I}_{n}+\boldsymbol{BA}-\boldsymbol{BA}=(\boldsymbol{I}_{n}+\boldsymbol{BA})-\boldsymbol{B}(\boldsymbol{I}_{m}+\boldsymbol{AB})^{-1}\boldsymbol{A}(\boldsymbol{I}_{n}+\boldsymbol{BA})\\
&=(\boldsymbol{I}_{n}-\boldsymbol{B}(\boldsymbol{I}_{m}+\boldsymbol{AB})^{-1}\boldsymbol{A})(\boldsymbol{I}_{n}+\boldsymbol{BA}).
\end{align*}
于是\((\boldsymbol{I}_{n}+\boldsymbol{BA})^{-1}=\boldsymbol{I}_{n}-\boldsymbol{B}(\boldsymbol{I}_{m}+\boldsymbol{AB})^{-1}\boldsymbol{A}\).
\end{proof}

\begin{example}\label{example:凑因子例2}
设\(\boldsymbol{A},\boldsymbol{B}\)均为\(n\)阶可逆阵,使得\(\boldsymbol{A}^{-1}+\boldsymbol{B}^{-1}\)可逆,证明:\(\boldsymbol{A}+\boldsymbol{B}\)也可逆,并且
\[
(\boldsymbol{A}+\boldsymbol{B})^{-1}=\boldsymbol{A}^{-1}-\boldsymbol{A}^{-1}(\boldsymbol{A}^{-1}+\boldsymbol{B}^{-1})^{-1}\boldsymbol{A}^{-1}.
\]
\end{example}
\begin{proof}
注意到\(\boldsymbol{A}+\boldsymbol{B}=\boldsymbol{A}(\boldsymbol{A}^{-1}+\boldsymbol{B}^{-1})\boldsymbol{B}\),故\(\boldsymbol{A}+\boldsymbol{B}\)可逆.由\hyperref[proposition:矩阵可逆的重要结论1]{命题\ref{proposition:矩阵可逆的重要结论1}}可得
\[
(\boldsymbol{I}_{n}+\boldsymbol{A}^{-1}\boldsymbol{B})^{-1}=\boldsymbol{I}_{n}-\boldsymbol{A}^{-1}(\boldsymbol{I}_{n}+\boldsymbol{B}\boldsymbol{A}^{-1})^{-1}\boldsymbol{B}=\boldsymbol{I}_{n}-\boldsymbol{A}^{-1}(\boldsymbol{A}^{-1}+\boldsymbol{B}^{-1})^{-1},
\]
于是
\begin{align*}
(\boldsymbol{A}+\boldsymbol{B})^{-1}&=(\boldsymbol{A}(\boldsymbol{I}_{n}+\boldsymbol{A}^{-1}\boldsymbol{B}))^{-1}=(\boldsymbol{I}_{n}+\boldsymbol{A}^{-1}\boldsymbol{B})^{-1}\boldsymbol{A}^{-1}\\
&=\boldsymbol{A}^{-1}-\boldsymbol{A}^{-1}(\boldsymbol{A}^{-1}+\boldsymbol{B}^{-1})^{-1}\boldsymbol{A}^{-1}.\square
\end{align*}
\end{proof}

\begin{example}[$\,\,$Sherman-Morrison-Woodbury公式]\label{Sherman-Morrison-Woodbury公式}

设\(\boldsymbol{A}\)为\(n\)阶可逆阵,\(\boldsymbol{C}\)为\(m\)阶可逆阵,\(\boldsymbol{B}\)为\(n\times m\)矩阵,\(\boldsymbol{D}\)为\(m\times n\)矩阵,使得\(\boldsymbol{C}^{-1}+\boldsymbol{D}\boldsymbol{A}^{-1}\boldsymbol{B}\)可逆.求证:\(\boldsymbol{A}+\boldsymbol{B}\boldsymbol{C}\boldsymbol{D}\)也可逆,并且
\[
(\boldsymbol{A}+\boldsymbol{B}\boldsymbol{C}\boldsymbol{D})^{-1}=\boldsymbol{A}^{-1}-\boldsymbol{A}^{-1}\boldsymbol{B}(\boldsymbol{C}^{-1}+\boldsymbol{D}\boldsymbol{A}^{-1}\boldsymbol{B})^{-1}\boldsymbol{D}\boldsymbol{A}^{-1}.
\]
\end{example}
\begin{remark}
若已知矩阵逆的表达式,也可以采取利用矩阵逆的定义直接验证的方法进行证明.
\end{remark}
\begin{proof}
注意到\(\boldsymbol{A}+\boldsymbol{B}\boldsymbol{C}\boldsymbol{D}=\boldsymbol{A}(\boldsymbol{I}_{n}+\boldsymbol{A}^{-1}\boldsymbol{B}\boldsymbol{C}\boldsymbol{D})\),将\(\boldsymbol{A}^{-1}\boldsymbol{B}\)和\(\boldsymbol{C}\boldsymbol{D}\)分别看成整体,此时
\(\boldsymbol{I}_{m}+(\boldsymbol{C}\boldsymbol{D})(\boldsymbol{A}^{-1}\boldsymbol{B})=\boldsymbol{C}(\boldsymbol{C}^{-1}+\boldsymbol{D}\boldsymbol{A}^{-1}\boldsymbol{B})\)可逆,故由\hyperref[proposition:矩阵可逆的重要结论1]{命题\ref{proposition:矩阵可逆的重要结论1}}的结论可知\(\boldsymbol{I}_{n}+(\boldsymbol{A}^{-1}\boldsymbol{B})(\boldsymbol{C}\boldsymbol{D})\)也可逆,并且
\begin{align*}
(\boldsymbol{I}_{n}+\boldsymbol{A}^{-1}\boldsymbol{B}\boldsymbol{C}\boldsymbol{D})^{-1}&=\boldsymbol{I}_{n}-\boldsymbol{A}^{-1}\boldsymbol{B}(\boldsymbol{I}_{m}+\boldsymbol{C}\boldsymbol{D}\boldsymbol{A}^{-1}\boldsymbol{B})^{-1}\boldsymbol{C}\boldsymbol{D}\\
&=\boldsymbol{I}_{n}-\boldsymbol{A}^{-1}\boldsymbol{B}(\boldsymbol{C}^{-1}+\boldsymbol{D}\boldsymbol{A}^{-1}\boldsymbol{B})^{-1}\boldsymbol{D}.
\end{align*}
于是\(\boldsymbol{A}+\boldsymbol{B}\boldsymbol{C}\boldsymbol{D}=\boldsymbol{A}(\boldsymbol{I}_{n}+\boldsymbol{A}^{-1}\boldsymbol{B}\boldsymbol{C}\boldsymbol{D})\)也可逆,并且
\[
(\boldsymbol{A}+\boldsymbol{B}\boldsymbol{C}\boldsymbol{D})^{-1}=\boldsymbol{A}^{-1}-\boldsymbol{A}^{-1}\boldsymbol{B}(\boldsymbol{C}^{-1}+\boldsymbol{D}\boldsymbol{A}^{-1}\boldsymbol{B})^{-1}\boldsymbol{D}\boldsymbol{A}^{-1}.
\]
\end{proof}


\subsection{练习}


\begin{exercise}
计算下列矩阵的\(k\)次幂,其中\(k\)为正整数:

(1) \(A=\begin{pmatrix}
a & 1 & 0 \\
0 & a & 1 \\
0 & 0 & a
\end{pmatrix}\);\quad
(2) \(A=\begin{pmatrix}
1 & 2 & 4 \\
2 & 4 & 8 \\
3 & 6 & 12
\end{pmatrix}\).
\end{exercise}
\begin{note}
第(2)问核心想法是利用\hyperref[proposition:可以写成两个矩阵(向量)乘积的矩阵]{命题\ref{proposition:可以写成两个矩阵(向量)乘积的矩阵}}.
\end{note}
\begin{solution}
(1)设\(J=\begin{pmatrix}
0 & 1 & 0 \\
0 & 0 & 1 \\
0 & 0 & 0
\end{pmatrix}\),则\(A = aI_{3}+J\).注意到\(aI_{3}\)和\(J\)乘法可交换,$J$是幂零阵并且\(J^{3}=O\),因此我们可用二项式定理来求\(A\)的\(k\)次幂:
\begin{align*}
A^{k}&=(aI_{3}+J)^{k}=(aI_{3})^{k}+C_{k}^{1}(aI_{3})^{k - 1}J+C_{k}^{2}(aI_{3})^{k - 2}J^{2}\\
&=a^{k}I_{3}+C_{k}^{1}a^{k - 1}J+C_{k}^{2}a^{k - 2}J^{2}=\begin{pmatrix}
a^{k}&C_{k}^{1}a^{k - 1}&C_{k}^{2}a^{k - 2}\\
0&a^{k}&C_{k}^{1}a^{k - 1}\\
0&0&a^{k}
\end{pmatrix}
\end{align*}

(2)注意到\(A\)的列向量成比例,故可设\(\alpha=(1,2,3)\),\(\beta=(1,2,4)\),则\(A = \alpha\beta'\).
由矩阵乘法的结合律并注意到\(\beta\alpha' = 17\),可得
\begin{align*}
A^{k}&=(\alpha\beta')(\alpha\beta')\cdots(\alpha\beta')=\alpha(\beta'\alpha)(\beta'\alpha)\cdots(\beta'\alpha)\beta'\\
&=(\beta'\alpha)^{k - 1}\alpha\beta'=17^{k - 1}A=
\begin{pmatrix}
17^{k - 1}&2\cdot17^{k - 1}&4\cdot17^{k - 1}\\
2\cdot17^{k - 1}&4\cdot17^{k - 1}&8\cdot17^{k - 1}\\
3\cdot17^{k - 1}&6\cdot17^{k - 1}&12\cdot17^{k - 1}
\end{pmatrix}
\end{align*}
\end{solution}

\begin{exercise}
设\(k\)是正整数,计算\(\begin{pmatrix}
\cos\theta & \sin\theta\\
-\sin\theta & \cos\theta
\end{pmatrix}^k\).
\end{exercise}
\begin{solution}
已知$k=1$时,有\(\begin{pmatrix}
\cos\theta & \sin\theta\\
-\sin\theta & \cos\theta
\end{pmatrix}\).假设$k=n$时,有$\left( \begin{matrix}
\cos \theta&		\sin \theta\\
-\sin \theta&		\cos \theta\\
\end{matrix} \right) ^n=\left( \begin{matrix}
\cos n\theta&		\sin n\theta\\
-\sin n\theta&		\cos n\theta\\
\end{matrix} \right) $.则当$k=n+1$时,有
\begin{align*}
&\left( \begin{matrix}
\cos \theta&		\sin \theta\\
-\sin \theta&		\cos \theta\\
\end{matrix} \right) ^{n+1}=\left( \begin{matrix}
\cos \theta&		\sin \theta\\
-\sin \theta&		\cos \theta\\
\end{matrix} \right) ^n\left( \begin{matrix}
\cos \theta&		\sin \theta\\
-\sin \theta&		\cos \theta\\
\end{matrix} \right) =\left( \begin{matrix}
\cos n\theta&		\sin n\theta\\
-\sin n\theta&		\cos n\theta\\
\end{matrix} \right) \left( \begin{matrix}
\cos \theta&		\sin \theta\\
-\sin \theta&		\cos \theta\\
\end{matrix} \right) 
\\
&=\left( \begin{matrix}
\cos n\theta \cos \theta -\sin n\theta \sin \theta&		\cos n\theta \sin \theta +\sin n\theta \cos \theta\\
-\left( \cos n\theta \sin \theta +\sin n\theta \cos \theta \right)&		\cos n\theta \cos \theta -\sin n\theta \sin \theta\\
\end{matrix} \right) 
=\left( \begin{matrix}
\cos \left( n+1 \right) \theta&		\sin \left( n+1 \right) \theta\\
-\sin \left( n+1 \right) \theta&		\cos \left( n+1 \right) \theta\\
\end{matrix} \right) .
\end{align*}
从而由数学归纳法可知,对任意正整数\(k\),有$\left( \begin{matrix}
\cos \theta&		\sin \theta\\
-\sin \theta&		\cos \theta\\
\end{matrix} \right) ^k=\left( \begin{matrix}
\cos k\theta&		\sin k\theta\\
-\sin k\theta&		\cos k\theta\\
\end{matrix} \right)$.
\end{solution}

\begin{exercise}
求矩阵\(A\)的逆阵:
\[
A = 
\begin{pmatrix}
1 & 2 & 3 & \cdots & n - 1 & n \\
n & 1 & 2 & \cdots & n - 2 & n - 1 \\
n - 1 & n & 1 & \cdots & n - 3 & n - 2 \\
\vdots & \vdots & \vdots & & \vdots & \vdots \\
2 & 3 & 4 & \cdots & n & 1
\end{pmatrix}.
\]
\end{exercise}
\begin{solution}
对\(\left( \begin{matrix}
A&		I_n\\
\end{matrix} \right) \)用初等变换法,将所有行加到第一行上,再将第一行乘以\(s^{-1}\),其中\(s = \frac{1}{2}n(n + 1)\),得到
\setcounter{MaxMatrixCols}{20} % 将矩阵最大列数设置为20
\begin{gather*}
\left( \begin{matrix}
1&		2&		3&		\cdots&		n-1&		n&		1&		0&		0&		\cdots&		0&		0\\
n&		1&		2&		\cdots&		n-2&		n-1&		0&		1&		0&		\cdots&		0&		0\\
n-1&		n&		1&		\cdots&		n-3&		n-2&		0&		0&		1&		\cdots&		0&		0\\
\vdots&		\vdots&		\vdots&		&		\vdots&		\vdots&		\vdots&		\vdots&		\vdots&		&		\vdots&		\vdots\\
2&		3&		4&		\cdots&		n&		1&		0&		0&		0&		\cdots&		0&		1\\
\end{matrix} \right) \rightarrow
\\
\left( \begin{matrix}
1&		1&		1&		\cdots&		1&		1&		\frac{1}{s}&		\frac{1}{s}&		\frac{1}{s}&		\cdots&		\frac{1}{s}&		\frac{1}{s}\\
n&		1&		2&		\cdots&		n-2&		n-1&		0&		1&		0&		\cdots&		0&		0\\
n-1&		n&		1&		\cdots&		n-3&		n-2&		0&		0&		1&		\cdots&		0&		0\\
\vdots&		\vdots&		\vdots&		&		\vdots&		\vdots&		\vdots&		\vdots&		\vdots&		&		\vdots&		\vdots\\
2&		3&		4&		\cdots&		n&		1&		0&		0&		0&		\cdots&		0&		1\\
\end{matrix} \right) .
\end{gather*}
从第二行起依次减去下一行,得到
\begin{gather*}
\left( \begin{matrix}
1&		1&		1&		\cdots&		1&		1&		\frac{1}{s}&		\frac{1}{s}&		\frac{1}{s}&		\cdots&		\frac{1}{s}&		\frac{1}{s}\\
1&		1-n&		1&		\cdots&		1&		1&		0&		1&		-1&		\cdots&		0&		0\\
1&		1&		1-n&		\cdots&		1&		1&		0&		0&		1&		\cdots&		0&		0\\
\vdots&		\vdots&		\vdots&		&		\vdots&		\vdots&		\vdots&		\vdots&		\vdots&		&		\vdots&		\vdots\\
2&		3&		4&		\cdots&		n&		1&		0&		0&		0&		\cdots&		0&		1\\
\end{matrix} \right) .
\end{gather*}
消去第一列除第一行外的所有元素后,得到
\begin{gather*}
\left( \begin{matrix}
1&		1&		1&		\cdots&		1&		1&		\frac{1}{s}&		\frac{1}{s}&		\frac{1}{s}&		\cdots&		\frac{1}{s}&		\frac{1}{s}\\
0&		-n&		0&		\cdots&		0&		0&		-\frac{1}{s}&		\frac{s-1}{s}&		-\frac{s+1}{s}&		\cdots&		-\frac{1}{s}&		-\frac{1}{s}\\
0&		0&		-n&		\cdots&		0&		0&		-\frac{1}{s}&		-\frac{1}{s}&		\frac{s-1}{s}&		\cdots&		-\frac{1}{s}&		-\frac{1}{s}\\
\vdots&		\vdots&		\vdots&		&		\vdots&		\vdots&		\vdots&		\vdots&		\vdots&		&		\vdots&		\vdots\\
0&		1&		2&		\cdots&		n-2&		-1&		-\frac{2}{s}&		-\frac{2}{s}&		-\frac{2}{s}&		\cdots&		-\frac{2}{s}&		\frac{s-2}{s}\\
\end{matrix} \right) .
\end{gather*}
从第二行到第\(n - 1\)行分别乘以\(-\frac{1}{n}\),得到
\begin{gather*}
\left( \begin{matrix}
1&		1&		1&		\cdots&		1&		1&		\frac{1}{s}&		\frac{1}{s}&		\frac{1}{s}&		\cdots&		\frac{1}{s}&		\frac{1}{s}\\
0&		1&		0&		\cdots&		0&		0&		\frac{1}{ns}&		\frac{1-s}{ns}&		\frac{s+1}{ns}&		\cdots&		\frac{1}{ns}&		\frac{1}{ns}\\
0&		0&		1&		\cdots&		0&		0&		\frac{1}{ns}&		\frac{1}{ns}&		\frac{1-s}{ns}&		\cdots&		\frac{1}{ns}&		\frac{1}{ns}\\
\vdots&		\vdots&		\vdots&		&		\vdots&		\vdots&		\vdots&		\vdots&		\vdots&		&		\vdots&		\vdots\\
0&		1&		2&		\cdots&		n-2&		-1&		-\frac{2}{s}&		-\frac{2}{s}&		-\frac{2}{s}&		\cdots&		-\frac{2}{s}&		\frac{s-2}{s}\\
\end{matrix} \right) .
\end{gather*}
将第一行依次减去第二行,第三行,\(\cdots\),第\(n - 1\)行,得到
\begin{gather*}
\left( \begin{matrix}
1&		0&		0&		\cdots&		0&		1&		\frac{2}{ns}&		\frac{s+2}{ns}&		\frac{2}{ns}&		\cdots&		\frac{2}{ns}&		\frac{2-s}{ns}\\
0&		1&		0&		\cdots&		0&		0&		\frac{1}{ns}&		\frac{1-s}{ns}&		\frac{s+1}{ns}&		\cdots&		\frac{1}{ns}&		\frac{1}{ns}\\
0&		0&		1&		\cdots&		0&		0&		\frac{1}{ns}&		\frac{1}{ns}&		\frac{1-s}{ns}&		\cdots&		\frac{1}{ns}&		\frac{1}{ns}\\
\vdots&		\vdots&		\vdots&		&		\vdots&		\vdots&		\vdots&		\vdots&		\vdots&		&		\vdots&		\vdots\\
0&		1&		2&		\cdots&		n-2&		-1&		-\frac{2}{s}&		-\frac{2}{s}&		-\frac{2}{s}&		\cdots&		-\frac{2}{s}&		\frac{s-2}{s}\\
\end{matrix} \right) .
\end{gather*}
将最后一行加到第一行,再将最后一行乘以\(-1\),得到
\begin{gather*}
\left( \begin{matrix}
1&		0&		0&		\cdots&		0&		0&		\frac{1-s}{ns}&		\frac{1+s}{ns}&		\frac{1}{ns}&		\cdots&		\frac{1}{ns}&		\frac{1}{ns}\\
0&		1&		0&		\cdots&		0&		0&		\frac{1}{ns}&		\frac{1-s}{ns}&		\frac{s+1}{ns}&		\cdots&		\frac{1}{ns}&		\frac{1}{ns}\\
0&		0&		1&		\cdots&		0&		0&		\frac{1}{ns}&		\frac{1}{ns}&		\frac{1-s}{ns}&		\cdots&		\frac{1}{ns}&		\frac{1}{ns}\\
\vdots&		\vdots&		\vdots&		&		\vdots&		\vdots&		\vdots&		\vdots&		\vdots&		&		\vdots&		\vdots\\
0&		0&		0&		\cdots&		0&		1&		\frac{s+1}{ns}&		\frac{1}{ns}&		\frac{1}{ns}&		\cdots&		\frac{1}{ns}&		\frac{1-s}{ns}\\
\end{matrix} \right) .
\end{gather*}
因此
\begin{align*}
A^{-1}=\frac{1}{ns}\left( \begin{matrix}
1-s&		1+s&		1&		\cdots&		1&		1\\
1&		1-s&		1+s&		\cdots&		1&		1\\
1&		1&		1-s&		\cdots&		1&		1\\
\vdots&		\vdots&		\vdots&		&		\vdots&		\vdots\\
1+s&		1&		1&		\cdots&		1&		1-s\\
\end{matrix} \right) .
\end{align*}
\end{solution}

\begin{exercise}\label{exercise2.3}
求下列\(n\)阶矩阵的逆阵,其中\(a_i \neq 0(1\leq i\leq n)\):
\[
A = 
\begin{pmatrix}
1 + a_1 & 1 & 1 & \cdots & 1 \\
1 & 1 + a_2 & 1 & \cdots & 1 \\
1 & 1 & 1 + a_3 & \cdots & 1 \\
\vdots & \vdots & \vdots & & \vdots \\
1 & 1 & 1 & \cdots & 1 + a_n
\end{pmatrix}.
\]
\end{exercise}
\begin{solution}
对\(\left( \begin{matrix}
A&		I_n\\
\end{matrix} \right) \)用初等变换法,将第\(i\)行乘以\(a_i^{-1}(1\leq i\leq n)\),有
\begin{gather*}
\left( \begin{matrix}
1+a_1&		1&		1&		\cdots&		1&		1&		0&		0&		\cdots&		0\\
1&		1+a_2&		1&		\cdots&		1&		0&		1&		0&		\cdots&		0\\
1&		1&		1+a_3&		\cdots&		1&		0&		0&		1&		\cdots&		0\\
\vdots&		\vdots&		\vdots&		&		\vdots&		\vdots&		\vdots&		\vdots&		&		\vdots\\
1&		1&		1&		\cdots&		1+a_n&		0&		0&		0&		\cdots&		1\\
\end{matrix} \right) \rightarrow 
\\
\left( \begin{matrix}
1+\frac{1}{a_1}&		\frac{1}{a_1}&		\frac{1}{a_1}&		\cdots&		\frac{1}{a_1}&		\frac{1}{a_1}&		0&		0&		\cdots&		0\\
\frac{1}{a_2}&		1+\frac{1}{a_2}&		\frac{1}{a_2}&		\cdots&		\frac{1}{a_2}&		0&		\frac{1}{a_2}&		0&		\cdots&		0\\
\frac{1}{a_3}&		\frac{1}{a_3}&		1+\frac{1}{a_3}&		\cdots&		\frac{1}{a_3}&		0&		0&		\frac{1}{a_3}&		\cdots&		0\\
\vdots&		\vdots&		\vdots&		&		\vdots&		\vdots&		\vdots&		\vdots&		&		\vdots\\
\frac{1}{a_n}&		\frac{1}{a_n}&		\frac{1}{a_n}&		\cdots&		1+\frac{1}{a_n}&		0&		0&		0&		\cdots&		\frac{1}{a_n}\\
\end{matrix} \right) .
\end{gather*}
将下面的行都加到第一行上,并令\(s = 1 + \frac{1}{a_1} + \frac{1}{a_2} + \cdots + \frac{1}{a_n}\),则上面的矩阵变为
\begin{gather*}
\left( \begin{matrix}
s&		s&		s&		\cdots&		s&		\frac{1}{a_1}&		\frac{1}{a_2}&		\frac{1}{a_3}&		\cdots&		\frac{1}{a_n}\\
\frac{1}{a_2}&		1+\frac{1}{a_2}&		\frac{1}{a_2}&		\cdots&		\frac{1}{a_2}&		0&		\frac{1}{a_2}&		0&		\cdots&		0\\
\frac{1}{a_3}&		\frac{1}{a_3}&		1+\frac{1}{a_3}&		\cdots&		\frac{1}{a_3}&		0&		0&		\frac{1}{a_3}&		\cdots&		0\\
\vdots&		\vdots&		\vdots&		&		\vdots&		\vdots&		\vdots&		\vdots&		&		\vdots\\
\frac{1}{a_n}&		\frac{1}{a_n}&		\frac{1}{a_n}&		\cdots&		1+\frac{1}{a_n}&		0&		0&		0&		\cdots&		\frac{1}{a_n}\\
\end{matrix} \right) \rightarrow 
\\
\left( \begin{matrix}
1&		1&		1&		\cdots&		1&		\frac{1}{sa_1}&		\frac{1}{sa_2}&		\frac{1}{sa_3}&		\cdots&		\frac{1}{sa_n}\\
\frac{1}{a_2}&		1+\frac{1}{a_2}&		\frac{1}{a_2}&		\cdots&		\frac{1}{a_2}&		0&		\frac{1}{a_2}&		0&		\cdots&		0\\
\frac{1}{a_3}&		\frac{1}{a_3}&		1+\frac{1}{a_3}&		\cdots&		\frac{1}{a_3}&		0&		0&		\frac{1}{a_3}&		\cdots&		0\\
\vdots&		\vdots&		\vdots&		&		\vdots&		\vdots&		\vdots&		\vdots&		&		\vdots\\
\frac{1}{a_n}&		\frac{1}{a_n}&		\frac{1}{a_n}&		\cdots&		1+\frac{1}{a_n}&		0&		0&		0&		\cdots&		\frac{1}{a_n}\\
\end{matrix} \right) \rightarrow 
\\
\left( \begin{matrix}
1&		1&		1&		\cdots&		1&		\frac{1}{sa_1}&		\frac{1}{sa_2}&		\frac{1}{sa_3}&		\cdots&		\frac{1}{sa_n}\\
0&		1&		0&		\cdots&		0&		-\frac{1}{sa_1a_2}&		\frac{sa_2-1}{sa_{2}^{2}}&		-\frac{1}{sa_3a_2}&		\cdots&		-\frac{1}{sa_na_2}\\
0&		0&		1&		\cdots&		0&		-\frac{1}{sa_1a_3}&		-\frac{1}{sa_2a_3}&		\frac{sa_3-1}{sa_{3}^{2}}&		\cdots&		-\frac{1}{sa_na_3}\\
\vdots&		\vdots&		\vdots&		&		\vdots&		\vdots&		\vdots&		\vdots&		&		\vdots\\
0&		0&		0&		\cdots&		1&		-\frac{1}{sa_1a_n}&		-\frac{1}{sa_2a_n}&		-\frac{1}{sa_3a_n}&		\cdots&		\frac{sa_n-1}{sa_{n}^{2}}\\
\end{matrix} \right) .
\end{gather*}
再消去第一行的后$n-1$个1就得到
\begin{align*}
A^{-1}=-\frac{1}{s}\left( \begin{matrix}
\frac{1}{a_1}&		\frac{1}{a_2}&		\frac{1}{a_3}&		\cdots&		\frac{1}{a_n}\\
-\frac{1}{a_1a_2}&		\frac{sa_2-1}{a_{2}^{2}}&		-\frac{1}{a_3a_2}&		\cdots&		-\frac{1}{a_na_2}\\
-\frac{1}{a_1a_3}&		-\frac{1}{a_2a_3}&		\frac{sa_3-1}{a_{3}^{2}}&		\cdots&		-\frac{1}{a_na_3}\\
\vdots&		\vdots&		\vdots&		&		\vdots\\
-\frac{1}{a_1a_n}&		-\frac{1}{a_2a_n}&		-\frac{1}{a_3a_n}&		\cdots&		\frac{sa_n-1}{a_{n}^{2}}\\
\end{matrix} \right) .
\end{align*}

\end{solution}

\begin{exercise}
求下列\(n\)阶矩阵的逆矩阵:
\[
A = 
\begin{pmatrix}
0 & 1 & 1 & \cdots & 1 \\
1 & 0 & 1 & \cdots & 1 \\
1 & 1 & 0 & \cdots & 1 \\
\vdots & \vdots & \vdots & & \vdots \\
1 & 1 & 1 & \cdots & 0
\end{pmatrix}
\]    
\end{exercise}
\begin{note}
解法一和解法二的核心想法是:先假设(猜测)矩阵$A$的逆矩阵与其具有相似的结构,再结合逆矩阵的定义,使用待定系数法求出矩阵$A$的逆矩阵.
\end{note}
\begin{solution}
{\color{blue}解法一:}
设\(\alpha = (1,1,\cdots,1)'\),则\(A = -I_n + \alpha\alpha'\).设\(B = cI_n + d\alpha\alpha'\),其中\(c,d\)为待定系数.则\(AB = -cI_n + (c + (n - 1)d)\alpha\alpha'\).令\(c = -1\),\(c + (n - 1)d = 0\),则\(d = \frac{1}{n - 1}\).于是\(AB = I_n\),从而\(A^{-1} = B = -I_n + \frac{1}{n - 1}\alpha\alpha'\).

{\color{blue}解法二\hyperref[Sherman-Morrison公式]{(Sherman-Morrison公式)}:}设\(\alpha = (1,1,\cdots,1)'\),则\(A = -I_n + \alpha\alpha'\).由\hyperref[Sherman-Morrison公式]{Sherman-Morrison公式}可得
\begin{align*}
\boldsymbol{A}^{-1}=\left( -\boldsymbol{I}_n+\boldsymbol{\alpha \alpha }^{\prime} \right) ^{-1}=\left( -\boldsymbol{I}_n \right) ^{-1}-\frac{1}{1+\boldsymbol{\alpha }^{\prime}\left( -\boldsymbol{I}_n \right) ^{-1}\boldsymbol{\alpha }}\left( -\boldsymbol{I}_n \right) ^{-1}\boldsymbol{\alpha \alpha }^{\prime}\left( -\boldsymbol{I}_n \right) ^{-1}=-\boldsymbol{I}_n+\frac{1}{n-1}\boldsymbol{\alpha \alpha }^{\prime}.
\end{align*}

{\color{blue}解法三(循环矩阵):}
设\(J\)为基础循环矩阵,则\(A = J + J^2 + \cdots + J^{n - 1}\).设\(B = cI_n + J + J^2 + \cdots + J^{n - 1}\)(因为循环矩阵的逆仍是循环矩阵),其中\(c\)为待定系数.则
\begin{align*}
AB = (n - 1)I_n + (c + n - 2)(J + J^2 + \cdots + J^{n - 1}) 
\end{align*}
只要令\(c = 2 - n\),则\(AB = (n - 1)I_n\).于是\(A^{-1} = \frac{1}{n - 1}B = \frac{2 - n}{n - 1}I_n + J + J^2 + \cdots + J^{n - 1}\).

{\color{blue}解法四(初等变换):}本题是\hyperref[exercise2.3]{练习\ref{exercise2.3}}的特例,都利用相同的初等变换方法求逆矩阵.
\end{solution}

\begin{exercise}
设\(\boldsymbol{A}\)是非零实矩阵且\(\boldsymbol{A}^* = \boldsymbol{A}'\).求证:\(\boldsymbol{A}\)是可逆阵.
\end{exercise}
\begin{proof}
设\(\boldsymbol{A} = (a_{ij})\),\(a_{ij}\)的代数余子式记为\(A_{ij}\),由已知,\(a_{ij} = A_{ij}\).由于\(\boldsymbol{A}\)是非零实矩阵,故必有某个\(a_{rs} \neq 0\),将\(|\boldsymbol{A}|\)按第\(r\)行展开,可得
\begin{align*}
|\boldsymbol{A}| = a_{r1}A_{r1} + \cdots + a_{rs}A_{rs} + \cdots + a_{rn}A_{rn} = a_{r1}^2 + \cdots + a_{rs}^2 + \cdots + a_{rn}^2 > 0.
\end{align*}
特别地,\(|\boldsymbol{A}| \neq 0\),即\(\boldsymbol{A}\)是可逆阵.
\end{proof}

\begin{exercise}
设\(\boldsymbol{A}\)是奇数阶矩阵,满足\(\boldsymbol{A}\boldsymbol{A}' = \boldsymbol{I}_{n}\)且\(|\boldsymbol{A}| > 0\),证明:\(\boldsymbol{I}_{n} - \boldsymbol{A}\)是奇异阵.
\end{exercise}
\begin{proof}
由\(1 = |\boldsymbol{I}_{n}| = |\boldsymbol{A}\boldsymbol{A}'| = |\boldsymbol{A}||\boldsymbol{A}'| = |\boldsymbol{A}|^{2}\)以及\(|\boldsymbol{A}| > 0\)可得\(|\boldsymbol{A}| = 1\).因为
\begin{align*}
|\boldsymbol{I}_{n} - \boldsymbol{A}| = |\boldsymbol{A}\boldsymbol{A}' - \boldsymbol{A}| = |\boldsymbol{A}||\boldsymbol{A}' - \boldsymbol{I}_{n}| = |(\boldsymbol{A} - \boldsymbol{I}_{n})'| = |\boldsymbol{A} - \boldsymbol{I}_{n}| = (-1)^{n}|\boldsymbol{I}_{n} - \boldsymbol{A}|.
\end{align*}
又\(n\)是奇数,故\(|\boldsymbol{I}_{n} - \boldsymbol{A}| = -|\boldsymbol{I}_{n} - \boldsymbol{A}|\),从而\(|\boldsymbol{I}_{n} - \boldsymbol{A}| = 0\),即\(\boldsymbol{I}_{n} - \boldsymbol{A}\)是奇异阵.
\end{proof}

\begin{exercise}
设\(\boldsymbol{A},\boldsymbol{B}\)为\(n\)阶可逆阵,满足\(\boldsymbol{A}^{2} = \boldsymbol{B}^{2}\)且\(|\boldsymbol{A}| + |\boldsymbol{B}| = 0\),求证:\(\boldsymbol{A} + \boldsymbol{B}\)是奇异阵.
\end{exercise}
\begin{proof}
由已知\(\boldsymbol{A},\boldsymbol{B}\)都是可逆阵且\(|\boldsymbol{B}| = -|\boldsymbol{A}|\),因此
\begin{align*}
|\boldsymbol{A}||\boldsymbol{A} + \boldsymbol{B}| = |\boldsymbol{A}^{2} + \boldsymbol{A}\boldsymbol{B}| = |\boldsymbol{B}^{2} + \boldsymbol{A}\boldsymbol{B}| = |\boldsymbol{B} + \boldsymbol{A}||\boldsymbol{B}| = -|\boldsymbol{A}||\boldsymbol{A} + \boldsymbol{B}|.
\end{align*}
于是\(|\boldsymbol{A}||\boldsymbol{A} + \boldsymbol{B}| = 0\).因为\(|\boldsymbol{A}| \neq 0\),故\(|\boldsymbol{A} + \boldsymbol{B}| = 0\),即\(\boldsymbol{A} + \boldsymbol{B}\)是奇异阵.
\end{proof}



\end{document}